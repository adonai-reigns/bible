\documentclass[twopage,twocolumn,showtrims]{memoir}


\setstocksize{228.6mm}{152.4mm}% 6in x 9in
\settrimmedsize{8.5in}{5.5in}{*}
\settrims{.25in}{.25in}

\setcolsepandrule{4.5mm}{0pt}

\setulmarginsandblock{14mm}{9mm}{1}

%\setlrmargins{16mm}{12mm}{*}
\setlrmarginsandblock{14.5mm}{7.5mm}{*}

%\setheadfoot{26mm}{12mm}
%\setheaderspaces{*}{24mm}{*}

\checkandfixthelayout 

%\setmarginnotes{1bp}{1bp}{1bp}

\quarkmarks

%\usepackage{atbegshi}
%\AtBeginShipout{\special{pdf: put @thispage <</TrimBox [36.0 36.0 612 810]>>}}
%\special{pdf: put @thispage <</TrimBox [36.0 36.0 612 810]>>}


%\usepackage[pass,b5paper,inner=16mm,outer=12mm,top=24mm,bottom=8mm]{geometry}
\usepackage{lipsum}
\usepackage{fixltx2e}
\usepackage{xspace}
\usepackage[usenames,dvipsnames,svgnames,table]{xcolor}
\usepackage{lettrine}
\usepackage{flushend}
\usepackage{fancyhdr}
\usepackage{hyperref}
\usepackage{microtype}
\usepackage[object=vectorian]{pgfornament}
\usepackage{fontspec}
\usepackage{ifpdf}
\usepackage{eso-pic}
\usepackage[british]{babel}
\usepackage{titlesec}
\usepackage{blindtext}
\usepackage{hyperref}



\setmainfont{notoserif}[
  % Files
  Path      = fonts/noto-serif/ ,
  % Fonts
  UprightFont     = NotoSerif-Regular.ttf ,
  UprightFeatures = { SmallCapsFont = NotoSerif-Regular.ttf, } ,
  BoldFont        = NotoSerif-Bold.ttf,
  BoldFeatures    = { SmallCapsFont = NotoSerif-Bold.ttf } ,
  ItalicFont      = NotoSerif-Italic.ttf ,
  BoldItalicFont  = NotoSerif-BoldItalic.ttf ,
  % Features
  Numbers = OldStyle ]


% old-style numbers don't look great as drop-caps
\newfontfamily{\lettrinefont}{AlegreyaSC}[
  % Files
  Path      = fonts/alegreya-sc/ ,
  % Fonts
  UprightFont     = *-Regular.ttf ,
  UprightFeatures = { SmallCapsFont = *-Regular.ttf } ,
  BoldFont        = *-Bold.ttf ,
  BoldFeatures    = { SmallCapsFont = *-Bold.ttf } ,
  ItalicFont      = *-Italic.ttf ,
  BoldItalicFont  = *-BoldItalic.ttf]



% old-style numbers don't look great as drop-caps
\newfontfamily{\headings}{AlegreyaSC}[
  % Files
  Path      = fonts/alegreya-sc/ ,
  % Fonts
  UprightFont     = *-Regular.ttf ,
  UprightFeatures = { SmallCapsFont = *-Regular.ttf } ,
  BoldFont        = *-Bold.ttf ,
  BoldFeatures    = { SmallCapsFont = *-Bold.ttf } ,
  ItalicFont      = *-Italic.ttf ,
  BoldItalicFont  = *-BoldItalic.ttf]



% book names font family and huge
\titleformat{\chapter}[display]{\Huge\headings\centering}{\chaptertitlename\ \thechapter}{20pt}{\Huge}
\titlespacing*{\chapter}{0pt}{0pt}{30pt}

\pagestyle{fancy}
\fancyhf{}

% the book name, chapter and verse number at the top of the page
\fancyhead[RO,LE]{\textbf{\headings{\Large \rightmark}}}

% the page number in the bottom centre of each page
\cfoot{\thepage}

\renewcommand{\headrulewidth}{0pt}


\hypersetup{colorlinks=true, linkcolor=black}

\newcommand{\framesize}{\textwidth}
\setlength{\headwidth}{\textwidth}
\setlength{\columnseprule}{0pt}

% space below the page header
\setheaderspaces{*}{0mm}{*}
\clubpenalty10000
\widowpenalty10000

\newlength{\versespacing}
\setlength{\versespacing}{.16667em plus .08333em}
\newcommand{\versespace}{\hspace{\versespacing}}
\frenchspacing

\newcommand\AtPageUpperRight[1]{\AtPageUpperLeft{%
 \put(\LenToUnit{\paperwidth},\LenToUnit{0\paperheight}){#1}%
 }}%
\newcommand\AtPageLowerRight[1]{\AtPageLowerLeft{%
 \put(\LenToUnit{\paperwidth},\LenToUnit{0\paperheight}){#1}%
 }}%

\newcommand{\sectionstyle}{\bfseries\raggedright\Large}
\setsecheadstyle{\sectionstyle}
\setbeforesecskip{0ex}
\setaftersecskip{0.1ex}

\newcommand{\subsectionstyle}{\bfseries\itshape\raggedright\large}
\setsubsecheadstyle{\subsectionstyle}
\setbeforesubsecskip{0ex}
\setaftersubsecskip{0.1ex}

\makeatletter
\newcommand\versenumcolor{red}
\newcommand\chapnumcolor{red}
\newlength{\biblechapskip}
  \setlength{\biblechapskip}{1em plus .33em minus .2em}
\newcounter{biblechapter}
\newcounter{bibleverse}[biblechapter]
\renewcommand\chaptername{Book}
\newcommand{\biblebook}[1]{%
  \setcounter{biblechapter}{0}
  \gdef\currbook{#1}
  \chapter*{#1}
  \addcontentsline{toc}{chapter}{#1}}
\newcount\biblechap@svdopt
\newenvironment{biblechapter}[1][\thebiblechapter]
  {\biblechap@svdopt=#1
  \ifnum\c@biblechapter=\biblechap@svdopt\else
    \advance\biblechap@svdopt by -1\fi
  \setcounter{biblechapter}{\the\biblechap@svdopt}
  \stepcounter{biblechapter}
  \setbeforesecskip{2mm}\setbeforesubsecskip{2mm}
  \lettrine[lines=3,lhang=0,findent=0pt,nindent=0pt,loversize=0.25]{\lettrinefont\color{\chapnumcolor}\,\thebiblechapter\,}{}\ignorespaces}
  {\par\vspace{\biblechapskip}\setbeforesecskip{0ex}\setbeforesubsecskip{0ex}}
\newcommand{\@verse}{\stepcounter{bibleverse}\markright{{\scshape\currbook} \thebiblechapter:\thebibleverse}}
\newcommand{\@showversenum}{\ifnum\c@bibleverse=1\else{\color{\versenumcolor}\textbf{\thebibleverse}~}\fi\ignorespaces}
\renewcommand{\verse}{\@verse\@showversenum}
\newcommand{\verseWithHeading}[1]{%
  \@verse%
  \ifnum\c@bibleverse=1{\headings{\sectionstyle#1}\newline}\else\vspace{\baselineskip}\newline\begin{headings}{\sectionstyle#1}\end{headings}\newline\fi\@showversenum}
\newcommand{\verseWithSubheading}[1]{%
  \@verse
  \ifnum\c@bibleverse=1{\headings{\subsectionstyle#1}\newline}\else\vspace{\baselineskip}\newline\begin{headings}{\subsectionstyle#1}\end{headings}\newline\fi\@showversenum}
\makeatother

%\newcommand{\startornaments}{\AddToShipoutPictureBG{%
%  \checkoddpage%
%  \ifoddpage%
%   \AtPageUpperRight{\put(-100,-55){\pgfornament[width=1.75cm,symmetry=h,color=black]{195}}}%
%   \AtPageLowerRight{\put(-100,55){\pgfornament[width=1.75cm,symmetry=v,color=black]{194}}}%
% \else%
%   \AtPageUpperLeft{\put(10,-55){\pgfornament[width=1.75cm,symmetry=h,color=black]{194}}}%
%   \AtPageLowerLeft{\put(10,55){\pgfornament[width=1.75cm,symmetry=v,color=black]{195}}}
% \fi}}

\newcommand{\stopornaments}{\ClearShipoutPictureBG}

\renewcommand{\printparttitle}[1]{%
  \thispagestyle{empty}%
  \addcontentsline{toc}{part}{#1}%
  \vspace*{\fill}%
  \begin{tikzpicture}[transform shape,every node/.style={inner sep=0pt}]%
    \node[minimum size=\framesize](vecbox){};%
  \node[inner sep=6pt, color=black] (text) at (vecbox.center){%
    \HUGE \headings{\textsc{#1}}};%
  \node[anchor=north, color=Goldenrod] (base) at (text.south){%
    \pgfornament[width=0.5*\framesize]{88}};%
  \end{tikzpicture}%
  \vspace*{\fill}}

\makechapterstyle{dash-embiggened}{%
  \chapterstyle{default}
  \setlength{\beforechapskip}{5\onelineskip}
  \renewcommand*{\printchaptername}{}
  \renewcommand*{\chapternamenum}{}
  \renewcommand*{\chapnumfont}{\normalfont\Huge}
  \settoheight{\midchapskip}{\chapnumfont 1}
  \renewcommand*{\printchapternum}{\centering \chapnumfont
    \rule[0.5\midchapskip]{1em}{0.4pt} \thechapter\
    \rule[0.5\midchapskip]{1em}{0.4pt}}
  \renewcommand*{\afterchapternum}{\par\nobreak\vskip 0.5\onelineskip}
  \renewcommand*{\printchapternonum}{\centering
                 \vphantom{\chapnumfont 1}\afterchapternum}
  \renewcommand*{\chaptitlefont}{\normalfont\HUGE\scshape}
  \renewcommand*{\printchaptertitle}[1]{\centering \chaptitlefont ##1}
  \setlength{\afterchapskip}{2.5\onelineskip}}

\chapterstyle{dash-embiggened}

\newcommand{\columnbreak}{\pagebreak}

\newcommand{\LORD}{\textsc{\headings{Lord}}\xspace}
\newcommand{\LORDs}{\textsc{\headings{Lord's}}\xspace}

\begin{document}

\title{\Huge \headings{The Holy Bible}}
\date{}
\author{}

\frontmatter

\begin{titlingpage}
\vspace*{\fill}

\begin{tikzpicture}[color=Gold,
    transform shape,
    every node/.style={inner sep=0pt}]
  \node[minimum size=\framesize,fill=Beige!10](vecbox){};
  \node[anchor=north west] at (vecbox.north west){%
    \pgfornament[width=0.2*\framesize]{131}};
  \node[anchor=north east] at (vecbox.north east){%
    \pgfornament[width=0.2*\framesize,symmetry=v]{131}};
  \node[anchor=south west] at (vecbox.south west){%
    \pgfornament[width=0.2*\framesize,symmetry=h]{131}};
  \node[anchor=south east] at (vecbox.south east){%
    \pgfornament[width=0.2*\framesize,symmetry=c]{131}};
  \node[anchor=north] at (vecbox.north){%
    \pgfornament[width=0.6*\framesize,symmetry=h]{85}};
  \node[anchor=south] at (vecbox.south){%
    \pgfornament[width=0.6*\framesize]{85}};
  \node[anchor=north,rotate=90] at (vecbox.west){%
    \pgfornament[width=0.6*\framesize,symmetry=h]{85}};
  \node[anchor=north,rotate=-90] at (vecbox.east){%
    \pgfornament[width=0.6*\framesize,symmetry=h]{85}};
  \node[inner sep=6pt, color=black] (text) at (vecbox.center){%
    \HUGE \textsc{The Holy Bible}};
  \node[anchor=north, color=Goldenrod] (base) at (text.south){%
    \pgfornament[width=0.5*\framesize]{71}};
  \node[anchor=south, color=Goldenrod] at (text.north){%
    \pgfornament[width=0.5*\framesize,symmetry=h]{71}};
\end{tikzpicture}

\vspace*{\fill}
\end{titlingpage}



\renewcommand{\contentsname}{\headings{Table of Contents}}
\hypersetup{colorlinks=true ,urlcolor=blue,urlbordercolor={0 1 1}}
\begin{headings}
\tableofcontents*
\end{headings}


\mainmatter
\part*{The Old Testament}

\startornaments
\biblebook{Genesis}

\begin{biblechapter} % Genesis 1
\verseWithHeading{The beginning} In the beginning God created the heaven and the earth.
\verse And the earth was without form, and void; and darkness was upon the face of the deep. And the Spirit of God moved upon the face of the waters.
\verse And God said, Let there be light: and there was light.
\verse And God saw the light, that it was good: and God divided the light from the darkness.
\verse And God called the light Day, and the darkness he called Night. And the evening and the morning were the first day.
\verse And God said, Let there be a firmament in the midst of the waters, and let it divide the waters from the waters.
\verse And God made the firmament, and divided the waters which were under the firmament from the waters which were above the firmament: and it was so.
\verse And God called the firmament Heaven. And the evening and the morning were the second day.
\verse And God said, Let the waters under the heaven be gathered together unto one place, and let the dry land appear: and it was so.
\verse And God called the dry land Earth; and the gathering together of the waters called he Seas: and God saw that it was good.
\verse And God said, Let the earth bring forth grass, the herb yielding seed, and the fruit tree yielding fruit after his kind, whose seed is in itself, upon the earth: and it was so.
\verse And the earth brought forth grass, and herb yielding seed after his kind, and the tree yielding fruit, whose seed was in itself, after his kind: and God saw that it was good.
\verse And the evening and the morning were the third day.
\verse And God said, Let there be lights in the firmament of the heaven to divide the day from the night; and let them be for signs, and for seasons, and for days, and years:
\verse And let them be for lights in the firmament of the heaven to give light upon the earth: and it was so.
\verse And God made two great lights; the greater light to rule the day, and the lesser light to rule the night: he made the stars also.
\verse And God set them in the firmament of the heaven to give light upon the earth,
\verse And to rule over the day and over the night, and to divide the light from the darkness: and God saw that it was good.
\verse And the evening and the morning were the fourth day.
\verse And God said, Let the waters bring forth abundantly the moving creature that hath life, and fowl that may fly above the earth in the open firmament of heaven.
\verse And God created great whales, and every living creature that moveth, which the waters brought forth abundantly, after their kind, and every winged fowl after his kind: and God saw that it was good.
\verse And God blessed them, saying, Be fruitful, and multiply, and fill the waters in the seas, and let fowl multiply in the earth.
\verse And the evening and the morning were the fifth day.
\verse And God said, Let the earth bring forth the living creature after his kind, cattle, and creeping thing, and beast of the earth after his kind: and it was so.
\verse And God made the beast of the earth after his kind, and cattle after their kind, and every thing that creepeth upon the earth after his kind: and God saw that it was good.
\verse And God said, Let us make man in our image, after our likeness: and let them have dominion over the fish of the sea, and over the fowl of the air, and over the cattle, and over all the earth, and over every creeping thing that creepeth upon the earth.
\verse So God created man in his own image, in the image of God created he him; male and female created he them.
\verse And God blessed them, and God said unto them, Be fruitful, and multiply, and replenish the earth, and subdue it: and have dominion over the fish of the sea, and over the fowl of the air, and over every living thing that moveth upon the earth.
\verse And God said, Behold, I have given you every herb bearing seed, which is upon the face of all the earth, and every tree, in the which is the fruit of a tree yielding seed; to you it shall be for meat.
\verse And to every beast of the earth, and to every fowl of the air, and to every thing that creepeth upon the earth, wherein there is life, I have given every green herb for meat: and it was so.
\verse And God saw every thing that he had made, and, behold, it was very good. And the evening and the morning were the sixth day.
\end{biblechapter}

\begin{biblechapter} % Genesis 2
\verse Thus the heavens and the earth were finished, and all the host of them.
\verse And on the seventh day God ended his work which he had made; and he rested on the seventh day from all his work which he had made.
\verse And God blessed the seventh day, and sanctified it: because that in it he had rested from all his work which God created and made.
\verseWithHeading{Adam and Eve} These are the generations of the heavens and of the earth when they were created, in the day that the \LORD God made the earth and the heavens,
\verse And every plant of the field before it was in the earth, and every herb of the field before it grew: for the \LORD God had not caused it to rain upon the earth, and there was not a man to till the ground.
\verse But there went up a mist from the earth, and watered the whole face of the ground.
\verse And the \LORD God formed man of the dust of the ground, and breathed into his nostrils the breath of life; and man became a living soul.
\verse And the \LORD God planted a garden eastward in Eden; and there he put the man whom he had formed.
\verse And out of the ground made the \LORD God to grow every tree that is pleasant to the sight, and good for food; the tree of life also in the midst of the garden, and the tree of knowledge of good and evil.
\verse And a river went out of Eden to water the garden; and from thence it was parted, and became into four heads.
\verse The name of the first is Pison: that is it which compasseth the whole land of Havilah, where there is gold;
\verse And the gold of that land is good: there is bdellium and the onyx stone.
\verse And the name of the second river is Gihon: the same is it that compasseth the whole land of Ethiopia.
\verse And the name of the third river is Hiddekel: that is it which goeth toward the east of Assyria. And the fourth river is Euphrates.
\verse And the \LORD God took the man, and put him into the garden of Eden to dress it and to keep it.
\verse And the \LORD God commanded the man, saying, Of every tree of the garden thou mayest freely eat:
\verse But of the tree of the knowledge of good and evil, thou shalt not eat of it: for in the day that thou eatest thereof thou shalt surely die.
\verse And the \LORD God said, It is not good that the man should be alone; I will make him an help meet for him.
\verse And out of the ground the \LORD God formed every beast of the field, and every fowl of the air; and brought them unto Adam to see what he would call them: and whatsoever Adam called every living creature, that was the name thereof.
\verse And Adam gave names to all cattle, and to the fowl of the air, and to every beast of the field; but for Adam there was not found an help meet for him.
\verse And the \LORD God caused a deep sleep to fall upon Adam, and he slept: and he took one of his ribs, and closed up the flesh instead thereof;
\verse And the rib, which the \LORD God had taken from man, made he a woman, and brought her unto the man.
\verse And Adam said, This is now bone of my bones, and flesh of my flesh: she shall be called Woman, because she was taken out of Man.
\verse Therefore shall a man leave his father and his mother, and shall cleave unto his wife: and they shall be one flesh.
\verse And they were both naked, the man and his wife, and were not ashamed.
\end{biblechapter}

\begin{biblechapter} % Genesis 3
\verseWithHeading{The fall} Now the serpent was more subtil than any beast of the field which the \LORD God had made. And he said unto the woman, Yea, hath God said, Ye shall not eat of every tree of the garden?
\verse And the woman said unto the serpent, We may eat of the fruit of the trees of the garden:
\verse But of the fruit of the tree which is in the midst of the garden, God hath said, Ye shall not eat of it, neither shall ye touch it, lest ye die.
\verse And the serpent said unto the woman, Ye shall not surely die:
\verse For God doth know that in the day ye eat thereof, then your eyes shall be opened, and ye shall be as gods, knowing good and evil.
\verse And when the woman saw that the tree was good for food, and that it was pleasant to the eyes, and a tree to be desired to make one wise, she took of the fruit thereof, and did eat, and gave also unto her husband with her; and he did eat.
\verse And the eyes of them both were opened, and they knew that they were naked; and they sewed fig leaves together, and made themselves aprons.
\verse And they heard the voice of the \LORD God walking in the garden in the cool of the day: and Adam and his wife hid themselves from the presence of the \LORD God amongst the trees of the garden.
\verse And the \LORD God called unto Adam, and said unto him, Where art thou?
\verse And he said, I heard thy voice in the garden, and I was afraid, because I was naked; and I hid myself.
\verse And he said, Who told thee that thou wast naked? Hast thou eaten of the tree, whereof I commanded thee that thou shouldest not eat?
\verse And the man said, The woman whom thou gavest to be with me, she gave me of the tree, and I did eat.
\verse And the \LORD God said unto the woman, What is this that thou hast done? And the woman said, The serpent beguiled me, and I did eat.
\verse And the \LORD God said unto the serpent, Because thou hast done this, thou art cursed above all cattle, and above every beast of the field; upon thy belly shalt thou go, and dust shalt thou eat all the days of thy life:
\verse And I will put enmity between thee and the woman, and between thy seed and her seed; it shall bruise thy head, and thou shalt bruise his heel.
\verse Unto the woman he said, I will greatly multiply thy sorrow and thy conception; in sorrow thou shalt bring forth children; and thy desire shall be to thy husband, and he shall rule over thee.
\verse And unto Adam he said, Because thou hast hearkened unto the voice of thy wife, and hast eaten of the tree, of which I commanded thee, saying, Thou shalt not eat of it: cursed is the ground for thy sake; in sorrow shalt thou eat of it all the days of thy life;
\verse Thorns also and thistles shall it bring forth to thee; and thou shalt eat the herb of the field;
\verse In the sweat of thy face shalt thou eat bread, till thou return unto the ground; for out of it wast thou taken: for dust thou art, and unto dust shalt thou return.
\verse And Adam called his wife's name Eve; because she was the mother of all living.
\verse Unto Adam also and to his wife did the \LORD God make coats of skins, and clothed them.
\verse And the \LORD God said, Behold, the man is become as one of us, to know good and evil: and now, lest he put forth his hand, and take also of the tree of life, and eat, and live for ever:
\verse Therefore the \LORD God sent him forth from the garden of Eden, to till the ground from whence he was taken.
\verse So he drove out the man; and he placed at the east of the garden of Eden Cherubims, and a flaming sword which turned every way, to keep the way of the tree of life.
\end{biblechapter}

\begin{biblechapter} % Genesis 4
\verseWithHeading{Cain and Abel} And Adam knew Eve his wife; and she conceived, and bare Cain, and said, I have gotten a man from the \LORD.
\verse And she again bare his brother Abel. And Abel was a keeper of sheep, but Cain was a tiller of the ground.
\verse And in process of time it came to pass, that Cain brought of the fruit of the ground an offering unto the \LORD.
\verse And Abel, he also brought of the firstlings of his flock and of the fat thereof. And the \LORD had respect unto Abel and to his offering:
\verse But unto Cain and to his offering he had not respect. And Cain was very wroth, and his countenance fell.
\verse And the \LORD said unto Cain, Why art thou wroth? and why is thy countenance fallen?
\verse If thou doest well, shalt thou not be accepted? and if thou doest not well, sin lieth at the door. And unto thee shall be his desire, and thou shalt rule over him.
\verse And Cain talked with Abel his brother: and it came to pass, when they were in the field, that Cain rose up against Abel his brother, and slew him.
\verse And the \LORD said unto Cain, Where is Abel thy brother? And he said, I know not: Am I my brother's keeper?
\verse And he said, What hast thou done? the voice of thy brother's blood crieth unto me from the ground.
\verse And now art thou cursed from the earth, which hath opened her mouth to receive thy brother's blood from thy hand;
\verse When thou tillest the ground, it shall not henceforth yield unto thee her strength; a fugitive and a vagabond shalt thou be in the earth.
\verse And Cain said unto the \LORD, My punishment is greater than I can bear.
\verse Behold, thou hast driven me out this day from the face of the earth; and from thy face shall I be hid; and I shall be a fugitive and a vagabond in the earth; and it shall come to pass, that every one that findeth me shall slay me.
\verse And the \LORD said unto him, Therefore whosoever slayeth Cain, vengeance shall be taken on him sevenfold. And the \LORD set a mark upon Cain, lest any finding him should kill him.
\verse And Cain went out from the presence of the \LORD, and dwelt in the land of Nod, on the east of Eden.
\verse And Cain knew his wife; and she conceived, and bare Enoch: and he builded a city, and called the name of the city, after the name of his son, Enoch.
\verse And unto Enoch was born Irad: and Irad begat Mehujael: and Mehujael begat Methusael: and Methusael begat Lamech.
\verse And Lamech took unto him two wives: the name of the one was Adah, and the name of the other Zillah.
\verse And Adah bare Jabal: he was the father of such as dwell in tents, and of such as have cattle.
\verse And his brother's name was Jubal: he was the father of all such as handle the harp and organ.
\verse And Zillah, she also bare Tubalcain, an instructer of every artificer in brass and iron: and the sister of Tubalcain was Naamah.
\verse And Lamech said unto his wives, Adah and Zillah, Hear my voice; ye wives of Lamech, hearken unto my speech: for I have slain a man to my wounding, and a young man to my hurt.
\verse If Cain shall be avenged sevenfold, truly Lamech seventy and sevenfold.
\verse And Adam knew his wife again; and she bare a son, and called his name Seth: For God, said she, hath appointed me another seed instead of Abel, whom Cain slew.
\verse And to Seth, to him also there was born a son; and he called his name Enos: then began men to call upon the name of the \LORD.
\end{biblechapter}

\begin{biblechapter} % Genesis 5
\verseWithHeading{From Adam to Noah} This is the book of the generations of Adam. In the day that God created man, in the likeness of God made he him;
\verse Male and female created he them; and blessed them, and called their name Adam, in the day when they were created.
\verse And Adam lived an hundred and thirty years, and begat a son in his own likeness, after his image; and called his name Seth:
\verse And the days of Adam after he had begotten Seth were eight hundred years: and he begat sons and daughters:
\verse And all the days that Adam lived were nine hundred and thirty years: and he died.
\verse And Seth lived an hundred and five years, and begat Enos:
\verse And Seth lived after he begat Enos eight hundred and seven years, and begat sons and daughters:
\verse And all the days of Seth were nine hundred and twelve years: and he died.
\verse And Enos lived ninety years, and begat Cainan:
\verse And Enos lived after he begat Cainan eight hundred and fifteen years, and begat sons and daughters:
\verse And all the days of Enos were nine hundred and five years: and he died.
\verse And Cainan lived seventy years, and begat Mahalaleel:
\verse And Cainan lived after he begat Mahalaleel eight hundred and forty years, and begat sons and daughters:
\verse And all the days of Cainan were nine hundred and ten years: and he died.
\verse And Mahalaleel lived sixty and five years, and begat Jared:
\verse And Mahalaleel lived after he begat Jared eight hundred and thirty years, and begat sons and daughters:
\verse And all the days of Mahalaleel were eight hundred ninety and five years: and he died.
\verse And Jared lived an hundred sixty and two years, and he begat Enoch:
\verse And Jared lived after he begat Enoch eight hundred years, and begat sons and daughters:
\verse And all the days of Jared were nine hundred sixty and two years: and he died.
\verse And Enoch lived sixty and five years, and begat Methuselah:
\verse And Enoch walked with God after he begat Methuselah three hundred years, and begat sons and daughters:
\verse And all the days of Enoch were three hundred sixty and five years:
\verse And Enoch walked with God: and he was not; for God took him.
\verse And Methuselah lived an hundred eighty and seven years, and begat Lamech:
\verse And Methuselah lived after he begat Lamech seven hundred eighty and two years, and begat sons and daughters:
\verse And all the days of Methuselah were nine hundred sixty and nine years: and he died.
\verse And Lamech lived an hundred eighty and two years, and begat a son:
\verse And he called his name Noah, saying, This same shall comfort us concerning our work and toil of our hands, because of the ground which the \LORD hath cursed.
\verse And Lamech lived after he begat Noah five hundred ninety and five years, and begat sons and daughters:
\verse And all the days of Lamech were seven hundred seventy and seven years: and he died.
\verse And Noah was five hundred years old: and Noah begat Shem, Ham, and Japheth.
\end{biblechapter}

\begin{biblechapter} % Genesis 6
\verseWithHeading{Wickedness in the world} And it came to pass, when men began to multiply on the face of the earth, and daughters were born unto them,
\verse That the sons of God saw the daughters of men that they were fair; and they took them wives of all which they chose.
\verse And the \LORD said, My spirit shall not always strive with man, for that he also is flesh: yet his days shall be an hundred and twenty years.
\verse There were giants in the earth in those days; and also after that, when the sons of God came in unto the daughters of men, and they bare children to them, the same became mighty men which were of old, men of renown.
\verse And God saw that the wickedness of man was great in the earth, and that every imagination of the thoughts of his heart was only evil continually.
\verse And it repented the \LORD that he had made man on the earth, and it grieved him at his heart.
\verse And the \LORD said, I will destroy man whom I have created from the face of the earth; both man, and beast, and the creeping thing, and the fowls of the air; for it repenteth me that I have made them.
\verse But Noah found grace in the eyes of the \LORD.
\verse These are the generations of Noah: Noah was a just man and perfect in his generations, and Noah walked with God.
\verse And Noah begat three sons, Shem, Ham, and Japheth.
\verse The earth also was corrupt before God, and the earth was filled with violence.
\verse And God looked upon the earth, and, behold, it was corrupt; for all flesh had corrupted his way upon the earth.
\verse And God said unto Noah, The end of all flesh is come before me; for the earth is filled with violence through them; and, behold, I will destroy them with the earth.
\verse Make thee an ark of gopher wood; rooms shalt thou make in the ark, and shalt pitch it within and without with pitch.
\verse And this is the fashion which thou shalt make it of: The length of the ark shall be three hundred cubits, the breadth of it fifty cubits, and the height of it thirty cubits.
\verse A window shalt thou make to the ark, and in a cubit shalt thou finish it above; and the door of the ark shalt thou set in the side thereof; with lower, second, and third stories shalt thou make it.
\verse And, behold, I, even I, do bring a flood of waters upon the earth, to destroy all flesh, wherein is the breath of life, from under heaven; and every thing that is in the earth shall die.
\verse But with thee will I establish my covenant; and thou shalt come into the ark, thou, and thy sons, and thy wife, and thy sons' wives with thee.
\verse And of every living thing of all flesh, two of every sort shalt thou bring into the ark, to keep them alive with thee; they shall be male and female.
\verse Of fowls after their kind, and of cattle after their kind, of every creeping thing of the earth after his kind, two of every sort shall come unto thee, to keep them alive.
\verse And take thou unto thee of all food that is eaten, and thou shalt gather it to thee; and it shall be for food for thee, and for them.
\verse Thus did Noah; according to all that God commanded him, so did he.
\end{biblechapter}

\begin{biblechapter} % Genesis 7
\verse And the \LORD said unto Noah, Come thou and all thy house into the ark; for thee have I seen righteous before me in this generation.
\verse Of every clean beast thou shalt take to thee by sevens, the male and his female: and of beasts that are not clean by two, the male and his female.
\verse Of fowls also of the air by sevens, the male and the female; to keep seed alive upon the face of all the earth.
\verse For yet seven days, and I will cause it to rain upon the earth forty days and forty nights; and every living substance that I have made will I destroy from off the face of the earth.
\verse And Noah did according unto all that the \LORD commanded him.
\verse And Noah was six hundred years old when the flood of waters was upon the earth.
\verse And Noah went in, and his sons, and his wife, and his sons' wives with him, into the ark, because of the waters of the flood.
\verse Of clean beasts, and of beasts that are not clean, and of fowls, and of every thing that creepeth upon the earth,
\verse There went in two and two unto Noah into the ark, the male and the female, as God had commanded Noah.
\verse And it came to pass after seven days, that the waters of the flood were upon the earth.
\verse In the six hundredth year of Noah's life, in the second month, the seventeenth day of the month, the same day were all the fountains of the great deep broken up, and the windows of heaven were opened.
\verse And the rain was upon the earth forty days and forty nights.
\verse In the selfsame day entered Noah, and Shem, and Ham, and Japheth, the sons of Noah, and Noah's wife, and the three wives of his sons with them, into the ark;
\verse They, and every beast after his kind, and all the cattle after their kind, and every creeping thing that creepeth upon the earth after his kind, and every fowl after his kind, every bird of every sort.
\verse And they went in unto Noah into the ark, two and two of all flesh, wherein is the breath of life.
\verse And they that went in, went in male and female of all flesh, as God had commanded him: and the \LORD shut him in.
\verse And the flood was forty days upon the earth; and the waters increased, and bare up the ark, and it was lift up above the earth.
\verse And the waters prevailed, and were increased greatly upon the earth; and the ark went upon the face of the waters.
\verse And the waters prevailed exceedingly upon the earth; and all the high hills, that were under the whole heaven, were covered.
\verse Fifteen cubits upward did the waters prevail; and the mountains were covered.
\verse And all flesh died that moved upon the earth, both of fowl, and of cattle, and of beast, and of every creeping thing that creepeth upon the earth, and every man:
\verse All in whose nostrils was the breath of life, of all that was in the dry land, died.
\verse And every living substance was destroyed which was upon the face of the ground, both man, and cattle, and the creeping things, and the fowl of the heaven; and they were destroyed from the earth: and Noah only remained alive, and they that were with him in the ark.
\verse And the waters prevailed upon the earth an hundred and fifty days.
\end{biblechapter}

\begin{biblechapter} % Genesis 8
\verse And God remembered Noah, and every living thing, and all the cattle that was with him in the ark: and God made a wind to pass over the earth, and the waters asswaged;
\verse The fountains also of the deep and the windows of heaven were stopped, and the rain from heaven was restrained;
\verse And the waters returned from off the earth continually: and after the end of the hundred and fifty days the waters were abated.
\verse And the ark rested in the seventh month, on the seventeenth day of the month, upon the mountains of Ararat.
\verse And the waters decreased continually until the tenth month: in the tenth month, on the first day of the month, were the tops of the mountains seen.
\verse And it came to pass at the end of forty days, that Noah opened the window of the ark which he had made:
\verse And he sent forth a raven, which went forth to and fro, until the waters were dried up from off the earth.
\verse Also he sent forth a dove from him, to see if the waters were abated from off the face of the ground;
\verse But the dove found no rest for the sole of her foot, and she returned unto him into the ark, for the waters were on the face of the whole earth: then he put forth his hand, and took her, and pulled her in unto him into the ark.
\verse And he stayed yet other seven days; and again he sent forth the dove out of the ark;
\verse And the dove came in to him in the evening; and, lo, in her mouth was an olive leaf pluckt off: so Noah knew that the waters were abated from off the earth.
\verse And he stayed yet other seven days; and sent forth the dove; which returned not again unto him any more.
\verse And it came to pass in the six hundredth and first year, in the first month, the first day of the month, the waters were dried up from off the earth: and Noah removed the covering of the ark, and looked, and, behold, the face of the ground was dry.
\verse And in the second month, on the seven and twentieth day of the month, was the earth dried.
\verse And God spake unto Noah, saying,
\verse Go forth of the ark, thou, and thy wife, and thy sons, and thy sons' wives with thee.
\verse Bring forth with thee every living thing that is with thee, of all flesh, both of fowl, and of cattle, and of every creeping thing that creepeth upon the earth; that they may breed abundantly in the earth, and be fruitful, and multiply upon the earth.
\verse And Noah went forth, and his sons, and his wife, and his sons' wives with him:
\verse Every beast, every creeping thing, and every fowl, and whatsoever creepeth upon the earth, after their kinds, went forth out of the ark.
\verse And Noah builded an altar unto the \LORD; and took of every clean beast, and of every clean fowl, and offered burnt offerings on the altar.
\verse And the \LORD smelled a sweet savour; and the \LORD said in his heart, I will not again curse the ground any more for man's sake; for the imagination of man's heart is evil from his youth; neither will I again smite any more every thing living, as I have done.
\verse While the earth remaineth, seedtime and harvest, and cold and heat, and summer and winter, and day and night shall not cease.
\end{biblechapter}

\begin{biblechapter} % Genesis 9
\verseWithHeading{God's covenant with Noah} And God blessed Noah and his sons, and said unto them, Be fruitful, and multiply, and replenish the earth.
\verse And the fear of you and the dread of you shall be upon every beast of the earth, and upon every fowl of the air, upon all that moveth upon the earth, and upon all the fishes of the sea; into your hand are they delivered.
\verse Every moving thing that liveth shall be meat for you; even as the green herb have I given you all things.
\verse But flesh with the life thereof, which is the blood thereof, shall ye not eat.
\verse And surely your blood of your lives will I require; at the hand of every beast will I require it, and at the hand of man; at the hand of every man's brother will I require the life of man.
\verse Whoso sheddeth man's blood, by man shall his blood be shed: for in the image of God made he man.
\verse And you, be ye fruitful, and multiply; bring forth abundantly in the earth, and multiply therein.
\verse And God spake unto Noah, and to his sons with him, saying,
\verse And I, behold, I establish my covenant with you, and with your seed after you;
\verse And with every living creature that is with you, of the fowl, of the cattle, and of every beast of the earth with you; from all that go out of the ark, to every beast of the earth.
\verse And I will establish my covenant with you; neither shall all flesh be cut off any more by the waters of a flood; neither shall there any more be a flood to destroy the earth.
\verse And God said, This is the token of the covenant which I make between me and you and every living creature that is with you, for perpetual generations:
\verse I do set my bow in the cloud, and it shall be for a token of a covenant between me and the earth.
\verse And it shall come to pass, when I bring a cloud over the earth, that the bow shall be seen in the cloud:
\verse And I will remember my covenant, which is between me and you and every living creature of all flesh; and the waters shall no more become a flood to destroy all flesh.
\verse And the bow shall be in the cloud; and I will look upon it, that I may remember the everlasting covenant between God and every living creature of all flesh that is upon the earth.
\verse And God said unto Noah, This is the token of the covenant, which I have established between me and all flesh that is upon the earth.
\verseWithHeading{The sons of Noah} And the sons of Noah, that went forth of the ark, were Shem, and Ham, and Japheth: and Ham is the father of Canaan.
\verse These are the three sons of Noah: and of them was the whole earth overspread.
\verse And Noah began to be an husbandman, and he planted a vineyard:
\verse And he drank of the wine, and was drunken; and he was uncovered within his tent.
\verse And Ham, the father of Canaan, saw the nakedness of his father, and told his two brethren without.
\verse And Shem and Japheth took a garment, and laid it upon both their shoulders, and went backward, and covered the nakedness of their father; and their faces were backward, and they saw not their father's nakedness.
\verse And Noah awoke from his wine, and knew what his younger son had done unto him.
\verse And he said, Cursed be Canaan; a servant of servants shall he be unto his brethren.
\verse And he said, Blessed be the \LORD God of Shem; and Canaan shall be his servant.
\verse God shall enlarge Japheth, and he shall dwell in the tents of Shem; and Canaan shall be his servant.
\verse And Noah lived after the flood three hundred and fifty years.
\verse And all the days of Noah were nine hundred and fifty years: and he died.
\end{biblechapter}

\begin{biblechapter} % Genesis 10
\verseWithHeading{The table of nations} Now these are the generations of the sons of Noah, Shem, Ham, and Japheth: and unto them were sons born after the flood.
\verseWithSubheading{The Japhethites} The sons of Japheth; Gomer, and Magog, and Madai, and Javan, and Tubal, and Meshech, and Tiras.
\verse And the sons of Gomer; Ashkenaz, and Riphath, and Togarmah.
\verse And the sons of Javan; Elishah, and Tarshish, Kittim, and Dodanim.
\verse By these were the isles of the Gentiles divided in their lands; every one after his tongue, after their families, in their nations.
\verseWithSubheading{The Hamites} And the sons of Ham; Cush, and Mizraim, and Phut, and Canaan.
\verse And the sons of Cush; Seba, and Havilah, and Sabtah, and Raamah, and Sabtecha: and the sons of Raamah; Sheba, and Dedan.
\verse And Cush begat Nimrod: he began to be a mighty one in the earth.
\verse He was a mighty hunter before the \LORD: wherefore it is said, Even as Nimrod the mighty hunter before the \LORD.
\verse And the beginning of his kingdom was Babel, and Erech, and Accad, and Calneh, in the land of Shinar.
\verse Out of that land went forth Asshur, and builded Nineveh, and the city Rehoboth, and Calah,
\verse And Resen between Nineveh and Calah: the same is a great city.
\verse And Mizraim begat Ludim, and Anamim, and Lehabim, and Naphtuhim,
\verse And Pathrusim, and Casluhim, (out of whom came Philistim,) and Caphtorim.
\verse And Canaan begat Sidon his firstborn, and Heth,
\verse And the Jebusite, and the Amorite, and the Girgasite,
\verse And the Hivite, and the Arkite, and the Sinite,
\verse And the Arvadite, and the Zemarite, and the Hamathite: and afterward were the families of the Canaanites spread abroad.
\verse And the border of the Canaanites was from Sidon, as thou comest to Gerar, unto Gaza; as thou goest, unto Sodom, and Gomorrah, and Admah, and Zeboim, even unto Lasha.
\verse These are the sons of Ham, after their families, after their tongues, in their countries, and in their nations.
\verseWithSubheading{The Semites} Unto Shem also, the father of all the children of Eber, the brother of Japheth the elder, even to him were children born.
\verse The children of Shem; Elam, and Asshur, and Arphaxad, and Lud, and Aram.
\verse And the children of Aram; Uz, and Hul, and Gether, and Mash.
\verse And Arphaxad begat Salah; and Salah begat Eber.
\verse And unto Eber were born two sons: the name of one was Peleg; for in his days was the earth divided; and his brother's name was Joktan.
\verse And Joktan begat Almodad, and Sheleph, and Hazarmaveth, and Jerah,
\verse And Hadoram, and Uzal, and Diklah,
\verse And Obal, and Abimael, and Sheba,
\verse And Ophir, and Havilah, and Jobab: all these were the sons of Joktan.
\verse And their dwelling was from Mesha, as thou goest unto Sephar a mount of the east.
\verse These are the sons of Shem, after their families, after their tongues, in their lands, after their nations.
\verse These are the families of the sons of Noah, after their generations, in their nations: and by these were the nations divided in the earth after the flood.
\end{biblechapter}

\begin{biblechapter} % Genesis 11
\verseWithHeading{The tower of Babel} And the whole earth was of one language, and of one speech.
\verse And it came to pass, as they journeyed from the east, that they found a plain in the land of Shinar; and they dwelt there.
\verse And they said one to another, Go to, let us make brick, and burn them throughly. And they had brick for stone, and slime had they for morter.
\verse And they said, Go to, let us build us a city and a tower, whose top may reach unto heaven; and let us make us a name, lest we be scattered abroad upon the face of the whole earth.
\verse And the \LORD came down to see the city and the tower, which the children of men builded.
\verse And the \LORD said, Behold, the people is one, and they have all one language; and this they begin to do: and now nothing will be restrained from them, which they have imagined to do.
\verse Go to, let us go down, and there confound their language, that they may not understand one another's speech.
\verse So the \LORD scattered them abroad from thence upon the face of all the earth: and they left off to build the city.
\verse Therefore is the name of it called Babel; because the \LORD did there confound the language of all the earth: and from thence did the \LORD scatter them abroad upon the face of all the earth.
\verseWithHeading{From Shem to Abram} These are the generations of Shem: Shem was an hundred years old, and begat Arphaxad two years after the flood:
\verse And Shem lived after he begat Arphaxad five hundred years, and begat sons and daughters.
\verse And Arphaxad lived five and thirty years, and begat Salah:
\verse And Arphaxad lived after he begat Salah four hundred and three years, and begat sons and daughters.
\verse And Salah lived thirty years, and begat Eber:
\verse And Salah lived after he begat Eber four hundred and three years, and begat sons and daughters.
\verse And Eber lived four and thirty years, and begat Peleg:
\verse And Eber lived after he begat Peleg four hundred and thirty years, and begat sons and daughters.
\verse And Peleg lived thirty years, and begat Reu:
\verse And Peleg lived after he begat Reu two hundred and nine years, and begat sons and daughters.
\verse And Reu lived two and thirty years, and begat Serug:
\verse And Reu lived after he begat Serug two hundred and seven years, and begat sons and daughters.
\verse And Serug lived thirty years, and begat Nahor:
\verse And Serug lived after he begat Nahor two hundred years, and begat sons and daughters.
\verse And Nahor lived nine and twenty years, and begat Terah:
\verse And Nahor lived after he begat Terah an hundred and nineteen years, and begat sons and daughters.
\verse And Terah lived seventy years, and begat Abram, Nahor, and Haran.
\verseWithHeading{Abram's Family} Now these are the generations of Terah: Terah begat Abram, Nahor, and Haran; and Haran begat Lot.
\verse And Haran died before his father Terah in the land of his nativity, in Ur of the Chaldees.
\verse And Abram and Nahor took them wives: the name of Abram's wife was Sarai; and the name of Nahor's wife, Milcah, the daughter of Haran, the father of Milcah, and the father of Iscah.
\verse But Sarai was barren; she had no child.
\verse And Terah took Abram his son, and Lot the son of Haran his son's son, and Sarai his daughter in law, his son Abram's wife; and they went forth with them from Ur of the Chaldees, to go into the land of Canaan; and they came unto Haran, and dwelt there.
\verse And the days of Terah were two hundred and five years: and Terah died in Haran.
\end{biblechapter}

\vfill\columnbreak % layout hack

\begin{biblechapter} % Genesis 12
\verseWithHeading{The call of Abram} Now the \LORD had said unto Abram, Get thee out of thy country, and from thy kindred, and from thy father's house, unto a land that I will shew thee:
\verse And I will make of thee a great nation, and I will bless thee, and make thy name great; and thou shalt be a blessing:
\verse And I will bless them that bless thee, and curse him that curseth thee: and in thee shall all families of the earth be blessed.
\verse So Abram departed, as the \LORD had spoken unto him; and Lot went with him: and Abram was seventy and five years old when he departed out of Haran.
\verse And Abram took Sarai his wife, and Lot his brother's son, and all their substance that they had gathered, and the souls that they had gotten in Haran; and they went forth to go into the land of Canaan; and into the land of Canaan they came.
\verse And Abram passed through the land unto the place of Sichem, unto the plain of Moreh. And the Canaanite was then in the land.
\verse And the \LORD appeared unto Abram, and said, Unto thy seed will I give this land: and there builded he an altar unto the \LORD, who appeared unto him.
\verse And he removed from thence unto a mountain on the east of Bethel, and pitched his tent, having Bethel on the west, and Hai on the east: and there he builded an altar unto the \LORD, and called upon the name of the \LORD.
\verse And Abram journeyed, going on still toward the south.
\verseWithHeading{Abram in Egypt} And there was a famine in the land: and Abram went down into Egypt to sojourn there; for the famine was grievous in the land.
\verse And it came to pass, when he was come near to enter into Egypt, that he said unto Sarai his wife, Behold now, I know that thou art a fair woman to look upon:
\verse Therefore it shall come to pass, when the Egyptians shall see thee, that they shall say, This is his wife: and they will kill me, but they will save thee alive.
\verse Say, I pray thee, thou art my sister: that it may be well with me for thy sake; and my soul shall live because of thee.
\verse And it came to pass, that, when Abram was come into Egypt, the Egyptians beheld the woman that she was very fair.
\verse The princes also of Pharaoh saw her, and commended her before Pharaoh: and the woman was taken into Pharaoh's house.
\verse And he entreated Abram well for her sake: and he had sheep, and oxen, and he asses, and menservants, and maidservants, and she asses, and camels.
\verse And the \LORD plagued Pharaoh and his house with great plagues because of Sarai Abram's wife.
\verse And Pharaoh called Abram, and said, What is this that thou hast done unto me? why didst thou not tell me that she was thy wife?
\verse Why saidst thou, She is my sister? so I might have taken her to me to wife: now therefore behold thy wife, take her, and go thy way.
\verse And Pharaoh commanded his men concerning him: and they sent him away, and his wife, and all that he had.
\end{biblechapter}

\begin{biblechapter} % Genesis 13
\verseWithHeading{Abram and Lot separate} And Abram went up out of Egypt, he, and his wife, and all that he had, and Lot with him, into the south.
\verse And Abram was very rich in cattle, in silver, and in gold.
\verse And he went on his journeys from the south even to Bethel, unto the place where his tent had been at the beginning, between Bethel and Hai;
\verse Unto the place of the altar, which he had made there at the first: and there Abram called on the name of the \LORD.
\verse And Lot also, which went with Abram, had flocks, and herds, and tents.
\verse And the land was not able to bear them, that they might dwell together: for their substance was great, so that they could not dwell together.
\verse And there was a strife between the herdmen of Abram's cattle and the herdmen of Lot's cattle: and the Canaanite and the Perizzite dwelled then in the land.
\verse And Abram said unto Lot, Let there be no strife, I pray thee, between me and thee, and between my herdmen and thy herdmen; for we be brethren.
\verse Is not the whole land before thee? separate thyself, I pray thee, from me: if thou wilt take the left hand, then I will go to the right; or if thou depart to the right hand, then I will go to the left.
\verse And Lot lifted up his eyes, and beheld all the plain of Jordan, that it was well watered every where, before the \LORD destroyed Sodom and Gomorrah, even as the garden of the \LORD, like the land of Egypt, as thou comest unto Zoar.
\verse Then Lot chose him all the plain of Jordan; and Lot journeyed east: and they separated themselves the one from the other.
\verse Abram dwelled in the land of Canaan, and Lot dwelled in the cities of the plain, and pitched his tent toward Sodom.
\verse But the men of Sodom were wicked and sinners before the \LORD exceedingly.
\verse And the \LORD said unto Abram, after that Lot was separated from him, Lift up now thine eyes, and look from the place where thou art northward, and southward, and eastward, and westward:
\verse For all the land which thou seest, to thee will I give it, and to thy seed for ever.
\verse And I will make thy seed as the dust of the earth: so that if a man can number the dust of the earth, then shall thy seed also be numbered.
\verse Arise, walk through the land in the length of it and in the breadth of it; for I will give it unto thee.
\verse Then Abram removed his tent, and came and dwelt in the plain of Mamre, which is in Hebron, and built there an altar unto the \LORD.
\end{biblechapter}

\begin{biblechapter} % Genesis 14
\verseWithHeading{Abram rescues Lot} And it came to pass in the days of Amraphel king of Shinar, Arioch king of Ellasar, Chedorlaomer king of Elam, and Tidal king of nations;
\verse That these made war with Bera king of Sodom, and with Birsha king of Gomorrah, Shinab king of Admah, and Shemeber king of Zeboiim, and the king of Bela, which is Zoar.
\verse All these were joined together in the vale of Siddim, which is the salt sea.
\verse Twelve years they served Chedorlaomer, and in the thirteenth year they rebelled.
\verse And in the fourteenth year came Chedorlaomer, and the kings that were with him, and smote the Rephaims in Ashteroth Karnaim, and the Zuzims in Ham, and the Emims in Shaveh Kiriathaim,
\verse And the Horites in their mount Seir, unto Elparan, which is by the wilderness.
\verse And they returned, and came to Enmishpat, which is Kadesh, and smote all the country of the Amalekites, and also the Amorites, that dwelt in Hazezontamar.
\verse And there went out the king of Sodom, and the king of Gomorrah, and the king of Admah, and the king of Zeboiim, and the king of Bela (the same is Zoar;) and they joined battle with them in the vale of Siddim;
\verse With Chedorlaomer the king of Elam, and with Tidal king of nations, and Amraphel king of Shinar, and Arioch king of Ellasar; four kings with five.
\verse And the vale of Siddim was full of slimepits; and the kings of Sodom and Gomorrah fled, and fell there; and they that remained fled to the mountain.
\verse And they took all the goods of Sodom and Gomorrah, and all their victuals, and went their way.
\verse And they took Lot, Abram's brother's son, who dwelt in Sodom, and his goods, and departed.
\verse And there came one that had escaped, and told Abram the Hebrew; for he dwelt in the plain of Mamre the Amorite, brother of Eshcol, and brother of Aner: and these were confederate with Abram.
\verse And when Abram heard that his brother was taken captive, he armed his trained servants, born in his own house, three hundred and eighteen, and pursued them unto Dan.
\verse And he divided himself against them, he and his servants, by night, and smote them, and pursued them unto Hobah, which is on the left hand of Damascus.
\verse And he brought back all the goods, and also brought again his brother Lot, and his goods, and the women also, and the people.
\verse And the king of Sodom went out to meet him after his return from the slaughter of Chedorlaomer, and of the kings that were with him, at the valley of Shaveh, which is the king's dale.
\verse And Melchizedek king of Salem brought forth bread and wine: and he was the priest of the most high God.
\verse And he blessed him, and said, Blessed be Abram of the most high God, possessor of heaven and earth:
\verse And blessed be the most high God, which hath delivered thine enemies into thy hand. And he gave him tithes of all.
\verse And the king of Sodom said unto Abram, Give me the persons, and take the goods to thyself.
\verse And Abram said to the king of Sodom, I have lift up mine hand unto the \LORD, the most high God, the possessor of heaven and earth,
\verse That I will not take from a thread even to a shoelatchet, and that I will not take any thing that is thine, lest thou shouldest say, I have made Abram rich:
\verse Save only that which the young men have eaten, and the portion of the men which went with me, Aner, Eshcol, and Mamre; let them take their portion.
\end{biblechapter}

\begin{biblechapter} % Genesis 15
\verseWithHeading{The \LORDs covenant with \newline Abram} After these things the word of the \LORD came unto Abram in a vision, saying, Fear not, Abram: I am thy shield, and thy exceeding great reward.
\verse And Abram said, Lord God, what wilt thou give me, seeing I go childless, and the steward of my house is this Eliezer of Damascus?
\verse And Abram said, Behold, to me thou hast given no seed: and, lo, one born in my house is mine heir.
\verse And, behold, the word of the \LORD came unto him, saying, This shall not be thine heir; but he that shall come forth out of thine own bowels shall be thine heir.
\verse And he brought him forth abroad, and said, Look now toward heaven, and tell the stars, if thou be able to number them: and he said unto him, So shall thy seed be.
\verse And he believed in the \LORD; and he counted it to him for righteousness.
\verse And he said unto him, I am the \LORD that brought thee out of Ur of the Chaldees, to give thee this land to inherit it.
\verse And he said, Lord God, whereby shall I know that I shall inherit it?
\verse And he said unto him, Take me an heifer of three years old, and a she goat of three years old, and a ram of three years old, and a turtledove, and a young pigeon.
\verse And he took unto him all these, and divided them in the midst, and laid each piece one against another: but the birds divided he not.
\verse And when the fowls came down upon the carcases, Abram drove them away.
\verse And when the sun was going down, a deep sleep fell upon Abram; and, lo, an horror of great darkness fell upon him.
\verse And he said unto Abram, Know of a surety that thy seed shall be a stranger in a land that is not theirs, and shall serve them; and they shall afflict them four hundred years;
\verse And also that nation, whom they shall serve, will I judge: and afterward shall they come out with great substance.
\verse And thou shalt go to thy fathers in peace; thou shalt be buried in a good old age.
\verse But in the fourth generation they shall come hither again: for the iniquity of the Amorites is not yet full.
\verse And it came to pass, that, when the sun went down, and it was dark, behold a smoking furnace, and a burning lamp that passed between those pieces.
\verse In the same day the \LORD made a covenant with Abram, saying, Unto thy seed have I given this land, from the river of Egypt unto the great river, the river Euphrates:
\verse The Kenites, and the Kenizzites, and the Kadmonites,
\verse And the Hittites, and the Perizzites, and the Rephaims,
\verse And the Amorites, and the Canaanites, and the Girgashites, and the Jebusites.
\end{biblechapter}

\begin{biblechapter} % Genesis 16
\verseWithHeading{Hagar and Ishmael} Now Sarai Abram's wife bare him no children: and she had an handmaid, an Egyptian, whose name was Hagar.
\verse And Sarai said unto Abram, Behold now, the \LORD hath restrained me from bearing: I pray thee, go in unto my maid; it may be that I may obtain children by her. And Abram hearkened to the voice of Sarai.
\verse And Sarai Abram's wife took Hagar her maid the Egyptian, after Abram had dwelt ten years in the land of Canaan, and gave her to her husband Abram to be his wife.
\verse And he went in unto Hagar, and she conceived: and when she saw that she had conceived, her mistress was despised in her eyes.
\verse And Sarai said unto Abram, My wrong be upon thee: I have given my maid into thy bosom; and when she saw that she had conceived, I was despised in her eyes: the \LORD judge between me and thee.
\verse But Abram said unto Sarai, Behold, thy maid is in thy hand; do to her as it pleaseth thee. And when Sarai dealt hardly with her, she fled from her face.
\verse And the angel of the \LORD found her by a fountain of water in the wilderness, by the fountain in the way to Shur.
\verse And he said, Hagar, Sarai's maid, whence camest thou? and whither wilt thou go? And she said, I flee from the face of my mistress Sarai.
\verse And the angel of the \LORD said unto her, Return to thy mistress, and submit thyself under her hands.
\verse And the angel of the \LORD said unto her, I will multiply thy seed exceedingly, that it shall not be numbered for multitude.
\verse And the angel of the \LORD said unto her, Behold, thou art with child, and shalt bear a son, and shalt call his name Ishmael; because the \LORD hath heard thy affliction.
\verse And he will be a wild man; his hand will be against every man, and every man's hand against him; and he shall dwell in the presence of all his brethren.
\verse And she called the name of the \LORD that spake unto her, Thou God seest me: for she said, Have I also here looked after him that seeth me?
\verse Wherefore the well was called Beerlahairoi; behold, it is between Kadesh and Bered.
\verse And Hagar bare Abram a son: and Abram called his son's name, which Hagar bare, Ishmael.
\verse And Abram was fourscore and six years old, when Hagar bare Ishmael to Abram.
\end{biblechapter}

\begin{biblechapter} % Genesis 17
\verseWithHeading{The covenant of circumcision} And when Abram was ninety years old and nine, the \LORD appeared to Abram, and said unto him, I am the Almighty God; walk before me, and be thou perfect.
\verse And I will make my covenant between me and thee, and will multiply thee exceedingly.
\verse And Abram fell on his face: and God talked with him, saying,
\verse As for me, behold, my covenant is with thee, and thou shalt be a father of many nations.
\verse Neither shall thy name any more be called Abram, but thy name shall be Abraham; for a father of many nations have I made thee.
\verse And I will make thee exceeding fruitful, and I will make nations of thee, and kings shall come out of thee.
\verse And I will establish my covenant between me and thee and thy seed after thee in their generations for an everlasting covenant, to be a God unto thee, and to thy seed after thee.
\verse And I will give unto thee, and to thy seed after thee, the land wherein thou art a stranger, all the land of Canaan, for an everlasting possession; and I will be their God.
\verse And God said unto Abraham, Thou shalt keep my covenant therefore, thou, and thy seed after thee in their generations.
\verse This is my covenant, which ye shall keep, between me and you and thy seed after thee; Every man child among you shall be circumcised.
\verse And ye shall circumcise the flesh of your foreskin; and it shall be a token of the covenant betwixt me and you.
\verse And he that is eight days old shall be circumcised among you, every man child in your generations, he that is born in the house, or bought with money of any stranger, which is not of thy seed.
\verse He that is born in thy house, and he that is bought with thy money, must needs be circumcised: and my covenant shall be in your flesh for an everlasting covenant.
\verse And the uncircumcised man child whose flesh of his foreskin is not circumcised, that soul shall be cut off from his people; he hath broken my covenant.
\verse And God said unto Abraham, As for Sarai thy wife, thou shalt not call her name Sarai, but Sarah shall her name be.
\verse And I will bless her, and give thee a son also of her: yea, I will bless her, and she shall be a mother of nations; kings of people shall be of her.
\verse Then Abraham fell upon his face, and laughed, and said in his heart, Shall a child be born unto him that is an hundred years old? and shall Sarah, that is ninety years old, bear?
\verse And Abraham said unto God, O that Ishmael might live before thee!
\verse And God said, Sarah thy wife shall bear thee a son indeed; and thou shalt call his name Isaac: and I will establish my covenant with him for an everlasting covenant, and with his seed after him.
\verse And as for Ishmael, I have heard thee: Behold, I have blessed him, and will make him fruitful, and will multiply him exceedingly; twelve princes shall he beget, and I will make him a great nation.
\verse But my covenant will I establish with Isaac, which Sarah shall bear unto thee at this set time in the next year.
\verse And he left off talking with him, and God went up from Abraham.
\verse And Abraham took Ishmael his son, and all that were born in his house, and all that were bought with his money, every male among the men of Abraham's house; and circumcised the flesh of their foreskin in the selfsame day, as God had said unto him.
\verse And Abraham was ninety years old and nine, when he was circumcised in the flesh of his foreskin.
\verse And Ishmael his son was thirteen years old, when he was circumcised in the flesh of his foreskin.
\verse In the selfsame day was Abraham circumcised, and Ishmael his son.
\verse And all the men of his house, born in the house, and bought with money of the stranger, were circumcised with him.
\end{biblechapter}

\begin{biblechapter} % Genesis 18
\verseWithHeading{The three visitors} And the \LORD appeared unto him in the plains of Mamre: and he sat in the tent door in the heat of the day;
\verse And he lift up his eyes and looked, and, lo, three men stood by him: and when he saw them, he ran to meet them from the tent door, and bowed himself toward the ground,
\verse And said, My Lord, if now I have found favour in thy sight, pass not away, I pray thee, from thy servant:
\verse Let a little water, I pray you, be fetched, and wash your feet, and rest yourselves under the tree:
\verse And I will fetch a morsel of bread, and comfort ye your hearts; after that ye shall pass on: for therefore are ye come to your servant. And they said, So do, as thou hast said.
\verse And Abraham hastened into the tent unto Sarah, and said, Make ready quickly three measures of fine meal, knead it, and make cakes upon the hearth.
\verse And Abraham ran unto the herd, and fetcht a calf tender and good, and gave it unto a young man; and he hasted to dress it.
\verse And he took butter, and milk, and the calf which he had dressed, and set it before them; and he stood by them under the tree, and they did eat.
\verse And they said unto him, Where is Sarah thy wife? And he said, Behold, in the tent.
\verse And he said, I will certainly return unto thee according to the time of life; and, lo, Sarah thy wife shall have a son. And Sarah heard it in the tent door, which was behind him.
\verse Now Abraham and Sarah were old and well stricken in age; and it ceased to be with Sarah after the manner of women.
\verse Therefore Sarah laughed within herself, saying, After I am waxed old shall I have pleasure, my lord being old also?
\verse And the \LORD said unto Abraham, Wherefore did Sarah laugh, saying, Shall I of a surety bear a child, which am old?
\verse Is any thing too hard for the \LORD? At the time appointed I will return unto thee, according to the time of life, and Sarah shall have a son.
\verse Then Sarah denied, saying, I laughed not; for she was afraid. And he said, Nay; but thou didst laugh.
\verseWithHeading{Abraham pleads for Sodom} And the men rose up from thence, and looked toward Sodom: and Abraham went with them to bring them on the way.
\verse And the \LORD said, Shall I hide from Abraham that thing which I do;
\verse Seeing that Abraham shall surely become a great and mighty nation, and all the nations of the earth shall be blessed in him?
\verse For I know him, that he will command his children and his household after him, and they shall keep the way of the \LORD, to do justice and judgment; that the \LORD may bring upon Abraham that which he hath spoken of him.
\verse And the \LORD said, Because the cry of Sodom and Gomorrah is great, and because their sin is very grievous;
\verse I will go down now, and see whether they have done altogether according to the cry of it, which is come unto me; and if not, I will know.
\verse And the men turned their faces from thence, and went toward Sodom: but Abraham stood yet before the \LORD.
\verse And Abraham drew near, and said, Wilt thou also destroy the righteous with the wicked?
\verse Peradventure there be fifty righteous within the city: wilt thou also destroy and not spare the place for the fifty righteous that are therein?
\verse That be far from thee to do after this manner, to slay the righteous with the wicked: and that the righteous should be as the wicked, that be far from thee: Shall not the Judge of all the earth do right?
\verse And the \LORD said, If I find in Sodom fifty righteous within the city, then I will spare all the place for their sakes.
\verse And Abraham answered and said, Behold now, I have taken upon me to speak unto the Lord, which am but dust and ashes:
\verse Peradventure there shall lack five of the fifty righteous: wilt thou destroy all the city for lack of five? And he said, If I find there forty and five, I will not destroy it.
\verse And he spake unto him yet again, and said, Peradventure there shall be forty found there. And he said, I will not do it for forty's sake.
\verse And he said unto him, Oh let not the Lord be angry, and I will speak: Peradventure there shall thirty be found there. And he said, I will not do it, if I find thirty there.
\verse And he said, Behold now, I have taken upon me to speak unto the Lord: Peradventure there shall be twenty found there. And he said, I will not destroy it for twenty's sake.
\verse And he said, Oh let not the Lord be angry, and I will speak yet but this once: Peradventure ten shall be found there. And he said, I will not destroy it for ten's sake.
\verse And the \LORD went his way, as soon as he had left communing with Abraham: and Abraham returned unto his place.
\end{biblechapter}

\begin{biblechapter} % Genesis 19
\verseWithHeading{Sodom and Gomorrah \newline destroyed} And there came two angels to Sodom at even; and Lot sat in the gate of Sodom: and Lot seeing them rose up to meet them; and he bowed himself with his face toward the ground;
\verse And he said, Behold now, my lords, turn in, I pray you, into your servant's house, and tarry all night, and wash your feet, and ye shall rise up early, and go on your ways. And they said, Nay; but we will abide in the street all night.
\verse And he pressed upon them greatly; and they turned in unto him, and entered into his house; and he made them a feast, and did bake unleavened bread, and they did eat.
\verse But before they lay down, the men of the city, even the men of Sodom, compassed the house round, both old and young, all the people from every quarter:
\verse And they called unto Lot, and said unto him, Where are the men which came in to thee this night? bring them out unto us, that we may know them.
\verse And Lot went out at the door unto them, and shut the door after him,
\verse And said, I pray you, brethren, do not so wickedly.
\verse Behold now, I have two daughters which have not known man; let me, I pray you, bring them out unto you, and do ye to them as is good in your eyes: only unto these men do nothing; for therefore came they under the shadow of my roof.
\verse And they said, Stand back. And they said again, This one fellow came in to sojourn, and he will needs be a judge: now will we deal worse with thee, than with them. And they pressed sore upon the man, even Lot, and came near to break the door.
\verse But the men put forth their hand, and pulled Lot into the house to them, and shut to the door.
\verse And they smote the men that were at the door of the house with blindness, both small and great: so that they wearied themselves to find the door.
\verse And the men said unto Lot, Hast thou here any besides? son in law, and thy sons, and thy daughters, and whatsoever thou hast in the city, bring them out of this place:
\verse For we will destroy this place, because the cry of them is waxen great before the face of the \LORD; and the \LORD hath sent us to destroy it.
\verse And Lot went out, and spake unto his sons in law, which married his daughters, and said, Up, get you out of this place; for the \LORD will destroy this city. But he seemed as one that mocked unto his sons in law.
\verse And when the morning arose, then the angels hastened Lot, saying, Arise, take thy wife, and thy two daughters, which are here; lest thou be consumed in the iniquity of the city.
\verse And while he lingered, the men laid hold upon his hand, and upon the hand of his wife, and upon the hand of his two daughters; the \LORD being merciful unto him: and they brought him forth, and set him without the city.
\verse And it came to pass, when they had brought them forth abroad, that he said, Escape for thy life; look not behind thee, neither stay thou in all the plain; escape to the mountain, lest thou be consumed.
\verse And Lot said unto them, Oh, not so, my Lord:
\verse Behold now, thy servant hath found grace in thy sight, and thou hast magnified thy mercy, which thou hast shewed unto me in saving my life; and I cannot escape to the mountain, lest some evil take me, and I die:
\verse Behold now, this city is near to flee unto, and it is a little one: Oh, let me escape thither, (is it not a little one?) and my soul shall live.
\verse And he said unto him, See, I have accepted thee concerning this thing also, that I will not overthrow this city, for the which thou hast spoken.
\verse Haste thee, escape thither; for I cannot do any thing till thou be come thither. Therefore the name of the city was called Zoar.
\verse The sun was risen upon the earth when Lot entered into Zoar.
\verse Then the \LORD rained upon Sodom and upon Gomorrah brimstone and fire from the \LORD out of heaven;
\verse And he overthrew those cities, and all the plain, and all the inhabitants of the cities, and that which grew upon the ground.
\verse But his wife looked back from behind him, and she became a pillar of salt.
\verse And Abraham gat up early in the morning to the place where he stood before the \LORD:
\verse And he looked toward Sodom and Gomorrah, and toward all the land of the plain, and beheld, and, lo, the smoke of the country went up as the smoke of a furnace.
\verse And it came to pass, when God destroyed the cities of the plain, that God remembered Abraham, and sent Lot out of the midst of the overthrow, when he overthrew the cities in the which Lot dwelt.
\verseWithHeading{Lot and his daughters} And Lot went up out of Zoar, and dwelt in the mountain, and his two daughters with him; for he feared to dwell in Zoar: and he dwelt in a cave, he and his two daughters.
\verse And the firstborn said unto the younger, Our father is old, and there is not a man in the earth to come in unto us after the manner of all the earth:
\verse Come, let us make our father drink wine, and we will lie with him, that we may preserve seed of our father.
\verse And they made their father drink wine that night: and the firstborn went in, and lay with her father; and he perceived not when she lay down, nor when she arose.
\verse And it came to pass on the morrow, that the firstborn said unto the younger, Behold, I lay yesternight with my father: let us make him drink wine this night also; and go thou in, and lie with him, that we may preserve seed of our father.
\verse And they made their father drink wine that night also: and the younger arose, and lay with him; and he perceived not when she lay down, nor when she arose.
\verse Thus were both the daughters of Lot with child by their father.
\verse And the firstborn bare a son, and called his name Moab: the same is the father of the Moabites unto this day.
\verse And the younger, she also bare a son, and called his name Benammi: the same is the father of the children of Ammon unto this day.
\end{biblechapter}

\begin{biblechapter} % Genesis 20
\verseWithHeading{Abraham and Abimelech} And Abraham journeyed from thence toward the south country, and dwelled between Kadesh and Shur, and sojourned in Gerar.
\verse And Abraham said of Sarah his wife, She is my sister: and Abimelech king of Gerar sent, and took Sarah.
\verse But God came to Abimelech in a dream by night, and said to him, Behold, thou art but a dead man, for the woman which thou hast taken; for she is a man's wife.
\verse But Abimelech had not come near her: and he said, Lord, wilt thou slay also a righteous nation?
\verse Said he not unto me, She is my sister? and she, even she herself said, He is my brother: in the integrity of my heart and innocency of my hands have I done this.
\verse And God said unto him in a dream, Yea, I know that thou didst this in the integrity of thy heart; for I also withheld thee from sinning against me: therefore suffered I thee not to touch her.
\verse Now therefore restore the man his wife; for he is a prophet, and he shall pray for thee, and thou shalt live: and if thou restore her not, know thou that thou shalt surely die, thou, and all that are thine.
\verse Therefore Abimelech rose early in the morning, and called all his servants, and told all these things in their ears: and the men were sore afraid.
\verse Then Abimelech called Abraham, and said unto him, What hast thou done unto us? and what have I offended thee, that thou hast brought on me and on my kingdom a great sin? thou hast done deeds unto me that ought not to be done.
\verse And Abimelech said unto Abraham, What sawest thou, that thou hast done this thing?
\verse And Abraham said, Because I thought, Surely the fear of God is not in this place; and they will slay me for my wife's sake.
\verse And yet indeed she is my sister; she is the daughter of my father, but not the daughter of my mother; and she became my wife.
\verse And it came to pass, when God caused me to wander from my father's house, that I said unto her, This is thy kindness which thou shalt shew unto me; at every place whither we shall come, say of me, He is my brother.
\verse And Abimelech took sheep, and oxen, and menservants, and womenservants, and gave them unto Abraham, and restored him Sarah his wife.
\verse And Abimelech said, Behold, my land is before thee: dwell where it pleaseth thee.
\verse And unto Sarah he said, Behold, I have given thy brother a thousand pieces of silver: behold, he is to thee a covering of the eyes, unto all that are with thee, and with all other: thus she was reproved.
\verse So Abraham prayed unto God: and God healed Abimelech, and his wife, and his maidservants; and they bare children.
\verse For the \LORD had fast closed up all the wombs of the house of Abimelech, because of Sarah Abraham's wife.
\end{biblechapter}

\begin{biblechapter} % Genesis 21
\verseWithHeading{The birth of Isaac} And the \LORD visited Sarah as he had said, and the \LORD did unto Sarah as he had spoken.
\verse For Sarah conceived, and bare Abraham a son in his old age, at the set time of which God had spoken to him.
\verse And Abraham called the name of his son that was born unto him, whom Sarah bare to him, Isaac.
\verse And Abraham circumcised his son Isaac being eight days old, as God had commanded him.
\verse And Abraham was an hundred years old, when his son Isaac was born unto him.
\verse And Sarah said, God hath made me to laugh, so that all that hear will laugh with me.
\verse And she said, Who would have said unto Abraham, that Sarah should have given children suck? for I have born him a son in his old age.
\verseWithHeading{Hagar and Ishmael sent away} And the child grew, and was weaned: and Abraham made a great feast the same day that Isaac was weaned.
\verse And Sarah saw the son of Hagar the Egyptian, which she had born unto Abraham, mocking.
\verse Wherefore she said unto Abraham, Cast out this bondwoman and her son: for the son of this bondwoman shall not be heir with my son, even with Isaac.
\verse And the thing was very grievous in Abraham's sight because of his son.
\verse And God said unto Abraham, Let it not be grievous in thy sight because of the lad, and because of thy bondwoman; in all that Sarah hath said unto thee, hearken unto her voice; for in Isaac shall thy seed be called.
\verse And also of the son of the bondwoman will I make a nation, because he is thy seed.
\verse And Abraham rose up early in the morning, and took bread, and a bottle of water, and gave it unto Hagar, putting it on her shoulder, and the child, and sent her away: and she departed, and wandered in the wilderness of Beersheba.
\verse And the water was spent in the bottle, and she cast the child under one of the shrubs.
\verse And she went, and sat her down over against him a good way off, as it were a bowshot: for she said, Let me not see the death of the child. And she sat over against him, and lift up her voice, and wept.
\verse And God heard the voice of the lad; and the angel of God called to Hagar out of heaven, and said unto her, What aileth thee, Hagar? fear not; for God hath heard the voice of the lad where he is.
\verse Arise, lift up the lad, and hold him in thine hand; for I will make him a great nation.
\verse And God opened her eyes, and she saw a well of water; and she went, and filled the bottle with water, and gave the lad drink.
\verse And God was with the lad; and he grew, and dwelt in the wilderness, and became an archer.
\verse And he dwelt in the wilderness of Paran: and his mother took him a wife out of the land of Egypt.
\verseWithHeading{The treaty at Beersheba} And it came to pass at that time, that Abimelech and Phichol the chief captain of his host spake unto Abraham, saying, God is with thee in all that thou doest:
\verse Now therefore swear unto me here by God that thou wilt not deal falsely with me, nor with my son, nor with my son's son: but according to the kindness that I have done unto thee, thou shalt do unto me, and to the land wherein thou hast sojourned.
\verse And Abraham said, I will swear.
\verse And Abraham reproved Abimelech because of a well of water, which Abimelech's servants had violently taken away.
\verse And Abimelech said, I wot not who hath done this thing: neither didst thou tell me, neither yet heard I of it, but to day.
\verse And Abraham took sheep and oxen, and gave them unto Abimelech; and both of them made a covenant.
\verse And Abraham set seven ewe lambs of the flock by themselves.
\verse And Abimelech said unto Abraham, What mean these seven ewe lambs which thou hast set by themselves?
\verse And he said, For these seven ewe lambs shalt thou take of my hand, that they may be a witness unto me, that I have digged this well.
\verse Wherefore he called that place Beersheba; because there they sware both of them.
\verse Thus they made a covenant at Beersheba: then Abimelech rose up, and Phichol the chief captain of his host, and they returned into the land of the Philistines.
\verse And Abraham planted a grove in Beersheba, and called there on the name of the \LORD, the everlasting God.
\verse And Abraham sojourned in the Philistines' land many days.
\end{biblechapter}

\begin{biblechapter} % Genesis 22
\verseWithHeading{Abraham tested} And it came to pass after these things, that God did tempt Abraham, and said unto him, Abraham: and he said, Behold, here I am.
\verse And he said, Take now thy son, thine only son Isaac, whom thou lovest, and get thee into the land of Moriah; and offer him there for a burnt offering upon one of the mountains which I will tell thee of.
\verse And Abraham rose up early in the morning, and saddled his ass, and took two of his young men with him, and Isaac his son, and clave the wood for the burnt offering, and rose up, and went unto the place of which God had told him.
\verse Then on the third day Abraham lifted up his eyes, and saw the place afar off.
\verse And Abraham said unto his young men, Abide ye here with the ass; and I and the lad will go yonder and worship, and come again to you.
\verse And Abraham took the wood of the burnt offering, and laid it upon Isaac his son; and he took the fire in his hand, and a knife; and they went both of them together.
\verse And Isaac spake unto Abraham his father, and said, My father: and he said, Here am I, my son. And he said, Behold the fire and the wood: but where is the lamb for a burnt offering?
\verse And Abraham said, My son, God will provide himself a lamb for a burnt offering: so they went both of them together.
\verse And they came to the place which God had told him of; and Abraham built an altar there, and laid the wood in order, and bound Isaac his son, and laid him on the altar upon the wood.
\verse And Abraham stretched forth his hand, and took the knife to slay his son.
\verse And the angel of the \LORD called unto him out of heaven, and said, Abraham, Abraham: and he said, Here am I.
\verse And he said, Lay not thine hand upon the lad, neither do thou any thing unto him: for now I know that thou fearest God, seeing thou hast not withheld thy son, thine only son from me.
\verse And Abraham lifted up his eyes, and looked, and behold behind him a ram caught in a thicket by his horns: and Abraham went and took the ram, and offered him up for a burnt offering in the stead of his son.
\verse And Abraham called the name of that place Jehovahjireh: as it is said to this day, In the mount of the \LORD it shall be seen.
\verse And the angel of the \LORD called unto Abraham out of heaven the second time,
\verse And said, By myself have I sworn, saith the \LORD, for because thou hast done this thing, and hast not withheld thy son, thine only son:
\verse That in blessing I will bless thee, and in multiplying I will multiply thy seed as the stars of the heaven, and as the sand which is upon the sea shore; and thy seed shall possess the gate of his enemies;
\verse And in thy seed shall all the nations of the earth be blessed; because thou hast obeyed my voice.
\verse So Abraham returned unto his young men, and they rose up and went together to Beersheba; and Abraham dwelt at Beersheba.
\verseWithHeading{Nahor's sons} And it came to pass after these things, that it was told Abraham, saying, Behold, Milcah, she hath also born children unto thy brother Nahor;
\verse Huz his firstborn, and Buz his brother, and Kemuel the father of Aram,
\verse And Chesed, and Hazo, and Pildash, and Jidlaph, and Bethuel.
\verse And Bethuel begat Rebekah: these eight Milcah did bear to Nahor, Abraham's brother.
\verse And his concubine, whose name was Reumah, she bare also Tebah, and Gaham, and Thahash, and Maachah.
\end{biblechapter}

\begin{biblechapter} % Genesis 23
\verseWithHeading{The death of Sarah} And Sarah was an hundred and seven and twenty years old: these were the years of the life of Sarah.
\verse And Sarah died in Kirjatharba; the same is Hebron in the land of Canaan: and Abraham came to mourn for Sarah, and to weep for her.
\verse And Abraham stood up from before his dead, and spake unto the sons of Heth, saying,
\verse I am a stranger and a sojourner with you: give me a possession of a buryingplace with you, that I may bury my dead out of my sight.
\verse And the children of Heth answered Abraham, saying unto him,
\verse Hear us, my lord: thou art a mighty prince among us: in the choice of our sepulchres bury thy dead; none of us shall withhold from thee his sepulchre, but that thou mayest bury thy dead.
\verse And Abraham stood up, and bowed himself to the people of the land, even to the children of Heth.
\verse And he communed with them, saying, If it be your mind that I should bury my dead out of my sight; hear me, and intreat for me to Ephron the son of Zohar,
\verse That he may give me the cave of Machpelah, which he hath, which is in the end of his field; for as much money as it is worth he shall give it me for a possession of a buryingplace amongst you.
\verse And Ephron dwelt among the children of Heth: and Ephron the Hittite answered Abraham in the audience of the children of Heth, even of all that went in at the gate of his city, saying,
\verse Nay, my lord, hear me: the field give I thee, and the cave that is therein, I give it thee; in the presence of the sons of my people give I it thee: bury thy dead.
\verse And Abraham bowed down himself before the people of the land.
\verse And he spake unto Ephron in the audience of the people of the land, saying, But if thou wilt give it, I pray thee, hear me: I will give thee money for the field; take it of me, and I will bury my dead there.
\verse And Ephron answered Abraham, saying unto him,
\verse My lord, hearken unto me: the land is worth four hundred shekels of silver; what is that betwixt me and thee? bury therefore thy dead.
\verse And Abraham hearkened unto Ephron; and Abraham weighed to Ephron the silver, which he had named in the audience of the sons of Heth, four hundred shekels of silver, current money with the merchant.
\verse And the field of Ephron, which was in Machpelah, which was before Mamre, the field, and the cave which was therein, and all the trees that were in the field, that were in all the borders round about, were made sure
\verse Unto Abraham for a possession in the presence of the children of Heth, before all that went in at the gate of his city.
\verse And after this, Abraham buried Sarah his wife in the cave of the field of Machpelah before Mamre: the same is Hebron in the land of Canaan.
\verse And the field, and the cave that is therein, were made sure unto Abraham for a possession of a buryingplace by the sons of Heth.
\end{biblechapter}

\begin{biblechapter} % Genesis 24
\verseWithHeading{Isaac and Rebekah} And Abraham was old, and well stricken in age: and the \LORD had blessed Abraham in all things.
\verse And Abraham said unto his eldest servant of his house, that ruled over all that he had, Put, I pray thee, thy hand under my thigh:
\verse And I will make thee swear by the \LORD, the God of heaven, and the God of the earth, that thou shalt not take a wife unto my son of the daughters of the Canaanites, among whom I dwell:
\verse But thou shalt go unto my country, and to my kindred, and take a wife unto my son Isaac.
\verse And the servant said unto him, Peradventure the woman will not be willing to follow me unto this land: must I needs bring thy son again unto the land from whence thou camest?
\verse And Abraham said unto him, Beware thou that thou bring not my son thither again.
\verse The \LORD God of heaven, which took me from my father's house, and from the land of my kindred, and which spake unto me, and that sware unto me, saying, Unto thy seed will I give this land; he shall send his angel before thee, and thou shalt take a wife unto my son from thence.
\verse And if the woman will not be willing to follow thee, then thou shalt be clear from this my oath: only bring not my son thither again.
\verse And the servant put his hand under the thigh of Abraham his master, and sware to him concerning that matter.
\verse And the servant took ten camels of the camels of his master, and departed; for all the goods of his master were in his hand: and he arose, and went to Mesopotamia, unto the city of Nahor.
\verse And he made his camels to kneel down without the city by a well of water at the time of the evening, even the time that women go out to draw water.
\verse And he said, O \LORD God of my master Abraham, I pray thee, send me good speed this day, and shew kindness unto my master Abraham.
\verse Behold, I stand here by the well of water; and the daughters of the men of the city come out to draw water:
\verse And let it come to pass, that the damsel to whom I shall say, Let down thy pitcher, I pray thee, that I may drink; and she shall say, Drink, and I will give thy camels drink also: let the same be she that thou hast appointed for thy servant Isaac; and thereby shall I know that thou hast shewed kindness unto my master.
\verse And it came to pass, before he had done speaking, that, behold, Rebekah came out, who was born to Bethuel, son of Milcah, the wife of Nahor, Abraham's brother, with her pitcher upon her shoulder.
\verse And the damsel was very fair to look upon, a virgin, neither had any man known her: and she went down to the well, and filled her pitcher, and came up.
\verse And the servant ran to meet her, and said, Let me, I pray thee, drink a little water of thy pitcher.
\verse And she said, Drink, my lord: and she hasted, and let down her pitcher upon her hand, and gave him drink.
\verse And when she had done giving him drink, she said, I will draw water for thy camels also, until they have done drinking.
\verse And she hasted, and emptied her pitcher into the trough, and ran again unto the well to draw water, and drew for all his camels.
\verse And the man wondering at her held his peace, to wit whether the \LORD had made his journey prosperous or not.
\verse And it came to pass, as the camels had done drinking, that the man took a golden earring of half a shekel weight, and two bracelets for her hands of ten shekels weight of gold;
\verse And said, Whose daughter art thou? tell me, I pray thee: is there room in thy father's house for us to lodge in?
\verse And she said unto him, I am the daughter of Bethuel the son of Milcah, which she bare unto Nahor.
\verse She said moreover unto him, We have both straw and provender enough, and room to lodge in.
\verse And the man bowed down his head, and worshipped the \LORD.
\verse And he said, Blessed be the \LORD God of my master Abraham, who hath not left destitute my master of his mercy and his truth: I being in the way, the \LORD led me to the house of my master's brethren.
\verse And the damsel ran, and told them of her mother's house these things.
\verse And Rebekah had a brother, and his name was Laban: and Laban ran out unto the man, unto the well.
\verse And it came to pass, when he saw the earring and bracelets upon his sister's hands, and when he heard the words of Rebekah his sister, saying, Thus spake the man unto me; that he came unto the man; and, behold, he stood by the camels at the well.
\verse And he said, Come in, thou blessed of the \LORD; wherefore standest thou without? for I have prepared the house, and room for the camels.
\verse And the man came into the house: and he ungirded his camels, and gave straw and provender for the camels, and water to wash his feet, and the men's feet that were with him.
\verse And there was set meat before him to eat: but he said, I will not eat, until I have told mine errand. And he said, Speak on.
\verse And he said, I am Abraham's servant.
\verse And the \LORD hath blessed my master greatly; and he is become great: and he hath given him flocks, and herds, and silver, and gold, and menservants, and maidservants, and camels, and asses.
\verse And Sarah my master's wife bare a son to my master when she was old: and unto him hath he given all that he hath.
\verse And my master made me swear, saying, Thou shalt not take a wife to my son of the daughters of the Canaanites, in whose land I dwell:
\verse But thou shalt go unto my father's house, and to my kindred, and take a wife unto my son.
\verse And I said unto my master, Peradventure the woman will not follow me.
\verse And he said unto me, The \LORD, before whom I walk, will send his angel with thee, and prosper thy way; and thou shalt take a wife for my son of my kindred, and of my father's house:
\verse Then shalt thou be clear from this my oath, when thou comest to my kindred; and if they give not thee one, thou shalt be clear from my oath.
\verse And I came this day unto the well, and said, O \LORD God of my master Abraham, if now thou do prosper my way which I go:
\verse Behold, I stand by the well of water; and it shall come to pass, that when the virgin cometh forth to draw water, and I say to her, Give me, I pray thee, a little water of thy pitcher to drink;
\verse And she say to me, Both drink thou, and I will also draw for thy camels: let the same be the woman whom the \LORD hath appointed out for my master's son.
\verse And before I had done speaking in mine heart, behold, Rebekah came forth with her pitcher on her shoulder; and she went down unto the well, and drew water: and I said unto her, Let me drink, I pray thee.
\verse And she made haste, and let down her pitcher from her shoulder, and said, Drink, and I will give thy camels drink also: so I drank, and she made the camels drink also.
\verse And I asked her, and said, Whose daughter art thou? And she said, The daughter of Bethuel, Nahor's son, whom Milcah bare unto him: and I put the earring upon her face, and the bracelets upon her hands.
\verse And I bowed down my head, and worshipped the \LORD, and blessed the \LORD God of my master Abraham, which had led me in the right way to take my master's brother's daughter unto his son.
\verse And now if ye will deal kindly and truly with my master, tell me: and if not, tell me; that I may turn to the right hand, or to the left.
\verse Then Laban and Bethuel answered and said, The thing proceedeth from the \LORD: we cannot speak unto thee bad or good.
\verse Behold, Rebekah is before thee, take her, and go, and let her be thy master's son's wife, as the \LORD hath spoken.
\verse And it came to pass, that, when Abraham's servant heard their words, he worshipped the \LORD, bowing himself to the earth.
\verse And the servant brought forth jewels of silver, and jewels of gold, and raiment, and gave them to Rebekah: he gave also to her brother and to her mother precious things.
\verse And they did eat and drink, he and the men that were with him, and tarried all night; and they rose up in the morning, and he said, Send me away unto my master.
\verse And her brother and her mother said, Let the damsel abide with us a few days, at the least ten; after that she shall go.
\verse And he said unto them, Hinder me not, seeing the \LORD hath prospered my way; send me away that I may go to my master.
\verse And they said, We will call the damsel, and enquire at her mouth.
\verse And they called Rebekah, and said unto her, Wilt thou go with this man? And she said, I will go.
\verse And they sent away Rebekah their sister, and her nurse, and Abraham's servant, and his men.
\verse And they blessed Rebekah, and said unto her, Thou art our sister, be thou the mother of thousands of millions, and let thy seed possess the gate of those which hate them.
\verse And Rebekah arose, and her damsels, and they rode upon the camels, and followed the man: and the servant took Rebekah, and went his way.
\verse And Isaac came from the way of the well Lahairoi; for he dwelt in the south country.
\verse And Isaac went out to meditate in the field at the eventide: and he lifted up his eyes, and saw, and, behold, the camels were coming.
\verse And Rebekah lifted up her eyes, and when she saw Isaac, she lighted off the camel.
\verse For she had said unto the servant, What man is this that walketh in the field to meet us? And the servant had said, It is my master: therefore she took a vail, and covered herself.
\verse And the servant told Isaac all things that he had done.
\verse And Isaac brought her into his mother Sarah's tent, and took Rebekah, and she became his wife; and he loved her: and Isaac was comforted after his mother's death.
\end{biblechapter}

\begin{biblechapter} % Genesis 25
\verseWithHeading{The death of Abraham} Then again Abraham took a wife, and her name was Keturah.
\verse And she bare him Zimran, and Jokshan, and Medan, and Midian, and Ishbak, and Shuah.
\verse And Jokshan begat Sheba, and Dedan. And the sons of Dedan were Asshurim, and Letushim, and Leummim.
\verse And the sons of Midian; Ephah, and Epher, and Hanoch, and Abida, and Eldaah. All these were the children of Keturah.
\verse And Abraham gave all that he had unto Isaac.
\verse But unto the sons of the concubines, which Abraham had, Abraham gave gifts, and sent them away from Isaac his son, while he yet lived, eastward, unto the east country.
\verse And these are the days of the years of Abraham's life which he lived, an hundred threescore and fifteen years.
\verse Then Abraham gave up the ghost, and died in a good old age, an old man, and full of years; and was gathered to his people.
\verse And his sons Isaac and Ishmael buried him in the cave of Machpelah, in the field of Ephron the son of Zohar the Hittite, which is before Mamre;
\verse The field which Abraham purchased of the sons of Heth: there was Abraham buried, and Sarah his wife.
\verse And it came to pass after the death of Abraham, that God blessed his son Isaac; and Isaac dwelt by the well Lahairoi.
\verseWithHeading{Ishmael's sons} Now these are the generations of Ishmael, Abraham's son, whom Hagar the Egyptian, Sarah's handmaid, bare unto Abraham:
\verse And these are the names of the sons of Ishmael, by their names, according to their generations: the firstborn of Ishmael, Nebajoth; and Kedar, and Adbeel, and Mibsam,
\verse And Mishma, and Dumah, and Massa,
\verse Hadar, and Tema, Jetur, Naphish, and Kedemah:
\verse These are the sons of Ishmael, and these are their names, by their towns, and by their castles; twelve princes according to their nations.
\verse And these are the years of the life of Ishmael, an hundred and thirty and seven years: and he gave up the ghost and died; and was gathered unto his people.
\verse And they dwelt from Havilah unto Shur, that is before Egypt, as thou goest toward Assyria: and he died in the presence of all his brethren.
\verseWithHeading{Jacob and Esau} And these are the generations of Isaac, Abraham's son: Abraham begat Isaac:
\verse And Isaac was forty years old when he took Rebekah to wife, the daughter of Bethuel the Syrian of Padanaram, the sister to Laban the Syrian.
\verse And Isaac intreated the \LORD for his wife, because she was barren: and the \LORD was intreated of him, and Rebekah his wife conceived.
\verse And the children struggled together within her; and she said, If it be so, why am I thus? And she went to enquire of the \LORD.
\verse And the \LORD said unto her, Two nations are in thy womb, and two manner of people shall be separated from thy bowels; and the one people shall be stronger than the other people; and the elder shall serve the younger.
\verse And when her days to be delivered were fulfilled, behold, there were twins in her womb.
\verse And the first came out red, all over like an hairy garment; and they called his name Esau.
\verse And after that came his brother out, and his hand took hold on Esau's heel; and his name was called Jacob: and Isaac was threescore years old when she bare them.
\verse And the boys grew: and Esau was a cunning hunter, a man of the field; and Jacob was a plain man, dwelling in tents.
\verse And Isaac loved Esau, because he did eat of his venison: but Rebekah loved Jacob.
\verse And Jacob sod pottage: and Esau came from the field, and he was faint:
\verse And Esau said to Jacob, Feed me, I pray thee, with that same red pottage; for I am faint: therefore was his name called Edom.
\verse And Jacob said, Sell me this day thy birthright.
\verse And Esau said, Behold, I am at the point to die: and what profit shall this birthright do to me?
\verse And Jacob said, Swear to me this day; and he sware unto him: and he sold his birthright unto Jacob.
\verse Then Jacob gave Esau bread and pottage of lentiles; and he did eat and drink, and rose up, and went his way: thus Esau despised his birthright.
\end{biblechapter}

\begin{biblechapter} % Genesis 26
\verseWithHeading{Isaac and Abimelech} And there was a famine in the land, beside the first famine that was in the days of Abraham. And Isaac went unto Abimelech king of the Philistines unto Gerar.
\verse And the \LORD appeared unto him, and said, Go not down into Egypt; dwell in the land which I shall tell thee of:
\verse Sojourn in this land, and I will be with thee, and will bless thee; for unto thee, and unto thy seed, I will give all these countries, and I will perform the oath which I sware unto Abraham thy father;
\verse And I will make thy seed to multiply as the stars of heaven, and will give unto thy seed all these countries; and in thy seed shall all the nations of the earth be blessed;
\verse Because that Abraham obeyed my voice, and kept my charge, my commandments, my statutes, and my laws.
\verse And Isaac dwelt in Gerar:
\verse And the men of the place asked him of his wife; and he said, She is my sister: for he feared to say, She is my wife; lest, said he, the men of the place should kill me for Rebekah; because she was fair to look upon.
\verse And it came to pass, when he had been there a long time, that Abimelech king of the Philistines looked out at a window, and saw, and, behold, Isaac was sporting with Rebekah his wife.
\verse And Abimelech called Isaac, and said, Behold, of a surety she is thy wife: and how saidst thou, She is my sister? And Isaac said unto him, Because I said, Lest I die for her.
\verse And Abimelech said, What is this thou hast done unto us? one of the people might lightly have lien with thy wife, and thou shouldest have brought guiltiness upon us.
\verse And Abimelech charged all his people, saying, He that toucheth this man or his wife shall surely be put to death.
\verse Then Isaac sowed in that land, and received in the same year an hundredfold: and the \LORD blessed him.
\verse And the man waxed great, and went forward, and grew until he became very great:
\verse For he had possession of flocks, and possession of herds, and great store of servants: and the Philistines envied him.
\verse For all the wells which his father's servants had digged in the days of Abraham his father, the Philistines had stopped them, and filled them with earth.
\verse And Abimelech said unto Isaac, Go from us; for thou art much mightier than we.
\verse And Isaac departed thence, and pitched his tent in the valley of Gerar, and dwelt there.
\verse And Isaac digged again the wells of water, which they had digged in the days of Abraham his father; for the Philistines had stopped them after the death of Abraham: and he called their names after the names by which his father had called them.
\verse And Isaac's servants digged in the valley, and found there a well of springing water.
\verse And the herdmen of Gerar did strive with Isaac's herdmen, saying, The water is ours: and he called the name of the well Esek; because they strove with him.
\verse And they digged another well, and strove for that also: and he called the name of it Sitnah.
\verse And he removed from thence, and digged another well; and for that they strove not: and he called the name of it Rehoboth; and he said, For now the \LORD hath made room for us, and we shall be fruitful in the land.
\verse And he went up from thence to Beersheba.
\verse And the \LORD appeared unto him the same night, and said, I am the God of Abraham thy father: fear not, for I am with thee, and will bless thee, and multiply thy seed for my servant Abraham's sake.
\verse And he builded an altar there, and called upon the name of the \LORD, and pitched his tent there: and there Isaac's servants digged a well.
\verse Then Abimelech went to him from Gerar, and Ahuzzath one of his friends, and Phichol the chief captain of his army.
\verse And Isaac said unto them, Wherefore come ye to me, seeing ye hate me, and have sent me away from you?
\verse And they said, We saw certainly that the \LORD was with thee: and we said, Let there be now an oath betwixt us, even betwixt us and thee, and let us make a covenant with thee;
\verse That thou wilt do us no hurt, as we have not touched thee, and as we have done unto thee nothing but good, and have sent thee away in peace: thou art now the blessed of the \LORD.
\verse And he made them a feast, and they did eat and drink.
\verse And they rose up betimes in the morning, and sware one to another: and Isaac sent them away, and they departed from him in peace.
\verse And it came to pass the same day, that Isaac's servants came, and told him concerning the well which they had digged, and said unto him, We have found water.
\verse And he called it Shebah: therefore the name of the city is Beersheba unto this day.
\verseWithHeading{Jacob takes Esau's blessing} And Esau was forty years old when he took to wife Judith the daughter of Beeri the Hittite, and Bashemath the daughter of Elon the Hittite:
\verse Which were a grief of mind unto Isaac and to Rebekah.
\end{biblechapter}

\begin{biblechapter} % Genesis 27
\verse And it came to pass, that when Isaac was old, and his eyes were dim, so that he could not see, he called Esau his eldest son, and said unto him, My son: and he said unto him, Behold, here am I.
\verse And he said, Behold now, I am old, I know not the day of my death:
\verse Now therefore take, I pray thee, thy weapons, thy quiver and thy bow, and go out to the field, and take me some venison;
\verse And make me savoury meat, such as I love, and bring it to me, that I may eat; that my soul may bless thee before I die.
\verse And Rebekah heard when Isaac spake to Esau his son. And Esau went to the field to hunt for venison, and to bring it.
\verse And Rebekah spake unto Jacob her son, saying, Behold, I heard thy father speak unto Esau thy brother, saying,
\verse Bring me venison, and make me savoury meat, that I may eat, and bless thee before the \LORD before my death.
\verse Now therefore, my son, obey my voice according to that which I command thee.
\verse Go now to the flock, and fetch me from thence two good kids of the goats; and I will make them savoury meat for thy father, such as he loveth:
\verse And thou shalt bring it to thy father, that he may eat, and that he may bless thee before his death.
\verse And Jacob said to Rebekah his mother, Behold, Esau my brother is a hairy man, and I am a smooth man:
\verse My father peradventure will feel me, and I shall seem to him as a deceiver; and I shall bring a curse upon me, and not a blessing.
\verse And his mother said unto him, Upon me be thy curse, my son: only obey my voice, and go fetch me them.
\verse And he went, and fetched, and brought them to his mother: and his mother made savoury meat, such as his father loved.
\verse And Rebekah took goodly raiment of her eldest son Esau, which were with her in the house, and put them upon Jacob her younger son:
\verse And she put the skins of the kids of the goats upon his hands, and upon the smooth of his neck:
\verse And she gave the savoury meat and the bread, which she had prepared, into the hand of her son Jacob.
\verse And he came unto his father, and said, My father: and he said, Here am I; who art thou, my son?
\verse And Jacob said unto his father, I am Esau thy firstborn; I have done according as thou badest me: arise, I pray thee, sit and eat of my venison, that thy soul may bless me.
\verse And Isaac said unto his son, How is it that thou hast found it so quickly, my son? And he said, Because the \LORD thy God brought it to me.
\verse And Isaac said unto Jacob, Come near, I pray thee, that I may feel thee, my son, whether thou be my very son Esau or not.
\verse And Jacob went near unto Isaac his father; and he felt him, and said, The voice is Jacob's voice, but the hands are the hands of Esau.
\verse And he discerned him not, because his hands were hairy, as his brother Esau's hands: so he blessed him.
\verse And he said, Art thou my very son Esau? And he said, I am.
\verse And he said, Bring it near to me, and I will eat of my son's venison, that my soul may bless thee. And he brought it near to him, and he did eat: and he brought him wine, and he drank.
\verse And his father Isaac said unto him, Come near now, and kiss me, my son.
\verse And he came near, and kissed him: and he smelled the smell of his raiment, and blessed him, and said, See, the smell of my son is as the smell of a field which the \LORD hath blessed:
\verse Therefore God give thee of the dew of heaven, and the fatness of the earth, and plenty of corn and wine:
\verse Let people serve thee, and nations bow down to thee: be lord over thy brethren, and let thy mother's sons bow down to thee: cursed be every one that curseth thee, and blessed be he that blesseth thee.
\verse And it came to pass, as soon as Isaac had made an end of blessing Jacob, and Jacob was yet scarce gone out from the presence of Isaac his father, that Esau his brother came in from his hunting.
\verse And he also had made savoury meat, and brought it unto his father, and said unto his father, Let my father arise, and eat of his son's venison, that thy soul may bless me.
\verse And Isaac his father said unto him, Who art thou? And he said, I am thy son, thy firstborn Esau.
\verse And Isaac trembled very exceedingly, and said, Who? where is he that hath taken venison, and brought it me, and I have eaten of all before thou camest, and have blessed him? yea, and he shall be blessed.
\verse And when Esau heard the words of his father, he cried with a great and exceeding bitter cry, and said unto his father, Bless me, even me also, O my father.
\verse And he said, Thy brother came with subtilty, and hath taken away thy blessing.
\verse And he said, Is not he rightly named Jacob? for he hath supplanted me these two times: he took away my birthright; and, behold, now he hath taken away my blessing. And he said, Hast thou not reserved a blessing for me?
\verse And Isaac answered and said unto Esau, Behold, I have made him thy lord, and all his brethren have I given to him for servants; and with corn and wine have I sustained him: and what shall I do now unto thee, my son?
\verse And Esau said unto his father, Hast thou but one blessing, my father? bless me, even me also, O my father. And Esau lifted up his voice, and wept.
\verse And Isaac his father answered and said unto him, Behold, thy dwelling shall be the fatness of the earth, and of the dew of heaven from above;
\verse And by thy sword shalt thou live, and shalt serve thy brother; and it shall come to pass when thou shalt have the dominion, that thou shalt break his yoke from off thy neck.
\verse And Esau hated Jacob because of the blessing wherewith his father blessed him: and Esau said in his heart, The days of mourning for my father are at hand; then will I slay my brother Jacob.
\verse And these words of Esau her elder son were told to Rebekah: and she sent and called Jacob her younger son, and said unto him, Behold, thy brother Esau, as touching thee, doth comfort himself, purposing to kill thee.
\verse Now therefore, my son, obey my voice; and arise, flee thou to Laban my brother to Haran;
\verse And tarry with him a few days, until thy brother's fury turn away;
\verse Until thy brother's anger turn away from thee, and he forget that which thou hast done to him: then I will send, and fetch thee from thence: why should I be deprived also of you both in one day?
\verse And Rebekah said to Isaac, I am weary of my life because of the daughters of Heth: if Jacob take a wife of the daughters of Heth, such as these which are of the daughters of the land, what good shall my life do me?
\end{biblechapter}

\begin{biblechapter} % Genesis 28
\verse And Isaac called Jacob, and blessed him, and charged him, and said unto him, Thou shalt not take a wife of the daughters of Canaan.
\verse Arise, go to Padanaram, to the house of Bethuel thy mother's father; and take thee a wife from thence of the daughters of Laban thy mother's brother.
\verse And God Almighty bless thee, and make thee fruitful, and multiply thee, that thou mayest be a multitude of people;
\verse And give thee the blessing of Abraham, to thee, and to thy seed with thee; that thou mayest inherit the land wherein thou art a stranger, which God gave unto Abraham.
\verse And Isaac sent away Jacob: and he went to Padanaram unto Laban, son of Bethuel the Syrian, the brother of Rebekah, Jacob's and Esau's mother.
\verse When Esau saw that Isaac had blessed Jacob, and sent him away to Padanaram, to take him a wife from thence; and that as he blessed him he gave him a charge, saying, Thou shalt not take a wife of the daughters of Canaan;
\verse And that Jacob obeyed his father and his mother, and was gone to Padanaram;
\verse And Esau seeing that the daughters of Canaan pleased not Isaac his father;
\verse Then went Esau unto Ishmael, and took unto the wives which he had Mahalath the daughter of Ishmael Abraham's son, the sister of Nebajoth, to be his wife.
\verseWithHeading{Jacob's dreams at Bethel} And Jacob went out from Beersheba, and went toward Haran.
\verse And he lighted upon a certain place, and tarried there all night, because the sun was set; and he took of the stones of that place, and put them for his pillows, and lay down in that place to sleep.
\verse And he dreamed, and behold a ladder set up on the earth, and the top of it reached to heaven: and behold the angels of God ascending and descending on it.
\verse And, behold, the \LORD stood above it, and said, I am the \LORD God of Abraham thy father, and the God of Isaac: the land whereon thou liest, to thee will I give it, and to thy seed;
\verse And thy seed shall be as the dust of the earth, and thou shalt spread abroad to the west, and to the east, and to the north, and to the south: and in thee and in thy seed shall all the families of the earth be blessed.
\verse And, behold, I am with thee, and will keep thee in all places whither thou goest, and will bring thee again into this land; for I will not leave thee, until I have done that which I have spoken to thee of.
\verse And Jacob awaked out of his sleep, and he said, Surely the \LORD is in this place; and I knew it not.
\verse And he was afraid, and said, How dreadful is this place! this is none other but the house of God, and this is the gate of heaven.
\verse And Jacob rose up early in the morning, and took the stone that he had put for his pillows, and set it up for a pillar, and poured oil upon the top of it.
\verse And he called the name of that place Bethel: but the name of that city was called Luz at the first.
\verse And Jacob vowed a vow, saying, If God will be with me, and will keep me in this way that I go, and will give me bread to eat, and raiment to put on,
\verse So that I come again to my father's house in peace; then shall the \LORD be my God:
\verse And this stone, which I have set for a pillar, shall be God's house: and of all that thou shalt give me I will surely give the tenth unto thee.
\end{biblechapter}

\begin{biblechapter} % Genesis 29
\verseWithHeading{Jacob arives in Padanaram} Then Jacob went on his journey, and came into the land of the people of the east.
\verse And he looked, and behold a well in the field, and, lo, there were three flocks of sheep lying by it; for out of that well they watered the flocks: and a great stone was upon the well's mouth.
\verse And thither were all the flocks gathered: and they rolled the stone from the well's mouth, and watered the sheep, and put the stone again upon the well's mouth in his place.
\verse And Jacob said unto them, My brethren, whence be ye? And they said, Of Haran are we.
\verse And he said unto them, Know ye Laban the son of Nahor? And they said, We know him.
\verse And he said unto them, Is he well? And they said, He is well: and, behold, Rachel his daughter cometh with the sheep.
\verse And he said, Lo, it is yet high day, neither is it time that the cattle should be gathered together: water ye the sheep, and go and feed them.
\verse And they said, We cannot, until all the flocks be gathered together, and till they roll the stone from the well's mouth; then we water the sheep.
\verse And while he yet spake with them, Rachel came with her father's sheep: for she kept them.
\verse And it came to pass, when Jacob saw Rachel the daughter of Laban his mother's brother, and the sheep of Laban his mother's brother, that Jacob went near, and rolled the stone from the well's mouth, and watered the flock of Laban his mother's brother.
\verse And Jacob kissed Rachel, and lifted up his voice, and wept.
\verse And Jacob told Rachel that he was her father's brother, and that he was Rebekah's son: and she ran and told her father.
\verse And it came to pass, when Laban heard the tidings of Jacob his sister's son, that he ran to meet him, and embraced him, and kissed him, and brought him to his house. And he told Laban all these things.
\verse And Laban said to him, Surely thou art my bone and my flesh. And he abode with him the space of a month.
\verseWithHeading{Jacob marries Leah and Rachel} And Laban said unto Jacob, Because thou art my brother, shouldest thou therefore serve me for nought? tell me, what shall thy wages be?
\verse And Laban had two daughters: the name of the elder was Leah, and the name of the younger was Rachel.
\verse Leah was tender eyed; but Rachel was beautiful and well favoured.
\verse And Jacob loved Rachel; and said, I will serve thee seven years for Rachel thy younger daughter.
\verse And Laban said, It is better that I give her to thee, than that I should give her to another man: abide with me.
\verse And Jacob served seven years for Rachel; and they seemed unto him but a few days, for the love he had to her.
\verse And Jacob said unto Laban, Give me my wife, for my days are fulfilled, that I may go in unto her.
\verse And Laban gathered together all the men of the place, and made a feast.
\verse And it came to pass in the evening, that he took Leah his daughter, and brought her to him; and he went in unto her.
\verse And Laban gave unto his daughter Leah Zilpah his maid for an handmaid.
\verse And it came to pass, that in the morning, behold, it was Leah: and he said to Laban, What is this thou hast done unto me? did not I serve with thee for Rachel? wherefore then hast thou beguiled me?
\verse And Laban said, It must not be so done in our country, to give the younger before the firstborn.
\verse Fulfil her week, and we will give thee this also for the service which thou shalt serve with me yet seven other years.
\verse And Jacob did so, and fulfilled her week: and he gave him Rachel his daughter to wife also.
\verse And Laban gave to Rachel his daughter Bilhah his handmaid to be her maid.
\verse And he went in also unto Rachel, and he loved also Rachel more than Leah, and served with him yet seven other years.
\verseWithHeading{Jacob's children} And when the \LORD saw that Leah was hated, he opened her womb: but Rachel was barren.
\verse And Leah conceived, and bare a son, and she called his name Reuben: for she said, Surely the \LORD hath looked upon my affliction; now therefore my husband will love me.
\verse And she conceived again, and bare a son; and said, Because the \LORD hath heard that I was hated, he hath therefore given me this son also: and she called his name Simeon.
\verse And she conceived again, and bare a son; and said, Now this time will my husband be joined unto me, because I have born him three sons: therefore was his name called Levi.
\verse And she conceived again, and bare a son: and she said, Now will I praise the \LORD: therefore she called his name Judah; and left bearing.
\end{biblechapter}

\begin{biblechapter} % Genesis 30
\verse And when Rachel saw that she bare Jacob no children, Rachel envied her sister; and said unto Jacob, Give me children, or else I die.
\verse And Jacob's anger was kindled against Rachel: and he said, Am I in God's stead, who hath withheld from thee the fruit of the womb?
\verse And she said, Behold my maid Bilhah, go in unto her; and she shall bear upon my knees, that I may also have children by her.
\verse And she gave him Bilhah her handmaid to wife: and Jacob went in unto her.
\verse And Bilhah conceived, and bare Jacob a son.
\verse And Rachel said, God hath judged me, and hath also heard my voice, and hath given me a son: therefore called she his name Dan.
\verse And Bilhah Rachel's maid conceived again, and bare Jacob a second son.
\verse And Rachel said, With great wrestlings have I wrestled with my sister, and I have prevailed: and she called his name Naphtali.
\verse When Leah saw that she had left bearing, she took Zilpah her maid, and gave her Jacob to wife.
\verse And Zilpah Leah's maid bare Jacob a son.
\verse And Leah said, A troop cometh: and she called his name Gad.
\verse And Zilpah Leah's maid bare Jacob a second son.
\verse And Leah said, Happy am I, for the daughters will call me blessed: and she called his name Asher.
\verse And Reuben went in the days of wheat harvest, and found mandrakes in the field, and brought them unto his mother Leah. Then Rachel said to Leah, Give me, I pray thee, of thy son's mandrakes.
\verse And she said unto her, Is it a small matter that thou hast taken my husband? and wouldest thou take away my son's mandrakes also? And Rachel said, Therefore he shall lie with thee to night for thy son's mandrakes.
\verse And Jacob came out of the field in the evening, and Leah went out to meet him, and said, Thou must come in unto me; for surely I have hired thee with my son's mandrakes. And he lay with her that night.
\verse And God hearkened unto Leah, and she conceived, and bare Jacob the fifth son.
\verse And Leah said, God hath given me my hire, because I have given my maiden to my husband: and she called his name Issachar.
\verse And Leah conceived again, and bare Jacob the sixth son.
\verse And Leah said, God hath endued me with a good dowry; now will my husband dwell with me, because I have born him six sons: and she called his name Zebulun.
\verse And afterwards she bare a daughter, and called her name Dinah.
\verse And God remembered Rachel, and God hearkened to her, and opened her womb.
\verse And she conceived, and bare a son; and said, God hath taken away my reproach:
\verse And she called his name Joseph; and said, The \LORD shall add to me another son.
\verseWithHeading{Jacob's flocks increase} And it came to pass, when Rachel had born Joseph, that Jacob said unto Laban, Send me away, that I may go unto mine own place, and to my country.
\verse Give me my wives and my children, for whom I have served thee, and let me go: for thou knowest my service which I have done thee.
\verse And Laban said unto him, I pray thee, if I have found favour in thine eyes, tarry: for I have learned by experience that the \LORD hath blessed me for thy sake.
\verse And he said, Appoint me thy wages, and I will give it.
\verse And he said unto him, Thou knowest how I have served thee, and how thy cattle was with me.
\verse For it was little which thou hadst before I came, and it is now increased unto a multitude; and the \LORD hath blessed thee since my coming: and now when shall I provide for mine own house also?
\verse And he said, What shall I give thee? And Jacob said, Thou shalt not give me any thing: if thou wilt do this thing for me, I will again feed and keep thy flock:
\verse I will pass through all thy flock to day, removing from thence all the speckled and spotted cattle, and all the brown cattle among the sheep, and the spotted and speckled among the goats: and of such shall be my hire.
\verse So shall my righteousness answer for me in time to come, when it shall come for my hire before thy face: every one that is not speckled and spotted among the goats, and brown among the sheep, that shall be counted stolen with me.
\verse And Laban said, Behold, I would it might be according to thy word.
\verse And he removed that day the he goats that were ringstraked and spotted, and all the she goats that were speckled and spotted, and every one that had some white in it, and all the brown among the sheep, and gave them into the hand of his sons.
\verse And he set three days' journey betwixt himself and Jacob: and Jacob fed the rest of Laban's flocks.
\verse And Jacob took him rods of green poplar, and of the hazel and chesnut tree; and pilled white strakes in them, and made the white appear which was in the rods.
\verse And he set the rods which he had pilled before the flocks in the gutters in the watering troughs when the flocks came to drink, that they should conceive when they came to drink.
\verse And the flocks conceived before the rods, and brought forth cattle ringstraked, speckled, and spotted.
\verse And Jacob did separate the lambs, and set the faces of the flocks toward the ringstraked, and all the brown in the flock of Laban; and he put his own flocks by themselves, and put them not unto Laban's cattle.
\verse And it came to pass, whensoever the stronger cattle did conceive, that Jacob laid the rods before the eyes of the cattle in the gutters, that they might conceive among the rods.
\verse But when the cattle were feeble, he put them not in: so the feebler were Laban's, and the stronger Jacob's.
\verse And the man increased exceedingly, and had much cattle, and maidservants, and menservants, and camels, and asses.
\end{biblechapter}

\begin{biblechapter} % Genesis 31
\verseWithHeading{Jacob flees from Laban} And he heard the words of Laban's sons, saying, Jacob hath taken away all that was our father's; and of that which was our father's hath he gotten all this glory.
\verse And Jacob beheld the countenance of Laban, and, behold, it was not toward him as before.
\verse And the \LORD said unto Jacob, Return unto the land of thy fathers, and to thy kindred; and I will be with thee.
\verse And Jacob sent and called Rachel and Leah to the field unto his flock,
\verse And said unto them, I see your father's countenance, that it is not toward me as before; but the God of my father hath been with me.
\verse And ye know that with all my power I have served your father.
\verse And your father hath deceived me, and changed my wages ten times; but God suffered him not to hurt me.
\verse If he said thus, The speckled shall be thy wages; then all the cattle bare speckled: and if he said thus, The ringstraked shall be thy hire; then bare all the cattle ringstraked.
\verse Thus God hath taken away the cattle of your father, and given them to me.
\verse And it came to pass at the time that the cattle conceived, that I lifted up mine eyes, and saw in a dream, and, behold, the rams which leaped upon the cattle were ringstraked, speckled, and grisled.
\verse And the angel of God spake unto me in a dream, saying, Jacob: And I said, Here am I.
\verse And he said, Lift up now thine eyes, and see, all the rams which leap upon the cattle are ringstraked, speckled, and grisled: for I have seen all that Laban doeth unto thee.
\verse I am the God of Bethel, where thou anointedst the pillar, and where thou vowedst a vow unto me: now arise, get thee out from this land, and return unto the land of thy kindred.
\verse And Rachel and Leah answered and said unto him, Is there yet any portion or inheritance for us in our father's house?
\verse Are we not counted of him strangers? for he hath sold us, and hath quite devoured also our money.
\verse For all the riches which God hath taken from our father, that is ours, and our children's: now then, whatsoever God hath said unto thee, do.
\verse Then Jacob rose up, and set his sons and his wives upon camels;
\verse And he carried away all his cattle, and all his goods which he had gotten, the cattle of his getting, which he had gotten in Padanaram, for to go to Isaac his father in the land of Canaan.
\verse And Laban went to shear his sheep: and Rachel had stolen the images that were her father's.
\verse And Jacob stole away unawares to Laban the Syrian, in that he told him not that he fled.
\verse So he fled with all that he had; and he rose up, and passed over the river, and set his face toward the mount Gilead.
\verseWithHeading{Laban pursues Jacob} And it was told Laban on the third day that Jacob was fled.
\verse And he took his brethren with him, and pursued after him seven days' journey; and they overtook him in the mount Gilead.
\verse And God came to Laban the Syrian in a dream by night, and said unto him, Take heed that thou speak not to Jacob either good or bad.
\verse Then Laban overtook Jacob. Now Jacob had pitched his tent in the mount: and Laban with his brethren pitched in the mount of Gilead.
\verse And Laban said to Jacob, What hast thou done, that thou hast stolen away unawares to me, and carried away my daughters, as captives taken with the sword?
\verse Wherefore didst thou flee away secretly, and steal away from me; and didst not tell me, that I might have sent thee away with mirth, and with songs, with tabret, and with harp?
\verse And hast not suffered me to kiss my sons and my daughters? thou hast now done foolishly in so doing.
\verse It is in the power of my hand to do you hurt: but the God of your father spake unto me yesternight, saying, Take thou heed that thou speak not to Jacob either good or bad.
\verse And now, though thou wouldest needs be gone, because thou sore longedst after thy father's house, yet wherefore hast thou stolen my gods?
\verse And Jacob answered and said to Laban, Because I was afraid: for I said, Peradventure thou wouldest take by force thy daughters from me.
\verse With whomsoever thou findest thy gods, let him not live: before our brethren discern thou what is thine with me, and take it to thee. For Jacob knew not that Rachel had stolen them.
\verse And Laban went into Jacob's tent, and into Leah's tent, and into the two maidservants' tents; but he found them not. Then went he out of Leah's tent, and entered into Rachel's tent.
\verse Now Rachel had taken the images, and put them in the camel's furniture, and sat upon them. And Laban searched all the tent, but found them not.
\verse And she said to her father, Let it not displease my lord that I cannot rise up before thee; for the custom of women is upon me. And he searched, but found not the images.
\verse And Jacob was wroth, and chode with Laban: and Jacob answered and said to Laban, What is my trespass? what is my sin, that thou hast so hotly pursued after me?
\verse Whereas thou hast searched all my stuff, what hast thou found of all thy household stuff? set it here before my brethren and thy brethren, that they may judge betwixt us both.
\verse This twenty years have I been with thee; thy ewes and thy she goats have not cast their young, and the rams of thy flock have I not eaten.
\verse That which was torn of beasts I brought not unto thee; I bare the loss of it; of my hand didst thou require it, whether stolen by day, or stolen by night.
\verse Thus I was; in the day the drought consumed me, and the frost by night; and my sleep departed from mine eyes.
\verse Thus have I been twenty years in thy house; I served thee fourteen years for thy two daughters, and six years for thy cattle: and thou hast changed my wages ten times.
\verse Except the God of my father, the God of Abraham, and the fear of Isaac, had been with me, surely thou hadst sent me away now empty. God hath seen mine affliction and the labour of my hands, and rebuked thee yesternight.
\verse And Laban answered and said unto Jacob, These daughters are my daughters, and these children are my children, and these cattle are my cattle, and all that thou seest is mine: and what can I do this day unto these my daughters, or unto their children which they have born?
\verse Now therefore come thou, let us make a covenant, I and thou; and let it be for a witness between me and thee.
\verse And Jacob took a stone, and set it up for a pillar.
\verse And Jacob said unto his brethren, Gather stones; and they took stones, and made an heap: and they did eat there upon the heap.
\verse And Laban called it Jegarsahadutha: but Jacob called it Galeed.
\verse And Laban said, This heap is a witness between me and thee this day. Therefore was the name of it called Galeed;
\verse And Mizpah; for he said, The \LORD watch between me and thee, when we are absent one from another.
\verse If thou shalt afflict my daughters, or if thou shalt take other wives beside my daughters, no man is with us; see, God is witness betwixt me and thee.
\verse And Laban said to Jacob, Behold this heap, and behold this pillar, which I have cast betwixt me and thee;
\verse This heap be witness, and this pillar be witness, that I will not pass over this heap to thee, and that thou shalt not pass over this heap and this pillar unto me, for harm.
\verse The God of Abraham, and the God of Nahor, the God of their father, judge betwixt us. And Jacob sware by the fear of his father Isaac.
\verse Then Jacob offered sacrifice upon the mount, and called his brethren to eat bread: and they did eat bread, and tarried all night in the mount.
\verse And early in the morning Laban rose up, and kissed his sons and his daughters, and blessed them: and Laban departed, and returned unto his place.
\end{biblechapter}

\begin{biblechapter} % Genesis 32
\verseWithHeading{Jacob prepares to meet Esau} And Jacob went on his way, and the angels of God met him.
\verse And when Jacob saw them, he said, This is God's host: and he called the name of that place Mahanaim.
\verse And Jacob sent messengers before him to Esau his brother unto the land of Seir, the country of Edom.
\verse And he commanded them, saying, Thus shall ye speak unto my lord Esau; Thy servant Jacob saith thus, I have sojourned with Laban, and stayed there until now:
\verse And I have oxen, and asses, flocks, and menservants, and womenservants: and I have sent to tell my lord, that I may find grace in thy sight.
\verse And the messengers returned to Jacob, saying, We came to thy brother Esau, and also he cometh to meet thee, and four hundred men with him.
\verse Then Jacob was greatly afraid and distressed: and he divided the people that was with him, and the flocks, and herds, and the camels, into two bands;
\verse And said, If Esau come to the one company, and smite it, then the other company which is left shall escape.
\verse And Jacob said, O God of my father Abraham, and God of my father Isaac, the \LORD which saidst unto me, Return unto thy country, and to thy kindred, and I will deal well with thee:
\verse I am not worthy of the least of all the mercies, and of all the truth, which thou hast shewed unto thy servant; for with my staff I passed over this Jordan; and now I am become two bands.
\verse Deliver me, I pray thee, from the hand of my brother, from the hand of Esau: for I fear him, lest he will come and smite me, and the mother with the children.
\verse And thou saidst, I will surely do thee good, and make thy seed as the sand of the sea, which cannot be numbered for multitude.
\verse And he lodged there that same night; and took of that which came to his hand a present for Esau his brother;
\verse Two hundred she goats, and twenty he goats, two hundred ewes, and twenty rams,
\verse Thirty milch camels with their colts, forty kine, and ten bulls, twenty she asses, and ten foals.
\verse And he delivered them into the hand of his servants, every drove by themselves; and said unto his servants, Pass over before me, and put a space betwixt drove and drove.
\verse And he commanded the foremost, saying, When Esau my brother meeteth thee, and asketh thee, saying, Whose art thou? and whither goest thou? and whose are these before thee?
\verse Then thou shalt say, They be thy servant Jacob's; it is a present sent unto my lord Esau: and, behold, also he is behind us.
\verse And so commanded he the second, and the third, and all that followed the droves, saying, On this manner shall ye speak unto Esau, when ye find him.
\verse And say ye moreover, Behold, thy servant Jacob is behind us. For he said, I will appease him with the present that goeth before me, and afterward I will see his face; peradventure he will accept of me.
\verse So went the present over before him: and himself lodged that night in the company.
\verseWithHeading{Jacob wrestles with God} And he rose up that night, and took his two wives, and his two womenservants, and his eleven sons, and passed over the ford Jabbok.
\verse And he took them, and sent them over the brook, and sent over that he had.
\verse And Jacob was left alone; and there wrestled a man with him until the breaking of the day.
\verse And when he saw that he prevailed not against him, he touched the hollow of his thigh; and the hollow of Jacob's thigh was out of joint, as he wrestled with him.
\verse And he said, Let me go, for the day breaketh. And he said, I will not let thee go, except thou bless me.
\verse And he said unto him, What is thy name? And he said, Jacob.
\verse And he said, Thy name shall be called no more Jacob, but Israel: for as a prince hast thou power with God and with men, and hast prevailed.
\verse And Jacob asked him, and said, Tell me, I pray thee, thy name. And he said, Wherefore is it that thou dost ask after my name? And he blessed him there.
\verse And Jacob called the name of the place Peniel: for I have seen God face to face, and my life is preserved.
\verse And as he passed over Penuel the sun rose upon him, and he halted upon his thigh.
\verse Therefore the children of Israel eat not of the sinew which shrank, which is upon the hollow of the thigh, unto this day: because he touched the hollow of Jacob's thigh in the sinew that shrank.
\end{biblechapter}

\begin{biblechapter} % Genesis 33
\verseWithHeading{Jacob meets Esau} And Jacob lifted up his eyes, and looked, and, behold, Esau came, and with him four hundred men. And he divided the children unto Leah, and unto Rachel, and unto the two handmaids.
\verse And he put the handmaids and their children foremost, and Leah and her children after, and Rachel and Joseph hindermost.
\verse And he passed over before them, and bowed himself to the ground seven times, until he came near to his brother.
\verse And Esau ran to meet him, and embraced him, and fell on his neck, and kissed him: and they wept.
\verse And he lifted up his eyes, and saw the women and the children; and said, Who are those with thee? And he said, The children which God hath graciously given thy servant.
\verse Then the handmaidens came near, they and their children, and they bowed themselves.
\verse And Leah also with her children came near, and bowed themselves: and after came Joseph near and Rachel, and they bowed themselves.
\verse And he said, What meanest thou by all this drove which I met? And he said, These are to find grace in the sight of my lord.
\verse And Esau said, I have enough, my brother; keep that thou hast unto thyself.
\verse And Jacob said, Nay, I pray thee, if now I have found grace in thy sight, then receive my present at my hand: for therefore I have seen thy face, as though I had seen the face of God, and thou wast pleased with me.
\verse Take, I pray thee, my blessing that is brought to thee; because God hath dealt graciously with me, and because I have enough. And he urged him, and he took it.
\verse And he said, Let us take our journey, and let us go, and I will go before thee.
\verse And he said unto him, My lord knoweth that the children are tender, and the flocks and herds with young are with me: and if men should overdrive them one day, all the flock will die.
\verse Let my lord, I pray thee, pass over before his servant: and I will lead on softly, according as the cattle that goeth before me and the children be able to endure, until I come unto my lord unto Seir.
\verse And Esau said, Let me now leave with thee some of the folk that are with me. And he said, What needeth it? let me find grace in the sight of my lord.
\verse So Esau returned that day on his way unto Seir.
\verse And Jacob journeyed to Succoth, and built him an house, and made booths for his cattle: therefore the name of the place is called Succoth.
\verse And Jacob came to Shalem, a city of Shechem, which is in the land of Canaan, when he came from Padanaram; and pitched his tent before the city.
\verse And he bought a parcel of a field, where he had spread his tent, at the hand of the children of Hamor, Shechem's father, for an hundred pieces of money.
\verse And he erected there an altar, and called it Elelohe-Israel.
\end{biblechapter}

\flushcolsend\columnbreak % layout hack

\begin{biblechapter} % Genesis 34
\verseWithHeading{Dinah and the Shechemites} And Dinah the daughter of Leah, which she bare unto Jacob, went out to see the daughters of the land.
\verse And when Shechem the son of Hamor the Hivite, prince of the country, saw her, he took her, and lay with her, and defiled her.
\verse And his soul clave unto Dinah the daughter of Jacob, and he loved the damsel, and spake kindly unto the damsel.
\verse And Shechem spake unto his father Hamor, saying, Get me this damsel to wife.
\verse And Jacob heard that he had defiled Dinah his daughter: now his sons were with his cattle in the field: and Jacob held his peace until they were come.
\verse And Hamor the father of Shechem went out unto Jacob to commune with him.
\verse And the sons of Jacob came out of the field when they heard it: and the men were grieved, and they were very wroth, because he had wrought folly in Israel in lying with Jacob's daughter; which thing ought not to be done.
\verse And Hamor communed with them, saying, The soul of my son Shechem longeth for your daughter: I pray you give her him to wife.
\verse And make ye marriages with us, and give your daughters unto us, and take our daughters unto you.
\verse And ye shall dwell with us: and the land shall be before you; dwell and trade ye therein, and get you possessions therein.
\verse And Shechem said unto her father and unto her brethren, Let me find grace in your eyes, and what ye shall say unto me I will give.
\verse Ask me never so much dowry and gift, and I will give according as ye shall say unto me: but give me the damsel to wife.
\verse And the sons of Jacob answered Shechem and Hamor his father deceitfully, and said, because he had defiled Dinah their sister:
\verse And they said unto them, We cannot do this thing, to give our sister to one that is uncircumcised; for that were a reproach unto us:
\verse But in this will we consent unto you: If ye will be as we be, that every male of you be circumcised;
\verse Then will we give our daughters unto you, and we will take your daughters to us, and we will dwell with you, and we will become one people.
\verse But if ye will not hearken unto us, to be circumcised; then will we take our daughter, and we will be gone.
\verse And their words pleased Hamor, and Shechem Hamor's son.
\verse And the young man deferred not to do the thing, because he had delight in Jacob's daughter: and he was more honourable than all the house of his father.
\verse And Hamor and Shechem his son came unto the gate of their city, and communed with the men of their city, saying,
\verse These men are peaceable with us; therefore let them dwell in the land, and trade therein; for the land, behold, it is large enough for them; let us take their daughters to us for wives, and let us give them our daughters.
\verse Only herein will the men consent unto us for to dwell with us, to be one people, if every male among us be circumcised, as they are circumcised.
\verse Shall not their cattle and their substance and every beast of theirs be ours? only let us consent unto them, and they will dwell with us.
\verse And unto Hamor and unto Shechem his son hearkened all that went out of the gate of his city; and every male was circumcised, all that went out of the gate of his city.
\verse And it came to pass on the third day, when they were sore, that two of the sons of Jacob, Simeon and Levi, Dinah's brethren, took each man his sword, and came upon the city boldly, and slew all the males.
\verse And they slew Hamor and Shechem his son with the edge of the sword, and took Dinah out of Shechem's house, and went out.
\verse The sons of Jacob came upon the slain, and spoiled the city, because they had defiled their sister.
\verse They took their sheep, and their oxen, and their asses, and that which was in the city, and that which was in the field,
\verse And all their wealth, and all their little ones, and their wives took they captive, and spoiled even all that was in the house.
\verse And Jacob said to Simeon and Levi, Ye have troubled me to make me to stink among the inhabitants of the land, among the Canaanites and the Perizzites: and I being few in number, they shall gather themselves together against me, and slay me; and I shall be destroyed, I and my house.
\verse And they said, Should he deal with our sister as with an harlot?
\end{biblechapter}

\begin{biblechapter} % Genesis 35
\verseWithHeading{Jacob returns to Bethel} And God said unto Jacob, Arise, go up to Bethel, and dwell there: and make there an altar unto God, that appeared unto thee when thou fleddest from the face of Esau thy brother.
\verse Then Jacob said unto his household, and to all that were with him, Put away the strange gods that are among you, and be clean, and change your garments:
\verse And let us arise, and go up to Bethel; and I will make there an altar unto God, who answered me in the day of my distress, and was with me in the way which I went.
\verse And they gave unto Jacob all the strange gods which were in their hand, and all their earrings which were in their ears; and Jacob hid them under the oak which was by Shechem.
\verse And they journeyed: and the terror of God was upon the cities that were round about them, and they did not pursue after the sons of Jacob.
\verse So Jacob came to Luz, which is in the land of Canaan, that is, Bethel, he and all the people that were with him.
\verse And he built there an altar, and called the place Elbethel: because there God appeared unto him, when he fled from the face of his brother.
\verse But Deborah Rebekah's nurse died, and she was buried beneath Bethel under an oak: and the name of it was called Allonbachuth.
\verse And God appeared unto Jacob again, when he came out of Padanaram, and blessed him.
\verse And God said unto him, Thy name is Jacob: thy name shall not be called any more Jacob, but Israel shall be thy name: and he called his name Israel.
\verse And God said unto him, I am God Almighty: be fruitful and multiply; a nation and a company of nations shall be of thee, and kings shall come out of thy loins;
\verse And the land which I gave Abraham and Isaac, to thee I will give it, and to thy seed after thee will I give the land.
\verse And God went up from him in the place where he talked with him.
\verse And Jacob set up a pillar in the place where he talked with him, even a pillar of stone: and he poured a drink offering thereon, and he poured oil thereon.
\verse And Jacob called the name of the place where God spake with him, Bethel.
\verseWithHeading{The deaths of Rachel and Isaac} And they journeyed from Bethel; and there was but a little way to come to Ephrath: and Rachel travailed, and she had hard labour.
\verse And it came to pass, when she was in hard labour, that the midwife said unto her, Fear not; thou shalt have this son also.
\verse And it came to pass, as her soul was in departing, (for she died) that she called his name Benoni: but his father called him Benjamin.
\verse And Rachel died, and was buried in the way to Ephrath, which is Bethlehem.
\verse And Jacob set a pillar upon her grave: that is the pillar of Rachel's grave unto this day.
\verse And Israel journeyed, and spread his tent beyond the tower of Edar.
\verse And it came to pass, when Israel dwelt in that land, that Reuben went and lay with Bilhah his father's concubine: and Israel heard it. Now the sons of Jacob were twelve:
\verse The sons of Leah; Reuben, Jacob's firstborn, and Simeon, and Levi, and Judah, and Issachar, and Zebulun:
\verse The sons of Rachel; Joseph, and Benjamin:
\verse And the sons of Bilhah, Rachel's handmaid; Dan, and Naphtali:
\verse And the sons of Zilpah, Leah's handmaid; Gad, and Asher: these are the sons of Jacob, which were born to him in Padanaram.
\verse And Jacob came unto Isaac his father unto Mamre, unto the city of Arbah, which is Hebron, where Abraham and Isaac sojourned.
\verse And the days of Isaac were an hundred and fourscore years.
\verse And Isaac gave up the ghost, and died, and was gathered unto his people, being old and full of days: and his sons Esau and Jacob buried him.
\end{biblechapter}

\begin{biblechapter} % Genesis 36
\verseWithHeading{Esau's descendents} Now these are the generations of Esau, who is Edom.
\verse Esau took his wives of the daughters of Canaan; Adah the daughter of Elon the Hittite, and Aholibamah the daughter of Anah the daughter of Zibeon the Hivite;
\verse And Bashemath Ishmael's daughter, sister of Nebajoth.
\verse And Adah bare to Esau Eliphaz; and Bashemath bare Reuel;
\verse And Aholibamah bare Jeush, and Jaalam, and Korah: these are the sons of Esau, which were born unto him in the land of Canaan.
\verse And Esau took his wives, and his sons, and his daughters, and all the persons of his house, and his cattle, and all his beasts, and all his substance, which he had got in the land of Canaan; and went into the country from the face of his brother Jacob.
\verse For their riches were more than that they might dwell together; and the land wherein they were strangers could not bear them because of their cattle.
\verse Thus dwelt Esau in mount Seir: Esau is Edom.
\verse And these are the generations of Esau the father of the Edomites in mount Seir:
\verse These are the names of Esau's sons; Eliphaz the son of Adah the wife of Esau, Reuel the son of Bashemath the wife of Esau.
\verse And the sons of Eliphaz were Teman, Omar, Zepho, and Gatam, and Kenaz.
\verse And Timna was concubine to Eliphaz Esau's son; and she bare to Eliphaz Amalek: these were the sons of Adah Esau's wife.
\verse And these are the sons of Reuel; Nahath, and Zerah, Shammah, and Mizzah: these were the sons of Bashemath Esau's wife.
\verse And these were the sons of Aholibamah, the daughter of Anah the daughter of Zibeon, Esau's wife: and she bare to Esau Jeush, and Jaalam, and Korah.
\verse These were dukes of the sons of Esau: the sons of Eliphaz the firstborn son of Esau; duke Teman, duke Omar, duke Zepho, duke Kenaz,
\verse Duke Korah, duke Gatam, and duke Amalek: these are the dukes that came of Eliphaz in the land of Edom; these were the sons of Adah.
\verse And these are the sons of Reuel Esau's son; duke Nahath, duke Zerah, duke Shammah, duke Mizzah: these are the dukes that came of Reuel in the land of Edom; these are the sons of Bashemath Esau's wife.
\verse And these are the sons of Aholibamah Esau's wife; duke Jeush, duke Jaalam, duke Korah: these were the dukes that came of Aholibamah the daughter of Anah, Esau's wife.
\verse These are the sons of Esau, who is Edom, and these are their dukes.
\verse These are the sons of Seir the Horite, who inhabited the land; Lotan, and Shobal, and Zibeon, and Anah,
\verse And Dishon, and Ezer, and Dishan: these are the dukes of the Horites, the children of Seir in the land of Edom.
\verse And the children of Lotan were Hori and Hemam; and Lotan's sister was Timna.
\verse And the children of Shobal were these; Alvan, and Manahath, and Ebal, Shepho, and Onam.
\verse And these are the children of Zibeon; both Ajah, and Anah: this was that Anah that found the mules in the wilderness, as he fed the asses of Zibeon his father.
\verse And the children of Anah were these; Dishon, and Aholibamah the daughter of Anah.
\verse And these are the children of Dishon; Hemdan, and Eshban, and Ithran, and Cheran.
\verse The children of Ezer are these; Bilhan, and Zaavan, and Akan.
\verse The children of Dishan are these; Uz, and Aran.
\verse These are the dukes that came of the Horites; duke Lotan, duke Shobal, duke Zibeon, duke Anah,
\verse Duke Dishon, duke Ezer, duke Dishan: these are the dukes that came of Hori, among their dukes in the land of Seir.
\verseWithHeading{The rulers of Edom} And these are the kings that reigned in the land of Edom, before there reigned any king over the children of Israel.
\verse And Bela the son of Beor reigned in Edom: and the name of his city was Dinhabah.
\verse And Bela died, and Jobab the son of Zerah of Bozrah reigned in his stead.
\verse And Jobab died, and Husham of the land of Temani reigned in his stead.
\verse And Husham died, and Hadad the son of Bedad, who smote Midian in the field of Moab, reigned in his stead: and the name of his city was Avith.
\verse And Hadad died, and Samlah of Masrekah reigned in his stead.
\verse And Samlah died, and Saul of Rehoboth by the river reigned in his stead.
\verse And Saul died, and Baalhanan the son of Achbor reigned in his stead.
\verse And Baalhanan the son of Achbor died, and Hadar reigned in his stead: and the name of his city was Pau; and his wife's name was Mehetabel, the daughter of Matred, the daughter of Mezahab.
\verse And these are the names of the dukes that came of Esau, according to their families, after their places, by their names; duke Timnah, duke Alvah, duke Jetheth,
\verse Duke Aholibamah, duke Elah, duke Pinon,
\verse Duke Kenaz, duke Teman, duke Mibzar,
\verse Duke Magdiel, duke Iram: these be the dukes of Edom, according to their habitations in the land of their possession: he is Esau the father of the Edomites.
\end{biblechapter}

\begin{biblechapter} % Genesis 37
\verseWithHeading{Joseph's dreams} And Jacob dwelt in the land wherein his father was a stranger, in the land of Canaan.
\verse These are the generations of Jacob. Joseph, being seventeen years old, was feeding the flock with his brethren; and the lad was with the sons of Bilhah, and with the sons of Zilpah, his father's wives: and Joseph brought unto his father their evil report.
\verse Now Israel loved Joseph more than all his children, because he was the son of his old age: and he made him a coat of many colours.
\verse And when his brethren saw that their father loved him more than all his brethren, they hated him, and could not speak peaceably unto him.
\verse And Joseph dreamed a dream, and he told it his brethren: and they hated him yet the more.
\verse And he said unto them, Hear, I pray you, this dream which I have dreamed:
\verse For, behold, we were binding sheaves in the field, and, lo, my sheaf arose, and also stood upright; and, behold, your sheaves stood round about, and made obeisance to my sheaf.
\verse And his brethren said to him, Shalt thou indeed reign over us? or shalt thou indeed have dominion over us? And they hated him yet the more for his dreams, and for his words.
\verse And he dreamed yet another dream, and told it his brethren, and said, Behold, I have dreamed a dream more; and, behold, the sun and the moon and the eleven stars made obeisance to me.
\verse And he told it to his father, and to his brethren: and his father rebuked him, and said unto him, What is this dream that thou hast dreamed? Shall I and thy mother and thy brethren indeed come to bow down ourselves to thee to the earth?
\verse And his brethren envied him; but his father observed the saying.
\verseWithHeading{Joseph sold by his brothers} And his brethren went to feed their father's flock in Shechem.
\verse And Israel said unto Joseph, Do not thy brethren feed the flock in Shechem? come, and I will send thee unto them. And he said to him, Here am I.
\verse And he said to him, Go, I pray thee, see whether it be well with thy brethren, and well with the flocks; and bring me word again. So he sent him out of the vale of Hebron, and he came to Shechem.
\verse And a certain man found him, and, behold, he was wandering in the field: and the man asked him, saying, What seekest thou?
\verse And he said, I seek my brethren: tell me, I pray thee, where they feed their flocks.
\verse And the man said, They are departed hence; for I heard them say, Let us go to Dothan. And Joseph went after his brethren, and found them in Dothan.
\verse And when they saw him afar off, even before he came near unto them, they conspired against him to slay him.
\verse And they said one to another, Behold, this dreamer cometh.
\verse Come now therefore, and let us slay him, and cast him into some pit, and we will say, Some evil beast hath devoured him: and we shall see what will become of his dreams.
\verse And Reuben heard it, and he delivered him out of their hands; and said, Let us not kill him.
\verse And Reuben said unto them, Shed no blood, but cast him into this pit that is in the wilderness, and lay no hand upon him; that he might rid him out of their hands, to deliver him to his father again.
\verse And it came to pass, when Joseph was come unto his brethren, that they stript Joseph out of his coat, his coat of many colours that was on him;
\verse And they took him, and cast him into a pit: and the pit was empty, there was no water in it.
\verse And they sat down to eat bread: and they lifted up their eyes and looked, and, behold, a company of Ishmeelites came from Gilead with their camels bearing spicery and balm and myrrh, going to carry it down to Egypt.
\verse And Judah said unto his brethren, What profit is it if we slay our brother, and conceal his blood?
\verse Come, and let us sell him to the Ishmeelites, and let not our hand be upon him; for he is our brother and our flesh. And his brethren were content.
\verse Then there passed by Midianites merchantmen; and they drew and lifted up Joseph out of the pit, and sold Joseph to the Ishmeelites for twenty pieces of silver: and they brought Joseph into Egypt.
\verse And Reuben returned unto the pit; and, behold, Joseph was not in the pit; and he rent his clothes.
\verse And he returned unto his brethren, and said, The child is not; and I, whither shall I go?
\verse And they took Joseph's coat, and killed a kid of the goats, and dipped the coat in the blood;
\verse And they sent the coat of many colours, and they brought it to their father; and said, This have we found: know now whether it be thy son's coat or no.
\verse And he knew it, and said, It is my son's coat; an evil beast hath devoured him; Joseph is without doubt rent in pieces.
\verse And Jacob rent his clothes, and put sackcloth upon his loins, and mourned for his son many days.
\verse And all his sons and all his daughters rose up to comfort him; but he refused to be comforted; and he said, For I will go down into the grave unto my son mourning. Thus his father wept for him.
\verse And the Midianites sold him into Egypt unto Potiphar, an officer of Pharaoh's, and captain of the guard.
\end{biblechapter}

\begin{biblechapter} % Genesis 38
\verseWithHeading{Joseph and Tamar} And it came to pass at that time, that Judah went down from his brethren, and turned in to a certain Adullamite, whose name was Hirah.
\verse And Judah saw there a daughter of a certain Canaanite, whose name was Shuah; and he took her, and went in unto her.
\verse And she conceived, and bare a son; and he called his name Er.
\verse And she conceived again, and bare a son; and she called his name Onan.
\verse And she yet again conceived, and bare a son; and called his name Shelah: and he was at Chezib, when she bare him.
\verse And Judah took a wife for Er his firstborn, whose name was Tamar.
\verse And Er, Judah's firstborn, was wicked in the sight of the \LORD; and the \LORD slew him.
\verse And Judah said unto Onan, Go in unto thy brother's wife, and marry her, and raise up seed to thy brother.
\verse And Onan knew that the seed should not be his; and it came to pass, when he went in unto his brother's wife, that he spilled it on the ground, lest that he should give seed to his brother.
\verse And the thing which he did displeased the \LORD: wherefore he slew him also.
\verse Then said Judah to Tamar his daughter in law, Remain a widow at thy father's house, till Shelah my son be grown: for he said, Lest peradventure he die also, as his brethren did. And Tamar went and dwelt in her father's house.
\verse And in process of time the daughter of Shuah Judah's wife died; and Judah was comforted, and went up unto his sheepshearers to Timnath, he and his friend Hirah the Adullamite.
\verse And it was told Tamar, saying, Behold thy father in law goeth up to Timnath to shear his sheep.
\verse And she put her widow's garments off from her, and covered her with a vail, and wrapped herself, and sat in an open place, which is by the way to Timnath; for she saw that Shelah was grown, and she was not given unto him to wife.
\verse When Judah saw her, he thought her to be an harlot; because she had covered her face.
\verse And he turned unto her by the way, and said, Go to, I pray thee, let me come in unto thee; (for he knew not that she was his daughter in law.) And she said, What wilt thou give me, that thou mayest come in unto me?
\verse And he said, I will send thee a kid from the flock. And she said, Wilt thou give me a pledge, till thou send it?
\verse And he said, What pledge shall I give thee? And she said, Thy signet, and thy bracelets, and thy staff that is in thine hand. And he gave it her, and came in unto her, and she conceived by him.
\verse And she arose, and went away, and laid by her vail from her, and put on the garments of her widowhood.
\verse And Judah sent the kid by the hand of his friend the Adullamite, to receive his pledge from the woman's hand: but he found her not.
\verse Then he asked the men of that place, saying, Where is the harlot, that was openly by the way side? And they said, There was no harlot in this place.
\verse And he returned to Judah, and said, I cannot find her; and also the men of the place said, that there was no harlot in this place.
\verse And Judah said, Let her take it to her, lest we be shamed: behold, I sent this kid, and thou hast not found her.
\verse And it came to pass about three months after, that it was told Judah, saying, Tamar thy daughter in law hath played the harlot; and also, behold, she is with child by whoredom. And Judah said, Bring her forth, and let her be burnt.
\verse When she was brought forth, she sent to her father in law, saying, By the man, whose these are, am I with child: and she said, Discern, I pray thee, whose are these, the signet, and bracelets, and staff.
\verse And Judah acknowledged them, and said, She hath been more righteous than I; because that I gave her not to Shelah my son. And he knew her again no more.
\verse And it came to pass in the time of her travail, that, behold, twins were in her womb.
\verse And it came to pass, when she travailed, that the one put out his hand: and the midwife took and bound upon his hand a scarlet thread, saying, This came out first.
\verse And it came to pass, as he drew back his hand, that, behold, his brother came out: and she said, How hast thou broken forth? this breach be upon thee: therefore his name was called Pharez.
\verse And afterward came out his brother, that had the scarlet thread upon his hand: and his name was called Zarah.
\end{biblechapter}

\begin{biblechapter} % Genesis 39
\verseWithHeading{Joseph and Potiphar's wife} And Joseph was brought down to Egypt; and Potiphar, an officer of Pharaoh, captain of the guard, an Egyptian, bought him of the hands of the Ishmeelites, which had brought him down thither.
\verse And the \LORD was with Joseph, and he was a prosperous man; and he was in the house of his master the Egyptian.
\verse And his master saw that the \LORD was with him, and that the \LORD made all that he did to prosper in his hand.
\verse And Joseph found grace in his sight, and he served him: and he made him overseer over his house, and all that he had he put into his hand.
\verse And it came to pass from the time that he had made him overseer in his house, and over all that he had, that the \LORD blessed the Egyptian's house for Joseph's sake; and the blessing of the \LORD was upon all that he had in the house, and in the field.
\verse And he left all that he had in Joseph's hand; and he knew not ought he had, save the bread which he did eat. And Joseph was a goodly person, and well favoured.
\verse And it came to pass after these things, that his master's wife cast her eyes upon Joseph; and she said, Lie with me.
\verse But he refused, and said unto his master's wife, Behold, my master wotteth not what is with me in the house, and he hath committed all that he hath to my hand;
\verse There is none greater in this house than I; neither hath he kept back any thing from me but thee, because thou art his wife: how then can I do this great wickedness, and sin against God?
\verse And it came to pass, as she spake to Joseph day by day, that he hearkened not unto her, to lie by her, or to be with her.
\verse And it came to pass about this time, that Joseph went into the house to do his business; and there was none of the men of the house there within.
\verse And she caught him by his garment, saying, Lie with me: and he left his garment in her hand, and fled, and got him out.
\verse And it came to pass, when she saw that he had left his garment in her hand, and was fled forth,
\verse That she called unto the men of her house, and spake unto them, saying, See, he hath brought in an Hebrew unto us to mock us; he came in unto me to lie with me, and I cried with a loud voice:
\verse And it came to pass, when he heard that I lifted up my voice and cried, that he left his garment with me, and fled, and got him out.
\verse And she laid up his garment by her, until his lord came home.
\verse And she spake unto him according to these words, saying, The Hebrew servant, which thou hast brought unto us, came in unto me to mock me:
\verse And it came to pass, as I lifted up my voice and cried, that he left his garment with me, and fled out.
\verse And it came to pass, when his master heard the words of his wife, which she spake unto him, saying, After this manner did thy servant to me; that his wrath was kindled.
\verse And Joseph's master took him, and put him into the prison, a place where the king's prisoners were bound: and he was there in the prison.
\verse But the \LORD was with Joseph, and shewed him mercy, and gave him favour in the sight of the keeper of the prison.
\verse And the keeper of the prison committed to Joseph's hand all the prisoners that were in the prison; and whatsoever they did there, he was the doer of it.
\verse The keeper of the prison looked not to any thing that was under his hand; because the \LORD was with him, and that which he did, the \LORD made it to prosper.
\end{biblechapter}

\begin{biblechapter} % Genesis 40
\verseWithHeading{The cupbearer and the \newline baker} And it came to pass after these things, that the butler of the king of Egypt and his baker had offended their lord the king of Egypt.
\verse And Pharaoh was wroth against two of his officers, against the chief of the butlers, and against the chief of the bakers.
\verse And he put them in ward in the house of the captain of the guard, into the prison, the place where Joseph was bound.
\verse And the captain of the guard charged Joseph with them, and he served them: and they continued a season in ward.
\verse And they dreamed a dream both of them, each man his dream in one night, each man according to the interpretation of his dream, the butler and the baker of the king of Egypt, which were bound in the prison.
\verse And Joseph came in unto them in the morning, and looked upon them, and, behold, they were sad.
\verse And he asked Pharaoh's officers that were with him in the ward of his lord's house, saying, Wherefore look ye so sadly to day?
\verse And they said unto him, We have dreamed a dream, and there is no interpreter of it. And Joseph said unto them, Do not interpretations belong to God? tell me them, I pray you.
\verse And the chief butler told his dream to Joseph, and said to him, In my dream, behold, a vine was before me;
\verse And in the vine were three branches: and it was as though it budded, and her blossoms shot forth; and the clusters thereof brought forth ripe grapes:
\verse And Pharaoh's cup was in my hand: and I took the grapes, and pressed them into Pharaoh's cup, and I gave the cup into Pharaoh's hand.
\verse And Joseph said unto him, This is the interpretation of it: The three branches are three days:
\verse Yet within three days shall Pharaoh lift up thine head, and restore thee unto thy place: and thou shalt deliver Pharaoh's cup into his hand, after the former manner when thou wast his butler.
\verse But think on me when it shall be well with thee, and shew kindness, I pray thee, unto me, and make mention of me unto Pharaoh, and bring me out of this house:
\verse For indeed I was stolen away out of the land of the Hebrews: and here also have I done nothing that they should put me into the dungeon.
\verse When the chief baker saw that the interpretation was good, he said unto Joseph, I also was in my dream, and, behold, I had three white baskets on my head:
\verse And in the uppermost basket there was of all manner of bakemeats for Pharaoh; and the birds did eat them out of the basket upon my head.
\verse And Joseph answered and said, This is the interpretation thereof: The three baskets are three days:
\verse Yet within three days shall Pharaoh lift up thy head from off thee, and shall hang thee on a tree; and the birds shall eat thy flesh from off thee.
\verse And it came to pass the third day, which was Pharaoh's birthday, that he made a feast unto all his servants: and he lifted up the head of the chief butler and of the chief baker among his servants.
\verse And he restored the chief butler unto his butlership again; and he gave the cup into Pharaoh's hand:
\verse But he hanged the chief baker: as Joseph had interpreted to them.
\verse Yet did not the chief butler remember Joseph, but forgat him.
\end{biblechapter}

\begin{biblechapter} % Genesis 41
\verseWithHeading{Pharaoh's dreams} And it came to pass at the end of two full years, that Pharaoh dreamed: and, behold, he stood by the river.
\verse And, behold, there came up out of the river seven well favoured kine and fatfleshed; and they fed in a meadow.
\verse And, behold, seven other kine came up after them out of the river, ill favoured and leanfleshed; and stood by the other kine upon the brink of the river.
\verse And the ill favoured and leanfleshed kine did eat up the seven well favoured and fat kine. So Pharaoh awoke.
\verse And he slept and dreamed the second time: and, behold, seven ears of corn came up upon one stalk, rank and good.
\verse And, behold, seven thin ears and blasted with the east wind sprung up after them.
\verse And the seven thin ears devoured the seven rank and full ears. And Pharaoh awoke, and, behold, it was a dream.
\verse And it came to pass in the morning that his spirit was troubled; and he sent and called for all the magicians of Egypt, and all the wise men thereof: and Pharaoh told them his dream; but there was none that could interpret them unto Pharaoh.
\verse Then spake the chief butler unto Pharaoh, saying, I do remember my faults this day:
\verse Pharaoh was wroth with his servants, and put me in ward in the captain of the guard's house, both me and the chief baker:
\verse And we dreamed a dream in one night, I and he; we dreamed each man according to the interpretation of his dream.
\verse And there was there with us a young man, an Hebrew, servant to the captain of the guard; and we told him, and he interpreted to us our dreams; to each man according to his dream he did interpret.
\verse And it came to pass, as he interpreted to us, so it was; me he restored unto mine office, and him he hanged.
\verse Then Pharaoh sent and called Joseph, and they brought him hastily out of the dungeon: and he shaved himself, and changed his raiment, and came in unto Pharaoh.
\verse And Pharaoh said unto Joseph, I have dreamed a dream, and there is none that can interpret it: and I have heard say of thee, that thou canst understand a dream to interpret it.
\verse And Joseph answered Pharaoh, saying, It is not in me: God shall give Pharaoh an answer of peace.
\verse And Pharaoh said unto Joseph, In my dream, behold, I stood upon the bank of the river:
\verse And, behold, there came up out of the river seven kine, fatfleshed and well favoured; and they fed in a meadow:
\verse And, behold, seven other kine came up after them, poor and very ill favoured and leanfleshed, such as I never saw in all the land of Egypt for badness:
\verse And the lean and the ill favoured kine did eat up the first seven fat kine:
\verse And when they had eaten them up, it could not be known that they had eaten them; but they were still ill favoured, as at the beginning. So I awoke.
\verse And I saw in my dream, and, behold, seven ears came up in one stalk, full and good:
\verse And, behold, seven ears, withered, thin, and blasted with the east wind, sprung up after them:
\verse And the thin ears devoured the seven good ears: and I told this unto the magicians; but there was none that could declare it to me.
\verse And Joseph said unto Pharaoh, The dream of Pharaoh is one: God hath shewed Pharaoh what he is about to do.
\verse The seven good kine are seven years; and the seven good ears are seven years: the dream is one.
\verse And the seven thin and ill favoured kine that came up after them are seven years; and the seven empty ears blasted with the east wind shall be seven years of famine.
\verse This is the thing which I have spoken unto Pharaoh: What God is about to do he sheweth unto Pharaoh.
\verse Behold, there come seven years of great plenty throughout all the land of Egypt:
\verse And there shall arise after them seven years of famine; and all the plenty shall be forgotten in the land of Egypt; and the famine shall consume the land;
\verse And the plenty shall not be known in the land by reason of that famine following; for it shall be very grievous.
\verse And for that the dream was doubled unto Pharaoh twice; it is because the thing is established by God, and God will shortly bring it to pass.
\verse Now therefore let Pharaoh look out a man discreet and wise, and set him over the land of Egypt.
\verse Let Pharaoh do this, and let him appoint officers over the land, and take up the fifth part of the land of Egypt in the seven plenteous years.
\verse And let them gather all the food of those good years that come, and lay up corn under the hand of Pharaoh, and let them keep food in the cities.
\verse And that food shall be for store to the land against the seven years of famine, which shall be in the land of Egypt; that the land perish not through the famine.
\verse And the thing was good in the eyes of Pharaoh, and in the eyes of all his servants.
\verse And Pharaoh said unto his servants, Can we find such a one as this is, a man in whom the Spirit of God is?
\verse And Pharaoh said unto Joseph, Forasmuch as God hath shewed thee all this, there is none so discreet and wise as thou art:
\verse Thou shalt be over my house, and according unto thy word shall all my people be ruled: only in the throne will I be greater than thou.
\verseWithHeading{Joseph in charge of Egypt} And Pharaoh said unto Joseph, See, I have set thee over all the land of Egypt.
\verse And Pharaoh took off his ring from his hand, and put it upon Joseph's hand, and arrayed him in vestures of fine linen, and put a gold chain about his neck;
\verse And he made him to ride in the second chariot which he had; and they cried before him, Bow the knee: and he made him ruler over all the land of Egypt.
\verse And Pharaoh said unto Joseph, I am Pharaoh, and without thee shall no man lift up his hand or foot in all the land of Egypt.
\verse And Pharaoh called Joseph's name Zaphnathpaaneah; and he gave him to wife Asenath the daughter of Potipherah priest of On. And Joseph went out over all the land of Egypt.
\verse And Joseph was thirty years old when he stood before Pharaoh king of Egypt. And Joseph went out from the presence of Pharaoh, and went throughout all the land of Egypt.
\verse And in the seven plenteous years the earth brought forth by handfuls.
\verse And he gathered up all the food of the seven years, which were in the land of Egypt, and laid up the food in the cities: the food of the field, which was round about every city, laid he up in the same.
\verse And Joseph gathered corn as the sand of the sea, very much, until he left numbering; for it was without number.
\verse And unto Joseph were born two sons before the years of famine came, which Asenath the daughter of Potipherah priest of On bare unto him.
\verse And Joseph called the name of the firstborn Manasseh: For God, said he, hath made me forget all my toil, and all my father's house.
\verse And the name of the second called he Ephraim: For God hath caused me to be fruitful in the land of my affliction.
\verse And the seven years of plenteousness, that was in the land of Egypt, were ended.
\verse And the seven years of dearth began to come, according as Joseph had said: and the dearth was in all lands; but in all the land of Egypt there was bread.
\verse And when all the land of Egypt was famished, the people cried to Pharaoh for bread: and Pharaoh said unto all the Egyptians, Go unto Joseph; what he saith to you, do.
\verse And the famine was over all the face of the earth: and Joseph opened all the storehouses, and sold unto the Egyptians; and the famine waxed sore in the land of Egypt.
\verse And all countries came into Egypt to Joseph for to buy corn; because that the famine was so sore in all lands.
\end{biblechapter}

\begin{biblechapter} % Genesis 42
\verseWithHeading{Joseph's brothers go to \newline Egypt} Now when Jacob saw that there was corn in Egypt, Jacob said unto his sons, Why do ye look one upon another?
\verse And he said, Behold, I have heard that there is corn in Egypt: get you down thither, and buy for us from thence; that we may live, and not die.
\verse And Joseph's ten brethren went down to buy corn in Egypt.
\verse But Benjamin, Joseph's brother, Jacob sent not with his brethren; for he said, Lest peradventure mischief befall him.
\verse And the sons of Israel came to buy corn among those that came: for the famine was in the land of Canaan.
\verse And Joseph was the governor over the land, and he it was that sold to all the people of the land: and Joseph's brethren came, and bowed down themselves before him with their faces to the earth.
\verse And Joseph saw his brethren, and he knew them, but made himself strange unto them, and spake roughly unto them; and he said unto them, Whence come ye? And they said, From the land of Canaan to buy food.
\verse And Joseph knew his brethren, but they knew not him.
\verse And Joseph remembered the dreams which he dreamed of them, and said unto them, Ye are spies; to see the nakedness of the land ye are come.
\verse And they said unto him, Nay, my lord, but to buy food are thy servants come.
\verse We are all one man's sons; we are true men, thy servants are no spies.
\verse And he said unto them, Nay, but to see the nakedness of the land ye are come.
\verse And they said, Thy servants are twelve brethren, the sons of one man in the land of Canaan; and, behold, the youngest is this day with our father, and one is not.
\verse And Joseph said unto them, That is it that I spake unto you, saying, Ye are spies:
\verse Hereby ye shall be proved: By the life of Pharaoh ye shall not go forth hence, except your youngest brother come hither.
\verse Send one of you, and let him fetch your brother, and ye shall be kept in prison, that your words may be proved, whether there be any truth in you: or else by the life of Pharaoh surely ye are spies.
\verse And he put them all together into ward three days.
\verse And Joseph said unto them the third day, This do, and live; for I fear God:
\verse If ye be true men, let one of your brethren be bound in the house of your prison: go ye, carry corn for the famine of your houses:
\verse But bring your youngest brother unto me; so shall your words be verified, and ye shall not die. And they did so.
\verse And they said one to another, We are verily guilty concerning our brother, in that we saw the anguish of his soul, when he besought us, and we would not hear; therefore is this distress come upon us.
\verse And Reuben answered them, saying, Spake I not unto you, saying, Do not sin against the child; and ye would not hear? therefore, behold, also his blood is required.
\verse And they knew not that Joseph understood them; for he spake unto them by an interpreter.
\verse And he turned himself about from them, and wept; and returned to them again, and communed with them, and took from them Simeon, and bound him before their eyes.
\verse Then Joseph commanded to fill their sacks with corn, and to restore every man's money into his sack, and to give them provision for the way: and thus did he unto them.
\verse And they laded their asses with the corn, and departed thence.
\verse And as one of them opened his sack to give his ass provender in the inn, he espied his money; for, behold, it was in his sack's mouth.
\verse And he said unto his brethren, My money is restored; and, lo, it is even in my sack: and their heart failed them, and they were afraid, saying one to another, What is this that God hath done unto us?
\verse And they came unto Jacob their father unto the land of Canaan, and told him all that befell unto them; saying,
\verse The man, who is the lord of the land, spake roughly to us, and took us for spies of the country.
\verse And we said unto him, We are true men; we are no spies:
\verse We be twelve brethren, sons of our father; one is not, and the youngest is this day with our father in the land of Canaan.
\verse And the man, the lord of the country, said unto us, Hereby shall I know that ye are true men; leave one of your brethren here with me, and take food for the famine of your households, and be gone:
\verse And bring your youngest brother unto me: then shall I know that ye are no spies, but that ye are true men: so will I deliver you your brother, and ye shall traffick in the land.
\verse And it came to pass as they emptied their sacks, that, behold, every man's bundle of money was in his sack: and when both they and their father saw the bundles of money, they were afraid.
\verse And Jacob their father said unto them, Me have ye bereaved of my children: Joseph is not, and Simeon is not, and ye will take Benjamin away: all these things are against me.
\verse And Reuben spake unto his father, saying, Slay my two sons, if I bring him not to thee: deliver him into my hand, and I will bring him to thee again.
\verse And he said, My son shall not go down with you; for his brother is dead, and he is left alone: if mischief befall him by the way in the which ye go, then shall ye bring down my gray hairs with sorrow to the grave.
\end{biblechapter}

\begin{biblechapter} % Genesis 43
\verseWithHeading{The second journey to Egypt} And the famine was sore in the land.
\verse And it came to pass, when they had eaten up the corn which they had brought out of Egypt, their father said unto them, Go again, buy us a little food.
\verse And Judah spake unto him, saying, The man did solemnly protest unto us, saying, Ye shall not see my face, except your brother be with you.
\verse If thou wilt send our brother with us, we will go down and buy thee food:
\verse But if thou wilt not send him, we will not go down: for the man said unto us, Ye shall not see my face, except your brother be with you.
\verse And Israel said, Wherefore dealt ye so ill with me, as to tell the man whether ye had yet a brother?
\verse And they said, The man asked us straitly of our state, and of our kindred, saying, Is your father yet alive? have ye another brother? and we told him according to the tenor of these words: could we certainly know that he would say, Bring your brother down?
\verse And Judah said unto Israel his father, Send the lad with me, and we will arise and go; that we may live, and not die, both we, and thou, and also our little ones.
\verse I will be surety for him; of my hand shalt thou require him: if I bring him not unto thee, and set him before thee, then let me bear the blame for ever:
\verse For except we had lingered, surely now we had returned this second time.
\verse And their father Israel said unto them, If it must be so now, do this; take of the best fruits in the land in your vessels, and carry down the man a present, a little balm, and a little honey, spices, and myrrh, nuts, and almonds:
\verse And take double money in your hand; and the money that was brought again in the mouth of your sacks, carry it again in your hand; peradventure it was an oversight:
\verse Take also your brother, and arise, go again unto the man:
\verse And God Almighty give you mercy before the man, that he may send away your other brother, and Benjamin. If I be bereaved of my children, I am bereaved.
\verse And the men took that present, and they took double money in their hand, and Benjamin; and rose up, and went down to Egypt, and stood before Joseph.
\verse And when Joseph saw Benjamin with them, he said to the ruler of his house, Bring these men home, and slay, and make ready; for these men shall dine with me at noon.
\verse And the man did as Joseph bade; and the man brought the men into Joseph's house.
\verse And the men were afraid, because they were brought into Joseph's house; and they said, Because of the money that was returned in our sacks at the first time are we brought in; that he may seek occasion against us, and fall upon us, and take us for bondmen, and our asses.
\verse And they came near to the steward of Joseph's house, and they communed with him at the door of the house,
\verse And said, O sir, we came indeed down at the first time to buy food:
\verse And it came to pass, when we came to the inn, that we opened our sacks, and, behold, every man's money was in the mouth of his sack, our money in full weight: and we have brought it again in our hand.
\verse And other money have we brought down in our hands to buy food: we cannot tell who put our money in our sacks.
\verse And he said, Peace be to you, fear not: your God, and the God of your father, hath given you treasure in your sacks: I had your money. And he brought Simeon out unto them.
\verse And the man brought the men into Joseph's house, and gave them water, and they washed their feet; and he gave their asses provender.
\verse And they made ready the present against Joseph came at noon: for they heard that they should eat bread there.
\verse And when Joseph came home, they brought him the present which was in their hand into the house, and bowed themselves to him to the earth.
\verse And he asked them of their welfare, and said, Is your father well, the old man of whom ye spake? Is he yet alive?
\verse And they answered, Thy servant our father is in good health, he is yet alive. And they bowed down their heads, and made obeisance.
\verse And he lifted up his eyes, and saw his brother Benjamin, his mother's son, and said, Is this your younger brother, of whom ye spake unto me? And he said, God be gracious unto thee, my son.
\verse And Joseph made haste; for his bowels did yearn upon his brother: and he sought where to weep; and he entered into his chamber, and wept there.
\verse And he washed his face, and went out, and refrained himself, and said, Set on bread.
\verse And they set on for him by himself, and for them by themselves, and for the Egyptians, which did eat with him, by themselves: because the Egyptians might not eat bread with the Hebrews; for that is an abomination unto the Egyptians.
\verse And they sat before him, the firstborn according to his birthright, and the youngest according to his youth: and the men marvelled one at another.
\verse And he took and sent messes unto them from before him: but Benjamin's mess was five times so much as any of theirs. And they drank, and were merry with him.
\end{biblechapter}

\begin{biblechapter} % Genesis 44
\verseWithHeading{A silver cup in a sack} And he commanded the steward of his house, saying, Fill the men's sacks with food, as much as they can carry, and put every man's money in his sack's mouth.
\verse And put my cup, the silver cup, in the sack's mouth of the youngest, and his corn money. And he did according to the word that Joseph had spoken.
\verse As soon as the morning was light, the men were sent away, they and their asses.
\verse And when they were gone out of the city, and not yet far off, Joseph said unto his steward, Up, follow after the men; and when thou dost overtake them, say unto them, Wherefore have ye rewarded evil for good?
\verse Is not this it in which my lord drinketh, and whereby indeed he divineth? ye have done evil in so doing.
\verse And he overtook them, and he spake unto them these same words.
\verse And they said unto him, Wherefore saith my lord these words? God forbid that thy servants should do according to this thing:
\verse Behold, the money, which we found in our sacks' mouths, we brought again unto thee out of the land of Canaan: how then should we steal out of thy lord's house silver or gold?
\verse With whomsoever of thy servants it be found, both let him die, and we also will be my lord's bondmen.
\verse And he said, Now also let it be according unto your words: he with whom it is found shall be my servant; and ye shall be blameless.
\verse Then they speedily took down every man his sack to the ground, and opened every man his sack.
\verse And he searched, and began at the eldest, and left at the youngest: and the cup was found in Benjamin's sack.
\verse Then they rent their clothes, and laded every man his ass, and returned to the city.
\verse And Judah and his brethren came to Joseph's house; for he was yet there: and they fell before him on the ground.
\verse And Joseph said unto them, What deed is this that ye have done? wot ye not that such a man as I can certainly divine?
\verse And Judah said, What shall we say unto my lord? what shall we speak? or how shall we clear ourselves? God hath found out the iniquity of thy servants: behold, we are my lord's servants, both we, and he also with whom the cup is found.
\verse And he said, God forbid that I should do so: but the man in whose hand the cup is found, he shall be my servant; and as for you, get you up in peace unto your father.
\verse Then Judah came near unto him, and said, Oh my lord, let thy servant, I pray thee, speak a word in my lord's ears, and let not thine anger burn against thy servant: for thou art even as Pharaoh.
\verse My lord asked his servants, saying, Have ye a father, or a brother?
\verse And we said unto my lord, We have a father, an old man, and a child of his old age, a little one; and his brother is dead, and he alone is left of his mother, and his father loveth him.
\verse And thou saidst unto thy servants, Bring him down unto me, that I may set mine eyes upon him.
\verse And we said unto my lord, The lad cannot leave his father: for if he should leave his father, his father would die.
\verse And thou saidst unto thy servants, Except your youngest brother come down with you, ye shall see my face no more.
\verse And it came to pass when we came up unto thy servant my father, we told him the words of my lord.
\verse And our father said, Go again, and buy us a little food.
\verse And we said, We cannot go down: if our youngest brother be with us, then will we go down: for we may not see the man's face, except our youngest brother be with us.
\verse And thy servant my father said unto us, Ye know that my wife bare me two sons:
\verse And the one went out from me, and I said, Surely he is torn in pieces; and I saw him not since:
\verse And if ye take this also from me, and mischief befall him, ye shall bring down my gray hairs with sorrow to the grave.
\verse Now therefore when I come to thy servant my father, and the lad be not with us; seeing that his life is bound up in the lad's life;
\verse It shall come to pass, when he seeth that the lad is not with us, that he will die: and thy servants shall bring down the gray hairs of thy servant our father with sorrow to the grave.
\verse For thy servant became surety for the lad unto my father, saying, If I bring him not unto thee, then I shall bear the blame to my father for ever.
\verse Now therefore, I pray thee, let thy servant abide instead of the lad a bondman to my lord; and let the lad go up with his brethren.
\verse For how shall I go up to my father, and the lad be not with me? lest peradventure I see the evil that shall come on my father.
\end{biblechapter}

\begin{biblechapter} % Genesis 45
\verseWithHeading{Joseph makes himself \newline known} Then Joseph could not refrain himself before all them that stood by him; and he cried, Cause every man to go out from me. And there stood no man with him, while Joseph made himself known unto his brethren.
\verse And he wept aloud: and the Egyptians and the house of Pharaoh heard.
\verse And Joseph said unto his brethren, I am Joseph; doth my father yet live? And his brethren could not answer him; for they were troubled at his presence.
\verse And Joseph said unto his brethren, Come near to me, I pray you. And they came near. And he said, I am Joseph your brother, whom ye sold into Egypt.
\verse Now therefore be not grieved, nor angry with yourselves, that ye sold me hither: for God did send me before you to preserve life.
\verse For these two years hath the famine been in the land: and yet there are five years, in the which there shall neither be earing nor harvest.
\verse And God sent me before you to preserve you a posterity in the earth, and to save your lives by a great deliverance.
\verse So now it was not you that sent me hither, but God: and he hath made me a father to Pharaoh, and lord of all his house, and a ruler throughout all the land of Egypt.
\verse Haste ye, and go up to my father, and say unto him, Thus saith thy son Joseph, God hath made me lord of all Egypt: come down unto me, tarry not:
\verse And thou shalt dwell in the land of Goshen, and thou shalt be near unto me, thou, and thy children, and thy children's children, and thy flocks, and thy herds, and all that thou hast:
\verse And there will I nourish thee; for yet there are five years of famine; lest thou, and thy household, and all that thou hast, come to poverty.
\verse And, behold, your eyes see, and the eyes of my brother Benjamin, that it is my mouth that speaketh unto you.
\verse And ye shall tell my father of all my glory in Egypt, and of all that ye have seen; and ye shall haste and bring down my father hither.
\verse And he fell upon his brother Benjamin's neck, and wept; and Benjamin wept upon his neck.
\verse Moreover he kissed all his brethren, and wept upon them: and after that his brethren talked with him.
\verse And the fame thereof was heard in Pharaoh's house, saying, Joseph's brethren are come: and it pleased Pharaoh well, and his servants.
\verse And Pharaoh said unto Joseph, Say unto thy brethren, This do ye; lade your beasts, and go, get you unto the land of Canaan;
\verse And take your father and your households, and come unto me: and I will give you the good of the land of Egypt, and ye shall eat the fat of the land.
\verse Now thou art commanded, this do ye; take you wagons out of the land of Egypt for your little ones, and for your wives, and bring your father, and come.
\verse Also regard not your stuff; for the good of all the land of Egypt is yours.
\verse And the children of Israel did so: and Joseph gave them wagons, according to the commandment of Pharaoh, and gave them provision for the way.
\verse To all of them he gave each man changes of raiment; but to Benjamin he gave three hundred pieces of silver, and five changes of raiment.
\verse And to his father he sent after this manner; ten asses laden with the good things of Egypt, and ten she asses laden with corn and bread and meat for his father by the way.
\verse So he sent his brethren away, and they departed: and he said unto them, See that ye fall not out by the way.
\verse And they went up out of Egypt, and came into the land of Canaan unto Jacob their father,
\verse And told him, saying, Joseph is yet alive, and he is governor over all the land of Egypt. And Jacob's heart fainted, for he believed them not.
\verse And they told him all the words of Joseph, which he had said unto them: and when he saw the wagons which Joseph had sent to carry him, the spirit of Jacob their father revived:
\verse And Israel said, It is enough; Joseph my son is yet alive: I will go and see him before I die.
\end{biblechapter}

\begin{biblechapter} % Genesis 46
\verseWithHeading{Jacob goes to Egypt} And Israel took his journey with all that he had, and came to Beersheba, and offered sacrifices unto the God of his father Isaac.
\verse And God spake unto Israel in the visions of the night, and said, Jacob, Jacob. And he said, Here am I.
\verse And he said, I am God, the God of thy father: fear not to go down into Egypt; for I will there make of thee a great nation:
\verse I will go down with thee into Egypt; and I will also surely bring thee up again: and Joseph shall put his hand upon thine eyes.
\verse And Jacob rose up from Beersheba: and the sons of Israel carried Jacob their father, and their little ones, and their wives, in the wagons which Pharaoh had sent to carry him.
\verse And they took their cattle, and their goods, which they had gotten in the land of Canaan, and came into Egypt, Jacob, and all his seed with him:
\verse His sons, and his sons' sons with him, his daughters, and his sons' daughters, and all his seed brought he with him into Egypt.
\verse And these are the names of the children of Israel, which came into Egypt, Jacob and his sons: Reuben, Jacob's firstborn.
\verse And the sons of Reuben; Hanoch, and Phallu, and Hezron, and Carmi.
\verse And the sons of Simeon; Jemuel, and Jamin, and Ohad, and Jachin, and Zohar, and Shaul the son of a Canaanitish woman.
\verse And the sons of Levi; Gershon, Kohath, and Merari.
\verse And the sons of Judah; Er, and Onan, and Shelah, and Pharez, and Zerah: but Er and Onan died in the land of Canaan. And the sons of Pharez were Hezron and Hamul.
\verse And the sons of Issachar; Tola, and Phuvah, and Job, and Shimron.
\verse And the sons of Zebulun; Sered, and Elon, and Jahleel.
\verse These be the sons of Leah, which she bare unto Jacob in Padanaram, with his daughter Dinah: all the souls of his sons and his daughters were thirty and three.
\verse And the sons of Gad; Ziphion, and Haggi, Shuni, and Ezbon, Eri, and Arodi, and Areli.
\verse And the sons of Asher; Jimnah, and Ishuah, and Isui, and Beriah, and Serah their sister: and the sons of Beriah; Heber, and Malchiel.
\verse These are the sons of Zilpah, whom Laban gave to Leah his daughter, and these she bare unto Jacob, even sixteen souls.
\verse The sons of Rachel Jacob's wife; Joseph, and Benjamin.
\verse And unto Joseph in the land of Egypt were born Manasseh and Ephraim, which Asenath the daughter of Potipherah priest of On bare unto him.
\verse And the sons of Benjamin were Belah, and Becher, and Ashbel, Gera, and Naaman, Ehi, and Rosh, Muppim, and Huppim, and Ard.
\verse These are the sons of Rachel, which were born to Jacob: all the souls were fourteen.
\verse And the sons of Dan; Hushim.
\verse And the sons of Naphtali; Jahzeel, and Guni, and Jezer, and Shill em.
\verse These are the sons of Bilhah, which Laban gave unto Rachel his daughter, and she bare these unto Jacob: all the souls were seven.
\verse All the souls that came with Jacob into Egypt, which came out of his loins, besides Jacob's sons' wives, all the souls were threescore and six;
\verse And the sons of Joseph, which were born him in Egypt, were two souls: all the souls of the house of Jacob, which came into Egypt, were threescore and ten.
\verse And he sent Judah before him unto Joseph, to direct his face unto Goshen; and they came into the land of Goshen.
\verse And Joseph made ready his chariot, and went up to meet Israel his father, to Goshen, and presented himself unto him; and he fell on his neck, and wept on his neck a good while.
\verse And Israel said unto Joseph, Now let me die, since I have seen thy face, because thou art yet alive.
\verse And Joseph said unto his brethren, and unto his father's house, I will go up, and shew Pharaoh, and say unto him, My brethren, and my father's house, which were in the land of Canaan, are come unto me;
\verse And the men are shepherds, for their trade hath been to feed cattle; and they have brought their flocks, and their herds, and all that they have.
\verse And it shall come to pass, when Pharaoh shall call you, and shall say, What is your occupation?
\verse That ye shall say, Thy servants' trade hath been about cattle from our youth even until now, both we, and also our fathers: that ye may dwell in the land of Goshen; for every shepherd is an abomination unto the Egyptians.
\end{biblechapter}

\begin{biblechapter} % Genesis 47
\verse Then Joseph came and told Pharaoh, and said, My father and my brethren, and their flocks, and their herds, and all that they have, are come out of the land of Canaan; and, behold, they are in the land of Goshen.
\verse And he took some of his brethren, even five men, and presented them unto Pharaoh.
\verse And Pharaoh said unto his brethren, What is your occupation? And they said unto Pharaoh, Thy servants are shepherds, both we, and also our fathers.
\verse They said moreover unto Pharaoh, For to sojourn in the land are we come; for thy servants have no pasture for their flocks; for the famine is sore in the land of Canaan: now therefore, we pray thee, let thy servants dwell in the land of Goshen.
\verse And Pharaoh spake unto Joseph, saying, Thy father and thy brethren are come unto thee:
\verse The land of Egypt is before thee; in the best of the land make thy father and brethren to dwell; in the land of Goshen let them dwell: and if thou knowest any men of activity among them, then make them rulers over my cattle.
\verse And Joseph brought in Jacob his father, and set him before Pharaoh: and Jacob blessed Pharaoh.
\verse And Pharaoh said unto Jacob, How old art thou?
\verse And Jacob said unto Pharaoh, The days of the years of my pilgrimage are an hundred and thirty years: few and evil have the days of the years of my life been, and have not attained unto the days of the years of the life of my fathers in the days of their pilgrimage.
\verse And Jacob blessed Pharaoh, and went out from before Pharaoh.
\verse And Joseph placed his father and his brethren, and gave them a possession in the land of Egypt, in the best of the land, in the land of Rameses, as Pharaoh had commanded.
\verse And Joseph nourished his father, and his brethren, and all his father's household, with bread, according to their families.
\verseWithHeading{Joseph and the famine} And there was no bread in all the land; for the famine was very sore, so that the land of Egypt and all the land of Canaan fainted by reason of the famine.
\verse And Joseph gathered up all the money that was found in the land of Egypt, and in the land of Canaan, for the corn which they bought: and Joseph brought the money into Pharaoh's house.
\verse And when money failed in the land of Egypt, and in the land of Canaan, all the Egyptians came unto Joseph, and said, Give us bread: for why should we die in thy presence? for the money faileth.
\verse And Joseph said, Give your cattle; and I will give you for your cattle, if money fail.
\verse And they brought their cattle unto Joseph: and Joseph gave them bread in exchange for horses, and for the flocks, and for the cattle of the herds, and for the asses: and he fed them with bread for all their cattle for that year.
\verse When that year was ended, they came unto him the second year, and said unto him, We will not hide it from my lord, how that our money is spent; my lord also hath our herds of cattle; there is not ought left in the sight of my lord, but our bodies, and our lands:
\verse Wherefore shall we die before thine eyes, both we and our land? buy us and our land for bread, and we and our land will be servants unto Pharaoh: and give us seed, that we may live, and not die, that the land be not desolate.
\verse And Joseph bought all the land of Egypt for Pharaoh; for the Egyptians sold every man his field, because the famine prevailed over them: so the land became Pharaoh's.
\verse And as for the people, he removed them to cities from one end of the borders of Egypt even to the other end thereof.
\verse Only the land of the priests bought he not; for the priests had a portion assigned them of Pharaoh, and did eat their portion which Pharaoh gave them: wherefore they sold not their lands.
\verse Then Joseph said unto the people, Behold, I have bought you this day and your land for Pharaoh: lo, here is seed for you, and ye shall sow the land.
\verse And it shall come to pass in the increase, that ye shall give the fifth part unto Pharaoh, and four parts shall be your own, for seed of the field, and for your food, and for them of your households, and for food for your little ones.
\verse And they said, Thou hast saved our lives: let us find grace in the sight of my lord, and we will be Pharaoh's servants.
\verse And Joseph made it a law over the land of Egypt unto this day, that Pharaoh should have the fifth part; except the land of the priests only, which became not Pharaoh's.
\verse And Israel dwelt in the land of Egypt, in the country of Goshen; and they had possessions therein, and grew, and multiplied exceedingly.
\verse And Jacob lived in the land of Egypt seventeen years: so the whole age of Jacob was an hundred forty and seven years.
\verse And the time drew nigh that Israel must die: and he called his son Joseph, and said unto him, If now I have found grace in thy sight, put, I pray thee, thy hand under my thigh, and deal kindly and truly with me; bury me not, I pray thee, in Egypt:
\verse But I will lie with my fathers, and thou shalt carry me out of Egypt, and bury me in their buryingplace. And he said, I will do as thou hast said.
\verse And he said, Swear unto me. And he sware unto him. And Israel bowed himself upon the bed's head.
\end{biblechapter}

\begin{biblechapter} % Genesis 48
\verseWithHeading{Manasseh and Ephraim} And it came to pass after these things, that one told Joseph, Behold, thy father is sick: and he took with him his two sons, Manasseh and Ephraim.
\verse And one told Jacob, and said, Behold, thy son Joseph cometh unto thee: and Israel strengthened himself, and sat upon the bed.
\verse And Jacob said unto Joseph, God Almighty appeared unto me at Luz in the land of Canaan, and blessed me,
\verse And said unto me, Behold, I will make thee fruitful, and multiply thee, and I will make of thee a multitude of people; and will give this land to thy seed after thee for an everlasting possession.
\verse And now thy two sons, Ephraim and Manasseh, which were born unto thee in the land of Egypt before I came unto thee into Egypt, are mine; as Reuben and Simeon, they shall be mine.
\verse And thy issue, which thou begettest after them, shall be thine, and shall be called after the name of their brethren in their inheritance.
\verse And as for me, when I came from Padan, Rachel died by me in the land of Canaan in the way, when yet there was but a little way to come unto Ephrath: and I buried her there in the way of Ephrath; the same is Bethlehem.
\verse And Israel beheld Joseph's sons, and said, Who are these?
\verse And Joseph said unto his father, They are my sons, whom God hath given me in this place. And he said, Bring them, I pray thee, unto me, and I will bless them.
\verse Now the eyes of Israel were dim for age, so that he could not see. And he brought them near unto him; and he kissed them, and embraced them.
\verse And Israel said unto Joseph, I had not thought to see thy face: and, lo, God hath shewed me also thy seed.
\verse And Joseph brought them out from between his knees, and he bowed himself with his face to the earth.
\verse And Joseph took them both, Ephraim in his right hand toward Israel's left hand, and Manasseh in his left hand toward Israel's right hand, and brought them near unto him.
\verse And Israel stretched out his right hand, and laid it upon Ephraim's head, who was the younger, and his left hand upon Manasseh's head, guiding his hands wittingly; for Manasseh was the firstborn.
\verse And he blessed Joseph, and said, God, before whom my fathers Abraham and Isaac did walk, the God which fed me all my life long unto this day,
\verse The Angel which redeemed me from all evil, bless the lads; and let my name be named on them, and the name of my fathers Abraham and Isaac; and let them grow into a multitude in the midst of the earth.
\verse And when Joseph saw that his father laid his right hand upon the head of Ephraim, it displeased him: and he held up his father's hand, to remove it from Ephraim's head unto Manasseh's head.
\verse And Joseph said unto his father, Not so, my father: for this is the firstborn; put thy right hand upon his head.
\verse And his father refused, and said, I know it, my son, I know it: he also shall become a people, and he also shall be great: but truly his younger brother shall be greater than he, and his seed shall become a multitude of nations.
\verse And he blessed them that day, saying, In thee shall Israel bless, saying, God make thee as Ephraim and as Manasseh: and he set Ephraim before Manasseh.
\verse And Israel said unto Joseph, Behold, I die: but God shall be with you, and bring you again unto the land of your fathers.
\verse Moreover I have given to thee one portion above thy brethren, which I took out of the hand of the Amorite with my sword and with my bow.
\end{biblechapter}

\begin{biblechapter} % Genesis 49
\verseWithHeading{Jacob blesses his sons} And Jacob called unto his sons, and said, Gather yourselves together, that I may tell you that which shall befall you in the last days.
\verse Gather yourselves together, and hear, ye sons of Jacob; and hearken unto Israel your father.
\verse Reuben, thou art my firstborn, my might, and the beginning of my strength, the excellency of dignity, and the excellency of power:
\verse Unstable as water, thou shalt not excel; because thou wentest up to thy father's bed; then defiledst thou it: he went up to my couch.
\verse Simeon and Levi are brethren; instruments of cruelty are in their habitations.
\verse O my soul, come not thou into their secret; unto their assembly, mine honour, be not thou united: for in their anger they slew a man, and in their selfwill they digged down a wall.
\verse Cursed be their anger, for it was fierce; and their wrath, for it was cruel: I will divide them in Jacob, and scatter them in Israel.
\verse Judah, thou art he whom thy brethren shall praise: thy hand shall be in the neck of thine enemies; thy father's children shall bow down before thee.
\verse Judah is a lion's whelp: from the prey, my son, thou art gone up: he stooped down, he couched as a lion, and as an old lion; who shall rouse him up?
\verse The sceptre shall not depart from Judah, nor a lawgiver from between his feet, until Shiloh come; and unto him shall the gathering of the people be.
\verse Binding his foal unto the vine, and his ass's colt unto the choice vine; he washed his garments in wine, and his clothes in the blood of grapes:
\verse His eyes shall be red with wine, and his teeth white with milk.
\verse Zebulun shall dwell at the haven of the sea; and he shall be for an haven of ships; and his border shall be unto Zidon.
\verse Issachar is a strong ass couching down between two burdens:
\verse And he saw that rest was good, and the land that it was pleasant; and bowed his shoulder to bear, and became a servant unto tribute.
\verse Dan shall judge his people, as one of the tribes of Israel.
\verse Dan shall be a serpent by the way, an adder in the path, that biteth the horse heels, so that his rider shall fall backward.
\verse I have waited for thy salvation, O \LORD.
\verse Gad, a troop shall overcome him: but he shall overcome at the last.
\verse Out of Asher his bread shall be fat, and he shall yield royal dainties.
\verse Naphtali is a hind let loose: he giveth goodly words.
\verse Joseph is a fruitful bough, even a fruitful bough by a well; whose branches run over the wall:
\verse The archers have sorely grieved him, and shot at him, and hated him:
\verse But his bow abode in strength, and the arms of his hands were made strong by the hands of the mighty God of Jacob; (from thence is the shepherd, the stone of Israel:)
\verse Even by the God of thy father, who shall help thee; and by the Almighty, who shall bless thee with blessings of heaven above, blessings of the deep that lieth under, blessings of the breasts, and of the womb:
\verse The blessings of thy father have prevailed above the blessings of my progenitors unto the utmost bound of the everlasting hills: they shall be on the head of Joseph, and on the crown of the head of him that was separate from his brethren.
\verse Benjamin shall ravin as a wolf: in the morning he shall devour the prey, and at night he shall divide the spoil.
\verse All these are the twelve tribes of Israel: and this is it that their father spake unto them, and blessed them; every one according to his blessing he blessed them.
\verseWithHeading{The death of Jacob} And he charged them, and said unto them, I am to be gathered unto my people: bury me with my fathers in the cave that is in the field of Ephron the Hittite,
\verse In the cave that is in the field of Machpelah, which is before Mamre, in the land of Canaan, which Abraham bought with the field of Ephron the Hittite for a possession of a buryingplace.
\verse There they buried Abraham and Sarah his wife; there they buried Isaac and Rebekah his wife; and there I buried Leah.
\verse The purchase of the field and of the cave that is therein was from the children of Heth.
\verse And when Jacob had made an end of commanding his sons, he gathered up his feet into the bed, and yielded up the ghost, and was gathered unto his people.
\end{biblechapter}

\begin{biblechapter} % Genesis 50
\verse And Joseph fell upon his father's face, and wept upon him, and kissed him.
\verse And Joseph commanded his servants the physicians to embalm his father: and the physicians embalmed Israel.
\verse And forty days were fulfilled for him; for so are fulfilled the days of those which are embalmed: and the Egyptians mourned for him threescore and ten days.
\verse And when the days of his mourning were past, Joseph spake unto the house of Pharaoh, saying, If now I have found grace in your eyes, speak, I pray you, in the ears of Pharaoh, saying,
\verse My father made me swear, saying, Lo, I die: in my grave which I have digged for me in the land of Canaan, there shalt thou bury me. Now therefore let me go up, I pray thee, and bury my father, and I will come again.
\verse And Pharaoh said, Go up, and bury thy father, according as he made thee swear.
\verse And Joseph went up to bury his father: and with him went up all the servants of Pharaoh, the elders of his house, and all the elders of the land of Egypt,
\verse And all the house of Joseph, and his brethren, and his father's house: only their little ones, and their flocks, and their herds, they left in the land of Goshen.
\verse And there went up with him both chariots and horsemen: and it was a very great company.
\verse And they came to the threshingfloor of Atad, which is beyond Jordan, and there they mourned with a great and very sore lamentation: and he made a mourning for his father seven days.
\verse And when the inhabitants of the land, the Canaanites, saw the mourning in the floor of Atad, they said, This is a grievous mourning to the Egyptians: wherefore the name of it was called Abelmizraim, which is beyond Jordan.
\verse And his sons did unto him according as he commanded them:
\verse For his sons carried him into the land of Canaan, and buried him in the cave of the field of Machpelah, which Abraham bought with the field for a possession of a buryingplace of Ephron the Hittite, before Mamre.
\verse And Joseph returned into Egypt, he, and his brethren, and all that went up with him to bury his father, after he had buried his father.
\verseWithHeading{Joseph reassures his brothers} And when Joseph's brethren saw that their father was dead, they said, Joseph will peradventure hate us, and will certainly requite us all the evil which we did unto him.
\verse And they sent a messenger unto Joseph, saying, Thy father did command before he died, saying,
\verse So shall ye say unto Joseph, Forgive, I pray thee now, the trespass of thy brethren, and their sin; for they did unto thee evil: and now, we pray thee, forgive the trespass of the servants of the God of thy father. And Joseph wept when they spake unto him.
\verse And his brethren also went and fell down before his face; and they said, Behold, we be thy servants.
\verse And Joseph said unto them, Fear not: for am I in the place of God?
\verse But as for you, ye thought evil against me; but God meant it unto good, to bring to pass, as it is this day, to save much people alive.
\verse Now therefore fear ye not: I will nourish you, and your little ones. And he comforted them, and spake kindly unto them.
\verseWithHeading{The death of Joseph} And Joseph dwelt in Egypt, he, and his father's house: and Joseph lived an hundred and ten years.
\verse And Joseph saw Ephraim's children of the third generation: the children also of Machir the son of Manasseh were brought up upon Joseph's knees.
\verse And Joseph said unto his brethren, I die: and God will surely visit you, and bring you out of this land unto the land which he sware to Abraham, to Isaac, and to Jacob.
\verse And Joseph took an oath of the children of Israel, saying, God will surely visit you, and ye shall carry up my bones from hence.
\verse So Joseph died, being an hundred and ten years old: and they embalmed him, and he was put in a coffin in Egypt.
\end{biblechapter}
\flushcolsend
\biblebook{Exodus}

\begin{biblechapter} % Exodus 1
\verseWithHeading{The Israelites oppressed} Now these are the names of the children of Israel, which came into Egypt; every man and his household came with Jacob.
\verse Reuben, Simeon, Levi, and Judah,
\verse Issachar, Zebulun, and Benjamin,
\verse Dan, and Naphtali, Gad, and Asher.
\verse And all the souls that came out of the loins of Jacob were seventy souls: for Joseph was in Egypt already.
\verse And Joseph died, and all his brethren, and all that generation.
\verse And the children of Israel were fruitful, and increased abundantly, and multiplied, and waxed exceeding mighty; and the land was filled with them.
\verse Now there arose up a new king over Egypt, which knew not Joseph.
\verse And he said unto his people, Behold, the people of the children of Israel are more and mightier than we:
\verse Come on, let us deal wisely with them; lest they multiply, and it come to pass, that, when there falleth out any war, they join also unto our enemies, and fight against us, and so get them up out of the land.
\verse Therefore they did set over them taskmasters to afflict them with their burdens. And they built for Pharaoh treasure cities, Pithom and Raamses.
\verse But the more they afflicted them, the more they multiplied and grew. And they were grieved because of the children of Israel.
\verse And the Egyptians made the children of Israel to serve with rigour:
\verse And they made their lives bitter with hard bondage, in morter, and in brick, and in all manner of service in the field: all their service, wherein they made them serve, was with rigour.
\verse And the king of Egypt spake to the Hebrew midwives, of which the name of the one was Shiphrah, and the name of the other Puah:
\verse And he said, When ye do the office of a midwife to the Hebrew women, and see them upon the stools; if it be a son, then ye shall kill him: but if it be a daughter, then she shall live.
\verse But the midwives feared God, and did not as the king of Egypt commanded them, but saved the men children alive.
\verse And the king of Egypt called for the midwives, and said unto them, Why have ye done this thing, and have saved the men children alive?
\verse And the midwives said unto Pharaoh, Because the Hebrew women are not as the Egyptian women; for they are lively, and are delivered ere the midwives come in unto them.
\verse Therefore God dealt well with the midwives: and the people multiplied, and waxed very mighty.
\verse And it came to pass, because the midwives feared God, that he made them houses.
\verse And Pharaoh charged all his people, saying, Every son that is born ye shall cast into the river, and every daughter ye shall save alive.
\end{biblechapter}

\begin{biblechapter} % Exodus 2
\verseWithHeading{The birth of Moses} And there went a man of the house of Levi, and took to wife a daughter of Levi.
\verse And the woman conceived, and bare a son: and when she saw him that he was a goodly child, she hid him three months.
\verse And when she could not longer hide him, she took for him an ark of bulrushes, and daubed it with slime and with pitch, and put the child therein; and she laid it in the flags by the river's brink.
\verse And his sister stood afar off, to wit what would be done to him.
\verse And the daughter of Pharaoh came down to wash herself at the river; and her maidens walked along by the river's side; and when she saw the ark among the flags, she sent her maid to fetch it.
\verse And when she had opened it, she saw the child: and, behold, the babe wept. And she had compassion on him, and said, This is one of the Hebrews' children.
\verse Then said his sister to Pharaoh's daughter, Shall I go and call to thee a nurse of the Hebrew women, that she may nurse the child for thee?
\verse And Pharaoh's daughter said to her, Go. And the maid went and called the child's mother.
\verse And Pharaoh's daughter said unto her, Take this child away, and nurse it for me, and I will give thee thy wages. And the woman took the child, and nursed it.
\verse And the child grew, and she brought him unto Pharaoh's daughter, and he became her son. And she called his name Moses: and she said, Because I drew him out of the water.
\verseWithHeading{Moses flees to Midian} And it came to pass in those days, when Moses was grown, that he went out unto his brethren, and looked on their burdens: and he spied an Egyptian smiting an Hebrew, one of his brethren.
\verse And he looked this way and that way, and when he saw that there was no man, he slew the Egyptian, and hid him in the sand.
\verse And when he went out the second day, behold, two men of the Hebrews strove together: and he said to him that did the wrong, Wherefore smitest thou thy fellow?
\verse And he said, Who made thee a prince and a judge over us? intendest thou to kill me, as thou killedst the Egyptian? And Moses feared, and said, Surely this thing is known.
\verse Now when Pharaoh heard this thing, he sought to slay Moses. But Moses fled from the face of Pharaoh, and dwelt in the land of Midian: and he sat down by a well.
\verse Now the priest of Midian had seven daughters: and they came and drew water, and filled the troughs to water their father's flock.
\verse And the shepherds came and drove them away: but Moses stood up and helped them, and watered their flock.
\verse And when they came to Reuel their father, he said, How is it that ye are come so soon to day?
\verse And they said, An Egyptian delivered us out of the hand of the shepherds, and also drew water enough for us, and watered the flock.
\verse And he said unto his daughters, And where is he? why is it that ye have left the man? call him, that he may eat bread.
\verse And Moses was content to dwell with the man: and he gave Moses Zipporah his daughter.
\verse And she bare him a son, and he called his name Gershom: for he said, I have been a stranger in a strange land.
\verse And it came to pass in process of time, that the king of Egypt died: and the children of Israel sighed by reason of the bondage, and they cried, and their cry came up unto God by reason of the bondage.
\verse And God heard their groaning, and God remembered his covenant with Abraham, with Isaac, and with Jacob.
\verse And God looked upon the children of Israel, and God had respect unto them.
\end{biblechapter}

\vfill\columnbreak % layout hack

\begin{biblechapter} % Exodus 3
\verseWithHeading{Moses and the burning bush} Now Moses kept the flock of Jethro his father in law, the priest of Midian: and he led the flock to the backside of the desert, and came to the mountain of God, even to Horeb.
\verse And the angel of the \LORD appeared unto him in a flame of fire out of the midst of a bush: and he looked, and, behold, the bush burned with fire, and the bush was not consumed.
\verse And Moses said, I will now turn aside, and see this great sight, why the bush is not burnt.
\verse And when the \LORD saw that he turned aside to see, God called unto him out of the midst of the bush, and said, Moses, Moses. And he said, Here am I.
\verse And he said, Draw not nigh hither: put off thy shoes from off thy feet, for the place whereon thou standest is holy ground.
\verse Moreover he said, I am the God of thy father, the God of Abraham, the God of Isaac, and the God of Jacob. And Moses hid his face; for he was afraid to look upon God.
\verse And the \LORD said, I have surely seen the affliction of my people which are in Egypt, and have heard their cry by reason of their taskmasters; for I know their sorrows;
\verse And I am come down to deliver them out of the hand of the Egyptians, and to bring them up out of that land unto a good land and a large, unto a land flowing with milk and honey; unto the place of the Canaanites, and the Hittites, and the Amorites, and the Perizzites, and the Hivites, and the Jebusites.
\verse Now therefore, behold, the cry of the children of Israel is come unto me: and I have also seen the oppression wherewith the Egyptians oppress them.
\verse Come now therefore, and I will send thee unto Pharaoh, that thou mayest bring forth my people the children of Israel out of Egypt.
\verse And Moses said unto God, Who am I, that I should go unto Pharaoh, and that I should bring forth the children of Israel out of Egypt?
\verse And he said, Certainly I will be with thee; and this shall be a token unto thee, that I have sent thee: When thou hast brought forth the people out of Egypt, ye shall serve God upon this mountain.
\verse And Moses said unto God, Behold, when I come unto the children of Israel, and shall say unto them, The God of your fathers hath sent me unto you; and they shall say to me, What is his name? what shall I say unto them?
\verse And God said unto Moses, \textsc{I am that I am}: and he said, Thus shalt thou say unto the children of Israel, \textsc{I am} hath sent me unto you.
\verse And God said moreover unto Moses, Thus shalt thou say unto the children of Israel, The \LORD God of your fathers, the God of Abraham, the God of Isaac, and the God of Jacob, hath sent me unto you: this is my name for ever, and this is my memorial unto all generations.
\verse Go, and gather the elders of Israel together, and say unto them, The \LORD God of your fathers, the God of Abraham, of Isaac, and of Jacob, appeared unto me, saying, I have surely visited you, and seen that which is done to you in Egypt:
\verse And I have said, I will bring you up out of the affliction of Egypt unto the land of the Canaanites, and the Hittites, and the Amorites, and the Perizzites, and the Hivites, and the Jebusites, unto a land flowing with milk and honey.
\verse And they shall hearken to thy voice: and thou shalt come, thou and the elders of Israel, unto the king of Egypt, and ye shall say unto him, The \LORD God of the Hebrews hath met with us: and now let us go, we beseech thee, three days' journey into the wilderness, that we may sacrifice to the \LORD our God.
\verse And I am sure that the king of Egypt will not let you go, no, not by a mighty hand.
\verse And I will stretch out my hand, and smite Egypt with all my wonders which I will do in the midst thereof: and after that he will let you go.
\verse And I will give this people favour in the sight of the Egyptians: and it shall come to pass, that, when ye go, ye shall not go empty:
\verse But every woman shall borrow of her neighbour, and of her that sojourneth in her house, jewels of silver, and jewels of gold, and raiment: and ye shall put them upon your sons, and upon your daughters; and ye shall spoil the Egyptians.
\end{biblechapter}

\begin{biblechapter} % Exodus 4
\verseWithHeading{Signs for Moses} And Moses answered and said, But, behold, they will not believe me, nor hearken unto my voice: for they will say, The \LORD hath not appeared unto thee.
\verse And the \LORD said unto him, What is that in thine hand? And he said, A rod.
\verse And he said, Cast it on the ground. And he cast it on the ground, and it became a serpent; and Moses fled from before it.
\verse And the \LORD said unto Moses, Put forth thine hand, and take it by the tail. And he put forth his hand, and caught it, and it became a rod in his hand:
\verse That they may believe that the \LORD God of their fathers, the God of Abraham, the God of Isaac, and the God of Jacob, hath appeared unto thee.
\verse And the \LORD said furthermore unto him, Put now thine hand into thy bosom. And he put his hand into his bosom: and when he took it out, behold, his hand was leprous as snow.
\verse And he said, Put thine hand into thy bosom again. And he put his hand into his bosom again; and plucked it out of his bosom, and, behold, it was turned again as his other flesh.
\verse And it shall come to pass, if they will not believe thee, neither hearken to the voice of the first sign, that they will believe the voice of the latter sign.
\verse And it shall come to pass, if they will not believe also these two signs, neither hearken unto thy voice, that thou shalt take of the water of the river, and pour it upon the dry land: and the water which thou takest out of the river shall become blood upon the dry land.
\verse And Moses said unto the \LORD, O my Lord, I am not eloquent, neither heretofore, nor since thou hast spoken unto thy servant: but I am slow of speech, and of a slow tongue.
\verse And the \LORD said unto him, Who hath made man's mouth? or who maketh the dumb, or deaf, or the seeing, or the blind? have not I the \LORD?
\verse Now therefore go, and I will be with thy mouth, and teach thee what thou shalt say.
\verse And he said, O my Lord, send, I pray thee, by the hand of him whom thou wilt send.
\verse And the anger of the \LORD was kindled against Moses, and he said, Is not Aaron the Levite thy brother? I know that he can speak well. And also, behold, he cometh forth to meet thee: and when he seeth thee, he will be glad in his heart.
\verse And thou shalt speak unto him, and put words in his mouth: and I will be with thy mouth, and with his mouth, and will teach you what ye shall do.
\verse And he shall be thy spokesman unto the people: and he shall be, even he shall be to thee instead of a mouth, and thou shalt be to him instead of God.
\verse And thou shalt take this rod in thine hand, wherewith thou shalt do signs.
\verseWithHeading{Moses returns to Egypt} And Moses went and returned to Jethro his father in law, and said unto him, Let me go, I pray thee, and return unto my brethren which are in Egypt, and see whether they be yet alive. And Jethro said to Moses, Go in peace.
\verse And the \LORD said unto Moses in Midian, Go, return into Egypt: for all the men are dead which sought thy life.
\verse And Moses took his wife and his sons, and set them upon an ass, and he returned to the land of Egypt: and Moses took the rod of God in his hand.
\verse And the \LORD said unto Moses, When thou goest to return into Egypt, see that thou do all those wonders before Pharaoh, which I have put in thine hand: but I will harden his heart, that he shall not let the people go.
\verse And thou shalt say unto Pharaoh, Thus saith the \LORD, Israel is my son, even my firstborn:
\verse And I say unto thee, Let my son go, that he may serve me: and if thou refuse to let him go, behold, I will slay thy son, even thy firstborn.
\verse And it came to pass by the way in the inn, that the \LORD met him, and sought to kill him.
\verse Then Zipporah took a sharp stone, and cut off the foreskin of her son, and cast it at his feet, and said, Surely a bloody husband art thou to me.
\verse So he let him go: then she said, A bloody husband thou art, because of the circumcision.
\verse And the \LORD said to Aaron, Go into the wilderness to meet Moses. And he went, and met him in the mount of God, and kissed him.
\verse And Moses told Aaron all the words of the \LORD who had sent him, and all the signs which he had commanded him.
\verse And Moses and Aaron went and gathered together all the elders of the children of Israel:
\verse And Aaron spake all the words which the \LORD had spoken unto Moses, and did the signs in the sight of the people.
\verse And the people believed: and when they heard that the \LORD had visited the children of Israel, and that he had looked upon their affliction, then they bowed their heads and worshipped.
\end{biblechapter}

\begin{biblechapter} % Exodus 5
\verseWithHeading{Bricks without straw} And afterward Moses and Aaron went in, and told Pharaoh, Thus saith the \LORD God of Israel, Let my people go, that they may hold a feast unto me in the wilderness.
\verse And Pharaoh said, Who is the \LORD, that I should obey his voice to let Israel go? I know not the \LORD, neither will I let Israel go.
\verse And they said, The God of the Hebrews hath met with us: let us go, we pray thee, three days' journey into the desert, and sacrifice unto the \LORD our God; lest he fall upon us with pestilence, or with the sword.
\verse And the king of Egypt said unto them, Wherefore do ye, Moses and Aaron, let the people from their works? get you unto your burdens.
\verse And Pharaoh said, Behold, the people of the land now are many, and ye make them rest from their burdens.
\verse And Pharaoh commanded the same day the taskmasters of the people, and their officers, saying,
\verse Ye shall no more give the people straw to make brick, as heretofore: let them go and gather straw for themselves.
\verse And the tale of the bricks, which they did make heretofore, ye shall lay upon them; ye shall not diminish ought thereof: for they be idle; therefore they cry, saying, Let us go and sacrifice to our God.
\verse Let there more work be laid upon the men, that they may labour therein; and let them not regard vain words.
\verse And the taskmasters of the people went out, and their officers, and they spake to the people, saying, Thus saith Pharaoh, I will not give you straw.
\verse Go ye, get you straw where ye can find it: yet not ought of your work shall be diminished.
\verse So the people were scattered abroad throughout all the land of Egypt to gather stubble instead of straw.
\verse And the taskmasters hasted them, saying, Fulfil your works, your daily tasks, as when there was straw.
\verse And the officers of the children of Israel, which Pharaoh's taskmasters had set over them, were beaten, and demanded, Wherefore have ye not fulfilled your task in making brick both yesterday and to day, as heretofore?
\verse Then the officers of the children of Israel came and cried unto Pharaoh, saying, Wherefore dealest thou thus with thy servants?
\verse There is no straw given unto thy servants, and they say to us, Make brick: and, behold, thy servants are beaten; but the fault is in thine own people.
\verse But he said, Ye are idle, ye are idle: therefore ye say, Let us go and do sacrifice to the \LORD.
\verse Go therefore now, and work; for there shall no straw be given you, yet shall ye deliver the tale of bricks.
\verse And the officers of the children of Israel did see that they were in evil case, after it was said, Ye shall not minish ought from your bricks of your daily task.
\verse And they met Moses and Aaron, who stood in the way, as they came forth from Pharaoh:
\verse And they said unto them, The \LORD look upon you, and judge; because ye have made our savour to be abhorred in the eyes of Pharaoh, and in the eyes of his servants, to put a sword in their hand to slay us.
\verseWithHeading{God promises deliverance} And Moses returned unto the \LORD, and said, Lord, wherefore hast thou so evil entreated this people? why is it that thou hast sent me?
\verse For since I came to Pharaoh to speak in thy name, he hath done evil to this people; neither hast thou delivered thy people at all.
\end{biblechapter}

\begin{biblechapter} % Exodus 6
\verse Then the \LORD said unto Moses, Now shalt thou see what I will do to Pharaoh: for with a strong hand shall he let them go, and with a strong hand shall he drive them out of his land.
\verse And God spake unto Moses, and said unto him, I am the \LORD:
\verse And I appeared unto Abraham, unto Isaac, and unto Jacob, by the name of God Almighty, but by my name \textsc{Jehovah} was I not known to them.
\verse And I have also established my covenant with them, to give them the land of Canaan, the land of their pilgrimage, wherein they were strangers.
\verse And I have also heard the groaning of the children of Israel, whom the Egyptians keep in bondage; and I have remembered my covenant.
\verse Wherefore say unto the children of Israel, I am the \LORD, and I will bring you out from under the burdens of the Egyptians, and I will rid you out of their bondage, and I will redeem you with a stretched out arm, and with great judgments:
\verse And I will take you to me for a people, and I will be to you a God: and ye shall know that I am the \LORD your God, which bringeth you out from under the burdens of the Egyptians.
\verse And I will bring you in unto the land, concerning the which I did swear to give it to Abraham, to Isaac, and to Jacob; and I will give it you for an heritage: I am the \LORD.
\verse And Moses spake so unto the children of Israel: but they hearkened not unto Moses for anguish of spirit, and for cruel bondage.
\verse And the \LORD spake unto Moses, saying,
\verse Go in, speak unto Pharaoh king of Egypt, that he let the children of Israel go out of his land.
\verse And Moses spake before the \LORD, saying, Behold, the children of Israel have not hearkened unto me; how then shall Pharaoh hear me, who am of uncircumcised lips?
\verseWithHeading{Family records of Moses and Aaron} And the \LORD spake unto Moses and unto Aaron, and gave them a charge unto the children of Israel, and unto Pharaoh king of Egypt, to bring the children of Israel out of the land of Egypt.
\verse These be the heads of their fathers' houses: The sons of Reuben the firstborn of Israel; Hanoch, and Pallu, Hezron, and Carmi: these be the families of Reuben.
\verse And the sons of Simeon; Jemuel, and Jamin, and Ohad, and Jachin, and Zohar, and Shaul the son of a Canaanitish woman: these are the families of Simeon.
\verse And these are the names of the sons of Levi according to their generations; Gershon, and Kohath, and Merari: and the years of the life of Levi were an hundred thirty and seven years.
\verse The sons of Gershon; Libni, and Shimi, according to their families.
\verse And the sons of Kohath; Amram, and Izhar, and Hebron, and Uzziel: and the years of the life of Kohath were an hundred thirty and three years.
\verse And the sons of Merari; Mahali and Mushi: these are the families of Levi according to their generations.
\verse And Amram took him Jochebed his father's sister to wife; and she bare him Aaron and Moses: and the years of the life of Amram were an hundred and thirty and seven years.
\verse And the sons of Izhar; Korah, and Nepheg, and Zichri.
\verse And the sons of Uzziel; Mishael, and Elzaphan, and Zithri.
\verse And Aaron took him Elisheba, daughter of Amminadab, sister of Naashon, to wife; and she bare him Nadab, and Abihu, Eleazar, and Ithamar.
\verse And the sons of Korah; Assir, and Elkanah, and Abiasaph: these are the families of the Korhites.
\verse And Eleazar Aaron's son took him one of the daughters of Putiel to wife; and she bare him Phinehas: these are the heads of the fathers of the Levites according to their families.
\verse These are that Aaron and Moses, to whom the \LORD said, Bring out the children of Israel from the land of Egypt according to their armies.
\verse These are they which spake to Pharaoh king of Egypt, to bring out the children of Israel from Egypt: these are that Moses and Aaron.
\verseWithHeading{Aaron to speak for Moses} And it came to pass on the day when the \LORD spake unto Moses in the land of Egypt,
\verse That the \LORD spake unto Moses, saying, I am the \LORD: speak thou unto Pharaoh king of Egypt all that I say unto thee.
\verse And Moses said before the \LORD, Behold, I am of uncircumcised lips, and how shall Pharaoh hearken unto me?
\end{biblechapter}

\begin{biblechapter} % Exodus 7
\verse And the \LORD said unto Moses, See, I have made thee a god to Pharaoh: and Aaron thy brother shall be thy prophet.
\verse Thou shalt speak all that I command thee: and Aaron thy brother shall speak unto Pharaoh, that he send the children of Israel out of his land.
\verse And I will harden Pharaoh's heart, and multiply my signs and my wonders in the land of Egypt.
\verse But Pharaoh shall not hearken unto you, that I may lay my hand upon Egypt, and bring forth mine armies, and my people the children of Israel, out of the land of Egypt by great judgments.
\verse And the Egyptians shall know that I am the \LORD, when I stretch forth mine hand upon Egypt, and bring out the children of Israel from among them.
\verse And Moses and Aaron did as the \LORD commanded them, so did they.
\verse And Moses was fourscore years old, and Aaron fourscore and three years old, when they spake unto Pharaoh.
\verseWithHeading{Aaron's staff becomes a snake} And the \LORD spake unto Moses and unto Aaron, saying,
\verse When Pharaoh shall speak unto you, saying, Shew a miracle for you: then thou shalt say unto Aaron, Take thy rod, and cast it before Pharaoh, and it shall become a serpent.
\verse And Moses and Aaron went in unto Pharaoh, and they did so as the \LORD had commanded: and Aaron cast down his rod before Pharaoh, and before his servants, and it became a serpent.
\verse Then Pharaoh also called the wise men and the sorcerers: now the magicians of Egypt, they also did in like manner with their enchantments.
\verse For they cast down every man his rod, and they became serpents: but Aaron's rod swallowed up their rods.
\verse And he hardened Pharaoh's heart, that he hearkened not unto them; as the \LORD had said.
\verseWithHeading{The plague of blood} And the \LORD said unto Moses, Pharaoh's heart is hardened, he refuseth to let the people go.
\verse Get thee unto Pharaoh in the morning; lo, he goeth out unto the water; and thou shalt stand by the river's brink against he come; and the rod which was turned to a serpent shalt thou take in thine hand.
\verse And thou shalt say unto him, The \LORD God of the Hebrews hath sent me unto thee, saying, Let my people go, that they may serve me in the wilderness: and, behold, hitherto thou wouldest not hear.
\verse Thus saith the \LORD, In this thou shalt know that I am the \LORD: behold, I will smite with the rod that is in mine hand upon the waters which are in the river, and they shall be turned to blood.
\verse And the fish that is in the river shall die, and the river shall stink; and the Egyptians shall lothe to drink of the water of the river.
\verse And the \LORD spake unto Moses, Say unto Aaron, Take thy rod, and stretch out thine hand upon the waters of Egypt, upon their streams, upon their rivers, and upon their ponds, and upon all their pools of water, that they may become blood; and that there may be blood throughout all the land of Egypt, both in vessels of wood, and in vessels of stone.
\verse And Moses and Aaron did so, as the \LORD commanded; and he lifted up the rod, and smote the waters that were in the river, in the sight of Pharaoh, and in the sight of his servants; and all the waters that were in the river were turned to blood.
\verse And the fish that was in the river died; and the river stank, and the Egyptians could not drink of the water of the river; and there was blood throughout all the land of Egypt.
\verse And the magicians of Egypt did so with their enchantments: and Pharaoh's heart was hardened, neither did he hearken unto them; as the \LORD had said.
\verse And Pharaoh turned and went into his house, neither did he set his heart to this also.
\verse And all the Egyptians digged round about the river for water to drink; for they could not drink of the water of the river.
\verse And seven days were fulfilled, after that the \LORD had smitten the river.
\end{biblechapter}

\begin{biblechapter} % Exodus 8
\verseWithHeading{The plague of frogs} And the \LORD spake unto Moses, Go unto Pharaoh, and say unto him, Thus saith the \LORD, Let my people go, that they may serve me.
\verse And if thou refuse to let them go, behold, I will smite all thy borders with frogs:
\verse And the river shall bring forth frogs abundantly, which shall go up and come into thine house, and into thy bedchamber, and upon thy bed, and into the house of thy servants, and upon thy people, and into thine ovens, and into thy kneadingtroughs:
\verse And the frogs shall come up both on thee, and upon thy people, and upon all thy servants.
\verse And the \LORD spake unto Moses, Say unto Aaron, Stretch forth thine hand with thy rod over the streams, over the rivers, and over the ponds, and cause frogs to come up upon the land of Egypt.
\verse And Aaron stretched out his hand over the waters of Egypt; and the frogs came up, and covered the land of Egypt.
\verse And the magicians did so with their enchantments, and brought up frogs upon the land of Egypt.
\verse Then Pharaoh called for Moses and Aaron, and said, Intreat the \LORD, that he may take away the frogs from me, and from my people; and I will let the people go, that they may do sacrifice unto the \LORD.
\verse And Moses said unto Pharaoh, Glory over me: when shall I intreat for thee, and for thy servants, and for thy people, to destroy the frogs from thee and thy houses, that they may remain in the river only?
\verse And he said, To morrow. And he said, Be it according to thy word: that thou mayest know that there is none like unto the \LORD our God.
\verse And the frogs shall depart from thee, and from thy houses, and from thy servants, and from thy people; they shall remain in the river only.
\verse And Moses and Aaron went out from Pharaoh: and Moses cried unto the \LORD because of the frogs which he had brought against Pharaoh.
\verse And the \LORD did according to the word of Moses; and the frogs died out of the houses, out of the villages, and out of the fields.
\verse And they gathered them together upon heaps: and the land stank.
\verse But when Pharaoh saw that there was respite, he hardened his heart, and hearkened not unto them; as the \LORD had said.
\verseWithHeading{The plague of gnats} And the \LORD said unto Moses, Say unto Aaron, Stretch out thy rod, and smite the dust of the land, that it may become lice throughout all the land of Egypt.
\verse And they did so; for Aaron stretched out his hand with his rod, and smote the dust of the earth, and it became lice in man, and in beast; all the dust of the land became lice throughout all the land of Egypt.
\verse And the magicians did so with their enchantments to bring forth lice, but they could not: so there were lice upon man, and upon beast.
\verse Then the magicians said unto Pharaoh, This is the finger of God: and Pharaoh's heart was hardened, and he hearkened not unto them; as the \LORD had said.
\verseWithHeading{The plague of flies} And the \LORD said unto Moses, Rise up early in the morning, and stand before Pharaoh; lo, he cometh forth to the water; and say unto him, Thus saith the \LORD, Let my people go, that they may serve me.
\verse Else, if thou wilt not let my people go, behold, I will send swarms of flies upon thee, and upon thy servants, and upon thy people, and into thy houses: and the houses of the Egyptians shall be full of swarms of flies, and also the ground whereon they are.
\verse And I will sever in that day the land of Goshen, in which my people dwell, that no swarms of flies shall be there; to the end thou mayest know that I am the \LORD in the midst of the earth.
\verse And I will put a division between my people and thy people: to morrow shall this sign be.
\verse And the \LORD did so; and there came a grievous swarm of flies into the house of Pharaoh, and into his servants' houses, and into all the land of Egypt: the land was corrupted by reason of the swarm of flies.
\verse And Pharaoh called for Moses and for Aaron, and said, Go ye, sacrifice to your God in the land.
\verse And Moses said, It is not meet so to do; for we shall sacrifice the abomination of the Egyptians to the \LORD our God: lo, shall we sacrifice the abomination of the Egyptians before their eyes, and will they not stone us?
\verse We will go three days' journey into the wilderness, and sacrifice to the \LORD our God, as he shall command us.
\verse And Pharaoh said, I will let you go, that ye may sacrifice to the \LORD your God in the wilderness; only ye shall not go very far away: intreat for me.
\verse And Moses said, Behold, I go out from thee, and I will intreat the \LORD that the swarms of flies may depart from Pharaoh, from his servants, and from his people, to morrow: but let not Pharaoh deal deceitfully any more in not letting the people go to sacrifice to the \LORD.
\verse And Moses went out from Pharaoh, and intreated the \LORD.
\verse And the \LORD did according to the word of Moses; and he removed the swarms of flies from Pharaoh, from his servants, and from his people; there remained not one.
\verse And Pharaoh hardened his heart at this time also, neither would he let the people go.
\end{biblechapter}

\begin{biblechapter} % Exodus 9
\verseWithHeading{The plague on livestock} Then the \LORD said unto Moses, Go in unto Pharaoh, and tell him, Thus saith the \LORD God of the Hebrews, Let my people go, that they may serve me.
\verse For if thou refuse to let them go, and wilt hold them still,
\verse Behold, the hand of the \LORD is upon thy cattle which is in the field, upon the horses, upon the asses, upon the camels, upon the oxen, and upon the sheep: there shall be a very grievous murrain.
\verse And the \LORD shall sever between the cattle of Israel and the cattle of Egypt: and there shall nothing die of all that is the children's of Israel.
\verse And the \LORD appointed a set time, saying, To morrow the \LORD shall do this thing in the land.
\verse And the \LORD did that thing on the morrow, and all the cattle of Egypt died: but of the cattle of the children of Israel died not one.
\verse And Pharaoh sent, and, behold, there was not one of the cattle of the Israelites dead. And the heart of Pharaoh was hardened, and he did not let the people go.
\verseWithHeading{The plague of boils} And the \LORD said unto Moses and unto Aaron, Take to you handfuls of ashes of the furnace, and let Moses sprinkle it toward the heaven in the sight of Pharaoh.
\verse And it shall become small dust in all the land of Egypt, and shall be a boil breaking forth with blains upon man, and upon beast, throughout all the land of Egypt.
\verse And they took ashes of the furnace, and stood before Pharaoh; and Moses sprinkled it up toward heaven; and it became a boil breaking forth with blains upon man, and upon beast.
\verse And the magicians could not stand before Moses because of the boils; for the boil was upon the magicians, and upon all the Egyptians.
\verse And the \LORD hardened the heart of Pharaoh, and he hearkened not unto them; as the \LORD had spoken unto Moses.
\verseWithHeading{The plague of hail} And the \LORD said unto Moses, Rise up early in the morning, and stand before Pharaoh, and say unto him, Thus saith the \LORD God of the Hebrews, Let my people go, that they may serve me.
\verse For I will at this time send all my plagues upon thine heart, and upon thy servants, and upon thy people; that thou mayest know that there is none like me in all the earth.
\verse For now I will stretch out my hand, that I may smite thee and thy people with pestilence; and thou shalt be cut off from the earth.
\verse And in very deed for this cause have I raised thee up, for to shew in thee my power; and that my name may be declared throughout all the earth.
\verse As yet exaltest thou thyself against my people, that thou wilt not let them go?
\verse Behold, to morrow about this time I will cause it to rain a very grievous hail, such as hath not been in Egypt since the foundation thereof even until now.
\verse Send therefore now, and gather thy cattle, and all that thou hast in the field; for upon every man and beast which shall be found in the field, and shall not be brought home, the hail shall come down upon them, and they shall die.
\verse He that feared the word of the \LORD among the servants of Pharaoh made his servants and his cattle flee into the houses:
\verse And he that regarded not the word of the \LORD left his servants and his cattle in the field.
\verse And the \LORD said unto Moses, Stretch forth thine hand toward heaven, that there may be hail in all the land of Egypt, upon man, and upon beast, and upon every herb of the field, throughout the land of Egypt.
\verse And Moses stretched forth his rod toward heaven: and the \LORD sent thunder and hail, and the fire ran along upon the ground; and the \LORD rained hail upon the land of Egypt.
\verse So there was hail, and fire mingled with the hail, very grievous, such as there was none like it in all the land of Egypt since it became a nation.
\verse And the hail smote throughout all the land of Egypt all that was in the field, both man and beast; and the hail smote every herb of the field, and brake every tree of the field.
\verse Only in the land of Goshen, where the children of Israel were, was there no hail.
\verse And Pharaoh sent, and called for Moses and Aaron, and said unto them, I have sinned this time: the \LORD is righteous, and I and my people are wicked.
\verse Intreat the \LORD (for it is enough) that there be no more mighty thunderings and hail; and I will let you go, and ye shall stay no longer.
\verse And Moses said unto him, As soon as I am gone out of the city, I will spread abroad my hands unto the \LORD; and the thunder shall cease, neither shall there be any more hail; that thou mayest know how that the earth is the \LORDs.
\verse But as for thee and thy servants, I know that ye will not yet fear the \LORD God.
\verse And the flax and the barley was smitten: for the barley was in the ear, and the flax was bolled.
\verse But the wheat and the rie were not smitten: for they were not grown up.
\verse And Moses went out of the city from Pharaoh, and spread abroad his hands unto the \LORD: and the thunders and hail ceased, and the rain was not poured upon the earth.
\verse And when Pharaoh saw that the rain and the hail and the thunders were ceased, he sinned yet more, and hardened his heart, he and his servants.
\verse And the heart of Pharaoh was hardened, neither would he let the children of Israel go; as the \LORD had spoken by Moses.
\end{biblechapter}

\begin{biblechapter} % Exodus 10
\verseWithHeading{The plague of locusts} And the \LORD said unto Moses, Go in unto Pharaoh: for I have hardened his heart, and the heart of his servants, that I might shew these my signs before him:
\verse And that thou mayest tell in the ears of thy son, and of thy son's son, what things I have wrought in Egypt, and my signs which I have done among them; that ye may know how that I am the \LORD.
\verse And Moses and Aaron came in unto Pharaoh, and said unto him, Thus saith the \LORD God of the Hebrews, How long wilt thou refuse to humble thyself before me? let my people go, that they may serve me.
\verse Else, if thou refuse to let my people go, behold, to morrow will I bring the locusts into thy coast:
\verse And they shall cover the face of the earth, that one cannot be able to see the earth: and they shall eat the residue of that which is escaped, which remaineth unto you from the hail, and shall eat every tree which groweth for you out of the field:
\verse And they shall fill thy houses, and the houses of all thy servants, and the houses of all the Egyptians; which neither thy fathers, nor thy fathers' fathers have seen, since the day that they were upon the earth unto this day. And he turned himself, and went out from Pharaoh.
\verse And Pharaoh's servants said unto him, How long shall this man be a snare unto us? let the men go, that they may serve the \LORD their God: knowest thou not yet that Egypt is destroyed?
\verse And Moses and Aaron were brought again unto Pharaoh: and he said unto them, Go, serve the \LORD your God: but who are they that shall go?
\verse And Moses said, We will go with our young and with our old, with our sons and with our daughters, with our flocks and with our herds will we go; for we must hold a feast unto the \LORD.
\verse And he said unto them, Let the \LORD be so with you, as I will let you go, and your little ones: look to it; for evil is before you.
\verse Not so: go now ye that are men, and serve the \LORD; for that ye did desire. And they were driven out from Pharaoh's presence.
\verse And the \LORD said unto Moses, Stretch out thine hand over the land of Egypt for the locusts, that they may come up upon the land of Egypt, and eat every herb of the land, even all that the hail hath left.
\verse And Moses stretched forth his rod over the land of Egypt, and the \LORD brought an east wind upon the land all that day, and all that night; and when it was morning, the east wind brought the locusts.
\verse And the locusts went up over all the land of Egypt, and rested in all the coasts of Egypt: very grievous were they; before them there were no such locusts as they, neither after them shall be such.
\verse For they covered the face of the whole earth, so that the land was darkened; and they did eat every herb of the land, and all the fruit of the trees which the hail had left: and there remained not any green thing in the trees, or in the herbs of the field, through all the land of Egypt.
\verse Then Pharaoh called for Moses and Aaron in haste; and he said, I have sinned against the \LORD your God, and against you.
\verse Now therefore forgive, I pray thee, my sin only this once, and intreat the \LORD your God, that he may take away from me this death only.
\verse And he went out from Pharaoh, and intreated the \LORD.
\verse And the \LORD turned a mighty strong west wind, which took away the locusts, and cast them into the Red sea; there remained not one locust in all the coasts of Egypt.
\verse But the \LORD hardened Pharaoh's heart, so that he would not let the children of Israel go.
\verseWithHeading{The plague of darkness} And the \LORD said unto Moses, Stretch out thine hand toward heaven, that there may be darkness over the land of Egypt, even darkness which may be felt.
\verse And Moses stretched forth his hand toward heaven; and there was a thick darkness in all the land of Egypt three days:
\verse They saw not one another, neither rose any from his place for three days: but all the children of Israel had light in their dwellings.
\verse And Pharaoh called unto Moses, and said, Go ye, serve the \LORD; only let your flocks and your herds be stayed: let your little ones also go with you.
\verse And Moses said, Thou must give us also sacrifices and burnt offerings, that we may sacrifice unto the \LORD our God.
\verse Our cattle also shall go with us; there shall not an hoof be left behind; for thereof must we take to serve the \LORD our God; and we know not with what we must serve the \LORD, until we come thither.
\verse But the \LORD hardened Pharaoh's heart, and he would not let them go.
\verse And Pharaoh said unto him, Get thee from me, take heed to thyself, see my face no more; for in that day thou seest my face thou shalt die.
\verse And Moses said, Thou hast spoken well, I will see thy face again no more.
\end{biblechapter}

\columnbreak % layout hack

\begin{biblechapter} % Exodus 11
\verseWithHeading{The plague on the firstborn} And the \LORD said unto Moses, Yet will I bring one plague more upon Pharaoh, and upon Egypt; afterwards he will let you go hence: when he shall let you go, he shall surely thrust you out hence altogether.
\verse Speak now in the ears of the people, and let every man borrow of his neighbour, and every woman of her neighbour, jewels of silver, and jewels of gold.
\verse And the \LORD gave the people favour in the sight of the Egyptians. Moreover the man Moses was very great in the land of Egypt, in the sight of Pharaoh's servants, and in the sight of the people.
\verse And Moses said, Thus saith the \LORD, About midnight will I go out into the midst of Egypt:
\verse And all the firstborn in the land of Egypt shall die, from the firstborn of Pharaoh that sitteth upon his throne, even unto the firstborn of the maidservant that is behind the mill; and all the firstborn of beasts.
\verse And there shall be a great cry throughout all the land of Egypt, such as there was none like it, nor shall be like it any more.
\verse But against any of the children of Israel shall not a dog move his tongue, against man or beast: that ye may know how that the \LORD doth put a difference between the Egyptians and Israel.
\verse And all these thy servants shall come down unto me, and bow down themselves unto me, saying, Get thee out, and all the people that follow thee: and after that I will go out. And he went out from Pharaoh in a great anger.
\verse And the \LORD said unto Moses, Pharaoh shall not hearken unto you; that my wonders may be multiplied in the land of Egypt.
\verse And Moses and Aaron did all these wonders before Pharaoh: and the \LORD hardened Pharaoh's heart, so that he would not let the children of Israel go out of his land.
\end{biblechapter}

\begin{biblechapter} % Exodus 12
\verseWithHeading{The Passover} And the \LORD spake unto Moses and Aaron in the land of Egypt, saying,
\verse This month shall be unto you the beginning of months: it shall be the first month of the year to you.
\verse Speak ye unto all the congregation of Israel, saying, In the tenth day of this month they shall take to them every man a lamb, according to the house of their fathers, a lamb for an house:
\verse And if the household be too little for the lamb, let him and his neighbour next unto his house take it according to the number of the souls; every man according to his eating shall make your count for the lamb.
\verse Your lamb shall be without blemish, a male of the first year: ye shall take it out from the sheep, or from the goats:
\verse And ye shall keep it up until the fourteenth day of the same month: and the whole assembly of the congregation of Israel shall kill it in the evening.
\verse And they shall take of the blood, and strike it on the two side posts and on the upper door post of the houses, wherein they shall eat it.
\verse And they shall eat the flesh in that night, roast with fire, and unleavened bread; and with bitter herbs they shall eat it.
\verse Eat not of it raw, nor sodden at all with water, but roast with fire; his head with his legs, and with the purtenance thereof.
\verse And ye shall let nothing of it remain until the morning; and that which remaineth of it until the morning ye shall burn with fire.
\verse And thus shall ye eat it; with your loins girded, your shoes on your feet, and your staff in your hand; and ye shall eat it in haste: it is the \LORDs Passover.
\verse For I will pass through the land of Egypt this night, and will smite all the firstborn in the land of Egypt, both man and beast; and against all the gods of Egypt I will execute judgment: I am the \LORD.
\verse And the blood shall be to you for a token upon the houses where ye are: and when I see the blood, I will pass over you, and the plague shall not be upon you to destroy you, when I smite the land of Egypt.
\verse And this day shall be unto you for a memorial; and ye shall keep it a feast to the \LORD throughout your generations; ye shall keep it a feast by an ordinance for ever.
\verse Seven days shall ye eat unleavened bread; even the first day ye shall put away leaven out of your houses: for whosoever eateth leavened bread from the first day until the seventh day, that soul shall be cut off from Israel.
\verse And in the first day there shall be an holy convocation, and in the seventh day there shall be an holy convocation to you; no manner of work shall be done in them, save that which every man must eat, that only may be done of you.
\verse And ye shall observe the feast of unleavened bread; for in this selfsame day have I brought your armies out of the land of Egypt: therefore shall ye observe this day in your generations by an ordinance for ever.
\verse In the first month, on the fourteenth day of the month at even, ye shall eat unleavened bread, until the one and twentieth day of the month at even.
\verse Seven days shall there be no leaven found in your houses: for whosoever eateth that which is leavened, even that soul shall be cut off from the congregation of Israel, whether he be a stranger, or born in the land.
\verse Ye shall eat nothing leavened; in all your habitations shall ye eat unleavened bread.
\verse Then Moses called for all the elders of Israel, and said unto them, Draw out and take you a lamb according to your families, and kill the Passover.
\verse And ye shall take a bunch of hyssop, and dip it in the blood that is in the bason, and strike the lintel and the two side posts with the blood that is in the bason; and none of you shall go out at the door of his house until the morning.
\verse For the \LORD will pass through to smite the Egyptians; and when he seeth the blood upon the lintel, and on the two side posts, the \LORD will pass over the door, and will not suffer the destroyer to come in unto your houses to smite you.
\verse And ye shall observe this thing for an ordinance to thee and to thy sons for ever.
\verse And it shall come to pass, when ye be come to the land which the \LORD will give you, according as he hath promised, that ye shall keep this service.
\verse And it shall come to pass, when your children shall say unto you, What mean ye by this service?
\verse That ye shall say, It is the sacrifice of the \LORDs Passover, who passed over the houses of the children of Israel in Egypt, when he smote the Egyptians, and delivered our houses. And the people bowed the head and worshipped.
\verse And the children of Israel went away, and did as the \LORD had commanded Moses and Aaron, so did they.
\verse And it came to pass, that at midnight the \LORD smote all the firstborn in the land of Egypt, from the firstborn of Pharaoh that sat on his throne unto the firstborn of the captive that was in the dungeon; and all the firstborn of cattle.
\verse And Pharaoh rose up in the night, he, and all his servants, and all the Egyptians; and there was a great cry in Egypt; for there was not a house where there was not one dead.
\verseWithHeading{The exodus} And he called for Moses and Aaron by night, and said, Rise up, and get you forth from among my people, both ye and the children of Israel; and go, serve the \LORD, as ye have said.
\verse Also take your flocks and your herds, as ye have said, and be gone; and bless me also.
\verse And the Egyptians were urgent upon the people, that they might send them out of the land in haste; for they said, We be all dead men.
\verse And the people took their dough before it was leavened, their kneadingtroughs being bound up in their clothes upon their shoulders.
\verse And the children of Israel did according to the word of Moses; and they borrowed of the Egyptians jewels of silver, and jewels of gold, and raiment:
\verse And the \LORD gave the people favour in the sight of the Egyptians, so that they lent unto them such things as they required. And they spoiled the Egyptians.
\verse And the children of Israel journeyed from Rameses to Succoth, about six hundred thousand on foot that were men, beside children.
\verse And a mixed multitude went up also with them; and flocks, and herds, even very much cattle.
\verse And they baked unleavened cakes of the dough which they brought forth out of Egypt, for it was not leavened; because they were thrust out of Egypt, and could not tarry, neither had they prepared for themselves any victual.
\verse Now the sojourning of the children of Israel, who dwelt in Egypt, was four hundred and thirty years.
\verse And it came to pass at the end of the four hundred and thirty years, even the selfsame day it came to pass, that all the hosts of the \LORD went out from the land of Egypt.
\verse It is a night to be much observed unto the \LORD for bringing them out from the land of Egypt: this is that night of the \LORD to be observed of all the children of Israel in their generations.
\verseWithHeading{Passover restrictions} And the \LORD said unto Moses and Aaron, This is the ordinance of the Passover: There shall no stranger eat thereof:
\verse But every man's servant that is bought for money, when thou hast circumcised him, then shall he eat thereof.
\verse A foreigner and an hired servant shall not eat thereof.
\verse In one house shall it be eaten; thou shalt not carry forth ought of the flesh abroad out of the house; neither shall ye break a bone thereof.
\verse All the congregation of Israel shall keep it.
\verse And when a stranger shall sojourn with thee, and will keep the Passover to the \LORD, let all his males be circumcised, and then let him come near and keep it; and he shall be as one that is born in the land: for no uncircumcised person shall eat thereof.
\verse One law shall be to him that is homeborn, and unto the stranger that sojourneth among you.
\verse Thus did all the children of Israel; as the \LORD commanded Moses and Aaron, so did they.
\verse And it came to pass the selfsame day, that the \LORD did bring the children of Israel out of the land of Egypt by their armies.
\end{biblechapter}

\begin{biblechapter} % Exodus 13
\verseWithHeading{Consecration of the firstborn} And the \LORD spake unto Moses, saying,
\verse Sanctify unto me all the firstborn, whatsoever openeth the womb among the children of Israel, both of man and of beast: it is mine.
\verse And Moses said unto the people, Remember this day, in which ye came out from Egypt, out of the house of bondage; for by strength of hand the \LORD brought you out from this place: there shall no leavened bread be eaten.
\verse This day came ye out in the month Abib.
\verse And it shall be when the \LORD shall bring thee into the land of the Canaanites, and the Hittites, and the Amorites, and the Hivites, and the Jebusites, which he sware unto thy fathers to give thee, a land flowing with milk and honey, that thou shalt keep this service in this month.
\verse Seven days thou shalt eat unleavened bread, and in the seventh day shall be a feast to the \LORD.
\verse Unleavened bread shall be eaten seven days; and there shall no leavened bread be seen with thee, neither shall there be leaven seen with thee in all thy quarters.
\verse And thou shalt shew thy son in that day, saying, This is done because of that which the \LORD did unto me when I came forth out of Egypt.
\verse And it shall be for a sign unto thee upon thine hand, and for a memorial between thine eyes, that the \LORDs law may be in thy mouth: for with a strong hand hath the \LORD brought thee out of Egypt.
\verse Thou shalt therefore keep this ordinance in his season from year to year.
\verse And it shall be when the \LORD shall bring thee into the land of the Canaanites, as he sware unto thee and to thy fathers, and shall give it thee,
\verse That thou shalt set apart unto the \LORD all that openeth the matrix, and every firstling that cometh of a beast which thou hast; the males shall be the \LORDs.
\verse And every firstling of an ass thou shalt redeem with a lamb; and if thou wilt not redeem it, then thou shalt break his neck: and all the firstborn of man among thy children shalt thou redeem.
\verse And it shall be when thy son asketh thee in time to come, saying, What is this? that thou shalt say unto him, By strength of hand the \LORD brought us out from Egypt, from the house of bondage:
\verse And it came to pass, when Pharaoh would hardly let us go, that the \LORD slew all the firstborn in the land of Egypt, both the firstborn of man, and the firstborn of beast: therefore I sacrifice to the \LORD all that openeth the matrix, being males; but all the firstborn of my children I redeem.
\verse And it shall be for a token upon thine hand, and for frontlets between thine eyes: for by strength of hand the \LORD brought us forth out of Egypt.
\verseWithHeading{Crossing the sea} And it came to pass, when Pharaoh had let the people go, that God led them not through the way of the land of the Philistines, although that was near; for God said, Lest peradventure the people repent when they see war, and they return to Egypt:
\verse But God led the people about, through the way of the wilderness of the Red sea: and the children of Israel went up harnessed out of the land of Egypt.
\verse And Moses took the bones of Joseph with him: for he had straitly sworn the children of Israel, saying, God will surely visit you; and ye shall carry up my bones away hence with you.
\verse And they took their journey from Succoth, and encamped in Etham, in the edge of the wilderness.
\verse And the \LORD went before them by day in a pillar of a cloud, to lead them the way; and by night in a pillar of fire, to give them light; to go by day and night:
\verse He took not away the pillar of the cloud by day, nor the pillar of fire by night, from before the people.
\end{biblechapter}

\begin{biblechapter} % Exodus 14
\verse And the \LORD spake unto Moses, saying,
\verse Speak unto the children of Israel, that they turn and encamp before Pihahiroth, between Migdol and the sea, over against Baalzephon: before it shall ye encamp by the sea.
\verse For Pharaoh will say of the children of Israel, They are entangled in the land, the wilderness hath shut them in.
\verse And I will harden Pharaoh's heart, that he shall follow after them; and I will be honoured upon Pharaoh, and upon all his host; that the Egyptians may know that I am the \LORD. And they did so.
\verse And it was told the king of Egypt that the people fled: and the heart of Pharaoh and of his servants was turned against the people, and they said, Why have we done this, that we have let Israel go from serving us?
\verse And he made ready his chariot, and took his people with him:
\verse And he took six hundred chosen chariots, and all the chariots of Egypt, and captains over every one of them.
\verse And the \LORD hardened the heart of Pharaoh king of Egypt, and he pursued after the children of Israel: and the children of Israel went out with an high hand.
\verse But the Egyptians pursued after them, all the horses and chariots of Pharaoh, and his horsemen, and his army, and overtook them encamping by the sea, beside Pihahiroth, before Baalzephon.
\verse And when Pharaoh drew nigh, the children of Israel lifted up their eyes, and, behold, the Egyptians marched after them; and they were sore afraid: and the children of Israel cried out unto the \LORD.
\verse And they said unto Moses, Because there were no graves in Egypt, hast thou taken us away to die in the wilderness? wherefore hast thou dealt thus with us, to carry us forth out of Egypt?
\verse Is not this the word that we did tell thee in Egypt, saying, Let us alone, that we may serve the Egyptians? For it had been better for us to serve the Egyptians, than that we should die in the wilderness.
\verse And Moses said unto the people, Fear ye not, stand still, and see the salvation of the \LORD, which he will shew to you to day: for the Egyptians whom ye have seen to day, ye shall see them again no more for ever.
\verse The \LORD shall fight for you, and ye shall hold your peace.
\verse And the \LORD said unto Moses, Wherefore criest thou unto me? speak unto the children of Israel, that they go forward:
\verse But lift thou up thy rod, and stretch out thine hand over the sea, and divide it: and the children of Israel shall go on dry ground through the midst of the sea.
\verse And I, behold, I will harden the hearts of the Egyptians, and they shall follow them: and I will get me honour upon Pharaoh, and upon all his host, upon his chariots, and upon his horsemen.
\verse And the Egyptians shall know that I am the \LORD, when I have gotten me honour upon Pharaoh, upon his chariots, and upon his horsemen.
\verse And the angel of God, which went before the camp of Israel, removed and went behind them; and the pillar of the cloud went from before their face, and stood behind them:
\verse And it came between the camp of the Egyptians and the camp of Israel; and it was a cloud and darkness to them, but it gave light by night to these: so that the one came not near the other all the night.
\verse And Moses stretched out his hand over the sea; and the \LORD caused the sea to go back by a strong east wind all that night, and made the sea dry land, and the waters were divided.
\verse And the children of Israel went into the midst of the sea upon the dry ground: and the waters were a wall unto them on their right hand, and on their left.
\verse And the Egyptians pursued, and went in after them to the midst of the sea, even all Pharaoh's horses, his chariots, and his horsemen.
\verse And it came to pass, that in the morning watch the \LORD looked unto the host of the Egyptians through the pillar of fire and of the cloud, and troubled the host of the Egyptians,
\verse And took off their chariot wheels, that they drave them heavily: so that the Egyptians said, Let us flee from the face of Israel; for the \LORD fighteth for them against the Egyptians.
\verse And the \LORD said unto Moses, Stretch out thine hand over the sea, that the waters may come again upon the Egyptians, upon their chariots, and upon their horsemen.
\verse And Moses stretched forth his hand over the sea, and the sea returned to his strength when the morning appeared; and the Egyptians fled against it; and the \LORD overthrew the Egyptians in the midst of the sea.
\verse And the waters returned, and covered the chariots, and the horsemen, and all the host of Pharaoh that came into the sea after them; there remained not so much as one of them.
\verse But the children of Israel walked upon dry land in the midst of the sea; and the waters were a wall unto them on their right hand, and on their left.
\verse Thus the \LORD saved Israel that day out of the hand of the Egyptians; and Israel saw the Egyptians dead upon the sea shore.
\verse And Israel saw that great work which the \LORD did upon the Egyptians: and the people feared the \LORD, and believed the \LORD, and his servant Moses.
\end{biblechapter}

\begin{biblechapter} % Exodus 15
\verseWithHeading{The song of Moses and \newline Miriam} Then sang Moses and the children of Israel this song unto the \LORD, and spake, saying, I will sing unto the \LORD, for he hath triumphed gloriously: the horse and his rider hath he thrown into the sea.
\verse The \LORD is my strength and song, and he is become my salvation: he is my God, and I will prepare him an habitation; my father's God, and I will exalt him.
\verse The \LORD is a man of war: the \LORD is his name.
\verse Pharaoh's chariots and his host hath he cast into the sea: his chosen captains also are drowned in the Red sea.
\verse The depths have covered them: they sank into the bottom as a stone.
\verse Thy right hand, O \LORD, is become glorious in power: thy right hand, O \LORD, hath dashed in pieces the enemy.
\verse And in the greatness of thine excellency thou hast overthrown them that rose up against thee: thou sentest forth thy wrath, which consumed them as stubble.
\verse And with the blast of thy nostrils the waters were gathered together, the floods stood upright as an heap, and the depths were congealed in the heart of the sea.
\verse The enemy said, I will pursue, I will overtake, I will divide the spoil; my lust shall be satisfied upon them; I will draw my sword, my hand shall destroy them.
\verse Thou didst blow with thy wind, the sea covered them: they sank as lead in the mighty waters.
\verse Who is like unto thee, O \LORD, among the gods? who is like thee, glorious in holiness, fearful in praises, doing wonders?
\verse Thou stretchedst out thy right hand, the earth swallowed them.
\verse Thou in thy mercy hast led forth the people which thou hast redeemed: thou hast guided them in thy strength unto thy holy habitation.
\verse The people shall hear, and be afraid: sorrow shall take hold on the inhabitants of Palestina.
\verse Then the dukes of Edom shall be amazed; the mighty men of Moab, trembling shall take hold upon them; all the inhabitants of Canaan shall melt away.
\verse Fear and dread shall fall upon them; by the greatness of thine arm they shall be as still as a stone; till thy people pass over, O \LORD, till the people pass over, which thou hast purchased.
\verse Thou shalt bring them in, and plant them in the mountain of thine inheritance, in the place, O \LORD, which thou hast made for thee to dwell in, in the Sanctuary, O Lord, which thy hands have established.
\verse The \LORD shall reign for ever and ever.
\verse For the horse of Pharaoh went in with his chariots and with his horsemen into the sea, and the \LORD brought again the waters of the sea upon them; but the children of Israel went on dry land in the midst of the sea.
\verse And Miriam the prophetess, the sister of Aaron, took a timbrel in her hand; and all the women went out after her with timbrels and with dances.
\verse And Miriam answered them, Sing ye to the \LORD, for he hath triumphed gloriously; the horse and his rider hath he thrown into the sea.
\verseWithHeading{The waters of Marah and Elim} So Moses brought Israel from the Red sea, and they went out into the wilderness of Shur; and they went three days in the wilderness, and found no water.
\verse And when they came to Marah, they could not drink of the waters of Marah, for they were bitter: therefore the name of it was called Marah.
\verse And the people murmured against Moses, saying, What shall we drink?
\verse And he cried unto the \LORD; and the \LORD shewed him a tree, which when he had cast into the waters, the waters were made sweet: there he made for them a statute and an ordinance, and there he proved them,
\verse And said, If thou wilt diligently hearken to the voice of the \LORD thy God, and wilt do that which is right in his sight, and wilt give ear to his commandments, and keep all his statutes, I will put none of these diseases upon thee, which I have brought upon the Egyptians: for I am the \LORD that healeth thee.
\verse And they came to Elim, where were twelve wells of water, and threescore and ten palm trees: and they encamped there by the waters.
\end{biblechapter}

\begin{biblechapter} % Exodus 16
\verseWithHeading{Manna and quail} And they took their journey from Elim, and all the congregation of the children of Israel came unto the wilderness of Sin, which is between Elim and Sinai, on the fifteenth day of the second month after their departing out of the land of Egypt.
\verse And the whole congregation of the children of Israel murmured against Moses and Aaron in the wilderness:
\verse And the children of Israel said unto them, Would to God we had died by the hand of the \LORD in the land of Egypt, when we sat by the flesh pots, and when we did eat bread to the full; for ye have brought us forth into this wilderness, to kill this whole assembly with hunger.
\verse Then said the \LORD unto Moses, Behold, I will rain bread from heaven for you; and the people shall go out and gather a certain rate every day, that I may prove them, whether they will walk in my law, or no.
\verse And it shall come to pass, that on the sixth day they shall prepare that which they bring in; and it shall be twice as much as they gather daily.
\verse And Moses and Aaron said unto all the children of Israel, At even, then ye shall know that the \LORD hath brought you out from the land of Egypt:
\verse And in the morning, then ye shall see the glory of the \LORD; for that he heareth your murmurings against the \LORD: and what are we, that ye murmur against us?
\verse And Moses said, This shall be, when the \LORD shall give you in the evening flesh to eat, and in the morning bread to the full; for that the \LORD heareth your murmurings which ye murmur against him: and what are we? your murmurings are not against us, but against the \LORD.
\verse And Moses spake unto Aaron, Say unto all the congregation of the children of Israel, Come near before the \LORD: for he hath heard your murmurings.
\verse And it came to pass, as Aaron spake unto the whole congregation of the children of Israel, that they looked toward the wilderness, and, behold, the glory of the \LORD appeared in the cloud.
\verse And the \LORD spake unto Moses, saying,
\verse I have heard the murmurings of the children of Israel: speak unto them, saying, At even ye shall eat flesh, and in the morning ye shall be filled with bread; and ye shall know that I am the \LORD your God.
\verse And it came to pass, that at even the quails came up, and covered the camp: and in the morning the dew lay round about the host.
\verse And when the dew that lay was gone up, behold, upon the face of the wilderness there lay a small round thing, as small as the hoar frost on the ground.
\verse And when the children of Israel saw it, they said one to another, It is manna: for they wist not what it was. And Moses said unto them, This is the bread which the \LORD hath given you to eat.
\verse This is the thing which the \LORD hath commanded, Gather of it every man according to his eating, an omer for every man, according to the number of your persons; take ye every man for them which are in his tents.
\verse And the children of Israel did so, and gathered, some more, some less.
\verse And when they did mete it with an omer, he that gathered much had nothing over, and he that gathered little had no lack; they gathered every man according to his eating.
\verse And Moses said, Let no man leave of it till the morning.
\verse Notwithstanding they hearkened not unto Moses; but some of them left of it until the morning, and it bred worms, and stank: and Moses was wroth with them.
\verse And they gathered it every morning, every man according to his eating: and when the sun waxed hot, it melted.
\verse And it came to pass, that on the sixth day they gathered twice as much bread, two omers for one man: and all the rulers of the congregation came and told Moses.
\verse And he said unto them, This is that which the \LORD hath said, To morrow is the rest of the holy Sabbath unto the \LORD: bake that which ye will bake to day, and seethe that ye will seethe; and that which remaineth over lay up for you to be kept until the morning.
\verse And they laid it up till the morning, as Moses bade: and it did not stink, neither was there any worm therein.
\verse And Moses said, Eat that to day; for to day is a Sabbath unto the \LORD: to day ye shall not find it in the field.
\verse Six days ye shall gather it; but on the seventh day, which is the Sabbath, in it there shall be none.
\verse And it came to pass, that there went out some of the people on the seventh day for to gather, and they found none.
\verse And the \LORD said unto Moses, How long refuse ye to keep my commandments and my laws?
\verse See, for that the \LORD hath given you the Sabbath, therefore he giveth you on the sixth day the bread of two days; abide ye every man in his place, let no man go out of his place on the seventh day.
\verse So the people rested on the seventh day.
\verse And the house of Israel called the name thereof Manna: and it was like coriander seed, white; and the taste of it was like wafers made with honey.
\verse And Moses said, This is the thing which the \LORD commandeth, Fill an omer of it to be kept for your generations; that they may see the bread wherewith I have fed you in the wilderness, when I brought you forth from the land of Egypt.
\verse And Moses said unto Aaron, Take a pot, and put an omer full of manna therein, and lay it up before the \LORD, to be kept for your generations.
\verse As the \LORD commanded Moses, so Aaron laid it up before the Testimony, to be kept.
\verse And the children of Israel did eat manna forty years, until they came to a land inhabited; they did eat manna, until they came unto the borders of the land of Canaan.
\verse Now an omer is the tenth part of an ephah.
\end{biblechapter}

\begin{biblechapter} % Exodus 17
\verseWithHeading{Water from the rock} And all the congregation of the children of Israel journeyed from the wilderness of Sin, after their journeys, according to the commandment of the \LORD, and pitched in Rephidim: and there was no water for the people to drink.
\verse Wherefore the people did chide with Moses, and said, Give us water that we may drink. And Moses said unto them, Why chide ye with me? wherefore do ye tempt the \LORD?
\verse And the people thirsted there for water; and the people murmured against Moses, and said, Wherefore is this that thou hast brought us up out of Egypt, to kill us and our children and our cattle with thirst?
\verse And Moses cried unto the \LORD, saying, What shall I do unto this people? they be almost ready to stone me.
\verse And the \LORD said unto Moses, Go on before the people, and take with thee of the elders of Israel; and thy rod, wherewith thou smotest the river, take in thine hand, and go.
\verse Behold, I will stand before thee there upon the rock in Horeb; and thou shalt smite the rock, and there shall come water out of it, that the people may drink. And Moses did so in the sight of the elders of Israel.
\verse And he called the name of the place Massah, and Meribah, because of the chiding of the children of Israel, and because they tempted the \LORD, saying, Is the \LORD among us, or not?
\verseWithHeading{The Amalekites defeated} Then came Amalek, and fought with Israel in Rephidim.
\verse And Moses said unto Joshua, Choose us out men, and go out, fight with Amalek: to morrow I will stand on the top of the hill with the rod of God in mine hand.
\verse So Joshua did as Moses had said to him, and fought with Amalek: and Moses, Aaron, and Hur went up to the top of the hill.
\verse And it came to pass, when Moses held up his hand, that Israel prevailed: and when he let down his hand, Amalek prevailed.
\verse But Moses' hands were heavy; and they took a stone, and put it under him, and he sat thereon; and Aaron and Hur stayed up his hands, the one on the one side, and the other on the other side; and his hands were steady until the going down of the sun.
\verse And Joshua discomfited Amalek and his people with the edge of the sword.
\verse And the \LORD said unto Moses, Write this for a memorial in a book, and rehearse it in the ears of Joshua: for I will utterly put out the remembrance of Amalek from under heaven.
\verse And Moses built an altar, and called the name of it Jehovahnissi:
\verse For he said, Because the \LORD hath sworn that the \LORD will have war with Amalek from generation to generation.
\end{biblechapter}

\begin{biblechapter} % Exodus 18
\verseWithHeading{Jethro visits Moses} When Jethro, the priest of Midian, Moses' father in law, heard of all that God had done for Moses, and for Israel his people, and that the \LORD had brought Israel out of Egypt;
\verse Then Jethro, Moses' father in law, took Zipporah, Moses' wife, after he had sent her back,
\verse And her two sons; of which the name of the one was Gershom; for he said, I have been an alien in a strange land:
\verse And the name of the other was Eliezer; for the God of my father, said he, was mine help, and delivered me from the sword of Pharaoh:
\verse And Jethro, Moses' father in law, came with his sons and his wife unto Moses into the wilderness, where he encamped at the mount of God:
\verse And he said unto Moses, I thy father in law Jethro am come unto thee, and thy wife, and her two sons with her.
\verse And Moses went out to meet his father in law, and did obeisance, and kissed him; and they asked each other of their welfare; and they came into the tent.
\verse And Moses told his father in law all that the \LORD had done unto Pharaoh and to the Egyptians for Israel's sake, and all the travail that had come upon them by the way, and how the \LORD delivered them.
\verse And Jethro rejoiced for all the goodness which the \LORD had done to Israel, whom he had delivered out of the hand of the Egyptians.
\verse And Jethro said, Blessed be the \LORD, who hath delivered you out of the hand of the Egyptians, and out of the hand of Pharaoh, who hath delivered the people from under the hand of the Egyptians.
\verse Now I know that the \LORD is greater than all gods: for in the thing wherein they dealt proudly he was above them.
\verse And Jethro, Moses' father in law, took a burnt offering and sacrifices for God: and Aaron came, and all the elders of Israel, to eat bread with Moses' father in law before God.
\verse And it came to pass on the morrow, that Moses sat to judge the people: and the people stood by Moses from the morning unto the evening.
\verse And when Moses' father in law saw all that he did to the people, he said, What is this thing that thou doest to the people? why sittest thou thyself alone, and all the people stand by thee from morning unto even?
\verse And Moses said unto his father in law, Because the people come unto me to enquire of God:
\verse When they have a matter, they come unto me; and I judge between one and another, and I do make them know the statutes of God, and his laws.
\verse And Moses' father in law said unto him, The thing that thou doest is not good.
\verse Thou wilt surely wear away, both thou, and this people that is with thee: for this thing is too heavy for thee; thou art not able to perform it thyself alone.
\verse Hearken now unto my voice, I will give thee counsel, and God shall be with thee: Be thou for the people to God-ward, that thou mayest bring the causes unto God:
\verse And thou shalt teach them ordinances and laws, and shalt shew them the way wherein they must walk, and the work that they must do.
\verse Moreover thou shalt provide out of all the people able men, such as fear God, men of truth, hating covetousness; and place such over them, to be rulers of thousands, and rulers of hundreds, rulers of fifties, and rulers of tens:
\verse And let them judge the people at all seasons: and it shall be, that every great matter they shall bring unto thee, but every small matter they shall judge: so shall it be easier for thyself, and they shall bear the burden with thee.
\verse If thou shalt do this thing, and God command thee so, then thou shalt be able to endure, and all this people shall also go to their place in peace.
\verse So Moses hearkened to the voice of his father in law, and did all that he had said.
\verse And Moses chose able men out of all Israel, and made them heads over the people, rulers of thousands, rulers of hundreds, rulers of fifties, and rulers of tens.
\verse And they judged the people at all seasons: the hard causes they brought unto Moses, but every small matter they judged themselves.
\verse And Moses let his father in law depart; and he went his way into his own land.
\end{biblechapter}

\begin{biblechapter} % Exodus 19
\verseWithHeading{At Mount Sinai} In the third month, when the children of Israel were gone forth out of the land of Egypt, the same day came they into the wilderness of Sinai.
\verse For they were departed from Rephidim, and were come to the desert of Sinai, and had pitched in the wilderness; and there Israel camped before the mount.
\verse And Moses went up unto God, and the \LORD called unto him out of the mountain, saying, Thus shalt thou say to the house of Jacob, and tell the children of Israel;
\verse Ye have seen what I did unto the Egyptians, and how I bare you on eagles' wings, and brought you unto myself.
\verse Now therefore, if ye will obey my voice indeed, and keep my covenant, then ye shall be a peculiar treasure unto me above all people: for all the earth is mine:
\verse And ye shall be unto me a kingdom of priests, and an holy nation. These are the words which thou shalt speak unto the children of Israel.
\verse And Moses came and called for the elders of the people, and laid before their faces all these words which the \LORD commanded him.
\verse And all the people answered together, and said, All that the \LORD hath spoken we will do. And Moses returned the words of the people unto the \LORD.
\verse And the \LORD said unto Moses, Lo, I come unto thee in a thick cloud, that the people may hear when I speak with thee, and believe thee for ever. And Moses told the words of the people unto the \LORD.
\verse And the \LORD said unto Moses, Go unto the people, and sanctify them to day and to morrow, and let them wash their clothes,
\verse And be ready against the third day: for the third day the \LORD will come down in the sight of all the people upon mount Sinai.
\verse And thou shalt set bounds unto the people round about, saying, Take heed to yourselves, that ye go not up into the mount, or touch the border of it: whosoever toucheth the mount shall be surely put to death:
\verse There shall not an hand touch it, but he shall surely be stoned, or shot through; whether it be beast or man, it shall not live: when the trumpet soundeth long, they shall come up to the mount.
\verse And Moses went down from the mount unto the people, and sanctified the people; and they washed their clothes.
\verse And he said unto the people, Be ready against the third day: come not at your wives.
\verse And it came to pass on the third day in the morning, that there were thunders and lightnings, and a thick cloud upon the mount, and the voice of the trumpet exceeding loud; so that all the people that was in the camp trembled.
\verse And Moses brought forth the people out of the camp to meet with God; and they stood at the nether part of the mount.
\verse And mount Sinai was altogether on a smoke, because the \LORD descended upon it in fire: and the smoke thereof ascended as the smoke of a furnace, and the whole mount quaked greatly.
\verse And when the voice of the trumpet sounded long, and waxed louder and louder, Moses spake, and God answered him by a voice.
\verse And the \LORD came down upon mount Sinai, on the top of the mount: and the \LORD called Moses up to the top of the mount; and Moses went up.
\verse And the \LORD said unto Moses, Go down, charge the people, lest they break through unto the \LORD to gaze, and many of them perish.
\verse And let the priests also, which come near to the \LORD, sanctify themselves, lest the \LORD break forth upon them.
\verse And Moses said unto the \LORD, The people cannot come up to mount Sinai: for thou chargedst us, saying, Set bounds about the mount, and sanctify it.
\verse And the \LORD said unto him, Away, get thee down, and thou shalt come up, thou, and Aaron with thee: but let not the priests and the people break through to come up unto the \LORD, lest he break forth upon them.
\verse So Moses went down unto the people, and spake unto them.
\end{biblechapter}

\flushcolsend\columnbreak % layout hack

\begin{biblechapter} % Exodus 20
\verseWithHeading{The Ten Commandments} And God spake all these words, saying,
\verse I am the \LORD thy God, which have brought thee out of the land of Egypt, out of the house of bondage.
\verse Thou shalt have no other gods before me.
\verse Thou shalt not make unto thee any graven image, or any likeness of any thing that is in heaven above, or that is in the earth beneath, or that is in the water under the earth:
\verse Thou shalt not bow down thyself to them, nor serve them: for I the \LORD thy God am a jealous God, visiting the iniquity of the fathers upon the children unto the third and fourth generation of them that hate me;
\verse And shewing mercy unto thousands of them that love me, and keep my commandments.
\verse Thou shalt not take the name of the \LORD thy God in vain; for the \LORD will not hold him guiltless that taketh his name in vain.
\verse Remember the Sabbath day, to keep it holy.
\verse Six days shalt thou labour, and do all thy work:
\verse But the seventh day is the Sabbath of the \LORD thy God: in it thou shalt not do any work, thou, nor thy son, nor thy daughter, thy manservant, nor thy maidservant, nor thy cattle, nor thy stranger that is within thy gates:
\verse For in six days the \LORD made heaven and earth, the sea, and all that in them is, and rested the seventh day: wherefore the \LORD blessed the Sabbath day, and hallowed it.
\verse Honour thy father and thy mother: that thy days may be long upon the land which the \LORD thy God giveth thee.
\verse Thou shalt not kill.
\verse Thou shalt not commit adultery.
\verse Thou shalt not steal.
\verse Thou shalt not bear false witness against thy neighbour.
\verse Thou shalt not covet thy neighbour's house, thou shalt not covet thy neighbour's wife, nor his manservant, nor his maidservant, nor his ox, nor his ass, nor any thing that is thy neighbour's.
\verse And all the people saw the thunderings, and the lightnings, and the noise of the trumpet, and the mountain smoking: and when the people saw it, they removed, and stood afar off.
\verse And they said unto Moses, Speak thou with us, and we will hear: but let not God speak with us, lest we die.
\verse And Moses said unto the people, Fear not: for God is come to prove you, and that his fear may be before your faces, that ye sin not.
\verse And the people stood afar off, and Moses drew near unto the thick darkness where God was.
\verseWithHeading{Idols and altars} And the \LORD said unto Moses, Thus thou shalt say unto the children of Israel, Ye have seen that I have talked with you from heaven.
\verse Ye shall not make with me gods of silver, neither shall ye make unto you gods of gold.
\verse An altar of earth thou shalt make unto me, and shalt sacrifice thereon thy burnt offerings, and thy peace offerings, thy sheep, and thine oxen: in all places where I record my name I will come unto thee, and I will bless thee.
\verse And if thou wilt make me an altar of stone, thou shalt not build it of hewn stone: for if thou lift up thy tool upon it, thou hast polluted it.
\verse Neither shalt thou go up by steps unto mine altar, that thy nakedness be not discovered thereon.
\end{biblechapter}

\begin{biblechapter} % Exodus 21
\verse Now these are the judgments which thou shalt set before them.
\verseWithHeading{Hebrew servants} If thou buy an Hebrew servant, six years he shall serve: and in the seventh he shall go out free for nothing.
\verse If he came in by himself, he shall go out by himself: if he were married, then his wife shall go out with him.
\verse If his master have given him a wife, and she have born him sons or daughters; the wife and her children shall be her master's, and he shall go out by himself.
\verse And if the servant shall plainly say, I love my master, my wife, and my children; I will not go out free:
\verse Then his master shall bring him unto the judges; he shall also bring him to the door, or unto the door post; and his master shall bore his ear through with an aul; and he shall serve him for ever.
\verse And if a man sell his daughter to be a maidservant, she shall not go out as the menservants do.
\verse If she please not her master, who hath betrothed her to himself, then shall he let her be redeemed: to sell her unto a strange nation he shall have no power, seeing he hath dealt deceitfully with her.
\verse And if he have betrothed her unto his son, he shall deal with her after the manner of daughters.
\verse If he take him another wife; her food, her raiment, and her duty of marriage, shall he not diminish.
\verse And if he do not these three unto her, then shall she go out free without money.
\verseWithHeading{Personal injuries} He that smiteth a man, so that he die, shall be surely put to death.
\verse And if a man lie not in wait, but God deliver him into his hand; then I will appoint thee a place whither he shall flee.
\verse But if a man come presumptuously upon his neighbour, to slay him with guile; thou shalt take him from mine altar, that he may die.
\verse And he that smiteth his father, or his mother, shall be surely put to death.
\verse And he that stealeth a man, and selleth him, or if he be found in his hand, he shall surely be put to death.
\verse And he that curseth his father, or his mother, shall surely be put to death.
\verse And if men strive together, and one smite another with a stone, or with his fist, and he die not, but keepeth his bed:
\verse If he rise again, and walk abroad upon his staff, then shall he that smote him be quit: only he shall pay for the loss of his time, and shall cause him to be thoroughly healed.
\verse And if a man smite his servant, or his maid, with a rod, and he die under his hand; he shall be surely punished.
\verse Notwithstanding, if he continue a day or two, he shall not be punished: for he is his money.
\verse If men strive, and hurt a woman with child, so that her fruit depart from her, and yet no mischief follow: he shall be surely punished, according as the woman's husband will lay upon him; and he shall pay as the judges determine.
\verse And if any mischief follow, then thou shalt give life for life,
\verse Eye for eye, tooth for tooth, hand for hand, foot for foot,
\verse Burning for burning, wound for wound, stripe for stripe.
\verse And if a man smite the eye of his servant, or the eye of his maid, that it perish; he shall let him go free for his eye's sake.
\verse And if he smite out his manservant's tooth, or his maidservant's tooth; he shall let him go free for his tooth's sake.
\verse If an ox gore a man or a woman, that they die: then the ox shall be surely stoned, and his flesh shall not be eaten; but the owner of the ox shall be quit.
\verse But if the ox were wont to push with his horn in time past, and it hath been testified to his owner, and he hath not kept him in, but that he hath killed a man or a woman; the ox shall be stoned, and his owner also shall be put to death.
\verse If there be laid on him a sum of money, then he shall give for the ransom of his life whatsoever is laid upon him.
\verse Whether he have gored a son, or have gored a daughter, according to this judgment shall it be done unto him.
\verse If the ox shall push a manservant or a maidservant; he shall give unto their master thirty shekels of silver, and the ox shall be stoned.
\verse And if a man shall open a pit, or if a man shall dig a pit, and not cover it, and an ox or an ass fall therein;
\verse The owner of the pit shall make it good, and give money unto the owner of them; and the dead beast shall be his.
\verse And if one man's ox hurt another's, that he die; then they shall sell the live ox, and divide the money of it; and the dead ox also they shall divide.
\verse Or if it be known that the ox hath used to push in time past, and his owner hath not kept him in; he shall surely pay ox for ox; and the dead shall be his own.
\end{biblechapter}

\begin{biblechapter} % Exodus 22
\verseWithHeading{Protection of property} If a man shall steal an ox, or a sheep, and kill it, or sell it; he shall restore five oxen for an ox, and four sheep for a sheep.
\verse If a thief be found breaking up, and be smitten that he die, there shall no blood be shed for him.
\verse If the sun be risen upon him, there shall be blood shed for him; for he should make full restitution; if he have nothing, then he shall be sold for his theft.
\verse If the theft be certainly found in his hand alive, whether it be ox, or ass, or sheep; he shall restore double.
\verse If a man shall cause a field or vineyard to be eaten, and shall put in his beast, and shall feed in another man's field; of the best of his own field, and of the best of his own vineyard, shall he make restitution.
\verse If fire break out, and catch in thorns, so that the stacks of corn, or the standing corn, or the field, be consumed therewith; he that kindled the fire shall surely make restitution.
\verse If a man shall deliver unto his neighbour money or stuff to keep, and it be stolen out of the man's house; if the thief be found, let him pay double.
\verse If the thief be not found, then the master of the house shall be brought unto the judges, to see whether he have put his hand unto his neighbour's goods.
\verse For all manner of trespass, whether it be for ox, for ass, for sheep, for raiment, or for any manner of lost thing, which another challengeth to be his, the cause of both parties shall come before the judges; and whom the judges shall condemn, he shall pay double unto his neighbour.
\verse If a man deliver unto his neighbour an ass, or an ox, or a sheep, or any beast, to keep; and it die, or be hurt, or driven away, no man seeing it:
\verse Then shall an oath of the \LORD be between them both, that he hath not put his hand unto his neighbour's goods; and the owner of it shall accept thereof, and he shall not make it good.
\verse And if it be stolen from him, he shall make restitution unto the owner thereof.
\verse If it be torn in pieces, then let him bring it for witness, and he shall not make good that which was torn.
\verse And if a man borrow ought of his neighbour, and it be hurt, or die, the owner thereof being not with it, he shall surely make it good.
\verse But if the owner thereof be with it, he shall not make it good: if it be an hired thing, it came for his hire.
\verseWithHeading{Social responsibility} And if a man entice a maid that is not betrothed, and lie with her, he shall surely endow her to be his wife.
\verse If her father utterly refuse to give her unto him, he shall pay money according to the dowry of virgins.
\verse Thou shalt not suffer a witch to live.
\verse Whosoever lieth with a beast shall surely be put to death.
\verse He that sacrificeth unto any god, save unto the \LORD only, he shall be utterly destroyed.
\verse Thou shalt neither vex a stranger, nor oppress him: for ye were strangers in the land of Egypt.
\verse Ye shall not afflict any widow, or fatherless child.
\verse If thou afflict them in any wise, and they cry at all unto me, I will surely hear their cry;
\verse And my wrath shall wax hot, and I will kill you with the sword; and your wives shall be widows, and your children fatherless.
\verse If thou lend money to any of my people that is poor by thee, thou shalt not be to him as an usurer, neither shalt thou lay upon him usury.
\verse If thou at all take thy neighbour's raiment to pledge, thou shalt deliver it unto him by that the sun goeth down:
\verse For that is his covering only, it is his raiment for his skin: wherein shall he sleep? and it shall come to pass, when he crieth unto me, that I will hear; for I am gracious.
\verse Thou shalt not revile the gods, nor curse the ruler of thy people.
\verse Thou shalt not delay to offer the first of thy ripe fruits, and of thy liquors: the firstborn of thy sons shalt thou give unto me.
\verse Likewise shalt thou do with thine oxen, and with thy sheep: seven days it shall be with his dam; on the eighth day thou shalt give it me.
\verse And ye shall be holy men unto me: neither shall ye eat any flesh that is torn of beasts in the field; ye shall cast it to the dogs.
\end{biblechapter}

\begin{biblechapter} % Exodus 23
\verseWithHeading{Justice and mercy} Thou shalt not raise a false report: put not thine hand with the wicked to be an unrighteous witness.
\verse Thou shalt not follow a multitude to do evil; neither shalt thou speak in a cause to decline after many to wrest judgment:
\verse Neither shalt thou countenance a poor man in his cause.
\verse If thou meet thine enemy's ox or his ass going astray, thou shalt surely bring it back to him again.
\verse If thou see the ass of him that hateth thee lying under his burden, and wouldest forbear to help him, thou shalt surely help with him.
\verse Thou shalt not wrest the judgment of thy poor in his cause.
\verse Keep thee far from a false matter; and the innocent and righteous slay thou not: for I will not justify the wicked.
\verse And thou shalt take no gift: for the gift blindeth the wise, and perverteth the words of the righteous.
\verse Also thou shalt not oppress a stranger: for ye know the heart of a stranger, seeing ye were strangers in the land of Egypt.
\columnbreak % layout hack
\verseWithHeading{The Sabbath} And six years thou shalt sow thy land, and shalt gather in the fruits thereof:
\verse But the seventh year thou shalt let it rest and lie still; that the poor of thy people may eat: and what they leave the beasts of the field shall eat. In like manner thou shalt deal with thy vineyard, and with thy oliveyard.
\verse Six days thou shalt do thy work, and on the seventh day thou shalt rest: that thine ox and thine ass may rest, and the son of thy handmaid, and the stranger, may be refreshed.
\verse And in all things that I have said unto you be circumspect: and make no mention of the name of other gods, neither let it be heard out of thy mouth.
\verseWithHeading{Annual festivals} Three times thou shalt keep a feast unto me in the year.
\verse Thou shalt keep the feast of unleavened bread: (thou shalt eat unleavened bread seven days, as I commanded thee, in the time appointed of the month Abib; for in it thou camest out from Egypt: and none shall appear before me empty:)
\verse And the feast of harvest, the firstfruits of thy labours, which thou hast sown in the field: and the feast of ingathering, which is in the end of the year, when thou hast gathered in thy labours out of the field.
\verse Three times in the year all thy males shall appear before the Lord God.
\verse Thou shalt not offer the blood of my sacrifice with leavened bread; neither shall the fat of my sacrifice remain until the morning.
\verse The first of the firstfruits of thy land thou shalt bring into the house of the \LORD thy God. Thou shalt not seethe a kid in his mother's milk.
\verseWithHeading{God's angel to prepare the way} Behold, I send an Angel before thee, to keep thee in the way, and to bring thee into the place which I have prepared.
\verse Beware of him, and obey his voice, provoke him not; for he will not pardon your transgressions: for my name is in him.
\verse But if thou shalt indeed obey his voice, and do all that I speak; then I will be an enemy unto thine enemies, and an adversary unto thine adversaries.
\verse For mine Angel shall go before thee, and bring thee in unto the Amorites, and the Hittites, and the Perizzites, and the Canaanites, the Hivites, and the Jebusites: and I will cut them off.
\verse Thou shalt not bow down to their gods, nor serve them, nor do after their works: but thou shalt utterly overthrow them, and quite break down their images.
\verse And ye shall serve the \LORD your God, and he shall bless thy bread, and thy water; and I will take sickness away from the midst of thee.
\verse There shall nothing cast their young, nor be barren, in thy land: the number of thy days I will fulfil.
\verse I will send my fear before thee, and will destroy all the people to whom thou shalt come, and I will make all thine enemies turn their backs unto thee.
\verse And I will send hornets before thee, which shall drive out the Hivite, the Canaanite, and the Hittite, from before thee.
\verse I will not drive them out from before thee in one year; lest the land become desolate, and the beast of the field multiply against thee.
\verse By little and little I will drive them out from before thee, until thou be increased, and inherit the land.
\verse And I will set thy bounds from the Red sea even unto the sea of the Philistines, and from the desert unto the river: for I will deliver the inhabitants of the land into your hand; and thou shalt drive them out before thee.
\verse Thou shalt make no covenant with them, nor with their gods.
\verse They shall not dwell in thy land, lest they make thee sin against me: for if thou serve their gods, it will surely be a snare unto thee.
\end{biblechapter}

\begin{biblechapter} % Exodus 24
\verseWithHeading{The covenant confirmed} And he said unto Moses, Come up unto the \LORD, thou, and Aaron, Nadab, and Abihu, and seventy of the elders of Israel; and worship ye afar off.
\verse And Moses alone shall come near the \LORD: but they shall not come nigh; neither shall the people go up with him.
\verse And Moses came and told the people all the words of the \LORD, and all the judgments: and all the people answered with one voice, and said, All the words which the \LORD hath said will we do.
\verse And Moses wrote all the words of the \LORD, and rose up early in the morning, and builded an altar under the hill, and twelve pillars, according to the twelve tribes of Israel.
\verse And he sent young men of the children of Israel, which offered burnt offerings, and sacrificed peace offerings of oxen unto the \LORD.
\verse And Moses took half of the blood, and put it in basons; and half of the blood he sprinkled on the altar.
\verse And he took the book of the covenant, and read in the audience of the people: and they said, All that the \LORD hath said will we do, and be obedient.
\verse And Moses took the blood, and sprinkled it on the people, and said, Behold the blood of the covenant, which the \LORD hath made with you concerning all these words.
\verse Then went up Moses, and Aaron, Nadab, and Abihu, and seventy of the elders of Israel:
\verse And they saw the God of Israel: and there was under his feet as it were a paved work of a sapphire stone, and as it were the body of heaven in his clearness.
\verse And upon the nobles of the children of Israel he laid not his hand: also they saw God, and did eat and drink.
\verse And the \LORD said unto Moses, Come up to me into the mount, and be there: and I will give thee tables of stone, and a law, and commandments which I have written; that thou mayest teach them.
\verse And Moses rose up, and his minister Joshua: and Moses went up into the mount of God.
\verse And he said unto the elders, Tarry ye here for us, until we come again unto you: and, behold, Aaron and Hur are with you: if any man have any matters to do, let him come unto them.
\verse And Moses went up into the mount, and a cloud covered the mount.
\verse And the glory of the \LORD abode upon mount Sinai, and the cloud covered it six days: and the seventh day he called unto Moses out of the midst of the cloud.
\verse And the sight of the glory of the \LORD was like devouring fire on the top of the mount in the eyes of the children of Israel.
\verse And Moses went into the midst of the cloud, and gat him up into the mount: and Moses was in the mount forty days and forty nights.
\end{biblechapter}

\begin{biblechapter} % Exodus 25
\verseWithHeading{Offerings for the tabernacle} And the \LORD spake unto Moses, saying,
\verse Speak unto the children of Israel, that they bring me an offering: of every man that giveth it willingly with his heart ye shall take my offering.
\verse And this is the offering which ye shall take of them; gold, and silver, and brass,
\verse And blue, and purple, and scarlet, and fine linen, and goats' hair,
\verse And rams' skins dyed red, and badgers' skins, and shittim wood,
\verse Oil for the light, spices for anointing oil, and for sweet incense,
\verse Onyx stones, and stones to be set in the ephod, and in the breastplate.
\verse And let them make me a sanctuary; that I may dwell among them.
\verse According to all that I shew thee, after the pattern of the tabernacle, and the pattern of all the instruments thereof, even so shall ye make it.
\verseWithHeading{The ark} And they shall make an ark of shittim wood: two cubits and a half shall be the length thereof, and a cubit and a half the breadth thereof, and a cubit and a half the height thereof.
\verse And thou shalt overlay it with pure gold, within and without shalt thou overlay it, and shalt make upon it a crown of gold round about.
\verse And thou shalt cast four rings of gold for it, and put them in the four corners thereof; and two rings shall be in the one side of it, and two rings in the other side of it.
\verse And thou shalt make staves of shittim wood, and overlay them with gold.
\verse And thou shalt put the staves into the rings by the sides of the ark, that the ark may be borne with them.
\verse The staves shall be in the rings of the ark: they shall not be taken from it.
\verse And thou shalt put into the ark the testimony which I shall give thee.
\verse And thou shalt make a mercy seat of pure gold: two cubits and a half shall be the length thereof, and a cubit and a half the breadth thereof.
\verse And thou shalt make two cherubims of gold, of beaten work shalt thou make them, in the two ends of the mercy seat.
\verse And make one cherub on the one end, and the other cherub on the other end: even of the mercy seat shall ye make the cherubims on the two ends thereof.
\verse And the cherubims shall stretch forth their wings on high, covering the mercy seat with their wings, and their faces shall look one to another; toward the mercy seat shall the faces of the cherubims be.
\verse And thou shalt put the mercy seat above upon the ark; and in the ark thou shalt put the testimony that I shall give thee.
\verse And there I will meet with thee, and I will commune with thee from above the mercy seat, from between the two cherubims which are upon the ark of the testimony, of all things which I will give thee in commandment unto the children of Israel.
\verseWithHeading{The table} Thou shalt also make a table of shittim wood: two cubits shall be the length thereof, and a cubit the breadth thereof, and a cubit and a half the height thereof.
\verse And thou shalt overlay it with pure gold, and make thereto a crown of gold round about.
\verse And thou shalt make unto it a border of an hand breadth round about, and thou shalt make a golden crown to the border thereof round about.
\verse And thou shalt make for it four rings of gold, and put the rings in the four corners that are on the four feet thereof.
\verse Over against the border shall the rings be for places of the staves to bear the table.
\verse And thou shalt make the staves of shittim wood, and overlay them with gold, that the table may be borne with them.
\verse And thou shalt make the dishes thereof, and spoons thereof, and covers thereof, and bowls thereof, to cover withal: of pure gold shalt thou make them.
\verse And thou shalt set upon the table shewbread before me alway.
\verseWithHeading{The candlestick} And thou shalt make a candlestick of pure gold: of beaten work shall the candlestick be made: his shaft, and his branches, his bowls, his knops, and his flowers, shall be of the same.
\verse And six branches shall come out of the sides of it; three branches of the candlestick out of the one side, and three branches of the candlestick out of the other side:
\verse Three bowls made like unto almonds, with a knop and a flower in one branch; and three bowls made like almonds in the other branch, with a knop and a flower: so in the six branches that come out of the candlestick.
\verse And in the candlestick shall be four bowls made like unto almonds, with their knops and their flowers.
\verse And there shall be a knop under two branches of the same, and a knop under two branches of the same, and a knop under two branches of the same, according to the six branches that proceed out of the candlestick.
\verse Their knops and their branches shall be of the same: all it shall be one beaten work of pure gold.
\verse And thou shalt make the seven lamps thereof: and they shall light the lamps thereof, that they may give light over against it.
\verse And the tongs thereof, and the snuffdishes thereof, shall be of pure gold.
\verse Of a talent of pure gold shall he make it, with all these vessels.
\verse And look that thou make them after their pattern, which was shewed thee in the mount.
\end{biblechapter}

\begin{biblechapter} % Exodus 26
\verseWithHeading{The tabernacle} Moreover thou shalt make the tabernacle with ten curtains of fine twined linen, and blue, and purple, and scarlet: with cherubims of cunning work shalt thou make them.
\verse The length of one curtain shall be eight and twenty cubits, and the breadth of one curtain four cubits: and every one of the curtains shall have one measure.
\verse The five curtains shall be coupled together one to another; and other five curtains shall be coupled one to another.
\verse And thou shalt make loops of blue upon the edge of the one curtain from the selvedge in the coupling; and likewise shalt thou make in the uttermost edge of another curtain, in the coupling of the second.
\verse Fifty loops shalt thou make in the one curtain, and fifty loops shalt thou make in the edge of the curtain that is in the coupling of the second; that the loops may take hold one of another.
\verse And thou shalt make fifty taches of gold, and couple the curtains together with the taches: and it shall be one tabernacle.
\verse And thou shalt make curtains of goats' hair to be a covering upon the tabernacle: eleven curtains shalt thou make.
\verse The length of one curtain shall be thirty cubits, and the breadth of one curtain four cubits: and the eleven curtains shall be all of one measure.
\verse And thou shalt couple five curtains by themselves, and six curtains by themselves, and shalt double the sixth curtain in the forefront of the tabernacle.
\verse And thou shalt make fifty loops on the edge of the one curtain that is outmost in the coupling, and fifty loops in the edge of the curtain which coupleth the second.
\verse And thou shalt make fifty taches of brass, and put the taches into the loops, and couple the tent together, that it may be one.
\verse And the remnant that remaineth of the curtains of the tent, the half curtain that remaineth, shall hang over the backside of the tabernacle.
\verse And a cubit on the one side, and a cubit on the other side of that which remaineth in the length of the curtains of the tent, it shall hang over the sides of the tabernacle on this side and on that side, to cover it.
\verse And thou shalt make a covering for the tent of rams' skins dyed red, and a covering above of badgers' skins.
\verse And thou shalt make boards for the tabernacle of shittim wood standing up.
\verse Ten cubits shall be the length of a board, and a cubit and a half shall be the breadth of one board.
\verse Two tenons shall there be in one board, set in order one against another: thus shalt thou make for all the boards of the tabernacle.
\verse And thou shalt make the boards for the tabernacle, twenty boards on the south side southward.
\verse And thou shalt make forty sockets of silver under the twenty boards; two sockets under one board for his two tenons, and two sockets under another board for his two tenons.
\verse And for the second side of the tabernacle on the north side there shall be twenty boards:
\verse And their forty sockets of silver; two sockets under one board, and two sockets under another board.
\verse And for the sides of the tabernacle westward thou shalt make six boards.
\verse And two boards shalt thou make for the corners of the tabernacle in the two sides.
\verse And they shall be coupled together beneath, and they shall be coupled together above the head of it unto one ring: thus shall it be for them both; they shall be for the two corners.
\verse And they shall be eight boards, and their sockets of silver, sixteen sockets; two sockets under one board, and two sockets under another board.
\verse And thou shalt make bars of shittim wood; five for the boards of the one side of the tabernacle,
\verse And five bars for the boards of the other side of the tabernacle, and five bars for the boards of the side of the tabernacle, for the two sides westward.
\verse And the middle bar in the midst of the boards shall reach from end to end.
\verse And thou shalt overlay the boards with gold, and make their rings of gold for places for the bars: and thou shalt overlay the bars with gold.
\verse And thou shalt rear up the tabernacle according to the fashion thereof which was shewed thee in the mount.
\verse And thou shalt make a vail of blue, and purple, and scarlet, and fine twined linen of cunning work: with cherubims shall it be made:
\verse And thou shalt hang it upon four pillars of shittim wood overlaid with gold: their hooks shall be of gold, upon the four sockets of silver.
\verse And thou shalt hang up the vail under the taches, that thou mayest bring in thither within the vail the ark of the testimony: and the vail shall divide unto you between the holy place and the most holy.
\verse And thou shalt put the mercy seat upon the ark of the testimony in the most holy place.
\verse And thou shalt set the table without the vail, and the candlestick over against the table on the side of the tabernacle toward the south: and thou shalt put the table on the north side.
\verse And thou shalt make an hanging for the door of the tent, of blue, and purple, and scarlet, and fine twined linen, wrought with needlework.
\verse And thou shalt make for the hanging five pillars of shittim wood, and overlay them with gold, and their hooks shall be of gold: and thou shalt cast five sockets of brass for them.
\end{biblechapter}

\begin{biblechapter} % Exodus 27
\verseWithHeading{The altar of burnt offering} And thou shalt make an altar of shittim wood, five cubits long, and five cubits broad; the altar shall be foursquare: and the height thereof shall be three cubits.
\verse And thou shalt make the horns of it upon the four corners thereof: his horns shall be of the same: and thou shalt overlay it with brass.
\verse And thou shalt make his pans to receive his ashes, and his shovels, and his basons, and his fleshhooks, and his firepans: all the vessels thereof thou shalt make of brass.
\verse And thou shalt make for it a grate of network of brass; and upon the net shalt thou make four brasen rings in the four corners thereof.
\verse And thou shalt put it under the compass of the altar beneath, that the net may be even to the midst of the altar.
\verse And thou shalt make staves for the altar, staves of shittim wood, and overlay them with brass.
\verse And the staves shall be put into the rings, and the staves shall be upon the two sides of the altar, to bear it.
\verse Hollow with boards shalt thou make it: as it was shewed thee in the mount, so shall they make it.
\verseWithHeading{The court} And thou shalt make the court of the tabernacle: for the south side southward there shall be hangings for the court of fine twined linen of an hundred cubits long for one side:
\verse And the twenty pillars thereof and their twenty sockets shall be of brass; the hooks of the pillars and their fillets shall be of silver.
\verse And likewise for the north side in length there shall be hangings of an hundred cubits long, and his twenty pillars and their twenty sockets of brass; the hooks of the pillars and their fillets of silver.
\verse And for the breadth of the court on the west side shall be hangings of fifty cubits: their pillars ten, and their sockets ten.
\verse And the breadth of the court on the east side eastward shall be fifty cubits.
\verse The hangings of one side of the gate shall be fifteen cubits: their pillars three, and their sockets three.
\verse And on the other side shall be hangings fifteen cubits: their pillars three, and their sockets three.
\verse And for the gate of the court shall be an hanging of twenty cubits, of blue, and purple, and scarlet, and fine twined linen, wrought with needlework: and their pillars shall be four, and their sockets four.
\verse All the pillars round about the court shall be filleted with silver; their hooks shall be of silver, and their sockets of brass.
\verse The length of the court shall be an hundred cubits, and the breadth fifty every where, and the height five cubits of fine twined linen, and their sockets of brass.
\verse All the vessels of the tabernacle in all the service thereof, and all the pins thereof, and all the pins of the court, shall be of brass.
\verseWithHeading{Oil for the light} And thou shalt command the children of Israel, that they bring thee pure oil olive beaten for the light, to cause the lamp to burn always.
\verse In the tabernacle of the congregation without the vail, which is before the testimony, Aaron and his sons shall order it from evening to morning before the \LORD: it shall be a statute for ever unto their generations on the behalf of the children of Israel.
\end{biblechapter}

\begin{biblechapter} % Exodus 28
\verseWithHeading{The priestly garments} And take thou unto thee Aaron thy brother, and his sons with him, from among the children of Israel, that he may minister unto me in the priest's office, even Aaron, Nadab and Abihu, Eleazar and Ithamar, Aaron's sons.
\verse And thou shalt make holy garments for Aaron thy brother for glory and for beauty.
\verse And thou shalt speak unto all that are wise hearted, whom I have filled with the spirit of wisdom, that they may make Aaron's garments to consecrate him, that he may minister unto me in the priest's office.
\verse And these are the garments which they shall make; a breastplate, and an ephod, and a robe, and a broidered coat, a mitre, and a girdle: and they shall make holy garments for Aaron thy brother, and his sons, that he may minister unto me in the priest's office.
\verse And they shall take gold, and blue, and purple, and scarlet, and fine linen.
\verseWithHeading{The ephod} And they shall make the ephod of gold, of blue, and of purple, of scarlet, and fine twined linen, with cunning work.
\verse It shall have the two shoulderpieces thereof joined at the two edges thereof; and so it shall be joined together.
\verse And the curious girdle of the ephod, which is upon it, shall be of the same, according to the work thereof; even of gold, of blue, and purple, and scarlet, and fine twined linen.
\verse And thou shalt take two onyx stones, and grave on them the names of the children of Israel:
\verse Six of their names on one stone, and the other six names of the rest on the other stone, according to their birth.
\verse With the work of an engraver in stone, like the engravings of a signet, shalt thou engrave the two stones with the names of the children of Israel: thou shalt make them to be set in ouches of gold.
\verse And thou shalt put the two stones upon the shoulders of the ephod for stones of memorial unto the children of Israel: and Aaron shall bear their names before the \LORD upon his two shoulders for a memorial.
\verse And thou shalt make ouches of gold;
\verse And two chains of pure gold at the ends; of wreathen work shalt thou make them, and fasten the wreathen chains to the ouches.
\verseWithHeading{The breastplate} And thou shalt make the breastplate of judgment with cunning work; after the work of the ephod thou shalt make it; of gold, of blue, and of purple, and of scarlet, and of fine twined linen, shalt thou make it.
\verse Foursquare it shall be being doubled; a span shall be the length thereof, and a span shall be the breadth thereof.
\verse And thou shalt set in it settings of stones, even four rows of stones: the first row shall be a sardius, a topaz, and a carbuncle: this shall be the first row.
\verse And the second row shall be an emerald, a sapphire, and a diamond.
\verse And the third row a ligure, an agate, and an amethyst.
\verse And the fourth row a beryl, and an onyx, and a jasper: they shall be set in gold in their inclosings.
\verse And the stones shall be with the names of the children of Israel, twelve, according to their names, like the engravings of a signet; every one with his name shall they be according to the twelve tribes.
\verse And thou shalt make upon the breastplate chains at the ends of wreathen work of pure gold.
\verse And thou shalt make upon the breastplate two rings of gold, and shalt put the two rings on the two ends of the breastplate.
\verse And thou shalt put the two wreathen chains of gold in the two rings which are on the ends of the breastplate.
\verse And the other two ends of the two wreathen chains thou shalt fasten in the two ouches, and put them on the shoulderpieces of the ephod before it.
\verse And thou shalt make two rings of gold, and thou shalt put them upon the two ends of the breastplate in the border thereof, which is in the side of the ephod inward.
\verse And two other rings of gold thou shalt make, and shalt put them on the two sides of the ephod underneath, toward the forepart thereof, over against the other coupling thereof, above the curious girdle of the ephod.
\verse And they shall bind the breastplate by the rings thereof unto the rings of the ephod with a lace of blue, that it may be above the curious girdle of the ephod, and that the breastplate be not loosed from the ephod.
\verse And Aaron shall bear the names of the children of Israel in the breastplate of judgment upon his heart, when he goeth in unto the holy place, for a memorial before the \LORD continually.
\verse And thou shalt put in the breastplate of judgment the Urim and the Thummim; and they shall be upon Aaron's heart, when he goeth in before the \LORD: and Aaron shall bear the judgment of the children of Israel upon his heart before the \LORD continually.
\verseWithHeading{Other priestly garments} And thou shalt make the robe of the ephod all of blue.
\verse And there shall be an hole in the top of it, in the midst thereof: it shall have a binding of woven work round about the hole of it, as it were the hole of an habergeon, that it be not rent.
\verse And beneath upon the hem of it thou shalt make pomegranates of blue, and of purple, and of scarlet, round about the hem thereof; and bells of gold between them round about:
\verse A golden bell and a pomegranate, a golden bell and a pomegranate, upon the hem of the robe round about.
\verse And it shall be upon Aaron to minister: and his sound shall be heard when he goeth in unto the holy place before the \LORD, and when he cometh out, that he die not.
\verse And thou shalt make a plate of pure gold, and grave upon it, like the engravings of a signet, \textsc{Holiness to the Lord}.
\verse And thou shalt put it on a blue lace, that it may be upon the mitre; upon the forefront of the mitre it shall be.
\verse And it shall be upon Aaron's forehead, that Aaron may bear the iniquity of the holy things, which the children of Israel shall hallow in all their holy gifts; and it shall be always upon his forehead, that they may be accepted before the \LORD.
\verse And thou shalt embroider the coat of fine linen, and thou shalt make the mitre of fine linen, and thou shalt make the girdle of needlework.
\verse And for Aaron's sons thou shalt make coats, and thou shalt make for them girdles, and bonnets shalt thou make for them, for glory and for beauty.
\verse And thou shalt put them upon Aaron thy brother, and his sons with him; and shalt anoint them, and consecrate them, and sanctify them, that they may minister unto me in the priest's office.
\verse And thou shalt make them linen breeches to cover their nakedness; from the loins even unto the thighs they shall reach:
\verse And they shall be upon Aaron, and upon his sons, when they come in unto the tabernacle of the congregation, or when they come near unto the altar to minister in the holy place; that they bear not iniquity, and die: it shall be a statute for ever unto him and his seed after him.
\end{biblechapter}

\begin{biblechapter} % Exodus 29
\verseWithHeading{Consecration of the priests} And this is the thing that thou shalt do unto them to hallow them, to minister unto me in the priest's office: Take one young bullock, and two rams without blemish,
\verse And unleavened bread, and cakes unleavened tempered with oil, and wafers unleavened anointed with oil: of wheaten flour shalt thou make them.
\verse And thou shalt put them into one basket, and bring them in the basket, with the bullock and the two rams.
\verse And Aaron and his sons thou shalt bring unto the door of the tabernacle of the congregation, and shalt wash them with water.
\verse And thou shalt take the garments, and put upon Aaron the coat, and the robe of the ephod, and the ephod, and the breastplate, and gird him with the curious girdle of the ephod:
\verse And thou shalt put the mitre upon his head, and put the holy crown upon the mitre.
\verse Then shalt thou take the anointing oil, and pour it upon his head, and anoint him.
\verse And thou shalt bring his sons, and put coats upon them.
\verse And thou shalt gird them with girdles, Aaron and his sons, and put the bonnets on them: and the priest's office shall be theirs for a perpetual statute: and thou shalt consecrate Aaron and his sons.
\verse And thou shalt cause a bullock to be brought before the tabernacle of the congregation: and Aaron and his sons shall put their hands upon the head of the bullock.
\verse And thou shalt kill the bullock before the \LORD, by the door of the tabernacle of the congregation.
\verse And thou shalt take of the blood of the bullock, and put it upon the horns of the altar with thy finger, and pour all the blood beside the bottom of the altar.
\verse And thou shalt take all the fat that covereth the inwards, and the caul that is above the liver, and the two kidneys, and the fat that is upon them, and burn them upon the altar.
\verse But the flesh of the bullock, and his skin, and his dung, shalt thou burn with fire without the camp: it is a sin offering.
\verse Thou shalt also take one ram; and Aaron and his sons shall put their hands upon the head of the ram.
\verse And thou shalt slay the ram, and thou shalt take his blood, and sprinkle it round about upon the altar.
\verse And thou shalt cut the ram in pieces, and wash the inwards of him, and his legs, and put them unto his pieces, and unto his head.
\verse And thou shalt burn the whole ram upon the altar: it is a burnt offering unto the \LORD: it is a sweet savour, an offering made by fire unto the \LORD.
\verse And thou shalt take the other ram; and Aaron and his sons shall put their hands upon the head of the ram.
\verse Then shalt thou kill the ram, and take of his blood, and put it upon the tip of the right ear of Aaron, and upon the tip of the right ear of his sons, and upon the thumb of their right hand, and upon the great toe of their right foot, and sprinkle the blood upon the altar round about.
\verse And thou shalt take of the blood that is upon the altar, and of the anointing oil, and sprinkle it upon Aaron, and upon his garments, and upon his sons, and upon the garments of his sons with him: and he shall be hallowed, and his garments, and his sons, and his sons' garments with him.
\verse Also thou shalt take of the ram the fat and the rump, and the fat that covereth the inwards, and the caul above the liver, and the two kidneys, and the fat that is upon them, and the right shoulder; for it is a ram of consecration:
\verse And one loaf of bread, and one cake of oiled bread, and one wafer out of the basket of the unleavened bread that is before the \LORD:
\verse And thou shalt put all in the hands of Aaron, and in the hands of his sons; and shalt wave them for a wave offering before the \LORD.
\verse And thou shalt receive them of their hands, and burn them upon the altar for a burnt offering, for a sweet savour before the \LORD: it is an offering made by fire unto the \LORD.
\verse And thou shalt take the breast of the ram of Aaron's consecration, and wave it for a wave offering before the \LORD: and it shall be thy part.
\verse And thou shalt sanctify the breast of the wave offering, and the shoulder of the heave offering, which is waved, and which is heaved up, of the ram of the consecration, even of that which is for Aaron, and of that which is for his sons:
\verse And it shall be Aaron's and his sons' by a statute for ever from the children of Israel: for it is an heave offering: and it shall be an heave offering from the children of Israel of the sacrifice of their peace offerings, even their heave offering unto the \LORD.
\verse And the holy garments of Aaron shall be his sons' after him, to be anointed therein, and to be consecrated in them.
\verse And that son that is priest in his stead shall put them on seven days, when he cometh into the tabernacle of the congregation to minister in the holy place.
\verse And thou shalt take the ram of the consecration, and seethe his flesh in the holy place.
\verse And Aaron and his sons shall eat the flesh of the ram, and the bread that is in the basket, by the door of the tabernacle of the congregation.
\verse And they shall eat those things wherewith the atonement was made, to consecrate and to sanctify them: but a stranger shall not eat thereof, because they are holy.
\verse And if ought of the flesh of the consecrations, or of the bread, remain unto the morning, then thou shalt burn the remainder with fire: it shall not be eaten, because it is holy.
\verse And thus shalt thou do unto Aaron, and to his sons, according to all things which I have commanded thee: seven days shalt thou consecrate them.
\verse And thou shalt offer every day a bullock for a sin offering for atonement: and thou shalt cleanse the altar, when thou hast made an atonement for it, and thou shalt anoint it, to sanctify it.
\verse Seven days thou shalt make an atonement for the altar, and sanctify it; and it shall be an altar most holy: whatsoever toucheth the altar shall be holy.
\verse Now this is that which thou shalt offer upon the altar; two lambs of the first year day by day continually.
\verse The one lamb thou shalt offer in the morning; and the other lamb thou shalt offer at even:
\verse And with the one lamb a tenth deal of flour mingled with the fourth part of an hin of beaten oil; and the fourth part of an hin of wine for a drink offering.
\verse And the other lamb thou shalt offer at even, and shalt do thereto according to the meat offering of the morning, and according to the drink offering thereof, for a sweet savour, an offering made by fire unto the \LORD.
\verse This shall be a continual burnt offering throughout your generations at the door of the tabernacle of the congregation before the \LORD: where I will meet you, to speak there unto thee.
\verse And there I will meet with the children of Israel, and the tabernacle shall be sanctified by my glory.
\verse And I will sanctify the tabernacle of the congregation, and the altar: I will sanctify also both Aaron and his sons, to minister to me in the priest's office.
\verse And I will dwell among the children of Israel, and will be their God.
\verse And they shall know that I am the \LORD their God, that brought them forth out of the land of Egypt, that I may dwell among them: I am the \LORD their God.
\end{biblechapter}

\begin{biblechapter} % Exodus 30
\verseWithHeading{The altar of incense} And thou shalt make an altar to burn incense upon: of shittim wood shalt thou make it.
\verse A cubit shall be the length thereof, and a cubit the breadth thereof; foursquare shall it be: and two cubits shall be the height thereof: the horns thereof shall be of the same.
\verse And thou shalt overlay it with pure gold, the top thereof, and the sides thereof round about, and the horns thereof; and thou shalt make unto it a crown of gold round about.
\verse And two golden rings shalt thou make to it under the crown of it, by the two corners thereof, upon the two sides of it shalt thou make it; and they shall be for places for the staves to bear it withal.
\verse And thou shalt make the staves of shittim wood, and overlay them with gold.
\verse And thou shalt put it before the vail that is by the ark of the testimony, before the mercy seat that is over the testimony, where I will meet with thee.
\verse And Aaron shall burn thereon sweet incense every morning: when he dresseth the lamps, he shall burn incense upon it.
\verse And when Aaron lighteth the lamps at even, he shall burn incense upon it, a perpetual incense before the \LORD throughout your generations.
\verse Ye shall offer no strange incense thereon, nor burnt sacrifice, nor meat offering; neither shall ye pour drink offering thereon.
\verse And Aaron shall make an atonement upon the horns of it once in a year with the blood of the sin offering of atonements: once in the year shall he make atonement upon it throughout your generations: it is most holy unto the \LORD.
\verseWithHeading{Atonement money} And the \LORD spake unto Moses, saying,
\verse When thou takest the sum of the children of Israel after their number, then shall they give every man a ransom for his soul unto the \LORD, when thou numberest them; that there be no plague among them, when thou numberest them.
\verse This they shall give, every one that passeth among them that are numbered, half a shekel after the shekel of the sanctuary: (a shekel is twenty gerahs:) an half shekel shall be the offering of the \LORD.
\verse Every one that passeth among them that are numbered, from twenty years old and above, shall give an offering unto the \LORD.
\verse The rich shall not give more, and the poor shall not give less than half a shekel, when they give an offering unto the \LORD, to make an atonement for your souls.
\verse And thou shalt take the atonement money of the children of Israel, and shalt appoint it for the service of the tabernacle of the congregation; that it may be a memorial unto the children of Israel before the \LORD, to make an atonement for your souls.
\verseWithHeading{Basin for washing} And the \LORD spake unto Moses, saying,
\verse Thou shalt also make a laver of brass, and his foot also of brass, to wash withal: and thou shalt put it between the tabernacle of the congregation and the altar, and thou shalt put water therein.
\verse For Aaron and his sons shall wash their hands and their feet thereat:
\verse When they go into the tabernacle of the congregation, they shall wash with water, that they die not; or when they come near to the altar to minister, to burn offering made by fire unto the \LORD:
\verse So they shall wash their hands and their feet, that they die not: and it shall be a statute for ever to them, even to him and to his seed throughout their generations.
\verseWithHeading{Anointing oil} Moreover the \LORD spake unto Moses, saying,
\verse Take thou also unto thee principal spices, of pure myrrh five hundred shekels, and of sweet cinnamon half so much, even two hundred and fifty shekels, and of sweet calamus two hundred and fifty shekels,
\verse And of cassia five hundred shekels, after the shekel of the sanctuary, and of oil olive an hin:
\verse And thou shalt make it an oil of holy ointment, an ointment compound after the art of the apothecary: it shall be an holy anointing oil.
\verse And thou shalt anoint the tabernacle of the congregation therewith, and the ark of the testimony,
\verse And the table and all his vessels, and the candlestick and his vessels, and the altar of incense,
\verse And the altar of burnt offering with all his vessels, and the laver and his foot.
\verse And thou shalt sanctify them, that they may be most holy: whatsoever toucheth them shall be holy.
\verse And thou shalt anoint Aaron and his sons, and consecrate them, that they may minister unto me in the priest's office.
\verse And thou shalt speak unto the children of Israel, saying, This shall be an holy anointing oil unto me throughout your generations.
\verse Upon man's flesh shall it not be poured, neither shall ye make any other like it, after the composition of it: it is holy, and it shall be holy unto you.
\verse Whosoever compoundeth any like it, or whosoever putteth any of it upon a stranger, shall even be cut off from his people.
\verseWithHeading{Incense} And the \LORD said unto Moses, Take unto thee sweet spices, stacte, and onycha, and galbanum; these sweet spices with pure frankincense: of each shall there be a like weight:
\verse And thou shalt make it a perfume, a confection after the art of the apothecary, tempered together, pure and holy:
\verse And thou shalt beat some of it very small, and put of it before the testimony in the tabernacle of the congregation, where I will meet with thee: it shall be unto you most holy.
\verse And as for the perfume which thou shalt make, ye shall not make to yourselves according to the composition thereof: it shall be unto thee holy for the \LORD.
\verse Whosoever shall make like unto that, to smell thereto, shall even be cut off from his people.
\end{biblechapter}

\begin{biblechapter} % Exodus 31
\verseWithHeading{Bezaleel and Aholiab} And the \LORD spake unto Moses, saying,
\verse See, I have called by name Bezaleel the son of Uri, the son of Hur, of the tribe of Judah:
\verse And I have filled him with the spirit of God, in wisdom, and in understanding, and in knowledge, and in all manner of workmanship,
\verse To devise cunning works, to work in gold, and in silver, and in brass,
\verse And in cutting of stones, to set them, and in carving of timber, to work in all manner of workmanship.
\verse And I, behold, I have given with him Aholiab, the son of Ahisamach, of the tribe of Dan: and in the hearts of all that are wise hearted I have put wisdom, that they may make all that I have commanded thee;
\verse The tabernacle of the congregation, and the ark of the testimony, and the mercy seat that is thereupon, and all the furniture of the tabernacle,
\verse And the table and his furniture, and the pure candlestick with all his furniture, and the altar of incense,
\verse And the altar of burnt offering with all his furniture, and the laver and his foot,
\verse And the cloths of service, and the holy garments for Aaron the priest, and the garments of his sons, to minister in the priest's office,
\verse And the anointing oil, and sweet incense for the holy place: according to all that I have commanded thee shall they do.
\verseWithHeading{The Sabbath} And the \LORD spake unto Moses, saying,
\verse Speak thou also unto the children of Israel, saying, Verily my Sabbaths ye shall keep: for it is a sign between me and you throughout your generations; that ye may know that I am the \LORD that doth sanctify you.
\verse Ye shall keep the Sabbath therefore; for it is holy unto you: every one that defileth it shall surely be put to death: for whosoever doeth any work therein, that soul shall be cut off from among his people.
\verse Six days may work be done; but in the seventh is the Sabbath of rest, holy to the \LORD: whosoever doeth any work in the Sabbath day, he shall surely be put to death.
\verse Wherefore the children of Israel shall keep the Sabbath, to observe the Sabbath throughout their generations, for a perpetual covenant.
\verse It is a sign between me and the children of Israel for ever: for in six days the \LORD made heaven and earth, and on the seventh day he rested, and was refreshed.
\verse And he gave unto Moses, when he had made an end of communing with him upon mount Sinai, two tables of testimony, tables of stone, written with the finger of God.
\end{biblechapter}

\begin{biblechapter} % Exodus 32
\verseWithHeading{The golden calf} And when the people saw that Moses delayed to come down out of the mount, the people gathered themselves together unto Aaron, and said unto him, Up, make us gods, which shall go before us; for as for this Moses, the man that brought us up out of the land of Egypt, we wot not what is become of him.
\verse And Aaron said unto them, Break off the golden earrings, which are in the ears of your wives, of your sons, and of your daughters, and bring them unto me.
\verse And all the people brake off the golden earrings which were in their ears, and brought them unto Aaron.
\verse And he received them at their hand, and fashioned it with a graving tool, after he had made it a molten calf: and they said, These be thy gods, O Israel, which brought thee up out of the land of Egypt.
\verse And when Aaron saw it, he built an altar before it; and Aaron made proclamation, and said, To morrow is a feast to the \LORD.
\verse And they rose up early on the morrow, and offered burnt offerings, and brought peace offerings; and the people sat down to eat and to drink, and rose up to play.
\verse And the \LORD said unto Moses, Go, get thee down; for thy people, which thou broughtest out of the land of Egypt, have corrupted themselves:
\verse They have turned aside quickly out of the way which I commanded them: they have made them a molten calf, and have worshipped it, and have sacrificed thereunto, and said, These be thy gods, O Israel, which have brought thee up out of the land of Egypt.
\verse And the \LORD said unto Moses, I have seen this people, and, behold, it is a stiffnecked people:
\verse Now therefore let me alone, that my wrath may wax hot against them, and that I may consume them: and I will make of thee a great nation.
\verse And Moses besought the \LORD his God, and said, \LORD, why doth thy wrath wax hot against thy people, which thou hast brought forth out of the land of Egypt with great power, and with a mighty hand?
\verse Wherefore should the Egyptians speak, and say, For mischief did he bring them out, to slay them in the mountains, and to consume them from the face of the earth? Turn from thy fierce wrath, and repent of this evil against thy people.
\verse Remember Abraham, Isaac, and Israel, thy servants, to whom thou swarest by thine own self, and saidst unto them, I will multiply your seed as the stars of heaven, and all this land that I have spoken of will I give unto your seed, and they shall inherit it for ever.
\verse And the \LORD repented of the evil which he thought to do unto his people.
\verse And Moses turned, and went down from the mount, and the two tables of the testimony were in his hand: the tables were written on both their sides; on the one side and on the other were they written.
\verse And the tables were the work of God, and the writing was the writing of God, graven upon the tables.
\verse And when Joshua heard the noise of the people as they shouted, he said unto Moses, There is a noise of war in the camp.
\verse And he said, It is not the voice of them that shout for mastery, neither is it the voice of them that cry for being overcome: but the noise of them that sing do I hear.
\verse And it came to pass, as soon as he came nigh unto the camp, that he saw the calf, and the dancing: and Moses' anger waxed hot, and he cast the tables out of his hands, and brake them beneath the mount.
\verse And he took the calf which they had made, and burnt it in the fire, and ground it to powder, and strawed it upon the water, and made the children of Israel drink of it.
\verse And Moses said unto Aaron, What did this people unto thee, that thou hast brought so great a sin upon them?
\verse And Aaron said, Let not the anger of my lord wax hot: thou knowest the people, that they are set on mischief.
\verse For they said unto me, Make us gods, which shall go before us: for as for this Moses, the man that brought us up out of the land of Egypt, we wot not what is become of him.
\verse And I said unto them, Whosoever hath any gold, let them break it off. So they gave it me: then I cast it into the fire, and there came out this calf.
\verse And when Moses saw that the people were naked; (for Aaron had made them naked unto their shame among their enemies:)
\verse Then Moses stood in the gate of the camp, and said, Who is on the \LORDs side? let him come unto me. And all the sons of Levi gathered themselves together unto him.
\verse And he said unto them, Thus saith the \LORD God of Israel, Put every man his sword by his side, and go in and out from gate to gate throughout the camp, and slay every man his brother, and every man his companion, and every man his neighbour.
\verse And the children of Levi did according to the word of Moses: and there fell of the people that day about three thousand men.
\verse For Moses had said, Consecrate yourselves to day to the \LORD, even every man upon his son, and upon his brother; that he may bestow upon you a blessing this day.
\verse And it came to pass on the morrow, that Moses said unto the people, Ye have sinned a great sin: and now I will go up unto the \LORD; peradventure I shall make an atonement for your sin.
\verse And Moses returned unto the \LORD, and said, Oh, this people have sinned a great sin, and have made them gods of gold.
\verse Yet now, if thou wilt forgive their sin--; and if not, blot me, I pray thee, out of thy book which thou hast written.
\verse And the \LORD said unto Moses, Whosoever hath sinned against me, him will I blot out of my book.
\verse Therefore now go, lead the people unto the place of which I have spoken unto thee: behold, mine Angel shall go before thee: nevertheless in the day when I visit I will visit their sin upon them.
\verse And the \LORD plagued the people, because they made the calf, which Aaron made.
\end{biblechapter}

\begin{biblechapter} % Exodus 33
\verse And the \LORD said unto Moses, Depart, and go up hence, thou and the people which thou hast brought up out of the land of Egypt, unto the land which I sware unto Abraham, to Isaac, and to Jacob, saying, Unto thy seed will I give it:
\verse And I will send an angel before thee; and I will drive out the Canaanite, the Amorite, and the Hittite, and the Perizzite, the Hivite, and the Jebusite:
\verse Unto a land flowing with milk and honey: for I will not go up in the midst of thee; for thou art a stiffnecked people: lest I consume thee in the way.
\verse And when the people heard these evil tidings, they mourned: and no man did put on him his ornaments.
\verse For the \LORD had said unto Moses, Say unto the children of Israel, Ye are a stiffnecked people: I will come up into the midst of thee in a moment, and consume thee: therefore now put off thy ornaments from thee, that I may know what to do unto thee.
\verse And the children of Israel stripped themselves of their ornaments by the mount Horeb.
\verseWithHeading{The Tabernacle of the Congregation} And Moses took the tabernacle, and pitched it without the camp, afar off from the camp, and called it the Tabernacle of the congregation. And it came to pass, that every one which sought the \LORD went out unto the tabernacle of the congregation, which was without the camp.
\verse And it came to pass, when Moses went out unto the tabernacle, that all the people rose up, and stood every man at his tent door, and looked after Moses, until he was gone into the tabernacle.
\verse And it came to pass, as Moses entered into the tabernacle, the cloudy pillar descended, and stood at the door of the tabernacle, and the \LORD talked with Moses.
\verse And all the people saw the cloudy pillar stand at the tabernacle door: and all the people rose up and worshipped, every man in his tent door.
\verse And the \LORD spake unto Moses face to face, as a man speaketh unto his friend. And he turned again into the camp: but his servant Joshua, the son of Nun, a young man, departed not out of the tabernacle.
\verseWithHeading{Moses and the glory of the \LORD} And Moses said unto the \LORD, See, thou sayest unto me, Bring up this people: and thou hast not let me know whom thou wilt send with me. Yet thou hast said, I know thee by name, and thou hast also found grace in my sight.
\verse Now therefore, I pray thee, if I have found grace in thy sight, shew me now thy way, that I may know thee, that I may find grace in thy sight: and consider that this nation is thy people.
\verse And he said, My presence shall go with thee, and I will give thee rest.
\verse And he said unto him, If thy presence go not with me, carry us not up hence.
\verse For wherein shall it be known here that I and thy people have found grace in thy sight? is it not in that thou goest with us? so shall we be separated, I and thy people, from all the people that are upon the face of the earth.
\verse And the \LORD said unto Moses, I will do this thing also that thou hast spoken: for thou hast found grace in my sight, and I know thee by name.
\verse And he said, I beseech thee, shew me thy glory.
\verse And he said, I will make all my goodness pass before thee, and I will proclaim the name of the \LORD before thee; and will be gracious to whom I will be gracious, and will shew mercy on whom I will shew mercy.
\verse And he said, Thou canst not see my face: for there shall no man see me, and live.
\verse And the \LORD said, Behold, there is a place by me, and thou shalt stand upon a rock:
\verse And it shall come to pass, while my glory passeth by, that I will put thee in a clift of the rock, and will cover thee with my hand while I pass by:
\verse And I will take away mine hand, and thou shalt see my back parts: but my face shall not be seen.
\end{biblechapter}

\begin{biblechapter} % Exodus 34
\verseWithHeading{The new stone tablets} And the \LORD said unto Moses, Hew thee two tables of stone like unto the first: and I will write upon these tables the words that were in the first tables, which thou brakest.
\verse And be ready in the morning, and come up in the morning unto mount Sinai, and present thyself there to me in the top of the mount.
\verse And no man shall come up with thee, neither let any man be seen throughout all the mount; neither let the flocks nor herds feed before that mount.
\verse And he hewed two tables of stone like unto the first; and Moses rose up early in the morning, and went up unto mount Sinai, as the \LORD had commanded him, and took in his hand the two tables of stone.
\verse And the \LORD descended in the cloud, and stood with him there, and proclaimed the name of the \LORD.
\verse And the \LORD passed by before him, and proclaimed, The \LORD, The \LORD God, merciful and gracious, longsuffering, and abundant in goodness and truth,
\verse Keeping mercy for thousands, forgiving iniquity and transgression and sin, and that will by no means clear the guilty; visiting the iniquity of the fathers upon the children, and upon the children's children, unto the third and to the fourth generation.
\verse And Moses made haste, and bowed his head toward the earth, and worshipped.
\verse And he said, If now I have found grace in thy sight, O Lord, let my Lord, I pray thee, go among us; for it is a stiffnecked people; and pardon our iniquity and our sin, and take us for thine inheritance.
\verse And he said, Behold, I make a covenant: before all thy people I will do marvels, such as have not been done in all the earth, nor in any nation: and all the people among which thou art shall see the work of the \LORD: for it is a terrible thing that I will do with thee.
\verse Observe thou that which I command thee this day: behold, I drive out before thee the Amorite, and the Canaanite, and the Hittite, and the Perizzite, and the Hivite, and the Jebusite.
\verse Take heed to thyself, lest thou make a covenant with the inhabitants of the land whither thou goest, lest it be for a snare in the midst of thee:
\verse But ye shall destroy their altars, break their images, and cut down their groves:
\verse For thou shalt worship no other god: for the \LORD, whose name is Jealous, is a jealous God:
\verse Lest thou make a covenant with the inhabitants of the land, and they go a whoring after their gods, and do sacrifice unto their gods, and one call thee, and thou eat of his sacrifice;
\verse And thou take of their daughters unto thy sons, and their daughters go a whoring after their gods, and make thy sons go a whoring after their gods.
\verse Thou shalt make thee no molten gods.
\verse The feast of unleavened bread shalt thou keep. Seven days thou shalt eat unleavened bread, as I commanded thee, in the time of the month Abib: for in the month Abib thou camest out from Egypt.
\verse All that openeth the matrix is mine; and every firstling among thy cattle, whether ox or sheep, that is male.
\verse But the firstling of an ass thou shalt redeem with a lamb: and if thou redeem him not, then shalt thou break his neck. All the firstborn of thy sons thou shalt redeem. And none shall appear before me empty.
\verse Six days thou shalt work, but on the seventh day thou shalt rest: in earing time and in harvest thou shalt rest.
\verse And thou shalt observe the feast of weeks, of the firstfruits of wheat harvest, and the feast of ingathering at the year's end.
\verse Thrice in the year shall all your men children appear before the Lord God, the God of Israel.
\verse For I will cast out the nations before thee, and enlarge thy borders: neither shall any man desire thy land, when thou shalt go up to appear before the \LORD thy God thrice in the year.
\verse Thou shalt not offer the blood of my sacrifice with leaven; neither shall the sacrifice of the feast of the Passover be left unto the morning.
\verse The first of the firstfruits of thy land thou shalt bring unto the house of the \LORD thy God. Thou shalt not seethe a kid in his mother's milk.
\verse And the \LORD said unto Moses, Write thou these words: for after the tenor of these words I have made a covenant with thee and with Israel.
\verse And he was there with the \LORD forty days and forty nights; he did neither eat bread, nor drink water. And he wrote upon the tables the words of the covenant, the ten commandments.
\verseWithHeading{The radiant face of Moses} And it came to pass, when Moses came down from mount Sinai with the two tables of testimony in Moses' hand, when he came down from the mount, that Moses wist not that the skin of his face shone while he talked with him.
\verse And when Aaron and all the children of Israel saw Moses, behold, the skin of his face shone; and they were afraid to come nigh him.
\verse And Moses called unto them; and Aaron and all the rulers of the congregation returned unto him: and Moses talked with them.
\verse And afterward all the children of Israel came nigh: and he gave them in commandment all that the \LORD had spoken with him in mount Sinai.
\verse And till Moses had done speaking with them, he put a vail on his face.
\verse But when Moses went in before the \LORD to speak with him, he took the vail off, until he came out. And he came out, and spake unto the children of Israel that which he was commanded.
\verse And the children of Israel saw the face of Moses, that the skin of Moses' face shone: and Moses put the vail upon his face again, until he went in to speak with him.
\end{biblechapter}

\begin{biblechapter} % Exodus 35
\verseWithHeading{Sabbath regulations} And Moses gathered all the congregation of the children of Israel together, and said unto them, These are the words which the \LORD hath commanded, that ye should do them.
\verse Six days shall work be done, but on the seventh day there shall be to you an holy day, a Sabbath of rest to the \LORD: whosoever doeth work therein shall be put to death.
\verse Ye shall kindle no fire throughout your habitations upon the Sabbath day.
\verseWithHeading{Materials for the tabernacle} And Moses spake unto all the congregation of the children of Israel, saying, This is the thing which the \LORD commanded, saying,
\verse Take ye from among you an offering unto the \LORD: whosoever is of a willing heart, let him bring it, an offering of the \LORD; gold, and silver, and brass,
\verse And blue, and purple, and scarlet, and fine linen, and goats' hair,
\verse And rams' skins dyed red, and badgers' skins, and shittim wood,
\verse And oil for the light, and spices for anointing oil, and for the sweet incense,
\verse And onyx stones, and stones to be set for the ephod, and for the breastplate.
\verse And every wise hearted among you shall come, and make all that the \LORD hath commanded;
\verse The tabernacle, his tent, and his covering, his taches, and his boards, his bars, his pillars, and his sockets,
\verse The ark, and the staves thereof, with the mercy seat, and the vail of the covering,
\verse The table, and his staves, and all his vessels, and the shewbread,
\verse The candlestick also for the light, and his furniture, and his lamps, with the oil for the light,
\verse And the incense altar, and his staves, and the anointing oil, and the sweet incense, and the hanging for the door at the entering in of the tabernacle,
\verse The altar of burnt offering, with his brasen grate, his staves, and all his vessels, the laver and his foot,
\verse The hangings of the court, his pillars, and their sockets, and the hanging for the door of the court,
\verse The pins of the tabernacle, and the pins of the court, and their cords,
\verse The cloths of service, to do service in the holy place, the holy garments for Aaron the priest, and the garments of his sons, to minister in the priest's office.
\verse And all the congregation of the children of Israel departed from the presence of Moses.
\verse And they came, every one whose heart stirred him up, and every one whom his spirit made willing, and they brought the \LORDs offering to the work of the tabernacle of the congregation, and for all his service, and for the holy garments.
\verse And they came, both men and women, as many as were willing hearted, and brought bracelets, and earrings, and rings, and tablets, all jewels of gold: and every man that offered offered an offering of gold unto the \LORD.
\verse And every man, with whom was found blue, and purple, and scarlet, and fine linen, and goats' hair, and red skins of rams, and badgers' skins, brought them.
\verse Every one that did offer an offering of silver and brass brought the \LORDs offering: and every man, with whom was found shittim wood for any work of the service, brought it.
\verse And all the women that were wise hearted did spin with their hands, and brought that which they had spun, both of blue, and of purple, and of scarlet, and of fine linen.
\verse And all the women whose heart stirred them up in wisdom spun goats' hair.
\verse And the rulers brought onyx stones, and stones to be set, for the ephod, and for the breastplate;
\verse And spice, and oil for the light, and for the anointing oil, and for the sweet incense.
\verse The children of Israel brought a willing offering unto the \LORD, every man and woman, whose heart made them willing to bring for all manner of work, which the \LORD had commanded to be made by the hand of Moses.
\verseWithHeading{Bezaleel and Aholiab} And Moses said unto the children of Israel, See, the \LORD hath called by name Bezaleel the son of Uri, the son of Hur, of the tribe of Judah;
\verse And he hath filled him with the spirit of God, in wisdom, in understanding, and in knowledge, and in all manner of workmanship;
\verse And to devise curious works, to work in gold, and in silver, and in brass,
\verse And in the cutting of stones, to set them, and in carving of wood, to make any manner of cunning work.
\verse And he hath put in his heart that he may teach, both he, and Aholiab, the son of Ahisamach, of the tribe of Dan.
\verse Them hath he filled with wisdom of heart, to work all manner of work, of the engraver, and of the cunning workman, and of the embroiderer, in blue, and in purple, in scarlet, and in fine linen, and of the weaver, even of them that do any work, and of those that devise cunning work.
\end{biblechapter}

\begin{biblechapter} % Exodus 36
\verse Then wrought Bezaleel and Aholiab, and every wise hearted man, in whom the \LORD put wisdom and understanding to know how to work all manner of work for the service of the sanctuary, according to all that the \LORD had commanded.
\verse And Moses called Bezaleel and Aholiab, and every wise hearted man, in whose heart the \LORD had put wisdom, even every one whose heart stirred him up to come unto the work to do it:
\verse And they received of Moses all the offering, which the children of Israel had brought for the work of the service of the sanctuary, to make it withal. And they brought yet unto him free offerings every morning.
\verse And all the wise men, that wrought all the work of the sanctuary, came every man from his work which they made;
\verse And they spake unto Moses, saying, The people bring much more than enough for the service of the work, which the \LORD commanded to make.
\verse And Moses gave commandment, and they caused it to be proclaimed throughout the camp, saying, Let neither man nor woman make any more work for the offering of the sanctuary. So the people were restrained from bringing.
\verse For the stuff they had was sufficient for all the work to make it, and too much.
\verseWithHeading{The tabernacle} And every wise hearted man among them that wrought the work of the tabernacle made ten curtains of fine twined linen, and blue, and purple, and scarlet: with cherubims of cunning work made he them.
\verse The length of one curtain was twenty and eight cubits, and the breadth of one curtain four cubits: the curtains were all of one size.
\verse And he coupled the five curtains one unto another: and the other five curtains he coupled one unto another.
\verse And he made loops of blue on the edge of one curtain from the selvedge in the coupling: likewise he made in the uttermost side of another curtain, in the coupling of the second.
\verse Fifty loops made he in one curtain, and fifty loops made he in the edge of the curtain which was in the coupling of the second: the loops held one curtain to another.
\verse And he made fifty taches of gold, and coupled the curtains one unto another with the taches: so it became one tabernacle.
\verse And he made curtains of goats' hair for the tent over the tabernacle: eleven curtains he made them.
\verse The length of one curtain was thirty cubits, and four cubits was the breadth of one curtain: the eleven curtains were of one size.
\verse And he coupled five curtains by themselves, and six curtains by themselves.
\verse And he made fifty loops upon the uttermost edge of the curtain in the coupling, and fifty loops made he upon the edge of the curtain which coupleth the second.
\verse And he made fifty taches of brass to couple the tent together, that it might be one.
\verse And he made a covering for the tent of rams' skins dyed red, and a covering of badgers' skins above that.
\verse And he made boards for the tabernacle of shittim wood, standing up.
\verse The length of a board was ten cubits, and the breadth of a board one cubit and a half.
\verse One board had two tenons, equally distant one from another: thus did he make for all the boards of the tabernacle.
\verse And he made boards for the tabernacle; twenty boards for the south side southward:
\verse And forty sockets of silver he made under the twenty boards; two sockets under one board for his two tenons, and two sockets under another board for his two tenons.
\verse And for the other side of the tabernacle, which is toward the north corner, he made twenty boards,
\verse And their forty sockets of silver; two sockets under one board, and two sockets under another board.
\verse And for the sides of the tabernacle westward he made six boards.
\verse And two boards made he for the corners of the tabernacle in the two sides.
\verse And they were coupled beneath, and coupled together at the head thereof, to one ring: thus he did to both of them in both the corners.
\verse And there were eight boards; and their sockets were sixteen sockets of silver, under every board two sockets.
\verse And he made bars of shittim wood; five for the boards of the one side of the tabernacle,
\verse And five bars for the boards of the other side of the tabernacle, and five bars for the boards of the tabernacle for the sides westward.
\verse And he made the middle bar to shoot through the boards from the one end to the other.
\verse And he overlaid the boards with gold, and made their rings of gold to be places for the bars, and overlaid the bars with gold.
\verse And he made a vail of blue, and purple, and scarlet, and fine twined linen: with cherubims made he it of cunning work.
\verse And he made thereunto four pillars of shittim wood, and overlaid them with gold: their hooks were of gold; and he cast for them four sockets of silver.
\verse And he made an hanging for the tabernacle door of blue, and purple, and scarlet, and fine twined linen, of needlework;
\verse And the five pillars of it with their hooks: and he overlaid their chapiters and their fillets with gold: but their five sockets were of brass.
\end{biblechapter}

\begin{biblechapter} % Exodus 37
\verseWithHeading{The ark} And Bezaleel made the ark of shittim wood: two cubits and a half was the length of it, and a cubit and a half the breadth of it, and a cubit and a half the height of it:
\verse And he overlaid it with pure gold within and without, and made a crown of gold to it round about.
\verse And he cast for it four rings of gold, to be set by the four corners of it; even two rings upon the one side of it, and two rings upon the other side of it.
\verse And he made staves of shittim wood, and overlaid them with gold.
\verse And he put the staves into the rings by the sides of the ark, to bear the ark.
\verse And he made the mercy seat of pure gold: two cubits and a half was the length thereof, and one cubit and a half the breadth thereof.
\verse And he made two cherubims of gold, beaten out of one piece made he them, on the two ends of the mercy seat;
\verse One cherub on the end on this side, and another cherub on the other end on that side: out of the mercy seat made he the cherubims on the two ends thereof.
\verse And the cherubims spread out their wings on high, and covered with their wings over the mercy seat, with their faces one to another; even to the mercy seatward were the faces of the cherubims.
\verseWithHeading{The table} And he made the table of shittim wood: two cubits was the length thereof, and a cubit the breadth thereof, and a cubit and a half the height thereof:
\verse And he overlaid it with pure gold, and made thereunto a crown of gold round about.
\verse Also he made thereunto a border of an handbreadth round about; and made a crown of gold for the border thereof round about.
\verse And he cast for it four rings of gold, and put the rings upon the four corners that were in the four feet thereof.
\verse Over against the border were the rings, the places for the staves to bear the table.
\verse And he made the staves of shittim wood, and overlaid them with gold, to bear the table.
\verse And he made the vessels which were upon the table, his dishes, and his spoons, and his bowls, and his covers to cover withal, of pure gold.
\verseWithHeading{The candlestick} And he made the candlestick of pure gold: of beaten work made he the candlestick; his shaft, and his branch, his bowls, his knops, and his flowers, were of the same:
\verse And six branches going out of the sides thereof; three branches of the candlestick out of the one side thereof, and three branches of the candlestick out of the other side thereof:
\verse Three bowls made after the fashion of almonds in one branch, a knop and a flower; and three bowls made like almonds in another branch, a knop and a flower: so throughout the six branches going out of the candlestick.
\verse And in the candlestick were four bowls made like almonds, his knops, and his flowers:
\verse And a knop under two branches of the same, and a knop under two branches of the same, and a knop under two branches of the same, according to the six branches going out of it.
\verse Their knops and their branches were of the same: all of it was one beaten work of pure gold.
\verse And he made his seven lamps, and his snuffers, and his snuffdishes, of pure gold.
\verse Of a talent of pure gold made he it, and all the vessels thereof.
\verseWithHeading{The incense} And he made the incense altar of shittim wood: the length of it was a cubit, and the breadth of it a cubit; it was foursquare; and two cubits was the height of it; the horns thereof were of the same.
\verse And he overlaid it with pure gold, both the top of it, and the sides thereof round about, and the horns of it: also he made unto it a crown of gold round about.
\verse And he made two rings of gold for it under the crown thereof, by the two corners of it, upon the two sides thereof, to be places for the staves to bear it withal.
\verse And he made the staves of shittim wood, and overlaid them with gold.
\verse And he made the holy anointing oil, and the pure incense of sweet spices, according to the work of the apothecary.
\end{biblechapter}

\begin{biblechapter} % Exodus 38
\verseWithHeading{The altar of burnt offering} And he made the altar of burnt offering of shittim wood: five cubits was the length thereof, and five cubits the breadth thereof; it was foursquare; and three cubits the height thereof.
\verse And he made the horns thereof on the four corners of it; the horns thereof were of the same: and he overlaid it with brass.
\verse And he made all the vessels of the altar, the pots, and the shovels, and the basons, and the fleshhooks, and the firepans: all the vessels thereof made he of brass.
\verse And he made for the altar a brasen grate of network under the compass thereof beneath unto the midst of it.
\verse And he cast four rings for the four ends of the grate of brass, to be places for the staves.
\verse And he made the staves of shittim wood, and overlaid them with brass.
\verse And he put the staves into the rings on the sides of the altar, to bear it withal; he made the altar hollow with boards.
\verseWithHeading{The basin for washing} And he made the laver of brass, and the foot of it of brass, of the lookingglasses of the women assembling, which assembled at the door of the tabernacle of the congregation.
\verseWithHeading{The court} And he made the court: on the south side southward the hangings of the court were of fine twined linen, an hundred cubits:
\verse Their pillars were twenty, and their brasen sockets twenty; the hooks of the pillars and their fillets were of silver.
\verse And for the north side the hangings were an hundred cubits, their pillars were twenty, and their sockets of brass twenty; the hooks of the pillars and their fillets of silver.
\verse And for the west side were hangings of fifty cubits, their pillars ten, and their sockets ten; the hooks of the pillars and their fillets of silver.
\verse And for the east side eastward fifty cubits.
\verse The hangings of the one side of the gate were fifteen cubits; their pillars three, and their sockets three.
\verse And for the other side of the court gate, on this hand and that hand, were hangings of fifteen cubits; their pillars three, and their sockets three.
\verse All the hangings of the court round about were of fine twined linen.
\verse And the sockets for the pillars were of brass; the hooks of the pillars and their fillets of silver; and the overlaying of their chapiters of silver; and all the pillars of the court were filleted with silver.
\verse And the hanging for the gate of the court was needlework, of blue, and purple, and scarlet, and fine twined linen: and twenty cubits was the length, and the height in the breadth was five cubits, answerable to the hangings of the court.
\verse And their pillars were four, and their sockets of brass four; their hooks of silver, and the overlaying of their chapiters and their fillets of silver.
\verse And all the pins of the tabernacle, and of the court round about, were of brass.
\verseWithHeading{The materials used} This is the sum of the tabernacle, even of the tabernacle of testimony, as it was counted, according to the commandment of Moses, for the service of the Levites, by the hand of Ithamar, son to Aaron the priest.
\verse And Bezaleel the son of Uri, the son of Hur, of the tribe of Judah, made all that the \LORD commanded Moses.
\verse And with him was Aholiab, son of Ahisamach, of the tribe of Dan, an engraver, and a cunning workman, and an embroiderer in blue, and in purple, and in scarlet, and fine linen.
\verse All the gold that was occupied for the work in all the work of the holy place, even the gold of the offering, was twenty and nine talents, and seven hundred and thirty shekels, after the shekel of the sanctuary.
\verse And the silver of them that were numbered of the congregation was an hundred talents, and a thousand seven hundred and threescore and fifteen shekels, after the shekel of the sanctuary:
\verse A bekah for every man, that is, half a shekel, after the shekel of the sanctuary, for every one that went to be numbered, from twenty years old and upward, for six hundred thousand and three thousand and five hundred and fifty men.
\verse And of the hundred talents of silver were cast the sockets of the sanctuary, and the sockets of the vail; an hundred sockets of the hundred talents, a talent for a socket.
\verse And of the thousand seven hundred seventy and five shekels he made hooks for the pillars, and overlaid their chapiters, and filleted them.
\verse And the brass of the offering was seventy talents, and two thousand and four hundred shekels.
\verse And therewith he made the sockets to the door of the tabernacle of the congregation, and the brasen altar, and the brasen grate for it, and all the vessels of the altar,
\verse And the sockets of the court round about, and the sockets of the court gate, and all the pins of the tabernacle, and all the pins of the court round about.
\end{biblechapter}

\begin{biblechapter} % Exodus 39
\verseWithHeading{The priestly garments} And of the blue, and purple, and scarlet, they made cloths of service, to do service in the holy place, and made the holy garments for Aaron; as the \LORD commanded Moses.
\verseWithHeading{The ephod} And he made the ephod of gold, blue, and purple, and scarlet, and fine twined linen.
\verse And they did beat the gold into thin plates, and cut it into wires, to work it in the blue, and in the purple, and in the scarlet, and in the fine linen, with cunning work.
\verse They made shoulderpieces for it, to couple it together: by the two edges was it coupled together.
\verse And the curious girdle of his ephod, that was upon it, was of the same, according to the work thereof; of gold, blue, and purple, and scarlet, and fine twined linen; as the \LORD commanded Moses.
\verse And they wrought onyx stones inclosed in ouches of gold, graven, as signets are graven, with the names of the children of Israel.
\verse And he put them on the shoulders of the ephod, that they should be stones for a memorial to the children of Israel; as the \LORD commanded Moses.
\verseWithHeading{The breastplate} And he made the breastplate of cunning work, like the work of the ephod; of gold, blue, and purple, and scarlet, and fine twined linen.
\verse It was foursquare; they made the breastplate double: a span was the length thereof, and a span the breadth thereof, being doubled.
\verse And they set in it four rows of stones: the first row was a sardius, a topaz, and a carbuncle: this was the first row.
\verse And the second row, an emerald, a sapphire, and a diamond.
\verse And the third row, a ligure, an agate, and an amethyst.
\verse And the fourth row, a beryl, an onyx, and a jasper: they were inclosed in ouches of gold in their inclosings.
\verse And the stones were according to the names of the children of Israel, twelve, according to their names, like the engravings of a signet, every one with his name, according to the twelve tribes.
\verse And they made upon the breastplate chains at the ends, of wreathen work of pure gold.
\verse And they made two ouches of gold, and two gold rings; and put the two rings in the two ends of the breastplate.
\verse And they put the two wreathen chains of gold in the two rings on the ends of the breastplate.
\verse And the two ends of the two wreathen chains they fastened in the two ouches, and put them on the shoulderpieces of the ephod, before it.
\verse And they made two rings of gold, and put them on the two ends of the breastplate, upon the border of it, which was on the side of the ephod inward.
\verse And they made two other golden rings, and put them on the two sides of the ephod underneath, toward the forepart of it, over against the other coupling thereof, above the curious girdle of the ephod.
\verse And they did bind the breastplate by his rings unto the rings of the ephod with a lace of blue, that it might be above the curious girdle of the ephod, and that the breastplate might not be loosed from the ephod; as the \LORD commanded Moses.
\verseWithHeading{Other priestly garments} And he made the robe of the ephod of woven work, all of blue.
\verse And there was an hole in the midst of the robe, as the hole of an habergeon, with a band round about the hole, that it should not rend.
\verse And they made upon the hems of the robe pomegranates of blue, and purple, and scarlet, and twined linen.
\verse And they made bells of pure gold, and put the bells between the pomegranates upon the hem of the robe, round about between the pomegranates;
\verse A bell and a pomegranate, a bell and a pomegranate, round about the hem of the robe to minister in; as the \LORD commanded Moses.
\verse And they made coats of fine linen of woven work for Aaron, and for his sons,
\verse And a mitre of fine linen, and goodly bonnets of fine linen, and linen breeches of fine twined linen,
\verse And a girdle of fine twined linen, and blue, and purple, and scarlet, of needlework; as the \LORD commanded Moses.
\verse And they made the plate of the holy crown of pure gold, and wrote upon it a writing, like to the engravings of a signet, \textsc{Holiness to the Lord}.
\verse And they tied unto it a lace of blue, to fasten it on high upon the mitre; as the \LORD commanded Moses.
\verseWithHeading{Moses inspects the tabernacle} Thus was all the work of the tabernacle of the tent of the congregation finished: and the children of Israel did according to all that the \LORD commanded Moses, so did they.
\verse And they brought the tabernacle unto Moses, the tent, and all his furniture, his taches, his boards, his bars, and his pillars, and his sockets,
\verse And the covering of rams' skins dyed red, and the covering of badgers' skins, and the vail of the covering,
\verse The ark of the testimony, and the staves thereof, and the mercy seat,
\verse The table, and all the vessels thereof, and the shewbread,
\verse The pure candlestick, with the lamps thereof, even with the lamps to be set in order, and all the vessels thereof, and the oil for light,
\verse And the golden altar, and the anointing oil, and the sweet incense, and the hanging for the tabernacle door,
\verse The brasen altar, and his grate of brass, his staves, and all his vessels, the laver and his foot,
\verse The hangings of the court, his pillars, and his sockets, and the hanging for the court gate, his cords, and his pins, and all the vessels of the service of the tabernacle, for the tent of the congregation,
\verse The cloths of service to do service in the holy place, and the holy garments for Aaron the priest, and his sons' garments, to minister in the priest's office.
\verse According to all that the \LORD commanded Moses, so the children of Israel made all the work.
\verse And Moses did look upon all the work, and, behold, they had done it as the \LORD had commanded, even so had they done it: and Moses blessed them.
\end{biblechapter}

\columnbreak % layout hack

\begin{biblechapter} % Exodus 40
\verseWithHeading{Setting up the tabernacle} And the \LORD spake unto Moses, saying,
\verse On the first day of the first month shalt thou set up the tabernacle of the tent of the congregation.
\verse And thou shalt put therein the ark of the testimony, and cover the ark with the vail.
\verse And thou shalt bring in the table, and set in order the things that are to be set in order upon it; and thou shalt bring in the candlestick, and light the lamps thereof.
\verse And thou shalt set the altar of gold for the incense before the ark of the testimony, and put the hanging of the door to the tabernacle.
\verse And thou shalt set the altar of the burnt offering before the door of the tabernacle of the tent of the congregation.
\verse And thou shalt set the laver between the tent of the congregation and the altar, and shalt put water therein.
\verse And thou shalt set up the court round about, and hang up the hanging at the court gate.
\verse And thou shalt take the anointing oil, and anoint the tabernacle, and all that is therein, and shalt hallow it, and all the vessels thereof: and it shall be holy.
\verse And thou shalt anoint the altar of the burnt offering, and all his vessels, and sanctify the altar: and it shall be an altar most holy.
\verse And thou shalt anoint the laver and his foot, and sanctify it.
\verse And thou shalt bring Aaron and his sons unto the door of the tabernacle of the congregation, and wash them with water.
\verse And thou shalt put upon Aaron the holy garments, and anoint him, and sanctify him; that he may minister unto me in the priest's office.
\verse And thou shalt bring his sons, and clothe them with coats:
\verse And thou shalt anoint them, as thou didst anoint their father, that they may minister unto me in the priest's office: for their anointing shall surely be an everlasting priesthood throughout their generations.
\verse Thus did Moses: according to all that the \LORD commanded him, so did he.
\verse And it came to pass in the first month in the second year, on the first day of the month, that the tabernacle was reared up.
\verse And Moses reared up the tabernacle, and fastened his sockets, and set up the boards thereof, and put in the bars thereof, and reared up his pillars.
\verse And he spread abroad the tent over the tabernacle, and put the covering of the tent above upon it; as the \LORD commanded Moses.
\verse And he took and put the testimony into the ark, and set the staves on the ark, and put the mercy seat above upon the ark:
\verse And he brought the ark into the tabernacle, and set up the vail of the covering, and covered the ark of the testimony; as the \LORD commanded Moses.
\verse And he put the table in the tent of the congregation, upon the side of the tabernacle northward, without the vail.
\verse And he set the bread in order upon it before the \LORD; as the \LORD had commanded Moses.
\verse And he put the candlestick in the tent of the congregation, over against the table, on the side of the tabernacle southward.
\verse And he lighted the lamps before the \LORD; as the \LORD commanded Moses.
\verse And he put the golden altar in the tent of the congregation before the vail:
\verse And he burnt sweet incense thereon; as the \LORD commanded Moses.
\verse And he set up the hanging at the door of the tabernacle.
\verse And he put the altar of burnt offering by the door of the tabernacle of the tent of the congregation, and offered upon it the burnt offering and the meat offering; as the \LORD commanded Moses.
\verse And he set the laver between the tent of the congregation and the altar, and put water there, to wash withal.
\verse And Moses and Aaron and his sons washed their hands and their feet thereat:
\verse When they went into the tent of the congregation, and when they came near unto the altar, they washed; as the \LORD commanded Moses.
\verse And he reared up the court round about the tabernacle and the altar, and set up the hanging of the court gate. So Moses finished the work.
\verseWithHeading{The glory of the \LORD} Then a cloud covered the tent of the congregation, and the glory of the \LORD filled the tabernacle.
\verse And Moses was not able to enter into the tent of the congregation, because the cloud abode thereon, and the glory of the \LORD filled the tabernacle.
\verse And when the cloud was taken up from over the tabernacle, the children of Israel went onward in all their journeys:
\verse But if the cloud were not taken up, then they journeyed not till the day that it was taken up.
\verse For the cloud of the \LORD was upon the tabernacle by day, and fire was on it by night, in the sight of all the house of Israel, throughout all their journeys.
\end{biblechapter}
\flushcolsend
\biblebook{Leviticus}

\begin{biblechapter} % Leviticus 1
\verseWithHeading{The burnt offering} And the \LORD called unto Moses, and spake unto him out of the tabernacle of the congregation, saying,
\verse Speak unto the children of Israel, and say unto them, If any man of you bring an offering unto the \LORD, ye shall bring your offering of the cattle, even of the herd, and of the flock.
\verse If his offering be a burnt sacrifice of the herd, let him offer a male without blemish: he shall offer it of his own voluntary will at the door of the tabernacle of the congregation before the \LORD.
\verse And he shall put his hand upon the head of the burnt offering; and it shall be accepted for him to make atonement for him.
\verse And he shall kill the bullock before the \LORD: and the priests, Aaron's sons, shall bring the blood, and sprinkle the blood round about upon the altar that is by the door of the tabernacle of the congregation.
\verse And he shall flay the burnt offering, and cut it into his pieces.
\verse And the sons of Aaron the priest shall put fire upon the altar, and lay the wood in order upon the fire:
\verse And the priests, Aaron's sons, shall lay the parts, the head, and the fat, in order upon the wood that is on the fire which is upon the altar:
\verse But his inwards and his legs shall he wash in water: and the priest shall burn all on the altar, to be a burnt sacrifice, an offering made by fire, of a sweet savour unto the \LORD.
\verse And if his offering be of the flocks, namely, of the sheep, or of the goats, for a burnt sacrifice; he shall bring it a male without blemish.
\verse And he shall kill it on the side of the altar northward before the \LORD: and the priests, Aaron's sons, shall sprinkle his blood round about upon the altar.
\verse And he shall cut it into his pieces, with his head and his fat: and the priest shall lay them in order on the wood that is on the fire which is upon the altar:
\verse But he shall wash the inwards and the legs with water: and the priest shall bring it all, and burn it upon the altar: it is a burnt sacrifice, an offering made by fire, of a sweet savour unto the \LORD.
\verse And if the burnt sacrifice for his offering to the \LORD be of fowls, then he shall bring his offering of turtledoves, or of young pigeons.
\verse And the priest shall bring it unto the altar, and wring off his head, and burn it on the altar; and the blood thereof shall be wrung out at the side of the altar:
\verse And he shall pluck away his crop with his feathers, and cast it beside the altar on the east part, by the place of the ashes:
\verse And he shall cleave it with the wings thereof, but shall not divide it asunder: and the priest shall burn it upon the altar, upon the wood that is upon the fire: it is a burnt sacrifice, an offering made by fire, of a sweet savour unto the \LORD.
\end{biblechapter}

\begin{biblechapter} % Leviticus 2
\verseWithHeading{The meat offering} And when any will offer a meat offering unto the \LORD, his offering shall be of fine flour; and he shall pour oil upon it, and put frankincense thereon:
\verse And he shall bring it to Aaron's sons the priests: and he shall take thereout his handful of the flour thereof, and of the oil thereof, with all the frankincense thereof; and the priest shall burn the memorial of it upon the altar, to be an offering made by fire, of a sweet savour unto the \LORD:
\verse And the remnant of the meat offering shall be Aaron's and his sons': it is a thing most holy of the offerings of the \LORD made by fire.
\verse And if thou bring an oblation of a meat offering baken in the oven, it shall be unleavened cakes of fine flour mingled with oil, or unleavened wafers anointed with oil.
\verse And if thy oblation be a meat offering baken in a pan, it shall be of fine flour unleavened, mingled with oil.
\verse Thou shalt part it in pieces, and pour oil thereon: it is a meat offering.
\verse And if thy oblation be a meat offering baken in the fryingpan, it shall be made of fine flour with oil.
\verse And thou shalt bring the meat offering that is made of these things unto the \LORD: and when it is presented unto the priest, he shall bring it unto the altar.
\verse And the priest shall take from the meat offering a memorial thereof, and shall burn it upon the altar: it is an offering made by fire, of a sweet savour unto the \LORD.
\verse And that which is left of the meat offering shall be Aaron's and his sons': it is a thing most holy of the offerings of the \LORD made by fire.
\verse No meat offering, which ye shall bring unto the \LORD, shall be made with leaven: for ye shall burn no leaven, nor any honey, in any offering of the \LORD made by fire.
\verse As for the oblation of the firstfruits, ye shall offer them unto the \LORD: but they shall not be burnt on the altar for a sweet savour.
\verse And every oblation of thy meat offering shalt thou season with salt; neither shalt thou suffer the salt of the covenant of thy God to be lacking from thy meat offering: with all thine offerings thou shalt offer salt.
\verse And if thou offer a meat offering of thy firstfruits unto the \LORD, thou shalt offer for the meat offering of thy firstfruits green ears of corn dried by the fire, even corn beaten out of full ears.
\verse And thou shalt put oil upon it, and lay frankincense thereon: it is a meat offering.
\verse And the priest shall burn the memorial of it, part of the beaten corn thereof, and part of the oil thereof, with all the frankincense thereof: it is an offering made by fire unto the \LORD.
\end{biblechapter}

\begin{biblechapter} % Leviticus 3
\verseWithHeading{The peace offering} And if his oblation be a sacrifice of peace offering, if he offer it of the herd; whether it be a male or female, he shall offer it without blemish before the \LORD.
\verse And he shall lay his hand upon the head of his offering, and kill it at the door of the tabernacle of the congregation: and Aaron's sons the priests shall sprinkle the blood upon the altar round about.
\verse And he shall offer of the sacrifice of the peace offering an offering made by fire unto the \LORD; the fat that covereth the inwards, and all the fat that is upon the inwards,
\verse And the two kidneys, and the fat that is on them, which is by the flanks, and the caul above the liver, with the kidneys, it shall he take away.
\verse And Aaron's sons shall burn it on the altar upon the burnt sacrifice, which is upon the wood that is on the fire: it is an offering made by fire, of a sweet savour unto the \LORD.
\verse And if his offering for a sacrifice of peace offering unto the \LORD be of the flock; male or female, he shall offer it without blemish.
\verse If he offer a lamb for his offering, then shall he offer it before the \LORD.
\verse And he shall lay his hand upon the head of his offering, and kill it before the tabernacle of the congregation: and Aaron's sons shall sprinkle the blood thereof round about upon the altar.
\verse And he shall offer of the sacrifice of the peace offering an offering made by fire unto the \LORD; the fat thereof, and the whole rump, it shall he take off hard by the backbone; and the fat that covereth the inwards, and all the fat that is upon the inwards,
\verse And the two kidneys, and the fat that is upon them, which is by the flanks, and the caul above the liver, with the kidneys, it shall he take away.
\verse And the priest shall burn it upon the altar: it is the food of the offering made by fire unto the \LORD.
\verse And if his offering be a goat, then he shall offer it before the \LORD.
\verse And he shall lay his hand upon the head of it, and kill it before the tabernacle of the congregation: and the sons of Aaron shall sprinkle the blood thereof upon the altar round about.
\verse And he shall offer thereof his offering, even an offering made by fire unto the \LORD; the fat that covereth the inwards, and all the fat that is upon the inwards,
\verse And the two kidneys, and the fat that is upon them, which is by the flanks, and the caul above the liver, with the kidneys, it shall he take away.
\verse And the priest shall burn them upon the altar: it is the food of the offering made by fire for a sweet savour: all the fat is the \LORDs.
\verse It shall be a perpetual statute for your generations throughout all your dwellings, that ye eat neither fat nor blood.
\end{biblechapter}

\begin{biblechapter} % Leviticus 4
\verseWithHeading{The sin offering} And the \LORD spake unto Moses, saying,
\verse Speak unto the children of Israel, saying, If a soul shall sin through ignorance against any of the commandments of the \LORD concerning things which ought not to be done, and shall do against any of them:
\verse If the priest that is anointed do sin according to the sin of the people; then let him bring for his sin, which he hath sinned, a young bullock without blemish unto the \LORD for a sin offering.
\verse And he shall bring the bullock unto the door of the tabernacle of the congregation before the \LORD; and shall lay his hand upon the bullock's head, and kill the bullock before the \LORD.
\verse And the priest that is anointed shall take of the bullock's blood, and bring it to the tabernacle of the congregation:
\verse And the priest shall dip his finger in the blood, and sprinkle of the blood seven times before the \LORD, before the vail of the sanctuary.
\verse And the priest shall put some of the blood upon the horns of the altar of sweet incense before the \LORD, which is in the tabernacle of the congregation; and shall pour all the blood of the bullock at the bottom of the altar of the burnt offering, which is at the door of the tabernacle of the congregation.
\verse And he shall take off from it all the fat of the bullock for the sin offering; the fat that covereth the inwards, and all the fat that is upon the inwards,
\verse And the two kidneys, and the fat that is upon them, which is by the flanks, and the caul above the liver, with the kidneys, it shall he take away,
\verse As it was taken off from the bullock of the sacrifice of peace offerings: and the priest shall burn them upon the altar of the burnt offering.
\verse And the skin of the bullock, and all his flesh, with his head, and with his legs, and his inwards, and his dung,
\verse Even the whole bullock shall he carry forth without the camp unto a clean place, where the ashes are poured out, and burn him on the wood with fire: where the ashes are poured out shall he be burnt.
\verse And if the whole congregation of Israel sin through ignorance, and the thing be hid from the eyes of the assembly, and they have done somewhat against any of the commandments of the \LORD concerning things which should not be done, and are guilty;
\verse When the sin, which they have sinned against it, is known, then the congregation shall offer a young bullock for the sin, and bring him before the tabernacle of the congregation.
\verse And the elders of the congregation shall lay their hands upon the head of the bullock before the \LORD: and the bullock shall be killed before the \LORD.
\verse And the priest that is anointed shall bring of the bullock's blood to the tabernacle of the congregation:
\verse And the priest shall dip his finger in some of the blood, and sprinkle it seven times before the \LORD, even before the vail.
\verse And he shall put some of the blood upon the horns of the altar which is before the \LORD, that is in the tabernacle of the congregation, and shall pour out all the blood at the bottom of the altar of the burnt offering, which is at the door of the tabernacle of the congregation.
\verse And he shall take all his fat from him, and burn it upon the altar.
\verse And he shall do with the bullock as he did with the bullock for a sin offering, so shall he do with this: and the priest shall make an atonement for them, and it shall be forgiven them.
\verse And he shall carry forth the bullock without the camp, and burn him as he burned the first bullock: it is a sin offering for the congregation.
\verse When a ruler hath sinned, and done somewhat through ignorance against any of the commandments of the \LORD his God concerning things which should not be done, and is guilty;
\verse Or if his sin, wherein he hath sinned, come to his knowledge; he shall bring his offering, a kid of the goats, a male without blemish:
\verse And he shall lay his hand upon the head of the goat, and kill it in the place where they kill the burnt offering before the \LORD: it is a sin offering.
\verse And the priest shall take of the blood of the sin offering with his finger, and put it upon the horns of the altar of burnt offering, and shall pour out his blood at the bottom of the altar of burnt offering.
\verse And he shall burn all his fat upon the altar, as the fat of the sacrifice of peace offerings: and the priest shall make an atonement for him as concerning his sin, and it shall be forgiven him.
\verse And if any one of the common people sin through ignorance, while he doeth somewhat against any of the commandments of the \LORD concerning things which ought not to be done, and be guilty;
\verse Or if his sin, which he hath sinned, come to his knowledge: then he shall bring his offering, a kid of the goats, a female without blemish, for his sin which he hath sinned.
\verse And he shall lay his hand upon the head of the sin offering, and slay the sin offering in the place of the burnt offering.
\verse And the priest shall take of the blood thereof with his finger, and put it upon the horns of the altar of burnt offering, and shall pour out all the blood thereof at the bottom of the altar.
\verse And he shall take away all the fat thereof, as the fat is taken away from off the sacrifice of peace offerings; and the priest shall burn it upon the altar for a sweet savour unto the \LORD; and the priest shall make an atonement for him, and it shall be forgiven him.
\verse And if he bring a lamb for a sin offering, he shall bring it a female without blemish.
\verse And he shall lay his hand upon the head of the sin offering, and slay it for a sin offering in the place where they kill the burnt offering.
\verse And the priest shall take of the blood of the sin offering with his finger, and put it upon the horns of the altar of burnt offering, and shall pour out all the blood thereof at the bottom of the altar:
\verse And he shall take away all the fat thereof, as the fat of the lamb is taken away from the sacrifice of the peace offerings; and the priest shall burn them upon the altar, according to the offerings made by fire unto the \LORD: and the priest shall make an atonement for his sin that he hath committed, and it shall be forgiven him.
\end{biblechapter}

\flushcolsend\columnbreak % layout hack

\begin{biblechapter} % Leviticus 5
\verse And if a soul sin, and hear the voice of swearing, and is a witness, whether he hath seen or known of it; if he do not utter it, then he shall bear his iniquity.
\verse Or if a soul touch any unclean thing, whether it be a carcase of an unclean beast, or a carcase of unclean cattle, or the carcase of unclean creeping things, and if it be hidden from him; he also shall be unclean, and guilty.
\verse Or if he touch the uncleanness of man, whatsoever uncleanness it be that a man shall be defiled withal, and it be hid from him; when he knoweth of it, then he shall be guilty.
\verse Or if a soul swear, pronouncing with his lips to do evil, or to do good, whatsoever it be that a man shall pronounce with an oath, and it be hid from him; when he knoweth of it, then he shall be guilty in one of these.
\verse And it shall be, when he shall be guilty in one of these things, that he shall confess that he hath sinned in that thing:
\verse And he shall bring his trespass offering unto the \LORD for his sin which he hath sinned, a female from the flock, a lamb or a kid of the goats, for a sin offering; and the priest shall make an atonement for him concerning his sin.
\verse And if he be not able to bring a lamb, then he shall bring for his trespass, which he hath committed, two turtledoves, or two young pigeons, unto the \LORD; one for a sin offering, and the other for a burnt offering.
\verse And he shall bring them unto the priest, who shall offer that which is for the sin offering first, and wring off his head from his neck, but shall not divide it asunder:
\verse And he shall sprinkle of the blood of the sin offering upon the side of the altar; and the rest of the blood shall be wrung out at the bottom of the altar: it is a sin offering.
\verse And he shall offer the second for a burnt offering, according to the manner: and the priest shall make an atonement for him for his sin which he hath sinned, and it shall be forgiven him.
\verse But if he be not able to bring two turtledoves, or two young pigeons, then he that sinned shall bring for his offering the tenth part of an ephah of fine flour for a sin offering; he shall put no oil upon it, neither shall he put any frankincense thereon: for it is a sin offering.
\verse Then shall he bring it to the priest, and the priest shall take his handful of it, even a memorial thereof, and burn it on the altar, according to the offerings made by fire unto the \LORD: it is a sin offering.
\verse And the priest shall make an atonement for him as touching his sin that he hath sinned in one of these, and it shall be forgiven him: and the remnant shall be the priest's, as a meat offering.
\vfill\columnbreak % layout hack
\verseWithHeading{The trespass offering} And the \LORD spake unto Moses, saying,
\verse If a soul commit a trespass, and sin through ignorance, in the holy things of the \LORD; then he shall bring for his trespass unto the \LORD a ram without blemish out of the flocks, with thy estimation by shekels of silver, after the shekel of the sanctuary, for a trespass offering:
\verse And he shall make amends for the harm that he hath done in the holy thing, and shall add the fifth part thereto, and give it unto the priest: and the priest shall make an atonement for him with the ram of the trespass offering, and it shall be forgiven him.
\verse And if a soul sin, and commit any of these things which are forbidden to be done by the commandments of the \LORD; though he wist it not, yet is he guilty, and shall bear his iniquity.
\verse And he shall bring a ram without blemish out of the flock, with thy estimation, for a trespass offering, unto the priest: and the priest shall make an atonement for him concerning his ignorance wherein he erred and wist it not, and it shall be forgiven him.
\verse It is a trespass offering: he hath certainly trespassed against the \LORD.
\end{biblechapter}

\begin{biblechapter} % Leviticus 6
\verse And the \LORD spake unto Moses, saying,
\verse If a soul sin, and commit a trespass against the \LORD, and lie unto his neighbour in that which was delivered him to keep, or in fellowship, or in a thing taken away by violence, or hath deceived his neighbour;
\verse Or have found that which was lost, and lieth concerning it, and sweareth falsely; in any of all these that a man doeth, sinning therein:
\verse Then it shall be, because he hath sinned, and is guilty, that he shall restore that which he took violently away, or the thing which he hath deceitfully gotten, or that which was delivered him to keep, or the lost thing which he found,
\verse Or all that about which he hath sworn falsely; he shall even restore it in the principal, and shall add the fifth part more thereto, and give it unto him to whom it appertaineth, in the day of his trespass offering.
\verse And he shall bring his trespass offering unto the \LORD, a ram without blemish out of the flock, with thy estimation, for a trespass offering, unto the priest:
\verse And the priest shall make an atonement for him before the \LORD: and it shall be forgiven him for any thing of all that he hath done in trespassing therein.
\verseWithHeading{The burnt offering} And the \LORD spake unto Moses, saying,
\verse Command Aaron and his sons, saying, This is the law of the burnt offering: It is the burnt offering, because of the burning upon the altar all night unto the morning, and the fire of the altar shall be burning in it.
\verse And the priest shall put on his linen garment, and his linen breeches shall he put upon his flesh, and take up the ashes which the fire hath consumed with the burnt offering on the altar, and he shall put them beside the altar.
\verse And he shall put off his garments, and put on other garments, and carry forth the ashes without the camp unto a clean place.
\verse And the fire upon the altar shall be burning in it; it shall not be put out: and the priest shall burn wood on it every morning, and lay the burnt offering in order upon it; and he shall burn thereon the fat of the peace offerings.
\verse The fire shall ever be burning upon the altar; it shall never go out.
\verseWithHeading{The meat offering} And this is the law of the meat offering: the sons of Aaron shall offer it before the \LORD, before the altar.
\verse And he shall take of it his handful, of the flour of the meat offering, and of the oil thereof, and all the frankincense which is upon the meat offering, and shall burn it upon the altar for a sweet savour, even the memorial of it, unto the \LORD.
\verse And the remainder thereof shall Aaron and his sons eat: with unleavened bread shall it be eaten in the holy place; in the court of the tabernacle of the congregation they shall eat it.
\verse It shall not be baken with leaven. I have given it unto them for their portion of my offerings made by fire; it is most holy, as is the sin offering, and as the trespass offering.
\verse All the males among the children of Aaron shall eat of it. It shall be a statute for ever in your generations concerning the offerings of the \LORD made by fire: every one that toucheth them shall be holy.
\verse And the \LORD spake unto Moses, saying,
\verse This is the offering of Aaron and of his sons, which they shall offer unto the \LORD in the day when he is anointed; the tenth part of an ephah of fine flour for a meat offering perpetual, half of it in the morning, and half thereof at night.
\verse In a pan it shall be made with oil; and when it is baken, thou shalt bring it in: and the baken pieces of the meat offering shalt thou offer for a sweet savour unto the \LORD.
\verse And the priest of his sons that is anointed in his stead shall offer it: it is a statute for ever unto the \LORD; it shall be wholly burnt.
\verse For every meat offering for the priest shall be wholly burnt: it shall not be eaten.
\verseWithHeading{The sin offering} And the \LORD spake unto Moses, saying,
\verse Speak unto Aaron and to his sons, saying, This is the law of the sin offering: In the place where the burnt offering is killed shall the sin offering be killed before the \LORD: it is most holy.
\verse The priest that offereth it for sin shall eat it: in the holy place shall it be eaten, in the court of the tabernacle of the congregation.
\verse Whatsoever shall touch the flesh thereof shall be holy: and when there is sprinkled of the blood thereof upon any garment, thou shalt wash that whereon it was sprinkled in the holy place.
\verse But the earthen vessel wherein it is sodden shall be broken: and if it be sodden in a brasen pot, it shall be both scoured, and rinsed in water.
\verse All the males among the priests shall eat thereof: it is most holy.
\verse And no sin offering, whereof any of the blood is brought into the tabernacle of the congregation to reconcile withal in the holy place, shall be eaten: it shall be burnt in the fire.
\end{biblechapter}

\begin{biblechapter} % Leviticus 7
\verseWithHeading{The trespass offering} Likewise this is the law of the trespass offering: it is most holy.
\verse In the place where they kill the burnt offering shall they kill the trespass offering: and the blood thereof shall he sprinkle round about upon the altar.
\verse And he shall offer of it all the fat thereof; the rump, and the fat that covereth the inwards,
\verse And the two kidneys, and the fat that is on them, which is by the flanks, and the caul that is above the liver, with the kidneys, it shall he take away:
\verse And the priest shall burn them upon the altar for an offering made by fire unto the \LORD: it is a trespass offering.
\verse Every male among the priests shall eat thereof: it shall be eaten in the holy place: it is most holy.
\verse As the sin offering is, so is the trespass offering: there is one law for them: the priest that maketh atonement therewith shall have it.
\verse And the priest that offereth any man's burnt offering, even the priest shall have to himself the skin of the burnt offering which he hath offered.
\verse And all the meat offering that is baken in the oven, and all that is dressed in the fryingpan, and in the pan, shall be the priest's that offereth it.
\verse And every meat offering, mingled with oil, and dry, shall all the sons of Aaron have, one as much as another.
\verseWithHeading{The peace offering} And this is the law of the sacrifice of peace offerings, which he shall offer unto the \LORD.
\verse If he offer it for a thanksgiving, then he shall offer with the sacrifice of thanksgiving unleavened cakes mingled with oil, and unleavened wafers anointed with oil, and cakes mingled with oil, of fine flour, fried.
\verse Besides the cakes, he shall offer for his offering leavened bread with the sacrifice of thanksgiving of his peace offerings.
\verse And of it he shall offer one out of the whole oblation for an heave offering unto the \LORD, and it shall be the priest's that sprinkleth the blood of the peace offerings.
\verse And the flesh of the sacrifice of his peace offerings for thanksgiving shall be eaten the same day that it is offered; he shall not leave any of it until the morning.
\verse But if the sacrifice of his offering be a vow, or a voluntary offering, it shall be eaten the same day that he offereth his sacrifice: and on the morrow also the remainder of it shall be eaten:
\verse But the remainder of the flesh of the sacrifice on the third day shall be burnt with fire.
\verse And if any of the flesh of the sacrifice of his peace offerings be eaten at all on the third day, it shall not be accepted, neither shall it be imputed unto him that offereth it: it shall be an abomination, and the soul that eateth of it shall bear his iniquity.
\verse And the flesh that toucheth any unclean thing shall not be eaten; it shall be burnt with fire: and as for the flesh, all that be clean shall eat thereof.
\verse But the soul that eateth of the flesh of the sacrifice of peace offerings, that pertain unto the \LORD, having his uncleanness upon him, even that soul shall be cut off from his people.
\verse Moreover the soul that shall touch any unclean thing, as the uncleanness of man, or any unclean beast, or any abominable unclean thing, and eat of the flesh of the sacrifice of peace offerings, which pertain unto the \LORD, even that soul shall be cut off from his people.
\verseWithHeading{Eating fat and blood forbidden} And the \LORD spake unto Moses, saying,
\verse Speak unto the children of Israel, saying, Ye shall eat no manner of fat, of ox, or of sheep, or of goat.
\verse And the fat of the beast that dieth of itself, and the fat of that which is torn with beasts, may be used in any other use: but ye shall in no wise eat of it.
\verse For whosoever eateth the fat of the beast, of which men offer an offering made by fire unto the \LORD, even the soul that eateth it shall be cut off from his people.
\verse Moreover ye shall eat no manner of blood, whether it be of fowl or of beast, in any of your dwellings.
\verse Whatsoever soul it be that eateth any manner of blood, even that soul shall be cut off from his people.
\verseWithHeading{The priests' share} And the \LORD spake unto Moses, saying,
\verse Speak unto the children of Israel, saying, He that offereth the sacrifice of his peace offerings unto the \LORD shall bring his oblation unto the \LORD of the sacrifice of his peace offerings.
\verse His own hands shall bring the offerings of the \LORD made by fire, the fat with the breast, it shall he bring, that the breast may be waved for a wave offering before the \LORD.
\verse And the priest shall burn the fat upon the altar: but the breast shall be Aaron's and his sons'.
\verse And the right shoulder shall ye give unto the priest for an heave offering of the sacrifices of your peace offerings.
\verse He among the sons of Aaron, that offereth the blood of the peace offerings, and the fat, shall have the right shoulder for his part.
\verse For the wave breast and the heave shoulder have I taken of the children of Israel from off the sacrifices of their peace offerings, and have given them unto Aaron the priest and unto his sons by a statute for ever from among the children of Israel.
\verse This is the portion of the anointing of Aaron, and of the anointing of his sons, out of the offerings of the \LORD made by fire, in the day when he presented them to minister unto the \LORD in the priest's office;
\verse Which the \LORD commanded to be given them of the children of Israel, in the day that he anointed them, by a statute for ever throughout their generations.
\verse This is the law of the burnt offering, of the meat offering, and of the sin offering, and of the trespass offering, and of the consecrations, and of the sacrifice of the peace offerings;
\verse Which the \LORD commanded Moses in mount Sinai, in the day that he commanded the children of Israel to offer their oblations unto the \LORD, in the wilderness of Sinai.
\end{biblechapter}

\begin{biblechapter} % Leviticus 8
\verseWithHeading{The ordination of Aaron and his \newline sons} And the \LORD spake unto Moses, saying,
\verse Take Aaron and his sons with him, and the garments, and the anointing oil, and a bullock for the sin offering, and two rams, and a basket of unleavened bread;
\verse And gather thou all the congregation together unto the door of the tabernacle of the congregation.
\verse And Moses did as the \LORD commanded him; and the assembly was gathered together unto the door of the tabernacle of the congregation.
\verse And Moses said unto the congregation, This is the thing which the \LORD commanded to be done.
\verse And Moses brought Aaron and his sons, and washed them with water.
\verse And he put upon him the coat, and girded him with the girdle, and clothed him with the robe, and put the ephod upon him, and he girded him with the curious girdle of the ephod, and bound it unto him therewith.
\verse And he put the breastplate upon him: also he put in the breastplate the Urim and the Thummim.
\verse And he put the mitre upon his head; also upon the mitre, even upon his forefront, did he put the golden plate, the holy crown; as the \LORD commanded Moses.
\verse And Moses took the anointing oil, and anointed the tabernacle and all that was therein, and sanctified them.
\verse And he sprinkled thereof upon the altar seven times, and anointed the altar and all his vessels, both the laver and his foot, to sanctify them.
\verse And he poured of the anointing oil upon Aaron's head, and anointed him, to sanctify him.
\verse And Moses brought Aaron's sons, and put coats upon them, and girded them with girdles, and put bonnets upon them; as the \LORD commanded Moses.
\verse And he brought the bullock for the sin offering: and Aaron and his sons laid their hands upon the head of the bullock for the sin offering.
\verse And he slew it; and Moses took the blood, and put it upon the horns of the altar round about with his finger, and purified the altar, and poured the blood at the bottom of the altar, and sanctified it, to make reconciliation upon it.
\verse And he took all the fat that was upon the inwards, and the caul above the liver, and the two kidneys, and their fat, and Moses burned it upon the altar.
\verse But the bullock, and his hide, his flesh, and his dung, he burnt with fire without the camp; as the \LORD commanded Moses.
\verse And he brought the ram for the burnt offering: and Aaron and his sons laid their hands upon the head of the ram.
\verse And he killed it; and Moses sprinkled the blood upon the altar round about.
\verse And he cut the ram into pieces; and Moses burnt the head, and the pieces, and the fat.
\verse And he washed the inwards and the legs in water; and Moses burnt the whole ram upon the altar: it was a burnt sacrifice for a sweet savour, and an offering made by fire unto the \LORD; as the \LORD commanded Moses.
\verse And he brought the other ram, the ram of consecration: and Aaron and his sons laid their hands upon the head of the ram.
\verse And he slew it; and Moses took of the blood of it, and put it upon the tip of Aaron's right ear, and upon the thumb of his right hand, and upon the great toe of his right foot.
\verse And he brought Aaron's sons, and Moses put of the blood upon the tip of their right ear, and upon the thumbs of their right hands, and upon the great toes of their right feet: and Moses sprinkled the blood upon the altar round about.
\verse And he took the fat, and the rump, and all the fat that was upon the inwards, and the caul above the liver, and the two kidneys, and their fat, and the right shoulder:
\verse And out of the basket of unleavened bread, that was before the \LORD, he took one unleavened cake, and a cake of oiled bread, and one wafer, and put them on the fat, and upon the right shoulder:
\verse And he put all upon Aaron's hands, and upon his sons' hands, and waved them for a wave offering before the \LORD.
\verse And Moses took them from off their hands, and burnt them on the altar upon the burnt offering: they were consecrations for a sweet savour: it is an offering made by fire unto the \LORD.
\verse And Moses took the breast, and waved it for a wave offering before the \LORD: for of the ram of consecration it was Moses' part; as the \LORD commanded Moses.
\verse And Moses took of the anointing oil, and of the blood which was upon the altar, and sprinkled it upon Aaron, and upon his garments, and upon his sons, and upon his sons' garments with him; and sanctified Aaron, and his garments, and his sons, and his sons' garments with him.
\verse And Moses said unto Aaron and to his sons, Boil the flesh at the door of the tabernacle of the congregation: and there eat it with the bread that is in the basket of consecrations, as I commanded, saying, Aaron and his sons shall eat it.
\verse And that which remaineth of the flesh and of the bread shall ye burn with fire.
\verse And ye shall not go out of the door of the tabernacle of the congregation in seven days, until the days of your consecration be at an end: for seven days shall he consecrate you.
\verse As he hath done this day, so the \LORD hath commanded to do, to make an atonement for you.
\verse Therefore shall ye abide at the door of the tabernacle of the congregation day and night seven days, and keep the charge of the \LORD, that ye die not: for so I am commanded.
\verse So Aaron and his sons did all things which the \LORD commanded by the hand of Moses.
\end{biblechapter}

\begin{biblechapter} % Leviticus 9
\verseWithHeading{The priests begin their ministry} And it came to pass on the eighth day, that Moses called Aaron and his sons, and the elders of Israel;
\verse And he said unto Aaron, Take thee a young calf for a sin offering, and a ram for a burnt offering, without blemish, and offer them before the \LORD.
\verse And unto the children of Israel thou shalt speak, saying, Take ye a kid of the goats for a sin offering; and a calf and a lamb, both of the first year, without blemish, for a burnt offering;
\verse Also a bullock and a ram for peace offerings, to sacrifice before the \LORD; and a meat offering mingled with oil: for to day the \LORD will appear unto you.
\verse And they brought that which Moses commanded before the tabernacle of the congregation: and all the congregation drew near and stood before the \LORD.
\verse And Moses said, This is the thing which the \LORD commanded that ye should do: and the glory of the \LORD shall appear unto you.
\verse And Moses said unto Aaron, Go unto the altar, and offer thy sin offering, and thy burnt offering, and make an atonement for thyself, and for the people: and offer the offering of the people, and make an atonement for them; as the \LORD commanded.
\verse Aaron therefore went unto the altar, and slew the calf of the sin offering, which was for himself.
\verse And the sons of Aaron brought the blood unto him: and he dipped his finger in the blood, and put it upon the horns of the altar, and poured out the blood at the bottom of the altar:
\verse But the fat, and the kidneys, and the caul above the liver of the sin offering, he burnt upon the altar; as the \LORD commanded Moses.
\verse And the flesh and the hide he burnt with fire without the camp.
\verse And he slew the burnt offering; and Aaron's sons presented unto him the blood, which he sprinkled round about upon the altar.
\verse And they presented the burnt offering unto him, with the pieces thereof, and the head: and he burnt them upon the altar.
\verse And he did wash the inwards and the legs, and burnt them upon the burnt offering on the altar.
\verse And he brought the people's offering, and took the goat, which was the sin offering for the people, and slew it, and offered it for sin, as the first.
\verse And he brought the burnt offering, and offered it according to the manner.
\verse And he brought the meat offering, and took an handful thereof, and burnt it upon the altar, beside the burnt sacrifice of the morning.
\verse He slew also the bullock and the ram for a sacrifice of peace offerings, which was for the people: and Aaron's sons presented unto him the blood, which he sprinkled upon the altar round about,
\verse And the fat of the bullock and of the ram, the rump, and that which covereth the inwards, and the kidneys, and the caul above the liver:
\verse And they put the fat upon the breasts, and he burnt the fat upon the altar:
\verse And the breasts and the right shoulder Aaron waved for a wave offering before the \LORD; as Moses commanded.
\verse And Aaron lifted up his hand toward the people, and blessed them, and came down from offering of the sin offering, and the burnt offering, and peace offerings.
\verse And Moses and Aaron went into the tabernacle of the congregation, and came out, and blessed the people: and the glory of the \LORD appeared unto all the people.
\verse And there came a fire out from before the \LORD, and consumed upon the altar the burnt offering and the fat: which when all the people saw, they shouted, and fell on their faces.
\end{biblechapter}

\begin{biblechapter} % Leviticus 10
\verseWithHeading{The death of Nadab and \newline Abihu} And Nadab and Abihu, the sons of Aaron, took either of them his censer, and put fire therein, and put incense thereon, and offered strange fire before the \LORD, which he commanded them not.
\verse And there went out fire from the \LORD, and devoured them, and they died before the \LORD.
\verse Then Moses said unto Aaron, This is it that the \LORD spake, saying, I will be sanctified in them that come nigh me, and before all the people I will be glorified. And Aaron held his peace.
\verse And Moses called Mishael and Elzaphan, the sons of Uzziel the uncle of Aaron, and said unto them, Come near, carry your brethren from before the sanctuary out of the camp.
\verse So they went near, and carried them in their coats out of the camp; as Moses had said.
\verse And Moses said unto Aaron, and unto Eleazar and unto Ithamar, his sons, Uncover not your heads, neither rend your clothes; lest ye die, and lest wrath come upon all the people: but let your brethren, the whole house of Israel, bewail the burning which the \LORD hath kindled.
\verse And ye shall not go out from the door of the tabernacle of the congregation, lest ye die: for the anointing oil of the \LORD is upon you. And they did according to the word of Moses.
\verse And the \LORD spake unto Aaron, saying,
\verse Do not drink wine nor strong drink, thou, nor thy sons with thee, when ye go into the tabernacle of the congregation, lest ye die: it shall be a statute for ever throughout your generations:
\verse And that ye may put difference between holy and unholy, and between unclean and clean;
\verse And that ye may teach the children of Israel all the statutes which the \LORD hath spoken unto them by the hand of Moses.
\verse And Moses spake unto Aaron, and unto Eleazar and unto Ithamar, his sons that were left, Take the meat offering that remaineth of the offerings of the \LORD made by fire, and eat it without leaven beside the altar: for it is most holy:
\verse And ye shall eat it in the holy place, because it is thy due, and thy sons' due, of the sacrifices of the \LORD made by fire: for so I am commanded.
\verse And the wave breast and heave shoulder shall ye eat in a clean place; thou, and thy sons, and thy daughters with thee: for they be thy due, and thy sons' due, which are given out of the sacrifices of peace offerings of the children of Israel.
\verse The heave shoulder and the wave breast shall they bring with the offerings made by fire of the fat, to wave it for a wave offering before the \LORD; and it shall be thine, and thy sons' with thee, by a statute for ever; as the \LORD hath commanded.
\verse And Moses diligently sought the goat of the sin offering, and, behold, it was burnt: and he was angry with Eleazar and Ithamar, the sons of Aaron which were left alive, saying,
\verse Wherefore have ye not eaten the sin offering in the holy place, seeing it is most holy, and God hath given it you to bear the iniquity of the congregation, to make atonement for them before the \LORD?
\verse Behold, the blood of it was not brought in within the holy place: ye should indeed have eaten it in the holy place, as I commanded.
\verse And Aaron said unto Moses, Behold, this day have they offered their sin offering and their burnt offering before the \LORD; and such things have befallen me: and if I had eaten the sin offering to day, should it have been accepted in the sight of the \LORD?
\verse And when Moses heard that, he was content.
\end{biblechapter}

\begin{biblechapter} % Leviticus 11
\verseWithHeading{Clean and unclean food} And the \LORD spake unto Moses and to Aaron, saying unto them,
\verse Speak unto the children of Israel, saying, These are the beasts which ye shall eat among all the beasts that are on the earth.
\verse Whatsoever parteth the hoof, and is clovenfooted, and cheweth the cud, among the beasts, that shall ye eat.
\verse Nevertheless these shall ye not eat of them that chew the cud, or of them that divide the hoof: as the camel, because he cheweth the cud, but divideth not the hoof; he is unclean unto you.
\verse And the coney, because he cheweth the cud, but divideth not the hoof; he is unclean unto you.
\verse And the hare, because he cheweth the cud, but divideth not the hoof; he is unclean unto you.
\verse And the swine, though he divide the hoof, and be clovenfooted, yet he cheweth not the cud; he is unclean to you.
\verse Of their flesh shall ye not eat, and their carcase shall ye not touch; they are unclean to you.
\verse These shall ye eat of all that are in the waters: whatsoever hath fins and scales in the waters, in the seas, and in the rivers, them shall ye eat.
\verse And all that have not fins and scales in the seas, and in the rivers, of all that move in the waters, and of any living thing which is in the waters, they shall be an abomination unto you:
\verse They shall be even an abomination unto you; ye shall not eat of their flesh, but ye shall have their carcases in abomination.
\verse Whatsoever hath no fins nor scales in the waters, that shall be an abomination unto you.
\verse And these are they which ye shall have in abomination among the fowls; they shall not be eaten, they are an abomination: the eagle, and the ossifrage, and the ospray,
\verse And the vulture, and the kite after his kind;
\verse Every raven after his kind;
\verse And the owl, and the night hawk, and the cuckow, and the hawk after his kind,
\verse And the little owl, and the cormorant, and the great owl,
\verse And the swan, and the pelican, and the gier eagle,
\verse And the stork, the heron after her kind, and the lapwing, and the bat.
\verse All fowls that creep, going upon all four, shall be an abomination unto you.
\verse Yet these may ye eat of every flying creeping thing that goeth upon all four, which have legs above their feet, to leap withal upon the earth;
\verse Even these of them ye may eat; the locust after his kind, and the bald locust after his kind, and the beetle after his kind, and the grasshopper after his kind.
\verse But all other flying creeping things, which have four feet, shall be an abomination unto you.
\verse And for these ye shall be unclean: whosoever toucheth the carcase of them shall be unclean until the even.
\verse And whosoever beareth ought of the carcase of them shall wash his clothes, and be unclean until the even.
\verse The carcases of every beast which divideth the hoof, and is not clovenfooted, nor cheweth the cud, are unclean unto you: every one that toucheth them shall be unclean.
\verse And whatsoever goeth upon his paws, among all manner of beasts that go on all four, those are unclean unto you: whoso toucheth their carcase shall be unclean until the even.
\verse And he that beareth the carcase of them shall wash his clothes, and be unclean until the even: they are unclean unto you.
\verse These also shall be unclean unto you among the creeping things that creep upon the earth; the weasel, and the mouse, and the tortoise after his kind,
\verse And the ferret, and the chameleon, and the lizard, and the snail, and the mole.
\verse These are unclean to you among all that creep: whosoever doth touch them, when they be dead, shall be unclean until the even.
\verse And upon whatsoever any of them, when they are dead, doth fall, it shall be unclean; whether it be any vessel of wood, or raiment, or skin, or sack, whatsoever vessel it be, wherein any work is done, it must be put into water, and it shall be unclean until the even; so it shall be cleansed.
\verse And every earthen vessel, whereinto any of them falleth, whatsoever is in it shall be unclean; and ye shall break it.
\verse Of all meat which may be eaten, that on which such water cometh shall be unclean: and all drink that may be drunk in every such vessel shall be unclean.
\verse And every thing whereupon any part of their carcase falleth shall be unclean; whether it be oven, or ranges for pots, they shall be broken down: for they are unclean, and shall be unclean unto you.
\verse Nevertheless a fountain or pit, wherein there is plenty of water, shall be clean: but that which toucheth their carcase shall be unclean.
\verse And if any part of their carcase fall upon any sowing seed which is to be sown, it shall be clean.
\verse But if any water be put upon the seed, and any part of their carcase fall thereon, it shall be unclean unto you.
\verse And if any beast, of which ye may eat, die; he that toucheth the carcase thereof shall be unclean until the even.
\verse And he that eateth of the carcase of it shall wash his clothes, and be unclean until the even: he also that beareth the carcase of it shall wash his clothes, and be unclean until the even.
\verse And every creeping thing that creepeth upon the earth shall be an abomination; it shall not be eaten.
\verse Whatsoever goeth upon the belly, and whatsoever goeth upon all four, or whatsoever hath more feet among all creeping things that creep upon the earth, them ye shall not eat; for they are an abomination.
\verse Ye shall not make yourselves abominable with any creeping thing that creepeth, neither shall ye make yourselves unclean with them, that ye should be defiled thereby.
\verse For I am the \LORD your God: ye shall therefore sanctify yourselves, and ye shall be holy; for I am holy: neither shall ye defile yourselves with any manner of creeping thing that creepeth upon the earth.
\verse For I am the \LORD that bringeth you up out of the land of Egypt, to be your God: ye shall therefore be holy, for I am holy.
\verse This is the law of the beasts, and of the fowl, and of every living creature that moveth in the waters, and of every creature that creepeth upon the earth:
\verse To make a difference between the unclean and the clean, and between the beast that may be eaten and the beast that may not be eaten.
\end{biblechapter}

\begin{biblechapter} % Leviticus 12
\verseWithHeading{Purification after childbirth} And the \LORD spake unto Moses, saying,
\verse Speak unto the children of Israel, saying, If a woman have conceived seed, and born a man child: then she shall be unclean seven days; according to the days of the separation for her infirmity shall she be unclean.
\verse And in the eighth day the flesh of his foreskin shall be circumcised.
\verse And she shall then continue in the blood of her purifying three and thirty days; she shall touch no hallowed thing, nor come into the sanctuary, until the days of her purifying be fulfilled.
\verse But if she bear a maid child, then she shall be unclean two weeks, as in her separation: and she shall continue in the blood of her purifying threescore and six days.
\verse And when the days of her purifying are fulfilled, for a son, or for a daughter, she shall bring a lamb of the first year for a burnt offering, and a young pigeon, or a turtledove, for a sin offering, unto the door of the tabernacle of the congregation, unto the priest:
\verse Who shall offer it before the \LORD, and make an atonement for her; and she shall be cleansed from the issue of her blood. This is the law for her that hath born a male or a female.
\verse And if she be not able to bring a lamb, then she shall bring two turtles, or two young pigeons; the one for the burnt offering, and the other for a sin offering: and the priest shall make an atonement for her, and she shall be clean.
\end{biblechapter}

\begin{biblechapter} % Leviticus 13
\verseWithHeading{Leprosy of the flesh} And the \LORD spake unto Moses and Aaron, saying,
\verse When a man shall have in the skin of his flesh a rising, a scab, or bright spot, and it be in the skin of his flesh like the plague of leprosy; then he shall be brought unto Aaron the priest, or unto one of his sons the priests:
\verse And the priest shall look on the plague in the skin of the flesh: and when the hair in the plague is turned white, and the plague in sight be deeper than the skin of his flesh, it is a plague of leprosy: and the priest shall look on him, and pronounce him unclean.
\verse If the bright spot be white in the skin of his flesh, and in sight be not deeper than the skin, and the hair thereof be not turned white; then the priest shall shut up him that hath the plague seven days:
\verse And the priest shall look on him the seventh day: and, behold, if the plague in his sight be at a stay, and the plague spread not in the skin; then the priest shall shut him up seven days more:
\verse And the priest shall look on him again the seventh day: and, behold, if the plague be somewhat dark, and the plague spread not in the skin, the priest shall pronounce him clean: it is but a scab: and he shall wash his clothes, and be clean.
\verse But if the scab spread much abroad in the skin, after that he hath been seen of the priest for his cleansing, he shall be seen of the priest again:
\verse And if the priest see that, behold, the scab spreadeth in the skin, then the priest shall pronounce him unclean: it is a leprosy.
\verse When the plague of leprosy is in a man, then he shall be brought unto the priest;
\verse And the priest shall see him: and, behold, if the rising be white in the skin, and it have turned the hair white, and there be quick raw flesh in the rising;
\verse It is an old leprosy in the skin of his flesh, and the priest shall pronounce him unclean, and shall not shut him up: for he is unclean.
\verse And if a leprosy break out abroad in the skin, and the leprosy cover all the skin of him that hath the plague from his head even to his foot, wheresoever the priest looketh;
\verse Then the priest shall consider: and, behold, if the leprosy have covered all his flesh, he shall pronounce him clean that hath the plague: it is all turned white: he is clean.
\verse But when raw flesh appeareth in him, he shall be unclean.
\verse And the priest shall see the raw flesh, and pronounce him to be unclean: for the raw flesh is unclean: it is a leprosy.
\verse Or if the raw flesh turn again, and be changed unto white, he shall come unto the priest;
\verse And the priest shall see him: and, behold, if the plague be turned into white; then the priest shall pronounce him clean that hath the plague: he is clean.
\verse The flesh also, in which, even in the skin thereof, was a boil, and is healed,
\verse And in the place of the boil there be a white rising, or a bright spot, white, and somewhat reddish, and it be shewed to the priest;
\verse And if, when the priest seeth it, behold, it be in sight lower than the skin, and the hair thereof be turned white; the priest shall pronounce him unclean: it is a plague of leprosy broken out of the boil.
\verse But if the priest look on it, and, behold, there be no white hairs therein, and if it be not lower than the skin, but be somewhat dark; then the priest shall shut him up seven days:
\verse And if it spread much abroad in the skin, then the priest shall pronounce him unclean: it is a plague.
\verse But if the bright spot stay in his place, and spread not, it is a burning boil; and the priest shall pronounce him clean.
\verse Or if there be any flesh, in the skin whereof there is a hot burning, and the quick flesh that burneth have a white bright spot, somewhat reddish, or white;
\verse Then the priest shall look upon it: and, behold, if the hair in the bright spot be turned white, and it be in sight deeper than the skin; it is a leprosy broken out of the burning: wherefore the priest shall pronounce him unclean: it is the plague of leprosy.
\verse But if the priest look on it, and, behold, there be no white hair in the bright spot, and it be no lower than the other skin, but be somewhat dark; then the priest shall shut him up seven days:
\verse And the priest shall look upon him the seventh day: and if it be spread much abroad in the skin, then the priest shall pronounce him unclean: it is the plague of leprosy.
\verse And if the bright spot stay in his place, and spread not in the skin, but it be somewhat dark; it is a rising of the burning, and the priest shall pronounce him clean: for it is an inflammation of the burning.
\verse If a man or woman have a plague upon the head or the beard;
\verse Then the priest shall see the plague: and, behold, if it be in sight deeper than the skin; and there be in it a yellow thin hair; then the priest shall pronounce him unclean: it is a dry scall, even a leprosy upon the head or beard.
\verse And if the priest look on the plague of the scall, and, behold, it be not in sight deeper than the skin, and that there is no black hair in it; then the priest shall shut up him that hath the plague of the scall seven days:
\verse And in the seventh day the priest shall look on the plague: and, behold, if the scall spread not, and there be in it no yellow hair, and the scall be not in sight deeper than the skin;
\verse He shall be shaven, but the scall shall he not shave; and the priest shall shut up him that hath the scall seven days more:
\verse And in the seventh day the priest shall look on the scall: and, behold, if the scall be not spread in the skin, nor be in sight deeper than the skin; then the priest shall pronounce him clean: and he shall wash his clothes, and be clean.
\verse But if the scall spread much in the skin after his cleansing;
\verse Then the priest shall look on him: and, behold, if the scall be spread in the skin, the priest shall not seek for yellow hair; he is unclean.
\verse But if the scall be in his sight at a stay, and that there is black hair grown up therein; the scall is healed, he is clean: and the priest shall pronounce him clean.
\verse If a man also or a woman have in the skin of their flesh bright spots, even white bright spots;
\verse Then the priest shall look: and, behold, if the bright spots in the skin of their flesh be darkish white; it is a freckled spot that groweth in the skin; he is clean.
\verse And the man whose hair is fallen off his head, he is bald; yet is he clean.
\verse And he that hath his hair fallen off from the part of his head toward his face, he is forehead bald: yet is he clean.
\verse And if there be in the bald head, or bald forehead, a white reddish sore; it is a leprosy sprung up in his bald head, or his bald forehead.
\verse Then the priest shall look upon it: and, behold, if the rising of the sore be white reddish in his bald head, or in his bald forehead, as the leprosy appeareth in the skin of the flesh;
\verse He is a leprous man, he is unclean: the priest shall pronounce him utterly unclean; his plague is in his head.
\verse And the leper in whom the plague is, his clothes shall be rent, and his head bare, and he shall put a covering upon his upper lip, and shall cry, Unclean, unclean.
\verse All the days wherein the plague shall be in him he shall be defiled; he is unclean: he shall dwell alone; without the camp shall his habitation be.
\verseWithHeading{Leprosy of the garments} The garment also that the plague of leprosy is in, whether it be a woollen garment, or a linen garment;
\verse Whether it be in the warp, or woof; of linen, or of woollen; whether in a skin, or in any thing made of skin;
\verse And if the plague be greenish or reddish in the garment, or in the skin, either in the warp, or in the woof, or in any thing of skin; it is a plague of leprosy, and shall be shewed unto the priest:
\verse And the priest shall look upon the plague, and shut up it that hath the plague seven days:
\verse And he shall look on the plague on the seventh day: if the plague be spread in the garment, either in the warp, or in the woof, or in a skin, or in any work that is made of skin; the plague is a fretting leprosy; it is unclean.
\verse He shall therefore burn that garment, whether warp or woof, in woollen or in linen, or any thing of skin, wherein the plague is: for it is a fretting leprosy; it shall be burnt in the fire.
\verse And if the priest shall look, and, behold, the plague be not spread in the garment, either in the warp, or in the woof, or in any thing of skin;
\verse Then the priest shall command that they wash the thing wherein the plague is, and he shall shut it up seven days more:
\verse And the priest shall look on the plague, after that it is washed: and, behold, if the plague have not changed his colour, and the plague be not spread; it is unclean; thou shalt burn it in the fire; it is fret inward, whether it be bare within or without.
\verse And if the priest look, and, behold, the plague be somewhat dark after the washing of it; then he shall rend it out of the garment, or out of the skin, or out of the warp, or out of the woof:
\verse And if it appear still in the garment, either in the warp, or in the woof, or in any thing of skin; it is a spreading plague: thou shalt burn that wherein the plague is with fire.
\verse And the garment, either warp, or woof, or whatsoever thing of skin it be, which thou shalt wash, if the plague be departed from them, then it shall be washed the second time, and shall be clean.
\verse This is the law of the plague of leprosy in a garment of woollen or linen, either in the warp, or woof, or any thing of skins, to pronounce it clean, or to pronounce it unclean.
\end{biblechapter}

\begin{biblechapter} % Leviticus 14
\verseWithHeading{Cleansing leprosy of the flesh} And the \LORD spake unto Moses, saying,
\verse This shall be the law of the leper in the day of his cleansing: He shall be brought unto the priest:
\verse And the priest shall go forth out of the camp; and the priest shall look, and, behold, if the plague of leprosy be healed in the leper;
\verse Then shall the priest command to take for him that is to be cleansed two birds alive and clean, and cedar wood, and scarlet, and hyssop:
\verse And the priest shall command that one of the birds be killed in an earthen vessel over running water:
\verse As for the living bird, he shall take it, and the cedar wood, and the scarlet, and the hyssop, and shall dip them and the living bird in the blood of the bird that was killed over the running water:
\verse And he shall sprinkle upon him that is to be cleansed from the leprosy seven times, and shall pronounce him clean, and shall let the living bird loose into the open field.
\verse And he that is to be cleansed shall wash his clothes, and shave off all his hair, and wash himself in water, that he may be clean: and after that he shall come into the camp, and shall tarry abroad out of his tent seven days.
\verse But it shall be on the seventh day, that he shall shave all his hair off his head and his beard and his eyebrows, even all his hair he shall shave off: and he shall wash his clothes, also he shall wash his flesh in water, and he shall be clean.
\verse And on the eighth day he shall take two he lambs without blemish, and one ewe lamb of the first year without blemish, and three tenth deals of fine flour for a meat offering, mingled with oil, and one log of oil.
\verse And the priest that maketh him clean shall present the man that is to be made clean, and those things, before the \LORD, at the door of the tabernacle of the congregation:
\verse And the priest shall take one he lamb, and offer him for a trespass offering, and the log of oil, and wave them for a wave offering before the \LORD:
\verse And he shall slay the lamb in the place where he shall kill the sin offering and the burnt offering, in the holy place: for as the sin offering is the priest's, so is the trespass offering: it is most holy:
\verse And the priest shall take some of the blood of the trespass offering, and the priest shall put it upon the tip of the right ear of him that is to be cleansed, and upon the thumb of his right hand, and upon the great toe of his right foot:
\verse And the priest shall take some of the log of oil, and pour it into the palm of his own left hand:
\verse And the priest shall dip his right finger in the oil that is in his left hand, and shall sprinkle of the oil with his finger seven times before the \LORD:
\verse And of the rest of the oil that is in his hand shall the priest put upon the tip of the right ear of him that is to be cleansed, and upon the thumb of his right hand, and upon the great toe of his right foot, upon the blood of the trespass offering:
\verse And the remnant of the oil that is in the priest's hand he shall pour upon the head of him that is to be cleansed: and the priest shall make an atonement for him before the \LORD.
\verse And the priest shall offer the sin offering, and make an atonement for him that is to be cleansed from his uncleanness; and afterward he shall kill the burnt offering:
\verse And the priest shall offer the burnt offering and the meat offering upon the altar: and the priest shall make an atonement for him, and he shall be clean.
\verse And if he be poor, and cannot get so much; then he shall take one lamb for a trespass offering to be waved, to make an atonement for him, and one tenth deal of fine flour mingled with oil for a meat offering, and a log of oil;
\verse And two turtledoves, or two young pigeons, such as he is able to get; and the one shall be a sin offering, and the other a burnt offering.
\verse And he shall bring them on the eighth day for his cleansing unto the priest, unto the door of the tabernacle of the congregation, before the \LORD.
\verse And the priest shall take the lamb of the trespass offering, and the log of oil, and the priest shall wave them for a wave offering before the \LORD:
\verse And he shall kill the lamb of the trespass offering, and the priest shall take some of the blood of the trespass offering, and put it upon the tip of the right ear of him that is to be cleansed, and upon the thumb of his right hand, and upon the great toe of his right foot:
\verse And the priest shall pour of the oil into the palm of his own left hand:
\verse And the priest shall sprinkle with his right finger some of the oil that is in his left hand seven times before the \LORD:
\verse And the priest shall put of the oil that is in his hand upon the tip of the right ear of him that is to be cleansed, and upon the thumb of his right hand, and upon the great toe of his right foot, upon the place of the blood of the trespass offering:
\verse And the rest of the oil that is in the priest's hand he shall put upon the head of him that is to be cleansed, to make an atonement for him before the \LORD.
\verse And he shall offer the one of the turtledoves, or of the young pigeons, such as he can get;
\verse Even such as he is able to get, the one for a sin offering, and the other for a burnt offering, with the meat offering: and the priest shall make an atonement for him that is to be cleansed before the \LORD.
\verse This is the law of him in whom is the plague of leprosy, whose hand is not able to get that which pertaineth to his cleansing.
\verseWithHeading{Cleansing leprosy of the garments} And the \LORD spake unto Moses and unto Aaron, saying,
\verse When ye be come into the land of Canaan, which I give to you for a possession, and I put the plague of leprosy in a house of the land of your possession;
\verse And he that owneth the house shall come and tell the priest, saying, It seemeth to me there is as it were a plague in the house:
\verse Then the priest shall command that they empty the house, before the priest go into it to see the plague, that all that is in the house be not made unclean: and afterward the priest shall go in to see the house:
\verse And he shall look on the plague, and, behold, if the plague be in the walls of the house with hollow strakes, greenish or reddish, which in sight are lower than the wall;
\verse Then the priest shall go out of the house to the door of the house, and shut up the house seven days:
\verse And the priest shall come again the seventh day, and shall look: and, behold, if the plague be spread in the walls of the house;
\verse Then the priest shall command that they take away the stones in which the plague is, and they shall cast them into an unclean place without the city:
\verse And he shall cause the house to be scraped within round about, and they shall pour out the dust that they scrape off without the city into an unclean place:
\verse And they shall take other stones, and put them in the place of those stones; and he shall take other morter, and shall plaister the house.
\verse And if the plague come again, and break out in the house, after that he hath taken away the stones, and after he hath scraped the house, and after it is plaistered;
\verse Then the priest shall come and look, and, behold, if the plague be spread in the house, it is a fretting leprosy in the house: it is unclean.
\verse And he shall break down the house, the stones of it, and the timber thereof, and all the morter of the house; and he shall carry them forth out of the city into an unclean place.
\verse Moreover he that goeth into the house all the while that it is shut up shall be unclean until the even.
\verse And he that lieth in the house shall wash his clothes; and he that eateth in the house shall wash his clothes.
\verse And if the priest shall come in, and look upon it, and, behold, the plague hath not spread in the house, after the house was plaistered: then the priest shall pronounce the house clean, because the plague is healed.
\verse And he shall take to cleanse the house two birds, and cedar wood, and scarlet, and hyssop:
\verse And he shall kill the one of the birds in an earthen vessel over running water:
\verse And he shall take the cedar wood, and the hyssop, and the scarlet, and the living bird, and dip them in the blood of the slain bird, and in the running water, and sprinkle the house seven times:
\verse And he shall cleanse the house with the blood of the bird, and with the running water, and with the living bird, and with the cedar wood, and with the hyssop, and with the scarlet:
\verse But he shall let go the living bird out of the city into the open fields, and make an atonement for the house: and it shall be clean.
\verse This is the law for all manner of plague of leprosy, and scall,
\verse And for the leprosy of a garment, and of a house,
\verse And for a rising, and for a scab, and for a bright spot:
\verse To teach when it is unclean, and when it is clean: this is the law of leprosy.
\end{biblechapter}

\begin{biblechapter} % Leviticus 15
\verseWithHeading{Uncleanliness of issue} And the \LORD spake unto Moses and to Aaron, saying,
\verse Speak unto the children of Israel, and say unto them, When any man hath a running issue out of his flesh, because of his issue he is unclean.
\verse And this shall be his uncleanness in his issue: whether his flesh run with his issue, or his flesh be stopped from his issue, it is his uncleanness.
\verse Every bed, whereon he lieth that hath the issue, is unclean: and every thing, whereon he sitteth, shall be unclean.
\verse And whosoever toucheth his bed shall wash his clothes, and bathe himself in water, and be unclean until the even.
\verse And he that sitteth on any thing whereon he sat that hath the issue shall wash his clothes, and bathe himself in water, and be unclean until the even.
\verse And he that toucheth the flesh of him that hath the issue shall wash his clothes, and bathe himself in water, and be unclean until the even.
\verse And if he that hath the issue spit upon him that is clean; then he shall wash his clothes, and bathe himself in water, and be unclean until the even.
\verse And what saddle soever he rideth upon that hath the issue shall be unclean.
\verse And whosoever toucheth any thing that was under him shall be unclean until the even: and he that beareth any of those things shall wash his clothes, and bathe himself in water, and be unclean until the even.
\verse And whomsoever he toucheth that hath the issue, and hath not rinsed his hands in water, he shall wash his clothes, and bathe himself in water, and be unclean until the even.
\verse And the vessel of earth, that he toucheth which hath the issue, shall be broken: and every vessel of wood shall be rinsed in water.
\verse And when he that hath an issue is cleansed of his issue; then he shall number to himself seven days for his cleansing, and wash his clothes, and bathe his flesh in running water, and shall be clean.
\verse And on the eighth day he shall take to him two turtledoves, or two young pigeons, and come before the \LORD unto the door of the tabernacle of the congregation, and give them unto the priest:
\verse And the priest shall offer them, the one for a sin offering, and the other for a burnt offering; and the priest shall make an atonement for him before the \LORD for his issue.
\verse And if any man's seed of copulation go out from him, then he shall wash all his flesh in water, and be unclean until the even.
\verse And every garment, and every skin, whereon is the seed of copulation, shall be washed with water, and be unclean until the even.
\verse The woman also with whom man shall lie with seed of copulation, they shall both bathe themselves in water, and be unclean until the even.
\verse And if a woman have an issue, and her issue in her flesh be blood, she shall be put apart seven days: and whosoever toucheth her shall be unclean until the even.
\verse And every thing that she lieth upon in her separation shall be unclean: every thing also that she sitteth upon shall be unclean.
\verse And whosoever toucheth her bed shall wash his clothes, and bathe himself in water, and be unclean until the even.
\verse And whosoever toucheth any thing that she sat upon shall wash his clothes, and bathe himself in water, and be unclean until the even.
\verse And if it be on her bed, or on any thing whereon she sitteth, when he toucheth it, he shall be unclean until the even.
\verse And if any man lie with her at all, and her flowers be upon him, he shall be unclean seven days; and all the bed whereon he lieth shall be unclean.
\verse And if a woman have an issue of her blood many days out of the time of her separation, or if it run beyond the time of her separation; all the days of the issue of her uncleanness shall be as the days of her separation: she shall be unclean.
\verse Every bed whereon she lieth all the days of her issue shall be unto her as the bed of her separation: and whatsoever she sitteth upon shall be unclean, as the uncleanness of her separation.
\verse And whosoever toucheth those things shall be unclean, and shall wash his clothes, and bathe himself in water, and be unclean until the even.
\verse But if she be cleansed of her issue, then she shall number to herself seven days, and after that she shall be clean.
\verse And on the eighth day she shall take unto her two turtles, or two young pigeons, and bring them unto the priest, to the door of the tabernacle of the congregation.
\verse And the priest shall offer the one for a sin offering, and the other for a burnt offering; and the priest shall make an atonement for her before the \LORD for the issue of her uncleanness.
\verse Thus shall ye separate the children of Israel from their uncleanness; that they die not in their uncleanness, when they defile my tabernacle that is among them.
\verse This is the law of him that hath an issue, and of him whose seed goeth from him, and is defiled therewith;
\verse And of her that is sick of her flowers, and of him that hath an issue, of the man, and of the woman, and of him that lieth with her that is unclean.
\end{biblechapter}

\begin{biblechapter} % Leviticus 16
\verseWithHeading{The Day of Atonement} And the \LORD spake unto Moses after the death of the two sons of Aaron, when they offered before the \LORD, and died;
\verse And the \LORD said unto Moses, Speak unto Aaron thy brother, that he come not at all times into the holy place within the vail before the mercy seat, which is upon the ark; that he die not: for I will appear in the cloud upon the mercy seat.
\verse Thus shall Aaron come into the holy place: with a young bullock for a sin offering, and a ram for a burnt offering.
\verse He shall put on the holy linen coat, and he shall have the linen breeches upon his flesh, and shall be girded with a linen girdle, and with the linen mitre shall he be attired: these are holy garments; therefore shall he wash his flesh in water, and so put them on.
\verse And he shall take of the congregation of the children of Israel two kids of the goats for a sin offering, and one ram for a burnt offering.
\verse And Aaron shall offer his bullock of the sin offering, which is for himself, and make an atonement for himself, and for his house.
\verse And he shall take the two goats, and present them before the \LORD at the door of the tabernacle of the congregation.
\verse And Aaron shall cast lots upon the two goats; one lot for the \LORD, and the other lot for the scapegoat.
\verse And Aaron shall bring the goat upon which the \LORDs lot fell, and offer him for a sin offering.
\verse But the goat, on which the lot fell to be the scapegoat, shall be presented alive before the \LORD, to make an atonement with him, and to let him go for a scapegoat into the wilderness.
\verse And Aaron shall bring the bullock of the sin offering, which is for himself, and shall make an atonement for himself, and for his house, and shall kill the bullock of the sin offering which is for himself:
\verse And he shall take a censer full of burning coals of fire from off the altar before the \LORD, and his hands full of sweet incense beaten small, and bring it within the vail:
\verse And he shall put the incense upon the fire before the \LORD, that the cloud of the incense may cover the mercy seat that is upon the testimony, that he die not:
\verse And he shall take of the blood of the bullock, and sprinkle it with his finger upon the mercy seat eastward; and before the mercy seat shall he sprinkle of the blood with his finger seven times.
\verse Then shall he kill the goat of the sin offering, that is for the people, and bring his blood within the vail, and do with that blood as he did with the blood of the bullock, and sprinkle it upon the mercy seat, and before the mercy seat:
\verse And he shall make an atonement for the holy place, because of the uncleanness of the children of Israel, and because of their transgressions in all their sins: and so shall he do for the tabernacle of the congregation, that remaineth among them in the midst of their uncleanness.
\verse And there shall be no man in the tabernacle of the congregation when he goeth in to make an atonement in the holy place, until he come out, and have made an atonement for himself, and for his household, and for all the congregation of Israel.
\verse And he shall go out unto the altar that is before the \LORD, and make an atonement for it; and shall take of the blood of the bullock, and of the blood of the goat, and put it upon the horns of the altar round about.
\verse And he shall sprinkle of the blood upon it with his finger seven times, and cleanse it, and hallow it from the uncleanness of the children of Israel.
\verse And when he hath made an end of reconciling the holy place, and the tabernacle of the congregation, and the altar, he shall bring the live goat:
\verse And Aaron shall lay both his hands upon the head of the live goat, and confess over him all the iniquities of the children of Israel, and all their transgressions in all their sins, putting them upon the head of the goat, and shall send him away by the hand of a fit man into the wilderness:
\verse And the goat shall bear upon him all their iniquities unto a land not inhabited: and he shall let go the goat in the wilderness.
\verse And Aaron shall come into the tabernacle of the congregation, and shall put off the linen garments, which he put on when he went into the holy place, and shall leave them there:
\verse And he shall wash his flesh with water in the holy place, and put on his garments, and come forth, and offer his burnt offering, and the burnt offering of the people, and make an atonement for himself, and for the people.
\verse And the fat of the sin offering shall he burn upon the altar.
\verse And he that let go the goat for the scapegoat shall wash his clothes, and bathe his flesh in water, and afterward come into the camp.
\verse And the bullock for the sin offering, and the goat for the sin offering, whose blood was brought in to make atonement in the holy place, shall one carry forth without the camp; and they shall burn in the fire their skins, and their flesh, and their dung.
\verse And he that burneth them shall wash his clothes, and bathe his flesh in water, and afterward he shall come into the camp.
\verse And this shall be a statute for ever unto you: that in the seventh month, on the tenth day of the month, ye shall afflict your souls, and do no work at all, whether it be one of your own country, or a stranger that sojourneth among you:
\verse For on that day shall the priest make an atonement for you, to cleanse you, that ye may be clean from all your sins before the \LORD.
\verse It shall be a Sabbath of rest unto you, and ye shall afflict your souls, by a statute for ever.
\verse And the priest, whom he shall anoint, and whom he shall consecrate to minister in the priest's office in his father's stead, shall make the atonement, and shall put on the linen clothes, even the holy garments:
\verse And he shall make an atonement for the holy sanctuary, and he shall make an atonement for the tabernacle of the congregation, and for the altar, and he shall make an atonement for the priests, and for all the people of the congregation.
\verse And this shall be an everlasting statute unto you, to make an atonement for the children of Israel for all their sins once a year. And he did as the \LORD commanded Moses.
\end{biblechapter}

\begin{biblechapter} % Leviticus 17
\verseWithHeading{Eating blood forbidden} And the \LORD spake unto Moses, saying,
\verse Speak unto Aaron, and unto his sons, and unto all the children of Israel, and say unto them; This is the thing which the \LORD hath commanded, saying,
\verse What man soever there be of the house of Israel, that killeth an ox, or lamb, or goat, in the camp, or that killeth it out of the camp,
\verse And bringeth it not unto the door of the tabernacle of the congregation, to offer an offering unto the \LORD before the tabernacle of the \LORD; blood shall be imputed unto that man; he hath shed blood; and that man shall be cut off from among his people:
\verse To the end that the children of Israel may bring their sacrifices, which they offer in the open field, even that they may bring them unto the \LORD, unto the door of the tabernacle of the congregation, unto the priest, and offer them for peace offerings unto the \LORD.
\verse And the priest shall sprinkle the blood upon the altar of the \LORD at the door of the tabernacle of the congregation, and burn the fat for a sweet savour unto the \LORD.
\verse And they shall no more offer their sacrifices unto devils, after whom they have gone a whoring. This shall be a statute for ever unto them throughout their generations.
\verse And thou shalt say unto them, Whatsoever man there be of the house of Israel, or of the strangers which sojourn among you, that offereth a burnt offering or sacrifice,
\verse And bringeth it not unto the door of the tabernacle of the congregation, to offer it unto the \LORD; even that man shall be cut off from among his people.
\verse And whatsoever man there be of the house of Israel, or of the strangers that sojourn among you, that eateth any manner of blood; I will even set my face against that soul that eateth blood, and will cut him off from among his people.
\verse For the life of the flesh is in the blood: and I have given it to you upon the altar to make an atonement for your souls: for it is the blood that maketh an atonement for the soul.
\verse Therefore I said unto the children of Israel, No soul of you shall eat blood, neither shall any stranger that sojourneth among you eat blood.
\verse And whatsoever man there be of the children of Israel, or of the strangers that sojourn among you, which hunteth and catcheth any beast or fowl that may be eaten; he shall even pour out the blood thereof, and cover it with dust.
\verse For it is the life of all flesh; the blood of it is for the life thereof: therefore I said unto the children of Israel, Ye shall eat the blood of no manner of flesh: for the life of all flesh is the blood thereof: whosoever eateth it shall be cut off.
\verse And every soul that eateth that which died of itself, or that which was torn with beasts, whether it be one of your own country, or a stranger, he shall both wash his clothes, and bathe himself in water, and be unclean until the even: then shall he be clean.
\verse But if he wash them not, nor bathe his flesh; then he shall bear his iniquity.
\end{biblechapter}

\begin{biblechapter} % Leviticus 18
\verseWithHeading{Unlawful nakedness} And the \LORD spake unto Moses, saying,
\verse Speak unto the children of Israel, and say unto them, I am the \LORD your God.
\verse After the doings of the land of Egypt, wherein ye dwelt, shall ye not do: and after the doings of the land of Canaan, whither I bring you, shall ye not do: neither shall ye walk in their ordinances.
\verse Ye shall do my judgments, and keep mine ordinances, to walk therein: I am the \LORD your God.
\verse Ye shall therefore keep my statutes, and my judgments: which if a man do, he shall live in them: I am the \LORD.
\verse None of you shall approach to any that is near of kin to him, to uncover their nakedness: I am the \LORD.
\verse The nakedness of thy father, or the nakedness of thy mother, shalt thou not uncover: she is thy mother; thou shalt not uncover her nakedness.
\verse The nakedness of thy father's wife shalt thou not uncover: it is thy father's nakedness.
\verse The nakedness of thy sister, the daughter of thy father, or daughter of thy mother, whether she be born at home, or born abroad, even their nakedness thou shalt not uncover.
\verse The nakedness of thy son's daughter, or of thy daughter's daughter, even their nakedness thou shalt not uncover: for theirs is thine own nakedness.
\verse The nakedness of thy father's wife's daughter, begotten of thy father, she is thy sister, thou shalt not uncover her nakedness.
\verse Thou shalt not uncover the nakedness of thy father's sister: she is thy father's near kinswoman.
\verse Thou shalt not uncover the nakedness of thy mother's sister: for she is thy mother's near kinswoman.
\verse Thou shalt not uncover the nakedness of thy father's brother, thou shalt not approach to his wife: she is thine aunt.
\verse Thou shalt not uncover the nakedness of thy daughter in law: she is thy son's wife; thou shalt not uncover her nakedness.
\verse Thou shalt not uncover the nakedness of thy brother's wife: it is thy brother's nakedness.
\verse Thou shalt not uncover the nakedness of a woman and her daughter, neither shalt thou take her son's daughter, or her daughter's daughter, to uncover her nakedness; for they are her near kinswomen: it is wickedness.
\verse Neither shalt thou take a wife to her sister, to vex her, to uncover her nakedness, beside the other in her life time.
\verse Also thou shalt not approach unto a woman to uncover her nakedness, as long as she is put apart for her uncleanness.
\verse Moreover thou shalt not lie carnally with thy neighbour's wife, to defile thyself with her.
\verse And thou shalt not let any of thy seed pass through the fire to Molech, neither shalt thou profane the name of thy God: I am the \LORD.
\verse Thou shalt not lie with mankind, as with womankind: it is abomination.
\verse Neither shalt thou lie with any beast to defile thyself therewith: neither shall any woman stand before a beast to lie down thereto: it is confusion.
\verse Defile not ye yourselves in any of these things: for in all these the nations are defiled which I cast out before you:
\verse And the land is defiled: therefore I do visit the iniquity thereof upon it, and the land itself vomiteth out her inhabitants.
\verse Ye shall therefore keep my statutes and my judgments, and shall not commit any of these abominations; neither any of your own nation, nor any stranger that sojourneth among you:
\verse (For all these abominations have the men of the land done, which were before you, and the land is defiled;)
\verse That the land spue not you out also, when ye defile it, as it spued out the nations that were before you.
\verse For whosoever shall commit any of these abominations, even the souls that commit them shall be cut off from among their people.
\verse Therefore shall ye keep mine ordinance, that ye commit not any one of these abominable customs, which were committed before you, and that ye defile not yourselves therein: I am the \LORD your God.
\end{biblechapter}

\columnbreak % layout hack

\begin{biblechapter} % Leviticus 19
\verseWithHeading{Various laws} And the \LORD spake unto Moses, saying,
\verse Speak unto all the congregation of the children of Israel, and say unto them, Ye shall be holy: for I the \LORD your God am holy.
\verse Ye shall fear every man his mother, and his father, and keep my Sabbaths: I am the \LORD your God.
\verse Turn ye not unto idols, nor make to yourselves molten gods: I am the \LORD your God.
\verse And if ye offer a sacrifice of peace offerings unto the \LORD, ye shall offer it at your own will.
\verse It shall be eaten the same day ye offer it, and on the morrow: and if ought remain until the third day, it shall be burnt in the fire.
\verse And if it be eaten at all on the third day, it is abominable; it shall not be accepted.
\verse Therefore every one that eateth it shall bear his iniquity, because he hath profaned the hallowed thing of the \LORD: and that soul shall be cut off from among his people.
\verse And when ye reap the harvest of your land, thou shalt not wholly reap the corners of thy field, neither shalt thou gather the gleanings of thy harvest.
\verse And thou shalt not glean thy vineyard, neither shalt thou gather every grape of thy vineyard; thou shalt leave them for the poor and stranger: I am the \LORD your God.
\verse Ye shall not steal, neither deal falsely, neither lie one to another.
\verse And ye shall not swear by my name falsely, neither shalt thou profane the name of thy God: I am the \LORD.
\verse Thou shalt not defraud thy neighbour, neither rob him: the wages of him that is hired shall not abide with thee all night until the morning.
\verse Thou shalt not curse the deaf, nor put a stumblingblock before the blind, but shalt fear thy God: I am the \LORD.
\verse Ye shall do no unrighteousness in judgment: thou shalt not respect the person of the poor, nor honour the person of the mighty: but in righteousness shalt thou judge thy neighbour.
\verse Thou shalt not go up and down as a talebearer among thy people: neither shalt thou stand against the blood of thy neighbour: I am the \LORD.
\verse Thou shalt not hate thy brother in thine heart: thou shalt in any wise rebuke thy neighbour, and not suffer sin upon him.
\verse Thou shalt not avenge, nor bear any grudge against the children of thy people, but thou shalt love thy neighbour as thyself: I am the \LORD.
\verse Ye shall keep my statutes. Thou shalt not let thy cattle gender with a diverse kind: thou shalt not sow thy field with mingled seed: neither shall a garment mingled of linen and woollen come upon thee.
\verse And whosoever lieth carnally with a woman, that is a bondmaid, betrothed to an husband, and not at all redeemed, nor freedom given her; she shall be scourged; they shall not be put to death, because she was not free.
\verse And he shall bring his trespass offering unto the \LORD, unto the door of the tabernacle of the congregation, even a ram for a trespass offering.
\verse And the priest shall make an atonement for him with the ram of the trespass offering before the \LORD for his sin which he hath done: and the sin which he hath done shall be forgiven him.
\verse And when ye shall come into the land, and shall have planted all manner of trees for food, then ye shall count the fruit thereof as uncircumcised: three years shall it be as uncircumcised unto you: it shall not be eaten of.
\verse But in the fourth year all the fruit thereof shall be holy to praise the \LORD withal.
\verse And in the fifth year shall ye eat of the fruit thereof, that it may yield unto you the increase thereof: I am the \LORD your God.
\verse Ye shall not eat any thing with the blood: neither shall ye use enchantment, nor observe times.
\verse Ye shall not round the corners of your heads, neither shalt thou mar the corners of thy beard.
\verse Ye shall not make any cuttings in your flesh for the dead, nor print any marks upon you: I am the \LORD.
\verse Do not prostitute thy daughter, to cause her to be a whore; lest the land fall to whoredom, and the land become full of wickedness.
\verse Ye shall keep my Sabbaths, and reverence my sanctuary: I am the \LORD.
\verse Regard not them that have familiar spirits, neither seek after wizards, to be defiled by them: I am the \LORD your God.
\verse Thou shalt rise up before the hoary head, and honour the face of the old man, and fear thy God: I am the \LORD.
\verse And if a stranger sojourn with thee in your land, ye shall not vex him.
\verse But the stranger that dwelleth with you shall be unto you as one born among you, and thou shalt love him as thyself; for ye were strangers in the land of Egypt: I am the \LORD your God.
\verse Ye shall do no unrighteousness in judgment, in meteyard, in weight, or in measure.
\verse Just balances, just weights, a just ephah, and a just hin, shall ye have: I am the \LORD your God, which brought you out of the land of Egypt.
\verse Therefore shall ye observe all my statutes, and all my judgments, and do them: I am the \LORD.
\end{biblechapter}

\begin{biblechapter} % Leviticus 20
\verseWithHeading{Punishments for sin} And the \LORD spake unto Moses, saying,
\verse Again, thou shalt say to the children of Israel, Whosoever he be of the children of Israel, or of the strangers that sojourn in Israel, that giveth any of his seed unto Molech; he shall surely be put to death: the people of the land shall stone him with stones.
\verse And I will set my face against that man, and will cut him off from among his people; because he hath given of his seed unto Molech, to defile my sanctuary, and to profane my holy name.
\verse And if the people of the land do any ways hide their eyes from the man, when he giveth of his seed unto Molech, and kill him not:
\verse Then I will set my face against that man, and against his family, and will cut him off, and all that go a whoring after him, to commit whoredom with Molech, from among their people.
\verse And the soul that turneth after such as have familiar spirits, and after wizards, to go a whoring after them, I will even set my face against that soul, and will cut him off from among his people.
\verse Sanctify yourselves therefore, and be ye holy: for I am the \LORD your God.
\verse And ye shall keep my statutes, and do them: I am the \LORD which sanctify you.
\verse For every one that curseth his father or his mother shall be surely put to death: he hath cursed his father or his mother; his blood shall be upon him.
\verse And the man that committeth adultery with another man's wife, even he that committeth adultery with his neighbour's wife, the adulterer and the adulteress shall surely be put to death.
\verse And the man that lieth with his father's wife hath uncovered his father's nakedness: both of them shall surely be put to death; their blood shall be upon them.
\verse And if a man lie with his daughter in law, both of them shall surely be put to death: they have wrought confusion; their blood shall be upon them.
\verse If a man also lie with mankind, as he lieth with a woman, both of them have committed an abomination: they shall surely be put to death; their blood shall be upon them.
\verse And if a man take a wife and her mother, it is wickedness: they shall be burnt with fire, both he and they; that there be no wickedness among you.
\verse And if a man lie with a beast, he shall surely be put to death: and ye shall slay the beast.
\verse And if a woman approach unto any beast, and lie down thereto, thou shalt kill the woman, and the beast: they shall surely be put to death; their blood shall be upon them.
\verse And if a man shall take his sister, his father's daughter, or his mother's daughter, and see her nakedness, and she see his nakedness; it is a wicked thing; and they shall be cut off in the sight of their people: he hath uncovered his sister's nakedness; he shall bear his iniquity.
\verse And if a man shall lie with a woman having her sickness, and shall uncover her nakedness; he hath discovered her fountain, and she hath uncovered the fountain of her blood: and both of them shall be cut off from among their people.
\verse And thou shalt not uncover the nakedness of thy mother's sister, nor of thy father's sister: for he uncovereth his near kin: they shall bear their iniquity.
\verse And if a man shall lie with his uncle's wife, he hath uncovered his uncle's nakedness: they shall bear their sin; they shall die childless.
\verse And if a man shall take his brother's wife, it is an unclean thing: he hath uncovered his brother's nakedness; they shall be childless.
\verse Ye shall therefore keep all my statutes, and all my judgments, and do them: that the land, whither I bring you to dwell therein, spue you not out.
\verse And ye shall not walk in the manners of the nation, which I cast out before you: for they committed all these things, and therefore I abhorred them.
\verse But I have said unto you, Ye shall inherit their land, and I will give it unto you to possess it, a land that floweth with milk and honey: I am the \LORD your God, which have separated you from other people.
\verse Ye shall therefore put difference between clean beasts and unclean, and between unclean fowls and clean: and ye shall not make your souls abominable by beast, or by fowl, or by any manner of living thing that creepeth on the ground, which I have separated from you as unclean.
\verse And ye shall be holy unto me: for I the \LORD am holy, and have severed you from other people, that ye should be mine.
\verse A man also or woman that hath a familiar spirit, or that is a wizard, shall surely be put to death: they shall stone them with stones: their blood shall be upon them.
\end{biblechapter}

\begin{biblechapter} % Leviticus 21
\verseWithHeading{Rules for priests} And the \LORD said unto Moses, Speak unto the priests the sons of Aaron, and say unto them, There shall none be defiled for the dead among his people:
\verse But for his kin, that is near unto him, that is, for his mother, and for his father, and for his son, and for his daughter, and for his brother,
\verse And for his sister a virgin, that is nigh unto him, which hath had no husband; for her may he be defiled.
\verse But he shall not defile himself, being a chief man among his people, to profane himself.
\verse They shall not make baldness upon their head, neither shall they shave off the corner of their beard, nor make any cuttings in their flesh.
\verse They shall be holy unto their God, and not profane the name of their God: for the offerings of the \LORD made by fire, and the bread of their God, they do offer: therefore they shall be holy.
\verse They shall not take a wife that is a whore, or profane; neither shall they take a woman put away from her husband: for he is holy unto his God.
\verse Thou shalt sanctify him therefore; for he offereth the bread of thy God: he shall be holy unto thee: for I the \LORD, which sanctify you, am holy.
\verse And the daughter of any priest, if she profane herself by playing the whore, she profaneth her father: she shall be burnt with fire.
\verse And he that is the high priest among his brethren, upon whose head the anointing oil was poured, and that is consecrated to put on the garments, shall not uncover his head, nor rend his clothes;
\verse Neither shall he go in to any dead body, nor defile himself for his father, or for his mother;
\verse Neither shall he go out of the sanctuary, nor profane the sanctuary of his God; for the crown of the anointing oil of his God is upon him: I am the \LORD.
\verse And he shall take a wife in her virginity.
\verse A widow, or a divorced woman, or profane, or an harlot, these shall he not take: but he shall take a virgin of his own people to wife.
\verse Neither shall he profane his seed among his people: for I the \LORD do sanctify him.
\verse And the \LORD spake unto Moses, saying,
\verse Speak unto Aaron, saying, Whosoever he be of thy seed in their generations that hath any blemish, let him not approach to offer the bread of his God.
\verse For whatsoever man he be that hath a blemish, he shall not approach: a blind man, or a lame, or he that hath a flat nose, or any thing superfluous,
\verse Or a man that is brokenfooted, or brokenhanded,
\verse Or crookbackt, or a dwarf, or that hath a blemish in his eye, or be scurvy, or scabbed, or hath his stones broken;
\verse No man that hath a blemish of the seed of Aaron the priest shall come nigh to offer the offerings of the \LORD made by fire: he hath a blemish; he shall not come nigh to offer the bread of his God.
\verse He shall eat the bread of his God, both of the most holy, and of the holy.
\verse Only he shall not go in unto the vail, nor come nigh unto the altar, because he hath a blemish; that he profane not my sanctuaries: for I the \LORD do sanctify them.
\verse And Moses told it unto Aaron, and to his sons, and unto all the children of Israel.
\end{biblechapter}

\begin{biblechapter} % Leviticus 22
\verse And the \LORD spake unto Moses, saying,
\verse Speak unto Aaron and to his sons, that they separate themselves from the holy things of the children of Israel, and that they profane not my holy name in those things which they hallow unto me: I am the \LORD.
\verse Say unto them, Whosoever he be of all your seed among your generations, that goeth unto the holy things, which the children of Israel hallow unto the \LORD, having his uncleanness upon him, that soul shall be cut off from my presence: I am the \LORD.
\verse What man soever of the seed of Aaron is a leper, or hath a running issue; he shall not eat of the holy things, until he be clean. And whoso toucheth any thing that is unclean by the dead, or a man whose seed goeth from him;
\verse Or whosoever toucheth any creeping thing, whereby he may be made unclean, or a man of whom he may take uncleanness, whatsoever uncleanness he hath;
\verse The soul which hath touched any such shall be unclean until even, and shall not eat of the holy things, unless he wash his flesh with water.
\verse And when the sun is down, he shall be clean, and shall afterward eat of the holy things; because it is his food.
\verse That which dieth of itself, or is torn with beasts, he shall not eat to defile himself therewith: I am the \LORD.
\verse They shall therefore keep mine ordinance, lest they bear sin for it, and die therefore, if they profane it: I the \LORD do sanctify them.
\verse There shall no stranger eat of the holy thing: a sojourner of the priest, or an hired servant, shall not eat of the holy thing.
\verse But if the priest buy any soul with his money, he shall eat of it, and he that is born in his house: they shall eat of his meat.
\verse If the priest's daughter also be married unto a stranger, she may not eat of an offering of the holy things.
\verse But if the priest's daughter be a widow, or divorced, and have no child, and is returned unto her father's house, as in her youth, she shall eat of her father's meat: but there shall no stranger eat thereof.
\verse And if a man eat of the holy thing unwittingly, then he shall put the fifth part thereof unto it, and shall give it unto the priest with the holy thing.
\verse And they shall not profane the holy things of the children of Israel, which they offer unto the \LORD;
\verse Or suffer them to bear the iniquity of trespass, when they eat their holy things: for I the \LORD do sanctify them.
\verseWithHeading{Unacceptable sacrifices} And the \LORD spake unto Moses, saying,
\verse Speak unto Aaron, and to his sons, and unto all the children of Israel, and say unto them, Whatsoever he be of the house of Israel, or of the strangers in Israel, that will offer his oblation for all his vows, and for all his freewill offerings, which they will offer unto the \LORD for a burnt offering;
\verse Ye shall offer at your own will a male without blemish, of the beeves, of the sheep, or of the goats.
\verse But whatsoever hath a blemish, that shall ye not offer: for it shall not be acceptable for you.
\verse And whosoever offereth a sacrifice of peace offerings unto the \LORD to accomplish his vow, or a freewill offering in beeves or sheep, it shall be perfect to be accepted; there shall be no blemish therein.
\verse Blind, or broken, or maimed, or having a wen, or scurvy, or scabbed, ye shall not offer these unto the \LORD, nor make an offering by fire of them upon the altar unto the \LORD.
\verse Either a bullock or a lamb that hath any thing superfluous or lacking in his parts, that mayest thou offer for a freewill offering; but for a vow it shall not be accepted.
\verse Ye shall not offer unto the \LORD that which is bruised, or crushed, or broken, or cut; neither shall ye make any offering thereof in your land.
\verse Neither from a stranger's hand shall ye offer the bread of your God of any of these; because their corruption is in them, and blemishes be in them: they shall not be accepted for you.
\verse And the \LORD spake unto Moses, saying,
\verse When a bullock, or a sheep, or a goat, is brought forth, then it shall be seven days under the dam; and from the eighth day and thenceforth it shall be accepted for an offering made by fire unto the \LORD.
\verse And whether it be cow or ewe, ye shall not kill it and her young both in one day.
\verse And when ye will offer a sacrifice of thanksgiving unto the \LORD, offer it at your own will.
\verse On the same day it shall be eaten up; ye shall leave none of it until the morrow: I am the \LORD.
\verse Therefore shall ye keep my commandments, and do them: I am the \LORD.
\verse Neither shall ye profane my holy name; but I will be hallowed among the children of Israel: I am the \LORD which hallow you,
\verse That brought you out of the land of Egypt, to be your God: I am the \LORD.
\end{biblechapter}

\begin{biblechapter} % Leviticus 23
\verseWithHeading{The appointed feasts} And the \LORD spake unto Moses, saying,
\verse Speak unto the children of Israel, and say unto them, Concerning the feasts of the \LORD, which ye shall proclaim to be holy convocations, even these are my feasts.
\verseWithSubheading{The Sabbath} Six days shall work be done: but the seventh day is the Sabbath of rest, an holy convocation; ye shall do no work therein: it is the Sabbath of the \LORD in all your dwellings.
\verseWithSubheading{The Passover and Unleavened Bread} These are the feasts of the \LORD, even holy convocations, which ye shall proclaim in their seasons.
\verse In the fourteenth day of the first month at even is the \LORDs Passover.
\verse And on the fifteenth day of the same month is the feast of unleavened bread unto the \LORD: seven days ye must eat unleavened bread.
\verse In the first day ye shall have an holy convocation: ye shall do no servile work therein.
\verse But ye shall offer an offering made by fire unto the \LORD seven days: in the seventh day is an holy convocation: ye shall do no servile work therein.
\verseWithSubheading{Offering the Firstfruits} And the \LORD spake unto Moses, saying,
\verse Speak unto the children of Israel, and say unto them, When ye be come into the land which I give unto you, and shall reap the harvest thereof, then ye shall bring a sheaf of the firstfruits of your harvest unto the priest:
\verse And he shall wave the sheaf before the \LORD, to be accepted for you: on the morrow after the Sabbath the priest shall wave it.
\verse And ye shall offer that day when ye wave the sheaf an he lamb without blemish of the first year for a burnt offering unto the \LORD.
\verse And the meat offering thereof shall be two tenth deals of fine flour mingled with oil, an offering made by fire unto the \LORD for a sweet savour: and the drink offering thereof shall be of wine, the fourth part of an hin.
\verse And ye shall eat neither bread, nor parched corn, nor green ears, until the selfsame day that ye have brought an offering unto your God: it shall be a statute for ever throughout your generations in all your dwellings.
\verseWithSubheading{The Feast of Weeks} And ye shall count unto you from the morrow after the Sabbath, from the day that ye brought the sheaf of the wave offering; seven Sabbaths shall be complete:
\verse Even unto the morrow after the seventh Sabbath shall ye number fifty days; and ye shall offer a new meat offering unto the \LORD.
\verse Ye shall bring out of your habitations two wave loaves of two tenth deals: they shall be of fine flour; they shall be baken with leaven; they are the firstfruits unto the \LORD.
\verse And ye shall offer with the bread seven lambs without blemish of the first year, and one young bullock, and two rams: they shall be for a burnt offering unto the \LORD, with their meat offering, and their drink offerings, even an offering made by fire, of sweet savour unto the \LORD.
\verse Then ye shall sacrifice one kid of the goats for a sin offering, and two lambs of the first year for a sacrifice of peace offerings.
\verse And the priest shall wave them with the bread of the firstfruits for a wave offering before the \LORD, with the two lambs: they shall be holy to the \LORD for the priest.
\verse And ye shall proclaim on the selfsame day, that it may be an holy convocation unto you: ye shall do no servile work therein: it shall be a statute for ever in all your dwellings throughout your generations.
\verse And when ye reap the harvest of your land, thou shalt not make clean riddance of the corners of thy field when thou reapest, neither shalt thou gather any gleaning of thy harvest: thou shalt leave them unto the poor, and to the stranger: I am the \LORD your God.
\verseWithSubheading{The Feast of Trumpets} And the \LORD spake unto Moses, saying,
\verse Speak unto the children of Israel, saying, In the seventh month, in the first day of the month, shall ye have a Sabbath, a memorial of blowing of trumpets, an holy convocation.
\verse Ye shall do no servile work therein: but ye shall offer an offering made by fire unto the \LORD.
\verseWithSubheading{The Day of Atonement} And the \LORD spake unto Moses, saying,
\verse Also on the tenth day of this seventh month there shall be a day of atonement: it shall be an holy convocation unto you; and ye shall afflict your souls, and offer an offering made by fire unto the \LORD.
\verse And ye shall do no work in that same day: for it is a day of atonement, to make an atonement for you before the \LORD your God.
\verse For whatsoever soul it be that shall not be afflicted in that same day, he shall be cut off from among his people.
\verse And whatsoever soul it be that doeth any work in that same day, the same soul will I destroy from among his people.
\verse Ye shall do no manner of work: it shall be a statute for ever throughout your generations in all your dwellings.
\verse It shall be unto you a Sabbath of rest, and ye shall afflict your souls: in the ninth day of the month at even, from even unto even, shall ye celebrate your Sabbath.
\verseWithSubheading{The Feast of Tabernacles} And the \LORD spake unto Moses, saying,
\verse Speak unto the children of Israel, saying, The fifteenth day of this seventh month shall be the feast of tabernacles for seven days unto the \LORD.
\verse On the first day shall be an holy convocation: ye shall do no servile work therein.
\verse Seven days ye shall offer an offering made by fire unto the \LORD: on the eighth day shall be an holy convocation unto you; and ye shall offer an offering made by fire unto the \LORD: it is a solemn assembly; and ye shall do no servile work therein.
\verse These are the feasts of the \LORD, which ye shall proclaim to be holy convocations, to offer an offering made by fire unto the \LORD, a burnt offering, and a meat offering, a sacrifice, and drink offerings, every thing upon his day:
\verse Beside the Sabbaths of the \LORD, and beside your gifts, and beside all your vows, and beside all your freewill offerings, which ye give unto the \LORD.
\verse Also in the fifteenth day of the seventh month, when ye have gathered in the fruit of the land, ye shall keep a feast unto the \LORD seven days: on the first day shall be a Sabbath, and on the eighth day shall be a Sabbath.
\verse And ye shall take you on the first day the boughs of goodly trees, branches of palm trees, and the boughs of thick trees, and willows of the brook; and ye shall rejoice before the \LORD your God seven days.
\verse And ye shall keep it a feast unto the \LORD seven days in the year. It shall be a statute for ever in your generations: ye shall celebrate it in the seventh month.
\verse Ye shall dwell in booths seven days; all that are Israelites born shall dwell in booths:
\verse That your generations may know that I made the children of Israel to dwell in booths, when I brought them out of the land of Egypt: I am the \LORD your God.
\verse And Moses declared unto the children of Israel the feasts of the \LORD.
\end{biblechapter}

\begin{biblechapter} % Leviticus 24
\verseWithHeading{Olive oil and bread set \newline before the \LORD} And the \LORD spake unto Moses, saying,
\verse Command the children of Israel, that they bring unto thee pure oil olive beaten for the light, to cause the lamps to burn continually.
\verse Without the vail of the testimony, in the tabernacle of the congregation, shall Aaron order it from the evening unto the morning before the \LORD continually: it shall be a statute for ever in your generations.
\verse He shall order the lamps upon the pure candlestick before the \LORD continually.
\verse And thou shalt take fine flour, and bake twelve cakes thereof: two tenth deals shall be in one cake.
\verse And thou shalt set them in two rows, six on a row, upon the pure table before the \LORD.
\verse And thou shalt put pure frankincense upon each row, that it may be on the bread for a memorial, even an offering made by fire unto the \LORD.
\verse Every Sabbath he shall set it in order before the \LORD continually, being taken from the children of Israel by an everlasting covenant.
\verse And it shall be Aaron's and his sons'; and they shall eat it in the holy place: for it is most holy unto him of the offerings of the \LORD made by fire by a perpetual statute.
\verseWithHeading{A blasphemer put to death} And the son of an Israelitish woman, whose father was an Egyptian, went out among the children of Israel: and this son of the Israelitish woman and a man of Israel strove together in the camp;
\verse And the Israelitish woman's son blasphemed the name of the \LORD, and cursed. And they brought him unto Moses: (and his mother's name was Shelomith, the daughter of Dibri, of the tribe of Dan:)
\verse And they put him in ward, that the mind of the \LORD might be shewed them.
\verse And the \LORD spake unto Moses, saying,
\verse Bring forth him that hath cursed without the camp; and let all that heard him lay their hands upon his head, and let all the congregation stone him.
\verse And thou shalt speak unto the children of Israel, saying, Whosoever curseth his God shall bear his sin.
\verse And he that blasphemeth the name of the \LORD, he shall surely be put to death, and all the congregation shall certainly stone him: as well the stranger, as he that is born in the land, when he blasphemeth the name of the \LORD, shall be put to death.
\verse And he that killeth any man shall surely be put to death.
\verse And he that killeth a beast shall make it good; beast for beast.
\verse And if a man cause a blemish in his neighbour; as he hath done, so shall it be done to him;
\verse Breach for breach, eye for eye, tooth for tooth: as he hath caused a blemish in a man, so shall it be done to him again.
\verse And he that killeth a beast, he shall restore it: and he that killeth a man, he shall be put to death.
\verse Ye shall have one manner of law, as well for the stranger, as for one of your own country: for I am the \LORD your God.
\verse And Moses spake to the children of Israel, that they should bring forth him that had cursed out of the camp, and stone him with stones. And the children of Israel did as the \LORD commanded Moses.
\end{biblechapter}

\begin{biblechapter} % Leviticus 25
\verseWithHeading{The Sabbath Year} And the \LORD spake unto Moses in mount Sinai, saying,
\verse Speak unto the children of Israel, and say unto them, When ye come into the land which I give you, then shall the land keep a Sabbath unto the \LORD.
\verse Six years thou shalt sow thy field, and six years thou shalt prune thy vineyard, and gather in the fruit thereof;
\verse But in the seventh year shall be a Sabbath of rest unto the land, a Sabbath for the \LORD: thou shalt neither sow thy field, nor prune thy vineyard.
\verse That which groweth of its own accord of thy harvest thou shalt not reap, neither gather the grapes of thy vine undressed: for it is a year of rest unto the land.
\verse And the Sabbath of the land shall be meat for you; for thee, and for thy servant, and for thy maid, and for thy hired servant, and for thy stranger that sojourneth with thee,
\verse And for thy cattle, and for the beast that are in thy land, shall all the increase thereof be meat.
\flushcolsend\columnbreak % layout hack
\verseWithHeading{The Year of Jubilee} And thou shalt number seven Sabbaths of years unto thee, seven times seven years; and the space of the seven Sabbaths of years shall be unto thee forty and nine years.
\verse Then shalt thou cause the trumpet of the jubile to sound on the tenth day of the seventh month, in the day of atonement shall ye make the trumpet sound throughout all your land.
\verse And ye shall hallow the fiftieth year, and proclaim liberty throughout all the land unto all the inhabitants thereof: it shall be a jubile unto you; and ye shall return every man unto his possession, and ye shall return every man unto his family.
\verse A jubile shall that fiftieth year be unto you: ye shall not sow, neither reap that which groweth of itself in it, nor gather the grapes in it of thy vine undressed.
\verse For it is the jubile; it shall be holy unto you: ye shall eat the increase thereof out of the field.
\verse In the year of this jubile ye shall return every man unto his possession.
\verse And if thou sell ought unto thy neighbour, or buyest ought of thy neighbour's hand, ye shall not oppress one another:
\verse According to the number of years after the jubile thou shalt buy of thy neighbour, and according unto the number of years of the fruits he shall sell unto thee:
\verse According to the multitude of years thou shalt increase the price thereof, and according to the fewness of years thou shalt diminish the price of it: for according to the number of the years of the fruits doth he sell unto thee.
\verse Ye shall not therefore oppress one another; but thou shalt fear thy God: for I am the \LORD your God.
\verse Wherefore ye shall do my statutes, and keep my judgments, and do them; and ye shall dwell in the land in safety.
\verse And the land shall yield her fruit, and ye shall eat your fill, and dwell therein in safety.
\verse And if ye shall say, What shall we eat the seventh year? behold, we shall not sow, nor gather in our increase:
\verse Then I will command my blessing upon you in the sixth year, and it shall bring forth fruit for three years.
\verse And ye shall sow the eighth year, and eat yet of old fruit until the ninth year; until her fruits come in ye shall eat of the old store.
\verse The land shall not be sold for ever: for the land is mine; for ye are strangers and sojourners with me.
\verse And in all the land of your possession ye shall grant a redemption for the land.
\verse If thy brother be waxen poor, and hath sold away some of his possession, and if any of his kin come to redeem it, then shall he redeem that which his brother sold.
\verse And if the man have none to redeem it, and himself be able to redeem it;
\verse Then let him count the years of the sale thereof, and restore the overplus unto the man to whom he sold it; that he may return unto his possession.
\verse But if he be not able to restore it to him, then that which is sold shall remain in the hand of him that hath bought it until the year of jubile: and in the jubile it shall go out, and he shall return unto his possession.
\verse And if a man sell a dwelling house in a walled city, then he may redeem it within a whole year after it is sold; within a full year may he redeem it.
\verse And if it be not redeemed within the space of a full year, then the house that is in the walled city shall be established for ever to him that bought it throughout his generations: it shall not go out in the jubile.
\verse But the houses of the villages which have no wall round about them shall be counted as the fields of the country: they may be redeemed, and they shall go out in the jubile.
\verse Notwithstanding the cities of the Levites, and the houses of the cities of their possession, may the Levites redeem at any time.
\verse And if a man purchase of the Levites, then the house that was sold, and the city of his possession, shall go out in the year of jubile: for the houses of the cities of the Levites are their possession among the children of Israel.
\verse But the field of the suburbs of their cities may not be sold; for it is their perpetual possession.
\verse And if thy brother be waxen poor, and fallen in decay with thee; then thou shalt relieve him: yea, though he be a stranger, or a sojourner; that he may live with thee.
\verse Take thou no usury of him, or increase: but fear thy God; that thy brother may live with thee.
\verse Thou shalt not give him thy money upon usury, nor lend him thy victuals for increase.
\verse I am the \LORD your God, which brought you forth out of the land of Egypt, to give you the land of Canaan, and to be your God.
\verse And if thy brother that dwelleth by thee be waxen poor, and be sold unto thee; thou shalt not compel him to serve as a bondservant:
\verse But as an hired servant, and as a sojourner, he shall be with thee, and shall serve thee unto the year of jubile:
\verse And then shall he depart from thee, both he and his children with him, and shall return unto his own family, and unto the possession of his fathers shall he return.
\verse For they are my servants, which I brought forth out of the land of Egypt: they shall not be sold as bondmen.
\verse Thou shalt not rule over him with rigour; but shalt fear thy God.
\verse Both thy bondmen, and thy bondmaids, which thou shalt have, shall be of the heathen that are round about you; of them shall ye buy bondmen and bondmaids.
\verse Moreover of the children of the strangers that do sojourn among you, of them shall ye buy, and of their families that are with you, which they begat in your land: and they shall be your possession.
\verse And ye shall take them as an inheritance for your children after you, to inherit them for a possession; they shall be your bondmen for ever: but over your brethren the children of Israel, ye shall not rule one over another with rigour.
\verse And if a sojourner or stranger wax rich by thee, and thy brother that dwelleth by him wax poor, and sell himself unto the stranger or sojourner by thee, or to the stock of the stranger's family:
\verse After that he is sold he may be redeemed again; one of his brethren may redeem him:
\verse Either his uncle, or his uncle's son, may redeem him, or any that is nigh of kin unto him of his family may redeem him; or if he be able, he may redeem himself.
\verse And he shall reckon with him that bought him from the year that he was sold to him unto the year of jubile: and the price of his sale shall be according unto the number of years, according to the time of an hired servant shall it be with him.
\verse If there be yet many years behind, according unto them he shall give again the price of his redemption out of the money that he was bought for.
\verse And if there remain but few years unto the year of jubile, then he shall count with him, and according unto his years shall he give him again the price of his redemption.
\verse And as a yearly hired servant shall he be with him: and the other shall not rule with rigour over him in thy sight.
\verse And if he be not redeemed in these years, then he shall go out in the year of jubile, both he, and his children with him.
\verse For unto me the children of Israel are servants; they are my servants whom I brought forth out of the land of Egypt: I am the \LORD your God.
\end{biblechapter}

\begin{biblechapter} % Leviticus 26
\verseWithHeading{Reward for obedience} Ye shall make you no idols nor graven image, neither rear you up a standing image, neither shall ye set up any image of stone in your land, to bow down unto it: for I am the \LORD your God.
\verse Ye shall keep my Sabbaths, and reverence my sanctuary: I am the \LORD.
\verse If ye walk in my statutes, and keep my commandments, and do them;
\verse Then I will give you rain in due season, and the land shall yield her increase, and the trees of the field shall yield their fruit.
\verse And your threshing shall reach unto the vintage, and the vintage shall reach unto the sowing time: and ye shall eat your bread to the full, and dwell in your land safely.
\verse And I will give peace in the land, and ye shall lie down, and none shall make you afraid: and I will rid evil beasts out of the land, neither shall the sword go through your land.
\verse And ye shall chase your enemies, and they shall fall before you by the sword.
\verse And five of you shall chase an hundred, and an hundred of you shall put ten thousand to flight: and your enemies shall fall before you by the sword.
\verse For I will have respect unto you, and make you fruitful, and multiply you, and establish my covenant with you.
\verse And ye shall eat old store, and bring forth the old because of the new.
\verse And I will set my tabernacle among you: and my soul shall not abhor you.
\verse And I will walk among you, and will be your God, and ye shall be my people.
\verse I am the \LORD your God, which brought you forth out of the land of Egypt, that ye should not be their bondmen; and I have broken the bands of your yoke, and made you go upright.
\verseWithHeading{Punishment for disobedience} But if ye will not hearken unto me, and will not do all these commandments;
\verse And if ye shall despise my statutes, or if your soul abhor my judgments, so that ye will not do all my commandments, but that ye break my covenant:
\verse I also will do this unto you; I will even appoint over you terror, consumption, and the burning ague, that shall consume the eyes, and cause sorrow of heart: and ye shall sow your seed in vain, for your enemies shall eat it.
\verse And I will set my face against you, and ye shall be slain before your enemies: they that hate you shall reign over you; and ye shall flee when none pursueth you.
\verse And if ye will not yet for all this hearken unto me, then I will punish you seven times more for your sins.
\verse And I will break the pride of your power; and I will make your heaven as iron, and your earth as brass:
\verse And your strength shall be spent in vain: for your land shall not yield her increase, neither shall the trees of the land yield their fruits.
\verse And if ye walk contrary unto me, and will not hearken unto me; I will bring seven times more plagues upon you according to your sins.
\verse I will also send wild beasts among you, which shall rob you of your children, and destroy your cattle, and make you few in number; and your high ways shall be desolate.
\verse And if ye will not be reformed by me by these things, but will walk contrary unto me;
\verse Then will I also walk contrary unto you, and will punish you yet seven times for your sins.
\verse And I will bring a sword upon you, that shall avenge the quarrel of my covenant: and when ye are gathered together within your cities, I will send the pestilence among you; and ye shall be delivered into the hand of the enemy.
\verse And when I have broken the staff of your bread, ten women shall bake your bread in one oven, and they shall deliver you your bread again by weight: and ye shall eat, and not be satisfied.
\verse And if ye will not for all this hearken unto me, but walk contrary unto me;
\verse Then I will walk contrary unto you also in fury; and I, even I, will chastise you seven times for your sins.
\verse And ye shall eat the flesh of your sons, and the flesh of your daughters shall ye eat.
\verse And I will destroy your high places, and cut down your images, and cast your carcases upon the carcases of your idols, and my soul shall abhor you.
\verse And I will make your cities waste, and bring your sanctuaries unto desolation, and I will not smell the savour of your sweet odours.
\verse And I will bring the land into desolation: and your enemies which dwell therein shall be astonished at it.
\verse And I will scatter you among the heathen, and will draw out a sword after you: and your land shall be desolate, and your cities waste.
\verse Then shall the land enjoy her Sabbaths, as long as it lieth desolate, and ye be in your enemies' land; even then shall the land rest, and enjoy her Sabbaths.
\verse As long as it lieth desolate it shall rest; because it did not rest in your Sabbaths, when ye dwelt upon it.
\verse And upon them that are left alive of you I will send a faintness into their hearts in the lands of their enemies; and the sound of a shaken leaf shall chase them; and they shall flee, as fleeing from a sword; and they shall fall when none pursueth.
\verse And they shall fall one upon another, as it were before a sword, when none pursueth: and ye shall have no power to stand before your enemies.
\verse And ye shall perish among the heathen, and the land of your enemies shall eat you up.
\verse And they that are left of you shall pine away in their iniquity in your enemies' lands; and also in the iniquities of their fathers shall they pine away with them.
\verse If they shall confess their iniquity, and the iniquity of their fathers, with their trespass which they trespassed against me, and that also they have walked contrary unto me;
\verse And that I also have walked contrary unto them, and have brought them into the land of their enemies; if then their uncircumcised hearts be humbled, and they then accept of the punishment of their iniquity:
\verse Then will I remember my covenant with Jacob, and also my covenant with Isaac, and also my covenant with Abraham will I remember; and I will remember the land.
\verse The land also shall be left of them, and shall enjoy her Sabbaths, while she lieth desolate without them: and they shall accept of the punishment of their iniquity: because, even because they despised my judgments, and because their soul abhorred my statutes.
\verse And yet for all that, when they be in the land of their enemies, I will not cast them away, neither will I abhor them, to destroy them utterly, and to break my covenant with them: for I am the \LORD their God.
\verse But I will for their sakes remember the covenant of their ancestors, whom I brought forth out of the land of Egypt in the sight of the heathen, that I might be their God: I am the \LORD.
\verse These are the statutes and judgments and laws, which the \LORD made between him and the children of Israel in mount Sinai by the hand of Moses.
\end{biblechapter}

\begin{biblechapter} % Leviticus 27
\verseWithHeading{Redeeming what is the \newline \LORDs} And the \LORD spake unto Moses, saying,
\verse Speak unto the children of Israel, and say unto them, When a man shall make a singular vow, the persons shall be for the \LORD by thy estimation.
\verse And thy estimation shall be of the male from twenty years old even unto sixty years old, even thy estimation shall be fifty shekels of silver, after the shekel of the sanctuary.
\verse And if it be a female, then thy estimation shall be thirty shekels.
\verse And if it be from five years old even unto twenty years old, then thy estimation shall be of the male twenty shekels, and for the female ten shekels.
\verse And if it be from a month old even unto five years old, then thy estimation shall be of the male five shekels of silver, and for the female thy estimation shall be three shekels of silver.
\verse And if it be from sixty years old and above; if it be a male, then thy estimation shall be fifteen shekels, and for the female ten shekels.
\verse But if he be poorer than thy estimation, then he shall present himself before the priest, and the priest shall value him; according to his ability that vowed shall the priest value him.
\verse And if it be a beast, whereof men bring an offering unto the \LORD, all that any man giveth of such unto the \LORD shall be holy.
\verse He shall not alter it, nor change it, a good for a bad, or a bad for a good: and if he shall at all change beast for beast, then it and the exchange thereof shall be holy.
\verse And if it be any unclean beast, of which they do not offer a sacrifice unto the \LORD, then he shall present the beast before the priest:
\verse And the priest shall value it, whether it be good or bad: as thou valuest it, who art the priest, so shall it be.
\verse But if he will at all redeem it, then he shall add a fifth part thereof unto thy estimation.
\verse And when a man shall sanctify his house to be holy unto the \LORD, then the priest shall estimate it, whether it be good or bad: as the priest shall estimate it, so shall it stand.
\verse And if he that sanctified it will redeem his house, then he shall add the fifth part of the money of thy estimation unto it, and it shall be his.
\verse And if a man shall sanctify unto the \LORD some part of a field of his possession, then thy estimation shall be according to the seed thereof: an homer of barley seed shall be valued at fifty shekels of silver.
\verse If he sanctify his field from the year of jubile, according to thy estimation it shall stand.
\verse But if he sanctify his field after the jubile, then the priest shall reckon unto him the money according to the years that remain, even unto the year of the jubile, and it shall be abated from thy estimation.
\verse And if he that sanctified the field will in any wise redeem it, then he shall add the fifth part of the money of thy estimation unto it, and it shall be assured to him.
\verse And if he will not redeem the field, or if he have sold the field to another man, it shall not be redeemed any more.
\verse But the field, when it goeth out in the jubile, shall be holy unto the \LORD, as a field devoted; the possession thereof shall be the priest's.
\verse And if a man sanctify unto the \LORD a field which he hath bought, which is not of the fields of his possession;
\verse Then the priest shall reckon unto him the worth of thy estimation, even unto the year of the jubile: and he shall give thine estimation in that day, as a holy thing unto the \LORD.
\verse In the year of the jubile the field shall return unto him of whom it was bought, even to him to whom the possession of the land did belong.
\verse And all thy estimations shall be according to the shekel of the sanctuary: twenty gerahs shall be the shekel.
\verse Only the firstling of the beasts, which should be the \LORDs firstling, no man shall sanctify it; whether it be ox, or sheep: it is the \LORDs.
\verse And if it be of an unclean beast, then he shall redeem it according to thine estimation, and shall add a fifth part of it thereto: or if it be not redeemed, then it shall be sold according to thy estimation.
\verse Notwithstanding no devoted thing, that a man shall devote unto the \LORD of all that he hath, both of man and beast, and of the field of his possession, shall be sold or redeemed: every devoted thing is most holy unto the \LORD.
\verse None devoted, which shall be devoted of men, shall be redeemed; but shall surely be put to death.
\verse And all the tithe of the land, whether of the seed of the land, or of the fruit of the tree, is the \LORDs: it is holy unto the \LORD.
\verse And if a man will at all redeem ought of his tithes, he shall add thereto the fifth part thereof.
\verse And concerning the tithe of the herd, or of the flock, even of whatsoever passeth under the rod, the tenth shall be holy unto the \LORD.
\verse He shall not search whether it be good or bad, neither shall he change it: and if he change it at all, then both it and the change thereof shall be holy; it shall not be redeemed.
\verse These are the commandments, which the \LORD commanded Moses for the children of Israel in mount Sinai.
\end{biblechapter}
\flushcolsend
\biblebook{Numbers}

\begin{biblechapter} % Numbers 1
\verseWithHeading{The census} And the \LORD spake unto Moses in the wilderness of Sinai, in the tabernacle of the congregation, on the first day of the second month, in the second year after they were come out of the land of Egypt, saying,
\verse Take ye the sum of all the congregation of the children of Israel, after their families, by the house of their fathers, with the number of their names, every male by their polls;
\verse From twenty years old and upward, all that are able to go forth to war in Israel: thou and Aaron shall number them by their armies.
\verse And with you there shall be a man of every tribe; every one head of the house of his fathers.
\verse And these are the names of the men that shall stand with you: of the tribe of Reuben; Elizur the son of Shedeur.
\verse Of Simeon; Shelumiel the son of Zurishaddai.
\verse Of Judah; Nahshon the son of Amminadab.
\verse Of Issachar; Nethaneel the son of Zuar.
\verse Of Zebulun; Eliab the son of Helon.
\verse Of the children of Joseph: of Ephraim; Elishama the son of Ammihud: of Manasseh; Gamaliel the son of Pedahzur.
\verse Of Benjamin; Abidan the son of Gideoni.
\verse Of Dan; Ahiezer the son of Ammishaddai.
\verse Of Asher; Pagiel the son of Ocran.
\verse Of Gad; Eliasaph the son of Deuel.
\verse Of Naphtali; Ahira the son of Enan.
\verse These were the renowned of the congregation, princes of the tribes of their fathers, heads of thousands in Israel.
\verse And Moses and Aaron took these men which are expressed by their names:
\verse And they assembled all the congregation together on the first day of the second month, and they declared their pedigrees after their families, by the house of their fathers, according to the number of the names, from twenty years old and upward, by their polls.
\verse As the \LORD commanded Moses, so he numbered them in the wilderness of Sinai.
\verse And the children of Reuben, Israel's eldest son, by their generations, after their families, by the house of their fathers, according to the number of the names, by their polls, every male from twenty years old and upward, all that were able to go forth to war;
\verse Those that were numbered of them, even of the tribe of Reuben, were forty and six thousand and five hundred.
\verse Of the children of Simeon, by their generations, after their families, by the house of their fathers, those that were numbered of them, according to the number of the names, by their polls, every male from twenty years old and upward, all that were able to go forth to war;
\verse Those that were numbered of them, even of the tribe of Simeon, were fifty and nine thousand and three hundred.
\verse Of the children of Gad, by their generations, after their families, by the house of their fathers, according to the number of the names, from twenty years old and upward, all that were able to go forth to war;
\verse Those that were numbered of them, even of the tribe of Gad, were forty and five thousand six hundred and fifty.
\verse Of the children of Judah, by their generations, after their families, by the house of their fathers, according to the number of the names, from twenty years old and upward, all that were able to go forth to war;
\verse Those that were numbered of them, even of the tribe of Judah, were threescore and fourteen thousand and six hundred.
\verse Of the children of Issachar, by their generations, after their families, by the house of their fathers, according to the number of the names, from twenty years old and upward, all that were able to go forth to war;
\verse Those that were numbered of them, even of the tribe of Issachar, were fifty and four thousand and four hundred.
\verse Of the children of Zebulun, by their generations, after their families, by the house of their fathers, according to the number of the names, from twenty years old and upward, all that were able to go forth to war;
\verse Those that were numbered of them, even of the tribe of Zebulun, were fifty and seven thousand and four hundred.
\verse Of the children of Joseph, namely, of the children of Ephraim, by their generations, after their families, by the house of their fathers, according to the number of the names, from twenty years old and upward, all that were able to go forth to war;
\verse Those that were numbered of them, even of the tribe of Ephraim, were forty thousand and five hundred.
\verse Of the children of Manasseh, by their generations, after their families, by the house of their fathers, according to the number of the names, from twenty years old and upward, all that were able to go forth to war;
\verse Those that were numbered of them, even of the tribe of Manasseh, were thirty and two thousand and two hundred.
\verse Of the children of Benjamin, by their generations, after their families, by the house of their fathers, according to the number of the names, from twenty years old and upward, all that were able to go forth to war;
\verse Those that were numbered of them, even of the tribe of Benjamin, were thirty and five thousand and four hundred.
\verse Of the children of Dan, by their generations, after their families, by the house of their fathers, according to the number of the names, from twenty years old and upward, all that were able to go forth to war;
\verse Those that were numbered of them, even of the tribe of Dan, were threescore and two thousand and seven hundred.
\verse Of the children of Asher, by their generations, after their families, by the house of their fathers, according to the number of the names, from twenty years old and upward, all that were able to go forth to war;
\verse Those that were numbered of them, even of the tribe of Asher, were forty and one thousand and five hundred.
\verse Of the children of Naphtali, throughout their generations, after their families, by the house of their fathers, according to the number of the names, from twenty years old and upward, all that were able to go forth to war;
\verse Those that were numbered of them, even of the tribe of Naphtali, were fifty and three thousand and four hundred.
\verse These are those that were numbered, which Moses and Aaron numbered, and the princes of Israel, being twelve men: each one was for the house of his fathers.
\verse So were all those that were numbered of the children of Israel, by the house of their fathers, from twenty years old and upward, all that were able to go forth to war in Israel;
\verse Even all they that were numbered were six hundred thousand and three thousand and five hundred and fifty.
\verse But the Levites after the tribe of their fathers were not numbered among them.
\verse For the \LORD had spoken unto Moses, saying,
\verse Only thou shalt not number the tribe of Levi, neither take the sum of them among the children of Israel:
\verse But thou shalt appoint the Levites over the tabernacle of testimony, and over all the vessels thereof, and over all things that belong to it: they shall bear the tabernacle, and all the vessels thereof; and they shall minister unto it, and shall encamp round about the tabernacle.
\verse And when the tabernacle setteth forward, the Levites shall take it down: and when the tabernacle is to be pitched, the Levites shall set it up: and the stranger that cometh nigh shall be put to death.
\verse And the children of Israel shall pitch their tents, every man by his own camp, and every man by his own standard, throughout their hosts.
\verse But the Levites shall pitch round about the tabernacle of testimony, that there be no wrath upon the congregation of the children of Israel: and the Levites shall keep the charge of the tabernacle of testimony.
\verse And the children of Israel did according to all that the \LORD commanded Moses, so did they.
\end{biblechapter}

\begin{biblechapter} % Numbers 2
\verseWithHeading{The arrangement of the tribal \newline camps} And the \LORD spake unto Moses and unto Aaron, saying,
\verse Every man of the children of Israel shall pitch by his own standard, with the ensign of their father's house: far off about the tabernacle of the congregation shall they pitch.
\verse And on the east side toward the rising of the sun shall they of the standard of the camp of Judah pitch throughout their armies: and Nahshon the son of Amminadab shall be captain of the children of Judah.
\verse And his host, and those that were numbered of them, were threescore and fourteen thousand and six hundred.
\verse And those that do pitch next unto him shall be the tribe of Issachar: and Nethaneel the son of Zuar shall be captain of the children of Issachar.
\verse And his host, and those that were numbered thereof, were fifty and four thousand and four hundred.
\verse Then the tribe of Zebulun: and Eliab the son of Helon shall be captain of the children of Zebulun.
\verse And his host, and those that were numbered thereof, were fifty and seven thousand and four hundred.
\verse All that were numbered in the camp of Judah were an hundred thousand and fourscore thousand and six thousand and four hundred, throughout their armies. These shall first set forth.
\verse On the south side shall be the standard of the camp of Reuben according to their armies: and the captain of the children of Reuben shall be Elizur the son of Shedeur.
\verse And his host, and those that were numbered thereof, were forty and six thousand and five hundred.
\verse And those which pitch by him shall be the tribe of Simeon: and the captain of the children of Simeon shall be Shelumiel the son of Zurishaddai.
\verse And his host, and those that were numbered of them, were fifty and nine thousand and three hundred.
\verse Then the tribe of Gad: and the captain of the sons of Gad shall be Eliasaph the son of Reuel.
\verse And his host, and those that were numbered of them, were forty and five thousand and six hundred and fifty.
\verse All that were numbered in the camp of Reuben were an hundred thousand and fifty and one thousand and four hundred and fifty, throughout their armies. And they shall set forth in the second rank.
\verse Then the tabernacle of the congregation shall set forward with the camp of the Levites in the midst of the camp: as they encamp, so shall they set forward, every man in his place by their standards.
\verse On the west side shall be the standard of the camp of Ephraim according to their armies: and the captain of the sons of Ephraim shall be Elishama the son of Ammihud.
\verse And his host, and those that were numbered of them, were forty thousand and five hundred.
\verse And by him shall be the tribe of Manasseh: and the captain of the children of Manasseh shall be Gamaliel the son of Pedahzur.
\verse And his host, and those that were numbered of them, were thirty and two thousand and two hundred.
\verse Then the tribe of Benjamin: and the captain of the sons of Benjamin shall be Abidan the son of Gideoni.
\verse And his host, and those that were numbered of them, were thirty and five thousand and four hundred.
\verse All that were numbered of the camp of Ephraim were an hundred thousand and eight thousand and an hundred, throughout their armies. And they shall go forward in the third rank.
\verse The standard of the camp of Dan shall be on the north side by their armies: and the captain of the children of Dan shall be Ahiezer the son of Ammishaddai.
\verse And his host, and those that were numbered of them, were threescore and two thousand and seven hundred.
\verse And those that encamp by him shall be the tribe of Asher: and the captain of the children of Asher shall be Pagiel the son of Ocran.
\verse And his host, and those that were numbered of them, were forty and one thousand and five hundred.
\verse Then the tribe of Naphtali: and the captain of the children of Naphtali shall be Ahira the son of Enan.
\verse And his host, and those that were numbered of them, were fifty and three thousand and four hundred.
\verse All they that were numbered in the camp of Dan were an hundred thousand and fifty and seven thousand and six hundred. They shall go hindmost with their standards.
\verse These are those which were numbered of the children of Israel by the house of their fathers: all those that were numbered of the camps throughout their hosts were six hundred thousand and three thousand and five hundred and fifty.
\verse But the Levites were not numbered among the children of Israel; as the \LORD commanded Moses.
\verse And the children of Israel did according to all that the \LORD commanded Moses: so they pitched by their standards, and so they set forward, every one after their families, according to the house of their fathers.
\end{biblechapter}

\begin{biblechapter} % Numbers 3
\verseWithHeading{The Levites} These also are the generations of Aaron and Moses in the day that the \LORD spake with Moses in mount Sinai.
\verse And these are the names of the sons of Aaron; Nadab the firstborn, and Abihu, Eleazar, and Ithamar.
\verse These are the names of the sons of Aaron, the priests which were anointed, whom he consecrated to minister in the priest's office.
\verse And Nadab and Abihu died before the \LORD, when they offered strange fire before the \LORD, in the wilderness of Sinai, and they had no children: and Eleazar and Ithamar ministered in the priest's office in the sight of Aaron their father.
\verse And the \LORD spake unto Moses, saying,
\verse Bring the tribe of Levi near, and present them before Aaron the priest, that they may minister unto him.
\verse And they shall keep his charge, and the charge of the whole congregation before the tabernacle of the congregation, to do the service of the tabernacle.
\verse And they shall keep all the instruments of the tabernacle of the congregation, and the charge of the children of Israel, to do the service of the tabernacle.
\verse And thou shalt give the Levites unto Aaron and to his sons: they are wholly given unto him out of the children of Israel.
\verse And thou shalt appoint Aaron and his sons, and they shall wait on their priest's office: and the stranger that cometh nigh shall be put to death.
\verse And the \LORD spake unto Moses, saying,
\verse And I, behold, I have taken the Levites from among the children of Israel instead of all the firstborn that openeth the matrix among the children of Israel: therefore the Levites shall be mine;
\verse Because all the firstborn are mine; for on the day that I smote all the firstborn in the land of Egypt I hallowed unto me all the firstborn in Israel, both man and beast: mine shall they be: I am the \LORD.
\verse And the \LORD spake unto Moses in the wilderness of Sinai, saying,
\verse Number the children of Levi after the house of their fathers, by their families: every male from a month old and upward shalt thou number them.
\verse And Moses numbered them according to the word of the \LORD, as he was commanded.
\verse And these were the sons of Levi by their names; Gershon, and Kohath, and Merari.
\verse And these are the names of the sons of Gershon by their families; Libni, and Shimei.
\verse And the sons of Kohath by their families; Amram, and Izehar, Hebron, and Uzziel.
\verse And the sons of Merari by their families; Mahli, and Mushi. These are the families of the Levites according to the house of their fathers.
\verse Of Gershon was the family of the Libnites, and the family of the Shimites: these are the families of the Gershonites.
\verse Those that were numbered of them, according to the number of all the males, from a month old and upward, even those that were numbered of them were seven thousand and five hundred.
\verse The families of the Gershonites shall pitch behind the tabernacle westward.
\verse And the chief of the house of the father of the Gershonites shall be Eliasaph the son of Lael.
\verse And the charge of the sons of Gershon in the tabernacle of the congregation shall be the tabernacle, and the tent, the covering thereof, and the hanging for the door of the tabernacle of the congregation,
\verse And the hangings of the court, and the curtain for the door of the court, which is by the tabernacle, and by the altar round about, and the cords of it for all the service thereof.
\verse And of Kohath was the family of the Amramites, and the family of the Izeharites, and the family of the Hebronites, and the family of the Uzzielites: these are the families of the Kohathites.
\verse In the number of all the males, from a month old and upward, were eight thousand and six hundred, keeping the charge of the sanctuary.
\verse The families of the sons of Kohath shall pitch on the side of the tabernacle southward.
\verse And the chief of the house of the father of the families of the Kohathites shall be Elizaphan the son of Uzziel.
\verse And their charge shall be the ark, and the table, and the candlestick, and the altars, and the vessels of the sanctuary wherewith they minister, and the hanging, and all the service thereof.
\verse And Eleazar the son of Aaron the priest shall be chief over the chief of the Levites, and have the oversight of them that keep the charge of the sanctuary.
\verse Of Merari was the family of the Mahlites, and the family of the Mushites: these are the families of Merari.
\verse And those that were numbered of them, according to the number of all the males, from a month old and upward, were six thousand and two hundred.
\verse And the chief of the house of the father of the families of Merari was Zuriel the son of Abihail: these shall pitch on the side of the tabernacle northward.
\verse And under the custody and charge of the sons of Merari shall be the boards of the tabernacle, and the bars thereof, and the pillars thereof, and the sockets thereof, and all the vessels thereof, and all that serveth thereto,
\verse And the pillars of the court round about, and their sockets, and their pins, and their cords.
\verse But those that encamp before the tabernacle toward the east, even before the tabernacle of the congregation eastward, shall be Moses, and Aaron and his sons, keeping the charge of the sanctuary for the charge of the children of Israel; and the stranger that cometh nigh shall be put to death.
\verse All that were numbered of the Levites, which Moses and Aaron numbered at the commandment of the \LORD, throughout their families, all the males from a month old and upward, were twenty and two thousand.
\verse And the \LORD said unto Moses, Number all the firstborn of the males of the children of Israel from a month old and upward, and take the number of their names.
\verse And thou shalt take the Levites for me (I am the \LORD) instead of all the firstborn among the children of Israel; and the cattle of the Levites instead of all the firstlings among the cattle of the children of Israel.
\verse And Moses numbered, as the \LORD commanded him, all the firstborn among the children of Israel.
\verse And all the firstborn males by the number of names, from a month old and upward, of those that were numbered of them, were twenty and two thousand two hundred and threescore and thirteen.
\verse And the \LORD spake unto Moses, saying,
\verse Take the Levites instead of all the firstborn among the children of Israel, and the cattle of the Levites instead of their cattle; and the Levites shall be mine: I am the \LORD.
\verse And for those that are to be redeemed of the two hundred and threescore and thirteen of the firstborn of the children of Israel, which are more than the Levites;
\verse Thou shalt even take five shekels apiece by the poll, after the shekel of the sanctuary shalt thou take them: (the shekel is twenty gerahs:)
\verse And thou shalt give the money, wherewith the odd number of them is to be redeemed, unto Aaron and to his sons.
\verse And Moses took the redemption money of them that were over and above them that were redeemed by the Levites:
\verse Of the firstborn of the children of Israel took he the money; a thousand three hundred and threescore and five shekels, after the shekel of the sanctuary:
\verse And Moses gave the money of them that were redeemed unto Aaron and to his sons, according to the word of the \LORD, as the \LORD commanded Moses.
\end{biblechapter}

\begin{biblechapter} % Numbers 4
\verseWithHeading{The Kohathites} And the \LORD spake unto Moses and unto Aaron, saying,
\verse Take the sum of the sons of Kohath from among the sons of Levi, after their families, by the house of their fathers,
\verse From thirty years old and upward even until fifty years old, all that enter into the host, to do the work in the tabernacle of the congregation.
\verse This shall be the service of the sons of Kohath in the tabernacle of the congregation, about the most holy things:
\verse And when the camp setteth forward, Aaron shall come, and his sons, and they shall take down the covering vail, and cover the ark of testimony with it:
\verse And shall put thereon the covering of badgers' skins, and shall spread over it a cloth wholly of blue, and shall put in the staves thereof.
\verse And upon the table of shewbread they shall spread a cloth of blue, and put thereon the dishes, and the spoons, and the bowls, and covers to cover withal: and the continual bread shall be thereon:
\verse And they shall spread upon them a cloth of scarlet, and cover the same with a covering of badgers' skins, and shall put in the staves thereof.
\verse And they shall take a cloth of blue, and cover the candlestick of the light, and his lamps, and his tongs, and his snuffdishes, and all the oil vessels thereof, wherewith they minister unto it:
\verse And they shall put it and all the vessels thereof within a covering of badgers' skins, and shall put it upon a bar.
\verse And upon the golden altar they shall spread a cloth of blue, and cover it with a covering of badgers' skins, and shall put to the staves thereof:
\verse And they shall take all the instruments of ministry, wherewith they minister in the sanctuary, and put them in a cloth of blue, and cover them with a covering of badgers' skins, and shall put them on a bar:
\verse And they shall take away the ashes from the altar, and spread a purple cloth thereon:
\verse And they shall put upon it all the vessels thereof, wherewith they minister about it, even the censers, the fleshhooks, and the shovels, and the basons, all the vessels of the altar; and they shall spread upon it a covering of badgers' skins, and put to the staves of it.
\verse And when Aaron and his sons have made an end of covering the sanctuary, and all the vessels of the sanctuary, as the camp is to set forward; after that, the sons of Kohath shall come to bear it: but they shall not touch any holy thing, lest they die. These things are the burden of the sons of Kohath in the tabernacle of the congregation.
\verse And to the office of Eleazar the son of Aaron the priest pertaineth the oil for the light, and the sweet incense, and the daily meat offering, and the anointing oil, and the oversight of all the tabernacle, and of all that therein is, in the sanctuary, and in the vessels thereof.
\verse And the \LORD spake unto Moses and unto Aaron, saying,
\verse Cut ye not off the tribe of the families of the Kohathites from among the Levites:
\verse But thus do unto them, that they may live, and not die, when they approach unto the most holy things: Aaron and his sons shall go in, and appoint them every one to his service and to his burden:
\verse But they shall not go in to see when the holy things are covered, lest they die.
\verseWithHeading{The Gershonites} And the \LORD spake unto Moses, saying,
\verse Take also the sum of the sons of Gershon, throughout the houses of their fathers, by their families;
\verse From thirty years old and upward until fifty years old shalt thou number them; all that enter in to perform the service, to do the work in the tabernacle of the congregation.
\verse This is the service of the families of the Gershonites, to serve, and for burdens:
\verse And they shall bear the curtains of the tabernacle, and the tabernacle of the congregation, his covering, and the covering of the badgers' skins that is above upon it, and the hanging for the door of the tabernacle of the congregation,
\verse And the hangings of the court, and the hanging for the door of the gate of the court, which is by the tabernacle and by the altar round about, and their cords, and all the instruments of their service, and all that is made for them: so shall they serve.
\verse At the appointment of Aaron and his sons shall be all the service of the sons of the Gershonites, in all their burdens, and in all their service: and ye shall appoint unto them in charge all their burdens.
\verse This is the service of the families of the sons of Gershon in the tabernacle of the congregation: and their charge shall be under the hand of Ithamar the son of Aaron the priest.
\verseWithHeading{The Merarites} As for the sons of Merari, thou shalt number them after their families, by the house of their fathers;
\verse From thirty years old and upward even unto fifty years old shalt thou number them, every one that entereth into the service, to do the work of the tabernacle of the congregation.
\verse And this is the charge of their burden, according to all their service in the tabernacle of the congregation; the boards of the tabernacle, and the bars thereof, and the pillars thereof, and sockets thereof,
\verse And the pillars of the court round about, and their sockets, and their pins, and their cords, with all their instruments, and with all their service: and by name ye shall reckon the instruments of the charge of their burden.
\verse This is the service of the families of the sons of Merari, according to all their service, in the tabernacle of the congregation, under the hand of Ithamar the son of Aaron the priest.
\verseWithHeading{The numbering of the Levite clans} And Moses and Aaron and the chief of the congregation numbered the sons of the Kohathites after their families, and after the house of their fathers,
\verse From thirty years old and upward even unto fifty years old, every one that entereth into the service, for the work in the tabernacle of the congregation:
\verse And those that were numbered of them by their families were two thousand seven hundred and fifty.
\verse These were they that were numbered of the families of the Kohathites, all that might do service in the tabernacle of the congregation, which Moses and Aaron did number according to the commandment of the \LORD by the hand of Moses.
\verse And those that were numbered of the sons of Gershon, throughout their families, and by the house of their fathers,
\verse From thirty years old and upward even unto fifty years old, every one that entereth into the service, for the work in the tabernacle of the congregation,
\verse Even those that were numbered of them, throughout their families, by the house of their fathers, were two thousand and six hundred and thirty.
\verse These are they that were numbered of the families of the sons of Gershon, of all that might do service in the tabernacle of the congregation, whom Moses and Aaron did number according to the commandment of the \LORD.
\verse And those that were numbered of the families of the sons of Merari, throughout their families, by the house of their fathers,
\verse From thirty years old and upward even unto fifty years old, every one that entereth into the service, for the work in the tabernacle of the congregation,
\verse Even those that were numbered of them after their families, were three thousand and two hundred.
\verse These be those that were numbered of the families of the sons of Merari, whom Moses and Aaron numbered according to the word of the \LORD by the hand of Moses.
\verse All those that were numbered of the Levites, whom Moses and Aaron and the chief of Israel numbered, after their families, and after the house of their fathers,
\verse From thirty years old and upward even unto fifty years old, every one that came to do the service of the ministry, and the service of the burden in the tabernacle of the congregation,
\verse Even those that were numbered of them, were eight thousand and five hundred and fourscore.
\verse According to the commandment of the \LORD they were numbered by the hand of Moses, every one according to his service, and according to his burden: thus were they numbered of him, as the \LORD commanded Moses.
\end{biblechapter}

\begin{biblechapter} % Numbers 5
\verseWithHeading{The purity of the camp} And the \LORD spake unto Moses, saying,
\verse Command the children of Israel, that they put out of the camp every leper, and every one that hath an issue, and whosoever is defiled by the dead:
\verse Both male and female shall ye put out, without the camp shall ye put them; that they defile not their camps, in the midst whereof I dwell.
\verse And the children of Israel did so, and put them out without the camp: as the \LORD spake unto Moses, so did the children of Israel.
\verseWithHeading{Restitution for wrongs} And the \LORD spake unto Moses, saying,
\verse Speak unto the children of Israel, When a man or woman shall commit any sin that men commit, to do a trespass against the \LORD, and that person be guilty;
\verse Then they shall confess their sin which they have done: and he shall recompense his trespass with the principal thereof, and add unto it the fifth part thereof, and give it unto him against whom he hath trespassed.
\verse But if the man have no kinsman to recompense the trespass unto, let the trespass be recompensed unto the \LORD, even to the priest; beside the ram of the atonement, whereby an atonement shall be made for him.
\verse And every offering of all the holy things of the children of Israel, which they bring unto the priest, shall be his.
\verse And every man's hallowed things shall be his: whatsoever any man giveth the priest, it shall be his.
\verseWithHeading{The test for an unfaithful wife} And the \LORD spake unto Moses, saying,
\verse Speak unto the children of Israel, and say unto them, If any man's wife go aside, and commit a trespass against him,
\verse And a man lie with her carnally, and it be hid from the eyes of her husband, and be kept close, and she be defiled, and there be no witness against her, neither she be taken with the manner;
\verse And the spirit of jealousy come upon him, and he be jealous of his wife, and she be defiled: or if the spirit of jealousy come upon him, and he be jealous of his wife, and she be not defiled:
\verse Then shall the man bring his wife unto the priest, and he shall bring her offering for her, the tenth part of an ephah of barley meal; he shall pour no oil upon it, nor put frankincense thereon; for it is an offering of jealousy, an offering of memorial, bringing iniquity to remembrance.
\verse And the priest shall bring her near, and set her before the \LORD:
\verse And the priest shall take holy water in an earthen vessel; and of the dust that is in the floor of the tabernacle the priest shall take, and put it into the water:
\verse And the priest shall set the woman before the \LORD, and uncover the woman's head, and put the offering of memorial in her hands, which is the jealousy offering: and the priest shall have in his hand the bitter water that causeth the curse:
\verse And the priest shall charge her by an oath, and say unto the woman, If no man have lain with thee, and if thou hast not gone aside to uncleanness with another instead of thy husband, be thou free from this bitter water that causeth the curse:
\verse But if thou hast gone aside to another instead of thy husband, and if thou be defiled, and some man have lain with thee beside thine husband:
\verse Then the priest shall charge the woman with an oath of cursing, and the priest shall say unto the woman, The \LORD make thee a curse and an oath among thy people, when the \LORD doth make thy thigh to rot, and thy belly to swell;
\verse And this water that causeth the curse shall go into thy bowels, to make thy belly to swell, and thy thigh to rot: And the woman shall say, Amen, amen.
\verse And the priest shall write these curses in a book, and he shall blot them out with the bitter water:
\verse And he shall cause the woman to drink the bitter water that causeth the curse: and the water that causeth the curse shall enter into her, and become bitter.
\verse Then the priest shall take the jealousy offering out of the woman's hand, and shall wave the offering before the \LORD, and offer it upon the altar:
\verse And the priest shall take an handful of the offering, even the memorial thereof, and burn it upon the altar, and afterward shall cause the woman to drink the water.
\verse And when he hath made her to drink the water, then it shall come to pass, that, if she be defiled, and have done trespass against her husband, that the water that causeth the curse shall enter into her, and become bitter, and her belly shall swell, and her thigh shall rot: and the woman shall be a curse among her people.
\verse And if the woman be not defiled, but be clean; then she shall be free, and shall conceive seed.
\verse This is the law of jealousies, when a wife goeth aside to another instead of her husband, and is defiled;
\verse Or when the spirit of jealousy cometh upon him, and he be jealous over his wife, and shall set the woman before the \LORD, and the priest shall execute upon her all this law.
\verse Then shall the man be guiltless from iniquity, and this woman shall bear her iniquity.
\end{biblechapter}

\begin{biblechapter} % Numbers 6
\verseWithHeading{The Nazirite} And the \LORD spake unto Moses, saying,
\verse Speak unto the children of Israel, and say unto them, When either man or woman shall separate themselves to vow a vow of a Nazarite, to separate themselves unto the \LORD:
\verse He shall separate himself from wine and strong drink, and shall drink no vinegar of wine, or vinegar of strong drink, neither shall he drink any liquor of grapes, nor eat moist grapes, or dried.
\verse All the days of his separation shall he eat nothing that is made of the vine tree, from the kernels even to the husk.
\verse All the days of the vow of his separation there shall no razor come upon his head: until the days be fulfilled, in the which he separateth himself unto the \LORD, he shall be holy, and shall let the locks of the hair of his head grow.
\verse All the days that he separateth himself unto the \LORD he shall come at no dead body.
\verse He shall not make himself unclean for his father, or for his mother, for his brother, or for his sister, when they die: because the consecration of his God is upon his head.
\verse All the days of his separation he is holy unto the \LORD.
\verse And if any man die very suddenly by him, and he hath defiled the head of his consecration; then he shall shave his head in the day of his cleansing, on the seventh day shall he shave it.
\verse And on the eighth day he shall bring two turtles, or two young pigeons, to the priest, to the door of the tabernacle of the congregation:
\verse And the priest shall offer the one for a sin offering, and the other for a burnt offering, and make an atonement for him, for that he sinned by the dead, and shall hallow his head that same day.
\verse And he shall consecrate unto the \LORD the days of his separation, and shall bring a lamb of the first year for a trespass offering: but the days that were before shall be lost, because his separation was defiled.
\verse And this is the law of the Nazarite, when the days of his separation are fulfilled: he shall be brought unto the door of the tabernacle of the congregation:
\verse And he shall offer his offering unto the \LORD, one he lamb of the first year without blemish for a burnt offering, and one ewe lamb of the first year without blemish for a sin offering, and one ram without blemish for peace offerings,
\verse And a basket of unleavened bread, cakes of fine flour mingled with oil, and wafers of unleavened bread anointed with oil, and their meat offering, and their drink offerings.
\verse And the priest shall bring them before the \LORD, and shall offer his sin offering, and his burnt offering:
\verse And he shall offer the ram for a sacrifice of peace offerings unto the \LORD, with the basket of unleavened bread: the priest shall offer also his meat offering, and his drink offering.
\verse And the Nazarite shall shave the head of his separation at the door of the tabernacle of the congregation, and shall take the hair of the head of his separation, and put it in the fire which is under the sacrifice of the peace offerings.
\verse And the priest shall take the sodden shoulder of the ram, and one unleavened cake out of the basket, and one unleavened wafer, and shall put them upon the hands of the Nazarite, after the hair of his separation is shaven:
\verse And the priest shall wave them for a wave offering before the \LORD: this is holy for the priest, with the wave breast and heave shoulder: and after that the Nazarite may drink wine.
\verse This is the law of the Nazarite who hath vowed, and of his offering unto the \LORD for his separation, beside that that his hand shall get: according to the vow which he vowed, so he must do after the law of his separation.
\verseWithHeading{The priestly blessing} And the \LORD spake unto Moses, saying,
\verse Speak unto Aaron and unto his sons, saying, On this wise ye shall bless the children of Israel, saying unto them,
\verse The \LORD bless thee, and keep thee:
\verse The \LORD make his face shine upon thee, and be gracious unto thee:
\verse The \LORD lift up his countenance upon thee, and give thee peace.
\verse And they shall put my name upon the children of Israel; and I will bless them.
\end{biblechapter}

\begin{biblechapter} % Numbers 7
\verseWithHeading{Offerings at the dedication of the \newline tabernacle} And it came to pass on the day that Moses had fully set up the tabernacle, and had anointed it, and sanctified it, and all the instruments thereof, both the altar and all the vessels thereof, and had anointed them, and sanctified them;
\verse That the princes of Israel, heads of the house of their fathers, who were the princes of the tribes, and were over them that were numbered, offered:
\verse And they brought their offering before the \LORD, six covered wagons, and twelve oxen; a wagon for two of the princes, and for each one an ox: and they brought them before the tabernacle.
\verse And the \LORD spake unto Moses, saying,
\verse Take it of them, that they may be to do the service of the tabernacle of the congregation; and thou shalt give them unto the Levites, to every man according to his service.
\verse And Moses took the wagons and the oxen, and gave them unto the Levites.
\verse Two wagons and four oxen he gave unto the sons of Gershon, according to their service:
\verse And four wagons and eight oxen he gave unto the sons of Merari, according unto their service, under the hand of Ithamar the son of Aaron the priest.
\verse But unto the sons of Kohath he gave none: because the service of the sanctuary belonging unto them was that they should bear upon their shoulders.
\verse And the princes offered for dedicating of the altar in the day that it was anointed, even the princes offered their offering before the altar.
\verse And the \LORD said unto Moses, They shall offer their offering, each prince on his day, for the dedicating of the altar.
\verse And he that offered his offering the first day was Nahshon the son of Amminadab, of the tribe of Judah:
\verse And his offering was one silver charger, the weight thereof was an hundred and thirty shekels, one silver bowl of seventy shekels, after the shekel of the sanctuary; both of them were full of fine flour mingled with oil for a meat offering:
\verse One spoon of ten shekels of gold, full of incense:
\verse One young bullock, one ram, one lamb of the first year, for a burnt offering:
\verse One kid of the goats for a sin offering:
\verse And for a sacrifice of peace offerings, two oxen, five rams, five he goats, five lambs of the first year: this was the offering of Nahshon the son of Amminadab.
\verse On the second day Nethaneel the son of Zuar, prince of Issachar, did offer:
\verse He offered for his offering one silver charger, the weight whereof was an hundred and thirty shekels, one silver bowl of seventy shekels, after the shekel of the sanctuary; both of them full of fine flour mingled with oil for a meat offering:
\verse One spoon of gold of ten shekels, full of incense:
\verse One young bullock, one ram, one lamb of the first year, for a burnt offering:
\verse One kid of the goats for a sin offering:
\verse And for a sacrifice of peace offerings, two oxen, five rams, five he goats, five lambs of the first year: this was the offering of Nethaneel the son of Zuar.
\verse On the third day Eliab the son of Helon, prince of the children of Zebulun, did offer:
\verse His offering was one silver charger, the weight whereof was an hundred and thirty shekels, one silver bowl of seventy shekels, after the shekel of the sanctuary; both of them full of fine flour mingled with oil for a meat offering:
\verse One golden spoon of ten shekels, full of incense:
\verse One young bullock, one ram, one lamb of the first year, for a burnt offering:
\verse One kid of the goats for a sin offering:
\verse And for a sacrifice of peace offerings, two oxen, five rams, five he goats, five lambs of the first year: this was the offering of Eliab the son of Helon.
\verse On the fourth day Elizur the son of Shedeur, prince of the children of Reuben, did offer:
\verse His offering was one silver charger of the weight of an hundred and thirty shekels, one silver bowl of seventy shekels, after the shekel of the sanctuary; both of them full of fine flour mingled with oil for a meat offering:
\verse One golden spoon of ten shekels, full of incense:
\verse One young bullock, one ram, one lamb of the first year, for a burnt offering:
\verse One kid of the goats for a sin offering:
\verse And for a sacrifice of peace offerings, two oxen, five rams, five he goats, five lambs of the first year: this was the offering of Elizur the son of Shedeur.
\verse On the fifth day Shelumiel the son of Zurishaddai, prince of the children of Simeon, did offer:
\verse His offering was one silver charger, the weight whereof was an hundred and thirty shekels, one silver bowl of seventy shekels, after the shekel of the sanctuary; both of them full of fine flour mingled with oil for a meat offering:
\verse One golden spoon of ten shekels, full of incense:
\verse One young bullock, one ram, one lamb of the first year, for a burnt offering:
\verse One kid of the goats for a sin offering:
\verse And for a sacrifice of peace offerings, two oxen, five rams, five he goats, five lambs of the first year: this was the offering of Shelumiel the son of Zurishaddai.
\verse On the sixth day Eliasaph the son of Deuel, prince of the children of Gad, offered:
\verse His offering was one silver charger of the weight of an hundred and thirty shekels, a silver bowl of seventy shekels, after the shekel of the sanctuary; both of them full of fine flour mingled with oil for a meat offering:
\verse One golden spoon of ten shekels, full of incense:
\verse One young bullock, one ram, one lamb of the first year, for a burnt offering:
\verse One kid of the goats for a sin offering:
\verse And for a sacrifice of peace offerings, two oxen, five rams, five he goats, five lambs of the first year: this was the offering of Eliasaph the son of Deuel.
\verse On the seventh day Elishama the son of Ammihud, prince of the children of Ephraim, offered:
\verse His offering was one silver charger, the weight whereof was an hundred and thirty shekels, one silver bowl of seventy shekels, after the shekel of the sanctuary; both of them full of fine flour mingled with oil for a meat offering:
\verse One golden spoon of ten shekels, full of incense:
\verse One young bullock, one ram, one lamb of the first year, for a burnt offering:
\verse One kid of the goats for a sin offering:
\verse And for a sacrifice of peace offerings, two oxen, five rams, five he goats, five lambs of the first year: this was the offering of Elishama the son of Ammihud.
\verse On the eighth day offered Gamaliel the son of Pedahzur, prince of the children of Manasseh:
\verse His offering was one silver charger of the weight of an hundred and thirty shekels, one silver bowl of seventy shekels, after the shekel of the sanctuary; both of them full of fine flour mingled with oil for a meat offering:
\verse One golden spoon of ten shekels, full of incense:
\verse One young bullock, one ram, one lamb of the first year, for a burnt offering:
\verse One kid of the goats for a sin offering:
\verse And for a sacrifice of peace offerings, two oxen, five rams, five he goats, five lambs of the first year: this was the offering of Gamaliel the son of Pedahzur.
\verse On the ninth day Abidan the son of Gideoni, prince of the children of Benjamin, offered:
\verse His offering was one silver charger, the weight whereof was an hundred and thirty shekels, one silver bowl of seventy shekels, after the shekel of the sanctuary; both of them full of fine flour mingled with oil for a meat offering:
\verse One golden spoon of ten shekels, full of incense:
\verse One young bullock, one ram, one lamb of the first year, for a burnt offering:
\verse One kid of the goats for a sin offering:
\verse And for a sacrifice of peace offerings, two oxen, five rams, five he goats, five lambs of the first year: this was the offering of Abidan the son of Gideoni.
\verse On the tenth day Ahiezer the son of Ammishaddai, prince of the children of Dan, offered:
\verse His offering was one silver charger, the weight whereof was an hundred and thirty shekels, one silver bowl of seventy shekels, after the shekel of the sanctuary; both of them full of fine flour mingled with oil for a meat offering:
\verse One golden spoon of ten shekels, full of incense:
\verse One young bullock, one ram, one lamb of the first year, for a burnt offering:
\verse One kid of the goats for a sin offering:
\verse And for a sacrifice of peace offerings, two oxen, five rams, five he goats, five lambs of the first year: this was the offering of Ahiezer the son of Ammishaddai.
\verse On the eleventh day Pagiel the son of Ocran, prince of the children of Asher, offered:
\verse His offering was one silver charger, the weight whereof was an hundred and thirty shekels, one silver bowl of seventy shekels, after the shekel of the sanctuary; both of them full of fine flour mingled with oil for a meat offering:
\verse One golden spoon of ten shekels, full of incense:
\verse One young bullock, one ram, one lamb of the first year, for a burnt offering:
\verse One kid of the goats for a sin offering:
\verse And for a sacrifice of peace offerings, two oxen, five rams, five he goats, five lambs of the first year: this was the offering of Pagiel the son of Ocran.
\verse On the twelfth day Ahira the son of Enan, prince of the children of Naphtali, offered:
\verse His offering was one silver charger, the weight whereof was an hundred and thirty shekels, one silver bowl of seventy shekels, after the shekel of the sanctuary; both of them full of fine flour mingled with oil for a meat offering:
\verse One golden spoon of ten shekels, full of incense:
\verse One young bullock, one ram, one lamb of the first year, for a burnt offering:
\verse One kid of the goats for a sin offering:
\verse And for a sacrifice of peace offerings, two oxen, five rams, five he goats, five lambs of the first year: this was the offering of Ahira the son of Enan.
\verse This was the dedication of the altar, in the day when it was anointed, by the princes of Israel: twelve chargers of silver, twelve silver bowls, twelve spoons of gold:
\verse Each charger of silver weighing an hundred and thirty shekels, each bowl seventy: all the silver vessels weighed two thousand and four hundred shekels, after the shekel of the sanctuary:
\verse The golden spoons were twelve, full of incense, weighing ten shekels apiece, after the shekel of the sanctuary: all the gold of the spoons was an hundred and twenty shekels.
\verse All the oxen for the burnt offering were twelve bullocks, the rams twelve, the lambs of the first year twelve, with their meat offering: and the kids of the goats for sin offering twelve.
\verse And all the oxen for the sacrifice of the peace offerings were twenty and four bullocks, the rams sixty, the he goats sixty, the lambs of the first year sixty. This was the dedication of the altar, after that it was anointed.
\verse And when Moses was gone into the tabernacle of the congregation to speak with him, then he heard the voice of one speaking unto him from off the mercy seat that was upon the ark of testimony, from between the two cherubims: and he spake unto him.
\end{biblechapter}

\begin{biblechapter} % Numbers 8
\verseWithHeading{Lighting the lamps} And the \LORD spake unto Moses, saying,
\verse Speak unto Aaron, and say unto him, When thou lightest the lamps, the seven lamps shall give light over against the candlestick.
\verse And Aaron did so; he lighted the lamps thereof over against the candlestick, as the \LORD commanded Moses.
\verse And this work of the candlestick was of beaten gold, unto the shaft thereof, unto the flowers thereof, was beaten work: according unto the pattern which the \LORD had shewed Moses, so he made the candlestick.
\vfill\columnbreak % layout hack
\verseWithHeading{The setting apart of the Levites} And the \LORD spake unto Moses, saying,
\verse Take the Levites from among the children of Israel, and cleanse them.
\verse And thus shalt thou do unto them, to cleanse them: Sprinkle water of purifying upon them, and let them shave all their flesh, and let them wash their clothes, and so make themselves clean.
\verse Then let them take a young bullock with his meat offering, even fine flour mingled with oil, and another young bullock shalt thou take for a sin offering.
\verse And thou shalt bring the Levites before the tabernacle of the congregation: and thou shalt gather the whole assembly of the children of Israel together:
\verse And thou shalt bring the Levites before the \LORD: and the children of Israel shall put their hands upon the Levites:
\verse And Aaron shall offer the Levites before the \LORD for an offering of the children of Israel, that they may execute the service of the \LORD.
\verse And the Levites shall lay their hands upon the heads of the bullocks: and thou shalt offer the one for a sin offering, and the other for a burnt offering, unto the \LORD, to make an atonement for the Levites.
\verse And thou shalt set the Levites before Aaron, and before his sons, and offer them for an offering unto the \LORD.
\verse Thus shalt thou separate the Levites from among the children of Israel: and the Levites shall be mine.
\verse And after that shall the Levites go in to do the service of the tabernacle of the congregation: and thou shalt cleanse them, and offer them for an offering.
\verse For they are wholly given unto me from among the children of Israel; instead of such as open every womb, even instead of the firstborn of all the children of Israel, have I taken them unto me.
\verse For all the firstborn of the children of Israel are mine, both man and beast: on the day that I smote every firstborn in the land of Egypt I sanctified them for myself.
\verse And I have taken the Levites for all the firstborn of the children of Israel.
\verse And I have given the Levites as a gift to Aaron and to his sons from among the children of Israel, to do the service of the children of Israel in the tabernacle of the congregation, and to make an atonement for the children of Israel: that there be no plague among the children of Israel, when the children of Israel come nigh unto the sanctuary.
\verse And Moses, and Aaron, and all the congregation of the children of Israel, did to the Levites according unto all that the \LORD commanded Moses concerning the Levites, so did the children of Israel unto them.
\verse And the Levites were purified, and they washed their clothes; and Aaron offered them as an offering before the \LORD; and Aaron made an atonement for them to cleanse them.
\verse And after that went the Levites in to do their service in the tabernacle of the congregation before Aaron, and before his sons: as the \LORD had commanded Moses concerning the Levites, so did they unto them.
\verse And the \LORD spake unto Moses, saying,
\verse This is it that belongeth unto the Levites: from twenty and five years old and upward they shall go in to wait upon the service of the tabernacle of the congregation:
\verse And from the age of fifty years they shall cease waiting upon the service thereof, and shall serve no more:
\verse But shall minister with their brethren in the tabernacle of the congregation, to keep the charge, and shall do no service. Thus shalt thou do unto the Levites touching their charge.
\end{biblechapter}

\begin{biblechapter} % Numbers 9
\verseWithHeading{The Passover} And the \LORD spake unto Moses in the wilderness of Sinai, in the first month of the second year after they were come out of the land of Egypt, saying,
\verse Let the children of Israel also keep the Passover at his appointed season.
\verse In the fourteenth day of this month, at even, ye shall keep it in his appointed season: according to all the rites of it, and according to all the ceremonies thereof, shall ye keep it.
\verse And Moses spake unto the children of Israel, that they should keep the Passover.
\verse And they kept the Passover on the fourteenth day of the first month at even in the wilderness of Sinai: according to all that the \LORD commanded Moses, so did the children of Israel.
\verse And there were certain men, who were defiled by the dead body of a man, that they could not keep the Passover on that day: and they came before Moses and before Aaron on that day:
\verse And those men said unto him, We are defiled by the dead body of a man: wherefore are we kept back, that we may not offer an offering of the \LORD in his appointed season among the children of Israel?
\verse And Moses said unto them, Stand still, and I will hear what the \LORD will command concerning you.
\verse And the \LORD spake unto Moses, saying,
\verse Speak unto the children of Israel, saying, If any man of you or of your posterity shall be unclean by reason of a dead body, or be in a journey afar off, yet he shall keep the Passover unto the \LORD.
\verse The fourteenth day of the second month at even they shall keep it, and eat it with unleavened bread and bitter herbs.
\verse They shall leave none of it unto the morning, nor break any bone of it: according to all the ordinances of the Passover they shall keep it.
\verse But the man that is clean, and is not in a journey, and forbeareth to keep the Passover, even the same soul shall be cut off from among his people: because he brought not the offering of the \LORD in his appointed season, that man shall bear his sin.
\verse And if a stranger shall sojourn among you, and will keep the Passover unto the \LORD; according to the ordinance of the Passover, and according to the manner thereof, so shall he do: ye shall have one ordinance, both for the stranger, and for him that was born in the land.
\verseWithHeading{The cloud above the tabernacle} And on the day that the tabernacle was reared up the cloud covered the tabernacle, namely, the tent of the testimony: and at even there was upon the tabernacle as it were the appearance of fire, until the morning.
\verse So it was alway: the cloud covered it by day, and the appearance of fire by night.
\verse And when the cloud was taken up from the tabernacle, then after that the children of Israel journeyed: and in the place where the cloud abode, there the children of Israel pitched their tents.
\verse At the commandment of the \LORD the children of Israel journeyed, and at the commandment of the \LORD they pitched: as long as the cloud abode upon the tabernacle they rested in their tents.
\verse And when the cloud tarried long upon the tabernacle many days, then the children of Israel kept the charge of the \LORD, and journeyed not.
\verse And so it was, when the cloud was a few days upon the tabernacle; according to the commandment of the \LORD they abode in their tents, and according to the commandment of the \LORD they journeyed.
\verse And so it was, when the cloud abode from even unto the morning, and that the cloud was taken up in the morning, then they journeyed: whether it was by day or by night that the cloud was taken up, they journeyed.
\verse Or whether it were two days, or a month, or a year, that the cloud tarried upon the tabernacle, remaining thereon, the children of Israel abode in their tents, and journeyed not: but when it was taken up, they journeyed.
\verse At the commandment of the \LORD they rested in the tents, and at the commandment of the \LORD they journeyed: they kept the charge of the \LORD, at the commandment of the \LORD by the hand of Moses.
\end{biblechapter}

\begin{biblechapter} % Numbers 10
\verseWithHeading{The silver trumpets} And the \LORD spake unto Moses, saying,
\verse Make thee two trumpets of silver; of a whole piece shalt thou make them: that thou mayest use them for the calling of the assembly, and for the journeying of the camps.
\verse And when they shall blow with them, all the assembly shall assemble themselves to thee at the door of the tabernacle of the congregation.
\verse And if they blow but with one trumpet, then the princes, which are heads of the thousands of Israel, shall gather themselves unto thee.
\verse When ye blow an alarm, then the camps that lie on the east parts shall go forward.
\verse When ye blow an alarm the second time, then the camps that lie on the south side shall take their journey: they shall blow an alarm for their journeys.
\verse But when the congregation is to be gathered together, ye shall blow, but ye shall not sound an alarm.
\verse And the sons of Aaron, the priests, shall blow with the trumpets; and they shall be to you for an ordinance for ever throughout your generations.
\verse And if ye go to war in your land against the enemy that oppresseth you, then ye shall blow an alarm with the trumpets; and ye shall be remembered before the \LORD your God, and ye shall be saved from your enemies.
\verse Also in the day of your gladness, and in your solemn days, and in the beginnings of your months, ye shall blow with the trumpets over your burnt offerings, and over the sacrifices of your peace offerings; that they may be to you for a memorial before your God: I am the \LORD your God.
\verseWithHeading{The Israelites leave Sinai} And it came to pass on the twentieth day of the second month, in the second year, that the cloud was taken up from off the tabernacle of the testimony.
\verse And the children of Israel took their journeys out of the wilderness of Sinai; and the cloud rested in the wilderness of Paran.
\verse And they first took their journey according to the commandment of the \LORD by the hand of Moses.
\verse In the first place went the standard of the camp of the children of Judah according to their armies: and over his host was Nahshon the son of Amminadab.
\verse And over the host of the tribe of the children of Issachar was Nethaneel the son of Zuar.
\verse And over the host of the tribe of the children of Zebulun was Eliab the son of Helon.
\verse And the tabernacle was taken down; and the sons of Gershon and the sons of Merari set forward, bearing the tabernacle.
\verse And the standard of the camp of Reuben set forward according to their armies: and over his host was Elizur the son of Shedeur.
\verse And over the host of the tribe of the children of Simeon was Shelumiel the son of Zurishaddai.
\verse And over the host of the tribe of the children of Gad was Eliasaph the son of Deuel.
\verse And the Kohathites set forward, bearing the sanctuary: and the other did set up the tabernacle against they came.
\verse And the standard of the camp of the children of Ephraim set forward according to their armies: and over his host was Elishama the son of Ammihud.
\verse And over the host of the tribe of the children of Manasseh was Gamaliel the son of Pedahzur.
\verse And over the host of the tribe of the children of Benjamin was Abidan the son of Gideoni.
\verse And the standard of the camp of the children of Dan set forward, which was the rereward of all the camps throughout their hosts: and over his host was Ahiezer the son of Ammishaddai.
\verse And over the host of the tribe of the children of Asher was Pagiel the son of Ocran.
\verse And over the host of the tribe of the children of Naphtali was Ahira the son of Enan.
\verse Thus were the journeyings of the children of Israel according to their armies, when they set forward.
\verse And Moses said unto Hobab, the son of Raguel the Midianite, Moses' father in law, We are journeying unto the place of which the \LORD said, I will give it you: come thou with us, and we will do thee good: for the \LORD hath spoken good concerning Israel.
\verse And he said unto him, I will not go; but I will depart to mine own land, and to my kindred.
\verse And he said, Leave us not, I pray thee; forasmuch as thou knowest how we are to encamp in the wilderness, and thou mayest be to us instead of eyes.
\verse And it shall be, if thou go with us, yea, it shall be, that what goodness the \LORD shall do unto us, the same will we do unto thee.
\verse And they departed from the mount of the \LORD three days' journey: and the ark of the covenant of the \LORD went before them in the three days' journey, to search out a resting place for them.
\verse And the cloud of the \LORD was upon them by day, when they went out of the camp.
\verse And it came to pass, when the ark set forward, that Moses said, Rise up, \LORD, and let thine enemies be scattered; and let them that hate thee flee before thee.
\verse And when it rested, he said, Return, O \LORD, unto the many thousands of Israel.
\end{biblechapter}

\begin{biblechapter} % Numbers 11
\verseWithHeading{Fire from the \LORD} And when the people complained, it displeased the \LORD: and the \LORD heard it; and his anger was kindled; and the fire of the \LORD burnt among them, and consumed them that were in the uttermost parts of the camp.
\verse And the people cried unto Moses; and when Moses prayed unto the \LORD, the fire was quenched.
\verse And he called the name of the place Taberah: because the fire of the \LORD burnt among them.
\verseWithHeading{Quail from the \LORD} And the mixt multitude that was among them fell a lusting: and the children of Israel also wept again, and said, Who shall give us flesh to eat?
\verse We remember the fish, which we did eat in Egypt freely; the cucumbers, and the melons, and the leeks, and the onions, and the garlick:
\verse But now our soul is dried away: there is nothing at all, beside this manna, before our eyes.
\verse And the manna was as coriander seed, and the colour thereof as the colour of bdellium.
\verse And the people went about, and gathered it, and ground it in mills, or beat it in a mortar, and baked it in pans, and made cakes of it: and the taste of it was as the taste of fresh oil.
\verse And when the dew fell upon the camp in the night, the manna fell upon it.
\verse Then Moses heard the people weep throughout their families, every man in the door of his tent: and the anger of the \LORD was kindled greatly; Moses also was displeased.
\verse And Moses said unto the \LORD, Wherefore hast thou afflicted thy servant? and wherefore have I not found favour in thy sight, that thou layest the burden of all this people upon me?
\verse Have I conceived all this people? have I begotten them, that thou shouldest say unto me, Carry them in thy bosom, as a nursing father beareth the sucking child, unto the land which thou swarest unto their fathers?
\verse Whence should I have flesh to give unto all this people? for they weep unto me, saying, Give us flesh, that we may eat.
\verse I am not able to bear all this people alone, because it is too heavy for me.
\verse And if thou deal thus with me, kill me, I pray thee, out of hand, if I have found favour in thy sight; and let me not see my wretchedness.
\verse And the \LORD said unto Moses, Gather unto me seventy men of the elders of Israel, whom thou knowest to be the elders of the people, and officers over them; and bring them unto the tabernacle of the congregation, that they may stand there with thee.
\verse And I will come down and talk with thee there: and I will take of the spirit which is upon thee, and will put it upon them; and they shall bear the burden of the people with thee, that thou bear it not thyself alone.
\verse And say thou unto the people, Sanctify yourselves against to morrow, and ye shall eat flesh: for ye have wept in the ears of the \LORD, saying, Who shall give us flesh to eat? for it was well with us in Egypt: therefore the \LORD will give you flesh, and ye shall eat.
\verse Ye shall not eat one day, nor two days, nor five days, neither ten days, nor twenty days;
\verse But even a whole month, until it come out at your nostrils, and it be loathsome unto you: because that ye have despised the \LORD which is among you, and have wept before him, saying, Why came we forth out of Egypt?
\verse And Moses said, The people, among whom I am, are six hundred thousand footmen; and thou hast said, I will give them flesh, that they may eat a whole month.
\verse Shall the flocks and the herds be slain for them, to suffice them? or shall all the fish of the sea be gathered together for them, to suffice them?
\verse And the \LORD said unto Moses, Is the \LORDs hand waxed short? thou shalt see now whether my word shall come to pass unto thee or not.
\verse And Moses went out, and told the people the words of the \LORD, and gathered the seventy men of the elders of the people, and set them round about the tabernacle.
\verse And the \LORD came down in a cloud, and spake unto him, and took of the spirit that was upon him, and gave it unto the seventy elders: and it came to pass, that, when the spirit rested upon them, they prophesied, and did not cease.
\verse But there remained two of the men in the camp, the name of the one was Eldad, and the name of the other Medad: and the spirit rested upon them; and they were of them that were written, but went not out unto the tabernacle: and they prophesied in the camp.
\verse And there ran a young man, and told Moses, and said, Eldad and Medad do prophesy in the camp.
\verse And Joshua the son of Nun, the servant of Moses, one of his young men, answered and said, My lord Moses, forbid them.
\verse And Moses said unto him, Enviest thou for my sake? would God that all the \LORDs people were prophets, and that the \LORD would put his spirit upon them!
\verse And Moses gat him into the camp, he and the elders of Israel.
\verse And there went forth a wind from the \LORD, and brought quails from the sea, and let them fall by the camp, as it were a day's journey on this side, and as it were a day's journey on the other side, round about the camp, and as it were two cubits high upon the face of the earth.
\verse And the people stood up all that day, and all that night, and all the next day, and they gathered the quails: he that gathered least gathered ten homers: and they spread them all abroad for themselves round about the camp.
\verse And while the flesh was yet between their teeth, ere it was chewed, the wrath of the \LORD was kindled against the people, and the \LORD smote the people with a very great plague.
\verse And he called the name of that place Kibrothhattaavah: because there they buried the people that lusted.
\verse And the people journeyed from Kibrothhattaavah unto Hazeroth; and abode at Hazeroth.
\end{biblechapter}

\begin{biblechapter} % Numbers 12
\verseWithHeading{Miriam and Aaron oppose \newline Moses} And Miriam and Aaron spake against Moses because of the Ethiopian woman whom he had married: for he had married an Ethiopian woman.
\verse And they said, Hath the \LORD indeed spoken only by Moses? hath he not spoken also by us? And the \LORD heard it.
\verse (Now the man Moses was very meek, above all the men which were upon the face of the earth.)
\verse And the \LORD spake suddenly unto Moses, and unto Aaron, and unto Miriam, Come out ye three unto the tabernacle of the congregation. And they three came out.
\verse And the \LORD came down in the pillar of the cloud, and stood in the door of the tabernacle, and called Aaron and Miriam: and they both came forth.
\verse And he said, Hear now my words: If there be a prophet among you, I the \LORD will make myself known unto him in a vision, and will speak unto him in a dream.
\verse My servant Moses is not so, who is faithful in all mine house.
\verse With him will I speak mouth to mouth, even apparently, and not in dark speeches; and the similitude of the \LORD shall he behold: wherefore then were ye not afraid to speak against my servant Moses?
\verse And the anger of the \LORD was kindled against them; and he departed.
\verse And the cloud departed from off the tabernacle; and, behold, Miriam became leprous, white as snow: and Aaron looked upon Miriam, and, behold, she was leprous.
\verse And Aaron said unto Moses, Alas, my lord, I beseech thee, lay not the sin upon us, wherein we have done foolishly, and wherein we have sinned.
\verse Let her not be as one dead, of whom the flesh is half consumed when he cometh out of his mother's womb.
\verse And Moses cried unto the \LORD, saying, Heal her now, O God, I beseech thee.
\verse And the \LORD said unto Moses, If her father had but spit in her face, should she not be ashamed seven days? let her be shut out from the camp seven days, and after that let her be received in again.
\verse And Miriam was shut out from the camp seven days: and the people journeyed not till Miriam was brought in again.
\verse And afterward the people removed from Hazeroth, and pitched in the wilderness of Paran.
\end{biblechapter}

\begin{biblechapter} % Numbers 13
\verseWithHeading{Exploring Canaan} And the \LORD spake unto Moses, saying,
\verse Send thou men, that they may search the land of Canaan, which I give unto the children of Israel: of every tribe of their fathers shall ye send a man, every one a ruler among them.
\verse And Moses by the commandment of the \LORD sent them from the wilderness of Paran: all those men were heads of the children of Israel.
\verse And these were their names: of the tribe of Reuben, Shammua the son of Zaccur.
\verse Of the tribe of Simeon, Shaphat the son of Hori.
\verse Of the tribe of Judah, Caleb the son of Jephunneh.
\verse Of the tribe of Issachar, Igal the son of Joseph.
\verse Of the tribe of Ephraim, Oshea the son of Nun.
\verse Of the tribe of Benjamin, Palti the son of Raphu.
\verse Of the tribe of Zebulun, Gaddiel the son of Sodi.
\verse Of the tribe of Joseph, namely, of the tribe of Manasseh, Gaddi the son of Susi.
\verse Of the tribe of Dan, Ammiel the son of Gemalli.
\verse Of the tribe of Asher, Sethur the son of Michael.
\verse Of the tribe of Naphtali, Nahbi the son of Vophsi.
\verse Of the tribe of Gad, Geuel the son of Machi.
\verse These are the names of the men which Moses sent to spy out the land. And Moses called Oshea the son of Nun Jehoshua.
\verse And Moses sent them to spy out the land of Canaan, and said unto them, Get you up this way southward, and go up into the mountain:
\verse And see the land, what it is; and the people that dwelleth therein, whether they be strong or weak, few or many;
\verse And what the land is that they dwell in, whether it be good or bad; and what cities they be that they dwell in, whether in tents, or in strong holds;
\verse And what the land is, whether it be fat or lean, whether there be wood therein, or not. And be ye of good courage, and bring of the fruit of the land. Now the time was the time of the firstripe grapes.
\verse So they went up, and searched the land from the wilderness of Zin unto Rehob, as men come to Hamath.
\verse And they ascended by the south, and came unto Hebron; where Ahiman, Sheshai, and Talmai, the children of Anak, were. (Now Hebron was built seven years before Zoan in Egypt.)
\verse And they came unto the brook of Eshcol, and cut down from thence a branch with one cluster of grapes, and they bare it between two upon a staff; and they brought of the pomegranates, and of the figs.
\verse The place was called the brook Eshcol, because of the cluster of grapes which the children of Israel cut down from thence.
\verse And they returned from searching of the land after forty days.
\verseWithHeading{Report on the exploration} And they went and came to Moses, and to Aaron, and to all the congregation of the children of Israel, unto the wilderness of Paran, to Kadesh; and brought back word unto them, and unto all the congregation, and shewed them the fruit of the land.
\verse And they told him, and said, We came unto the land whither thou sentest us, and surely it floweth with milk and honey; and this is the fruit of it.
\verse Nevertheless the people be strong that dwell in the land, and the cities are walled, and very great: and moreover we saw the children of Anak there.
\verse The Amalekites dwell in the land of the south: and the Hittites, and the Jebusites, and the Amorites, dwell in the mountains: and the Canaanites dwell by the sea, and by the coast of Jordan.
\verse And Caleb stilled the people before Moses, and said, Let us go up at once, and possess it; for we are well able to overcome it.
\verse But the men that went up with him said, We be not able to go up against the people; for they are stronger than we.
\verse And they brought up an evil report of the land which they had searched unto the children of Israel, saying, The land, through which we have gone to search it, is a land that eateth up the inhabitants thereof; and all the people that we saw in it are men of a great stature.
\verse And there we saw the giants, the sons of Anak, which come of the giants: and we were in our own sight as grasshoppers, and so we were in their sight.
\end{biblechapter}

\begin{biblechapter} % Numbers 14
\verseWithHeading{The people rebel} And all the congregation lifted up their voice, and cried; and the people wept that night.
\verse And all the children of Israel murmured against Moses and against Aaron: and the whole congregation said unto them, Would God that we had died in the land of Egypt! or would God we had died in this wilderness!
\verse And wherefore hath the \LORD brought us unto this land, to fall by the sword, that our wives and our children should be a prey? were it not better for us to return into Egypt?
\verse And they said one to another, Let us make a captain, and let us return into Egypt.
\verse Then Moses and Aaron fell on their faces before all the assembly of the congregation of the children of Israel.
\verse And Joshua the son of Nun, and Caleb the son of Jephunneh, which were of them that searched the land, rent their clothes:
\verse And they spake unto all the company of the children of Israel, saying, The land, which we passed through to search it, is an exceeding good land.
\verse If the \LORD delight in us, then he will bring us into this land, and give it us; a land which floweth with milk and honey.
\verse Only rebel not ye against the \LORD, neither fear ye the people of the land; for they are bread for us: their defence is departed from them, and the \LORD is with us: fear them not.
\verse But all the congregation bade stone them with stones. And the glory of the \LORD appeared in the tabernacle of the congregation before all the children of Israel.
\verse And the \LORD said unto Moses, How long will this people provoke me? and how long will it be ere they believe me, for all the signs which I have shewed among them?
\verse I will smite them with the pestilence, and disinherit them, and will make of thee a greater nation and mightier than they.
\verse And Moses said unto the \LORD, Then the Egyptians shall hear it, (for thou broughtest up this people in thy might from among them;)
\verse And they will tell it to the inhabitants of this land: for they have heard that thou \LORD art among this people, that thou \LORD art seen face to face, and that thy cloud standeth over them, and that thou goest before them, by day time in a pillar of a cloud, and in a pillar of fire by night.
\verse Now if thou shalt kill all this people as one man, then the nations which have heard the fame of thee will speak, saying,
\verse Because the \LORD was not able to bring this people into the land which he sware unto them, therefore he hath slain them in the wilderness.
\verse And now, I beseech thee, let the power of my Lord be great, according as thou hast spoken, saying,
\verse The \LORD is longsuffering, and of great mercy, forgiving iniquity and transgression, and by no means clearing the guilty, visiting the iniquity of the fathers upon the children unto the third and fourth generation.
\verse Pardon, I beseech thee, the iniquity of this people according unto the greatness of thy mercy, and as thou hast forgiven this people, from Egypt even until now.
\verse And the \LORD said, I have pardoned according to thy word:
\verse But as truly as I live, all the earth shall be filled with the glory of the \LORD.
\verse Because all those men which have seen my glory, and my miracles, which I did in Egypt and in the wilderness, and have tempted me now these ten times, and have not hearkened to my voice;
\verse Surely they shall not see the land which I sware unto their fathers, neither shall any of them that provoked me see it:
\verse But my servant Caleb, because he had another spirit with him, and hath followed me fully, him will I bring into the land whereinto he went; and his seed shall possess it.
\verse (Now the Amalekites and the Canaanites dwelt in the valley.) To morrow turn you, and get you into the wilderness by the way of the Red sea.
\verse And the \LORD spake unto Moses and unto Aaron, saying,
\verse How long shall I bear with this evil congregation, which murmur against me? I have heard the murmurings of the children of Israel, which they murmur against me.
\verse Say unto them, As truly as I live, saith the \LORD, as ye have spoken in mine ears, so will I do to you:
\verse Your carcases shall fall in this wilderness; and all that were numbered of you, according to your whole number, from twenty years old and upward, which have murmured against me,
\verse Doubtless ye shall not come into the land, concerning which I sware to make you dwell therein, save Caleb the son of Jephunneh, and Joshua the son of Nun.
\verse But your little ones, which ye said should be a prey, them will I bring in, and they shall know the land which ye have despised.
\verse But as for you, your carcases, they shall fall in this wilderness.
\verse And your children shall wander in the wilderness forty years, and bear your whoredoms, until your carcases be wasted in the wilderness.
\verse After the number of the days in which ye searched the land, even forty days, each day for a year, shall ye bear your iniquities, even forty years, and ye shall know my breach of promise.
\verse I the \LORD have said, I will surely do it unto all this evil congregation, that are gathered together against me: in this wilderness they shall be consumed, and there they shall die.
\verse And the men, which Moses sent to search the land, who returned, and made all the congregation to murmur against him, by bringing up a slander upon the land,
\verse Even those men that did bring up the evil report upon the land, died by the plague before the \LORD.
\verse But Joshua the son of Nun, and Caleb the son of Jephunneh, which were of the men that went to search the land, lived still.
\verse And Moses told these sayings unto all the children of Israel: and the people mourned greatly.
\verse And they rose up early in the morning, and gat them up into the top of the mountain, saying, Lo, we be here, and will go up unto the place which the \LORD hath promised: for we have sinned.
\verse And Moses said, Wherefore now do ye transgress the commandment of the \LORD? but it shall not prosper.
\verse Go not up, for the \LORD is not among you; that ye be not smitten before your enemies.
\verse For the Amalekites and the Canaanites are there before you, and ye shall fall by the sword: because ye are turned away from the \LORD, therefore the \LORD will not be with you.
\verse But they presumed to go up unto the hill top: nevertheless the ark of the covenant of the \LORD, and Moses, departed not out of the camp.
\verse Then the Amalekites came down, and the Canaanites which dwelt in that hill, and smote them, and discomfited them, even unto Hormah.
\end{biblechapter}

\vfill\columnbreak % layout hack

\begin{biblechapter} % Numbers 15
\verseWithHeading{Supplementary offerings} And the \LORD spake unto Moses, saying,
\verse Speak unto the children of Israel, and say unto them, When ye be come into the land of your habitations, which I give unto you,
\verse And will make an offering by fire unto the \LORD, a burnt offering, or a sacrifice in performing a vow, or in a freewill offering, or in your solemn feasts, to make a sweet savour unto the \LORD, of the herd, or of the flock:
\verse Then shall he that offereth his offering unto the \LORD bring a meat offering of a tenth deal of flour mingled with the fourth part of an hin of oil.
\verse And the fourth part of an hin of wine for a drink offering shalt thou prepare with the burnt offering or sacrifice, for one lamb.
\verse Or for a ram, thou shalt prepare for a meat offering two tenth deals of flour mingled with the third part of an hin of oil.
\verse And for a drink offering thou shalt offer the third part of an hin of wine, for a sweet savour unto the \LORD.
\verse And when thou preparest a bullock for a burnt offering, or for a sacrifice in performing a vow, or peace offerings unto the \LORD:
\verse Then shall he bring with a bullock a meat offering of three tenth deals of flour mingled with half an hin of oil.
\verse And thou shalt bring for a drink offering half an hin of wine, for an offering made by fire, of a sweet savour unto the \LORD.
\verse Thus shall it be done for one bullock, or for one ram, or for a lamb, or a kid.
\verse According to the number that ye shall prepare, so shall ye do to every one according to their number.
\verse All that are born of the country shall do these things after this manner, in offering an offering made by fire, of a sweet savour unto the \LORD.
\verse And if a stranger sojourn with you, or whosoever be among you in your generations, and will offer an offering made by fire, of a sweet savour unto the \LORD; as ye do, so he shall do.
\verse One ordinance shall be both for you of the congregation, and also for the stranger that sojourneth with you, an ordinance for ever in your generations: as ye are, so shall the stranger be before the \LORD.
\verse One law and one manner shall be for you, and for the stranger that sojourneth with you.
\verse And the \LORD spake unto Moses, saying,
\verse Speak unto the children of Israel, and say unto them, When ye come into the land whither I bring you,
\verse Then it shall be, that, when ye eat of the bread of the land, ye shall offer up an heave offering unto the \LORD.
\verse Ye shall offer up a cake of the first of your dough for an heave offering: as ye do the heave offering of the threshingfloor, so shall ye heave it.
\verse Of the first of your dough ye shall give unto the \LORD an heave offering in your generations.
\verseWithHeading{Offerings for unintentional sins} And if ye have erred, and not observed all these commandments, which the \LORD hath spoken unto Moses,
\verse Even all that the \LORD hath commanded you by the hand of Moses, from the day that the \LORD commanded Moses, and henceforward among your generations;
\verse Then it shall be, if ought be committed by ignorance without the knowledge of the congregation, that all the congregation shall offer one young bullock for a burnt offering, for a sweet savour unto the \LORD, with his meat offering, and his drink offering, according to the manner, and one kid of the goats for a sin offering.
\verse And the priest shall make an atonement for all the congregation of the children of Israel, and it shall be forgiven them; for it is ignorance: and they shall bring their offering, a sacrifice made by fire unto the \LORD, and their sin offering before the \LORD, for their ignorance:
\verse And it shall be forgiven all the congregation of the children of Israel, and the stranger that sojourneth among them; seeing all the people were in ignorance.
\verse And if any soul sin through ignorance, then he shall bring a she goat of the first year for a sin offering.
\verse And the priest shall make an atonement for the soul that sinneth ignorantly, when he sinneth by ignorance before the \LORD, to make an atonement for him; and it shall be forgiven him.
\verse Ye shall have one law for him that sinneth through ignorance, both for him that is born among the children of Israel, and for the stranger that sojourneth among them.
\verse But the soul that doeth ought presumptuously, whether he be born in the land, or a stranger, the same reproacheth the \LORD; and that soul shall be cut off from among his people.
\verse Because he hath despised the word of the \LORD, and hath broken his commandment, that soul shall utterly be cut off; his iniquity shall be upon him.
\verseWithHeading{The Sabbath-breaker put to death} And while the children of Israel were in the wilderness, they found a man that gathered sticks upon the Sabbath day.
\verse And they that found him gathering sticks brought him unto Moses and Aaron, and unto all the congregation.
\verse And they put him in ward, because it was not declared what should be done to him.
\verse And the \LORD said unto Moses, The man shall be surely put to death: all the congregation shall stone him with stones without the camp.
\verse And all the congregation brought him without the camp, and stoned him with stones, and he died; as the \LORD commanded Moses.
\verseWithHeading{Fringes on garments} And the \LORD spake unto Moses, saying,
\verse Speak unto the children of Israel, and bid them that they make them fringes in the borders of their garments throughout their generations, and that they put upon the fringe of the borders a ribband of blue:
\verse And it shall be unto you for a fringe, that ye may look upon it, and remember all the commandments of the \LORD, and do them; and that ye seek not after your own heart and your own eyes, after which ye use to go a whoring:
\verse That ye may remember, and do all my commandments, and be holy unto your God.
\verse I am the \LORD your God, which brought you out of the land of Egypt, to be your God: I am the \LORD your God.
\end{biblechapter}

\begin{biblechapter} % Numbers 16
\verseWithHeading{Korah, Dathan, and Abiram} Now Korah, the son of Izhar, the son of Kohath, the son of Levi, and Dathan and Abiram, the sons of Eliab, and On, the son of Peleth, sons of Reuben, took men:
\verse And they rose up before Moses, with certain of the children of Israel, two hundred and fifty princes of the assembly, famous in the congregation, men of renown:
\verse And they gathered themselves together against Moses and against Aaron, and said unto them, Ye take too much upon you, seeing all the congregation are holy, every one of them, and the \LORD is among them: wherefore then lift ye up yourselves above the congregation of the \LORD?
\verse And when Moses heard it, he fell upon his face:
\verse And he spake unto Korah and unto all his company, saying, Even to morrow the \LORD will shew who are his, and who is holy; and will cause him to come near unto him: even him whom he hath chosen will he cause to come near unto him.
\verse This do; Take you censers, Korah, and all his company;
\verse And put fire therein, and put incense in them before the \LORD to morrow: and it shall be that the man whom the \LORD doth choose, he shall be holy: ye take too much upon you, ye sons of Levi.
\verse And Moses said unto Korah, Hear, I pray you, ye sons of Levi:
\verse Seemeth it but a small thing unto you, that the God of Israel hath separated you from the congregation of Israel, to bring you near to himself to do the service of the tabernacle of the \LORD, and to stand before the congregation to minister unto them?
\verse And he hath brought thee near to him, and all thy brethren the sons of Levi with thee: and seek ye the priesthood also?
\verse For which cause both thou and all thy company are gathered together against the \LORD: and what is Aaron, that ye murmur against him?
\verse And Moses sent to call Dathan and Abiram, the sons of Eliab: which said, We will not come up:
\verse Is it a small thing that thou hast brought us up out of a land that floweth with milk and honey, to kill us in the wilderness, except thou make thyself altogether a prince over us?
\verse Moreover thou hast not brought us into a land that floweth with milk and honey, or given us inheritance of fields and vineyards: wilt thou put out the eyes of these men? we will not come up.
\verse And Moses was very wroth, and said unto the \LORD, Respect not thou their offering: I have not taken one ass from them, neither have I hurt one of them.
\verse And Moses said unto Korah, Be thou and all thy company before the \LORD, thou, and they, and Aaron, to morrow:
\verse And take every man his censer, and put incense in them, and bring ye before the \LORD every man his censer, two hundred and fifty censers; thou also, and Aaron, each of you his censer.
\verse And they took every man his censer, and put fire in them, and laid incense thereon, and stood in the door of the tabernacle of the congregation with Moses and Aaron.
\verse And Korah gathered all the congregation against them unto the door of the tabernacle of the congregation: and the glory of the \LORD appeared unto all the congregation.
\verse And the \LORD spake unto Moses and unto Aaron, saying,
\verse Separate yourselves from among this congregation, that I may consume them in a moment.
\verse And they fell upon their faces, and said, O God, the God of the spirits of all flesh, shall one man sin, and wilt thou be wroth with all the congregation?
\verse And the \LORD spake unto Moses, saying,
\verse Speak unto the congregation, saying, Get you up from about the tabernacle of Korah, Dathan, and Abiram.
\verse And Moses rose up and went unto Dathan and Abiram; and the elders of Israel followed him.
\verse And he spake unto the congregation, saying, Depart, I pray you, from the tents of these wicked men, and touch nothing of theirs, lest ye be consumed in all their sins.
\verse So they gat up from the tabernacle of Korah, Dathan, and Abiram, on every side: and Dathan and Abiram came out, and stood in the door of their tents, and their wives, and their sons, and their little children.
\verse And Moses said, Hereby ye shall know that the \LORD hath sent me to do all these works; for I have not done them of mine own mind.
\verse If these men die the common death of all men, or if they be visited after the visitation of all men; then the \LORD hath not sent me.
\verse But if the \LORD make a new thing, and the earth open her mouth, and swallow them up, with all that appertain unto them, and they go down quick into the pit; then ye shall understand that these men have provoked the \LORD.
\verse And it came to pass, as he had made an end of speaking all these words, that the ground clave asunder that was under them:
\verse And the earth opened her mouth, and swallowed them up, and their houses, and all the men that appertained unto Korah, and all their goods.
\verse They, and all that appertained to them, went down alive into the pit, and the earth closed upon them: and they perished from among the congregation.
\verse And all Israel that were round about them fled at the cry of them: for they said, Lest the earth swallow us up also.
\verse And there came out a fire from the \LORD, and consumed the two hundred and fifty men that offered incense.
\verse And the \LORD spake unto Moses, saying,
\verse Speak unto Eleazar the son of Aaron the priest, that he take up the censers out of the burning, and scatter thou the fire yonder; for they are hallowed.
\verse The censers of these sinners against their own souls, let them make them broad plates for a covering of the altar: for they offered them before the \LORD, therefore they are hallowed: and they shall be a sign unto the children of Israel.
\verse And Eleazar the priest took the brasen censers, wherewith they that were burnt had offered; and they were made broad plates for a covering of the altar:
\verse To be a memorial unto the children of Israel, that no stranger, which is not of the seed of Aaron, come near to offer incense before the \LORD; that he be not as Korah, and as his company: as the \LORD said to him by the hand of Moses.
\verse But on the morrow all the congregation of the children of Israel murmured against Moses and against Aaron, saying, Ye have killed the people of the \LORD.
\verse And it came to pass, when the congregation was gathered against Moses and against Aaron, that they looked toward the tabernacle of the congregation: and, behold, the cloud covered it, and the glory of the \LORD appeared.
\verse And Moses and Aaron came before the tabernacle of the congregation.
\verse And the \LORD spake unto Moses, saying,
\verse Get you up from among this congregation, that I may consume them as in a moment. And they fell upon their faces.
\verse And Moses said unto Aaron, Take a censer, and put fire therein from off the altar, and put on incense, and go quickly unto the congregation, and make an atonement for them: for there is wrath gone out from the \LORD; the plague is begun.
\verse And Aaron took as Moses commanded, and ran into the midst of the congregation; and, behold, the plague was begun among the people: and he put on incense, and made an atonement for the people.
\verse And he stood between the dead and the living; and the plague was stayed.
\verse Now they that died in the plague were fourteen thousand and seven hundred, beside them that died about the matter of Korah.
\verse And Aaron returned unto Moses unto the door of the tabernacle of the congregation: and the plague was stayed.
\end{biblechapter}

\begin{biblechapter} % Numbers 17
\verseWithHeading{The budding of Aaron's staff} And the \LORD spake unto Moses, saying,
\verse Speak unto the children of Israel, and take of every one of them a rod according to the house of their fathers, of all their princes according to the house of their fathers twelve rods: write thou every man's name upon his rod.
\verse And thou shalt write Aaron's name upon the rod of Levi: for one rod shall be for the head of the house of their fathers.
\verse And thou shalt lay them up in the tabernacle of the congregation before the testimony, where I will meet with you.
\verse And it shall come to pass, that the man's rod, whom I shall choose, shall blossom: and I will make to cease from me the murmurings of the children of Israel, whereby they murmur against you.
\verse And Moses spake unto the children of Israel, and every one of their princes gave him a rod apiece, for each prince one, according to their fathers' houses, even twelve rods: and the rod of Aaron was among their rods.
\verse And Moses laid up the rods before the \LORD in the tabernacle of witness.
\verse And it came to pass, that on the morrow Moses went into the tabernacle of witness; and, behold, the rod of Aaron for the house of Levi was budded, and brought forth buds, and bloomed blossoms, and yielded almonds.
\verse And Moses brought out all the rods from before the \LORD unto all the children of Israel: and they looked, and took every man his rod.
\verse And the \LORD said unto Moses, Bring Aaron's rod again before the testimony, to be kept for a token against the rebels; and thou shalt quite take away their murmurings from me, that they die not.
\verse And Moses did so: as the \LORD commanded him, so did he.
\verse And the children of Israel spake unto Moses, saying, Behold, we die, we perish, we all perish.
\verse Whosoever cometh any thing near unto the tabernacle of the \LORD shall die: shall we be consumed with dying?
\end{biblechapter}

\vfill\columnbreak % layout hack

\begin{biblechapter} % Numbers 18
\verseWithHeading{Duties of priests and Levites} And the \LORD said unto Aaron, Thou and thy sons and thy father's house with thee shall bear the iniquity of the sanctuary: and thou and thy sons with thee shall bear the iniquity of your priesthood.
\verse And thy brethren also of the tribe of Levi, the tribe of thy father, bring thou with thee, that they may be joined unto thee, and minister unto thee: but thou and thy sons with thee shall minister before the tabernacle of witness.
\verse And they shall keep thy charge, and the charge of all the tabernacle: only they shall not come nigh the vessels of the sanctuary and the altar, that neither they, nor ye also, die.
\verse And they shall be joined unto thee, and keep the charge of the tabernacle of the congregation, for all the service of the tabernacle: and a stranger shall not come nigh unto you.
\verse And ye shall keep the charge of the sanctuary, and the charge of the altar: that there be no wrath any more upon the children of Israel.
\verse And I, behold, I have taken your brethren the Levites from among the children of Israel: to you they are given as a gift for the \LORD, to do the service of the tabernacle of the congregation.
\verse Therefore thou and thy sons with thee shall keep your priest's office for every thing of the altar, and within the vail; and ye shall serve: I have given your priest's office unto you as a service of gift: and the stranger that cometh nigh shall be put to death.
\verseWithHeading{Offerings for priests and Levites} And the \LORD spake unto Aaron, Behold, I also have given thee the charge of mine heave offerings of all the hallowed things of the children of Israel; unto thee have I given them by reason of the anointing, and to thy sons, by an ordinance for ever.
\verse This shall be thine of the most holy things, reserved from the fire: every oblation of theirs, every meat offering of theirs, and every sin offering of theirs, and every trespass offering of theirs, which they shall render unto me, shall be most holy for thee and for thy sons.
\verse In the most holy place shalt thou eat it; every male shall eat it: it shall be holy unto thee.
\verse And this is thine; the heave offering of their gift, with all the wave offerings of the children of Israel: I have given them unto thee, and to thy sons and to thy daughters with thee, by a statute for ever: every one that is clean in thy house shall eat of it.
\verse All the best of the oil, and all the best of the wine, and of the wheat, the firstfruits of them which they shall offer unto the \LORD, them have I given thee.
\verse And whatsoever is first ripe in the land, which they shall bring unto the \LORD, shall be thine; every one that is clean in thine house shall eat of it.
\verse Every thing devoted in Israel shall be thine.
\verse Every thing that openeth the matrix in all flesh, which they bring unto the \LORD, whether it be of men or beasts, shall be thine: nevertheless the firstborn of man shalt thou surely redeem, and the firstling of unclean beasts shalt thou redeem.
\verse And those that are to be redeemed from a month old shalt thou redeem, according to thine estimation, for the money of five shekels, after the shekel of the sanctuary, which is twenty gerahs.
\verse But the firstling of a cow, or the firstling of a sheep, or the firstling of a goat, thou shalt not redeem; they are holy: thou shalt sprinkle their blood upon the altar, and shalt burn their fat for an offering made by fire, for a sweet savour unto the \LORD.
\verse And the flesh of them shall be thine, as the wave breast and as the right shoulder are thine.
\verse All the heave offerings of the holy things, which the children of Israel offer unto the \LORD, have I given thee, and thy sons and thy daughters with thee, by a statute for ever: it is a covenant of salt for ever before the \LORD unto thee and to thy seed with thee.
\verse And the \LORD spake unto Aaron, Thou shalt have no inheritance in their land, neither shalt thou have any part among them: I am thy part and thine inheritance among the children of Israel.
\verse And, behold, I have given the children of Levi all the tenth in Israel for an inheritance, for their service which they serve, even the service of the tabernacle of the congregation.
\verse Neither must the children of Israel henceforth come nigh the tabernacle of the congregation, lest they bear sin, and die.
\verse But the Levites shall do the service of the tabernacle of the congregation, and they shall bear their iniquity: it shall be a statute for ever throughout your generations, that among the children of Israel they have no inheritance.
\verse But the tithes of the children of Israel, which they offer as an heave offering unto the \LORD, I have given to the Levites to inherit: therefore I have said unto them, Among the children of Israel they shall have no inheritance.
\verse And the \LORD spake unto Moses, saying,
\verse Thus speak unto the Levites, and say unto them, When ye take of the children of Israel the tithes which I have given you from them for your inheritance, then ye shall offer up an heave offering of it for the \LORD, even a tenth part of the tithe.
\verse And this your heave offering shall be reckoned unto you, as though it were the corn of the threshingfloor, and as the fulness of the winepress.
\verse Thus ye also shall offer an heave offering unto the \LORD of all your tithes, which ye receive of the children of Israel; and ye shall give thereof the \LORDs heave offering to Aaron the priest.
\verse Out of all your gifts ye shall offer every heave offering of the \LORD, of all the best thereof, even the hallowed part thereof out of it.
\verse Therefore thou shalt say unto them, When ye have heaved the best thereof from it, then it shall be counted unto the Levites as the increase of the threshingfloor, and as the increase of the winepress.
\verse And ye shall eat it in every place, ye and your households: for it is your reward for your service in the tabernacle of the congregation.
\verse And ye shall bear no sin by reason of it, when ye have heaved from it the best of it: neither shall ye pollute the holy things of the children of Israel, lest ye die.
\end{biblechapter}

\begin{biblechapter} % Numbers 19
\verseWithHeading{The water of cleansing} And the \LORD spake unto Moses and unto Aaron, saying,
\verse This is the ordinance of the law which the \LORD hath commanded, saying, Speak unto the children of Israel, that they bring thee a red heifer without spot, wherein is no blemish, and upon which never came yoke:
\verse And ye shall give her unto Eleazar the priest, that he may bring her forth without the camp, and one shall slay her before his face:
\verse And Eleazar the priest shall take of her blood with his finger, and sprinkle of her blood directly before the tabernacle of the congregation seven times:
\verse And one shall burn the heifer in his sight; her skin, and her flesh, and her blood, with her dung, shall he burn:
\verse And the priest shall take cedar wood, and hyssop, and scarlet, and cast it into the midst of the burning of the heifer.
\verse Then the priest shall wash his clothes, and he shall bathe his flesh in water, and afterward he shall come into the camp, and the priest shall be unclean until the even.
\verse And he that burneth her shall wash his clothes in water, and bathe his flesh in water, and shall be unclean until the even.
\verse And a man that is clean shall gather up the ashes of the heifer, and lay them up without the camp in a clean place, and it shall be kept for the congregation of the children of Israel for a water of separation: it is a purification for sin.
\verse And he that gathereth the ashes of the heifer shall wash his clothes, and be unclean until the even: and it shall be unto the children of Israel, and unto the stranger that sojourneth among them, for a statute for ever.
\verse He that toucheth the dead body of any man shall be unclean seven days.
\verse He shall purify himself with it on the third day, and on the seventh day he shall be clean: but if he purify not himself the third day, then the seventh day he shall not be clean.
\verse Whosoever toucheth the dead body of any man that is dead, and purifieth not himself, defileth the tabernacle of the \LORD; and that soul shall be cut off from Israel: because the water of separation was not sprinkled upon him, he shall be unclean; his uncleanness is yet upon him.
\verse This is the law, when a man dieth in a tent: all that come into the tent, and all that is in the tent, shall be unclean seven days.
\verse And every open vessel, which hath no covering bound upon it, is unclean.
\verse And whosoever toucheth one that is slain with a sword in the open fields, or a dead body, or a bone of a man, or a grave, shall be unclean seven days.
\verse And for an unclean person they shall take of the ashes of the burnt heifer of purification for sin, and running water shall be put thereto in a vessel:
\verse And a clean person shall take hyssop, and dip it in the water, and sprinkle it upon the tent, and upon all the vessels, and upon the persons that were there, and upon him that touched a bone, or one slain, or one dead, or a grave:
\verse And the clean person shall sprinkle upon the unclean on the third day, and on the seventh day: and on the seventh day he shall purify himself, and wash his clothes, and bathe himself in water, and shall be clean at even.
\verse But the man that shall be unclean, and shall not purify himself, that soul shall be cut off from among the congregation, because he hath defiled the sanctuary of the \LORD: the water of separation hath not been sprinkled upon him; he is unclean.
\verse And it shall be a perpetual statute unto them, that he that sprinkleth the water of separation shall wash his clothes; and he that toucheth the water of separation shall be unclean until even.
\verse And whatsoever the unclean person toucheth shall be unclean; and the soul that toucheth it shall be unclean until even.
\end{biblechapter}

\begin{biblechapter} % Numbers 20
\verseWithHeading{Water from the rock} Then came the children of Israel, even the whole congregation, into the desert of Zin in the first month: and the people abode in Kadesh; and Miriam died there, and was buried there.
\verse And there was no water for the congregation: and they gathered themselves together against Moses and against Aaron.
\verse And the people chode with Moses, and spake, saying, Would God that we had died when our brethren died before the \LORD!
\verse And why have ye brought up the congregation of the \LORD into this wilderness, that we and our cattle should die there?
\verse And wherefore have ye made us to come up out of Egypt, to bring us in unto this evil place? it is no place of seed, or of figs, or of vines, or of pomegranates; neither is there any water to drink.
\verse And Moses and Aaron went from the presence of the assembly unto the door of the tabernacle of the congregation, and they fell upon their faces: and the glory of the \LORD appeared unto them.
\verse And the \LORD spake unto Moses, saying,
\verse Take the rod, and gather thou the assembly together, thou, and Aaron thy brother, and speak ye unto the rock before their eyes; and it shall give forth his water, and thou shalt bring forth to them water out of the rock: so thou shalt give the congregation and their beasts drink.
\verse And Moses took the rod from before the \LORD, as he commanded him.
\verse And Moses and Aaron gathered the congregation together before the rock, and he said unto them, Hear now, ye rebels; must we fetch you water out of this rock?
\verse And Moses lifted up his hand, and with his rod he smote the rock twice: and the water came out abundantly, and the congregation drank, and their beasts also.
\verse And the \LORD spake unto Moses and Aaron, Because ye believed me not, to sanctify me in the eyes of the children of Israel, therefore ye shall not bring this congregation into the land which I have given them.
\verse This is the water of Meribah; because the children of Israel strove with the \LORD, and he was sanctified in them.
\verseWithHeading{Edom denies Israel passage} And Moses sent messengers from Kadesh unto the king of Edom, Thus saith thy brother Israel, Thou knowest all the travail that hath befallen us:
\verse How our fathers went down into Egypt, and we have dwelt in Egypt a long time; and the Egyptians vexed us, and our fathers:
\verse And when we cried unto the \LORD, he heard our voice, and sent an angel, and hath brought us forth out of Egypt: and, behold, we are in Kadesh, a city in the uttermost of thy border:
\verse Let us pass, I pray thee, through thy country: we will not pass through the fields, or through the vineyards, neither will we drink of the water of the wells: we will go by the king's high way, we will not turn to the right hand nor to the left, until we have passed thy borders.
\verse And Edom said unto him, Thou shalt not pass by me, lest I come out against thee with the sword.
\verse And the children of Israel said unto him, We will go by the high way: and if I and my cattle drink of thy water, then I will pay for it: I will only, without doing any thing else, go through on my feet.
\verse And he said, Thou shalt not go through. And Edom came out against him with much people, and with a strong hand.
\verse Thus Edom refused to give Israel passage through his border: wherefore Israel turned away from him.
\verseWithHeading{The death of Aaron} And the children of Israel, even the whole congregation, journeyed from Kadesh, and came unto mount Hor.
\verse And the \LORD spake unto Moses and Aaron in mount Hor, by the coast of the land of Edom, saying,
\verse Aaron shall be gathered unto his people: for he shall not enter into the land which I have given unto the children of Israel, because ye rebelled against my word at the water of Meribah.
\verse Take Aaron and Eleazar his son, and bring them up unto mount Hor:
\verse And strip Aaron of his garments, and put them upon Eleazar his son: and Aaron shall be gathered unto his people, and shall die there.
\verse And Moses did as the \LORD commanded: and they went up into mount Hor in the sight of all the congregation.
\verse And Moses stripped Aaron of his garments, and put them upon Eleazar his son; and Aaron died there in the top of the mount: and Moses and Eleazar came down from the mount.
\verse And when all the congregation saw that Aaron was dead, they mourned for Aaron thirty days, even all the house of Israel.
\end{biblechapter}

\begin{biblechapter} % Numbers 21
\verseWithHeading{Arad destroyed} And when king Arad the Canaanite, which dwelt in the south, heard tell that Israel came by the way of the spies; then he fought against Israel, and took some of them prisoners.
\verse And Israel vowed a vow unto the \LORD, and said, If thou wilt indeed deliver this people into my hand, then I will utterly destroy their cities.
\verse And the \LORD hearkened to the voice of Israel, and delivered up the Canaanites; and they utterly destroyed them and their cities: and he called the name of the place Hormah.
\verse And they journeyed from mount Hor by the way of the Red sea, to compass the land of Edom: and the soul of the people was much discouraged because of the way.
\verse And the people spake against God, and against Moses, Wherefore have ye brought us up out of Egypt to die in the wilderness? for there is no bread, neither is there any water; and our soul loatheth this light bread.
\verse And the \LORD sent fiery serpents among the people, and they bit the people; and much people of Israel died.
\verse Therefore the people came to Moses, and said, We have sinned, for we have spoken against the \LORD, and against thee; pray unto the \LORD, that he take away the serpents from us. And Moses prayed for the people.
\verse And the \LORD said unto Moses, Make thee a fiery serpent, and set it upon a pole: and it shall come to pass, that every one that is bitten, when he looketh upon it, shall live.
\verse And Moses made a serpent of brass, and put it upon a pole, and it came to pass, that if a serpent had bitten any man, when he beheld the serpent of brass, he lived.
\verseWithHeading{The journey to Moab} And the children of Israel set forward, and pitched in Oboth.
\verse And they journeyed from Oboth, and pitched at Ijeabarim, in the wilderness which is before Moab, toward the sunrising.
\verse From thence they removed, and pitched in the valley of Zared.
\verse From thence they removed, and pitched on the other side of Arnon, which is in the wilderness that cometh out of the coasts of the Amorites: for Arnon is the border of Moab, between Moab and the Amorites.
\verse Wherefore it is said in the book of the wars of the \LORD, What he did in the Red sea, and in the brooks of Arnon,
\verse And at the stream of the brooks that goeth down to the dwelling of Ar, and lieth upon the border of Moab.
\verse And from thence they went to Beer: that is the well whereof the \LORD spake unto Moses, Gather the people together, and I will give them water.
\verse Then Israel sang this song, Spring up, O well; sing ye unto it:
\verse The princes digged the well, the nobles of the people digged it, by the direction of the lawgiver, with their staves. And from the wilderness they went to Mattanah:
\verse And from Mattanah to Nahaliel: and from Nahaliel to Bamoth:
\verse And from Bamoth in the valley, that is in the country of Moab, to the top of Pisgah, which looketh toward Jeshimon.
\verseWithHeading{Defeat of Sihon and Og} And Israel sent messengers unto Sihon king of the Amorites, saying,
\verse Let me pass through thy land: we will not turn into the fields, or into the vineyards; we will not drink of the waters of the well: but we will go along by the king's high way, until we be past thy borders.
\verse And Sihon would not suffer Israel to pass through his border: but Sihon gathered all his people together, and went out against Israel into the wilderness: and he came to Jahaz, and fought against Israel.
\verse And Israel smote him with the edge of the sword, and possessed his land from Arnon unto Jabbok, even unto the children of Ammon: for the border of the children of Ammon was strong.
\verse And Israel took all these cities: and Israel dwelt in all the cities of the Amorites, in Heshbon, and in all the villages thereof.
\verse For Heshbon was the city of Sihon the king of the Amorites, who had fought against the former king of Moab, and taken all his land out of his hand, even unto Arnon.
\verse Wherefore they that speak in proverbs say, Come into Heshbon, let the city of Sihon be built and prepared:
\verse For there is a fire gone out of Heshbon, a flame from the city of Sihon: it hath consumed Ar of Moab, and the lords of the high places of Arnon.
\verse Woe to thee, Moab! thou art undone, O people of Chemosh: he hath given his sons that escaped, and his daughters, into captivity unto Sihon king of the Amorites.
\verse We have shot at them; Heshbon is perished even unto Dibon, and we have laid them waste even unto Nophah, which reacheth unto Medeba.
\verse Thus Israel dwelt in the land of the Amorites.
\verse And Moses sent to spy out Jaazer, and they took the villages thereof, and drove out the Amorites that were there.
\verse And they turned and went up by the way of Bashan: and Og the king of Bashan went out against them, he, and all his people, to the battle at Edrei.
\verse And the \LORD said unto Moses, Fear him not: for I have delivered him into thy hand, and all his people, and his land; and thou shalt do to him as thou didst unto Sihon king of the Amorites, which dwelt at Heshbon.
\verse So they smote him, and his sons, and all his people, until there was none left him alive: and they possessed his land.
\end{biblechapter}

\begin{biblechapter} % Numbers 22
\verseWithHeading{Balak summons Balaam} And the children of Israel set forward, and pitched in the plains of Moab on this side Jordan by Jericho.
\verse And Balak the son of Zippor saw all that Israel had done to the Amorites.
\verse And Moab was sore afraid of the people, because they were many: and Moab was distressed because of the children of Israel.
\verse And Moab said unto the elders of Midian, Now shall this company lick up all that are round about us, as the ox licketh up the grass of the field. And Balak the son of Zippor was king of the Moabites at that time.
\verse He sent messengers therefore unto Balaam the son of Beor to Pethor, which is by the river of the land of the children of his people, to call him, saying, Behold, there is a people come out from Egypt: behold, they cover the face of the earth, and they abide over against me:
\verse Come now therefore, I pray thee, curse me this people; for they are too mighty for me: peradventure I shall prevail, that we may smite them, and that I may drive them out of the land: for I wot that he whom thou blessest is blessed, and he whom thou cursest is cursed.
\verse And the elders of Moab and the elders of Midian departed with the rewards of divination in their hand; and they came unto Balaam, and spake unto him the words of Balak.
\verse And he said unto them, Lodge here this night, and I will bring you word again, as the \LORD shall speak unto me: and the princes of Moab abode with Balaam.
\verse And God came unto Balaam, and said, What men are these with thee?
\verse And Balaam said unto God, Balak the son of Zippor, king of Moab, hath sent unto me, saying,
\verse Behold, there is a people come out of Egypt, which covereth the face of the earth: come now, curse me them; peradventure I shall be able to overcome them, and drive them out.
\verse And God said unto Balaam, Thou shalt not go with them; thou shalt not curse the people: for they are blessed.
\verse And Balaam rose up in the morning, and said unto the princes of Balak, Get you into your land: for the \LORD refuseth to give me leave to go with you.
\verse And the princes of Moab rose up, and they went unto Balak, and said, Balaam refuseth to come with us.
\verse And Balak sent yet again princes, more, and more honourable than they.
\verse And they came to Balaam, and said to him, Thus saith Balak the son of Zippor, Let nothing, I pray thee, hinder thee from coming unto me:
\verse For I will promote thee unto very great honour, and I will do whatsoever thou sayest unto me: come therefore, I pray thee, curse me this people.
\verse And Balaam answered and said unto the servants of Balak, If Balak would give me his house full of silver and gold, I cannot go beyond the word of the \LORD my God, to do less or more.
\verse Now therefore, I pray you, tarry ye also here this night, that I may know what the \LORD will say unto me more.
\verse And God came unto Balaam at night, and said unto him, If the men come to call thee, rise up, and go with them; but yet the word which I shall say unto thee, that shalt thou do.
\verseWithHeading{Balaam's donkey} And Balaam rose up in the morning, and saddled his ass, and went with the princes of Moab.
\verse And God's anger was kindled because he went: and the angel of the \LORD stood in the way for an adversary against him. Now he was riding upon his ass, and his two servants were with him.
\verse And the ass saw the angel of the \LORD standing in the way, and his sword drawn in his hand: and the ass turned aside out of the way, and went into the field: and Balaam smote the ass, to turn her into the way.
\verse But the angel of the \LORD stood in a path of the vineyards, a wall being on this side, and a wall on that side.
\verse And when the ass saw the angel of the \LORD, she thrust herself unto the wall, and crushed Balaam's foot against the wall: and he smote her again.
\verse And the angel of the \LORD went further, and stood in a narrow place, where was no way to turn either to the right hand or to the left.
\verse And when the ass saw the angel of the \LORD, she fell down under Balaam: and Balaam's anger was kindled, and he smote the ass with a staff.
\verse And the \LORD opened the mouth of the ass, and she said unto Balaam, What have I done unto thee, that thou hast smitten me these three times?
\verse And Balaam said unto the ass, Because thou hast mocked me: I would there were a sword in mine hand, for now would I kill thee.
\verse And the ass said unto Balaam, Am not I thine ass, upon which thou hast ridden ever since I was thine unto this day? was I ever wont to do so unto thee? And he said, Nay.
\verse Then the \LORD opened the eyes of Balaam, and he saw the angel of the \LORD standing in the way, and his sword drawn in his hand: and he bowed down his head, and fell flat on his face.
\verse And the angel of the \LORD said unto him, Wherefore hast thou smitten thine ass these three times? behold, I went out to withstand thee, because thy way is perverse before me:
\verse And the ass saw me, and turned from me these three times: unless she had turned from me, surely now also I had slain thee, and saved her alive.
\verse And Balaam said unto the angel of the \LORD, I have sinned; for I knew not that thou stoodest in the way against me: now therefore, if it displease thee, I will get me back again.
\verse And the angel of the \LORD said unto Balaam, Go with the men: but only the word that I shall speak unto thee, that thou shalt speak. So Balaam went with the princes of Balak.
\verse And when Balak heard that Balaam was come, he went out to meet him unto a city of Moab, which is in the border of Arnon, which is in the utmost coast.
\verse And Balak said unto Balaam, Did I not earnestly send unto thee to call thee? wherefore camest thou not unto me? am I not able indeed to promote thee to honour?
\verse And Balaam said unto Balak, Lo, I am come unto thee: have I now any power at all to say any thing? the word that God putteth in my mouth, that shall I speak.
\verse And Balaam went with Balak, and they came unto Kirjathhuzoth.
\verse And Balak offered oxen and sheep, and sent to Balaam, and to the princes that were with him.
\verse And it came to pass on the morrow, that Balak took Balaam, and brought him up into the high places of Baal, that thence he might see the utmost part of the people.
\end{biblechapter}

\begin{biblechapter} % Numbers 23
\verseWithHeading{Balaam's first message} And Balaam said unto Balak, Build me here seven altars, and prepare me here seven oxen and seven rams.
\verse And Balak did as Balaam had spoken; and Balak and Balaam offered on every altar a bullock and a ram.
\verse And Balaam said unto Balak, Stand by thy burnt offering, and I will go: peradventure the \LORD will come to meet me: and whatsoever he sheweth me I will tell thee. And he went to an high place.
\verse And God met Balaam: and he said unto him, I have prepared seven altars, and I have offered upon every altar a bullock and a ram.
\verse And the \LORD put a word in Balaam's mouth, and said, Return unto Balak, and thus thou shalt speak.
\verse And he returned unto him, and, lo, he stood by his burnt sacrifice, he, and all the princes of Moab.
\verse And he took up his parable, and said, Balak the king of Moab hath brought me from Aram, out of the mountains of the east, saying, Come, curse me Jacob, and come, defy Israel.
\verse How shall I curse, whom God hath not cursed? or how shall I defy, whom the \LORD hath not defied?
\verse For from the top of the rocks I see him, and from the hills I behold him: lo, the people shall dwell alone, and shall not be reckoned among the nations.
\verse Who can count the dust of Jacob, and the number of the fourth part of Israel? Let me die the death of the righteous, and let my last end be like his!
\verse And Balak said unto Balaam, What hast thou done unto me? I took thee to curse mine enemies, and, behold, thou hast blessed them altogether.
\verse And he answered and said, Must I not take heed to speak that which the \LORD hath put in my mouth?
\verseWithHeading{Balaam's second message} And Balak said unto him, Come, I pray thee, with me unto another place, from whence thou mayest see them: thou shalt see but the utmost part of them, and shalt not see them all: and curse me them from thence.
\verse And he brought him into the field of Zophim, to the top of Pisgah, and built seven altars, and offered a bullock and a ram on every altar.
\verse And he said unto Balak, Stand here by thy burnt offering, while I meet the \LORD yonder.
\verse And the \LORD met Balaam, and put a word in his mouth, and said, Go again unto Balak, and say thus.
\verse And when he came to him, behold, he stood by his burnt offering, and the princes of Moab with him. And Balak said unto him, What hath the \LORD spoken?
\verse And he took up his parable, and said, Rise up, Balak, and hear; hearken unto me, thou son of Zippor:
\verse God is not a man, that he should lie; neither the son of man, that he should repent: hath he said, and shall he not do it? or hath he spoken, and shall he not make it good?
\verse Behold, I have received commandment to bless: and he hath blessed; and I cannot reverse it.
\verse He hath not beheld iniquity in Jacob, neither hath he seen perverseness in Israel: the \LORD his God is with him, and the shout of a king is among them.
\verse God brought them out of Egypt; he hath as it were the strength of an unicorn.
\verse Surely there is no enchantment against Jacob, neither is there any divination against Israel: according to this time it shall be said of Jacob and of Israel, What hath God wrought!
\verse Behold, the people shall rise up as a great lion, and lift up himself as a young lion: he shall not lie down until he eat of the prey, and drink the blood of the slain.
\verse And Balak said unto Balaam, Neither curse them at all, nor bless them at all.
\verse But Balaam answered and said unto Balak, Told not I thee, saying, All that the \LORD speaketh, that I must do?
\verseWithHeading{Balaam's third message} And Balak said unto Balaam, Come, I pray thee, I will bring thee unto another place; peradventure it will please God that thou mayest curse me them from thence.
\verse And Balak brought Balaam unto the top of Peor, that looketh toward Jeshimon.
\verse And Balaam said unto Balak, Build me here seven altars, and prepare me here seven bullocks and seven rams.
\verse And Balak did as Balaam had said, and offered a bullock and a ram on every altar.
\end{biblechapter}

\begin{biblechapter} % Numbers 24
\verse And when Balaam saw that it pleased the \LORD to bless Israel, he went not, as at other times, to seek for enchantments, but he set his face toward the wilderness.
\verse And Balaam lifted up his eyes, and he saw Israel abiding in his tents according to their tribes; and the spirit of God came upon him.
\verse And he took up his parable, and said, Balaam the son of Beor hath said, and the man whose eyes are open hath said:
\verse He hath said, which heard the words of God, which saw the vision of the Almighty, falling into a trance, but having his eyes open:
\verse How goodly are thy tents, O Jacob, and thy tabernacles, O Israel!
\verse As the valleys are they spread forth, as gardens by the river's side, as the trees of lign aloes which the \LORD hath planted, and as cedar trees beside the waters.
\verse He shall pour the water out of his buckets, and his seed shall be in many waters, and his king shall be higher than Agag, and his kingdom shall be exalted.
\verse God brought him forth out of Egypt; he hath as it were the strength of an unicorn: he shall eat up the nations his enemies, and shall break their bones, and pierce them through with his arrows.
\verse He couched, he lay down as a lion, and as a great lion: who shall stir him up? Blessed is he that blesseth thee, and cursed is he that curseth thee.
\verse And Balak's anger was kindled against Balaam, and he smote his hands together: and Balak said unto Balaam, I called thee to curse mine enemies, and, behold, thou hast altogether blessed them these three times.
\verse Therefore now flee thou to thy place: I thought to promote thee unto great honour; but, lo, the \LORD hath kept thee back from honour.
\verse And Balaam said unto Balak, Spake I not also to thy messengers which thou sentest unto me, saying,
\verse If Balak would give me his house full of silver and gold, I cannot go beyond the commandment of the \LORD, to do either good or bad of mine own mind; but what the \LORD saith, that will I speak?
\verse And now, behold, I go unto my people: come therefore, and I will advertise thee what this people shall do to thy people in the latter days.
\verseWithHeading{Balaam's fourth message} And he took up his parable, and said, Balaam the son of Beor hath said, and the man whose eyes are open hath said:
\verse He hath said, which heard the words of God, and knew the knowledge of the most High, which saw the vision of the Almighty, falling into a trance, but having his eyes open:
\verse I shall see him, but not now: I shall behold him, but not nigh: there shall come a Star out of Jacob, and a Sceptre shall rise out of Israel, and shall smite the corners of Moab, and destroy all the children of Sheth.
\verse And Edom shall be a possession, Seir also shall be a possession for his enemies; and Israel shall do valiantly.
\verse Out of Jacob shall come he that shall have dominion, and shall destroy him that remaineth of the city.
\verseWithHeading{Balaam's fifth message} And when he looked on Amalek, he took up his parable, and said, Amalek was the first of the nations; but his latter end shall be that he perish for ever.
\verseWithHeading{Balaam's sixth message} And he looked on the Kenites, and took up his parable, and said, Strong is thy dwellingplace, and thou puttest thy nest in a rock.
\verse Nevertheless the Kenite shall be wasted, until Asshur shall carry thee away captive.
\verseWithHeading{Balaam's seventh message} And he took up his parable, and said, Alas, who shall live when God doeth this!
\verse And ships shall come from the coast of Chittim, and shall afflict Asshur, and shall afflict Eber, and he also shall perish for ever.
\verse And Balaam rose up, and went and returned to his place: and Balak also went his way.
\end{biblechapter}

\begin{biblechapter} % Numbers 25
\verseWithHeading{Moab seduces Israel} And Israel abode in Shittim, and the people began to commit whoredom with the daughters of Moab.
\verse And they called the people unto the sacrifices of their gods: and the people did eat, and bowed down to their gods.
\verse And Israel joined himself unto Baalpeor: and the anger of the \LORD was kindled against Israel.
\verse And the \LORD said unto Moses, Take all the heads of the people, and hang them up before the \LORD against the sun, that the fierce anger of the \LORD may be turned away from Israel.
\verse And Moses said unto the judges of Israel, Slay ye every one his men that were joined unto Baalpeor.
\verse And, behold, one of the children of Israel came and brought unto his brethren a Midianitish woman in the sight of Moses, and in the sight of all the congregation of the children of Israel, who were weeping before the door of the tabernacle of the congregation.
\verse And when Phinehas, the son of Eleazar, the son of Aaron the priest, saw it, he rose up from among the congregation, and took a javelin in his hand;
\verse And he went after the man of Israel into the tent, and thrust both of them through, the man of Israel, and the woman through her belly. So the plague was stayed from the children of Israel.
\verse And those that died in the plague were twenty and four thousand.
\verse And the \LORD spake unto Moses, saying,
\verse Phinehas, the son of Eleazar, the son of Aaron the priest, hath turned my wrath away from the children of Israel, while he was zealous for my sake among them, that I consumed not the children of Israel in my jealousy.
\verse Wherefore say, Behold, I give unto him my covenant of peace:
\verse And he shall have it, and his seed after him, even the covenant of an everlasting priesthood; because he was zealous for his God, and made an atonement for the children of Israel.
\verse Now the name of the Israelite that was slain, even that was slain with the Midianitish woman, was Zimri, the son of Salu, a prince of a chief house among the Simeonites.
\verse And the name of the Midianitish woman that was slain was Cozbi, the daughter of Zur; he was head over a people, and of a chief house in Midian.
\verse And the \LORD spake unto Moses, saying,
\verse Vex the Midianites, and smite them:
\verse For they vex you with their wiles, wherewith they have beguiled you in the matter of Peor, and in the matter of Cozbi, the daughter of a prince of Midian, their sister, which was slain in the day of the plague for Peor's sake.
\end{biblechapter}

\begin{biblechapter} % Numbers 26
\verseWithHeading{The second census} And it came to pass after the plague, that the \LORD spake unto Moses and unto Eleazar the son of Aaron the priest, saying,
\verse Take the sum of all the congregation of the children of Israel, from twenty years old and upward, throughout their fathers' house, all that are able to go to war in Israel.
\verse And Moses and Eleazar the priest spake with them in the plains of Moab by Jordan near Jericho, saying,
\verse Take the sum of the people, from twenty years old and upward; as the \LORD commanded Moses and the children of Israel, which went forth out of the land of Egypt.
\verse Reuben, the eldest son of Israel: the children of Reuben; Hanoch, of whom cometh the family of the Hanochites: of Pallu, the family of the Palluites:
\verse Of Hezron, the family of the Hezronites: of Carmi, the family of the Carmites.
\verse These are the families of the Reubenites: and they that were numbered of them were forty and three thousand and seven hundred and thirty.
\verse And the sons of Pallu; Eliab.
\verse And the sons of Eliab; Nemuel, and Dathan, and Abiram. This is that Dathan and Abiram, which were famous in the congregation, who strove against Moses and against Aaron in the company of Korah, when they strove against the \LORD:
\verse And the earth opened her mouth, and swallowed them up together with Korah, when that company died, what time the fire devoured two hundred and fifty men: and they became a sign.
\verse Notwithstanding the children of Korah died not.
\verse The sons of Simeon after their families: of Nemuel, the family of the Nemuelites: of Jamin, the family of the Jaminites: of Jachin, the family of the Jachinites:
\verse Of Zerah, the family of the Zarhites: of Shaul, the family of the Shaulites.
\verse These are the families of the Simeonites, twenty and two thousand and two hundred.
\verse The children of Gad after their families: of Zephon, the family of the Zephonites: of Haggi, the family of the Haggites: of Shuni, the family of the Shunites:
\verse Of Ozni, the family of the Oznites: of Eri, the family of the Erites:
\verse Of Arod, the family of the Arodites: of Areli, the family of the Arelites.
\verse These are the families of the children of Gad according to those that were numbered of them, forty thousand and five hundred.
\verse The sons of Judah were Er and Onan: and Er and Onan died in the land of Canaan.
\verse And the sons of Judah after their families were; of Shelah, the family of the Shelanites: of Pharez, the family of the Pharzites: of Zerah, the family of the Zarhites.
\verse And the sons of Pharez were; of Hezron, the family of the Hezronites: of Hamul, the family of the Hamulites.
\verse These are the families of Judah according to those that were numbered of them, threescore and sixteen thousand and five hundred.
\verse Of the sons of Issachar after their families: of Tola, the family of the Tolaites: of Pua, the family of the Punites:
\verse Of Jashub, the family of the Jashubites: of Shimron, the family of the Shimronites.
\verse These are the families of Issachar according to those that were numbered of them, threescore and four thousand and three hundred.
\verse Of the sons of Zebulun after their families: of Sered, the family of the Sardites: of Elon, the family of the Elonites: of Jahleel, the family of the Jahleelites.
\verse These are the families of the Zebulunites according to those that were numbered of them, threescore thousand and five hundred.
\verse The sons of Joseph after their families were Manasseh and Ephraim.
\verse Of the sons of Manasseh: of Machir, the family of the Machirites: and Machir begat Gilead: of Gilead come the family of the Gileadites.
\verse These are the sons of Gilead: of Jeezer, the family of the Jeezerites: of Helek, the family of the Helekites:
\verse And of Asriel, the family of the Asrielites: and of Shechem, the family of the Shechemites:
\verse And of Shemida, the family of the Shemidaites: and of Hepher, the family of the Hepherites.
\verse And Zelophehad the son of Hepher had no sons, but daughters: and the names of the daughters of Zelophehad were Mahlah, and Noah, Hoglah, Milcah, and Tirzah.
\verse These are the families of Manasseh, and those that were numbered of them, fifty and two thousand and seven hundred.
\verse These are the sons of Ephraim after their families: of Shuthelah, the family of the Shuthalhites: of Becher, the family of the Bachrites: of Tahan, the family of the Tahanites.
\verse And these are the sons of Shuthelah: of Eran, the family of the Eranites.
\verse These are the families of the sons of Ephraim according to those that were numbered of them, thirty and two thousand and five hundred. These are the sons of Joseph after their families.
\verse The sons of Benjamin after their families: of Bela, the family of the Belaites: of Ashbel, the family of the Ashbelites: of Ahiram, the family of the Ahiramites:
\verse Of Shupham, the family of the Shuphamites: of Hupham, the family of the Huphamites.
\verse And the sons of Bela were Ard and Naaman: of Ard, the family of the Ardites: and of Naaman, the family of the Naamites.
\verse These are the sons of Benjamin after their families: and they that were numbered of them were forty and five thousand and six hundred.
\verse These are the sons of Dan after their families: of Shuham, the family of the Shuhamites. These are the families of Dan after their families.
\verse All the families of the Shuhamites, according to those that were numbered of them, were threescore and four thousand and four hundred.
\verse Of the children of Asher after their families: of Jimna, the family of the Jimnites: of Jesui, the family of the Jesuites: of Beriah, the family of the Beriites.
\verse Of the sons of Beriah: of Heber, the family of the Heberites: of Malchiel, the family of the Malchielites.
\verse And the name of the daughter of Asher was Sarah.
\verse These are the families of the sons of Asher according to those that were numbered of them; who were fifty and three thousand and four hundred.
\verse Of the sons of Naphtali after their families: of Jahzeel, the family of the Jahzeelites: of Guni, the family of the Gunites:
\verse Of Jezer, the family of the Jezerites: of Shillem, the family of the Shillemites.
\verse These are the families of Naphtali according to their families: and they that were numbered of them were forty and five thousand and four hundred.
\verse These were the numbered of the children of Israel, six hundred thousand and a thousand seven hundred and thirty.
\verse And the \LORD spake unto Moses, saying,
\verse Unto these the land shall be divided for an inheritance according to the number of names.
\verse To many thou shalt give the more inheritance, and to few thou shalt give the less inheritance: to every one shall his inheritance be given according to those that were numbered of him.
\verse Notwithstanding the land shall be divided by lot: according to the names of the tribes of their fathers they shall inherit.
\verse According to the lot shall the possession thereof be divided between many and few.
\verse And these are they that were numbered of the Levites after their families: of Gershon, the family of the Gershonites: of Kohath, the family of the Kohathites: of Merari, the family of the Merarites.
\verse These are the families of the Levites: the family of the Libnites, the family of the Hebronites, the family of the Mahlites, the family of the Mushites, the family of the Korathites. And Kohath begat Amram.
\verse And the name of Amram's wife was Jochebed, the daughter of Levi, whom her mother bare to Levi in Egypt: and she bare unto Amram Aaron and Moses, and Miriam their sister.
\verse And unto Aaron was born Nadab, and Abihu, Eleazar, and Ithamar.
\verse And Nadab and Abihu died, when they offered strange fire before the \LORD.
\verse And those that were numbered of them were twenty and three thousand, all males from a month old and upward: for they were not numbered among the children of Israel, because there was no inheritance given them among the children of Israel.
\verse These are they that were numbered by Moses and Eleazar the priest, who numbered the children of Israel in the plains of Moab by Jordan near Jericho.
\verse But among these there was not a man of them whom Moses and Aaron the priest numbered, when they numbered the children of Israel in the wilderness of Sinai.
\verse For the \LORD had said of them, They shall surely die in the wilderness. And there was not left a man of them, save Caleb the son of Jephunneh, and Joshua the son of Nun.
\end{biblechapter}

\begin{biblechapter} % Numbers 27
\verseWithHeading{Zelophehad's daughters} Then came the daughters of Zelophehad, the son of Hepher, the son of Gilead, the son of Machir, the son of Manasseh, of the families of Manasseh the son of Joseph: and these are the names of his daughters; Mahlah, Noah, and Hoglah, and Milcah, and Tirzah.
\verse And they stood before Moses, and before Eleazar the priest, and before the princes and all the congregation, by the door of the tabernacle of the congregation, saying,
\verse Our father died in the wilderness, and he was not in the company of them that gathered themselves together against the \LORD in the company of Korah; but died in his own sin, and had no sons.
\verse Why should the name of our father be done away from among his family, because he hath no son? Give unto us therefore a possession among the brethren of our father.
\verse And Moses brought their cause before the \LORD.
\verse And the \LORD spake unto Moses, saying,
\verse The daughters of Zelophehad speak right: thou shalt surely give them a possession of an inheritance among their father's brethren; and thou shalt cause the inheritance of their father to pass unto them.
\verse And thou shalt speak unto the children of Israel, saying, If a man die, and have no son, then ye shall cause his inheritance to pass unto his daughter.
\verse And if he have no daughter, then ye shall give his inheritance unto his brethren.
\verse And if he have no brethren, then ye shall give his inheritance unto his father's brethren.
\verse And if his father have no brethren, then ye shall give his inheritance unto his kinsman that is next to him of his family, and he shall possess it: and it shall be unto the children of Israel a statute of judgment, as the \LORD commanded Moses.
\verseWithHeading{Joshua to succeed Moses} And the \LORD said unto Moses, Get thee up into this mount Abarim, and see the land which I have given unto the children of Israel.
\verse And when thou hast seen it, thou also shalt be gathered unto thy people, as Aaron thy brother was gathered.
\verse For ye rebelled against my commandment in the desert of Zin, in the strife of the congregation, to sanctify me at the water before their eyes: that is the water of Meribah in Kadesh in the wilderness of Zin.
\verse And Moses spake unto the \LORD, saying,
\verse Let the \LORD, the God of the spirits of all flesh, set a man over the congregation,
\verse Which may go out before them, and which may go in before them, and which may lead them out, and which may bring them in; that the congregation of the \LORD be not as sheep which have no shepherd.
\verse And the \LORD said unto Moses, Take thee Joshua the son of Nun, a man in whom is the spirit, and lay thine hand upon him;
\verse And set him before Eleazar the priest, and before all the congregation; and give him a charge in their sight.
\verse And thou shalt put some of thine honour upon him, that all the congregation of the children of Israel may be obedient.
\verse And he shall stand before Eleazar the priest, who shall ask counsel for him after the judgment of Urim before the \LORD: at his word shall they go out, and at his word they shall come in, both he, and all the children of Israel with him, even all the congregation.
\verse And Moses did as the \LORD commanded him: and he took Joshua, and set him before Eleazar the priest, and before all the congregation:
\verse And he laid his hands upon him, and gave him a charge, as the \LORD commanded by the hand of Moses.
\end{biblechapter}

\begin{biblechapter} % Numbers 28
\verseWithHeading{Daily offerings} And the \LORD spake unto Moses, saying,
\verse Command the children of Israel, and say unto them, My offering, and my bread for my sacrifices made by fire, for a sweet savour unto me, shall ye observe to offer unto me in their due season.
\verse And thou shalt say unto them, This is the offering made by fire which ye shall offer unto the \LORD; two lambs of the first year without spot day by day, for a continual burnt offering.
\verse The one lamb shalt thou offer in the morning, and the other lamb shalt thou offer at even;
\verse And a tenth part of an ephah of flour for a meat offering, mingled with the fourth part of an hin of beaten oil.
\verse It is a continual burnt offering, which was ordained in mount Sinai for a sweet savour, a sacrifice made by fire unto the \LORD.
\verse And the drink offering thereof shall be the fourth part of an hin for the one lamb: in the holy place shalt thou cause the strong wine to be poured unto the \LORD for a drink offering.
\verse And the other lamb shalt thou offer at even: as the meat offering of the morning, and as the drink offering thereof, thou shalt offer it, a sacrifice made by fire, of a sweet savour unto the \LORD.
\verseWithHeading{Sabbath offerings} And on the Sabbath day two lambs of the first year without spot, and two tenth deals of flour for a meat offering, mingled with oil, and the drink offering thereof:
\verse This is the burnt offering of every Sabbath, beside the continual burnt offering, and his drink offering.
\verseWithHeading{Monthly offerings} And in the beginnings of your months ye shall offer a burnt offering unto the \LORD; two young bullocks, and one ram, seven lambs of the first year without spot;
\verse And three tenth deals of flour for a meat offering, mingled with oil, for one bullock; and two tenth deals of flour for a meat offering, mingled with oil, for one ram;
\verse And a several tenth deal of flour mingled with oil for a meat offering unto one lamb; for a burnt offering of a sweet savour, a sacrifice made by fire unto the \LORD.
\verse And their drink offerings shall be half an hin of wine unto a bullock, and the third part of an hin unto a ram, and a fourth part of an hin unto a lamb: this is the burnt offering of every month throughout the months of the year.
\verse And one kid of the goats for a sin offering unto the \LORD shall be offered, beside the continual burnt offering, and his drink offering.
\verseWithHeading{Passover offerings} And in the fourteenth day of the first month is the Passover of the \LORD.
\verse And in the fifteenth day of this month is the feast: seven days shall unleavened bread be eaten.
\verse In the first day shall be an holy convocation; ye shall do no manner of servile work therein:
\verse But ye shall offer a sacrifice made by fire for a burnt offering unto the \LORD; two young bullocks, and one ram, and seven lambs of the first year: they shall be unto you without blemish:
\verse And their meat offering shall be of flour mingled with oil: three tenth deals shall ye offer for a bullock, and two tenth deals for a ram;
\verse A several tenth deal shalt thou offer for every lamb, throughout the seven lambs:
\verse And one goat for a sin offering, to make an atonement for you.
\verse Ye shall offer these beside the burnt offering in the morning, which is for a continual burnt offering.
\verse After this manner ye shall offer daily, throughout the seven days, the meat of the sacrifice made by fire, of a sweet savour unto the \LORD: it shall be offered beside the continual burnt offering, and his drink offering.
\verse And on the seventh day ye shall have an holy convocation; ye shall do no servile work.
\verseWithHeading{The Feast of Weeks} Also in the day of the firstfruits, when ye bring a new meat offering unto the \LORD, after your weeks be out, ye shall have an holy convocation; ye shall do no servile work:
\verse But ye shall offer the burnt offering for a sweet savour unto the \LORD; two young bullocks, one ram, seven lambs of the first year;
\verse And their meat offering of flour mingled with oil, three tenth deals unto one bullock, two tenth deals unto one ram,
\verse A several tenth deal unto one lamb, throughout the seven lambs;
\verse And one kid of the goats, to make an atonement for you.
\verse Ye shall offer them beside the continual burnt offering, and his meat offering, (they shall be unto you without blemish) and their drink offerings.
\end{biblechapter}

\begin{biblechapter} % Numbers 29
\verseWithHeading{The Feast of Trumpets} And in the seventh month, on the first day of the month, ye shall have an holy convocation; ye shall do no servile work: it is a day of blowing the trumpets unto you.
\verse And ye shall offer a burnt offering for a sweet savour unto the \LORD; one young bullock, one ram, and seven lambs of the first year without blemish:
\verse And their meat offering shall be of flour mingled with oil, three tenth deals for a bullock, and two tenth deals for a ram,
\verse And one tenth deal for one lamb, throughout the seven lambs:
\verse And one kid of the goats for a sin offering, to make an atonement for you:
\verse Beside the burnt offering of the month, and his meat offering, and the daily burnt offering, and his meat offering, and their drink offerings, according unto their manner, for a sweet savour, a sacrifice made by fire unto the \LORD.
\verseWithHeading{The Day of Atonement} And ye shall have on the tenth day of this seventh month an holy convocation; and ye shall afflict your souls: ye shall not do any work therein:
\verse But ye shall offer a burnt offering unto the \LORD for a sweet savour; one young bullock, one ram, and seven lambs of the first year; they shall be unto you without blemish:
\verse And their meat offering shall be of flour mingled with oil, three tenth deals to a bullock, and two tenth deals to one ram,
\verse A several tenth deal for one lamb, throughout the seven lambs:
\verse One kid of the goats for a sin offering; beside the sin offering of atonement, and the continual burnt offering, and the meat offering of it, and their drink offerings.
\verseWithHeading{The Feast of Tabernacles} And on the fifteenth day of the seventh month ye shall have an holy convocation; ye shall do no servile work, and ye shall keep a feast unto the \LORD seven days:
\verse And ye shall offer a burnt offering, a sacrifice made by fire, of a sweet savour unto the \LORD; thirteen young bullocks, two rams, and fourteen lambs of the first year; they shall be without blemish:
\verse And their meat offering shall be of flour mingled with oil, three tenth deals unto every bullock of the thirteen bullocks, two tenth deals to each ram of the two rams,
\verse And a several tenth deal to each lamb of the fourteen lambs:
\verse And one kid of the goats for a sin offering; beside the continual burnt offering, his meat offering, and his drink offering.
\verse And on the second day ye shall offer twelve young bullocks, two rams, fourteen lambs of the first year without spot:
\verse And their meat offering and their drink offerings for the bullocks, for the rams, and for the lambs, shall be according to their number, after the manner:
\verse And one kid of the goats for a sin offering; beside the continual burnt offering, and the meat offering thereof, and their drink offerings.
\verse And on the third day eleven bullocks, two rams, fourteen lambs of the first year without blemish;
\verse And their meat offering and their drink offerings for the bullocks, for the rams, and for the lambs, shall be according to their number, after the manner:
\verse And one goat for a sin offering; beside the continual burnt offering, and his meat offering, and his drink offering.
\verse And on the fourth day ten bullocks, two rams, and fourteen lambs of the first year without blemish:
\verse Their meat offering and their drink offerings for the bullocks, for the rams, and for the lambs, shall be according to their number, after the manner:
\verse And one kid of the goats for a sin offering; beside the continual burnt offering, his meat offering, and his drink offering.
\verse And on the fifth day nine bullocks, two rams, and fourteen lambs of the first year without spot:
\verse And their meat offering and their drink offerings for the bullocks, for the rams, and for the lambs, shall be according to their number, after the manner:
\verse And one goat for a sin offering; beside the continual burnt offering, and his meat offering, and his drink offering.
\verse And on the sixth day eight bullocks, two rams, and fourteen lambs of the first year without blemish:
\verse And their meat offering and their drink offerings for the bullocks, for the rams, and for the lambs, shall be according to their number, after the manner:
\verse And one goat for a sin offering; beside the continual burnt offering, his meat offering, and his drink offering.
\verse And on the seventh day seven bullocks, two rams, and fourteen lambs of the first year without blemish:
\verse And their meat offering and their drink offerings for the bullocks, for the rams, and for the lambs, shall be according to their number, after the manner:
\verse And one goat for a sin offering; beside the continual burnt offering, his meat offering, and his drink offering.
\verse On the eighth day ye shall have a solemn assembly: ye shall do no servile work therein:
\verse But ye shall offer a burnt offering, a sacrifice made by fire, of a sweet savour unto the \LORD: one bullock, one ram, seven lambs of the first year without blemish:
\verse Their meat offering and their drink offerings for the bullock, for the ram, and for the lambs, shall be according to their number, after the manner:
\verse And one goat for a sin offering; beside the continual burnt offering, and his meat offering, and his drink offering.
\verse These things ye shall do unto the \LORD in your set feasts, beside your vows, and your freewill offerings, for your burnt offerings, and for your meat offerings, and for your drink offerings, and for your peace offerings.
\verse And Moses told the children of Israel according to all that the \LORD commanded Moses.
\end{biblechapter}

\begin{biblechapter} % Numbers 30
\verseWithHeading{Vows} And Moses spake unto the heads of the tribes concerning the children of Israel, saying, This is the thing which the \LORD hath commanded.
\verse If a man vow a vow unto the \LORD, or swear an oath to bind his soul with a bond; he shall not break his word, he shall do according to all that proceedeth out of his mouth.
\verse If a woman also vow a vow unto the \LORD, and bind herself by a bond, being in her father's house in her youth;
\verse And her father hear her vow, and her bond wherewith she hath bound her soul, and her father shall hold his peace at her: then all her vows shall stand, and every bond wherewith she hath bound her soul shall stand.
\verse But if her father disallow her in the day that he heareth; not any of her vows, or of her bonds wherewith she hath bound her soul, shall stand: and the \LORD shall forgive her, because her father disallowed her.
\verse And if she had at all an husband, when she vowed, or uttered ought out of her lips, wherewith she bound her soul;
\verse And her husband heard it, and held his peace at her in the day that he heard it: then her vows shall stand, and her bonds wherewith she bound her soul shall stand.
\verse But if her husband disallowed her on the day that he heard it; then he shall make her vow which she vowed, and that which she uttered with her lips, wherewith she bound her soul, of none effect: and the \LORD shall forgive her.
\verse But every vow of a widow, and of her that is divorced, wherewith they have bound their souls, shall stand against her.
\verse And if she vowed in her husband's house, or bound her soul by a bond with an oath;
\verse And her husband heard it, and held his peace at her, and disallowed her not: then all her vows shall stand, and every bond wherewith she bound her soul shall stand.
\verse But if her husband hath utterly made them void on the day he heard them; then whatsoever proceeded out of her lips concerning her vows, or concerning the bond of her soul, shall not stand: her husband hath made them void; and the \LORD shall forgive her.
\verse Every vow, and every binding oath to afflict the soul, her husband may establish it, or her husband may make it void.
\verse But if her husband altogether hold his peace at her from day to day; then he establisheth all her vows, or all her bonds, which are upon her: he confirmeth them, because he held his peace at her in the day that he heard them.
\verse But if he shall any ways make them void after that he hath heard them; then he shall bear her iniquity.
\verse These are the statutes, which the \LORD commanded Moses, between a man and his wife, between the father and his daughter, being yet in her youth in her father's house.
\end{biblechapter}

\begin{biblechapter} % Numbers 31
\verseWithHeading{Vengeance on the Midianites} And the \LORD spake unto Moses, saying,
\verse Avenge the children of Israel of the Midianites: afterward shalt thou be gathered unto thy people.
\verse And Moses spake unto the people, saying, Arm some of yourselves unto the war, and let them go against the Midianites, and avenge the \LORD of Midian.
\verse Of every tribe a thousand, throughout all the tribes of Israel, shall ye send to the war.
\verse So there were delivered out of the thousands of Israel, a thousand of every tribe, twelve thousand armed for war.
\verse And Moses sent them to the war, a thousand of every tribe, them and Phinehas the son of Eleazar the priest, to the war, with the holy instruments, and the trumpets to blow in his hand.
\verse And they warred against the Midianites, as the \LORD commanded Moses; and they slew all the males.
\verse And they slew the kings of Midian, beside the rest of them that were slain; namely, Evi, and Rekem, and Zur, and Hur, and Reba, five kings of Midian: Balaam also the son of Beor they slew with the sword.
\verse And the children of Israel took all the women of Midian captives, and their little ones, and took the spoil of all their cattle, and all their flocks, and all their goods.
\verse And they burnt all their cities wherein they dwelt, and all their goodly castles, with fire.
\verse And they took all the spoil, and all the prey, both of men and of beasts.
\verse And they brought the captives, and the prey, and the spoil, unto Moses, and Eleazar the priest, and unto the congregation of the children of Israel, unto the camp at the plains of Moab, which are by Jordan near Jericho.
\verse And Moses, and Eleazar the priest, and all the princes of the congregation, went forth to meet them without the camp.
\verse And Moses was wroth with the officers of the host, with the captains over thousands, and captains over hundreds, which came from the battle.
\verse And Moses said unto them, Have ye saved all the women alive?
\verse Behold, these caused the children of Israel, through the counsel of Balaam, to commit trespass against the \LORD in the matter of Peor, and there was a plague among the congregation of the \LORD.
\verse Now therefore kill every male among the little ones, and kill every woman that hath known man by lying with him.
\verse But all the women children, that have not known a man by lying with him, keep alive for yourselves.
\verse And do ye abide without the camp seven days: whosoever hath killed any person, and whosoever hath touched any slain, purify both yourselves and your captives on the third day, and on the seventh day.
\verse And purify all your raiment, and all that is made of skins, and all work of goats' hair, and all things made of wood.
\verse And Eleazar the priest said unto the men of war which went to the battle, This is the ordinance of the law which the \LORD commanded Moses;
\verse Only the gold, and the silver, the brass, the iron, the tin, and the lead,
\verse Every thing that may abide the fire, ye shall make it go through the fire, and it shall be clean: nevertheless it shall be purified with the water of separation: and all that abideth not the fire ye shall make go through the water.
\verse And ye shall wash your clothes on the seventh day, and ye shall be clean, and afterward ye shall come into the camp.
\flushcolsend\columnbreak % layout hack
\verseWithHeading{Dividing the spoils} And the \LORD spake unto Moses, saying,
\verse Take the sum of the prey that was taken, both of man and of beast, thou, and Eleazar the priest, and the chief fathers of the congregation:
\verse And divide the prey into two parts; between them that took the war upon them, who went out to battle, and between all the congregation:
\verse And levy a tribute unto the \LORD of the men of war which went out to battle: one soul of five hundred, both of the persons, and of the beeves, and of the asses, and of the sheep:
\verse Take it of their half, and give it unto Eleazar the priest, for an heave offering of the \LORD.
\verse And of the children of Israel's half, thou shalt take one portion of fifty, of the persons, of the beeves, of the asses, and of the flocks, of all manner of beasts, and give them unto the Levites, which keep the charge of the tabernacle of the \LORD.
\verse And Moses and Eleazar the priest did as the \LORD commanded Moses.
\verse And the booty, being the rest of the prey which the men of war had caught, was six hundred thousand and seventy thousand and five thousand sheep,
\verse And threescore and twelve thousand beeves,
\verse And threescore and one thousand asses,
\verse And thirty and two thousand persons in all, of women that had not known man by lying with him.
\verse And the half, which was the portion of them that went out to war, was in number three hundred thousand and seven and thirty thousand and five hundred sheep:
\verse And the \LORDs tribute of the sheep was six hundred and threescore and fifteen.
\verse And the beeves were thirty and six thousand; of which the \LORDs tribute was threescore and twelve.
\verse And the asses were thirty thousand and five hundred; of which the \LORDs tribute was threescore and one.
\verse And the persons were sixteen thousand; of which the \LORDs tribute was thirty and two persons.
\verse And Moses gave the tribute, which was the \LORDs heave offering, unto Eleazar the priest, as the \LORD commanded Moses.
\verse And of the children of Israel's half, which Moses divided from the men that warred,
\verse (Now the half that pertained unto the congregation was three hundred thousand and thirty thousand and seven thousand and five hundred sheep,
\verse And thirty and six thousand beeves,
\verse And thirty thousand asses and five hundred,
\verse And sixteen thousand persons;)
\verse Even of the children of Israel's half, Moses took one portion of fifty, both of man and of beast, and gave them unto the Levites, which kept the charge of the tabernacle of the \LORD; as the \LORD commanded Moses.
\verse And the officers which were over thousands of the host, the captains of thousands, and captains of hundreds, came near unto Moses:
\verse And they said unto Moses, Thy servants have taken the sum of the men of war which are under our charge, and there lacketh not one man of us.
\verse We have therefore brought an oblation for the \LORD, what every man hath gotten, of jewels of gold, chains, and bracelets, rings, earrings, and tablets, to make an atonement for our souls before the \LORD.
\verse And Moses and Eleazar the priest took the gold of them, even all wrought jewels.
\verse And all the gold of the offering that they offered up to the \LORD, of the captains of thousands, and of the captains of hundreds, was sixteen thousand seven hundred and fifty shekels.
\verse (For the men of war had taken spoil, every man for himself.)
\verse And Moses and Eleazar the priest took the gold of the captains of thousands and of hundreds, and brought it into the tabernacle of the congregation, for a memorial for the children of Israel before the \LORD.
\end{biblechapter}

\begin{biblechapter} % Numbers 32
\verseWithHeading{The Transjordan tribes} Now the children of Reuben and the children of Gad had a very great multitude of cattle: and when they saw the land of Jazer, and the land of Gilead, that, behold, the place was a place for cattle;
\verse The children of Gad and the children of Reuben came and spake unto Moses, and to Eleazar the priest, and unto the princes of the congregation, saying,
\verse Ataroth, and Dibon, and Jazer, and Nimrah, and Heshbon, and Elealeh, and Shebam, and Nebo, and Beon,
\verse Even the country which the \LORD smote before the congregation of Israel, is a land for cattle, and thy servants have cattle:
\verse Wherefore, said they, if we have found grace in thy sight, let this land be given unto thy servants for a possession, and bring us not over Jordan.
\verse And Moses said unto the children of Gad and to the children of Reuben, Shall your brethren go to war, and shall ye sit here?
\verse And wherefore discourage ye the heart of the children of Israel from going over into the land which the \LORD hath given them?
\verse Thus did your fathers, when I sent them from Kadeshbarnea to see the land.
\verse For when they went up unto the valley of Eshcol, and saw the land, they discouraged the heart of the children of Israel, that they should not go into the land which the \LORD had given them.
\verse And the \LORDs anger was kindled the same time, and he sware, saying,
\verse Surely none of the men that came up out of Egypt, from twenty years old and upward, shall see the land which I sware unto Abraham, unto Isaac, and unto Jacob; because they have not wholly followed me:
\verse Save Caleb the son of Jephunneh the Kenezite, and Joshua the son of Nun: for they have wholly followed the \LORD.
\verse And the \LORDs anger was kindled against Israel, and he made them wander in the wilderness forty years, until all the generation, that had done evil in the sight of the \LORD, was consumed.
\verse And, behold, ye are risen up in your fathers' stead, an increase of sinful men, to augment yet the fierce anger of the \LORD toward Israel.
\verse For if ye turn away from after him, he will yet again leave them in the wilderness; and ye shall destroy all this people.
\verse And they came near unto him, and said, We will build sheepfolds here for our cattle, and cities for our little ones:
\verse But we ourselves will go ready armed before the children of Israel, until we have brought them unto their place: and our little ones shall dwell in the fenced cities because of the inhabitants of the land.
\verse We will not return unto our houses, until the children of Israel have inherited every man his inheritance.
\verse For we will not inherit with them on yonder side Jordan, or forward; because our inheritance is fallen to us on this side Jordan eastward.
\verse And Moses said unto them, If ye will do this thing, if ye will go armed before the \LORD to war,
\verse And will go all of you armed over Jordan before the \LORD, until he hath driven out his enemies from before him,
\verse And the land be subdued before the \LORD: then afterward ye shall return, and be guiltless before the \LORD, and before Israel; and this land shall be your possession before the \LORD.
\verse But if ye will not do so, behold, ye have sinned against the \LORD: and be sure your sin will find you out.
\verse Build you cities for your little ones, and folds for your sheep; and do that which hath proceeded out of your mouth.
\verse And the children of Gad and the children of Reuben spake unto Moses, saying, Thy servants will do as my lord commandeth.
\verse Our little ones, our wives, our flocks, and all our cattle, shall be there in the cities of Gilead:
\verse But thy servants will pass over, every man armed for war, before the \LORD to battle, as my lord saith.
\verse So concerning them Moses commanded Eleazar the priest, and Joshua the son of Nun, and the chief fathers of the tribes of the children of Israel:
\verse And Moses said unto them, If the children of Gad and the children of Reuben will pass with you over Jordan, every man armed to battle, before the \LORD, and the land shall be subdued before you; then ye shall give them the land of Gilead for a possession:
\verse But if they will not pass over with you armed, they shall have possessions among you in the land of Canaan.
\verse And the children of Gad and the children of Reuben answered, saying, As the \LORD hath said unto thy servants, so will we do.
\verse We will pass over armed before the \LORD into the land of Canaan, that the possession of our inheritance on this side Jordan may be ours.
\verse And Moses gave unto them, even to the children of Gad, and to the children of Reuben, and unto half the tribe of Manasseh the son of Joseph, the kingdom of Sihon king of the Amorites, and the kingdom of Og king of Bashan, the land, with the cities thereof in the coasts, even the cities of the country round about.
\verse And the children of Gad built Dibon, and Ataroth, and Aroer,
\verse And Atroth, Shophan, and Jaazer, and Jogbehah,
\verse And Bethnimrah, and Bethharan, fenced cities: and folds for sheep.
\verse And the children of Reuben built Heshbon, and Elealeh, and Kirjathaim,
\verse And Nebo, and Baalmeon, (their names being changed,) and Shibmah: and gave other names unto the cities which they builded.
\verse And the children of Machir the son of Manasseh went to Gilead, and took it, and dispossessed the Amorite which was in it.
\verse And Moses gave Gilead unto Machir the son of Manasseh; and he dwelt therein.
\verse And Jair the son of Manasseh went and took the small towns thereof, and called them Havothjair.
\verse And Nobah went and took Kenath, and the villages thereof, and called it Nobah, after his own name.
\end{biblechapter}

\begin{biblechapter} % Numbers 33
\verseWithHeading{Stages on Israel's journey} These are the journeys of the children of Israel, which went forth out of the land of Egypt with their armies under the hand of Moses and Aaron.
\verse And Moses wrote their goings out according to their journeys by the commandment of the \LORD: and these are their journeys according to their goings out.
\verse And they departed from Rameses in the first month, on the fifteenth day of the first month; on the morrow after the Passover the children of Israel went out with an high hand in the sight of all the Egyptians.
\verse For the Egyptians buried all their firstborn, which the \LORD had smitten among them: upon their gods also the \LORD executed judgments.
\verse And the children of Israel removed from Rameses, and pitched in Succoth.
\verse And they departed from Succoth, and pitched in Etham, which is in the edge of the wilderness.
\verse And they removed from Etham, and turned again unto Pihahiroth, which is before Baalzephon: and they pitched before Migdol.
\verse And they departed from before Pihahiroth, and passed through the midst of the sea into the wilderness, and went three days' journey in the wilderness of Etham, and pitched in Marah.
\verse And they removed from Marah, and came unto Elim: and in Elim were twelve fountains of water, and threescore and ten palm trees; and they pitched there.
\verse And they removed from Elim, and encamped by the Red sea.
\verse And they removed from the Red sea, and encamped in the wilderness of Sin.
\verse And they took their journey out of the wilderness of Sin, and encamped in Dophkah.
\verse And they departed from Dophkah, and encamped in Alush.
\verse And they removed from Alush, and encamped at Rephidim, where was no water for the people to drink.
\verse And they departed from Rephidim, and pitched in the wilderness of Sinai.
\verse And they removed from the desert of Sinai, and pitched at Kibrothhattaavah.
\verse And they departed from Kibrothhattaavah, and encamped at Hazeroth.
\verse And they departed from Hazeroth, and pitched in Rithmah.
\verse And they departed from Rithmah, and pitched at Rimmonparez.
\verse And they departed from Rimmonparez, and pitched in Libnah.
\verse And they removed from Libnah, and pitched at Rissah.
\verse And they journeyed from Rissah, and pitched in Kehelathah.
\verse And they went from Kehelathah, and pitched in mount Shapher.
\verse And they removed from mount Shapher, and encamped in Haradah.
\verse And they removed from Haradah, and pitched in Makheloth.
\verse And they removed from Makheloth, and encamped at Tahath.
\verse And they departed from Tahath, and pitched at Tarah.
\verse And they removed from Tarah, and pitched in Mithcah.
\verse And they went from Mithcah, and pitched in Hashmonah.
\verse And they departed from Hashmonah, and encamped at Moseroth.
\verse And they departed from Moseroth, and pitched in Benejaakan.
\verse And they removed from Benejaakan, and encamped at Horhagidgad.
\verse And they went from Horhagidgad, and pitched in Jotbathah.
\verse And they removed from Jotbathah, and encamped at Ebronah.
\verse And they departed from Ebronah, and encamped at Eziongaber.
\verse And they removed from Eziongaber, and pitched in the wilderness of Zin, which is Kadesh.
\verse And they removed from Kadesh, and pitched in mount Hor, in the edge of the land of Edom.
\verse And Aaron the priest went up into mount Hor at the commandment of the \LORD, and died there, in the fortieth year after the children of Israel were come out of the land of Egypt, in the first day of the fifth month.
\verse And Aaron was an hundred and twenty and three years old when he died in mount Hor.
\verse And king Arad the Canaanite, which dwelt in the south in the land of Canaan, heard of the coming of the children of Israel.
\verse And they departed from mount Hor, and pitched in Zalmonah.
\verse And they departed from Zalmonah, and pitched in Punon.
\verse And they departed from Punon, and pitched in Oboth.
\verse And they departed from Oboth, and pitched in Ijeabarim, in the border of Moab.
\verse And they departed from Iim, and pitched in Dibongad.
\verse And they removed from Dibongad, and encamped in Almondiblathaim.
\verse And they removed from Almondiblathaim, and pitched in the mountains of Abarim, before Nebo.
\verse And they departed from the mountains of Abarim, and pitched in the plains of Moab by Jordan near Jericho.
\verse And they pitched by Jordan, from Bethjesimoth even unto Abelshittim in the plains of Moab.
\verse And the \LORD spake unto Moses in the plains of Moab by Jordan near Jericho, saying,
\verse Speak unto the children of Israel, and say unto them, When ye are passed over Jordan into the land of Canaan;
\verse Then ye shall drive out all the inhabitants of the land from before you, and destroy all their pictures, and destroy all their molten images, and quite pluck down all their high places:
\verse And ye shall dispossess the inhabitants of the land, and dwell therein: for I have given you the land to possess it.
\verse And ye shall divide the land by lot for an inheritance among your families: and to the more ye shall give the more inheritance, and to the fewer ye shall give the less inheritance: every man's inheritance shall be in the place where his lot falleth; according to the tribes of your fathers ye shall inherit.
\verse But if ye will not drive out the inhabitants of the land from before you; then it shall come to pass, that those which ye let remain of them shall be pricks in your eyes, and thorns in your sides, and shall vex you in the land wherein ye dwell.
\verse Moreover it shall come to pass, that I shall do unto you, as I thought to do unto them.
\end{biblechapter}

\begin{biblechapter} % Numbers 34
\verseWithHeading{Boundaries of Canaan} And the \LORD spake unto Moses, saying,
\verse Command the children of Israel, and say unto them, When ye come into the land of Canaan; (this is the land that shall fall unto you for an inheritance, even the land of Canaan with the coasts thereof:)
\verse Then your south quarter shall be from the wilderness of Zin along by the coast of Edom, and your south border shall be the outmost coast of the salt sea eastward:
\verse And your border shall turn from the south to the ascent of Akrabbim, and pass on to Zin: and the going forth thereof shall be from the south to Kadeshbarnea, and shall go on to Hazaraddar, and pass on to Azmon:
\verse And the border shall fetch a compass from Azmon unto the river of Egypt, and the goings out of it shall be at the sea.
\verse And as for the western border, ye shall even have the great sea for a border: this shall be your west border.
\verse And this shall be your north border: from the great sea ye shall point out for you mount Hor:
\verse From mount Hor ye shall point out your border unto the entrance of Hamath; and the goings forth of the border shall be to Zedad:
\verse And the border shall go on to Ziphron, and the goings out of it shall be at Hazarenan: this shall be your north border.
\verse And ye shall point out your east border from Hazarenan to Shepham:
\verse And the coast shall go down from Shepham to Riblah, on the east side of Ain; and the border shall descend, and shall reach unto the side of the sea of Chinnereth eastward:
\verse And the border shall go down to Jordan, and the goings out of it shall be at the salt sea: this shall be your land with the coasts thereof round about.
\verse And Moses commanded the children of Israel, saying, This is the land which ye shall inherit by lot, which the \LORD commanded to give unto the nine tribes, and to the half tribe:
\verse For the tribe of the children of Reuben according to the house of their fathers, and the tribe of the children of Gad according to the house of their fathers, have received their inheritance; and half the tribe of Manasseh have received their inheritance:
\verse The two tribes and the half tribe have received their inheritance on this side Jordan near Jericho eastward, toward the sunrising.
\verse And the \LORD spake unto Moses, saying,
\verse These are the names of the men which shall divide the land unto you: Eleazar the priest, and Joshua the son of Nun.
\verse And ye shall take one prince of every tribe, to divide the land by inheritance.
\verse And the names of the men are these: Of the tribe of Judah, Caleb the son of Jephunneh.
\verse And of the tribe of the children of Simeon, Shemuel the son of Ammihud.
\verse Of the tribe of Benjamin, Elidad the son of Chislon.
\verse And the prince of the tribe of the children of Dan, Bukki the son of Jogli.
\verse The prince of the children of Joseph, for the tribe of the children of Manasseh, Hanniel the son of Ephod.
\verse And the prince of the tribe of the children of Ephraim, Kemuel the son of Shiphtan.
\verse And the prince of the tribe of the children of Zebulun, Elizaphan the son of Parnach.
\verse And the prince of the tribe of the children of Issachar, Paltiel the son of Azzan.
\verse And the prince of the tribe of the children of Asher, Ahihud the son of Shelomi.
\verse And the prince of the tribe of the children of Naphtali, Pedahel the son of Ammihud.
\verse These are they whom the \LORD commanded to divide the inheritance unto the children of Israel in the land of Canaan.
\end{biblechapter}

\begin{biblechapter} % Numbers 35
\verseWithHeading{Towns for the Levites} And the \LORD spake unto Moses in the plains of Moab by Jordan near Jericho, saying,
\verse Command the children of Israel, that they give unto the Levites of the inheritance of their possession cities to dwell in; and ye shall give also unto the Levites suburbs for the cities round about them.
\verse And the cities shall they have to dwell in; and the suburbs of them shall be for their cattle, and for their goods, and for all their beasts.
\verse And the suburbs of the cities, which ye shall give unto the Levites, shall reach from the wall of the city and outward a thousand cubits round about.
\verse And ye shall measure from without the city on the east side two thousand cubits, and on the south side two thousand cubits, and on the west side two thousand cubits, and on the north side two thousand cubits; and the city shall be in the midst: this shall be to them the suburbs of the cities.
\verseWithHeading{Cities of refuge} And among the cities which ye shall give unto the Levites there shall be six cities for refuge, which ye shall appoint for the manslayer, that he may flee thither: and to them ye shall add forty and two cities.
\verse So all the cities which ye shall give to the Levites shall be forty and eight cities: them shall ye give with their suburbs.
\verse And the cities which ye shall give shall be of the possession of the children of Israel: from them that have many ye shall give many; but from them that have few ye shall give few: every one shall give of his cities unto the Levites according to his inheritance which he inheriteth.
\verse And the \LORD spake unto Moses, saying,
\verse Speak unto the children of Israel, and say unto them, When ye be come over Jordan into the land of Canaan;
\verse Then ye shall appoint you cities to be cities of refuge for you; that the slayer may flee thither, which killeth any person at unawares.
\verse And they shall be unto you cities for refuge from the avenger; that the manslayer die not, until he stand before the congregation in judgment.
\verse And of these cities which ye shall give six cities shall ye have for refuge.
\verse Ye shall give three cities on this side Jordan, and three cities shall ye give in the land of Canaan, which shall be cities of refuge.
\verse These six cities shall be a refuge, both for the children of Israel, and for the stranger, and for the sojourner among them: that every one that killeth any person unawares may flee thither.
\verse And if he smite him with an instrument of iron, so that he die, he is a murderer: the murderer shall surely be put to death.
\verse And if he smite him with throwing a stone, wherewith he may die, and he die, he is a murderer: the murderer shall surely be put to death.
\verse Or if he smite him with an hand weapon of wood, wherewith he may die, and he die, he is a murderer: the murderer shall surely be put to death.
\verse The revenger of blood himself shall slay the murderer: when he meeteth him, he shall slay him.
\verse But if he thrust him of hatred, or hurl at him by laying of wait, that he die;
\verse Or in enmity smite him with his hand, that he die: he that smote him shall surely be put to death; for he is a murderer: the revenger of blood shall slay the murderer, when he meeteth him.
\verse But if he thrust him suddenly without enmity, or have cast upon him any thing without laying of wait,
\verse Or with any stone, wherewith a man may die, seeing him not, and cast it upon him, that he die, and was not his enemy, neither sought his harm:
\verse Then the congregation shall judge between the slayer and the revenger of blood according to these judgments:
\verse And the congregation shall deliver the slayer out of the hand of the revenger of blood, and the congregation shall restore him to the city of his refuge, whither he was fled: and he shall abide in it unto the death of the high priest, which was anointed with the holy oil.
\verse But if the slayer shall at any time come without the border of the city of his refuge, whither he was fled;
\verse And the revenger of blood find him without the borders of the city of his refuge, and the revenger of blood kill the slayer; he shall not be guilty of blood:
\verse Because he should have remained in the city of his refuge until the death of the high priest: but after the death of the high priest the slayer shall return into the land of his possession.
\verse So these things shall be for a statute of judgment unto you throughout your generations in all your dwellings.
\verse Whoso killeth any person, the murderer shall be put to death by the mouth of witnesses: but one witness shall not testify against any person to cause him to die.
\verse Moreover ye shall take no satisfaction for the life of a murderer, which is guilty of death: but he shall be surely put to death.
\verse And ye shall take no satisfaction for him that is fled to the city of his refuge, that he should come again to dwell in the land, until the death of the priest.
\verse So ye shall not pollute the land wherein ye are: for blood it defileth the land: and the land cannot be cleansed of the blood that is shed therein, but by the blood of him that shed it.
\verse Defile not therefore the land which ye shall inhabit, wherein I dwell: for I the \LORD dwell among the children of Israel.
\end{biblechapter}

\begin{biblechapter} % Numbers 36
\verseWithHeading{Inheritance of Zelophehad's \newline daughters} And the chief fathers of the families of the children of Gilead, the son of Machir, the son of Manasseh, of the families of the sons of Joseph, came near, and spake before Moses, and before the princes, the chief fathers of the children of Israel:
\verse And they said, The \LORD commanded my lord to give the land for an inheritance by lot to the children of Israel: and my lord was commanded by the \LORD to give the inheritance of Zelophehad our brother unto his daughters.
\verse And if they be married to any of the sons of the other tribes of the children of Israel, then shall their inheritance be taken from the inheritance of our fathers, and shall be put to the inheritance of the tribe whereunto they are received: so shall it be taken from the lot of our inheritance.
\verse And when the jubile of the children of Israel shall be, then shall their inheritance be put unto the inheritance of the tribe whereunto they are received: so shall their inheritance be taken away from the inheritance of the tribe of our fathers.
\verse And Moses commanded the children of Israel according to the word of the \LORD, saying, The tribe of the sons of Joseph hath said well.
\verse This is the thing which the \LORD doth command concerning the daughters of Zelophehad, saying, Let them marry to whom they think best; only to the family of the tribe of their father shall they marry.
\verse So shall not the inheritance of the children of Israel remove from tribe to tribe: for every one of the children of Israel shall keep himself to the inheritance of the tribe of his fathers.
\verse And every daughter, that possesseth an inheritance in any tribe of the children of Israel, shall be wife unto one of the family of the tribe of her father, that the children of Israel may enjoy every man the inheritance of his fathers.
\verse Neither shall the inheritance remove from one tribe to another tribe; but every one of the tribes of the children of Israel shall keep himself to his own inheritance.
\verse Even as the \LORD commanded Moses, so did the daughters of Zelophehad:
\verse For Mahlah, Tirzah, and Hoglah, and Milcah, and Noah, the daughters of Zelophehad, were married unto their father's brothers' sons:
\verse And they were married into the families of the sons of Manasseh the son of Joseph, and their inheritance remained in the tribe of the family of their father.
\verse These are the commandments and the judgments, which the \LORD commanded by the hand of Moses unto the children of Israel in the plains of Moab by Jordan near Jericho.
\end{biblechapter}
\flushcolsend
\biblebook{Deuteronomy}

\begin{biblechapter} % Deuteronomy 1
\verseWithHeading{The command to leave Horeb} These be the words which Moses spake unto all Israel on this side Jordan in the wilderness, in the plain over against the Red sea, between Paran, and Tophel, and Laban, and Hazeroth, and Dizahab.
\verse (There are eleven days' journey from Horeb by the way of mount Seir unto Kadeshbarnea.)
\verse And it came to pass in the fortieth year, in the eleventh month, on the first day of the month, that Moses spake unto the children of Israel, according unto all that the \LORD had given him in commandment unto them;
\verse After he had slain Sihon the king of the Amorites, which dwelt in Heshbon, and Og the king of Bashan, which dwelt at Astaroth in Edrei:
\verse On this side Jordan, in the land of Moab, began Moses to declare this law, saying,
\verse The \LORD our God spake unto us in Horeb, saying, Ye have dwelt long enough in this mount:
\verse Turn you, and take your journey, and go to the mount of the Amorites, and unto all the places nigh thereunto, in the plain, in the hills, and in the vale, and in the south, and by the sea side, to the land of the Canaanites, and unto Lebanon, unto the great river, the river Euphrates.
\verse Behold, I have set the land before you: go in and possess the land which the \LORD sware unto your fathers, Abraham, Isaac, and Jacob, to give unto them and to their seed after them.
\verseWithHeading{The appointment of leaders} And I spake unto you at that time, saying, I am not able to bear you myself alone:
\verse The \LORD your God hath multiplied you, and, behold, ye are this day as the stars of heaven for multitude.
\verse (The \LORD God of your fathers make you a thousand times so many more as ye are, and bless you, as he hath promised you!)
\verse How can I myself alone bear your cumbrance, and your burden, and your strife?
\verse Take you wise men, and understanding, and known among your tribes, and I will make them rulers over you.
\verse And ye answered me, and said, The thing which thou hast spoken is good for us to do.
\verse So I took the chief of your tribes, wise men, and known, and made them heads over you, captains over thousands, and captains over hundreds, and captains over fifties, and captains over tens, and officers among your tribes.
\verse And I charged your judges at that time, saying, Hear the causes between your brethren, and judge righteously between every man and his brother, and the stranger that is with him.
\verse Ye shall not respect persons in judgment; but ye shall hear the small as well as the great; ye shall not be afraid of the face of man; for the judgment is God's: and the cause that is too hard for you, bring it unto me, and I will hear it.
\verse And I commanded you at that time all the things which ye should do.
\verseWithHeading{Spies sent out} And when we departed from Horeb, we went through all that great and terrible wilderness, which ye saw by the way of the mountain of the Amorites, as the \LORD our God commanded us; and we came to Kadeshbarnea.
\verse And I said unto you, Ye are come unto the mountain of the Amorites, which the \LORD our God doth give unto us.
\verse Behold, the \LORD thy God hath set the land before thee: go up and possess it, as the \LORD God of thy fathers hath said unto thee; fear not, neither be discouraged.
\verse And ye came near unto me every one of you, and said, We will send men before us, and they shall search us out the land, and bring us word again by what way we must go up, and into what cities we shall come.
\verse And the saying pleased me well: and I took twelve men of you, one of a tribe:
\verse And they turned and went up into the mountain, and came unto the valley of Eshcol, and searched it out.
\verse And they took of the fruit of the land in their hands, and brought it down unto us, and brought us word again, and said, It is a good land which the \LORD our God doth give us.
\verseWithHeading{Rebellion against the \LORD} Notwithstanding ye would not go up, but rebelled against the commandment of the \LORD your God:
\verse And ye murmured in your tents, and said, Because the \LORD hated us, he hath brought us forth out of the land of Egypt, to deliver us into the hand of the Amorites, to destroy us.
\verse Whither shall we go up? our brethren have discouraged our heart, saying, The people is greater and taller than we; the cities are great and walled up to heaven; and moreover we have seen the sons of the Anakims there.
\verse Then I said unto you, Dread not, neither be afraid of them.
\verse The \LORD your God which goeth before you, he shall fight for you, according to all that he did for you in Egypt before your eyes;
\verse And in the wilderness, where thou hast seen how that the \LORD thy God bare thee, as a man doth bear his son, in all the way that ye went, until ye came into this place.
\verse Yet in this thing ye did not believe the \LORD your God,
\verse Who went in the way before you, to search you out a place to pitch your tents in, in fire by night, to shew you by what way ye should go, and in a cloud by day.
\verse And the \LORD heard the voice of your words, and was wroth, and sware, saying,
\verse Surely there shall not one of these men of this evil generation see that good land, which I sware to give unto your fathers,
\verse Save Caleb the son of Jephunneh; he shall see it, and to him will I give the land that he hath trodden upon, and to his children, because he hath wholly followed the \LORD.
\verse Also the \LORD was angry with me for your sakes, saying, Thou also shalt not go in thither.
\verse But Joshua the son of Nun, which standeth before thee, he shall go in thither: encourage him: for he shall cause Israel to inherit it.
\verse Moreover your little ones, which ye said should be a prey, and your children, which in that day had no knowledge between good and evil, they shall go in thither, and unto them will I give it, and they shall possess it.
\verse But as for you, turn you, and take your journey into the wilderness by the way of the Red sea.
\verse Then ye answered and said unto me, We have sinned against the \LORD, we will go up and fight, according to all that the \LORD our God commanded us. And when ye had girded on every man his weapons of war, ye were ready to go up into the hill.
\verse And the \LORD said unto me, Say unto them, Go not up, neither fight; for I am not among you; lest ye be smitten before your enemies.
\verse So I spake unto you; and ye would not hear, but rebelled against the commandment of the \LORD, and went presumptuously up into the hill.
\verse And the Amorites, which dwelt in that mountain, came out against you, and chased you, as bees do, and destroyed you in Seir, even unto Hormah.
\verse And ye returned and wept before the \LORD; but the \LORD would not hearken to your voice, nor give ear unto you.
\verse So ye abode in Kadesh many days, according unto the days that ye abode there.
\end{biblechapter}

\begin{biblechapter} % Deuteronomy 2
\verseWithHeading{Wandering in the wilderness} Then we turned, and took our journey into the wilderness by the way of the Red sea, as the \LORD spake unto me: and we compassed mount Seir many days.
\verse And the \LORD spake unto me, saying,
\verse Ye have compassed this mountain long enough: turn you northward.
\verse And command thou the people, saying, Ye are to pass through the coast of your brethren the children of Esau, which dwell in Seir; and they shall be afraid of you: take ye good heed unto yourselves therefore:
\verse Meddle not with them; for I will not give you of their land, no, not so much as a foot breadth; because I have given mount Seir unto Esau for a possession.
\verse Ye shall buy meat of them for money, that ye may eat; and ye shall also buy water of them for money, that ye may drink.
\verse For the \LORD thy God hath blessed thee in all the works of thy hand: he knoweth thy walking through this great wilderness: these forty years the \LORD thy God hath been with thee; thou hast lacked nothing.
\verse And when we passed by from our brethren the children of Esau, which dwelt in Seir, through the way of the plain from Elath, and from Eziongaber, we turned and passed by the way of the wilderness of Moab.
\verse And the \LORD said unto me, Distress not the Moabites, neither contend with them in battle: for I will not give thee of their land for a possession; because I have given Ar unto the children of Lot for a possession.
\verse The Emims dwelt therein in times past, a people great, and many, and tall, as the Anakims;
\verse Which also were accounted giants, as the Anakims; but the Moabites call them Emims.
\verse The Horims also dwelt in Seir beforetime; but the children of Esau succeeded them, when they had destroyed them from before them, and dwelt in their stead; as Israel did unto the land of his possession, which the \LORD gave unto them.
\verse Now rise up, said I, and get you over the brook Zered. And we went over the brook Zered.
\verse And the space in which we came from Kadeshbarnea, until we were come over the brook Zered, was thirty and eight years; until all the generation of the men of war were wasted out from among the host, as the \LORD sware unto them.
\verse For indeed the hand of the \LORD was against them, to destroy them from among the host, until they were consumed.
\verse So it came to pass, when all the men of war were consumed and dead from among the people,
\verse That the \LORD spake unto me, saying,
\verse Thou art to pass over through Ar, the coast of Moab, this day:
\verse And when thou comest nigh over against the children of Ammon, distress them not, nor meddle with them: for I will not give thee of the land of the children of Ammon any possession; because I have given it unto the children of Lot for a possession.
\verse (That also was accounted a land of giants: giants dwelt therein in old time; and the Ammonites call them Zamzummims;
\verse A people great, and many, and tall, as the Anakims; but the \LORD destroyed them before them; and they succeeded them, and dwelt in their stead:
\verse As he did to the children of Esau, which dwelt in Seir, when he destroyed the Horims from before them; and they succeeded them, and dwelt in their stead even unto this day:
\verse And the Avims which dwelt in Hazerim, even unto Azzah, the Caphtorims, which came forth out of Caphtor, destroyed them, and dwelt in their stead.)
\verseWithHeading{Defeat Sihon king of Heshbon} Rise ye up, take your journey, and pass over the river Arnon: behold, I have given into thine hand Sihon the Amorite, king of Heshbon, and his land: begin to possess it, and contend with him in battle.
\verse This day will I begin to put the dread of thee and the fear of thee upon the nations that are under the whole heaven, who shall hear report of thee, and shall tremble, and be in anguish because of thee.
\verse And I sent messengers out of the wilderness of Kedemoth unto Sihon king of Heshbon with words of peace, saying,
\verse Let me pass through thy land: I will go along by the high way, I will neither turn unto the right hand nor to the left.
\verse Thou shalt sell me meat for money, that I may eat; and give me water for money, that I may drink: only I will pass through on my feet;
\verse (As the children of Esau which dwell in Seir, and the Moabites which dwell in Ar, did unto me;) until I shall pass over Jordan into the land which the \LORD our God giveth us.
\verse But Sihon king of Heshbon would not let us pass by him: for the \LORD thy God hardened his spirit, and made his heart obstinate, that he might deliver him into thy hand, as appeareth this day.
\verse And the \LORD said unto me, Behold, I have begun to give Sihon and his land before thee: begin to possess, that thou mayest inherit his land.
\verse Then Sihon came out against us, he and all his people, to fight at Jahaz.
\verse And the \LORD our God delivered him before us; and we smote him, and his sons, and all his people.
\verse And we took all his cities at that time, and utterly destroyed the men, and the women, and the little ones, of every city, we left none to remain:
\verse Only the cattle we took for a prey unto ourselves, and the spoil of the cities which we took.
\verse From Aroer, which is by the brink of the river of Arnon, and from the city that is by the river, even unto Gilead, there was not one city too strong for us: the \LORD our God delivered all unto us:
\verse Only unto the land of the children of Ammon thou camest not, nor unto any place of the river Jabbok, nor unto the cities in the mountains, nor unto whatsoever the \LORD our God forbad us.
\end{biblechapter}

\begin{biblechapter} % Deuteronomy 3
\verseWithHeading{Defeat Og king of Bashan} Then we turned, and went up the way to Bashan: and Og the king of Bashan came out against us, he and all his people, to battle at Edrei.
\verse And the \LORD said unto me, Fear him not: for I will deliver him, and all his people, and his land, into thy hand; and thou shalt do unto him as thou didst unto Sihon king of the Amorites, which dwelt at Heshbon.
\verse So the \LORD our God delivered into our hands Og also, the king of Bashan, and all his people: and we smote him until none was left to him remaining.
\verse And we took all his cities at that time, there was not a city which we took not from them, threescore cities, all the region of Argob, the kingdom of Og in Bashan.
\verse All these cities were fenced with high walls, gates, and bars; beside unwalled towns a great many.
\verse And we utterly destroyed them, as we did unto Sihon king of Heshbon, utterly destroying the men, women, and children, of every city.
\verse But all the cattle, and the spoil of the cities, we took for a prey to ourselves.
\verse And we took at that time out of the hand of the two kings of the Amorites the land that was on this side Jordan, from the river of Arnon unto mount Hermon;
\verse (Which Hermon the Sidonians call Sirion; and the Amorites call it Shenir;)
\verse All the cities of the plain, and all Gilead, and all Bashan, unto Salchah and Edrei, cities of the kingdom of Og in Bashan.
\verse For only Og king of Bashan remained of the remnant of giants; behold, his bedstead was a bedstead of iron; is it not in Rabbath of the children of Ammon? nine cubits was the length thereof, and four cubits the breadth of it, after the cubit of a man.
\verseWithHeading{Division of the land} And this land, which we possessed at that time, from Aroer, which is by the river Arnon, and half mount Gilead, and the cities thereof, gave I unto the Reubenites and to the Gadites.
\verse And the rest of Gilead, and all Bashan, being the kingdom of Og, gave I unto the half tribe of Manasseh; all the region of Argob, with all Bashan, which was called the land of giants.
\verse Jair the son of Manasseh took all the country of Argob unto the coasts of Geshuri and Maachathi; and called them after his own name, Bashanhavothjair, unto this day.
\verse And I gave Gilead unto Machir.
\verse And unto the Reubenites and unto the Gadites I gave from Gilead even unto the river Arnon half the valley, and the border even unto the river Jabbok, which is the border of the children of Ammon;
\verse The plain also, and Jordan, and the coast thereof, from Chinnereth even unto the sea of the plain, even the salt sea, under Ashdothpisgah eastward.
\verse And I commanded you at that time, saying, The \LORD your God hath given you this land to possess it: ye shall pass over armed before your brethren the children of Israel, all that are meet for the war.
\verse But your wives, and your little ones, and your cattle, (for I know that ye have much cattle,) shall abide in your cities which I have given you;
\verse Until the \LORD have given rest unto your brethren, as well as unto you, and until they also possess the land which the \LORD your God hath given them beyond Jordan: and then shall ye return every man unto his possession, which I have given you.
\verseWithHeading{Moses forbidden to cross the Jordan} And I commanded Joshua at that time, saying, Thine eyes have seen all that the \LORD your God hath done unto these two kings: so shall the \LORD do unto all the kingdoms whither thou passest.
\verse Ye shall not fear them: for the \LORD your God he shall fight for you.
\verse And I besought the \LORD at that time, saying,
\verse O Lord God, thou hast begun to shew thy servant thy greatness, and thy mighty hand: for what God is there in heaven or in earth, that can do according to thy works, and according to thy might?
\verse I pray thee, let me go over, and see the good land that is beyond Jordan, that goodly mountain, and Lebanon.
\verse But the \LORD was wroth with me for your sakes, and would not hear me: and the \LORD said unto me, Let it suffice thee; speak no more unto me of this matter.
\verse Get thee up into the top of Pisgah, and lift up thine eyes westward, and northward, and southward, and eastward, and behold it with thine eyes: for thou shalt not go over this Jordan.
\verse But charge Joshua, and encourage him, and strengthen him: for he shall go over before this people, and he shall cause them to inherit the land which thou shalt see.
\verse So we abode in the valley over against Bethpeor.
\end{biblechapter}

\begin{biblechapter} % Deuteronomy 4
\verseWithHeading{Obedience commanded} Now therefore hearken, O Israel, unto the statutes and unto the judgments, which I teach you, for to do them, that ye may live, and go in and possess the land which the \LORD God of your fathers giveth you.
\verse Ye shall not add unto the word which I command you, neither shall ye diminish ought from it, that ye may keep the commandments of the \LORD your God which I command you.
\verse Your eyes have seen what the \LORD did because of Baalpeor: for all the men that followed Baalpeor, the \LORD thy God hath destroyed them from among you.
\verse But ye that did cleave unto the \LORD your God are alive every one of you this day.
\verse Behold, I have taught you statutes and judgments, even as the \LORD my God commanded me, that ye should do so in the land whither ye go to possess it.
\verse Keep therefore and do them; for this is your wisdom and your understanding in the sight of the nations, which shall hear all these statutes, and say, Surely this great nation is a wise and understanding people.
\verse For what nation is there so great, who hath God so nigh unto them, as the \LORD our God is in all things that we call upon him for?
\verse And what nation is there so great, that hath statutes and judgments so righteous as all this law, which I set before you this day?
\verse Only take heed to thyself, and keep thy soul diligently, lest thou forget the things which thine eyes have seen, and lest they depart from thy heart all the days of thy life: but teach them thy sons, and thy sons' sons;
\verse Specially the day that thou stoodest before the \LORD thy God in Horeb, when the \LORD said unto me, Gather me the people together, and I will make them hear my words, that they may learn to fear me all the days that they shall live upon the earth, and that they may teach their children.
\verse And ye came near and stood under the mountain; and the mountain burned with fire unto the midst of heaven, with darkness, clouds, and thick darkness.
\verse And the \LORD spake unto you out of the midst of the fire: ye heard the voice of the words, but saw no similitude; only ye heard a voice.
\verse And he declared unto you his covenant, which he commanded you to perform, even ten commandments; and he wrote them upon two tables of stone.
\verse And the \LORD commanded me at that time to teach you statutes and judgments, that ye might do them in the land whither ye go over to possess it.
\verseWithHeading{Idolatry forbidden} Take ye therefore good heed unto yourselves; for ye saw no manner of similitude on the day that the \LORD spake unto you in Horeb out of the midst of the fire:
\verse Lest ye corrupt yourselves, and make you a graven image, the similitude of any figure, the likeness of male or female,
\verse The likeness of any beast that is on the earth, the likeness of any winged fowl that flieth in the air,
\verse The likeness of any thing that creepeth on the ground, the likeness of any fish that is in the waters beneath the earth:
\verse And lest thou lift up thine eyes unto heaven, and when thou seest the sun, and the moon, and the stars, even all the host of heaven, shouldest be driven to worship them, and serve them, which the \LORD thy God hath divided unto all nations under the whole heaven.
\verse But the \LORD hath taken you, and brought you forth out of the iron furnace, even out of Egypt, to be unto him a people of inheritance, as ye are this day.
\verse Furthermore the \LORD was angry with me for your sakes, and sware that I should not go over Jordan, and that I should not go in unto that good land, which the \LORD thy God giveth thee for an inheritance:
\verse But I must die in this land, I must not go over Jordan: but ye shall go over, and possess that good land.
\verse Take heed unto yourselves, lest ye forget the covenant of the \LORD your God, which he made with you, and make you a graven image, or the likeness of any thing, which the \LORD thy God hath forbidden thee.
\verse For the \LORD thy God is a consuming fire, even a jealous God.
\verse When thou shalt beget children, and children's children, and ye shall have remained long in the land, and shall corrupt yourselves, and make a graven image, or the likeness of any thing, and shall do evil in the sight of the \LORD thy God, to provoke him to anger:
\verse I call heaven and earth to witness against you this day, that ye shall soon utterly perish from off the land whereunto ye go over Jordan to possess it; ye shall not prolong your days upon it, but shall utterly be destroyed.
\verse And the \LORD shall scatter you among the nations, and ye shall be left few in number among the heathen, whither the \LORD shall lead you.
\verse And there ye shall serve gods, the work of men's hands, wood and stone, which neither see, nor hear, nor eat, nor smell.
\verse But if from thence thou shalt seek the \LORD thy God, thou shalt find him, if thou seek him with all thy heart and with all thy soul.
\verse When thou art in tribulation, and all these things are come upon thee, even in the latter days, if thou turn to the \LORD thy God, and shalt be obedient unto his voice;
\verse (For the \LORD thy God is a merciful God;) he will not forsake thee, neither destroy thee, nor forget the covenant of thy fathers which he sware unto them.
\verseWithHeading{The \LORD is God} For ask now of the days that are past, which were before thee, since the day that God created man upon the earth, and ask from the one side of heaven unto the other, whether there hath been any such thing as this great thing is, or hath been heard like it?
\verse Did ever people hear the voice of God speaking out of the midst of the fire, as thou hast heard, and live?
\verse Or hath God assayed to go and take him a nation from the midst of another nation, by temptations, by signs, and by wonders, and by war, and by a mighty hand, and by a stretched out arm, and by great terrors, according to all that the \LORD your God did for you in Egypt before your eyes?
\verse Unto thee it was shewed, that thou mightest know that the \LORD he is God; there is none else beside him.
\verse Out of heaven he made thee to hear his voice, that he might instruct thee: and upon earth he shewed thee his great fire; and thou heardest his words out of the midst of the fire.
\verse And because he loved thy fathers, therefore he chose their seed after them, and brought thee out in his sight with his mighty power out of Egypt;
\verse To drive out nations from before thee greater and mightier than thou art, to bring thee in, to give thee their land for an inheritance, as it is this day.
\verse Know therefore this day, and consider it in thine heart, that the \LORD he is God in heaven above, and upon the earth beneath: there is none else.
\verse Thou shalt keep therefore his statutes, and his commandments, which I command thee this day, that it may go well with thee, and with thy children after thee, and that thou mayest prolong thy days upon the earth, which the \LORD thy God giveth thee, for ever.
\verseWithHeading{Cities of refuge} Then Moses severed three cities on this side Jordan toward the sunrising;
\verse That the slayer might flee thither, which should kill his neighbour unawares, and hated him not in times past; and that fleeing unto one of these cities he might live:
\verse Namely, Bezer in the wilderness, in the plain country, of the Reubenites; and Ramoth in Gilead, of the Gadites; and Golan in Bashan, of the Manassites.
\verseWithHeading{Introduction of the law} And this is the law which Moses set before the children of Israel:
\verse These are the testimonies, and the statutes, and the judgments, which Moses spake unto the children of Israel, after they came forth out of Egypt,
\verse On this side Jordan, in the valley over against Bethpeor, in the land of Sihon king of the Amorites, who dwelt at Heshbon, whom Moses and the children of Israel smote, after they were come forth out of Egypt:
\verse And they possessed his land, and the land of Og king of Bashan, two kings of the Amorites, which were on this side Jordan toward the sunrising;
\verse From Aroer, which is by the bank of the river Arnon, even unto mount Sion, which is Hermon,
\verse And all the plain on this side Jordan eastward, even unto the sea of the plain, under the springs of Pisgah.
\end{biblechapter}

\begin{biblechapter} % Deuteronomy 5
\verseWithHeading{The Ten Commandments} And Moses called all Israel, and said unto them, Hear, O Israel, the statutes and judgments which I speak in your ears this day, that ye may learn them, and keep, and do them.
\verse The \LORD our God made a covenant with us in Horeb.
\verse The \LORD made not this covenant with our fathers, but with us, even us, who are all of us here alive this day.
\verse The \LORD talked with you face to face in the mount out of the midst of the fire,
\verse (I stood between the \LORD and you at that time, to shew you the word of the \LORD: for ye were afraid by reason of the fire, and went not up into the mount;) saying,
\verse I am the \LORD thy God, which brought thee out of the land of Egypt, from the house of bondage.
\verse Thou shalt have none other gods before me.
\verse Thou shalt not make thee any graven image, or any likeness of any thing that is in heaven above, or that is in the earth beneath, or that is in the waters beneath the earth:
\verse Thou shalt not bow down thyself unto them, nor serve them: for I the \LORD thy God am a jealous God, visiting the iniquity of the fathers upon the children unto the third and fourth generation of them that hate me,
\verse And shewing mercy unto thousands of them that love me and keep my commandments.
\verse Thou shalt not take the name of the \LORD thy God in vain: for the \LORD will not hold him guiltless that taketh his name in vain.
\verse Keep the Sabbath day to sanctify it, as the \LORD thy God hath commanded thee.
\verse Six days thou shalt labour, and do all thy work:
\verse But the seventh day is the Sabbath of the \LORD thy God: in it thou shalt not do any work, thou, nor thy son, nor thy daughter, nor thy manservant, nor thy maidservant, nor thine ox, nor thine ass, nor any of thy cattle, nor thy stranger that is within thy gates; that thy manservant and thy maidservant may rest as well as thou.
\verse And remember that thou wast a servant in the land of Egypt, and that the \LORD thy God brought thee out thence through a mighty hand and by a stretched out arm: therefore the \LORD thy God commanded thee to keep the Sabbath day.
\verse Honour thy father and thy mother, as the \LORD thy God hath commanded thee; that thy days may be prolonged, and that it may go well with thee, in the land which the \LORD thy God giveth thee.
\verse Thou shalt not kill.
\verse Neither shalt thou commit adultery.
\verse Neither shalt thou steal.
\verse Neither shalt thou bear false witness against thy neighbour.
\verse Neither shalt thou desire thy neighbour's wife, neither shalt thou covet thy neighbour's house, his field, or his manservant, or his maidservant, his ox, or his ass, or any thing that is thy neighbour's.
\verse These words the \LORD spake unto all your assembly in the mount out of the midst of the fire, of the cloud, and of the thick darkness, with a great voice: and he added no more. And he wrote them in two tables of stone, and delivered them unto me.
\verse And it came to pass, when ye heard the voice out of the midst of the darkness, (for the mountain did burn with fire,) that ye came near unto me, even all the heads of your tribes, and your elders;
\verse And ye said, Behold, the \LORD our God hath shewed us his glory and his greatness, and we have heard his voice out of the midst of the fire: we have seen this day that God doth talk with man, and he liveth.
\verse Now therefore why should we die? for this great fire will consume us: if we hear the voice of the \LORD our God any more, then we shall die.
\verse For who is there of all flesh, that hath heard the voice of the living God speaking out of the midst of the fire, as we have, and lived?
\verse Go thou near, and hear all that the \LORD our God shall say: and speak thou unto us all that the \LORD our God shall speak unto thee; and we will hear it, and do it.
\verse And the \LORD heard the voice of your words, when ye spake unto me; and the \LORD said unto me, I have heard the voice of the words of this people, which they have spoken unto thee: they have well said all that they have spoken.
\verse O that there were such an heart in them, that they would fear me, and keep all my commandments always, that it might be well with them, and with their children for ever!
\verse Go say to them, Get you into your tents again.
\verse But as for thee, stand thou here by me, and I will speak unto thee all the commandments, and the statutes, and the judgments, which thou shalt teach them, that they may do them in the land which I give them to possess it.
\verse Ye shall observe to do therefore as the \LORD your God hath commanded you: ye shall not turn aside to the right hand or to the left.
\verse Ye shall walk in all the ways which the \LORD your God hath commanded you, that ye may live, and that it may be well with you, and that ye may prolong your days in the land which ye shall possess.
\end{biblechapter}

\begin{biblechapter} % Deuteronomy 6
\verseWithHeading{Love the \LORD your God} Now these are the commandments, the statutes, and the judgments, which the \LORD your God commanded to teach you, that ye might do them in the land whither ye go to possess it:
\verse That thou mightest fear the \LORD thy God, to keep all his statutes and his commandments, which I command thee, thou, and thy son, and thy son's son, all the days of thy life; and that thy days may be prolonged.
\verse Hear therefore, O Israel, and observe to do it; that it may be well with thee, and that ye may increase mightily, as the \LORD God of thy fathers hath promised thee, in the land that floweth with milk and honey.
\verse Hear, O Israel: The \LORD our God is one \LORD:
\verse And thou shalt love the \LORD thy God with all thine heart, and with all thy soul, and with all thy might.
\verse And these words, which I command thee this day, shall be in thine heart:
\verse And thou shalt teach them diligently unto thy children, and shalt talk of them when thou sittest in thine house, and when thou walkest by the way, and when thou liest down, and when thou risest up.
\verse And thou shalt bind them for a sign upon thine hand, and they shall be as frontlets between thine eyes.
\verse And thou shalt write them upon the posts of thy house, and on thy gates.
\verse And it shall be, when the \LORD thy God shall have brought thee into the land which he sware unto thy fathers, to Abraham, to Isaac, and to Jacob, to give thee great and goodly cities, which thou buildedst not,
\verse And houses full of all good things, which thou filledst not, and wells digged, which thou diggedst not, vineyards and olive trees, which thou plantedst not; when thou shalt have eaten and be full;
\verse Then beware lest thou forget the \LORD, which brought thee forth out of the land of Egypt, from the house of bondage.
\verse Thou shalt fear the \LORD thy God, and serve him, and shalt swear by his name.
\verse Ye shall not go after other gods, of the gods of the people which are round about you;
\verse (For the \LORD thy God is a jealous God among you) lest the anger of the \LORD thy God be kindled against thee, and destroy thee from off the face of the earth.
\verse Ye shall not tempt the \LORD your God, as ye tempted him in Massah.
\verse Ye shall diligently keep the commandments of the \LORD your God, and his testimonies, and his statutes, which he hath commanded thee.
\verse And thou shalt do that which is right and good in the sight of the \LORD: that it may be well with thee, and that thou mayest go in and possess the good land which the \LORD sware unto thy fathers,
\verse To cast out all thine enemies from before thee, as the \LORD hath spoken.
\verse And when thy son asketh thee in time to come, saying, What mean the testimonies, and the statutes, and the judgments, which the \LORD our God hath commanded you?
\verse Then thou shalt say unto thy son, We were Pharaoh's bondmen in Egypt; and the \LORD brought us out of Egypt with a mighty hand:
\verse And the \LORD shewed signs and wonders, great and sore, upon Egypt, upon Pharaoh, and upon all his household, before our eyes:
\verse And he brought us out from thence, that he might bring us in, to give us the land which he sware unto our fathers.
\verse And the \LORD commanded us to do all these statutes, to fear the \LORD our God, for our good always, that he might preserve us alive, as it is at this day.
\verse And it shall be our righteousness, if we observe to do all these commandments before the \LORD our God, as he hath commanded us.
\end{biblechapter}

\begin{biblechapter} % Deuteronomy 7
\verseWithHeading{Driving out the nations} When the \LORD thy God shall bring thee into the land whither thou goest to possess it, and hath cast out many nations before thee, the Hittites, and the Girgashites, and the Amorites, and the Canaanites, and the Perizzites, and the Hivites, and the Jebusites, seven nations greater and mightier than thou;
\verse And when the \LORD thy God shall deliver them before thee; thou shalt smite them, and utterly destroy them; thou shalt make no covenant with them, nor shew mercy unto them:
\verse Neither shalt thou make marriages with them; thy daughter thou shalt not give unto his son, nor his daughter shalt thou take unto thy son.
\verse For they will turn away thy son from following me, that they may serve other gods: so will the anger of the \LORD be kindled against you, and destroy thee suddenly.
\verse But thus shall ye deal with them; ye shall destroy their altars, and break down their images, and cut down their groves, and burn their graven images with fire.
\verse For thou art an holy people unto the \LORD thy God: the \LORD thy God hath chosen thee to be a special people unto himself, above all people that are upon the face of the earth.
\verse The \LORD did not set his love upon you, nor choose you, because ye were more in number than any people; for ye were the fewest of all people:
\verse But because the \LORD loved you, and because he would keep the oath which he had sworn unto your fathers, hath the \LORD brought you out with a mighty hand, and redeemed you out of the house of bondmen, from the hand of Pharaoh king of Egypt.
\verse Know therefore that the \LORD thy God, he is God, the faithful God, which keepeth covenant and mercy with them that love him and keep his commandments to a thousand generations;
\verse And repayeth them that hate him to their face, to destroy them: he will not be slack to him that hateth him, he will repay him to his face.
\verse Thou shalt therefore keep the commandments, and the statutes, and the judgments, which I command thee this day, to do them.
\verse Wherefore it shall come to pass, if ye hearken to these judgments, and keep, and do them, that the \LORD thy God shall keep unto thee the covenant and the mercy which he sware unto thy fathers:
\verse And he will love thee, and bless thee, and multiply thee: he will also bless the fruit of thy womb, and the fruit of thy land, thy corn, and thy wine, and thine oil, the increase of thy kine, and the flocks of thy sheep, in the land which he sware unto thy fathers to give thee.
\verse Thou shalt be blessed above all people: there shall not be male or female barren among you, or among your cattle.
\verse And the \LORD will take away from thee all sickness, and will put none of the evil diseases of Egypt, which thou knowest, upon thee; but will lay them upon all them that hate thee.
\verse And thou shalt consume all the people which the \LORD thy God shall deliver thee; thine eye shall have no pity upon them: neither shalt thou serve their gods; for that will be a snare unto thee.
\verse If thou shalt say in thine heart, These nations are more than I; how can I dispossess them?
\verse Thou shalt not be afraid of them: but shalt well remember what the \LORD thy God did unto Pharaoh, and unto all Egypt;
\verse The great temptations which thine eyes saw, and the signs, and the wonders, and the mighty hand, and the stretched out arm, whereby the \LORD thy God brought thee out: so shall the \LORD thy God do unto all the people of whom thou art afraid.
\verse Moreover the \LORD thy God will send the hornet among them, until they that are left, and hide themselves from thee, be destroyed.
\verse Thou shalt not be affrighted at them: for the \LORD thy God is among you, a mighty God and terrible.
\verse And the \LORD thy God will put out those nations before thee by little and little: thou mayest not consume them at once, lest the beasts of the field increase upon thee.
\verse But the \LORD thy God shall deliver them unto thee, and shall destroy them with a mighty destruction, until they be destroyed.
\verse And he shall deliver their kings into thine hand, and thou shalt destroy their name from under heaven: there shall no man be able to stand before thee, until thou have destroyed them.
\verse The graven images of their gods shall ye burn with fire: thou shalt not desire the silver or gold that is on them, nor take it unto thee, lest thou be snared therein: for it is an abomination to the \LORD thy God.
\verse Neither shalt thou bring an abomination into thine house, lest thou be a cursed thing like it: but thou shalt utterly detest it, and thou shalt utterly abhor it; for it is a cursed thing.
\end{biblechapter}

\begin{biblechapter} % Deuteronomy 8
\verseWithHeading{Do not forget the \LORD} All the commandments which I command thee this day shall ye observe to do, that ye may live, and multiply, and go in and possess the land which the \LORD sware unto your fathers.
\verse And thou shalt remember all the way which the \LORD thy God led thee these forty years in the wilderness, to humble thee, and to prove thee, to know what was in thine heart, whether thou wouldest keep his commandments, or no.
\verse And he humbled thee, and suffered thee to hunger, and fed thee with manna, which thou knewest not, neither did thy fathers know; that he might make thee know that man doth not live by bread only, but by every word that proceedeth out of the mouth of the \LORD doth man live.
\verse Thy raiment waxed not old upon thee, neither did thy foot swell, these forty years.
\verse Thou shalt also consider in thine heart, that, as a man chasteneth his son, so the \LORD thy God chasteneth thee.
\verse Therefore thou shalt keep the commandments of the \LORD thy God, to walk in his ways, and to fear him.
\verse For the \LORD thy God bringeth thee into a good land, a land of brooks of water, of fountains and depths that spring out of valleys and hills;
\verse A land of wheat, and barley, and vines, and fig trees, and pomegranates; a land of oil olive, and honey;
\verse A land wherein thou shalt eat bread without scarceness, thou shalt not lack any thing in it; a land whose stones are iron, and out of whose hills thou mayest dig brass.
\verse When thou hast eaten and art full, then thou shalt bless the \LORD thy God for the good land which he hath given thee.
\verse Beware that thou forget not the \LORD thy God, in not keeping his commandments, and his judgments, and his statutes, which I command thee this day:
\verse Lest when thou hast eaten and art full, and hast built goodly houses, and dwelt therein;
\verse And when thy herds and thy flocks multiply, and thy silver and thy gold is multiplied, and all that thou hast is multiplied;
\verse Then thine heart be lifted up, and thou forget the \LORD thy God, which brought thee forth out of the land of Egypt, from the house of bondage;
\verse Who led thee through that great and terrible wilderness, wherein were fiery serpents, and scorpions, and drought, where there was no water; who brought thee forth water out of the rock of flint;
\verse Who fed thee in the wilderness with manna, which thy fathers knew not, that he might humble thee, and that he might prove thee, to do thee good at thy latter end;
\verse And thou say in thine heart, My power and the might of mine hand hath gotten me this wealth.
\verse But thou shalt remember the \LORD thy God: for it is he that giveth thee power to get wealth, that he may establish his covenant which he sware unto thy fathers, as it is this day.
\verse And it shall be, if thou do at all forget the \LORD thy God, and walk after other gods, and serve them, and worship them, I testify against you this day that ye shall surely perish.
\verse As the nations which the \LORD destroyeth before your face, so shall ye perish; because ye would not be obedient unto the voice of the \LORD your God.
\end{biblechapter}

\begin{biblechapter} % Deuteronomy 9
\verseWithHeading{Not because of Israel's \newline righteousness} Hear, O Israel: Thou art to pass over Jordan this day, to go in to possess nations greater and mightier than thyself, cities great and fenced up to heaven,
\verse A people great and tall, the children of the Anakims, whom thou knowest, and of whom thou hast heard say, Who can stand before the children of Anak!
\verse Understand therefore this day, that the \LORD thy God is he which goeth over before thee; as a consuming fire he shall destroy them, and he shall bring them down before thy face: so shalt thou drive them out, and destroy them quickly, as the \LORD hath said unto thee.
\verse Speak not thou in thine heart, after that the \LORD thy God hath cast them out from before thee, saying, For my righteousness the \LORD hath brought me in to possess this land: but for the wickedness of these nations the \LORD doth drive them out from before thee.
\verse Not for thy righteousness, or for the uprightness of thine heart, dost thou go to possess their land: but for the wickedness of these nations the \LORD thy God doth drive them out from before thee, and that he may perform the word which the \LORD sware unto thy fathers, Abraham, Isaac, and Jacob.
\verse Understand therefore, that the \LORD thy God giveth thee not this good land to possess it for thy righteousness; for thou art a stiffnecked people.
\verseWithHeading{The golden calf} Remember, and forget not, how thou provokedst the \LORD thy God to wrath in the wilderness: from the day that thou didst depart out of the land of Egypt, until ye came unto this place, ye have been rebellious against the \LORD.
\verse Also in Horeb ye provoked the \LORD to wrath, so that the \LORD was angry with you to have destroyed you.
\verse When I was gone up into the mount to receive the tables of stone, even the tables of the covenant which the \LORD made with you, then I abode in the mount forty days and forty nights, I neither did eat bread nor drink water:
\verse And the \LORD delivered unto me two tables of stone written with the finger of God; and on them was written according to all the words, which the \LORD spake with you in the mount out of the midst of the fire in the day of the assembly.
\verse And it came to pass at the end of forty days and forty nights, that the \LORD gave me the two tables of stone, even the tables of the covenant.
\verse And the \LORD said unto me, Arise, get thee down quickly from hence; for thy people which thou hast brought forth out of Egypt have corrupted themselves; they are quickly turned aside out of the way which I commanded them; they have made them a molten image.
\verse Furthermore the \LORD spake unto me, saying, I have seen this people, and, behold, it is a stiffnecked people:
\verse Let me alone, that I may destroy them, and blot out their name from under heaven: and I will make of thee a nation mightier and greater than they.
\verse So I turned and came down from the mount, and the mount burned with fire: and the two tables of the covenant were in my two hands.
\verse And I looked, and, behold, ye had sinned against the \LORD your God, and had made you a molten calf: ye had turned aside quickly out of the way which the \LORD had commanded you.
\verse And I took the two tables, and cast them out of my two hands, and brake them before your eyes.
\verse And I fell down before the \LORD, as at the first, forty days and forty nights: I did neither eat bread, nor drink water, because of all your sins which ye sinned, in doing wickedly in the sight of the \LORD, to provoke him to anger.
\verse For I was afraid of the anger and hot displeasure, wherewith the \LORD was wroth against you to destroy you. But the \LORD hearkened unto me at that time also.
\verse And the \LORD was very angry with Aaron to have destroyed him: and I prayed for Aaron also the same time.
\verse And I took your sin, the calf which ye had made, and burnt it with fire, and stamped it, and ground it very small, even until it was as small as dust: and I cast the dust thereof into the brook that descended out of the mount.
\verse And at Taberah, and at Massah, and at Kibrothhattaavah, ye provoked the \LORD to wrath.
\verse Likewise when the \LORD sent you from Kadeshbarnea, saying, Go up and possess the land which I have given you; then ye rebelled against the commandment of the \LORD your God, and ye believed him not, nor hearkened to his voice.
\verse Ye have been rebellious against the \LORD from the day that I knew you.
\verse Thus I fell down before the \LORD forty days and forty nights, as I fell down at the first; because the \LORD had said he would destroy you.
\verse I prayed therefore unto the \LORD, and said, O Lord God, destroy not thy people and thine inheritance, which thou hast redeemed through thy greatness, which thou hast brought forth out of Egypt with a mighty hand.
\verse Remember thy servants, Abraham, Isaac, and Jacob; look not unto the stubbornness of this people, nor to their wickedness, nor to their sin:
\verse Lest the land whence thou broughtest us out say, Because the \LORD was not able to bring them into the land which he promised them, and because he hated them, he hath brought them out to slay them in the wilderness.
\verse Yet they are thy people and thine inheritance, which thou broughtest out by thy mighty power and by thy stretched out arm.
\end{biblechapter}

\begin{biblechapter} % Deuteronomy 10
\verseWithHeading{Tablets like the first ones} At that time the \LORD said unto me, Hew thee two tables of stone like unto the first, and come up unto me into the mount, and make thee an ark of wood.
\verse And I will write on the tables the words that were in the first tables which thou brakest, and thou shalt put them in the ark.
\verse And I made an ark of shittim wood, and hewed two tables of stone like unto the first, and went up into the mount, having the two tables in mine hand.
\verse And he wrote on the tables, according to the first writing, the ten commandments, which the \LORD spake unto you in the mount out of the midst of the fire in the day of the assembly: and the \LORD gave them unto me.
\verse And I turned myself and came down from the mount, and put the tables in the ark which I had made; and there they be, as the \LORD commanded me.
\verse And the children of Israel took their journey from Beeroth of the children of Jaakan to Mosera: there Aaron died, and there he was buried; and Eleazar his son ministered in the priest's office in his stead.
\verse From thence they journeyed unto Gudgodah; and from Gudgodah to Jotbath, a land of rivers of waters.
\verse At that time the \LORD separated the tribe of Levi, to bear the ark of the covenant of the \LORD, to stand before the \LORD to minister unto him, and to bless in his name, unto this day.
\verse Wherefore Levi hath no part nor inheritance with his brethren; the \LORD is his inheritance, according as the \LORD thy God promised him.
\verse And I stayed in the mount, according to the first time, forty days and forty nights; and the \LORD hearkened unto me at that time also, and the \LORD would not destroy thee.
\verse And the \LORD said unto me, Arise, take thy journey before the people, that they may go in and possess the land, which I sware unto their fathers to give unto them.
\verseWithHeading{Fear the \LORD} And now, Israel, what doth the \LORD thy God require of thee, but to fear the \LORD thy God, to walk in all his ways, and to love him, and to serve the \LORD thy God with all thy heart and with all thy soul,
\verse To keep the commandments of the \LORD, and his statutes, which I command thee this day for thy good?
\verse Behold, the heaven and the heaven of heavens is the \LORDs thy God, the earth also, with all that therein is.
\verse Only the \LORD had a delight in thy fathers to love them, and he chose their seed after them, even you above all people, as it is this day.
\verse Circumcise therefore the foreskin of your heart, and be no more stiffnecked.
\verse For the \LORD your God is God of gods, and Lord of lords, a great God, a mighty, and a terrible, which regardeth not persons, nor taketh reward:
\verse He doth execute the judgment of the fatherless and widow, and loveth the stranger, in giving him food and raiment.
\verse Love ye therefore the stranger: for ye were strangers in the land of Egypt.
\verse Thou shalt fear the \LORD thy God; him shalt thou serve, and to him shalt thou cleave, and swear by his name.
\verse He is thy praise, and he is thy God, that hath done for thee these great and terrible things, which thine eyes have seen.
\verse Thy fathers went down into Egypt with threescore and ten persons; and now the \LORD thy God hath made thee as the stars of heaven for multitude.
\end{biblechapter}

\begin{biblechapter} % Deuteronomy 11
\verseWithHeading{Love and obey the \LORD} Therefore thou shalt love the \LORD thy God, and keep his charge, and his statutes, and his judgments, and his commandments, alway.
\verse And know ye this day: for I speak not with your children which have not known, and which have not seen the chastisement of the \LORD your God, his greatness, his mighty hand, and his stretched out arm,
\verse And his miracles, and his acts, which he did in the midst of Egypt unto Pharaoh the king of Egypt, and unto all his land;
\verse And what he did unto the army of Egypt, unto their horses, and to their chariots; how he made the water of the Red sea to overflow them as they pursued after you, and how the \LORD hath destroyed them unto this day;
\verse And what he did unto you in the wilderness, until ye came into this place;
\verse And what he did unto Dathan and Abiram, the sons of Eliab, the son of Reuben: how the earth opened her mouth, and swallowed them up, and their households, and their tents, and all the substance that was in their possession, in the midst of all Israel:
\verse But your eyes have seen all the great acts of the \LORD which he did.
\verse Therefore shall ye keep all the commandments which I command you this day, that ye may be strong, and go in and possess the land, whither ye go to possess it;
\verse And that ye may prolong your days in the land, which the \LORD sware unto your fathers to give unto them and to their seed, a land that floweth with milk and honey.
\verse For the land, whither thou goest in to possess it, is not as the land of Egypt, from whence ye came out, where thou sowedst thy seed, and wateredst it with thy foot, as a garden of herbs:
\verse But the land, whither ye go to possess it, is a land of hills and valleys, and drinketh water of the rain of heaven:
\verse A land which the \LORD thy God careth for: the eyes of the \LORD thy God are always upon it, from the beginning of the year even unto the end of the year.
\verse And it shall come to pass, if ye shall hearken diligently unto my commandments which I command you this day, to love the \LORD your God, and to serve him with all your heart and with all your soul,
\verse That I will give you the rain of your land in his due season, the first rain and the latter rain, that thou mayest gather in thy corn, and thy wine, and thine oil.
\verse And I will send grass in thy fields for thy cattle, that thou mayest eat and be full.
\verse Take heed to yourselves, that your heart be not deceived, and ye turn aside, and serve other gods, and worship them;
\verse And then the \LORDs wrath be kindled against you, and he shut up the heaven, that there be no rain, and that the land yield not her fruit; and lest ye perish quickly from off the good land which the \LORD giveth you.
\verse Therefore shall ye lay up these my words in your heart and in your soul, and bind them for a sign upon your hand, that they may be as frontlets between your eyes.
\verse And ye shall teach them your children, speaking of them when thou sittest in thine house, and when thou walkest by the way, when thou liest down, and when thou risest up.
\verse And thou shalt write them upon the door posts of thine house, and upon thy gates:
\verse That your days may be multiplied, and the days of your children, in the land which the \LORD sware unto your fathers to give them, as the days of heaven upon the earth.
\verse For if ye shall diligently keep all these commandments which I command you, to do them, to love the \LORD your God, to walk in all his ways, and to cleave unto him;
\verse Then will the \LORD drive out all these nations from before you, and ye shall possess greater nations and mightier than yourselves.
\verse Every place whereon the soles of your feet shall tread shall be yours: from the wilderness and Lebanon, from the river, the river Euphrates, even unto the uttermost sea shall your coast be.
\verse There shall no man be able to stand before you: for the \LORD your God shall lay the fear of you and the dread of you upon all the land that ye shall tread upon, as he hath said unto you.
\verse Behold, I set before you this day a blessing and a curse;
\verse A blessing, if ye obey the commandments of the \LORD your God, which I command you this day:
\verse And a curse, if ye will not obey the commandments of the \LORD your God, but turn aside out of the way which I command you this day, to go after other gods, which ye have not known.
\verse And it shall come to pass, when the \LORD thy God hath brought thee in unto the land whither thou goest to possess it, that thou shalt put the blessing upon mount Gerizim, and the curse upon mount Ebal.
\verse Are they not on the other side Jordan, by the way where the sun goeth down, in the land of the Canaanites, which dwell in the champaign over against Gilgal, beside the plains of Moreh?
\verse For ye shall pass over Jordan to go in to possess the land which the \LORD your God giveth you, and ye shall possess it, and dwell therein.
\verse And ye shall observe to do all the statutes and judgments which I set before you this day.
\end{biblechapter}

\begin{biblechapter} % Deuteronomy 12
\verseWithHeading{The one place of worship} These are the statutes and judgments, which ye shall observe to do in the land, which the \LORD God of thy fathers giveth thee to possess it, all the days that ye live upon the earth.
\verse Ye shall utterly destroy all the places, wherein the nations which ye shall possess served their gods, upon the high mountains, and upon the hills, and under every green tree:
\verse And ye shall overthrow their altars, and break their pillars, and burn their groves with fire; and ye shall hew down the graven images of their gods, and destroy the names of them out of that place.
\verse Ye shall not do so unto the \LORD your God.
\verse But unto the place which the \LORD your God shall choose out of all your tribes to put his name there, even unto his habitation shall ye seek, and thither thou shalt come:
\verse And thither ye shall bring your burnt offerings, and your sacrifices, and your tithes, and heave offerings of your hand, and your vows, and your freewill offerings, and the firstlings of your herds and of your flocks:
\verse And there ye shall eat before the \LORD your God, and ye shall rejoice in all that ye put your hand unto, ye and your households, wherein the \LORD thy God hath blessed thee.
\verse Ye shall not do after all the things that we do here this day, every man whatsoever is right in his own eyes.
\verse For ye are not as yet come to the rest and to the inheritance, which the \LORD your God giveth you.
\verse But when ye go over Jordan, and dwell in the land which the \LORD your God giveth you to inherit, and when he giveth you rest from all your enemies round about, so that ye dwell in safety;
\verse Then there shall be a place which the \LORD your God shall choose to cause his name to dwell there; thither shall ye bring all that I command you; your burnt offerings, and your sacrifices, your tithes, and the heave offering of your hand, and all your choice vows which ye vow unto the \LORD:
\verse And ye shall rejoice before the \LORD your God, ye, and your sons, and your daughters, and your menservants, and your maidservants, and the Levite that is within your gates; forasmuch as he hath no part nor inheritance with you.
\verse Take heed to thyself that thou offer not thy burnt offerings in every place that thou seest:
\verse But in the place which the \LORD shall choose in one of thy tribes, there thou shalt offer thy burnt offerings, and there thou shalt do all that I command thee.
\verse Notwithstanding thou mayest kill and eat flesh in all thy gates, whatsoever thy soul lusteth after, according to the blessing of the \LORD thy God which he hath given thee: the unclean and the clean may eat thereof, as of the roebuck, and as of the hart.
\verse Only ye shall not eat the blood; ye shall pour it upon the earth as water.
\verse Thou mayest not eat within thy gates the tithe of thy corn, or of thy wine, or of thy oil, or the firstlings of thy herds or of thy flock, nor any of thy vows which thou vowest, nor thy freewill offerings, or heave offering of thine hand:
\verse But thou must eat them before the \LORD thy God in the place which the \LORD thy God shall choose, thou, and thy son, and thy daughter, and thy manservant, and thy maidservant, and the Levite that is within thy gates: and thou shalt rejoice before the \LORD thy God in all that thou puttest thine hands unto.
\verse Take heed to thyself that thou forsake not the Levite as long as thou livest upon the earth.
\verse When the \LORD thy God shall enlarge thy border, as he hath promised thee, and thou shalt say, I will eat flesh, because thy soul longeth to eat flesh; thou mayest eat flesh, whatsoever thy soul lusteth after.
\verse If the place which the \LORD thy God hath chosen to put his name there be too far from thee, then thou shalt kill of thy herd and of thy flock, which the \LORD hath given thee, as I have commanded thee, and thou shalt eat in thy gates whatsoever thy soul lusteth after.
\verse Even as the roebuck and the hart is eaten, so thou shalt eat them: the unclean and the clean shall eat of them alike.
\verse Only be sure that thou eat not the blood: for the blood is the life; and thou mayest not eat the life with the flesh.
\verse Thou shalt not eat it; thou shalt pour it upon the earth as water.
\verse Thou shalt not eat it; that it may go well with thee, and with thy children after thee, when thou shalt do that which is right in the sight of the \LORD.
\verse Only thy holy things which thou hast, and thy vows, thou shalt take, and go unto the place which the \LORD shall choose:
\verse And thou shalt offer thy burnt offerings, the flesh and the blood, upon the altar of the \LORD thy God: and the blood of thy sacrifices shall be poured out upon the altar of the \LORD thy God, and thou shalt eat the flesh.
\verse Observe and hear all these words which I command thee, that it may go well with thee, and with thy children after thee for ever, when thou doest that which is good and right in the sight of the \LORD thy God.
\verse When the \LORD thy God shall cut off the nations from before thee, whither thou goest to possess them, and thou succeedest them, and dwellest in their land;
\verse Take heed to thyself that thou be not snared by following them, after that they be destroyed from before thee; and that thou enquire not after their gods, saying, How did these nations serve their gods? even so will I do likewise.
\verse Thou shalt not do so unto the \LORD thy God: for every abomination to the \LORD, which he hateth, have they done unto their gods; for even their sons and their daughters they have burnt in the fire to their gods.
\verse What thing soever I command you, observe to do it: thou shalt not add thereto, nor diminish from it.
\end{biblechapter}

\begin{biblechapter} % Deuteronomy 13
\verseWithHeading{Worshipping other gods} If there arise among you a prophet, or a dreamer of dreams, and giveth thee a sign or a wonder,
\verse And the sign or the wonder come to pass, whereof he spake unto thee, saying, Let us go after other gods, which thou hast not known, and let us serve them;
\verse Thou shalt not hearken unto the words of that prophet, or that dreamer of dreams: for the \LORD your God proveth you, to know whether ye love the \LORD your God with all your heart and with all your soul.
\verse Ye shall walk after the \LORD your God, and fear him, and keep his commandments, and obey his voice, and ye shall serve him, and cleave unto him.
\verse And that prophet, or that dreamer of dreams, shall be put to death; because he hath spoken to turn you away from the \LORD your God, which brought you out of the land of Egypt, and redeemed you out of the house of bondage, to thrust thee out of the way which the \LORD thy God commanded thee to walk in. So shalt thou put the evil away from the midst of thee.
\verse If thy brother, the son of thy mother, or thy son, or thy daughter, or the wife of thy bosom, or thy friend, which is as thine own soul, entice thee secretly, saying, Let us go and serve other gods, which thou hast not known, thou, nor thy fathers;
\verse Namely, of the gods of the people which are round about you, nigh unto thee, or far off from thee, from the one end of the earth even unto the other end of the earth;
\verse Thou shalt not consent unto him, nor hearken unto him; neither shall thine eye pity him, neither shalt thou spare, neither shalt thou conceal him:
\verse But thou shalt surely kill him; thine hand shall be first upon him to put him to death, and afterwards the hand of all the people.
\verse And thou shalt stone him with stones, that he die; because he hath sought to thrust thee away from the \LORD thy God, which brought thee out of the land of Egypt, from the house of bondage.
\verse And all Israel shall hear, and fear, and shall do no more any such wickedness as this is among you.
\verse If thou shalt hear say in one of thy cities, which the \LORD thy God hath given thee to dwell there, saying,
\verse Certain men, the children of Belial, are gone out from among you, and have withdrawn the inhabitants of their city, saying, Let us go and serve other gods, which ye have not known;
\verse Then shalt thou enquire, and make search, and ask diligently; and, behold, if it be truth, and the thing certain, that such abomination is wrought among you;
\verse Thou shalt surely smite the inhabitants of that city with the edge of the sword, destroying it utterly, and all that is therein, and the cattle thereof, with the edge of the sword.
\verse And thou shalt gather all the spoil of it into the midst of the street thereof, and shalt burn with fire the city, and all the spoil thereof every whit, for the \LORD thy God: and it shall be an heap for ever; it shall not be built again.
\verse And there shall cleave nought of the cursed thing to thine hand: that the \LORD may turn from the fierceness of his anger, and shew thee mercy, and have compassion upon thee, and multiply thee, as he hath sworn unto thy fathers;
\verse When thou shalt hearken to the voice of the \LORD thy God, to keep all his commandments which I command thee this day, to do that which is right in the eyes of the \LORD thy God.
\end{biblechapter}

\begin{biblechapter} % Deuteronomy 14
\verseWithHeading{Clean and unclean food} Ye are the children of the \LORD your God: ye shall not cut yourselves, nor make any baldness between your eyes for the dead.
\verse For thou art an holy people unto the \LORD thy God, and the \LORD hath chosen thee to be a peculiar people unto himself, above all the nations that are upon the earth.
\verse Thou shalt not eat any abominable thing.
\verse These are the beasts which ye shall eat: the ox, the sheep, and the goat,
\verse The hart, and the roebuck, and the fallow deer, and the wild goat, and the pygarg, and the wild ox, and the chamois.
\verse And every beast that parteth the hoof, and cleaveth the cleft into two claws, and cheweth the cud among the beasts, that ye shall eat.
\verse Nevertheless these ye shall not eat of them that chew the cud, or of them that divide the cloven hoof; as the camel, and the hare, and the coney: for they chew the cud, but divide not the hoof; therefore they are unclean unto you.
\verse And the swine, because it divideth the hoof, yet cheweth not the cud, it is unclean unto you: ye shall not eat of their flesh, nor touch their dead carcase.
\verse These ye shall eat of all that are in the waters: all that have fins and scales shall ye eat:
\verse And whatsoever hath not fins and scales ye may not eat; it is unclean unto you.
\verse Of all clean birds ye shall eat.
\verse But these are they of which ye shall not eat: the eagle, and the ossifrage, and the ospray,
\verse And the glede, and the kite, and the vulture after his kind,
\verse And every raven after his kind,
\verse And the owl, and the night hawk, and the cuckow, and the hawk after his kind,
\verse The little owl, and the great owl, and the swan,
\verse And the pelican, and the gier eagle, and the cormorant,
\verse And the stork, and the heron after her kind, and the lapwing, and the bat.
\verse And every creeping thing that flieth is unclean unto you: they shall not be eaten.
\verse But of all clean fowls ye may eat.
\verse Ye shall not eat of any thing that dieth of itself: thou shalt give it unto the stranger that is in thy gates, that he may eat it; or thou mayest sell it unto an alien: for thou art an holy people unto the \LORD thy God. Thou shalt not seethe a kid in his mother's milk.
\verseWithHeading{Tithes} Thou shalt truly tithe all the increase of thy seed, that the field bringeth forth year by year.
\verse And thou shalt eat before the \LORD thy God, in the place which he shall choose to place his name there, the tithe of thy corn, of thy wine, and of thine oil, and the firstlings of thy herds and of thy flocks; that thou mayest learn to fear the \LORD thy God always.
\verse And if the way be too long for thee, so that thou art not able to carry it; or if the place be too far from thee, which the \LORD thy God shall choose to set his name there, when the \LORD thy God hath blessed thee:
\verse Then shalt thou turn it into money, and bind up the money in thine hand, and shalt go unto the place which the \LORD thy God shall choose:
\verse And thou shalt bestow that money for whatsoever thy soul lusteth after, for oxen, or for sheep, or for wine, or for strong drink, or for whatsoever thy soul desireth: and thou shalt eat there before the \LORD thy God, and thou shalt rejoice, thou, and thine household,
\verse And the Levite that is within thy gates; thou shalt not forsake him; for he hath no part nor inheritance with thee.
\verse At the end of three years thou shalt bring forth all the tithe of thine increase the same year, and shalt lay it up within thy gates:
\verse And the Levite, (because he hath no part nor inheritance with thee,) and the stranger, and the fatherless, and the widow, which are within thy gates, shall come, and shall eat and be satisfied; that the \LORD thy God may bless thee in all the work of thine hand which thou doest.
\end{biblechapter}

\begin{biblechapter} % Deuteronomy 15
\verseWithHeading{The year for releasing debts} At the end of every seven years thou shalt make a release.
\verse And this is the manner of the release: Every creditor that lendeth ought unto his neighbour shall release it; he shall not exact it of his neighbour, or of his brother; because it is called the \LORDs release.
\verse Of a foreigner thou mayest exact it again: but that which is thine with thy brother thine hand shall release;
\verse Save when there shall be no poor among you; for the \LORD shall greatly bless thee in the land which the \LORD thy God giveth thee for an inheritance to possess it:
\verse Only if thou carefully hearken unto the voice of the \LORD thy God, to observe to do all these commandments which I command thee this day.
\verse For the \LORD thy God blesseth thee, as he promised thee: and thou shalt lend unto many nations, but thou shalt not borrow; and thou shalt reign over many nations, but they shall not reign over thee.
\verse If there be among you a poor man of one of thy brethren within any of thy gates in thy land which the \LORD thy God giveth thee, thou shalt not harden thine heart, nor shut thine hand from thy poor brother:
\verse But thou shalt open thine hand wide unto him, and shalt surely lend him sufficient for his need, in that which he wanteth.
\verse Beware that there be not a thought in thy wicked heart, saying, The seventh year, the year of release, is at hand; and thine eye be evil against thy poor brother, and thou givest him nought; and he cry unto the \LORD against thee, and it be sin unto thee.
\verse Thou shalt surely give him, and thine heart shall not be grieved when thou givest unto him: because that for this thing the \LORD thy God shall bless thee in all thy works, and in all that thou puttest thine hand unto.
\verse For the poor shall never cease out of the land: therefore I command thee, saying, Thou shalt open thine hand wide unto thy brother, to thy poor, and to thy needy, in thy land.
\verseWithHeading{Freeing servants} And if thy brother, an Hebrew man, or an Hebrew woman, be sold unto thee, and serve thee six years; then in the seventh year thou shalt let him go free from thee.
\verse And when thou sendest him out free from thee, thou shalt not let him go away empty:
\verse Thou shalt furnish him liberally out of thy flock, and out of thy floor, and out of thy winepress: of that wherewith the \LORD thy God hath blessed thee thou shalt give unto him.
\verse And thou shalt remember that thou wast a bondman in the land of Egypt, and the \LORD thy God redeemed thee: therefore I command thee this thing to day.
\verse And it shall be, if he say unto thee, I will not go away from thee; because he loveth thee and thine house, because he is well with thee;
\verse Then thou shalt take an aul, and thrust it through his ear unto the door, and he shall be thy servant for ever. And also unto thy maidservant thou shalt do likewise.
\verse It shall not seem hard unto thee, when thou sendest him away free from thee; for he hath been worth a double hired servant to thee, in serving thee six years: and the \LORD thy God shall bless thee in all that thou doest.
\verseWithHeading{The firstborn animals} All the firstling males that come of thy herd and of thy flock thou shalt sanctify unto the \LORD thy God: thou shalt do no work with the firstling of thy bullock, nor shear the firstling of thy sheep.
\verse Thou shalt eat it before the \LORD thy God year by year in the place which the \LORD shall choose, thou and thy household.
\verse And if there be any blemish therein, as if it be lame, or blind, or have any ill blemish, thou shalt not sacrifice it unto the \LORD thy God.
\verse Thou shalt eat it within thy gates: the unclean and the clean person shall eat it alike, as the roebuck, and as the hart.
\verse Only thou shalt not eat the blood thereof; thou shalt pour it upon the ground as water.
\end{biblechapter}

\begin{biblechapter} % Deuteronomy 16
\verseWithHeading{The Passover} Observe the month of Abib, and keep the Passover unto the \LORD thy God: for in the month of Abib the \LORD thy God brought thee forth out of Egypt by night.
\verse Thou shalt therefore sacrifice the Passover unto the \LORD thy God, of the flock and the herd, in the place which the \LORD shall choose to place his name there.
\verse Thou shalt eat no leavened bread with it; seven days shalt thou eat unleavened bread therewith, even the bread of affliction; for thou camest forth out of the land of Egypt in haste: that thou mayest remember the day when thou camest forth out of the land of Egypt all the days of thy life.
\verse And there shall be no leavened bread seen with thee in all thy coast seven days; neither shall there any thing of the flesh, which thou sacrificedst the first day at even, remain all night until the morning.
\verse Thou mayest not sacrifice the Passover within any of thy gates, which the \LORD thy God giveth thee:
\verse But at the place which the \LORD thy God shall choose to place his name in, there thou shalt sacrifice the Passover at even, at the going down of the sun, at the season that thou camest forth out of Egypt.
\verse And thou shalt roast and eat it in the place which the \LORD thy God shall choose: and thou shalt turn in the morning, and go unto thy tents.
\verse Six days thou shalt eat unleavened bread: and on the seventh day shall be a solemn assembly to the \LORD thy God: thou shalt do no work therein.
\verseWithHeading{The Feast of Weeks} Seven weeks shalt thou number unto thee: begin to number the seven weeks from such time as thou beginnest to put the sickle to the corn.
\verse And thou shalt keep the feast of weeks unto the \LORD thy God with a tribute of a freewill offering of thine hand, which thou shalt give unto the \LORD thy God, according as the \LORD thy God hath blessed thee:
\verse And thou shalt rejoice before the \LORD thy God, thou, and thy son, and thy daughter, and thy manservant, and thy maidservant, and the Levite that is within thy gates, and the stranger, and the fatherless, and the widow, that are among you, in the place which the \LORD thy God hath chosen to place his name there.
\verse And thou shalt remember that thou wast a bondman in Egypt: and thou shalt observe and do these statutes.
\verseWithHeading{The Feast of Tabernacles} Thou shalt observe the feast of tabernacles seven days, after that thou hast gathered in thy corn and thy wine:
\verse And thou shalt rejoice in thy feast, thou, and thy son, and thy daughter, and thy manservant, and thy maidservant, and the Levite, the stranger, and the fatherless, and the widow, that are within thy gates.
\verse Seven days shalt thou keep a solemn feast unto the \LORD thy God in the place which the \LORD shall choose: because the \LORD thy God shall bless thee in all thine increase, and in all the works of thine hands, therefore thou shalt surely rejoice.
\verse Three times in a year shall all thy males appear before the \LORD thy God in the place which he shall choose; in the feast of unleavened bread, and in the feast of weeks, and in the feast of tabernacles: and they shall not appear before the \LORD empty:
\verse Every man shall give as he is able, according to the blessing of the \LORD thy God which he hath given thee.
\verseWithHeading{Judges} Judges and officers shalt thou make thee in all thy gates, which the \LORD thy God giveth thee, throughout thy tribes: and they shall judge the people with just judgment.
\verse Thou shalt not wrest judgment; thou shalt not respect persons, neither take a gift: for a gift doth blind the eyes of the wise, and pervert the words of the righteous.
\verse That which is altogether just shalt thou follow, that thou mayest live, and inherit the land which the \LORD thy God giveth thee.
\verseWithHeading{Worshipping other gods} Thou shalt not plant thee a grove of any trees near unto the altar of the \LORD thy God, which thou shalt make thee.
\verse Neither shalt thou set thee up any image; which the \LORD thy God hateth.
\end{biblechapter}

\begin{biblechapter} % Deuteronomy 17
\verse Thou shalt not sacrifice unto the \LORD thy God any bullock, or sheep, wherein is blemish, or any evilfavouredness: for that is an abomination unto the \LORD thy God.
\verse If there be found among you, within any of thy gates which the \LORD thy God giveth thee, man or woman, that hath wrought wickedness in the sight of the \LORD thy God, in transgressing his covenant,
\verse And hath gone and served other gods, and worshipped them, either the sun, or moon, or any of the host of heaven, which I have not commanded;
\verse And it be told thee, and thou hast heard of it, and enquired diligently, and, behold, it be true, and the thing certain, that such abomination is wrought in Israel:
\verse Then shalt thou bring forth that man or that woman, which have committed that wicked thing, unto thy gates, even that man or that woman, and shalt stone them with stones, till they die.
\verse At the mouth of two witnesses, or three witnesses, shall he that is worthy of death be put to death; but at the mouth of one witness he shall not be put to death.
\verse The hands of the witnesses shall be first upon him to put him to death, and afterward the hands of all the people. So thou shalt put the evil away from among you.
\verseWithHeading{The law courts} If there arise a matter too hard for thee in judgment, between blood and blood, between plea and plea, and between stroke and stroke, being matters of controversy within thy gates: then shalt thou arise, and get thee up into the place which the \LORD thy God shall choose;
\verse And thou shalt come unto the priests the Levites, and unto the judge that shall be in those days, and enquire; and they shall shew thee the sentence of judgment:
\verse And thou shalt do according to the sentence, which they of that place which the \LORD shall choose shall shew thee; and thou shalt observe to do according to all that they inform thee:
\verse According to the sentence of the law which they shall teach thee, and according to the judgment which they shall tell thee, thou shalt do: thou shalt not decline from the sentence which they shall shew thee, to the right hand, nor to the left.
\verse And the man that will do presumptuously, and will not hearken unto the priest that standeth to minister there before the \LORD thy God, or unto the judge, even that man shall die: and thou shalt put away the evil from Israel.
\verse And all the people shall hear, and fear, and do no more presumptuously.
\verseWithHeading{The king} When thou art come unto the land which the \LORD thy God giveth thee, and shalt possess it, and shalt dwell therein, and shalt say, I will set a king over me, like as all the nations that are about me;
\verse Thou shalt in any wise set him king over thee, whom the \LORD thy God shall choose: one from among thy brethren shalt thou set king over thee: thou mayest not set a stranger over thee, which is not thy brother.
\verse But he shall not multiply horses to himself, nor cause the people to return to Egypt, to the end that he should multiply horses: forasmuch as the \LORD hath said unto you, Ye shall henceforth return no more that way.
\verse Neither shall he multiply wives to himself, that his heart turn not away: neither shall he greatly multiply to himself silver and gold.
\verse And it shall be, when he sitteth upon the throne of his kingdom, that he shall write him a copy of this law in a book out of that which is before the priests the Levites:
\verse And it shall be with him, and he shall read therein all the days of his life: that he may learn to fear the \LORD his God, to keep all the words of this law and these statutes, to do them:
\verse That his heart be not lifted up above his brethren, and that he turn not aside from the commandment, to the right hand, or to the left: to the end that he may prolong his days in his kingdom, he, and his children, in the midst of Israel.
\end{biblechapter}

\begin{biblechapter} % Deuteronomy 18
\verseWithHeading{Offerings for priests and \newline Levites} The priests the Levites, and all the tribe of Levi, shall have no part nor inheritance with Israel: they shall eat the offerings of the \LORD made by fire, and his inheritance.
\verse Therefore shall they have no inheritance among their brethren: the \LORD is their inheritance, as he hath said unto them.
\verse And this shall be the priest's due from the people, from them that offer a sacrifice, whether it be ox or sheep; and they shall give unto the priest the shoulder, and the two cheeks, and the maw.
\verse The firstfruit also of thy corn, of thy wine, and of thine oil, and the first of the fleece of thy sheep, shalt thou give him.
\verse For the \LORD thy God hath chosen him out of all thy tribes, to stand to minister in the name of the \LORD, him and his sons for ever.
\verse And if a Levite come from any of thy gates out of all Israel, where he sojourned, and come with all the desire of his mind unto the place which the \LORD shall choose;
\verse Then he shall minister in the name of the \LORD his God, as all his brethren the Levites do, which stand there before the \LORD.
\verse They shall have like portions to eat, beside that which cometh of the sale of his patrimony.
\verseWithHeading{Occult practices} When thou art come into the land which the \LORD thy God giveth thee, thou shalt not learn to do after the abominations of those nations.
\verse There shall not be found among you any one that maketh his son or his daughter to pass through the fire, or that useth divination, or an observer of times, or an enchanter, or a witch,
\verse Or a charmer, or a consulter with familiar spirits, or a wizard, or a necromancer.
\verse For all that do these things are an abomination unto the \LORD: and because of these abominations the \LORD thy God doth drive them out from before thee.
\verse Thou shalt be perfect with the \LORD thy God.
\verseWithHeading{The prophet} For these nations, which thou shalt possess, hearkened unto observers of times, and unto diviners: but as for thee, the \LORD thy God hath not suffered thee so to do.
\verse The \LORD thy God will raise up unto thee a Prophet from the midst of thee, of thy brethren, like unto me; unto him ye shall hearken;
\verse According to all that thou desiredst of the \LORD thy God in Horeb in the day of the assembly, saying, Let me not hear again the voice of the \LORD my God, neither let me see this great fire any more, that I die not.
\verse And the \LORD said unto me, They have well spoken that which they have spoken.
\verse I will raise them up a Prophet from among their brethren, like unto thee, and will put my words in his mouth; and he shall speak unto them all that I shall command him.
\verse And it shall come to pass, that whosoever will not hearken unto my words which he shall speak in my name, I will require it of him.
\verse But the prophet, which shall presume to speak a word in my name, which I have not commanded him to speak, or that shall speak in the name of other gods, even that prophet shall die.
\verse And if thou say in thine heart, How shall we know the word which the \LORD hath not spoken?
\verse When a prophet speaketh in the name of the \LORD, if the thing follow not, nor come to pass, that is the thing which the \LORD hath not spoken, but the prophet hath spoken it presumptuously: thou shalt not be afraid of him.
\end{biblechapter}

\begin{biblechapter} % Deuteronomy 19
\verseWithHeading{Cities of refuge} When the \LORD thy God hath cut off the nations, whose land the \LORD thy God giveth thee, and thou succeedest them, and dwellest in their cities, and in their houses;
\verse Thou shalt separate three cities for thee in the midst of thy land, which the \LORD thy God giveth thee to possess it.
\verse Thou shalt prepare thee a way, and divide the coasts of thy land, which the \LORD thy God giveth thee to inherit, into three parts, that every slayer may flee thither.
\verse And this is the case of the slayer, which shall flee thither, that he may live: Whoso killeth his neighbour ignorantly, whom he hated not in time past;
\verse As when a man goeth into the wood with his neighbour to hew wood, and his hand fetcheth a stroke with the axe to cut down the tree, and the head slippeth from the helve, and lighteth upon his neighbour, that he die; he shall flee unto one of those cities, and live:
\verse Lest the avenger of the blood pursue the slayer, while his heart is hot, and overtake him, because the way is long, and slay him; whereas he was not worthy of death, inasmuch as he hated him not in time past.
\verse Wherefore I command thee, saying, Thou shalt separate three cities for thee.
\verse And if the \LORD thy God enlarge thy coast, as he hath sworn unto thy fathers, and give thee all the land which he promised to give unto thy fathers;
\verse If thou shalt keep all these commandments to do them, which I command thee this day, to love the \LORD thy God, and to walk ever in his ways; then shalt thou add three cities more for thee, beside these three:
\verse That innocent blood be not shed in thy land, which the \LORD thy God giveth thee for an inheritance, and so blood be upon thee.
\verse But if any man hate his neighbour, and lie in wait for him, and rise up against him, and smite him mortally that he die, and fleeth into one of these cities:
\verse Then the elders of his city shall send and fetch him thence, and deliver him into the hand of the avenger of blood, that he may die.
\verse Thine eye shall not pity him, but thou shalt put away the guilt of innocent blood from Israel, that it may go well with thee.
\verse Thou shalt not remove thy neighbour's landmark, which they of old time have set in thine inheritance, which thou shalt inherit in the land that the \LORD thy God giveth thee to possess it.
\verseWithHeading{Witnesses} One witness shall not rise up against a man for any iniquity, or for any sin, in any sin that he sinneth: at the mouth of two witnesses, or at the mouth of three witnesses, shall the matter be established.
\verse If a false witness rise up against any man to testify against him that which is wrong;
\verse Then both the men, between whom the controversy is, shall stand before the \LORD, before the priests and the judges, which shall be in those days;
\verse And the judges shall make diligent inquisition: and, behold, if the witness be a false witness, and hath testified falsely against his brother;
\verse Then shall ye do unto him, as he had thought to have done unto his brother: so shalt thou put the evil away from among you.
\verse And those which remain shall hear, and fear, and shall henceforth commit no more any such evil among you.
\verse And thine eye shall not pity; but life shall go for life, eye for eye, tooth for tooth, hand for hand, foot for foot.
\end{biblechapter}

\begin{biblechapter} % Deuteronomy 20
\verseWithHeading{Going to war} When thou goest out to battle against thine enemies, and seest horses, and chariots, and a people more than thou, be not afraid of them: for the \LORD thy God is with thee, which brought thee up out of the land of Egypt.
\verse And it shall be, when ye are come nigh unto the battle, that the priest shall approach and speak unto the people,
\verse And shall say unto them, Hear, O Israel, ye approach this day unto battle against your enemies: let not your hearts faint, fear not, and do not tremble, neither be ye terrified because of them;
\verse For the \LORD your God is he that goeth with you, to fight for you against your enemies, to save you.
\verse And the officers shall speak unto the people, saying, What man is there that hath built a new house, and hath not dedicated it? let him go and return to his house, lest he die in the battle, and another man dedicate it.
\verse And what man is he that hath planted a vineyard, and hath not yet eaten of it? let him also go and return unto his house, lest he die in the battle, and another man eat of it.
\verse And what man is there that hath betrothed a wife, and hath not taken her? let him go and return unto his house, lest he die in the battle, and another man take her.
\verse And the officers shall speak further unto the people, and they shall say, What man is there that is fearful and fainthearted? let him go and return unto his house, lest his brethren's heart faint as well as his heart.
\verse And it shall be, when the officers have made an end of speaking unto the people, that they shall make captains of the armies to lead the people.
\verse When thou comest nigh unto a city to fight against it, then proclaim peace unto it.
\verse And it shall be, if it make thee answer of peace, and open unto thee, then it shall be, that all the people that is found therein shall be tributaries unto thee, and they shall serve thee.
\verse And if it will make no peace with thee, but will make war against thee, then thou shalt besiege it:
\verse And when the \LORD thy God hath delivered it into thine hands, thou shalt smite every male thereof with the edge of the sword:
\verse But the women, and the little ones, and the cattle, and all that is in the city, even all the spoil thereof, shalt thou take unto thyself; and thou shalt eat the spoil of thine enemies, which the \LORD thy God hath given thee.
\verse Thus shalt thou do unto all the cities which are very far off from thee, which are not of the cities of these nations.
\verse But of the cities of these people, which the \LORD thy God doth give thee for an inheritance, thou shalt save alive nothing that breatheth:
\verse But thou shalt utterly destroy them; namely, the Hittites, and the Amorites, the Canaanites, and the Perizzites, the Hivites, and the Jebusites; as the \LORD thy God hath commanded thee:
\verse That they teach you not to do after all their abominations, which they have done unto their gods; so should ye sin against the \LORD your God.
\verse When thou shalt besiege a city a long time, in making war against it to take it, thou shalt not destroy the trees thereof by forcing an axe against them: for thou mayest eat of them, and thou shalt not cut them down (for the tree of the field is man's life) to employ them in the siege:
\verse Only the trees which thou knowest that they be not trees for meat, thou shalt destroy and cut them down; and thou shalt build bulwarks against the city that maketh war with thee, until it be subdued.
\end{biblechapter}

\begin{biblechapter} % Deuteronomy 21
\verseWithHeading{Atonement for an unsolved \newline murder} If one be found slain in the land which the \LORD thy God giveth thee to possess it, lying in the field, and it be not known who hath slain him:
\verse Then thy elders and thy judges shall come forth, and they shall measure unto the cities which are round about him that is slain:
\verse And it shall be, that the city which is next unto the slain man, even the elders of that city shall take an heifer, which hath not been wrought with, and which hath not drawn in the yoke;
\verse And the elders of that city shall bring down the heifer unto a rough valley, which is neither eared nor sown, and shall strike off the heifer's neck there in the valley:
\verse And the priests the sons of Levi shall come near; for them the \LORD thy God hath chosen to minister unto him, and to bless in the name of the \LORD; and by their word shall every controversy and every stroke be tried:
\verse And all the elders of that city, that are next unto the slain man, shall wash their hands over the heifer that is beheaded in the valley:
\verse And they shall answer and say, Our hands have not shed this blood, neither have our eyes seen it.
\verse Be merciful, O \LORD, unto thy people Israel, whom thou hast redeemed, and lay not innocent blood unto thy people of Israel's charge. And the blood shall be forgiven them.
\verse So shalt thou put away the guilt of innocent blood from among you, when thou shalt do that which is right in the sight of the \LORD.
\verseWithHeading{Marrying a captive woman} When thou goest forth to war against thine enemies, and the \LORD thy God hath delivered them into thine hands, and thou hast taken them captive,
\verse And seest among the captives a beautiful woman, and hast a desire unto her, that thou wouldest have her to thy wife;
\verse Then thou shalt bring her home to thine house; and she shall shave her head, and pare her nails;
\verse And she shall put the raiment of her captivity from off her, and shall remain in thine house, and bewail her father and her mother a full month: and after that thou shalt go in unto her, and be her husband, and she shall be thy wife.
\verse And it shall be, if thou have no delight in her, then thou shalt let her go whither she will; but thou shalt not sell her at all for money, thou shalt not make merchandise of her, because thou hast humbled her.
\verseWithHeading{The right of the firstborn} If a man have two wives, one beloved, and another hated, and they have born him children, both the beloved and the hated; and if the firstborn son be hers that was hated:
\verse Then it shall be, when he maketh his sons to inherit that which he hath, that he may not make the son of the beloved firstborn before the son of the hated, which is indeed the firstborn:
\verse But he shall acknowledge the son of the hated for the firstborn, by giving him a double portion of all that he hath: for he is the beginning of his strength; the right of the firstborn is his.
\verseWithHeading{A rebellious son} If a man have a stubborn and rebellious son, which will not obey the voice of his father, or the voice of his mother, and that, when they have chastened him, will not hearken unto them:
\verse Then shall his father and his mother lay hold on him, and bring him out unto the elders of his city, and unto the gate of his place;
\verse And they shall say unto the elders of his city, This our son is stubborn and rebellious, he will not obey our voice; he is a glutton, and a drunkard.
\verse And all the men of his city shall stone him with stones, that he die: so shalt thou put evil away from among you; and all Israel shall hear, and fear.
\verseWithHeading{Various laws} And if a man have committed a sin worthy of death, and he be to be put to death, and thou hang him on a tree:
\verse His body shall not remain all night upon the tree, but thou shalt in any wise bury him that day; (for he that is hanged is accursed of God;) that thy land be not defiled, which the \LORD thy God giveth thee for an inheritance.
\end{biblechapter}

\begin{biblechapter} % Deuteronomy 22
\verse Thou shalt not see thy brother's ox or his sheep go astray, and hide thyself from them: thou shalt in any case bring them again unto thy brother.
\verse And if thy brother be not nigh unto thee, or if thou know him not, then thou shalt bring it unto thine own house, and it shall be with thee until thy brother seek after it, and thou shalt restore it to him again.
\verse In like manner shalt thou do with his ass; and so shalt thou do with his raiment; and with all lost thing of thy brother's, which he hath lost, and thou hast found, shalt thou do likewise: thou mayest not hide thyself.
\verse Thou shalt not see thy brother's ass or his ox fall down by the way, and hide thyself from them: thou shalt surely help him to lift them up again.
\verse The woman shall not wear that which pertaineth unto a man, neither shall a man put on a woman's garment: for all that do so are abomination unto the \LORD thy God.
\verse If a bird's nest chance to be before thee in the way in any tree, or on the ground, whether they be young ones, or eggs, and the dam sitting upon the young, or upon the eggs, thou shalt not take the dam with the young:
\verse But thou shalt in any wise let the dam go, and take the young to thee; that it may be well with thee, and that thou mayest prolong thy days.
\verse When thou buildest a new house, then thou shalt make a battlement for thy roof, that thou bring not blood upon thine house, if any man fall from thence.
\verse Thou shalt not sow thy vineyard with divers seeds: lest the fruit of thy seed which thou hast sown, and the fruit of thy vineyard, be defiled.
\verse Thou shalt not plow with an ox and an ass together.
\verse Thou shalt not wear a garment of divers sorts, as of woollen and linen together.
\verse Thou shalt make thee fringes upon the four quarters of thy vesture, wherewith thou coverest thyself.
\verseWithHeading{Marriage violations} If any man take a wife, and go in unto her, and hate her,
\verse And give occasions of speech against her, and bring up an evil name upon her, and say, I took this woman, and when I came to her, I found her not a maid:
\verse Then shall the father of the damsel, and her mother, take and bring forth the tokens of the damsel's virginity unto the elders of the city in the gate:
\verse And the damsel's father shall say unto the elders, I gave my daughter unto this man to wife, and he hateth her;
\verse And, lo, he hath given occasions of speech against her, saying, I found not thy daughter a maid; and yet these are the tokens of my daughter's virginity. And they shall spread the cloth before the elders of the city.
\verse And the elders of that city shall take that man and chastise him;
\verse And they shall amerce him in an hundred shekels of silver, and give them unto the father of the damsel, because he hath brought up an evil name upon a virgin of Israel: and she shall be his wife; he may not put her away all his days.
\verse But if this thing be true, and the tokens of virginity be not found for the damsel:
\verse Then they shall bring out the damsel to the door of her father's house, and the men of her city shall stone her with stones that she die: because she hath wrought folly in Israel, to play the whore in her father's house: so shalt thou put evil away from among you.
\verse If a man be found lying with a woman married to an husband, then they shall both of them die, both the man that lay with the woman, and the woman: so shalt thou put away evil from Israel.
\verse If a damsel that is a virgin be betrothed unto an husband, and a man find her in the city, and lie with her;
\verse Then ye shall bring them both out unto the gate of that city, and ye shall stone them with stones that they die; the damsel, because she cried not, being in the city; and the man, because he hath humbled his neighbour's wife: so thou shalt put away evil from among you.
\verse But if a man find a betrothed damsel in the field, and the man force her, and lie with her: then the man only that lay with her shall die:
\verse But unto the damsel thou shalt do nothing; there is in the damsel no sin worthy of death: for as when a man riseth against his neighbour, and slayeth him, even so is this matter:
\verse For he found her in the field, and the betrothed damsel cried, and there was none to save her.
\verse If a man find a damsel that is a virgin, which is not betrothed, and lay hold on her, and lie with her, and they be found;
\verse Then the man that lay with her shall give unto the damsel's father fifty shekels of silver, and she shall be his wife; because he hath humbled her, he may not put her away all his days.
\verse A man shall not take his father's wife, nor discover his father's skirt.
\end{biblechapter}

\begin{biblechapter} % Deuteronomy 23
\verseWithHeading{Exclusion from the \newline congregation} He that is wounded in the stones, or hath his privy member cut off, shall not enter into the congregation of the \LORD.
\verse A bastard shall not enter into the congregation of the \LORD; even to his tenth generation shall he not enter into the congregation of the \LORD.
\verse An Ammonite or Moabite shall not enter into the congregation of the \LORD; even to their tenth generation shall they not enter into the congregation of the \LORD for ever:
\verse Because they met you not with bread and with water in the way, when ye came forth out of Egypt; and because they hired against thee Balaam the son of Beor of Pethor of Mesopotamia, to curse thee.
\verse Nevertheless the \LORD thy God would not hearken unto Balaam; but the \LORD thy God turned the curse into a blessing unto thee, because the \LORD thy God loved thee.
\verse Thou shalt not seek their peace nor their prosperity all thy days for ever.
\verse Thou shalt not abhor an Edomite; for he is thy brother: thou shalt not abhor an Egyptian; because thou wast a stranger in his land.
\verse The children that are begotten of them shall enter into the congregation of the \LORD in their third generation.
\verseWithHeading{Uncleanliness in the camp} When the host goeth forth against thine enemies, then keep thee from every wicked thing.
\verse If there be among you any man, that is not clean by reason of uncleanness that chanceth him by night, then shall he go abroad out of the camp, he shall not come within the camp:
\verse But it shall be, when evening cometh on, he shall wash himself with water: and when the sun is down, he shall come into the camp again.
\verse Thou shalt have a place also without the camp, whither thou shalt go forth abroad:
\verse And thou shalt have a paddle upon thy weapon; and it shall be, when thou wilt ease thyself abroad, thou shalt dig therewith, and shalt turn back and cover that which cometh from thee:
\verse For the \LORD thy God walketh in the midst of thy camp, to deliver thee, and to give up thine enemies before thee; therefore shall thy camp be holy: that he see no unclean thing in thee, and turn away from thee.
\flushcolsend\columnbreak % layout hack
\verseWithHeading{Various laws} Thou shalt not deliver unto his master the servant which is escaped from his master unto thee:
\verse He shall dwell with thee, even among you, in that place which he shall choose in one of thy gates, where it liketh him best: thou shalt not oppress him.
\verse There shall be no whore of the daughters of Israel, nor a sodomite of the sons of Israel.
\verse Thou shalt not bring the hire of a whore, or the price of a dog, into the house of the \LORD thy God for any vow: for even both these are abomination unto the \LORD thy God.
\verse Thou shalt not lend upon usury to thy brother; usury of money, usury of victuals, usury of any thing that is lent upon usury:
\verse Unto a stranger thou mayest lend upon usury; but unto thy brother thou shalt not lend upon usury: that the \LORD thy God may bless thee in all that thou settest thine hand to in the land whither thou goest to possess it.
\verse When thou shalt vow a vow unto the \LORD thy God, thou shalt not slack to pay it: for the \LORD thy God will surely require it of thee; and it would be sin in thee.
\verse But if thou shalt forbear to vow, it shall be no sin in thee.
\verse That which is gone out of thy lips thou shalt keep and perform; even a freewill offering, according as thou hast vowed unto the \LORD thy God, which thou hast promised with thy mouth.
\verse When thou comest into thy neighbour's vineyard, then thou mayest eat grapes thy fill at thine own pleasure; but thou shalt not put any in thy vessel.
\verse When thou comest into the standing corn of thy neighbour, then thou mayest pluck the ears with thine hand; but thou shalt not move a sickle unto thy neighbour's standing corn.
\end{biblechapter}

\begin{biblechapter} % Deuteronomy 24
\verse When a man hath taken a wife, and married her, and it come to pass that she find no favour in his eyes, because he hath found some uncleanness in her: then let him write her a bill of divorcement, and give it in her hand, and send her out of his house.
\verse And when she is departed out of his house, she may go and be another man's wife.
\verse And if the latter husband hate her, and write her a bill of divorcement, and giveth it in her hand, and sendeth her out of his house; or if the latter husband die, which took her to be his wife;
\verse Her former husband, which sent her away, may not take her again to be his wife, after that she is defiled; for that is abomination before the \LORD: and thou shalt not cause the land to sin, which the \LORD thy God giveth thee for an inheritance.
\verse When a man hath taken a new wife, he shall not go out to war, neither shall he be charged with any business: but he shall be free at home one year, and shall cheer up his wife which he hath taken.
\verse No man shall take the nether or the upper millstone to pledge: for he taketh a man's life to pledge.
\verse If a man be found stealing any of his brethren of the children of Israel, and maketh merchandise of him, or selleth him; then that thief shall die; and thou shalt put evil away from among you.
\verse Take heed in the plague of leprosy, that thou observe diligently, and do according to all that the priests the Levites shall teach you: as I commanded them, so ye shall observe to do.
\verse Remember what the \LORD thy God did unto Miriam by the way, after that ye were come forth out of Egypt.
\verse When thou dost lend thy brother any thing, thou shalt not go into his house to fetch his pledge.
\verse Thou shalt stand abroad, and the man to whom thou dost lend shall bring out the pledge abroad unto thee.
\verse And if the man be poor, thou shalt not sleep with his pledge:
\verse In any case thou shalt deliver him the pledge again when the sun goeth down, that he may sleep in his own raiment, and bless thee: and it shall be righteousness unto thee before the \LORD thy God.
\verse Thou shalt not oppress an hired servant that is poor and needy, whether he be of thy brethren, or of thy strangers that are in thy land within thy gates:
\verse At his day thou shalt give him his hire, neither shall the sun go down upon it; for he is poor, and setteth his heart upon it: lest he cry against thee unto the \LORD, and it be sin unto thee.
\verse The fathers shall not be put to death for the children, neither shall the children be put to death for the fathers: every man shall be put to death for his own sin.
\verse Thou shalt not pervert the judgment of the stranger, nor of the fatherless; nor take a widow's raiment to pledge:
\verse But thou shalt remember that thou wast a bondman in Egypt, and the \LORD thy God redeemed thee thence: therefore I command thee to do this thing.
\verse When thou cuttest down thine harvest in thy field, and hast forgot a sheaf in the field, thou shalt not go again to fetch it: it shall be for the stranger, for the fatherless, and for the widow: that the \LORD thy God may bless thee in all the work of thine hands.
\verse When thou beatest thine olive tree, thou shalt not go over the boughs again: it shall be for the stranger, for the fatherless, and for the widow.
\verse When thou gatherest the grapes of thy vineyard, thou shalt not glean it afterward: it shall be for the stranger, for the fatherless, and for the widow.
\verse And thou shalt remember that thou wast a bondman in the land of Egypt: therefore I command thee to do this thing.
\end{biblechapter}

\begin{biblechapter} % Deuteronomy 25
\verse If there be a controversy between men, and they come unto judgment, that the judges may judge them; then they shall justify the righteous, and condemn the wicked.
\verse And it shall be, if the wicked man be worthy to be beaten, that the judge shall cause him to lie down, and to be beaten before his face, according to his fault, by a certain number.
\verse Forty stripes he may give him, and not exceed: lest, if he should exceed, and beat him above these with many stripes, then thy brother should seem vile unto thee.
\verse Thou shalt not muzzle the ox when he treadeth out the corn.
\verse If brethren dwell together, and one of them die, and have no child, the wife of the dead shall not marry without unto a stranger: her husband's brother shall go in unto her, and take her to him to wife, and perform the duty of an husband's brother unto her.
\verse And it shall be, that the firstborn which she beareth shall succeed in the name of his brother which is dead, that his name be not put out of Israel.
\verse And if the man like not to take his brother's wife, then let his brother's wife go up to the gate unto the elders, and say, My husband's brother refuseth to raise up unto his brother a name in Israel, he will not perform the duty of my husband's brother.
\verse Then the elders of his city shall call him, and speak unto him: and if he stand to it, and say, I like not to take her;
\verse Then shall his brother's wife come unto him in the presence of the elders, and loose his shoe from off his foot, and spit in his face, and shall answer and say, So shall it be done unto that man that will not build up his brother's house.
\verse And his name shall be called in Israel, The house of him that hath his shoe loosed.
\verse When men strive together one with another, and the wife of the one draweth near for to deliver her husband out of the hand of him that smiteth him, and putteth forth her hand, and taketh him by the secrets:
\verse Then thou shalt cut off her hand, thine eye shall not pity her.
\verse Thou shalt not have in thy bag divers weights, a great and a small.
\verse Thou shalt not have in thine house divers measures, a great and a small.
\verse But thou shalt have a perfect and just weight, a perfect and just measure shalt thou have: that thy days may be lengthened in the land which the \LORD thy God giveth thee.
\verse For all that do such things, and all that do unrighteously, are an abomination unto the \LORD thy God.
\verse Remember what Amalek did unto thee by the way, when ye were come forth out of Egypt;
\verse How he met thee by the way, and smote the hindmost of thee, even all that were feeble behind thee, when thou wast faint and weary; and he feared not God.
\verse Therefore it shall be, when the \LORD thy God hath given thee rest from all thine enemies round about, in the land which the \LORD thy God giveth thee for an inheritance to possess it, that thou shalt blot out the remembrance of Amalek from under heaven; thou shalt not forget it.
\end{biblechapter}

\begin{biblechapter} % Deuteronomy 26
\verseWithHeading{Firstfruits and tithes} And it shall be, when thou art come in unto the land which the \LORD thy God giveth thee for an inheritance, and possessest it, and dwellest therein;
\verse That thou shalt take of the first of all the fruit of the earth, which thou shalt bring of thy land that the \LORD thy God giveth thee, and shalt put it in a basket, and shalt go unto the place which the \LORD thy God shall choose to place his name there.
\verse And thou shalt go unto the priest that shall be in those days, and say unto him, I profess this day unto the \LORD thy God, that I am come unto the country which the \LORD sware unto our fathers for to give us.
\verse And the priest shall take the basket out of thine hand, and set it down before the altar of the \LORD thy God.
\verse And thou shalt speak and say before the \LORD thy God, A Syrian ready to perish was my father, and he went down into Egypt, and sojourned there with a few, and became there a nation, great, mighty, and populous:
\verse And the Egyptians evil entreated us, and afflicted us, and laid upon us hard bondage:
\verse And when we cried unto the \LORD God of our fathers, the \LORD heard our voice, and looked on our affliction, and our labour, and our oppression:
\verse And the \LORD brought us forth out of Egypt with a mighty hand, and with an outstretched arm, and with great terribleness, and with signs, and with wonders:
\verse And he hath brought us into this place, and hath given us this land, even a land that floweth with milk and honey.
\verse And now, behold, I have brought the firstfruits of the land, which thou, O \LORD, hast given me. And thou shalt set it before the \LORD thy God, and worship before the \LORD thy God:
\verse And thou shalt rejoice in every good thing which the \LORD thy God hath given unto thee, and unto thine house, thou, and the Levite, and the stranger that is among you.
\verse When thou hast made an end of tithing all the tithes of thine increase the third year, which is the year of tithing, and hast given it unto the Levite, the stranger, the fatherless, and the widow, that they may eat within thy gates, and be filled;
\verse Then thou shalt say before the \LORD thy God, I have brought away the hallowed things out of mine house, and also have given them unto the Levite, and unto the stranger, to the fatherless, and to the widow, according to all thy commandments which thou hast commanded me: I have not transgressed thy commandments, neither have I forgotten them:
\verse I have not eaten thereof in my mourning, neither have I taken away ought thereof for any unclean use, nor given ought thereof for the dead: but I have hearkened to the voice of the \LORD my God, and have done according to all that thou hast commanded me.
\verse Look down from thy holy habitation, from heaven, and bless thy people Israel, and the land which thou hast given us, as thou swarest unto our fathers, a land that floweth with milk and honey.
\verseWithHeading{Follow the \LORDs commands} This day the \LORD thy God hath commanded thee to do these statutes and judgments: thou shalt therefore keep and do them with all thine heart, and with all thy soul.
\verse Thou hast avouched the \LORD this day to be thy God, and to walk in his ways, and to keep his statutes, and his commandments, and his judgments, and to hearken unto his voice:
\verse And the \LORD hath avouched thee this day to be his peculiar people, as he hath promised thee, and that thou shouldest keep all his commandments;
\verse And to make thee high above all nations which he hath made, in praise, and in name, and in honour; and that thou mayest be an holy people unto the \LORD thy God, as he hath spoken.
\end{biblechapter}

\begin{biblechapter} % Deuteronomy 27
\verseWithHeading{The altar on Mount Ebal} And Moses with the elders of Israel commanded the people, saying, Keep all the commandments which I command you this day.
\verse And it shall be on the day when ye shall pass over Jordan unto the land which the \LORD thy God giveth thee, that thou shalt set thee up great stones, and plaister them with plaister:
\verse And thou shalt write upon them all the words of this law, when thou art passed over, that thou mayest go in unto the land which the \LORD thy God giveth thee, a land that floweth with milk and honey; as the \LORD God of thy fathers hath promised thee.
\verse Therefore it shall be when ye be gone over Jordan, that ye shall set up these stones, which I command you this day, in mount Ebal, and thou shalt plaister them with plaister.
\verse And there shalt thou build an altar unto the \LORD thy God, an altar of stones: thou shalt not lift up any iron tool upon them.
\verse Thou shalt build the altar of the \LORD thy God of whole stones: and thou shalt offer burnt offerings thereon unto the \LORD thy God:
\verse And thou shalt offer peace offerings, and shalt eat there, and rejoice before the \LORD thy God.
\verse And thou shalt write upon the stones all the words of this law very plainly.
\verseWithHeading{Curses from Mount Ebal} And Moses and the priests the Levites spake unto all Israel, saying, Take heed, and hearken, O Israel; this day thou art become the people of the \LORD thy God.
\verse Thou shalt therefore obey the voice of the \LORD thy God, and do his commandments and his statutes, which I command thee this day.
\verse And Moses charged the people the same day, saying,
\verse These shall stand upon mount Gerizim to bless the people, when ye are come over Jordan; Simeon, and Levi, and Judah, and Issachar, and Joseph, and Benjamin:
\verse And these shall stand upon mount Ebal to curse; Reuben, Gad, and Asher, and Zebulun, Dan, and Naphtali.
\verse And the Levites shall speak, and say unto all the men of Israel with a loud voice,
\verse Cursed be the man that maketh any graven or molten image, an abomination unto the \LORD, the work of the hands of the craftsman, and putteth it in a secret place. And all the people shall answer and say, Amen.
\verse Cursed be he that setteth light by his father or his mother. And all the people shall say, Amen.
\verse Cursed be he that removeth his neighbour's landmark. And all the people shall say, Amen.
\verse Cursed be he that maketh the blind to wander out of the way. And all the people shall say, Amen.
\verse Cursed be he that perverteth the judgment of the stranger, fatherless, and widow. And all the people shall say, Amen.
\verse Cursed be he that lieth with his father's wife; because he uncovereth his father's skirt. And all the people shall say, Amen.
\verse Cursed be he that lieth with any manner of beast. And all the people shall say, Amen.
\verse Cursed be he that lieth with his sister, the daughter of his father, or the daughter of his mother. And all the people shall say, Amen.
\verse Cursed be he that lieth with his mother in law. And all the people shall say, Amen.
\verse Cursed be he that smiteth his neighbour secretly. And all the people shall say, Amen.
\verse Cursed be he that taketh reward to slay an innocent person. And all the people shall say, Amen.
\verse Cursed be he that confirmeth not all the words of this law to do them. And all the people shall say, Amen.
\end{biblechapter}

\begin{biblechapter} % Deuteronomy 28
\verseWithHeading{Blessings for obedience} And it shall come to pass, if thou shalt hearken diligently unto the voice of the \LORD thy God, to observe and to do all his commandments which I command thee this day, that the \LORD thy God will set thee on high above all nations of the earth:
\verse And all these blessings shall come on thee, and overtake thee, if thou shalt hearken unto the voice of the \LORD thy God.
\verse Blessed shalt thou be in the city, and blessed shalt thou be in the field.
\verse Blessed shall be the fruit of thy body, and the fruit of thy ground, and the fruit of thy cattle, the increase of thy kine, and the flocks of thy sheep.
\verse Blessed shall be thy basket and thy store.
\verse Blessed shalt thou be when thou comest in, and blessed shalt thou be when thou goest out.
\verse The \LORD shall cause thine enemies that rise up against thee to be smitten before thy face: they shall come out against thee one way, and flee before thee seven ways.
\verse The \LORD shall command the blessing upon thee in thy storehouses, and in all that thou settest thine hand unto; and he shall bless thee in the land which the \LORD thy God giveth thee.
\verse The \LORD shall establish thee an holy people unto himself, as he hath sworn unto thee, if thou shalt keep the commandments of the \LORD thy God, and walk in his ways.
\verse And all people of the earth shall see that thou art called by the name of the \LORD; and they shall be afraid of thee.
\verse And the \LORD shall make thee plenteous in goods, in the fruit of thy body, and in the fruit of thy cattle, and in the fruit of thy ground, in the land which the \LORD sware unto thy fathers to give thee.
\verse The \LORD shall open unto thee his good treasure, the heaven to give the rain unto thy land in his season, and to bless all the work of thine hand: and thou shalt lend unto many nations, and thou shalt not borrow.
\verse And the \LORD shall make thee the head, and not the tail; and thou shalt be above only, and thou shalt not be beneath; if that thou hearken unto the commandments of the \LORD thy God, which I command thee this day, to observe and to do them:
\verse And thou shalt not go aside from any of the words which I command thee this day, to the right hand, or to the left, to go after other gods to serve them.
\verseWithHeading{Curses for disobedience} But it shall come to pass, if thou wilt not hearken unto the voice of the \LORD thy God, to observe to do all his commandments and his statutes which I command thee this day; that all these curses shall come upon thee, and overtake thee:
\verse Cursed shalt thou be in the city, and cursed shalt thou be in the field.
\verse Cursed shall be thy basket and thy store.
\verse Cursed shall be the fruit of thy body, and the fruit of thy land, the increase of thy kine, and the flocks of thy sheep.
\verse Cursed shalt thou be when thou comest in, and cursed shalt thou be when thou goest out.
\verse The \LORD shall send upon thee cursing, vexation, and rebuke, in all that thou settest thine hand unto for to do, until thou be destroyed, and until thou perish quickly; because of the wickedness of thy doings, whereby thou hast forsaken me.
\verse The \LORD shall make the pestilence cleave unto thee, until he have consumed thee from off the land, whither thou goest to possess it.
\verse The \LORD shall smite thee with a consumption, and with a fever, and with an inflammation, and with an extreme burning, and with the sword, and with blasting, and with mildew; and they shall pursue thee until thou perish.
\verse And thy heaven that is over thy head shall be brass, and the earth that is under thee shall be iron.
\verse The \LORD shall make the rain of thy land powder and dust: from heaven shall it come down upon thee, until thou be destroyed.
\verse The \LORD shall cause thee to be smitten before thine enemies: thou shalt go out one way against them, and flee seven ways before them: and shalt be removed into all the kingdoms of the earth.
\verse And thy carcase shall be meat unto all fowls of the air, and unto the beasts of the earth, and no man shall fray them away.
\verse The \LORD will smite thee with the botch of Egypt, and with the emerods, and with the scab, and with the itch, whereof thou canst not be healed.
\verse The \LORD shall smite thee with madness, and blindness, and astonishment of heart:
\verse And thou shalt grope at noonday, as the blind gropeth in darkness, and thou shalt not prosper in thy ways: and thou shalt be only oppressed and spoiled evermore, and no man shall save thee.
\verse Thou shalt betroth a wife, and another man shall lie with her: thou shalt build an house, and thou shalt not dwell therein: thou shalt plant a vineyard, and shalt not gather the grapes thereof.
\verse Thine ox shall be slain before thine eyes, and thou shalt not eat thereof: thine ass shall be violently taken away from before thy face, and shall not be restored to thee: thy sheep shall be given unto thine enemies, and thou shalt have none to rescue them.
\verse Thy sons and thy daughters shall be given unto another people, and thine eyes shall look, and fail with longing for them all the day long: and there shall be no might in thine hand.
\verse The fruit of thy land, and all thy labours, shall a nation which thou knowest not eat up; and thou shalt be only oppressed and crushed alway:
\verse So that thou shalt be mad for the sight of thine eyes which thou shalt see.
\verse The \LORD shall smite thee in the knees, and in the legs, with a sore botch that cannot be healed, from the sole of thy foot unto the top of thy head.
\verse The \LORD shall bring thee, and thy king which thou shalt set over thee, unto a nation which neither thou nor thy fathers have known; and there shalt thou serve other gods, wood and stone.
\verse And thou shalt become an astonishment, a proverb, and a byword, among all nations whither the \LORD shall lead thee.
\verse Thou shalt carry much seed out into the field, and shalt gather but little in; for the locust shall consume it.
\verse Thou shalt plant vineyards, and dress them, but shalt neither drink of the wine, nor gather the grapes; for the worms shall eat them.
\verse Thou shalt have olive trees throughout all thy coasts, but thou shalt not anoint thyself with the oil; for thine olive shall cast his fruit.
\verse Thou shalt beget sons and daughters, but thou shalt not enjoy them; for they shall go into captivity.
\verse All thy trees and fruit of thy land shall the locust consume.
\verse The stranger that is within thee shall get up above thee very high; and thou shalt come down very low.
\verse He shall lend to thee, and thou shalt not lend to him: he shall be the head, and thou shalt be the tail.
\verse Moreover all these curses shall come upon thee, and shall pursue thee, and overtake thee, till thou be destroyed; because thou hearkenedst not unto the voice of the \LORD thy God, to keep his commandments and his statutes which he commanded thee:
\verse And they shall be upon thee for a sign and for a wonder, and upon thy seed for ever.
\verse Because thou servedst not the \LORD thy God with joyfulness, and with gladness of heart, for the abundance of all things;
\verse Therefore shalt thou serve thine enemies which the \LORD shall send against thee, in hunger, and in thirst, and in nakedness, and in want of all things: and he shall put a yoke of iron upon thy neck, until he have destroyed thee.
\verse The \LORD shall bring a nation against thee from far, from the end of the earth, as swift as the eagle flieth; a nation whose tongue thou shalt not understand;
\verse A nation of fierce countenance, which shall not regard the person of the old, nor shew favour to the young:
\verse And he shall eat the fruit of thy cattle, and the fruit of thy land, until thou be destroyed: which also shall not leave thee either corn, wine, or oil, or the increase of thy kine, or flocks of thy sheep, until he have destroyed thee.
\verse And he shall besiege thee in all thy gates, until thy high and fenced walls come down, wherein thou trustedst, throughout all thy land: and he shall besiege thee in all thy gates throughout all thy land, which the \LORD thy God hath given thee.
\verse And thou shalt eat the fruit of thine own body, the flesh of thy sons and of thy daughters, which the \LORD thy God hath given thee, in the siege, and in the straitness, wherewith thine enemies shall distress thee:
\verse So that the man that is tender among you, and very delicate, his eye shall be evil toward his brother, and toward the wife of his bosom, and toward the remnant of his children which he shall leave:
\verse So that he will not give to any of them of the flesh of his children whom he shall eat: because he hath nothing left him in the siege, and in the straitness, wherewith thine enemies shall distress thee in all thy gates.
\verse The tender and delicate woman among you, which would not adventure to set the sole of her foot upon the ground for delicateness and tenderness, her eye shall be evil toward the husband of her bosom, and toward her son, and toward her daughter,
\verse And toward her young one that cometh out from between her feet, and toward her children which she shall bear: for she shall eat them for want of all things secretly in the siege and straitness, wherewith thine enemy shall distress thee in thy gates.
\verse If thou wilt not observe to do all the words of this law that are written in this book, that thou mayest fear this glorious and fearful name, THE \LORD THY God;
\verse Then the \LORD will make thy plagues wonderful, and the plagues of thy seed, even great plagues, and of long continuance, and sore sicknesses, and of long continuance.
\verse Moreover he will bring upon thee all the diseases of Egypt, which thou wast afraid of; and they shall cleave unto thee.
\verse Also every sickness, and every plague, which is not written in the book of this law, them will the \LORD bring upon thee, until thou be destroyed.
\verse And ye shall be left few in number, whereas ye were as the stars of heaven for multitude; because thou wouldest not obey the voice of the \LORD thy God.
\verse And it shall come to pass, that as the \LORD rejoiced over you to do you good, and to multiply you; so the \LORD will rejoice over you to destroy you, and to bring you to nought; and ye shall be plucked from off the land whither thou goest to possess it.
\verse And the \LORD shall scatter thee among all people, from the one end of the earth even unto the other; and there thou shalt serve other gods, which neither thou nor thy fathers have known, even wood and stone.
\verse And among these nations shalt thou find no ease, neither shall the sole of thy foot have rest: but the \LORD shall give thee there a trembling heart, and failing of eyes, and sorrow of mind:
\verse And thy life shall hang in doubt before thee; and thou shalt fear day and night, and shalt have none assurance of thy life:
\verse In the morning thou shalt say, Would God it were even! and at even thou shalt say, Would God it were morning! for the fear of thine heart wherewith thou shalt fear, and for the sight of thine eyes which thou shalt see.
\verse And the \LORD shall bring thee into Egypt again with ships, by the way whereof I spake unto thee, Thou shalt see it no more again: and there ye shall be sold unto your enemies for bondmen and bondwomen, and no man shall buy you.
\end{biblechapter}

\begin{biblechapter} % Deuteronomy 29
\verseWithHeading{Renewal of the covenant} These are the words of the covenant, which the \LORD commanded Moses to make with the children of Israel in the land of Moab, beside the covenant which he made with them in Horeb.
\verse And Moses called unto all Israel, and said unto them, Ye have seen all that the \LORD did before your eyes in the land of Egypt unto Pharaoh, and unto all his servants, and unto all his land;
\verse The great temptations which thine eyes have seen, the signs, and those great miracles:
\verse Yet the \LORD hath not given you an heart to perceive, and eyes to see, and ears to hear, unto this day.
\verse And I have led you forty years in the wilderness: your clothes are not waxen old upon you, and thy shoe is not waxen old upon thy foot.
\verse Ye have not eaten bread, neither have ye drunk wine or strong drink: that ye might know that I am the \LORD your God.
\verse And when ye came unto this place, Sihon the king of Heshbon, and Og the king of Bashan, came out against us unto battle, and we smote them:
\verse And we took their land, and gave it for an inheritance unto the Reubenites, and to the Gadites, and to the half tribe of Manasseh.
\verse Keep therefore the words of this covenant, and do them, that ye may prosper in all that ye do.
\verse Ye stand this day all of you before the \LORD your God; your captains of your tribes, your elders, and your officers, with all the men of Israel,
\verse Your little ones, your wives, and thy stranger that is in thy camp, from the hewer of thy wood unto the drawer of thy water:
\verse That thou shouldest enter into covenant with the \LORD thy God, and into his oath, which the \LORD thy God maketh with thee this day:
\verse That he may establish thee to day for a people unto himself, and that he may be unto thee a God, as he hath said unto thee, and as he hath sworn unto thy fathers, to Abraham, to Isaac, and to Jacob.
\verse Neither with you only do I make this covenant and this oath;
\verse But with him that standeth here with us this day before the \LORD our God, and also with him that is not here with us this day:
\verse (For ye know how we have dwelt in the land of Egypt; and how we came through the nations which ye passed by;
\verse And ye have seen their abominations, and their idols, wood and stone, silver and gold, which were among them:)
\verse Lest there should be among you man, or woman, or family, or tribe, whose heart turneth away this day from the \LORD our God, to go and serve the gods of these nations; lest there should be among you a root that beareth gall and wormwood;
\verse And it come to pass, when he heareth the words of this curse, that he bless himself in his heart, saying, I shall have peace, though I walk in the imagination of mine heart, to add drunkenness to thirst:
\verse The \LORD will not spare him, but then the anger of the \LORD and his jealousy shall smoke against that man, and all the curses that are written in this book shall lie upon him, and the \LORD shall blot out his name from under heaven.
\verse And the \LORD shall separate him unto evil out of all the tribes of Israel, according to all the curses of the covenant that are written in this book of the law:
\verse So that the generation to come of your children that shall rise up after you, and the stranger that shall come from a far land, shall say, when they see the plagues of that land, and the sicknesses which the \LORD hath laid upon it;
\verse And that the whole land thereof is brimstone, and salt, and burning, that it is not sown, nor beareth, nor any grass groweth therein, like the overthrow of Sodom, and Gomorrah, Admah, and Zeboim, which the \LORD overthrew in his anger, and in his wrath:
\verse Even all nations shall say, Wherefore hath the \LORD done thus unto this land? what meaneth the heat of this great anger?
\verse Then men shall say, Because they have forsaken the covenant of the \LORD God of their fathers, which he made with them when he brought them forth out of the land of Egypt:
\verse For they went and served other gods, and worshipped them, gods whom they knew not, and whom he had not given unto them:
\verse And the anger of the \LORD was kindled against this land, to bring upon it all the curses that are written in this book:
\verse And the \LORD rooted them out of their land in anger, and in wrath, and in great indignation, and cast them into another land, as it is this day.
\verse The secret things belong unto the \LORD our God: but those things which are revealed belong unto us and to our children for ever, that we may do all the words of this law.
\end{biblechapter}

\begin{biblechapter} % Deuteronomy 30
\verseWithHeading{Prosperity after turning to \newline the \LORD} And it shall come to pass, when all these things are come upon thee, the blessing and the curse, which I have set before thee, and thou shalt call them to mind among all the nations, whither the \LORD thy God hath driven thee,
\verse And shalt return unto the \LORD thy God, and shalt obey his voice according to all that I command thee this day, thou and thy children, with all thine heart, and with all thy soul;
\verse That then the \LORD thy God will turn thy captivity, and have compassion upon thee, and will return and gather thee from all the nations, whither the \LORD thy God hath scattered thee.
\verse If any of thine be driven out unto the outmost parts of heaven, from thence will the \LORD thy God gather thee, and from thence will he fetch thee:
\verse And the \LORD thy God will bring thee into the land which thy fathers possessed, and thou shalt possess it; and he will do thee good, and multiply thee above thy fathers.
\verse And the \LORD thy God will circumcise thine heart, and the heart of thy seed, to love the \LORD thy God with all thine heart, and with all thy soul, that thou mayest live.
\verse And the \LORD thy God will put all these curses upon thine enemies, and on them that hate thee, which persecuted thee.
\verse And thou shalt return and obey the voice of the \LORD, and do all his commandments which I command thee this day.
\verse And the \LORD thy God will make thee plenteous in every work of thine hand, in the fruit of thy body, and in the fruit of thy cattle, and in the fruit of thy land, for good: for the \LORD will again rejoice over thee for good, as he rejoiced over thy fathers:
\verse If thou shalt hearken unto the voice of the \LORD thy God, to keep his commandments and his statutes which are written in this book of the law, and if thou turn unto the \LORD thy God with all thine heart, and with all thy soul.
\verseWithHeading{The offer of life or death} For this commandment which I command thee this day, it is not hidden from thee, neither is it far off.
\verse It is not in heaven, that thou shouldest say, Who shall go up for us to heaven, and bring it unto us, that we may hear it, and do it?
\verse Neither is it beyond the sea, that thou shouldest say, Who shall go over the sea for us, and bring it unto us, that we may hear it, and do it?
\verse But the word is very nigh unto thee, in thy mouth, and in thy heart, that thou mayest do it.
\verse See, I have set before thee this day life and good, and death and evil;
\verse In that I command thee this day to love the \LORD thy God, to walk in his ways, and to keep his commandments and his statutes and his judgments, that thou mayest live and multiply: and the \LORD thy God shall bless thee in the land whither thou goest to possess it.
\verse But if thine heart turn away, so that thou wilt not hear, but shalt be drawn away, and worship other gods, and serve them;
\verse I denounce unto you this day, that ye shall surely perish, and that ye shall not prolong your days upon the land, whither thou passest over Jordan to go to possess it.
\verse I call heaven and earth to record this day against you, that I have set before you life and death, blessing and cursing: therefore choose life, that both thou and thy seed may live:
\verse That thou mayest love the \LORD thy God, and that thou mayest obey his voice, and that thou mayest cleave unto him: for he is thy life, and the length of thy days: that thou mayest dwell in the land which the \LORD sware unto thy fathers, to Abraham, to Isaac, and to Jacob, to give them.
\end{biblechapter}

\begin{biblechapter} % Deuteronomy 31
\verseWithHeading{Joshua to Moses} And Moses went and spake these words unto all Israel.
\verse And he said unto them, I am an hundred and twenty years old this day; I can no more go out and come in: also the \LORD hath said unto me, Thou shalt not go over this Jordan.
\verse The \LORD thy God, he will go over before thee, and he will destroy these nations from before thee, and thou shalt possess them: and Joshua, he shall go over before thee, as the \LORD hath said.
\verse And the \LORD shall do unto them as he did to Sihon and to Og, kings of the Amorites, and unto the land of them, whom he destroyed.
\verse And the \LORD shall give them up before your face, that ye may do unto them according unto all the commandments which I have commanded you.
\verse Be strong and of a good courage, fear not, nor be afraid of them: for the \LORD thy God, he it is that doth go with thee; he will not fail thee, nor forsake thee.
\verse And Moses called unto Joshua, and said unto him in the sight of all Israel, Be strong and of a good courage: for thou must go with this people unto the land which the \LORD hath sworn unto their fathers to give them; and thou shalt cause them to inherit it.
\verse And the \LORD, he it is that doth go before thee; he will be with thee, he will not fail thee, neither forsake thee: fear not, neither be dismayed.
\verseWithHeading{The reading of the law} And Moses wrote this law, and delivered it unto the priests the sons of Levi, which bare the ark of the covenant of the \LORD, and unto all the elders of Israel.
\verse And Moses commanded them, saying, At the end of every seven years, in the solemnity of the year of release, in the feast of tabernacles,
\verse When all Israel is come to appear before the \LORD thy God in the place which he shall choose, thou shalt read this law before all Israel in their hearing.
\verse Gather the people together, men, and women, and children, and thy stranger that is within thy gates, that they may hear, and that they may learn, and fear the \LORD your God, and observe to do all the words of this law:
\verse And that their children, which have not known any thing, may hear, and learn to fear the \LORD your God, as long as ye live in the land whither ye go over Jordan to possess it.
\verseWithHeading{Israel's rebellion predicted} And the \LORD said unto Moses, Behold, thy days approach that thou must die: call Joshua, and present yourselves in the tabernacle of the congregation, that I may give him a charge. And Moses and Joshua went, and presented themselves in the tabernacle of the congregation.
\verse And the \LORD appeared in the tabernacle in a pillar of a cloud: and the pillar of the cloud stood over the door of the tabernacle.
\verse And the \LORD said unto Moses, Behold, thou shalt sleep with thy fathers; and this people will rise up, and go a whoring after the gods of the strangers of the land, whither they go to be among them, and will forsake me, and break my covenant which I have made with them.
\verse Then my anger shall be kindled against them in that day, and I will forsake them, and I will hide my face from them, and they shall be devoured, and many evils and troubles shall befall them; so that they will say in that day, Are not these evils come upon us, because our God is not among us?
\verse And I will surely hide my face in that day for all the evils which they shall have wrought, in that they are turned unto other gods.
\verse Now therefore write ye this song for you, and teach it the children of Israel: put it in their mouths, that this song may be a witness for me against the children of Israel.
\verse For when I shall have brought them into the land which I sware unto their fathers, that floweth with milk and honey; and they shall have eaten and filled themselves, and waxen fat; then will they turn unto other gods, and serve them, and provoke me, and break my covenant.
\verse And it shall come to pass, when many evils and troubles are befallen them, that this song shall testify against them as a witness; for it shall not be forgotten out of the mouths of their seed: for I know their imagination which they go about, even now, before I have brought them into the land which I sware.
\verse Moses therefore wrote this song the same day, and taught it the children of Israel.
\verse And he gave Joshua the son of Nun a charge, and said, Be strong and of a good courage: for thou shalt bring the children of Israel into the land which I sware unto them: and I will be with thee.
\verse And it came to pass, when Moses had made an end of writing the words of this law in a book, until they were finished,
\verse That Moses commanded the Levites, which bare the ark of the covenant of the \LORD, saying,
\verse Take this book of the law, and put it in the side of the ark of the covenant of the \LORD your God, that it may be there for a witness against thee.
\verse For I know thy rebellion, and thy stiff neck: behold, while I am yet alive with you this day, ye have been rebellious against the \LORD; and how much more after my death?
\verse Gather unto me all the elders of your tribes, and your officers, that I may speak these words in their ears, and call heaven and earth to record against them.
\verse For I know that after my death ye will utterly corrupt yourselves, and turn aside from the way which I have commanded you; and evil will befall you in the latter days; because ye will do evil in the sight of the \LORD, to provoke him to anger through the work of your hands.
\verseWithHeading{The song of Moses} And Moses spake in the ears of all the congregation of Israel the words of this song, until they were ended.
\end{biblechapter}

\begin{biblechapter} % Deuteronomy 32
\verse Give ear, O ye heavens, and I will speak; and hear, O earth, the words of my mouth.
\verse My doctrine shall drop as the rain, my speech shall distil as the dew, as the small rain upon the tender herb, and as the showers upon the grass:
\verse Because I will publish the name of the \LORD: ascribe ye greatness unto our God.
\verse He is the Rock, his work is perfect: for all his ways are judgment: a God of truth and without iniquity, just and right is he.
\verse They have corrupted themselves, their spot is not the spot of his children: they are a perverse and crooked generation.
\verse Do ye thus requite the \LORD, O foolish people and unwise? is not he thy father that hath bought thee? hath he not made thee, and established thee?
\verse Remember the days of old, consider the years of many generations: ask thy father, and he will shew thee; thy elders, and they will tell thee.
\verse When the most High divided to the nations their inheritance, when he separated the sons of Adam, he set the bounds of the people according to the number of the children of Israel.
\verse For the \LORDs portion is his people; Jacob is the lot of his inheritance.
\verse He found him in a desert land, and in the waste howling wilderness; he led him about, he instructed him, he kept him as the apple of his eye.
\verse As an eagle stirreth up her nest, fluttereth over her young, spreadeth abroad her wings, taketh them, beareth them on her wings:
\verse So the \LORD alone did lead him, and there was no strange god with him.
\verse He made him ride on the high places of the earth, that he might eat the increase of the fields; and he made him to suck honey out of the rock, and oil out of the flinty rock;
\verse Butter of kine, and milk of sheep, with fat of lambs, and rams of the breed of Bashan, and goats, with the fat of kidneys of wheat; and thou didst drink the pure blood of the grape.
\verse But Jeshurun waxed fat, and kicked: thou art waxen fat, thou art grown thick, thou art covered with fatness; then he forsook God which made him, and lightly esteemed the Rock of his salvation.
\verse They provoked him to jealousy with strange gods, with abominations provoked they him to anger.
\verse They sacrificed unto devils, not to God; to gods whom they knew not, to new gods that came newly up, whom your fathers feared not.
\verse Of the Rock that begat thee thou art unmindful, and hast forgotten God that formed thee.
\verse And when the \LORD saw it, he abhorred them, because of the provoking of his sons, and of his daughters.
\verse And he said, I will hide my face from them, I will see what their end shall be: for they are a very froward generation, children in whom is no faith.
\verse They have moved me to jealousy with that which is not God; they have provoked me to anger with their vanities: and I will move them to jealousy with those which are not a people; I will provoke them to anger with a foolish nation.
\verse For a fire is kindled in mine anger, and shall burn unto the lowest hell, and shall consume the earth with her increase, and set on fire the foundations of the mountains.
\verse I will heap mischiefs upon them; I will spend mine arrows upon them.
\verse They shall be burnt with hunger, and devoured with burning heat, and with bitter destruction: I will also send the teeth of beasts upon them, with the poison of serpents of the dust.
\verse The sword without, and terror within, shall destroy both the young man and the virgin, the suckling also with the man of gray hairs.
\verse I said, I would scatter them into corners, I would make the remembrance of them to cease from among men:
\verse Were it not that I feared the wrath of the enemy, lest their adversaries should behave themselves strangely, and lest they should say, Our hand is high, and the \LORD hath not done all this.
\verse For they are a nation void of counsel, neither is there any understanding in them.
\verse O that they were wise, that they understood this, that they would consider their latter end!
\verse How should one chase a thousand, and two put ten thousand to flight, except their Rock had sold them, and the \LORD had shut them up?
\verse For their rock is not as our Rock, even our enemies themselves being judges.
\verse For their vine is of the vine of Sodom, and of the fields of Gomorrah: their grapes are grapes of gall, their clusters are bitter:
\verse Their wine is the poison of dragons, and the cruel venom of asps.
\verse Is not this laid up in store with me, and sealed up among my treasures?
\verse To me belongeth vengeance, and recompence; their foot shall slide in due time: for the day of their calamity is at hand, and the things that shall come upon them make haste.
\verse For the \LORD shall judge his people, and repent himself for his servants, when he seeth that their power is gone, and there is none shut up, or left.
\verse And he shall say, Where are their gods, their rock in whom they trusted,
\verse Which did eat the fat of their sacrifices, and drank the wine of their drink offerings? let them rise up and help you, and be your protection.
\verse See now that I, even I, am he, and there is no god with me: I kill, and I make alive; I wound, and I heal: neither is there any that can deliver out of my hand.
\verse For I lift up my hand to heaven, and say, I live for ever.
\verse If I whet my glittering sword, and mine hand take hold on judgment; I will render vengeance to mine enemies, and will reward them that hate me.
\verse I will make mine arrows drunk with blood, and my sword shall devour flesh; and that with the blood of the slain and of the captives, from the beginning of revenges upon the enemy.
\verse Rejoice, O ye nations, with his people: for he will avenge the blood of his servants, and will render vengeance to his adversaries, and will be merciful unto his land, and to his people.
\verse And Moses came and spake all the words of this song in the ears of the people, he, and Hoshea the son of Nun.
\verse And Moses made an end of speaking all these words to all Israel:
\verse And he said unto them, Set your hearts unto all the words which I testify among you this day, which ye shall command your children to observe to do, all the words of this law.
\verse For it is not a vain thing for you; because it is your life: and through this thing ye shall prolong your days in the land, whither ye go over Jordan to possess it.
\verseWithHeading{Moses to die on Mount Nebo} And the \LORD spake unto Moses that selfsame day, saying,
\verse Get thee up into this mountain Abarim, unto mount Nebo, which is in the land of Moab, that is over against Jericho; and behold the land of Canaan, which I give unto the children of Israel for a possession:
\verse And die in the mount whither thou goest up, and be gathered unto thy people; as Aaron thy brother died in mount Hor, and was gathered unto his people:
\verse Because ye trespassed against me among the children of Israel at the waters of Meribah-Kadesh, in the wilderness of Zin; because ye sanctified me not in the midst of the children of Israel.
\verse Yet thou shalt see the land before thee; but thou shalt not go thither unto the land which I give the children of Israel.
\end{biblechapter}

\begin{biblechapter} % Deuteronomy 33
\verseWithHeading{Moses blesses the tribes} And this is the blessing, wherewith Moses the man of God blessed the children of Israel before his death.
\verse And he said, The \LORD came from Sinai, and rose up from Seir unto them; he shined forth from mount Paran, and he came with ten thousands of saints: from his right hand went a fiery law for them.
\verse Yea, he loved the people; all his saints are in thy hand: and they sat down at thy feet; every one shall receive of thy words.
\verse Moses commanded us a law, even the inheritance of the congregation of Jacob.
\verse And he was king in Jeshurun, when the heads of the people and the tribes of Israel were gathered together.
\verse Let Reuben live, and not die; and let not his men be few.
\verse And this is the blessing of Judah: and he said, Hear, \LORD, the voice of Judah, and bring him unto his people: let his hands be sufficient for him; and be thou an help to him from his enemies.
\verse And of Levi he said, Let thy Thummim and thy Urim be with thy holy one, whom thou didst prove at Massah, and with whom thou didst strive at the waters of Meribah;
\verse Who said unto his father and to his mother, I have not seen him; neither did he acknowledge his brethren, nor knew his own children: for they have observed thy word, and kept thy covenant.
\verse They shall teach Jacob thy judgments, and Israel thy law: they shall put incense before thee, and whole burnt sacrifice upon thine altar.
\verse Bless, \LORD, his substance, and accept the work of his hands: smite through the loins of them that rise against him, and of them that hate him, that they rise not again.
\verse And of Benjamin he said, The beloved of the \LORD shall dwell in safety by him; and the \LORD shall cover him all the day long, and he shall dwell between his shoulders.
\verse And of Joseph he said, Blessed of the \LORD be his land, for the precious things of heaven, for the dew, and for the deep that coucheth beneath,
\verse And for the precious fruits brought forth by the sun, and for the precious things put forth by the moon,
\verse And for the chief things of the ancient mountains, and for the precious things of the lasting hills,
\verse And for the precious things of the earth and fulness thereof, and for the good will of him that dwelt in the bush: let the blessing come upon the head of Joseph, and upon the top of the head of him that was separated from his brethren.
\verse His glory is like the firstling of his bullock, and his horns are like the horns of unicorns: with them he shall push the people together to the ends of the earth: and they are the ten thousands of Ephraim, and they are the thousands of Manasseh.
\verse And of Zebulun he said, Rejoice, Zebulun, in thy going out; and, Issachar, in thy tents.
\verse They shall call the people unto the mountain; there they shall offer sacrifices of righteousness: for they shall suck of the abundance of the seas, and of treasures hid in the sand.
\verse And of Gad he said, Blessed be he that enlargeth Gad: he dwelleth as a lion, and teareth the arm with the crown of the head.
\verse And he provided the first part for himself, because there, in a portion of the lawgiver, was he seated; and he came with the heads of the people, he executed the justice of the \LORD, and his judgments with Israel.
\verse And of Dan he said, Dan is a lion's whelp: he shall leap from Bashan.
\verse And of Naphtali he said, O Naphtali, satisfied with favour, and full with the blessing of the \LORD: possess thou the west and the south.
\verse And of Asher he said, Let Asher be blessed with children; let him be acceptable to his brethren, and let him dip his foot in oil.
\verse Thy shoes shall be iron and brass; and as thy days, so shall thy strength be.
\verse There is none like unto the God of Jeshurun, who rideth upon the heaven in thy help, and in his excellency on the sky.
\verse The eternal God is thy refuge, and underneath are the everlasting arms: and he shall thrust out the enemy from before thee; and shall say, Destroy them.
\verse Israel then shall dwell in safety alone: the fountain of Jacob shall be upon a land of corn and wine; also his heavens shall drop down dew.
\verse Happy art thou, O Israel: who is like unto thee, O people saved by the \LORD, the shield of thy help, and who is the sword of thy excellency! and thine enemies shall be found liars unto thee; and thou shalt tread upon their high places.
\end{biblechapter}

\begin{biblechapter} % Deuteronomy 34
\verseWithHeading{The death of Moses} And Moses went up from the plains of Moab unto the mountain of Nebo, to the top of Pisgah, that is over against Jericho. And the \LORD shewed him all the land of Gilead, unto Dan,
\verse And all Naphtali, and the land of Ephraim, and Manasseh, and all the land of Judah, unto the utmost sea,
\verse And the south, and the plain of the valley of Jericho, the city of palm trees, unto Zoar.
\verse And the \LORD said unto him, This is the land which I sware unto Abraham, unto Isaac, and unto Jacob, saying, I will give it unto thy seed: I have caused thee to see it with thine eyes, but thou shalt not go over thither.
\verse So Moses the servant of the \LORD died there in the land of Moab, according to the word of the \LORD.
\verse And he buried him in a valley in the land of Moab, over against Bethpeor: but no man knoweth of his sepulchre unto this day.
\verse And Moses was an hundred and twenty years old when he died: his eye was not dim, nor his natural force abated.
\verse And the children of Israel wept for Moses in the plains of Moab thirty days: so the days of weeping and mourning for Moses were ended.
\verse And Joshua the son of Nun was full of the spirit of wisdom; for Moses had laid his hands upon him: and the children of Israel hearkened unto him, and did as the \LORD commanded Moses.
\verse And there arose not a prophet since in Israel like unto Moses, whom the \LORD knew face to face,
\verse In all the signs and the wonders, which the \LORD sent him to do in the land of Egypt to Pharaoh, and to all his servants, and to all his land,
\verse And in all that mighty hand, and in all the great terror which Moses shewed in the sight of all Israel.
\end{biblechapter}
\flushcolsend
\biblebook{Joshua}

\begin{biblechapter} % Joshua 1
\verseWithHeading{Joshua installed as leader} Now after the death of Moses the servant of the \LORD it came to pass, that the \LORD spake unto Joshua the son of Nun, Moses' minister, saying,
\verse Moses my servant is dead; now therefore arise, go over this Jordan, thou, and all this people, unto the land which I do give to them, even to the children of Israel.
\verse Every place that the sole of your foot shall tread upon, that have I given unto you, as I said unto Moses.
\verse From the wilderness and this Lebanon even unto the great river, the river Euphrates, all the land of the Hittites, and unto the great sea toward the going down of the sun, shall be your coast.
\verse There shall not any man be able to stand before thee all the days of thy life: as I was with Moses, so I will be with thee: I will not fail thee, nor forsake thee.
\verse Be strong and of a good courage: for unto this people shalt thou divide for an inheritance the land, which I sware unto their fathers to give them.
\verse Only be thou strong and very courageous, that thou mayest observe to do according to all the law, which Moses my servant commanded thee: turn not from it to the right hand or to the left, that thou mayest prosper whithersoever thou goest.
\verse This book of the law shall not depart out of thy mouth; but thou shalt meditate therein day and night, that thou mayest observe to do according to all that is written therein: for then thou shalt make thy way prosperous, and then thou shalt have good success.
\verse Have not I commanded thee? Be strong and of a good courage; be not afraid, neither be thou dismayed: for the \LORD thy God is with thee whithersoever thou goest.
\verse Then Joshua commanded the officers of the people, saying,
\verse Pass through the host, and command the people, saying, Prepare you victuals; for within three days ye shall pass over this Jordan, to go in to possess the land, which the \LORD your God giveth you to possess it.
\verse And to the Reubenites, and to the Gadites, and to half the tribe of Manasseh, spake Joshua, saying,
\verse Remember the word which Moses the servant of the \LORD commanded you, saying, The \LORD your God hath given you rest, and hath given you this land.
\verse Your wives, your little ones, and your cattle, shall remain in the land which Moses gave you on this side Jordan; but ye shall pass before your brethren armed, all the mighty men of valour, and help them;
\verse Until the \LORD have given your brethren rest, as he hath given you, and they also have possessed the land which the \LORD your God giveth them: then ye shall return unto the land of your possession, and enjoy it, which Moses the \LORDs servant gave you on this side Jordan toward the sunrising.
\verse And they answered Joshua, saying, All that thou commandest us we will do, and whithersoever thou sendest us, we will go.
\verse According as we hearkened unto Moses in all things, so will we hearken unto thee: only the \LORD thy God be with thee, as he was with Moses.
\verse Whosoever he be that doth rebel against thy commandment, and will not hearken unto thy words in all that thou commandest him, he shall be put to death: only be strong and of a good courage.
\end{biblechapter}

\begin{biblechapter} % Joshua 2
\verseWithHeading{Rahab and the spies} And Joshua the son of Nun sent out of Shittim two men to spy secretly, saying, Go view the land, even Jericho. And they went, and came into an harlot's house, named Rahab, and lodged there.
\verse And it was told the king of Jericho, saying, Behold, there came men in hither to night of the children of Israel to search out the country.
\verse And the king of Jericho sent unto Rahab, saying, Bring forth the men that are come to thee, which are entered into thine house: for they be come to search out all the country.
\verse And the woman took the two men, and hid them, and said thus, There came men unto me, but I wist not whence they were:
\verse And it came to pass about the time of shutting of the gate, when it was dark, that the men went out: whither the men went I wot not: pursue after them quickly; for ye shall overtake them.
\verse But she had brought them up to the roof of the house, and hid them with the stalks of flax, which she had laid in order upon the roof.
\verse And the men pursued after them the way to Jordan unto the fords: and as soon as they which pursued after them were gone out, they shut the gate.
\verse And before they were laid down, she came up unto them upon the roof;
\verse And she said unto the men, I know that the \LORD hath given you the land, and that your terror is fallen upon us, and that all the inhabitants of the land faint because of you.
\verse For we have heard how the \LORD dried up the water of the Red sea for you, when ye came out of Egypt; and what ye did unto the two kings of the Amorites, that were on the other side Jordan, Sihon and Og, whom ye utterly destroyed.
\verse And as soon as we had heard these things, our hearts did melt, neither did there remain any more courage in any man, because of you: for the \LORD your God, he is God in heaven above, and in earth beneath.
\verse Now therefore, I pray you, swear unto me by the \LORD, since I have shewed you kindness, that ye will also shew kindness unto my father's house, and give me a true token:
\verse And that ye will save alive my father, and my mother, and my brethren, and my sisters, and all that they have, and deliver our lives from death.
\verse And the men answered her, Our life for yours, if ye utter not this our business. And it shall be, when the \LORD hath given us the land, that we will deal kindly and truly with thee.
\verse Then she let them down by a cord through the window: for her house was upon the town wall, and she dwelt upon the wall.
\verse And she said unto them, Get you to the mountain, lest the pursuers meet you; and hide yourselves there three days, until the pursuers be returned: and afterward may ye go your way.
\verse And the men said unto her, We will be blameless of this thine oath which thou hast made us swear.
\verse Behold, when we come into the land, thou shalt bind this line of scarlet thread in the window which thou didst let us down by: and thou shalt bring thy father, and thy mother, and thy brethren, and all thy father's household, home unto thee.
\verse And it shall be, that whosoever shall go out of the doors of thy house into the street, his blood shall be upon his head, and we will be guiltless: and whosoever shall be with thee in the house, his blood shall be on our head, if any hand be upon him.
\verse And if thou utter this our business, then we will be quit of thine oath which thou hast made us to swear.
\verse And she said, According unto your words, so be it. And she sent them away, and they departed: and she bound the scarlet line in the window.
\verse And they went, and came unto the mountain, and abode there three days, until the pursuers were returned: and the pursuers sought them throughout all the way, but found them not.
\verse So the two men returned, and descended from the mountain, and passed over, and came to Joshua the son of Nun, and told him all things that befell them:
\verse And they said unto Joshua, Truly the \LORD hath delivered into our hands all the land; for even all the inhabitants of the country do faint because of us.
\end{biblechapter}

\begin{biblechapter} % Joshua 3
\verseWithHeading{Crossing the Jordan} And Joshua rose early in the morning; and they removed from Shittim, and came to Jordan, he and all the children of Israel, and lodged there before they passed over.
\verse And it came to pass after three days, that the officers went through the host;
\verse And they commanded the people, saying, When ye see the ark of the covenant of the \LORD your God, and the priests the Levites bearing it, then ye shall remove from your place, and go after it.
\verse Yet there shall be a space between you and it, about two thousand cubits by measure: come not near unto it, that ye may know the way by which ye must go: for ye have not passed this way heretofore.
\verse And Joshua said unto the people, Sanctify yourselves: for to morrow the \LORD will do wonders among you.
\verse And Joshua spake unto the priests, saying, Take up the ark of the covenant, and pass over before the people. And they took up the ark of the covenant, and went before the people.
\verse And the \LORD said unto Joshua, This day will I begin to magnify thee in the sight of all Israel, that they may know that, as I was with Moses, so I will be with thee.
\verse And thou shalt command the priests that bear the ark of the covenant, saying, When ye are come to the brink of the water of Jordan, ye shall stand still in Jordan.
\verse And Joshua said unto the children of Israel, Come hither, and hear the words of the \LORD your God.
\verse And Joshua said, Hereby ye shall know that the living God is among you, and that he will without fail drive out from before you the Canaanites, and the Hittites, and the Hivites, and the Perizzites, and the Girgashites, and the Amorites, and the Jebusites.
\verse Behold, the ark of the covenant of the Lord of all the earth passeth over before you into Jordan.
\verse Now therefore take you twelve men out of the tribes of Israel, out of every tribe a man.
\verse And it shall come to pass, as soon as the soles of the feet of the priests that bear the ark of the \LORD, the Lord of all the earth, shall rest in the waters of Jordan, that the waters of Jordan shall be cut off from the waters that come down from above; and they shall stand upon an heap.
\verse And it came to pass, when the people removed from their tents, to pass over Jordan, and the priests bearing the ark of the covenant before the people;
\verse And as they that bare the ark were come unto Jordan, and the feet of the priests that bare the ark were dipped in the brim of the water, (for Jordan overfloweth all his banks all the time of harvest,)
\verse That the waters which came down from above stood and rose up upon an heap very far from the city Adam, that is beside Zaretan: and those that came down toward the sea of the plain, even the salt sea, failed, and were cut off: and the people passed over right against Jericho.
\verse And the priests that bare the ark of the covenant of the \LORD stood firm on dry ground in the midst of Jordan, and all the Israelites passed over on dry ground, until all the people were passed clean over Jordan.
\end{biblechapter}

\begin{biblechapter} % Joshua 4
\verse And it came to pass, when all the people were clean passed over Jordan, that the \LORD spake unto Joshua, saying,
\verse Take you twelve men out of the people, out of every tribe a man,
\verse And command ye them, saying, Take you hence out of the midst of Jordan, out of the place where the priests' feet stood firm, twelve stones, and ye shall carry them over with you, and leave them in the lodging place, where ye shall lodge this night.
\verse Then Joshua called the twelve men, whom he had prepared of the children of Israel, out of every tribe a man:
\verse And Joshua said unto them, Pass over before the ark of the \LORD your God into the midst of Jordan, and take ye up every man of you a stone upon his shoulder, according unto the number of the tribes of the children of Israel:
\verse That this may be a sign among you, that when your children ask their fathers in time to come, saying, What mean ye by these stones?
\verse Then ye shall answer them, That the waters of Jordan were cut off before the ark of the covenant of the \LORD; when it passed over Jordan, the waters of Jordan were cut off: and these stones shall be for a memorial unto the children of Israel for ever.
\verse And the children of Israel did so as Joshua commanded, and took up twelve stones out of the midst of Jordan, as the \LORD spake unto Joshua, according to the number of the tribes of the children of Israel, and carried them over with them unto the place where they lodged, and laid them down there.
\verse And Joshua set up twelve stones in the midst of Jordan, in the place where the feet of the priests which bare the ark of the covenant stood: and they are there unto this day.
\verse For the priests which bare the ark stood in the midst of Jordan, until every thing was finished that the \LORD commanded Joshua to speak unto the people, according to all that Moses commanded Joshua: and the people hasted and passed over.
\verse And it came to pass, when all the people were clean passed over, that the ark of the \LORD passed over, and the priests, in the presence of the people.
\verse And the children of Reuben, and the children of Gad, and half the tribe of Manasseh, passed over armed before the children of Israel, as Moses spake unto them:
\verse About forty thousand prepared for war passed over before the \LORD unto battle, to the plains of Jericho.
\verse On that day the \LORD magnified Joshua in the sight of all Israel; and they feared him, as they feared Moses, all the days of his life.
\verse And the \LORD spake unto Joshua, saying,
\verse Command the priests that bear the ark of the testimony, that they come up out of Jordan.
\verse Joshua therefore commanded the priests, saying, Come ye up out of Jordan.
\verse And it came to pass, when the priests that bare the ark of the covenant of the \LORD were come up out of the midst of Jordan, and the soles of the priests' feet were lifted up unto the dry land, that the waters of Jordan returned unto their place, and flowed over all his banks, as they did before.
\verse And the people came up out of Jordan on the tenth day of the first month, and encamped in Gilgal, in the east border of Jericho.
\verse And those twelve stones, which they took out of Jordan, did Joshua pitch in Gilgal.
\verse And he spake unto the children of Israel, saying, When your children shall ask their fathers in time to come, saying, What mean these stones?
\verse Then ye shall let your children know, saying, Israel came over this Jordan on dry land.
\verse For the \LORD your God dried up the waters of Jordan from before you, until ye were passed over, as the \LORD your God did to the Red sea, which he dried up from before us, until we were gone over:
\verse That all the people of the earth might know the hand of the \LORD, that it is mighty: that ye might fear the \LORD your God for ever.
\end{biblechapter}

\flushcolsend\columnbreak % layout hack

\begin{biblechapter} % Joshua 5
\verse And it came to pass, when all the kings of the Amorites, which were on the side of Jordan westward, and all the kings of the Canaanites, which were by the sea, heard that the \LORD had dried up the waters of Jordan from before the children of Israel, until we were passed over, that their heart melted, neither was there spirit in them any more, because of the children of Israel.
\verseWithHeading{Circumcision and the Passover at Gilgal} At that time the \LORD said unto Joshua, Make thee sharp knives, and circumcise again the children of Israel the second time.
\verse And Joshua made him sharp knives, and circumcised the children of Israel at the hill of the foreskins.
\verse And this is the cause why Joshua did circumcise: All the people that came out of Egypt, that were males, even all the men of war, died in the wilderness by the way, after they came out of Egypt.
\verse Now all the people that came out were circumcised: but all the people that were born in the wilderness by the way as they came forth out of Egypt, them they had not circumcised.
\verse For the children of Israel walked forty years in the wilderness, till all the people that were men of war, which came out of Egypt, were consumed, because they obeyed not the voice of the \LORD: unto whom the \LORD sware that he would not shew them the land, which the \LORD sware unto their fathers that he would give us, a land that floweth with milk and honey.
\verse And their children, whom he raised up in their stead, them Joshua circumcised: for they were uncircumcised, because they had not circumcised them by the way.
\verse And it came to pass, when they had done circumcising all the people, that they abode in their places in the camp, till they were whole.
\verse And the \LORD said unto Joshua, This day have I rolled away the reproach of Egypt from off you. Wherefore the name of the place is called Gilgal unto this day.
\verse And the children of Israel encamped in Gilgal, and kept the Passover on the fourteenth day of the month at even in the plains of Jericho.
\verse And they did eat of the old corn of the land on the morrow after the Passover, unleavened cakes, and parched corn in the selfsame day.
\verse And the manna ceased on the morrow after they had eaten of the old corn of the land; neither had the children of Israel manna any more; but they did eat of the fruit of the land of Canaan that year.
\verse And it came to pass, when Joshua was by Jericho, that he lifted up his eyes and looked, and, behold, there stood a man over against him with his sword drawn in his hand: and Joshua went unto him, and said unto him, Art thou for us, or for our adversaries?
\verse And he said, Nay; but as captain of the host of the \LORD am I now come. And Joshua fell on his face to the earth, and did worship, and said unto him, What saith my lord unto his servant?
\verse And the captain of the \LORDs host said unto Joshua, Loose thy shoe from off thy foot; for the place whereon thou standest is holy. And Joshua did so.
\end{biblechapter}

\begin{biblechapter} % Joshua 6
\verse Now Jericho was straitly shut up because of the children of Israel: none went out, and none came in.
\verse And the \LORD said unto Joshua, See, I have given into thine hand Jericho, and the king thereof, and the mighty men of valour.
\verse And ye shall compass the city, all ye men of war, and go round about the city once. Thus shalt thou do six days.
\verse And seven priests shall bear before the ark seven trumpets of rams' horns: and the seventh day ye shall compass the city seven times, and the priests shall blow with the trumpets.
\verse And it shall come to pass, that when they make a long blast with the ram's horn, and when ye hear the sound of the trumpet, all the people shall shout with a great shout; and the wall of the city shall fall down flat, and the people shall ascend up every man straight before him.
\verse And Joshua the son of Nun called the priests, and said unto them, Take up the ark of the covenant, and let seven priests bear seven trumpets of rams' horns before the ark of the \LORD.
\verse And he said unto the people, Pass on, and compass the city, and let him that is armed pass on before the ark of the \LORD.
\verse And it came to pass, when Joshua had spoken unto the people, that the seven priests bearing the seven trumpets of rams' horns passed on before the \LORD, and blew with the trumpets: and the ark of the covenant of the \LORD followed them.
\verse And the armed men went before the priests that blew with the trumpets, and the rereward came after the ark, the priests going on, and blowing with the trumpets.
\verse And Joshua had commanded the people, saying, Ye shall not shout, nor make any noise with your voice, neither shall any word proceed out of your mouth, until the day I bid you shout; then shall ye shout.
\verse So the ark of the \LORD compassed the city, going about it once: and they came into the camp, and lodged in the camp.
\verse And Joshua rose early in the morning, and the priests took up the ark of the \LORD.
\verse And seven priests bearing seven trumpets of rams' horns before the ark of the \LORD went on continually, and blew with the trumpets: and the armed men went before them; but the rereward came after the ark of the \LORD, the priests going on, and blowing with the trumpets.
\verse And the second day they compassed the city once, and returned into the camp: so they did six days.
\verse And it came to pass on the seventh day, that they rose early about the dawning of the day, and compassed the city after the same manner seven times: only on that day they compassed the city seven times.
\verse And it came to pass at the seventh time, when the priests blew with the trumpets, Joshua said unto the people, Shout; for the \LORD hath given you the city.
\verse And the city shall be accursed, even it, and all that are therein, to the \LORD: only Rahab the harlot shall live, she and all that are with her in the house, because she hid the messengers that we sent.
\verse And ye, in any wise keep yourselves from the accursed thing, lest ye make yourselves accursed, when ye take of the accursed thing, and make the camp of Israel a curse, and trouble it.
\verse But all the silver, and gold, and vessels of brass and iron, are consecrated unto the \LORD: they shall come into the treasury of the \LORD.
\verse So the people shouted when the priests blew with the trumpets: and it came to pass, when the people heard the sound of the trumpet, and the people shouted with a great shout, that the wall fell down flat, so that the people went up into the city, every man straight before him, and they took the city.
\verse And they utterly destroyed all that was in the city, both man and woman, young and old, and ox, and sheep, and ass, with the edge of the sword.
\verse But Joshua had said unto the two men that had spied out the country, Go into the harlot's house, and bring out thence the woman, and all that she hath, as ye sware unto her.
\verse And the young men that were spies went in, and brought out Rahab, and her father, and her mother, and her brethren, and all that she had; and they brought out all her kindred, and left them without the camp of Israel.
\verse And they burnt the city with fire, and all that was therein: only the silver, and the gold, and the vessels of brass and of iron, they put into the treasury of the house of the \LORD.
\verse And Joshua saved Rahab the harlot alive, and her father's household, and all that she had; and she dwelleth in Israel even unto this day; because she hid the messengers, which Joshua sent to spy out Jericho.
\verse And Joshua adjured them at that time, saying, Cursed be the man before the \LORD, that riseth up and buildeth this city Jericho: he shall lay the foundation thereof in his firstborn, and in his youngest son shall he set up the gates of it.
\verse So the \LORD was with Joshua; and his fame was noised throughout all the country.
\end{biblechapter}

\begin{biblechapter} % Joshua 7
\verseWithHeading{Achan's sin} But the children of Israel committed a trespass in the accursed thing: for Achan, the son of Carmi, the son of Zabdi, the son of Zerah, of the tribe of Judah, took of the accursed thing: and the anger of the \LORD was kindled against the children of Israel.
\verse And Joshua sent men from Jericho to Ai, which is beside Bethaven, on the east side of Bethel, and spake unto them, saying, Go up and view the country. And the men went up and viewed Ai.
\verse And they returned to Joshua, and said unto him, Let not all the people go up; but let about two or three thousand men go up and smite Ai; and make not all the people to labour thither; for they are but few.
\verse So there went up thither of the people about three thousand men: and they fled before the men of Ai.
\verse And the men of Ai smote of them about thirty and six men: for they chased them from before the gate even unto Shebarim, and smote them in the going down: wherefore the hearts of the people melted, and became as water.
\verse And Joshua rent his clothes, and fell to the earth upon his face before the ark of the \LORD until the eventide, he and the elders of Israel, and put dust upon their heads.
\verse And Joshua said, Alas, O Lord God, wherefore hast thou at all brought this people over Jordan, to deliver us into the hand of the Amorites, to destroy us? would to God we had been content, and dwelt on the other side Jordan!
\verse O Lord, what shall I say, when Israel turneth their backs before their enemies!
\verse For the Canaanites and all the inhabitants of the land shall hear of it, and shall environ us round, and cut off our name from the earth: and what wilt thou do unto thy great name?
\verse And the \LORD said unto Joshua, Get thee up; wherefore liest thou thus upon thy face?
\verse Israel hath sinned, and they have also transgressed my covenant which I commanded them: for they have even taken of the accursed thing, and have also stolen, and dissembled also, and they have put it even among their own stuff.
\verse Therefore the children of Israel could not stand before their enemies, but turned their backs before their enemies, because they were accursed: neither will I be with you any more, except ye destroy the accursed from among you.
\verse Up, sanctify the people, and say, Sanctify yourselves against to morrow: for thus saith the \LORD God of Israel, There is an accursed thing in the midst of thee, O Israel: thou canst not stand before thine enemies, until ye take away the accursed thing from among you.
\verse In the morning therefore ye shall be brought according to your tribes: and it shall be, that the tribe which the \LORD taketh shall come according to the families thereof; and the family which the \LORD shall take shall come by households; and the household which the \LORD shall take shall come man by man.
\verse And it shall be, that he that is taken with the accursed thing shall be burnt with fire, he and all that he hath: because he hath transgressed the covenant of the \LORD, and because he hath wrought folly in Israel.
\verse So Joshua rose up early in the morning, and brought Israel by their tribes; and the tribe of Judah was taken:
\verse And he brought the family of Judah; and he took the family of the Zarhites: and he brought the family of the Zarhites man by man; and Zabdi was taken:
\verse And he brought his household man by man; and Achan, the son of Carmi, the son of Zabdi, the son of Zerah, of the tribe of Judah, was taken.
\verse And Joshua said unto Achan, My son, give, I pray thee, glory to the \LORD God of Israel, and make confession unto him; and tell me now what thou hast done; hide it not from me.
\verse And Achan answered Joshua, and said, Indeed I have sinned against the \LORD God of Israel, and thus and thus have I done:
\verse When I saw among the spoils a goodly Babylonish garment, and two hundred shekels of silver, and a wedge of gold of fifty shekels weight, then I coveted them, and took them; and, behold, they are hid in the earth in the midst of my tent, and the silver under it.
\verse So Joshua sent messengers, and they ran unto the tent; and, behold, it was hid in his tent, and the silver under it.
\verse And they took them out of the midst of the tent, and brought them unto Joshua, and unto all the children of Israel, and laid them out before the \LORD.
\verse And Joshua, and all Israel with him, took Achan the son of Zerah, and the silver, and the garment, and the wedge of gold, and his sons, and his daughters, and his oxen, and his asses, and his sheep, and his tent, and all that he had: and they brought them unto the valley of Achor.
\verse And Joshua said, Why hast thou troubled us? the \LORD shall trouble thee this day. And all Israel stoned him with stones, and burned them with fire, after they had stoned them with stones.
\verse And they raised over him a great heap of stones unto this day. So the \LORD turned from the fierceness of his anger. Wherefore the name of that place was called, The valley of Achor, unto this day.
\end{biblechapter}

\begin{biblechapter} % Joshua 8
\verseWithHeading{Ai destroyed} And the \LORD said unto Joshua, Fear not, neither be thou dismayed: take all the people of war with thee, and arise, go up to Ai: see, I have given into thy hand the king of Ai, and his people, and his city, and his land:
\verse And thou shalt do to Ai and her king as thou didst unto Jericho and her king: only the spoil thereof, and the cattle thereof, shall ye take for a prey unto yourselves: lay thee an ambush for the city behind it.
\verse So Joshua arose, and all the people of war, to go up against Ai: and Joshua chose out thirty thousand mighty men of valour, and sent them away by night.
\verse And he commanded them, saying, Behold, ye shall lie in wait against the city, even behind the city: go not very far from the city, but be ye all ready:
\verse And I, and all the people that are with me, will approach unto the city: and it shall come to pass, when they come out against us, as at the first, that we will flee before them,
\verse (For they will come out after us) till we have drawn them from the city; for they will say, They flee before us, as at the first: therefore we will flee before them.
\verse Then ye shall rise up from the ambush, and seize upon the city: for the \LORD your God will deliver it into your hand.
\verse And it shall be, when ye have taken the city, that ye shall set the city on fire: according to the commandment of the \LORD shall ye do. See, I have commanded you.
\verse Joshua therefore sent them forth: and they went to lie in ambush, and abode between Bethel and Ai, on the west side of Ai: but Joshua lodged that night among the people.
\verse And Joshua rose up early in the morning, and numbered the people, and went up, he and the elders of Israel, before the people to Ai.
\verse And all the people, even the people of war that were with him, went up, and drew nigh, and came before the city, and pitched on the north side of Ai: now there was a valley between them and Ai.
\verse And he took about five thousand men, and set them to lie in ambush between Bethel and Ai, on the west side of the city.
\verse And when they had set the people, even all the host that was on the north of the city, and their liers in wait on the west of the city, Joshua went that night into the midst of the valley.
\verse And it came to pass, when the king of Ai saw it, that they hasted and rose up early, and the men of the city went out against Israel to battle, he and all his people, at a time appointed, before the plain; but he wist not that there were liers in ambush against him behind the city.
\verse And Joshua and all Israel made as if they were beaten before them, and fled by the way of the wilderness.
\verse And all the people that were in Ai were called together to pursue after them: and they pursued after Joshua, and were drawn away from the city.
\verse And there was not a man left in Ai or Bethel, that went not out after Israel: and they left the city open, and pursued after Israel.
\verse And the \LORD said unto Joshua, Stretch out the spear that is in thy hand toward Ai; for I will give it into thine hand. And Joshua stretched out the spear that he had in his hand toward the city.
\verse And the ambush arose quickly out of their place, and they ran as soon as he had stretched out his hand: and they entered into the city, and took it, and hasted and set the city on fire.
\verse And when the men of Ai looked behind them, they saw, and, behold, the smoke of the city ascended up to heaven, and they had no power to flee this way or that way: and the people that fled to the wilderness turned back upon the pursuers.
\verse And when Joshua and all Israel saw that the ambush had taken the city, and that the smoke of the city ascended, then they turned again, and slew the men of Ai.
\verse And the other issued out of the city against them; so they were in the midst of Israel, some on this side, and some on that side: and they smote them, so that they let none of them remain or escape.
\verse And the king of Ai they took alive, and brought him to Joshua.
\verse And it came to pass, when Israel had made an end of slaying all the inhabitants of Ai in the field, in the wilderness wherein they chased them, and when they were all fallen on the edge of the sword, until they were consumed, that all the Israelites returned unto Ai, and smote it with the edge of the sword.
\verse And so it was, that all that fell that day, both of men and women, were twelve thousand, even all the men of Ai.
\verse For Joshua drew not his hand back, wherewith he stretched out the spear, until he had utterly destroyed all the inhabitants of Ai.
\verse Only the cattle and the spoil of that city Israel took for a prey unto themselves, according unto the word of the \LORD which he commanded Joshua.
\verse And Joshua burnt Ai, and made it an heap for ever, even a desolation unto this day.
\verse And the king of Ai he hanged on a tree until eventide: and as soon as the sun was down, Joshua commanded that they should take his carcase down from the tree, and cast it at the entering of the gate of the city, and raise thereon a great heap of stones, that remaineth unto this day.
\verseWithHeading{The covenant renewed at mount Ebal} Then Joshua built an altar unto the \LORD God of Israel in mount Ebal,
\verse As Moses the servant of the \LORD commanded the children of Israel, as it is written in the book of the law of Moses, an altar of whole stones, over which no man hath lift up any iron: and they offered thereon burnt offerings unto the \LORD, and sacrificed peace offerings.
\verse And he wrote there upon the stones a copy of the law of Moses, which he wrote in the presence of the children of Israel.
\verse And all Israel, and their elders, and officers, and their judges, stood on this side the ark and on that side before the priests the Levites, which bare the ark of the covenant of the \LORD, as well the stranger, as he that was born among them; half of them over against mount Gerizim, and half of them over against mount Ebal; as Moses the servant of the \LORD had commanded before, that they should bless the people of Israel.
\verse And afterward he read all the words of the law, the blessings and cursings, according to all that is written in the book of the law.
\verse There was not a word of all that Moses commanded, which Joshua read not before all the congregation of Israel, with the women, and the little ones, and the strangers that were conversant among them.
\end{biblechapter}

\begin{biblechapter} % Joshua 9
\verseWithHeading{The Gibeonite deception} And it came to pass, when all the kings which were on this side Jordan, in the hills, and in the valleys, and in all the coasts of the great sea over against Lebanon, the Hittite, and the Amorite, the Canaanite, the Perizzite, the Hivite, and the Jebusite, heard thereof;
\verse That they gathered themselves together, to fight with Joshua and with Israel, with one accord.
\verse And when the inhabitants of Gibeon heard what Joshua had done unto Jericho and to Ai,
\verse They did work wilily, and went and made as if they had been ambassadors, and took old sacks upon their asses, and wine bottles, old, and rent, and bound up;
\verse And old shoes and clouted upon their feet, and old garments upon them; and all the bread of their provision was dry and mouldy.
\verse And they went to Joshua unto the camp at Gilgal, and said unto him, and to the men of Israel, We be come from a far country: now therefore make ye a league with us.
\verse And the men of Israel said unto the Hivites, Peradventure ye dwell among us; and how shall we make a league with you?
\verse And they said unto Joshua, We are thy servants. And Joshua said unto them, Who are ye? and from whence come ye?
\verse And they said unto him, From a very far country thy servants are come because of the name of the \LORD thy God: for we have heard the fame of him, and all that he did in Egypt,
\verse And all that he did to the two kings of the Amorites, that were beyond Jordan, to Sihon king of Heshbon, and to Og king of Bashan, which was at Ashtaroth.
\verse Wherefore our elders and all the inhabitants of our country spake to us, saying, Take victuals with you for the journey, and go to meet them, and say unto them, We are your servants: therefore now make ye a league with us.
\verse This our bread we took hot for our provision out of our houses on the day we came forth to go unto you; but now, behold, it is dry, and it is mouldy:
\verse And these bottles of wine, which we filled, were new; and, behold, they be rent: and these our garments and our shoes are become old by reason of the very long journey.
\verse And the men took of their victuals, and asked not counsel at the mouth of the \LORD.
\verse And Joshua made peace with them, and made a league with them, to let them live: and the princes of the congregation sware unto them.
\verse And it came to pass at the end of three days after they had made a league with them, that they heard that they were their neighbours, and that they dwelt among them.
\verse And the children of Israel journeyed, and came unto their cities on the third day. Now their cities were Gibeon, and Chephirah, and Beeroth, and Kirjathjearim.
\verse And the children of Israel smote them not, because the princes of the congregation had sworn unto them by the \LORD God of Israel. And all the congregation murmured against the princes.
\verse But all the princes said unto all the congregation, We have sworn unto them by the \LORD God of Israel: now therefore we may not touch them.
\verse This we will do to them; we will even let them live, lest wrath be upon us, because of the oath which we sware unto them.
\verse And the princes said unto them, Let them live; but let them be hewers of wood and drawers of water unto all the congregation; as the princes had promised them.
\verse And Joshua called for them, and he spake unto them, saying, Wherefore have ye beguiled us, saying, We are very far from you; when ye dwell among us?
\verse Now therefore ye are cursed, and there shall none of you be freed from being bondmen, and hewers of wood and drawers of water for the house of my God.
\verse And they answered Joshua, and said, Because it was certainly told thy servants, how that the \LORD thy God commanded his servant Moses to give you all the land, and to destroy all the inhabitants of the land from before you, therefore we were sore afraid of our lives because of you, and have done this thing.
\verse And now, behold, we are in thine hand: as it seemeth good and right unto thee to do unto us, do.
\verse And so did he unto them, and delivered them out of the hand of the children of Israel, that they slew them not.
\verse And Joshua made them that day hewers of wood and drawers of water for the congregation, and for the altar of the \LORD, even unto this day, in the place which he should choose.
\end{biblechapter}

\begin{biblechapter} % Joshua 10
\verseWithHeading{The sun stands still} Now it came to pass, when Adonizedek king of Jerusalem had heard how Joshua had taken Ai, and had utterly destroyed it; as he had done to Jericho and her king, so he had done to Ai and her king; and how the inhabitants of Gibeon had made peace with Israel, and were among them;
\verse That they feared greatly, because Gibeon was a great city, as one of the royal cities, and because it was greater than Ai, and all the men thereof were mighty.
\verse Wherefore Adonizedek king of Jerusalem sent unto Hoham king of Hebron, and unto Piram king of Jarmuth, and unto Japhia king of Lachish, and unto Debir king of Eglon, saying,
\verse Come up unto me, and help me, that we may smite Gibeon: for it hath made peace with Joshua and with the children of Israel.
\verse Therefore the five kings of the Amorites, the king of Jerusalem, the king of Hebron, the king of Jarmuth, the king of Lachish, the king of Eglon, gathered themselves together, and went up, they and all their hosts, and encamped before Gibeon, and made war against it.
\verse And the men of Gibeon sent unto Joshua to the camp to Gilgal, saying, Slack not thy hand from thy servants; come up to us quickly, and save us, and help us: for all the kings of the Amorites that dwell in the mountains are gathered together against us.
\verse So Joshua ascended from Gilgal, he, and all the people of war with him, and all the mighty men of valour.
\verse And the \LORD said unto Joshua, Fear them not: for I have delivered them into thine hand; there shall not a man of them stand before thee.
\verse Joshua therefore came unto them suddenly, and went up from Gilgal all night.
\verse And the \LORD discomfited them before Israel, and slew them with a great slaughter at Gibeon, and chased them along the way that goeth up to Bethhoron, and smote them to Azekah, and unto Makkedah.
\verse And it came to pass, as they fled from before Israel, and were in the going down to Bethhoron, that the \LORD cast down great stones from heaven upon them unto Azekah, and they died: they were more which died with hailstones than they whom the children of Israel slew with the sword.
\verse Then spake Joshua to the \LORD in the day when the \LORD delivered up the Amorites before the children of Israel, and he said in the sight of Israel, Sun, stand thou still upon Gibeon; and thou, Moon, in the valley of Ajalon.
\verse And the sun stood still, and the moon stayed, until the people had avenged themselves upon their enemies. Is not this written in the book of Jasher? So the sun stood still in the midst of heaven, and hasted not to go down about a whole day.
\verse And there was no day like that before it or after it, that the \LORD hearkened unto the voice of a man: for the \LORD fought for Israel.
\verse And Joshua returned, and all Israel with him, unto the camp to Gilgal.
\verseWithHeading{Five Amorite kings killed} But these five kings fled, and hid themselves in a cave at Makkedah.
\verse And it was told Joshua, saying, The five kings are found hid in a cave at Makkedah.
\verse And Joshua said, Roll great stones upon the mouth of the cave, and set men by it for to keep them:
\verse And stay ye not, but pursue after your enemies, and smite the hindmost of them; suffer them not to enter into their cities: for the \LORD your God hath delivered them into your hand.
\verse And it came to pass, when Joshua and the children of Israel had made an end of slaying them with a very great slaughter, till they were consumed, that the rest which remained of them entered into fenced cities.
\verse And all the people returned to the camp to Joshua at Makkedah in peace: none moved his tongue against any of the children of Israel.
\verse Then said Joshua, Open the mouth of the cave, and bring out those five kings unto me out of the cave.
\verse And they did so, and brought forth those five kings unto him out of the cave, the king of Jerusalem, the king of Hebron, the king of Jarmuth, the king of Lachish, and the king of Eglon.
\verse And it came to pass, when they brought out those kings unto Joshua, that Joshua called for all the men of Israel, and said unto the captains of the men of war which went with him, Come near, put your feet upon the necks of these kings. And they came near, and put their feet upon the necks of them.
\verse And Joshua said unto them, Fear not, nor be dismayed, be strong and of good courage: for thus shall the \LORD do to all your enemies against whom ye fight.
\verse And afterward Joshua smote them, and slew them, and hanged them on five trees: and they were hanging upon the trees until the evening.
\verse And it came to pass at the time of the going down of the sun, that Joshua commanded, and they took them down off the trees, and cast them into the cave wherein they had been hid, and laid great stones in the cave's mouth, which remain until this very day.
\vfill\columnbreak % layout hack
\verseWithHeading{Southern cities conquered} And that day Joshua took Makkedah, and smote it with the edge of the sword, and the king thereof he utterly destroyed, them, and all the souls that were therein; he let none remain: and he did to the king of Makkedah as he did unto the king of Jericho.
\verse Then Joshua passed from Makkedah, and all Israel with him, unto Libnah, and fought against Libnah:
\verse And the \LORD delivered it also, and the king thereof, into the hand of Israel; and he smote it with the edge of the sword, and all the souls that were therein; he let none remain in it; but did unto the king thereof as he did unto the king of Jericho.
\verse And Joshua passed from Libnah, and all Israel with him, unto Lachish, and encamped against it, and fought against it:
\verse And the \LORD delivered Lachish into the hand of Israel, which took it on the second day, and smote it with the edge of the sword, and all the souls that were therein, according to all that he had done to Libnah.
\verse Then Horam king of Gezer came up to help Lachish; and Joshua smote him and his people, until he had left him none remaining.
\verse And from Lachish Joshua passed unto Eglon, and all Israel with him; and they encamped against it, and fought against it:
\verse And they took it on that day, and smote it with the edge of the sword, and all the souls that were therein he utterly destroyed that day, according to all that he had done to Lachish.
\verse And Joshua went up from Eglon, and all Israel with him, unto Hebron; and they fought against it:
\verse And they took it, and smote it with the edge of the sword, and the king thereof, and all the cities thereof, and all the souls that were therein; he left none remaining, according to all that he had done to Eglon; but destroyed it utterly, and all the souls that were therein.
\verse And Joshua returned, and all Israel with him, to Debir; and fought against it:
\verse And he took it, and the king thereof, and all the cities thereof; and they smote them with the edge of the sword, and utterly destroyed all the souls that were therein; he left none remaining: as he had done to Hebron, so he did to Debir, and to the king thereof; as he had done also to Libnah, and to her king.
\verse So Joshua smote all the country of the hills, and of the south, and of the vale, and of the springs, and all their kings: he left none remaining, but utterly destroyed all that breathed, as the \LORD God of Israel commanded.
\verse And Joshua smote them from Kadeshbarnea even unto Gaza, and all the country of Goshen, even unto Gibeon.
\verse And all these kings and their land did Joshua take at one time, because the \LORD God of Israel fought for Israel.
\verse And Joshua returned, and all Israel with him, unto the camp to Gilgal.
\end{biblechapter}

\begin{biblechapter} % Joshua 11
\verseWithHeading{Northern kings defeated} And it came to pass, when Jabin king of Hazor had heard those things, that he sent to Jobab king of Madon, and to the king of Shimron, and to the king of Achshaph,
\verse And to the kings that were on the north of the mountains, and of the plains south of Chinneroth, and in the valley, and in the borders of Dor on the west,
\verse And to the Canaanite on the east and on the west, and to the Amorite, and the Hittite, and the Perizzite, and the Jebusite in the mountains, and to the Hivite under Hermon in the land of Mizpeh.
\verse And they went out, they and all their hosts with them, much people, even as the sand that is upon the sea shore in multitude, with horses and chariots very many.
\verse And when all these kings were met together, they came and pitched together at the waters of Merom, to fight against Israel.
\verse And the \LORD said unto Joshua, Be not afraid because of them: for to morrow about this time will I deliver them up all slain before Israel: thou shalt hough their horses, and burn their chariots with fire.
\verse So Joshua came, and all the people of war with him, against them by the waters of Merom suddenly; and they fell upon them.
\verse And the \LORD delivered them into the hand of Israel, who smote them, and chased them unto great Zidon, and unto Misrephothmaim, and unto the valley of Mizpeh eastward; and they smote them, until they left them none remaining.
\verse And Joshua did unto them as the \LORD bade him: he houghed their horses, and burnt their chariots with fire.
\verse And Joshua at that time turned back, and took Hazor, and smote the king thereof with the sword: for Hazor beforetime was the head of all those kingdoms.
\verse And they smote all the souls that were therein with the edge of the sword, utterly destroying them: there was not any left to breathe: and he burnt Hazor with fire.
\verse And all the cities of those kings, and all the kings of them, did Joshua take, and smote them with the edge of the sword, and he utterly destroyed them, as Moses the servant of the \LORD commanded.
\verse But as for the cities that stood still in their strength, Israel burned none of them, save Hazor only; that did Joshua burn.
\verse And all the spoil of these cities, and the cattle, the children of Israel took for a prey unto themselves; but every man they smote with the edge of the sword, until they had destroyed them, neither left they any to breathe.
\verse As the \LORD commanded Moses his servant, so did Moses command Joshua, and so did Joshua; he left nothing undone of all that the \LORD commanded Moses.
\verse So Joshua took all that land, the hills, and all the south country, and all the land of Goshen, and the valley, and the plain, and the mountain of Israel, and the valley of the same;
\verse Even from the mount Halak, that goeth up to Seir, even unto Baalgad in the valley of Lebanon under mount Hermon: and all their kings he took, and smote them, and slew them.
\verse Joshua made war a long time with all those kings.
\verse There was not a city that made peace with the children of Israel, save the Hivites the inhabitants of Gibeon: all other they took in battle.
\verse For it was of the \LORD to harden their hearts, that they should come against Israel in battle, that he might destroy them utterly, and that they might have no favour, but that he might destroy them, as the \LORD commanded Moses.
\verse And at that time came Joshua, and cut off the Anakims from the mountains, from Hebron, from Debir, from Anab, and from all the mountains of Judah, and from all the mountains of Israel: Joshua destroyed them utterly with their cities.
\verse There was none of the Anakims left in the land of the children of Israel: only in Gaza, in Gath, and in Ashdod, there remained.
\verse So Joshua took the whole land, according to all that the \LORD said unto Moses; and Joshua gave it for an inheritance unto Israel according to their divisions by their tribes. And the land rested from war.
\end{biblechapter}

\begin{biblechapter} % Joshua 12
\verseWithHeading{List of defeated kings} Now these are the kings of the land, which the children of Israel smote, and possessed their land on the other side Jordan toward the rising of the sun, from the river Arnon unto mount Hermon, and all the plain on the east:
\verse Sihon king of the Amorites, who dwelt in Heshbon, and ruled from Aroer, which is upon the bank of the river Arnon, and from the middle of the river, and from half Gilead, even unto the river Jabbok, which is the border of the children of Ammon;
\verse And from the plain to the sea of Chinneroth on the east, and unto the sea of the plain, even the salt sea on the east, the way to Bethjeshimoth; and from the south, under Ashdothpisgah:
\verse And the coast of Og king of Bashan, which was of the remnant of the giants, that dwelt at Ashtaroth and at Edrei,
\verse And reigned in mount Hermon, and in Salcah, and in all Bashan, unto the border of the Geshurites and the Maachathites, and half Gilead, the border of Sihon king of Heshbon.
\verse Them did Moses the servant of the \LORD and the children of Israel smite: and Moses the servant of the \LORD gave it for a possession unto the Reubenites, and the Gadites, and the half tribe of Manasseh.
\verse And these are the kings of the country which Joshua and the children of Israel smote on this side Jordan on the west, from Baalgad in the valley of Lebanon even unto the mount Halak, that goeth up to Seir; which Joshua gave unto the tribes of Israel for a possession according to their divisions;
\verse In the mountains, and in the valleys, and in the plains, and in the springs, and in the wilderness, and in the south country; the Hittites, the Amorites, and the Canaanites, the Perizzites, the Hivites, and the Jebusites:
\verse The king of Jericho, one; the king of Ai, which is beside Bethel, one;
\verse The king of Jerusalem, one; the king of Hebron, one;
\verse The king of Jarmuth, one; the king of Lachish, one;
\verse The king of Eglon, one; the king of Gezer, one;
\verse The king of Debir, one; the king of Geder, one;
\verse The king of Hormah, one; the king of Arad, one;
\verse The king of Libnah, one; the king of Adullam, one;
\verse The king of Makkedah, one; the king of Bethel, one;
\verse The king of Tappuah, one; the king of Hepher, one;
\verse The king of Aphek, one; the king of Lasharon, one;
\verse The king of Madon, one; the king of Hazor, one;
\verse The king of Shimronmeron, one; the king of Achshaph, one;
\verse The king of Taanach, one; the king of Megiddo, one;
\verse The king of Kedesh, one; the king of Jokneam of Carmel, one;
\verse The king of Dor in the coast of Dor, one; the king of the nations of Gilgal, one;
\verse The king of Tirzah, one: all the kings thirty and one.
\end{biblechapter}

\begin{biblechapter} % Joshua 13
\verseWithHeading{Land still to be taken} Now Joshua was old and stricken in years; and the \LORD said unto him, Thou art old and stricken in years, and there remaineth yet very much land to be possessed.
\verse This is the land that yet remaineth: all the borders of the Philistines, and all Geshuri,
\verse From Sihor, which is before Egypt, even unto the borders of Ekron northward, which is counted to the Canaanite: five lords of the Philistines; the Gazathites, and the Ashdothites, the Eshkalonites, the Gittites, and the Ekronites; also the Avites:
\verse From the south, all the land of the Canaanites, and Mearah that is beside the Sidonians, unto Aphek, to the borders of the Amorites:
\verse And the land of the Giblites, and all Lebanon, toward the sunrising, from Baalgad under mount Hermon unto the entering into Hamath.
\verse All the inhabitants of the hill country from Lebanon unto Misrephothmaim, and all the Sidonians, them will I drive out from before the children of Israel: only divide thou it by lot unto the Israelites for an inheritance, as I have commanded thee.
\verse Now therefore divide this land for an inheritance unto the nine tribes, and the half tribe of Manasseh,
\verseWithHeading{Division of the land east of the Jordan} With whom the Reubenites and the Gadites have received their inheritance, which Moses gave them, beyond Jordan eastward, even as Moses the servant of the \LORD gave them;
\verse From Aroer, that is upon the bank of the river Arnon, and the city that is in the midst of the river, and all the plain of Medeba unto Dibon;
\verse And all the cities of Sihon king of the Amorites, which reigned in Heshbon, unto the border of the children of Ammon;
\verse And Gilead, and the border of the Geshurites and Maachathites, and all mount Hermon, and all Bashan unto Salcah;
\verse All the kingdom of Og in Bashan, which reigned in Ashtaroth and in Edrei, who remained of the remnant of the giants: for these did Moses smite, and cast them out.
\verse Nevertheless the children of Israel expelled not the Geshurites, nor the Maachathites: but the Geshurites and the Maachathites dwell among the Israelites until this day.
\verse Only unto the tribe of Levi he gave none inheritance; the sacrifices of the \LORD God of Israel made by fire are their inheritance, as he said unto them.
\verse And Moses gave unto the tribe of the children of Reuben inheritance according to their families.
\verse And their coast was from Aroer, that is on the bank of the river Arnon, and the city that is in the midst of the river, and all the plain by Medeba;
\verse Heshbon, and all her cities that are in the plain; Dibon, and Bamothbaal, and Bethbaalmeon,
\verse And Jahazah, and Kedemoth, and Mephaath,
\verse And Kirjathaim, and Sibmah, and Zarethshahar in the mount of the valley,
\verse And Bethpeor, and Ashdothpisgah, and Bethjeshimoth,
\verse And all the cities of the plain, and all the kingdom of Sihon king of the Amorites, which reigned in Heshbon, whom Moses smote with the princes of Midian, Evi, and Rekem, and Zur, and Hur, and Reba, which were dukes of Sihon, dwelling in the country.
\verse Balaam also the son of Beor, the soothsayer, did the children of Israel slay with the sword among them that were slain by them.
\verse And the border of the children of Reuben was Jordan, and the border thereof. This was the inheritance of the children of Reuben after their families, the cities and the villages thereof.
\verse And Moses gave inheritance unto the tribe of Gad, even unto the children of Gad according to their families.
\verse And their coast was Jazer, and all the cities of Gilead, and half the land of the children of Ammon, unto Aroer that is before Rabbah;
\verse And from Heshbon unto Ramathmizpeh, and Betonim; and from Mahanaim unto the border of Debir;
\verse And in the valley, Betharam, and Bethnimrah, and Succoth, and Zaphon, the rest of the kingdom of Sihon king of Heshbon, Jordan and his border, even unto the edge of the sea of Chinnereth on the other side Jordan eastward.
\verse This is the inheritance of the children of Gad after their families, the cities, and their villages.
\verse And Moses gave inheritance unto the half tribe of Manasseh: and this was the possession of the half tribe of the children of Manasseh by their families.
\verse And their coast was from Mahanaim, all Bashan, all the kingdom of Og king of Bashan, and all the towns of Jair, which are in Bashan, threescore cities:
\verse And half Gilead, and Ashtaroth, and Edrei, cities of the kingdom of Og in Bashan, were pertaining unto the children of Machir the son of Manasseh, even to the one half of the children of Machir by their families.
\verse These are the countries which Moses did distribute for inheritance in the plains of Moab, on the other side Jordan, by Jericho, eastward.
\verse But unto the tribe of Levi Moses gave not any inheritance: the \LORD God of Israel was their inheritance, as he said unto them.
\end{biblechapter}

\begin{biblechapter} % Joshua 14
\verseWithHeading{Division of the land west of \newline the Jordan} And these are the countries which the children of Israel inherited in the land of Canaan, which Eleazar the priest, and Joshua the son of Nun, and the heads of the fathers of the tribes of the children of Israel, distributed for inheritance to them.
\verse By lot was their inheritance, as the \LORD commanded by the hand of Moses, for the nine tribes, and for the half tribe.
\verse For Moses had given the inheritance of two tribes and an half tribe on the other side Jordan: but unto the Levites he gave none inheritance among them.
\verse For the children of Joseph were two tribes, Manasseh and Ephraim: therefore they gave no part unto the Levites in the land, save cities to dwell in, with their suburbs for their cattle and for their substance.
\verse As the \LORD commanded Moses, so the children of Israel did, and they divided the land.
\verseWithHeading{Land allotted to Caleb} Then the children of Judah came unto Joshua in Gilgal: and Caleb the son of Jephunneh the Kenezite said unto him, Thou knowest the thing that the \LORD said unto Moses the man of God concerning me and thee in Kadeshbarnea.
\verse Forty years old was I when Moses the servant of the \LORD sent me from Kadeshbarnea to espy out the land; and I brought him word again as it was in mine heart.
\verse Nevertheless my brethren that went up with me made the heart of the people melt: but I wholly followed the \LORD my God.
\verse And Moses sware on that day, saying, Surely the land whereon thy feet have trodden shall be thine inheritance, and thy children's for ever, because thou hast wholly followed the \LORD my God.
\verse And now, behold, the \LORD hath kept me alive, as he said, these forty and five years, even since the \LORD spake this word unto Moses, while the children of Israel wandered in the wilderness: and now, lo, I am this day fourscore and five years old.
\verse As yet I am as strong this day as I was in the day that Moses sent me: as my strength was then, even so is my strength now, for war, both to go out, and to come in.
\verse Now therefore give me this mountain, whereof the \LORD spake in that day; for thou heardest in that day how the Anakims were there, and that the cities were great and fenced: if so be the \LORD will be with me, then I shall be able to drive them out, as the \LORD said.
\verse And Joshua blessed him, and gave unto Caleb the son of Jephunneh Hebron for an inheritance.
\verse Hebron therefore became the inheritance of Caleb the son of Jephunneh the Kenezite unto this day, because that he wholly followed the \LORD God of Israel.
\verse And the name of Hebron before was Kirjatharba; which Arba was a great man among the Anakims. And the land had rest from war.
\end{biblechapter}

\begin{biblechapter} % Joshua 15
\verseWithHeading{Land allotted to Judah} This then was the lot of the tribe of the children of Judah by their families; even to the border of Edom the wilderness of Zin southward was the uttermost part of the south coast.
\verse And their south border was from the shore of the salt sea, from the bay that looketh southward:
\verse And it went out to the south side to Maalehacrabbim, and passed along to Zin, and ascended up on the south side unto Kadeshbarnea, and passed along to Hezron, and went up to Adar, and fetched a compass to Karkaa:
\verse From thence it passed toward Azmon, and went out unto the river of Egypt; and the goings out of that coast were at the sea: this shall be your south coast.
\verse And the east border was the salt sea, even unto the end of Jordan. And their border in the north quarter was from the bay of the sea at the uttermost part of Jordan:
\verse And the border went up to Bethhogla, and passed along by the north of Betharabah; and the border went up to the stone of Bohan the son of Reuben:
\verse And the border went up toward Debir from the valley of Achor, and so northward, looking toward Gilgal, that is before the going up to Adummim, which is on the south side of the river: and the border passed toward the waters of Enshemesh, and the goings out thereof were at Enrogel:
\verse And the border went up by the valley of the son of Hinnom unto the south side of the Jebusite; the same is Jerusalem: and the border went up to the top of the mountain that lieth before the valley of Hinnom westward, which is at the end of the valley of the giants northward:
\verse And the border was drawn from the top of the hill unto the fountain of the water of Nephtoah, and went out to the cities of mount Ephron; and the border was drawn to Baalah, which is Kirjathjearim:
\verse And the border compassed from Baalah westward unto mount Seir, and passed along unto the side of mount Jearim, which is Chesalon, on the north side, and went down to Bethshemesh, and passed on to Timnah:
\verse And the border went out unto the side of Ekron northward: and the border was drawn to Shicron, and passed along to mount Baalah, and went out unto Jabneel; and the goings out of the border were at the sea.
\verse And the west border was to the great sea, and the coast thereof. This is the coast of the children of Judah round about according to their families.
\verse And unto Caleb the son of Jephunneh he gave a part among the children of Judah, according to the commandment of the \LORD to Joshua, even the city of Arba the father of Anak, which city is Hebron.
\verse And Caleb drove thence the three sons of Anak, Sheshai, and Ahiman, and Talmai, the children of Anak.
\verse And he went up thence to the inhabitants of Debir: and the name of Debir before was Kirjathsepher.
\verse And Caleb said, He that smiteth Kirjathsepher, and taketh it, to him will I give Achsah my daughter to wife.
\verse And Othniel the son of Kenaz, the brother of Caleb, took it: and he gave him Achsah his daughter to wife.
\verse And it came to pass, as she came unto him, that she moved him to ask of her father a field: and she lighted off her ass; and Caleb said unto her, What wouldest thou?
\verse Who answered, Give me a blessing; for thou hast given me a south land; give me also springs of water. And he gave her the upper springs, and the nether springs.
\verse This is the inheritance of the tribe of the children of Judah according to their families.
\verse And the uttermost cities of the tribe of the children of Judah toward the coast of Edom southward were Kabzeel, and Eder, and Jagur,
\verse And Kinah, and Dimonah, and Adadah,
\verse And Kedesh, and Hazor, and Ithnan,
\verse Ziph, and Telem, and Bealoth,
\verse And Hazor, Hadattah, and Kerioth, and Hezron, which is Hazor,
\verse Amam, and Shema, and Moladah,
\verse And Hazargaddah, and Heshmon, and Bethpalet,
\verse And Hazarshual, and Beersheba, and Bizjothjah,
\verse Baalah, and Iim, and Azem,
\verse And Eltolad, and Chesil, and Hormah,
\verse And Ziklag, and Madmannah, and Sansannah,
\verse And Lebaoth, and Shilhim, and Ain, and Rimmon: all the cities are twenty and nine, with their villages:
\verse And in the valley, Eshtaol, and Zoreah, and Ashnah,
\verse And Zanoah, and Engannim, Tappuah, and Enam,
\verse Jarmuth, and Adullam, Socoh, and Azekah,
\verse And Sharaim, and Adithaim, and Gederah, and Gederothaim; fourteen cities with their villages:
\verse Zenan, and Hadashah, and Migdalgad,
\verse And Dilean, and Mizpeh, and Joktheel,
\verse Lachish, and Bozkath, and Eglon,
\verse And Cabbon, and Lahmam, and Kithlish,
\verse And Gederoth, Bethdagon, and Naamah, and Makkedah; sixteen cities with their villages:
\verse Libnah, and Ether, and Ashan,
\verse And Jiphtah, and Ashnah, and Nezib,
\verse And Keilah, and Achzib, and Mareshah; nine cities with their villages:
\verse Ekron, with her towns and her villages:
\verse From Ekron even unto the sea, all that lay near Ashdod, with their villages:
\verse Ashdod with her towns and her villages, Gaza with her towns and her villages, unto the river of Egypt, and the great sea, and the border thereof:
\verse And in the mountains, Shamir, and Jattir, and Socoh,
\verse And Dannah, and Kirjathsannah, which is Debir,
\verse And Anab, and Eshtemoh, and Anim,
\verse And Goshen, and Holon, and Giloh; eleven cities with their villages:
\verse Arab, and Dumah, and Eshean,
\verse And Janum, and Bethtappuah, and Aphekah,
\verse And Humtah, and Kirjatharba, which is Hebron, and Zior; nine cities with their villages:
\verse Maon, Carmel, and Ziph, and Juttah,
\verse And Jezreel, and Jokdeam, and Zanoah,
\verse Cain, Gibeah, and Timnah; ten cities with their villages:
\verse Halhul, Bethzur, and Gedor,
\verse And Maarath, and Bethanoth, and Eltekon; six cities with their villages:
\verse Kirjathbaal, which is Kirjathjearim, and Rabbah; two cities with their villages:
\verse In the wilderness, Betharabah, Middin, and Secacah,
\verse And Nibshan, and the city of Salt, and Engedi; six cities with their villages.
\verse As for the Jebusites the inhabitants of Jerusalem, the children of Judah could not drive them out: but the Jebusites dwell with the children of Judah at Jerusalem unto this day.
\end{biblechapter}

\begin{biblechapter} % Joshua 16
\verseWithHeading{Land allotted to Ephraim and \newline Manasseh} And the lot of the children of Joseph fell from Jordan by Jericho, unto the water of Jericho on the east, to the wilderness that goeth up from Jericho throughout mount Bethel,
\verse And goeth out from Bethel to Luz, and passeth along unto the borders of Archi to Ataroth,
\verse And goeth down westward to the coast of Japhleti, unto the coast of Bethhoron the nether, and to Gezer: and the goings out thereof are at the sea.
\verse So the children of Joseph, Manasseh and Ephraim, took their inheritance.
\verse And the border of the children of Ephraim according to their families was thus: even the border of their inheritance on the east side was Atarothaddar, unto Bethhoron the upper;
\verse And the border went out toward the sea to Michmethah on the north side; and the border went about eastward unto Taanathshiloh, and passed by it on the east to Janohah;
\verse And it went down from Janohah to Ataroth, and to Naarath, and came to Jericho, and went out at Jordan.
\verse The border went out from Tappuah westward unto the river Kanah; and the goings out thereof were at the sea. This is the inheritance of the tribe of the children of Ephraim by their families.
\verse And the separate cities for the children of Ephraim were among the inheritance of the children of Manasseh, all the cities with their villages.
\verse And they drave not out the Canaanites that dwelt in Gezer: but the Canaanites dwell among the Ephraimites unto this day, and serve under tribute.
\end{biblechapter}

\begin{biblechapter} % Joshua 17
\verse There was also a lot for the tribe of Manasseh; for he was the firstborn of Joseph; to wit, for Machir the firstborn of Manasseh, the father of Gilead: because he was a man of war, therefore he had Gilead and Bashan.
\verse There was also a lot for the rest of the children of Manasseh by their families; for the children of Abiezer, and for the children of Helek, and for the children of Asriel, and for the children of Shechem, and for the children of Hepher, and for the children of Shemida: these were the male children of Manasseh the son of Joseph by their families.
\verse But Zelophehad, the son of Hepher, the son of Gilead, the son of Machir, the son of Manasseh, had no sons, but daughters: and these are the names of his daughters, Mahlah, and Noah, Hoglah, Milcah, and Tirzah.
\verse And they came near before Eleazar the priest, and before Joshua the son of Nun, and before the princes, saying, The \LORD commanded Moses to give us an inheritance among our brethren. Therefore according to the commandment of the \LORD he gave them an inheritance among the brethren of their father.
\verse And there fell ten portions to Manasseh, beside the land of Gilead and Bashan, which were on the other side Jordan;
\verse Because the daughters of Manasseh had an inheritance among his sons: and the rest of Manasseh's sons had the land of Gilead.
\verse And the coast of Manasseh was from Asher to Michmethah, that lieth before Shechem; and the border went along on the right hand unto the inhabitants of Entappuah.
\verse Now Manasseh had the land of Tappuah: but Tappuah on the border of Manasseh belonged to the children of Ephraim;
\verse And the coast descended unto the river Kanah, southward of the river: these cities of Ephraim are among the cities of Manasseh: the coast of Manasseh also was on the north side of the river, and the outgoings of it were at the sea:
\verse Southward it was Ephraim's, and northward it was Manasseh's, and the sea is his border; and they met together in Asher on the north, and in Issachar on the east.
\verse And Manasseh had in Issachar and in Asher Bethshean and her towns, and Ibleam and her towns, and the inhabitants of Dor and her towns, and the inhabitants of Endor and her towns, and the inhabitants of Taanach and her towns, and the inhabitants of Megiddo and her towns, even three countries.
\verse Yet the children of Manasseh could not drive out the inhabitants of those cities; but the Canaanites would dwell in that land.
\verse Yet it came to pass, when the children of Israel were waxen strong, that they put the Canaanites to tribute; but did not utterly drive them out.
\verse And the children of Joseph spake unto Joshua, saying, Why hast thou given me but one lot and one portion to inherit, seeing I am a great people, forasmuch as the \LORD hath blessed me hitherto?
\verse And Joshua answered them, If thou be a great people, then get thee up to the wood country, and cut down for thyself there in the land of the Perizzites and of the giants, if mount Ephraim be too narrow for thee.
\verse And the children of Joseph said, The hill is not enough for us: and all the Canaanites that dwell in the land of the valley have chariots of iron, both they who are of Bethshean and her towns, and they who are of the valley of Jezreel.
\verse And Joshua spake unto the house of Joseph, even to Ephraim and to Manasseh, saying, Thou art a great people, and hast great power: thou shalt not have one lot only:
\verse But the mountain shall be thine; for it is a wood, and thou shalt cut it down: and the outgoings of it shall be thine: for thou shalt drive out the Canaanites, though they have iron chariots, and though they be strong.
\end{biblechapter}

\begin{biblechapter} % Joshua 18
\verseWithHeading{Division of the rest of the \newline land} And the whole congregation of the children of Israel assembled together at Shiloh, and set up the tabernacle of the congregation there. And the land was subdued before them.
\verse And there remained among the children of Israel seven tribes, which had not yet received their inheritance.
\verse And Joshua said unto the children of Israel, How long are ye slack to go to possess the land, which the \LORD God of your fathers hath given you?
\verse Give out from among you three men for each tribe: and I will send them, and they shall rise, and go through the land, and describe it according to the inheritance of them; and they shall come again to me.
\verse And they shall divide it into seven parts: Judah shall abide in their coast on the south, and the house of Joseph shall abide in their coasts on the north.
\verse Ye shall therefore describe the land into seven parts, and bring the description hither to me, that I may cast lots for you here before the \LORD our God.
\verse But the Levites have no part among you; for the priesthood of the \LORD is their inheritance: and Gad, and Reuben, and half the tribe of Manasseh, have received their inheritance beyond Jordan on the east, which Moses the servant of the \LORD gave them.
\verse And the men arose, and went away: and Joshua charged them that went to describe the land, saying, Go and walk through the land, and describe it, and come again to me, that I may here cast lots for you before the \LORD in Shiloh.
\verse And the men went and passed through the land, and described it by cities into seven parts in a book, and came again to Joshua to the host at Shiloh.
\verse And Joshua cast lots for them in Shiloh before the \LORD: and there Joshua divided the land unto the children of Israel according to their divisions.
\verseWithHeading{Land allotted to Benjamin} And the lot of the tribe of the children of Benjamin came up according to their families: and the coast of their lot came forth between the children of Judah and the children of Joseph.
\verse And their border on the north side was from Jordan; and the border went up to the side of Jericho on the north side, and went up through the mountains westward; and the goings out thereof were at the wilderness of Bethaven.
\verse And the border went over from thence toward Luz, to the side of Luz, which is Bethel, southward; and the border descended to Atarothadar, near the hill that lieth on the south side of the nether Bethhoron.
\verse And the border was drawn thence, and compassed the corner of the sea southward, from the hill that lieth before Bethhoron southward; and the goings out thereof were at Kirjathbaal, which is Kirjathjearim, a city of the children of Judah: this was the west quarter.
\verse And the south quarter was from the end of Kirjathjearim, and the border went out on the west, and went out to the well of waters of Nephtoah:
\verse And the border came down to the end of the mountain that lieth before the valley of the son of Hinnom, and which is in the valley of the giants on the north, and descended to the valley of Hinnom, to the side of Jebusi on the south, and descended to Enrogel,
\verse And was drawn from the north, and went forth to Enshemesh, and went forth toward Geliloth, which is over against the going up of Adummim, and descended to the stone of Bohan the son of Reuben,
\verse And passed along toward the side over against Arabah northward, and went down unto Arabah:
\verse And the border passed along to the side of Bethhoglah northward: and the outgoings of the border were at the north bay of the salt sea at the south end of Jordan: this was the south coast.
\verse And Jordan was the border of it on the east side. This was the inheritance of the children of Benjamin, by the coasts thereof round about, according to their families.
\verse Now the cities of the tribe of the children of Benjamin according to their families were Jericho, and Bethhoglah, and the valley of Keziz,
\verse And Betharabah, and Zemaraim, and Bethel,
\verse And Avim, and Parah, and Ophrah,
\verse And Chepharhaammonai, and Ophni, and Gaba; twelve cities with their villages:
\verse Gibeon, and Ramah, and Beeroth,
\verse And Mizpeh, and Chephirah, and Mozah,
\verse And Rekem, and Irpeel, and Taralah,
\verse And Zelah, Eleph, and Jebusi, which is Jerusalem, Gibeath, and Kirjath; fourteen cities with their villages. This is the inheritance of the children of Benjamin according to their families.
\end{biblechapter}

\begin{biblechapter} % Joshua 19
\verseWithHeading{Land allotted to Simeon} And the second lot came forth to Simeon, even for the tribe of the children of Simeon according to their families: and their inheritance was within the inheritance of the children of Judah.
\verse And they had in their inheritance Beersheba, or Sheba, and Moladah,
\verse And Hazarshual, and Balah, and Azem,
\verse And Eltolad, and Bethul, and Hormah,
\verse And Ziklag, and Bethmarcaboth, and Hazarsusah,
\verse And Bethlebaoth, and Sharuhen; thirteen cities and their villages:
\verse Ain, Remmon, and Ether, and Ashan; four cities and their villages:
\verse And all the villages that were round about these cities to Baalathbeer, Ramath of the south. This is the inheritance of the tribe of the children of Simeon according to their families.
\verse Out of the portion of the children of Judah was the inheritance of the children of Simeon: for the part of the children of Judah was too much for them: therefore the children of Simeon had their inheritance within the inheritance of them.
\verseWithHeading{Land allotted to Zebulun} And the third lot came up for the children of Zebulun according to their families: and the border of their inheritance was unto Sarid:
\verse And their border went up toward the sea, and Maralah, and reached to Dabbasheth, and reached to the river that is before Jokneam;
\verse And turned from Sarid eastward toward the sunrising unto the border of Chislothtabor, and then goeth out to Daberath, and goeth up to Japhia,
\verse And from thence passeth on along on the east to Gittahhepher, to Ittahkazin, and goeth out to Remmonmethoar to Neah;
\verse And the border compasseth it on the north side to Hannathon: and the outgoings thereof are in the valley of Jiphthahel:
\verse And Kattath, and Nahallal, and Shimron, and Idalah, and Bethlehem: twelve cities with their villages.
\verse This is the inheritance of the children of Zebulun according to their families, these cities with their villages.
\verseWithHeading{Land allotted to Issachar} And the fourth lot came out to Issachar, for the children of Issachar according to their families.
\verse And their border was toward Jezreel, and Chesulloth, and Shunem,
\verse And Hapharaim, and Shion, and Anaharath,
\verse And Rabbith, and Kishion, and Abez,
\verse And Remeth, and Engannim, and Enhaddah, and Bethpazzez;
\verse And the coast reacheth to Tabor, and Shahazimah, and Bethshemesh; and the outgoings of their border were at Jordan: sixteen cities with their villages.
\verse This is the inheritance of the tribe of the children of Issachar according to their families, the cities and their villages.
\verseWithHeading{Land allotted to Asher} And the fifth lot came out for the tribe of the children of Asher according to their families.
\verse And their border was Helkath, and Hali, and Beten, and Achshaph,
\verse And Alammelech, and Amad, and Misheal; and reacheth to Carmel westward, and to Shihorlibnath;
\verse And turneth toward the sunrising to Bethdagon, and reacheth to Zebulun, and to the valley of Jiphthahel toward the north side of Bethemek, and Neiel, and goeth out to Cabul on the left hand,
\verse And Hebron, and Rehob, and Hammon, and Kanah, even unto great Zidon;
\verse And then the coast turneth to Ramah, and to the strong city Tyre; and the coast turneth to Hosah; and the outgoings thereof are at the sea from the coast to Achzib:
\verse Ummah also, and Aphek, and Rehob: twenty and two cities with their villages.
\verse This is the inheritance of the tribe of the children of Asher according to their families, these cities with their villages.
\verseWithHeading{Land allotted to Naphtali} The sixth lot came out to the children of Naphtali, even for the children of Naphtali according to their families.
\verse And their coast was from Heleph, from Allon to Zaanannim, and Adami, Nekeb, and Jabneel, unto Lakum; and the outgoings thereof were at Jordan:
\verse And then the coast turneth westward to Aznothtabor, and goeth out from thence to Hukkok, and reacheth to Zebulun on the south side, and reacheth to Asher on the west side, and to Judah upon Jordan toward the sunrising.
\verse And the fenced cities are Ziddim, Zer, and Hammath, Rakkath, and Chinnereth,
\verse And Adamah, and Ramah, and Hazor,
\verse And Kedesh, and Edrei, and Enhazor,
\verse And Iron, and Migdalel, Horem, and Bethanath, and Bethshemesh; nineteen cities with their villages.
\verse This is the inheritance of the tribe of the children of Naphtali according to their families, the cities and their villages.
\verseWithHeading{Land allotted to Dan} And the seventh lot came out for the tribe of the children of Dan according to their families.
\verse And the coast of their inheritance was Zorah, and Eshtaol, and Irshemesh,
\verse And Shaalabbin, and Ajalon, and Jethlah,
\verse And Elon, and Thimnathah, and Ekron,
\verse And Eltekeh, and Gibbethon, and Baalath,
\verse And Jehud, and Beneberak, and Gathrimmon,
\verse And Mejarkon, and Rakkon, with the border before Japho.
\verse And the coast of the children of Dan went out too little for them: therefore the children of Dan went up to fight against Leshem, and took it, and smote it with the edge of the sword, and possessed it, and dwelt therein, and called Leshem, Dan, after the name of Dan their father.
\verse This is the inheritance of the tribe of the children of Dan according to their families, these cities with their villages.
\verseWithHeading{Land allotted to Joshua} When they had made an end of dividing the land for inheritance by their coasts, the children of Israel gave an inheritance to Joshua the son of Nun among them:
\verse According to the word of the \LORD they gave him the city which he asked, even Timnathserah in mount Ephraim: and he built the city, and dwelt therein.
\verse These are the inheritances, which Eleazar the priest, and Joshua the son of Nun, and the heads of the fathers of the tribes of the children of Israel, divided for an inheritance by lot in Shiloh before the \LORD, at the door of the tabernacle of the congregation. So they made an end of dividing the country.
\end{biblechapter}

\begin{biblechapter} % Joshua 20
\verseWithHeading{Cities of refuge} The \LORD also spake unto Joshua, saying,
\verse Speak to the children of Israel, saying, Appoint out for you cities of refuge, whereof I spake unto you by the hand of Moses:
\verse That the slayer that killeth any person unawares and unwittingly may flee thither: and they shall be your refuge from the avenger of blood.
\verse And when he that doth flee unto one of those cities shall stand at the entering of the gate of the city, and shall declare his cause in the ears of the elders of that city, they shall take him into the city unto them, and give him a place, that he may dwell among them.
\verse And if the avenger of blood pursue after him, then they shall not deliver the slayer up into his hand; because he smote his neighbour unwittingly, and hated him not beforetime.
\verse And he shall dwell in that city, until he stand before the congregation for judgment, and until the death of the high priest that shall be in those days: then shall the slayer return, and come unto his own city, and unto his own house, unto the city from whence he fled.
\verse And they appointed Kedesh in Galilee in mount Naphtali, and Shechem in mount Ephraim, and Kirjatharba, which is Hebron, in the mountain of Judah.
\verse And on the other side Jordan by Jericho eastward, they assigned Bezer in the wilderness upon the plain out of the tribe of Reuben, and Ramoth in Gilead out of the tribe of Gad, and Golan in Bashan out of the tribe of Manasseh.
\verse These were the cities appointed for all the children of Israel, and for the stranger that sojourneth among them, that whosoever killeth any person at unawares might flee thither, and not die by the hand of the avenger of blood, until he stood before the congregation.
\end{biblechapter}

\begin{biblechapter} % Joshua 21
\verseWithHeading{Cities for the Levites} Then came near the heads of the fathers of the Levites unto Eleazar the priest, and unto Joshua the son of Nun, and unto the heads of the fathers of the tribes of the children of Israel;
\verse And they spake unto them at Shiloh in the land of Canaan, saying, The \LORD commanded by the hand of Moses to give us cities to dwell in, with the suburbs thereof for our cattle.
\verse And the children of Israel gave unto the Levites out of their inheritance, at the commandment of the \LORD, these cities and their suburbs.
\verse And the lot came out for the families of the Kohathites: and the children of Aaron the priest, which were of the Levites, had by lot out of the tribe of Judah, and out of the tribe of Simeon, and out of the tribe of Benjamin, thirteen cities.
\verse And the rest of the children of Kohath had by lot out of the families of the tribe of Ephraim, and out of the tribe of Dan, and out of the half tribe of Manasseh, ten cities.
\verse And the children of Gershon had by lot out of the families of the tribe of Issachar, and out of the tribe of Asher, and out of the tribe of Naphtali, and out of the half tribe of Manasseh in Bashan, thirteen cities.
\verse The children of Merari by their families had out of the tribe of Reuben, and out of the tribe of Gad, and out of the tribe of Zebulun, twelve cities.
\verse And the children of Israel gave by lot unto the Levites these cities with their suburbs, as the \LORD commanded by the hand of Moses.
\verse And they gave out of the tribe of the children of Judah, and out of the tribe of the children of Simeon, these cities which are here mentioned by name,
\verse Which the children of Aaron, being of the families of the Kohathites, who were of the children of Levi, had: for theirs was the first lot.
\verse And they gave them the city of Arba the father of Anak, which city is Hebron, in the hill country of Judah, with the suburbs thereof round about it.
\verse But the fields of the city, and the villages thereof, gave they to Caleb the son of Jephunneh for his possession.
\verse Thus they gave to the children of Aaron the priest Hebron with her suburbs, to be a city of refuge for the slayer; and Libnah with her suburbs,
\verse And Jattir with her suburbs, and Eshtemoa with her suburbs,
\verse And Holon with her suburbs, and Debir with her suburbs,
\verse And Ain with her suburbs, and Juttah with her suburbs, and Bethshemesh with her suburbs; nine cities out of those two tribes.
\verse And out of the tribe of Benjamin, Gibeon with her suburbs, Geba with her suburbs,
\verse Anathoth with her suburbs, and Almon with her suburbs; four cities.
\verse All the cities of the children of Aaron, the priests, were thirteen cities with their suburbs.
\verse And the families of the children of Kohath, the Levites which remained of the children of Kohath, even they had the cities of their lot out of the tribe of Ephraim.
\verse For they gave them Shechem with her suburbs in mount Ephraim, to be a city of refuge for the slayer; and Gezer with her suburbs,
\verse And Kibzaim with her suburbs, and Bethhoron with her suburbs; four cities.
\verse And out of the tribe of Dan, Eltekeh with her suburbs, Gibbethon with her suburbs,
\verse Aijalon with her suburbs, Gathrimmon with her suburbs; four cities.
\verse And out of the half tribe of Manasseh, Tanach with her suburbs, and Gathrimmon with her suburbs; two cities.
\verse All the cities were ten with their suburbs for the families of the children of Kohath that remained.
\verse And unto the children of Gershon, of the families of the Levites, out of the other half tribe of Manasseh they gave Golan in Bashan with her suburbs, to be a city of refuge for the slayer; and Beeshterah with her suburbs; two cities.
\verse And out of the tribe of Issachar, Kishon with her suburbs, Dabareh with her suburbs,
\verse Jarmuth with her suburbs, Engannim with her suburbs; four cities.
\verse And out of the tribe of Asher, Mishal with her suburbs, Abdon with her suburbs,
\verse Helkath with her suburbs, and Rehob with her suburbs; four cities.
\verse And out of the tribe of Naphtali, Kedesh in Galilee with her suburbs, to be a city of refuge for the slayer; and Hammothdor with her suburbs, and Kartan with her suburbs; three cities.
\verse All the cities of the Gershonites according to their families were thirteen cities with their suburbs.
\verse And unto the families of the children of Merari, the rest of the Levites, out of the tribe of Zebulun, Jokneam with her suburbs, and Kartah with her suburbs,
\verse Dimnah with her suburbs, Nahalal with her suburbs; four cities.
\verse And out of the tribe of Reuben, Bezer with her suburbs, and Jahazah with her suburbs,
\verse Kedemoth with her suburbs, and Mephaath with her suburbs; four cities.
\verse And out of the tribe of Gad, Ramoth in Gilead with her suburbs, to be a city of refuge for the slayer; and Mahanaim with her suburbs,
\verse Heshbon with her suburbs, Jazer with her suburbs; four cities in all.
\verse So all the cities for the children of Merari by their families, which were remaining of the families of the Levites, were by their lot twelve cities.
\verse All the cities of the Levites within the possession of the children of Israel were forty and eight cities with their suburbs.
\verse These cities were every one with their suburbs round about them: thus were all these cities.
\verse And the \LORD gave unto Israel all the land which he sware to give unto their fathers; and they possessed it, and dwelt therein.
\verse And the \LORD gave them rest round about, according to all that he sware unto their fathers: and there stood not a man of all their enemies before them; the \LORD delivered all their enemies into their hand.
\verse There failed not ought of any good thing which the \LORD had spoken unto the house of Israel; all came to pass.
\end{biblechapter}

\begin{biblechapter} % Joshua 22
\verseWithHeading{Eastern tribes return home} Then Joshua called the Reubenites, and the Gadites, and the half tribe of Manasseh,
\verse And said unto them, Ye have kept all that Moses the servant of the \LORD commanded you, and have obeyed my voice in all that I commanded you:
\verse Ye have not left your brethren these many days unto this day, but have kept the charge of the commandment of the \LORD your God.
\verse And now the \LORD your God hath given rest unto your brethren, as he promised them: therefore now return ye, and get you unto your tents, and unto the land of your possession, which Moses the servant of the \LORD gave you on the other side Jordan.
\verse But take diligent heed to do the commandment and the law, which Moses the servant of the \LORD charged you, to love the \LORD your God, and to walk in all his ways, and to keep his commandments, and to cleave unto him, and to serve him with all your heart and with all your soul.
\verse So Joshua blessed them, and sent them away: and they went unto their tents.
\verse Now to the one half of the tribe of Manasseh Moses had given possession in Bashan: but unto the other half thereof gave Joshua among their brethren on this side Jordan westward. And when Joshua sent them away also unto their tents, then he blessed them,
\verse And he spake unto them, saying, Return with much riches unto your tents, and with very much cattle, with silver, and with gold, and with brass, and with iron, and with very much raiment: divide the spoil of your enemies with your brethren.
\verse And the children of Reuben and the children of Gad and the half tribe of Manasseh returned, and departed from the children of Israel out of Shiloh, which is in the land of Canaan, to go unto the country of Gilead, to the land of their possession, whereof they were possessed, according to the word of the \LORD by the hand of Moses.
\verse And when they came unto the borders of Jordan, that are in the land of Canaan, the children of Reuben and the children of Gad and the half tribe of Manasseh built there an altar by Jordan, a great altar to see to.
\verse And the children of Israel heard say, Behold, the children of Reuben and the children of Gad and the half tribe of Manasseh have built an altar over against the land of Canaan, in the borders of Jordan, at the passage of the children of Israel.
\verse And when the children of Israel heard of it, the whole congregation of the children of Israel gathered themselves together at Shiloh, to go up to war against them.
\verse And the children of Israel sent unto the children of Reuben, and to the children of Gad, and to the half tribe of Manasseh, into the land of Gilead, Phinehas the son of Eleazar the priest,
\verse And with him ten princes, of each chief house a prince throughout all the tribes of Israel; and each one was an head of the house of their fathers among the thousands of Israel.
\verse And they came unto the children of Reuben, and to the children of Gad, and to the half tribe of Manasseh, unto the land of Gilead, and they spake with them, saying,
\verse Thus saith the whole congregation of the \LORD, What trespass is this that ye have committed against the God of Israel, to turn away this day from following the \LORD, in that ye have builded you an altar, that ye might rebel this day against the \LORD?
\verse Is the iniquity of Peor too little for us, from which we are not cleansed until this day, although there was a plague in the congregation of the \LORD,
\verse But that ye must turn away this day from following the \LORD? and it will be, seeing ye rebel to day against the \LORD, that to morrow he will be wroth with the whole congregation of Israel.
\verse Notwithstanding, if the land of your possession be unclean, then pass ye over unto the land of the possession of the \LORD, wherein the \LORDs tabernacle dwelleth, and take possession among us: but rebel not against the \LORD, nor rebel against us, in building you an altar beside the altar of the \LORD our God.
\verse Did not Achan the son of Zerah commit a trespass in the accursed thing, and wrath fell on all the congregation of Israel? and that man perished not alone in his iniquity.
\verse Then the children of Reuben and the children of Gad and the half tribe of Manasseh answered, and said unto the heads of the thousands of Israel,
\verse The \LORD God of gods, the \LORD God of gods, he knoweth, and Israel he shall know; if it be in rebellion, or if in transgression against the \LORD, (save us not this day,)
\verse That we have built us an altar to turn from following the \LORD, or if to offer thereon burnt offering or meat offering, or if to offer peace offerings thereon, let the \LORD himself require it;
\verse And if we have not rather done it for fear of this thing, saying, In time to come your children might speak unto our children, saying, What have ye to do with the \LORD God of Israel?
\verse For the \LORD hath made Jordan a border between us and you, ye children of Reuben and children of Gad; ye have no part in the \LORD: so shall your children make our children cease from fearing the \LORD.
\verse Therefore we said, Let us now prepare to build us an altar, not for burnt offering, nor for sacrifice:
\verse But that it may be a witness between us, and you, and our generations after us, that we might do the service of the \LORD before him with our burnt offerings, and with our sacrifices, and with our peace offerings; that your children may not say to our children in time to come, Ye have no part in the \LORD.
\verse Therefore said we, that it shall be, when they should so say to us or to our generations in time to come, that we may say again, Behold the pattern of the altar of the \LORD, which our fathers made, not for burnt offerings, nor for sacrifices; but it is a witness between us and you.
\verse God forbid that we should rebel against the \LORD, and turn this day from following the \LORD, to build an altar for burnt offerings, for meat offerings, or for sacrifices, beside the altar of the \LORD our God that is before his tabernacle.
\verse And when Phinehas the priest, and the princes of the congregation and heads of the thousands of Israel which were with him, heard the words that the children of Reuben and the children of Gad and the children of Manasseh spake, it pleased them.
\verse And Phinehas the son of Eleazar the priest said unto the children of Reuben, and to the children of Gad, and to the children of Manasseh, This day we perceive that the \LORD is among us, because ye have not committed this trespass against the \LORD: now ye have delivered the children of Israel out of the hand of the \LORD.
\verse And Phinehas the son of Eleazar the priest, and the princes, returned from the children of Reuben, and from the children of Gad, out of the land of Gilead, unto the land of Canaan, to the children of Israel, and brought them word again.
\verse And the thing pleased the children of Israel; and the children of Israel blessed God, and did not intend to go up against them in battle, to destroy the land wherein the children of Reuben and Gad dwelt.
\verse And the children of Reuben and the children of Gad called the altar Ed: for it shall be a witness between us that the \LORD is God.
\end{biblechapter}

\begin{biblechapter} % Joshua 23
\verseWithHeading{Joshua's farewell to the \newline leaders} And it came to pass a long time after that the \LORD had given rest unto Israel from all their enemies round about, that Joshua waxed old and stricken in age.
\verse And Joshua called for all Israel, and for their elders, and for their heads, and for their judges, and for their officers, and said unto them, I am old and stricken in age:
\verse And ye have seen all that the \LORD your God hath done unto all these nations because of you; for the \LORD your God is he that hath fought for you.
\verse Behold, I have divided unto you by lot these nations that remain, to be an inheritance for your tribes, from Jordan, with all the nations that I have cut off, even unto the great sea westward.
\verse And the \LORD your God, he shall expel them from before you, and drive them from out of your sight; and ye shall possess their land, as the \LORD your God hath promised unto you.
\verse Be ye therefore very courageous to keep and to do all that is written in the book of the law of Moses, that ye turn not aside therefrom to the right hand or to the left;
\verse That ye come not among these nations, these that remain among you; neither make mention of the name of their gods, nor cause to swear by them, neither serve them, nor bow yourselves unto them:
\verse But cleave unto the \LORD your God, as ye have done unto this day.
\verse For the \LORD hath driven out from before you great nations and strong: but as for you, no man hath been able to stand before you unto this day.
\verse One man of you shall chase a thousand: for the \LORD your God, he it is that fighteth for you, as he hath promised you.
\verse Take good heed therefore unto yourselves, that ye love the \LORD your God.
\verse Else if ye do in any wise go back, and cleave unto the remnant of these nations, even these that remain among you, and shall make marriages with them, and go in unto them, and they to you:
\verse Know for a certainty that the \LORD your God will no more drive out any of these nations from before you; but they shall be snares and traps unto you, and scourges in your sides, and thorns in your eyes, until ye perish from off this good land which the \LORD your God hath given you.
\verse And, behold, this day I am going the way of all the earth: and ye know in all your hearts and in all your souls, that not one thing hath failed of all the good things which the \LORD your God spake concerning you; all are come to pass unto you, and not one thing hath failed thereof.
\verse Therefore it shall come to pass, that as all good things are come upon you, which the \LORD your God promised you; so shall the \LORD bring upon you all evil things, until he have destroyed you from off this good land which the \LORD your God hath given you.
\verse When ye have transgressed the covenant of the \LORD your God, which he commanded you, and have gone and served other gods, and bowed yourselves to them; then shall the anger of the \LORD be kindled against you, and ye shall perish quickly from off the good land which he hath given unto you.
\end{biblechapter}

\begin{biblechapter} % Joshua 24
\verseWithHeading{The covenant renewed at \newline Shechem} And Joshua gathered all the tribes of Israel to Shechem, and called for the elders of Israel, and for their heads, and for their judges, and for their officers; and they presented themselves before God.
\verse And Joshua said unto all the people, Thus saith the \LORD God of Israel, Your fathers dwelt on the other side of the flood in old time, even Terah, the father of Abraham, and the father of Nachor: and they served other gods.
\verse And I took your father Abraham from the other side of the flood, and led him throughout all the land of Canaan, and multiplied his seed, and gave him Isaac.
\verse And I gave unto Isaac Jacob and Esau: and I gave unto Esau mount Seir, to possess it; but Jacob and his children went down into Egypt.
\verse I sent Moses also and Aaron, and I plagued Egypt, according to that which I did among them: and afterward I brought you out.
\verse And I brought your fathers out of Egypt: and ye came unto the sea; and the Egyptians pursued after your fathers with chariots and horsemen unto the Red sea.
\verse And when they cried unto the \LORD, he put darkness between you and the Egyptians, and brought the sea upon them, and covered them; and your eyes have seen what I have done in Egypt: and ye dwelt in the wilderness a long season.
\verse And I brought you into the land of the Amorites, which dwelt on the other side Jordan; and they fought with you: and I gave them into your hand, that ye might possess their land; and I destroyed them from before you.
\verse Then Balak the son of Zippor, king of Moab, arose and warred against Israel, and sent and called Balaam the son of Beor to curse you:
\verse But I would not hearken unto Balaam; therefore he blessed you still: so I delivered you out of his hand.
\verse And ye went over Jordan, and came unto Jericho: and the men of Jericho fought against you, the Amorites, and the Perizzites, and the Canaanites, and the Hittites, and the Girgashites, the Hivites, and the Jebusites; and I delivered them into your hand.
\verse And I sent the hornet before you, which drave them out from before you, even the two kings of the Amorites; but not with thy sword, nor with thy bow.
\verse And I have given you a land for which ye did not labour, and cities which ye built not, and ye dwell in them; of the vineyards and oliveyards which ye planted not do ye eat.
\verse Now therefore fear the \LORD, and serve him in sincerity and in truth: and put away the gods which your fathers served on the other side of the flood, and in Egypt; and serve ye the \LORD.
\verse And if it seem evil unto you to serve the \LORD, choose you this day whom ye will serve; whether the gods which your fathers served that were on the other side of the flood, or the gods of the Amorites, in whose land ye dwell: but as for me and my house, we will serve the \LORD.
\verse And the people answered and said, God forbid that we should forsake the \LORD, to serve other gods;
\verse For the \LORD our God, he it is that brought us up and our fathers out of the land of Egypt, from the house of bondage, and which did those great signs in our sight, and preserved us in all the way wherein we went, and among all the people through whom we passed:
\verse And the \LORD drave out from before us all the people, even the Amorites which dwelt in the land: therefore will we also serve the \LORD; for he is our God.
\verse And Joshua said unto the people, Ye cannot serve the \LORD: for he is an holy God; he is a jealous God; he will not forgive your transgressions nor your sins.
\verse If ye forsake the \LORD, and serve strange gods, then he will turn and do you hurt, and consume you, after that he hath done you good.
\verse And the people said unto Joshua, Nay; but we will serve the \LORD.
\verse And Joshua said unto the people, Ye are witnesses against yourselves that ye have chosen you the \LORD, to serve him. And they said, We are witnesses.
\verse Now therefore put away, said he, the strange gods which are among you, and incline your heart unto the \LORD God of Israel.
\verse And the people said unto Joshua, The \LORD our God will we serve, and his voice will we obey.
\verse So Joshua made a covenant with the people that day, and set them a statute and an ordinance in Shechem.
\verse And Joshua wrote these words in the book of the law of God, and took a great stone, and set it up there under an oak, that was by the sanctuary of the \LORD.
\verse And Joshua said unto all the people, Behold, this stone shall be a witness unto us; for it hath heard all the words of the \LORD which he spake unto us: it shall be therefore a witness unto you, lest ye deny your God.
\verse So Joshua let the people depart, every man unto his inheritance.
\verseWithHeading{The death of Joshua} And it came to pass after these things, that Joshua the son of Nun, the servant of the \LORD, died, being an hundred and ten years old.
\verse And they buried him in the border of his inheritance in Timnathserah, which is in mount Ephraim, on the north side of the hill of Gaash.
\verse And Israel served the \LORD all the days of Joshua, and all the days of the elders that overlived Joshua, and which had known all the works of the \LORD, that he had done for Israel.
\verse And the bones of Joseph, which the children of Israel brought up out of Egypt, buried they in Shechem, in a parcel of ground which Jacob bought of the sons of Hamor the father of Shechem for an hundred pieces of silver: and it became the inheritance of the children of Joseph.
\verse And Eleazar the son of Aaron died; and they buried him in a hill that pertained to Phinehas his son, which was given him in mount Ephraim.
\end{biblechapter}
\flushcolsend
\biblebook{Judges}

\begin{biblechapter} % Judges 1
\verseWithHeading{Israel fights the remaining \newline Canaanites} Now after the death of Joshua it came to pass, that the children of Israel asked the \LORD, saying, Who shall go up for us against the Canaanites first, to fight against them?
\verse And the \LORD said, Judah shall go up: behold, I have delivered the land into his hand.
\verse And Judah said unto Simeon his brother, Come up with me into my lot, that we may fight against the Canaanites; and I likewise will go with thee into thy lot. So Simeon went with him.
\verse And Judah went up; and the \LORD delivered the Canaanites and the Perizzites into their hand: and they slew of them in Bezek ten thousand men.
\verse And they found Adonibezek in Bezek: and they fought against him, and they slew the Canaanites and the Perizzites.
\verse But Adonibezek fled; and they pursued after him, and caught him, and cut off his thumbs and his great toes.
\verse And Adonibezek said, Threescore and ten kings, having their thumbs and their great toes cut off, gathered their meat under my table: as I have done, so God hath requited me. And they brought him to Jerusalem, and there he died.
\verse Now the children of Judah had fought against Jerusalem, and had taken it, and smitten it with the edge of the sword, and set the city on fire.
\verse And afterward the children of Judah went down to fight against the Canaanites, that dwelt in the mountain, and in the south, and in the valley.
\verse And Judah went against the Canaanites that dwelt in Hebron: (now the name of Hebron before was Kirjatharba:) and they slew Sheshai, and Ahiman, and Talmai.
\verse And from thence he went against the inhabitants of Debir: and the name of Debir before was Kirjathsepher:
\verse And Caleb said, He that smiteth Kirjathsepher, and taketh it, to him will I give Achsah my daughter to wife.
\verse And Othniel the son of Kenaz, Caleb's younger brother, took it: and he gave him Achsah his daughter to wife.
\verse And it came to pass, when she came to him, that she moved him to ask of her father a field: and she lighted from off her ass; and Caleb said unto her, What wilt thou?
\verse And she said unto him, Give me a blessing: for thou hast given me a south land; give me also springs of water. And Caleb gave her the upper springs and the nether springs.
\verse And the children of the Kenite, Moses' father in law, went up out of the city of palm trees with the children of Judah into the wilderness of Judah, which lieth in the south of Arad; and they went and dwelt among the people.
\verse And Judah went with Simeon his brother, and they slew the Canaanites that inhabited Zephath, and utterly destroyed it. And the name of the city was called Hormah.
\verse Also Judah took Gaza with the coast thereof, and Askelon with the coast thereof, and Ekron with the coast thereof.
\verse And the \LORD was with Judah; and he drave out the inhabitants of the mountain; but could not drive out the inhabitants of the valley, because they had chariots of iron.
\verse And they gave Hebron unto Caleb, as Moses said: and he expelled thence the three sons of Anak.
\verse And the children of Benjamin did not drive out the Jebusites that inhabited Jerusalem; but the Jebusites dwell with the children of Benjamin in Jerusalem unto this day.
\verse And the house of Joseph, they also went up against Bethel: and the \LORD was with them.
\verse And the house of Joseph sent to descry Bethel. (Now the name of the city before was Luz.)
\verse And the spies saw a man come forth out of the city, and they said unto him, Shew us, we pray thee, the entrance into the city, and we will shew thee mercy.
\verse And when he shewed them the entrance into the city, they smote the city with the edge of the sword; but they let go the man and all his family.
\verse And the man went into the land of the Hittites, and built a city, and called the name thereof Luz: which is the name thereof unto this day.
\verse Neither did Manasseh drive out the inhabitants of Bethshean and her towns, nor Taanach and her towns, nor the inhabitants of Dor and her towns, nor the inhabitants of Ibleam and her towns, nor the inhabitants of Megiddo and her towns: but the Canaanites would dwell in that land.
\verse And it came to pass, when Israel was strong, that they put the Canaanites to tribute, and did not utterly drive them out.
\verse Neither did Ephraim drive out the Canaanites that dwelt in Gezer; but the Canaanites dwelt in Gezer among them.
\verse Neither did Zebulun drive out the inhabitants of Kitron, nor the inhabitants of Nahalol; but the Canaanites dwelt among them, and became tributaries.
\verse Neither did Asher drive out the inhabitants of Accho, nor the inhabitants of Zidon, nor of Ahlab, nor of Achzib, nor of Helbah, nor of Aphik, nor of Rehob:
\verse But the Asherites dwelt among the Canaanites, the inhabitants of the land: for they did not drive them out.
\verse Neither did Naphtali drive out the inhabitants of Bethshemesh, nor the inhabitants of Bethanath; but he dwelt among the Canaanites, the inhabitants of the land: nevertheless the inhabitants of Bethshemesh and of Bethanath became tributaries unto them.
\verse And the Amorites forced the children of Dan into the mountain: for they would not suffer them to come down to the valley:
\verse But the Amorites would dwell in mount Heres in Aijalon, and in Shaalbim: yet the hand of the house of Joseph prevailed, so that they became tributaries.
\verse And the coast of the Amorites was from the going up to Akrabbim, from the rock, and upward.
\end{biblechapter}

\begin{biblechapter} % Judges 2
\verseWithHeading{The angel of the \LORD at Bochim} And an angel of the \LORD came up from Gilgal to Bochim, and said, I made you to go up out of Egypt, and have brought you unto the land which I sware unto your fathers; and I said, I will never break my covenant with you.
\verse And ye shall make no league with the inhabitants of this land; ye shall throw down their altars: but ye have not obeyed my voice: why have ye done this?
\verse Wherefore I also said, I will not drive them out from before you; but they shall be as thorns in your sides, and their gods shall be a snare unto you.
\verse And it came to pass, when the angel of the \LORD spake these words unto all the children of Israel, that the people lifted up their voice, and wept.
\verse And they called the name of that place Bochim: and they sacrificed there unto the \LORD.
\vfill\columnbreak % layout hack
\verseWithHeading{Disobedience and defeat} And when Joshua had let the people go, the children of Israel went every man unto his inheritance to possess the land.
\verse And the people served the \LORD all the days of Joshua, and all the days of the elders that outlived Joshua, who had seen all the great works of the \LORD, that he did for Israel.
\verse And Joshua the son of Nun, the servant of the \LORD, died, being an hundred and ten years old.
\verse And they buried him in the border of his inheritance in Timnathheres, in the mount of Ephraim, on the north side of the hill Gaash.
\verse And also all that generation were gathered unto their fathers: and there arose another generation after them, which knew not the \LORD, nor yet the works which he had done for Israel.
\verse And the children of Israel did evil in the sight of the \LORD, and served Baalim:
\verse And they forsook the \LORD God of their fathers, which brought them out of the land of Egypt, and followed other gods, of the gods of the people that were round about them, and bowed themselves unto them, and provoked the \LORD to anger.
\verse And they forsook the \LORD, and served Baal and Ashtaroth.
\verse And the anger of the \LORD was hot against Israel, and he delivered them into the hands of spoilers that spoiled them, and he sold them into the hands of their enemies round about, so that they could not any longer stand before their enemies.
\verse Whithersoever they went out, the hand of the \LORD was against them for evil, as the \LORD had said, and as the \LORD had sworn unto them: and they were greatly distressed.
\verse Nevertheless the \LORD raised up judges, which delivered them out of the hand of those that spoiled them.
\verse And yet they would not hearken unto their judges, but they went a whoring after other gods, and bowed themselves unto them: they turned quickly out of the way which their fathers walked in, obeying the commandments of the \LORD; but they did not so.
\verse And when the \LORD raised them up judges, then the \LORD was with the judge, and delivered them out of the hand of their enemies all the days of the judge: for it repented the \LORD because of their groanings by reason of them that oppressed them and vexed them.
\verse And it came to pass, when the judge was dead, that they returned, and corrupted themselves more than their fathers, in following other gods to serve them, and to bow down unto them; they ceased not from their own doings, nor from their stubborn way.
\verse And the anger of the \LORD was hot against Israel; and he said, Because that this people hath transgressed my covenant which I commanded their fathers, and have not hearkened unto my voice;
\verse I also will not henceforth drive out any from before them of the nations which Joshua left when he died:
\verse That through them I may prove Israel, whether they will keep the way of the \LORD to walk therein, as their fathers did keep it, or not.
\verse Therefore the \LORD left those nations, without driving them out hastily; neither delivered he them into the hand of Joshua.
\end{biblechapter}

\begin{biblechapter} % Judges 3
\verse Now these are the nations which the \LORD left, to prove Israel by them, even as many of Israel as had not known all the wars of Canaan;
\verse Only that the generations of the children of Israel might know, to teach them war, at the least such as before knew nothing thereof;
\verse Namely, five lords of the Philistines, and all the Canaanites, and the Sidonians, and the Hivites that dwelt in mount Lebanon, from mount Baalhermon unto the entering in of Hamath.
\verse And they were to prove Israel by them, to know whether they would hearken unto the commandments of the \LORD, which he commanded their fathers by the hand of Moses.
\verse And the children of Israel dwelt among the Canaanites, Hittites, and Amorites, and Perizzites, and Hivites, and Jebusites:
\verse And they took their daughters to be their wives, and gave their daughters to their sons, and served their gods.
\verseWithHeading{Othniel} And the children of Israel did evil in the sight of the \LORD, and forgat the \LORD their God, and served Baalim and the groves.
\verse Therefore the anger of the \LORD was hot against Israel, and he sold them into the hand of Chushanrishathaim king of Mesopotamia: and the children of Israel served Chushanrishathaim eight years.
\verse And when the children of Israel cried unto the \LORD, the \LORD raised up a deliverer to the children of Israel, who delivered them, even Othniel the son of Kenaz, Caleb's younger brother.
\verse And the Spirit of the \LORD came upon him, and he judged Israel, and went out to war: and the \LORD delivered Chushanrishathaim king of Mesopotamia into his hand; and his hand prevailed against Chushanrishathaim.
\verse And the land had rest forty years. And Othniel the son of Kenaz died.
\verseWithHeading{Ehud} And the children of Israel did evil again in the sight of the \LORD: and the \LORD strengthened Eglon the king of Moab against Israel, because they had done evil in the sight of the \LORD.
\verse And he gathered unto him the children of Ammon and Amalek, and went and smote Israel, and possessed the city of palm trees.
\verse So the children of Israel served Eglon the king of Moab eighteen years.
\verse But when the children of Israel cried unto the \LORD, the \LORD raised them up a deliverer, Ehud the son of Gera, a Benjamite, a man lefthanded: and by him the children of Israel sent a present unto Eglon the king of Moab.
\verse But Ehud made him a dagger which had two edges, of a cubit length; and he did gird it under his raiment upon his right thigh.
\verse And he brought the present unto Eglon king of Moab: and Eglon was a very fat man.
\verse And when he had made an end to offer the present, he sent away the people that bare the present.
\verse But he himself turned again from the quarries that were by Gilgal, and said, I have a secret errand unto thee, O king: who said, Keep silence. And all that stood by him went out from him.
\verse And Ehud came unto him; and he was sitting in a summer parlour, which he had for himself alone. And Ehud said, I have a message from God unto thee. And he arose out of his seat.
\verse And Ehud put forth his left hand, and took the dagger from his right thigh, and thrust it into his belly:
\verse And the haft also went in after the blade; and the fat closed upon the blade, so that he could not draw the dagger out of his belly; and the dirt came out.
\verse Then Ehud went forth through the porch, and shut the doors of the parlour upon him, and locked them.
\verse When he was gone out, his servants came; and when they saw that, behold, the doors of the parlour were locked, they said, Surely he covereth his feet in his summer chamber.
\verse And they tarried till they were ashamed: and, behold, he opened not the doors of the parlour; therefore they took a key, and opened them: and, behold, their lord was fallen down dead on the earth.
\verse And Ehud escaped while they tarried, and passed beyond the quarries, and escaped unto Seirath.
\verse And it came to pass, when he was come, that he blew a trumpet in the mountain of Ephraim, and the children of Israel went down with him from the mount, and he before them.
\verse And he said unto them, Follow after me: for the \LORD hath delivered your enemies the Moabites into your hand. And they went down after him, and took the fords of Jordan toward Moab, and suffered not a man to pass over.
\verse And they slew of Moab at that time about ten thousand men, all lusty, and all men of valour; and there escaped not a man.
\verse So Moab was subdued that day under the hand of Israel. And the land had rest fourscore years.
\verseWithHeading{Shamgar} And after him was Shamgar the son of Anath, which slew of the Philistines six hundred men with an ox goad: and he also delivered Israel.
\end{biblechapter}

\begin{biblechapter} % Judges 4
\verseWithHeading{Deborah} And the children of Israel again did evil in the sight of the \LORD, when Ehud was dead.
\verse And the \LORD sold them into the hand of Jabin king of Canaan, that reigned in Hazor; the captain of whose host was Sisera, which dwelt in Harosheth of the Gentiles.
\verse And the children of Israel cried unto the \LORD: for he had nine hundred chariots of iron; and twenty years he mightily oppressed the children of Israel.
\verse And Deborah, a prophetess, the wife of Lapidoth, she judged Israel at that time.
\verse And she dwelt under the palm tree of Deborah between Ramah and Bethel in mount Ephraim: and the children of Israel came up to her for judgment.
\verse And she sent and called Barak the son of Abinoam out of Kedeshnaphtali, and said unto him, Hath not the \LORD God of Israel commanded, saying, Go and draw toward mount Tabor, and take with thee ten thousand men of the children of Naphtali and of the children of Zebulun?
\verse And I will draw unto thee to the river Kishon Sisera, the captain of Jabin's army, with his chariots and his multitude; and I will deliver him into thine hand.
\verse And Barak said unto her, If thou wilt go with me, then I will go: but if thou wilt not go with me, then I will not go.
\verse And she said, I will surely go with thee: notwithstanding the journey that thou takest shall not be for thine honour; for the \LORD shall sell Sisera into the hand of a woman. And Deborah arose, and went with Barak to Kedesh.
\verse And Barak called Zebulun and Naphtali to Kedesh; and he went up with ten thousand men at his feet: and Deborah went up with him.
\verse Now Heber the Kenite, which was of the children of Hobab the father in law of Moses, had severed himself from the Kenites, and pitched his tent unto the plain of Zaanaim, which is by Kedesh.
\verse And they shewed Sisera that Barak the son of Abinoam was gone up to mount Tabor.
\verse And Sisera gathered together all his chariots, even nine hundred chariots of iron, and all the people that were with him, from Harosheth of the Gentiles unto the river of Kishon.
\verse And Deborah said unto Barak, Up; for this is the day in which the \LORD hath delivered Sisera into thine hand: is not the \LORD gone out before thee? So Barak went down from mount Tabor, and ten thousand men after him.
\verse And the \LORD discomfited Sisera, and all his chariots, and all his host, with the edge of the sword before Barak; so that Sisera lighted down off his chariot, and fled away on his feet.
\verse But Barak pursued after the chariots, and after the host, unto Harosheth of the Gentiles: and all the host of Sisera fell upon the edge of the sword; and there was not a man left.
\verse Howbeit Sisera fled away on his feet to the tent of Jael the wife of Heber the Kenite: for there was peace between Jabin the king of Hazor and the house of Heber the Kenite.
\verse And Jael went out to meet Sisera, and said unto him, Turn in, my lord, turn in to me; fear not. And when he had turned in unto her into the tent, she covered him with a mantle.
\verse And he said unto her, Give me, I pray thee, a little water to drink; for I am thirsty. And she opened a bottle of milk, and gave him drink, and covered him.
\verse Again he said unto her, Stand in the door of the tent, and it shall be, when any man doth come and enquire of thee, and say, Is there any man here? that thou shalt say, No.
\verse Then Jael Heber's wife took a nail of the tent, and took an hammer in her hand, and went softly unto him, and smote the nail into his temples, and fastened it into the ground: for he was fast asleep and weary. So he died.
\verse And, behold, as Barak pursued Sisera, Jael came out to meet him, and said unto him, Come, and I will shew thee the man whom thou seekest. And when he came into her tent, behold, Sisera lay dead, and the nail was in his temples.
\verse So God subdued on that day Jabin the king of Canaan before the children of Israel.
\verse And the hand of the children of Israel prospered, and prevailed against Jabin the king of Canaan, until they had destroyed Jabin king of Canaan.
\end{biblechapter}

\begin{biblechapter} % Judges 5
\verseWithHeading{The song of Deborah} Then sang Deborah and Barak the son of Abinoam on that day, saying,
\verse Praise ye the \LORD for the avenging of Israel, when the people willingly offered themselves.
\verse Hear, O ye kings; give ear, O ye princes; I, even I, will sing unto the \LORD; I will sing praise to the \LORD God of Israel.
\verse \LORD, when thou wentest out of Seir, when thou marchedst out of the field of Edom, the earth trembled, and the heavens dropped, the clouds also dropped water.
\verse The mountains melted from before the \LORD, even that Sinai from before the \LORD God of Israel.
\verse In the days of Shamgar the son of Anath, in the days of Jael, the highways were unoccupied, and the travellers walked through byways.
\verse The inhabitants of the villages ceased, they ceased in Israel, until that I Deborah arose, that I arose a mother in Israel.
\verse They chose new gods; then was war in the gates: was there a shield or spear seen among forty thousand in Israel?
\verse My heart is toward the governors of Israel, that offered themselves willingly among the people. Bless ye the \LORD.
\verse Speak, ye that ride on white asses, ye that sit in judgment, and walk by the way.
\verse They that are delivered from the noise of archers in the places of drawing water, there shall they rehearse the righteous acts of the \LORD, even the righteous acts toward the inhabitants of his villages in Israel: then shall the people of the \LORD go down to the gates.
\verse Awake, awake, Deborah: awake, awake, utter a song: arise, Barak, and lead thy captivity captive, thou son of Abinoam.
\verse Then he made him that remaineth have dominion over the nobles among the people: the \LORD made me have dominion over the mighty.
\verse Out of Ephraim was there a root of them against Amalek; after thee, Benjamin, among thy people; out of Machir came down governors, and out of Zebulun they that handle the pen of the writer.
\verse And the princes of Issachar were with Deborah; even Issachar, and also Barak: he was sent on foot into the valley. For the divisions of Reuben there were great thoughts of heart.
\verse Why abodest thou among the sheepfolds, to hear the bleatings of the flocks? For the divisions of Reuben there were great searchings of heart.
\verse Gilead abode beyond Jordan: and why did Dan remain in ships? Asher continued on the sea shore, and abode in his breaches.
\verse Zebulun and Naphtali were a people that jeoparded their lives unto the death in the high places of the field.
\verse The kings came and fought, then fought the kings of Canaan in Taanach by the waters of Megiddo; they took no gain of money.
\verse They fought from heaven; the stars in their courses fought against Sisera.
\verse The river of Kishon swept them away, that ancient river, the river Kishon. O my soul, thou hast trodden down strength.
\verse Then were the horsehoofs broken by the means of the pransings, the pransings of their mighty ones.
\verse Curse ye Meroz, said the angel of the \LORD, curse ye bitterly the inhabitants thereof; because they came not to the help of the \LORD, to the help of the \LORD against the mighty.
\verse Blessed above women shall Jael the wife of Heber the Kenite be, blessed shall she be above women in the tent.
\verse He asked water, and she gave him milk; she brought forth butter in a lordly dish.
\verse She put her hand to the nail, and her right hand to the workmen's hammer; and with the hammer she smote Sisera, she smote off his head, when she had pierced and stricken through his temples.
\verse At her feet he bowed, he fell, he lay down: at her feet he bowed, he fell: where he bowed, there he fell down dead.
\verse The mother of Sisera looked out at a window, and cried through the lattice, Why is his chariot so long in coming? why tarry the wheels of his chariots?
\verse Her wise ladies answered her, yea, she returned answer to herself,
\verse Have they not sped? have they not divided the prey; to every man a damsel or two; to Sisera a prey of divers colours, a prey of divers colours of needlework, of divers colours of needlework on both sides, meet for the necks of them that take the spoil?
\verse So let all thine enemies perish, O \LORD: but let them that love him be as the sun when he goeth forth in his might. And the land had rest forty years.
\end{biblechapter}

\begin{biblechapter} % Judges 6
\verseWithHeading{Gideon} And the children of Israel did evil in the sight of the \LORD: and the \LORD delivered them into the hand of Midian seven years.
\verse And the hand of Midian prevailed against Israel: and because of the Midianites the children of Israel made them the dens which are in the mountains, and caves, and strong holds.
\verse And so it was, when Israel had sown, that the Midianites came up, and the Amalekites, and the children of the east, even they came up against them;
\verse And they encamped against them, and destroyed the increase of the earth, till thou come unto Gaza, and left no sustenance for Israel, neither sheep, nor ox, nor ass.
\verse For they came up with their cattle and their tents, and they came as grasshoppers for multitude; for both they and their camels were without number: and they entered into the land to destroy it.
\verse And Israel was greatly impoverished because of the Midianites; and the children of Israel cried unto the \LORD.
\verse And it came to pass, when the children of Israel cried unto the \LORD because of the Midianites,
\verse That the \LORD sent a prophet unto the children of Israel, which said unto them, Thus saith the \LORD God of Israel, I brought you up from Egypt, and brought you forth out of the house of bondage;
\verse And I delivered you out of the hand of the Egyptians, and out of the hand of all that oppressed you, and drave them out from before you, and gave you their land;
\verse And I said unto you, I am the \LORD your God; fear not the gods of the Amorites, in whose land ye dwell: but ye have not obeyed my voice.
\verse And there came an angel of the \LORD, and sat under an oak which was in Ophrah, that pertained unto Joash the Abiezrite: and his son Gideon threshed wheat by the winepress, to hide it from the Midianites.
\verse And the angel of the \LORD appeared unto him, and said unto him, The \LORD is with thee, thou mighty man of valour.
\verse And Gideon said unto him, Oh my Lord, if the \LORD be with us, why then is all this befallen us? and where be all his miracles which our fathers told us of, saying, Did not the \LORD bring us up from Egypt? but now the \LORD hath forsaken us, and delivered us into the hands of the Midianites.
\verse And the \LORD looked upon him, and said, Go in this thy might, and thou shalt save Israel from the hand of the Midianites: have not I sent thee?
\verse And he said unto him, Oh my Lord, wherewith shall I save Israel? behold, my family is poor in Manasseh, and I am the least in my father's house.
\verse And the \LORD said unto him, Surely I will be with thee, and thou shalt smite the Midianites as one man.
\verse And he said unto him, If now I have found grace in thy sight, then shew me a sign that thou talkest with me.
\verse Depart not hence, I pray thee, until I come unto thee, and bring forth my present, and set it before thee. And he said, I will tarry until thou come again.
\verse And Gideon went in, and made ready a kid, and unleavened cakes of an ephah of flour: the flesh he put in a basket, and he put the broth in a pot, and brought it out unto him under the oak, and presented it.
\verse And the angel of God said unto him, Take the flesh and the unleavened cakes, and lay them upon this rock, and pour out the broth. And he did so.
\verse Then the angel of the \LORD put forth the end of the staff that was in his hand, and touched the flesh and the unleavened cakes; and there rose up fire out of the rock, and consumed the flesh and the unleavened cakes. Then the angel of the \LORD departed out of his sight.
\verse And when Gideon perceived that he was an angel of the \LORD, Gideon said, Alas, O Lord GOD! for because I have seen an angel of the \LORD face to face.
\verse And the \LORD said unto him, Peace be unto thee; fear not: thou shalt not die.
\verse Then Gideon built an altar there unto the \LORD, and called it Jehovahshalom: unto this day it is yet in Ophrah of the Abiezrites.
\verse And it came to pass the same night, that the \LORD said unto him, Take thy father's young bullock, even the second bullock of seven years old, and throw down the altar of Baal that thy father hath, and cut down the grove that is by it:
\verse And build an altar unto the \LORD thy God upon the top of this rock, in the ordered place, and take the second bullock, and offer a burnt sacrifice with the wood of the grove which thou shalt cut down.
\verse Then Gideon took ten men of his servants, and did as the \LORD had said unto him: and so it was, because he feared his father's household, and the men of the city, that he could not do it by day, that he did it by night.
\verse And when the men of the city arose early in the morning, behold, the altar of Baal was cast down, and the grove was cut down that was by it, and the second bullock was offered upon the altar that was built.
\verse And they said one to another, Who hath done this thing? And when they enquired and asked, they said, Gideon the son of Joash hath done this thing.
\verse Then the men of the city said unto Joash, Bring out thy son, that he may die: because he hath cast down the altar of Baal, and because he hath cut down the grove that was by it.
\verse And Joash said unto all that stood against him, Will ye plead for Baal? will ye save him? he that will plead for him, let him be put to death whilst it is yet morning: if he be a god, let him plead for himself, because one hath cast down his altar.
\verse Therefore on that day he called him Jerubbaal, saying, Let Baal plead against him, because he hath thrown down his altar.
\verse Then all the Midianites and the Amalekites and the children of the east were gathered together, and went over, and pitched in the valley of Jezreel.
\verse But the Spirit of the \LORD came upon Gideon, and he blew a trumpet; and Abiezer was gathered after him.
\verse And he sent messengers throughout all Manasseh; who also was gathered after him: and he sent messengers unto Asher, and unto Zebulun, and unto Naphtali; and they came up to meet them.
\verse And Gideon said unto God, If thou wilt save Israel by mine hand, as thou hast said,
\verse Behold, I will put a fleece of wool in the floor; and if the dew be on the fleece only, and it be dry upon all the earth beside, then shall I know that thou wilt save Israel by mine hand, as thou hast said.
\verse And it was so: for he rose up early on the morrow, and thrust the fleece together, and wringed the dew out of the fleece, a bowl full of water.
\verse And Gideon said unto God, Let not thine anger be hot against me, and I will speak but this once: let me prove, I pray thee, but this once with the fleece; let it now be dry only upon the fleece, and upon all the ground let there be dew.
\verse And God did so that night: for it was dry upon the fleece only, and there was dew on all the ground.
\end{biblechapter}

\begin{biblechapter} % Judges 7
\verseWithHeading{Gideon defeats the Midianites} Then Jerubbaal, who is Gideon, and all the people that were with him, rose up early, and pitched beside the well of Harod: so that the host of the Midianites were on the north side of them, by the hill of Moreh, in the valley.
\verse And the \LORD said unto Gideon, The people that are with thee are too many for me to give the Midianites into their hands, lest Israel vaunt themselves against me, saying, Mine own hand hath saved me.
\verse Now therefore go to, proclaim in the ears of the people, saying, Whosoever is fearful and afraid, let him return and depart early from mount Gilead. And there returned of the people twenty and two thousand; and there remained ten thousand.
\verse And the \LORD said unto Gideon, The people are yet too many; bring them down unto the water, and I will try them for thee there: and it shall be, that of whom I say unto thee, This shall go with thee, the same shall go with thee; and of whomsoever I say unto thee, This shall not go with thee, the same shall not go.
\verse So he brought down the people unto the water: and the \LORD said unto Gideon, Every one that lappeth of the water with his tongue, as a dog lappeth, him shalt thou set by himself; likewise every one that boweth down upon his knees to drink.
\verse And the number of them that lapped, putting their hand to their mouth, were three hundred men: but all the rest of the people bowed down upon their knees to drink water.
\verse And the \LORD said unto Gideon, By the three hundred men that lapped will I save you, and deliver the Midianites into thine hand: and let all the other people go every man unto his place.
\verse So the people took victuals in their hand, and their trumpets: and he sent all the rest of Israel every man unto his tent, and retained those three hundred men: and the host of Midian was beneath him in the valley.
\verse And it came to pass the same night, that the \LORD said unto him, Arise, get thee down unto the host; for I have delivered it into thine hand.
\verse But if thou fear to go down, go thou with Phurah thy servant down to the host:
\verse And thou shalt hear what they say; and afterward shall thine hands be strengthened to go down unto the host. Then went he down with Phurah his servant unto the outside of the armed men that were in the host.
\verse And the Midianites and the Amalekites and all the children of the east lay along in the valley like grasshoppers for multitude; and their camels were without number, as the sand by the sea side for multitude.
\verse And when Gideon was come, behold, there was a man that told a dream unto his fellow, and said, Behold, I dreamed a dream, and, lo, a cake of barley bread tumbled into the host of Midian, and came unto a tent, and smote it that it fell, and overturned it, that the tent lay along.
\verse And his fellow answered and said, This is nothing else save the sword of Gideon the son of Joash, a man of Israel: for into his hand hath God delivered Midian, and all the host.
\verse And it was so, when Gideon heard the telling of the dream, and the interpretation thereof, that he worshipped, and returned into the host of Israel, and said, Arise; for the \LORD hath delivered into your hand the host of Midian.
\verse And he divided the three hundred men into three companies, and he put a trumpet in every man's hand, with empty pitchers, and lamps within the pitchers.
\verse And he said unto them, Look on me, and do likewise: and, behold, when I come to the outside of the camp, it shall be that, as I do, so shall ye do.
\verse When I blow with a trumpet, I and all that are with me, then blow ye the trumpets also on every side of all the camp, and say, The sword of the \LORD, and of Gideon.
\verse So Gideon, and the hundred men that were with him, came unto the outside of the camp in the beginning of the middle watch; and they had but newly set the watch: and they blew the trumpets, and brake the pitchers that were in their hands.
\verse And the three companies blew the trumpets, and brake the pitchers, and held the lamps in their left hands, and the trumpets in their right hands to blow withal: and they cried, The sword of the \LORD, and of Gideon.
\verse And they stood every man in his place round about the camp: and all the host ran, and cried, and fled.
\verse And the three hundred blew the trumpets, and the \LORD set every man's sword against his fellow, even throughout all the host: and the host fled to Bethshittah in Zererath, and to the border of Abelmeholah, unto Tabbath.
\verse And the men of Israel gathered themselves together out of Naphtali, and out of Asher, and out of all Manasseh, and pursued after the Midianites.
\verse And Gideon sent messengers throughout all mount Ephraim, saying, Come down against the Midianites, and take before them the waters unto Bethbarah and Jordan. Then all the men of Ephraim gathered themselves together, and took the waters unto Bethbarah and Jordan.
\verse And they took two princes of the Midianites, Oreb and Zeeb; and they slew Oreb upon the rock Oreb, and Zeeb they slew at the winepress of Zeeb, and pursued Midian, and brought the heads of Oreb and Zeeb to Gideon on the other side Jordan.
\end{biblechapter}

\begin{biblechapter} % Judges 8
\verseWithHeading{Zebah and Zalmunna} And the men of Ephraim said unto him, Why hast thou served us thus, that thou calledst us not, when thou wentest to fight with the Midianites? And they did chide with him sharply.
\verse And he said unto them, What have I done now in comparison of you? Is not the gleaning of the grapes of Ephraim better than the vintage of Abiezer?
\verse God hath delivered into your hands the princes of Midian, Oreb and Zeeb: and what was I able to do in comparison of you? Then their anger was abated toward him, when he had said that.
\verse And Gideon came to Jordan, and passed over, he, and the three hundred men that were with him, faint, yet pursuing them.
\verse And he said unto the men of Succoth, Give, I pray you, loaves of bread unto the people that follow me; for they be faint, and I am pursuing after Zebah and Zalmunna, kings of Midian.
\verse And the princes of Succoth said, Are the hands of Zebah and Zalmunna now in thine hand, that we should give bread unto thine army?
\verse And Gideon said, Therefore when the \LORD hath delivered Zebah and Zalmunna into mine hand, then I will tear your flesh with the thorns of the wilderness and with briers.
\verse And he went up thence to Penuel, and spake unto them likewise: and the men of Penuel answered him as the men of Succoth had answered him.
\verse And he spake also unto the men of Penuel, saying, When I come again in peace, I will break down this tower.
\verse Now Zebah and Zalmunna were in Karkor, and their hosts with them, about fifteen thousand men, all that were left of all the hosts of the children of the east: for there fell an hundred and twenty thousand men that drew sword.
\verse And Gideon went up by the way of them that dwelt in tents on the east of Nobah and Jogbehah, and smote the host: for the host was secure.
\verse And when Zebah and Zalmunna fled, he pursued after them, and took the two kings of Midian, Zebah and Zalmunna, and discomfited all the host.
\verse And Gideon the son of Joash returned from battle before the sun was up,
\verse And caught a young man of the men of Succoth, and enquired of him: and he described unto him the princes of Succoth, and the elders thereof, even threescore and seventeen men.
\verse And he came unto the men of Succoth, and said, Behold Zebah and Zalmunna, with whom ye did upbraid me, saying, Are the hands of Zebah and Zalmunna now in thine hand, that we should give bread unto thy men that are weary?
\verse And he took the elders of the city, and thorns of the wilderness and briers, and with them he taught the men of Succoth.
\verse And he beat down the tower of Penuel, and slew the men of the city.
\verse Then said he unto Zebah and Zalmunna, What manner of men were they whom ye slew at Tabor? And they answered, As thou art, so were they; each one resembled the children of a king.
\verse And he said, They were my brethren, even the sons of my mother: as the \LORD liveth, if ye had saved them alive, I would not slay you.
\verse And he said unto Jether his firstborn, Up, and slay them. But the youth drew not his sword: for he feared, because he was yet a youth.
\verse Then Zebah and Zalmunna said, Rise thou, and fall upon us: for as the man is, so is his strength. And Gideon arose, and slew Zebah and Zalmunna, and took away the ornaments that were on their camels' necks.
\verseWithHeading{Gideon's ephod} Then the men of Israel said unto Gideon, Rule thou over us, both thou, and thy son, and thy son's son also: for thou hast delivered us from the hand of Midian.
\verse And Gideon said unto them, I will not rule over you, neither shall my son rule over you: the \LORD shall rule over you.
\verse And Gideon said unto them, I would desire a request of you, that ye would give me every man the earrings of his prey. (For they had golden earrings, because they were Ishmaelites.)
\verse And they answered, We will willingly give them. And they spread a garment, and did cast therein every man the earrings of his prey.
\verse And the weight of the golden earrings that he requested was a thousand and seven hundred shekels of gold; beside ornaments, and collars, and purple raiment that was on the kings of Midian, and beside the chains that were about their camels' necks.
\verse And Gideon made an ephod thereof, and put it in his city, even in Ophrah: and all Israel went thither a whoring after it: which thing became a snare unto Gideon, and to his house.
\verseWithHeading{The death of Gideon} Thus was Midian subdued before the children of Israel, so that they lifted up their heads no more. And the country was in quietness forty years in the days of Gideon.
\verse And Jerubbaal the son of Joash went and dwelt in his own house.
\verse And Gideon had threescore and ten sons of his body begotten: for he had many wives.
\verse And his concubine that was in Shechem, she also bare him a son, whose name he called Abimelech.
\verse And Gideon the son of Joash died in a good old age, and was buried in the sepulchre of Joash his father, in Ophrah of the Abiezrites.
\verse And it came to pass, as soon as Gideon was dead, that the children of Israel turned again, and went a whoring after Baalim, and made Baalberith their god.
\verse And the children of Israel remembered not the \LORD their God, who had delivered them out of the hands of all their enemies on every side:
\verse Neither shewed they kindness to the house of Jerubbaal, namely, Gideon, according to all the goodness which he had shewed unto Israel.
\end{biblechapter}

\begin{biblechapter} % Judges 9
\verseWithHeading{Abimelech} And Abimelech the son of Jerubbaal went to Shechem unto his mother's brethren, and communed with them, and with all the family of the house of his mother's father, saying,
\verse Speak, I pray you, in the ears of all the men of Shechem, Whether is better for you, either that all the sons of Jerubbaal, which are threescore and ten persons, reign over you, or that one reign over you? remember also that I am your bone and your flesh.
\verse And his mother's brethren spake of him in the ears of all the men of Shechem all these words: and their hearts inclined to follow Abimelech; for they said, He is our brother.
\verse And they gave him threescore and ten pieces of silver out of the house of Baalberith, wherewith Abimelech hired vain and light persons, which followed him.
\verse And he went unto his father's house at Ophrah, and slew his brethren the sons of Jerubbaal, being threescore and ten persons, upon one stone: notwithstanding yet Jotham the youngest son of Jerubbaal was left; for he hid himself.
\verse And all the men of Shechem gathered together, and all the house of Millo, and went, and made Abimelech king, by the plain of the pillar that was in Shechem.
\verse And when they told it to Jotham, he went and stood in the top of mount Gerizim, and lifted up his voice, and cried, and said unto them, Hearken unto me, ye men of Shechem, that God may hearken unto you.
\verse The trees went forth on a time to anoint a king over them; and they said unto the olive tree, Reign thou over us.
\verse But the olive tree said unto them, Should I leave my fatness, wherewith by me they honour God and man, and go to be promoted over the trees?
\verse And the trees said to the fig tree, Come thou, and reign over us.
\verse But the fig tree said unto them, Should I forsake my sweetness, and my good fruit, and go to be promoted over the trees?
\verse Then said the trees unto the vine, Come thou, and reign over us.
\verse And the vine said unto them, Should I leave my wine, which cheereth God and man, and go to be promoted over the trees?
\verse Then said all the trees unto the bramble, Come thou, and reign over us.
\verse And the bramble said unto the trees, If in truth ye anoint me king over you, then come and put your trust in my shadow: and if not, let fire come out of the bramble, and devour the cedars of Lebanon.
\verse Now therefore, if ye have done truly and sincerely, in that ye have made Abimelech king, and if ye have dealt well with Jerubbaal and his house, and have done unto him according to the deserving of his hands;
\verse (For my father fought for you, and adventured his life far, and delivered you out of the hand of Midian:
\verse And ye are risen up against my father's house this day, and have slain his sons, threescore and ten persons, upon one stone, and have made Abimelech, the son of his maidservant, king over the men of Shechem, because he is your brother;)
\verse If ye then have dealt truly and sincerely with Jerubbaal and with his house this day, then rejoice ye in Abimelech, and let him also rejoice in you:
\verse But if not, let fire come out from Abimelech, and devour the men of Shechem, and the house of Millo; and let fire come out from the men of Shechem, and from the house of Millo, and devour Abimelech.
\verse And Jotham ran away, and fled, and went to Beer, and dwelt there, for fear of Abimelech his brother.
\verse When Abimelech had reigned three years over Israel,
\verse Then God sent an evil spirit between Abimelech and the men of Shechem; and the men of Shechem dealt treacherously with Abimelech:
\verse That the cruelty done to the threescore and ten sons of Jerubbaal might come, and their blood be laid upon Abimelech their brother, which slew them; and upon the men of Shechem, which aided him in the killing of his brethren.
\verse And the men of Shechem set liers in wait for him in the top of the mountains, and they robbed all that came along that way by them: and it was told Abimelech.
\verse And Gaal the son of Ebed came with his brethren, and went over to Shechem: and the men of Shechem put their confidence in him.
\verse And they went out into the fields, and gathered their vineyards, and trode the grapes, and made merry, and went into the house of their god, and did eat and drink, and cursed Abimelech.
\verse And Gaal the son of Ebed said, Who is Abimelech, and who is Shechem, that we should serve him? is not he the son of Jerubbaal? and Zebul his officer? serve the men of Hamor the father of Shechem: for why should we serve him?
\verse And would to God this people were under my hand! then would I remove Abimelech. And he said to Abimelech, Increase thine army, and come out.
\verse And when Zebul the ruler of the city heard the words of Gaal the son of Ebed, his anger was kindled.
\verse And he sent messengers unto Abimelech privily, saying, Behold, Gaal the son of Ebed and his brethren be come to Shechem; and, behold, they fortify the city against thee.
\verse Now therefore up by night, thou and the people that is with thee, and lie in wait in the field:
\verse And it shall be, that in the morning, as soon as the sun is up, thou shalt rise early, and set upon the city: and, behold, when he and the people that is with him come out against thee, then mayest thou do to them as thou shalt find occasion.
\verse And Abimelech rose up, and all the people that were with him, by night, and they laid wait against Shechem in four companies.
\verse And Gaal the son of Ebed went out, and stood in the entering of the gate of the city: and Abimelech rose up, and the people that were with him, from lying in wait.
\verse And when Gaal saw the people, he said to Zebul, Behold, there come people down from the top of the mountains. And Zebul said unto him, Thou seest the shadow of the mountains as if they were men.
\verse And Gaal spake again and said, See there come people down by the middle of the land, and another company come along by the plain of Meonenim.
\verse Then said Zebul unto him, Where is now thy mouth, wherewith thou saidst, Who is Abimelech, that we should serve him? is not this the people that thou hast despised? go out, I pray now, and fight with them.
\verse And Gaal went out before the men of Shechem, and fought with Abimelech.
\verse And Abimelech chased him, and he fled before him, and many were overthrown and wounded, even unto the entering of the gate.
\verse And Abimelech dwelt at Arumah: and Zebul thrust out Gaal and his brethren, that they should not dwell in Shechem.
\verse And it came to pass on the morrow, that the people went out into the field; and they told Abimelech.
\verse And he took the people, and divided them into three companies, and laid wait in the field, and looked, and, behold, the people were come forth out of the city; and he rose up against them, and smote them.
\verse And Abimelech, and the company that was with him, rushed forward, and stood in the entering of the gate of the city: and the two other companies ran upon all the people that were in the fields, and slew them.
\verse And Abimelech fought against the city all that day; and he took the city, and slew the people that was therein, and beat down the city, and sowed it with salt.
\verse And when all the men of the tower of Shechem heard that, they entered into an hold of the house of the god Berith.
\verse And it was told Abimelech, that all the men of the tower of Shechem were gathered together.
\verse And Abimelech gat him up to mount Zalmon, he and all the people that were with him; and Abimelech took an axe in his hand, and cut down a bough from the trees, and took it, and laid it on his shoulder, and said unto the people that were with him, What ye have seen me do, make haste, and do as I have done.
\verse And all the people likewise cut down every man his bough, and followed Abimelech, and put them to the hold, and set the hold on fire upon them; so that all the men of the tower of Shechem died also, about a thousand men and women.
\verse Then went Abimelech to Thebez, and encamped against Thebez, and took it.
\verse But there was a strong tower within the city, and thither fled all the men and women, and all they of the city, and shut it to them, and gat them up to the top of the tower.
\verse And Abimelech came unto the tower, and fought against it, and went hard unto the door of the tower to burn it with fire.
\verse And a certain woman cast a piece of a millstone upon Abimelech's head, and all to brake his skull.
\verse Then he called hastily unto the young man his armourbearer, and said unto him, Draw thy sword, and slay me, that men say not of me, A woman slew him. And his young man thrust him through, and he died.
\verse And when the men of Israel saw that Abimelech was dead, they departed every man unto his place.
\verse Thus God rendered the wickedness of Abimelech, which he did unto his father, in slaying his seventy brethren:
\verse And all the evil of the men of Shechem did God render upon their heads: and upon them came the curse of Jotham the son of Jerubbaal.
\end{biblechapter}

\begin{biblechapter} % Judges 10
\verseWithHeading{Tola} And after Abimelech there arose to defend Israel Tola the son of Puah, the son of Dodo, a man of Issachar; and he dwelt in Shamir in mount Ephraim.
\verse And he judged Israel twenty and three years, and died, and was buried in Shamir.
\verseWithHeading{Jair} And after him arose Jair, a Gileadite, and judged Israel twenty and two years.
\verse And he had thirty sons that rode on thirty ass colts, and they had thirty cities, which are called Havothjair unto this day, which are in the land of Gilead.
\verse And Jair died, and was buried in Camon.
\verseWithHeading{Jephthah} And the children of Israel did evil again in the sight of the \LORD, and served Baalim, and Ashtaroth, and the gods of Syria, and the gods of Zidon, and the gods of Moab, and the gods of the children of Ammon, and the gods of the Philistines, and forsook the \LORD, and served not him.
\verse And the anger of the \LORD was hot against Israel, and he sold them into the hands of the Philistines, and into the hands of the children of Ammon.
\verse And that year they vexed and oppressed the children of Israel: eighteen years, all the children of Israel that were on the other side Jordan in the land of the Amorites, which is in Gilead.
\verse Moreover the children of Ammon passed over Jordan to fight also against Judah, and against Benjamin, and against the house of Ephraim; so that Israel was sore distressed.
\verse And the children of Israel cried unto the \LORD, saying, We have sinned against thee, both because we have forsaken our God, and also served Baalim.
\verse And the \LORD said unto the children of Israel, Did not I deliver you from the Egyptians, and from the Amorites, from the children of Ammon, and from the Philistines?
\verse The Zidonians also, and the Amalekites, and the Maonites, did oppress you; and ye cried to me, and I delivered you out of their hand.
\verse Yet ye have forsaken me, and served other gods: wherefore I will deliver you no more.
\verse Go and cry unto the gods which ye have chosen; let them deliver you in the time of your tribulation.
\verse And the children of Israel said unto the \LORD, We have sinned: do thou unto us whatsoever seemeth good unto thee; deliver us only, we pray thee, this day.
\verse And they put away the strange gods from among them, and served the \LORD: and his soul was grieved for the misery of Israel.
\verse Then the children of Ammon were gathered together, and encamped in Gilead. And the children of Israel assembled themselves together, and encamped in Mizpeh.
\verse And the people and princes of Gilead said one to another, What man is he that will begin to fight against the children of Ammon? he shall be head over all the inhabitants of Gilead.
\end{biblechapter}

\begin{biblechapter} % Judges 11
\verse Now Jephthah the Gileadite was a mighty man of valour, and he was the son of an harlot: and Gilead begat Jephthah.
\verse And Gilead's wife bare him sons; and his wife's sons grew up, and they thrust out Jephthah, and said unto him, Thou shalt not inherit in our father's house; for thou art the son of a strange woman.
\verse Then Jephthah fled from his brethren, and dwelt in the land of Tob: and there were gathered vain men to Jephthah, and went out with him.
\verse And it came to pass in process of time, that the children of Ammon made war against Israel.
\verse And it was so, that when the children of Ammon made war against Israel, the elders of Gilead went to fetch Jephthah out of the land of Tob:
\verse And they said unto Jephthah, Come, and be our captain, that we may fight with the children of Ammon.
\verse And Jephthah said unto the elders of Gilead, Did not ye hate me, and expel me out of my father's house? and why are ye come unto me now when ye are in distress?
\verse And the elders of Gilead said unto Jephthah, Therefore we turn again to thee now, that thou mayest go with us, and fight against the children of Ammon, and be our head over all the inhabitants of Gilead.
\verse And Jephthah said unto the elders of Gilead, If ye bring me home again to fight against the children of Ammon, and the \LORD deliver them before me, shall I be your head?
\verse And the elders of Gilead said unto Jephthah, The \LORD be witness between us, if we do not so according to thy words.
\verse Then Jephthah went with the elders of Gilead, and the people made him head and captain over them: and Jephthah uttered all his words before the \LORD in Mizpeh.
\verse And Jephthah sent messengers unto the king of the children of Ammon, saying, What hast thou to do with me, that thou art come against me to fight in my land?
\verse And the king of the children of Ammon answered unto the messengers of Jephthah, Because Israel took away my land, when they came up out of Egypt, from Arnon even unto Jabbok, and unto Jordan: now therefore restore those lands again peaceably.
\verse And Jephthah sent messengers again unto the king of the children of Ammon:
\verse And said unto him, Thus saith Jephthah, Israel took not away the land of Moab, nor the land of the children of Ammon:
\verse But when Israel came up from Egypt, and walked through the wilderness unto the Red sea, and came to Kadesh;
\verse Then Israel sent messengers unto the king of Edom, saying, Let me, I pray thee, pass through thy land: but the king of Edom would not hearken thereto. And in like manner they sent unto the king of Moab: but he would not consent: and Israel abode in Kadesh.
\verse Then they went along through the wilderness, and compassed the land of Edom, and the land of Moab, and came by the east side of the land of Moab, and pitched on the other side of Arnon, but came not within the border of Moab: for Arnon was the border of Moab.
\verse And Israel sent messengers unto Sihon king of the Amorites, the king of Heshbon; and Israel said unto him, Let us pass, we pray thee, through thy land into my place.
\verse But Sihon trusted not Israel to pass through his coast: but Sihon gathered all his people together, and pitched in Jahaz, and fought against Israel.
\verse And the \LORD God of Israel delivered Sihon and all his people into the hand of Israel, and they smote them: so Israel possessed all the land of the Amorites, the inhabitants of that country.
\verse And they possessed all the coasts of the Amorites, from Arnon even unto Jabbok, and from the wilderness even unto Jordan.
\verse So now the \LORD God of Israel hath dispossessed the Amorites from before his people Israel, and shouldest thou possess it?
\verse Wilt not thou possess that which Chemosh thy god giveth thee to possess? So whomsoever the \LORD our God shall drive out from before us, them will we possess.
\verse And now art thou any thing better than Balak the son of Zippor, king of Moab? did he ever strive against Israel, or did he ever fight against them,
\verse While Israel dwelt in Heshbon and her towns, and in Aroer and her towns, and in all the cities that be along by the coasts of Arnon, three hundred years? why therefore did ye not recover them within that time?
\verse Wherefore I have not sinned against thee, but thou doest me wrong to war against me: the \LORD the Judge be judge this day between the children of Israel and the children of Ammon.
\verse Howbeit the king of the children of Ammon hearkened not unto the words of Jephthah which he sent him.
\verse Then the Spirit of the \LORD came upon Jephthah, and he passed over Gilead, and Manasseh, and passed over Mizpeh of Gilead, and from Mizpeh of Gilead he passed over unto the children of Ammon.
\verse And Jephthah vowed a vow unto the \LORD, and said, If thou shalt without fail deliver the children of Ammon into mine hands,
\verse Then it shall be, that whatsoever cometh forth of the doors of my house to meet me, when I return in peace from the children of Ammon, shall surely be the \LORDs, and I will offer it up for a burnt offering.
\verse So Jephthah passed over unto the children of Ammon to fight against them; and the \LORD delivered them into his hands.
\verse And he smote them from Aroer, even till thou come to Minnith, even twenty cities, and unto the plain of the vineyards, with a very great slaughter. Thus the children of Ammon were subdued before the children of Israel.
\verse And Jephthah came to Mizpeh unto his house, and, behold, his daughter came out to meet him with timbrels and with dances: and she was his only child; beside her he had neither son nor daughter.
\verse And it came to pass, when he saw her, that he rent his clothes, and said, Alas, my daughter! thou hast brought me very low, and thou art one of them that trouble me: for I have opened my mouth unto the \LORD, and I cannot go back.
\verse And she said unto him, My father, if thou hast opened thy mouth unto the \LORD, do to me according to that which hath proceeded out of thy mouth; forasmuch as the \LORD hath taken vengeance for thee of thine enemies, even of the children of Ammon.
\verse And she said unto her father, Let this thing be done for me: let me alone two months, that I may go up and down upon the mountains, and bewail my virginity, I and my fellows.
\verse And he said, Go. And he sent her away for two months: and she went with her companions, and bewailed her virginity upon the mountains.
\verse And it came to pass at the end of two months, that she returned unto her father, who did with her according to his vow which he had vowed: and she knew no man. And it was a custom in Israel,
\verse That the daughters of Israel went yearly to lament the daughter of Jephthah the Gileadite four days in a year.
\end{biblechapter}

\begin{biblechapter} % Judges 12
\verseWithHeading{Jephthah and Ephraim} And the men of Ephraim gathered themselves together, and went northward, and said unto Jephthah, Wherefore passedst thou over to fight against the children of Ammon, and didst not call us to go with thee? we will burn thine house upon thee with fire.
\verse And Jephthah said unto them, I and my people were at great strife with the children of Ammon; and when I called you, ye delivered me not out of their hands.
\verse And when I saw that ye delivered me not, I put my life in my hands, and passed over against the children of Ammon, and the \LORD delivered them into my hand: wherefore then are ye come up unto me this day, to fight against me?
\verse Then Jephthah gathered together all the men of Gilead, and fought with Ephraim: and the men of Gilead smote Ephraim, because they said, Ye Gileadites are fugitives of Ephraim among the Ephraimites, and among the Manassites.
\verse And the Gileadites took the passages of Jordan before the Ephraimites: and it was so, that when those Ephraimites which were escaped said, Let me go over; that the men of Gilead said unto him, Art thou an Ephraimite? If he said, Nay;
\verse Then said they unto him, Say now Shibboleth: and he said Sibboleth: for he could not frame to pronounce it right. Then they took him, and slew him at the passages of Jordan: and there fell at that time of the Ephraimites forty and two thousand.
\verse And Jephthah judged Israel six years. Then died Jephthah the Gileadite, and was buried in one of the cities of Gilead.
\verseWithHeading{Ibzan, Elon, and Abdon} And after him Ibzan of Bethlehem judged Israel.
\verse And he had thirty sons, and thirty daughters, whom he sent abroad, and took in thirty daughters from abroad for his sons. And he judged Israel seven years.
\verse Then died Ibzan, and was buried at Bethlehem.
\verse And after him Elon, a Zebulonite, judged Israel; and he judged Israel ten years.
\verse And Elon the Zebulonite died, and was buried in Aijalon in the country of Zebulun.
\verse And after him Abdon the son of Hillel, a Pirathonite, judged Israel.
\verse And he had forty sons and thirty nephews, that rode on threescore and ten ass colts: and he judged Israel eight years.
\verse And Abdon the son of Hillel the Pirathonite died, and was buried in Pirathon in the land of Ephraim, in the mount of the Amalekites.
\end{biblechapter}

\begin{biblechapter} % Judges 13
\verseWithHeading{The birth of Samson} And the children of Israel did evil again in the sight of the \LORD; and the \LORD delivered them into the hand of the Philistines forty years.
\verse And there was a certain man of Zorah, of the family of the Danites, whose name was Manoah; and his wife was barren, and bare not.
\verse And the angel of the \LORD appeared unto the woman, and said unto her, Behold now, thou art barren, and bearest not: but thou shalt conceive, and bear a son.
\verse Now therefore beware, I pray thee, and drink not wine nor strong drink, and eat not any unclean thing:
\verse For, lo, thou shalt conceive, and bear a son; and no razor shall come on his head: for the child shall be a Nazarite unto God from the womb: and he shall begin to deliver Israel out of the hand of the Philistines.
\verse Then the woman came and told her husband, saying, A man of God came unto me, and his countenance was like the countenance of an angel of God, very terrible: but I asked him not whence he was, neither told he me his name:
\verse But he said unto me, Behold, thou shalt conceive, and bear a son; and now drink no wine nor strong drink, neither eat any unclean thing: for the child shall be a Nazarite to God from the womb to the day of his death.
\verse Then Manoah intreated the \LORD, and said, O my Lord, let the man of God which thou didst send come again unto us, and teach us what we shall do unto the child that shall be born.
\verse And God hearkened to the voice of Manoah; and the angel of God came again unto the woman as she sat in the field: but Manoah her husband was not with her.
\verse And the woman made haste, and ran, and shewed her husband, and said unto him, Behold, the man hath appeared unto me, that came unto me the other day.
\verse And Manoah arose, and went after his wife, and came to the man, and said unto him, Art thou the man that spakest unto the woman? And he said, I am.
\verse And Manoah said, Now let thy words come to pass. How shall we order the child, and how shall we do unto him?
\verse And the angel of the \LORD said unto Manoah, Of all that I said unto the woman let her beware.
\verse She may not eat of any thing that cometh of the vine, neither let her drink wine or strong drink, nor eat any unclean thing: all that I commanded her let her observe.
\verse And Manoah said unto the angel of the \LORD, I pray thee, let us detain thee, until we shall have made ready a kid for thee.
\verse And the angel of the \LORD said unto Manoah, Though thou detain me, I will not eat of thy bread: and if thou wilt offer a burnt offering, thou must offer it unto the \LORD. For Manoah knew not that he was an angel of the \LORD.
\verse And Manoah said unto the angel of the \LORD, What is thy name, that when thy sayings come to pass we may do thee honour?
\verse And the angel of the \LORD said unto him, Why askest thou thus after my name, seeing it is secret?
\verse So Manoah took a kid with a meat offering, and offered it upon a rock unto the \LORD: and the angel did wondrously; and Manoah and his wife looked on.
\verse For it came to pass, when the flame went up toward heaven from off the altar, that the angel of the \LORD ascended in the flame of the altar. And Manoah and his wife looked on it, and fell on their faces to the ground.
\verse But the angel of the \LORD did no more appear to Manoah and to his wife. Then Manoah knew that he was an angel of the \LORD.
\verse And Manoah said unto his wife, We shall surely die, because we have seen God.
\verse But his wife said unto him, If the \LORD were pleased to kill us, he would not have received a burnt offering and a meat offering at our hands, neither would he have shewed us all these things, nor would as at this time have told us such things as these.
\verse And the woman bare a son, and called his name Samson: and the child grew, and the \LORD blessed him.
\verse And the Spirit of the \LORD began to move him at times in the camp of Dan between Zorah and Eshtaol.
\end{biblechapter}

\begin{biblechapter} % Judges 14
\verseWithHeading{Samson's marriage} And Samson went down to Timnath, and saw a woman in Timnath of the daughters of the Philistines.
\verse And he came up, and told his father and his mother, and said, I have seen a woman in Timnath of the daughters of the Philistines: now therefore get her for me to wife.
\verse Then his father and his mother said unto him, Is there never a woman among the daughters of thy brethren, or among all my people, that thou goest to take a wife of the uncircumcised Philistines? And Samson said unto his father, Get her for me; for she pleaseth me well.
\verse But his father and his mother knew not that it was of the \LORD, that he sought an occasion against the Philistines: for at that time the Philistines had dominion over Israel.
\verse Then went Samson down, and his father and his mother, to Timnath, and came to the vineyards of Timnath: and, behold, a young lion roared against him.
\verse And the Spirit of the \LORD came mightily upon him, and he rent him as he would have rent a kid, and he had nothing in his hand: but he told not his father or his mother what he had done.
\verse And he went down, and talked with the woman; and she pleased Samson well.
\verse And after a time he returned to take her, and he turned aside to see the carcase of the lion: and, behold, there was a swarm of bees and honey in the carcase of the lion.
\verse And he took thereof in his hands, and went on eating, and came to his father and mother, and he gave them, and they did eat: but he told not them that he had taken the honey out of the carcase of the lion.
\verse So his father went down unto the woman: and Samson made there a feast; for so used the young men to do.
\verse And it came to pass, when they saw him, that they brought thirty companions to be with him.
\verse And Samson said unto them, I will now put forth a riddle unto you: if ye can certainly declare it me within the seven days of the feast, and find it out, then I will give you thirty sheets and thirty change of garments:
\verse But if ye cannot declare it me, then shall ye give me thirty sheets and thirty change of garments. And they said unto him, Put forth thy riddle, that we may hear it.
\verse And he said unto them, Out of the eater came forth meat, and out of the strong came forth sweetness. And they could not in three days expound the riddle.
\verse And it came to pass on the seventh day, that they said unto Samson's wife, Entice thy husband, that he may declare unto us the riddle, lest we burn thee and thy father's house with fire: have ye called us to take that we have? is it not so?
\verse And Samson's wife wept before him, and said, Thou dost but hate me, and lovest me not: thou hast put forth a riddle unto the children of my people, and hast not told it me. And he said unto her, Behold, I have not told it my father nor my mother, and shall I tell it thee?
\verse And she wept before him the seven days, while their feast lasted: and it came to pass on the seventh day, that he told her, because she lay sore upon him: and she told the riddle to the children of her people.
\verse And the men of the city said unto him on the seventh day before the sun went down, What is sweeter than honey? and what is stronger than a lion? And he said unto them, If ye had not plowed with my heifer, ye had not found out my riddle.
\verse And the Spirit of the \LORD came upon him, and he went down to Ashkelon, and slew thirty men of them, and took their spoil, and gave change of garments unto them which expounded the riddle. And his anger was kindled, and he went up to his father's house.
\verse But Samson's wife was given to his companion, whom he had used as his friend.
\end{biblechapter}

\begin{biblechapter} % Judges 15
\verseWithHeading{Samson's vengeance on the \newline Philistines} But it came to pass within a while after, in the time of wheat harvest, that Samson visited his wife with a kid; and he said, I will go in to my wife into the chamber. But her father would not suffer him to go in.
\verse And her father said, I verily thought that thou hadst utterly hated her; therefore I gave her to thy companion: is not her younger sister fairer than she? take her, I pray thee, instead of her.
\verse And Samson said concerning them, Now shall I be more blameless than the Philistines, though I do them a displeasure.
\verse And Samson went and caught three hundred foxes, and took firebrands, and turned tail to tail, and put a firebrand in the midst between two tails.
\verse And when he had set the brands on fire, he let them go into the standing corn of the Philistines, and burnt up both the shocks, and also the standing corn, with the vineyards and olives.
\verse Then the Philistines said, Who hath done this? And they answered, Samson, the son in law of the Timnite, because he had taken his wife, and given her to his companion. And the Philistines came up, and burnt her and her father with fire.
\verse And Samson said unto them, Though ye have done this, yet will I be avenged of you, and after that I will cease.
\verse And he smote them hip and thigh with a great slaughter: and he went down and dwelt in the top of the rock Etam.
\verse Then the Philistines went up, and pitched in Judah, and spread themselves in Lehi.
\verse And the men of Judah said, Why are ye come up against us? And they answered, To bind Samson are we come up, to do to him as he hath done to us.
\verse Then three thousand men of Judah went to the top of the rock Etam, and said to Samson, Knowest thou not that the Philistines are rulers over us? what is this that thou hast done unto us? And he said unto them, As they did unto me, so have I done unto them.
\verse And they said unto him, We are come down to bind thee, that we may deliver thee into the hand of the Philistines. And Samson said unto them, Swear unto me, that ye will not fall upon me yourselves.
\verse And they spake unto him, saying, No; but we will bind thee fast, and deliver thee into their hand: but surely we will not kill thee. And they bound him with two new cords, and brought him up from the rock.
\verse And when he came unto Lehi, the Philistines shouted against him: and the Spirit of the \LORD came mightily upon him, and the cords that were upon his arms became as flax that was burnt with fire, and his bands loosed from off his hands.
\verse And he found a new jawbone of an ass, and put forth his hand, and took it, and slew a thousand men therewith.
\verse And Samson said, With the jawbone of an ass, heaps upon heaps, with the jaw of an ass have I slain a thousand men.
\verse And it came to pass, when he had made an end of speaking, that he cast away the jawbone out of his hand, and called that place Ramathlehi.
\verse And he was sore athirst, and called on the \LORD, and said, Thou hast given this great deliverance into the hand of thy servant: and now shall I die for thirst, and fall into the hand of the uncircumcised?
\verse But God clave an hollow place that was in the jaw, and there came water thereout; and when he had drunk, his spirit came again, and he revived: wherefore he called the name thereof Enhakkore, which is in Lehi unto this day.
\verse And he judged Israel in the days of the Philistines twenty years.
\end{biblechapter}

\begin{biblechapter} % Judges 16
\verseWithHeading{Samson and Delilah} Then went Samson to Gaza, and saw there an harlot, and went in unto her.
\verse And it was told the Gazites, saying, Samson is come hither. And they compassed him in, and laid wait for him all night in the gate of the city, and were quiet all the night, saying, In the morning, when it is day, we shall kill him.
\verse And Samson lay till midnight, and arose at midnight, and took the doors of the gate of the city, and the two posts, and went away with them, bar and all, and put them upon his shoulders, and carried them up to the top of an hill that is before Hebron.
\verse And it came to pass afterward, that he loved a woman in the valley of Sorek, whose name was Delilah.
\verse And the lords of the Philistines came up unto her, and said unto her, Entice him, and see wherein his great strength lieth, and by what means we may prevail against him, that we may bind him to afflict him: and we will give thee every one of us eleven hundred pieces of silver.
\verse And Delilah said to Samson, Tell me, I pray thee, wherein thy great strength lieth, and wherewith thou mightest be bound to afflict thee.
\verse And Samson said unto her, If they bind me with seven green withs that were never dried, then shall I be weak, and be as another man.
\verse Then the lords of the Philistines brought up to her seven green withs which had not been dried, and she bound him with them.
\verse Now there were men lying in wait, abiding with her in the chamber. And she said unto him, The Philistines be upon thee, Samson. And he brake the withs, as a thread of tow is broken when it toucheth the fire. So his strength was not known.
\verse And Delilah said unto Samson, Behold, thou hast mocked me, and told me lies: now tell me, I pray thee, wherewith thou mightest be bound.
\verse And he said unto her, If they bind me fast with new ropes that never were occupied, then shall I be weak, and be as another man.
\verse Delilah therefore took new ropes, and bound him therewith, and said unto him, The Philistines be upon thee, Samson. And there were liers in wait abiding in the chamber. And he brake them from off his arms like a thread.
\verse And Delilah said unto Samson, Hitherto thou hast mocked me, and told me lies: tell me wherewith thou mightest be bound. And he said unto her, If thou weavest the seven locks of my head with the web.
\verse And she fastened it with the pin, and said unto him, The Philistines be upon thee, Samson. And he awaked out of his sleep, and went away with the pin of the beam, and with the web.
\verse And she said unto him, How canst thou say, I love thee, when thine heart is not with me? thou hast mocked me these three times, and hast not told me wherein thy great strength lieth.
\verse And it came to pass, when she pressed him daily with her words, and urged him, so that his soul was vexed unto death;
\verse That he told her all his heart, and said unto her, There hath not come a razor upon mine head; for I have been a Nazarite unto God from my mother's womb: if I be shaven, then my strength will go from me, and I shall become weak, and be like any other man.
\verse And when Delilah saw that he had told her all his heart, she sent and called for the lords of the Philistines, saying, Come up this once, for he hath shewed me all his heart. Then the lords of the Philistines came up unto her, and brought money in their hand.
\verse And she made him sleep upon her knees; and she called for a man, and she caused him to shave off the seven locks of his head; and she began to afflict him, and his strength went from him.
\verse And she said, The Philistines be upon thee, Samson. And he awoke out of his sleep, and said, I will go out as at other times before, and shake myself. And he wist not that the \LORD was departed from him.
\verse But the Philistines took him, and put out his eyes, and brought him down to Gaza, and bound him with fetters of brass; and he did grind in the prison house.
\verse Howbeit the hair of his head began to grow again after he was shaven.
\verseWithHeading{The death of Samson} Then the lords of the Philistines gathered them together for to offer a great sacrifice unto Dagon their god, and to rejoice: for they said, Our god hath delivered Samson our enemy into our hand.
\verse And when the people saw him, they praised their god: for they said, Our god hath delivered into our hands our enemy, and the destroyer of our country, which slew many of us.
\verse And it came to pass, when their hearts were merry, that they said, Call for Samson, that he may make us sport. And they called for Samson out of the prison house; and he made them sport: and they set him between the pillars.
\verse And Samson said unto the lad that held him by the hand, Suffer me that I may feel the pillars whereupon the house standeth, that I may lean upon them.
\verse Now the house was full of men and women; and all the lords of the Philistines were there; and there were upon the roof about three thousand men and women, that beheld while Samson made sport.
\verse And Samson called unto the \LORD, and said, O Lord GOD, remember me, I pray thee, and strengthen me, I pray thee, only this once, O God, that I may be at once avenged of the Philistines for my two eyes.
\verse And Samson took hold of the two middle pillars upon which the house stood, and on which it was borne up, of the one with his right hand, and of the other with his left.
\verse And Samson said, Let me die with the Philistines. And he bowed himself with all his might; and the house fell upon the lords, and upon all the people that were therein. So the dead which he slew at his death were more than they which he slew in his life.
\verse Then his brethren and all the house of his father came down, and took him, and brought him up, and buried him between Zorah and Eshtaol in the buryingplace of Manoah his father. And he judged Israel twenty years.
\end{biblechapter}

\begin{biblechapter} % Judges 17
\verseWithHeading{Micah's idols} And there was a man of mount Ephraim, whose name was Micah.
\verse And he said unto his mother, The eleven hundred shekels of silver that were taken from thee, about which thou cursedst, and spakest of also in mine ears, behold, the silver is with me; I took it. And his mother said, Blessed be thou of the \LORD, my son.
\verse And when he had restored the eleven hundred shekels of silver to his mother, his mother said, I had wholly dedicated the silver unto the \LORD from my hand for my son, to make a graven image and a molten image: now therefore I will restore it unto thee.
\verse Yet he restored the money unto his mother; and his mother took two hundred shekels of silver, and gave them to the founder, who made thereof a graven image and a molten image: and they were in the house of Micah.
\verse And the man Micah had an house of gods, and made an ephod, and teraphim, and consecrated one of his sons, who became his priest.
\verse In those days there was no king in Israel, but every man did that which was right in his own eyes.
\verse And there was a young man out of Bethlehemjudah of the family of Judah, who was a Levite, and he sojourned there.
\verse And the man departed out of the city from Bethlehemjudah to sojourn where he could find a place: and he came to mount Ephraim to the house of Micah, as he journeyed.
\verse And Micah said unto him, Whence comest thou? And he said unto him, I am a Levite of Bethlehemjudah, and I go to sojourn where I may find a place.
\verse And Micah said unto him, Dwell with me, and be unto me a father and a priest, and I will give thee ten shekels of silver by the year, and a suit of apparel, and thy victuals. So the Levite went in.
\verse And the Levite was content to dwell with the man; and the young man was unto him as one of his sons.
\verse And Micah consecrated the Levite; and the young man became his priest, and was in the house of Micah.
\verse Then said Micah, Now know I that the \LORD will do me good, seeing I have a Levite to my priest.
\end{biblechapter}

\begin{biblechapter} % Judges 18
\verseWithHeading{The Danites settle in Laish} In those days there was no king in Israel: and in those days the tribe of the Danites sought them an inheritance to dwell in; for unto that day all their inheritance had not fallen unto them among the tribes of Israel.
\verse And the children of Dan sent of their family five men from their coasts, men of valour, from Zorah, and from Eshtaol, to spy out the land, and to search it; and they said unto them, Go, search the land: who when they came to mount Ephraim, to the house of Micah, they lodged there.
\verse When they were by the house of Micah, they knew the voice of the young man the Levite: and they turned in thither, and said unto him, Who brought thee hither? and what makest thou in this place? and what hast thou here?
\verse And he said unto them, Thus and thus dealeth Micah with me, and hath hired me, and I am his priest.
\verse And they said unto him, Ask counsel, we pray thee, of God, that we may know whether our way which we go shall be prosperous.
\verse And the priest said unto them, Go in peace: before the \LORD is your way wherein ye go.
\verse Then the five men departed, and came to Laish, and saw the people that were therein, how they dwelt careless, after the manner of the Zidonians, quiet and secure; and there was no magistrate in the land, that might put them to shame in any thing; and they were far from the Zidonians, and had no business with any man.
\verse And they came unto their brethren to Zorah and Eshtaol: and their brethren said unto them, What say ye?
\verse And they said, Arise, that we may go up against them: for we have seen the land, and, behold, it is very good: and are ye still? be not slothful to go, and to enter to possess the land.
\verse When ye go, ye shall come unto a people secure, and to a large land: for God hath given it into your hands; a place where there is no want of any thing that is in the earth.
\verse And there went from thence of the family of the Danites, out of Zorah and out of Eshtaol, six hundred men appointed with weapons of war.
\verse And they went up, and pitched in Kirjathjearim, in Judah: wherefore they called that place Mahanehdan unto this day: behold, it is behind Kirjathjearim.
\verse And they passed thence unto mount Ephraim, and came unto the house of Micah.
\verse Then answered the five men that went to spy out the country of Laish, and said unto their brethren, Do ye know that there is in these houses an ephod, and teraphim, and a graven image, and a molten image? now therefore consider what ye have to do.
\verse And they turned thitherward, and came to the house of the young man the Levite, even unto the house of Micah, and saluted him.
\verse And the six hundred men appointed with their weapons of war, which were of the children of Dan, stood by the entering of the gate.
\verse And the five men that went to spy out the land went up, and came in thither, and took the graven image, and the ephod, and the teraphim, and the molten image: and the priest stood in the entering of the gate with the six hundred men that were appointed with weapons of war.
\verse And these went into Micah's house, and fetched the carved image, the ephod, and the teraphim, and the molten image. Then said the priest unto them, What do ye?
\verse And they said unto him, Hold thy peace, lay thine hand upon thy mouth, and go with us, and be to us a father and a priest: is it better for thee to be a priest unto the house of one man, or that thou be a priest unto a tribe and a family in Israel?
\verse And the priest's heart was glad, and he took the ephod, and the teraphim, and the graven image, and went in the midst of the people.
\verse So they turned and departed, and put the little ones and the cattle and the carriage before them.
\verse And when they were a good way from the house of Micah, the men that were in the houses near to Micah's house were gathered together, and overtook the children of Dan.
\verse And they cried unto the children of Dan. And they turned their faces, and said unto Micah, What aileth thee, that thou comest with such a company?
\verse And he said, Ye have taken away my gods which I made, and the priest, and ye are gone away: and what have I more? and what is this that ye say unto me, What aileth thee?
\verse And the children of Dan said unto him, Let not thy voice be heard among us, lest angry fellows run upon thee, and thou lose thy life, with the lives of thy household.
\verse And the children of Dan went their way: and when Micah saw that they were too strong for him, he turned and went back unto his house.
\verse And they took the things which Micah had made, and the priest which he had, and came unto Laish, unto a people that were at quiet and secure: and they smote them with the edge of the sword, and burnt the city with fire.
\verse And there was no deliverer, because it was far from Zidon, and they had no business with any man; and it was in the valley that lieth by Bethrehob. And they built a city, and dwelt therein.
\verse And they called the name of the city Dan, after the name of Dan their father, who was born unto Israel: howbeit the name of the city was Laish at the first.
\verse And the children of Dan set up the graven image: and Jonathan, the son of Gershom, the son of Manasseh, he and his sons were priests to the tribe of Dan until the day of the captivity of the land.
\verse And they set them up Micah's graven image, which he made, all the time that the house of God was in Shiloh.
\end{biblechapter}

\begin{biblechapter} % Judges 19
\verseWithHeading{The Levite and his concubine} And it came to pass in those days, when there was no king in Israel, that there was a certain Levite sojourning on the side of mount Ephraim, who took to him a concubine out of Bethlehemjudah.
\verse And his concubine played the whore against him, and went away from him unto her father's house to Bethlehemjudah, and was there four whole months.
\verse And her husband arose, and went after her, to speak friendly unto her, and to bring her again, having his servant with him, and a couple of asses: and she brought him into her father's house: and when the father of the damsel saw him, he rejoiced to meet him.
\verse And his father in law, the damsel's father, retained him; and he abode with him three days: so they did eat and drink, and lodged there.
\verse And it came to pass on the fourth day, when they arose early in the morning, that he rose up to depart: and the damsel's father said unto his son in law, Comfort thine heart with a morsel of bread, and afterward go your way.
\verse And they sat down, and did eat and drink both of them together: for the damsel's father had said unto the man, Be content, I pray thee, and tarry all night, and let thine heart be merry.
\verse And when the man rose up to depart, his father in law urged him: therefore he lodged there again.
\verse And he arose early in the morning on the fifth day to depart: and the damsel's father said, Comfort thine heart, I pray thee. And they tarried until afternoon, and they did eat both of them.
\verse And when the man rose up to depart, he, and his concubine, and his servant, his father in law, the damsel's father, said unto him, Behold, now the day draweth toward evening, I pray you tarry all night: behold, the day groweth to an end, lodge here, that thine heart may be merry; and to morrow get you early on your way, that thou mayest go home.
\verse But the man would not tarry that night, but he rose up and departed, and came over against Jebus, which is Jerusalem; and there were with him two asses saddled, his concubine also was with him.
\verse And when they were by Jebus, the day was far spent; and the servant said unto his master, Come, I pray thee, and let us turn in into this city of the Jebusites, and lodge in it.
\verse And his master said unto him, We will not turn aside hither into the city of a stranger, that is not of the children of Israel; we will pass over to Gibeah.
\verse And he said unto his servant, Come, and let us draw near to one of these places to lodge all night, in Gibeah, or in Ramah.
\verse And they passed on and went their way; and the sun went down upon them when they were by Gibeah, which belongeth to Benjamin.
\verse And they turned aside thither, to go in and to lodge in Gibeah: and when he went in, he sat him down in a street of the city: for there was no man that took them into his house to lodging.
\verse And, behold, there came an old man from his work out of the field at even, which was also of mount Ephraim; and he sojourned in Gibeah: but the men of the place were Benjamites.
\verse And when he had lifted up his eyes, he saw a wayfaring man in the street of the city: and the old man said, Whither goest thou? and whence comest thou?
\verse And he said unto him, We are passing from Bethlehemjudah toward the side of mount Ephraim; from thence am I: and I went to Bethlehemjudah, but I am now going to the house of the \LORD; and there is no man that receiveth me to house.
\verse Yet there is both straw and provender for our asses; and there is bread and wine also for me, and for thy handmaid, and for the young man which is with thy servants: there is no want of any thing.
\verse And the old man said, Peace be with thee; howsoever let all thy wants lie upon me; only lodge not in the street.
\verse So he brought him into his house, and gave provender unto the asses: and they washed their feet, and did eat and drink.
\verse Now as they were making their hearts merry, behold, the men of the city, certain sons of Belial, beset the house round about, and beat at the door, and spake to the master of the house, the old man, saying, Bring forth the man that came into thine house, that we may know him.
\verse And the man, the master of the house, went out unto them, and said unto them, Nay, my brethren, nay, I pray you, do not so wickedly; seeing that this man is come into mine house, do not this folly.
\verse Behold, here is my daughter a maiden, and his concubine; them I will bring out now, and humble ye them, and do with them what seemeth good unto you: but unto this man do not so vile a thing.
\verse But the men would not hearken to him: so the man took his concubine, and brought her forth unto them; and they knew her, and abused her all the night until the morning: and when the day began to spring, they let her go.
\verse Then came the woman in the dawning of the day, and fell down at the door of the man's house where her lord was, till it was light.
\verse And her lord rose up in the morning, and opened the doors of the house, and went out to go his way: and, behold, the woman his concubine was fallen down at the door of the house, and her hands were upon the threshold.
\verse And he said unto her, Up, and let us be going. But none answered. Then the man took her up upon an ass, and the man rose up, and gat him unto his place.
\verse And when he was come into his house, he took a knife, and laid hold on his concubine, and divided her, together with her bones, into twelve pieces, and sent her into all the coasts of Israel.
\verse And it was so, that all that saw it said, There was no such deed done nor seen from the day that the children of Israel came up out of the land of Egypt unto this day: consider of it, take advice, and speak your minds.
\end{biblechapter}

\begin{biblechapter} % Judges 20
\verseWithHeading{The Israelites punish the \newline Benjaminites} Then all the children of Israel went out, and the congregation was gathered together as one man, from Dan even to Beersheba, with the land of Gilead, unto the \LORD in Mizpeh.
\verse And the chief of all the people, even of all the tribes of Israel, presented themselves in the assembly of the people of God, four hundred thousand footmen that drew sword.
\verse (Now the children of Benjamin heard that the children of Israel were gone up to Mizpeh.) Then said the children of Israel, Tell us, how was this wickedness?
\verse And the Levite, the husband of the woman that was slain, answered and said, I came into Gibeah that belongeth to Benjamin, I and my concubine, to lodge.
\verse And the men of Gibeah rose against me, and beset the house round about upon me by night, and thought to have slain me: and my concubine have they forced, that she is dead.
\verse And I took my concubine, and cut her in pieces, and sent her throughout all the country of the inheritance of Israel: for they have committed lewdness and folly in Israel.
\verse Behold, ye are all children of Israel; give here your advice and counsel.
\verse And all the people arose as one man, saying, We will not any of us go to his tent, neither will we any of us turn into his house.
\verse But now this shall be the thing which we will do to Gibeah; we will go up by lot against it;
\verse And we will take ten men of an hundred throughout all the tribes of Israel, and an hundred of a thousand, and a thousand out of ten thousand, to fetch victual for the people, that they may do, when they come to Gibeah of Benjamin, according to all the folly that they have wrought in Israel.
\verse So all the men of Israel were gathered against the city, knit together as one man.
\verse And the tribes of Israel sent men through all the tribe of Benjamin, saying, What wickedness is this that is done among you?
\verse Now therefore deliver us the men, the children of Belial, which are in Gibeah, that we may put them to death, and put away evil from Israel. But the children of Benjamin would not hearken to the voice of their brethren the children of Israel:
\verse But the children of Benjamin gathered themselves together out of the cities unto Gibeah, to go out to battle against the children of Israel.
\verse And the children of Benjamin were numbered at that time out of the cities twenty and six thousand men that drew sword, beside the inhabitants of Gibeah, which were numbered seven hundred chosen men.
\verse Among all this people there were seven hundred chosen men lefthanded; every one could sling stones at an hair breadth, and not miss.
\verse And the men of Israel, beside Benjamin, were numbered four hundred thousand men that drew sword: all these were men of war.
\verse And the children of Israel arose, and went up to the house of God, and asked counsel of God, and said, Which of us shall go up first to the battle against the children of Benjamin? And the \LORD said, Judah shall go up first.
\verse And the children of Israel rose up in the morning, and encamped against Gibeah.
\verse And the men of Israel went out to battle against Benjamin; and the men of Israel put themselves in array to fight against them at Gibeah.
\verse And the children of Benjamin came forth out of Gibeah, and destroyed down to the ground of the Israelites that day twenty and two thousand men.
\verse And the people the men of Israel encouraged themselves, and set their battle again in array in the place where they put themselves in array the first day.
\verse (And the children of Israel went up and wept before the \LORD until even, and asked counsel of the \LORD, saying, Shall I go up again to battle against the children of Benjamin my brother? And the \LORD said, Go up against him.)
\verse And the children of Israel came near against the children of Benjamin the second day.
\verse And Benjamin went forth against them out of Gibeah the second day, and destroyed down to the ground of the children of Israel again eighteen thousand men; all these drew the sword.
\verse Then all the children of Israel, and all the people, went up, and came unto the house of God, and wept, and sat there before the \LORD, and fasted that day until even, and offered burnt offerings and peace offerings before the \LORD.
\verse And the children of Israel enquired of the \LORD, (for the ark of the covenant of God was there in those days,
\verse And Phinehas, the son of Eleazar, the son of Aaron, stood before it in those days,) saying, Shall I yet again go out to battle against the children of Benjamin my brother, or shall I cease? And the \LORD said, Go up; for to morrow I will deliver them into thine hand.
\verse And Israel set liers in wait round about Gibeah.
\verse And the children of Israel went up against the children of Benjamin on the third day, and put themselves in array against Gibeah, as at other times.
\verse And the children of Benjamin went out against the people, and were drawn away from the city; and they began to smite of the people, and kill, as at other times, in the highways, of which one goeth up to the house of God, and the other to Gibeah in the field, about thirty men of Israel.
\verse And the children of Benjamin said, They are smitten down before us, as at the first. But the children of Israel said, Let us flee, and draw them from the city unto the highways.
\verse And all the men of Israel rose up out of their place, and put themselves in array at Baaltamar: and the liers in wait of Israel came forth out of their places, even out of the meadows of Gibeah.
\verse And there came against Gibeah ten thousand chosen men out of all Israel, and the battle was sore: but they knew not that evil was near them.
\verse And the \LORD smote Benjamin before Israel: and the children of Israel destroyed of the Benjamites that day twenty and five thousand and an hundred men: all these drew the sword.
\verse So the children of Benjamin saw that they were smitten: for the men of Israel gave place to the Benjamites, because they trusted unto the liers in wait which they had set beside Gibeah.
\verse And the liers in wait hasted, and rushed upon Gibeah; and the liers in wait drew themselves along, and smote all the city with the edge of the sword.
\verse Now there was an appointed sign between the men of Israel and the liers in wait, that they should make a great flame with smoke rise up out of the city.
\verse And when the men of Israel retired in the battle, Benjamin began to smite and kill of the men of Israel about thirty persons: for they said, Surely they are smitten down before us, as in the first battle.
\verse But when the flame began to arise up out of the city with a pillar of smoke, the Benjamites looked behind them, and, behold, the flame of the city ascended up to heaven.
\verse And when the men of Israel turned again, the men of Benjamin were amazed: for they saw that evil was come upon them.
\verse Therefore they turned their backs before the men of Israel unto the way of the wilderness; but the battle overtook them; and them which came out of the cities they destroyed in the midst of them.
\verse Thus they inclosed the Benjamites round about, and chased them, and trode them down with ease over against Gibeah toward the sunrising.
\verse And there fell of Benjamin eighteen thousand men; all these were men of valour.
\verse And they turned and fled toward the wilderness unto the rock of Rimmon: and they gleaned of them in the highways five thousand men; and pursued hard after them unto Gidom, and slew two thousand men of them.
\verse So that all which fell that day of Benjamin were twenty and five thousand men that drew the sword; all these were men of valour.
\verse But six hundred men turned and fled to the wilderness unto the rock Rimmon, and abode in the rock Rimmon four months.
\verse And the men of Israel turned again upon the children of Benjamin, and smote them with the edge of the sword, as well the men of every city, as the beast, and all that came to hand: also they set on fire all the cities that they came to.
\end{biblechapter}

\columnbreak % layout hack

\begin{biblechapter} % Judges 21
\verseWithHeading{Wives for the Benjaminites} Now the men of Israel had sworn in Mizpeh, saying, There shall not any of us give his daughter unto Benjamin to wife.
\verse And the people came to the house of God, and abode there till even before God, and lifted up their voices, and wept sore;
\verse And said, O \LORD God of Israel, why is this come to pass in Israel, that there should be to day one tribe lacking in Israel?
\verse And it came to pass on the morrow, that the people rose early, and built there an altar, and offered burnt offerings and peace offerings.
\verse And the children of Israel said, Who is there among all the tribes of Israel that came not up with the congregation unto the \LORD? For they had made a great oath concerning him that came not up to the \LORD to Mizpeh, saying, He shall surely be put to death.
\verse And the children of Israel repented them for Benjamin their brother, and said, There is one tribe cut off from Israel this day.
\verse How shall we do for wives for them that remain, seeing we have sworn by the \LORD that we will not give them of our daughters to wives?
\verse And they said, What one is there of the tribes of Israel that came not up to Mizpeh to the \LORD? And, behold, there came none to the camp from Jabeshgilead to the assembly.
\verse For the people were numbered, and, behold, there were none of the inhabitants of Jabeshgilead there.
\verse And the congregation sent thither twelve thousand men of the valiantest, and commanded them, saying, Go and smite the inhabitants of Jabeshgilead with the edge of the sword, with the women and the children.
\verse And this is the thing that ye shall do, Ye shall utterly destroy every male, and every woman that hath lain by man.
\verse And they found among the inhabitants of Jabeshgilead four hundred young virgins, that had known no man by lying with any male: and they brought them unto the camp to Shiloh, which is in the land of Canaan.
\verse And the whole congregation sent some to speak to the children of Benjamin that were in the rock Rimmon, and to call peaceably unto them.
\verse And Benjamin came again at that time; and they gave them wives which they had saved alive of the women of Jabeshgilead: and yet so they sufficed them not.
\verse And the people repented them for Benjamin, because that the \LORD had made a breach in the tribes of Israel.
\verse Then the elders of the congregation said, How shall we do for wives for them that remain, seeing the women are destroyed out of Benjamin?
\verse And they said, There must be an inheritance for them that be escaped of Benjamin, that a tribe be not destroyed out of Israel.
\verse Howbeit we may not give them wives of our daughters: for the children of Israel have sworn, saying, Cursed be he that giveth a wife to Benjamin.
\verse Then they said, Behold, there is a feast of the \LORD in Shiloh yearly in a place which is on the north side of Bethel, on the east side of the highway that goeth up from Bethel to Shechem, and on the south of Lebonah.
\verse Therefore they commanded the children of Benjamin, saying, Go and lie in wait in the vineyards;
\verse And see, and, behold, if the daughters of Shiloh come out to dance in dances, then come ye out of the vineyards, and catch you every man his wife of the daughters of Shiloh, and go to the land of Benjamin.
\verse And it shall be, when their fathers or their brethren come unto us to complain, that we will say unto them, Be favourable unto them for our sakes: because we reserved not to each man his wife in the war: for ye did not give unto them at this time, that ye should be guilty.
\verse And the children of Benjamin did so, and took them wives, according to their number, of them that danced, whom they caught: and they went and returned unto their inheritance, and repaired the cities, and dwelt in them.
\verse And the children of Israel departed thence at that time, every man to his tribe and to his family, and they went out from thence every man to his inheritance.
\verse In those days there was no king in Israel: every man did that which was right in his own eyes.
\end{biblechapter}
\flushcolsend
\biblebook{Ruth}

\begin{biblechapter} % Ruth 1
\verseWithHeading{Naomi loses her husband and sons} Now it came to pass in the days when the judges ruled, that there was a famine in the land. And a certain man of Bethlehemjudah went to sojourn in the country of Moab, he, and his wife, and his two sons.
\verse And the name of the man was Elimelech, and the name of his wife Naomi, and the name of his two sons Mahlon and Chilion, Ephrathites of Bethlehemjudah. And they came into the country of Moab, and continued there.
\verse And Elimelech Naomi's husband died; and she was left, and her two sons.
\verse And they took them wives of the women of Moab; the name of the one was Orpah, and the name of the other Ruth: and they dwelled there about ten years.
\verse And Mahlon and Chilion died also both of them; and the woman was left of her two sons and her husband.
\verseWithHeading{Naomi and Ruth return to Bethlehem} Then she arose with her daughters in law, that she might return from the country of Moab: for she had heard in the country of Moab how that the \LORD had visited his people in giving them bread.
\verse Wherefore she went forth out of the place where she was, and her two daughters in law with her; and they went on the way to return unto the land of Judah.
\verse And Naomi said unto her two daughters in law, Go, return each to her mother's house: the \LORD deal kindly with you, as ye have dealt with the dead, and with me.
\verse The \LORD grant you that ye may find rest, each of you in the house of her husband. Then she kissed them; and they lifted up their voice, and wept.
\verse And they said unto her, Surely we will return with thee unto thy people.
\verse And Naomi said, Turn again, my daughters: why will ye go with me? are there yet any more sons in my womb, that they may be your husbands?
\verse Turn again, my daughters, go your way; for I am too old to have an husband. If I should say, I have hope, if I should have an husband also to night, and should also bear sons;
\verse Would ye tarry for them till they were grown? would ye stay for them from having husbands? nay, my daughters; for it grieveth me much for your sakes that the hand of the \LORD is gone out against me.
\verse And they lifted up their voice, and wept again: and Orpah kissed her mother in law; but Ruth clave unto her.
\verse And she said, Behold, thy sister in law is gone back unto her people, and unto her gods: return thou after thy sister in law.
\verse And Ruth said, Intreat me not to leave thee, or to return from following after thee: for whither thou goest, I will go; and where thou lodgest, I will lodge: thy people shall be my people, and thy God my God:
\verse Where thou diest, will I die, and there will I be buried: the \LORD do so to me, and more also, if ought but death part thee and me.
\verse When she saw that she was stedfastly minded to go with her, then she left speaking unto her.
\verse So they two went until they came to Bethlehem. And it came to pass, when they were come to Bethlehem, that all the city was moved about them, and they said, Is this Naomi?
\verse And she said unto them, Call me not Naomi, call me Mara: for the Almighty hath dealt very bitterly with me.
\verse I went out full, and the \LORD hath brought me home again empty: why then call ye me Naomi, seeing the \LORD hath testified against me, and the Almighty hath afflicted me?
\verse So Naomi returned, and Ruth the Moabitess, her daughter in law, with her, which returned out of the country of Moab: and they came to Bethlehem in the beginning of barley harvest.
\end{biblechapter}

\begin{biblechapter} % Ruth 2
\verseWithHeading{Naomi meets Boaz in the \newline cornfield} And Naomi had a kinsman of her husband's, a mighty man of wealth, of the family of Elimelech; and his name was Boaz.
\verse And Ruth the Moabitess said unto Naomi, Let me now go to the field, and glean ears of corn after him in whose sight I shall find grace. And she said unto her, Go, my daughter.
\verse And she went, and came, and gleaned in the field after the reapers: and her hap was to light on a part of the field belonging unto Boaz, who was of the kindred of Elimelech.
\verse And, behold, Boaz came from Bethlehem, and said unto the reapers, The \LORD be with you. And they answered him, The \LORD bless thee.
\verse Then said Boaz unto his servant that was set over the reapers, Whose damsel is this?
\verse And the servant that was set over the reapers answered and said, It is the Moabitish damsel that came back with Naomi out of the country of Moab:
\verse And she said, I pray you, let me glean and gather after the reapers among the sheaves: so she came, and hath continued even from the morning until now, that she tarried a little in the house.
\verse Then said Boaz unto Ruth, Hearest thou not, my daughter? Go not to glean in another field, neither go from hence, but abide here fast by my maidens:
\verse Let thine eyes be on the field that they do reap, and go thou after them: have I not charged the young men that they shall not touch thee? and when thou art athirst, go unto the vessels, and drink of that which the young men have drawn.
\verse Then she fell on her face, and bowed herself to the ground, and said unto him, Why have I found grace in thine eyes, that thou shouldest take knowledge of me, seeing I am a stranger?
\verse And Boaz answered and said unto her, It hath fully been shewed me, all that thou hast done unto thy mother in law since the death of thine husband: and how thou hast left thy father and thy mother, and the land of thy nativity, and art come unto a people which thou knewest not heretofore.
\verse The \LORD recompense thy work, and a full reward be given thee of the \LORD God of Israel, under whose wings thou art come to trust.
\verse Then she said, Let me find favour in thy sight, my lord; for that thou hast comforted me, and for that thou hast spoken friendly unto thine handmaid, though I be not like unto one of thine handmaidens.
\verse And Boaz said unto her, At mealtime come thou hither, and eat of the bread, and dip thy morsel in the vinegar. And she sat beside the reapers: and he reached her parched corn, and she did eat, and was sufficed, and left.
\verse And when she was risen up to glean, Boaz commanded his young men, saying, Let her glean even among the sheaves, and reproach her not:
\verse And let fall also some of the handfuls of purpose for her, and leave them, that she may glean them, and rebuke her not.
\verse So she gleaned in the field until even, and beat out that she had gleaned: and it was about an ephah of barley.
\verse And she took it up, and went into the city: and her mother in law saw what she had gleaned: and she brought forth, and gave to her that she had reserved after she was sufficed.
\verse And her mother in law said unto her, Where hast thou gleaned to day? and where wroughtest thou? blessed be he that did take knowledge of thee. And she shewed her mother in law with whom she had wrought, and said, The man's name with whom I wrought to day is Boaz.
\verse And Naomi said unto her daughter in law, Blessed be he of the \LORD, who hath not left off his kindness to the living and to the dead. And Naomi said unto her, The man is near of kin unto us, one of our next kinsmen.
\verse And Ruth the Moabitess said, He said unto me also, Thou shalt keep fast by my young men, until they have ended all my harvest.
\verse And Naomi said unto Ruth her daughter in law, It is good, my daughter, that thou go out with his maidens, that they meet thee not in any other field.
\verse So she kept fast by the maidens of Boaz to glean unto the end of barley harvest and of wheat harvest; and dwelt with her mother in law.
\end{biblechapter}

\begin{biblechapter} % Ruth 3
\verseWithHeading{Ruth and Boaz at the threshing \newline floor} Then Naomi her mother in law said unto her, My daughter, shall I not seek rest for thee, that it may be well with thee?
\verse And now is not Boaz of our kindred, with whose maidens thou wast? Behold, he winnoweth barley to night in the threshingfloor.
\verse Wash thyself therefore, and anoint thee, and put thy raiment upon thee, and get thee down to the floor: but make not thyself known unto the man, until he shall have done eating and drinking.
\verse And it shall be, when he lieth down, that thou shalt mark the place where he shall lie, and thou shalt go in, and uncover his feet, and lay thee down; and he will tell thee what thou shalt do.
\verse And she said unto her, All that thou sayest unto me I will do.
\verse And she went down unto the floor, and did according to all that her mother in law bade her.
\verse And when Boaz had eaten and drunk, and his heart was merry, he went to lie down at the end of the heap of corn: and she came softly, and uncovered his feet, and laid her down.
\verse And it came to pass at midnight, that the man was afraid, and turned himself: and, behold, a woman lay at his feet.
\verse And he said, Who art thou? And she answered, I am Ruth thine handmaid: spread therefore thy skirt over thine handmaid; for thou art a near kinsman.
\verse And he said, Blessed be thou of the \LORD, my daughter: for thou hast shewed more kindness in the latter end than at the beginning, inasmuch as thou followedst not young men, whether poor or rich.
\verse And now, my daughter, fear not; I will do to thee all that thou requirest: for all the city of my people doth know that thou art a virtuous woman.
\verse And now it is true that I am thy near kinsman: howbeit there is a kinsman nearer than I.
\verse Tarry this night, and it shall be in the morning, that if he will perform unto thee the part of a kinsman, well; let him do the kinsman's part: but if he will not do the part of a kinsman to thee, then will I do the part of a kinsman to thee, as the \LORD liveth: lie down until the morning.
\verse And she lay at his feet until the morning: and she rose up before one could know another. And he said, Let it not be known that a woman came into the floor.
\verse Also he said, Bring the vail that thou hast upon thee, and hold it. And when she held it, he measured six measures of barley, and laid it on her: and she went into the city.
\verse And when she came to her mother in law, she said, Who art thou, my daughter? And she told her all that the man had done to her.
\verse And she said, These six measures of barley gave he me; for he said to me, Go not empty unto thy mother in law.
\verse Then said she, Sit still, my daughter, until thou know how the matter will fall: for the man will not be in rest, until he have finished the thing this day.
\end{biblechapter}

\begin{biblechapter} % Ruth 4
\verseWithHeading{Boaz marries Ruth} Then went Boaz up to the gate, and sat him down there: and, behold, the kinsman of whom Boaz spake came by; unto whom he said, Ho, such a one! turn aside, sit down here. And he turned aside, and sat down.
\verse And he took ten men of the elders of the city, and said, Sit ye down here. And they sat down.
\verse And he said unto the kinsman, Naomi, that is come again out of the country of Moab, selleth a parcel of land, which was our brother Elimelech's:
\verse And I thought to advertise thee, saying, Buy it before the inhabitants, and before the elders of my people. If thou wilt redeem it, redeem it: but if thou wilt not redeem it, then tell me, that I may know: for there is none to redeem it beside thee; and I am after thee. And he said, I will redeem it.
\verse Then said Boaz, What day thou buyest the field of the hand of Naomi, thou must buy it also of Ruth the Moabitess, the wife of the dead, to raise up the name of the dead upon his inheritance.
\verse And the kinsman said, I cannot redeem it for myself, lest I mar mine own inheritance: redeem thou my right to thyself; for I cannot redeem it.
\verse Now this was the manner in former time in Israel concerning redeeming and concerning changing, for to confirm all things; a man plucked off his shoe, and gave it to his neighbour: and this was a testimony in Israel.
\verse Therefore the kinsman said unto Boaz, Buy it for thee. So he drew off his shoe.
\verse And Boaz said unto the elders, and unto all the people, Ye are witnesses this day, that I have bought all that was Elimelech's, and all that was Chilion's and Mahlon's, of the hand of Naomi.
\verse Moreover Ruth the Moabitess, the wife of Mahlon, have I purchased to be my wife, to raise up the name of the dead upon his inheritance, that the name of the dead be not cut off from among his brethren, and from the gate of his place: ye are witnesses this day.
\verse And all the people that were in the gate, and the elders, said, We are witnesses. The \LORD make the woman that is come into thine house like Rachel and like Leah, which two did build the house of Israel: and do thou worthily in Ephratah, and be famous in Bethlehem:
\verse And let thy house be like the house of Pharez, whom Tamar bare unto Judah, of the seed which the \LORD shall give thee of this young woman.
\verseWithHeading{Naomi gains a son} So Boaz took Ruth, and she was his wife: and when he went in unto her, the \LORD gave her conception, and she bare a son.
\verse And the women said unto Naomi, Blessed be the \LORD, which hath not left thee this day without a kinsman, that his name may be famous in Israel.
\verse And he shall be unto thee a restorer of thy life, and a nourisher of thine old age: for thy daughter in law, which loveth thee, which is better to thee than seven sons, hath born him.
\verse And Naomi took the child, and laid it in her bosom, and became nurse unto it.
\verse And the women her neighbours gave it a name, saying, There is a son born to Naomi; and they called his name Obed: he is the father of Jesse, the father of David.
\verseWithHeading{The genealogy of David} Now these are the generations of Pharez: Pharez begat Hezron,
\verse And Hezron begat Ram, and Ram begat Amminadab,
\verse And Amminadab begat Nahshon, and Nahshon begat Salmon,
\verse And Salmon begat Boaz, and Boaz begat Obed,
\verse And Obed begat Jesse, and Jesse begat David.
\end{biblechapter}
\flushcolsend

\end{document}
