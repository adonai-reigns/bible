\onecolumn

\customsection[Preface]{The Lexham English Bible\\\large Fourth Edition*}



\setlength{\fboxsep}{.1625in}
\setlength{\fboxrule}{0.5pt}
\noindent\fbox{%
    \parbox{\textwidth - (.1625in*2 + 1pt)}{%
    
    \begin{headings}
        \begin{centering}
            \bfseries{*\ul{Publisher's Proofing Edition}, February 2020}
        \end{centering}
    \end{headings}

    \begin{headings}
        \begin{itemize}[itemsep=0in, leftmargin=0.125in]
            \item This copy of the Lexham English Bible is for the publisher to use while reviewing the content. The publisher has not yet endorsed the content.
            \item Text in this translation should be compared to reliable manuscripts for validation before being used as an authoritative statement.
            \item This disclaimer does not validate other translations and it does not imply that other translations are necessarily more reliable than this one.
        \end{itemize}
    \end{headings}
    
    }%
}


\begingroup
\fontsize{12pt}{11pt}\selectfont

With approximately one hundred different English translations of the Bible already published, the reader may well wonder 
why yet another English version has been produced. Those actually engaged in the work of translating the Bible might 
answer that the quest for increased accuracy, the incorporation of new scholarly discoveries in the fields of semantics, 
lexicography, linguistics, new archaeological discoveries, and the continuing evolution of the English language all 
contribute to the need for producing new translations. But in the case of the Lexham English Bible (LEB), the answer to 
this question is much simpler; in fact, it is merely twofold.\par

First, the LEB achieves an unparalleled level of transparency with the original language text because the LEB had as 
its starting point the Lexham Hebrew-English Interlinear Bible and the Lexham Greek-English Interlinear New Testament. 
It was produced with the specific purpose of being used alongside the original language text of the Bible. Existing 
translations, however excellent they may be in terms of English style and idiom, are frequently so far removed from 
the original language texts of Scripture that straightforward comparison is difficult for the average user. Of course 
distance between the original language text and the English translation is not a criticism of any modern English 
translation. To a large extent this distance is the result of the philosophy of translation chosen for a particular 
English version, and it is almost always the result of an attempt to convey the meaning of the original in a clearer 
and more easily understandable way to the contemporary reader. However, there are many readers, particularly those 
who have studied some biblical Hebrew, Aramaic, or Greek, who desire a translation that facilitates straightforward 
and easy comparisons between the translation and the original language text. The ability to make such comparisons 
easily in software formats like Logos Bible Software makes the need for an English translation specifically designed 
for such comparison even more acute.\par

Second, the LEB is designed from the beginning to make extensive use of the most up-to-date lexical reference works 
available. For the Old Testament this is primarily The Hebrew and Aramaic Lexicon of the Old Testament (HALOT), and for 
New Testament this is primarily the third edition of Walter Bauer's A Greek-English Lexicon of the New Testament and 
Other Early Christian Literature (BDAG). Users can be assured that the LEB as a translation is based on the best scholarly 
research available. The Hebrew text on which the LEB Old Testament is based is that of \textbf{Biblia Hebraica Stuttgartensia}. 
The Greek text on which the LEB New Testament is based is that of The Greek New Testament: \textbf{SBL Edition (SBLGNT)}, a new 
edition produced by Michael W. Holmes in conjunction with the Society of Biblical Literature and Logos Bible Software. 
In its evaluation of textual variation, the SBLGNT uses modern text-critical methodology along with guidance from the 
most recently available articles, monographs, and technical commentaries to establish the text of the Greek New Testament.\par

Naturally, when these two factors are taken into consideration, it should not be surprising that the character of the LEB 
as a translation is fairly literal. This is a necessary by-product of the desire to have the English translation correspond 
transparently to the original language text. Nevertheless, a serious attempt has been made within these constraints to produce 
a clear and readable English translation instead of a woodenly literal one.\par

There are three areas in particular that need to be addressed to make a translation like the LEB more accessible to readers 
today, while at the same time maintaining easy comparison with the original language text. First, differences in word order 
have to be addressed. In this regard, the LEB follows standard English word order, not the word order of biblical Hebrew, 
Aramaic, or Koiné Greek. Anyone who needs to see the word order of the original languages can readily consult the Lexham 
Hebrew-English Interlinear Bible or the Lexham Greek-English Interlinear New Testament, which contain a sequence line which 
gives this information. Second, some expressions in biblical languages are idiomatic, so that a literal translation would 
be meaningless or would miscommunicate the true meaning. \st{The LEB uses lower corner brackets to indicate such expressions, 
with a literal rendering given in a note}. Third, words which have no equivalent in the original language text must sometimes 
be supplied in the English translation. Because the LEB is designed to be used alongside the original language texts of 
Scripture, \st{these supplied words are indicated with \textit{[italics]}}. In some cases the need for such supplied words is obvious, 
but in other cases where it is less clear \st{a note has been included}.\par
Finally, the reader should remember that any Bible translation, to be useful to the person using it, must actually be read. 
We encourage every user of the LEB, whether reading it alongside the original languages text or not, to remember that once 
we understand the meaning of a biblical text we are responsible to apply it first in our own lives, and then to share it 
with those around us.\\
\textit{--The Editors}\par



\textquote[Heb 4:12]{\textit{For the word of God is living and active and sharper than any double-edged sword, and piercing as far as the division 
of soul and spirit, both joints and marrow, and able to judge the reflections and thoughts of the heart.}}

\clearpage

\subsection*{The Team: Contributors, Editors and Translators}

\begin{multicols}{3}
\begin{headings}
\vspace{-\topsep}
\begin{itemize}[rightmargin=0.125in, leftmargin=0.125in]
\fontsize{9.5pt}{11pt}\selectfont
\setlength{\parskip}{0pt} \setlength{\itemsep}{0pt plus 1pt}
\setlength{\columnsep}{0}
    \item W. Hall Harris III
    \item Elliot Ritzema
    \item Rick Brannan
    \item Douglas Mangum
    \item John Dunham
    \item Jeffrey A. Reimer
    \item Micah Wierenga
    \item W. Hall Harris III
    \item Michael S. Heiser
    \item Jeremy Penner
    \item David M. Fouts
    \item Eugene E. Carpenter
    \item Gordon H. Johnston
    \item H. Daniel Zacharias
    \item William D. Barrick
    \item Michael A. Grisanti
    \item Chip McDaniel
    \item Israel Loken
    \item Ken M. Penner
    \item Dorian G. Coover-Cox
    \item Amy L. Pfeister
    \vfill
\end{itemize}

\vspace{-\topsep}
\end{headings}
\end{multicols}

\subsection*{License}
You can give away the Lexham English Bible, but you can't sell it on its own.\\
If the LEB comprises less than 25\% of the 
content of a larger work, you can sell it as part of that work.
If you give away the LEB for use with a commercial product, or sell a work containing more than 1,000 verses from the LEB, 
you must annually report the number of units sold, distributed, and/or downloaded.
You must always attribute quotations of the LEB.
If you quote less than 100 verses of the LEB in a single work you can attribute it by simply adding (LEB) after the quotation. 
Longer quotations, or use of 100 or more verses in a single work, must be accompanied by the following statement:
Scripture quotations marked (LEB) are from the Lexham English Bible. Copyright 2012 Logos Bible Software. Lexham is a 
registered trademark of Logos Bible Software.
In electronic use, link "LEB" and "Lexham English Bible" to http://lexhamenglishbible.com, and "Logos Bible Software" 
to http://logos.com. If all quotations are unmarked and from the LEB, you may remove "marked (LEB) are" from the statement.
In support of non-English Bible translation, non-profit organizations may use 50\% as the maximum portion the LEB may 
comprise of a work offered for sale. (This specifically allows the creation and commercial sale of diglot Bibles.)


\subsection*{Trademarks}
Lexham is a registered trademark of Logos Bible Software. You may use LEB or Lexham English Bible to refer to the 
Lexham English Bible, but may not use the Lexham trademark as any part of the name of a larger work quoting or containing it.

\vfill
\centering{%
\noindent\normalsize\textbf The Lexham English Bible, Fourth Edition, Copyright 2010, 2012\\
Logos Bible Software, 1313 Commercial St., Bellingham, WA 98225\\
\url{http://www.logos.com}{http://www.logos.com}
}


\endgroup



\clearpage
\twocolumn
\justify
\thispagestyle{empty}
\mbox{}








