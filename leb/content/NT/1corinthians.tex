\biblebook{1 Corinthians}

\begin{biblechapter} % 1 Corinthians 1
\verseWithHeading{Greeting} Paul, called to be an apostle of Christ Jesus through the will of God, and Sosthenes our brother,
\verse to the church of God sanctified in Christ Jesus that is in Corinth,\lebnote{Some manuscripts have “to the church of God that is in Corinth, sanctified in Christ Jesus”} called to be saints, together with all those who call upon the name of our Lord Jesus Christ in every place, their Lord\lebnote{:|NP|:*Here “Lord” must be supplied to indicate the referent; otherwise this could be understood as “their place and ours”} and ours.
\verse Grace to you and peace from God our Father and the Lord Jesus Christ.
\verseWithHeading{Thanksgiving for the Corinthian Believers} I give thanks to my God always concerning you, because of the grace of God which was given to you in Christ Jesus,
\verse that\lebnote{:|NP|:*Or “because”; the conjunction either (1) introduces a content clause (“that”) giving the content of Paul’s thanks, or (2) introduces a causal clause (“because”) giving the reason for Paul’s thanks} in everything you were made rich in him, in all speech and all knowledge,
\verse just as the testimony about Christ has been confirmed in you,
\verse so that you do not lack in any spiritual gift as you\lebnote{:|NP|:*Here “as” is supplied as a component of the participle (“eagerly await”) which is understood as temporal} eagerly await the revelation of our Lord Jesus Christ,
\verse who will also confirm you until the end, blameless in the day of our Lord Jesus Christ.
\verse God is faithful, by whom you were called to fellowship with his Son Jesus Christ our Lord.
\verseWithHeading{Divisions in the Church at Corinth} Now I exhort you, brothers, by the name of our Lord Jesus Christ, that you all say the same thing and there not be divisions among you, and that you be made complete in the same mind and with the same purpose.
\verse For it has been made clear to me concerning you, my brothers, by \textit{Chloe’s people}\lebnote{Literally “those of Chloe”}, that there are quarrels among you.
\verse But I say this, that each of you is saying, “I am with Paul,” and “I am with Apollos,” and “I am with Cephas,” and “I am with Christ.”
\verse Has Christ been divided? Paul was not crucified for you, was he?\lebnote{:|NP|:*The negative construction in Greek anticipates a negative answer here} Or were you baptized in the name of Paul?
\verse I give thanks\lebnote{Some manuscripts have “I give thanks to God”} that I baptized none of you except Crispus and Gaius,
\verse lest anyone should say that you were baptized in my name.
\verse (Now I also baptized the household of Stephanas. Beyond that I do not know if I baptized anyone else.)
\verse For Christ did not send me to baptize, but to proclaim the gospel, not with \textit{clever speech}\lebnote{Literally “wisdom of word”}, lest the cross of Christ be emptied.
\verseWithHeading{Christ Crucified, the Power and Wisdom of God} For the message about the cross is foolishness to those who are perishing, but to us who are being saved it is the power of God.
\verse For it is written, “I will destroy the wisdom of the wise, 
and the intelligence of the intelligent I will confound.”\lebnote{:|NP|:A quotation from Isa 29:14}
\verse Where is the wise person? Where is the scribe? Where is the debater of this age? Has not God made foolish the wisdom of the world?
\verse For since, in the wisdom of God, the world through its wisdom did not know God, God was pleased through the foolishness of preaching to save those who believe.
\verse For indeed, Jews ask for sign miracles and Greeks seek wisdom,
\verse but we preach Christ crucified, to the Jews a cause for stumbling, but to the Gentiles foolishness,
\verse but to those who are called, both Jews and Greeks, Christ is the power of God and the wisdom of God.
\verse For the foolishness of God is wiser than human wisdom,\lebnote{Literally “man”} and the weakness of God is stronger than human strength.\lebnote{Literally “man”}
\verseWithHeading{Boast in the Lord} For consider your calling, brothers, that not many were wise according to human standards,\lebnote{Literally “according to flesh”} not many were powerful, not many were well born.
\verse But the foolish things of the world God chose in order that he might put to shame the wise, and the weak things of the world God chose in order that he might put to shame the strong,
\verse and the insignificant of the world, and the despised, God chose, the things that are not, in order that he might abolish the things that are,
\verse so that all flesh may not boast before God.
\verse But from him you are in Christ Jesus, who became wisdom to us from God, and righteousness and sanctification and redemption,
\verse so that, just as it is written, “The one who boasts, let him boast in the Lord.”\lebnote{:|NP|:A quotation from Jer 9:24}
\end{biblechapter}

\begin{biblechapter} % 1 Corinthians 2
\verseWithHeading{Paul’s Approach to Ministry in Corinth} And I, when I\lebnote{:|NP|:*Here “when” is supplied as a component of the participle (“came”) which is understood as temporal} came to you, brothers, did not come with superiority of speech or of wisdom, proclaiming to you the testimony\lebnote{Some manuscripts have “mystery”} of God.
\verse For I decided not to know anything among you except Jesus Christ and him crucified.
\verse And I came to you in weakness and in fear and with much trembling,
\verse and my speech and my preaching were not with the persuasiveness\lebnote{Some manuscripts have “with persuasive words”} of wisdom, but with a demonstration of the Spirit and power,
\verse in order that your faith would not be in the wisdom of men, but in the power of God.
\verseWithHeading{The Wisdom Revealed by the Spirit} Now we do speak wisdom among the mature, but wisdom not of this age or of the rulers of this age, who are perishing,
\verse but we speak the hidden wisdom of God in a mystery, which God predestined before the ages for our glory,
\verse which none of the rulers of this age knew. For if they had known it, they would not have crucified the Lord of glory.
\verse But just as it is written, “Things which eye has not seen and ear has not heard, 
and have not entered into the heart of man, 
all that\lebnote{Some manuscripts have “which”} God has prepared for those who love him.”\lebnote{:|NP|:A quotation from Isa 64:4}
\verse For\lebnote{Some manuscripts have “But to us God has revealed them”} to us God has revealed them through the Spirit. For the Spirit searches all things, even the depths of God.
\verse For who among men knows the things of a man, except the spirit of the man that is in him? Thus also no one knows the things of God except the Spirit of God.
\verse Now we have received not the spirit of the world, but the Spirit who is from God, in order that we may know the things freely given to us by God,
\verse things which we also speak, not in words taught by human wisdom, but in words taught by the Spirit, explaining spiritual things to spiritual people.\lebnote{Or “in spiritual words”}
\verse But the natural man does not accept the things of the Spirit of God, for they are foolishness to him, and he is not able to understand them, because they are spiritually discerned.
\verse Now the spiritual person discerns all things, but he himself is judged by no one.
\verse “For who has known the mind of the Lord; who has advised him?”\lebnote{:|NP|:A quotation from Isa 40:13} But we have the mind of Christ.
\end{biblechapter}

\begin{biblechapter} % 1 Corinthians 3
\verseWithHeading{Divisiveness and Immaturity} And I, brothers, was not able to speak to you as to spiritual people, but as to fleshly people, as to infants in Christ.
\verse I gave you milk to drink, not solid food, for you were not yet able to eat it. But now you are still not able,
\verse for you are still fleshly. For where there is jealousy and strife among you, are you not fleshly, and do you not live like unregenerate people?\lebnote{That is, like people who do not possess the Spirit at all}
\verse For whenever anyone says, “I am with Paul,” and another, “I am with Apollos,” are you not merely human?
\verse Therefore, what is Apollos and what is Paul? Servants through whom you believed, and to each as the Lord gave.
\verse I planted, Apollos watered, but God was causing it to grow.
\verse So then, neither the one who plants nor the one who waters is anything, but God who is causing it to grow.
\verse Now the one who plants and the one who waters are one, but each one will receive his own reward according to his own labor.
\verse For we are God’s fellow workers; you are God’s field, God’s building.
\verse According to the grace of God given to me, like a skilled master builder I laid a foundation, and another is building upon it. But each one must direct his attention to how he is building upon it.
\verse For no one is able to lay another foundation than the one which is laid, which is Jesus Christ.
\verse Now if anyone builds upon the foundation with gold, silver, precious stones, wood, grass, straw,
\verse the work of each one will become evident. For the day will reveal it, because it will be revealed with fire, and the fire itself will test the work of each one, of what sort it is.
\verse If anyone’s work that he has built upon it remains, he will receive a reward.
\verse If anyone’s work is burned up, he will suffer loss, but he himself will be saved, but so as through fire.
\verse Do you not know that you are God’s temple and the Spirit of God dwells in you?
\verse If anyone destroys God’s temple, God will destroy this one. For God’s temple is holy, which you are.
\verse Let no one deceive himself. If anyone thinks himself to be wise among you in this age, let him become a fool, in order that he may become wise.
\verse For the wisdom of this world is foolishness with God, for it is written, “The one who catches the wise in their craftiness,”\lebnote{:|NP|:A quotation from Job 5:13}
\verse and again, “The Lord knows the thoughts of the wise, that they are futile.”\lebnote{:|NP|:A quotation from Ps 94:11}
\verse So then, let no one boast in people. For all things are yours,
\verse whether Paul or Apollos or Cephas or the world or life or death or things present or things to come, all things are yours,
\verse and you are Christ’s, and Christ is God’s.
\end{biblechapter}

\begin{biblechapter} % 1 Corinthians 4
\verseWithHeading{Christ’s Servant, God’s Steward} Thus let a person consider us as servants of Christ and stewards of God’s mysteries.
\verse In this case, moreover, it is sought in stewards that one be found faithful.
\verse But to me it is a very little matter that I be judged by you or by a human court,\lebnote{Literally “day”} but I do not even judge myself.
\verse For I am conscious of nothing against myself, but not by this am I vindicated. But the one who judges me is the Lord.
\verse Therefore do not judge anything before the time, until the Lord should come, who will both enlighten the hidden things of darkness and will reveal the counsels of hearts, and then praise will come to each one from God.
\verseWithHeading{The Apostles’ Humility} Now I have applied these things, brothers, to myself and Apollos for your sake, in order that in us you may learn not to go beyond what is written, lest someone be inflated with pride on behalf of one person against the other.
\verse For who concedes you superiority? And what do you have that you did not receive? But if indeed you received it, why do you boast as if you\lebnote{:|NP|:*Here “if” is supplied as a component of the participle (“receive”) which is understood as conditional} did not receive it?
\verse Already you are satiated! Already you are rich! Apart from us you reign as kings! And would that indeed you reigned as kings, in order that we also might reign as kings with you!
\verse For, I think, God has exhibited us apostles last of all, as condemned to death, because we have become a spectacle to the world and to angels and to people.
\verse We are fools for the sake of Christ, but you are prudent in Christ! We are weak, but you are strong! You are honored, but we are dishonored!
\verse Until the present hour we are both hungry and thirsty and poorly clothed and roughly treated and homeless,
\verse and we toil, working with our own hands. When we are\lebnote{:|NP|:*Here “when” is supplied as a component of the participle (“reviled”) which is understood as temporal} reviled, we bless; when we are\lebnote{:|NP|:*Here “when” is supplied as a component of the participle (“persecuted”) which is understood as temporal} persecuted, we endure;
\verse when we are\lebnote{:|NP|:*Here “when” is supplied as a component of the participle (“slandered”) which is understood as temporal} slandered, we encourage. We have become like the refuse of the world, the offscouring of all things, until now.
\verseWithHeading{Paul’s Concern for the Corinthian Believers} I am not writing these things to shame you, but admonishing you as my dear children.
\verse For if you have ten thousand guardians in Christ, yet you do not have many fathers, for in Christ Jesus I fathered you through the gospel.
\verse Therefore I exhort you, become imitators of me.
\verse Because of this, I have sent to you Timothy, who is my dear and faithful child in the Lord, who will remind you of my ways in Christ Jesus, just as I teach everywhere in every church.
\verse But some have become arrogant, as if I were not coming to you.
\verse But I am coming to you soon, if the Lord wills, and I will know not the talk of the ones who have become arrogant, but the power.
\verse For the kingdom of God is not with talk, but with power.
\verse What do you want? Shall I come to you with a rod, or with love and a spirit of gentleness?
\end{biblechapter}

\begin{biblechapter} % 1 Corinthians 5
\verseWithHeading{Immoral Behavior and Church Discipline} It is reported everywhere that there is sexual immorality among you, and sexual immorality of such a kind which does not even exist among the Gentiles, so that someone has the wife of his father.
\verse And you are inflated with pride, and should you not rather have mourned, so that the one who has done this deed would be removed from your midst?
\verse For although I\lebnote{:|NP|:*Here “although” is supplied as a component of the participle (“am absent”) which is understood as concessive} am absent in body but present in spirit, I have already passed judgment on the one who has done this in this way, as if I\lebnote{:|NP|:*Here “if” is supplied as a component of the participle (“were present”) which is understood as conditional} were present.
\verse In the name of our Lord Jesus, when\lebnote{:|NP|:*Here “when” is supplied as a component of the temporal genitive absolute participle (“are assembled”)} you are assembled, and my spirit, together with the power of our Lord Jesus,
\verse I have decided\lebnote{:|NP|:*The words “I have decided” are implied from the statement “I have already passed judgment” in v. 3} to hand over such a person to Satan for the destruction of the flesh, in order that his spirit may be saved in the day of the Lord.
\verse Your boasting is not good. Do you not know that a little leaven leavens the whole batch of dough?
\verse Clean out the old leaven in order that you may be a new batch of dough, just as you are unleavened. For Christ our Passover has been sacrificed.
\verse So then, let us celebrate the feast, not with the old leaven or with the leaven of wickedness and sinfulness, but with the unleavened bread of sincerity and truth.
\verse I wrote to you in the letter not to associate with sexually immoral people.
\verse By no means did I mean the sexually immoral people of this world or the greedy people and swindlers or idolaters, since then you would have to depart out of the world.
\verse But now I have written to you not to associate with any so-called brother, if he is a sexually immoral person or a greedy person or an idolater or an abusive person or a drunkard or a swindler—with such a person not even to eat.
\verse For what is it to me to judge those outside? Should you not judge those inside?
\verse But those outside God will judge. Remove the evil person \textit{from among yourselves}\lebnote{Literally “from you of them”}.
\end{biblechapter}

\begin{biblechapter} % 1 Corinthians 6
\verseWithHeading{Lawsuits between Believers} Does anyone among you, if he\lebnote{:|NP|:*Here “if” is supplied as a component of the participle (“has”) which is understood as conditional} has a matter against someone else, dare to go to court before the unrighteous, and not before the saints?
\verse Or do you not know that the saints will judge the world? And if by you the world is judged, are you unworthy of the most insignificant courts?
\verse Do you not know that we will judge angels, not to mention ordinary matters?
\verse Therefore, if you have courts with regard to ordinary matters, do you seat\lebnote{Or “seat” (imperative); the Greek verb can be either indicative mood or imperative mood by its form} these despised people in the church?
\verse I say this to your shame. So is there not anyone wise among you who will be able to render a decision between his brothers?
\verse But brother goes to court with brother, and this before unbelievers!
\verse Therefore it is already completely a loss for you that you have lawsuits with one another. Why not rather be wronged? Why not rather be defrauded?
\verse But you wrong and defraud, and do this to brothers!
\verse Or do you not know that the unrighteous will not inherit the kingdom of God? Do not be deceived! Neither sexually immoral people, nor idolaters, nor adulterers, nor passive homosexual partners, nor dominant homosexual partners,
\verse nor thieves, nor greedy persons, not drunkards, not abusive persons, not swindlers will inherit the kingdom of God.
\verse And some of you were these things, but you were washed, but you were sanctified, but you were justified in the name of the Lord Jesus\lebnote{Some manuscripts have “of the Lord Jesus Christ”} and by the Spirit of our God.
\verseWithHeading{Avoid Sexual Immorality} All things are permitted for me, but not all things are profitable. All things are permitted for me, but I will not be controlled by anything.
\verse Food is for the stomach, and the stomach for food, but God will abolish \textit{both of them}\lebnote{Literally “both this and these”}. Now the body is not for sexual immorality, but for the Lord, and the Lord for the body.
\verse And God both raised up the Lord and will raise us up by his power.
\verse Do you not know that your bodies are members of Christ? Therefore, shall I take away the members of Christ and make them members of a prostitute? May it never be!
\verse Or do you not know that the one who joins himself to a prostitute is one body with her? For it says, “The two will become one flesh.”\lebnote{:|NP|:A quotation from Gen 2:24}
\verse But the one who joins himself to the Lord is one spirit with him.
\verse Flee sexual immorality. Every sin that a person commits is outside his body, but the one who commits sexual immorality sins against his own body.
\verse Or do you not know that your body is the temple of the Holy Spirit who is in you, whom you have from God, and you are not your own?
\verse For you were bought at a price; therefore glorify God with your body.
\end{biblechapter}

\begin{biblechapter} % 1 Corinthians 7
\verseWithHeading{Concerning Christian Marriage} Now concerning the things about which you wrote: “It is good for a man not to touch\lebnote{I.e., in a sexual sense} a woman.”
\verse But because of sexual immorality, let each man have\lebnote{I.e., in the sense of “have sexual relations with”} his own wife and let each woman have her own husband.
\verse The husband must fulfill his obligation to his wife, and likewise also the wife to her husband.
\verse The wife does not have authority over her own body, but her husband does. And likewise also the husband does not have authority over his own body, but his wife does.
\verse Do not defraud one another, except perhaps by agreement, for a time, in order that you may devote yourselves to prayer, and then you should be \textit{together}\lebnote{Literally “at the same”} again, lest Satan tempt you because of your lack of self control.
\verse But I say this as a concession, not as a command.
\verse I wish all people could be like myself, but each one has his own gift from God, one in this way and another in that way.
\verse Now I say to the unmarried and to the widows: It is good for them if they remain as I am.
\verse But if they cannot control themselves, they should marry, for it is better to marry than to burn with sexual desire.
\verse To the married I command—not I, but the Lord—a wife must not separate from her husband.
\verse But if indeed she does separate, she must remain unmarried or be reconciled to her husband. And a husband must not divorce his wife.
\verse Now to the rest I say—not the Lord—if any brother has an unbelieving wife and she consents to live with him, he must not divorce her.
\verse And if any wife has an unbelieving husband and he consents to live with her, she must not divorce her husband.
\verse For the unbelieving husband is sanctified by his wife, and the unbelieving wife is sanctified by the brother, since otherwise your children are unclean, but now they are holy.
\verse But if the unbeliever leaves, let him leave. The brother or the sister is not bound in such cases. But God has called us\lebnote{Some manuscripts have “you” (plural)} in peace.
\verse For how do you know, wife, whether you will save your husband? Or how do you know, husband, whether you will save your wife?
\verse But to each one as the Lord has apportioned. As God has called each one, thus let him live—and thus I order in all the churches.
\verse Was anyone called after\lebnote{:|NP|:*Here “after” is supplied as a component of the participle (“being circumcised”) which is understood as temporal} being circumcised? He must not undo his circumcision. Was anyone called in uncircumcision? He must not become circumcised.
\verse Circumcision is nothing and uncircumcision is nothing, but the keeping of the commandments of God.
\verse Each one in the calling in which he was called—in this he should remain.
\verse Were you called while a slave? Do not let it be a concern to you. But if indeed you are able to become free, rather make use of it.
\verse For the one who is called in the Lord while a slave is the Lord’s freedperson. Likewise the one who is called while free is a slave of Christ.
\verse You were bought at a price; do not become slaves of men.
\verse Each one in the situation in which he was called, brothers—in this he should remain with God.
\verseWithHeading{Concerning the Unmarried} Now concerning virgins I do not have a command from the Lord, but I am giving an opinion as one shown mercy by the Lord to be trustworthy.
\verse Therefore, I consider this to be good because of the impending distress, that it is good for a man to be thus.
\verse Are you bound to a wife? Do not seek release. Are you free from a wife? Do not seek a wife.
\verse But if you marry, you have not sinned, and if the virgin marries, she has not sinned. But such people will have affliction in the flesh, and I would spare you.
\verse But I say this, brothers: the time is shortened, that from now on even those who have wives should be as if they do not have wives,
\verse and those who weep as if they do not weep, and those who rejoice as if they do not rejoice, and those who buy as if they do not possess,
\verse and those who make use of the world as if they do not make full use of it. For the present form of this world is passing away.
\verse But I want you to be free from care. The unmarried person cares for the things of the Lord, how he may please the Lord.
\verse But the one who is married cares for the things of the world, how he may please his wife,
\verse and he is divided. And the unmarried woman or the virgin cares for the things of the Lord, in order that she may be holy both in body and in spirit. But the married woman cares for the things of the world, how she may please her husband.
\verse Now I am saying this for your own benefit, not that I may put a restriction on you, but to promote appropriate and devoted service to the Lord without distraction.
\verse But if anyone thinks he is behaving dishonorably concerning his virgin, if she is past her prime\lebnote{Or “if his passions are strong” (it is not clear in context whether this term refers to the man or to the woman)} and it ought to be thus, let him do what he wishes. He does not sin. Let them marry.
\verse But he who stands firm in his heart, not having necessity, but has authority concerning his own will, and has decided this in his own heart, to keep his own virgin, he will do well.
\verse So then, the one who marries\lebnote{Or perhaps “the one who gives in marriage”} his own virgin does well, and the one who does not marry her will do better.
\verse A wife is bound for as long a time as her husband lives. But if her husband \textit{dies}\lebnote{Literally “falls asleep”}, she is free to marry whomever she wishes, only in the Lord.
\verse But she is happier if she remains thus, according to my opinion—and I think I have the Spirit of God.
\end{biblechapter}

\begin{biblechapter} % 1 Corinthians 8
\verseWithHeading{Concerning Food Sacrificed to Idols} Now concerning food sacrificed to idols, we know that “we all have knowledge.”\lebnote{Considered by many interpreters to be a slogan used by the Corinthians to justify their behavior} Knowledge puffs up, but love builds up.
\verse If anyone thinks he knows anything, he has not yet known as it is necessary to know.
\verse But if anyone loves God, this one is known by him.
\verse Therefore, concerning the eating of food sacrificed to idols, we know that “an idol is nothing in the world” and that “there is no God except one.”\lebnote{Considered by many interpreters to be slogans used by the Corinthians to justify their behavior}
\verse For even if after all there are so-called gods, whether in heaven or on earth, just as there are many gods and many lords,
\verse yet to us there is one God, the Father, 
from whom are all things, and we are for him, 
and there is one Lord, Jesus Christ, 
through whom are all things, and we are through him.
\verse But this knowledge is not in everyone. But some, being accustomed until now to the idol, eat this food as food sacrificed to idols, and their conscience, because it\lebnote{:|NP|:*Here “because” is supplied as a component of the participle (“is”) which is understood as causal} is weak, is defiled.
\verse But food does not bring us close to God. For neither if we eat do we have more, nor if we do not eat do we lack.\lebnote{Some manuscripts omit “For” and have “Neither if we do not eat do we lack, nor if we do eat do we have more”}
\verse But watch out lest somehow this right of yours becomes a cause for stumbling to the weak.
\verse For if someone should see you who has knowledge reclining for a meal in an idol’s temple, will not his conscience, because it\lebnote{:|NP|:*Here “because” is supplied as a component of the participle (“is”) which is understood as causal} is weak, be strengthened so that he eats the food sacrificed to idols?
\verse For the one who is weak—the brother for whom Christ died—is destroyed by your knowledge.
\verse Now if you\lebnote{:|NP|:*Here “if” is supplied as a component of the participle (“sin”) which is understood as conditional} sin in this way against the brothers and wound their conscience, which is weak, you sin against Christ.
\verse Therefore, if food causes my brother to sin, I will never eat meat \textit{forever}\lebnote{Literally “for the age”}, in order that I may not cause my brother to sin.
\end{biblechapter}

\begin{biblechapter} % 1 Corinthians 9
\verseWithHeading{Paul Gives Up His Rights as an Apostle} Am I not free? Am I not an apostle? Have I not seen Jesus our Lord? Are you not my work in the Lord?
\verse If to others I am not an apostle, yet indeed I am to you, for you are my seal of apostleship in the Lord.
\verse My defense to those who examine me is this:
\verse Do we not have the right to eat and drink?
\verse Do we not have the right to take along a sister as wife, like the rest of the apostles and the brothers of the Lord and Cephas?
\verse Or do only I and Barnabas not have the right \textit{to refrain from working}\lebnote{Literally “not to work”}?
\verse Who ever serves as a soldier at his own expense? Who plants a vineyard and does not eat the fruit of it? Who\lebnote{Some manuscripts have “Or who”} shepherds a flock and does not drink\lebnote{Literally “eat”} from the milk of the flock?
\verse I am not saying these things according to a human perspective. Or does the law not also say these things?
\verse For in the law of Moses it is written, “You must not muzzle an ox while it\lebnote{:|NP|:*Here “while” is supplied as a component of the participle (“threshing”) which is understood as temporal} is threshing.”\lebnote{:|NP|:A quotation from Deut 25:4} It is not about oxen God is concerned, is it?\lebnote{:|NP|:*The negative construction in Greek anticipates a negative answer here}
\verse Or doubtless does he speak \textit{for our sake}\lebnote{Literally “for the sake of us”}? For it is written \textit{for our sake}\lebnote{Literally “for the sake of us”}, because the one who plows ought to plow in hope and the one who threshes ought to do so in hope of a share.
\verse If we have sown spiritual things among you, is it too great a thing if we reap material things from you?
\verse If others share this right over you, do we not do so even more? Yet we have not made use of this right, but we endure all things, in order that we may not cause any hindrance to the gospel of Christ.
\verse Do you not know that those performing the holy services eat the things from the temple, and those attending to the altar have a share with the altar?
\verse In the same way also the Lord ordered those who proclaim the gospel to live from the gospel.
\verse But I have not made use of any of these rights. And I am not writing these things in order that it may be thus with me. For it would be better to me rather to die than for anyone to deprive me of my reason for boasting.
\verse For if I proclaim the gospel, it is not to me a reason for boasting, for necessity is imposed on me. For woe is to me if I do not proclaim the gospel.
\verse For if I do this voluntarily, I have a reward, but if I do so unwillingly, I have been entrusted with a stewardship.
\verse What then is my reward? That when I\lebnote{:|NP|:*Here “when” is supplied as a component of the participle (“proclaim the gospel”) which is understood as temporal} proclaim the gospel, I may offer the gospel free of charge, in order not to make full use of my right in the gospel.
\verse For although I\lebnote{:|NP|:*Here “although” is supplied as a component of the participle (“am”) which is understood as concessive} am free from all people, I have enslaved myself to all, in order that I may gain more.
\verse I have become like a Jew to the Jews, in order that I may gain the Jews. To those under the law I became as under the law (although I\lebnote{:|NP|:*Here “although” is supplied as a component of the participle (“am”) which is understood as concessive} myself am not under the law) in order that I may gain those under the law.
\verse To those outside the law I became as outside the law (although I\lebnote{:|NP|:*Here “although” is supplied as a component of the participle (“am”) which is understood as concessive} am not outside the law of God, but subject to the law of Christ) in order that I may gain those outside the law.
\verse To the weak I became weak, in order that I may gain the weak. I have become all things to all people, in order that by all means I may save some.
\verse I do all this for the sake of the gospel, in order that I may become a participant with it.
\verse Do you not know that those who run in the stadium all run, but one receives the prize? Run in such a way that you may win.
\verse And everyone who competes exercises self-control in all things. Thus those do so in order that they may receive a perishable crown, but we an imperishable one.
\verse Therefore I run in this way, not as running aimlessly; I box in this way, not as beating the air.
\verse But I discipline my body and subjugate it, lest somehow after\lebnote{:|NP|:*Here “after” is supplied as a component of the participle (“preaching”) which is understood as temporal} preaching to others, I myself should become disqualified.
\end{biblechapter}

\begin{biblechapter} % 1 Corinthians 10
\verseWithHeading{A History Lesson from Israel} For I do not want you to be ignorant, brothers, that our fathers were all under the cloud and all went through the sea,
\verse and all were baptized into Moses in the cloud and in the sea,
\verse and all ate the same spiritual food,
\verse and all drank the same spiritual drink. For they drank from the spiritual rock that followed them, and the rock was Christ.
\verse But God was not pleased with the majority of them, for they were struck down in the desert.
\verse Now these things happened as examples for us, so that we should not be desirers of evil things, just as those also desired them,
\verse and not become idolaters, as some of them did, just as it is written, “The people sat down to eat and drink, and stood up to play,”\lebnote{:|NP|:A quotation from Exod 32:6}
\verse nor commit sexual immorality, as some of them committed sexual immorality, and twenty-three thousand fell in one day,
\verse nor put Christ to the test, as some of them tested him, and were destroyed by snakes,
\verse nor grumble, just as some of them grumbled, and were destroyed by the destroyer.
\verse Now these things happened to those people as an example, but are written for our instruction, on whom the ends of the ages have come.
\verse Therefore, the one who thinks that he stands must watch out lest he fall.
\verse Temptation has not come upon you except what is common to humanity. But God is faithful, who will not permit you to be tempted beyond what you are able, but will also make a way out together with the temptation, so that you may be able to endure it.
\verseWithHeading{Warning Against Idolatry} Therefore, my dear friends, flee from idolatry.
\verse I am speaking as to sensible people; you judge what I am saying.
\verse The cup of blessing which we bless, is it not a participation in the blood of Christ? The bread which we break, is it not a participation in the body of Christ?
\verse Because there is one bread, we who are many are one body, for we all share from the one bread.
\verse Consider Israel according to the flesh: are not the ones who eat the sacrifices sharers in the altar?
\verse Therefore, what am I saying? That food sacrificed to idols is anything, or that an idol is anything?
\verse No, but that the things which they sacrifice, they sacrifice to demons and not to God, and I do not want you to become sharers with demons.
\verse You are not able to drink the cup of the Lord and the cup of demons. You are not able to share the table of the Lord and the table of demons.
\verse Or are we attempting to provoke the Lord to jealousy? We are not stronger than he is, are we?\lebnote{:|NP|:*The negative construction in Greek anticipates a negative answer here}
\verseWithHeading{Freedom in Christ} All things are permitted, but not all things are profitable. All things are permitted, but not all things build up.
\verse Let no one seek \textit{his own good}\lebnote{Literally “the - of himself”; “good” is implied} but the good of the other.
\verse Eat everything that is sold in the meat market, \textit{asking no questions}\lebnote{Literally “questioning nothing”} for the sake of the conscience,
\verse for “the earth is the Lord’s, and its fullness.”\lebnote{:|NP|:A quotation from Ps 24:1, and an allusion to Ps 50:12; 89:11}
\verse If any of the unbelievers invites you, and you want to go, eat everything that is set before you, \textit{asking no questions}\lebnote{Literally “questioning nothing”} for the sake of the conscience.
\verse But if someone says to you, “This is offered to idols,” do not eat it, for the sake of that one who informed you and the conscience.
\verse Now I am not speaking about your own conscience, but the conscience of the other person. For why is my freedom judged by another’s conscience?
\verse If I partake with thankfulness, why am I slandered concerning that for which I give thanks?
\verse Therefore, whether you eat or you drink or whatever you do, do all things for the glory of God.
\verse \textit{Give no offense}\lebnote{Literally “be blameless”} both to Jews and to Greeks and to the church of God,
\verse just as I also please all people in all things, not seeking my own benefit, but the benefit of the many, in order that they may be saved.
\end{biblechapter}

\begin{biblechapter} % 1 Corinthians 11
\verseWithHeading{Concerning Head Coverings in Worship} Become imitators of me, just as I also am of Christ.
\verse Now I praise you that you remember me in all things, and just as I handed over to you the traditions, you hold fast to them.
\verse But I want you to know that Christ is the head of every man, and the man is the head of the woman, and the head of Christ is God.
\verse Every man who prays or prophesies while\lebnote{:|NP|:*Here “while” is supplied as a component of the participle (“having”) which is understood as temporal} having something\lebnote{:|NP|:*Here the direct object must be supplied from context, but the exact nature of the object is not clear from the context, so “something” is used in the English translation} on his head dishonors his head,
\verse but every woman who prays or prophesies with uncovered head dishonors her head, for she is one and the same with the one whose head is shaved.
\verse For if a woman does not cover herself, let her hair be shorn off. But if it is shameful for a woman to have her head\lebnote{Literally “to be”} shorn or shaved, let her cover her head.\lebnote{Literally “herself”}
\verse For indeed a man ought not to cover his head, because he\lebnote{:|NP|:*Here “because” is supplied as a component of the participle (“is”) which is understood as causal} is the image and glory of God, but the woman is the glory of the man.
\verse For man is not from woman, but woman from man.
\verse For indeed man was not created for the sake of the woman, but woman for the sake of the man.
\verse Because of this, the woman ought to have a symbol of authority on her head, on account of the angels.
\verse Nevertheless, neither is woman anything apart from man, nor is man anything apart from woman in the Lord.
\verse For just as the woman is from the man, thus also the man is through the woman. But all things are from God.
\verse You judge \textit{for yourselves}\lebnote{Literally “in you yourselves”}: is it fitting for a woman to pray to God with her head\lebnote{:|NP|:*Here the words “with her head” have been supplied as a necessary clarification} uncovered?
\verse And does not nature itself teach you that a man, if he wears long hair, it is a dishonor to him?
\verse But a woman, if she wears long hair, it is her glory, because her hair is given\lebnote{Some manuscripts have “is given to her”} for a covering.
\verse But if anyone is disposed to be contentious, we have no such custom, nor do the churches of God.
\verseWithHeading{Improper Conduct at the Lord’s Supper} But in giving this instruction I do not praise you, because you come together not for the better but for the worse.
\verse For in the first place, when you\lebnote{:|NP|:*Here “when” is supplied as a component of the temporal genitive absolute participle (“come together”)} come together as a church, I hear there are divisions among you, and in part I believe it.
\verse For indeed it is necessary that there be factions among you, in order that those who are approved may become evident among you.
\verse Therefore, when\lebnote{:|NP|:*Here “when” is supplied as a component of the temporal genitive absolute participle (“come together”)} you come together in the same place, it is not to eat the Lord’s supper.
\verse For when you\lebnote{:|NP|:*Here “when” is supplied as a component of the temporal infinitive (“eat”)} eat it, each one of you goes ahead to take his own supper, and one is hungry and another is drunk.
\verse For do you not have houses for eating and drinking? Or do you despise the church of God and humiliate those who do not have anything? What shall I say to you? Shall I praise you? In this I will not praise you!
\verse For I received from the Lord what I also passed on to you, that the Lord Jesus, on the night in which he was betrayed, took bread,
\verse and after he\lebnote{:|NP|:*Here “after” is supplied as a component of the participle (“had given thanks”) which is understood as temporal} had given thanks, he broke it and said, “This is my body which is for you. Do this in remembrance of me.”
\verse Likewise also the cup, after they had eaten, saying, “This cup is the new covenant in my blood. Do this, as often as you drink it, in remembrance of me.”
\verse For as often as you eat this bread and drink this cup, you proclaim the Lord’s death until he comes.
\verse So then, whoever eats the bread or drinks the cup of the Lord in an unworthy manner will be guilty of the body and the blood of the Lord.
\verse But let a person examine himself, and in this way let him eat from the bread and let him drink from the cup.
\verse For the one who eats and drinks, if he\lebnote{:|NP|:*Here “if” is supplied as a component of the participle (“recognize”) which is understood as conditional} does not recognize the body, eats and drinks judgment against himself.
\verse Because of this, many are weak and sick among you, and quite a few \textit{have died}\lebnote{Literally “have fallen asleep”}.
\verse But if we were evaluating ourselves, we would not be judged.
\verse But if we\lebnote{:|NP|:*Here “if” is supplied as a component of the participle (“are judged”) which is understood as conditional} are judged by the Lord, we are being disciplined, in order that we will not be condemned with the world.
\verse So then, my brothers, when you\lebnote{:|NP|:*Here “when” is supplied as a component of the participle (“come together”) which is understood as temporal} come together in order to eat the Lord’s supper,\lebnote{:|NP|:*Here the direct object (“the Lord’s supper”) is supplied from context in the English translation} wait for one another.
\verse If anyone is hungry, let him eat at home, lest you come together for judgment. And I will give directions about the remaining matters whenever I come.
\end{biblechapter}

\begin{biblechapter} % 1 Corinthians 12
\verseWithHeading{Varieties of Spiritual Gifts} Now concerning spiritual gifts,\lebnote{Or “spiritual things”; possibly “those who possess spiritual gifts”} brothers, I do not want you to be ignorant.
\verse You know that when you were pagans, you were led astray to the speechless idols, however you were led.
\verse Therefore I make known to you that no one speaking by the Spirit of God says, “Jesus is accursed,” and no one is able to say “Jesus is Lord” except by the Holy Spirit.
\verse Now there are varieties of gifts, but the same Spirit,
\verse and there are varieties of ministries, and the same Lord,
\verse and there are varieties of activities, but the same God, who works all things in all people.
\verse But to each one is given the manifestation of the Spirit for what is beneficial to all.
\verse For to one is given a word of wisdom through the Spirit, and to another a word of knowledge by the same Spirit,
\verse to another faith by the same Spirit, to another\lebnote{Some manuscripts have “and to another”} gifts of healing by the one Spirit,
\verse to another\lebnote{Some manuscripts have “and to another”} \textit{miraculous powers}\lebnote{Literally “activities of power”}, to another\lebnote{Some manuscripts have “and to another”} prophecy, to another\lebnote{Some manuscripts have “and to another”} distinguishing of spirits, to another kinds of tongues, to another\lebnote{Some manuscripts have “and to another”} interpretation of tongues.
\verse But in all these things one and the same Spirit is at work, distributing to each one individually just as he wishes.
\verseWithHeading{Unity in the Midst of Diversity} For just as the body is one and has many members, but all the members of the body, although they\lebnote{:|NP|:*Here “although” is supplied as a component of the participle (“are”) which is understood as concessive} are many, are one body, thus also Christ.
\verse For by\lebnote{Or “in”} one Spirit we were all baptized into one body, whether Jews or Greeks, whether slaves or free persons, and all were made to drink one Spirit.
\verse For the body is not one member, but many.
\verse If the foot should say, “Because I am not a hand, I am not a part of the body,” not because of this is it not a part of the body.
\verse And if the ear should say, “Because I am not an eye, I am not a part of the body, not because of this is it not a part of the body.
\verse If the whole body were an eye, where would the hearing be? If the whole were hearing, where would the sense of smell be?
\verse But now God has placed the members, each one of them, in the body just as he wanted.
\verse And if they all were one member, where would the body be?
\verse But now there are many members, but one body.
\verse Now the eye is not able to say to the hand, “I do not have need of you,” or again, the head to the feet, “I do not have need of you.”
\verse But by much more the members of the body which are thought to be weaker are necessary,
\verse and the parts of the body which we think to be less honorable, these we clothe with more abundant honor, and our unpresentable parts come to have more abundant presentability,
\verse but our presentable parts do not have need of this. Yet God composed the body by giving more abundant honor to the part which lacked it,
\verse in order that there not be a division in the body, but the members would have the same concern for one another.
\verse And if one member suffers, all the members suffer together; if a member\lebnote{Some manuscripts have “one member”} is honored, all the members rejoice with it.
\verse Now you are the body of Christ, and members \textit{of it individually}\lebnote{Literally “by part”},
\verse and whom God has appointed in the church: first, apostles, second, prophets, third, teachers, then miracles, then gifts of healing, helps, administrations, kinds of tongues.
\verse Not all are apostles, are they?\lebnote{:|NP|:*The negative construction in Greek anticipates a negative answer here} Not all are prophets, are they?\lebnote{:|NP|:*The negative construction in Greek anticipates a negative answer here} Not all are teachers, are they?\lebnote{:|NP|:*The negative construction in Greek anticipates a negative answer here} Not all are workers of miracles, are they?\lebnote{:|NP|:*The negative construction in Greek anticipates a negative answer here}
\verse Not all have gifts of healing, do they?\lebnote{:|NP|:*The negative construction in Greek anticipates a negative answer here} Not all speak with tongues, do they?\lebnote{:|NP|:*The negative construction in Greek anticipates a negative answer here} Not all interpret, do they?\lebnote{:|NP|:*The negative construction in Greek anticipates a negative answer here}
\verse But strive\lebnote{Or “you are striving” (some understand the form of this verb to be indicative mood rather than imperative mood)} for the greater gifts. And I will show you a still more excellent way.
\end{biblechapter}

\begin{biblechapter} % 1 Corinthians 13
\verseWithHeading{Love, the More Excellent Way} If I speak with the tongues of men and of angels, but do not have love, I have become a ringing brass gong or a clashing cymbal.
\verse And if I have the gift of prophecy and I know all mysteries and all knowledge, and if I have all faith so that I can remove mountains, but do not have love, I am nothing.
\verse And if I parcel out all my possessions, and if I hand over my body in order that I will be burned,\lebnote{Some manuscripts have “in order that I may boast”} but do not have love, it benefits me nothing.
\verse Love is patient, love is kind, love is not jealous, it does not boast, it does not become conceited,
\verse it does not behave dishonorably, it \textit{is not selfish}\lebnote{Literally “does not seek the things of itself”}, it does not become angry, it does not keep a record of wrongs,
\verse it does not rejoice at unrighteousness, but rejoices with the truth,
\verse bears all things, believes all things, hopes all things, endures all things.
\verse Love never ends. But if there are prophecies, they will pass away. If there are tongues, they will cease. If there is knowledge, it will pass away.
\verse For we know in part and we prophesy in part,
\verse but whenever the perfect comes, the partial will pass away.
\verse When I was a child, I spoke like a child, I thought like a child, I reasoned like a child. When I became a man, I set aside the things of a child.
\verse For now we see through a mirror \textit{indirectly}\lebnote{Literally “in an indirect image”}, but then face to face. Now I know in part, but then I will know completely, just as I have also been completely known.
\verse And now these three things remain: faith, hope, and love. But the greatest of these is love.
\end{biblechapter}

\begin{biblechapter} % 1 Corinthians 14
\verseWithHeading{Appropriate Use of Prophecy and Tongues} Pursue love, and strive for spiritual gifts, but especially that you may prophesy.
\verse For the one who speaks in a tongue does not speak to people but to God, because no one understands, but by the Spirit\lebnote{Or “in the Spirit”; or “in his spirit”} he speaks mysteries.
\verse But the one who prophesies speaks to people edification and encouragement and consolation.
\verse The one who speaks in a tongue edifies himself, but the one who prophesies edifies the church.
\verse Now I want you all to speak with tongues, but even more that you may prophesy. The one who prophesies is greater than the one who speaks with tongues, unless he interprets, in order that the church may receive edification.
\verse But now, brothers, if I come to you speaking with tongues, how do I benefit you, unless I speak to you either with a revelation or with knowledge or with a prophecy or with a teaching?
\verse Likewise, the inanimate things which produce a sound, whether flute or lyre, if they do not produce a distinction in the tones, how will it be known what is played on the flute or on the lyre?
\verse For indeed, if the trumpet produces an indistinct sound, who will prepare for battle?
\verse And so you through the tongue, unless you produce a clear message, how will it be known what is spoken? For you will be speaking into the air.
\verse There are probably so many kinds of languages in the world, and none without meaning.
\verse Therefore, if I do not know the meaning of the language, I will be a barbarian to the one who is speaking, and the one who is speaking will be a barbarian in my judgment.
\verse In this way also you, since you are zealous of spiritual gifts, seek for the edification of the church, in order that you may abound.
\verse Therefore the one who speaks in a tongue must pray that he may interpret.
\verse For if I pray in a tongue, my spirit prays but my mind is unproductive.
\verse \textit{Therefore what should I do}\lebnote{Literally “what therefore is it”}? I will pray with my spirit, but I will also pray with my mind. I will sing praise with my spirit, but I will also sing praise with my mind.
\verse For otherwise, if you praise in your spirit, how will the one who fills the place of the outsider say the “amen” at your thanksgiving, because he does not know what you are saying?
\verse For indeed you are giving thanks well, but the other person is not edified.
\verse I give thanks to God that I speak with tongues more than all of you,
\verse but in the church I prefer to speak five words with my mind, in order that I may instruct other people, than ten thousand words in a tongue.
\verse Brothers, do not become children in your understanding, but with respect to wickedness be as a child, and in your understanding be mature.
\verse In the law it is written: “By those who speak a foreign language 
and by the lips of others 
I will speak to this people, 
and not even in this way will they obey me,”\lebnote{:|NP|:A quotation from Isa 28:11–12}
\verse says the Lord.
\verse So then, tongues are for a sign not to those who believe, but to unbelievers, but prophecy is not for unbelievers, but for those who believe.
\verse Therefore, if the whole church comes together at the same time and all speak with tongues, and outsiders or unbelievers enter, will they not say that you are out of your minds?
\verse But if all prophesy, and some unbeliever or outsider enters, he is convicted by all, he is judged by all,
\verseWithHeading{Specific Instructions for Orderly Worship Services} \textit{Therefore what should you do}\lebnote{Literally “what therefore is it”}, brothers? Whenever you come together, each one of you has a psalm, has a teaching, has a revelation, has a tongue, has an interpretation. All things must be done for edification.
\verse If anyone speaks in a tongue, it must be on one occasion two or at most three, and \textit{one after the other}\lebnote{Literally “in turn”}, and one must interpret.
\verse But if there is no interpreter, he must be silent in the church, but let him speak to himself and to God.
\verse Let two or three prophets speak, and the others evaluate.
\verse And if something is revealed to another who is seated, the first must be silent.
\verse For you are all able to prophesy \textit{in turn}\lebnote{Literally “one at a time}, in order that all may learn and all may be encouraged,
\verse and the spirits of prophets are subject to prophets.
\verse For God is not a God of disorder but of peace.
\verse As in all the churches of the saints,
\verse the women must be silent in the churches, for it is not permitted for them to speak, but they must be in submission, just as the law also says.
\verse But if they want to learn something, let them ask their own husbands at home, for it is shameful for a woman to speak in church.
\verse If anyone thinks he is a prophet or spiritual,\lebnote{Or “or one who possesses the Spirit”} he should recognize that the things which I am writing to you are of the Lord.\lebnote{Some manuscripts have “are a command of the Lord”}
\verse But if anyone ignores this, he is ignored.\lebnote{Or “if anyone is ignorant, let him be ignorant”}
\verse So then, my brothers, desire to prophesy, and do not prevent speaking with tongues.
\verse But let all things be done decently and according to proper procedure.
\end{biblechapter}

\begin{biblechapter} % 1 Corinthians 15
\verseWithHeading{Paul’s Gospel and the Resurrection of Christ} Now I make known to you, brothers, the gospel which I proclaimed to you, which you have also received, in which you also stand,
\verse by which you are also being saved, if you hold fast to the message I proclaimed to you, unless you believed to no purpose.
\verse For I passed on to you \textit{as of first importance}\lebnote{Literally “among the first things”} what I also received, that Christ died for our sins according to the scriptures,
\verse and that he was buried, and that he was raised up on the third day according to the scriptures,
\verse and that he appeared to Cephas, then to the twelve,
\verse then he appeared to more than five hundred brothers at once, the majority of whom remain until now, but some have fallen asleep.
\verse Then he appeared to James, then to all the apostles,
\verse and last of all, as it were to one born at the wrong time, he appeared also to me.
\verse For I am the least of the apostles, not worthy to be called an apostle, because I persecuted the church of God.
\verse But by the grace of God I am what I am, and his grace to me has not been in vain, but I labored even more than all of them, and not I, but the grace of God with me.
\verse Therefore whether I or those, in this way we preached, and in this way you believed.
\verseWithHeading{Concerning the Resurrection of the Dead} Now if Christ is preached as raised up from the dead, how do some among you say that there is no resurrection of the dead?
\verse But if there is no resurrection of the dead, Christ has not been raised either.
\verse But if Christ has not been raised, then\lebnote{Some manuscripts have “then both”} our preaching is in vain, and your faith is in vain.
\verse And also we are found to be false witnesses of God, because we testified against God that he raised Christ, whom he did not raise if after all, then, the dead are not raised.
\verse For if the dead are not raised, Christ has not been raised either.
\verse But if Christ has not been raised, your faith is empty; you are still in your sins.
\verse And as a further result, those who have fallen asleep in Christ have perished.
\verse If \textit{we have put our hope}\lebnote{Literally “we are having put our hope”} in Christ in this life only, we are of all people most pitiable.
\verse But now Christ has been raised from the dead, the first fruits of those who have fallen asleep.
\verse For since through a man came death, also through a man came the resurrection of the dead.
\verse For just as in Adam all die, so also in Christ all will be made alive.
\verse But each in his own group: Christ the first fruits, then those who are Christ’s at his coming,
\verse then the end, when he hands over the kingdom to the God and Father, when he has abolished all rule and all authority and power.
\verse For it is necessary for him to reign until he has put all his enemies under his feet.
\verse The last enemy to be abolished is death.
\verse For “he subjected all things under his feet.”\lebnote{:|NP|:A quotation from Ps 8:6} But when it says “all things” are subjected, it is clear that the one who subjected all things to him is not included.
\verse But whenever all things are subjected to him, then the Son himself will be subjected\lebnote{Some manuscripts have “also will be subjected”} to the one who subjected all things to him, in order that God may be all in all.
\verse Otherwise, why do they do it, those who are being baptized on behalf of the dead? If the dead are not raised at all, why indeed are they being baptized on behalf of them?
\verse And why are we in danger every hour?
\verse I die every day—yes indeed, by my boasting in you,\lebnote{Some manuscripts have “in you, brothers,”} which I have in Christ Jesus our Lord!
\verse If according to a human perspective I fought wild beasts at Ephesus, what benefit is it to me? If the dead are not raised, let us eat and drink, for tomorrow we die.\lebnote{An allusion to Isa 22:13; 56:12}
\verse Do not be deceived! “Bad company corrupts good morals.”\lebnote{:|NP|:A quotation from the Greek poet Menander’s comedy \textit{Thais}, 218}
\verse Sober up correctly and \textit{stop sinning}\lebnote{Literally “do not sin”}, for some have no knowledge of God—I say this to your shame.
\verseWithHeading{Questions Concerning the Resurrection Body} But someone will say, “How are the dead raised? And with what sort of body do they come?”
\verse Foolish person! What you sow does not come to life unless it dies.
\verse And what you sow is not the body which it will become, but you sow the bare seed, whether perhaps of wheat or of some of the rest.
\verse But God gives to it a body just as he wishes, and to each one of the seeds its own body.
\verse Not all flesh is the same, but there is one flesh of human beings, and another flesh of animals, and another flesh of birds, and another of fish,
\verse and heavenly bodies and earthly bodies. But the glory of the heavenly bodies is of one kind, and the glory of the earthly bodies is of another kind.
\verse There is one glory of the sun, and another glory of the moon, and another glory of the stars, for star differs from star in glory.
\verse Thus also is the resurrection of the dead. It is sown in corruption, it is raised in incorruptibility.
\verse It is sown in dishonor, it is raised in glory. It is sown in weakness, it is raised in power.
\verse It is sown a natural body, it is raised a spiritual body. If there is a natural body, there is also a spiritual body.
\verse Thus also it is written, “The first man, Adam, became a living soul”;\lebnote{:|NP|:A quotation from Gen 2:7} the last Adam became a life-giving spirit.
\verse But the spiritual is not first, but the natural; then the spiritual.
\verse The first man is from the earth, made of earth; the second man is from heaven.
\verse As the one who is made of earth, so also are those who are made of earth, and as the heavenly, so also are those who are heavenly.
\verse And just as we have borne the image of the one who is made of earth, we will also bear the image of the heavenly.
\verse But I say this, brothers, that flesh and blood is not able to inherit the kingdom of God, nor can corruption inherit incorruptibility.
\verse Behold, I tell you a mystery: we will not all fall asleep, but we will all be changed,
\verse in a moment, in the blink of an eye, at the last trumpet. For the trumpet will sound, and the dead will be raised imperishable, and we will be changed.
\verse For it is necessary for this perishable body to put on incorruptibility, and this mortal body to put on immortality.
\verse But whenever this perishable body puts on incorruptibility and this mortal body puts on immortality, then the saying that is written will take place:
\verse “Death is swallowed up in victory.
\verse Now the sting of death is sin, and the power of sin is the law.
\verse But thanks be to God, who gives us the victory through our Lord Jesus Christ!
\verse So then, my dear brothers, be steadfast, immovable, always abounding in the work of the Lord, because you\lebnote{:|NP|:*Here “because” is supplied as a component of the participle (“know”) which is understood as causal} know that your labor is not in vain in the Lord.
\end{biblechapter}

\begin{biblechapter} % 1 Corinthians 16
\verseWithHeading{Concerning the Collection for the Saints} Now concerning the collection for the saints: just as I gave directions about it to the churches of Galatia, so you do also.
\verse On the first day of the week, each one of you \textit{put aside}\lebnote{Literally “put from himself”} something, saving up \textit{to whatever extent he has prospered}\lebnote{Literally “whatever if anything he has been prospered”}, in order that whenever I come, at that time collections do not take place.
\verse And whenever I arrive, whomever you approve by letters, I will send these to take your gift to Jerusalem.
\verse And if it is worthwhile for me to go also, they will travel with me.
\verseWithHeading{Travel Plans for Paul and Associates} But I will come to you whenever I go through Macedonia (for I am going through Macedonia),
\verse and perhaps I will stay with you, or even spend the winter, so that you may send me on my way wherever I may go.
\verse For I do not want to see you now in passing, for I hope to remain some time with you, if the Lord allows it.
\verse But I will remain in Ephesus until Pentecost,
\verse for a great and effective door has opened for me, and there are many opponents.
\verse But if Timothy comes, see that he is with you without cause to fear, for he is carrying out the Lord’s work, as I also am.
\verse Therefore do not let anyone disdain him, but send him on his way in peace in order that he may come to me, for I am expecting him with the brothers.
\verse Now concerning Apollos our brother, I urged him many times that he should come to you with the brothers, and he was not at all willing that he should come now, but he will come whenever he has an opportunity.
\verseWithHeading{Concluding Exhortations} Be on the alert, stand firm in the faith, act courageously, be strong.
\verse All your actions must be done in love.
\verse Now I urge you, brothers—you know about the household of Stephanas, that they are the first fruits of Achaia, and they have devoted themselves to the ministry for the saints—
\verse that you also be subject to such people, and to all those who work together and labor.
\verse Now I rejoice over the arrival of Stephanas and Fortunatus and Achaicus, because these make up for your absence,
\verse for they have refreshed my spirit and yours. Therefore recognize such people.
\verseWithHeading{Final Greetings and Benediction} The churches of the province of Asia\lebnote{That is, the Roman province of Asia, known today as Asia Minor} greet you. Aquila and Prisca greet you in the Lord many times, together with the church in their house.
\verse All the brothers greet you. Greet one another with a holy kiss.
\verse The greeting is by my hand—Paul’s.
\verse If anyone does not love the Lord, let him be accursed. O Lord, come!\lebnote{The Aramaic expression \textit{marana tha} (“O Lord, come!”) can also be rendered \textit{maran atha} (“our Lord has come”); it is used here by Paul without explanation}
\verse The grace of the Lord Jesus be with you.
\verse My love be with all of you in Christ Jesus.
\end{biblechapter}

