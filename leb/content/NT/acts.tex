\biblebook{Acts}

\begin{biblechapter} % Acts 1
\verseWithHeading{The Preface} I produced the former account, O Theophilus, about all \textit{that}\lebnote{Literally “of what”} Jesus began to do and to teach,
\verse until the day he was taken up, after he\lebnote{:|NP|:*Here “after” is supplied as a component of the participle (“had given orders”) which is understood as temporal} had given orders through the Holy Spirit to the apostles whom he had chosen,
\verse to whom he also presented himself alive after he suffered, with many convincing proofs, appearing to them over a period of forty days and speaking the things about the kingdom of God.
\verse And while he\lebnote{:|NP|:*Here “while” is supplied as a component of the participle (“was with”) which is understood as temporal} was with\lebnote{Or “was assembling with,” or “was sharing a meal with”} them,\lebnote{:|NP|:*Here the direct object is supplied from context in the English translation} he commanded them, “Do not depart from Jerusalem, but wait for what was promised by the Father, which you heard about from me.
\verse For John baptized with water, but you will be baptized with the Holy Spirit \textit{not many days from now}.”\lebnote{Literally “not many days after these”}
\verseWithHeading{The Ascension} So when\lebnote{:|NP|:*Here “when” is supplied as a component of the participle (“had come together”) which is understood as temporal} they had come together, they began asking\lebnote{:|NP|:*The imperfect tense has been translated as ingressive here (“began asking”)} him, saying, “Lord, is it at this time you are restoring the kingdom to Israel?”
\verse But he said to them, “It is not for you to know the times or seasons that the Father has set by his own authority.
\verse But you will receive power when\lebnote{:|NP|:*Here “when” is supplied as a component of the temporal genitive absolute participle (“has come”)} the Holy Spirit has come upon you, and you will be my witnesses in Jerusalem, and in all Judea and Samaria, and to the farthest part of the earth.”
\verse And after he\lebnote{:|NP|:*Here “after” is supplied as a component of the participle (“had said”) which is understood as temporal} had said these things, while\lebnote{:|NP|:*Here “while” is supplied as a component of the temporal genitive absolute participle (“were watching”)} they were watching, he was taken up, and a cloud received him from their sight.
\verse And as they were staring into the sky while\lebnote{:|NP|:*Here “while” is supplied as a component of the temporal genitive absolute participle (“was departing”)} he was departing, behold, two men in white clothing stood by them
\verse who also said, “Men \textit{of Galilee},\lebnote{Literally “Galileans”} why do you stand there looking\lebnote{Some manuscripts have “gazing”} into the sky? This Jesus who was taken up from you into heaven like this will come back in the same way you saw him departing into heaven!”
\verseWithHeading{Matthias Chosen to Replace Judas} Then they returned to Jerusalem from the mountain that is called Olive Grove\lebnote{This is a variation of the name “Mount of Olives”} which is near Jerusalem, \textit{a Sabbath day’s journey away}.\lebnote{Literally “having a journey of a Sabbath”}
\verse And when they had entered, they went up to the upstairs room where they were staying—Peter and John and James and Andrew, Philip and Thomas, Bartholomew and Matthew, James son of Alphaeus and Simon the Zealot and Judas son of James.
\verse All these were busily engaged with one mind in prayer, together with the women and Mary the mother of Jesus and with\lebnote{Some manuscripts omit “with”} his brothers.
\verse And in those days Peter stood up in the midst of the brothers (and it was a crowd of persons of about one hundred twenty at the same place) and\lebnote{:|NP|:*Here “and” is supplied because the previous participle (“stood up”) has been translated as a finite verb} said,
\verse “Men and brothers, it was necessary that the scripture be fulfilled, which the Holy Spirit proclaimed beforehand through the mouth of David concerning Judas, who became a guide to those who arrested Jesus,
\verse because he was counted among us and received a share in this ministry.”
\verse (Now this man acquired a field for the wages of his wickedness, and falling headlong, he burst open in the middle and all his intestines spilled out.
\verse And it became known to all who live in Jerusalem, so that that field was called in their own language\lebnote{That is, Aramaic} “Akeldama,” that is, “Field of Blood.”)
\verse “For it is written in the book of Psalms, ‘Let his residence become deserted, 
and let there be no one to live in it,’\lebnote{:|NP|:A quotation from Ps 69:25} and, ‘Let another person take his position.’\lebnote{:|NP|:A quotation from Ps 109:8}
\verse Therefore it is necessary for one of the men who have accompanied us during all the time in which the Lord Jesus went in and went out among us,
\verse beginning from the baptism of John until the day on which he was taken up from us—one of these men must become a witness of his resurrection together with us.”
\verse And they proposed two men, Joseph called Barsabbas (who was called Justus) and Matthias.
\verse And they prayed and\lebnote{:|NP|:*Here “and” is supplied because the previous participle (“prayed”) has been translated as a finite verb} said, “You, Lord, who know the hearts of all, show clearly which one of these two you have chosen
\verse to take the place in this ministry and apostleship from which Judas turned aside to depart to his own place.”
\verse And they cast lots for them, and the lot fell on Matthias, and he was added to serve\lebnote{:|NP|:*The words “to serve” are not in the Greek text, but are implied} with the eleven apostles.
\end{biblechapter}

\begin{biblechapter} % Acts 2
\verseWithHeading{Pentecost and the Coming of the Holy Spirit} And when the day of Pentecost had come, they were all together in the same place.
\verse And suddenly a sound like a violent rushing wind came from heaven and filled the whole house where they were sitting.
\verse And divided\lebnote{Or perhaps “distributed”} tongues like fire appeared to them and rested on each one of them.
\verse And they were all filled with the Holy Spirit and began to speak in other languages\lebnote{Or “tongues”} as the Spirit gave them ability to speak out.
\verse Now there were Jews residing in Jerusalem, devout men from every nation under heaven.
\verse And when\lebnote{:|NP|:*Here “when” is supplied as a component of the temporal genitive absolute participle (“happened”)} this sound occurred, the crowd gathered and was in confusion, because each one was hearing them speaking in his own language.
\verse And they were astounded and astonished, saying, “Behold, are not all these who are speaking Galileans?
\verse And how do we hear, each one of us, in \textit{our own native language}?\lebnote{Literally “our own language in which we were born”}
\verse Parthians and Medes and Elamites and those residing in Mesopotamia, Judea and Cappadocia, Pontus and Asia,\lebnote{A reference to the Roman province of Asia (modern Asia Minor)}
\verse Phrygia and Pamphylia, Egypt and the parts of Libya toward Cyrene, and the Romans who were in town,
\verse both Jews and proselytes, Cretans and Arabs—we hear them speaking in our own languages the great deeds of God!”
\verse And all were amazed and greatly perplexed, saying \textit{to one another},\lebnote{Literally “one to the other”} “\textit{What can this mean?}”\lebnote{Literally “what does this want to be”}
\verse But others jeered and\lebnote{:|NP|:*Here “and” is supplied because the previous participle (“jeered”) has been translated as a finite verb} said, “They are full of sweet new wine!”
\verseWithHeading{Peter’s Sermon on the Day of Pentecost} But Peter, standing with the eleven, raised his voice and declared to them, “Judean men, and all those who live in Jerusalem, let this be known to you, and pay attention to my words!
\verse For these men are not drunk, as you assume, because it is the third hour of the day.
\verse But this is what was spoken through the prophet Joel:
\verse ‘And it will be in the last days,’ God says, 
‘I will pour out my Spirit on all flesh, 
and your sons and your daughters will prophesy, 
and your young men will see visions, 
and your old men will dream dreams.
\verse And even on my male slaves and on my female slaves 
I will pour out my Spirit in those days, and they will prophesy.
\verse And I will cause wonders in the heaven\lebnote{Or “sky”} above 
and signs on the earth below, 
blood and fire and vapor of smoke.
\verse The sun will be changed to darkness 
and the moon to blood, 
before the great and glorious day of the Lord comes.
\verse And it will be that everyone who calls upon the name of the Lord will be saved.’\lebnote{:|NP|:A quotation from Joel 2:28–32}
\verse “Israelite men, listen to these words! Jesus the Nazarene, a man attested to you by God with deeds of power and wonders and signs that God did through him in your midst, just as you yourselves know—
\verse this man, delivered up by the determined plan and foreknowledge of God, you executed by\lebnote{:|NP|:*Here “by” is supplied as a component of the participle (“nailing to”) which is understood as means} nailing to a cross\lebnote{:|NP|:*The words “a cross” are not in the Greek text but are implied by the nature of the verb} through the hand of lawless men.
\verse God raised \textit{him}\lebnote{Literally “whom”; it is necessary to specify “him” in the translation to avoid confusion with the “lawless men” in the previous verse} up, having brought to an end the pains of death, because it was not possible for him to be held by it.
\verse For David says with reference to him,
\verse ‘I saw the Lord before me \textit{continually},\lebnote{Literally “through everything”} 
for he is at my right hand so that I will not be shaken.
\verse For this reason my heart was glad 
and my tongue rejoiced greatly, 
furthermore also my flesh will live in hope,
\verse because you will not abandon my soul in Hades, 
nor will you permit your Holy One to experience decay.
\verse “Men and brothers, it is possible to speak with confidence to you about the patriarch David, that he both died and was buried, and his tomb is with us until this day.
\verse Therefore, because he\lebnote{:|NP|:*Here “because” is supplied as a component of the participle (“was”) which is understood as causal} was a prophet and knew that God had sworn to him with an oath to seat \textit{one of his descendants}\lebnote{Literally “from the fruit of his loins”} on his throne,
\verse by\lebnote{:|NP|:*Here “by” is supplied as a component of the participle (“having foreseen”) which is understood as means} having foreseen this,\lebnote{:|NP|:*Here the direct object is supplied from context in the English translation} he spoke about the resurrection of the Christ,\lebnote{Or “Messiah”} that neither was he abandoned in Hades nor did his flesh experience decay.
\verse This Jesus God raised up, of which\lebnote{Or “of whom,” referring to Jesus} we all are witnesses.
\verse Therefore, having been exalted to the right hand of God and having received the promise of the Holy Spirit from the Father, he has poured out this that you see and hear.\lebnote{Some manuscripts have “both see and hear”}
\verse For David did not ascend into heaven, but he himself says,
\verse ‘The Lord said to my Lord, 
“Sit at my right hand,
\verse Therefore let all the house of Israel know beyond a doubt, that God has made him both Lord and Christ—this Jesus whom you crucified!”
\verseWithHeading{The Response to Peter’s Sermon} Now when they\lebnote{:|NP|:*Here “when” is supplied as a component of the participle (“heard”) which is understood as temporal} heard this,\lebnote{:|NP|:*Here the direct object is supplied from context in the English translation} they were pierced to the heart and said to Peter and the other apostles, “What should we do, men and brothers?”
\verse And Peter said\lebnote{Some manuscripts explicitly add “said”} to them, “Repent and be baptized, each one of you, in the name of Jesus Christ for the forgiveness of your sins, and you will receive the gift of the Holy Spirit.
\verse For the promise is for you and for your children, and for all those who are far away, as many as the Lord our God calls to himself.”
\verse And with many other words he solemnly urged and exhorted them, saying, “Be saved from this crooked generation!”
\verse So those who accepted his message were baptized, and on that day about three thousand souls were added.
\verseWithHeading{The Fellowship of the First Believers} And they were devoting themselves to the teaching of the apostles and to fellowship, to the breaking of bread and to prayers.
\verse And fear came on every soul, and many wonders and signs were being performed by the apostles.
\verse And all who believed were in the same place, and had everything in common.
\verse And they began selling\lebnote{:|NP|:*The imperfect tense has been translated as ingressive here (“began selling”)} their\lebnote{:|NP|:*Literally “the”; the Greek article is used here as a possessive pronoun} possessions and property, and distributing these things to all, to the degree that anyone had need.
\verse And every day, devoting themselves to meeting\lebnote{:|NP|:*The words “to meeting” are not in the Greek text but are implied} with one purpose in the temple courts\lebnote{:|NP|:*Here “courts” is supplied to distinguish this area from the interior of the temple building itself} and breaking bread from house to house, they were eating their food with joy and simplicity of heart,
\verse praising God and having favor with all the people. And the Lord was adding every day to the total of those who were being saved.
\end{biblechapter}

\begin{biblechapter} % Acts 3
\verseWithHeading{A Lame Beggar Healed at the Temple} Now Peter and John were going up to the temple at the hour of prayer, the ninth hour.
\verse And a certain man was being carried who was lame \textit{from birth}.\lebnote{Literally “from his mother’s womb”} \textit{He}\lebnote{Literally “who,” but a new sentence was begun here in the translation in keeping with English style} was placed every day at the gate of the temple called “Beautiful,” so that he could ask for charitable gifts from those who were going into the temple courts.\lebnote{:|NP|:*Here “courts” is supplied to distinguish this area from the interior of the temple building itself}
\verse When he\lebnote{:|NP|:*Here “when” is supplied as a component of the participle (“saw”) which is understood as temporal} saw Peter and John about to go into the temple courts,\lebnote{:|NP|:*Here “courts” is supplied to distinguish this area from the interior of the temple building itself} he began asking to receive alms.
\verse And Peter looked intently at him, together with John, and\lebnote{:|NP|:*Here “and” is supplied because the previous participle (“looked intently”) has been translated as a finite verb} said, “Look at us!”
\verse So he fixed his attention on them, expecting to receive something from them.
\verse But Peter said, “Silver and gold \textit{I do not possess},\lebnote{Literally “is not to me”} but what I have, this I give to you—in the name of Jesus Christ the Nazarene, walk!”\lebnote{Some manuscripts have “get up and walk”}
\verse And taking hold of him by the right hand, he raised him up, and immediately his feet and ankles were made strong.
\verse And leaping up, he stood and began walking around\lebnote{:|NP|:*The imperfect tense has been translated as ingressive here (“began walking around”)} and entered into the temple courts\lebnote{:|NP|:*Here “courts” is supplied to distinguish this area from the interior of the temple building itself} with them, walking and leaping and praising God.
\verse And all the people saw him walking and praising God,
\verse And they recognized him, that this one\lebnote{Some manuscripts have “that he himself”} was the one who used to sit asking for alms at the Beautiful Gate of the temple, and they were filled with awe and astonishment at what had happened to him.
\verseWithHeading{Peter’s Sermon in Solomon’s Portico} And while\lebnote{:|NP|:*Here “while” is supplied as a component of the temporal genitive absolute participle (“was holding fast to”)} he was holding fast to Peter and John, all the people ran together to them in the portico called Solomon’s, utterly astonished.
\verse And when he\lebnote{:|NP|:*Here “when” is supplied as a component of the participle (“saw”) which is understood as temporal} saw it,\lebnote{:|NP|:*Here the direct object is supplied from context in the English translation} Peter replied to the people, “Men and Israelites, why are you astonished at this? Or why are you staring at us, as if by our own power or godliness we have made him walk?
\verse The God of Abraham and of Isaac\lebnote{Some manuscripts have “the God of Isaac”} and of Jacob,\lebnote{Some manuscripts have “the God of Jacob”} the God of our fathers, has glorified his servant Jesus, whom you handed over and denied in the presence of Pilate, after\lebnote{Or “although”; this genitive absolute construction can be understood as either temporal “after” or concessive “although”} he had decided to release him.\lebnote{:|NP|:*Here the direct object is supplied from context in the English translation}
\verse But you denied the Holy and Righteous One and demanded that a man—a murderer—be granted to you.
\verse And you killed the originator of life, whom God raised from the dead, of which we are witnesses!
\verse And on the basis of faith in his name, his name has made this man strong, whom you see and know, and the faith that is through him has given him this perfect health in the presence of you all.
\verse And now, brothers, I know that you acted in ignorance, just as your rulers did also.
\verse But the things which God foretold through the mouth of all the prophets, that his Christ\lebnote{Or “Messiah”} would suffer, he has fulfilled in this way.
\verse Therefore repent and turn back, so that your sins may be blotted out,
\verse so that times of refreshing may come from the presence of the Lord, and he may send the Christ\lebnote{Or “Messiah”} appointed for you—Jesus,
\verse whom heaven must receive until the times of the restoration of all things, about which God spoke through the mouth of his holy prophets from earliest times.
\verse Moses said, ‘The Lord God\lebnote{Some manuscripts have “The Lord your God”} will raise up for you a prophet like me from your brothers. You will listen to him in everything that he says to you.\lebnote{:|NP|:A quotation from Deut 18:15}
\verse And it will be that every soul who does not listen to that prophet will be destroyed utterly from the people.’\lebnote{:|NP|:A quotation from Deut 18:19 and Lev 23:29}
\verse And indeed, all the prophets from Samuel and all those who followed him\lebnote{:|NP|:*Here the direct object is supplied from context in the English translation} have spoken about and proclaimed these days.
\verse You are the sons of the prophets and of the covenant that God ordained with your fathers, saying to Abraham, ‘And in your offspring all the nations of the earth will be blessed.’\lebnote{:|NP|:A quotation from Gen 22:18}
\verse God, after he\lebnote{:|NP|:*Here “after” is supplied as a component of the participle (“had raised up”) which is understood as temporal} had raised up his servant, sent him to you first, to bless you by turning each of you back from your wickedness!”
\end{biblechapter}

\begin{biblechapter} % Acts 4
\verseWithHeading{Peter and John Arrested} And while\lebnote{:|NP|:*Here “while” is supplied as a component of the temporal genitive absolute participle (“were speaking”)} they were speaking to the people, the priests and the captain of the temple and the Sadducees approached them,
\verse greatly annoyed because they were teaching the people and proclaiming in Jesus the resurrection from the dead.
\verse And they laid hands on them and put them\lebnote{:|NP|:*Here the direct object is supplied from context in the English translation} in custody until the next day, because it was already evening.
\verse But many of those who listened to the message believed, and the number of the men was approximately five thousand.
\verseWithHeading{Peter and John on Trial Before the Sanhedrin} And it happened that on the next day, their rulers and elders and scribes came together in Jerusalem,
\verse and Annas the high priest, and Caiaphas and John and Alexander, and all those who were from the high priest’s family.
\verse And they made them stand in their\lebnote{:|NP|:*Literally “the”; the Greek article is used here as a possessive pronoun} midst and\lebnote{:|NP|:*Here “and” is supplied because the previous participle (“stand”) has been translated as a finite verb} began to ask,\lebnote{:|NP|:*The imperfect tense has been translated as ingressive here (“began to ask”)} “By what power or by what name did you do this?”
\verse Then Peter, filled with the Holy Spirit, said to them, “Rulers of the people and elders,
\verse if we are being examined today concerning a good deed done to a sick man—by what means\lebnote{Or “through whom”} this man was healed—
\verse let it be known to all of you and to all the people of Israel that by the name of Jesus Christ the Nazarene, whom you crucified, whom God raised from the dead—by him this man stands before you healthy!
\verse This one\lebnote{“This one” refers to Jesus} is the stone that was rejected by you, the builders, that has become \textit{the cornerstone}.\lebnote{Literally “the head of the corner”; this verse is an allusion to Ps 118:22}
\verse And there is salvation in no one else, for there is no other name under heaven that is given among people by which we must be saved.”
\verse Now when they\lebnote{:|NP|:*Here “when” is supplied as a component of the participle (“saw”) which is understood as temporal} saw the boldness of Peter and John, and understood that they were uneducated and untrained men, they were astonished, and recognized them, that they had been with Jesus.
\verse And because they\lebnote{:|NP|:*Here “because” is supplied as a component of the participle (“saw”) which is understood as causal} saw the man who had been healed standing there with them, they had nothing to say in return.
\verse But after they\lebnote{:|NP|:*Here “after” is supplied as a component of the participle (“had ordered”) which is understood as temporal} had ordered them to go outside the Sanhedrin,\lebnote{Or “council”} they began to confer\lebnote{:|NP|:*The imperfect tense has been translated as ingressive here (“began to confer”)} with one another,
\verse saying, “What should we do with these men? For that a remarkable sign has taken place through them is evident to all those who live in Jerusalem, and we are not able to deny it!\lebnote{:|NP|:*Here the direct object is supplied from context in the English translation}
\verse But in order that it may not spread much further among the people, let us warn them to speak no more in this name \textit{to anyone at all}.”\lebnote{Literally “to no man”}
\verse And they called them back and\lebnote{:|NP|:*Here “and” is supplied because the previous participle (“called”) has been translated as a finite verb} commanded them\lebnote{:|NP|:*Here the direct object is supplied from context in the English translation} not to speak or to teach at all in the name of Jesus.
\verse But Peter and John answered and\lebnote{:|NP|:*Here “and” is supplied because the previous participle (“answered”) has been translated as a finite verb} said to them, “Whether it is right in the sight of God to listen to you rather than God, you decide!
\verse For we are not able to refrain from speaking about the things that we have seen and heard.”
\verse So after\lebnote{:|NP|:*Here “after” is supplied as a component of the participle (“threatening … further”) which is understood as temporal} threatening them\lebnote{:|NP|:*Here the direct object is supplied from context in the English translation} further, they released them, finding no way to punish them on account of the people, because they were all praising God for what had happened.
\verse For the man on whom this sign of healing had been performed was more than forty years old.
\verseWithHeading{The Believers Pray for Continued Bold Witness} And when they\lebnote{:|NP|:*Here “when” is supplied as a component of the participle (“were released”) which is understood as temporal} were released, they went to their own people and reported all that the chief priests and the elders had said to them.
\verse And when they\lebnote{:|NP|:*Here “when” is supplied as a component of the participle (“heard”) which is understood as temporal} heard it,\lebnote{:|NP|:*Here the direct object is supplied from context in the English translation} they lifted their voices with one mind to God and said, “Master, you are the one who made the heaven and the earth and the sea and all the things in them,
\verse the one who said by the Holy Spirit through the mouth of our father David, your servant,
\verse ‘Why do the nations\lebnote{Or “Gentiles”; the same Greek word can be translated “nations” or “Gentiles” depending on the context} rage, 
and the peoples conspire in vain?
\verse For in truth both Herod and Pontius Pilate, together with the Gentiles and the peoples of Israel, assembled together in this city against your holy servant Jesus whom you anointed,
\verse to do all that your hand and plan\lebnote{Some manuscripts have “and your plan”} had predestined to take place.
\verse And now, Lord, concern yourself with their threats and grant your slaves to speak your message with all boldness,
\verse as you extend your hand to heal and signs and wonders are performed through the name of your holy servant Jesus.”
\verse And when\lebnote{:|NP|:*Here “when” is supplied as a component of the temporal genitive absolute participle (“had prayed”)} they had prayed, the place in which they were gathered was shaken, and they were all filled with the Holy Spirit and began to speak\lebnote{:|NP|:*The imperfect tense has been translated as ingressive here (“began to speak”)} the word\lebnote{Or “message”} of God with boldness.
\verseWithHeading{The Believers Share All Things in Common} Now the group of those who believed were one heart and soul, and no one said anything of what belonged to him was his own, but all things were theirs in common.
\verse And with great power the apostles were giving testimony to the resurrection of the Lord Jesus, and great grace was on them all.
\verse For there was not even anyone needy among them, because all those who were owners of plots of land or houses were selling them\lebnote{:|NP|:*Here the direct object is supplied from context in the English translation} and\lebnote{:|NP|:*Here “and” is supplied because the previous participle (“were selling”) has been translated as a finite verb} bringing the proceeds of the things that were sold
\verse and placing them\lebnote{:|NP|:*Here the direct object is supplied from context in the English translation} at the feet of the apostles. And it was being distributed to each as anyone had need.
\verse So Joseph, who was called Barnabas by the apostles (which is translated “son of encouragement”), a Levite of Cyprus by nationality,
\verse sold a field\lebnote{Or “a farm”} that belonged to him and\lebnote{:|NP|:*Here “and” is supplied because the previous participle (“sold”) has been translated as a finite verb} brought the money and placed it\lebnote{:|NP|:*Here the direct object is supplied from context in the English translation} at the feet of the apostles.
\end{biblechapter}

\begin{biblechapter} % Acts 5
\verseWithHeading{Ananias and Sapphira Lie to the Holy Spirit} Now a certain man \textit{named}\lebnote{Literally “by name”} Ananias, together with his wife Sapphira, sold a piece of property,
\verse and he kept back for himself some of the proceeds, and his\lebnote{:|NP|:*Literally “the”; the Greek article is used here as a possessive pronoun} wife was aware of it.\lebnote{:|NP|:*Here the participle “was aware of” in this genitive absolute construction has been translated as a finite verb in keeping with English style} And he brought a certain part and\lebnote{:|NP|:*Here “and” is supplied because the previous participle (“brought”) has been translated as a finite verb} placed it\lebnote{:|NP|:*Here the direct object is supplied from context in the English translation} at the feet of the apostles.
\verse But Peter said, “Ananias, for what reason has Satan filled your heart, that you lied to the Holy Spirit and kept back for yourself some of the proceeds of the piece of land?
\verse When it\lebnote{:|NP|:*Here “when” is supplied as a component of the participle (“remained”) which is understood as temporal} remained to you, did it not remain yours? And when it\lebnote{:|NP|:*Here “when” is supplied as a component of the participle (“was sold”) which is understood as temporal} was sold, was it at your disposal? How is it that you have contrived this deed in your heart? You have not lied to people, but to God!”
\verse And when\lebnote{:|NP|:*Here “when” is supplied as a component of the participle (“heard”) which is understood as temporal} Ananias heard these words, he fell down and\lebnote{:|NP|:*Here “and” is supplied because the previous participle (“fell down”) has been translated as a finite verb} died. And great fear came on all those who heard about it.\lebnote{:|NP|:*Here the direct object is supplied from context in the English translation}
\verse So the young men stood up, wrapped him up, and carried him\lebnote{:|NP|:*Here the direct object is supplied from context in the English translation} out and\lebnote{:|NP|:*Here “and” is supplied because the previous participle (“carried … out”) has been translated as a finite verb} buried him.\lebnote{:|NP|:*Here the direct object is supplied from context in the English translation}
\verse And it happened that there was an interval of about three hours, and his wife came in, not knowing what had happened.
\verse And Peter said to her, “Tell me whether you both\lebnote{:|NP|:*Here “both” reflects the second person plural verb, which refers to both Ananias and Sapphira} were paid this much for the piece of land.” And she said, “Yes, this much.”
\verse So Peter said to her, “How is it that it was agreed by you two\lebnote{:|NP|:*Here “two” is supplied in the translation to indicate that the pronoun (“you”) is plural in the Greek text} to test the Spirit of the Lord? Behold, the feet of those who buried your husband are at the door, and they will carry you out!”
\verse And immediately she fell down at his feet and died. So when\lebnote{:|NP|:*Here “when” is supplied as a component of the participle (“came in”) which is understood as temporal} the young men came in, they found her dead, and carried her\lebnote{:|NP|:*Here the direct object is supplied from context in the English translation} out and\lebnote{:|NP|:*Here “and” is supplied because the previous participle (“carried … out”) has been translated as a finite verb} buried her\lebnote{:|NP|:*Here the direct object is supplied from context in the English translation} with her husband.
\verse And great fear came on the whole church and on all who heard about these things.
\verseWithHeading{Many Signs and Wonders Are Performed by the Apostles} Now many signs and wonders were being performed among the people through the hands of the apostles. And they were all together\lebnote{Or perhaps “by common consent”} in Solomon’s Portico.
\verse And none of the rest dared to join them, but the people spoke highly of them.
\verse And even more believers in the Lord\lebnote{Or “even more believers were being added to the Lord”} were being added, large numbers of both men and women,
\verse so that they even carried out the sick into the streets and put them\lebnote{:|NP|:*Here the direct object is supplied from context in the English translation} on cots and mats\lebnote{Or “mattresses”} so that when\lebnote{:|NP|:*Here “when” is supplied as a component of the temporal genitive absolute participle (“came by”)} Peter came by, at least his\lebnote{:|NP|:*Literally “the”; the Greek article is used here as a possessive pronoun} shadow would fall on some of them.
\verse And the people of the towns around Jerusalem also came together, bringing the sick and those tormented by unclean spirits, who were all being healed.
\verseWithHeading{The Apostles Arrested and Imprisoned} Now the high priest rose up and all those who were with him (that is, the party of the Sadducees), and\lebnote{:|NP|:*Here “and” is supplied because the previous participle (“rose up”) has been translated as a finite verb} they were filled with jealousy.
\verse And they laid hands on the apostles and put them in the public prison.
\verse But during the night an angel of the Lord opened the doors of the prison and led them out and\lebnote{:|NP|:*Here “and” is supplied because the two previous participles (“opened” and “led”) have been translated as finite verbs} said,
\verse “Go and stand in the temple courts\lebnote{:|NP|:*Here “courts” is supplied to distinguish this area from the interior of the temple building itself} and\lebnote{:|NP|:*Here “and” is supplied because the previous participle (“stand”) has been translated as a finite verb} proclaim to the people all the words of this life.”
\verse And when they\lebnote{:|NP|:*Here “when” is supplied as a component of the participle (“heard”) which is understood as temporal} heard this,\lebnote{:|NP|:*Here the direct object is supplied from context in the English translation} they entered at daybreak into the temple courts\lebnote{:|NP|:*Here “courts” is supplied to distinguish this area from the interior of the temple building itself} and began teaching.\lebnote{:|NP|:*The imperfect tense has been translated as ingressive here (“began teaching”)}
\verse Now when\lebnote{:|NP|:*Here “when” is supplied as a component of the participle (“arrived”) which is understood as temporal} the high priest and those with him arrived, they summoned the Sanhedrin—even the whole council of elders of the sons of Israel—and sent to the prison to have them brought.
\verse But the officers who came\lebnote{Or “when they came”} did not find them in the prison, and they returned and\lebnote{:|NP|:*Here “and” is supplied because the previous participle (“returned”) has been translated as a finite verb} reported,
\verse saying, “We found the prison locked with all security and the guards standing at the doors, but when we\lebnote{:|NP|:*Here “when” is supplied as a component of the participle (“opened”) which is understood as temporal} opened them,\lebnote{:|NP|:*Here the direct object is supplied from context in the English translation} we found no one inside!”
\verse Now when both the captain of the temple and the chief priests heard these words, they were greatly perplexed concerning them, as to what this might be.
\verse But someone came and\lebnote{:|NP|:*Here “and” is supplied because the previous participle (“came”) has been translated as a finite verb} reported to them, “Behold, the men whom you put in prison are standing in the temple courts\lebnote{:|NP|:*Here “courts” is supplied to distinguish this area from the interior of the temple building itself} and teaching the people!”
\verseWithHeading{The Apostles on Trial Before the Sanhedrin} And when they\lebnote{:|NP|:*Here “when” is supplied as a component of the participle (“had brought”) which is understood as temporal} had brought them, they made them\lebnote{:|NP|:*Here the direct object is supplied from context in the English translation} stand in the Sanhedrin,\lebnote{Or “council”} and the high priest put a question to them,
\verse saying, “\textit{We strictly commanded}\lebnote{Literally “we commanded with a commandment”} you\lebnote{Some manuscripts have “Did we not strictly command you”} not to teach in this name? And behold, you have filled Jerusalem with your teaching! And you are intending to bring upon us the blood of this man!”
\verse But Peter and the apostles answered and\lebnote{:|NP|:*Here “and” is supplied because the previous participle (“answered”) has been translated as a finite verb} said, “It is necessary to obey God rather than men!
\verse The God of our fathers raised up Jesus, whom you killed by\lebnote{:|NP|:*Here “by” is supplied as a component of the participle (“hanging”) which is understood as means} hanging him\lebnote{:|NP|:*Here the direct object is supplied from context in the English translation} on a tree.
\verse This one God has exalted to his right hand as Leader and Savior to grant repentance to Israel and forgiveness of sins.
\verse And we are witnesses of these things, and so is\lebnote{:|NP|:*The words “so is” are not in the Greek text but are implied} the Holy Spirit whom God has given to those who obey him.”
\verse Now when\lebnote{:|NP|:*Here “when” is supplied as a component of the participle (“heard”) which is understood as temporal} they heard this,\lebnote{:|NP|:*Here the direct object is supplied from context in the English translation} they were infuriated, and were wanting to execute them.
\verse But a certain man stood up in the Sanhedrin,\lebnote{Or “council”} a Pharisee \textit{named}\lebnote{Literally “by name”} Gamaliel, a teacher of the law respected by all the people, and\lebnote{:|NP|:*Here “and” is supplied because the previous participle (“stood up”) has been translated as a finite verb} gave orders to put the men outside for a short time.
\verse And he said to them, “Men and Israelites, take care for yourselves what you are about to do to these men!
\verse For before these days, Theudas rose up saying he was somebody. A number of men, about four hundred, joined \textit{him}.\lebnote{Literally “to whom”} \textit{He}\lebnote{Literally “who”} was executed, and all who followed him were dispersed and came to nothing.
\verse After this man, Judas the Galilean rose up in the days of the census and \textit{caused people to follow him in revolt}.\lebnote{Literally “caused people to revolt after him”} And that one perished, and all who followed him were scattered.
\verse And now I tell you, keep away from these men, and leave them alone, because if this plan or this matter is from people, it will be overthrown.\lebnote{Or “it will fail”}
\verse But if it is from God, you will not be able to overthrow them, lest you even be found fighting against God.” So they were persuaded by him.
\verse And they summoned the apostles, beat them,\lebnote{:|NP|:*Here the direct object is supplied from context in the English translation} commanded them\lebnote{:|NP|:*Here the direct object is supplied from context in the English translation} not to speak in the name of Jesus, and released them.\lebnote{:|NP|:*Here the direct object is supplied from context in the English translation}
\verse So they went out from the presence of the Sanhedrin\lebnote{Or “council”} rejoicing, because they had been considered worthy to be dishonored for the sake of the name.
\verse Every day, both in the temple courts\lebnote{:|NP|:*Here “courts” is supplied to distinguish this area from the interior of the temple building itself} and from house to house, they did not stop teaching and proclaiming the good news that the Christ\lebnote{Or “Messiah”} was Jesus.
\end{biblechapter}

\begin{biblechapter} % Acts 6
\verseWithHeading{The First Seven Deacons Appointed} Now in these days, as\lebnote{:|NP|:*Here “as” is supplied as a component of the temporal genitive absolute participle (“were increasing”)} the disciples were increasing in number,\lebnote{:|NP|:*The words “in number” are not in the Greek text but are implied} a complaint arose by the \textit{Greek-speaking Jews}\lebnote{Literally “Hellenists”} against the \textit{Hebraic Jews}\lebnote{Literally “Hebrews”} because\lebnote{Or “that”} their widows were being overlooked in the daily distribution of food.\lebnote{:|NP|:*The words “of food” are not in the Greek text but are implied}
\verse So the twelve summoned the community of disciples and\lebnote{:|NP|:*Here “and” is supplied because the previous participle (“summoned”) has been translated as a finite verb} said, “It is not desirable that we neglect the word of God to serve tables.
\verse So, brothers, select from among you seven men \textit{of good reputation},\lebnote{Literally “well spoken of”} full of the Spirit and wisdom, whom we will put in charge of this need.
\verse But we will devote ourselves to prayer and to the ministry of the word.”
\verse And the statement pleased the whole group, and they chose Stephen (a man full of faith and of the Holy Spirit), and Philip, and Prochorus, and Nicanor, and Timon, and Parmenas, and Nicolaus (a convert from Antioch),
\verse whom they stood before the apostles. And they prayed and\lebnote{:|NP|:*Here “and” is supplied because the previous participle (“prayed”) has been translated as a finite verb} placed their\lebnote{:|NP|:*Literally “the”; the Greek article is used here as a possessive pronoun} hands on them.
\verse And the word of God kept spreading, and the number of disciples in Jerusalem was increasing greatly, and a large number of priests began obeying\lebnote{:|NP|:*The imperfect tense has been translated as ingressive here (“began obeying”)} the faith.
\verseWithHeading{Stephen Arrested} Now Stephen, full of grace and power, was performing great wonders and signs among the people.
\verse But some of those from the Synagogue of the Freedmen (\textit{as it was called}),\lebnote{Literally “called”} both Cyrenians and Alexandrians, and those from Cilicia and Asia,\lebnote{A reference to the Roman province of Asia (modern Asia Minor)} stood up and\lebnote{:|NP|:*Here “and” is supplied because the participle (“disputed”) has been translated as a finite verb in keeping with English style} disputed with Stephen.
\verse And they were not able to resist the wisdom and the Spirit with which he was speaking.
\verse Then they secretly instigated men who said, “We have heard him speaking blasphemous words against Moses and God!”
\verse And they incited the people and the elders and the scribes, and they came up and\lebnote{:|NP|:*Here “and” is supplied because the previous participle (“came up”) has been translated as a finite verb} seized him and brought him\lebnote{:|NP|:*Here the direct object is supplied from context in the English translation} to the Sanhedrin.\lebnote{Or “council”}
\verse And they put forward false witnesses who said, “This man does not stop speaking words against the holy place\lebnote{Some manuscripts have “this holy place”} and the law!
\verse For we have heard him saying that this Nazarene Jesus will destroy this place and will change the customs that Moses handed down to us.”
\verse And as they\lebnote{:|NP|:*Here “as” is supplied as a component of the participle (“looked intently”) which is understood as temporal} looked intently at him, all those who were sitting in the Sanhedrin\lebnote{Or “council”} saw his face was like the face of an angel.
\end{biblechapter}

\begin{biblechapter} % Acts 7
\verseWithHeading{Stephen’s Defense} And the high priest said, “Is it so concerning these things?”
\verse So he said, “Men—brothers and fathers—listen: The God of glory appeared to our father Abraham while he\lebnote{:|NP|:*Here “while” is supplied as a component of the participle (“was”) which is understood as temporal} was in Mesopotamia, before he settled in Haran,
\verse and said to him, ‘Go out from your land and from your relatives and come to the land that I will show you.’
\verse Then he went out from the land of the Chaldeans and\lebnote{:|NP|:*Here “and” is supplied because the previous participle (“went out”) has been translated as a finite verb} settled in Haran. And from there, after his father died, he caused him to move to this land in which you now live.
\verse And he did not give him an inheritance in it—not even a footstep\lebnote{Literally “a step of a foot”}—and he promised to give it\lebnote{:|NP|:*Here the direct object is supplied from context in the English translation} to him for his possession, and to his descendants after him, \textit{although he did not have}\lebnote{Literally “not being to him”}\lebnote{:|NP|:*Here “although” is supplied in the translation as a component of the participle (“was”) which is understood as concessive} a child.
\verse But God spoke like this: ‘His descendants will be foreigners in a foreign land, and they will enslave them and mistreat them\lebnote{:|NP|:*Here the direct object is supplied from context in the English translation} four hundred years,
\verse and the nation \textit{that}\lebnote{Literally “to which”} they will serve as slaves, I will judge,’ God said, ‘and after these things they will come out\lebnote{:|NP|:Verses 6–7 are a quotation from Gen 15:13–14} and will worship me in this place.’\lebnote{The final phrase is an allusion to Exod 3:12}
\verse And he gave him the covenant of circumcision, and so he became the father of Isaac and circumcised him on the eighth day, and Isaac did so with\lebnote{:|NP|:*Here the words “did so with” are not in the Greek text but are implied; in view of the “covenant of circumcision” mentioned earlier in the verse, it is probable that circumcision and not just fatherhood is involved} Jacob, and Jacob did so with\lebnote{:|NP|:*Here the words “did so with” are not in the Greek text but are implied; see the note on the same phrase earlier in this verse} the twelve patriarchs.
\verse And the patriarchs, because they\lebnote{:|NP|:*Here “because” is supplied as a component of the participle (“were jealous of”) which is understood as causal} were jealous of Joseph, sold him\lebnote{:|NP|:*Here the direct object is supplied from context in the English translation} into Egypt. And God was with him,
\verse and rescued him from all his afflictions and granted him favor and wisdom in the sight of Pharaoh king of Egypt. And he appointed him ruler over Egypt and all\lebnote{Some manuscripts have “over all”} his household.
\verse And a famine came over all Egypt and Canaan and great affliction, and our fathers could not find food.
\verse So when\lebnote{:|NP|:*Here “when” is supplied as a component of the participle (“heard”) which is understood as temporal} Jacob heard there was grain in Egypt, he sent out our fathers first.
\verse And on the second visit\lebnote{:|NP|:*The word “visit” is not in the Greek text but is implied} Joseph was made known to his brothers, and the family of Joseph became known to Pharaoh.
\verse So Joseph sent and\lebnote{:|NP|:*Here “and” is supplied because the previous participle (“sent”) has been translated as a finite verb} summoned his father Jacob and all his\lebnote{:|NP|:*Literally “the”; the Greek article is used here as a possessive pronoun} relatives, seventy-five persons in all.
\verse And Jacob went down to Egypt and died, he and our fathers.
\verse And they were brought back to Shechem and buried in the tomb that Abraham had bought for a sum of silver from the sons of Hamor in Shechem.
\verse “But as the time of the promise that God had made to Abraham was drawing near, the people increased and multiplied in Egypt
\verse until another king arose over Egypt who did not know Joseph.
\verse This man deceitfully took advantage of our\lebnote{:|NP|:*Literally “the”; the Greek article is used here as a possessive pronoun} people and\lebnote{:|NP|:*Here “and” is supplied because the previous participle (“deceitfully took advantage of”) has been translated as a finite verb} mistreated our ancestors, \textit{causing them to abandon their infants}\lebnote{Literally “making their infants be abandoned”} so that they would not be kept alive.
\verse At this time Moses was born, and he was beautiful to God. \textit{He}\lebnote{Literally “who”} was brought up for three months in his\lebnote{:|NP|:*Literally “the”; the Greek article is used here as a possessive pronoun} father’s house,
\verse and when\lebnote{:|NP|:*Here “when” is supplied as a component of the temporal genitive absolute participle (“was abandoned”)} he was abandoned, the daughter of Pharaoh took him up and brought him up \textit{as her own son}.\lebnote{Literally “for a son to herself”}
\verse And Moses was educated in all the wisdom of the Egyptians, and was powerful in his words and deeds.
\verse “But when \textit{he was forty years old},\lebnote{Literally “a period of time of forty years was fulfilled for him”} it entered in his heart to visit his brothers, the sons of Israel.
\verse And when he\lebnote{:|NP|:*Here “when” is supplied as a component of the participle (“saw”) which is understood as temporal} saw one of them being unjustly harmed, he defended him\lebnote{:|NP|:*Here the direct object is supplied from context in the English translation} and \textit{avenged}\lebnote{Literally “produced vengeance for”} the one who had been oppressed by\lebnote{:|NP|:*Here “by” is supplied as a component of the participle (“striking down”) which is understood as means} striking down the Egyptian.
\verse And he thought his\lebnote{:|NP|:*Literally “the”; the Greek article is used here as a possessive pronoun} brothers would understand that God was granting deliverance to them by his hand, but they did not understand.
\verse And on the following day, he made an appearance to them while they\lebnote{:|NP|:*Here “while” is supplied as a component of the participle (“were fighting”) which is understood as temporal} were fighting and was attempting to reconcile\lebnote{:|NP|:*Here the imperfect verb has been translated as conative (“was attempting to reconcile”)} them in peace, saying, ‘Men and brothers, why are you doing wrong to one another?’
\verse But the one who was doing wrong to his\lebnote{:|NP|:*Literally “the”; the Greek article is used here as a possessive pronoun} neighbor pushed him aside, saying, ‘Who appointed you a ruler and a judge over us?
\verse You do not want to do away with me \textit{the same way}\lebnote{Literally “in the manner in which”} you did away with the Egyptian yesterday, do you?’\lebnote{:|NP|:A quotation from Exod 2:14; the negative construction in Greek anticipates a negative answer here, indicated by “do you”}
\verse And at this statement, Moses fled and became a foreigner in the land of Midian, where he became the father of two sons.
\verse “And when\lebnote{:|NP|:*Here “when” is supplied as a component of the temporal genitive absolute participle (“had been completed”)} forty years had been completed, an angel appeared to him in the desert of Mount Sinai in the flame of a burning bush.
\verse And when\lebnote{:|NP|:*Here “when” is supplied as a component of the participle (“saw”) which is understood as temporal} Moses saw it,\lebnote{:|NP|:*Here the direct object is supplied from context in the English translation} he was astonished at the sight, and when\lebnote{:|NP|:*Here “when” is supplied as a component of the temporal genitive absolute participle (“approached”)} he approached to look at it,\lebnote{:|NP|:*Here the direct object is supplied from context in the English translation} the voice of the Lord came:
\verse ‘I am the God of your fathers, the God of Abraham and of Isaac and of Jacob!’\lebnote{:|NP|:A quotation from Exod 3:6} So Moses began trembling and\lebnote{:|NP|:*Here “and” is supplied because the previous participle (“began”) has been translated as a finite verb} did not dare to look at it.\lebnote{:|NP|:*Here the direct object is supplied from context in the English translation}
\verse And the Lord said to him, ‘Untie the sandals from your feet, for the place on which you are standing is holy ground.
\verse \textit{I have certainly seen}\lebnote{Literally “seeing I have seen”} the mistreatment of my people who are in Egypt and have heard their groaning, and I have come down to deliver them. And now come, I will send you to Egypt.’\lebnote{:|NP|:A quotation from Exod 3:5, 7–8, 10}
\verse This Moses whom they had repudiated, saying, ‘Who appointed you a ruler and a judge?’\lebnote{:|NP|:A quotation from Exod 2:14 (see v. 27 above)}—this man God sent as both ruler and redeemer with \textit{the help}\lebnote{Literally “hand”} of the angel who appeared to him in the bush.
\verse This man led them out, performing wonders and signs in the land of Egypt and at the Red Sea and in the wilderness for forty years.
\verse “This is the Moses who said to the sons of Israel, ‘God will raise up for you a prophet like me from among your brothers.’\lebnote{:|NP|:A quotation from Deut 18:15}
\verse This is the one who was in the congregation in the wilderness with the angel who spoke to him at Mount Sinai, and who with our fathers received living oracles to give to us,
\verse to whom our fathers were not willing to become obedient, but rejected him\lebnote{:|NP|:*Here the direct object is supplied from context in the English translation} and turned back in their hearts to Egypt,
\verse saying to Aaron, ‘Make us gods who will go on before us! For this Moses, who led us out from the land of Egypt—we do not know what has happened to him!’\lebnote{:|NP|:A quotation from Exod 32:1, 23}
\verse And they manufactured a calf in those days, and offered up a sacrifice to the idol, and began rejoicing\lebnote{:|NP|:*The imperfect tense has been translated as ingressive here (“began rejoicing”)} in the works of their hands.
\verse But God turned away and gave them over to worship the host of heaven, just as it is written in the book of the prophets:
\verse ‘You did not bring offerings and sacrifices to me 
for forty years in the wilderness, did you,\lebnote{:|NP|:*The negative construction in Greek anticipates a negative answer here, indicated by “did you”} house of Israel?
\verse The tabernacle of the testimony \textit{belonged}\lebnote{Literally “was”} to our fathers in the wilderness, just as the one who spoke to Moses directed him\lebnote{:|NP|:*Here the direct object is supplied from context in the English translation} to make it according to the design that he had seen,
\verse and which, after\lebnote{:|NP|:*Here “after” is supplied as a component of the participle (“receiving”) which is understood as temporal} receiving it\lebnote{:|NP|:*Here the direct object is supplied from context in the English translation} in turn, our fathers brought in with Joshua \textit{when they dispossessed the}\lebnote{Literally “in the possession of the”} nations that God drove out from the presence of our fathers, until the days of David,
\verse who found favor in the sight of God and asked to find a habitation for the God of Jacob.\lebnote{Some manuscripts have “for the house of Jacob”}
\verse But Solomon built a house for him.
\verse But the Most High does not live in houses\lebnote{Or “temples made by human hands”; either word (“houses” or “temples”) is understood here} made by human hands, just as the prophet says,
\verse ‘Heaven is my throne 
and earth is the footstool for my feet. 
What kind of house will you build for me, says the Lord, 
or what is the place of my rest?
\verse Did not my hand make all these things?’\lebnote{:|NP|:A quotation from Isa 66:1–2}
\verse “You stiff-necked people and uncircumcised in hearts and in your\lebnote{:|NP|:*Literally “the”; the Greek article is used here as a possessive pronoun} ears! You constantly resist the Holy Spirit! As your fathers did, so also do you!
\verse Which of the prophets did your fathers not persecute? And they killed those who announced beforehand about the coming of the Righteous One, whose betrayers and murderers you have now become,
\verse you who received the law by directions of angels and have not observed it!”
\verseWithHeading{Stephen’s Martyrdom} Now when they\lebnote{:|NP|:*Here “when” is supplied as a component of the participle (“heard”) which is understood as temporal} heard these things, they were infuriated in their hearts and gnashed their\lebnote{:|NP|:*Literally “the”; the Greek article is used here as a possessive pronoun} teeth at him.
\verse But he, being full of the Holy Spirit, looked intently into heaven and\lebnote{:|NP|:*Here “and” is supplied because the previous participle (“looked intently”) has been translated as a finite verb} saw the glory of God, and Jesus standing at the right hand of God.
\verse And he said, “Behold, I see the heavens opened and the Son of Man standing at the right hand of God!”
\verse But crying out with a loud voice, they stopped their ears and rushed at him with one purpose.
\verse And after they\lebnote{:|NP|:*Here “after” is supplied as a component of the participle (“had driven”) which is understood as temporal} had driven him\lebnote{:|NP|:*Here the direct object is supplied from context in the English translation} out of the city, they began to stone\lebnote{:|NP|:*The imperfect tense has been translated as ingressive here (“began stoning”)} him,\lebnote{:|NP|:*Here the direct object is supplied from context in the English translation} and the witnesses laid aside their cloaks at the feet of a young man named Saul.
\verse And they kept on stoning Stephen as he\lebnote{:|NP|:*Here “as” is supplied as a component of the participle (“was calling out”) which is understood as temporal} was calling out and saying, “Lord Jesus, receive my spirit!”
\verse And falling to his\lebnote{:|NP|:*Literally “the”; the Greek article is used here as a possessive pronoun} knees, he cried out with a loud voice, “Lord, do not hold this sin against them!” And after he\lebnote{:|NP|:*Here “after” is supplied as a component of the participle (“said”) which is understood as temporal} said this, he fell asleep.\lebnote{Or “he passed away”}
\end{biblechapter}

\begin{biblechapter} % Acts 8
\verseWithHeading{Saul Attempts to Destroy the Church} And Saul was agreeing with his murder. Now there happened on that day a great persecution against the church in Jerusalem, and they were all scattered throughout the regions of Judea and Samaria, except the apostles.
\verse And devout men buried Stephen and made loud lamentation over him.
\verse But Saul was attempting to destroy the church. Entering \textit{house after house},\lebnote{Literally “from house” to house} he dragged off both men and women and\lebnote{:|NP|:*Here “and” is supplied because the previous participle (“dragged off”) has been translated as a finite verb} delivered them\lebnote{:|NP|:*Here the direct object is supplied from context in the English translation} to prison.
\verseWithHeading{Philip Proclaims Christ in Samaria} Now those who had been scattered went about proclaiming the good news of the word.
\verse And Philip came down to the city of Samaria and\lebnote{:|NP|:*Here “and” is supplied because the previous participle (“came down”) has been translated as a finite verb} began proclaiming\lebnote{:|NP|:*The imperfect tense has been translated as ingressive here (“began proclaiming”)} the Christ\lebnote{Or “Messiah”} to them.
\verse And the crowds with one mind were paying attention to what was being said by Philip, as they heard him\lebnote{:|NP|:*Here the direct object is supplied from context in the English translation} and saw the signs that he was performing.
\verse For many of those who had unclean spirits, they were coming out of them,\lebnote{:|NP|:*The words “of them” are supplied in the translation to indicate that the unclean spirits were coming out of the people} crying out with a loud voice, and many who were paralyzed and lame were healed.
\verse And there was great joy in that city.
\verseWithHeading{Simon the Magician} Now a certain man \textit{named}\lebnote{Literally “by name”} Simon had been in the city practicing magic and astonishing the people of Samaria, saying he was someone great.
\verse They were all paying attention to \textit{him},\lebnote{Literally “whom”} from the least to the greatest, saying, “This man is the power of God that is called ‘Great.’ ”
\verse And they were paying attention to him because for a long time he had astonished them with his\lebnote{:|NP|:*Literally “the”; the Greek article is used here as a possessive pronoun} magic.
\verse But when they believed Philip as he\lebnote{:|NP|:*Here “as” is supplied as a component of the participle (“was proclaiming the good news”) which is understood as temporal} was proclaiming the good news about the kingdom of God and the name of Jesus Christ, both men and women were being baptized.
\verse And Simon himself also believed, and after he\lebnote{:|NP|:*Here “after” is supplied as a component of the participle (“was baptized”) which is understood as temporal} was baptized he was \textit{keeping close company with}\lebnote{Literally “attaching himself to”} Philip. And when he\lebnote{:|NP|:*Here “when” is supplied as a component of the participle (“saw”) which is understood as temporal} saw the signs and great miracles that were taking place, he was astonished.
\verse Now when\lebnote{:|NP|:*Here “when” is supplied as a component of the participle (“heard”) which is understood as temporal} the apostles in Jerusalem heard that Samaria had accepted the word of God, they sent Peter and John to them,
\verse who went down and\lebnote{:|NP|:*Here “and” is supplied because the previous participle (“went down”) has been translated as a finite verb} prayed for them so that they would receive the Holy Spirit.
\verse (For he had not yet fallen on any of them, \textit{but they had only been baptized}\lebnote{Literally “but they were only having been baptized”} in the name of the Lord Jesus.)
\verse Then they placed their\lebnote{:|NP|:*Literally “the”; the Greek article is used here as a possessive pronoun} hands on them and they received the Holy Spirit.
\verse Now Simon, when he\lebnote{:|NP|:*Here “when” is supplied as a component of the participle (“saw”) which is understood as temporal} saw that the Spirit was given through the laying on of the apostles’ hands, offered them money,
\verse saying, “Give to me also this power, so that whomever I place my\lebnote{:|NP|:*Literally “the”; the Greek article is used here as a possessive pronoun} hands on may receive the Holy Spirit!”
\verse But Peter said to him, “May your silver \textit{be destroyed along with you},\lebnote{Literally “be for destruction with you”} because you thought \textit{you could acquire}\lebnote{Literally “to acquire”} the gift of God by means of money!
\verse \textit{You have no}\lebnote{Literally “there is for you no”} part or share in this matter, because your heart is not right before God.
\verse Therefore repent of this wickedness of yours, and ask the Lord if perhaps the intent of your heart may be forgiven you!”
\verse For I see you are in \textit{a state of bitter envy}\lebnote{Literally “the gall of bitterness”} and \textit{bound by unrighteousness}.”\lebnote{Literally “the fetter of unrighteousness”}
\verse But Simon answered and\lebnote{:|NP|:*Here “and” is supplied because the previous participle (“answered”) has been translated as a finite verb} said, “You pray to the Lord for me so that nothing of what you have said will come upon me.”
\verse So when\lebnote{:|NP|:*Here “when” is supplied as a component of the participle (“had solemnly testified”) which is understood as temporal} they had solemnly testified and spoken the word of the Lord, they turned back toward Jerusalem, and were proclaiming the good news to many villages of the Samaritans.
\verseWithHeading{Philip and the Ethiopian Eunuch} Now an angel of the Lord spoke to Philip, saying, “Get up and go toward the south\lebnote{Or “go about noon”} on the road that goes down from Jerusalem to Gaza.” (This is a desert road.)
\verse And he got up and\lebnote{:|NP|:*Here “and” is supplied because the previous participle (“got up”) has been translated as a finite verb} went, and behold, there was a man, an Ethiopian eunuch (a court official of Candace,\lebnote{Or “the Candace” (the title of the queen of Ethiopia)} queen of the Ethiopians, who was over all her treasury) who had come to worship in Jerusalem
\verse and was returning and sitting in his chariot, and reading aloud the prophet Isaiah.
\verse And the Spirit said to Philip, “Approach and join this chariot.”
\verse So Philip ran up to it\lebnote{:|NP|:*Here the direct object is supplied from context in the English translation} and\lebnote{:|NP|:*Here “and” is supplied because the previous participle (“ran up to”) has been translated as a finite verb} heard him reading aloud Isaiah the prophet and said, “So then, do you understand what you are reading?”
\verse And he said, “So how could I, unless someone will guide me?” And he invited Philip to come up and\lebnote{:|NP|:*Here “and” is supplied because the previous participle (“come up”) has been translated as an infinitive} sit with him.
\verse Now the passage of scripture that he was reading aloud was this:
\verse “He was led like a sheep to the slaughter, 
and like a lamb before its shearer is silent, 
so he did not open his mouth.
\verse And the eunuch answered and\lebnote{:|NP|:*Here “and” is supplied because the previous participle (“answered”) has been translated as a finite verb} said to Philip, “I ask you, about whom does the prophet say this—about himself or about someone else?”
\verse So Philip opened his mouth and beginning from this scripture, proclaimed the good news to him about Jesus.
\verse And as they were traveling down the road, they came to some water. And the eunuch said, “Look! Water! What prevents me from being baptized?”\lebnote{A few later manuscripts add v. 37, with minor variations: “He said to him, ‘If you believe with your whole heart, you may.’ And he answered and said, ‘I believe that Jesus Christ is the Son of God.’ ” The verse is almost certainly not an original part of the text of Acts.}
\verse And he ordered the chariot to stop, and they both went down into the water—Philip and the eunuch—and he baptized him.
\verse And when they came up out of the water, the Spirit of the Lord carried Philip away, and the eunuch did not see him any longer, for he went on his way rejoicing.
\verse But Philip found himself at Azotus, and as he\lebnote{:|NP|:*Here “as” is supplied as a component of the participle (“passed through”) which is understood as temporal} passed through, he proclaimed the good news to all the towns until he came to Caesarea.
\end{biblechapter}

\begin{biblechapter} % Acts 9
\verseWithHeading{Saul’s Conversion on the Damascus Road} But Saul, still breathing threats and murder against the disciples of the Lord, went to the high priest
\verse and\lebnote{:|NP|:*Here “and” is supplied because the participle in the previous verse (“went to”) has been translated as a finite verb} asked for letters from him to the synagogues in Damascus, so that if he found any who were of the Way, both men and women, he could bring them\lebnote{:|NP|:*Here the direct object is supplied from context in the English translation} tied up\lebnote{Or figuratively “bring them under arrest”} to Jerusalem.
\verse Now as he proceeded, it happened that when he approached Damascus, suddenly a light from heaven flashed around him.
\verse And falling to the ground, he heard a voice saying to him, “Saul, Saul, why are you persecuting me?”
\verse So he said, “Who are you, Lord?” And he said, “I am Jesus, whom you are persecuting!
\verse But get up and enter into the city, and it will be told to you \textit{what you must do}.”\lebnote{Literally “what thing it is necessary that you do”}
\verse (Now the men who were traveling together with him stood speechless, because they\lebnote{:|NP|:*Here “because” is supplied as a component of the participle (“heard”) which is understood as causal} heard the voice but saw no one.)
\verse So Saul got up from the ground, but although\lebnote{:|NP|:*Here “although” is supplied as a component of the genitive absolute participle (“were open”) which is understood as concessive} his eyes were open he could see nothing. And leading him by the hand, they brought him into Damascus.
\verse And he was \textit{unable to see}\lebnote{Literally “not seeing”} for three days, and he did not eat or drink.
\verseWithHeading{Ananias Sent to Saul} Now there was a certain disciple in Damascus \textit{named}\lebnote{Literally “by name”} Ananias, and the Lord said to him in a vision, “Ananias!” And he said, “Behold, here I am, Lord!”
\verse And the Lord said to him, “Get up, go to the street called ‘Straight’ and in the house of Judas look for \textit{a man named Saul from Tarsus}.\lebnote{Literally “Saul by name of Tarsus”} For behold, he is praying,
\verse and he has seen in a vision a man \textit{named}\lebnote{Literally “by name”} Ananias coming in and placing hands\lebnote{Some manuscripts have “placing his hands”} on him so that he may regain his sight.”
\verse But Ananias replied, “Lord, I have heard from many people about this man, how much harm he has done to your saints in Jerusalem,
\verse and here he has authority from the chief priests to tie up\lebnote{Or figuratively “to arrest” or “to imprison”} all who call upon your name!”
\verse But the Lord said to him, “Go, because this man is my chosen instrument to carry my name before Gentiles\lebnote{The same Greek word can be translated “nations” or “Gentiles” depending on the context} and kings and the sons of Israel.
\verse For I will show him how much he must suffer for the sake of my name.”
\verse So Ananias departed and entered into the house, and placing his\lebnote{:|NP|:*Literally “the”; the Greek article is used here as a possessive pronoun} hands on him, he said, “Brother Saul, the Lord Jesus, who appeared to you on the road by which you came, has sent me so that you may regain your sight and be filled with the Holy Spirit.”
\verse And immediately something like scales fell from his eyes and he regained his sight and got up and\lebnote{:|NP|:*Here “and” is supplied because the previous participle (“got up”) has been translated as a finite verb} was baptized,
\verse and after\lebnote{:|NP|:*Here “after” is supplied as a component of the participle (“taking”) which is understood as temporal} taking food, he regained his strength. And he was with the disciples in Damascus several days.
\verseWithHeading{Saul Proclaims Christ in Damascus} And immediately he began proclaiming\lebnote{:|NP|:*The imperfect tense has been translated as ingressive here (“began proclaiming”)} Jesus in the synagogues: “This one is the Son of God!”
\verse And all who heard him\lebnote{:|NP|:*Here the direct object is supplied from context in the English translation} were amazed, and were saying, “Is this not the one who was wreaking havoc in Jerusalem on those who call upon this name, and had come here for this reason, that he could bring them tied up\lebnote{Or figuratively “bring them under arrest”} to the chief priests?”
\verse But Saul was increasing in strength even more, and was confounding the Jews who lived in Damascus by\lebnote{:|NP|:*Here “by” is supplied as a component of the participle (“proving”) which is understood as means} proving that this one is the Christ.\lebnote{Or “Messiah”}
\verse And when many days had elapsed, the Jews plotted to do away with him.
\verse But their plot became known to Saul, and they were also watching the gates both day and night so that they could do away with him.
\verse But his disciples took him\lebnote{:|NP|:*Here the direct object is supplied from context in the English translation} at night and\lebnote{:|NP|:*Here “and” is supplied because the previous participle (“took”) has been translated as a finite verb} let him down through the wall by\lebnote{:|NP|:*Here “by” is supplied as a component of the participle (“lowering”) which is understood as means} lowering him\lebnote{:|NP|:*Here the direct object is supplied from context in the English translation} in a basket.
\verseWithHeading{Saul in Jerusalem} And when he\lebnote{:|NP|:*Here “when” is supplied as a component of the participle (“arrived”) which is understood as temporal} arrived in Jerusalem, he was attempting to associate with the disciples, and they were all afraid of him, because they\lebnote{:|NP|:*Here “because” is supplied as a component of the participle (“believe”) which is understood as causal} did not believe that he was a disciple.
\verse But Barnabas took him and\lebnote{:|NP|:*Here “and” is supplied because the previous participle (“took”) has been translated as a finite verb} brought him\lebnote{:|NP|:*Here the direct object is supplied from context in the English translation} to the apostles and related to them how he had seen the Lord on the road and that he had spoken to him, and how in Damascus he had spoken boldly in the name of Jesus.
\verse And he was going in and going out among them in Jerusalem, speaking boldly in the name of the Lord.
\verse And he was speaking and debating with the \textit{Greek-speaking Jews},\lebnote{Literally “Hellenists”} but they were trying to do away with him.
\verse And when\lebnote{:|NP|:*Here “when” is supplied as a component of the participle (“found out”) which is understood as temporal} the brothers found out, they brought him down to Caesarea and sent him away to Tarsus.
\verse Then the church throughout all of Judea and Galilee and Samaria had peace, being strengthened. And living in the fear of the Lord and the encouragement of the Holy Spirit, it was increasing in numbers.\lebnote{:|NP|:*The words “in numbers” are not in the Greek text but are implied}
\verseWithHeading{Aeneas Healed} Now it happened that as\lebnote{:|NP|:*Here “as” is supplied as a component of the participle (“was traveling”) which is understood as temporal} Peter was traveling through all the places,\lebnote{:|NP|:*The words “the places” are not in the Greek text but are implied} he also came down to the saints who lived in Lydda.
\verse And he found there a certain man \textit{named}\lebnote{Literally “by name”} Aeneas who was paralyzed, who had been lying on a mat\lebnote{Or “mattress”} for eight years.
\verse And Peter said to him, “Aeneas, Jesus Christ heals you! Get up and make your bed yourself!” And immediately he got up.
\verse And all those who lived in Lydda and Sharon saw him, who all\lebnote{:|NP|:*Here “all” is supplied to indicate the relative pronoun is plural} indeed turned to the Lord.
\verseWithHeading{Dorcas Raised} Now in Joppa there was a certain female disciple \textit{named}\lebnote{Literally “by name”} Tabitha (which translated means “Dorcas”).\lebnote{“Dorcas” is the Greek translation of the Aramaic name “Tabitha” which means “deer” or “gazelle”} She was full of good deeds and charitable giving which she was constantly doing.\lebnote{:|NP|:*Here the imperfect verb is translated as a customary imperfect (“was constantly doing”)}
\verse Now it happened that in those days after\lebnote{:|NP|:*Here “after” is supplied as a component of the participle (“becoming sick”) which is understood as temporal} becoming sick, she died. And after\lebnote{:|NP|:*Here “after” is supplied as a component of the participle (“washing”) which is understood as temporal} washing her,\lebnote{:|NP|:*Here the direct object is supplied from context in the English translation} they placed her in an upstairs room.
\verse And because\lebnote{:|NP|:*Here “because” is supplied as a component of the participle (“was”) which is understood as causal} Lydda was near Joppa, the disciples, when they\lebnote{:|NP|:*Here “when” is supplied as a component of the participle (“heard”) which is understood as temporal} heard that Peter was in \textit{Lydda},\lebnote{Literally “in it”} sent two men to him, urging, “Do not delay to come to us!”
\verse So Peter got up and\lebnote{:|NP|:*Here “and” is supplied because the previous participle (“got up”) has been translated as a finite verb} accompanied them. When he\lebnote{:|NP|:*Here “when” is supplied as a component of the participle (“arrived”) which is understood as temporal} arrived, they brought him\lebnote{:|NP|:*Here the direct object is supplied from context in the English translation} up to the upstairs room, and all the widows came to him, weeping and showing him\lebnote{:|NP|:*Here the direct object is supplied from context in the English translation} tunics and other clothing that Dorcas used to make while she\lebnote{:|NP|:*Here “while” is supplied as a component of the participle (“was”) which is understood as temporal} was with them.
\verse But Peter sent them all outside, and, falling to his\lebnote{:|NP|:*Literally “the”; the Greek article is used here as a possessive pronoun} knees, he prayed. And turning toward the body, he said, “Tabitha, get up!” And she opened her eyes, and when she\lebnote{:|NP|:*Here “when” is supplied as a component of the participle (“saw”) which is understood as temporal} saw Peter, she sat up.
\verse And he gave her his hand and\lebnote{:|NP|:*Here “and” is supplied because the previous participle (“gave”) has been translated as a finite verb} raised her up. And he called the saints and the widows and\lebnote{:|NP|:*Here “and” is supplied because the previous participle (“called”) has been translated as a finite verb} presented her alive.
\verse And it became known throughout all Joppa, and many believed in the Lord.
\verse And it happened that he stayed many days in Joppa with a certain Simon, a tanner.\lebnote{Or “with a certain Simon Berseus”; most modern English versions treat the word as Simon’s profession (“Simon the tanner”), but the word may actually be a surname (“Simon Berseus” or “Simon Tanner”)}
\end{biblechapter}

\begin{biblechapter} % Acts 10
\verseWithHeading{Cornelius Has a Vision} Now there was a certain man in Caesarea \textit{named}\lebnote{Literally “by name”} Cornelius, a centurion of what was called the Italian Cohort,
\verse devout and fearing God together with all his household, doing many charitable deeds for the people and praying to God \textit{continually}.\lebnote{Literally “through everything”}
\verse About the ninth hour of the day, he saw clearly in a vision an angel of God coming to him and saying to him, “Cornelius.”
\verse And he stared at him and became terrified and\lebnote{:|NP|:*Here “and” is supplied because the previous participle (“became”) has been translated as a finite verb} said, “What is it, Lord?” And he said to him, “Your prayers and your charitable deeds have gone up for a memorial offering before God.
\verse And now, send men to Joppa and summon a certain Simon, who is also called Peter.
\verse This man is staying as a guest with a certain Simon, a tanner,\lebnote{Or “with a certain Simon Berseus”; most modern English versions treat the word as Simon’s profession (“Simon the tanner”), but the word may actually be a surname (“Simon Berseus” or “Simon Tanner”)} whose house is by the sea.”
\verse And when the angel who spoke to him departed, he summoned two of the household slaves and a devout soldier from those who attended him,
\verse and after he\lebnote{:|NP|:*Here “after” is supplied as a component of the participle (“had explained”) which is understood as temporal} had explained everything to them, he sent them to Joppa.
\verseWithHeading{Peter Has a Vision} And the next day, as\lebnote{:|NP|:*Here “as” is supplied as a component of the temporal genitive absolute participle (“were on their way”)} they were on their way and approaching the city, Peter went up on the housetop to pray at about the sixth hour.
\verse And he became hungry and wanted to eat. But while\lebnote{:|NP|:*Here “while” is supplied as a component of the temporal genitive absolute participle (“were preparing”)} they were preparing the food,\lebnote{:|NP|:*Here the direct object is supplied from context in the English translation} a trance came over him.
\verse And he saw heaven opened and an object something like a large sheet coming down, being let down to the earth by its four corners,
\verse in which were all the four-footed animals and reptiles of the earth and birds of the sky.
\verse And a voice came to him, “Get up, Peter, slaughter and eat!”
\verse But Peter said, “Certainly not, Lord! For I have never eaten anything common and unclean!”
\verse And the voice came again to him for the second time: “The things which God has made clean, you must not consider unclean!”
\verse And this happened three times, and immediately the object was taken up into heaven.
\verse Now while Peter was greatly perplexed within himself as to what the vision that he had seen might be, behold, the men who had been sent by Cornelius, having found the house of Simon by asking around, stood at the gate.
\verse And they called out and\lebnote{:|NP|:*Here “and” is supplied because the previous participle (“called out”) has been translated as a finite verb} asked if Simon who was also called Peter was staying there as a guest.
\verse And while\lebnote{:|NP|:*Here “while” is supplied as a component of the temporal genitive absolute participle (“was reflecting”)} Peter was reflecting about the vision, the Spirit said to him, “Behold, men\lebnote{Some manuscripts have “three men”} are looking for you.
\verse But get up, go down, and go with them—not hesitating at all, because I have sent them.”
\verse So Peter went down to the men and\lebnote{:|NP|:*Here “and” is supplied because the previous participle (“went down”) has been translated as a finite verb} said, “Behold, I am he whom you are looking for! What is the reason for which you have come?”
\verse And they said, “Cornelius, a centurion, a righteous and God-fearing man—and well spoken of by the whole nation of the Jews—was directed by a holy angel to summon you to his house and to hear words from you.”
\verse So he invited them in and\lebnote{:|NP|:*Here “and” is supplied because the previous participle (“invited … in”) has been translated as a finite verb} entertained them as guests, and on the next day he got up and\lebnote{:|NP|:*Here “and” is supplied because the previous participle (“got up”) has been translated as a finite verb} went away with them. And some of the brothers from Joppa accompanied him.
\verse And on the next day he entered into Caesarea.
\verseWithHeading{Peter Visits Cornelius} Now Cornelius was waiting for them, and\lebnote{:|NP|:*Here “and” is supplied because the previous participle (“waiting for”) has been translated as a finite verb} had called together his relatives and close friends.
\verse So it happened that when Peter entered, Cornelius met him, fell at his\lebnote{:|NP|:*Literally “the”; the Greek article is used here as a possessive pronoun} feet, and\lebnote{:|NP|:*Here “and” is supplied because the previous participle (“fell”) has been translated as a finite verb} worshiped him.\lebnote{:|NP|:*Here the direct object is supplied from context in the English translation}
\verse But Peter helped him up, saying, “Get up! I myself am also a man!”
\verse And as he\lebnote{:|NP|:*Here “as” is supplied as a component of the participle (“conversed with”) which is understood as temporal} conversed with him, he went in and found many people gathered.
\verse And he said to them, “You know that it is forbidden for a Jewish man to associate with or to approach a foreigner. And to me God has shown that I should call no man common or unclean.
\verse And Cornelius said, “\textit{Four days ago at this hour},\lebnote{Literally “from the fourth day until this hour”} the ninth, I was praying in my house. And behold, a man in shining clothing stood before me
\verse and said, ‘Cornelius, your prayer has been heard, and your charitable deeds have been remembered before God.
\verse Therefore send to Joppa and summon Simon who is also called Peter. This man is staying as a guest in the house of Simon, a tanner,\lebnote{Or “of Simon Berseus”; most modern English versions treat the word as Simon’s profession (“Simon the tanner”), but the word may actually be a surname (“Simon Berseus” or “Simon Tanner”)} by the sea.
\verse Therefore I sent for you at once, and you \textit{were kind enough to come}.\lebnote{Literally “have done rightly coming”} So now we all are present before God to hear all the things that have been commanded to you by the Lord.”
\verse So Peter opened his\lebnote{:|NP|:*Literally “the”; the Greek article is used here as a possessive pronoun} mouth and\lebnote{:|NP|:*Here “and” is supplied because the previous participle (“opened”) has been translated as a finite verb} said, “In truth I understand that God is not one who shows partiality,
\verse but in every nation the one who fears him and who does what is right is acceptable to him.
\verse As for\lebnote{:|NP|:*The words “As for” are not in the Greek text, but are supplied in the translation in keeping with English style} the message that he sent to the sons of Israel, proclaiming the good news of peace through Jesus Christ—this one is Lord of all—
\verse you know the thing that happened throughout all Judea, beginning from Galilee, after the baptism that John proclaimed:
\verse Jesus of Nazareth—how God anointed him with the Holy Spirit and with power, who went about doing good and healing all who were oppressed by the devil, because God was with him.
\verse And we are witnesses of all the things that he did both in the land of the Judeans and in Jerusalem, whom they also executed by\lebnote{:|NP|:*Here “by” is supplied as a component of the participle (“hanging”) which is understood as means} hanging him\lebnote{:|NP|:*Here the direct object is supplied from context in the English translation} on a tree.
\verse God raised this one up on the third day and granted that he should become visible,
\verse not to all the people but to us who had been chosen beforehand by God as witnesses, who ate and drank with him after he rose from the dead.
\verse And he commanded us to preach to the people and to testify solemnly that this one is the one appointed\lebnote{Or “one who is designated”} by God as judge of the living and of the dead.
\verse To this one all the prophets testify, that through his name everyone who believes in him receives forgiveness of sins.”
\verseWithHeading{The Holy Spirit Given to Gentiles} While\lebnote{:|NP|:*Here “while” is supplied as a component of the temporal genitive absolute participle (“was … speaking”)} Peter was still speaking these words, the Holy Spirit fell on all those who were listening to the message.
\verse And those believers from the circumcision who had accompanied Peter were astonished that the gift of the Holy Spirit had been poured out even on the Gentiles,
\verse for they heard them speaking in tongues and glorifying God. Then Peter said,
\verse “Surely no one can withhold the water for these people to be baptized, who have received the Holy Spirit as we also did!”
\verse So he ordered that they be baptized in the name of Jesus Christ. Then they asked him to stay for several days.
\end{biblechapter}

\begin{biblechapter} % Acts 11
\verseWithHeading{Peter’s Explanation to the Church in Jerusalem} Now the apostles and the brothers who were throughout Judea heard that the Gentiles also had accepted the word\lebnote{Or “message”} of God.
\verse So when Peter went up to Jerusalem, those of the circumcision took issue with him,
\verse saying, “You went to men \textit{who were uncircumcised}\lebnote{Literally “who had uncircumcision} and ate with them!”
\verse But Peter began and\lebnote{:|NP|:*Here “and” is supplied because the previous participle (“began”) has been translated as a finite verb} explained it\lebnote{:|NP|:*Here the direct object is supplied from context in the English translation} to them in an orderly sequence, saying,
\verse “I was in the city of Joppa praying, and in a trance I saw a vision—an object something like a large sheet coming down, being let down from heaven by its four corners, and it came to me.
\verse As I\lebnote{:|NP|:*Here “as” is supplied as a component of the participle (“looked intently”) which is understood as temporal} looked intently into it, I was considering it,\lebnote{:|NP|:*Here the direct object is supplied from context in the English translation} and I saw the four-footed animals of the earth and the wild animals and the reptiles and the birds of the sky.
\verse And I also heard a voice saying to me, ‘Get up, Peter, slaughter and eat!’
\verse But I said, ‘Certainly not, Lord! For nothing common or unclean has ever entered into my mouth!’
\verse But the voice replied from heaven for the second time, ‘The things which God has made clean, you must not consider unclean!’
\verse And this happened three times, and everything was pulled up into heaven again.
\verse And behold, at once three men who had been sent to me from Caesarea approached the house in which we were staying.\lebnote{:|NP|:*Here the word “staying” is not in the Greek text but is implied}
\verse And the Spirit told me to accompany them, not hesitating at all. So these six brothers also went with me, and we entered into the man’s house.
\verse And he reported to us how he had seen the angel standing in his house and saying, ‘Send to Joppa and summon Simon, who is also called Peter,
\verse who will speak words to you by which you will be saved, you and all your household.’
\verse And as I was beginning to speak, the Holy Spirit fell on them, just as also on us at the beginning.
\verse And I remembered the word of the Lord, how he said, ‘John baptized with water, but you will be baptized with the Holy Spirit.’\lebnote{An allusion to Acts 1:5}
\verse Therefore if God gave them the same gift as also to us when we\lebnote{:|NP|:*Here “when” is supplied as a component of the participle (“believed”) which is understood as temporal} believed in the Lord Jesus Christ, who was I to be able to hinder God?”
\verse And when they\lebnote{:|NP|:*Here “when” is supplied as a component of the participle (“heard”) which is understood as temporal} heard these things, they became silent\lebnote{:|NP|:*Here the aorist verb is translated as ingressive (“became silent”)} and praised God, saying, “Then God has granted the repentance leading to life to the Gentiles also!”
\verseWithHeading{Developments in the Church in Antioch} Now those who had been scattered because of the persecution that took place over Stephen traveled as far as Phoenicia and Cyprus and Antioch, proclaiming the message to no one except Jews alone.
\verse But some of them were men from Cyprus and Cyrene, who, when they\lebnote{:|NP|:*Here “when” is supplied as a component of the participle (“came”) which is understood as temporal} came to Antioch, began to speak\lebnote{:|NP|:*The imperfect tense has been translated as ingressive here (“began to speak”)} to the Hellenists\lebnote{:|NP|:*Here this term could refer to (1) Greek-speaking Jews or (2) Greek-speaking non-Jews (i.e., Gentiles)} also, proclaiming the good news about the Lord Jesus.
\verse And the hand of the Lord was with them, and a large number who believed turned to the Lord.
\verse \textit{And the report came to the attention}\lebnote{Literally “and the report was heard in the ears”} of the church that was in Jerusalem about them, and they sent out Barnabas as far as\lebnote{Some manuscripts have “to go as far as”} Antioch,
\verse who, when he\lebnote{:|NP|:*Here “when” is supplied as a component of the participle (“arrived”) which is understood as temporal} arrived and saw the grace of God, rejoiced and encouraged them all to remain true to the Lord with \textit{devoted hearts},\lebnote{Literally “purpose of heart”}
\verse because he was a good man and full of the Holy Spirit and of faith. And a large number were added\lebnote{Or “were brought”} to the Lord.
\verse So he departed for Tarsus to look for Saul.
\verse And when he\lebnote{:|NP|:*Here “when” is supplied as a component of the participle (“found”) which is understood as temporal} found him,\lebnote{:|NP|:*Here the direct object is supplied from context in the English translation} he brought him\lebnote{:|NP|:*Here the direct object is supplied from context in the English translation} to Antioch. And it happened to them also that they met together for a whole year with the church and taught a large number of people.\lebnote{:|NP|:*The words “of people” are not in the Greek text but are implied} And in Antioch the disciples were first called Christians.
\verse Now in those days prophets came down from Jerusalem to Antioch.
\verse And one of them \textit{named}\lebnote{Literally “by name”} Agabus stood up and\lebnote{:|NP|:*Here “and” is supplied because the previous participle (“stood up”) has been translated as a finite verb} indicated by the Spirit that a great famine was about to come over the whole inhabited earth (which took place in the time of Claudius).
\verse So from the disciples, \textit{according to their ability to give},\lebnote{Literally “to the degree that anyone was prospering”} each one of them determined to send financial aid\lebnote{:|NP|:*Here the direct object is supplied from context in the English translation} for support to the brothers who lived in Judea,
\verse which they also did, sending the aid\lebnote{:|NP|:*Here the direct object is supplied from context in the English translation} to the elders by the hand of Barnabas and Saul.
\end{biblechapter}

\begin{biblechapter} % Acts 12
\verseWithHeading{Herod Kills James and Imprisons Peter} Now at that time, Herod the king laid hands on some of those from the church to harm them.\lebnote{:|NP|:*Here the direct object is supplied from context in the English translation}
\verse So he executed James the brother of John with a sword.
\verse And when he\lebnote{:|NP|:*Here “when” is supplied as a component of the participle (“saw”) which is understood as temporal} saw that it was pleasing to the Jews, he proceeded to arrest Peter also. (\textit{Now this was during the feast}\lebnote{Literally “now these were the days”} of Unleavened Bread.)
\verse After he\lebnote{:|NP|:*Here “after” is supplied as a component of the participle (“had arrested”) which is understood as temporal} had arrested \textit{him},\lebnote{Literally “whom”} he also put him\lebnote{:|NP|:*Here the direct object is supplied from context in the English translation} in prison, handing him\lebnote{:|NP|:*Here the direct object is supplied from context in the English translation} over to four squads of soldiers to guard him, intending to bring him \textit{out for public trial}\lebnote{Literally “to the people”} after the Passover.
\verse Thus Peter was kept in the prison, but prayer was fervently being made to God by the church for him.
\verseWithHeading{Peter Rescued by an Angel} Now when Herod was about to bring him out, on that very night Peter was sleeping between two soldiers, bound with two chains, and guards before the door were watching the prison.
\verse And behold, an angel of the Lord stood near him,\lebnote{:|NP|:*Here the direct object is supplied from context in the English translation} and a light shone in the prison cell. And striking Peter’s side, he woke him up, saying, “Get up \textit{quickly}!”\lebnote{Literally “with quickness”} And his chains fell off of his\lebnote{:|NP|:*Literally “the”; the Greek article is used here as a possessive pronoun} hands.
\verse And the angel said to him, “Gird yourself and put on your sandals!” And he did so. And he said to him, “Wrap your cloak around you and follow me!”
\verse And he went out and\lebnote{:|NP|:*Here “and” is supplied because the previous participle (“went out”) has been translated as a finite verb} was following him.\lebnote{:|NP|:*Here the direct object is supplied from context in the English translation} And he did not know that what was being done by the angel was real, but was thinking he was seeing a vision.
\verse And after they\lebnote{:|NP|:*Here “after” is supplied as a component of the participle (“had passed”) which is understood as temporal} had passed the first and second guard, they came to the iron gate that leads to the city, which opened for them by itself, and they went out and\lebnote{:|NP|:*Here “and” is supplied because the previous participle (“went out”) has been translated as a finite verb} went forward along one narrow street, and at once the angel departed from him.
\verse And when\lebnote{:|NP|:*Here “when” is supplied as a component of the participle (“came”) which is understood as temporal} Peter came to himself, he said, “Now I know truly that the Lord has sent out his angel and rescued me from the hand of Herod and all \textit{that the Jewish people expected}!”\lebnote{Literally “the expectation of the people of the Jews”}
\verse And when he\lebnote{:|NP|:*Here “when” is supplied as a component of the participle (“realized”) which is understood as temporal} realized this,\lebnote{:|NP|:*Here the direct object is supplied from context in the English translation} he went to the house of Mary, the mother of John (who is also called Mark), where many people were gathered together and were praying.
\verse And when\lebnote{:|NP|:*Here “when” is supplied as a component of the temporal genitive absolute participle (“knocked”)} he knocked at the door of the gateway, a female slave \textit{named}\lebnote{Literally “by name”} Rhoda came up to answer.
\verse And recognizing Peter’s voice, because of her\lebnote{:|NP|:*Literally “the”; the Greek article is used here as a possessive pronoun} joy she did not open the gate, but ran in and\lebnote{:|NP|:*Here “and” is supplied because the previous participle (“ran in”) has been translated as a finite verb} announced that Peter was standing at the gate.
\verse But they said to her, “You are out of your mind!” But she kept insisting\lebnote{:|NP|:*This imperfect verb is translated as an iterative imperfect (“kept insisting”)} it was so. And they kept saying,\lebnote{:|NP|:*This imperfect verb is translated as an iterative imperfect (“kept saying”)} “It is his angel!”
\verse But Peter was continuing to knock, and when they\lebnote{:|NP|:*Here “when” is supplied as a component of the participle (“opened”) which is understood as temporal} opened the door\lebnote{:|NP|:*Here the direct object is supplied from context in the English translation} they saw him and were astonished.
\verse But motioning to them with his\lebnote{:|NP|:*Literally “the”; the Greek article is used here as a possessive pronoun} hand to be silent, he related to them how the Lord had brought him out of the prison. And he said, “Report these things to James and to the brothers,” and he departed and\lebnote{:|NP|:*Here “and” is supplied because the previous participle (“departed”) has been translated as a finite verb} went to another place.
\verse Now when\lebnote{:|NP|:*Here “when” is supplied as a component of the temporal genitive absolute participle (“came”)} day came, there was not a little commotion among the soldiers as to what then had become of Peter.
\verse And when\lebnote{:|NP|:*Here “when” is supplied as a component of the participle (“had searched for”) which is understood as temporal} Herod had searched for him and did not find him,\lebnote{:|NP|:*Here the direct object is supplied from context in the English translation} he questioned the guards and\lebnote{:|NP|:*Here “and” is supplied because the previous participle (“questioned”) has been translated as a finite verb} ordered that they be led away to execution. And he came down from Judea to Caesarea and\lebnote{:|NP|:*Here “and” is supplied because the previous participle (“came down”) has been translated as a finite verb} stayed there.
\verseWithHeading{Herod’s Gruesome Death} Now he was very angry with the Tyrians and Sidonians. So they came to him with one purpose, and after\lebnote{:|NP|:*Here “after” is supplied as a component of the participle (“persuading”) which is understood as temporal} persuading Blastus, \textit{the king’s chamberlain},\lebnote{Literally “the one over the bedroom of the king”} they asked for peace, because their country was supported with food from the king’s country.
\verse So on an appointed day Herod, after\lebnote{:|NP|:*Here “after” is supplied as a component of the participle (“putting on”) which is understood as temporal} putting on royal clothing and sitting down on the judgment seat, began to deliver a public address to them.
\verse But the people began to call out loudly,\lebnote{:|NP|:*The imperfect tense has been translated as ingressive here (“began to call out loudly”)} “The voice of a god and not of a man!”
\verse And immediately an angel of the Lord struck him down \textit{because}\lebnote{Literally “in return for which”} he did not give the glory to God. And he was eaten by worms and\lebnote{:|NP|:*Here “and” is supplied because the previous participle (“was”) has been translated as a finite verb} died.
\verse But the word of God kept on increasing\lebnote{:|NP|:*This imperfect verb has been translated as customary (“kept on increasing”)} and multiplying.
\verse So Barnabas and Saul returned to\lebnote{Some manuscripts read “from”} Jerusalem when they\lebnote{:|NP|:*Here “when” is supplied as a component of the participle (“had completed”) which is understood as temporal} had completed their\lebnote{:|NP|:*Literally “the”; the Greek article is used here as a possessive pronoun} service, having taken along with them\lebnote{:|NP|:*Here the direct object is supplied from context in the English translation} John (who is also called Mark).
\end{biblechapter}

\begin{biblechapter} % Acts 13
\verseWithHeading{Barnabas and Saul Sent Out from Antioch} Now there were prophets and teachers in Antioch in the church that was there: Barnabas, and Simeon (who was called Niger), and Lucius the Cyrenian, and Manaen (a close friend of Herod the tetrarch), and Saul.
\verse And while\lebnote{:|NP|:*Here “while” is supplied as a component of the temporal genitive absolute participle (“were serving”)} they were serving the Lord and fasting, the Holy Spirit said, “Set apart now for me Barnabas and Saul for the work to which I have called them.”
\verse Then, after they\lebnote{:|NP|:*Here “after” is supplied as a component of the participle (“had fasted”) which is understood as temporal} had fasted and prayed and placed their\lebnote{:|NP|:*Literally “the”; the Greek article is used here as a possessive pronoun} hands on them, they sent them\lebnote{:|NP|:*Here the direct object is supplied from context in the English translation} away.
\verseWithHeading{Confronting a Magician on Cyprus} Therefore, sent out by the Holy Spirit, they came down to Seleucia, and from there they sailed away to Cyprus.
\verse And when they\lebnote{:|NP|:*Here “when” is supplied as a component of the participle (“came”) which is understood as temporal} came to Salamis, they began to proclaim the word of God in the synagogues of the Jews. And they also had John as assistant.
\verse And when they\lebnote{:|NP|:*Here “when” is supplied as a component of the participle (“had crossed over”) which is understood as temporal} had crossed over the whole island as far as Paphos, they found a certain man, a magician, a Jewish false prophet whose name was Bar-Jesus,
\verse who was with the proconsul Sergius Paulus, an intelligent man. This man summoned Barnabas and Saul and\lebnote{:|NP|:*Here “and” is supplied because the previous participle (“summoned”) has been translated as a finite verb} wished to hear the word of God.
\verse But Elymas the magician (for his name is translated in this way) opposed them, attempting to turn the proconsul away from the faith.
\verse But Saul (also called Paul), filled with the Holy Spirit, looked intently at him
\verse and\lebnote{:|NP|:*Here “and” is supplied because the participle in the previous verse (“looked intently at”) has been translated as a finite verb} said, “O you who are full of all deceit and of all unscrupulousness, you son of the devil, you enemy of all righteousness! Will you not stop making crooked the straight paths of the Lord!
\verse And now behold, the hand of the Lord is against you, and you will be blind, not seeing the sun \textit{for a while}.\lebnote{Literally “until the time”} And immediately mist and darkness fell over him, and he was going around looking for people\lebnote{:|NP|:*In Greek the direct object (“people”) is understood and must be supplied in the English translation; since the following noun is plural, “people” rather than “someone” is used here} to lead him\lebnote{:|NP|:*Here the direct object is supplied from context in the English translation} by the hand.
\verse Then when\lebnote{:|NP|:*Here “when” is supplied as a component of the participle (“saw”) which is understood as temporal} the proconsul saw what had happened, he believed, because he\lebnote{:|NP|:*Here “because” is supplied as a component of the participle (“was astounded”) which is understood as causal} was astounded at the teaching about\lebnote{:|NP|:*Here “about” reflects an objective genitive (“the Lord” is the object of the teaching)} the Lord.
\verseWithHeading{Preaching in the Synagogue at Pisidian Antioch} Now \textit{Paul and his companions}\lebnote{Literally “those around Paul”} put out to sea from Paphos and\lebnote{:|NP|:*Here “and” is supplied because the previous participle (“put out to sea”) has been translated as a finite verb} came to Perga in Pamphylia, but John departed from them and\lebnote{:|NP|:*Here “and” is supplied because the previous participle (“departed”) has been translated as a finite verb} returned to Jerusalem.
\verse And they went on from Perga and\lebnote{:|NP|:*Here “and” is supplied because the previous participle (“went on”) has been translated as a finite verb} arrived at Pisidian Antioch. And they entered into the synagogue on the day of the Sabbath and\lebnote{:|NP|:*Here “and” is supplied because the previous participle (“entered”) has been translated as a finite verb} sat down.
\verse So after the reading from the law and the prophets, the rulers of the synagogue sent word\lebnote{:|NP|:*Here the direct object is supplied from context in the English translation} to them, saying, “Men and brothers, if there is any message of exhortation by you for the people, say it.”\lebnote{:|NP|:*Here the direct object is supplied from context in the English translation}
\verse So Paul stood up,\lebnote{:|NP|:*Here the participle (“stood up”) is translated as a finite verb because of English style} and motioning with his\lebnote{:|NP|:*Literally “the”; the Greek article is used here as a possessive pronoun} hand, he said, “Israelite men, and those who fear God, listen!
\verse The God of this people Israel chose our fathers and exalted the people during their\lebnote{:|NP|:*Literally “the”; the Greek article is used here as a possessive pronoun} stay in the land of Egypt, and with uplifted arm he led them out of it.
\verse And for a period of time of about forty years, he put up with them in the wilderness.
\verse And after\lebnote{:|NP|:*Here “after” is supplied as a component of the participle (“destroying”) which is understood as temporal} destroying seven nations in the land of Canaan, he gave their land to his people\lebnote{:|NP|:*The words “to his people” are supplied as a clarification of who received the land} as an inheritance.
\verse This took\lebnote{:|NP|:*The words “This took” are not in the Greek text but are supplied in keeping with English style} about four hundred and fifty years. And after these things, he gave them\lebnote{:|NP|:*Here the indirect object “them” is not in the Greek text but is implied} judges until Samuel the prophet.
\verse And then they asked for a king, and God gave them Saul son of Kish, a man from the tribe of Benjamin, for forty years.
\verse And after\lebnote{:|NP|:*Here “after” is supplied as a component of the participle (“removing”) which is understood as temporal} removing him, he raised up David for their king, about whom he also said, testifying, ‘I have found David the son of Jesse to be a man in accordance with my heart, who will carry out all my will.’\lebnote{:|NP|:A quotation from 1 Sam 13:14}
\verse From the descendants of this man, according to his promise, God brought to Israel a Savior, Jesus.
\verse Before \textit{his coming}\lebnote{Literally “the presence of his coming”} John had publicly proclaimed\lebnote{:|NP|:*Here the participle (“had publicly proclaimed”) has been translated as a finite verb in keeping with English style} a baptism of repentance to all the people of Israel.
\verse But while John was completing his\lebnote{:|NP|:*Literally “the”; the Greek article is used here as a possessive pronoun} mission, he said, ‘What do you suppose me to be? I am not he! But behold, one is coming after me of whom I am not worthy to untie the sandals of his\lebnote{:|NP|:*Literally “the”; the Greek article is used here as a possessive pronoun} feet!’
\verse “Men and brothers, sons of the family of Abraham and those among you who fear God—to us the message of this salvation has been sent!
\verse For those who live in Jerusalem and their rulers, because they\lebnote{:|NP|:*Here “because” is supplied as a component of the participle (“did not recognize”) which is understood as causal} did not recognize this one, and the voices of the prophets that are read on every Sabbath, fulfilled them\lebnote{:|NP|:*Here the direct object is supplied from context in the English translation} by\lebnote{:|NP|:*Here “by” is supplied as a component of the participle (“condemning”) which is understood as means} condemning him.\lebnote{:|NP|:*Here the direct object is supplied from context in the English translation}
\verse And although they\lebnote{:|NP|:*Here “although” is supplied as a component of the genitive absolute participle (“found”) which is understood as concessive} found no charge worthy of death, they asked Pilate that he be executed.
\verse And when they had carried out all the things that were written about him, they took him\lebnote{:|NP|:*Here the direct object is supplied from context in the English translation} down from the tree and\lebnote{:|NP|:*Here “and” is supplied because the previous participle (“took … down”) has been translated as a finite verb} placed him\lebnote{:|NP|:*Here the direct object is supplied from context in the English translation} in a tomb.
\verse But God raised him from the dead,
\verse who appeared for many days to those who had come up with him from Galilee to Jerusalem—who are now his witnesses to the people.
\verse And we proclaim the good news to you: that the promise that was made to the fathers,
\verse this promise\lebnote{It is necessary to repeat the word “promise” from the previous verse for clarity here} God has fulfilled to our children\lebnote{Some manuscripts have “to us their children”} by\lebnote{:|NP|:*Here “by” is supplied as a component of the participle (“raising”) which is understood as means} raising Jesus, as it is also written in the second psalm, ‘You are my Son; 
today I have fathered you.’\lebnote{:|NP|:A quotation from Ps 2:7}
\verse But that he has raised him from the dead, no more going to return to decay, he has spoken in this way: ‘I will give you the reliable divine decrees of David.’\lebnote{:|NP|:A quotation from Isa 55:3}
\verse Therefore he also says in another psalm,\lebnote{:|NP|:*The word “psalm” is not in the Greek text but is implied} ‘You will not permit your Holy One to experience decay.’\lebnote{:|NP|:A quotation from Ps 16:10}
\verse For David, after\lebnote{:|NP|:*Here “after” is supplied as a component of the participle (“serving”) which is understood as temporal} serving the purpose of God in his own generation, fell asleep and \textit{was buried with}\lebnote{Literally “was gathered to”} his fathers, and experienced decay.
\verse But he whom God raised up did not experience decay.
\verse “Therefore let it be known to you, men and brothers, that through this one forgiveness of sins is proclaimed to you, and from all the things from which you were not able to be justified by the law of Moses,
\verse by this one everyone who believes is justified!
\verse Watch out, therefore, lest what is stated by the prophets come upon you:\lebnote{:|NP|:*Here the direct object is supplied from context in the English translation}
\verse ‘Look, you scoffers, 
and be astonished and perish! 
For I am doing a work in your days, 
a work that you would never believe 
even if someone were to tell it\lebnote{:|NP|:*Here the direct object is supplied from context in the English translation} to you.’ ”\lebnote{:|NP|:A quotation from Hab 1:5}
\verseWithHeading{Response to the Message in Pisidian Antioch} And as\lebnote{:|NP|:*Here “as” is supplied as a component of the temporal genitive absolute participle (“were going out”)} they were going out, they began urging\lebnote{:|NP|:*The imperfect tense has been translated as ingressive here (“began urging”)} that these things be spoken about to them on the next Sabbath.
\verse And after\lebnote{:|NP|:*Here “after” is supplied as a component of the temporal genitive absolute participle (“had broken up”)} the synagogue had broken up, many of the Jews and the devout\lebnote{Or “God-fearing”} proselytes followed Paul and Barnabas, who were speaking to them and\lebnote{:|NP|:*Here “and” is supplied because the previous participle (“were speaking to”) has been translated as a finite verb} were persuading them to continue in the grace of God.
\verse And on the coming Sabbath, nearly the whole city came together to hear the word of the Lord.
\verse But when\lebnote{:|NP|:*Here “when” is supplied as a component of the participle (“saw”) which is understood as temporal} the Jews saw the crowds, they were filled with jealousy, and began contradicting what was being said by Paul by\lebnote{:|NP|:*Here “by” is supplied as a component of the participle (“reviling”) which is understood as means} reviling him.\lebnote{:|NP|:*Here the direct object is supplied from context in the English translation}
\verse Both Paul and Barnabas spoke boldly and\lebnote{:|NP|:*Here “and” is supplied because the previous participle (“spoke boldly”) has been translated as a finite verb} said, “It was necessary that the word of God be spoken first to you, since you reject it and do not consider yourselves worthy of eternal life! Behold, we are turning to the Gentiles!
\verse For so the Lord has commanded us: ‘I have appointed you \textit{to be}\lebnote{Literally “for”} a light for the Gentiles, 
\textit{that you would bring}\lebnote{Literally “that you would bring”} salvation to the end of the earth.’\lebnote{An allusion to Isa 42:6; 49:6}
\verse And when\lebnote{:|NP|:*Here “when” is supplied as a component of the participle (“heard”) which is understood as temporal} the Gentiles heard this,\lebnote{:|NP|:*Here the direct object is supplied from context in the English translation} they began to rejoice\lebnote{:|NP|:*The imperfect tense has been translated as ingressive here (“began to rejoice”)} and to glorify the word of the Lord. And all those who were designated for eternal life believed.
\verse So the word of the Lord was carried through the whole region.
\verse But the Jews incited the devout women of high social standing and the most prominent men of the city, and stirred up persecution against Paul and Barnabas and threw them out of their district.
\verse So after\lebnote{:|NP|:*Here “after” is supplied as a component of the participle (“shaking off”) which is understood as temporal} shaking off the dust from their feet against them, they went to Iconium.
\verse And the disciples were filled with joy and with the Holy Spirit.
\end{biblechapter}

\begin{biblechapter} % Acts 14
\verseWithHeading{Preaching in Iconium} Now it happened that in Iconium they entered \textit{together}\lebnote{Literally “according to the same”} into the synagogue of the Jews and spoke in such a way that a large number of both Jews and Greeks believed.
\verse But the Jews who were disobedient stirred up and \textit{poisoned the minds}\lebnote{Literally “embittered the souls”} of the Gentiles against the brothers.
\verse So they stayed there\lebnote{:|NP|:*Here the direct object is supplied from context in the English translation} for a considerable time, speaking boldly for the Lord, who testified to the message of his grace, granting signs and wonders to be performed through their hands.
\verse But the population of the city was divided, and \textit{some}\lebnote{Literally “those on the one hand”} were with the Jews \textit{and some}\lebnote{Literally “those on the other hand”} with the apostles.
\verse So when an inclination took place on the part of both the Gentiles and the Jews, together with their rulers, to mistreat them\lebnote{:|NP|:*Here the direct object is supplied from context in the English translation} and to stone them,
\verse they became aware of it\lebnote{:|NP|:*Here the direct object is supplied from context in the English translation} and\lebnote{:|NP|:*Here “and” is supplied because the previous participle (“became aware of”) has been translated as a finite verb} fled to the Lycaonian cities—Lystra and Derbe and the surrounding region.
\verse And there they were continuing to proclaim the good news.
\verseWithHeading{Mistaken for Gods in Lystra} And in Lystra a certain man was sitting powerless in his feet, lame from \textit{birth},\lebnote{Literally “his mother’s womb”} who had never walked.
\verse This man listened while\lebnote{:|NP|:*Here “while” is supplied as a component of the temporal genitive absolute participle (“was speaking”)} Paul was speaking. \textit{Paul},\lebnote{Literally “who”} looking intently at him and seeing that he had faith to be healed,
\verse said with a loud voice, “Stand upright on your feet!” And he leaped up and began walking.\lebnote{:|NP|:*The imperfect tense has been translated as ingressive here (“began walking”)}
\verse And when\lebnote{:|NP|:*Here “when” is supplied as a component of the participle (“saw”) which is understood as temporal} the crowds saw what Paul had done, they raised their voices in the Lycaonian language, saying, “The gods have become like men and\lebnote{:|NP|:*Here “and” is supplied because the previous participle (“have become like”) has been translated as a finite verb} have come down to us!”
\verse And they began calling\lebnote{:|NP|:*The imperfect tense has been translated as ingressive here (“began calling”)} Barnabas Zeus and Paul Hermes, because he was the \textit{principal speaker}.\lebnote{Literally “leader of the message”}
\verse And the priest of the temple\lebnote{:|NP|:*The word “temple” is not in the Greek text but is implied} of Zeus that was just outside the city brought bulls and garlands to the gates and\lebnote{:|NP|:*Here “and” is supplied because the previous participle (“brought”) has been translated as a finite verb} was wanting to offer sacrifice, along with the crowds.
\verse But when\lebnote{:|NP|:*Here “when” is supplied as a component of the participle (“heard about”) which is understood as temporal} the apostles Barnabas and Paul heard about it,\lebnote{:|NP|:*Here the direct object is supplied from context in the English translation} they tore their clothing and\lebnote{:|NP|:*Here “and” is supplied because the previous participle (“tore”) has been translated as a finite verb} rushed out into the crowd, shouting
\verse and saying, “Men, why are you doing these things? We also are men with the same nature as you, proclaiming the good news that you should turn from these worthless things to the living God, who made the heaven and the earth and the sea and all the things that are in them—
\verse who in generations that are past permitted all the nations\lebnote{Or “Gentiles”; the same Greek word can be translated “nations” or “Gentiles” depending on the context} to go their own ways.
\verse And yet he did not leave himself without witness by\lebnote{:|NP|:*Here “by” is supplied as a component of the participle (“doing good”) which is understood as means} doing good, giving you rain from heaven and fruitful seasons, satisfying you\lebnote{:|NP|:*Here the direct object is supplied from context in the English translation} with food and your hearts with gladness.”
\verse And although\lebnote{:|NP|:*Here “although” is supplied as a component of the participle (“said”) which is understood as concessive} they said these things, only with difficulty did they dissuade the crowds from offering sacrifice to them.
\verse But Jews arrived from Antioch and Iconium, and when they\lebnote{:|NP|:*Here “when” is supplied as a component of the participle (“had won over”) which is understood as temporal} had won over the crowds and stoned Paul, they dragged him\lebnote{:|NP|:*Here the direct object is supplied from context in the English translation} outside the city, thinking he was dead.
\verse But after\lebnote{:|NP|:*Here “after” is supplied as a component of the temporal genitive absolute participle (“surrounded”)} the disciples surrounded him, he got up and\lebnote{:|NP|:*Here “and” is supplied because the previous participle (“got up”) has been translated as a finite verb} went into the city. And on the next day he departed with Barnabas for Derbe.
\verseWithHeading{Paul and Barnabas Return to Antioch in Syria} And after they\lebnote{:|NP|:*Here “after” is supplied as a component of the participle (“had proclaimed the good news”) which is understood as temporal} had proclaimed the good news in that city and made many disciples, they returned to Lystra and to Iconium and to Antioch,
\verse strengthening the souls of the disciples, encouraging them\lebnote{:|NP|:*Here the direct object is supplied from context in the English translation} to continue in the faith and saying,\lebnote{:|NP|:*The word “saying” is not in the Greek text but is implied} “Through many persecutions\lebnote{Or “afflictions”} it is necessary for us to enter into the kingdom of God.”
\verse And when they\lebnote{:|NP|:*Here “when” is supplied as a component of the participle (“had appointed”) which is understood as temporal} had appointed elders for them in every church, after\lebnote{:|NP|:*Here “after” is supplied as a component of the participle (“praying”) which is understood as temporal} praying with fasting, they entrusted them to the Lord, in whom they had believed.
\verse And they passed through Pisidia and\lebnote{:|NP|:*Here “and” is supplied because the previous participle (“passed through”) has been translated as a finite verb} came to Pamphylia.
\verse And after\lebnote{:|NP|:*Here “after” is supplied as a component of the participle (“proclaimed”) which is understood as temporal} they proclaimed the message in Perga, they went down to Attalia,
\verse and from there they sailed away to Antioch where they had been commended to the grace of God for the work that they had completed.
\verse And when they\lebnote{:|NP|:*Here “when” is supplied as a component of the participle (“arrived”) which is understood as temporal} arrived and called the church together, they reported all that God had done with them, and that he had opened a door of faith for the Gentiles.\lebnote{Or “nations”; the same Greek word can be translated “nations” or “Gentiles” depending on the context}
\verse And they stayed no little time with the disciples.
\end{biblechapter}

\begin{biblechapter} % Acts 15
\verseWithHeading{The Jerusalem Council} And some men came down from Judea and\lebnote{:|NP|:*Here “and” is supplied because the previous participle (“came down”) has been translated as a finite verb} began teaching\lebnote{:|NP|:*The imperfect tense has been translated as ingressive here (“began teaching”)} the brothers, “Unless you are circumcised according to the custom prescribed by Moses, you cannot be saved.”
\verse And after\lebnote{:|NP|:*Here “after” is supplied as a component of the temporal genitive absolute participle (“was”)} there was no little strife and debate by Paul and Barnabas against them, they appointed Paul and Barnabas and some others from among them to go up to the apostles and elders in Jerusalem concerning this issue.
\verse So they were sent on their way by the church, and\lebnote{:|NP|:*Here “and” is supplied because the previous participle (“were sent on their way”) has been translated as a finite verb} passed through both Phoenicia and Samaria, telling in detail the conversion of the Gentiles and bringing great joy to all the brothers.
\verse And when they\lebnote{:|NP|:*Here “when” is supplied as a component of the participle (“arrived”) which is understood as temporal} arrived in Jerusalem, they were received by the church and the apostles and the elders, and reported all that God had done with them.
\verse But some of those who had believed from the party of the Pharisees stood up, saying, “It is necessary to circumcise them and to command them\lebnote{:|NP|:*Here the direct object is supplied from context in the English translation} to observe the law of Moses!”
\verse Both the apostles and the elders assembled to deliberate concerning this matter.
\verse And after\lebnote{:|NP|:*Here “after” is supplied as a component of the temporal genitive absolute participle (“was”)} there was much debate, Peter stood up and\lebnote{:|NP|:*Here “and” is supplied because the previous participle (“stood up”) has been translated as a finite verb} said to them, “Men and brothers, you know that in the early days\lebnote{Or “from ancient days”} God chose among you through my mouth that the Gentiles should hear the message of the gospel and believe.
\verse And God, who knows the heart, testified to them by\lebnote{:|NP|:*Here “by” is supplied as a component of the participle (“giving”) which is understood as means} giving them\lebnote{:|NP|:*Here the direct object is supplied from context in the English translation} the Holy Spirit, just as he also did to us.
\verse And he made no distinction between us and them, cleansing their hearts by faith.
\verse So now why are you putting God to the test by\lebnote{:|NP|:*Here “by” is supplied as a component of the infinitive (“placing”) which is understood as means} placing on the neck of the disciples a yoke that neither our fathers nor we have been able to bear?
\verse But we believe we will be saved through the grace of the Lord Jesus in \textit{the same}\lebnote{Literally “which”} way those also are.”
\verse And the whole group became silent and listened to Barnabas and Paul describing all the signs and wonders God had done among the Gentiles through them.
\verse And after they had stopped speaking, James answered, saying, “Men and brothers, listen to me!
\verse Simeon has described how God first concerned himself to take from among the Gentiles a people for his name.
\verse And with this the words of the prophets agree, just as it is written:
\verse ‘After these things I will return 
and build up again the tent of David that has fallen, 
and the parts of it that had been torn down I will build up again 
and will restore it,
\verse so that the rest of humanity may seek the Lord, 
even all the Gentiles\lebnote{The same Greek word can be translated “nations” or “Gentiles” depending on the context} \textit{who are called by my name},\lebnote{Literally “on whom my name has been called on them”} 
says the Lord,\lebnote{:|NP|:A quotation from Amos 9:11–12} who makes these things
\verse known from of old.’\lebnote{The last phrase of v. 17 and all of v. 18 is an allusion to Isa 45:21}
\verse Therefore I conclude we should not cause difficulty for those from among the Gentiles who turn to God,
\verse but we should write a letter to them to abstain from the pollution of idols and from sexual immorality and from what has been strangled and from blood.
\verse For Moses has those who proclaim him in every city from ancient generations, because he\lebnote{:|NP|:*Here “because” is supplied as a component of the participle (“is read aloud”) which is understood as causal} is read aloud in the synagogues on every Sabbath.”
\verseWithHeading{The Letter from the Council} Then it seemed best to the apostles and the elders, together with the whole church, to send men chosen from among them to Antioch with Paul and Barnabas—Judas who was called Barsabbas and Silas, men who were leaders among the brothers—
\verse writing this letter\lebnote{:|NP|:*Here the direct object is supplied from context in the English translation} \textit{to be delivered by them}:\lebnote{Literally “by their hand”} The apostles and the elders, brothers. To the brothers who are from among the Gentiles in Antioch and Syria and Cilicia. Greetings!
\verse Because we have heard that some have gone out from among us—to whom we gave no orders—and\lebnote{:|NP|:*Here “and” is supplied because the previous participle (“have gone out”) has been translated as a finite verb} have thrown you into confusion by words upsetting your \textit{minds},\lebnote{Literally “souls”}
\verse it seemed best to us, \textit{having reached a unanimous decision},\lebnote{Literally “having become of one mind”} and\lebnote{:|NP|:*Here “and” is supplied in keeping with English style} having chosen men, to send them\lebnote{:|NP|:*Here the direct object is supplied from context in the English translation} to you together with our dear friends Barnabas and Paul,
\verse men who have risked their lives on behalf of the name of our Lord Jesus Christ.
\verse Therefore we have sent Judas and Silas, and they will report the same things by word of mouth.
\verse For it seemed best to the Holy Spirit and to us to place on you no greater burden except these necessary things:
\verse that you abstain from food sacrificed to idols, and from blood, and from what has been strangled, and from sexual immorality. If you\lebnote{:|NP|:*Here “if” is supplied as a component of the participle (“keep”) which is understood as conditional} keep yourselves from \textit{these things}\lebnote{Literally “which things”} you will do well. Farewell.
\verseWithHeading{The Letter Is Delivered to Antioch} So when\lebnote{:|NP|:*Here “when” is supplied as a component of the participle (“were sent off”) which is understood as temporal} they were sent off, they came down to Antioch, and after\lebnote{:|NP|:*Here “after” is supplied as a component of the participle (“calling together”) which is understood as temporal} calling together the community, they delivered the letter.
\verse And when they\lebnote{:|NP|:*Here “when” is supplied as a component of the participle (“read … aloud”) which is understood as temporal} read it\lebnote{:|NP|:*Here the direct object is supplied from context in the English translation} aloud, they rejoiced at the encouragement.
\verse Both Judas and Silas, who were also prophets themselves, encouraged and strengthened the brothers by a long message.
\verse And after\lebnote{:|NP|:*Here “after” is supplied as a component of the participle (“spending”) which is understood as temporal} spending some time, they were sent away in peace from the brothers to those who had sent them.\lebnote{A few later manuscripts add v. 34, “But Silas decided to stay there.”}
\verse But Paul and Barnabas remained in Antioch teaching and proclaiming the word of the Lord with many others also.
\verseWithHeading{Paul and Barnabas Disagree and Part Company} And after some days, Paul said to Barnabas, “Come then,let us return and\lebnote{:|NP|:*Here “and” is supplied because the previous participle (“return”) has been translated as a finite verb} visit the brothers in every town in which we proclaimed the word of the Lord, to see how they are doing.”
\verse Now Barnabas wanted to take John who was called Mark along also,
\verse but Paul held the opinion they should not take this one along, who departed from them in Pamphylia and did not accompany them in the work.
\verse And a sharp disagreement took place, so that they separated from one another. And Barnabas took along Mark and\lebnote{:|NP|:*Here “and” is supplied because the previous participle (“took along”) has been translated as a finite verb} sailed away to Cyprus,
\verse but Paul chose Silas and\lebnote{:|NP|:*Here “and” is supplied because the previous participle (“chose”) has been translated as a finite verb} departed, after\lebnote{:|NP|:*Here “after” is supplied as a component of the participle (“being commended”) which is understood as temporal} being commended to the grace of the Lord by the brothers.
\verse And he traveled through Syria and Cilicia, strengthening the churches.
\end{biblechapter}

\begin{biblechapter} % Acts 16
\verseWithHeading{Timothy Accompanies Paul and Silas} And he came also to Derbe and to Lystra. And behold, a certain disciple was there \textit{named}\lebnote{Literally “by name”} Timothy, the son of a believing Jewish woman but of a Greek father,
\verse who was well spoken of by the brothers in Lystra and Iconium.
\verse Paul wanted this one to go with him, and he took him\lebnote{:|NP|:*Here the direct object is supplied from context in the English translation} and\lebnote{:|NP|:*Here “and” is supplied because the previous participle (“took”) has been translated as a finite verb} circumcised him because of the Jews who were in those places, for they all knew that his father was Greek.
\verse And as they went through the towns, they passed on to them to observe the rules that had been decided by the apostles and elders who were in Jerusalem.
\verse So the churches were being strengthened in the faith and were growing in number every day.
\verseWithHeading{Paul’s Vision of a Man of Macedonia} And they traveled through the Phrygian and Galatian region, having been prevented by the Holy Spirit from speaking the message in Asia.\lebnote{A reference to the Roman province of Asia (modern Asia Minor)}
\verse And when they\lebnote{:|NP|:*Here “when” is supplied as a component of the participle (“came”) which is understood as temporal} came to Mysia, they attempted to go into Bithynia, and the Spirit of Jesus did not permit them.
\verse So going through Mysia, they went down to Troas.
\verse And a vision appeared to Paul during the night: a certain Macedonian man was standing there and imploring him and saying, “Come over to Macedonia and\lebnote{:|NP|:*Here “and” is supplied because the previous participle (“come over”) has been translated as a finite verb} help us!”
\verse And when he had seen the vision, we wanted at once to go away to Macedonia, concluding that God had called us to proclaim the good news to them.
\verseWithHeading{The Conversion of Lydia at Philippi} So putting out to sea from Troas, we sailed a straight course to Samothrace, and on the following day to Neapolis,
\verse and from there to Philippi, which is a leading city of that district of Macedonia, a Roman colony. And we were staying in this city for some days.
\verse And on the day of the Sabbath, we went outside the city gate beside the river, where we thought there was a place of prayer, and we sat down and\lebnote{:|NP|:*Here “and” is supplied because the previous participle (“sat down”) has been translated as a finite verb} spoke to the women assembled there.
\verse And a certain woman \textit{named}\lebnote{Literally “by name”} Lydia from the city of Thyatira, a merchant dealing in purple cloth who showed reverence for God, was listening. The Lord opened \textit{her}\lebnote{Literally “whose”} heart to pay attention to what was being said by Paul.
\verse And after she was baptized, and her household, she urged us,\lebnote{:|NP|:*Here the direct object is supplied from context in the English translation} saying, “If you consider me to be a believer in the Lord, come to my house and\lebnote{:|NP|:*Here “and” is supplied because the previous participle (“come”) has been translated as a finite verb} stay.” And she prevailed upon us.
\verseWithHeading{Paul and Silas Imprisoned} And it happened that as\lebnote{:|NP|:*Here “as” is supplied as a component of the temporal genitive absolute participle (“were going”)} we were going to the place of prayer, a certain female slave who had a spirit of divination\lebnote{Literally “a spirit of Python”; Python was the name of the serpent or dragon that guarded the Delphic oracle at the foot of Mt. Parnassus and the word eventually came to be used for a spirit of divination} met us, who was bringing a large profit to her owners by\lebnote{:|NP|:*Here “by” is supplied as a component of the infinitive (“fortune-telling”) which is understood as means} fortune-telling.
\verse She followed Paul and us and\lebnote{:|NP|:*Here “and” is supplied because the previous participle (“followed”) has been translated as a finite verb} was crying out, saying, “These men are slaves of the Most High God, who are proclaiming to you the way of salvation!”
\verse And she was doing this for many days. But Paul, becoming greatly annoyed and turning around, said to the spirit, “I command you in the name of Jesus Christ to come out of her!” And it came out \textit{immediately}.\lebnote{Literally “that same hour”}
\verse But when\lebnote{:|NP|:*Here “when” is supplied as a component of the participle (“saw”) which is understood as temporal} her owners saw that their hope of profit was gone, they seized Paul and Silas and\lebnote{:|NP|:*Here “and” is supplied because the previous participle (“seized”) has been translated as a finite verb} dragged them\lebnote{:|NP|:*Here the direct object is supplied from context in the English translation} into the marketplace before the rulers.
\verse And when they\lebnote{:|NP|:*Here “when” is supplied as a component of the participle (“had brought”) which is understood as temporal} had brought them to the chief magistrates, they said, “These men are throwing our city into confusion, being Jews,
\verse and are proclaiming customs that are not permitted for us to accept or to practice, because we\lebnote{:|NP|:*Here “because” is supplied as a component of the participle (“are”) which is understood as causal} are Romans!”
\verse And the crowd joined in attacking them, and the chief magistrates tore off their clothing and\lebnote{:|NP|:*Here “and” is supplied because the previous participle (“tore off”) has been translated as a finite verb} gave orders to beat them\lebnote{:|NP|:*Here the direct object is supplied from context in the English translation} with rods.
\verse And after they\lebnote{:|NP|:*Here “after” is supplied as a component of the participle (“had inflicted”) which is understood as temporal} had inflicted many blows on them, they threw them\lebnote{:|NP|:*Here the direct object is supplied from context in the English translation} into prison, giving orders to the jailer to guard them securely.
\verse Having received such an order, \textit{he}\lebnote{Literally “who”} put them in the inner prison and fastened their feet in the stocks.\lebnote{Or possibly “to the block of wood,” referring to a log to which the prisoners were chained or tied}
\verseWithHeading{The Conversion of the Philippian Jailer} Now about midnight, Paul and Silas were praying and\lebnote{:|NP|:*Here “and” is supplied because the previous participle (“were praying”) has been translated as a finite verb} singing hymns to God, and the prisoners were listening to them.
\verse And suddenly there was a great earthquake, so that the foundations of the prison were shaken. And immediately all the doors were opened and all the bonds\lebnote{Or “chains”} were unfastened.
\verse And after\lebnote{:|NP|:*Here “after” is supplied as a component of the participle (“was”) which is understood as temporal} the jailer was awake and saw the doors of the prison open, he drew his\lebnote{:|NP|:*Literally “the”; the Greek article is used here as a possessive pronoun} sword and\lebnote{:|NP|:*Here “and” is supplied because the previous participle (“drew”) has been translated as a finite verb} was about to kill himself, because he\lebnote{:|NP|:*Here “because” is supplied as a component of the participle (“thought”) which is understood as causal} thought the prisoners had escaped.
\verse But Paul called out with a loud voice, saying, “Do no harm to yourself, for we are all here!”
\verse And demanding lights, he rushed in and, \textit{beginning to tremble},\lebnote{Literally “became trembling”} fell down at the feet of Paul and Silas.
\verse And he brought them outside and\lebnote{:|NP|:*Here “and” is supplied because the previous participle (“brought”) has been translated as a finite verb} said, “Sirs, what must I do so that I can be saved?”
\verse And they said, “Believe in the Lord Jesus and you will be saved, you and your household!”
\verse And they spoke the message of the Lord to him, together with all those in his house.
\verse And he took them at that very hour of the night and\lebnote{:|NP|:*Here “and” is supplied because the previous participle (“took”) has been translated as a finite verb} washed their\lebnote{:|NP|:*Literally “the”; the Greek article is used here as a possessive pronoun} wounds, and he himself was baptized at once, and all those of his household.\lebnote{:|NP|:*The word “household” is not in the Greek text, but is supplied from the previous verse}
\verse And he brought them up into his\lebnote{:|NP|:*Literally “the”; the Greek article is used here as a possessive pronoun} house and\lebnote{:|NP|:*Here “and” is supplied because the previous participle (“brought … up”) has been translated as a finite verb} set a meal before them,\lebnote{:|NP|:*Here the direct object is supplied from context in the English translation} and rejoiced greatly that he had believed in God with his whole household.
\verseWithHeading{Paul and Silas Receive an Official Apology} And when it\lebnote{:|NP|:*Here “when” is supplied as a component of the temporal genitive absolute participle (“was”)} was day, the chief magistrates sent the police officers, saying, “Release those men.”
\verse And the jailer reported these words to Paul: “The chief magistrates have sent an order\lebnote{:|NP|:*Here the direct object is supplied from context in the English translation} that you should be released. So come out now and\lebnote{:|NP|:*Here “and” is supplied because the previous participle (“come out”) has been translated as a finite verb} go in peace!”
\verse But Paul said to them, “They beat us in public without due process—men who are Roman citizens—and\lebnote{:|NP|:*Here “and” is supplied because the previous participle (“beat”) has been translated as a finite verb} threw us\lebnote{:|NP|:*Here the direct object is supplied from context in the English translation} into prison, and now they are wanting to release us secretly? Certainly not! Rather let them come themselves and\lebnote{:|NP|:*Here “and” is supplied because the previous participle (“come”) has been translated as a finite verb} bring us out!”
\verse So the police officers reported these words to the chief magistrates, and they were afraid when they\lebnote{:|NP|:*Here “when” is supplied as a component of the participle (“heard”) which is understood as temporal} heard that they were Roman citizens.
\verse And they came and\lebnote{:|NP|:*Here “and” is supplied because the previous participle (“came”) has been translated as a finite verb} apologized to\lebnote{Or “reassured”; or “conciliated”} them, and after they\lebnote{:|NP|:*Here “after” is supplied as a component of the participle (“brought … out”) which is understood as temporal} brought them\lebnote{:|NP|:*Here the direct object is supplied from context in the English translation} out they asked them\lebnote{:|NP|:*Here the direct object is supplied from context in the English translation} to depart from the city.
\verse And when they\lebnote{:|NP|:*Here “when” is supplied as a component of the participle (“came out”) which is understood as temporal} came out of the prison, they went to Lydia and when they\lebnote{:|NP|:*Here “when” is supplied as a component of the participle (“saw”) which is understood as temporal} saw them,\lebnote{:|NP|:*Here the direct object is supplied from context in the English translation} they encouraged the brothers and departed.
\end{biblechapter}

\begin{biblechapter} % Acts 17
\verseWithHeading{Attacked by a Mob in Thessalonica} Now after they\lebnote{:|NP|:*Here “after” is supplied as a component of the participle (“traveled through”) which is understood as temporal} traveled through Amphipolis and Apollonia, they came to Thessalonica, where there was a synagogue of the Jews.
\verse \textit{And as was his custom},\lebnote{Literally “and in accordance with what he was accustomed to”} Paul went in to them and on three Sabbath days he discussed with them from the scriptures,
\verse explaining and demonstrating that it was necessary for the Christ\lebnote{Or “Messiah”} to suffer and to rise from the dead, and saying, “This Jesus whom I am proclaiming to you is the Christ.”\lebnote{Or “Messiah”}
\verse And some of them were persuaded and joined Paul and Silas, and also a large number of God-fearing Greeks and not a few of the prominent women.
\verse But the Jews were filled with jealousy and, taking along some worthless men from the rabble in the marketplace and forming a mob, threw the city into an uproar. And attacking Jason’s house, they were looking for them to bring them\lebnote{:|NP|:*Here the direct object is supplied from context in the English translation} out to the popular assembly.
\verse And when they\lebnote{:|NP|:*Here “when” is supplied as a component of the participle (“find”) which is understood as temporal} did not find them, they dragged Jason and some brothers before the city officials, shouting, “These people who have stirred up trouble throughout the world\lebnote{Or “empire”} have come here also,
\verse whom Jason has entertained as guests! And these people are all acting contrary to the decrees of Caesar, saying there is another king, Jesus!”
\verse And they threw the crowd into confusion, and the city officials who heard these things.
\verse And after\lebnote{:|NP|:*Here “after” is supplied as a component of the participle (“taking”) which is understood as temporal} taking money as security from Jason and the rest, they released them.
\verseWithHeading{Paul and Silas in Berea} Now the brothers sent away both Paul and Silas at once, during the night, to Berea. \textit{They}\lebnote{Literally “who” (referring to Paul and Silas)} went into the synagogue of the Jews when they\lebnote{:|NP|:*Here “when” is supplied as a component of the participle (“arrived”) which is understood as temporal} arrived.
\verse Now these were more open-minded than those in Thessalonica. \textit{They}\lebnote{Literally “who”} accepted the message with all eagerness, examining the scriptures every day to see if these things were so.
\verse Therefore many of them believed, and not a few of the prominent Greek women and men.
\verse But when the Jews from Thessalonica found out that the message of God had been proclaimed by Paul in Berea also, they came there too, inciting and stirring up the crowds.
\verse So then the brothers sent Paul away at once to go to the sea, and both Silas and Timothy remained there.
\verse And those who conducted Paul brought him\lebnote{:|NP|:*Here the direct object is supplied from context in the English translation} as far as Athens, and after\lebnote{:|NP|:*Here “after” is supplied as a component of the participle (“receiving”) which is understood as temporal} receiving an order for Silas and Timothy that they should come to him as soon as possible, they went away.
\verseWithHeading{Paul in Athens} Now while Paul was waiting for them in Athens, his spirit was provoked within him when he\lebnote{:|NP|:*Here “when” is supplied as a component of the participle (“observed”) which is understood as temporal} observed the city was full of idols.
\verse So he was discussing in the synagogue with the Jews and the God-fearing Gentiles,\lebnote{:|NP|:*Here the word “Gentiles” is not in the Greek text but is implied} and in the marketplace every day with those who happened to be there.
\verse And even some of the Epicurean and Stoic philosophers were conversing with him, and some were saying, “What does this babbler want to say?” But others said,\lebnote{:|NP|:*The words “others said” are not in the Greek text but are implied} “He appears to be a proclaimer of foreign deities,” because he was proclaiming the good news about Jesus and the resurrection.
\verse And they took hold of him and\lebnote{:|NP|:*Here “and” is supplied because the previous participle (“took hold of”) has been translated as a finite verb} brought him\lebnote{:|NP|:*Here the direct object is supplied from context in the English translation} to the Areopagus, saying, “May we learn what is this new teaching being proclaimed by you?
\verse For you are bringing some astonishing things to our ears. Therefore we want to know what \textit{these things mean}.”\lebnote{Literally “these things want to be”}
\verse (Now all the Athenians and the foreigners who stayed there used to spend their time in nothing else than telling something or listening to something new.)
\verseWithHeading{Paul Speaks to the Areopagus} So Paul stood there in the middle of the Areopagus and\lebnote{:|NP|:*Here “and” is supplied because the previous participle (“stood there”) has been translated as a finite verb} said, “Men of Athens, I see you are very religious \textit{in every respect}.\lebnote{Literally “with respect to all things”}
\verse For as I\lebnote{:|NP|:*Here “as” is supplied as a component of the participle (“was passing through”) which is understood as temporal} was passing through and observing carefully your objects of worship, I even found an altar on which was inscribed, ‘To an unknown God.’ Therefore what you worship without knowing it,\lebnote{:|NP|:*Here the direct object is supplied from context in the English translation} this I proclaim to you—
\verse the God who made the world and all the things in it. This one, being Lord of heaven and earth, does not live in temples made by human hands,
\verse nor is he served by human hands as if he\lebnote{:|NP|:*Here “as if” is supplied as a component of the conditional adverbial participle (“needed”)} needed anything, because\lebnote{:|NP|:*Here “because” is supplied as a component of the participle (“gives”) which is understood as causal} he himself gives to everyone life and breath and everything.
\verse And he made from one man every nation of humanity to live on all the face of the earth, determining their fixed times and the fixed boundaries of their habitation,
\verse to search for God, if perhaps indeed they might feel around for him and find him.\lebnote{:|NP|:*Here the direct object is supplied from context in the English translation} And indeed he is not far away from each one of us,
\verse for in him we live and move and exist,\lebnote{Some interpreters hold that the phrase “in him we live and move and exist” is a quotation from Epimenides of Crete, but more likely it is a traditional Greek formula} as even some of \textit{your own}\lebnote{Literally “with respect to you”} poets have said: ‘For we also are \textit{his}\lebnote{Literally “of him”} offspring.’\lebnote{:|NP|:A quotation from Aratus, \textit{Phaenomena} 5}
\verse Therefore, because we\lebnote{:|NP|:*Here “because” is supplied as a component of the participle (“are”) which is understood as causal} are offspring of God, we ought not to think the divine being is like gold or silver or stone, an image formed by human skill and thought.
\verse Therefore although\lebnote{:|NP|:*Here “although” is supplied as a component of the participle (“has overlooked”) which is understood as concessive} God has overlooked the times of ignorance, he now commands all people everywhere to repent,
\verse because he has set a day on which he is going to judge the world in righteousness by the man who he has appointed, having provided proof to everyone by\lebnote{:|NP|:*Here “by” is supplied as a component of the participle (“raising”) which is understood as means} raising him from the dead.”
\verse Now when they\lebnote{:|NP|:*Here “when” is supplied as a component of the participle (“heard about”) which is understood as temporal} heard about the resurrection of the dead, some scoffed, but others said, “We will hear you about this again also.”
\verse So Paul went out from the midst of them.
\verse But some people\lebnote{:|NP|:*Here the Greek term “men” is used as a generic for “people”; note the presence of of a woman (Damaris) in the group} joined him and\lebnote{:|NP|:*Here “and” is supplied because the previous participle (“joined”) has been translated as a finite verb} believed, among whom also were Dionysius the Areopagite and a woman \textit{named}\lebnote{Literally “by name”} Damaris and others with them.
\end{biblechapter}

\begin{biblechapter} % Acts 18
\verseWithHeading{Paul, Silas, and Timothy in Corinth} After these things he departed from Athens and\lebnote{:|NP|:*Here “and” is supplied because the previous participle (“departed”) has been translated as a finite verb} went to Corinth.
\verse And he found a certain Jew \textit{named}\lebnote{Literally “by name”} Aquila, \textit{a native}\lebnote{Literally “by nationality”} of Pontus who had arrived recently from Italy \textit{along with}\lebnote{Literally “and”} Priscilla his wife, because Claudius had ordered all the Jews to depart from Rome, and\lebnote{:|NP|:*Here “and” is supplied because the previous participle (“found”) has been translated as a finite verb} he went to them.
\verse And because he was practicing the same trade, he stayed with them and worked, for they were tentmakers by trade.
\verse And he argued in the synagogue every Sabbath, attempting to persuade\lebnote{:|NP|:*Here the imperfect verb has been translated as conative (“attempting to persuade”)} both Jews and Greeks.
\verse Now when both Silas and Timothy came down from Macedonia, Paul began to be occupied with\lebnote{:|NP|:*The imperfect tense has been translated as ingressive here (“began to be occupied with”)} the message, solemnly testifying to the Jews that the Christ\lebnote{Or “Messiah”} was Jesus.
\verse And when\lebnote{:|NP|:*Here “when” is supplied as a component of the temporal genitive absolute participle (“resisted”)} they resisted and reviled him,\lebnote{:|NP|:*Here the direct object is supplied from context in the English translation} he shook out his\lebnote{:|NP|:*Literally “the”; the Greek article is used here as a possessive pronoun} clothes and\lebnote{:|NP|:*Here “and” is supplied because the previous participle (“shook out”) has been translated as a finite verb} said to them, “Your blood be on your own heads! I am guiltless! From now on I will go to the Gentiles!”
\verse And leaving there, he entered into the house of someone \textit{named}\lebnote{Literally “by name”} Titius Justus, a worshiper\lebnote{Or “a God-fearer”} of God whose house was next door to the synagogue.
\verse And Crispus, the ruler of the synagogue, believed in the Lord together with his whole household. And many of the Corinthians, when they\lebnote{:|NP|:*Here “when” is supplied as a component of the participle (“heard about”) which is understood as temporal} heard about it,\lebnote{:|NP|:*Here the direct object is supplied from context in the English translation} believed and were baptized.
\verse And the Lord said to Paul by a vision in the night, “Do not be afraid, but speak and do not keep silent,
\verse because I am with you and no one will attack you to harm you, because many people are mine in this city.”
\verse So he stayed a year and six months, teaching the word of God among them.
\verseWithHeading{Paul Accused Before the Proconsul Gallio} Now when\lebnote{:|NP|:*Here “when” is supplied as a component of the temporal genitive absolute participle (“was”)} Gallio was proconsul of Achaia, the Jews rose up with one purpose against Paul and brought him before the judgment seat,
\verse saying, “This man is persuading people to worship God contrary to the law!”
\verse But when\lebnote{:|NP|:*Here “when” is supplied as a component of the temporal genitive absolute participle (“was about”)} Paul was about to open his\lebnote{:|NP|:*Literally “the”; the Greek article is used here as a possessive pronoun} mouth, Gallio said to the Jews, “If it was some crime or wicked villainy, O Jews, \textit{I would have been justified in accepting}\lebnote{Literally “with respect to a motive I would have accepted”} your complaint.
\verse But if it is questions concerning a word and names and \textit{your own law},\lebnote{Literally “the according to you law”} see to it\lebnote{:|NP|:*Here the direct object is supplied from context in the English translation} yourselves! I do not wish to be a judge of these things.”
\verse And he drove them away from the judgment seat.
\verse So they all seized Sosthenes, the ruler of the synagogue, and\lebnote{:|NP|:*Here “and” is supplied because the previous participle (“seized”) has been translated as a finite verb} began beating\lebnote{:|NP|:*The imperfect tense has been translated as ingressive here (“began beating”)} him\lebnote{:|NP|:*Here the direct object is supplied from context in the English translation} in front of the judgment seat. And none of these things was a concern to Gallio.
\verseWithHeading{Paul Returns to Antioch in Syria} So Paul, after\lebnote{:|NP|:*Here “after” is supplied as a component of the participle (“remaining”) which is understood as temporal} remaining many days longer, said farewell to the brothers and\lebnote{:|NP|:*Here “and” is supplied because the previous participle (“said farewell”) has been translated as a finite verb} sailed away to Syria, and with him Priscilla and Aquila. He shaved his\lebnote{:|NP|:*Literally “the”; the Greek article is used here as a possessive pronoun} head at Cenchrea, because he had taken a vow.
\verse So they arrived at Ephesus, and those he left behind there, but he himself entered into the synagogue and\lebnote{:|NP|:*Here “and” is supplied because the previous participle (“entered”) has been translated as a finite verb} discussed with the Jews.
\verse And when\lebnote{:|NP|:*Here “when” is supplied as a component of the temporal genitive absolute participle (“asked”)} they asked him\lebnote{:|NP|:*Here the direct object is supplied from context in the English translation} to stay for a longer time, he did not give his consent,
\verse but saying farewell and telling them,\lebnote{:|NP|:*Here the direct object is supplied from context in the English translation} “I will return to you again if\lebnote{:|NP|:*Here “if” is supplied as a component of the conditional adverbial participle (“wills”)} God wills,” he set sail from Ephesus.
\verse And when he\lebnote{:|NP|:*Here “when” is supplied as a component of the participle (“arrived”) which is understood as temporal} arrived at Caesarea, he went up and greeted the church, and\lebnote{:|NP|:*Here “and” is supplied because the two previous participles (“went up” and “greeted”) have been translated as finite verbs} went down to Antioch.
\verse And after\lebnote{:|NP|:*Here “after” is supplied as a component of the participle (“spending”) which is understood as temporal} spending some time there, he departed, traveling through one place after another in the Galatian region and Phrygia, strengthening all the disciples.
\verseWithHeading{The Early Ministry of Apollos} Now a certain Jew \textit{named}\lebnote{Literally “by name”} Apollos, \textit{a native}\lebnote{Literally “by nationality”} Alexandrian, arrived in Ephesus—an eloquent man who was well-versed in the scriptures.
\verse This man had been instructed in the way of the Lord, and being enthusiastic in spirit, he was speaking and teaching accurately the things about Jesus, although he\lebnote{:|NP|:*Here “although” is supplied as a component of the participle (“knew”) which is understood as concessive} knew only the baptism of John.
\verse And he began to speak boldly in the synagogue, but when\lebnote{:|NP|:*Here “when” is supplied as a component of the participle (“heard”) which is understood as temporal} Priscilla and Aquila heard him, they took him aside and explained the way of God to him more accurately.
\verse And when\lebnote{:|NP|:*Here “when” is supplied as a component of the temporal genitive absolute participle (“wanted”)} he wanted to cross over to Achaia, the brothers encouraged him\lebnote{:|NP|:*Here the direct object is supplied from context in the English translation} and\lebnote{:|NP|:*Here “and” is supplied because the previous participle (“encouraged”) has been translated as a finite verb} wrote to the disciples to welcome him. When he\lebnote{:|NP|:*Here “when” is supplied as a component of the participle (“arrived”) which is understood as temporal} arrived, \textit{he}\lebnote{Literally “who”} assisted greatly those who had believed through grace.
\verse For he was vigorously refuting the Jews in public, demonstrating through the scriptures that the Christ\lebnote{Or “Messiah”} was Jesus.
\end{biblechapter}

\begin{biblechapter} % Acts 19
\verseWithHeading{Paul Finds Disciples of John the Baptist in Ephesus} And it happened that while Apollos was in Corinth, Paul traveled through the inland regions and\lebnote{:|NP|:*Here “and” is supplied because the previous participle (“traveled through”) has been translated as a finite verb} came\lebnote{Some manuscripts have “and came down”} to Ephesus and found some disciples.
\verse And he said to them, “Did you receive the Holy Spirit when you\lebnote{:|NP|:*Here “when” is supplied as a component of the participle (“believed”) which is understood as temporal} believed?” And they said to him, “But we have not even heard that there is a Holy Spirit!”
\verse And he said, “Into what then were you baptized?” And they said, “Into the baptism of John.”
\verse And Paul said, “John baptized with a baptism of repentance, telling the people that they should believe in the one who was to come after him—that is, in Jesus.”
\verse And when they\lebnote{:|NP|:*Here “when” is supplied as a component of the participle (“heard”) which is understood as temporal} heard this,\lebnote{:|NP|:*Here the direct object is supplied from context in the English translation} they were baptized in the name of the Lord Jesus.
\verse And when\lebnote{:|NP|:*Here “when” is supplied as a component of the temporal genitive absolute participle (“laid”)} Paul laid hands\lebnote{Some manuscripts have “placed his hands”} on them, the Holy Spirit came upon them and they began to speak\lebnote{:|NP|:*The imperfect tense has been translated as ingressive here (“began to speak”)} in tongues and to prophesy.
\verse (Now the total number of men was about twelve.)
\verse So he entered into the synagogue and\lebnote{:|NP|:*Here “and” is supplied because the previous participle (“entered”) has been translated as a finite verb} was speaking boldly for three months, discussing and attempting to convince\lebnote{The present tense participle has been translated as a conative present (“attempting to convince”)} them\lebnote{:|NP|:*Here the direct object is supplied from context in the English translation} concerning\lebnote{Some manuscripts have “of the things concerning”} the kingdom of God.
\verse But when some became hardened and were disobedient, reviling the Way before the congregation, he departed from them and\lebnote{:|NP|:*Here “and” is supplied because the previous participle (“departed”) has been translated as a finite verb} took away the disciples, leading discussions every day in the lecture hall of Tyrannus.
\verse And this took place for two years, so that all who lived in Asia\lebnote{A reference to the Roman province of Asia (modern Asia Minor)} heard the word of the Lord, both Jews and Greeks.
\verseWithHeading{Would-be Exorcists} And God was performing \textit{extraordinary}\lebnote{Literally “not the ordinary”} miracles by the hands of Paul,
\verse so that even handkerchiefs or work aprons \textit{that had touched his skin}\lebnote{Literally “from his skin”} were carried away to those who were sick, and their\lebnote{:|NP|:*Literally “the”; the Greek article is used here as a possessive pronoun} diseases left them and the evil spirits came out of them.\lebnote{:|NP|:*The words “of them” are not in the Greek text but are implied}
\verse But some itinerant Jewish exorcists also attempted to pronounce the name of the Lord Jesus over those who had evil spirits, saying, “I adjure you by Jesus whom Paul preaches!”
\verse (Now seven sons of a certain Sceva, a Jewish chief priest, were doing this.)
\verse But the evil spirit answered and\lebnote{:|NP|:*Here “and” is supplied because the previous participle (“answered”) has been translated as a finite verb} said to them, “Jesus I know, and Paul I am acquainted with, but who are you?”
\verse And the man \textit{who had the evil spirit}\lebnote{Literally “in whom the evil spirit was”} leaped on them, subdued all of them, and\lebnote{:|NP|:*Here “and” is supplied because the two previous participles (“leaped” and “subdued”) have been translated as finite verbs} prevailed against them, so that they ran away from that house naked and wounded.
\verse And this became known to all who lived in Ephesus, both Jews and Greeks, and fear fell upon them all, and the name of the Lord Jesus was exalted.
\verse And many of those who had believed came, confessing and disclosing their practices,
\verse and many of those who practiced magic brought together their\lebnote{:|NP|:*Literally “the”; the Greek article is used here as a possessive pronoun} books and\lebnote{:|NP|:*Here “and” is supplied because the previous participle (“brought together”) has been translated as a finite verb} burned them\lebnote{:|NP|:*Here the direct object is supplied from context in the English translation} up in the sight of everyone. And they counted up their value and found it was \textit{fifty thousand silver coins}.\lebnote{Literally “five ten thousands of silver coins”}
\verse In this way the word of the Lord was growing in power and was prevailing.
\verseWithHeading{A Major Riot in Ephesus} Now when these things were completed, Paul resolved in the Spirit to go to Jerusalem, passing through Macedonia and Achaia, saying, “After I have been there, it is necessary for me to see Rome also.”
\verse So after\lebnote{:|NP|:*Here “after” is supplied as a component of the participle (“sending”) which is understood as temporal} sending two of those who were assisting him, Timothy and Erastus, to Macedonia, he himself stayed some time in Asia.\lebnote{A reference to the Roman province of Asia (modern Asia Minor)}
\verse Now there happened at that time no little disturbance concerning the Way.
\verse For someone \textit{named}\lebnote{Literally “by name”} Demetrius, a silversmith who made silver replicas of the temple of Artemis, was bringing no little business to the craftsmen.
\verse \textit{These}\lebnote{Literally “whom”} he gathered together, and the workers occupied with such things, and\lebnote{:|NP|:*Here “and” is supplied because the previous participle (“gathered together”) has been translated as a finite verb} said, “Men, you know that from this business \textit{we get our prosperity},\lebnote{Literally “prosperity is to us”}
\verse and you see and hear that not only in Ephesus but in almost all of Asia\lebnote{A reference to the Roman province of Asia (modern Asia Minor)} this man Paul has persuaded and\lebnote{:|NP|:*Here “and” is supplied because the previous participle (“has persuaded”) has been translated as a finite verb} turned away a large crowd by\lebnote{:|NP|:*Here “by” is supplied as a component of the participle (“saying”) which is understood as means} saying that the gods made by hands are not gods.
\verse So not only is there a danger this line of business of ours will come into disrepute, but also the temple of the great goddess Artemis will be regarded as nothing—and she is about to be brought down even from her grandeur, she whom the whole of Asia\lebnote{A reference to the Roman province of Asia (modern Asia Minor)} and the entire world worship!”
\verse And when they\lebnote{:|NP|:*Here “when” is supplied as a component of the participle (“heard”) which is understood as temporal} heard this\lebnote{:|NP|:*Here the direct object is supplied from context in the English translation} and became full of rage, they began to shout,\lebnote{:|NP|:*The imperfect tense has been translated as ingressive here (“began to shout”)} saying, “Great is Artemis of the Ephesians!”
\verse And the city was filled with the tumult, and with one purpose they rushed into the theater, seizing Gaius and Aristarchus, Macedonians who were traveling companions of Paul.
\verse But when\lebnote{:|NP|:*Here “when” is supplied as a component of the temporal genitive absolute participle (“wanted”)} Paul wanted to enter into the popular assembly, the disciples would not let him.
\verse And even some of the Asiarchs\lebnote{Or “provincial authorities”} who were his friends sent word\lebnote{:|NP|:*Here the direct object is supplied from context in the English translation} to him and\lebnote{:|NP|:*Here “and” is supplied because the previous participle (“sent”) has been translated as a finite verb} were urging him\lebnote{:|NP|:*Here the direct object is supplied from context in the English translation} not to risk himself by going into the theater.
\verse So some were shouting one thing\lebnote{:|NP|:*Here the direct object is supplied from context in the English translation} and some another, for the assembly was in confusion, and the majority did not know \textit{why}\lebnote{Literally “on account of what”} they had assembled.
\verse And some of the crowd advised\lebnote{Or “concluded it was about”} Alexander, when\lebnote{:|NP|:*Here “when” is supplied as a component of the temporal genitive absolute participle (“put … forward”)} the Jews put him forward. But Alexander, motioning with his\lebnote{:|NP|:*Literally “the”; the Greek article is used here as a possessive pronoun} hand, was wanting to defend himself to the popular assembly.
\verse But when they\lebnote{:|NP|:*Here “when” is supplied as a component of the participle (“recognized”) which is understood as temporal} recognized that he was a Jew, they were shouting with one voice from all of them for about two hours, “Great is Artemis of the Ephesians!”
\verse And when\lebnote{:|NP|:*Here “when” is supplied as a component of the participle (“had quieted”) which is understood as temporal} the city secretary had quieted the crowd, he said, “Ephesian men, for who is there among men who does not know the Ephesian city is honorary temple keeper of the great Artemis and of her\lebnote{:|NP|:*Literally “the”; the Greek article is used here as a possessive pronoun} image fallen from heaven?
\verse Therefore because\lebnote{:|NP|:*Here “because” is supplied as a component of the causal genitive absolute participle (“are”)} these things are undeniable, it is necessary that you be quiet and do nothing rash!
\verse For you have brought these men here who are neither temple robbers nor blasphemers of our goddess.
\verse If then Demetrius and the craftsmen who are with him have a complaint against anyone, the court days are observed and there are proconsuls—let them bring charges against one another!
\verse But if you desire anything further, it will be settled in the lawful assembly.
\verse For indeed we are in danger of being accused of rioting concerning today, since there\lebnote{:|NP|:*Here “since” is supplied as a component of the causal genitive absolute participle (“is”)} is no cause in relation to which we will be able to give an account concerning this disorderly gathering!” And when he\lebnote{:|NP|:*Here “when” is supplied as a component of the participle (“had said”) which is understood as temporal} had said these things, he dismissed the assembly.\lebnote{Verse 41 in the English Bible is included as part of v. 40 in the standard editions of the Greek text}
\end{biblechapter}

\begin{biblechapter} % Acts 20
\verseWithHeading{Paul Travels Through Macedonia and Greece} Now after the turmoil had ceased, Paul summoned\lebnote{:|NP|:*Here the participle (“summoned”) has been translated as a finite verb in keeping with English style} the disciples, and after\lebnote{:|NP|:*Here “after” is supplied as a component of the participle (“encouraging”) which is understood as temporal} encouraging them,\lebnote{:|NP|:*Here the direct object is supplied from context in the English translation} he said farewell and\lebnote{:|NP|:*Here “and” is supplied because the previous participle (“departed”) has been translated as a finite verb} departed to travel to Macedonia.
\verse And after he\lebnote{:|NP|:*Here “after” is supplied as a component of the participle (“had gone through”) which is understood as temporal} had gone through those regions and encouraged them \textit{at length},\lebnote{Literally “with many a word”} he came to Greece
\verse and stayed\lebnote{:|NP|:*Here the participle (“stayed”) has been translated as a finite verb in keeping with English style} three months. Because\lebnote{:|NP|:*Here “because” is supplied as a component of the causal genitive absolute participle (“was made”)} a plot was made against him by the Jews as he\lebnote{:|NP|:*Here “as” is supplied as a component of the participle (“was about to”) which is understood as temporal} was about to set sail for Syria, he came to a decision to return through Macedonia.
\verse And Sopater son of Pyrrhus from Berea, and Aristarchus and Secundus from Thessalonica, and Gaius from Derbe, and Timothy, and Tychicus and Trophimus from Asia, were accompanying him.
\verse And these had gone on ahead and\lebnote{:|NP|:*Here “and” is supplied because the previous participle (“had gone on ahead”) has been translated as a finite verb} were waiting for us in Troas.
\verse And we sailed away from Philippi after the days of Unleavened Bread and came to them at Troas within five days, where we stayed seven days.
\verseWithHeading{Eutychus Falls from a Window} And on the first day of the week, when\lebnote{:|NP|:*Here “when” is supplied as a component of the temporal genitive absolute participle (“had assembled”)} we had assembled to break bread, Paul began conversing\lebnote{:|NP|:*The imperfect tense has been translated as ingressive here (“began conversing”)} with them, because he\lebnote{:|NP|:*Here “because” is supplied as a component of the participle (“was going to”) which is understood as causal} was going to leave on the next day, and he extended his\lebnote{:|NP|:*Literally “the”; the Greek article is used here as a possessive pronoun} message until midnight.
\verse Now there were quite a few lamps in the upstairs room where we were gathered.
\verse And a certain young man \textit{named}\lebnote{Literally “by name”} Eutychus who was sitting in the window was sinking into a deep sleep while\lebnote{:|NP|:*Here “while” is supplied as a component of the temporal genitive absolute participle (“was conversing”)} Paul was conversing at length. Being overcome by sleep, he fell down from the third story and was picked up dead.
\verse But Paul went down and\lebnote{:|NP|:*Here “and” is supplied because the previous participle (“went down”) has been translated as a finite verb} threw himself on him, and putting his arms around him,\lebnote{:|NP|:*Here the direct object is supplied from context in the English translation} said, “Do not be distressed, for his life is in him.”
\verse So he went up and broke bread,\lebnote{:|NP|:*This participle and the previous one (“went up”) have been translated as finite verbs in keeping with English style} and when he\lebnote{:|NP|:*Here “when” is supplied as a component of the participle (“had eaten”) which is understood as temporal} had eaten and talked for a long time, until dawn, then he departed.
\verse And they led the youth away alive, and \textit{were greatly}\lebnote{Literally “were not moderately”} comforted.
\verseWithHeading{The Voyage to Miletus} But we went on ahead to the ship and\lebnote{:|NP|:*Here “and” is supplied because the previous participle (“went on ahead”) has been translated as a finite verb} put out to sea for Assos, intending to take Paul on board there. For having made arrangements in this way, he himself was intending to travel by land.
\verse And when he met us at Assos, we took him on board and\lebnote{:|NP|:*Here “and” is supplied because the previous participle (“took … on board”) has been translated as a finite verb} went to Mitylene.
\verse And we sailed from there on the next day, and\lebnote{:|NP|:*Here “and” is supplied because the previous participle (“sailed”) has been translated as a finite verb} arrived opposite Chios. And on the next day we approached Samos, and on the following day we came to Miletus.
\verse For Paul had decided to sail past Ephesus so that he would not be having to spend time in Asia.\lebnote{A reference to the Roman province of Asia (modern Asia Minor)} For he was hurrying if it could be possible for him to be in Jerusalem on the day of Pentecost.
\verseWithHeading{Paul’s Farewell to the Ephesian Elders} And from Miletus he sent word\lebnote{:|NP|:*Here the direct object is supplied from context in the English translation} to Ephesus and\lebnote{:|NP|:*Here “and” is supplied because the previous participle (“sent”) has been translated as a finite verb} summoned the elders of the church.
\verse And when they came to him, he said to them, “You know from the first day on which I set foot in Asia\lebnote{A reference to the Roman province of Asia (modern Asia Minor)} how I was the whole time with you—
\verse serving the Lord with all humility and with tears, and with the trials that happened to me through the plots of the Jews—
\verse how I did not shrink from proclaiming to you anything that would be profitable, and from teaching you in public and from house to house,
\verse testifying both to Jews and to Greeks with respect to repentance toward God and faith in our Lord Jesus.
\verse “And now behold, bound by the Spirit I am traveling to Jerusalem, not knowing the things that will happen to me \textit{there},\lebnote{Literally “in it”}
\verse except that the Holy Spirit testifies to me in town after town, saying that bonds and persecutions await me.
\verse But I consider my\lebnote{:|NP|:*Literally “the”; the Greek article is used here as a possessive pronoun} life as worth \textit{nothing}\lebnote{Or “not a single word”; literally “not any thing”} to myself, in order to finish my mission and the ministry that I received from the Lord Jesus, to testify to the gospel of the grace of God.
\verse “And now behold, I know that all of you, among whom I went about proclaiming the kingdom, will see my face no more.
\verse Therefore I testify to you on this very day that I am guiltless of the blood of all of you,\lebnote{:|NP|:*The words “of you” are not in the Greek text but are implied}
\verse for I did not shrink from proclaiming to you the whole purpose of God.
\verse Be on guard for yourselves and for all the flock among which the Holy Spirit has appointed you as overseers, to shepherd the church of God which he obtained through the blood of his own Son.\lebnote{Or “through his own blood”; the Greek construction can be taken either way, with “Son” implied if the meaning is “through the blood of his own”}
\verse I know that after my departure fierce wolves will come in among you, not sparing the flock.
\verse And from among you yourselves men will arise, speaking \textit{perversions of the truth}\lebnote{Literally “crooked things”} in order to draw away the disciples after them.
\verse Therefore be on the alert, remembering that night and day for three years I did not stop warning each one of you\lebnote{:|NP|:*The words “of you” are not in the Greek text but are implied} with tears.
\verse “And now I entrust you to God and to the message of his grace, which is able to build you\lebnote{:|NP|:*Here the direct object is supplied from context in the English translation} up and to give you\lebnote{:|NP|:*Here the direct object is supplied from context in the English translation} the inheritance among all those who are sanctified.
\verse I have desired no one’s silver or gold or clothing!
\verse You yourselves know that these hands served to meet\lebnote{:|NP|:*The words “to meet” are not in the Greek text but are supplied for clarity} my needs and the needs of\lebnote{:|NP|:*The words “the needs of” are supplied in keeping with English style to maintain the parallelism with the earlier phrase “my needs”} those who were with me.
\verse I have shown you with respect to all things that by\lebnote{:|NP|:*Here “by” is supplied as a component of the participle (“working hard”) which is understood as means} working hard in this way it is necessary to help those who are in need, and to remember the words of the Lord Jesus that he himself said, “It is more blessed to give than to receive.”\lebnote{Perhaps an allusion to Matt 10:8; these exact words are not found in the NT gospels}
\verse And when he\lebnote{:|NP|:*Here “when” is supplied as a component of the participle (“had said”) which is understood as temporal} had said these things, he fell to his knees and\lebnote{:|NP|:*Here “and” is supplied because the previous participle (“fell to”) has been translated as a finite verb} prayed with them all.
\verse And there was considerable weeping by all, and \textit{hugging}\lebnote{Literally “falling on the neck of”} Paul, they kissed him,
\verse especially distressed at the statement that he had said, that they were going to see his face no more. And they accompanied him to the ship.
\end{biblechapter}

\begin{biblechapter} % Acts 21
\verseWithHeading{Paul Travels on to Jerusalem} And it happened that after we tore ourselves away\lebnote{:|NP|:*Here the participle (“tore ourselves away”) has been translated as a finite verb in keeping with English style} from them, we put out to sea, and\lebnote{:|NP|:*Here “and” is supplied because the previous infinitive (“put out to sea”) has been translated as a finite verb} running a straight course we came to Cos and on the next day to Rhodes, and from there to Patara.
\verse And finding a ship that was crossing over to Phoenicia, we went aboard and\lebnote{:|NP|:*Here “and” is supplied because the previous participle (“went aboard”) has been translated as a finite verb} put out to sea.
\verse And after we\lebnote{:|NP|:*Here “after” is supplied as a component of the participle (“sighted”) which is understood as temporal} sighted Cyprus and left it behind \textit{on the port side},\lebnote{Literally “left”} we sailed to Syria and arrived at Tyre, because the ship was to unload its\lebnote{:|NP|:*Literally “the”; the Greek article is used here as a possessive pronoun} cargo there.
\verse And we stayed there seven days after we\lebnote{:|NP|:*Here “after” is supplied as a component of the participle (“found”) which is understood as temporal} found the disciples, who kept telling Paul through the Spirit not to set foot in Jerusalem.
\verse And it happened that when our days were over, we departed and\lebnote{:|NP|:*Here “and” is supplied because the previous participle (“departed”) has been translated as a finite verb} went on our way, while\lebnote{:|NP|:*Here “while” is supplied as a component of the temporal genitive absolute participle (“accompanied”)} all of them accompanied us, together with their\lebnote{:|NP|:*The word “their” is not in the Greek text but is implied} wives and children, as far as outside the city. And after\lebnote{:|NP|:*Here “after” is supplied as a component of the participle (“falling to”) which is understood as temporal} falling to our knees on the beach and\lebnote{:|NP|:*Here “and” is supplied to join this and the previous participle (“falling to”) in keeping with English style} praying,
\verse we said farewell to one another and embarked in the ship, and they returned to their own homes.
\verse And when\lebnote{:|NP|:*Here “when” is supplied as a component of the participle (“had completed”) which is understood as temporal} we had completed the voyage from Tyre, we arrived at Ptolemais. And after we\lebnote{:|NP|:*Here “after” is supplied as a component of the participle (“greeted”) which is understood as temporal} greeted the brothers, we stayed one day with them.
\verse And on the next day we departed and\lebnote{:|NP|:*Here “and” is supplied to join this and the previous participle (“departed”) in keeping with English style} came to Caesarea, and entered into the house of Philip the evangelist, who was one of the seven, and\lebnote{:|NP|:*Here “and” is supplied to join this and the previous participle (“was”) in keeping with English style} stayed with him.
\verse (\textit{Now this man had}\lebnote{Literally “now to this man were”} four virgin daughters who prophesied.)
\verse And while we\lebnote{:|NP|:*Here “while” is supplied as a component of the temporal genitive absolute participle (“were staying”)} were staying there\lebnote{:|NP|:*Here the direct object is supplied from context in the English translation} many days, a certain prophet \textit{named}\lebnote{Literally “by name”} Agabus came down from Judea.
\verse And he came to us and took Paul’s belt. Tying up his own feet and hands, he said, “This is what the Holy Spirit says: ‘In this way the Jews in Jerusalem will tie up the man whose belt this is, and will deliver him\lebnote{:|NP|:*Here the direct object is supplied from context in the English translation} into the hands of the Gentiles.’ ”
\verse And when we heard these things, both we and the local residents urged him not to go up to Jerusalem.
\verse Then Paul replied, “What are you doing weeping and breaking my heart? For I am ready not only to be tied up, but even to die in Jerusalem for the name of the Lord Jesus!”
\verse And because\lebnote{:|NP|:*Here “because” is supplied as a component of the causal genitive absolute participle (“be persuaded”)} he would not be persuaded, we remained silent, saying, “The will of the Lord be done.”
\verse So after these days we got ready and\lebnote{:|NP|:*Here “and” is supplied because the previous participle (“got ready”) has been translated as a finite verb} went up to Jerusalem.
\verse And some of the disciples from Caesarea also traveled together with us, bringing us\lebnote{:|NP|:*Here the direct object is supplied from context in the English translation} to a certain Mnason of Cyprus, a disciple of long standing,\lebnote{Or perhaps “one of the original disciples”} with whom we were to be entertained as guests.
\verseWithHeading{Paul Visits the Leaders of the Jerusalem Church} And when\lebnote{:|NP|:*Here “when” is supplied as a component of the temporal genitive absolute participle (“came”)} we came to Jerusalem, the brothers welcomed us gladly.
\verse And on the next day Paul went in with us to James, and all the elders were present.
\verse And after\lebnote{:|NP|:*Here “after” is supplied as a component of the participle (“greeting”) which is understood as temporal} greeting them, he began to relate\lebnote{:|NP|:*The imperfect tense has been translated as ingressive here (“began to relate”)} \textit{one after the other}\lebnote{Literally “with respect to each one”} the things which God had done among the Gentiles through his ministry.
\verse And when\lebnote{:|NP|:*Here “when” is supplied as a component of the participle (“heard”) which is understood as temporal} they heard this,\lebnote{:|NP|:*Here the direct object is supplied from context in the English translation} they began to glorify\lebnote{:|NP|:*The imperfect tense has been translated as ingressive here (“began to glorify”)} God. And they said to him, “You see, brother, how many ten thousands there are among the Jews who have believed, and they are all zealous adherents of the law.
\verse And they have been informed about you that you are teaching all the Jews who are among the Gentiles the abandonment of Moses, telling them not to circumcise their\lebnote{:|NP|:*Literally “the”; the Greek article is used here as a possessive pronoun} children or to live according to our\lebnote{Literally “according to the”; the Greek article is used here as a possessive pronoun} customs.
\verse What then \textit{is to be done}?\lebnote{Literally “is it”} Doubtless they will all hear that you have come!
\verse Therefore do this that we tell you: \textit{we have}\lebnote{Literally “there are to us”} four men who have taken a vow upon themselves.\lebnote{Some manuscripts have “on themselves”}
\verse Take these men and\lebnote{:|NP|:*Here “and” is supplied because the previous participle (“take”) has been translated as a finite verb} purify yourself along with them and \textit{pay their expenses}\lebnote{Literally “spend on them”} so that they can shave their\lebnote{:|NP|:*Literally “the”; the Greek article is used here as a possessive pronoun} heads, and everyone will know that the things which they had been informed about you are nothing, but you yourself also agree with observing the law.
\verse But concerning the Gentiles who have believed, we have written a letter after\lebnote{:|NP|:*Here “after” is supplied as a component of the participle (“deciding”) which is understood as temporal} deciding they should avoid food sacrificed to idols and blood and what has been strangled and sexual immorality.”
\verse Then Paul took along the men on the next day, and\lebnote{:|NP|:*Here “and” has been supplied in keeping with English style} after he\lebnote{:|NP|:*Here “after” is supplied as a component of the participle (“had purified”) which is understood as temporal} had purified himself together with them, he entered into the temple courts,\lebnote{:|NP|:*Here “courts” is supplied to distinguish this area from the interior of the temple building itself} announcing the completion of the days of purification until \textit{the time}\lebnote{Literally “which time”} the offering would be presented on behalf of each one of them.
\verseWithHeading{Paul Arrested in the Temple Courts} But when the seven days were about to be completed, the Jews from Asia\lebnote{A reference to the Roman province of Asia (modern Asia Minor)} who had seen him in the temple courts\lebnote{:|NP|:*Here “courts” is supplied to distinguish this area from the interior of the temple building itself} stirred up the whole crowd and laid hands on him,
\verse shouting, “Israelite men, help! This is the man who is teaching everyone everywhere against the people and the law and this place! And furthermore he also brought Greeks into the temple, and has defiled this holy place!”
\verse (For they had previously seen Trophimus the Ephesian in the city with him, whom they thought that Paul had brought into the temple.)
\verse And the whole city was stirred up, and the people came running together, and they seized Paul and\lebnote{:|NP|:*Here “and” is supplied because the previous participle (“seized”) has been translated as a finite verb} dragged him outside of the temple courts,\lebnote{:|NP|:*Here “courts” is supplied to distinguish this area from the interior of the temple building itself} and immediately the doors were shut.
\verse And as they\lebnote{:|NP|:*Here “as” is supplied as a component of the temporal genitive absolute participle (“were seeking”)} were seeking to kill him, a report came up to the military tribune of the cohort that all Jerusalem was in confusion.
\verse \textit{He}\lebnote{Literally “who”} immediately took along soldiers and centurions and\lebnote{:|NP|:*Here “and” is supplied because the previous participle (“took along”) has been translated as a finite verb} ran down to them. And when\lebnote{:|NP|:*Here “when” is supplied as a component of the participle (“saw”) which is understood as temporal} they saw the military tribune and the soldiers, they stopped beating Paul.
\verse Then the military tribune came up and\lebnote{:|NP|:*Here “and” is supplied because the previous participle (“came up”) has been translated as a finite verb} arrested him and ordered him\lebnote{:|NP|:*Here the direct object is supplied from context in the English translation} to be bound with two chains, and inquired who he was and what it was that he had done.
\verse But some in the crowd were shouting one thing and others another, and because\lebnote{:|NP|:*Here “because” is supplied as a component of the causal genitive absolute participle (“able”)} he was not able to find out the truth on account of the commotion, he gave orders to bring him into the barracks.\lebnote{Or “headquarters”}
\verse And when he came to the steps, it happened that he had to be carried by the soldiers on account of the violence of the crowd,
\verse for the crowd of people was following them,\lebnote{:|NP|:*Here the direct object is supplied from context in the English translation} shouting, “Away with him!”
\verseWithHeading{Paul Permitted to Address the Crowd} And as he\lebnote{:|NP|:*Here “as” is supplied as a component of the participle (“was about to”) which is understood as temporal} was about to be brought into the barracks,\lebnote{Or “headquarters”} Paul said to the military tribune, “Is it permitted for me to say something to you?” And he said, “Do you know Greek?
\verse Then you are not the Egyptian who before these days raised a revolt and led out into the wilderness the four thousand men of the Assassins?”\lebnote{Also known as the “Sicarii” from the Latin word “sicarius” = dagger, after the short dagger used to assassinate political opponents}
\verse But Paul said, “I am a Jewish man from Tarsus in Cilicia, a citizen of no unimportant city. Now I ask you, allow me to speak to the people.”
\verse So when\lebnote{:|NP|:*Here “when” is supplied as a component of the temporal genitive absolute participle (“permitted”)} he permitted him,\lebnote{:|NP|:*Here the direct object is supplied from context in the English translation} Paul, standing there on the steps, motioned with his\lebnote{:|NP|:*Literally “the”; the Greek article is used here as a possessive pronoun} hand to the people. And when there\lebnote{:|NP|:*Here “when” is supplied as a component of the temporal genitive absolute participle (“was”)} was a great silence, he addressed them\lebnote{:|NP|:*Here the direct object is supplied from context in the English translation} in the Aramaic language, saying,
\end{biblechapter}

\begin{biblechapter} % Acts 22
\verse “Men—brothers and fathers—listen to my defense to you now!”
\verse And when they\lebnote{:|NP|:*Here “when” is supplied as a component of the participle (“heard”) which is understood as temporal} heard that he was addressing them in the Aramaic language, \textit{they became even more silent}.\lebnote{Literally “they showed even more silence”} And he said,
\verse “I am a Jewish man born in Tarsus in Cilicia, but brought up in this city at the feet of Gamaliel, educated according to the exactness of the law received from our fathers, being zealous for God, just as all of you are today.
\verse \textit{I}\lebnote{Literally “who”} persecuted this Way to the death, tying up and delivering to prison both men and women,
\verse as indeed the high priest and the whole council of elders can testify about me, from whom also I received letters to the brothers in Damascus, and\lebnote{:|NP|:*Here “and” is supplied because the previous participle (“received”) has been translated as a finite verb} was traveling there\lebnote{:|NP|:*Here the direct object is supplied from context in the English translation} to lead away those who were there also tied up to Jerusalem so that they could be punished.
\verseWithHeading{Paul Tells of His Conversion on the Damascus Road} “And it happened that as\lebnote{:|NP|:*Here “as” is supplied as a component of the participle (“was traveling”) which is understood as temporal} I was traveling and approaching Damascus around noon, suddenly a very bright light from heaven flashed around me,
\verse and I fell to the ground and heard a voice saying to me, ‘Saul, Saul, why are you persecuting me?’
\verse And I answered, ‘Who are you, Lord?’ And he said to me, ‘I am Jesus the Nazarene whom you are persecuting.’
\verse (Now those who were with me saw the light but did not hear the voice of the one who was speaking to me.)
\verse So I said, ‘What should I do, Lord?’ And the Lord said to me, ‘Get up and\lebnote{:|NP|:*Here “and” is supplied because the previous participle (“get up”) has been translated as a finite verb} proceed to Damascus, and there it will be told to you about all the things that have been appointed for you to do.’
\verse And as I could not see as a result of the brightness of that light, I arrived in Damascus led by the hand of those who were with me.
\verse And a certain Ananias, a devout man according to the law, well spoken of by all the Jews who live there,
\verse came to me and stood by me\lebnote{:|NP|:*Here the direct object is supplied from context in the English translation} and\lebnote{:|NP|:*Here “and” is supplied because the previous participle (“stood by”) has been translated as a finite verb} said to me, ‘Brother Saul, regain your sight!’ And at that same time I looked up at him and saw him.\lebnote{:|NP|:*The words “and saw him” are not in the Greek text but are supplied in the translation for clarity}
\verse And he said, ‘The God of our fathers has appointed you to know his will, and to see the Righteous One and to hear a voice from his mouth,
\verse because you will be a witness for him\lebnote{Or “to him”} to all people of what you have seen and heard.
\verse And now why are you delaying? Get up, be baptized, and wash away your sins, calling on his name!’
\verse “And it happened that when\lebnote{:|NP|:*Here “when” is supplied as a component of the participle (“returned”) which is understood as temporal} I returned to Jerusalem and I was praying in the temple courts,\lebnote{:|NP|:*Here “courts” is supplied to distinguish this area from the interior of the temple building itself} I was in a trance,
\verse and saw him saying to me, ‘Hurry and depart \textit{quickly}\lebnote{Literally “with quickness”} from Jerusalem, because they will not accept your testimony about me.’
\verse And I said, ‘Lord, they themselves know that from synagogue to synagogue I was imprisoning and beating those who believed in you.
\verse And when the blood of your witness Stephen was being shed, I myself also was standing near and was approving, and was guarding the cloaks of those who were killing him.’
\verse And he said to me, ‘Go, because I will send you far away to the Gentiles!’ ”
\verseWithHeading{Paul Questioned by the Military Tribune} Now they were listening to him until this word, and they raised their voices, saying, “Away with such a man from the earth! For it is not fitting for him to live!”
\verse And while\lebnote{:|NP|:*Here “while” is supplied as a component of the temporal genitive absolute participle (“were screaming”)} they were screaming and throwing off their\lebnote{:|NP|:*Literally “the”; the Greek article is used here as a possessive pronoun} cloaks and throwing dust into the air,
\verse the military tribune ordered him to be brought into the barracks,\lebnote{Or “headquarters”} saying he was to be examined with a lash so that he could find out for what reason they were crying out against him in this way.
\verse But when they had stretched him out for the lash,\lebnote{Or “with straps” (in order to lash him)} Paul said to the centurion standing there, “Is it permitted for you to flog a man who is a Roman citizen and uncondemned?”
\verse And when\lebnote{:|NP|:*Here “when” is supplied as a component of the participle (“heard”) which is understood as temporal} the centurion heard this,\lebnote{:|NP|:*Here the direct object is supplied from context in the English translation} he went to the military tribune and\lebnote{:|NP|:*Here “and” is supplied because the previous participle (“went to”) has been translated as a finite verb} reported it,\lebnote{:|NP|:*Here the direct object is supplied from context in the English translation} saying, “What are you about to do? For this man is a Roman citizen!”
\verse So the military tribune came and\lebnote{:|NP|:*Here “and” is supplied because the previous participle (“came”) has been translated as a finite verb} said to him, “Tell me, are you a Roman citizen?” And he said, “Yes.”
\verse And the military tribune replied, “I acquired this citizenship for a large sum of money.” And Paul said, “But I indeed was born a citizen.\lebnote{:|NP|:*The words “a citizen” are not in the Greek text but are implied}
\verse Then immediately those who were about to examine him kept away from him, and the military tribune also was afraid when he\lebnote{:|NP|:*Here “when” is supplied as a component of the participle (“realized”) which is understood as temporal} realized that he was a Roman citizen and that \textit{he had tied him up}.\lebnote{Literally “he was having tied him up”}
\verse But on the next day, because he\lebnote{:|NP|:*Here “because” is supplied as a component of the participle (“wanted”) which is understood as causal} wanted to know the true reason why he was being accused by the Jews, he released him and ordered the chief priests and the whole Sanhedrin to assemble, and he brought down Paul and\lebnote{:|NP|:*Here “and” is supplied because the previous participle (“brought down”) has been translated as a finite verb} had him\lebnote{:|NP|:*Here the direct object is supplied from context in the English translation} stand before them.
\end{biblechapter}

\begin{biblechapter} % Acts 23
\verseWithHeading{Paul Before the Sanhedrin} And looking intently at the Sanhedrin, Paul said, “Men and brothers, I have lived my life in all good conscience before God to this day.”
\verse So the high priest Ananias ordered those standing near him to strike his mouth.
\verse Then Paul said to him, “God is going to strike you, you whitewashed wall! And are you sitting there judging me according to the law, and acting contrary to the law do you order me to be struck?”
\verse And those who stood nearby said, “Are you reviling the high priest of God?”
\verse And Paul said, “I did not know, brothers, that he was high priest. For it is written, ‘You must not speak evil of a ruler of your people.’ ”\lebnote{:|NP|:A quotation from Exod 22:28}
\verse Now when\lebnote{:|NP|:*Here “when” is supplied as a component of the participle (“realized”) which is understood as temporal} Paul realized that one part were Sadducees and the other Pharisees, he shouted out in the Sanhedrin, “Men and brothers! I am a Pharisee, a son of Pharisees! I am being judged concerning the hope and the resurrection of the dead!”
\verse And when\lebnote{:|NP|:*Here “when” is supplied as a component of the temporal genitive absolute participle (“said”)} he said this, a dispute developed between the Pharisees and Sadducees, and the assembly was divided.
\verse (For the Sadducees say there is no resurrection or angel or spirit, but the Pharisees acknowledge them all.)
\verse And there was loud shouting, and some of the scribes from the party of the Pharisees stood up and\lebnote{:|NP|:*Here “and” is supplied because the previous participle (“stood up”) has been translated as a finite verb} contended sharply, saying, “We find nothing wrong with this man! But what if a spirit or an angel has spoken to him?”
\verse And when\lebnote{:|NP|:*Here “when” is supplied as a component of the temporal genitive absolute participle (“became”)} the dispute became severe, the military tribune, fearing lest Paul be torn apart by them, ordered the detachment to go down, take him away from their midst, and bring him\lebnote{:|NP|:*Here the direct object is supplied from context in the English translation} into the barracks.\lebnote{Or “headquarters”}
\verse And the next night the Lord stood by him and\lebnote{:|NP|:*Here “and” is supplied because the previous participle (“stood by”) has been translated as a finite verb} said, “Have courage, for as you have testified about me in Jerusalem, so you must also testify in Rome.”
\verseWithHeading{A Conspiracy to Kill Paul} And when it\lebnote{:|NP|:*Here “when” is supplied as a component of the temporal genitive absolute participle (“was”)} was day, the Jews made a conspiracy and\lebnote{:|NP|:*Here “and” is supplied because the previous participle (“made”) has been translated as a finite verb} bound themselves under a curse, saying they would\lebnote{Literally “saying neither to eat nor to drink”; the words “they would” are supplied for smoother English style} neither eat nor drink until they had killed Paul.
\verse Now there were more than forty who had made this conspiracy,
\verse who went to the chief priests and the elders and\lebnote{:|NP|:*Here “and” is supplied because the previous participle (“went to”) has been translated as a finite verb} said, “We have bound ourselves under a curse to partake of nothing until we have killed Paul.
\verse Therefore, now you along with the Sanhedrin explain to the military tribune that he should bring him down to you, as if you were going to determine more accurately the things concerning him. And we are ready to do away with him before he comes near.”
\verse But when\lebnote{:|NP|:*Here “when” is supplied as a component of the participle (“heard about”) which is understood as temporal} the son of Paul’s sister heard about the ambush, he came and entered into the barracks\lebnote{Or “headquarters”} and\lebnote{:|NP|:*Here “and” is supplied because the two previous participles (“came” and “entered”) have been translated as finite verbs} reported it\lebnote{:|NP|:*Here the direct object is supplied from context in the English translation} to Paul.
\verse So Paul called one of the centurions and\lebnote{:|NP|:*Here “and” is supplied because the previous participle (“called”) has been translated as a finite verb} said, “Bring this young man to the military tribune, because he has something to report to him.”
\verse So he took him and\lebnote{:|NP|:*Here “and” is supplied because the previous participle (“took”) has been translated as a finite verb} brought him\lebnote{:|NP|:*Here the direct object is supplied from context in the English translation} to the military tribune and said, “The prisoner Paul called me and\lebnote{:|NP|:*Here “and” is supplied because the previous participle (“called”) has been translated as a finite verb} asked me\lebnote{:|NP|:*Here the direct object is supplied from context in the English translation} to bring this young man to you because he\lebnote{:|NP|:*Here “because” is supplied as a component of the participle (“has”) which is understood as causal} has something to tell you.”
\verse And the military tribune, taking hold of his hand and withdrawing privately, asked, “What is it that you have to report to me?”
\verse And he said, “The Jews have agreed to ask you that you bring Paul down to the Sanhedrin tomorrow, as if they were going to inquire somewhat more accurately concerning him.
\verse You therefore do not be persuaded by them, because more than forty men of \textit{their number}\lebnote{Literally “them”} are lying in wait for him, who have bound themselves under a curse neither to eat nor to drink until they have done away with him. And now they are ready, waiting for \textit{you to agree}.”\lebnote{Literally “the assurance of agreement from you”}
\verse So the military tribune sent the young man away, directing him,\lebnote{:|NP|:*Here the direct object is supplied from context in the English translation} “Tell no one that you have revealed these things to me.”
\verse And he summoned two of the centurions and\lebnote{:|NP|:*Here “and” is supplied because the previous participle (“summoned”) has been translated as a finite verb} said, “Make ready from the third hour of the night two hundred soldiers and seventy horsemen and two hundred spearmen,\lebnote{A word of uncertain meaning, probably a military technical term} in order that they may proceed as far as Caesarea.
\verse And provide mounts so that they can put Paul on them and\lebnote{:|NP|:*Here “and” is supplied because the previous participle (“put … on”) has been translated as a finite verb} bring him\lebnote{:|NP|:*Here the direct object is supplied from context in the English translation} safely to Felix the governor.”
\verse \textit{He wrote}\lebnote{Literally “writing”} a letter that had this form:\lebnote{Or “content”}
\verse Claudius Lysias. To his excellency Governor Felix. Greetings!
\verse This man was seized by the Jews and was about to be killed by them when I\lebnote{:|NP|:*Here “when” is supplied as a component of the participle (“came upon”) which is understood as temporal} came upon them\lebnote{:|NP|:*Here the direct object is supplied from context in the English translation} with the detachment and\lebnote{:|NP|:*Here “and” is supplied because the previous participle (“came upon”) has been translated as a finite verb} rescued him,\lebnote{:|NP|:*Here the direct object is supplied from context in the English translation} because I\lebnote{:|NP|:*Here “because” is supplied as a component of the participle (“learned”) which is understood as causal} learned that he was a Roman citizen.
\verse And because I\lebnote{:|NP|:*Here “because” is supplied as a component of the participle (“wanted”) which is understood as causal} wanted to know the charge for which they were accusing him, I brought him\lebnote{:|NP|:*Here the direct object is supplied from context in the English translation} down to their Sanhedrin.\lebnote{Or “council”}
\verse I found \textit{he}\lebnote{Literally “whom”} was accused concerning controversial questions of their law, but having no charge deserving death or imprisonment.
\verse And when it\lebnote{:|NP|:*Here “when” is supplied as a component of the participle (“was made known”) which is understood as temporal} was made known to me there would be a plot against the man, I sent him\lebnote{:|NP|:*Here the direct object is supplied from context in the English translation} to you immediately, also ordering his\lebnote{:|NP|:*Literally “the”; the Greek article is used here as a possessive pronoun} accusers to speak against him\lebnote{Some manuscripts have “to state the charges against him” (literally, “to speak the things against him”)} before you.
\verse Therefore the soldiers, in accordance with \textit{their orders},\lebnote{Literally “what was ordered to them”} took Paul and\lebnote{:|NP|:*Here “and” is supplied because the previous participle (“took”) has been translated as a finite verb} brought him\lebnote{:|NP|:*Here the direct object is supplied from context in the English translation} to Antipatris during the night.
\verse And on the next day they let the horsemen go on with him, and\lebnote{:|NP|:*Here “and” is supplied because the previous participle (“let”) has been translated as a finite verb} they returned to the barracks.\lebnote{Or “headquarters”}
\verse \textit{The horsemen},\lebnote{Literally “who”} when they\lebnote{:|NP|:*Here “when” is supplied as a component of the participle (“came”) which is understood as temporal} came to Caesarea and delivered the letter to the governor, also presented Paul to him.
\verse So after\lebnote{:|NP|:*Here “after” is supplied as a component of the participle (“reading”) which is understood as temporal} reading the letter\lebnote{:|NP|:*Here the direct object is supplied from context in the English translation} and asking what province he was from, and learning that he was from Cilicia,
\verse he said, “I will give you a hearing whenever your accusers arrive also,” giving orders for him to be guarded in the praetorium\lebnote{The “praetorium” of Herod refers to the palace of Herod the Great in Caesarea Maritima} of Herod.
\end{biblechapter}

\begin{biblechapter} % Acts 24
\verseWithHeading{Paul Before Felix at Caesarea Maritima} And after five days the high priest Ananias came down with some elders and an attorney, a certain Tertullus, all of whom brought charges against Paul to the governor.
\verse And when\lebnote{:|NP|:*Here “when” is supplied as a component of the temporal genitive absolute participle (“had been summoned”)} he had been summoned, Tertullus began to accuse him,\lebnote{:|NP|:*Here the direct object is supplied from context in the English translation} saying, “We have experienced\lebnote{:|NP|:*Here this participle (“have experienced”) and the following participle (“are taking place”) have been translated as finite verbs in keeping with English style} much\lebnote{Or “many years,” with “years” understood} peace through you, and reforms are taking place in this nation through your foresight.
\verse Both in every way and everywhere we acknowledge this,\lebnote{:|NP|:*Here the direct object is supplied from context in the English translation} most excellent Felix, with all gratitude.
\verse But so that I may not impose on you for longer, I implore you to hear us briefly with your customary graciousness.
\verse For we have found\lebnote{:|NP|:*Here this participle (“found”) has been translated as a finite verb in keeping with English style} this man to be a public menace and one who causes riots among all the Jews throughout the Roman Empire\lebnote{Literally “the inhabited earth,” but here this is probably rhetorical hyperbole for the Roman Empire, especially since Felix, the Roman governor, is being addressed} and a ringleader of the sect of the Nazarenes,
\verse who even attempted to desecrate the temple, and we arrested \textit{him\lebnote{Literally “whom”}}.\lebnote{Some later manuscripts include the following additional material between v. 6 and v. 8: “and we wanted to judge him according to our law, (24:7) but Lysius the military tribune came and took him from our hands with much violence, (24:8) ordering his accusers to come before you.”}
\verse When\lebnote{:|NP|:*Here “when” is supplied as a component of the participle (“examine”) which is understood as temporal} you yourself examine him\lebnote{:|NP|:*Here the direct object is supplied from context in the English translation} you will be able to find out from \textit{him}\lebnote{Literally “whom”} about all these things of which we are accusing him.”
\verse And the Jews also joined in the attack, asserting these things were so.
\verse And when\lebnote{:|NP|:*Here “when” is supplied as a component of the temporal genitive absolute participle (“gestured”)} the governor gestured for him to speak, Paul replied, “Because I\lebnote{:|NP|:*Here “because” is supplied as a component of the participle (“know”) which is understood as causal} know you have been a judge over this nation for many years, I defend myself cheerfully with respect to the things concerning myself.
\verse You can ascertain that \textit{it has not been more than}\lebnote{Literally “there are not to me more than”} twelve days \textit{since}\lebnote{Literally “from which time”} I went up to Jerusalem to worship.
\verse And neither did they find me arguing with anyone or making a crowd develop in the temple courts\lebnote{:|NP|:*Here “courts” is supplied to distinguish this area from the interior of the temple building itself} nor in the synagogues nor throughout the city.
\verse Nor can they prove the things\lebnote{:|NP|:*Here the direct object is supplied from context in the English translation} to you concerning which they are now accusing me.
\verse But I do confess this to you, that according to the Way (which they call a sect), so I worship the God of our fathers, believing all things that are in accordance with the law and that are written in the prophets,
\verse having a hope in God which these men also themselves await: that there is going to be a resurrection of both the righteous and the unrighteous.
\verse \textit{For this reason}\lebnote{Literally “by this”} also I myself \textit{always}\lebnote{Literally “through everything”} do my best to have a clear conscience toward God and people.
\verse So after many years, I came to practice charitable giving and offerings to my people,\lebnote{Or “nation”}
\verse in which they found me purified in the temple courts,\lebnote{:|NP|:*Here “courts” is supplied to distinguish this area from the interior of the temple building itself} not with a crowd or with a disturbance.
\verse But there are some Jews from Asia\lebnote{A reference to the Roman province of Asia (modern Asia Minor)} who ought to be present before you and bring charges against me,\lebnote{:|NP|:*Here the direct object is supplied from context in the English translation} if they have anything against me,
\verse or these men themselves should say what crime they found when\lebnote{:|NP|:*Here “when” is supplied as a component of the temporal genitive absolute participle (“stood”)} I stood before the Sanhedrin,\lebnote{Or “council”}
\verse other than concerning this one declaration that I shouted while\lebnote{:|NP|:*Here “while” is supplied as a component of the participle (“standing there”) which is understood as temporal} standing there before them: ‘I am being judged before you today concerning the resurrection of the dead!’ ”
\verseWithHeading{Paul Held Awaiting Trial} But Felix, because he\lebnote{:|NP|:*Here “because” is supplied as a component of the participle (“understood”) which is understood as causal} understood the facts concerning the Way more accurately, put them off, saying, “When Lysias the military tribune comes down, I will decide \textit{your case}.”\lebnote{Literally “the case with respect to you”}
\verse He ordered\lebnote{:|NP|:*Here this participle (“ordered”) has been translated as a finite verb in keeping with English style} the centurion for him to be guarded and to have some freedom, and in no way to prevent any of his own people\lebnote{This could refer to either friends or relatives} from serving him.
\verse And after some days, when\lebnote{:|NP|:*Here “when” is supplied as a component of the participle (“arrived”) which is understood as temporal} Felix arrived with his wife Drusilla, who was Jewish, he sent for Paul and listened to him concerning faith in Christ Jesus.
\verse And while\lebnote{:|NP|:*Here “while” is supplied as a component of the temporal genitive absolute participle (“was discussing”)} he was discussing about righteousness and self control and the judgment that is to come, Felix became afraid and\lebnote{:|NP|:*Here “and” is supplied because the previous participle (“became”) has been translated as a finite verb} replied, “Go away for the present, and when I\lebnote{:|NP|:*Here “when” is supplied as a component of the participle (“have”) which is understood as temporal} have an opportunity, I will summon you.”
\verse At the same time he was also hoping that money would be given to him by Paul. For this reason also he sent for him as often as possible and\lebnote{:|NP|:*Here “and” is supplied because the previous participle (“sent for”) has been translated as a finite verb} talked with him.
\verse And when\lebnote{:|NP|:*Here “when” is supplied as a component of the temporal genitive absolute participle (“had passed”)} two years had passed, Felix received as successor Porcius Festus. And because he\lebnote{:|NP|:*Here “because” is supplied as a component of the participle (“wanted”) which is understood as causal} wanted to do a favor for the Jews, Felix left Paul behind \textit{as a prisoner}.\lebnote{Literally “bound”}
\end{biblechapter}

\begin{biblechapter} % Acts 25
\verseWithHeading{Paul Appeals to Caesar} Now when\lebnote{:|NP|:*Here “when” is supplied as a component of the participle (“set foot in”) which is understood as temporal} Festus set foot in the province, after three days he went up to Jerusalem from Caesarea.
\verse And the chief priests and the most prominent men of the Jews brought charges against Paul to him, and were urging him,
\verse asking for a favor against him, that he summon him to Jerusalem, because they\lebnote{:|NP|:*Here “because” is supplied as a component of the participle (“were preparing ”) which is understood as causal} were preparing an ambush to do away with him along the way.
\verse Then Festus replied that Paul was being kept at Caesarea, and he himself was about to go there\lebnote{:|NP|:*Here the direct object is supplied from context in the English translation} in a short time.
\verse So he said, “Let those among you who are prominent go down with me,\lebnote{:|NP|:*Here the direct object is supplied from context in the English translation} and\lebnote{:|NP|:*Here “and” is supplied because the previous participle (“go down with”) has been translated as a finite verb} if there is any wrong in the man, let them bring charges against him.”
\verse And after he\lebnote{:|NP|:*Here “after” is supplied as a component of the participle (“had stayed”) which is understood as temporal} had stayed among them not more than eight or ten days, he went down to Caesarea. On the next day he sat down on the judgment seat and\lebnote{:|NP|:*Here “and” is supplied because the previous participle (“sat down”) has been translated as a finite verb} gave orders for Paul to be brought.
\verse And when\lebnote{:|NP|:*Here “when” is supplied as a component of the temporal genitive absolute participle (“arrived”)} he arrived, the Jews who had come down from Jerusalem stood around him, bringing many and serious charges that they were not able to prove,
\verse while\lebnote{:|NP|:*Here “while” is supplied as a component of the temporal genitive absolute participle (“said in his defense”)} Paul said in his defense, “Neither against the law of the Jews nor against the temple nor against Caesar have I sinned with reference to anything!”
\verse But Festus, because he\lebnote{:|NP|:*Here “because” is supplied as a component of the participle (“wanted”) which is understood as causal} wanted to do a favor for the Jews, answered and\lebnote{:|NP|:*Here “and” is supplied because the previous participle (“answered”) has been translated as a finite verb} said to Paul, “Are you willing to go up to Jerusalem to be tried before me there concerning these things?”
\verse But Paul said, “I am standing before the judgment seat of Caesar, where it is necessary for me to be judged. I have done no wrong to the Jews, as you also know very well.
\verse If then I am doing wrong\lebnote{Or “I am in the wrong”} and have done anything deserving death, I am not trying to avoid\lebnote{:|NP|:*Here the present tense has been translated as conative (“trying to avoid”)} dying. But if there is nothing true of the things which these people are accusing me, no one can give me up to them. I appeal to Caesar!”
\verse Then Festus, after\lebnote{:|NP|:*Here “after” is supplied as a component of the participle (“discussing”) which is understood as temporal} discussing this\lebnote{:|NP|:*Here the direct object is supplied from context in the English translation} with his\lebnote{:|NP|:*Literally “the”; the Greek article is used here as a possessive pronoun} council, replied, “You have appealed to Caesar—to Caesar you will go!”
\verseWithHeading{Festus Asks King Agrippa for Advice} Now after\lebnote{:|NP|:*Here “after” is supplied as a component of the temporal genitive absolute participle (“had passed”)} some days had passed, King Agrippa and Bernice arrived at Caesarea to welcome Festus.
\verse And while they were staying there many days, Festus laid out the case against Paul to the king, saying, “There is a certain man left behind by Felix as a prisoner,
\verse concerning whom when\lebnote{:|NP|:*Here “when” is supplied as a component of the temporal genitive absolute participle (“was”)} I was in Jerusalem the chief priests and the elders of the Jews presented evidence, asking for a sentence of condemnation against him.
\verse To \textit{them}\lebnote{Literally “whom”} I replied that it was not the custom of the Romans to give up any man before the one who had been accused met his\lebnote{:|NP|:*Literally “the”; the Greek article is used here as a possessive pronoun} accusers face to face and received an opportunity for a defense concerning the accusation.
\verse Therefore, when\lebnote{:|NP|:*Here “when” is supplied as a component of the temporal genitive absolute participle (“had assembled”)} they had assembled here, I made\lebnote{:|NP|:*Here this participle (“made”) has been translated as a finite verb in keeping with English style} no delay; on the next day I sat down on the judgment seat and\lebnote{:|NP|:*Here “and” is supplied because the two previous participles (“made” and “sat down”) have been translated as finite verbs} gave orders for the man to be brought.
\verse When they\lebnote{:|NP|:*Here “when” is supplied as a component of the participle (“stood up”) which is understood as temporal} stood up, his\lebnote{:|NP|:*Literally “the”; the Greek article is used here as a possessive pronoun} accusers began bringing\lebnote{:|NP|:*The imperfect tense has been translated as ingressive here (“began bringing”)} no charge concerning \textit{him}\lebnote{Literally “whom”} of the evil deeds that I was suspecting,
\verse but they had some issues with him concerning their own religion, and concerning a certain Jesus, who was dead, whom Paul claimed to be alive.
\verse And because\lebnote{:|NP|:*Here “because” is supplied as a component of the participle (“was at a loss”) which is understood as causal} I was at a loss with regard to the investigation concerning these things, I asked if he was willing to go to Jerusalem and to be judged there concerning these things.
\verse But when\lebnote{:|NP|:*Here “when” is supplied as a component of the temporal genitive absolute participle (“appealed”)} Paul appealed that he be kept under guard for the decision of His Majesty the Emperor, I gave orders for him to be kept under guard until I could send him to Caesar.”
\verse So Agrippa said to Festus, “I want to hear the man myself also.” “Tomorrow,” he said, “you will hear him.”
\verse So on the next day, Agrippa and Bernice came with great pageantry and entered into the audience hall, along with military tribunes and the most prominent men of the city. And when\lebnote{:|NP|:*Here “when” is supplied as a component of the temporal genitive absolute participle (“gave the order”)} Festus gave the order, Paul was brought in.
\verse And Festus said, “King Agrippa and all who are present with us, you see this man about whom the whole population of the Jews appealed to me, both in Jerusalem and here, shouting that he must not live any longer.
\verse But I understood that he had done nothing deserving death himself, and when\lebnote{:|NP|:*Here “when” is supplied as a component of the temporal genitive absolute participle (“appealed to”)} this man appealed to His Majesty the Emperor, I decided to send him.\lebnote{:|NP|:*Here the direct object is supplied from context in the English translation}
\verse I do not have anything definite to write to my\lebnote{:|NP|:*Literally “the”; the Greek article is used here as a possessive pronoun} lord about \textit{him}.\lebnote{Literally “whom”} Therefore I have brought him before you all\lebnote{:|NP|:*Here “all” is supplied in the translation to indicate that the pronoun (“you”) is plural}—and especially before you, King Agrippa—so that after\lebnote{:|NP|:*Here “after” is supplied as a component of the temporal genitive absolute participle (“has taken place”)} this preliminary hearing has taken place, I may have something to write.
\verse For it seems unreasonable to me to send a prisoner and not to indicate the charges against him.”
\end{biblechapter}

\begin{biblechapter} % Acts 26
\verseWithHeading{Paul Makes His Defense Before King Agrippa} So Agrippa said to Paul, “It is permitted for you to speak for yourself.” Then Paul extended his\lebnote{:|NP|:*Literally “the”; the Greek article is used here as a possessive pronoun} hand and\lebnote{:|NP|:*Here “and” is supplied because the previous participle (“extended”) has been translated as a finite verb} began to defend himself:\lebnote{:|NP|:*The imperfect tense has been translated as ingressive here (“began to defend himself”)}
\verse “Concerning all the things of which I am accused by the Jews, King Agrippa, I consider myself fortunate that before you I am about to defend myself today,
\verse because\lebnote{:|NP|:*Here “because” is supplied as a component of the participle (“are”) which is understood as causal} you are especially acquainted with both all the customs and controversial questions with respect to the Jews. Therefore I beg you\lebnote{:|NP|:*Here the direct object is supplied from context in the English translation} to listen to me with patience.
\verse “Now all the Jews know my manner of life from my youth, that had taken place from the beginning among my own people\lebnote{Or “nation”} and in Jerusalem,
\verse having known me for a long time, if they are willing to testify, that in accordance with the strictest party of our religion I lived as a Pharisee.
\verse And now I stand here on trial on the basis of hope in the promise made by God to our fathers,
\verse to which our twelve tribes hope to attain as they earnestly serve him\lebnote{:|NP|:*Here the direct object is supplied from context in the English translation} night and day. Concerning this hope I am being accused by the Jews, O king!
\verse Why is it thought incredible by you people\lebnote{:|NP|:*Here “people” is supplied in the translation to indicate that the pronoun (“you”) is plural} that God raises the dead?
\verse Indeed, I myself thought it was necessary to do many things opposed to the name of Jesus the Nazarene,
\verse which I also did in Jerusalem, and not only did I lock up many of the saints in prison, having received authority from the chief priests, but also when\lebnote{:|NP|:*Here “when” is supplied as a component of the temporal genitive absolute participle (“were being executed”)} they were being executed, I cast my vote\lebnote{Literally “voting pebble,” but here “vote” rather than “voting pebble” is used in the translation to avoid the idea that this small stone was actually thrown at the accused (it was used as a method of voting)} against them.\lebnote{:|NP|:*Here the direct object is supplied from context in the English translation}
\verse And throughout all the synagogues I punished them often and\lebnote{:|NP|:*Here “and” is supplied because the previous participle (“punished”) has been translated as a finite verb} tried to force\lebnote{:|NP|:*The imperfect tense has been translated as conative here (“tried to force”)} them\lebnote{:|NP|:*Here the direct object is supplied from context in the English translation} to blaspheme, and because I\lebnote{:|NP|:*Here “because” is supplied as a component of the participle (“was enraged”) which is understood as causal} was enraged at them beyond measure, I was pursuing them\lebnote{:|NP|:*Here the direct object is supplied from context in the English translation} even as far as to foreign cities.
\verse In \textit{this activity}\lebnote{Literally “which”} I was traveling to Damascus with the authority and full power of the chief priests.
\verse In the middle of the day along the road, O king, I saw a light from heaven, more than the brightness of the sun, shining around me and those who were traveling with me.
\verse And when\lebnote{:|NP|:*Here “when” is supplied as a component of the temporal genitive absolute participle (“had … fallen”)} we had all fallen to the ground, I heard a voice saying to me in the Aramaic language, ‘Saul, Saul, why are you persecuting me? It is hard for you to kick against the goads!’
\verse So I said, ‘Who are you, Lord?’ And the Lord said, ‘I am Jesus whom you are persecuting.
\verse But get up and stand on your feet, because for this reason I have appeared to you, to appoint you a servant and witness both to the things in which you saw me and to the things in which I will appear to you,
\verse rescuing you from the people and from the Gentiles to whom I am sending you,
\verse to open their eyes so that they may turn from darkness to light and from the power of Satan to God, so that they may receive forgiveness of sins and a share among those who are sanctified by faith in me.’
\verse “Therefore, O King Agrippa, I was not disobedient to the heavenly vision,
\verse but to those in Damascus first, and in Jerusalem and all the region of Judea and to the Gentiles, I proclaimed that they should repent and turn to God, doing deeds worthy of repentance.
\verse On account of these things the Jews seized me in\lebnote{Some manuscripts have “while I was in”} the temple courts\lebnote{:|NP|:*Here “courts” is supplied to distinguish this area from the interior of the temple building itself} and\lebnote{:|NP|:*Here “and” is supplied because the previous participle (“seized”) has been translated as a finite verb} were attempting to kill me.\lebnote{:|NP|:*Here the direct object is supplied from context in the English translation}
\verse Therefore I have experienced help from God until this day, and\lebnote{:|NP|:*Here “and” is supplied because the previous participle (“have experienced”) has been translated as a finite verb} I stand here testifying to both small and great, saying nothing except what both the prophets and Moses have said were going to happen,
\verse that the Christ\lebnote{Or “Messiah”} was to suffer and that as the first of the resurrection from the dead, he was going to proclaim light both to the people and to the Gentiles.”
\verse And as\lebnote{:|NP|:*Here “as” is supplied as a component of the temporal genitive absolute participle (“was saying … in his defense”)} he was saying these things in his defense, Festus said with a loud voice, “You are out of your mind, Paul! Your\lebnote{:|NP|:*Literally “the”; the Greek article is used here as a possessive pronoun} great learning \textit{is driving}\lebnote{Literally “is turning”} you \textit{insane}!”\lebnote{Literally “to madness”}
\verse But Paul said, “I am not out of my mind, most excellent Festus, but am speaking words of truth and rationality.
\verse For the king knows about these things, to whom also I am speaking freely, for I am not convinced that these things in any way have escaped\lebnote{Some manuscripts have “that any of these things in any way has escaped”} his notice, because this \textit{was}\lebnote{Literally “is”} not \textit{done}\lebnote{Literally “having been done”} in a corner.
\verse Do you believe the prophets, King Agrippa? I know that you believe.”
\verse But Agrippa said to Paul, “In a short time are you persuading me to become a Christian?”\lebnote{Or “In a short time you are persuading me to become a Christian”}
\verse And Paul replied, “I pray to God, whether in a short time or in a long time, not only you but also all those who are listening to me today may become such people as I also am, except for these bonds!”
\verse Both the king and the governor got up, and Bernice and those who were sitting with them.
\verse And as they\lebnote{:|NP|:*Here “as” is supplied as a component of the temporal genitive absolute participle (“were going out”)} were going out, they were talking to one another, saying, “This man is not doing anything deserving death or imprisonment.”
\verse And Agrippa said to Festus, “This man could have been released if he had not appealed to Caesar.”
\end{biblechapter}

\begin{biblechapter} % Acts 27
\verseWithHeading{Paul and His Associates Sail for Rome} And when it was decided that we would sail away to Italy, they handed over Paul and some other prisoners to a centurion \textit{named}\lebnote{Literally “by name”} Julius of the Augustan\lebnote{The meaning and significance of the title “Augustan” is highly debated, as is the precise identification of this military unit; it may be an honorary unit designation given to auxiliary or provincial troops} Cohort.
\verse And we went aboard a ship from Adramyttium that was about to sail to the places along the coast\lebnote{:|NP|:*The word “coast” is not in the Greek text but is implied} of Asia\lebnote{A reference to the Roman province of Asia (modern Asia Minor)} and\lebnote{:|NP|:*Here “and” is supplied because the previous participle (“went aboard”) has been translated as a finite verb} put out to sea. Aristarchus, a Macedonian from Thessalonica, was with us.
\verse And on the next day, we put in at Sidon. And Julius, treating Paul kindly, allowed him\lebnote{:|NP|:*Here the direct object is supplied from context in the English translation} to go to his\lebnote{:|NP|:*Literally “the”; the Greek article is used here as a possessive pronoun} friends \textit{to be cared for}.\lebnote{Literally “to experience care”}
\verse And from there we put out to sea and\lebnote{:|NP|:*Here “and” is supplied because the previous participle (“put out to sea”) has been translated as a finite verb} sailed under the lee of Cyprus, because the winds were against us.\lebnote{:|NP|:*Here the direct object is supplied from context in the English translation}
\verse And after we\lebnote{:|NP|:*Here “after” is supplied as a component of the participle (“had sailed across”) which is understood as temporal} had sailed across the open sea along Cilicia and Pamphylia, we put in at Myra in Lycia.
\verse And there the centurion found an Alexandrian ship sailing for Italy and\lebnote{:|NP|:*Here “and” is supplied because the previous participle (“found”) has been translated as a finite verb} put us \textit{on board}\lebnote{Literally “into”} it.
\verse And sailing slowly, in many days and with difficulty we came\lebnote{:|NP|:*Here this participle (“came”) has been translated as a finite verb in keeping with English style} to Cnidus. Because\lebnote{:|NP|:*Here “because” is supplied as a component of the causal genitive absolute participle (“permit … to go further”)} the wind did not permit us to go further, we sailed under the lee of Crete off Salmone.
\verse And sailing along its coast with difficulty, we came to a certain place called Fair Havens, near which was the town of Lasea.
\verse And because\lebnote{:|NP|:*Here “because” is supplied as a component of the causal genitive absolute participle (“had passed”)} considerable time had passed and the voyage was now dangerous because even the Fast\lebnote{A reference to the Jewish Day of Atonement (Yom Kippur) which occurs in mid-autumn} was already over, Paul strongly recommended,
\verse saying to them, “Men, I perceive that the voyage is going \textit{to end}\lebnote{Literally “to be”} with disaster and great loss, not only of the cargo and the ship, but also of our lives!”
\verse But the centurion was convinced even more by the shipmaster and the shipowner than by what was said by Paul.
\verse And because\lebnote{:|NP|:*Here “because” is supplied as a component of the causal genitive absolute participle (“was”)} the harbor was unsuitable for spending the winter in, the majority decided on a plan to put out to sea from there, if somehow they could arrive at Phoenix, a harbor of Crete facing toward the southwest and toward the northwest, to spend the winter there.\lebnote{:|NP|:*Here the direct object is supplied from context in the English translation}
\verseWithHeading{A Violent Storm at Sea} And when\lebnote{:|NP|:*Here “when” is supplied as a component of the temporal genitive absolute participle (“began to blow gently”)} a southwest wind began to blow gently, because they\lebnote{:|NP|:*Here “because” is supplied as a component of the participle (“thought”) which is understood as causal} thought they could accomplish their purpose, they weighed anchor and\lebnote{:|NP|:*Here “and” is supplied because the previous participle (“weighed anchor”) has been translated as a finite verb} sailed close along Crete.
\verse But not long afterward a wind like a hurricane, called the northeaster,\lebnote{Literally “Euraquilo,” a violent northern wind} rushed down from it.\lebnote{That is, from the island of Crete}
\verse And when\lebnote{:|NP|:*Here “when” is supplied as a component of the temporal genitive absolute participle (“was caught”)} the ship was caught and was not able to head into the wind, we gave way and\lebnote{:|NP|:*Here “and” is supplied because the previous participle (“gave way”) has been translated as a finite verb} were driven along.
\verse And running under the lee of a certain small island called Cauda, we were able with difficulty to get the ship’s boat under control.
\verse After\lebnote{:|NP|:*Here “after” is supplied as a component of the participle (“hoisting”) which is understood as temporal} hoisting \textit{it up},\lebnote{Literally “which”} they made use of supports to undergird the ship. And because they\lebnote{:|NP|:*Here “because” is supplied as a component of the participle (“were afraid”) which is understood as causal} were afraid lest they run aground on the Syrtis, they lowered the sea anchor and\lebnote{:|NP|:*Here “and” is supplied because the previous participle (“lowered”) has been translated as a finite verb} thus were driven along.
\verse And because\lebnote{:|NP|:*Here “because” is supplied as a component of the causal genitive absolute participle (“battered by the storm”)} we were violently battered by the storm, on the next day \textit{they began}\lebnote{Literally “they began to carry out”} jettisoning the cargo,\lebnote{:|NP|:*Here the direct object is supplied from context in the English translation}
\verse and on the third day they threw overboard the gear of the ship with their own hands.
\verse But when\lebnote{:|NP|:*Here “when” is supplied as a component of the temporal genitive absolute participle (“appeared”)} neither sun nor stars appeared for many days, and with not a little bad weather confronting us,\lebnote{:|NP|:*Here the direct object is supplied from context in the English translation} finally all hope was abandoned that we would be saved.
\verse And because\lebnote{:|NP|:*Here “because” is supplied as a component of the causal genitive absolute participle (“were experiencing”)} many were experiencing lack of appetite, at that time Paul stood up in their midst and\lebnote{:|NP|:*Here “and” is supplied because the previous participle (“stood up”) has been translated as a finite verb} said, “Men, you ought to have followed my advice not to put out to sea from Crete, and thus avoided this damage and loss!
\verse And now I urge you to cheer up, for there will be no loss of life from among you, but only of the ship.
\verse For this night an angel of the God whose I am and whom I serve came to me,
\verse saying, ‘Do not be afraid, Paul! It is necessary for you to stand before Caesar, and behold, God has graciously granted you all who are sailing with you.’
\verse Therefore keep up your courage, men, for I believe God that it will be like this—according to \textit{the}\lebnote{Literally “which”} way it was told to me.
\verse But it is necessary that we run aground on some island.”
\verse And when the fourteenth night had come, as\lebnote{:|NP|:*Here “as” is supplied as a component of the temporal genitive absolute participle (“were being driven”)} we were being driven in the Adriatic Sea about the middle of the night, the sailors suspected \textit{they were approaching some land}.\lebnote{Literally “some land was approaching them”}
\verse And taking soundings, they found twenty fathoms. So going on a little further and taking soundings again, they found fifteen fathoms.
\verse And because they\lebnote{:|NP|:*Here “because” is supplied as a component of the participle (“were afraid”) which is understood as causal} were afraid lest somewhere we run aground against rough places, they threw down four anchors from the stern and\lebnote{:|NP|:*Here “and” is supplied because the previous participle (“threw down”) has been translated as a finite verb} prayed for day to come.
\verse And when\lebnote{:|NP|:*Here “when” is supplied as a component of the temporal genitive absolute participle (“were seeking”)} the sailors were seeking to escape from the ship and were lowering the ship’s boat into the sea, pretending as if they were going to lay out anchors from the bow,
\verse Paul said to the centurion and the soldiers, “Unless these men remain with the ship, you cannot be saved!”
\verse Then the soldiers cut away the ropes of the ship’s boat and let it fall away.\lebnote{Or “let it drift away”}
\verse And until the day was about to come, Paul was urging them all to take some food, saying, “Today is the fourteenth day you have waited anxiously, and\lebnote{:|NP|:*Here “and” is supplied because the previous participle (“have waited”) has been translated as a finite verb} you have continued without eating, having taken nothing.
\verse Therefore I urge you to take some food, for this is necessary for your preservation. For not a hair from your head will be lost.”
\verse And after he\lebnote{:|NP|:*Here “after” is supplied as a component of the participle (“said”) which is understood as temporal} said these things and took bread, he gave thanks to God in front of them all, and after\lebnote{:|NP|:*Here “after” is supplied as a component of the participle (“breaking”) which is understood as temporal} breaking it,\lebnote{:|NP|:*Here the direct object is supplied from context in the English translation} he began to eat.
\verse So they all were\lebnote{:|NP|:*Here this participle (“were”) has been translated as a finite verb in keeping with English style} encouraged and partook of food themselves.
\verse (Now we were in all two hundred seventy six persons on the ship.)
\verse And when they\lebnote{:|NP|:*Here “when” is supplied as a component of the participle (“had eaten their fill”) which is understood as temporal} had eaten their fill of food, they lightened the ship by\lebnote{:|NP|:*Here “by” is supplied as a component of the participle (“throwing”) which is understood as means} throwing the wheat\lebnote{Or “grain”} into the sea.
\verseWithHeading{The Shipwreck} Now when day came, they did not recognize the land, but they noticed a certain bay having a beach, onto which they decided to run the ship ashore if they could.
\verse And slipping the anchors, they left them\lebnote{:|NP|:*Here the direct object is supplied from context in the English translation} in the sea, at the same time loosening the ropes\lebnote{Or “bands” (referring to the linkage that tied the steering oars together)} of the steering oars. And hoisting the foresail to the wind that was blowing, they held course for the beach.
\verse But falling into a place of crosscurrents,\lebnote{Or “a reef”; literally “a place of two seas,” an expression of uncertain meaning but most likely a nautical technical term for some adverse sea condition} they ran the ship aground. And the bow stuck fast and\lebnote{:|NP|:*Here “and” is supplied because the previous participle (“stuck fast”) has been translated as a finite verb} stayed immovable, but the stern was being broken up by the violence.\lebnote{Some manuscripts have “the violence of the waves”}
\verse Now the plan of the soldiers was that they would kill the prisoners lest any escape by\lebnote{:|NP|:*Here “by” is supplied as a component of the participle (“swimming away”) which is understood as means} swimming away,
\verse but the centurion, because he\lebnote{:|NP|:*Here “because” is supplied as a component of the participle (“wanted”) which is understood as causal} wanted to save Paul, prevented them \textit{from doing what they intended},\lebnote{Literally “of the intention”} and gave orders that those who were able to swim should jump in first to get to the land,
\verse and then the rest, some of whom floated\lebnote{:|NP|:*The word “floated” is not in the Greek text, but is supplied in the translation for clarity} on planks and some of whom on anything that was from the ship. And in this way all were brought safely to the land.
\end{biblechapter}

\begin{biblechapter} % Acts 28
\verseWithHeading{Paul on Malta} And after we\lebnote{:|NP|:*Here “after” is supplied as a component of the participle (“were brought safely through”) which is understood as temporal} were brought safely through, then we found out that the island was called Malta.
\verse And the local inhabitants showed \textit{extraordinary}\lebnote{Literally “not the ordinary”} kindness to us, for they lit a fire and\lebnote{:|NP|:*Here “and” is supplied because the previous participle (“lit”) has been translated as a finite verb} welcomed us all, because of the rain that had begun and because of the cold.
\verse And when\lebnote{:|NP|:*Here “when” is supplied as a component of the temporal genitive absolute participle (“had gathered”)} Paul had gathered a large number of sticks and was placing them\lebnote{:|NP|:*Here the direct object is supplied from context in the English translation} on the fire, a viper came out because of the heat and\lebnote{:|NP|:*Here “and” is supplied because the previous participle (“came out”) has been translated as a finite verb} fastened itself on his hand.
\verse And when the local people saw the creature hanging from his hand, they began saying\lebnote{:|NP|:*The imperfect tense has been translated as ingressive here (“began saying”)} to one another, “Doubtless this man is a murderer whom, although he\lebnote{:|NP|:*Here “although” is supplied as a component of the participle (“was rescued”) which is understood as concessive} was rescued from the sea, Justice\lebnote{:|NP|:*Here personified as a goddess} has not permitted to live!”
\verse He, in turn, shook off the creature into the fire and\lebnote{:|NP|:*Here “and” is supplied because the previous participle (“shook off”) has been translated as a finite verb} suffered no harm.
\verse But they were expecting that he was going to swell up\lebnote{Or “to burn with fever” (either meaning is possible here)} or suddenly to fall down dead. So after\lebnote{:|NP|:*Here “after” is supplied as a component of the temporal genitive absolute participle (“had waited”)} they had waited for a long time and saw nothing unusual happen to him, they changed their minds and\lebnote{:|NP|:*Here “and” is supplied because the previous participle (“changed their minds”) has been translated as a finite verb} began saying\lebnote{:|NP|:*The imperfect tense has been translated as ingressive here (“began saying”)} that he was a god.
\verse Now in the regions around that place were fields belonging to the chief official of the island, \textit{named}\lebnote{Literally “by name”} Publius, who welcomed us and\lebnote{:|NP|:*Here “and” is supplied because the previous participle (“welcomed”) has been translated as a finite verb} entertained us\lebnote{:|NP|:*Here the direct object is supplied from context in the English translation} hospitably for three days.
\verse And it happened that the father of Publius was lying down, afflicted with fever and dysentery. Paul went\lebnote{:|NP|:*Here this participle (“went”) has been translated as a finite verb in keeping with English style} to \textit{him}\lebnote{Literally “whom”} and after\lebnote{:|NP|:*Here “after” is supplied as a component of the participle (“praying”) which is understood as temporal} praying, he placed his\lebnote{:|NP|:*Literally “the”; the Greek article is used here as a possessive pronoun} hands on him and\lebnote{:|NP|:*Here “and” is supplied because the previous participle (“placed”) has been translated as a finite verb} healed him.
\verse And after\lebnote{:|NP|:*Here “after” is supplied as a component of the temporal genitive absolute participle (“had taken place”)} this had taken place, the rest of those on the island who had diseases were coming and being healed also.
\verse They also honored us with many honors, and when we\lebnote{:|NP|:*Here “when” is supplied as a component of the participle (“putting out to sea”) which is understood as temporal} were putting out to sea, they gave us\lebnote{:|NP|:*Here the direct object is supplied from context in the English translation} the things \textit{we needed}.\lebnote{Literally “for the needs”}
\verseWithHeading{Paul Arrives in Rome at Last} Now after three months we put out to sea in a ship that had wintered at the island, an Alexandrian one \textit{with the twin gods Castor and Pollux as its insignia}.\lebnote{Literally “marked with the Dioscuri” (a joint name for the twin gods Castor and Pollux)}
\verse And putting in at Syracuse, we stayed there three days.
\verse From there we got underway\lebnote{Or “we sailed along” (with “the coast” understood); the exact meaning of the text as it stands is disputed and various additional terms have to be supplied in any case} and\lebnote{:|NP|:*Here “and” is supplied because the previous participle (“got underway”) has been translated as a finite verb} arrived at Rhegium, and after one day a south wind came up and\lebnote{:|NP|:*Here “and” is supplied because the previous participle (“came up”) has been translated as a finite verb} on the second day we came to Puteoli,
\verse where we found brothers, and\lebnote{:|NP|:*Here “and” is supplied because the previous participle (“found”) has been translated as a finite verb} were implored to stay with them seven days. And in this way we came to Rome.
\verse And from there the brothers, when they\lebnote{:|NP|:*Here “when” is supplied as a component of the participle (“heard”) which is understood as temporal} heard the news about us, came to meet us as far as the Forum of Appius and Three Taverns. When he\lebnote{:|NP|:*Here “when” is supplied as a component of the participle (“saw”) which is understood as temporal} saw them, Paul gave thanks to God and\lebnote{:|NP|:*Here “and” is supplied because the previous participle (“gave thanks”) has been translated as a finite verb} took courage.
\verse And when we entered into Rome, Paul was allowed to stay by himself with the soldier who was guarding him.
\verseWithHeading{Paul and the Jewish Community in Rome} Now it happened that after three days, he called together those who were the most prominent of the Jews. And when\lebnote{:|NP|:*Here “when” is supplied as a component of the temporal genitive absolute participle (“had assembled”)} they had assembled, he said to them, “Men and brothers, although\lebnote{:|NP|:*Here “although” is supplied as a component of the participle (“had done”) which is understood as concessive} I had done nothing against our\lebnote{:|NP|:*Literally “the”; the Greek article is used here as a possessive pronoun} people or the customs of our fathers, from Jerusalem I was delivered as a prisoner into the hands of the Romans,
\verse who, when they\lebnote{:|NP|:*Here “when” is supplied as a component of the participle (“had examined”) which is understood as temporal} had examined me, were wanting to release me,\lebnote{:|NP|:*Here the direct object is supplied from context in the English translation} because there was no basis for an accusation worthy of death with me.
\verse But because\lebnote{:|NP|:*Here “because” is supplied as a component of the causal genitive absolute participle (“objected”)} the Jews objected, I was forced to appeal to Caesar (not as if I\lebnote{:|NP|:*Here “if” is supplied as a component of the participle (“had”) which is understood as concessive} had any charge to bring against my own people).\lebnote{Or “nation”}
\verse Therefore for this reason I have requested to see you and to speak with you,\lebnote{:|NP|:*Here the direct object is supplied from context in the English translation} for because of the hope of Israel I am wearing this chain!”
\verse And they said to him, “We have received no letters about you from Judea, nor has any of the brothers come and\lebnote{:|NP|:*Here “and” is supplied because the previous participle (“come”) has been translated as a finite verb} reported or spoken anything evil about you.
\verse But we would like to hear from you what you think, for concerning this sect it is known to us that it is spoken against everywhere.”
\verse And when they\lebnote{:|NP|:*Here “when” is supplied as a component of the participle (“had set”) which is understood as temporal} had set a day with him, many more came to him at his lodging place, to whom he was explaining from early in the morning until evening, testifying about the kingdom of God and attempting to convince\lebnote{:|NP|:*Here the present tense has been translated as conative (“attempting to convince”)} them about Jesus from both the law of Moses and the prophets.
\verse And some were convinced by\lebnote{:|NP|:*Here “by” is supplied as a component of the participle (“what was said”) which is understood as means} what was said, but others refused to believe.
\verse So being in disagreement with one another, they began to leave after\lebnote{:|NP|:*Here “after” is supplied as a component of the temporal genitive absolute participle (“made”)} Paul made one statement: “The Holy Spirit spoke rightly through the prophet Isaiah to your fathers,
\verse saying,
\verse ‘Go to this people and say, 
“\textit{You will keep on hearing}\lebnote{Literally “hearing you will hear”} and will never understand, 
and \textit{you will keep on seeing}\lebnote{Literally “seeing you will see”} and will never perceive.
\verse Therefore let it be known to you that this salvation of God has been sent to the Gentiles. They also will listen!”\lebnote{Some later manuscripts include v. 29: “And when he had said these things, the Jews departed, having a great dispute among themselves.”}
\verse So he stayed two whole years in his own rented house, and welcomed all who came to him,
\verse proclaiming the kingdom of God and teaching the things concerning the Lord Jesus Christ with all boldness, without hindrance.
\end{biblechapter}

