\documentclass[twoside,twocolumn,a4paper,10pt]{memoir}
\usepackage{lipsum}
\usepackage{fixltx2e}
\usepackage{xspace}
\usepackage[usenames,dvipsnames,svgnames,table]{xcolor}
\usepackage{lettrine}
\usepackage{flushend}
\usepackage{fancyhdr}
\usepackage{hyperref}
\usepackage[object=vectorian]{pgfornament}

\pagestyle{fancy}
\fancyhf{}
\fancyhead[RO,LE]{\rightmark}
\cfoot{\thepage}
\renewcommand{\headrulewidth}{0pt}

\usepackage{fontspec}
\setmainfont{equity}[
  % Files
  Path      = \string~/s/fonts/equity/ ,
  Extension = .otf ,
  % Fonts
  UprightFont     = Equity Text A Regular ,
  UprightFeatures = { SmallCapsFont = Equity Caps A Regular } ,
  BoldFont        = Equity Text A Bold ,
  BoldFeatures    = { SmallCapsFont = Equity Caps A Bold } ,
  ItalicFont      = Equity Text A Italic ,
  BoldItalicFont  = Equity Text A Bold Italic ,
  % Features
  Numbers = OldStyle ]

% old-style numbers don't look great as drop-caps
\newfontfamily{\lettrinefont}{lettrine}[
  % Files
  Path      = \string~/s/fonts/equity/ ,
  Extension = .otf ,
  % Fonts
  UprightFont     = Equity Text A Regular ,
  UprightFeatures = { SmallCapsFont = Equity Caps A Regular } ,
  BoldFont        = Equity Text A Bold ,
  BoldFeatures    = { SmallCapsFont = Equity Caps A Bold } ,
  ItalicFont      = Equity Text A Italic ,
  BoldItalicFont  = Equity Text A Bold Italic]

\newcommand{\framesize}{\textwidth}
\setlength{\headwidth}{\textwidth}
\setlength{\columnseprule}{0pt}
\setlength{\columnsep}{30pt}
\setlength{\parindent}{0pt}
\setheadfoot{\baselineskip}{2\baselineskip}
\setheaderspaces{*}{12pt}{*}
\clubpenalty10000
\widowpenalty10000

\usepackage[british]{babel}

\newlength{\versespacing}
\setlength{\versespacing}{.16667em plus .08333em}
\newcommand{\versespace}{\hspace{\versespacing}}
\frenchspacing

\usepackage{eso-pic}
\newcommand\AtPageUpperRight[1]{\AtPageUpperLeft{%
 \put(\LenToUnit{\paperwidth},\LenToUnit{0\paperheight}){#1}%
 }}%
\newcommand\AtPageLowerRight[1]{\AtPageLowerLeft{%
 \put(\LenToUnit{\paperwidth},\LenToUnit{0\paperheight}){#1}%
 }}%

\makeatletter
\newcommand\versenumcolor{red}
\newcommand\chapnumcolor{red}
\newlength{\biblechapskip}
  \setlength{\biblechapskip}{1em plus .33em minus .2em}
\newcounter{biblechapter}
\newcounter{bibleverse}[biblechapter]
\renewcommand\chaptername{Book}
\newcommand{\biblebook}[1]{%
  \setcounter{biblechapter}{0}
  \gdef\currbook{#1}
  \chapter*{#1}
  \addcontentsline{toc}{chapter}{#1}}
\newcount\biblechap@svdopt
\newenvironment{biblechapter}[1][\thebiblechapter]
  {\biblechap@svdopt=#1
  \ifnum\c@biblechapter=\biblechap@svdopt\else
    \advance\biblechap@svdopt by -1\fi
  \setcounter{biblechapter}{\the\biblechap@svdopt}
  \stepcounter{biblechapter}
  \setbeforesecskip{1ex}\setbeforesubsecskip{1ex}
  \lettrine[lines=3,lhang=0,loversize=0.25]{\lettrinefont\color{\chapnumcolor}\thebiblechapter}{}\ignorespaces}
  {\par\vspace{\biblechapskip}\setbeforesecskip{0ex}\setbeforesubsecskip{0ex}}
\renewcommand{\verse}[1][\thebibleverse]{%
  \stepcounter{bibleverse}%
  \markright{{\scshape\currbook} \thebiblechapter:\thebibleverse}%
  \ifnum\c@bibleverse=1\else{\versespace\color{\versenumcolor}\textbf{\thebibleverse}}\fi~}%
\makeatother

\newcommand{\startornaments}{\AddToShipoutPictureBG{%
  \checkoddpage%
  \ifoddpage%
   \AtPageUpperRight{\put(-60,-35){\pgfornament[width=1.75cm,symmetry=h,color=black]{195}}}%
   \AtPageLowerRight{\put(-60,35){\pgfornament[width=1.75cm,symmetry=v,color=black]{194}}}%
 \else%
   \AtPageUpperLeft{\put(10,-35){\pgfornament[width=1.75cm,symmetry=h,color=black]{194}}}%
   \AtPageLowerLeft{\put(10,35){\pgfornament[width=1.75cm,symmetry=v,color=black]{195}}}
 \fi}}

\newcommand{\stopornaments}{\ClearShipoutPictureBG}

\renewcommand{\printparttitle}[1]{%
  \thispagestyle{empty}%
  \addcontentsline{toc}{part}{#1}%
  \vspace*{\fill}%
  \begin{tikzpicture}[transform shape,every node/.style={inner sep=0pt}]%
    \node[minimum size=\framesize](vecbox){};%
  \node[inner sep=6pt, color=black] (text) at (vecbox.center){%
    \HUGE \textsc{#1}};%
  \node[anchor=north, color=Goldenrod] (base) at (text.south){%
    \pgfornament[width=0.5*\framesize]{88}};%
  \end{tikzpicture}%
  \vspace*{\fill}}

\makechapterstyle{dash-embiggened}{%
  \chapterstyle{default}
  \setlength{\beforechapskip}{5\onelineskip}
  \renewcommand*{\printchaptername}{}
  \renewcommand*{\chapternamenum}{}
  \renewcommand*{\chapnumfont}{\normalfont\Huge}
  \settoheight{\midchapskip}{\chapnumfont 1}
  \renewcommand*{\printchapternum}{\centering \chapnumfont
    \rule[0.5\midchapskip]{1em}{0.4pt} \thechapter\
    \rule[0.5\midchapskip]{1em}{0.4pt}}
  \renewcommand*{\afterchapternum}{\par\nobreak\vskip 0.5\onelineskip}
  \renewcommand*{\printchapternonum}{\centering
                 \vphantom{\chapnumfont 1}\afterchapternum}
  \renewcommand*{\chaptitlefont}{\normalfont\HUGE\scshape}
  \renewcommand*{\printchaptertitle}[1]{\centering \chaptitlefont ##1}
  \setlength{\afterchapskip}{2.5\onelineskip}}

\chapterstyle{dash-embiggened}

\setbeforesecskip{0ex}
\setaftersecskip{0.1ex}
\setsecheadstyle{\bfseries\raggedright\Large}

\setbeforesubsecskip{0ex}
\setaftersubsecskip{0.1ex}
\setsubsecheadstyle{\bfseries\itshape\raggedright\large}

\newcommand{\columnbreak}{\pagebreak}

\newcommand{\LORD}{\textsc{Lord}\xspace}
\newcommand{\LORDs}{\textsc{Lord's}\xspace}

\title{The Holy Bible}
\date{}
\author{}

\begin{document}
\frontmatter

\begin{titlingpage}
\vspace*{\fill}

\begin{tikzpicture}[color=Gold,
    transform shape,
    every node/.style={inner sep=0pt}]
  \node[minimum size=\framesize,fill=Beige!10](vecbox){};
  \node[anchor=north west] at (vecbox.north west){%
    \pgfornament[width=0.2*\framesize]{131}};
  \node[anchor=north east] at (vecbox.north east){%
    \pgfornament[width=0.2*\framesize,symmetry=v]{131}};
  \node[anchor=south west] at (vecbox.south west){%
    \pgfornament[width=0.2*\framesize,symmetry=h]{131}};
  \node[anchor=south east] at (vecbox.south east){%
    \pgfornament[width=0.2*\framesize,symmetry=c]{131}};
  \node[anchor=north] at (vecbox.north){%
    \pgfornament[width=0.6*\framesize,symmetry=h]{85}};
  \node[anchor=south] at (vecbox.south){%
    \pgfornament[width=0.6*\framesize]{85}};
  \node[anchor=north,rotate=90] at (vecbox.west){%
    \pgfornament[width=0.6*\framesize,symmetry=h]{85}};
  \node[anchor=north,rotate=-90] at (vecbox.east){%
    \pgfornament[width=0.6*\framesize,symmetry=h]{85}};
  \node[inner sep=6pt, color=black] (text) at (vecbox.center){%
    \HUGE \textsc{The Holy Bible}};
  \node[anchor=north, color=Goldenrod] (base) at (text.south){%
    \pgfornament[width=0.5*\framesize]{71}};
  \node[anchor=south, color=Goldenrod] at (text.north){%
    \pgfornament[width=0.5*\framesize,symmetry=h]{71}};
\end{tikzpicture}

\vspace*{\fill}
\end{titlingpage}

\tableofcontents

\mainmatter
\part*{The Old Testament}

\startornaments
\biblebook{Genesis}

\begin{biblechapter} % Genesis 1
\verseWithHeading{The Creation} In the beginning, God created the heavens and the earth—
\verse Now the earth was formless and empty, and darkness \textit{was} over the face of the deep. And the Spirit of God \textit{was} hovering over the surface of the waters.
\verse And God said, “Let there be light!” And there was light.
\verse And God saw the light, that \textit{it was} good, and God caused \textit{there to be} a separation between the light and between the darkness.
\verse And God called the light Day, and the darkness he called Night. And there was evening and there was morning, \textit{the} first day.
\verse And God said, “Let there be a vaulted dome in the midst of the waters, and \textit{let it cause a separation between the waters}.”
\verse So God made the vaulted dome, and he caused a separation between the waters which \textit{were} under the vaulted dome and between the waters which were over the vaulted dome. And it was so.
\verse And God called the vaulted dome “heaven.” And there was evening, and there was morning, a second day.
\verse And God said, “Let the waters under heaven be gathered to one place, and let the dry ground appear.” And it was so.
\verse And God called the dry ground “earth,” and he called the collection of the waters “seas.” And God saw that \textit{it was} good.
\verse And God said, “Let the earth produce green plants \textit{that will} bear seed—fruit trees bearing fruit \textit{in which there is seed}—according to its kind, on the earth.” And it was so.
\verse And the earth brought forth green plants bearing seed according to its kind, and trees bearing fruit \textit{in which there was seed} according to its kind. And God saw that \textit{it was} good.
\verse And there was evening and there was morning, a third day.
\verse And God said, “Let there be lights in the vaulted dome of heaven \textit{to separate day from night}, and let them be as signs and for appointed times, and for days and years,
\verse and they shall be as lights in the vaulted dome of heaven to give light on the earth.” And it \textit{was} so.
\verse And God made two lights, the greater light to rule the day and the smaller light to rule the night, and the stars.
\verse And God placed them in the vaulted dome of heaven to give light on the earth
\verse and to rule over the day and over the night, and to \textit{separate light from darkness}. And God saw that \textit{it was} good.
\verse And there was evening and there was morning, a fourth day.
\verse And God said, “Let the waters swarm \textit{with} swarms of living creatures, and let birds fly over the earth across the face of the vaulted dome of heaven.
\verse So God created the great sea creatures and every living creature \textit{that} moves, \textit{with} which the waters swarm, according to their kind, and every bird \textit{with} wings according to its kind. And God saw that \textit{it was} good.
\verse And God blessed them, saying, “Be fruitful and multiply, and fill the waters in the seas, and let the birds multiply on the earth.”
\verse And there was evening, and there was morning, a fifth day.
\verse And God said, “Let the earth bring forth living creatures according to their kind: cattle and moving things, and wild animals according to their kind.” And it was so.
\verse So God made wild animals according to their kind and the cattle according to their kind, and every creeping thing of the earth according to its kind. And God saw that \textit{it was} good.
\verse And God said, “Let us make humankind in our image and according to our likeness, and let them rule over the fish of the sea, and over the birds of heaven, and over the cattle, and over all the earth, and over every moving thing that moves upon the earth.”
\verse So God created humankind in his image, in the likeness of God he created him, male and female he created them.
\verse And God blessed them, and God said to them, “Be fruitful and multiply, and fill the earth and subdue it, and rule over the fish of the sea and the birds of heaven, and over every animal that moves upon the earth.”
\verse And God said, “Look—I am giving to you every plant \textit{that} bears seed which \textit{is} on the face of the whole earth, and every kind of tree \textit{that bears fruit}. They shall be yours as food.”
\verse And to every kind of animal of the earth and to every bird of heaven, and to everything that moves upon the earth in which \textit{there is} life \textit{I am giving} every green plant as food.” And it was so.
\verse And God saw everything that he had made and, behold, \textit{it was} very good. And there was evening, and there was morning, a sixth day.
\end{biblechapter}

\begin{biblechapter} % Genesis 2
\verse And heaven and earth and all their array were finished.
\verse And on the seventh day God finished his work that he had done, and he rested on the seventh day from all his work that he had done.
\verse And God blessed the seventh day, and he sanctified it, because on it he rested from all his work \textit{of creating that \textit{there was} to do}.
\verseWithHeading{The Generations of Heaven and Earth} These are the generations of heaven and earth when they were created, in the day \textit{that} Adonai made earth and heaven—
\verse \textit{before any plant of the field was} on earth, and \textit{before} \textit{any plant of the field} had sprung up, because Adonai had not caused it to rain upon the earth, and there was no human being to cultivate the ground,
\verse but a stream \textit{would} rise from the earth and water the whole face of the ground—
\verse when Adonai formed the man \textit{of} dust from the ground, and he blew into his nostrils the breath of life, and the man became a living creature.
\verse And Adonai planted a garden in Eden in the east, and there he put the man whom he had formed.
\verse And Adonai caused to grow every tree \textit{that} was pleasing to the sight and good for food. And the tree of life \textit{was} in the midst of the garden, \textit{along with} the tree of the knowledge of good and evil.
\verse Now a river flowed out from Eden that watered the garden, and from there it diverged and became four branches.
\verse The name of the first \textit{is} the Pishon. It went around all the land of Havilah, where \textit{there is} gold.
\verse (The gold of that land \textit{is} good; bdellium and onyx stones \textit{are} there.)
\verse And the name of the second \textit{is} Gihon. It went around all the land of Cush.
\verse And the name of the third \textit{is} Tigris. It flows east of Assyria. And the fourth river \textit{is} the Euphrates.
\verse And Adonai took the man and set him in the garden of Eden to cultivate it and to keep it.
\verse And Adonai commanded the man, saying, “From every tree of the garden \textit{you may freely eat},
\verse but from the tree of the knowledge of good and evil you shall not eat, for in the day \textit{that you eat} from it \textit{you shall surely die}.”
\verse Then Adonai said, “\textit{it is} not good \textit{that} the man is alone. I will make for him a helper \textit{as his counterpart}.”
\verse And out of the ground Adonai formed every beast of the field and every bird of the sky, and he brought \textit{each} to the man to see what he would call it. And whatever the man called that living creature \textit{was} its name.
\verse And the man \textit{gave names} to every domesticated animal and to the birds of heaven and to all the wild animals. But for \textit{the} man there was not found a helper \textit{as his counterpart}.
\verse And Adonai caused a deep sleep to fall upon the man. While he slept, he took one of his ribs, and closed up \textit{the flesh where it had been}.
\verse And Adonai fashioned the rib which he had taken from the man into a woman and brought her to the man.
\verse And the man said, “\textit{She is now} bone from my bones 
and flesh from my flesh; 
\textit{she} shall be called ‘Woman,’ 
for \textit{she was taken} from man.”
\verse Therefore a man shall leave his father and his mother and shall cling to his wife, and they shall be as one flesh.
\verse And the man and his wife, both of them, were naked, and they were not ashamed.
\end{biblechapter}

\begin{biblechapter} % Genesis 3
\verseWithHeading{The Fall} Now the serpent was more crafty than any other \textit{wild animal} which Adonai had made. He said to the woman, “Did God indeed say, ‘You shall not eat from any tree in the garden’?”
\verse The woman said to the serpent, “From the fruit of the trees of the garden we may eat,
\verse but from the tree that is in the midst of the garden, God said, ‘You shall not eat from it, nor shall you touch it, lest you die’.”
\verse But the serpent said to the woman, “You shall not surely die.
\verse For God knows that on the day you \textit{both} eat from it, then your eyes will be opened and you \textit{both} shall be like gods, knowing good and evil.”
\verse When the woman saw that the tree \textit{was} good for food and that it \textit{was} a delight to the eyes, and the tree was desirable to make \textit{one} wise, then she took from its fruit and she ate. And she gave \textit{it} also to her husband with her, and he ate.
\verse Then the eyes of both of them were opened, and they knew that they \textit{were} naked. And they sewed together fig leaves and they made for themselves coverings.
\verse Then they heard the sound of Adonai walking in the garden \textit{at the windy time of day}. And the man and his wife hid themselves from the presence of Adonai among the trees of the garden.
\verse And Adonai called to the man and said to him, “Where \textit{are} you?”
\verse And he replied, “I heard the sound of you in the garden, and I was afraid because I \textit{am} naked, so I hid myself.”
\verse Then he asked, “Who told you that you \textit{were} naked? Have you eaten from \textit{the tree from which I forbade you to eat}?”
\verse And the man replied, “The woman whom you gave \textit{to be} with me—she gave to me from the tree and I ate.”
\verse Then Adonai said to the woman, “What \textit{is} this you have done?” And the woman said, “The serpent deceived me, and I ate.”
\verse Then Adonai said to the serpent,
\verse “Because you have done this, 
you \textit{will be} cursed 
more than any domesticated animal 
and more than any wild animal. 
On your belly you shall go 
and dust you shall eat 
all the days of your life.
\verse To the woman he said, “I will greatly increase 
\textit{your pain in childbearing}; 
in pain you shall bear children. 
And to your husband \textit{shall be} your desire. 
And he shall rule over you.”
\verse And to Adam he said, “Because you listened to the voice of your wife and you ate from the tree \textit{from which I forbade you to eat},
\verse the ground \textit{shall be} cursed on your account. 
In pain you shall eat \textit{from} it 
all the days of your life.
\verse And thorns and thistles shall sprout for you, 
and you shall eat the plants of the field.
\verse And the man \textit{named} his wife Eve, because she was the mother of all life.
\verse And Adonai made for Adam and for his wife garments of skin, and he clothed them.
\verse And Adonai said, “Look—the man has become as one of us, to know good and evil. \textit{What if} he stretches out his hand and takes also from the tree of life and eats, and lives forever?”
\verse And Adonai sent him out from the garden of Eden, to till the ground from which he was taken.
\verse So he drove the man out, and placed cherubim east of the garden of Eden, and \textit{a flaming, turning sword} to guard the way to the tree of life.
\end{biblechapter}

\begin{biblechapter} % Genesis 4
\verseWithHeading{Cain and Abel} Now Adam knew Eve his wife, and she conceived and bore Cain. And she said, “I have given birth to a man with \textit{the help of} Adonai.”
\verse \textit{Then she bore} his brother Abel. And Abel became a keeper of sheep, and Cain became a tiller of the ground.
\verse And \textit{in the course of time} Cain brought an offering from the fruit of the ground to Adonai,
\verse and Abel also brought \textit{an offering} from \textit{the choicest firstlings of his flock}. And Adonai looked with favor to Abel and to his offering,
\verse but to Cain and to his offering he did not look with favor. And Cain became very angry, and his face fell.
\verse And Adonai said to Cain, “Why are you angry, and why is your face fallen?
\verse If you do well \textit{will I not accept you}? But if you do not do well, sin is crouching at the door. And its desire \textit{is} for you, but you must rule over it.”
\verse Then Cain said to his brother Abel, \textit{“Let us go out into the field.”} And when they were in the field, Cain rose up against his brother Abel and killed him.
\verse Then Adonai said to Cain, “Where \textit{is} Abel your brother?” And he said, “I do not know; am I my brother’s keeper?”
\verse And he said, “What have you done? The voice of your brother’s blood is crying out to me from the ground.
\verse So now you are cursed from the ground, which has opened its mouth to receive the blood of your brother from your hand.
\verse When you till the ground \textit{it shall no longer yield its strength to you}. You shall be a wanderer and a fugitive on the earth.”
\verse And Cain said to Adonai, “My punishment \textit{is} greater than \textit{I can} bear.
\verse Look, you have driven me out today from the face of the ground, and from your face I must hide. I will be a wanderer and a fugitive on the earth, and it will happen that whoever finds me will kill me.”
\verse Then Adonai said to him, “Therefore, whoever kills Cain will be avenged sevenfold.” Then Adonai put a sign on Cain so that whoever found him would not kill him.
\verse And Cain went out from the presence of Adonai, and he settled in the land of Nod, east of Eden.
\verse And Cain knew his wife, and she conceived and gave birth to Enoch. And when he built a city \textit{he named the city after his son, Enoch}.
\verse And to Enoch was born Irad, and Irad fathered Mehujael, and Mehujael fathered Methushael, and Methushael fathered Lamech.
\verse And Lamech took to himself two wives. The name of the first \textit{was} Adah, and the name of the second \textit{was} Zillah.
\verse And Adah gave birth to Jabal; he was the father of those who live in tents and \textit{those who have} livestock.
\verse And the name of his brother \textit{was} Jubal; he was the father of all who play stringed instruments and wind instruments.
\verse Then Zillah also gave birth to Tubal-Cain who forged all \textit{kinds of} tools of bronze and iron. And the sister of Tubal-Cain \textit{was} Naamah.
\verse Then Lamech said to his wives,
\verse “Adah and Zillah, listen to my voice; 
O wives of Lamech, hear my words. 
I have killed a man for wounding me, 
Even a young man for injuring me.
\verse Then Adam knew his wife again, and she gave birth to a son. And she called his name Seth, for \textit{she said} “God has appointed to me another child in the place of Abel, because Cain killed him.”
\verse And as for Seth, he also fathered a son, and he called his name Enosh. At that time he began to call on the name of Adonai.
\end{biblechapter}

\begin{biblechapter} % Genesis 5
\verseWithHeading{Adam’s Descendants to Noah} This is the record of the generations of Adam. When God created Adam, he made him in the likeness of God.
\verse Male and female he created them. And he blessed them. And he called their name “Humankind” when they were created.
\verse And when Adam had lived one hundred and thirty years, he fathered a child in his likeness, according to his image. And he called his name Seth.
\verse And the days of Adam after he fathered Seth were eight hundred years. And he fathered sons and daughters.
\verse And all the days of Adam which he lived were nine hundred and thirty years, and he died.
\verse When Seth had lived one hundred and five years, he fathered Enosh.
\verse And after Seth had fathered Enosh he lived eight hundred and seven years, and fathered sons and daughters.
\verse And all the days of Seth were nine hundred and twelve years, and he died.
\verse When Enosh lived ninety years, he fathered Kenan.
\verse And after Enosh fathered Kenan he lived eight hundred and fifteen years, and fathered sons and daughters.
\verse And all the days of Enosh were nine hundred and five years, and he died.
\verse When Kenan had lived seventy years, he fathered Mahalalel.
\verse And after Kenan had fathered Mahalalel, he lived eight hundred and forty years, and fathered sons and daughters.
\verse And all the days of Kenan were nine hundred and ten years, and he died.
\verse When Mahalalel had lived sixty-five years, he fathered Jared.
\verse And after Mahalalel had fathered Jared, he lived eight hundred and thirty years, and fathered sons and daughters.
\verse And all the days of Mahalalel were eight hundred and ninety-five years, and he died.
\verse When Jared had lived one hundred and sixty-two years, he fathered Enoch.
\verse And after Jared had fathered Enoch, he lived eight hundred years, and fathered sons and daughters.
\verse And all the days of Jared were nine hundred and sixty-two years, and he died.
\verse When Enoch had lived sixty-five years, he fathered Methuselah.
\verse And Enoch walked with God after he fathered Methuselah three hundred years, and fathered sons and daughters.
\verse And all the days of Enoch were three hundred and sixty-five years.
\verse And Enoch walked with God, and he was no more, for God took him.
\verse When Methuselah had lived one hundred and eighty-seven years, he fathered Lamech.
\verse And after Methuselah had fathered Lamech, he lived seven hundred and eighty-two years, and fathered sons and daughters.
\verse And all the days of Methuselah were nine hundred and sixty-nine years, and he died.
\verse When Lamech had lived one hundred and eighty-two years, he fathered a son.
\verse And he called his name Noah, saying, “This one \textit{shall relieve us} from our work, and from the hard labor of our hands, from the ground which Adonai had cursed.
\verse And after Lamech had fathered Noah he lived five hundred and ninety-five years, and he fathered sons and daughters.
\verse All the days of Lamech were seven hundred and seventy-seven years, and he died.
\verse When Noah \textit{was five hundred years old}, Noah fathered Shem, Ham, and Japheth.
\end{biblechapter}

\begin{biblechapter} % Genesis 6
\verseWithHeading{Prelude to the Flood} And it happened \textit{that}, when humankind began to multiply on the face of the ground, daughters were born to them.
\verse Then the sons of God saw the daughters of humankind, that they \textit{were} beautiful. And they took for themselves wives from all that they chose.
\verse And Adonai said, “My Spirit shall not abide with humankind forever in that he \textit{is} also flesh. And his days \textit{shall be} one hundred and twenty years.”
\verse The Nephilim \textit{were} upon the earth in those days, and also afterward, when the sons of God went into the daughters of humankind, and they bore children to them.
\verse And Adonai saw that the evil of humankind \textit{was} great upon the earth, and every inclination of the thoughts of his heart \textit{was} always only evil.
\verse And Adonai regretted that he had made humankind on the earth, and \textit{he was grieved in his heart}.
\verse And Adonai said, “I will destroy humankind whom I created from upon the face of the earth, from humankind, to animals, to creeping things, and to the birds of heaven, for I regret that I have made them.”
\verse But Noah found favor in the eyes of Adonai.
\verse These \textit{are} the generations of Noah. Noah \textit{was} a righteous man, without defect in his generations. Noah walked with God.
\verse And Noah fathered three sons: Shem, Ham, and Japheth.
\verse And the earth \textit{was} corrupted before God, and the earth was filled \textit{with} violence.
\verse And God saw the earth, and behold, it was corrupt, for all flesh had corrupted its way upon the earth.
\verse And God said to Noah, “The end of all flesh \textit{has} come before me, for the earth was filled \textit{with} violence because of them. Now, look, I \textit{am going} to destroy them \textit{along} with the earth.
\verse Make for yourself an ark of cypress wood; you must make the ark \textit{with} rooms, then you must cover it with pitch, inside and outside.
\verse And this \textit{is} how you must make it: the length of the ark, three hundred cubits; its width fifty cubits; its height, thirty cubits.
\verse You must make a roof for the ark, and \textit{finish it to a cubit above}. And \textit{as for} the door of the ark, you must put \textit{it} in its side. You must make it \textit{with} a lower, second, and a third \textit{deck}.
\verse And I, behold, I \textit{am} about to bring the flood waters over the earth to destroy all flesh in which \textit{is} the breath of life from under the heaven; everything that \textit{is} on the earth shall perish.
\verse And I will establish my covenant with you, and you must go into the ark—you, and your sons, and your wife, and the wives of your sons with you.
\verse And of every living thing, from all flesh, you must bring two from every \textit{kind} into the ark to keep \textit{them} alive with you; they shall be male and female.
\verse From the birds according to their kind, and from the animals according to their kind, from every creeping thing \textit{on} the ground according to its kind—two from every kind shall come to you to keep \textit{them} alive.
\verse And \textit{as for} you, take for yourself from every kind of food that is eaten. And you must gather \textit{it} to yourself. And it shall be for you and for them for food.”
\verse And Noah did according to all that God commanded him; thus he did.
\end{biblechapter}

\begin{biblechapter} % Genesis 7
\verse Then Adonai said to Noah, “Go—you and all your household—into the ark, for I have seen you \textit{are} righteous before me in this generation.
\verse From all the clean animals you must take for yourself \textit{seven pairs}, a male and its mate. And from the animals that \textit{are} not clean \textit{you must take} two, a male and its mate,
\verse as well as from the birds of heaven \textit{seven pairs}, male and female, \textit{to keep their kind alive} on the face of the earth.
\verse For \textit{within seven days} I will send rain upon the earth \textit{for} forty days and forty nights. And I will blot out all the living creatures that I have made from upon the face of the ground.”
\verse And Noah did according to all that Adonai commanded him.
\verseWithHeading{The Flood} Noah \textit{was six hundred years old} when the flood waters came upon the earth.
\verse And Noah and his sons and his wife, and the wives of his sons with him, went into the ark because of the waters of the flood.
\verse Of clean animals, and of animals which \textit{are} not clean, and of the birds, and everything that creeps upon the ground,
\verse \textit{two of each} went to Noah, into the ark, male and female, as God had commanded Noah.
\verse And it happened \textit{that} after seven days the waters of the flood came over the earth.
\verse In the six hundredth year of the life of Noah, in the second month, on the seventeenth day of the month—on that day all the springs of the great deep were split open, and the windows of heaven were opened.
\verse And the rain came upon the earth forty days and forty nights.
\verse On this same day, Noah, Shem, Ham, and Japheth, the sons of Noah, and the wife of Noah and the three wives of his sons with them, went into the ark,
\verse they and all the living creatures according to their kind, and all the domesticated animals according to their kind, and all the creatures that creep upon the earth according to their kind, all the birds according to their kind, every winged creature.
\verse And they came to Noah to the ark, \textit{two of each}, from every living thing in which \textit{was} the breath of life.
\verse And those that came, male and female, of every living thing, came as God had commanded him. And Adonai shut the door behind him.
\verse And the flood came forty days and forty nights upon the earth. And the waters increased, and lifted the ark, and it rose up from the earth.
\verse And the waters prevailed and increased greatly upon the earth. And the ark went upon the surface of the waters.
\verse And the waters prevailed \textit{overwhelmingly} upon the earth, and they covered all the high mountains which were under the entire heaven.
\verse \textit{The waters swelled fifteen cubits above the mountains, covering them}.
\verse And every living thing that moved on the earth perished—the birds, and the domesticated animals, and the wild animals, and everything that swarmed on the earth, and all humankind.
\verse Everything in whose nostrils \textit{was} \textit{the breath of life}, among all that \textit{was} on dry land, died.
\verse And he blotted out every living thing upon the surface of the ground, from humankind, to animals, to creeping things, and to the birds of heaven; they were blotted out from the earth. Only Noah and those who \textit{were} with him in the ark remained.
\verse And the waters prevailed over the earth one hundred and fifty days.
\end{biblechapter}

\begin{biblechapter} % Genesis 8
\verseWithHeading{The Flood Subsides} And God remembered Noah and all the wild animals, and all the domesticated animals that \textit{were} with him in the ark. And God caused a wind to blow over the earth, and the waters subsided.
\verse And the fountains of the deep and the windows of the heavens were closed, and the rain from the heavens was restrained.
\verse And the waters receded from the earth \textit{gradually}, and the waters abated at the end of one hundred and fifty days.
\verse And the ark came to rest in the seventh month, on the seventeenth day of the month, on the mountains of Ararat.
\verse And the waters \textit{continued to recede} to the tenth month; in the tenth month, on the first of the month, the tops of the mountains appeared.
\verse And it happened \textit{that} at the end of forty days Noah opened the window of the ark that he had made.
\verse And he sent out a raven; \textit{it went to and fro} until the waters were dried up from upon the earth.
\verse And \textit{he sent out a dove} to see \textit{whether} the waters had subsided from upon the ground.
\verse But the dove did not find a resting place for the sole of her foot, and she returned to him into the ark, for the waters \textit{were still} on the face of the earth. And he stretched out his hand and took her, and brought her to himself into the ark.
\verse And he waited another seven days, and \textit{again he sent out} the dove from the ark.
\verse And the dove came to him \textit{in the evening}, and behold, a freshly-picked olive tree leaf \textit{was} in her mouth. And Noah knew that the waters had subsided from upon the earth.
\verse And he waited \textit{seven more days}, and he sent out the dove. But it did not return again to him.
\verse And it happened that, in the six hundred and first year, in the first \textit{month}, on the first \textit{day} of the month, the waters dried up from upon the earth. And Noah removed the covering of the ark and looked. And behold, the face of the ground was dried up.
\verse And in the second month, on the twenty-seventh day of the month, the earth was dry.
\verse And God spoke to Noah, saying:
\verse “Go out from the ark, you and your wife, and your sons, and your sons’ wives with you.
\verse Bring out with you all the living things which \textit{are} with you, from all the living creatures—birds, and animals, and everything that creeps on the earth, and let them swarm on the earth and be fruitful and multiply on the earth.”
\verse So Noah went out, with his sons and his wife, and the wives of his sons with him.
\verse Every animal, every creeping thing, and every bird, \textit{and} everything \textit{that} moves upon the earth, according to its families, went out from the ark.
\verse And Noah built an altar to Adonai, and he took from all the clean animals and from all the clean birds, and offered burnt offerings on the altar.
\verse And Adonai smelled the soothing fragrance, and Adonai said \textit{to himself}, “\textit{Never again will I curse} the ground for the sake of humankind, because the inclination of the heart of humankind \textit{is} evil from his youth. \textit{Nor will I ever again destroy} all life as I have done.
\verse \textit{As long as the earth endures}, seed and harvest, cold and heat, summer and winter, day and night will not cease.
\end{biblechapter}

\begin{biblechapter} % Genesis 9
\verseWithHeading{God’s Covenant with Noah and Humankind} And God blessed Noah and his sons, and said to them, “Be fruitful and multiply, and fill the earth.
\verse And fear of you and dread of you shall be upon every animal of the earth, and on every bird of heaven, \textit{and} on everything that moves upon the ground, and on all the fish of the sea. Into your hand they shall be given.
\verse Every moving thing that lives shall be for you as food. As \textit{I gave} the green plants to you, I have \textit{now} given you everything.
\verse Only you shall not eat \textit{raw flesh with blood in it}.
\verse And \textit{your lifeblood} I will require; from \textit{every animal} I will require it. And from the hand of humankind, from the hand of \textit{each} man to his brother I will require the life of humankind.
\verse “\textit{As for} the one shedding the blood of humankind, 
by humankind his blood shall be shed, 
for God made humankind in his own image.
\verse “And you, be fruitful and multiply, swarm on the earth and multiply in it.”
\verse And God said to Noah and to his sons with him,
\verse “As for me, behold, I am establishing my covenant with you and with your seed after you,
\verse and with every living creature that \textit{is} with you—the birds, the animals, and every animal of the earth with you, from all \textit{that} came out of the ark to all the animals of the earth.
\verse I am establishing my covenant with you, that never again will all flesh be cut off by the waters of a flood, nor will there ever be a flood that destroys the earth.”
\verse And God said, “This \textit{is} the sign of the covenant that I am making between me and you, and between every living creature that \textit{is} with you for future generations.
\verse My bow I have set in the clouds, and it shall be for a sign of \textit{the} covenant between me and between the earth.
\verse And when I make clouds appear over the earth the bow shall be seen in the clouds.
\verse Then I will remember my covenant that \textit{is} between me and you, and between every living creature, with all flesh. And the waters of a flood will never again \textit{cause the destruction} of all flesh.
\verse The bow shall be in the clouds, and I will see it, so as to remember \textit{the} everlasting covenant between God and between every living creature, with all flesh that \textit{is} upon the earth.”
\verse And God said to Noah, “This \textit{is} the sign of the covenant which I am establishing between me and all flesh that \textit{is} upon the earth.
\verseWithHeading{Noah’s Descendants} Now the sons of Noah who came out of the ark \textit{were} Shem, Ham, and Japheth. (Ham \textit{was} the father of Canaan.)
\verse These three \textit{were} the sons of Noah, and from these \textit{the whole earth was populated}.
\verse And Noah began \textit{to be} a man of the ground, and he planted a vineyard.
\verse And he drank some of the wine and became drunk, and he exposed himself in the midst of his tent.
\verse And Ham, the father of Canaan, saw the nakedness of his father, and he told his two brothers outside.
\verse Then Shem and Japheth took a garment, and the two of them put \textit{it} on \textit{their} shoulders and, walking backward, they covered the nakedness of their father. And their faces \textit{were turned} backward, so that they did not see the nakedness of their father.
\verse Then Noah awoke from his drunkenness, and he knew what his youngest son had done to him.
\verse And he said, “Cursed \textit{be} Canaan, 
a slave of slaves he shall be to his brothers.”
\verse Then he said,
\verse “Blessed \textit{be} Adonai, the God of Shem, 
and let Canaan be a slave to them.
\verse And Noah lived three hundred and fifty years after the flood.
\verse And all the days of Noah were nine hundred and fifty years, and he died.
\end{biblechapter}

\begin{biblechapter} % Genesis 10
\verseWithHeading{The Descendants of the Sons of Noah} These \textit{are} the generations of the sons of Noah—Shem, Ham, and Japheth. Children were born to them after the flood.
\verse The sons of Japheth: Gomer, Magog, Madai, Javan, Tubal, Meshech, and Tiras.
\verse And the sons of Gomer: Ashkenaz, Riphath, and Togarmah.
\verse And the sons of Javan: Elishah, Tarshish, Kittim, and Dodanim.
\verse From these the coastland peoples spread out through their lands, each according to his own language by their own families, in their nations.
\verse And the sons of Ham: Cush, Egypt, Put, and Canaan.
\verse And the sons of Cush: Seba, Havilah, Sabtah, Raamah, and Sabteca. The sons of Raamah: Sheba and Dedan.
\verse And Cush fathered Nimrod. \textit{He was the first on earth to be a mighty warrior}.
\verse He was a mighty hunter before Adonai. Therefore it was said, “Like Nimrod a mighty hunter before Adonai.”
\verse Now, the beginning of his kingdom \textit{was} Babel, Erech, Akkad, and Calneh, in the land of Shinar.
\verse From that land he went out \textit{to} Assyria, and he built Nineveh, Rehoboth-Ir, Calah,
\verse Resen between Nineveh and Calah; that \textit{is} the great city.
\verse And Egypt fathered Ludim, Anamim, Lehabim, Naphtuhim,
\verse Pathrusim, and Casluhim (from whom the Philistines came), and Caphtorim.
\verse Canaan fathered Sidon, his firstborn, and Heth,
\verse and the Jebusites, the Amorites, the Girgashites,
\verse the Hivites, the Arkites, the Sinites,
\verse the Arvadites, the Zemarites, and the Hamathites. Afterward the families of the Canaanites were spread abroad.
\verse And the territory of the Canaanites \textit{was} from Sidon \textit{in the direction of} Gerar as far as Gaza, and \textit{in the direction of} Sodom, Gomorrah, Admah, and Zeboiim, as far as Lasha.
\verse These \textit{are} the descendants of Ham, according to their families and their languages, in their lands, and in their nations.
\verse And to Shem, the father of all the children of Eber, the older brother of Japheth, \textit{children} were also born.
\verse The sons of Shem: Elam, Asshur, Arphaxad, Lud, and Aram.
\verse And the sons of Aram: Uz, Hul, Gether, and Mash.
\verse And Arphaxad fathered Shelah, and Shelah fathered Eber.
\verse And to Eber two sons were born. The name of the one was Peleg, for in his days the earth was divided, and the name of his brother \textit{was} Joktan.
\verse And Joktan fathered Almodad, Sheleph, Hazarmaveth, Jerah,
\verse Hadoram, Uzal, Diklah,
\verse Obal, Abimael, Sheba,
\verse Ophir, Havilah, and Jobab. All these \textit{were} the sons of Joktan.
\verse And their dwelling \textit{place} \textit{extended from} Mesha \textit{in the direction of} Sephar \textit{to} the hill country of the east.
\verse These \textit{are} the sons of Shem, according to their families, according to their languages, in their lands, and according to their nations.
\verse These are the families of the sons of Noah, according to their generations \textit{and} in their nations. And from these the nations spread abroad on the earth after the flood.
\end{biblechapter}

\begin{biblechapter} % Genesis 11
\verseWithHeading{The Tower of Babel} Now the whole earth \textit{had} one language and the same words.
\verse And as people migrated from the east they found a plain in the land of Shinar and settled there.
\verse And they said \textit{to each other}, “Come, let us make bricks and \textit{burn them thoroughly}.” And they had brick for stone and they had tar for mortar.
\verse And they said, “Come, let us build ourselves a city and a tower whose top \textit{reaches to} the heavens. And let us make a name for ourselves, lest we be scattered over the face of the whole earth.”
\verse Then Adonai came down to see the city and the tower that \textit{humankind} was building.
\verse And Adonai said, “Behold, \textit{they are one people with one language}, and \textit{this is only the beginning of what they will do}. So now nothing that they intend to do will be impossible for them.
\verse Come, let us go down and confuse their language there, so that they will not understand \textit{each other’s language}.”
\verse So Adonai scattered them from there over the face of the whole earth, and they stopped building the city.
\verse Therefore its name was called Babel, for there Adonai confused the language of the whole earth, and there Adonai scattered them over the face of the whole earth.
\verseWithHeading{The Descendants of Shem} These are the generations of Shem. When Shem \textit{was one hundred years old}, he fathered Arphaxad, two years after the flood.
\verse And Shem lived five hundred years after he fathered Arphaxad, and he fathered \textit{other} sons and daughters.
\verse When Arphaxad had lived thirty-five years, he fathered Shelah.
\verse And Arphaxad lived four hundred and three years after he fathered Shelah, and he fathered \textit{other} sons and daughters.
\verse When Shelah had lived thirty years, he fathered Eber.
\verse And Shelah lived four hundred and three years after he fathered Eber, and he fathered \textit{other} sons and daughters.
\verse When Eber had lived thirty-four years, he fathered Peleg.
\verse And Eber lived four hundred and thirty years after he fathered Peleg, and he fathered \textit{other} sons and daughters.
\verse When Peleg had lived thirty years, he fathered Reu.
\verse And Peleg lived two hundred and nine years after he fathered Reu, and he fathered \textit{other} sons and daughters.
\verse When Reu had lived thirty-two years, he fathered Serug.
\verse And Reu lived two hundred and seven years after he fathered Serug, and he fathered \textit{other} sons and daughters.
\verse When Serug had lived thirty years, he fathered Nahor.
\verse And Serug lived two hundred years after he fathered Nahor, and he fathered \textit{other} sons and daughters.
\verse When Nahor had lived twenty-nine years, he fathered Terah.
\verse And Nahor lived one hundred and nineteen years after he fathered Terah, and he fathered \textit{other} sons and daughters.
\verse When Terah had lived seventy years, he fathered Abram, Nahor, and Haran.
\verseWithHeading{The Descendants of Terah} Now these are the generations of Terah. Terah fathered Abram, Nahor, and Haran, and Haran fathered Lot.
\verse And Haran died in the presence of Terah his father in the land of his birth, in Ur of the Chaldeans.
\verse And Abram and Nahor took wives for themselves. The name of the wife of Abram \textit{was} Sarai, and the name of the wife of Nahor \textit{was} Milcah, the daughter of Haran, the father of Milcah and Iscah.
\verse And Sarai was barren; she had no child.
\verse And Terah took Abram his son, and Lot, the son of Haran, \textit{his grandson}, and Sarai his daughter-in-law, the wife of Abram his son, and went out with them from Ur of the Chaldeans to go to the land of Canaan. And they went to Haran, and they settled there.
\verse And the days of Terah \textit{were} two hundred and five years, and Terah died in Haran.
\end{biblechapter}

\begin{biblechapter} % Genesis 12
\verseWithHeading{The Call of Abram} And Adonai said to Abram, “Go out from your land and from your relatives, and from the house of your father, to the land that I will show you.
\verse And I will make you a great nation, and I will bless you, and I will make your name great. And you will be a blessing.
\verse And I will bless those who bless you, and those who curse you I will curse. And all families of the earth will be blessed in you.”
\verseWithHeading{Abram’s Journey} And Abram went \textit{out} as Adonai had told him, and Lot went with him. Now Abram \textit{was seventy-five years old} when he went out from Haran.
\verse And Abram took Sarai his wife, and Lot \textit{his nephew}, and all their possessions that they had gathered, and all the persons that they had acquired in Haran, and they went out to go to the land of Canaan. And they went to the land of Canaan.
\verse And Abram traveled through the land up to the place of Shechem, to the Oak of Moreh. Now the Canaanites \textit{were} in the land at that time.
\verse And Adonai appeared to Abram and said, “To your offspring I will give this land.” And he built an altar there to Adonai, who had appeared to him.
\verse And he moved on from there to the hill country, east of Bethel. And he pitched his tent at Bethel on the west, and at Ai on the east. And he built an altar there to Adonai. And he called on the name of Adonai.
\verse \textit{And Abram kept moving on}, toward the Negev.
\verse And there was a famine in the land. And Abram went down to Egypt to dwell as an alien there, for the famine was severe in the land.
\verse And it happened \textit{that} as he drew near to enter into Egypt, he said to Sarai his wife, “Look now, I know that you are a woman beautiful of appearance,
\verse and it shall happen \textit{that}, if the Egyptians see you, then they will say, ‘This \textit{is} his wife,’ then they will kill me but let you live.
\verse Please say you are my sister so that it will go well for me on your account. \textit{Then I will live} on account of you.”
\verse And it happened \textit{that} as Abram came into Egypt, the Egyptians saw the woman, that she \textit{was} very beautiful.
\verse And the officials of Pharaoh saw her, and they praised her \textit{beauty} to Pharaoh. And the woman was taken to the house of Pharaoh.
\verse And he dealt well with Abram on account of her, and he had sheep, cattle, male donkeys, male servants, female servants, female donkeys, and camels.
\verse Then Adonai afflicted Pharaoh and his household with severe plagues on account of the matter of Sarai the wife of Abram.
\verse Then Pharaoh called for Abram and said, “What \textit{is} this you have done to me? Why did you not tell me that she \textit{was} your wife?
\verse Why did you say ‘She \textit{is} my sister,’ so that I took her to myself as a wife? Now then, here \textit{is} your wife. Take her and go.”
\verse And Pharaoh commanded his men concerning him, and then sent him and his wife and all that \textit{was} with him away.
\end{biblechapter}

\begin{biblechapter} % Genesis 13
\verseWithHeading{The Parting of Abram and Lot} Then Abram went up from Egypt, he and his wife and all that \textit{was} with him. And Lot \textit{went} with him to the Negev.
\verse Now Abram \textit{was} very wealthy in livestock, in silver, and in gold.
\verse And he went according to his journey from the Negev, then to Bethel, to the place where his tent was at the beginning, between Bethel and Ai,
\verse to the place where he had made an altar at the beginning. And Abram called on the name of Adonai there.
\verse And Lot, who went with Abram, also had herds and tents.
\verse And the land could not \textit{support them} \textit{so as} to live together, because their possessions were \textit{so} many that they were not able to live together.
\verse And there was a quarrel between the herdsmen of the livestock of Abram and the herdsmen of the livestock of Lot. Now at that time the Canaanites and the Perizzites were living in the land.
\verse Then Abram said to Lot, “Please, let there not be quarreling between me and you, and between my shepherds and your shepherds, for we men \textit{are} brothers.
\verse Is not the whole land before you? Separate yourself from me. If \textit{you want what is on} the left, then I will go right; if \textit{you want what is on} the right, I will go left.”
\verse And Lot lifted up his eyes and saw the whole plain of the Jordan, that all of it \textit{was} well-watered land—\textit{this was} before Adonai destroyed Sodom and Gomorrah—like the garden of Adonai, like the land of Egypt \textit{in the direction of} Zoar.
\verse So Lot chose for himself all the plain of the Jordan. And Lot journeyed from the east, and so they separated \textit{from each other}.
\verse Abram settled in the land of Canaan, and Lot settled in the cities of the plain. And he pitched his tent toward Sodom.
\verse Now the men of Sodom \textit{were extremely wicked sinners against Adonai}.
\verse And Adonai said to Abram after Lot had separated from him, “Now, lift up your eyes and look from the place where you \textit{are} to the north, and to the south, and to the east and to the west,
\verse for all the land which you see I will give to you, and to your descendants, forever.
\verse I will make your descendants like the dust of the earth which, if anyone were able to count the dust of the earth, your descendants would be \textit{so} counted.
\verse Arise, go through the length of the land and through its breadth, for I will give it to you.”
\verse So Abram pitched his tent, and he came and settled at the oaks of Mamre, which \textit{were} at Hebron. And there he built an altar to Adonai.
\end{biblechapter}

\begin{biblechapter} % Genesis 14
\verseWithHeading{Abram Rescues Lot} And it happened \textit{that} in the days of Amraphel, the king of Shinar, Arioch, the king of Ellasar, Kedorlaomer, the king of Elam, and Tidal, the king of Goiim,
\verse made war with Bera, the king of Sodom, and Birsha, the king of Gomorrah, Shinab, the king of Admah, and Shemeber, the king of Zeboiim, and the king of Bela (that \textit{is}, Zoar).
\verse All these joined forces at the valley of Siddim (that \textit{is}, the sea of the salt).
\verse Twelve years they had served Kedorlaomer, but in the thirteenth year they rebelled.
\verse In the fourteenth year Kedorlaomer and the kings who \textit{were} with him came and defeated the Rephaim in Ashteroth-Karnaim, and the Zuzim in Ham, and the Emim in Shaveh-Kiriathaim,
\verse And the Horites in their hill country of Seir, as far as El-Paran, which is at the wilderness.
\verse Then they turned back and came to En-Mishpat (that \textit{is}, Kadesh). And they defeated the whole territory of the Amalekites, and also the Amorites who were living in Hazazon-Tamar.
\verse Then the king of Sodom, the king of Gomorrah, the king of Admah, the king of Zeboiim, and the king of Bela (that \textit{is}, Zoar) went out, and \textit{they took up battle position} in the Valley of Siddim
\verse with Kedorlaomer, king of Elam, and Tidal, king of Goiim, and Amraphel, king of Shinar, and Arioch, king of Ellasar, four kings against five.
\verse Now the Valley of Siddim \textit{was full of tar pits}. And the kings of Sodom and Gomorrah fled and \textit{fell into them}, but the rest fled to the mountains.
\verse So they took all the possessions of Sodom and Gomorrah and all their provisions, and they left.
\verse And they took Lot, the son of the brother of Abram, and his possessions, and left. (Now he had been living in Sodom.)
\verse Then one who escaped came and told Abram the Hebrew. And he was living at the oaks of Mamre the Amorite, brother of Eshcol and brother of Aner. \textit{They were allies with Abram}.
\verse When Abram heard that his \textit{relative} was taken captive, he summoned his trained men, born in his house, three hundred and eighteen \textit{of them}, and he went in pursuit up to Dan.
\verse And he divided \textit{his trained men} against them at night, he and his servants. And he defeated them and pursued them to Hobah, which \textit{is} north of Damascus.
\verse And he brought back all the possessions. And he also brought back Lot, his \textit{relative}, and his possessions, and the women and the people as well.
\verseWithHeading{Abram Meets Melchizedek} After his return from defeating Kedorlaomer and the kings who \textit{were} with him, the king of Sodom went out to meet him at the Valley of Shaveh (that \textit{is}, the Valley of the King).
\verse And Melchizedek, the king of Salem, brought out bread and wine. (He was the priest of God Most High).
\verse And he blessed him and said,
\verse “Blessed \textit{be} Abram by God Most High, 
Maker of heaven and earth.
\verse And he gave to him a tenth of everything.
\verse And the king of Sodom said to Abram, “Give me the people, but the possessions take for yourself.”
\verse And Abram said to the king of Sodom, “I have raised my hand to Adonai, God Most High, Maker of heaven and earth,
\verse \textit{that neither a thread nor} a thong of a sandal would I take from all that \textit{belongs} to you, that you might not say, ‘I made Abram rich.’
\end{biblechapter}

\begin{biblechapter} % Genesis 15
\verseWithHeading{Adonai’s Covenant with Abram} After these things the word of Adonai came to Abram in a vision, saying: “Do not be afraid, Abram; I \textit{am} your shield, \textit{and} your reward \textit{shall be} very great.”
\verse Then Abram said, “O Adonai, my Lord, what will you give me? \textit{I continue to be} childless, and \textit{my heir} is Eliezer of Damascus.”
\verse And Abram said, “Look, you have not given me a descendant, and here, \textit{a member of my household} \textit{is} \textit{my heir}.”
\verse And behold, the word of Adonai \textit{came} to him saying, “This \textit{person} will not \textit{be your heir}, but \textit{your own son will be your heir}.”
\verse And he brought him outside and said, “Look toward the heavens and count the stars if you are able to count them.” And he said to him, “So shall your offspring be.”
\verse And he believed in Adonai, and he reckoned it to him \textit{as} righteousness.
\verse And he said to him, “I \textit{am} Adonai, who brought you out from Ur of the Chaldeans to give this land to you, to possess it.”
\verse And he said, “O Adonai, how shall I know \textit{that} I will possess it?”
\verse And he said to him, “Take for me a three-year-old heifer, and a three-year-old female goat, and a three-year-old ram, and a turtledove and a young pigeon.”
\verse And he took for him all these and cut them in pieces down the middle. And he put each piece opposite \textit{the other}, but the birds he did not cut.
\verse And the birds of prey came down on the carcasses, but Abram drove them away.
\verse And it happened, as the sun \textit{went down}, then a deep sleep fell upon Abram and, behold, a great terrifying darkness fell upon him.
\verse And he said to Abram, “\textit{You must surely know} that your descendants shall be \textit{as} aliens in a land \textit{not their own}. And they shall serve them and they shall oppress them four hundred years.
\verse And also the nation that they serve I will judge. Then afterward they shall go out with great possessions.
\verse And \textit{as for} you, you shall go to your ancestors in peace; you shall be buried in a good old age.
\verse And the fourth generation shall return here, for the guilt of the Amorites \textit{is not yet complete}.”
\verse And after the sun had gone down and it \textit{was} dusk, behold, a smoking firepot and a flaming torch passed between those half pieces.
\verse On that day Adonai \textit{made} a covenant with Abram saying, “To your offspring I will give this land, from the river of Egypt to the great river, the Euphrates river,
\verse \textit{the land of} the Kenites, the Kenizzites, the Kadmonites,
\verse the Hittites, the Perizzites, the Rephaim,
\verse the Amorites, the Canaanites, the Girgashites, and the Jebusites.”
\end{biblechapter}

\begin{biblechapter} % Genesis 16
\verseWithHeading{Sarai and Hagar} Now Sarai, the wife of Abram, had borne him no children. And she had a female Egyptian servant, and her name \textit{was} Hagar.
\verse And Sarai said to Abram, “Look, please, Adonai has prevented me from bearing children. Please go in to my servant; perhaps \textit{I will have children by her}.” And Abram listened to the voice of Sarai.
\verse Then Sarai, the wife of Abram, took Hagar, her Egyptian servant, after Abram had lived ten years in the land of Canaan, and gave her to Abram her husband as his wife.
\verse And he went in to Hagar, and she conceived. And \textit{when} she saw that she had conceived, then her mistress grew small in her eyes.
\verse And Sarai said to Abram, “may my harm \textit{be} upon you. \textit{I had my servant sleep with you}, and \textit{when} she saw that she had conceived, \textit{she no longer respected me}. May Adonai judge between me and you!”
\verse And Abram said to Sarai, “Look, your servant \textit{is} \textit{under your authority}. Do to her that which \textit{is} good in your eyes.” And Sarai mistreated her, and she fled from her presence.
\verseWithHeading{Hagar and the Angel of Adonai} And the angel of Adonai found her at a spring of water in the wilderness, at the spring by the road of Shur.
\verse And he said to Hagar, the servant of Sarai, “\textit{From where} have you come, and where are you going?” And she said, “I am fleeing from the presence of Sarai my mistress.”
\verse Then the angel of Adonai said to her, “Return to your mistress and submit yourself under \textit{her authority}.”
\verse And the angel of Adonai said to her, “\textit{I will greatly multiply} your offspring, so that they cannot be counted for \textit{their} abundance.”
\verse And the angel of Adonai said to her:
\verse “Behold, you are pregnant 
and shall have a son. 
And you shall call his name Ishmael, 
for Adonai has listened to your suffering.
\verse So she called the name of Adonai who spoke to her, “You \textit{are} El-Roi,” for she said, “Here I have seen after he who sees me.”
\verse Therefore the well was called Beer-Lahai-Roi; behold, it \textit{is} between Kadesh and Bered.
\verse And Hagar had a child for Abram, a son. And Abram called the name of his son whom Hagar bore to him, Ishmael.
\verse And Abram \textit{was} eighty-six years old when Hagar bore Ishmael to Abram.
\end{biblechapter}

\begin{biblechapter} % Genesis 17
\verseWithHeading{Abram and Circumcision, the Sign of the Covenant} When Abram \textit{was} ninety-nine years old Adonai appeared to Abram. And he said to him, “I \textit{am} El-Shaddai; walk before me and be blameless
\verse so that I may make my covenant between me and you, and may multiply you \textit{exceedingly}.”
\verse Then Abram fell upon his face and God spoke with him, saying,
\verse “\textit{As for} me, behold, my covenant \textit{shall be} with you, and you shall be the father of a multitude of nations.
\verse Your name shall no longer be called Abram, but your name shall be Abraham, for I will make you the father of a multitude of nations.
\verse And I will make you \textit{exceedingly} fruitful. I will make you a nation, and kings shall go out from you.
\verse And I will establish my covenant between me and you, and between your offspring after you, throughout their generations as an everlasting covenant to be as God for you and to your offspring after you.
\verse And I will give to you and to your offspring after you \textit{the land in which you are living as an alien}, all the land of Canaan, as an everlasting property. And I will be to them as God.”
\verse And God said to Abraham, “Now \textit{as for} you, you must keep my covenant, you and your offspring after you, throughout their generations.
\verse This \textit{is} my covenant which you shall keep, between me and you, and \textit{also} with your offspring after you: Every male among you shall be circumcised.
\verse And you shall circumcise the flesh of your foreskin, and it shall be a sign of the covenant between me and you.
\verse And \textit{at eight days of age} you shall yourselves circumcise every male \textit{belonging} to your generations \textit{and} \textit{the servant born in your house and the one bought from any foreigner} who is not from your offspring.
\verse \textit{You must certainly circumcise} \textit{the servant born in your house and the one bought from any foreigner}. And my covenant shall be with your flesh as an everlasting covenant.
\verse And \textit{as for any} uncircumcised male who has not circumcised the flesh of his foreskin, that person shall be cut off from his people. He has broken my covenant.
\verse And God said to Abraham, “\textit{as for} Sarai your wife, you shall not call her name Sarai, for Sarah \textit{shall be} her name.
\verse And I will bless her; moreover, I give to you from her a son. And I will bless her, and \textit{she shall give rise to nations}. Kings of peoples shall come from her.”
\verse And Abraham fell upon his face and laughed. And he said in his heart, “\textit{Can a child be born to a man a hundred years old}, or \textit{can Sarah bear a child at ninety}?”
\verse And Abraham said to God, “Oh that Ishmael might live before you!”
\verse And God said, “No, but Sarah your wife shall bear a son for you, and you shall call his name Isaac. And I will establish my covenant with him as an everlasting covenant to his offspring after him.
\verse And \textit{as for} Ishmael, I have heard you. Behold, I will bless him and I will make him fruitful, and I will multiply him \textit{exceedingly}. He shall father twelve princes, and I will make him a great nation.
\verse But my covenant I will establish with Isaac, whom Sarah shall bear to you at this appointed time next year.”
\verse When he finished speaking with him, God went up from Abraham.
\verse And Abraham took Ishmael his son and all who were born of his house, and all \textit{those} acquired by his money, every male among the men of Abraham’s house, and he circumcised the flesh of their foreskin on the same day that God spoke with him.
\verse Abraham \textit{was} ninety-nine years old when he circumcised the flesh of his foreskin.
\verse And Ishmael his son \textit{was} thirteen years old when he circumcised the flesh of his foreskin.
\verse Abraham and his son Ishmael \textit{were} circumcised on the same day.
\verse And all the men of his house, \textit{those born in the house, and those acquired by money from a foreigner}, were circumcised with him.
\end{biblechapter}

\begin{biblechapter} % Genesis 18
\verseWithHeading{Adonai Appears to Abraham as a Man} And Adonai appeared to him by the oaks of Mamre. And he was sitting in the doorway of the tent at the heat of the day.
\verse And he lifted up his eyes and saw, and behold, three men were standing near him. And he saw \textit{them} and ran from the doorway of the tent to meet them. And he bowed down to the ground.
\verse And he said, “My lord, if I have found favor in your eyes do not pass by your servant.
\verse Let a little water be brought and wash your feet, and rest under the tree.
\verse And let me bring a piece of bread, then refresh \textit{yourselves}. Afterward you can pass on, \textit{once} you have passed by with your servant.” Then they said, “Do so as you have said.”
\verse Then Abraham hastened into the tent to Sarah, and he said, “Quickly—make three seahs of fine flour for kneading and make bread cakes!”
\verse And Abraham ran to the cattle and took a \textit{calf}, tender and good, and gave it to the servant, and he made haste to prepare it.
\verse Then he took curds and milk, and the calf which he prepared, and set \textit{it} before them. And he was standing by them under the tree while they ate.
\verse And they said to him, “Where \textit{is} Sarah your wife?” And he said, “Here, in the tent.”
\verse And he said, “I will certainly return to you \textit{in the spring}, and look, Sarah your wife \textit{will have} a son.” Now Sarah \textit{was} listening at the doorway of the tent, and which \textit{was} behind him.
\verse Now Abraham and Sarah \textit{were} old, \textit{advanced in age}; \textit{the way of women} had ceased to be for Sarah.
\verse So Sarah laughed to herself saying, “After I am worn out and my husband is old, shall \textit{this} pleasure be to me?”
\verse Then Adonai said to Abraham, “What \textit{is} this \textit{that} Sarah laughed, saying, ‘Is it indeed true \textit{that} I will bear a child, now \textit{that} I have grown old?’
\verse Is anything too difficult for Adonai? At the appointed time I will return to you \textit{in the spring} and Sarah \textit{shall have} a son.”
\verse But Sarah denied \textit{it}, saying, “I did not laugh,” because she was afraid. He said, “No, but you did laugh.”
\verse Then the men set out from there, and they looked down upon Sodom. And Abraham went with them \textit{to send them on their way}.
\verse Then Adonai said, “Shall I conceal from Abraham what I \textit{am going} to do?
\verse Abraham will surely become a great and strong nation, and all the nations of the earth will be blessed on account of him.
\verse For I have chosen him, that he will command his children and his household after him that they will keep the way of Adonai, to do righteousness and justice, so that Adonai may bring upon Abraham that which he said to him.”
\verse Then Adonai said, “Because the outcry of Sodom and Gomorrah \textit{is} great and because their sin \textit{is} very \textit{serious},
\verse I will go down and I will see. Have they done altogether according to its cry of distress \textit{which} has come to me? If not, I will know.”
\verseWithHeading{Abraham Intercedes for Sodom} And the men turned from there and went toward Sodom. And Abraham \textit{was} still standing before Adonai.
\verse And Abraham drew near \textit{to Adonai} and said, “Will you also sweep away the righteous with the wicked?
\verse If perhaps there are fifty righteous in the midst of the city, will you also sweep \textit{them} away and not forgive the place on account of the fifty righteous in her midst?
\verse Far be it from you to do such a thing as this, to kill \textit{the} righteous with \textit{the} wicked, that the righteous would be as the wicked! Far be it from you! Will not the Judge of all the earth do justice?”
\verse And Adonai said, “If I find fifty righteous in Sodom, in the midst of the city, then I will forgive the whole place for their sake.”
\verse Then Abraham answered and said, “Look, please, I was bold to speak to my Lord, but I \textit{am} dust and ashes.
\verse Perhaps the fifty righteous are lacking five—will you destroy the whole city on account of the five?” And he answered, “I will not destroy \textit{it} if I find forty-five there.”
\verse And \textit{once again he spoke} to him and said, “What if forty are found there?” And he answered, “I will not do \textit{it} on account of the forty.”
\verse And he said, “Please, let not my Lord be angry, and I will speak. What if thirty be found there?” And he answered, “I will not do \textit{it} if I find thirty there.”
\verse And he said, “Please, now, I was bold to speak to my Lord. What if twenty be found there?” And he answered, “I will not destroy \textit{it} for the sake of the twenty.”
\verse And he said, “Please, let not my Lord be angry, and I will speak only once more. What if ten are found there?” And he answered, “I will not destroy \textit{it} for the sake of the ten.”
\verse Then Adonai left, as he finished speaking to Abraham, and Abraham returned to his place.
\end{biblechapter}

\begin{biblechapter} % Genesis 19
\verseWithHeading{The Rescue of Lot from Sodom} And the two angels came to Sodom in the evening. And Lot was sitting in the gateway of Sodom. Then Lot saw \textit{them} and stood up to meet them. And he bowed down \textit{with his} face to the ground.
\verse And he said, “Behold, my lords, please turn aside into the house of your servant and spend the night and wash your feet. Then you can rise early and go on your way.” And they said, “No, but we will spend the night in the square.”
\verse But \textit{he urged them strongly}, and they turned aside with him and came into his house. And he made a meal for them and baked unleavened bread, and they ate.
\verse Before they laid down, the men of the city, the men of Sodom, both young and old, all the people \textit{to the last man}, surrounded the house.
\verse And they called to Lot and said to him, “Where \textit{are} the men who came to you tonight? Bring them out to us so that we may know them.”
\verse But Lot went out to them at the entrance, and he shut the door behind him.
\verse And he said, “No, my brothers, please do not do \textit{such a} wrong \textit{thing}.
\verse Behold, I have two daughters who have not known a man. Please, let me bring them out to you; then do to them as \textit{it seems} good in your eyes. Only to these men do not do \textit{this} thing, since they came under\textit{ my roof} for protection.”
\verse But they said, “Step aside!” Then they said, “\textit{This fellow} came to dwell as a foreigner and he acts as a judge! Now we shall do worse to you than them!” And they pressed very hard against the man, against Lot, and they drew near to break the door.
\verse Then the men reached out \textit{with} their hands and brought Lot in to them, into the house, and they shut the door.
\verse And the men who \textit{were} at the entrance of the house they struck with blindness, both small and great, and they were unable to find the entrance.
\verse Then the men said to Lot, “Who \textit{is} here with you? Bring out from the place \textit{your} sons-in-law, and your sons and your daughters, and all who \textit{are} with you in the city.
\verse For we are \textit{about to} destroy this place, because their cry has become great before Adonai. Adonai sent us to destroy it.”
\verse Then Lot went out and spoke to his sons-in-law \textit{who were} taking his daughters and said, “Get up! Go out from this place, because Adonai \textit{is going} to destroy the city!” But \textit{it seemed like a joke} in the eyes of his sons-in-law.
\verse And as the dawn came up the angels urged Lot saying, “Get up, take your wife and your two daughters \textit{who are staying with you}, lest you be destroyed with the punishment of the city.”
\verse But \textit{when} he lingered, the men seized him by his hand and his wife’s hand, and his two daughters by hand, on account of the mercy of Adonai upon him. And they brought him out and set him outside of the city.
\verse And after bringing them outside \textit{one} said, “Flee for your life; do not look behind you, and do not stand anywhere in the plain. Flee to the mountains lest you be destroyed.”
\verse And Lot said to them, “No, please, my lords.
\verse Behold, your servant has found favor in your eyes and \textit{you have shown me great kindness} in saving my life. But I cannot flee to the mountains, lest the disaster overtake me and I die.
\verse Behold, this city \textit{is} near \textit{enough} to flee there, and it \textit{is a} little \textit{one}. Please, let me flee there. Is it not a little \textit{one}? Then my life shall be saved.”
\verse And he said to him, “Behold, \textit{I will grant this favor as well}; that I will not overthrow the city of which you have spoken.
\verse Escape there quickly, for I cannot do \textit{this} thing until you get there.” Therefore, there name of the city was called Zoar.
\verseWithHeading{The Destruction of Sodom} \textit{After} the sun \textit{had risen} upon the earth and Lot had entered Zoar,
\verse Adonai rained down from heaven upon Sodom and Gomorrah brimstone and fire from Adonai.
\verse And he overthrew those cities and the whole plain, and the inhabitants of the cities and the vegetation of the ground.
\verse But his wife looked \textit{back}, and she became a pillar of salt.
\verse And Abraham arose early in the morning \textit{and went} to the place where he had stood before Adonai.
\verse And he looked down upon the surface of Sodom and Gomorrah, and upon the whole surface of the land, the plain. And he saw that, behold, the smoke of the land went up like the smoke of a smelting furnace.
\verse So it was, when God destroyed the cities of the plain that God remembered Abraham and sent Lot out from the midst of the overthrow, when he overthrew the cities in which Lot lived.
\verseWithHeading{Lot and His Daughters} And Lot went out from Zoar and settled in the hill country with his two daughters, for he was afraid to stay in Zoar. So he lived in a cave, he and his two daughters.
\verse And the firstborn \textit{daughter} said to the younger one, “Our father is old, and there is no man in the land to come in to us according to the manner of all the land.
\verse Come, let us give our father wine to drink and let us lie with him that \textit{we may secure descendants through our father}.”
\verse And they gave their father wine to drink that night, and the firstborn went and lay with her father, but he did not know when she lay down or when she got up.
\verse And it happened \textit{that}, the next day the firstborn said to the younger one, “Look, I laid with my father last night. Let us give him wine to drink also tonight, then go and lie with him that \textit{we may secure descendants through our father}.”
\verse And they gave their father wine to drink again that night, and the younger got up and lay with him, but he did not know when she lay down or when she got up.
\verse And the two daughters of Lot became pregnant by their father.
\verse The firstborn gave birth to a son, and she called his name Moab. He \textit{is} the father of Moab until this day.
\verse And the younger, she also gave birth to a son, and she called his name Ben-Ammi. He \textit{is} the father of the \textit{Ammonites} until this day.
\end{biblechapter}

\begin{biblechapter} % Genesis 20
\verseWithHeading{Abraham and Abimelech} And Abraham journeyed from there to the land of the Negev, and he settled between Kadesh and Shur. And he dwelled as an alien in Gerar.
\verse And Abraham said about Sarah his wife, “She \textit{is} my sister.” And Abimelech king of Gerar sent and took Sarah.
\verse And God came to Abimelech in a dream at night. And he said to him, “Look, you \textit{are} a dead man on account of the woman you have taken. For she \textit{is} \textit{a married woman}.”
\verse Now Abimelech had not approached her, so he said, “my Lord, will you even kill a righteous people?”
\verse Did not he himself say to me, ‘She \textit{is} my sister’? And she herself said, ‘He \textit{is} my brother.’ With integrity of my heart and with cleanness of my hands I did this.”
\verse Then God said to him in the dream, “Yes, I know that in the integrity of your heart you did this, and I also \textit{kept you from sinning} against me. Therefore, I did not allow you to touch her.
\verse So now, return the wife of the man, for he \textit{is} a prophet, so that he will pray for you and you will live. And \textit{if you do not return her}, know that you will certainly die, and all that \textit{are} yours.”
\verse So Abimelech rose early in the morning. And he called all his servants and \textit{told them all these things}, and the men were very afraid.
\verse And Abimelech called for Abraham and said to him, “What have you done to us? And how have I sinned against you that you brought upon me and upon my kingdom a great sin? You have done things to me that should not be done.”
\verse And Abimelech said to Abraham, “\textit{What were you thinking} that you did this thing?”
\verse And Abraham said, “Because I thought, surely there is no fear of God in this place; they will kill me on account of the matter of my wife.
\verse \textit{Besides}, she \textit{is} my sister, the daughter of my father, but not the daughter of my mother. And she became my wife.
\verse And it happened \textit{that} as God caused me to wander from the house of my father I said to her, ‘This \textit{is} your loyal kindness that you must do for me at every place where we come: say concerning me, “He \textit{is} my brother.” ’ ”
\verse And Abimelech took sheep and cattle and male slaves and female slaves, and he gave \textit{them} to Abraham. And he returned Sarah his wife to him.
\verse And Abimelech said, “Here \textit{is} my land before you; settle \textit{where it pleases you}.”
\verse And to Sarah he said, “Look, I have given a thousand \textit{pieces of} silver to your brother. It \textit{shall be} \textit{an exoneration}. \textit{You are vindicated before all who are with you}.”
\verse And Abraham prayed to God, and God healed Abimelech and his wife and his female servants so that they \textit{could} bear children \textit{again}.
\verse For Adonai had completely closed up all the wombs of the house of Abimelech because of the matter of Sarah, the wife of Abraham.
\end{biblechapter}

\begin{biblechapter} % Genesis 21
\verseWithHeading{The Birth of Isaac} And Adonai visited Sarah as he had said. And Adonai did to Sarah as he had promised.
\verse And she conceived, and Sarah bore to Abraham a son in his old age at the appointed time that God had told him.
\verse And Abraham called the name of his son who was born to him, whom Sarah bore to him, Isaac.
\verse And Abraham circumcised Isaac his son \textit{when he was} eight days old, as God had commanded him.
\verse And Abraham \textit{was} one hundred years old when Isaac his son was born to him.
\verse And Sarah said, “God has made laughter for me; all who hear will laugh for me.”
\verse And she said, “Who would announce to Abraham \textit{that} Sarah would nurse children? Yet I have borne a son \textit{to Abraham} in his old age.”
\verseWithHeading{Hagar and Ishmael} And the child grew and was weaned. And Abraham made a great feast on the day Isaac was weaned.
\verse And Sarah saw the son of Hagar the Egyptian, whom she had borne Abraham, mocking.
\verse Then she said to Abraham, “Drive out this slave woman and her son, for the son of this slave woman will not be heir with my son, with Isaac.”
\verse And the matter \textit{displeased Abraham very much} on account of his son.
\verse Then God said to Abraham, “\textit{Do not be displeased} on account of the boy and on account of the slave woman. \textit{Listen to everything that Sarah said to you}, for through Isaac \textit{your} offspring will be named.
\verse And I will also make the son of the slave woman into a nation, for he is your offspring.”
\verse Then Abraham rose up early in the morning and took bread and a skin of water and gave \textit{it} to Hagar, putting \textit{it} on her shoulder. And he sent her away with the child, and she went, wandering about in the wilderness, in Beersheba.
\verse And when the water was finished from the skin, she put the child under one of the bushes.
\verse And she went and \textit{she sat a good distance away}, for she said, “Let me not see the child’s death.” So she sat away from him and lifted up her voice and wept.
\verse And God heard the cry of the boy and the angel of God called to Hagar from the heavens and said to her, “\textit{What is the matter} Hagar? Do not be afraid, for God has heard the cry of the boy \textit{from where he is}.
\verse Get up, take up the boy and take him with your hand, for I will make him a great nation.”
\verse And God opened her eyes, and she saw a well of water. And she went and filled the skin with water and gave a drink to the boy.
\verse And God was with the boy, and he grew and lived in the wilderness. And he became \textit{an expert with a bow}.
\verse And he lived in the wilderness of Paran. And his mother took a wife for him from the land of Egypt.
\verseWithHeading{The Covenant Between Abraham and Abimelech} And it happened \textit{that} at that time, Abimelech and Phicol, the commander of his army, said to Abraham, “God \textit{is} with you, in all that you do.
\verse So now, swear to me here by God \textit{that} you will not deal with me falsely, or with my descendants, or my posterity. According to the kindness that I have done to you, you shall \textit{pledge} to do with me and with the land where you have dwelled as an alien.”
\verse And Abraham said, “I swear.”
\verse Then Abraham complained to Abimelech on account of the well of water that servants of Abimelech had seized.
\verse And Abimelech said, “I do not know who did this thing, neither did you tell me, nor have I heard \textit{of it} except for today.”
\verse And Abraham took sheep and cattle and gave \textit{them} to Abimelech. And the two of them \textit{made} a covenant.
\verse Then Abraham set \textit{off} seven ewe-lambs of the flock by themselves.
\verse And Abimelech said to Abraham, “What \textit{is the meaning of} these seven ewe-lambs that you have set \textit{off} by themselves?”
\verse And he said, “You shall take the seven ewe-lambs from my hand \textit{as proof on my behalf} that I dug this well.”
\verse Therefore that place is called Beersheba, because there the two of them swore an oath.
\verse And they \textit{made} a covenant at Beersheba. And Abimelech, and Phicol his army commander stood and returned to the land of the Philistines.
\verse And he planted a tamarisk tree in Beersheba, and there he called on the name of Adonai, \textit{the everlasting God}.
\verse And Abraham dwelled as an alien in the land of the Philistines many days.
\end{biblechapter}

\begin{biblechapter} % Genesis 22
\verseWithHeading{God Tests Abraham} And it happened \textit{that} after these things, God tested Abraham. And he said to him, “Abraham!” And he said, “Here I \textit{am}.”
\verse And he said, “Take your son, your only child, Isaac, whom you love, and go to the land of Moriah, and offer him there as a burnt offering on one of the mountains where I will tell you.”
\verse And Abraham rose up early in the morning and saddled his donkey. And he took two of his servants with him, and Isaac his son. And he chopped wood for a burnt offering. And he got up and went to the place which God had told him.
\verse On the third day Abraham lifted up his eyes, and he saw the place at a distance.
\verse And Abraham said to his servants, “You stay here with the donkey, and I and the boy will go up there. We will worship, then we will return to you.”
\verse And Abraham took the wood of the burnt offering and placed \textit{it} on Isaac his son. And he took the fire in his hand and the knife, and the two of them went together.
\verse And Isaac said to Abraham his father, “My father!” And he said, “Here I \textit{am}, my son.” And he said, “Here is the fire and the wood, but where is the lamb for a burnt offering?”
\verse And Abraham said, “\textit{God will provide} the lamb for a burnt offering, my son.” And the two of them went together.
\verse And they came to the place that God had told him. And Abraham built an altar there and arranged the wood. Then he bound Isaac his son and placed him on the altar atop the wood.
\verse And Abraham stretched out his hand and took the knife to slaughter his son.
\verse And the angel of Adonai called to him from heaven and said, “Abraham! Abraham!” And he said, “Here I \textit{am}.”
\verse And he said, “Do not stretch out your hand against the boy; do not do anything to him. For now I know that you are \textit{one who fears} God, since you have not withheld your son, your only child, from me.”
\verse And Abraham lifted up his eyes and looked. And behold, a ram was caught in the thicket by his horns. And Abraham went and took the ram, and offered it as a burnt offering in place of his son.
\verse And Abraham called the name of that place “Adonai \textit{will provide},” \textit{for which reason} it is said today, “on the mountain of Adonai \textit{it shall be provided}.”
\verse And the angel of Adonai called to Abraham a second time from heaven.
\verse And he said, “I swear by myself, declares Adonai, that because you have done this thing and have not withheld your son, your only child,
\verse that I will certainly bless you and greatly multiply your offspring as the stars of heaven, and as the sand that is by the shore of the sea. And your offspring will take possession of the gate of his enemies.
\verse All the nations of the earth will be blessed through your offspring, because you have listened to my voice.”
\verse And Abraham returned to his servants, and they got up and went together to Beersheba. And Abraham lived in Beersheba.
\verse And it happened \textit{that} after these things, it was told to Abraham, “Look, Milcah has also borne children to your brother Nahor:
\verse Uz his firstborn and Buz his brother, and Kemuel the father of Aram,
\verse and Kesed, Hazo, Pildash, Jidlaph, and Bethuel.”
\verse (Now, Bethuel fathered Rebekah). These eight Milcah bore to Nahor, the brother of Abraham.
\verse And his concubine, whose name was Reumah, also bore Tebah, Gaham, Tahash, and Maacah.
\end{biblechapter}

\begin{biblechapter} % Genesis 23
\verseWithHeading{Sarah’s Death and Burial} And \textit{Sarah lived} one hundred and twenty-seven years; \textit{these were} the years of the life of Sarah.
\verse And Sarah died in Kiriath Arba; that \textit{is} Hebron, in the land of Canaan.
\verse And Abraham went to mourn for Sarah and to weep for her. And Abraham rose up from his dead, and he spoke to the Hittites \textit{and} said,
\verse “I \textit{am} a stranger and an alien among you; give to me \textit{my own burial site} among you so that I may bury my dead from before me.”
\verse And the Hittites answered Abraham \textit{and} said to him,
\verse “Hear us, my lord, you \textit{are} a mighty prince in our midst. Bury your dead in the choicest of our burial sites. None of us \textit{will withhold his burial site} from you \textit{for} burying your dead.”
\verse And Abraham rose up and bowed to the people of the land, to the Hittites.
\verse And he spoke with them, saying, “\textit{If you are willing} \textit{that} I bury my dead from before me, hear me and intercede for me with Ephron the son of Zohar,
\verse that he may sell to me the cave of Machpelah which \textit{belongs to him}, which \textit{is} at the end of his field. At full value let him sell \textit{it} to me in your midst as \textit{a burial site}.”
\verse Now Ephron was sitting among the Hittites. And Ephron the Hittite answered Abraham in the hearing of the Hittites with respect to all \textit{who were} entering the gate of his city, \textit{and} said,
\verse “No, my lord, hear me. I give you the field and the cave which \textit{is} in it, I \textit{also} give it to you in the sight of the children of my people I give it to you. Bury your dead.”
\verse And Abraham bowed before the people of the land.
\verse And he spoke to Ephron in the hearing of the people of the land, saying, “\textit{If only you will hear me}—I give the price of the field. Take \textit{it} from me that I may bury my dead there.”
\verse And Ephron answered Abraham, saying to him,
\verse “My lord, hear me. A \textit{piece of} land \textit{worth} four hundred shekels of silver—what \textit{is} that between me and you? Bury your dead.”
\verse Then Abraham listened to Ephron, and Abraham weighed for Ephron the silver that he had named in the hearing of the Hittites: four hundred shekels of silver \textit{at the merchants’ current rate}.
\verse So the field of Ephron which \textit{was} in the Machpelah, which \textit{was} near Mamre—the field and the cave which \textit{was} in it, with all the trees that \textit{were} in the field, which \textit{were} within all its surrounding boundaries—\textit{passed}
\verse to Abraham as a property in the presence of the Hittites, with respect to all \textit{who were} entering the gate of his city.
\verse And thus afterward Abraham buried Sarah his wife in the cave of the field of Machpelah near Mamre; that \textit{is} Hebron, in the land of Canaan.
\verse And the field and the cave which \textit{was} in it \textit{passed} to Abraham as \textit{a burial site} from the Hittites.
\end{biblechapter}

\begin{biblechapter} % Genesis 24
\verseWithHeading{Isaac and Rebekah} Now Abraham \textit{was} old, \textit{advanced in age}, and Adonai had blessed Abraham in everything.
\verse And Abraham said to his servant, the oldest of his house, who had charge of all he had, “Please put your hand under my thigh
\verse that I may make you swear by Adonai, the God of heaven and the God of earth, that you will not take a wife for my son from the daughters of the Canaanites in whose midst I am dwelling,
\verse but that you will go to my land and to my family, and take a wife for my son, for Isaac.”
\verse And the servant said to him, “Perhaps the woman will not be willing \textit{to follow} me to this land—must I then return your son to the land from whence you came?”
\verse Abraham said to him, “\textit{You must take care} that you do not return my son there.
\verse Adonai, the God of heaven who took me from the house of my father and from the land of my family, and who spoke to me and swore to me, saying, ‘to your offspring I will give this land,’ he will send his angel before you, and you shall take a wife for my son from there.
\verse And if the woman is not willing \textit{to follow} you, then you shall be released from this oath of mine—only you must not return my son there.”
\verse Then the servant put his hand under the thigh of Abraham his master, and he swore to him concerning this matter.
\verse And the servant took ten camels from his master’s camels, and he went with all \textit{kinds of} his master’s good things in his hand. And he arose and went to Aram-Naharaim, to the city of Nahor.
\verse And he made the camels kneel outside the city at the well of water, at the time of evening, toward the time \textit{the women} went out to draw water.
\verse And he said, “O Adonai, God of my master Abraham, please grant me success today and show loyal love to my master Abraham.
\verse Behold, I am standing by the spring of water, and the daughters of the men of the city are going out to draw water.
\verse And let it be \textit{that} the girl to whom I shall say, ‘Please, offer your jar that I may drink’ and \textit{who} says, ‘Drink—and I will also water your camels,’ she \textit{is the one} you have chosen for your servant, for Isaac. By her I will know that you have shown loyal love to my master.”
\verse And it happened \textit{that} before he had finished speaking, behold, Rebekah—who was born to Bethuel, son of Milcah, the wife of Nahor, the brother of Abraham—came out, and her jar \textit{was} on her shoulder.
\verse Now the girl \textit{was} very pleasing in appearance. \textit{She was} a virgin; no man had known her. And she went down to the spring, filled her jar, and came up.
\verse And the servant ran to meet her. And he said, “Please, let me drink a little of the water from your jar.”
\verse And she said, “Drink, my lord.” And she quickly lowered her jar in her hand and gave him a drink.
\verse When she finished giving him a drink she said, “I will also draw water for your camels until they finish drinking.”
\verse And she quickly emptied her jar into the trough and ran again to the well to draw water. And she drew water for all his camels.
\verse And the man \textit{was} gazing at her silently to know \textit{if} Adonai had made his journey successful or not.
\verse And it happened \textit{that} as the camels finished drinking the man took a gold ring of a half shekel in weight and two bracelets for her arms, ten shekels in weight,
\verse and said, “Please tell me, whose daughter \textit{are} you? Is there a place \textit{at} the house of your father for us to spend the night?”
\verse And she said to him, “I \textit{am} the daughter of Bethuel, son of Milcah, whom she bore to Nahor.”
\verse Then she said to him, “We have both straw and fodder in abundance, as well as a place to spend the night.”
\verse And the man knelt down and worshiped Adonai.
\verse And he said, “Blessed \textit{be} Adonai, God of my master Abraham, who has not withheld his loyal love and his faithfulness from my master. I \textit{was} on the way \textit{and} Adonai led me \textit{to} the house of my master’s brother.”
\verse Then the girl ran and reported these things to the household of her mother.
\verse Now Rebekah had a brother, and his name \textit{was} Laban. And Laban ran out to the man toward the spring.
\verse And when he saw the ring and the bracelets on the arms of his sister and heard the words of Rebekah his sister, \textit{who} said, “Thus the man spoke to me,” he went to the man. And behold, \textit{he was} standing with the camels at the spring.
\verse And he said, “Come, O blessed \textit{one} of Adonai. Why do you stand outside? Now I have prepared the house and a place for the camels.”
\verse And the man came to the house and unloaded the camels. And he gave straw and fodder to the camels, and water to wash his feet and the feet of the men who \textit{were} with him.
\verse \textit{And food was placed before him} to eat. And he said, “I will not eat until \textit{I have told my errand}.” And he said, “Speak.”
\verse And he said, “I \textit{am} the servant of Abraham.
\verse Now Adonai has blessed my master exceedingly, and he has become great. He has given to him sheep and cattle, silver and gold, male slaves and female slaves, and camels and donkeys.
\verse And Sarah, the wife of my master, has borne a son to my master after her old age. And he has given to him all that he has.
\verse And my master made me swear, saying, ‘Do not take a wife for my son from the daughters of the Canaanites in whose land I am living.
\verse But you shall go to the house of my father, and to my family, and you shall take a wife for my son.’
\verse And I said to my master, ‘Perhaps the woman will not \textit{follow} me.’
\verse And he said to me, ‘Adonai, before whom I have walked, shall send his angel with you and will make your journey successful. And you shall take a wife for my son from my family, and from the house of my father.
\verse Then you shall be released from my oath, when you come to my family. And if they will not give \textit{a woman} to you, then you will be released from my oath.’
\verse Then today I came to the spring, and I said, ‘O Adonai, God of my master Abraham, \textit{if you would please make my journey successful}, upon which I am going.
\verse Behold, I am standing by the spring of water. Let it be \textit{that} the young woman who comes out to draw water and to whom I say, “Please give me a little water to drink from your jar,”
\verse let her say to me, “Drink; I will also draw water for your camels,” she \textit{is} the woman whom Adonai has appointed for the son of my master.’
\verse I had not yet finished speaking to myself when, behold, Rebekah \textit{was} coming out with her jar on her shoulder. And she went down to the spring and drew water. And I said to her, ‘Please give me a drink.’
\verse And she hastened and let down her jar \textit{from her shoulder} and said, ‘Drink, and I will give a drink to your camels also.’ Then I drank and she gave a drink to the camels also.
\verse Then I asked her and said, ‘Whose daughter \textit{are} you?’ And she said, ‘The daughter of Bethuel, son of Nahor, whom Milcah bore to him.’ And I put the ring on her nose and the bracelets on her arms.
\verse And I knelt down and worshiped Adonai, and I praised Adonai, the God of my master Abraham, who led me on the right way, to take the daughter of the brother of my master for his son.
\verse So now, \textit{if you are going to deal loyally and truly} with my master, tell me. And if not, tell me, so that I may turn to \textit{the} right or to \textit{the} left.”
\verse Then Laban and Bethuel answered, and they said, “The matter has gone out from Adonai; we are not able to speak bad or good to you.
\verse Here \textit{is} Rebekah before you. Take \textit{her} and go; let her be a wife for the son of your master as Adonai has spoken.”
\verse And it happened \textit{that} when the servant of Abraham heard their words he bowed down to the ground to Adonai.
\verse And the servant brought out silver jewelry and gold jewelry, and garments, and he gave \textit{them} to Rebekah. And he gave precious gifts to her brother and to her mother.
\verse And he and the men who \textit{were} with him ate and drank, and they spent the night. And they got up in the morning, and he said, “Let me go to my master.”
\verse And her brother and her mother said, “Let the girl remain with us ten days \textit{or so}; after \textit{that} she may go.”
\verse And he said to them, “Do not delay me. Now, Adonai has made my journey successful. Let me go. I must go to my master.”
\verse And they said, “Let us call the girl and ask \textit{her opinion}.”
\verse And they called Rebekah and said to her, “Will you go with this man?” And she said, “I will go.”
\verse So they sent away Rebekah their sister, and her nurse, and the servant of Abraham and his men.
\verse And they blessed Rebekah and said to her, “You \textit{are} our sister; may you become countless thousands; and may your offspring take possession of the gate of his enemies.”
\verse And Rebekah and her maidservants arose, and they mounted the camels and \textit{followed} the man. And the servant took Rebekah and left.
\verse Now Isaac \textit{was} coming from the direction of Beer-Lahai-Roi. And he \textit{was} living in the land of the Negev.
\verse And Isaac went out to meditate in the field \textit{early in the evening}, and he lifted up his eyes and saw—behold, camels were coming.
\verse And Rebekah lifted up her eyes and saw Isaac. And she got down from the camel.
\verse And she said to the servant, “Who \textit{is} this man walking around in the field to meet us?” And the servant said, “That \textit{is} my master.” And she took her veil and covered herself.
\verse And the servant told Isaac all the things that he had done.
\verse And Isaac brought her to the tent of Sarah his mother. And he took Rebekah, and she became his wife. And Isaac loved her and was comforted after \textit{the death of} his mother.
\end{biblechapter}

\begin{biblechapter} % Genesis 25
\verseWithHeading{The Death and Descendants of Abraham} Now Abraham again took a wife, and her name \textit{was} Keturah.
\verse And she bore to him Zimran, Jokshan, Medan, Midian, Ishbak, and Shuah.
\verse And Jokshan fathered Sheba and Dedan. And the sons of Dedan were Asshurim and Letushim and Leummim.
\verse And the sons of Midian \textit{were} Ephah, Epher, Hanoch, Abidah, and Eldaah. All of these \textit{were} the children of Keturah.
\verse And Abraham gave all he had to Isaac.
\verse But to the sons of Abraham’s concubines Abraham gave gifts. And while he \textit{was} still living he sent them away eastward, \textit{away} from his son Isaac, to the land of the east.
\verse Now these \textit{are} the days of the years of \textit{the life of Abraham}: one hundred and seventy-five years.
\verse And Abraham passed away and died in a good old age, old and full of years. And he was gathered to his people.
\verse And Isaac and Ishmael his sons buried him in the cave of Machpelah, in the field of Ephron, son of Zohar the Hittite, that \textit{was} east of Mamre,
\verse the field that Abraham had bought from the Hittites. There Abraham was buried and Sarah his wife.
\verse And it happened \textit{that} after the death of Abraham God blessed Isaac his son, and Isaac settled at Beer-Lahai-Roi.
\verse Now these \textit{are} the generations of Ishmael, the son of Abraham, that Hagar the Egyptian, the maidservant of Sarah, bore to Abraham.
\verse And these are the names of the sons of Ishmael, by their names according to their family records. The firstborn of Ishmael \textit{was} Nebaioth, then Kedar, Adbeel, Mibsam,
\verse Mishma, Dumah, Massa,
\verse Hadad, Tema, Jetur, Naphish, and Kedemah.
\verse These \textit{are} the sons of Ishmael, and these \textit{are} their names by their villages and by their encampments—12 leaders according to their tribes.
\verse Now these \textit{are} the years of the life of Ishmael: 137 years. And he passed away and died, and was gathered to his people.
\verse They settled from Havilah to Shur, which \textit{was} opposite Egypt, going toward Asshur, opposite; he \textit{settled} opposite all his brothers.
\verseWithHeading{Jacob and Esau} Now these \textit{are} the generations of Isaac, the son of Abraham. Abraham fathered Isaac,
\verse And Isaac was \textit{forty years old} when he took Rebekah, the daughter of Bethuel the Aramean of Paddan-Aram, the sister of Laban the Aramean, as his wife.
\verse And Isaac prayed to Adonai on behalf of his wife, for she \textit{was} barren. And Adonai responded to his prayer, and Rebekah his wife conceived.
\verse And the children in her womb jostled each other, and she said, “\textit{If it is going to be like this, why be pregnant}?” And she went to inquire of Adonai.
\verse And Adonai said to her, “Two nations \textit{are} in your womb, and two peoples \textit{from birth} shall be divided. And \textit{one people shall be stronger than the other}. And \textit{the} elder shall serve \textit{the} younger.”
\verse And when her days to give birth were completed, then—behold—twins \textit{were} in her womb.
\verse And the first came out red, all \textit{his body} \textit{was} like a hairy coat, so they called his name Esau.
\verse And afterward his brother came out, and his hand grasped the heel of Esau, so his name was called Jacob. And Isaac \textit{was sixty years old} at their birth.
\verse And the boys grew up. And Esau \textit{was} a skilled hunter, a man of the field, but Jacob \textit{was} a peaceful man, living \textit{in} tents.
\verse And Isaac loved Esau because \textit{he could eat of his game}, but Rebekah loved Jacob.
\verse Once Jacob cooked a thick stew, and Esau came in from the field, and he was exhausted.
\verse And Esau said to Jacob, “Give me \textit{some of that red stuff} to gulp down, for I am exhausted!” (Therefore his name was called Edom).
\verse Then Jacob said, “Sell me your birthright \textit{first}.”
\verse And Esau said, “Look, I am going to die; now what \textit{is} this birthright to me?”
\verse Then Jacob said, “Swear to me \textit{first}.” And he swore to him, and sold his birthright to Jacob.
\verse Then Jacob gave Esau bread, and thick lentil stew, and he ate and drank. Then he got up and went away. So Esau despised his birthright.
\end{biblechapter}

\begin{biblechapter} % Genesis 26
\verseWithHeading{Isaac and Abimelech} And there was a famine in the land, besides the former famine which was in the days of Abraham. And Isaac went to Abimelech, king of the Philistines, to Gerar.
\verse And Adonai appeared to him and said, “Do not go down to Egypt; dwell in the land which I will show to you.
\verse Dwell as an alien in this land, and I will be with you, and will bless you, for I will give all these lands to you and to your descendants, and I will establish the oath that I swore to Abraham you father.
\verse And I will multiply your descendants like the stars of heaven, and I will give to your descendants all these lands. And all nations of the earth will be blessed through your offspring,
\verse because Abraham listened to my voice and kept my charge: my commandments, my statutes, and my laws.”
\verse So Isaac settled in Gerar.
\verse When the men of the place asked concerning his wife, he said, “She \textit{is} my sister,” for he was afraid to say, “my wife,” thinking “the men of the place will kill me on account of Rebekah, for \textit{she was beautiful}.”
\verse And it happened \textit{that}, \textit{when he had been there a long time}, Abimelech the king of the Philistines looked through the window, and saw—behold—Isaac \textit{was} fondling Rebekah his wife.
\verse And Abimelech called Isaac and said, “Surely she \textit{is} your wife. Now why did you say ‘She \textit{is} my sister’?” And Isaac said to him, “Because I thought I would die on account of her.”
\verse And Abimelech said, “What \textit{is} this you have done to us? One of the people might easily have slept with your wife! Then you would have brought guilt upon us!”
\verse Then Abimelech instructed all the people, saying, “The \textit{one who} touches this man or his wife shall certainly die.”
\verse And Isaac sowed in that land and reaped in that \textit{same} year a hundredfold, and Adonai blessed him.
\verse And the man \textit{became wealthier and wealthier} until he was exceedingly wealthy.
\verse And he possessed sheep and cattle and many servants, so that the Philistines envied him.
\verse And the Philistines stopped up all the wells that the servants of his father had dug in the days of Abraham his father. They filled them with earth.
\verse And Abimelech said to Isaac, “Go \textit{away} from us, for you have become much too powerful for us.”
\verse So Isaac departed from there and camped in the valley of Gerar, and settled there.
\verse And Isaac dug again the wells of water which they had dug in the days of his father Abraham, which the Philistines had stopped up after the death of Abraham. And he gave to them \textit{the same names} which his father had given them.
\verse And when the servants of Isaac dug in the valley, they found a well of fresh water there.
\verse Then the herdsmen of Gerar quarreled with the herdsmen of Isaac, saying, “The water is ours.” And he called the name of the well Esek, because they contended with him.
\verse And they dug another well, and they quarreled over it also. And he called its name Sitnah.
\verse Then he moved from there and dug another well, and they did not quarrel over it. And he called its name Rehoboth, and said, “Now Adonai has made room for us, and we shall be fruitful in the land.”
\verse And from there he went up to Beersheba.
\verse And Adonai appeared to him that night and said, “I \textit{am} the God of your father Abraham. Do not be afraid, for I \textit{am} with you, and I will bless you and make your descendants numerous for the sake of my servant Abraham.”
\verse And he built an altar there and called on the name of Adonai. And he pitched his tent there, and the servants of Isaac dug a well there.
\verse Then Abimelech went to him from Gerar with Ahuzzath his friend and Phicol his army commander.
\verse And Isaac said to them, “Why have you come to me? You hate me and sent me away from you.”
\verse And they said, “We see clearly that Adonai has been with you, so we thought let there be an oath between us—between us and you—and let us \textit{make} a covenant with you
\verse that you may not do us harm just as we have not touched you, but have only done good to you and sent you away in peace. You \textit{are} now blessed by Adonai.”
\verse So he made a meal for them, and they ate and drank.
\verse And they arose early in the morning and each one swore to the other, and Isaac sent them away. And they left him in peace.
\verse And it happened \textit{that} on that same day the servants of Isaac came and told him about the well that they had dug. And they said, “We have found water!”
\verse And he called it Sheba. Therefore the name of the city \textit{is} Beersheba unto this day.
\verse And \textit{when} Esau was forty years old he took as wife Judith, daughter of Beeri the Hittite, and Basemath, daughter of Elon the Hittite.
\verse And \textit{they made life bitter} for Isaac and Rebekah.
\end{biblechapter}

\begin{biblechapter} % Genesis 27
\verseWithHeading{Jacob Steals Esau’s Blessing} And it happened \textit{that} when Isaac \textit{was} old and \textit{his eyesight was weak}, he called Esau his older son and said to him, “My son.” And he said to him, “Here I \textit{am}.”
\verse And he said, “Look, I \textit{am} old; I do not know the day of my death.
\verse So now, take your weapons, your quiver and your bow, and go out to the field and hunt food for me.
\verse Then make for me tasty food like I love, and bring \textit{it} to me. And I will eat \textit{it} so that I can bless you before I die.
\verse Now Rebekah \textit{was} listening as Isaac spoke to Esau his son, and \textit{when} Esau went to the field to hunt wild game to bring \textit{back},
\verse Rebekah said to Jacob her son, “Look, I heard your father speaking to Esau your brother saying,
\verse ‘Bring wild game to me and prepare tasty food so I can eat \textit{it} and bless you before Adonai before my death.’
\verse So now, my son, listen to my voice, to what I command you.
\verse Go to the flock and take two good young goats from it for me, and I will prepare them \textit{as} tasty food for your father, just as he likes.
\verse Then you must take it to your father and he will eat \textit{it} so that he may bless you before his death.”
\verse Then Jacob said to his mother, “Behold, Esau my brother \textit{is} a hairy man, but I \textit{am} a smooth man.
\verse Perhaps my father will feel me and I will be in his eyes \textit{as} a mocker, and he will bring upon me a curse and not a blessing.”
\verse Then his mother said to him, “Your curse be upon me, my son, only listen to my voice—go and get \textit{them} for me.”
\verse So he went and took \textit{them}, and brought \textit{them} to his mother, and his mother prepared tasty food as his father liked.
\verse Then Rebekah took \textit{some of} her older son Esau’s best garments that \textit{were} with her in the house, and she put \textit{them} on Jacob her younger son.
\verse And she put the skins of the young goats over his hands and over the smooth \textit{part of} his neck.
\verse And she put the tasty food and the bread that she had made into the hand of Jacob, her son.
\verse And he went to his father and said, “My father.” And he said, “Here I \textit{am}. Who \textit{are} you, my son?”
\verse And Jacob said to his father, “I \textit{am} Esau, your firstborn. I have done as you told me. Please get up, sit up and eat from my wild game so that you may bless me.”
\verse Then Isaac said to his son, “\textit{How} did you find \textit{it} so quickly, my son?” And he said, “Because Adonai your God \textit{caused me to find it}.”
\verse Then Isaac said to Jacob, “Please, come near and let me feel you, my son. \textit{Are you really} my son Esau or not?”
\verse And Jacob drew near to Isaac his father. And he felt him and said, “The voice \textit{is} the voice of Jacob, but the hands \textit{are} the hands of Esau.”
\verse And he did not recognize him because his hands were hairy like the hands of Esau his brother. And he blessed him.
\verse And he said, “\textit{Are you really} my son Esau?” And he said, “I \textit{am}.”
\verse Then he said, “Bring \textit{it} near to me that I may eat from the game of my son, so that I may bless you.” And he brought \textit{it} to him, and he ate. And he brought wine to him, and he drank.
\verse Then his father Isaac said to him, “Come near and kiss me, my son.”
\verse And he drew near and kissed him. And he smelled the smell of his garments, and he blessed him and said,
\verse “Look, the smell of my son \textit{is} like the smell of a field that Adonai has blessed!
\verse May God give you of the dew of heaven 
and of the fatness of the earth, 
and abundance of grain and new wine.
\verse And as soon as Isaac had finished blessing Jacob, \textit{immediately after} Jacob had gone out from the presence of Isaac his father, Esau his brother came \textit{back} from his hunting.
\verse He too prepared tasty food and brought \textit{it} to his father. And he said to his father, “Let my father arise and eat from the wild game of his son, that you may bless me.”
\verse And Isaac his father said to him, “Who \textit{are} you?” And he said, “I \textit{am} your son, your firstborn, Esau.”
\verse Then Isaac \textit{trembled violently}. Then he said, “Who then \textit{was} he that hunted wild game and brought \textit{it} to me, and I ate \textit{it} all before you came, and I blessed him? Moreover, he will be blessed!”
\verse When Esau heard the words of his father he cried out \textit{with} a great and exceedingly bitter cry of distress. And he said to his father, “Bless me as well, my father!”
\verse And he said, “Your brother came in deceit and took your blessing.”
\verse Then he said, “\textit{Isn’t that why he is named Jacob}? He has deceived me these two times. He took my birthright and, look, now he has taken my blessing!” Then he said, “Have you not reserved a blessing for me?”
\verse Then Isaac answered and said to Esau, “Behold, I have made him lord over you and I have given him all his brothers as servants, and \textit{with} grain and wine I have sustained him. Now what can I do for you, my son?”
\verse And Esau said to his father, “Have you only one blessing, my father? Bless me also, my father!” And Esau lifted up his voice and wept.
\verse Then Isaac his father answered and said to him,
\verse “Your home shall be from the fatness of the land, 
and from the dew of heaven above.
\verse Then Esau held a grudge against Jacob on account of the blessing with which his father had blessed him. And Esau said in his heart, “The days of mourning for my father are coming, then I will kill Jacob my brother.”
\verse But the words of Esau her older son were told to Rebekah. And she sent and called for her younger son Jacob. And she said to him, “Look, Esau your brother \textit{is} consoling himself concerning you, \textit{intending} to kill you.
\verse Now then, my son, listen to my voice; arise and flee to Haran to Laban my brother.
\verse Stay with him a few days until the wrath of your brother has turned—
\verse until the anger of your brother turns from you and he has forgotten what you have done to him. Then I will send and bring you from there. Why should I lose the two of you in one day?”
\verse Then Rebekah said to Isaac, “I loathe my life because of the Hittite women. If Jacob takes a wife from Hittite women like these, from the \textit{native women}, \textit{what am I living for}?”
\end{biblechapter}

\begin{biblechapter} % Genesis 28
\verseWithHeading{Jacob Flees to Haran} Then Isaac called Jacob and blessed him. And he instructed him and said to him, “You must not take a wife from the daughters of Canaan.
\verse Arise, go to Paddan-Aram, to the house of Bethuel, your mother’s father, and take for yourself a wife from there, from the daughters of Laban your mother’s brother.
\verse Now, may El-Shaddai bless you, and make you fruitful, and multiply you, so that you become an assembly of peoples.
\verse And may he give you the blessing of Abraham, to you and to your descendants with you, that you may take possession of the land of your sojourning, which God gave to Abraham.”
\verse Then Isaac sent Jacob away, and he went to Paddan-Aram, to Laban the son of Bethuel the Aramean, the brother of Rebekah, the mother of Jacob and Esau.
\verse Now Esau saw that Isaac had blessed Jacob and sent him away to Paddan-Aram, to take for himself a wife from there, and he blessed him and instructed him, saying, “You must not take a wife from the daughters of Canaan,”
\verse and \textit{that} Jacob listened to his father and to his mother and went to Paddan-Aram.
\verse Then Esau saw that the daughters of Canaan \textit{were} evil in the eyes of Isaac his father,
\verse then Esau went to Ishmael and took Mahalath, the daughter of Ishmael, son of Abraham, sister of Nebaioth, as a wife, in addition to the wives he had.
\verseWithHeading{Jacob’s Dream} Then Jacob went out from Beersheba and went to Haran.
\verse And he arrived at a \textit{certain} place and spent the night there, because the sun had set. And he took \textit{one} of the stones of the place and put \textit{it} under his head and slept at that place.
\verse And he dreamed, and behold, a stairway was set on the earth, and its top touched the heavens. And behold, angels of God \textit{were} going up and going down on it.
\verse And behold, Adonai \textit{was} standing beside him, and he said, “I \textit{am} Adonai, the God of Abraham your father, and the God of Isaac. The ground on which you \textit{were} sleeping I will give to you and to your descendants.
\verse Your descendants shall be like the dust of the earth, and you will spread out to the west, and to the east, and to the north and to the south. And all the families of the earth will be blessed through you and through your descendants.
\verse Now behold, I \textit{am} with you, and I will keep you wherever you go. And I will bring you to this land, for I will not leave you until I have done what I have promised to you.”
\verse Then Jacob awoke from his sleep and said, “Surely Adonai \textit{is indeed} in this place and I did not know!”
\verse Then he was afraid and said, “How awesome \textit{is} this place! \textit{This is nothing else than the house of God}, and this is the gate of heaven!”
\verse And Jacob rose early in the morning, and he took the stone that he had put under his head and set it up \textit{as} a stone pillar, and poured oil on top of it.
\verse And he called the name of that place Bethel; however, the name of the city \textit{was} formerly Luz.
\verse And Jacob made a vow saying, “If God will be with me and protect me on this way that I am going, and gives me food to eat and clothing to wear,
\verse and \textit{if} I return in peace to the house of my father, then Adonai will become my God.
\verse And this stone that I have set up \textit{as} a pillar shall be the house of God, and \textit{of} all that you give to me I will certainly give a tenth to you.”
\end{biblechapter}

\begin{biblechapter} % Genesis 29
\verseWithHeading{Jacob Flees to Haran} And Jacob \textit{continued his journey} and went to the land of the Easterners.
\verse And he looked, and behold, \textit{there was} a well in the field, and behold, there \textit{were} three flocks of sheep lying beside it, for out of that well the flocks were watered. And the stone on the mouth of the well \textit{was} large.
\verse And \textit{when} all the flocks were gathered there, they rolled away the stone from the mouth of the well. And they watered the sheep and returned the stone upon the mouth of the well to its place.
\verse And Jacob said to them, “My brothers, where \textit{are} you from?” And they said, “We \textit{are} from Haran.”
\verse And he said to them, “Do you know Laban, son of Nahor?” And they said, “We know \textit{him}.”
\verse And he said to them, “\textit{Is he well}?” And they said, “\textit{He is} well. Now look, Rachel his daughter is coming with the sheep.”
\verse And he said, “Look, \textit{it is} still \textit{broad daylight}; it is not the time \textit{for} the livestock to be gathered. Give water to the sheep and go, pasture them.”
\verse And they said, “We are not able, until all the flocks are gathered. Then the stone is rolled away from the mouth of the well, and we water the sheep.”
\verse While he was speaking with them, Rachel came with the sheep which belonged to her father, for she was pasturing \textit{them}.
\verse And it happened \textit{that}, when Jacob saw Rachel, the daughter of Laban, his mother’s brother, and the sheep of Laban, his mother’s brother, Jacob drew near and rolled away the stone from the mouth of the well and watered the sheep of Laban, his mother’s brother.
\verse And Jacob kissed Rachel, and lifted up his voice and wept.
\verse And Jacob told Rachel that he \textit{was} the relative of her father, and that he \textit{was} the son of Rebekah. And she ran and told her father.
\verse And it happened \textit{that} when Laban heard the message about Jacob, the son of his sister, he ran to meet him. And he embraced him and kissed him, and brought him to his house. And he told Laban all these things.
\verse And Laban said to him, “Surely you \textit{are} my flesh and my bone!” And he stayed with him a month.
\verseWithHeading{Jacob’s Marriages} Then Laban said to Jacob, “\textit{Just} because you \textit{are} my brother should you work for me for nothing? Tell me what your wage \textit{should be}.”
\verse Now Laban had two daughters. The name of the older \textit{was} Leah, and the name of the younger \textit{was} Rachel.
\verse Now the eyes of Leah \textit{were} dull, but Rachel was beautiful in form and appearance.
\verse And Jacob loved Rachel and said, “I will serve you seven years for Rachel your younger daughter.”
\verse Then Laban said, “Better \textit{that} I give her to you than I give her to another man. Stay with me.”
\verse And Jacob worked for Rachel seven years, but they were as a few days in his eyes because he loved her.
\verse And Jacob said to Laban, “Give \textit{me} my wife, that I may go in to her, for \textit{my time} is completed.”
\verse So Laban gathered all the men of the place and prepared a feast.
\verse And it happened \textit{that} in the evening he took Leah his daughter and brought her to him, and he went in to her.
\verse And Laban gave Zilpah his female servant to her, to Leah his daughter \textit{as} a female servant.
\verse And it happened \textit{that} in the morning, behold, it \textit{was} Leah! And he said to Laban, “What \textit{is} this you have done to me? Did I not serve with you for Rachel? Now why did you deceive me?”
\verse Then Laban said, “\textit{It is not the custom} in our country to give the younger before the firstborn.
\verse Complete the week of this one, then I will also give you the other, \textit{on the condition that you will work for me} another seven years.”
\verse And Jacob did so. So he completed the week of this \textit{one}, then he gave Rachel his daughter to him as a wife.
\verse And Laban gave Bilhah his female servant to Rachel his daughter as a female servant.
\verse Then he also went in to Rachel, and he loved Rachel more than Leah. And he served with him yet another seven years.
\verseWithHeading{Jacob’s Children} When Adonai saw that Leah \textit{was} unloved he opened her womb, but Rachel \textit{was} barren.
\verse Then Leah conceived and gave birth to a son, and she called his name Reuben, for she said, “Because Adonai has noticed my misery, that I \textit{am} unloved. Now my husband will love me.”
\verse And she conceived again and gave birth to a son. And she said, “\textit{It is} because Adonai has heard that I \textit{am} unloved that he gave me this \textit{son} also.” And she called his name Simeon.
\verse And she conceived again and gave birth to a son. Then she said, “Now this time my husband will be joined to me, for I have borne him three sons.” Therefore, she called his name Levi.
\verse And she conceived again and gave birth to a son. And she said, “This time I will praise Adonai.” Therefore she called his name Judah. And she ceased bearing children.
\end{biblechapter}

\begin{biblechapter} % Genesis 30
\verseWithHeading{Jacob’s Children} When Rachel saw that she could not bear children to Jacob, Rachel envied her sister. And she said to Jacob, “Give me children—if not, I will die!”
\verse And Jacob \textit{became angry} with Rachel. And he said, “\textit{Am} I in the place of God, who has withheld from you the fruit of the womb?”
\verse Then she said, “Here \textit{is} my servant girl Bilhah; go in to her that she may bear children \textit{as my surrogate}. Then I will even \textit{have children} by her.”
\verse Then she gave him Bilhah, her female servant, as a wife, and Jacob went in to her
\verse And Bilhah conceived and gave birth to a son for Jacob.
\verse Then Rachel said, “God has judged me, and has also heard my voice, and has given me a son.” Therefore she called his name Dan.
\verse And Bilhah, Rachel’s servant, conceived again and bore a second son to Jacob.
\verse And Rachel said, “I have struggled a mighty struggle with my sister and have prevailed.” And she called his name Naphtali.
\verse When Leah saw that she had ceased bearing children, she took Zilpah her female servant and gave her to Jacob as a wife.
\verse And Zilpah, the female slave of Leah, bore a son to Jacob.
\verse Then Leah said, “Good fortune!” And she called his name Gad.
\verse And Zilpah, Leah’s female servant, bore a second son to Jacob.
\verse Then Leah said, “How happy \textit{am} I! For women have called me happy.” So she called his name Asher.
\verse And in the days of the wheat harvest, Reuben went and found mandrakes in the field and he brought them to Leah his mother. And Rachel said to Leah, “Please give me some of your son’s mandrakes.”
\verse And she said to her, “\textit{Is} your taking my husband \textit{such} a small \textit{thing} that you will also take the mandrakes of my son?” Then Rachel said, “Then he may sleep with you tonight in exchange for your son’s mandrakes.”
\verse When Jacob came in from the field in the evening, Leah went out to meet him. And she said, “Come in to me, for \textit{I have hired} you with my son’s mandrakes.” And he slept with her that night.
\verse And God listened to Leah and she conceived and gave birth to a fifth son for Jacob.
\verse Then Leah said, “God has given \textit{me} my wage since I gave my servant girl to my husband.” And she called his name Issachar.
\verse And Leah conceived again and gave birth to a sixth son for Jacob.
\verse And Leah said, “God has endowed me with a good gift. This time my husband will acknowledge me, because I bore him six sons.” And she called his name Zebulun.
\verse And afterward she gave birth to a daughter. And she called her name Dinah.
\verse Then God remembered Rachel and listened to her, and God opened her womb.
\verse And she conceived and gave birth to a son. And she said, “God has taken away my disgrace.”
\verse And she called his name Joseph, saying, “Adonai has added to me another son.”
\verseWithHeading{Jacob’s Prosperity} And it happened \textit{that} as soon as Rachel had given birth to Joseph, Jacob said to Laban, “Send me away that I may go to my place and my land.
\verse Give \textit{me} my wives and my children for which I have served you, and let me go. For you yourself know my service that I have rendered to you.”
\verse But Laban said to him, “Please, if I have found favor in your eyes, I have learned by divination that Adonai has blessed me because of you.”
\verse And he said, “Name your wage to me and I will give \textit{it}.”
\verse Then he said to him, “You yourself know how I have served you and how your livestock have been with me.
\verse For you had little before me, and it has increased abundantly. And Adonai has blessed you \textit{wherever I turned}. So then, when shall I provide for my own family also?”
\verse And he said, “What shall I give you?” And Jacob said, “Do not give me anything. If you will do this thing for me, I will again feed your flocks and keep \textit{them}.
\verse Let me pass through all your flocks today, removing all the speckled and spotted sheep from them, along with every dark-colored sheep among the sheep, and the spotted and speckled among the goats. That shall be my wages.
\verse And my righteousness will answer for me \textit{later} when you come concerning my wages before you. Every \textit{one} that \textit{is} not speckled or spotted among the goats, or dark-colored among the sheep shall be stolen \textit{if it is} with me.”
\verse Then Laban said, “Look! Very well. It shall be according to your word.”
\verse But that day he removed the streaked and spotted male goats and all the speckled and spotted female goats, all that \textit{had} white on it, and every dark-colored ram, and \textit{put them in the charge of his sons}.
\verse And he put a journey of three days between him and Jacob, and Jacob pastured the remainder of Laban’s flock.
\verse Then Jacob took fresh branches of poplar, almond, and plane trees and peeled white strips on them, exposing the white which \textit{was} on the branches.
\verse And he set the branches that he had peeled in front of the flocks, in the troughs \textit{and} in the water containers. And they were in heat when they came to drink.
\verse And the flocks mated by the branches, so the flocks bore streaked, speckled, and spotted.
\verse And Jacob separated the lambs and turned the faces of the flocks toward the streaked and all the dark-colored in Laban’s flocks. And he put his own herds apart, and did not put them with the flocks of Laban.
\verse And whenever any of the stronger of the flocks were in heat, Jacob put the branches \textit{in full view} of the flock in the troughs that they might mate among the branches.
\verse But with the more feeble of the flock he would not put \textit{them there}. So the feebler were Laban’s and the stronger \textit{were} Jacob’s.
\verse And the man became \textit{exceedingly} rich and had large flocks, female slaves, male slaves, camels, and donkeys.
\end{biblechapter}

\begin{biblechapter} % Genesis 31
\verseWithHeading{Jacob Flees from Laban} Now he heard the words of the sons of Laban, saying, “Jacob has taken all that our father has,” and “From that which \textit{was} our father’s he has gained all this wealth.”
\verse Then Jacob saw the face of Laban and, behold, \textit{it was not like it had been in the past}.
\verse And Adonai said to Jacob, “Return to the land of your ancestors and to your family, and I will be with you.”
\verse So Jacob sent and called Rachel and Leah to the field, to his flocks,
\verse and he said to them, “Look, I see the face of your father, that \textit{it is not like it has been toward me in the past}. But the God of my father is with me.
\verse Now you yourselves know that I have served your father with all my strength,
\verse and your father has cheated me and changed my wages ten times, but God has not allowed him to harm me.
\verse If thus he said, ‘Speckled shall be your wage,’ then all the flock bore speckled. And \textit{if} he said, ‘Streaked shall be your wage,’ then all the flock bore streaked.
\verse God has taken away your father’s livestock and given \textit{them} to me.
\verse Now it happened \textit{that} at the time of the mating of the flock I lifted up my eyes and saw in a dream, and behold, the rams mounting the flock \textit{were} streaked, speckled, and dappled.
\verse Then the angel of God said to me in the dream, ‘Jacob,’ and I said, ‘Here I \textit{am}.’
\verse And he said, ‘Lift up your eyes and see—all the rams mounting the flock \textit{are} streaked, speckled, and dappled, for I have seen all that Laban is doing to you.
\verse I \textit{am} the God of Bethel where you anointed a stone pillar, where you made a vow to me. Now get up, go out from this land and return to the land of your birth.’ ”
\verse Then Rachel and Leah answered and said to him, “\textit{Is there} yet a portion for us, and an inheritance in the house of our father?
\verse Are we not regarded \textit{as} foreigners by him, because he has sold us and completely consumed our money?
\verse For all the wealth that God has taken away from our father, it belongs to us and to our sons. So now, all that God has said to you, do.”
\verse Then Jacob got up and put his children and his wives on the camels.
\verse And he drove all his livestock and his possessions that he had acquired, the livestock of his possession that he had acquired in Paddan-Aram, in order to go to Isaac his father, to the land of Canaan.
\verse Now Laban had gone to shear his sheep, and Rachel stole the idols that belonged to her father.
\verse And Jacob \textit{tricked} Laban the Aramean by not telling him that he \textit{intended to} flee.
\verse Then he fled with all that he had, and arose and crossed the Euphrates and set his face toward the hill country of Gilead.
\verse And on the third day it was told to Laban that Jacob had fled.
\verse Then he took his kinsmen with him and pursued after him, a seven-day journey, and he caught up with him in the hill country of Gilead.
\verse And God came to Laban the Aramean in a dream at night and said to him, “\textit{Take care} that you not speak with Jacob, whether good or evil.”
\verse And Laban overtook Jacob. Now Jacob had pitched his tent in the hill country, and Laban and his kinsmen pitched \textit{their tents} in the hill country of Gilead.
\verse Then Laban said to Jacob, “What have you done that you \textit{tricked me} and have carried off my daughters like captives of the sword?
\verse Why did you hide \textit{your intention} to flee and \textit{trick me}, and did not tell me so that I would have sent you away with joy and song and tambourine and lyre?
\verse And \textit{why} did you not give me opportunity to kiss my grandsons and my daughters \textit{goodbye}? Now you have behaved foolishly \textit{by} doing \textit{this}.
\verse \textit{It is in my power} to do harm to you, but the God of your father spoke to me last night saying, ‘\textit{Take care} from speaking with Jacob, whether good or evil.’
\verse Now, you have surely gone because you desperately longed for the house of your father, \textit{but} why did you steal my gods?”
\verse Then Jacob answered and said to Laban, “Because I \textit{was} afraid, for I thought, ‘Lest you take your daughters from me by force.’
\verse \textit{But} with whomever you find your gods, he shall not live. In the presence of your kinsmen \textit{now} identify what \textit{is} with me \textit{that is} yours and take it.” Now Jacob did not know that Rachel had stolen them.
\verse Then Laban went into Jacob’s tent and Leah’s tent and the tent of the two female servants and did not find \textit{his gods}. And he came out of Leah’s tent and went into Rachel’s tent.
\verse Now Rachel had taken the idols and put them in the saddle bag of the camel and sat on them. And Jacob searched the whole tent thoroughly but did not find them.
\verse And she said to her father, “Let there not be anger in the eyes of my lord, for I am not able to rise before you, for the way of women \textit{is} with me. And he searched carefully and did not find the idols.
\verse Then Jacob became angry and quarreled with Laban. Jacob answered and said to Laban, “What \textit{is} my offense? What \textit{is} my sin that you pursued after me?
\verse For you have searched all my possessions and what did you find among all the possessions of my household? Set it before my kinsmen and your kinsmen that they may decide between the two of us!
\verse These twenty years I \textit{was} with you; your ewes and your female goats did not miscarry, and the rams of your flocks I did not eat.
\verse I brought no mangled carcass to you—I bore its loss. From my hand you sought it, whether stolen by day or stolen by night.
\verse \textit{There} I was, during the day the heat consumed me, and the cold by night, and my sleep fled from my eyes.
\verse These twenty years \textit{I have been} in your house. I served you fourteen years for your two daughters and six years for your flock, and you have changed my wages ten times.
\verse If the God of my father, the God of Abraham and the Fear of Isaac had not been with me, indeed now you would have sent me away empty-handed. God saw my misery and the labor of my hands and rebuked you last night.”
\verse Then Laban answered and said to Jacob, “The daughters \textit{are} my daughters and the grandsons \textit{are} my grandsons, and the flocks \textit{are} my flocks, and all that you see, it \textit{is} mine. Now, what can I do for these my daughters today, or for their children whom they have borne?
\verse So now, come, let us \textit{make} a covenant, you and I, and let it be a witness between me and you.”
\verse And Jacob took a stone and set it up \textit{as} a stone pillar.
\verse And Jacob said to his kinsmen, “Gather stones.” And they took stones and made a pile of stones, and they ate there by the pile of stones.
\verse And Laban called it Jegar Sahadutha, but Jacob called it Galeed.
\verse Then Laban said, “This pile of stones \textit{is} a witness between me and you today.” Therefore its name is called Galeed,
\verse and Mizpah, because he said, “Adonai watch between me and you when \textit{we are out of sight of each other}.
\verse If you mistreat my daughters, and if you take wives besides my daughters, \textit{when} there is no man with us, see—God \textit{is} a witness between me and you.”
\verse And Laban said to Jacob, “See, this pile of stones, and see the pillar that I have set up between me and you.
\verse This pile of stones \textit{is} a witness, and the pillar \textit{is} a witness, that I will not pass beyond this pile of stones to you, and that you will not pass beyond this pile of stones and this pillar to me intending harm.
\verse May the God of Abraham and the God of Nahor, the God of their father judge between us.” Then Jacob swore by the Fear of his father Isaac.
\verse And Jacob sacrificed a sacrifice on the hill, and he called his kinsmen to eat the meal. And they ate the meal and spent the night on the hill.
\verse  And Laban arose early in the morning and kissed his grandsons and his daughters, and blessed them. Then Laban departed and returned to his homeland.
\end{biblechapter}

\begin{biblechapter} % Genesis 32
\verseWithHeading{Jacob Fears Esau} And Jacob went on his way, and angels of God met him.
\verse And when he saw them, Jacob said, “This \textit{is} the camp of God!” And he called the name of that place Mahanaim.
\verse Then Jacob sent messengers before him to Esau his brother, to the land of Seir, the territory of Edom.
\verse And he instructed them, saying, “Thus you must say to my lord, to Esau, ‘Thus says your servant Jacob, I have dwelled as an alien with Laban, and I have remained \textit{there} until now.
\verse And I have acquired cattle, male donkeys, flocks, and male and female slaves, and I have sent to tell my lord, to find favor in your eyes.’ ”
\verse And the messengers returned to Jacob \textit{and} said, “We came to your brother, to Esau, and he is coming to meet you, and four hundred men \textit{are} with him.”
\verse Then Jacob was very frightened and distressed. So he divided the people, flocks, cattle, and camels that \textit{were} with him into two companies.
\verse And he thought, “If Esau comes to one company and destroys it, the remaining company will be \textit{able} to escape.”
\verse Then Jacob said, “O God of my father Abraham, and God of my father Isaac, O Adonai, who said to me, ‘Return to your land and to your family, and I will deal well with you.’
\verse \textit{I am not worthy} of all the loyal love and all the faithfulness that you have shown your servant, for with \textit{only} my staff I crossed this Jordan, and now I have become two camps.
\verse Please rescue me from the hand of my brother, from the hand of Esau, for I fear him, lest he come and attack mother and children \textit{alike}.
\verse Now you yourself said, ‘I will surely deal well with you and make your offspring as the sand of the sea that cannot be counted for abundance.’ ”
\verse And he lodged there that night. Then he took \textit{from what he had with him} a gift for Esau his brother:
\verse two hundred female goats, twenty male goats, two hundred ewes, twenty rams,
\verse thirty milk camels with their young, forty cows, ten bulls, twenty female donkeys, and ten male donkeys.
\verse And he put \textit{them} under the hand of his servants, \textit{herd by herd}, and said to his servants, “Cross on ahead before me, and put some distance \textit{between herds}.
\verse And he instructed the foremost, saying, “When Esau my brother comes upon you and asks you, saying, ‘Whose \textit{are} you and where are you going? To whom do these \textit{animals} belong ahead of you?’
\verse Then you must say, ‘To your servant, to Jacob. It \textit{is} a gift sent to my lord, to Esau. Now behold, he \textit{is} also \textit{coming} after us.’ ”
\verse And he also instructed the second \textit{servant} and the third, and everyone \textit{else} who \textit{was} behind the herds, saying, “You must speak to Esau according to this word when you find him.
\verse And moreover, you shall say, ‘Look, your servant Jacob \textit{is} behind us.’ ” For he thought, “\textit{Let me appease him} with the gift going before me, and afterward I will see his face. Perhaps he will \textit{show me favor}.”
\verse So the gift passed on before him, but he himself spent that night in the camp.
\verseWithHeading{Jacob Wrestles with God} That night he arose and took his two wives, his two female servants, and his eleven children and crossed the ford of the Jabbok.
\verse And he took them and sent them across the stream. Then he sent across all his possessions.
\verse And Jacob remained alone, and a man wrestled with him until the breaking of the dawn.
\verse And when he saw that he could not prevail against him, he struck his hip socket, so that Jacob’s hip socket was sprained as he wrestled with him.
\verse Then he said, “Let me go, for dawn is breaking.” But he answered, “I will not let you go unless you bless me.”
\verse Then he said to him, “What \textit{is} your name?” And he said, “Jacob.”
\verse And he said, “Your name shall no longer be called Jacob, but Israel, for you have struggled with God and with men and have prevailed.”
\verse Then Jacob asked and said, “Please tell me your name.” And he said, “Why do you ask this—for my name?” And he blessed him there.
\verse Then Jacob called the name of the place Peniel \textit{which means} “I have seen God face to face and my life was spared.”
\verse Then the sun rose upon him as he passed Penuel, and he was limping because of his hip.
\verse Therefore the \textit{Israelites} do not eat the sinew of the sciatic nerve that \textit{is} upon the socket of the hip unto this day, because he struck the socket of the thigh of Jacob at the sinew of the sciatic nerve.
\end{biblechapter}

\begin{biblechapter} % Genesis 33
\verseWithHeading{Jacob Meets Esau and Settles at Shechem} And Jacob lifted up his eyes and looked. And behold, Esau \textit{was} coming and four hundred men \textit{were} with him. And he divided the children among Leah and among Rachel, and among the two of his female servants.
\verse And he put the female slaves and their children first, then Leah and her children next, then Rachel with Joseph last.
\verse And he himself passed on before them and bowed down to the ground seven times until he came to his brother.
\verse But Esau ran to meet him, and embraced him, and fell upon his neck and kissed him, and they wept.
\verse Then Esau lifted up his eyes and saw the women and the children and said, “Who \textit{are} these with you?” And he said, “The children whom God has graciously given your servant.”
\verse Then the female servants drew near, they and their children, and they bowed down.
\verse Then Leah and her children drew near and bowed down, and afterward Joseph and Rachel drew near and they bowed down.
\verse And he said, “\textit{What do you mean by} all this company that I have met?” Then he said, “To find favor in the eyes of my lord.”
\verse Then Esau said, “\textit{I have enough} my brother; \textit{keep what you have}.”
\verse And Jacob said, “No, please, if I have found favor in your eyes, you must take my gift from my hand, for then I have seen your face \textit{which is} like seeing the face of God, and you have received me.
\verse Please take my gift which has been brought to you, for God has dealt graciously with me, and because \textit{I have enough}.” And he urged him, so he took \textit{it}.
\verse Then he said, “Let us journey and go \textit{on}, and I will go ahead of you.”
\verse But he said to him, “My lord knows that the children \textit{are} frail, and the flocks and the cattle \textit{which are} nursing \textit{are a concern} to me. Now \textit{if} they drove them hard for a day all the flocks would die.
\verse Let my lord pass on before his servant and I will move along slowly at the pace of the livestock that are ahead of me, and at the pace of the children until I come to my lord in Seir.”
\verse And Esau said, “Let me leave some of my people with you.” But he said, “\textit{What need is there}? Let me find favor in the eyes of my lord.”
\verse So Esau turned that day on his way to Seir.
\verse But Jacob traveled on to Succoth, and he built for himself a house, and he made shelters for his livestock. Therefore he called the name of the place Succoth.
\verse And Jacob came safely to the city of Shechem which \textit{is} in the land of Canaan, \textit{on his way} from Paddan-Aram. And he camped before the city.
\verse And he bought a piece of land where he pitched his tent for one hundred pieces of money from the hand of the sons of Hamor, father of Shechem.
\verse And there he erected an altar and called it “El Elohe Israel.”
\end{biblechapter}

\begin{biblechapter} % Genesis 34
\verseWithHeading{The Rape of Dinah and the Massacre at Shechem} Now Dinah the daughter of Leah, whom she had borne to Jacob, went out to see the daughters of the land.
\verse And Shechem, the son of Hamor the Hivite, the prince of the land, saw her. And he took her and lay with her and raped her.
\verse And his soul clung to Dinah, the daughter of Jacob, and he loved the girl and spoke \textit{tenderly} to the girl.
\verse So Shechem said to Hamor his father, saying, “Get this girl for me as a wife.”
\verse And Jacob heard that Dinah his daughter had been defiled, but his sons were with his flocks in the field. And Jacob kept silent until they came.
\verse And Hamor, father of Shechem, went out to Jacob to speak with him.
\verse And the sons of Jacob came in from the field when they heard \textit{it}. And the men were distressed and very angry because he had done a disgraceful thing in Israel by having sexual relations with the daughter of Jacob—\textit{something that} should not be done.
\verse And Hamor spoke with them saying, “Shechem my son \textit{is in love with} your daughter. Please give her to him for a wife.
\verse Make marriages with us. Give us your daughters and take our daughters for yourselves.
\verse You shall dwell with us and the land shall be before you; settle and trade in it, and acquire \textit{property} in it.”
\verse Then Shechem said to her father and to her brothers, “Let me find favor in your eyes, and whatever you say to me I will do.
\verse \textit{Make the bride price and gift as high as you like}; I will give what you say to me. But give me the girl as a wife.”
\verse Then the sons of Jacob answered Shechem and his father Hamor speaking deceitfully, because he had defiled Dinah their sister.
\verse And they said to them, “We cannot do this thing, to give our sister to a man who \textit{is} uncircumcised, for that \textit{is} a disgrace for us.
\verse Only on this \textit{condition} will we give consent to you; if you will become like us—every male among you to be circumcised.
\verse Then we will give our daughters to you, and we will take for ourselves your daughters, and we will live with you and become one family.
\verse But if you will not listen to us, to be circumcised, then we will take our daughters and we will go.”
\verse And their words were good in the eyes of Hamor and in the eyes of Shechem, the son of Hamor.
\verse And the young man did not delay to do the thing, for he wanted the daughter of Jacob. Now he \textit{was} the most honored of his father’s house.
\verse Then Hamor and his son Shechem came to the gate of their city, and they spoke to the men of their city, saying,
\verse “These men \textit{are} at peace with us. Let them dwell in the land and let them trade in it. Now, behold, the land is \textit{broad enough for them}. Let us take their daughters as wives, and let us give our daughters to them.
\verse Only on this \textit{condition} will they give consent to us, to live with us \textit{and} to become one family—when every male among us \textit{is} circumcised as they are circumcised.
\verse Will not their livestock and their property and all their animals \textit{be} ours? Only let us give consent to them so they will live among us.”
\verse And all those who went out of the gate of his city listened to Hamor and Shechem. Every male was circumcised, all those who went out of the gate of his city.
\verse And it happened \textit{that} on the third day, while they were in pain, two of the sons of Jacob, Simeon and Levi, the brothers of Dinah, each took his sword and came against the unsuspecting city and killed all the males.
\verse They killed Hamor and his son Shechem with the edge of the sword, and they took Dinah from the house of Shechem and went out.
\verse The \textit{other} sons of Jacob came upon the slain and plundered the city, because they had defiled their sister.
\verse They took their flocks and their cattle and their donkeys, and whatever \textit{was} in the field.
\verse They captured and plundered all that \textit{was} in the houses—all their wealth, their little ones, and their women.
\verse Then Jacob said to Simeon and Levi, “You have brought trouble on me, making me stink among the inhabitants of the land, among the Canaanites and the Perizzites! I \textit{am} few in number! If they gather against me and attack me, I will be destroyed—I and my household!”
\verse But they said, “Shall he treat our sister like a prostitute?”
\end{biblechapter}

\begin{biblechapter} % Genesis 35
\verseWithHeading{Jacob Goes Back to Bethel} And God said to Jacob, “Arise, go up to Bethel and dwell there, and make an altar to the God who appeared to you when you fled from before Esau your brother.”
\verse Then Jacob said to his household and to all who \textit{were} with him, “Get rid of the foreign gods that \textit{are} in your midst and purify yourselves and change your garments.
\verse Then let us make ready and let us go up to Bethel, so that I can make an altar there to the God who answered me in the day of my trouble, and who has been with me on the way that I have gone.”
\verse So they gave to Jacob all the foreign gods that \textit{were} in their hands, and the ornamental rings that \textit{were} in their ears. And Jacob buried them under the oak which \textit{was} near Shechem.
\verse Then they set out on their journey, and the terror of God was upon the cities that \textit{were} all around them, so that they did not pursue after the sons of Jacob.
\verse And Jacob came to Luz which \textit{was} in the land of Canaan (that \textit{is} Bethel), he and all the people that \textit{were} with him.
\verse And he built an altar there and called the place El-Bethel, for there God had appeared to him when he fled before his brother.
\verse And Deborah, the nurse of Rebekah, died. And she was buried below Bethel, under the oak. And its name was called Allon-Bacuth.
\verse And God appeared to Jacob again when he came from Paddan-Aram, and he blessed him.
\verse And God said to him, “Your name \textit{is} Jacob. Your name shall no longer be called Jacob, but Israel shall be your name.” Then his name was called Israel.
\verse And God said to him, “I \textit{am} El-Shaddai. Be fruitful and multiply. A nation and an assemblage of nations shall be from you, and kings shall go out from your loins.
\verse And \textit{as for} the land that I gave to Abraham and to Isaac, I will give it to you. And to your descendants after you I will give the land.
\verse And God went up from him at the place where he spoke with him.
\verse And Jacob set up a pillar at the place where God had spoken with him, a pillar of stone. And he poured out a drink offering upon it, and poured oil on it.
\verse And Jacob called the name of the place where God had spoken with him Bethel.
\verseWithHeading{The Death of Rachel} Then they journeyed from Bethel. And \textit{when they were still some distance} from Ephrath, Rachel went into labor. And she had hard labor.
\verse And \textit{when her labor was the most difficult} the midwife said to her, “Do not be afraid \textit{for you have another son}.”
\verse And it happened \textit{that} when her life was departing (for she was dying), she called his name Ben-Oni. But his father called him Benjamin.
\verse And Rachel died and she was buried on the way to Ephrath (that \textit{is}, Bethlehem).
\verse And Jacob erected a pillar at her burial site. That \textit{is} the pillar of the burial site of Rachel unto this day.
\verse And Israel journeyed \textit{on} and pitched his tent beyond the tower of Eder.
\verse And while Israel was living in that land Reuben went and had sexual relations with Bilhah, his father’s concubine. And Israel heard \textit{about it}.
\verseWithHeading{The Twelve Sons of Jacob} Now the sons of Jacob \textit{were} twelve.
\verse The sons of Leah: The firstborn of Jacob \textit{was} Reuben. Then Simeon, Levi, Judah, Issachar, and Zebulun.
\verse The sons of Rachel: Joseph and Benjamin.
\verse The sons of Bilhah, the female servant of Rachel: Dan and Naphtali.
\verseWithHeading{The Death of Isaac} And Jacob came to Isaac his father \textit{at} Mamre, \textit{or} Kiriath-Arba (that \textit{is}, Hebron), where Abraham and Isaac dwelled as aliens.
\verse Now the days of Isaac were one hundred and eighty years.
\verse And Isaac passed away and died, and was gathered to his people, old and full of days. And his sons Esau and Jacob buried him.
\end{biblechapter}

\begin{biblechapter} % Genesis 36
\verseWithHeading{The Descendants of Esau} Now these \textit{are} the descendants of Esau (that \textit{is}, Edom).
\verse Esau took his wives from the daughters of Canaan: Adah, daughter of Elon, the Hittite, and Oholibamah, daughter of Anah, the daughter of Zibeon, the Hivite,
\verse and Basemath, the daughter of Ishmael, the sister of Nebaioth.
\verse And Adah bore to Esau Eliphaz; and Basemath bore Reuel;
\verse and Oholibamah bore Jeush and Jalam, and Korah. These \textit{are} the sons of Esau who were born to him in the land of Canaan.
\verse And Esau took his wives and his sons and his daughters, and all the persons of his household, and his sheep and goats, and all his cattle, and all the goods that he had acquired in the land of Canaan, and went to a land away from his brother Jacob.
\verse For their possessions were \textit{too many to live together}, so that the land of their sojourning was not able to support them on account of their livestock.
\verse So Esau dwelled in the hill country of Seir (Esau, that \textit{is} Edom).
\verse Now these \textit{are} the descendants of Esau, the father of Edom, in the hill country of Seir.
\verse These \textit{are} the names of the sons of Esau: Eliphaz, the son of Adah, the wife of Esau; Reuel, the son of Basemath, the wife of Esau.
\verse The sons of Eliphaz were Teman, Omar, Zepho, Gatam, and Kenaz.
\verse (Now Timnah was the concubine of Eliphaz, the son of Esau. And she bore Amalek to Eliphaz.) These \textit{are} the sons of Adah, the wife of Esau.
\verse Now these \textit{are} the sons of Reuel: Nahath, Zerah, Shammah, and Mizzah. These \textit{are} the sons of Basemath, the wife of Esau.
\verse Now these \textit{are} the sons of Oholibamah, the daughter of Anah, daughter of Zibeon, the wife of Esau: She bore to Esau Jeush, Jalam, and Korah.
\verse These \textit{are} the chiefs of the sons of Esau. The sons of Eliphaz, the firstborn of Esau: the chiefs of Teman, Omar, Zepho, Kenaz,
\verse Korah, Gatam, and Amalek. These \textit{are} the chiefs of Eliphaz in the land of Edom. These \textit{are} the sons of Adah.
\verse Now these \textit{are} the sons Reuel, the son of Esau: the chiefs Nahath, Zerah, Shammah, and Mizzah. These \textit{are} the chiefs of Reuel in the land of Edom. These \textit{are} the sons of Basemath, the wife of Esau.
\verse Now these \textit{are} the sons of Oholibamah, the wife of Esau: the chiefs Jeush, Jalam, and Korah. These \textit{are} the chiefs born of Oholibamah, the daughter of Anah, the wife of Esau.
\verse These \textit{are} the sons of Esau, and these \textit{are} their chiefs (that \textit{is}, Edom).
\verse These \textit{are} the sons of Seir, the Horite, the inhabitants of the land: Lotan, Shobal, Zibeon, Anah,
\verse Dishon, Ezer, and Dishan. These \textit{are} the chiefs of the Horites, the sons of Seir in the land of Edom.
\verse And the sons of Lotan were Hori and Hemam. And Lotan’s sister \textit{was} Timna.
\verse Now these \textit{are} the sons of Shobal: Alvan, Manahath, Ebal, Shepho, and Onam.
\verse Now these \textit{are} the sons of Zibeon: Aiah and Anah—he \textit{is} Anah who found the hot springs in the desert while he pastured the donkeys of Zibeon his father.
\verse Now these \textit{are} the sons of Anah: Dishon and Oholibamah, the daughter of Anah.
\verse Now these \textit{are} the sons of Dishon: Hemdan, Eshban, Ithran, and Keran.
\verse These \textit{are} the sons of Ezer: Bilhan, Zaavan, and Akan.
\verse These \textit{are} the sons of Dishan: Uz and Aran.
\verse These \textit{are} the chiefs of the Horites: the chiefs Lotan, Shobal, Zibeon, Anah,
\verse Dishon, Ezer, and Dishan. These \textit{are} the chiefs of the Horites, according to their chiefs in the land of Seir.
\verseWithHeading{The Kings of Edom} Now these \textit{are} the kings who reigned in the land of Edom before any king ruled over the \textit{Israelites}.
\verse Bela the son of Beor reigned in Edom. And the name of his city \textit{was} Dinhabah.
\verse And Bela died, and Jobab, the son of Zerah from Bozrah, reigned in his place.
\verse And Jobab died, and Husham from the land of the Temanites reigned in his place.
\verse And Husham died, and Hadad, son of Bedad, who defeated Midian in the field of Moab reigned in his place. And the name of his city \textit{was} Avith.
\verse And Hadad died, and Samlah from Masrekah reigned in his place.
\verse And Samlah died, and Shaul from Rehoboth \textit{on} the Euphrates reigned in his place.
\verse And Shaul died, and Baal-Hanan, the son of Acbor, reigned in his place.
\verse And Baal-Hanan the son of Acbor died, and Hadar reigned in his place. And the name of his city \textit{was} Pau, and the name of his wife \textit{was} Mehetabel, the daughter of Matred, daughter of Mezahab.
\verse Now these \textit{are} the names of the chiefs of Esau according to their families, according to their dwelling places, by their names: the chiefs Timna, Alvah, Jetheth,
\verse Oholibamah, Elah, Pinon,
\verse Kenaz, Teman, Mibzar,
\verse Magdiel, and Iram. These \textit{are} the chiefs of Edom (that \textit{is}, Esau, the father of Edom) according to their settlements in the land of their possession.
\end{biblechapter}

\begin{biblechapter} % Genesis 37
\verseWithHeading{The Dreams of Joseph} And Jacob settled in the land of the sojourning of his father, in the land of Canaan.
\verse These \textit{are} the generations of Jacob. Joseph, \textit{being} seventeen years old, was shepherding the flock with his brothers. Now he \textit{was} a helper with the sons of Bilhah and the sons of Zilpah, the wives of his father. And Joseph brought a bad report of them to his father.
\verse Now Israel loved Joseph more than all his sons, for he \textit{was} a son of his old age. And he made a robe with long sleeves for him.
\verse When his brothers saw that their father loved him more than all his brothers, they hated him and were not able to speak peaceably to him.
\verse And Joseph dreamed a dream, and he told \textit{it} to his brothers. And \textit{they hated him even more}.
\verse And he said to them, “Listen now to this dream that I dreamed.
\verse Now behold, we were binding sheaves in the midst of the field and, behold, my sheaf stood up and it remained standing. Then behold, your sheaves gathered around and bowed down to my sheaf.”
\verse Then his brothers said to him, “Will you really rule over us?” And \textit{they hated him even more} on account of his dream and because of his words.
\verse Then he dreamed yet another dream and told it to his brothers. And he said, “Behold, I dreamed a dream again, and behold, the sun and the moon and eleven stars were bowing down to me.”
\verse And he told \textit{it} to his father and to his brothers. And his father rebuked him and said to him, “What \textit{is} this dream that you have dreamed? Will I and your mother and your brothers indeed come to bow down to the ground to you?”
\verse And his brothers were jealous of him, but his father kept the matter \textit{in mind}.
\verseWithHeading{Joseph Sold Into Slavery by his Brothers} Now his brothers went to pasture the flock of their father in Shechem.
\verse And Israel said to Joseph, “Are not your brothers pasturing in Shechem? Come, let me send you to them.” And he said, “Here I \textit{am}.”
\verse Then he said to him, “Go now, see \textit{if it goes well for your brothers and for the flock}, then return word to me.” And he sent him from the valley of Hebron, and he arrived at Shechem.
\verse And a man found him, and behold, he was wandering about in a field. And the man asked him, “What do you seek?”
\verse And he said, “I am seeking my brothers. Tell me, please, where they are pasturing.”
\verse And the man said, “They have moved on from here, for I heard \textit{them} saying, ‘Let us go to Dothan.’ ” Then Joseph went after his brothers and found them in Dothan.
\verse And they saw him from a distance. And before he drew near to them, they conspired against him to kill him.
\verse And each said to his brothers, “Look, this master of dreams is coming.
\verse Now then, come, let us kill him and throw him in one of the pits. Then we will say a wild animal devoured him. Then we will see what his dreams become.”
\verse And Reuben heard \textit{it} and delivered him from their hand and said, “We must not take his life.”
\verse And Reuben said to them, “You must not shed blood. Throw him into this pit that \textit{is} in the desert, but do not lay a hand on him”—so that he might rescue him from their hand to return him to his father.
\verse And it happened \textit{that} as Joseph came to his brothers they stripped Joseph of his robe, the robe with long sleeves, that \textit{was} upon him.
\verse And they took him and threw him into the pit (the pit \textit{was} empty; there was no water in it).
\verse Then they sat down to eat \textit{some} food. And they lifted up their eyes and looked, and behold, a caravan of Ishmaelites was coming from Gilead. And their camels were carrying aromatic gum and balm and spices \textit{on the way} to Egypt.
\verse Then Judah said to his brothers, “What profit \textit{is there} if we kill our brother and conceal his blood?
\verse Come, let us sell him to the Ishmaelites, but our hand shall not be against him, for he \textit{is} our brother, our own flesh.” And his brothers agreed.
\verse Then Midianite traders passed by. And they drew Joseph up and brought \textit{him} up from the pit, and they sold Joseph to the Ishmaelites for twenty \textit{pieces of} silver. And they brought Joseph to Egypt.
\verse Then Reuben returned to the pit and, behold, Joseph was not in the pit. And he tore his clothes.
\verse And he returned to his brothers and said, “The boy \textit{is gone}! Now I, \textit{what can I do}?”
\verse Then they took the robe of Joseph and slaughtered a goat, and dipped the robe in the blood.
\verse Then they sent the robe with long sleeves and they brought \textit{it} to their father and said, “We found this; please examine \textit{it}. \textit{Is} it the robe of your son or not?”
\verse And he recognized it and said, “The robe of my son! A wild animal has devoured him! Joseph \textit{is} surely torn to pieces!”
\verse And Jacob tore his clothes and put sackcloth on his loins and mourned for his son many days.
\verse And all his sons and daughters tried to console him, but he refused to be consoled. And he said, “No, I shall go down to my son, to Sheol, mourning.” And his father wept for him.
\verse And the Midianites sold him in Egypt to Potiphar, a court official of Pharaoh, a commander of the imperial guard.
\end{biblechapter}

\begin{biblechapter} % Genesis 38
\verseWithHeading{Judah and Tamar} And it happened \textit{that} at that time Judah went down from his brothers and pitched his tent near a certain Adullamite, whose name \textit{was} Hirah.
\verse And Judah saw the daughter of a certain Canaanite there whose name \textit{was} Shua. And he took her and went in to her.
\verse And she conceived and bore a son, and he called his name Er.
\verse And she conceived again and bore a son, and he called his name Onan.
\verse And once again she bore a son, and she called his name Shelah. And he was in Chezib when she bore him.
\verse And Judah took a wife for Er his firstborn, and her name \textit{was} Tamar.
\verse And Er, the firstborn of Judah, was evil in the eyes of Adonai, and Adonai killed him.
\verse Then Judah said to Onan, “Go in to the wife of your brother and perform the duty of a brother-in-law to her, and raise up offspring for your brother.”
\verse But Onan knew that the offspring would not be for him, so whenever he went in to the wife of his brother he would waste \textit{it} on the ground so as not to give offspring to his brother.
\verse And what he did was evil in the sight of Adonai, so he killed him also.
\verse Then Judah said to Tamar, his daughter-in-law, “Stay a widow in your father’s house until Shelah my son grows up,” for \textit{he feared he would also die} like his brother. So Tamar went and stayed in the house of her father.
\verse \textit{And in the course of time} the daughter of Shua, the wife of Judah, died. When Judah was consoled he went up to his sheepshearers, he and his friend Hirah the Adullamite, to Timnah.
\verse And it was told to Tamar, saying, “Look, your father-in-law is going up to Timnah to shear his sheep.”
\verse So she removed the clothes of her widowhood and covered \textit{herself} with the veil and disguised herself. And she sat at the entrance to Eynayim, which \textit{is} on the way to Timnah, for she saw that Shelah was grown but she had not been given to him as a wife.
\verse And Judah saw her and reckoned her to \textit{be} a prostitute, for she had covered her face.
\verse And he turned aside to her at the roadside and said, “Please come, let me come in to you,” for he did not know that she \textit{was} his daughter-in-law. And she said, “What will you give to me that you may come in to me?”
\verse And he said, “I will send a kid from the goats of the flock.” And she said, “\textit{Only} if you give a pledge until you send \textit{it}.”
\verse And he said, “What \textit{is} the pledge that I must give to you?” And she said, “your seal, your cord, and your staff that \textit{is} in your hand.” And he gave \textit{them} to her and went in to her. And she conceived by him.
\verse And she arose and left, and she removed her veil from herself and put on the garments of her widowhood.
\verse And Judah sent the kid from the goats by the hand of his friend the Adullamite to take \textit{back} the pledge from the hand of the woman, but he could not find her.
\verse So he asked the men of her place, saying, “Where \textit{is} that cult prostitute \textit{that was} at Eynayim by the roadside?” And they said, “There is no cult prostitute here.”
\verse Then he returned to Judah and said, “I could not find her. Morever, the men of the place said, ‘There is no cult prostitute here.’ ”
\verse And Judah said, “Let her take \textit{them} for herself, lest we be \textit{laughed at}. Behold, I sent this kid, but you could not find her.”
\verse And \textit{about three months later} it was told to Judah, “Tamar your daughter-in-law has played the whore, and now, behold, she has conceived by prostitution.” And Judah said, “Bring her out and let her be burned.”
\verse She was brought out, but she sent to her father-in-law saying, “By the man to whom these \textit{belong} I have conceived.” And she said, “Now discern to whom these \textit{belong}: the seal and cord and the staff.”
\verse Then Judah recognized \textit{them} and said, “She is more righteous than I, since I did not give her to my son Shelah.” And he did not know her again.
\verse And it happened \textit{that} at the time she gave birth that, behold, twins \textit{were} in her womb.
\verse And it happened \textit{that} at her labor one \textit{child} put out a hand. And the midwife took \textit{it} and tied a crimson thread on his hand saying, “This \textit{one} came out first.”
\verse Then his hand drew back and, behold, his brother came out, and she said, “What a breach you have made for yourself!” And she called his name Perez.
\verse And afterward his brother who \textit{had} the crimson thread on his hand came out. And his name was called Zerah.
\end{biblechapter}

\begin{biblechapter} % Genesis 39
\verseWithHeading{Joseph in Potiphar’s House} Now Joseph had been brought down to Egypt, and Potiphar, a court official of Pharaoh, commander of the guard, an Egyptian, bought him from the hand of the Ishmaelites who had brought him down there.
\verse And Adonai was with Joseph, and he became a successful man. And he was in the house of his master, the Egyptian.
\verse And his master observed that Adonai \textit{was} with him, and everything that \textit{was} in his hand to do Adonai made successful.
\verse And Joseph found favor in his eyes and he served him. Then he appointed him over his house and all that he owned he put into his hand.
\verse And it happened \textit{that} from the time he appointed him over his house and over all that he had, Adonai blessed the house of the Egyptian on account of Joseph. And the blessing of Adonai was upon all that he had in the house and in the field.
\verse And he left all that he had in the hand of Joseph, and \textit{he did not worry about anything} except the food that he ate. Now Joseph was \textit{well built and handsome}.
\verse And it happened \textit{that} after these things his master’s wife cast her eyes on Joseph, and she said, “Lie with me.”
\verse But he refused and said to his master’s wife, “Look, my master \textit{does not worry about} what \textit{is} in the house, and everything he owns he has put in my hand.
\verse He has no greater \textit{authority} in this house than me, and he has not withheld anything from me except you, since you \textit{are} his wife. Now how could I do this great wickedness and sin against God?”
\verse And it happened \textit{that} as she spoke to Joseph \textit{day after day}, he did not heed her to lie beside her or to be with her.
\verse \textit{But one particular day} he came into the house to do his work and none of the men of the house were there in the house,
\verse she seized him by his garment \textit{and} said, “Lie with me!” And he left his garment in her hand and fled, and he went outside.
\verse And it happened \textit{that} when she saw that he left his garment in her hand and fled outside,
\verse she called to the men of her house and said to them, “Look! He brought a Hebrew man to us to mock us! He came to me to lie with me, and I cried out with a loud voice.
\verse And when he heard \textit{me}, that I raised my voice and called out, he left his garment beside me and fled, and he went outside.”
\verse Then she put his garment beside her until his master came to his house.
\verse Then she spoke to him according to these words, saying, “The Hebrew slave that you brought to us came to me to make fun of me.
\verse And it happened \textit{that} as I raised my voice and called out, he left his garment beside me and fled outside.”
\verse And when his master heard the words of his wife that she spoke to him, “\textit{This is what your servant did to me},” \textit{he became very angry}.
\verse And Joseph’s master took him and put him into prison, the place that the king’s prisoners were confined. And he was there in prison.
\verse And Adonai was with Joseph, and showed loyal love to him, and gave him favor in the eyes of the chief of the prison.
\verse And the chief of the prison put all the prisoners that \textit{were} in the prison into the hand of Joseph. And everything that was done there, he \textit{was} the one who did \textit{it}.
\verse The chief of the prison \textit{did not worry about} anything in his hand, since Adonai \textit{was} with him. And whatever he did Adonai made \textit{it} successful.
\end{biblechapter}

\begin{biblechapter} % Genesis 40
\verseWithHeading{Joseph Interprets Dreams in Prison} And it happened \textit{that} after these things the cupbearer of the king of Egypt and \textit{his} baker did wrong against their lord, against the king of Egypt.
\verse And Pharaoh was angry with his two officials, with the chief cupbearer and chief baker.
\verse And he put them in custody in the house of the chief of the guard, into the prison where Joseph was confined.
\verse And the chief of the guard appointed Joseph \textit{to be} with them, and he attended them. And they were in custody \textit{many days}.
\verse And the two of them, the cupbearer and the baker of the king of Egypt, who \textit{were} confined in the prison, dreamed a dream, each his own dream, with its own interpretation.
\verse When Joseph came to them in the morning he looked at them, and behold, they were troubled.
\verse And he asked the court officials of Pharaoh that \textit{were} with him in the custody of his master’s house, “Why \textit{are} your faces sad today?”
\verse And they said to him, “We \textit{each} dreamed a dream, but there is no one to interpret it.” And Joseph said to them, “Do not interpretations belong to God? Please tell \textit{them} to me.”
\verse Then the chief cupbearer told his dream to Joseph, and he said to him, “In my dream, now behold, \textit{there was} a vine before me,
\verse and on the vine \textit{were} three branches. And as it budded, its blossoms came up, \textit{and} its clusters of grapes grew ripe.
\verse And the cup of Pharaoh \textit{was} in my hand, and I took the grapes and squeezed them into the cup of Pharaoh. Then I placed the cup into the hand of Pharaoh.”
\verse Then Joseph said to him, “This \textit{is} its interpretation: The three branches, they \textit{are} three days.
\verse In three days Pharaoh will lift up your head and will restore you to your office. And you shall put the cup of Pharaoh into his hand as \textit{was} formerly the custom, when you were his cupbearer.
\verse But remember me when it goes well with you, and please may you show kindness with respect to me, and mention me to Pharaoh, and bring me out of this house.
\verse For I was surely kidnapped from the land of the Hebrews, and here also I have done nothing that they should put me in this pit.”
\verse And when the chief baker saw that the interpretation \textit{was} good he said to Joseph, “I also \textit{dreamed}. In my dream, now behold, \textit{there were} three baskets of bread upon my head.
\verse And in the upper basket \textit{were} all sorts of baked foods for Pharaoh, but the birds were eating them out of the basket upon my head.”
\verse Then Joseph answered and said, “This \textit{is} its interpretation: The three baskets, they \textit{are} three days.
\verse In three days Pharaoh will lift your head from you and hang you on a pole, and the birds will eat your flesh from you.”
\verse And it happened \textit{that} on the third day, \textit{which was} Pharaoh’s birthday, he made a feast for all his servants. And he lifted up the head of the chief cupbearer and the head of the chief baker in the midst of his servants.
\verse And he restored the chief cupbearer to his cupbearing \textit{position}. And he placed the cup in the hand of Pharaoh.
\verse But the chief baker he hanged as Joseph had interpreted to them.
\verse But the chief cupbearer did not remember Joseph, but forgot him.
\end{biblechapter}

\begin{biblechapter} % Genesis 41
\verseWithHeading{Joseph Interprets Pharaoh’s Dreams} And it happened \textit{that} after \textit{two full years} Pharaoh dreamed, and behold, he was standing by the Nile.
\verse And behold, seven cows, \textit{well built and fat}, were coming up from the Nile, and they grazed among the reeds.
\verse And behold, seven other cows came up after them from the Nile, \textit{ugly and gaunt}, and they stood beside those cows on the bank of the Nile.
\verse And the \textit{ugly and gaunt} cows ate the seven \textit{well built and fat} cows. Then Pharaoh awoke.
\verse And he fell asleep and dreamed a second time, and behold, seven ears of grain, plump and good, were coming out of one stalk.
\verse And behold, seven thin ears of grain, scorched by the east wind, sprouted up after them.
\verse And the thin ears of grain swallowed up the seven plump and full ears of grain. Then Pharaoh awoke, and behold, \textit{it was} a dream.
\verse And it happened \textit{that} in the morning his spirit was troubled, and he sent and called all of the magicians of Egypt, and all its wise men, and Pharaoh told his dream to them. But \textit{they had no interpretation} for Pharaoh.
\verse Then the chief of the cupbearers spoke with Pharaoh, saying, “I remember my sins today.
\verse Pharaoh was angry with his servants, and he put me and the chief baker in the custody of the house of the chief of the guard.
\verse And we dreamed a dream one night, I and he, \textit{each with a dream that had a meaning}.
\verse And there with us \textit{was} a young man, a Hebrew servant of the chief of the guard, and we told him \textit{the dream}, and he interpreted our dreams for us, each according to his dream he interpreted.
\verse And it happened just as he interpreted to us, so it was. He restored me to my office, and him he hanged.”
\verse Then Pharaoh sent and called \textit{for} Joseph, and they brought him quickly from the prison. And he shaved and changed his clothing, and came to Pharaoh.
\verse Then Pharaoh said to Joseph, “I dreamed a dream, but there is none to interpret it. Now, I have heard concerning you \textit{that when} you hear a dream \textit{you can} interpret it.”
\verse Then Joseph answered Pharaoh saying, “\textit{It is not in my power}; God will answer \textit{concerning} the well-being of Pharaoh.”
\verse And Pharaoh said to Joseph, “\textit{Now} in my dream, behold, I was standing on the bank of the Nile,
\verse and behold, seven cows, \textit{well built and fat}, were coming up from the Nile, and they grazed among the reeds.
\verse And behold, seven other cows came up after them from the Nile, very \textit{ugly and gaunt}—never have I seen \textit{any} as them in all the land of Egypt for ugliness.
\verse And the thin and ugly cows ate the former seven healthy cows.
\verse But \textit{when} they went into their bellies it could not be known that they went into their bellies, for their appearance \textit{was} as ugly as at the beginning. Then I awoke.
\verse Then I saw in my dream and behold, seven ears of grain were coming out of one stalk, full and good.
\verse And behold, seven withered ears of grain, thin \textit{and} scorched by the east wind, sprouted up after them.
\verse And the thin ears of grain swallowed up the seven good ears of grain. And I told the magicians, but there was none to explain \textit{it} to me.”
\verse Then Joseph said to Pharaoh, “The dreams of Pharaoh \textit{are} one. God has revealed to Pharaoh what he is about to do.
\verse The seven good cows, they are seven years, and the seven good ears of grain, they \textit{are} seven years. The dreams \textit{are} one.
\verse And the seven thin and ugly cows coming up after them, they \textit{are} seven years, and the seven empty ears of grain, scorched by the east wind, they are \textit{also} seven years of famine.
\verse This \textit{is} the word that I have spoken to Pharaoh; God has shown Pharaoh what he is about to do.
\verse Behold, seven years of great abundance are coming throughout the whole land of Egypt.
\verse Then seven years of famine will arise after them, and all the abundance in the land of Egypt will be forgotten. The famine will consume the land.
\verse Abundance in the land will not be known because of the famine \textit{that follows}, for it will be very heavy.
\verse Now concerning the repetition of the dream twice to Pharaoh, \textit{it is} because the matter \textit{is} established by God, and God will do \textit{it} quickly.
\verse Now then, let Pharaoh select a man \textit{who is} discerning and wise, and let him set him over the land of Egypt.
\verse Let Pharaoh do \textit{this}, and let him appoint supervisors over the land, and let him take one-fifth from the land of Egypt in the seven years of abundance.
\verse Then let them gather all the food of these coming good years and let them pile up grain under the hand of Pharaoh \textit{for} food in the cities, and let them keep \textit{it}.
\verse Then the food shall be as a deposit for the land for the seven years of the famine that will be in the land of Egypt, that the land will not perish on account of the famine.”
\verseWithHeading{Joseph Rises to Power} And the plan was good in the eyes of Pharaoh and in the eyes of all his servants.
\verse Then Pharaoh said to his servants, “Can we find a man like this in whom is the spirit of God?”
\verse Then Pharaoh said to Joseph, “Since God has made all of this known to you there is no one as discerning and wise as you.
\verse You shall be over my house, and to your word all my people shall submit. Only \textit{with respect to} the throne will I be greater than you.”
\verse Then Pharaoh said to Joseph, “See, I have set you over all the land of Egypt.”
\verse Then Pharaoh removed his signet ring from his finger and put it on the finger of Joseph. And he clothed him with garments of fine linen, and he put a chain of gold around his neck.
\verse And he had him ride in his second chariot. And they cried out before him, “Kneel!” And Pharaoh set him over all the land of Egypt.
\verse Then Pharaoh said to Joseph, “I \textit{am} Pharaoh, but without your consent no one will lift his hand or his foot in all the land of Egypt.”
\verse And Pharaoh called the name of Joseph Zaphenath-paneah and gave him Asenath, the daughter of Potiphera, priest of On, as a wife. And Joseph went out over the land of Egypt.
\verse Now Joseph \textit{was thirty years old} when he stood before Pharaoh, the king of Egypt. And Joseph went out from the presence of Pharaoh and traveled through the whole land of Egypt.
\verse And the land produced a plenty in the seven years of abundance.
\verse And he gathered all the food of the seven years which \textit{occurred} in the land of Egypt. And he stored the food in the cities. The food of the field that surrounded \textit{each} city he stored in its midst.
\verse And Joseph piled up grain like the sand of the sea in great abundance until he stopped counting \textit{it}, for \textit{it could not be counted}.
\verse Before the years of famine came, Asenath, daughter of Potiphera priest of On, bore two sons to him.
\verse And Joseph called the name of the firstborn Manasseh, for \textit{he said}, “God has caused me to forget all my hardship and all my father’s house.”
\verse And the name of the second he called Ephraim, for \textit{he said}, “God has made me fruitful in the land of my misfortune.”
\verse And the seven years of abundance which \textit{were} in the land of Egypt came to an end.
\verse And the seven years of famine began to come as Joseph had said. And there was famine in all of the countries, but in the land of Egypt there was food.
\verse And when all the land of Egypt was hungry the people cried out to Pharaoh for food. And Pharaoh said to all the land of Egypt, “Go to Joseph; what he says to you, you must do.”
\verse And the famine was over the whole land, and Joseph opened all the storehouses and sold \textit{food} to the Egyptians. And the famine was severe in the land of Egypt.
\verse And every land came to Egypt to Joseph to buy grain, for the famine was severe in every land.
\end{biblechapter}

\begin{biblechapter} % Genesis 42
\verseWithHeading{Joseph’s Brothers Go to Egypt for Food} When Jacob realized that there was grain in Egypt, Jacob said to his sons, “Why do you look at one another?”
\verse Then he said, “Look, I have heard that there is grain in Egypt. Go down there and buy grain for us there that we may live and not die.”
\verse And the ten brothers of Joseph went down to buy grain from Egypt.
\verse But Jacob did not send Benjamin, the brother of Joseph, for \textit{he feared harm would come to him}.
\verse Then the sons of Israel went to buy grain amid those \textit{other people} who went \textit{as well}, for there was famine in the land of Canaan.
\verse Now Joseph was the governor over the land. He \textit{was} the one who sold \textit{food} to all the people of the land. And the brothers of Joseph came and bowed down to him with their faces to the ground.
\verse And Joseph saw his brothers and recognized them, but he pretended to be a stranger to them. And he spoke with them harshly and said to them, “From where have you come?” And they said, “From the land of Canaan to buy food.”
\verse And Joseph recognized his brothers, but they did not recognize him.
\verse And Joseph remembered the dreams which he had dreamed concerning them, and he said to them, “You are spies! You have come to see the nakedness of the land!”
\verse And they said to him, “No, my lord, but your servants have come to buy food.
\verse We all are sons of one man. We \textit{are} honest \textit{men}. We, your servants, are not spies.”
\verse Then he said to them, “No, but you have come to see the nakedness of the land.”
\verse Then they said, “We, your servants, \textit{are} twelve brothers, the sons of one man in the land of Canaan, but behold, the youngest \textit{is} with our father today, and one is no more.”
\verse But Joseph said to them, “It \textit{is} what I said to you—you \textit{are} spies.
\verse By this you shall be tested. By the life of Pharaoh you will not go out from here unless your youngest brother comes here.
\verse Send one of you, and let him bring your brother, but you will be kept in prison so that your words might be tested \textit{to see} if \textit{there is} truth with you. And if not, by the life of Pharaoh surely you \textit{are} spies.”
\verse Then he gathered them into the prison for three days.
\verse On the third day Joseph said to them, “Do this and you will live; I fear God.
\verse If you \textit{are} honest, let one of your brothers be kept in prison \textit{where you are now being kept}, but \textit{the rest of} you go, carry grain for the famine for your households.
\verse You must bring your youngest brother to me, and then your words will be confirmed and you will not die.” And they did so.
\verse Then each said to his brother, “Surely we \textit{are} guilty on account of our brother when we saw the anguish of his soul when he pleaded for mercy to us and we would not listen. Therefore this trouble has come to us.”
\verse Then Reuben answered them, saying, “Did I not say to you, do not sin against the boy? But you did not listen, and now, behold, his blood has been sought.”
\verse Now they did not know that Joseph understood, for the interpreter \textit{was} between them.
\verse And he turned away from them and wept. Then he returned to them and spoke to them, and took Simeon from them and tied him up in front of them.
\verse Then Joseph gave orders to fill their bags with grain and to return their money to each sack, and to give them provisions for the journey. Thus he did for them.
\verse Then they loaded their grain upon their donkeys and went \textit{away} from there.
\verse And one \textit{of them later} opened his sack to give fodder to his donkey at the lodging place and saw his money—behold, it \textit{was} in the mouth of his sack.
\verse And he said to his brothers, “My money was returned and moreover, behold, \textit{it is} in my sack!” Then \textit{their hearts failed them} and each of them trembled \textit{and} said, “What \textit{is} this God has done to us?”
\verse And when they came to Jacob their father in the land of Canaan they told him everything \textit{that} had happened to them, saying,
\verse “The man, the lord of the land, spoke harshly to us and treated us as \textit{if we were} spying out the land.
\verse But we said to him, ‘We \textit{are} honest; we are not spies.
\verse We \textit{are} twelve brothers, the sons of our father. One is no more and the youngest \textit{is} with our father now in the land of Canaan.’
\verse Then the man, the lord of the land, said to us, ‘By this I will know that you \textit{are} honest. Leave one \textit{brother} with me, and take \textit{food for} the famine in your households and go.
\verse And bring your youngest brother to me. Then I will know that you \textit{are} not spies but you \textit{are} honest. And I will give your brother \textit{back} to you, and you will trade in the land.’ ”
\verse And it happened \textit{that when} they emptied their sacks, behold, each one’s pouch of money \textit{was} in his sack. And when they and their father saw the pouches of their money, they were greatly distressed.
\verse And Jacob their father said to them, “You have bereaved me—Joseph is no more and Simeon is no more, and Benjamin you would take! All of this \textit{is} against me!
\verse Then Reuben said to his father, “You may kill my two sons if I do not bring him back to you. Put him in my hand and I myself will return him to you.”
\verse But he said, “My son shall not go down with you, for his brother is dead and he alone remains. \textit{If} harm meets him on the journey that you would take, you would bring down my gray head in sorrow to Sheol.”
\end{biblechapter}

\begin{biblechapter} % Genesis 43
\verseWithHeading{Joseph’s Brothers Return to Egypt} Now the famine in the land \textit{was} severe.
\verse And it happened \textit{that} as they finished eating the grain which they had brought from Egypt their father said to them, “Return and buy a little food for us.”
\verse Then Judah said to him, “The man solemnly admonished us, saying, ‘You shall not see my face unless your brother \textit{is} with you.’
\verse \textit{If you will send} our brother with us, we will go down and buy food for you,
\verse but \textit{if you will not send} \textit{him}, we will not go down, for the man said to us, ‘You shall not see my face unless your brother \textit{is} with you.’ ”
\verse Then Israel said, “Why did you bring trouble to me by telling the man you still had a brother?”
\verse And they said, “The man asked explicitly about us and about our family, saying, ‘Is your father still alive? Do you have a brother?’ And we answered him according to these words. How could we know that he would say, ‘Bring down your brother’?”
\verse Then Judah said to his father Israel, “Send the boy with me, and let us arise and go, so that we will live and not die—you, we, and our children.
\verse I myself will be surety for him. You may seek him from my hand. If I do not bring him back to you and present him before you, then I will stand guilty before you forever.
\verse Surely if we had not hesitated by this \textit{time} we would have returned twice.”
\verse Then their father Israel said to them, “If \textit{it must be} so then do this. Take some of the best products of the land in your bags and take them down to the man as a gift—a little balm and honey, aromatic gum and myrrh, and pistachios and almonds.
\verse And take double \textit{the} money in your hands. Take back the money that was returned in the mouth of your sacks. Perhaps it \textit{was} a mistake.
\verse And take your brother. Now arise and return to the man.
\verse And may El-Shaddai grant you compassion before the man that he may release your other brother to you and Benjamin. As for me, if I am bereaved, I am bereaved.”
\verse So the men took this gift, and they took double money in their hands, and Benjamin, and they rose up and went down to Egypt and stood before Joseph.
\verse When Joseph saw Benjamin with them he said to the one who \textit{was} over his household, “Bring the men into the house and slaughter and prepare \textit{an animal}, for the men shall eat with me at noon.”
\verse And the man did as Joseph had said, and the man brought the men into the house of Joseph.
\verse And the men were afraid when they were brought into the house of Joseph. And they said “We were brought \textit{here} on account of the money that was returned to our sacks the first time, that he might attack us and fall upon us to take us as slaves with our donkeys.”
\verse So they approached the man who \textit{was} over Joseph’s house and spoke to him at the doorway of the house.
\verse And they said, “Please, my lord, we surely came down once before to buy food,
\verse but when we came to the place of lodging and we opened our sacks, then behold, each one’s money \textit{was} in the mouth of his sack—our money in its \textit{full} weight—so we have returned \textit{with} it in our hands.
\verse Now, other money we have brought down in our hand to buy food. We do not know who put our money in our sacks.”
\verse And he said, “Peace to you; do not be afraid. Your God and the God of your father must have given you a treasure in your sacks; your money came to me.” And he brought Simeon out to them.
\verse Then the man brought the men into Joseph’s house and he gave them water and washed their feet, and gave fodder to their donkeys.
\verse Then they laid out the gift until Joseph came at noon, for they had heard that they were to eat food there.
\verse And when Joseph came into the house they brought the gift that \textit{was} in their hand into the house to him, and they bowed down before him to the ground.
\verse And \textit{he greeted them} and said, “Is your father well, the old man of whom you spoke? Is he still alive?”
\verse And they said, “Your servant our father \textit{is} well; he is still alive.” And they knelt and bowed down.
\verse Then he lifted up his eyes and saw Benjamin his brother, the son of his mother, and said, “Is this your youngest brother of whom you told me?” And he continued, “God be gracious to you, my son.”
\verse Then Joseph \textit{hurried away}, \textit{being overcome with emotion} toward his brother, and sought for \textit{a place} to cry. Then he went into a room and wept there.
\verse Then he washed his face and went out, now controlling himself, and said, “Serve the food.”
\verse And they served him by himself, and them by themselves, and the Egyptians who were eating with him by themselves, for Egyptians \textit{could not dine} with Hebrews, because that \textit{was} a detestable thing to Egyptians.
\verse And they were seated before him \textit{from} the firstborn according to his birthright \textit{to} the youngest according to his youth. And the men \textit{looked at one another} amazed.
\verse And portions were served to them from \textit{his table}, and the portion of Benjamin was five times greater than the portion of any of them. And they drank and became drunk with him.
\end{biblechapter}

\begin{biblechapter} % Genesis 44
\verseWithHeading{Joseph Tests His Brothers} Then he commanded \textit{the one} who \textit{was} over his household, saying, “Fill the sacks of the men \textit{with} food as much as they are able to carry, and put each one’s money in the mouth of his sack.
\verse And my cup—the cup of silver—you shall put into the mouth of the sack of the youngest, and the money for his grain. And he did according to the word of Joseph that he had commanded.
\verse \textit{When} the morning light \textit{came} the men were sent away, they and their donkeys.
\verse They went out of the city, \textit{and} had not gone far when Joseph said to \textit{the one} who \textit{was} over his house, “Arise! Pursue after the men and overtake them. Then you shall say to them, ‘Why have you repaid evil for good?
\verse Is this not that from which my master drinks? Now he himself certainly practices divination with it. You have done evil \textit{in} what you have done.’ ”
\verse When he overtook them he spoke these words to them.
\verse And they said to him, “Why has my lord spoken according to these words? Far be it from your servants to do such a thing!
\verse Behold, the money that we found in the mouth of our sacks we returned to you from the land of Canaan. Now why would we steal silver or gold from the house of my lord?
\verse Whoever is found with it from among your servants shall die. And moreover, we will become slaves to my lord.”
\verse Then he said, “Now also according to your words, thus will it be. He who is found with it shall be my slave, but you shall be innocent.”
\verse Then each man quickly brought down his sack to the ground, and each one opened his sack.
\verse And he searched, beginning with the oldest and finishing with the youngest. And the cup was found in the sack of Benjamin.
\verse Then they tore their clothes, and each one loaded his donkey and they returned to the city.
\verse And Judah and his brothers came to the house of Joseph—now he \textit{was} still there—they fell before him to the ground.
\verse Then Joseph said to them, “What is this deed that you have done? Did you not know that a man who \textit{is} like me surely practices divination?”
\verse And Judah said, “What can we say to my lord? What can we speak? Now how can we show ourselves innocent? God has found the guilt of your servants! Behold, we \textit{are} slaves to my lord, both we and also he in whose hand the cup was found.”
\verse But he said, “Far be it from me to do this! The man in whose hand the cup was found, he will become my slave. But as for you, go up in peace to your father.”
\verse But Judah drew near to him and said, “Please my lord, let your servant speak a word in the ears of my lord, and \textit{let not your anger burn} against your servant, for \textit{you are like Pharaoh himself}.
\verse My lord had asked his servants, saying, ‘Do you have a father or a brother?’
\verse And we said to my lord, ‘We have an aged father, and a younger \textit{brother}, the child of his old age, and his brother died, and he alone remains from his mother, and his father loves him.’
\verse Then you said to your servants, ‘Bring him down to me that I may set my eyes upon him.’
\verse Then we said to my lord, ‘The boy cannot leave his father; if he should leave his father, then he would die.’
\verse Then you said to your servants, ‘Unless your youngest brother comes down with you, you shall not again see my face.’
\verse And it happened \textit{that} we went up to your servant, my father, and told him the words of my lord.
\verse And when our father said, ‘Buy a little food for us,’
\verse then we said, ‘We cannot go down. If our youngest brother \textit{is} with us, then we shall go down. For we will not be able to see the face of the man unless our youngest brother \textit{is} with us.’
\verse Then your servant, my father, said to us, ‘You yourselves know that my wife bore two sons to me.
\verse One went out from me, and I said, “Surely he must have been torn to pieces,” and I have never seen him since.
\verse And if you take this one also from me, and he encounters harm, you will bring down my gray head in sorrow to Sheol.’
\verse So now, when I come to your servant, my father, and the boy is not with us—now his life is bound up with his life—
\verse it shall happen \textit{that} when he sees that the boy is gone, he will die. And your servants will bring down the gray head of your servant, our father, to Sheol with sorrow.
\verse For your servant is pledged as surety for the boy by my father, saying, If I do not bring him to you, then I shall be culpable to my father forever.
\verse So then, please let your servant remain in place of the boy as a slave to my lord, and let the boy go up with his brothers.
\verse For how can I go up to my father if the boy is not with me? \textit{I do not want to see} the misery which will find my father.”
\end{biblechapter}

\begin{biblechapter} % Genesis 45
\verseWithHeading{Joseph Reveals His Identity} Then Joseph was not able to control himself before all who were standing by him. And he cried out, “Make every man go out from me!” So no one stood with him when Joseph made himself known to his brothers.
\verse And \textit{he wept loudly}, so that the Egyptians heard \textit{it} and the household of Pharaoh heard \textit{it}.
\verse Then Joseph said to his brothers, “I \textit{am} Joseph! Is my father still alive?” And his brothers were unable to answer him, for they were dismayed at his presence.
\verse So Joseph said to his brothers, “Come near to me, please.” And they drew near. And he said, “I \textit{am} Joseph, your brother, whom you sold into Egypt.
\verse So now, do not be distressed and do not be angry \textit{with yourselves} that you sold me here, for God sent me as deliverance before you.
\verse For these two years the famine \textit{has been} in the midst of the land, but \textit{there will be} five more years where there is no plowing or harvest.
\verse And God sent me before you \textit{all} to preserve for you a remnant in the land and to keep alive among you many survivors.
\verse So now, you yourselves did not send me here, but God put me here as father to Pharaoh and as master of all his household, and a ruler over all the land of Egypt.
\verse Hurry, and go up to my father and say to him, ‘Thus says your son Joseph, God has made me lord of all Egypt. Come down to me and do not delay.
\verse You shall settle in the land of Goshen so that you will be near me, you and your children and your grandchildren, and your flocks and your herds and all that you have.
\verse And I will provide for you there, because \textit{there are} still five years of famine—lest you and your household and all that you have become destitute.’
\verse Now behold, your eyes see, and the eyes of my brother Benjamin see, that \textit{it is I} who am speaking to you.
\verse And you must tell my father of all my honor in Egypt and all that you have seen. Now hurry and bring my father here.”
\verse Then he fell upon the neck of his brother Benjamin and wept, and Benjamin wept upon his neck.
\verse And he kissed all his brothers and wept upon them. And afterward his brothers spoke with him.
\verse Then the report was heard \textit{in} the house of Pharaoh, saying, “Joseph’s brothers have come.” And it pleased Pharaoh and his servants.
\verse Then Pharaoh said to Joseph, “Say to your brothers: ‘Do this—load your donkeys and go back to the land of Canaan,
\verse and take your father and your households and come to me, and I will give you the best of the land of Egypt, and you shall eat the fat of the land.’
\verse And you \textit{Joseph}, are commanded \textit{to say} this: ‘Do this! Take wagons from the land of Egypt for your little ones and your wives, and bring your father and come!
\verse \textit{Do not worry} about your possessions, for the best of all the land of Egypt is yours.’ ”
\verse And the sons of Israel did so. And Joseph gave them wagons at the word of Pharaoh, and gave them provisions for the journey.
\verse To each and to all of them he gave sets of clothing, but to Benjamin he gave three hundred pieces of silver and five sets of clothing.
\verse And to his father he sent \textit{as follows}: ten donkeys carrying the best of Egypt, and ten donkeys carrying grain and food and provisions for his father for the journey.
\verse Then he sent his brothers away, and when they departed he said to them, “Do not be agitated on the journey.”
\verse So they went up from Egypt and came to the land of Canaan to Jacob their father.
\verse And they spoke to him, saying, “Joseph \textit{is} still alive, and he \textit{is} ruler over all the land of Egypt.” And his heart \textit{went numb}, because he did not believe him.
\verse Then they told him all the words of Joseph that he had spoken to them. And when he saw the wagons that Joseph had sent to carry him, then the spirit of Jacob their father revived.
\verse And Israel said, “\textit{It is} enough. Joseph my son \textit{is} still alive. I will go and see him before I die.”
\end{biblechapter}

\begin{biblechapter} % Genesis 46
\verseWithHeading{Jacob and His Offspring Go to Egypt} So Israel journeyed with all that he had, and he came to Beersheba and offered sacrifices to the God of his father, Isaac.
\verse And God spoke to Israel in visions of the night and said, “Jacob, Jacob.” And he said, “Here I \textit{am}.”
\verse Then he said, “I \textit{am} the God of your father. Do not be afraid to go down to Egypt, for I will make you a great nation there.
\verse I myself will go down with you to Egypt, and I myself will also bring you up. And Joseph will place his hand over your eyes.”
\verse So Jacob arose from Beersheba. And the sons of Israel carried their father Jacob, and their little ones and their wives in the wagons Pharaoh had sent to transport him.
\verse And they took their livestock and their possessions that they had acquired in the land of Canaan. And they came to Egypt, Jacob and all his offspring with him,
\verse his sons and his sons’ sons with him, his daughters and his daughters’ daughters with him, into Egypt.
\verse Now these \textit{are} the names of the sons of Israel, who came into Egypt, Jacob and his sons. Reuben, the firstborn of Jacob
\verse and the sons of Reuben: Enoch, Pallu, Hezron, and Carmi.
\verse The sons of Simeon: Jemuel, Jamin, Ohad, Jakin, Zohar, and Shaul, the son of a Canaanite woman.
\verse The sons of Levi: Gershon, Kohath, and Merari.
\verse The sons of Judah: Er, Onan, Shelah, Perez, and Zerah (but Er and Onan died in the land of Canaan). And the sons of Perez were Hezron and Hamul.
\verse The sons of Issachar: Tolah, Puvah, Iob, and Shimron.
\verse The sons of Zebulun: Sered, Elon, and Jahleel.
\verse These \textit{are} the sons of Leah that she bore to Jacob in Paddan-Aram, and Dinah his daughter. His sons and daughters \textit{were} thirty-three persons in all.
\verse The sons of Gad: Ziphion, Haggi, Shuni, Ezbon, Eri, Arodi, and Areli.
\verse The sons of Asher: Imnah, Ishvah, Ishvi, and Beriah, and their sister Serah. And the sons of Beriah: Heber and Malkiel.
\verse There \textit{are} the sons of Zilpah, whom Laban gave to Leah his daughter, and she bore these to Jacob—sixteen persons.
\verse The sons of Rachel, Jacob’s wife: Joseph and Benjamin.
\verse And Ephraim and Manasseh, whom Asenath, daughter of Potiphera, priest of On bore to him, were born to Joseph in the land of Egypt.
\verse The sons of Benjamin: Bela, Beker, Ashbel, Gera, Naaman, Ehi, Rosh, Muppim, Huppim, and Ard.
\verse These \textit{are} the sons of Rachel who were born to Jacob—fourteen persons in all.
\verse The sons of Dan: Hushim.
\verse The sons of Naphtali: Jahzeel, Guni, Jezer, and Shillem.
\verse These \textit{are} the sons of Bilhah whom Laban gave to Rachel his daughter, and she bore these to Jacob—seven persons in all.
\verse All the persons belonging to Jacob who came to Egypt \textit{who were his descendants}, not including the wives of the sons of Jacob \textit{were} sixty-six persons in all.
\verse And the sons of Joseph who were born to him in Egypt \textit{were} two persons. All the persons of the house of Jacob who came to Egypt \textit{were} seventy.
\verse He had sent Judah ahead of him to Joseph to appear before him in Goshen. And they came to the land of Goshen.
\verse Then Joseph harnessed his chariot and went up to meet Israel his father in Goshen. He presented himself to him and fell upon his neck and wept upon his neck a long time.
\verse Then Israel said to Joseph, “Now let me die since I have seen your face, for you are still alive.”
\verse Then Joseph said to his brothers and to his father’s household, “I will go up and report to Pharaoh, and I will say to him, ‘My brothers and my father’s household who \textit{were} in the land of Canaan have come to me.
\verse And the men \textit{are} shepherds, for they are men of livestock, and they have brought their flocks and their cattle and all that they have.’
\verse And it shall be \textit{that} when Pharaoh calls you he will say, ‘What \textit{is} your occupation?’
\verse Then you must say, ‘You servants \textit{are} men of livestock from our childhood until now, both we and also our ancestors,’ so that you may dwell in the land of Goshen, for every shepherd \textit{is} a detestable thing to Egyptians.”
\end{biblechapter}

\begin{biblechapter} % Genesis 47
\verseWithHeading{Jacob Settles in Goshen} So Joseph went and reported to Pharaoh. And he said, “My father and my brothers, with their flocks and their herds, and all that they have, have come from the land of Canaan. Now \textit{they are} here in the land of Goshen.”
\verse And from among his brothers he took five men and presented them before Pharaoh.
\verse And Pharaoh said to his brothers, “What \textit{is} your occupation?” And they said to Pharaoh, “Your servants \textit{are} keepers of sheep, both we and also our ancestors.”
\verse And they said to Pharaoh, “We have come to sojourn in the land, for there is no pasture for your servant’s flocks, for the famine \textit{is} severe in the land of Canaan. So now, please let your servants dwell in the land of Goshen.”
\verse Then Pharaoh said to Joseph, “Your father and your brothers have come to you.
\verse The land of Egypt \textit{is} before you. Settle your father and your brothers in the best of the land. Let them live in the land of Goshen, and if you know there is among them men of ability, then appoint them overseers of my own livestock.”
\verse Then Joseph brought his father Jacob and presented him before Pharaoh. And Jacob blessed Pharaoh.
\verse Then Pharaoh said to Jacob, “\textit{How old are you}?”
\verse And Jacob said to Pharaoh, “The days of the years of my sojourning \textit{are} one hundred and thirty years. Few and hard have been the days of the years of my life, and they have not reached the days of the years of the lives of my ancestors in the days of their sojourning.”
\verse And Jacob blessed Pharaoh, and he went out from the presence of Pharaoh.
\verse And Joseph settled his father and his brothers, and he gave them property in the land of Egypt in the best part of the land, in the land of Rameses, as Pharaoh had instructed.
\verse And Joseph provided his father and his brothers and all the household of his father with food, according to the number of their children.
\verseWithHeading{The Famine in Egypt Continues} Now there was no food in all the land, for the famine \textit{was} very severe. And the land of Egypt languished, with the land of Canaan, on account of the famine.
\verse And Joseph collected all the money found in the land of Egypt and in the land of Canaan in exchange for the grain that they were buying. And Joseph brought the money into the house of Pharaoh.
\verse And when the money was spent in the land of Egypt and from the land of Canaan, all of Egypt came to Joseph, saying, “Give us food! Why should we die before you? For the money is used up.”
\verse And Joseph said, “Give your livestock and I will give you \textit{food} in exchange for your livestock if \textit{your} money is used up.”
\verse So they brought their herds to Joseph, and Joseph gave food to them in exchange for horses, their flocks, and their cattle and donkeys. And he provided them with food in exchange for all their livestock that year.
\verse When that year ended, they came to him in the following year and said to him, “We cannot hide from my lord that \textit{our} money and livestock belong to my lord. Nothing remains before my lord except our bodies and our land.
\verse Why should we die in front of you, both we and our land? Buy us and our land in exchange for food, then we and our land will be servants to Pharaoh. Then give us seed and we shall live and not die, and the land will not become desolate.”
\verse So Joseph bought all the land of Egypt for Pharaoh, for each Egyptian sold his field, for the famine \textit{was} severe upon them. And the land became Pharaoh’s.
\verse As for the people, he transferred them to the cities, from one end of the territory of Egypt to the other.
\verse Only the land of the priests he did not buy, for \textit{there was} an allotment for the priests from Pharaoh, and they \textit{lived on} the allotment that Pharaoh gave to them. Therefore they did not sell their land.
\verse And Joseph said to the people, “Look, I have bought you and your land this day for Pharaoh. Here \textit{is} seed for you so you can sow the land.
\verse And it shall happen \textit{that} at the harvest, you must give a fifth to Pharaoh and four-fifths shall be yours, as seed for the field and for your food and for those who \textit{are} in your households, and as food for your little ones.”
\verse And they said, “You have saved our lives. \textit{If} we have found favor in the eyes of my lord, we will be servants to Pharaoh.”
\verse So Joseph made it a statute unto this day concerning the land of Egypt: one fifth to Pharaoh. Only the land of the priests alone did not belong to Pharaoh.
\verse So Israel settled in the land of Egypt, in the land of Goshen. And they acquired possessions in it and were fruitful and multiplied greatly.
\verse And Jacob lived in the land of Egypt seventeen years. And the days of Jacob, the years of his life, were one hundred and forty-seven years.
\verse When \textit{the time of Israel’s death drew near}, he called to his son, to Joseph. And he said to him, “If I have found favor in your eyes, please put your hand under my thigh, that you might \textit{vow} to deal kindly and faithfully with me. Please do not bury me in Egypt,
\verse but let me lie with my ancestors. Carry me out of Egypt and bury me in their burial site.” And he said, “I will do according to your word.”
\verse Then he said, “Swear to me.” And he swore to him. Then Israel bowed himself on the head of the bed.
\end{biblechapter}

\begin{biblechapter} % Genesis 48
\verseWithHeading{Jacob Blesses Ephraim and Manasseh} And it happened \textit{that} after these things, it was said to Joseph, “Behold, your father \textit{is} ill.” And he took his two sons with him, Ephraim and Manasseh.
\verse And it was told to Jacob, “Behold, your son Joseph has come to you.” Then Israel strengthened himself and he sat up in the bed.
\verse Then Jacob said to Joseph, “El-Shaddai appeared to me in Luz, in the land of Canaan, and blessed me,
\verse and said to me, ‘Behold, I will make you fruitful and make you numerous, and will make you a company of nations. And I will give this land to your offspring after you \textit{as} an everlasting possession.’
\verse And now, your two sons who were born to you in the land of Egypt before my coming to you in Egypt, are mine. Ephraim and Manasseh shall be mine as Reuben and Simeon \textit{are}.
\verse And your children whom you father after them shall be yours. By the name of their brothers they shall be called, with respect to their inheritance.
\verse As for me, when I came to Paddan-Aram Rachel died \textit{to my sorrow} in the land of Canaan on the way when \textit{there was} still some distance to go to Ephrath. And I buried her there on the way to Ephrath (that \textit{is}, Bethlehem).”
\verse When Israel saw the sons of Joseph he said, “Who \textit{are} these?”
\verse Then Joseph said to his father, “They \textit{are} my sons whom God has given me here.” And he said, “Please bring them to me that I may bless them.”
\verse Now the eyes of Israel were dim on account of old age; he was not able to see. So he brought them near to him, and he kissed them and embraced them.
\verse And Israel said to Joseph, “I did not expect to see your face and behold, God has also shown me your offspring.”
\verse Then Joseph removed them from his knees and bowed down with his face to the ground.
\verse And Joseph took the two of them, Ephraim at his right \textit{to} the left of Israel, and Manasseh at his left \textit{to} the right of Israel. And he brought them near to him.
\verse And Israel stretched out his right hand and put \textit{it} on the head of Ephraim (now he was the younger), and his left hand on the head of Manasseh, crossing his hands, for Manasseh \textit{was} the firstborn.
\verse And he blessed Joseph and said,
\verse “The God before whom my fathers, Abraham and Isaac, walked, 
The God who shepherded me \textit{all my life} unto this day,
\verse When Joseph saw that his father put his right hand on the head of Ephraim, he was displeased. And he took hold of his father’s hand to remove it from the head of Ephraim \textit{over} to the head of Manasseh.
\verse And Joseph said to his father, “Not so, my father; because this one \textit{is} the firstborn. Put your right \textit{hand} upon his head.”
\verse But his father refused and said, “I know, my son; I know. He also shall become a people, and he also shall be great, but his younger brother shall be greater than him, and his offspring shall become a multitude of nations.”
\verse So he blessed them that day, saying, Through you Israel shall pronounce blessing, saying, 
‘May God make you like Ephraim and like Manasseh.’ ”
\verse So he put Ephraim before Manasseh.
\verse And Israel said to Joseph, “Behold, I \textit{am about} to die, but God will be with you and will bring you back to the land of your ancestors.
\end{biblechapter}

\begin{biblechapter} % Genesis 49
\verseWithHeading{Jacob Blesses His Twelve Sons} Then Jacob called his sons and said, “Gather together so that I can tell you what will happen with you in \textit{days to come}.
\verse Assemble and hear, O sons of Jacob! 
Listen to Israel your father!
\verse Reuben, you \textit{are} my firstborn, 
my strength, and the firstfruit of my vigor, 
excelling in rank and excelling in power.
\verse Unstable as water, you shall not excel \textit{any longer}, 
for you went up upon the bed of your father, 
then defiled \textit{it}. You went up upon my couch!
\verse Simeon and Levi \textit{are} brothers; 
weapons of violence \textit{are} their swords. 
Let me not come into their council.
\verse Let not my person be joined to their company. 
For in their anger they killed men, 
and at their pleasure they hamstrung cattle.
\verse Cursed be their anger, for \textit{it is} fierce, 
and their wrath, for \textit{it is} cruel. 
I will divide them in Jacob, 
and I will scatter them in Israel.
\verse Judah, \textit{as for} you, your brothers shall praise you. 
Your hand \textit{shall be} on the neck of your enemies. 
The sons of your father shall bow down to you.
\verse Judah \textit{is} a lion’s cub. 
From the prey, my son, you have gone up. 
He bowed down; he crouched like a lion and as a lioness. 
Who shall rouse him?
\verse The scepter shall not depart from Judah, 
nor the ruler’s staff between his feet, 
until Shiloh comes. 
And to him shall be the obedience of nations.
\verse Binding his donkey to the vine 
and his donkey’s colt to the choice vine, 
he washes his clothing in the wine 
and his garment in the blood of grapes.
\verse The eyes \textit{are} darker than wine, 
and the teeth whiter than milk.
\verse Zebulun shall settle by the shore of the sea. 
He \textit{shall become} a haven for ships, 
and his border \textit{shall be} at Sidon.
\verse Issachar \textit{is} a strong donkey, 
crouching between the sheepfolds.
\verse He saw a resting place that \textit{was} good, 
and land that \textit{was} pleasant. 
So he bowed his shoulder to the burden 
and became a servant of forced labor.
\verse Dan shall judge his people 
as one of the tribes of Israel.
\verse Dan shall be a serpent on the way, 
a viper on the road 
that bites the heels of a horse, 
so that its rider falls backward.
\verse I wait for your salvation, O Adonai.
\verse Bandits shall attack Gad, 
but he shall attack \textit{their} heels.
\verse Asher’s food \textit{is} delicious, 
and he shall provide from the king’s delicacies.
\verse Naphtali \textit{is} a doe running free 
that puts forth beautiful words.
\verse Joseph \textit{is} the bough of a fruitful vine, 
a fruitful bough by a spring. 
His branches climb over the wall.
\verse \textit{The archers} fiercely attacked him. 
They shot arrows \textit{at him} and were hostile to him.
\verse But his bow remained in a steady position; 
\textit{his arms} were made agile 
by the hands of the Mighty One of Jacob. 
From there \textit{is} the Shepherd, the Rock of Israel.
\verse Because of the God of your father he will help you 
and \textit{by} Shaddai he will bless you 
with the blessings of heaven above, 
blessings of the deep that crouches beneath, 
blessings of the breasts and the womb.
\verse The blessings of your father 
are superior to the blessings of my ancestors, 
to the bounty of the everlasting hills. 
May they be on the head of Joseph, 
and on the forehead of the prince of his brothers.
\verse Benjamin \textit{is} a devouring wolf, 
devouring the prey in the morning, 
and dividing the plunder in the evening.
\verseWithHeading{The Death and Burial of Jacob} All these \textit{are} the twelve tribes of Israel, and this \textit{is} what their father said to them when he blessed them, each according to their blessing.
\verse Then he instructed them and said to them, “I am \textit{about to be} gathered to my people. Bury me among my ancestors in the cave that \textit{is} in the field of Ephron the Hittite,
\verse in the cave that \textit{is} in the field of Machpelah that \textit{is} before Mamre in the land of Canaan, which Abraham bought with the field from Ephron the Hittite as a burial site.
\verse There they buried Abraham and Sarah his wife. There they buried Isaac and Rebekah his wife. And there I buried Leah—
\verse the purchase of the field and the cave which \textit{was} in it from the Hittites.”
\verse When Jacob finished instructing his sons he drew his feet up to the bed. Then he took his last breath and was gathered to his people.
\end{biblechapter}

\begin{biblechapter} % Genesis 50
\verseWithHeading{Jacob’s Funeral and Joseph’s Remaining Time in Egypt} Then Joseph fell on the face of his father and wept upon him and kissed him.
\verse And Joseph instructed his servants the physicians to embalm his father. So the physicians embalmed Israel.
\verse Forty days \textit{were required for it}, for thus \textit{are} the days \textit{required for} embalming. And the Egyptians wept for him seventy days.
\verse When the days of his weeping had passed, Joseph spoke to the household of Pharaoh, saying, “If I have found favor in your eyes, please speak in the hearing of Pharaoh, saying,
\verse ‘My father made me swear, saying, “Behold, I \textit{am about} to die. In the tomb that I have hewed out for myself in the land of Canaan—there you must bury me.” So then, please let me go up and let me bury my father; then I will return.’ ”
\verse Then Pharaoh said, “Go up and bury your father as he made you swear.”
\verse So Joseph went up to bury his father. And all the servants of Pharaoh, the elders of his household, and all the elders of the land of Egypt, went up with him,
\verse with all the household of Joseph, his brothers, and the household of his father. They left only their little children and their flocks and their herds in the land of Goshen.
\verse And there also went up with him chariots and horsemen. The company \textit{was} very great.
\verse When they came to the threshing floor of Atad, which \textit{was} beyond the Jordan, they lamented there with a very great and sorrowful wailing. And he made a mourning ceremony for his father seven days.
\verse And when the Canaanites, the inhabitants of the land, saw the mourning ceremony at the threshing floor of Atad they said, “This \textit{is} a severe mourning for the Egyptians.” Therefore its name was called Abel-Mizraim, which \textit{is} beyond the Jordan.
\verse Thus his sons did to him just as he had instructed them.
\verse And his sons carried him to the land of Canaan and buried him in the cave of the field of Machpelah, which field Abraham had bought as a burial site from Ephron the Hittite before Mamre.
\verse And after burying his father, Joseph returned to Egypt, he and his brothers and all who had gone up with him to bury his father.
\verse And when the brothers of Joseph saw that their father \textit{was} dead, they said, “It may be \textit{that} Joseph will hold a grudge against us and pay us back dearly for all the evil that we did to him.”
\verse So they sent \textit{word} to Joseph saying, “Your father commanded \textit{us} before his death, saying,
\verse “Thus you must say to Joseph, ‘O, please now forgive the transgression of your brothers and their sin, for they did evil to you.’ So now, please forgive the transgression of the servants of the God of your father.” And Joseph wept when they spoke to him.
\verse Then his brothers went also and fell before him and said, “Behold, we \textit{are} your servants.”
\verse Then Joseph said to them, “Do not be afraid, for \textit{am} I in the place of God?
\verse As for you, you planned evil against me, \textit{but} God planned it for good, in order to do this—to keep many people alive—as \textit{it is} today.
\verse So then, do not be afraid. I myself will provide for you and your little ones. And he consoled them and \textit{spoke kindly} to them.
\verseWithHeading{The Death of Joseph} So Joseph remained in Egypt, he and the house of his father. And Joseph lived one hundred and ten years.
\verse And Joseph saw Ephraim’s children to the third generation. Moreover, the children of Makir, son of Manasseh, were born on the knees of Joseph.
\verse And Joseph said to his brothers, “I \textit{am about} to die, but God will certainly visit you and bring you up from this land to the land that he swore to Abraham, to Isaac, and to Jacob.”
\verse Then Joseph made the sons of Israel swear an oath, saying, “God will surely visit you, and you shall bring up my bones from here.”
\verse So Joseph died, \textit{being} one hundred and ten years old. They embalmed him and he was placed in a coffin in Egypt.
\end{biblechapter}

\flushcolsend
\biblebook{Exodus}

\begin{biblechapter} % Exodus 1
\verseWithHeading{Israel and Oppression in Egypt} And these are the names of the sons of Israel who came to Egypt; with Jacob, they each came with his \textit{family}:
\verse Reuben, Simeon, Levi, and Judah;
\verse Issachar, Zebulun, and Benjamin;
\verse Dan and Naphtali, Gad and Asher.
\verse And \textit{all those who descended from Jacob} were seventy individuals, and Joseph was in Egypt.
\verse And Joseph died and all of his brothers and all of that generation.
\verse And the \textit{Israelites} were fruitful and multiplied and were many and were very, very numerous, and the land was filled with them.
\verse And a new king rose over Egypt who did not know Joseph.
\verse And he said to his people, “Look, the people of the \textit{Israelites} are greater and more numerous than us.
\verse Come, we must deal shrewdly with them, lest they become many, and when war happens, they also will join our enemies and will fight against us and go up from the land.”
\verse And they appointed commanders of forced labor over them in order to oppress them with their \textit{forced labor}, and they built storage cities for Pharaoh—Pithom and Rameses.
\verse And as he oppressed them, so they became many, and so they spread out, and the Egyptians were afraid because of the presence of the \textit{Israelites}.
\verse And the Egyptians ruthlessly compelled the \textit{Israelites} to work.
\verse And they made their lives bitter with hard work with mortar and with bricks and with all sorts of work in the field—with all their work in which they ruthlessly enslaved them.
\verse And the king of Egypt said to the Hebrew midwives—of whom the name of the one was Shiphrah and the name of the second was Puah—
\verse and he said, “When you help the Hebrews give birth, you will look upon the pairs of testicles; if he is a son, you will put him to death, and if she is a daughter, she will live.”
\verse But the midwives feared God, and they did not do as the king of Egypt had said to them. They let the boys live.
\verse And the king of Egypt summoned the midwives, and he said to them, “Why have you done this thing and let the boys live?”
\verse And the midwives said to Pharaoh, “Because the Hebrew women are not like the Egyptian women, because they are vigorous; before the midwife comes to them, they have given birth.”
\verse And God did the midwives good, and the Israelite people became many and were very numerous.
\verse \textit{And so} because the midwives feared God, he gave them \textit{families}.
\verse And Pharaoh commanded all his people, saying, “Every son who is born you will throw into the Nile, and every daughter you will let live.”
\end{biblechapter}

\begin{biblechapter} % Exodus 2
\verseWithHeading{The Birth and Early Life of Moses} And a man from the \textit{family} of Levi went, and he took \textit{a descendent of Levi}.
\verse And the woman conceived, and she gave birth to a son, and she saw him, that he was a fine baby, and she hid him three months.
\verse But when she could no longer hide him, she got a papyrus basket for him, and she coated it with tar and with pitch, and she placed the boy in it, and she placed it among the reeds on the bank of the Nile.
\verse And his sister stood at a distance to know what would be done to him.
\verse And the daughter of Pharaoh went down to wash at the Nile, while her maidservants were walking alongside the Nile, and she saw the basket in the midst of the reeds, and she sent her slave woman for it and took it
\verse and opened it and saw him—the boy—and it was a lad weeping, and she had compassion for him and said, “This must be from the boys of the Hebrews.”
\verse And his sister said to the daughter of Pharaoh, “Shall I go and call for you a woman from the Hebrews who is nursing so that she will nurse the boy for you?”
\verse And the daughter of Pharaoh said to her, “Go.” And the girl went, and she called the mother of the boy.
\verse And the daughter of Pharaoh said, “Take this boy and nurse him for me, and I myself will give you wages, and the woman took the boy, and she nursed him.
\verse And the boy grew, and she brought him to the daughter of Pharaoh, and he became her son, and she called his name Moses, and she said, “Because I drew him out from the water.”
\verse \textit{And then} in those days when Moses had grown up, he went out to his brothers, and he saw their \textit{forced labor}, and he saw an Egyptian man striking a Hebrew man, one of his brothers.
\verse And he turned here and there, and he saw no one, and he struck the Egyptian, and he hid him in the sand.
\verse And he went out on the second day, and there were two Hebrew men fighting, and he said to the guilty one, “Why do you strike your neighbor?”
\verse And he said, “Who \textit{appointed you as a commander} and a judge over us? Are you intending to kill me like you killed the Egyptian?” And Moses was afraid, and he said, “Surely the matter has become known.”
\verse And Pharaoh heard this matter, and he sought to kill Moses, and Moses fled from Pharaoh, and he lived in the land of Midian, and he lived at \textit{a certain well}.
\verse Now the priest of Midian had seven daughters, and they came and drew water and filled the troughs to water their father’s flock.
\verse And the shepherds came and drove them away, but Moses stood up and came to their rescue and watered their flock.
\verse And they came to Reuel, their father, and he said, “\textit{Why have you come so quickly} today?”
\verse And they said, “An Egyptian man delivered us from the hand of the shepherds, and he even drew water for us and watered the flock.”
\verse And he said to his daughters, “Where is he? \textit{Why then} have you left the man? \textit{Call him so that he can eat some food}.”
\verse And Moses agreed to stay with the man, and he gave Zipporah his daughter to Moses.
\verse And she bore a son, and he called his name Gershom because he said, “I am an alien in a foreign land.”
\verse \textit{And then} during those many days, the king of Egypt died, and the \textit{Israelites} groaned because of the work, and they cried out, and their cry for help because of the work went up to God.
\verse And God heard their groaning, and God remembered his covenant with Abraham, with Isaac, and with Jacob,
\verse and God saw the \textit{Israelites}, and God took notice.
\end{biblechapter}

\begin{biblechapter} % Exodus 3
\verseWithHeading{Adonai’s Plan to Rescue the Israelites} And Moses was a shepherd with the flock of Jethro, his father-in-law, the priest of Midian, and he led the flock to the west of the desert, and he came to the mountain of God, to Horeb.
\verse And the angel of Adonai appeared to him in a flame of fire from the midst of a bush, and he looked, and there was the bush burning with fire, but the bush was not being consumed.
\verse And Moses said, “Let me turn aside and see this great sight. Why does the bush not burn up?”
\verse And Adonai saw that he turned aside to see, and God called to him from the midst of the bush, and he said, “Moses, Moses.” And he said, “Here I am.”
\verse And he said, “You must not come near to here. Take off your sandals from on your feet, because the place on which you are standing, it is holy ground.”
\verse And he said, “I am the God of your father, the God of Abraham, the God of Isaac, and the God of Jacob.” And Moses hid his face because he was afraid of looking at God.
\verse And Adonai said, “Surely I have seen the misery of my people who are in Egypt, and I have heard their cry of distress because of their oppressors, for I know their sufferings.
\verse And I have come down to deliver them from the hand of the Egyptians and to bring them up from this land to a good and wide land, to a land flowing with milk and honey, to the place of the Canaanites and the Hittites and the Amorites and the Perizzites and the Hivites and the Jebusites.
\verse And now, look, the cry of distress of the \textit{Israelites} has come to me, and also I see the oppression with which the Egyptians are oppressing them.
\verse And now come, and I will send you to Pharaoh, and you must bring my people, the \textit{Israelites}, out from Egypt.”
\verse But Moses said to God, “Who am I that I should go to Pharaoh and that I should bring the \textit{Israelites} out from Egypt?”
\verse And he said, “Because I am with you, and this will be the sign for you that I myself have sent you: When you bring the people out from Egypt, you will serve God on this mountain.”
\verse But Moses said to God, “Look, if I go to the \textit{Israelites} and I say to them, ‘The God of your ancestors has sent me to you,’ and they say to me, ‘What is his name?’ then what shall I say to them?”
\verse And God said to Moses, “I am that I am.” And he said, “So you must say to the \textit{Israelites}, ‘I am sent me to you.’ ”
\verse And God said again to Moses, “So you must say to the \textit{Israelites}, ‘Adonai, the God of your ancestors, the God of Abraham, the God of Isaac, and the God of Jacob, has sent me to you. This is my name forever, and this is my remembrance from generation to generation.’
\verse Go and gather the elders of Israel and say to them, ‘Adonai, the God of your ancestors, appeared to me, the God of Abraham, Isaac, and Jacob, saying, “I have carefully attended to you and what has been done to you in Egypt.”
\verse And I said, “I will bring you up from the misery of Egypt to the land of the Canaanites and the Hittites and the Amorites and the Perizzites and the Hivites and the Jebusites, to a land flowing with milk and honey.” ’
\verse And they will listen to your voice, and you will go, you and the elders of Israel, to the king of Egypt, and you will say to him, ‘Adonai, the God of the Hebrews has met with us, and now let us please go on a journey of three days into the desert, and let us sacrifice to Adonai our God.’
\verse But I myself know that the king of Egypt will not allow you to go \textit{unless compelled by a strong hand}.
\verse And I will stretch out my hand, and I will strike Egypt with all of my wonders that I will do in its midst, and \textit{afterward} he will release you.
\verse And I will give this people favor in the eyes of the Egyptians, \textit{and then} when you go, you will not go empty-handed.
\verse And a woman will ask from her neighbor and from the woman dwelling as an alien in her house for objects of silver and objects of gold and garments, and you will put them on your sons and on your daughters; and you will plunder Egypt.”
\end{biblechapter}

\begin{biblechapter} % Exodus 4
\verseWithHeading{Adonai Provides Signs and Help for Speaking} And Moses answered, and he said, “And if they do not believe me and they do not listen to my voice, but they say, ‘Adonai did not appear to you?’ ”
\verse And Adonai said to him, “What is this in your hand?” And he said, “A staff.”
\verse And he said, “Throw it onto the ground.” And he threw it onto the ground, and it became a snake, and Moses fled from it.
\verse And Adonai said to Moses, “Reach out your hand and grasp it by its tail”—” (And he reached out his hand and grabbed it, and it became a staff in his palm.)—
\verse “so that they may believe that Adonai, the God of their ancestors, appeared to you, the God of Abraham, the God of Isaac, and the God of Jacob.”
\verse And Adonai said to him again, “Put your hand into the fold of your garment.” And he put his hand into the fold of his garment, and he took it out, and, \textit{to his surprise}, his hand was afflicted with a skin disease, like snow.
\verse And he said, “Return your hand to the fold of your garment.” And he returned his hand to the fold of his garment, and he took it out from the fold of his garment, and, \textit{to his surprise}, it was restored like the rest of his body.
\verse “\textit{And} if they do not believe you and they will not listen to the voice of the former sign, then they will believe the voice of the latter sign.
\verse \textit{And} if they also do not believe the second of these signs and they will not listen to your voice, then you must take water from the Nile and pour it onto the dry ground, and the water that you take from the Nile will become blood on the dry ground.”
\verse And Moses said to Adonai, “Please, Lord, I am not a man of words, \textit{neither recently nor in the past nor since your speaking} to your servant, because I am heavy of mouth and of tongue.”
\verse And Adonai said to him, “Who gave a mouth to humankind, or who makes mute or deaf or sighted or blind? Is it not I, Adonai?
\verse So then go, and I myself will be with your mouth, and I will teach you what you must speak.”
\verse And he said, “Please, Lord, do send \textit{anyone else whom you wish to send}.”
\verse \textit{And Adonai was angry with} Moses and said, “Is there not Aaron your brother the Levite? I know that he certainly can speak, and also there he is coming out to meet you, and when he sees you, he will rejoice in his heart.
\verse And you will speak to him, and you will put words in his mouth, and I myself will be with your mouth and with his mouth, and I will teach you what you must do.
\verse And he will speak for you to the people, \textit{and then} he will be to you as a mouth, and you will be to him as a god.
\verse And you must take this staff in your hand, with which you will do the signs.”
\verseWithHeading{Moses Leaves Midian and Returns to Egypt} And Moses went, and he returned to Jethro his father-in-law, and he said to him, “Please let me go, and let me return to my brothers who are in Egypt, and let me see whether they are yet alive. And Jethro said to Moses, “Go in peace.”
\verse And Adonai said to Moses in Midian, “Go, return to Egypt because all the men have died who were seeking your life.”
\verse And Moses took his wife and his sons and had them ride on the donkey, and he returned to the land of Egypt, and Moses took the staff of God in his hand.
\verse And Adonai said to Moses, “When you go to return to Egypt, see all of the wonders that I have put in your hand, and do them before Pharaoh, and I myself will harden his heart, and he will not release the people.
\verse And you must say to Pharaoh, ‘Thus says Adonai, “Israel is my son, my firstborn.”
\verse And I said to you, “Release my son and let him serve me,” but you refused to release him. Look, I am about to kill your son, your firstborn.’ ”
\verse \textit{And} on the way, at the place of overnight lodging, Adonai encountered him and sought to kill him.
\verse But Zipporah took a flint knife, and she cut off the foreskin of her son, and she touched his feet, and she said, “Yes, you are a bridegroom of blood to me.”
\verse And he left him alone. At that time she said, “A bridegroom of blood,” because of the circumcision.
\verse And Adonai said to Aaron, “Go to the desert to meet Moses.” And he went and encountered him at the mountain of God and kissed him.
\verse And Moses told Aaron all the words of Adonai, who had sent him—and all the signs that he had commanded him.
\verse And Moses and Aaron went, and they gathered all of the elders of the \textit{Israelites}.
\verse And Aaron spoke all the words that Adonai had spoken to Moses, and he did the signs before the eyes of the people.
\verse And the people believed when they heard that Adonai had attended to the \textit{Israelites} and that he had seen their misery, and they knelt down and they worshiped.
\end{biblechapter}

\begin{biblechapter} % Exodus 5
\verseWithHeading{Pharaoh Rejects Adonai’s Authority and Makes Israel’s Troubles Worse} And afterward, Moses and Aaron went, and they said to Pharaoh, “Thus says Adonai the God of Israel, ‘Release my people so that they may hold a festival for me in the desert.’ ”
\verse And Pharaoh said, “Who is Adonai that I should listen to his voice to release Israel? I do not know Adonai, and also I will not release Israel.”
\verse And they said, “The God of the Hebrews has met with us. Please let us go on a three-day journey into the desert, and let us sacrifice to Adonai our God, lest he strike us with plague or with sword.”
\verse And the king of Egypt said, “Why, Moses and Aaron, do you take the people from their work? Go to your \textit{forced labor}!”
\verse And Pharaoh said, “Look, the people of the land are now many, and you want to stop them from their \textit{forced labor}.”
\verse And on that day Pharaoh commanded the slave drivers over the people and his foremen, saying,
\verse “You must no longer give straw to the people to make the bricks like \textit{before}. Let them go and gather straw for themselves.
\verse But the quota of the bricks that they were making \textit{before} \textit{you must require of them}. You must not reduce from it, because they are lazy. Therefore they are crying out, saying, ‘Let us go and sacrifice to our God.’
\verse Let the work be heavier on the men so that they will do it and not pay attention to words of deception.”
\verse And the slave drivers of the people and their foremen went out, and they spoke to the people, saying, “Thus says Pharaoh, ‘I am not giving you straw.
\verse You go, get straw for yourselves from whatever you find because not a thing is being reduced from your work.’ ”
\verse And the people spread out in all the land of Egypt to gather stubble for the straw.
\verse And the slave drivers were insisting, saying, “Finish your work \textit{for each day} on its day, as \textit{when there was straw}.”
\verse And the foremen of the \textit{Israelites}, whom Pharaoh’s slave drivers had appointed over them, were beaten by men who were saying, “Why have you not completed your portion of brickmaking \textit{as before, both yesterday and today}?”
\verseWithHeading{The Foremen Complain to Pharaoh and Moses, and Moses Complains to Adonai} And the foremen of the \textit{Israelites} came and cried out to Pharaoh, saying, “Why do you treat your servants like this?
\verse Straw is not being given to your servants, but they are saying to us, ‘Make bricks!’ and, look, your servants are being beaten, but it is the fault of your people.”
\verse And he said, “You are lazy, lazy! Therefore you are saying, ‘Let us go; let us sacrifice to Adonai.’
\verse And now go, work, but straw will not be given to you, and you must give the full quota of bricks.”
\verse And the foremen of the \textit{Israelites} saw they were in trouble \textit{with the saying}, “You will not reduce from your bricks \textit{for each day} on its day.”
\verse And they met Moses and Aaron, who were waiting to meet them when they were going out from Pharaoh.
\verse And they said to them, “May Adonai look upon you and judge because you have caused our fragrance to stink in the eyes of Pharaoh and in the eyes of his servants so as to put a sword into their hand to kill us.”
\verse And Moses returned to Adonai and said, “Lord, why have you brought trouble to this people? Why ever did you send me?
\verse And from the time I came to Pharaoh to speak in your name, he has brought trouble to this people, and you have certainly not delivered your people.”
\end{biblechapter}

\begin{biblechapter} % Exodus 6
\verseWithHeading{Adonai Discusses His Name and Israel’s Future} And Adonai said to Moses, “Now you will see what I will do to Pharaoh, because with a strong hand he will release them, and with a strong hand he will drive them out from his land.”
\verse And God spoke to Moses, and he said to him, “I am Adonai.
\verse And I appeared to Abraham, to Isaac, and to Jacob as God Shaddai, but by my name Adonai I was not known to them.
\verse And I not only established my covenant with them to give to them the land of Canaan, the land of their sojournings, in which they dwelt as aliens,
\verse but also I myself heard the groaning of the \textit{Israelites}, whom the Egyptians are making to work, and I remembered my covenant.
\verse Therefore say to the \textit{Israelites}, ‘I am Adonai, and I will bring you out from under the \textit{forced labor} of Egypt, and I will deliver you from their slavery, and I will redeem you with an outstretched arm and with great punishments.
\verse And I will take you \textit{as my people}, and I will be \textit{your God}, and you will know that I am Adonai your God, who brought you out from under the \textit{forced labor} of Egypt.
\verse And I will bring you to the land \textit{that I swore} to give to Abraham, to Isaac, and to Jacob, and I will give it to you as a possession. I am Adonai.”
\verse And Moses spoke thus to the \textit{Israelites}, but they did not listen to Moses, because of \textit{discouragement} and because of hard work.
\verse And Adonai spoke to Moses, saying,
\verse “Go, speak to Pharaoh, the king of Egypt, and let him release the \textit{Israelites} from his land.”
\verse And Moses spoke before Adonai, saying, “Look, the \textit{Israelites} do not listen to me, and how will Pharaoh listen to me, since I am \textit{a poor speaker}?”
\verse And Adonai spoke to Moses and to Aaron, and he commanded them to go to the \textit{Israelites} and to Pharaoh, the king of Egypt, to bring the \textit{Israelites} out from the land of Egypt.
\verseWithHeading{The Genealogy of Moses and Aaron} These are the heads of \textit{their families}. The sons of Reuben, the firstborn of Israel, are Enoch and Pallu, Hezron and Carmi. These are the clans of Reuben.
\verse And the sons of Simeon are Jemuel and Jamin and Ohad and Jakin and Zohar and Shaul the son of the Canaanitess. These are the clans of Simeon.
\verse And these are the names of the sons of Levi according to their genealogies: Gershon and Kohath and Merari, and the years of the life of Levi were \textit{one hundred and thirty-seven years}.
\verse The sons of Gershon are Libni and Shimei according to their clans.
\verse And the sons of Kohath are Amram and Izhar and Hebron and Uzziel, and the years of the life of Kohath were \textit{one hundred and thirty-three years}.
\verse And the sons of Merari are Mahli and Mushi. These are clans of the Levites according to their genealogies.
\verse And Amram took Jochebed his aunt for himself as a wife, and she bore for him Aaron and Moses, and the years of the life of Amram were \textit{one hundred and thirty-seven years}.
\verse And the sons of Izhar are Korah and Nepheg and Zikri.
\verse And the sons of Uzziel are Mishael and Elzaphan and Sithri.
\verse And Aaron took Elisheba the daughter of Amminadab, the sister of Nahshon, for himself as a wife, and she bore for him Nadab and Abihu, Eleazar and Ithamar.
\verse And the sons of Korah are Assir and Elkanah and Abiasaph. These are the clans of the Korahites.
\verse And Eleazar the son of Aaron took for himself one from the daughters of Putiel as a wife, and she bore for him Phinehas. These are the heads of the \textit{families} of the Levites according to their clans.
\verse It was that Aaron and Moses to whom Adonai said, “Bring the \textit{Israelites} out from the land of Egypt according to their divisions.”
\verse They were those who spoke to Pharaoh, the king of Egypt, in order to bring the \textit{Israelites} out from Egypt. It was that Moses and Aaron.
\verse \textit{And so it was on a certain day} Adonai spoke to Moses in the land of Egypt.
\verse And Adonai spoke to Moses, saying, “I am Adonai. Speak to Pharaoh, the king of Egypt, all that I am speaking to you.”
\verse And Moses said before Adonai, “Look, I am \textit{a poor speaker}. And how will Pharaoh listen to me?”
\end{biblechapter}

\begin{biblechapter} % Exodus 7
\verseWithHeading{Adonai Reviews Plans for Bringing the Israelites out of Egypt} And Adonai said to Moses, “See, I have made you as a god to Pharaoh, and Aaron your brother will be your prophet.
\verse You will speak all that I will command you, and Aaron your brother will speak to Pharaoh, and he will release the \textit{Israelites} from his land.
\verse And I myself will harden the heart of Pharaoh, and I will make my signs and my wonders numerous in the land of Egypt.
\verse And Pharaoh will not listen to you, and I will put my hand into Egypt and bring out my divisions, my people, the \textit{Israelites}, from the land of Egypt with great punishments.
\verse And the Egyptians will know that I am Adonai when I stretch out my hand over Egypt and bring the \textit{Israelites} out from their midst.”
\verse And Moses and Aaron did it; as Adonai commanded them, so they did.
\verse (And Moses was \textit{eighty years old}, and Aaron was \textit{eighty-three years old} when they spoke to Pharaoh.)
\verseWithHeading{Provision of a Wonder: Aaron’s Staff Becomes a Snake} And Adonai said to Moses and to Aaron, saying,
\verse “When Pharaoh speaks to you, saying, ‘Do a wonder for yourselves,’ you will say to Aaron, ‘Take your staff and throw it before Pharaoh, and it will become a snake.’ ”
\verse And Moses and Aaron came to Pharaoh, and they did so, as Adonai had commanded. And Aaron threw his staff before Pharaoh and before his servants, and it became a snake.
\verse And Pharaoh also called the wise men and the sorcerers, and they also, the magicians of Egypt, did likewise with their secret arts.
\verse Each threw down his staff, and they became snakes, and Aaron’s staff swallowed up their staffs.
\verse And Pharaoh’s heart was hard, and he did not listen to them, as Adonai had said.
\verseWithHeading{Plague One: Blood} And Adonai said to Moses, “Pharaoh’s heart is \textit{insensitive}; he refuses to release the people.
\verse Go to Pharaoh in the morning. Look, he is going out to the water, and you must wait to meet him on the bank of the Nile, and you must take in your hand the staff that was changed into a snake.
\verse And you must say to him, ‘Adonai, the God of the Hebrews, has sent me to you, saying, “Release my people that they may serve me in the desert, and, look, you have not listened until now.”
\verse Thus says Adonai, “By this you will know that I am Adonai. Look, I am about to strike with the staff that is in my hand the water that is in the Nile, and it will be changed to blood.
\verse And the fish that are in the Nile will die, and the Nile will stink, and the Egyptians will be unable to drink water from the Nile.” ’ ”
\verse And Adonai said to Moses, “Say to Aaron, ‘Take your staff and stretch your hand out over the waters of Egypt and over their rivers, over their canals, and over their pools and over all of their reservoirs of water, so that they become blood,’ and blood will be in all the land of Egypt and in vessels of wood and of stone.”
\verse And Moses and Aaron did so, as Adonai had commanded, and he raised the staff and struck the water that was in the Nile before the eyes of Pharaoh and before the eyes of his servants, and all of the water that was in the Nile was changed to blood.
\verse And the fish that were in the Nile died, and the Nile stank, and the Egyptians were not able to drink water from the Nile, and the blood was in all the land of Egypt.
\verse And the magicians of Egypt did likewise with their secret arts, and Pharaoh’s heart was hard, and he did not listen to them, as Adonai had spoken.
\verse And Pharaoh turned and went to his house, and \textit{he did not take also this to heart}.
\verse And all of the Egyptians dug around the Nile for water to drink, because they were unable to drink from the water of the Nile.
\verse And seven days passed after Adonai struck the Nile.
\end{biblechapter}

\begin{biblechapter} % Exodus 8
\verseWithHeading{Plague Two: Frogs}  And Adonai said to Moses, “Go to Pharaoh, and say to him, ‘Thus says Adonai, “Release my people so that they may serve me.”
\verse And if you are refusing to release, look, I am going to plague all of your territory with frogs.
\verse And the Nile will swarm with frogs, and they will go up and come into your house and into your \textit{bedroom} and onto your bed and into the house of your servants and among your people and into your ovens and into your kneading troughs.
\verse And the frogs will go up against you and against your people and against all of your servants.’ ”
\verse And Adonai said to Moses, “Say to Aaron, ‘Stretch out your hand with your staff over the rivers, over the canals, and over the pools, and bring up the frogs on the land of Egypt.’ ”
\verse And Aaron stretched out his hand over the waters of Egypt, and the frogs went up and covered the land of Egypt.
\verse And the magicians did likewise with their secret arts, and they brought up frogs over the land of Egypt.
\verse And Pharaoh called Moses and Aaron and said, “Pray to Adonai, and let him remove the frogs from me and from my people, and let me release the people so that they can sacrifice to Adonai.”
\verse And Moses said to Pharaoh, “\textit{I leave to you the honor} over me. When shall I pray for you and for your servants and for your people to cut off the frogs from you and from your houses? They will be left only in the Nile.”
\verse And he said, “Tomorrow.” And he said, “Let it be according to your word so that you will know that there is no one like Adonai our God.
\verse And the frogs will depart from you and from your house and from your servants. They will be left only in the Nile.”
\verse And Moses and Aaron went out from Pharaoh, and Moses cried out to Adonai over the matter of the frogs that he had brought on Pharaoh.
\verse And Adonai did according to the word of Moses, and the frogs died from the houses, from the courtyards, and from the fields.
\verse And they piled them in countless heaps, and the land stank.
\verse And Pharaoh saw that there was relief, and he made his heart \textit{insensitive}, and he did not listen to them, as Adonai had spoken.
\verseWithHeading{Plague Three: Gnats} And Adonai said to Moses, “Say to Aaron, ‘Stretch out your staff and strike the dust of the land, and it will become gnats in all the land of Egypt.’ ”
\verse And he did so, and Aaron stretched out his hand with his staff, and he struck the dust of the land, and it became gnats on the humans and on the animals; all of the dust of the land became gnats in all the land of Egypt.
\verse And the magicians did so with their secret arts to bring out the gnats, but they were not able, and the gnats were on the humans and on the animals.
\verse And the magicians said to Pharaoh, “It is the finger of God.” But the heart of Pharaoh was hard, and he did not listen to them, as Adonai had spoken.
\verseWithHeading{Plague Four: Flies} And Adonai said to Moses, “Start early in the morning and stand before Pharaoh. Look, he is going out to the water, and you must say to him, ‘Thus says Adonai, “Release my people so that they may serve me.”
\verse But if you are not about to release my people, look, I am about to send out flies among you and among your servants and among your people and in your houses; and the houses of Egypt will fill up with the flies and also the ground that they are on.
\verse But on that day I will distinguish the land of Goshen, where my people are staying, by there not being flies there, so that you will know that I am Adonai in the midst of the land.
\verse And I will put a distinction between my people and your people; this sign will be tomorrow.’ ”
\verse And Adonai did so, and a \textit{severe} swarm of flies came to the house of Pharaoh and the house of his servants and in all the land of Egypt; the land was ruined because of the flies.
\verse And Pharaoh called Moses and Aaron, and he said, “Go, sacrifice to your God in the land.”
\verse And Moses said, “To do so is not right, because we will sacrifice to Adonai our God a thing detestable to the Egyptians. Look, if we sacrifice before their eyes the thing detestable to the Egyptians, will they not stone us?
\verse We will go a journey of three days into the desert, and we will sacrifice to Adonai our God according to what he says to us.”
\verse And Pharaoh said, “I myself will release you, and you will sacrifice to Adonai your God in the desert. Only surely you must not go far. Pray for me.”
\verse And Moses said, “Look, I am going out from you, and I will pray to Adonai so that the flies depart from Pharaoh, from his servants, and from his people tomorrow. Only let not Pharaoh again deceive us by not releasing the people to sacrifice to Adonai.”
\verse And Moses went out from Pharaoh and prayed to Adonai.
\verse And Adonai did according to the word of Moses and removed the flies from Pharaoh, from his servants, and from his people; not one was left.
\verse And Pharaoh made his heart \textit{insensitive} also this time, and he did not release the people.
\end{biblechapter}

\begin{biblechapter} % Exodus 9
\verseWithHeading{Plague Five: Livestock Death} And Adonai said to Moses, “Go to Pharaoh and say to him, ‘Thus says Adonai, the God of the Hebrews, “Release my people so that they may serve me.”
\verse But if you are refusing to release and you still are keeping hold of them,
\verse look, the hand of Adonai is about to be present with a very \textit{severe} plague on your livestock that are in the field, on the horses, on the donkeys, on the camels, on the cattle, and on the sheep and goats.
\verse But Adonai will make a distinction between the livestock of Israel and the livestock of Egypt, and not a thing will die from all that belongs to the \textit{Israelites}.’ ”
\verse And Adonai set an appointed time, saying, “Tomorrow Adonai will do this thing in the land.”
\verse And Adonai did this thing the next day; all the livestock of Egypt died, but from the livestock of the \textit{Israelites} not one died.
\verse And Pharaoh sent to check, and \textit{it turned out} not even one from the livestock of Israel had died, but Pharaoh’s heart was \textit{insensitive}, and he did not release the people.
\verseWithHeading{Plague Six: Painful Sores} And Adonai said to Moses and to Aaron, “Take for yourselves full handfuls of soot from a smelting furnace, and let Moses sprinkle it toward the heavens before the eyes of Pharaoh.
\verse And it will become fine dust over all the land of Egypt, and it will become on humans and on animals a skin sore sprouting blisters in all the land of Egypt.”
\verse And they took the soot of the smelting furnace, and they stood before Pharaoh, and Moses sprinkled it toward the heavens, and it became skin sores sprouting blisters on humans and on animals.
\verse And the magicians were not able to stand before Moses because of the skin sores, for the skin sores were on the magicians and on all the Egyptians.
\verse And Adonai hardened Pharaoh’s heart, and he did not listen to them, as Adonai had spoken to Moses.
\verseWithHeading{Plague Seven: Hail} And Adonai said to Moses, “Start early in the morning and stand before Pharaoh. Look, he is going out to the water, and you must say to him, ‘Thus says Adonai, the God of the Hebrews, “Release my people so that they may serve me.
\verse For at this time I am sending all of my plagues \textit{to you personally} and among your servants and among your people so that you will know that there is no one like me in all the earth.
\verse For now I could have stretched out my hand, and I could have struck you and your people with the plague, and you would have perished from the earth.
\verse But for the sake of this I have caused you to stand—for the sake of showing you my strength and in order to proclaim my name in all the earth.
\verse Still you are behaving haughtily to my people by not releasing them.
\verse Look, about this time tomorrow, I am going to cause very severe hail to rain, the like of which has not been in Egypt from the day it was founded until now.
\verse And now send word; bring into safety your livestock and all that belongs to you in the field. The hail will come down on every human and animal that is found in the field and not gathered into the house, and they will die.” ’ ”
\verse Anyone from the servants of Pharaoh who feared the word of Adonai caused his servants and livestock to flee to the houses.
\verse But whoever did not \textit{give regard to} the word of Adonai abandoned his servants and his livestock in the field.
\verse And Adonai said to Moses, “Stretch out your hand to the heavens, and let there be hail in all the land of Egypt, on human and on animal and on all the vegetation of the field in the land of Egypt.”
\verse And Moses stretched out his staff to the heavens, and Adonai gave thunder and hail, and fire went to the earth, and Adonai caused hail to rain on the land of Egypt.
\verse And there was hail, and fire was flashing back and forth in the midst of the very severe hail, the like of which was not in all the land of Egypt since it had become a nation.
\verse And the hail struck in all the land of Egypt all that was in the field, from human to animal, and the hail struck all the vegetation of the field and smashed every tree of the field.
\verse Only in the land of Goshen, where the \textit{Israelites} were, there was no hail.
\verse And Pharaoh sent and called Moses and Aaron and said to them, “I have sinned this time. Adonai is the righteous one, and I and my people are the wicked ones.
\verse Pray to Adonai. The thunder of God and hail \textit{are enough}, and I will release you, and \textit{you will no longer have to stay}.”
\verse And Moses said to him, “At my leaving the city, I will spread out my hands to Adonai. The thunder will stop, and the hail will be no more, so that you will know that the earth belongs to Adonai.
\verse But as for you and your servants, I know that you do not yet fear the presence of Adonai.”
\verse And the flax and the barley were struck, because the barley was in the ear and the flax was in bud.
\verse But the wheat and the spelt were not struck, because they are late-ripening.
\verse And Moses went from Pharaoh out of the city, and he spread his hands to Adonai, and the thunder and the hail stopped, and rain did not pour on the earth.
\verse And Pharaoh saw that the rain and the hail and the thunder stopped, and \textit{he again sinned} and made his heart \textit{insensitive}, he and his servants.
\verse And Pharaoh’s heart was hard, and he did not release the \textit{Israelites}, as Adonai had said \textit{by the agency of Moses}.
\end{biblechapter}

\begin{biblechapter} % Exodus 10
\verseWithHeading{Plague Eight: Locusts} And Adonai said to Moses, “Go to Pharaoh, for I have made his heart \textit{insensitive} and the heart of his servants in order to put these signs of mine in his midst,
\verse so that you will tell in the ears of your child and \textit{your grandchild} that I dealt harshly with the Egyptians and so that you will tell about my signs that I have done among them, and so you will know that I am Adonai.”
\verse And Moses and Aaron came to Pharaoh, and they said to him, “Thus says Adonai, the God of the Hebrews, ‘Until when will you refuse to submit before me? Release my people so that they may serve me.
\verse But if you are refusing to release my people, look, I am about to bring locusts into your territory tomorrow.
\verse And they will cover the surface of the land, and no one will be able to see the land, and they will eat the remainder of what is left—what is left over for you from the hail—and they will eat every sprouting tree belonging to you from the field.
\verse And your houses will be full, and the houses of all your servants and the houses of all Egypt, something that your fathers and \textit{your grandfathers} never saw from the day they were on the earth until this day.’ ” And he turned and went out from Pharaoh.
\verse And the servants of Pharaoh said to him, “Until when will this be a snare for us? Release the men so that they may serve Adonai their God. Do you not yet know that Egypt is destroyed?”
\verse And Moses and Aaron were brought back to Pharaoh, and he said to them, “Serve Adonai your God. \textit{Who are the ones going}?
\verse And Moses said, “With our young and with our old we will go; with our sons and with our daughters, with our sheep and goats and with our cattle we will go because it is the feast of Adonai for us.”
\verse And he said to them, “Let Adonai be thus with you as soon as I release you and your dependents. See that evil is before your faces.
\verse \textit{No indeed}; just the men go and serve Adonai, since this is what you are seeking.” And he drove them out from the presence of Pharaoh.
\verse And Adonai said to Moses, “Stretch out your hand over the land of Egypt with the locusts so that they may come up over the land of Egypt, and let them eat all the vegetation of the land, all that the hail left behind.”
\verse And Moses stretched out his staff over the land of Egypt, and Adonai drove an east wind into the land all that day and all night. The morning came, and the east wind had brought the locusts.
\verse And the locusts went up over all the land of Egypt, and they settled in all the territory of Egypt, very \textit{severe}. Before it there were not locusts like them, nor will there be after it.
\verse And they covered the surface of all the land, and the land was dark with them, and they ate all the vegetation of the land and all the fruit of the trees that the hail had left, and no green was left in the trees nor in the vegetation of the field in all the land of Egypt.
\verse And Pharaoh hurried to call Moses and Aaron, and he said, “I have sinned against Adonai your God and against you.
\verse And now forgive my sin surely this time, and pray to Adonai your God so that he may only remove from me this death.”
\verse And he went out from Pharaoh, and he prayed to Adonai.
\verse And Adonai turned a very strong \textit{west wind} and lifted up the locusts and thrust them into the \textit{Red Sea}, and not one locust remained in all the territory of Egypt.
\verse And Adonai hardened Pharaoh’s heart, and he did not release the \textit{Israelites}.
\verseWithHeading{Plague Nine: Darkness} And Adonai said to Moses, “Stretch out your hand toward the heavens so that there may be darkness over the land of Egypt and so that a person can feel darkness.”
\verse And Moses stretched out his hand toward the heavens, and there was darkness of night in all the land of Egypt for three days.
\verse No one could see his brother, and \textit{because of it no one could move from where they were} for three days, but there was light for the \textit{Israelites} in their dwellings.
\verse And Pharaoh called Moses and said, “Go, serve Adonai. Only your sheep and goats and your cattle must be left behind. Your dependents may also go with you.”
\verse And Moses said, “Even if you yourself put into our hand sacrifices and burnt offerings and we offer them to Adonai our God,
\verse our livestock must also go with us. Not a hoof can be left because we must take from them to serve Adonai our God. And we will not know with what we are to serve Adonai until we come there.”
\verse And Adonai hardened Pharaoh’s heart, and he was not willing to release them.
\verse And Pharaoh said to him, “Go from me. \textit{Be careful} not to see my face again, because on the day of your seeing my face you will die.”
\verse And Moses said, “\textit{That is right}. \textit{I will not again see your face}.”
\end{biblechapter}

\begin{biblechapter} % Exodus 11
\verseWithHeading{Announcement of the Tenth Plague: Death of the Firstborn} And Adonai said to Moses, “Still one plague I will bring upon Pharaoh and upon Egypt; afterward he will release you from here. At the moment of his releasing, he will certainly drive you completely out from here.
\verse Speak in the ears of the people, and let them ask, a man from his neighbor and a woman from her neighbor, for objects of silver and objects of gold.”
\verse And Adonai gave the people favor in the eyes of Egypt. Also the man Moses was very great in the land of Egypt, in the eyes of the servants of Pharaoh and in the eyes of the people.
\verse And Moses said, “Thus says Adonai, ‘About the middle of the night I will go out through the midst of Egypt,
\verse and every firstborn in the land of Egypt will die, from the firstborn of Pharaoh who sits on his throne to the firstborn of the slave woman who is behind the pair of millstones and every firstborn animal.
\verse And there will be a great cry of distress in all the land of Egypt, the like of which has not been nor will be again.
\verse But against all the \textit{Israelites}, from a man to an animal, a dog will not even \textit{bark}, so that you will know that Adonai makes a distinction between Egypt and Israel.’
\verse And all of these your servants will come down to me and bow to me, saying, ‘Go out, you and all the people who are at your feet.’ And afterward I will go out.” And he went out from Pharaoh \textit{in great anger}.
\verse And Adonai said to Moses, “Pharaoh will not listen to you, \textit{so that my wonders may multiply} in the land of Egypt.”
\verse And Moses and Aaron did all these wonders before Pharaoh, and Adonai hardened Pharaoh’s heart, and he did not release the \textit{Israelites} from his land.
\end{biblechapter}

\begin{biblechapter} % Exodus 12
\verseWithHeading{Instructions for the Feast of Passover} And Adonai said to Moses and to Aaron in the land of Egypt, saying,
\verse “This month will be the beginning of months; it will be for you the first of the months of the year.
\verse Speak to all the community of Israel, saying, ‘On the tenth of this month, they will each take for themselves \textit{a lamb for the family}, a lamb for the household.
\verse And if the household is too small for a lamb, he and the neighbor nearest to his house will take one according to the number of persons; you will count out portions of the lamb \textit{according to how much each one can eat}.
\verse The lamb for you must be a male, without defect, in its first year; you will take it from the sheep or from the goats.
\verse “\textit{You will keep it} until the fourteenth day of this month, and all the assembly of the community of Israel will slaughter it \textit{at twilight}.
\verse And they will take some of the blood and put it on the two doorposts and on the lintel on the houses in which they eat it.
\verse And they will eat the meat on this night; they will eat it fire-roasted and with unleavened bread on \textit{bitter herbs}.
\verse You must not eat any of it raw or boiled, boiled in the water, but rather roasted with fire, its head with its legs and with its inner parts.
\verse And you must not leave any of it until morning; anything left from it until morning you must burn in the fire.
\verse And this is how you will eat it—with your waists fastened, your sandals on your feet, and your staff in your hand, and you will eat it in haste. It is Adonai’s Passover.
\verse “And I will go through the land of Egypt during this night, and I will strike all of the firstborn in the land of Egypt, from human to animal, and I will do punishments among all of the gods of Egypt. I am Adonai.
\verse And the blood will be a sign for you on the houses where you are, and I will see the blood, and I will pass over you, and there will not be a destructive plague among you when I strike the land of Egypt.
\verse “And this day will become a memorial for you, and you will celebrate it as a religious feast for Adonai throughout your generations; you will celebrate it as a lasting statute.
\verse You will eat unleavened bread for seven days. Surely on the first day you shall remove yeast from your houses, because anyone who eats food with yeast from the first day until the seventh day—that person will be cut off from Israel.
\verse It will be for you on the first day a holy assembly and on the seventh day a holy assembly; no work will be done on them; only what is eaten by every person, it alone will be prepared for you.
\verse “And you will keep the Feast of Unleavened Bread, because on this very day I brought out your divisions from the land of Egypt, and you will keep this day for your generations as a lasting statute.
\verse On the first day, on the fourteenth day of the month, in the evening, you will eat unleavened bread until the evening of the twenty-first day of the month.
\verse For seven days yeast must not be found in your houses, because \textit{anyone eating food with yeast} will be cut off from the community of Israel—whether an alien or a native of the land.
\verse You will eat no food with yeast; in all of your dwellings you will eat unleavened bread.”
\verse And Moses called all the elders of Israel, and he said to them, “Select and take for yourselves sheep for your clans and slaughter the Passover sacrifice.
\verse And take a bunch of hyssop and dip it into the blood that is in the basin and apply some of the blood that is in the basin to the lintel and the two doorposts. And you will not go out, anyone from the doorway of his house, until morning.
\verse And Adonai will go through to strike Egypt, and he will see the blood on the lintel and on the two doorposts, and Adonai will pass over the doorway and will not allow the destroyer to come to your houses to strike you.
\verse “And you will keep this event as a rule for you and for your children forever.
\verse \textit{And} when you come into the land that Adonai will give to you, as he said, you will keep this \textit{religious custom}.
\verse \textit{And} when your children say to you, ‘What is this \textit{religious custom} for you?’
\verse you will say, ‘It is a Passover sacrifice for Adonai, who passed over the houses of the \textit{Israelites} in Egypt when he struck Egypt; and he delivered our houses.’ ” And the people knelt down and they worshiped.
\verse And the \textit{Israelites} went, and they did as Adonai had commanded Moses and Aaron; so they did.
\verseWithHeading{Death of Firstborn and Deliverance from Egypt} \textit{And} in the middle of the night, Adonai struck all of the firstborn in the land of Egypt, from the firstborn of Pharaoh sitting on his throne to the firstborn of the captive who was in the prison house and every firstborn of an animal.
\verse And Pharaoh got up at night, he and all his servants and all Egypt, and a great cry of distress was in Egypt because there was not a house where there was no one dead.
\verse And he called Moses and Aaron at night, and he said, “Get up, go out from the midst of my people, both you as well as the \textit{Israelites}, and go, serve Adonai, as you have said.
\verse Take both your sheep and goats as well as your cattle, and go, and bless also me.”
\verse And the Egyptians urged the people in order to hurry their release from the land, because they said, “All of us will die!”
\verse And the people lifted up their dough before it had yeast; their kneading troughs were wrapped up in their cloaks on their shoulder.
\verse And the \textit{Israelites} did according to the word of Moses, and they asked from the Egyptians for objects of silver and objects of gold and for clothing.
\verse And Adonai gave the people favor in the eyes of the Egyptians, and they granted their requests, and they plundered the Egyptians.
\verse And the \textit{Israelites} set out from Rameses to Succoth; the men were about six hundred thousand on foot, besides dependents.
\verse And also a \textit{mixed multitude} went up with them and sheep and goats and cattle, very numerous livestock.
\verse And they baked the dough that they had brought out from Egypt as cakes, unleavened bread, because it had no yeast when they were driven out from Egypt, and they were not able to delay, and also they had not made provisions for themselves.
\verse And the period of dwelling of the \textit{Israelites} that they dwelled in Egypt was four hundred and thirty years.
\verse And at the end of four hundred and thirty years, on this exact day, all of Adonai’s divisions went out from the land of Egypt.
\verse It is a night of vigils belonging to Adonai for bringing them out from the land of Egypt; it is this night belonging to Adonai with vigils for all of the \textit{Israelites} throughout their generations.
\verse And Adonai said to Moses and Aaron, “This is the statute of the Passover: No foreigner may eat it.
\verse But any slave of a man, an acquisition by money, and you have circumcised him, then he may eat it.
\verse A temporary resident and a hired worker may not eat it.
\verse It will be eaten in one house; you will not bring part of the meat out from the house to the outside; and you will not break a bone of it.
\verse All of the community of Israel will prepare it.
\verse And when an alien dwells with you and he wants to prepare the Passover for Adonai, every male belonging to him must be circumcised, and then he may come near to prepare it, and he will be as the native of the land, but any uncircumcised man may not eat it.
\verse One law will be for the native and for the alien who is dwelling in your midst.”
\verse And all the \textit{Israelites} did as Adonai had commanded Moses and Aaron; so they did.
\verse And it was on exactly this day Adonai brought the \textit{Israelites} out from the land of Egypt by their divisions.
\end{biblechapter}

\begin{biblechapter} % Exodus 13
\verseWithHeading{Unleavened Bread and Dedication of Firstborn to Commemorate the Rescue from Egypt} And Adonai spoke to Moses, saying,
\verse “Consecrate to me every firstborn, the first offspring of every womb among the \textit{Israelites}, among humans and among domestic animals; \textit{it belongs to me}.”
\verse And Moses said to the people, “Remember this day when you went out from Egypt, from a house of slaves, because with strength of hand Adonai brought you out from here, and food with yeast will not be eaten.
\verse Today you are going out in the month of Abib.
\verse And when Adonai brings you to the land of the Canaanites and the Hittites and the Amorites and the Hivites and the Jebusites—which he swore to your ancestors to give to you, a land flowing with milk and honey—you will perform this service in this month.
\verse Seven days you will eat unleavened bread, and on the seventh day will be a feast for Adonai.
\verse Unleavened bread will be eaten the seven days; food with yeast will not be seen for you; and yeast will not be seen for you in all your territory.
\verse And you shall tell your son on that day, saying, ‘This is because of what Adonai did for me when I came out from Egypt.’
\verse And it will be as a sign on your hand and as a memorial between your eyes so that the law of Adonai will be in your mouth, that with a strong hand Adonai brought you out from Egypt.
\verse And you will keep this statute at its appointed time \textit{from year to year}.
\verse “And when Adonai brings you to the land of the Canaanite, as he swore to you and to your ancestors, and he gives it to you,
\verse you will hand over every first offspring of a womb to Adonai, and every first offspring dropped by a domestic animal that will belong to you, the males will be for Adonai.
\verse And every first offspring of a donkey you will redeem with small livestock, and if you will not redeem it, then you will break its neck, and every firstborn human among your sons you will redeem.
\verse And when your son asks you \textit{in the future}, saying, ‘What is this?’ you will say to him, ‘With strength of hand Adonai brought us out from Egypt, from a house of slaves.
\verse And when Pharaoh was stubborn to release us, Adonai killed every firstborn in the land of Egypt, from firstborn human to firstborn domestic animal. Therefore I am sacrificing to Adonai every first offspring of a womb, the males, and every firstborn of my sons I redeem.’
\verse And it will be as a sign on your hand and as symbolic ornaments between your eyes that with strength of hand Adonai brought us out from Egypt.”
\verseWithHeading{Summary of Travel} And when Pharaoh released the people, God did not lead them the way of the land of the Philistines, though it was nearer, because God said, “Lest the people change their mind when they see war and return to Egypt.”
\verse So God led the people around by the way of the desert to the \textit{Red Sea}, and the \textit{Israelites} went up in battle array from the land of Egypt.
\verse And Moses took the bones of Joseph with him because Joseph had made the \textit{Israelites} solemnly swear an oath, saying, “God will surely attend to you, and you will take up my bones from here with you.”
\verse And they set out from Succoth, and they encamped at Etham on the edge of the desert.
\verse And Adonai was going before them by day in a column of cloud to lead them on the way and by night in a column of fire to give light to them to go by day and night.
\verse The column of cloud by day and the column of fire by night did not depart from before the people.
\end{biblechapter}

\begin{biblechapter} % Exodus 14
\verseWithHeading{Adonai Rescues Israel at the Red Sea} And Adonai spoke to Moses, saying,
\verse “Speak to the \textit{Israelites} so that they turn and encamp before Pi-hahiroth, between Migdol \textit{and the sea}; before Baal Zephon, which is opposite it, you will camp by the sea.
\verse And Pharaoh will say of the \textit{Israelites}, ‘They are wandering around in the land. The desert has closed in on them.’
\verse And I will harden the heart of Pharaoh, and he will chase after them, and I will be glorified through Pharaoh and through all his army, and the Egyptians will know that I am Adonai.” And they did so.
\verse And it was told to the king of Egypt that the people fled, and the heart of Pharaoh was changed and that of his servants toward the people, and they said, “What is this we have done, that we have released Israel from serving us!”
\verse And he harnessed his chariot and took with him his people.
\verse And he took six hundred select chariots and all the chariots of Egypt and officers over all of them.
\verse And Adonai hardened the heart of Pharaoh the king of Egypt, and he chased after the \textit{Israelites}. (Now the \textit{Israelites} were going out \textit{boldly}.)
\verse And the Egyptians chased after them, and they overtook them encamped at the sea—all the horses of the chariots of Pharaoh and his charioteers and his army—at Pi-hahiroth before Baal Zephon.
\verse And Pharaoh approached, and the \textit{Israelites} lifted their eyes, and there were the Egyptians traveling after them! And they were very afraid, and the \textit{Israelites} cried out to Adonai.
\verse And they said to Moses, “Because there are no graves in Egypt? Is that why you have taken us to die in the desert? What is this you have done to us by bringing us out from Egypt!
\verse Isn’t this the word we spoke to you in Egypt, saying, ‘Leave us alone so that we can serve Egypt!’ because serving Egypt is better for us than our dying in the desert.”
\verse And Moses said to the people, “You must not be afraid. Stand still and see the salvation of Adonai, which he will accomplish for you today, because the Egyptians whom you see today you will see never again.
\verse Adonai will fight for you, and you must be quiet.”
\verse And Adonai said to Moses, “Why do you cry out to me? Speak to the \textit{Israelites} so that they set out.
\verse And you, lift up your staff and stretch out your hand over the sea and divide it so that the \textit{Israelites} can go in the middle of the sea on the dry land.
\verse And as for me, look, I am about to harden the heart of the Egyptians so that they come after them, and I will display my glory through Pharaoh and through all of his army, through his chariots and through his charioteers.
\verse And the Egyptians will know that I am Adonai when I display my glory through Pharaoh, through his chariots, and through his charioteers.”
\verse And the angel of God who was going before the camp of Israel set out and went behind them. And the column of cloud set out ahead of them, and it stood still behind them,
\verse so that it came between the camp of Egypt and the camp of Israel. And \textit{it was a dark cloud}, but it gave light to the night, so that \textit{neither approached the other} all night.
\verse And Moses stretched out his hand over the sea, and Adonai moved the sea with a strong east wind all night, and he made the sea become dry ground, and the waters were divided.
\verse And the \textit{Israelites} entered the middle of the sea on the dry land. The waters were a wall for them on their right and on their left.
\verse And the Egyptians gave chase and entered after them—all the horses of Pharaoh, his chariots, and his charioteers—into the middle of the sea.
\verse And during the morning watch, Adonai looked down to the Egyptian camp from in the column of fire and cloud, and he threw the Egyptian camp into a panic.
\verse And he removed the wheels of their chariots so that they drove them with difficulty, and the Egyptians said, “We must flee away from Israel because Adonai is fighting for them against Egypt.”
\verse And Adonai said to Moses, “Stretch out your hand over the sea, and let the waters return over the Egyptians, over their chariots, and over their charioteers.”
\verse And Moses stretched out his hand over the sea, and the sea returned \textit{at daybreak} to its normal level, and the Egyptians were fleeing \textit{because of it}, and Adonai swept the Egyptians into the middle of the sea.
\verse And the waters returned and covered the chariots and the charioteers—all the army of Pharaoh coming after them into the sea. Not \textit{even} one survived among them.
\verse But the \textit{Israelites} walked on the dry land in the middle of the sea. The waters were a wall for them on their right and on their left.
\verse And Adonai saved Israel on that day from the hand of Egypt, and Israel saw the Egyptians dead on the shore of the sea.
\verse And Israel saw the great hand that Adonai displayed against Egypt, and the people feared Adonai, and they believed in Adonai and in Moses his servant.
\end{biblechapter}

\begin{biblechapter} % Exodus 15
\verseWithHeading{Song of Victory at the Sea} Then Moses and the \textit{Israelites} sang this song to Adonai, \textit{and they said},
\verse “Let me sing to Adonai because he is highly exalted; 
the horse and its rider he hurled into the sea.
\verse Yah is my strength and song, and he has become my salvation; 
this is my God, and I will praise him—the God of my father—and I will exalt him.
\verse Adonai is a man of war; Adonai is his name.
\verse The chariots of Pharaoh and his army he cast into the sea, 
and his choice adjutants were sunk in the \textit{Red Sea}.
\verse The deep waters covered them; 
they went down into the depths like a stone.
\verse Adonai, your right hand is glorious in power; 
Adonai, your right hand destroyed the enemy.
\verse And in the greatness of your majesty you overthrew those standing up to you; 
you released your fierce anger, and it consumed them like stubble.
\verse And by the breath of your nostrils waters were piled up; 
waves stood like a heap; 
deep waters in the middle of the sea congealed.
\verse The enemy said, ‘I will pursue, I will overtake, I will divide plunder, 
my desire will be full of them, I will draw my sword, my hand will destroy them.’
\verse You blew with your breath; the sea covered them; 
they dropped like lead in the mighty waters.
\verse Who is like you among the gods, Adonai? 
Who is like you—glorious in holiness, awesome in praiseworthy actions, doing wonders?
\verse You stretched out your right hand; 
the earth swallowed them.
\verse In your loyal love you led the people whom you redeemed; 
in your strength you guided them to the abode of your holiness.
\verse Peoples heard; they trembled; 
anguish seized the inhabitants of Philistia.
\verse Then the chiefs of Edom were horrified; great distress seized the leaders of Moab; 
all of the inhabitants of Canaan melted away.
\verse Terror and dread fell on them; 
at the greatness of your arm they became silent like the stone, 
until your people passed by, Adonai, 
until the people whom you bought passed by.
\verse You brought them and planted them on the mountain of your inheritance, 
a place you made for yourself to inhabit, Adonai, 
a sanctuary, Lord, that your hands established.
\verse When the horses of Pharaoh came into the sea with his chariots and with his charioteers, Adonai brought back upon them the waters of the sea, and the \textit{Israelites} traveled on dry ground through the middle of the sea.
\verse And Miriam the prophetess, the sister of Aaron, took her tambourine in her hand, and all of the women went out after her with tambourines and with dances.
\verse And Miriam answered, “Sing to Adonai because he is highly exalted; the horse and its rider he hurled into the sea.”
\verseWithHeading{Adonai Provides Water at Marah} And Moses caused Israel to set out from the \textit{Red Sea}, and they went out into the desert of Shur, and they traveled three days in the desert, and they did not find water.
\verse And they came to Marah, and they were not able to drink water from Marah because it was bitter. Therefore \textit{it was named} Marah.
\verse And the people grumbled against Moses, saying, “What shall we drink?”
\verse And he cried out to Adonai, and Adonai showed him a piece of wood, and he threw it into the water, and the water became sweet. There he made a rule and regulation for them, and there he tested them.
\verse And he said, “If you carefully listen to the voice of Adonai your God and you do what is right in his eyes and give heed to his commands and you keep all his rules, then I will not bring about on you any of the diseases that I brought about on Egypt, because I am Adonai your healer.
\verse And they came to Elim, and twelve springs of water and seventy palm trees were there, and they encamped there at the water.
\end{biblechapter}

\begin{biblechapter} % Exodus 16
\verseWithHeading{Adonai Provides Food in the Desert} And they set out from Elim, and all the community of the \textit{Israelites} came to the desert of Sin, which is between Elim \textit{and Sinai}, in the fifteenth day of the second month of their going out from the land of Egypt.
\verse And all the community of the \textit{Israelites} grumbled against Moses and against Aaron in the desert.
\verse And the \textit{Israelites} said to them, “\textit{If only we had died} by the hand of Adonai in the land of Egypt, when we sat by the pots of meat, when we ate bread \textit{until we were full}, because you have brought us out to this desert to kill all of this assembly with hunger.”
\verse And Adonai said to Moses, “Look, I am going to rain down for you bread from the heavens, and the people will go out and gather enough for the day on its day; in that way I will test them: Will they go according to my law or not?
\verse And then on the sixth day, they will prepare what they bring, and it will be twice over what they will gather every other day.”
\verse And Moses and Aaron said to all the \textit{Israelites}, “In the evening, you will know that Adonai has brought you out from the land of Egypt,
\verse and in the morning, you will see the glory of Adonai, \textit{for he hears} your grumblings against Adonai, and what are we that you grumble against us?”
\verse And Moses said, “When in the evening Adonai gives you meat to eat and bread in the morning \textit{to fill up on}, \textit{for he hears} your grumblings that you grumble against him—and what are we? Your grumblings are not against us but against Adonai.”
\verse And Moses said to Aaron, “Say to all the community of the \textit{Israelites}, ‘Come near before Adonai because he has heard your grumblings.’ ”
\verse And at the moment of Aaron’s speaking to all the community of the \textit{Israelites}, they turned to the desert, \textit{and just then} the glory of Adonai appeared in the cloud.
\verse And Adonai spoke to Moses, saying,
\verse “I have heard the grumblings of the \textit{Israelites}. Speak to them, saying, ‘\textit{At twilight} you will eat meat, and in the morning you will be full with bread, and you will know that I am Adonai your God.’ ”
\verse And so it was, in the evening, the quail came up and covered the camp, and in the morning, a layer of dew was all around the camp.
\verse And the layer of dew came up, \textit{and there} on the face of the desert was a fine granular substance, fine like frost on the ground.
\verse And the \textit{Israelites} saw, and they said \textit{to each other}, “What is this?” because they did not know what it was. And Moses said to them, “That is the bread that Adonai has given to you as food.
\verse This is the word that Adonai commanded, ‘Gather from it, \textit{each according to what he can eat}, an omer per person according to the number of you. You each shall take enough for whoever is in his tent.’ ”
\verse And the \textit{Israelites} did so, and they gathered, some more and some less.
\verse And when they measured with the omer, the one gathering more had no surplus, and the one gathering less had no lack; they gathered \textit{each according to what he could eat}.
\verse And Moses said to them, “Let no one leave any of it until morning.”
\verse But they did not listen to Moses. Some people left some of it until morning, and it bred worms and stank. And Moses was angry with them.
\verse And they gathered it morning by morning, \textit{each according to what he could eat}, and it melted when the sun was hot.
\verse And when it was the sixth day, they gathered twice as much bread, two omers for one person, and all the leaders of the community came and told Moses.
\verse And he said to them, “This is what Adonai has said. Tomorrow is a rest period, a holy Sabbath for Adonai. Bake what you want to bake, and boil what you want to boil. Put aside all the surplus for yourselves for safekeeping until the morning.”
\verse And they put it aside until the morning, as Moses had commanded, and it did not make a stench, and not a maggot was in it.
\verse And Moses said, “Eat it today, because today is a Sabbath for Adonai. Today you will not find it in the field.
\verse Six days you will gather it, but on the seventh day, the Sabbath, it will not be present on it.”
\verse And on the seventh day some of the people went out to gather, and they did not find any.
\verse And Adonai said to Moses, “How long do you refuse to keep my commands and my laws?
\verse See, because Adonai has given to you the Sabbath, therefore he is giving to you on the sixth day bread for two days. Stay, \textit{each in his location}; let no one go from his place on the seventh day.”
\verse And the people rested on the seventh day.
\verse And the house of Israel called its name “manna.” And it was like coriander seed, white, and its taste was like a wafer with honey.
\verse And Moses said, “This is the word that Adonai has commanded. ‘A full omer of it is for safekeeping for your generations so that they will see the bread that I fed you in the desert when I brought you from the land of Egypt.’ ”
\verse And Moses said to Aaron, “Take one jar and put there a full omer of manna. Leave it before Adonai for safekeeping for your generations.”
\verse As Adonai had commanded Moses, so Aaron left it before the testimony for safekeeping.
\verse And the \textit{Israelites} ate the manna forty years, until their coming to an inhabited land; they ate the manna until their coming to the border of the land of Canaan.
\verse (And an omer is a tenth of an ephah.)
\end{biblechapter}

\begin{biblechapter} % Exodus 17
\verseWithHeading{Water from a Rock} And all the community of the \textit{Israelites} set out from the desert of Sin for their journeys according to the command of Adonai, and they camped in Rephidim, and there was no water for the people to drink.
\verse And the people quarreled with Moses, and they said, “Give us water so that we can drink.” And Moses said to them, “Why do you quarrel with me? Why do you test Adonai?”
\verse And the people thirsted for water, and the people grumbled against Moses and said, “Why \textit{ever} did you bring us up from Egypt to kill me and my sons and my cattle with thirst?”
\verse And Moses cried out to Adonai, saying, “What will I do with this people? A little longer and they will stone me.”
\verse And Adonai said to Moses, “Go on before the people and take with you some from the elders of Israel, and the staff with which you struck the Nile take in your hand, and go.
\verse Look, I will be standing before you there on the rock in Horeb, and you will strike the rock, and water will come out from it, and the people will drink.”
\verse And Moses did so before the eyes of the elders of Israel.
\verseWithHeading{Battle with the Amalekites} And Amalek came and fought with Israel at Rephidim.
\verse And Moses said to Joshua, “Choose men for us, and go out, fight against Amalek tomorrow. I will be standing on the top of the hill, and the staff of God will be in my hand.”
\verse And Joshua did as Moses had said to him to fight with Amalek. And Moses, Aaron, and Hur went up to the top of the hill.
\verse And when Moses raised his hand, Israel would prevail, but when he rested his hand, Amalek would prevail.
\verse But the hands of Moses were heavy, and they took a stone and placed it under him, and he sat on it; Aaron and Hur supported his hands, \textit{one on each side}, and his hands were steady until \textit{sundown}.
\verse And Joshua defeated Amalek and his people with the \textit{edge of the sword}.
\verse And Adonai said to Moses, “Write this as a memorial in the scroll and \textit{recite it in the hearing of} Joshua, because I will utterly blot out the remembrance of Amalek from under the heavens.”
\verse And Moses built an altar, and he called its name Adonai Is My Banner.
\verse And he said, “Because a hand was against the throne of Yah, a war will be for Adonai with Amalek from generation to generation.”
\end{biblechapter}

\begin{biblechapter} % Exodus 18
\verseWithHeading{Jethro’s Visit to Moses at the Mountain of God} And Jethro, the priest of Midian, the father-in-law of Moses, heard all that God had done for Moses and for Israel, his people, that Adonai had brought Israel out from Egypt.
\verse And Jethro, the father-in-law of Moses, took Zipporah the wife of Moses after her sending away,
\verse and her two sons—the one whose name was Gershom, for he had said, “I have been an alien in a foreign land,”
\verse and the one whose name was Eliezer, for “the God of my father was my help, and he delivered me from the sword of Pharaoh.”
\verse And Jethro, the father-in-law of Moses, came and his sons and his wife to Moses, to the desert where he was camping there at the mountain of God.
\verse And he said to Moses, “I, your father-in-law Jethro, am coming to you and your wife and her two sons with her.”
\verse And Moses went out to meet his father-in-law, and he bowed, and he kissed him, and \textit{they each asked about the other’s welfare}, and they came into the tent.
\verse And Moses told his father-in-law all that Adonai had done to Pharaoh and to Egypt on account of Israel, all the hardship that had found them on the way, and how Adonai delivered them.
\verse And Jethro rejoiced over all the good that Adonai had done for Israel when he delivered them from the hand of Egypt.
\verse And Jethro said, “Blessed be Adonai, who has delivered you from the hand of Egypt and from the hand of Pharaoh—who has delivered the people from under the hand of Egypt.
\verse Now I know that Adonai is greater than all the gods, \textit{even in the matter where they the Egyptians dealt arrogantly against the Israelites}.”
\verse And Jethro, the father-in-law of Moses, took a burnt offering and sacrifices for God, and Aaron and all the elders of Israel came to eat bread with the father-in-law of Moses before God.
\verse \textit{And} the next day, Moses sat to judge the people, and the people stood before Moses from the morning until the evening.
\verse And the father-in-law of Moses saw all that he was doing for the people, and he said, “What is this thing that you are doing for the people? Why are you sitting alone and all the people are standing by you from morning until evening?”
\verse And Moses said to his father-in-law, “Because the people come to me to seek God.
\verse When \textit{they have an issue}, it comes to me, and I judge between a man and his neighbor, and I make known God’s rule and his instructions.”
\verse And the father-in-law of Moses said to him, “The thing that you are doing is not good.
\verse Surely you will wear out, both you and this people who are with you, because the thing is too \textit{difficult} for you. You are not able to do it alone.
\verse Now listen to my voice; I will advise you, and may God be with you. You be for the people before God, and you bring the issues to God.
\verse And you warn them of the rules and the instructions, and you make known to them the way in which they must walk and the work that they must do.
\verse And you will select from all the people men of ability, fearers of God, trustworthy men, haters of dishonest gain, and you will appoint such men over them as commanders of thousands, commanders of hundreds, commanders of fifties, and commanders of tens.
\verse And let them judge the people all the time, \textit{and} every major issue they will bring to you, and every minor issue they will judge themselves. And so lighten it for yourself, and they will bear it with you.
\verse If you will do this thing and God will command you, then you will be able to endure, and also each of the people will go to his home in peace.”
\verse And Moses listened to the voice of his father-in-law, and he did all that he had said.
\verse And Moses chose men of ability from all Israel, and he appointed them as heads over the people, as commanders of thousands, commanders of hundreds, commanders of fifties, and commanders of tens.
\verse And they judged the people all the time; the difficult issues they would bring to Moses, and every minor issue they would judge themselves.
\verse And Moses let his father-in-law go, and he went to his land.
\end{biblechapter}

\begin{biblechapter} % Exodus 19
\verseWithHeading{Preparation for Receiving the Covenant at Mount Sinai} In the third month after the \textit{Israelites} went out from the land of Egypt, on this day they came to the Sinai desert.
\verse They set out from Rephidim, and they came to the desert of Sinai, and they camped in the desert, and Israel camped there in front of the mountain.
\verse And Moses went up to God, and Adonai called to him from the mountain, saying, “Thus you will say to the house of Jacob and you will tell the \textit{Israelites},
\verse ‘You yourselves have seen what I did to Egypt and how I bore you on eagles’ wings and I brought you to me.
\verse And now if you will carefully listen to my voice and keep my covenant, you will be a treasured possession for me out of all the peoples, \textit{for all the earth is mine},
\verse but you, you will belong to me as a kingdom of priests and a holy nation.’ These are the words that you will speak to the \textit{Israelites}.”
\verse And Moses came and called the elders of the people, and he placed before them all these words that Adonai had commanded him.
\verse And all the people together answered and said, “All that Adonai has spoken we will do.” And Moses brought back the words of the people to Adonai.
\verse And Adonai said to Moses, “Look, I am going to come to you in \textit{a thick cloud} in order that the people will hear when I speak with you and will also trust in you forever.” And Moses told the words of the people to Adonai.
\verse And Adonai said to Moses, “Go to the people and consecrate them today and tomorrow. They must wash their clothes,
\verse and they must be prepared for the third day, because on the third day, Adonai will go down on Mount Sinai before the eyes of all the people.
\verse And you must set limits for the people all around, saying, ‘Guard yourselves \textit{against} going up to the mountain and touching its edge. Anyone touching the mountain will certainly be put to death.
\verse Not a hand will touch it, because he will certainly be stoned or certainly be shot; whether an animal or a man, he will not live.’ At the blowing of the ram’s horn they may go up to the mountain.”
\verse And Moses went down from the mountain to the people, and he consecrated the people, and they washed their clothes.
\verse And he said to the people, “Be ready \textit{for the third day}. Do not go near to a woman.”
\verse \textit{And} on the third day, when it was morning, there was thunder and lightning, and a heavy cloud over the mountain and a very loud ram’s horn sound, and all the people who were in the camp trembled.
\verse And Moses brought the people out from the camp to meet God, and they took their stand at the foot of the mountain.
\verse And Mount Sinai was all wrapped in smoke because Adonai went down on it in the fire, and its smoke went up like the smoke of a smelting furnace, and the whole mountain trembled greatly.
\verse And the sound of the ram’s horn became \textit{louder and louder}, and Moses would speak, and God would answer him with a voice.
\verse And Adonai went down on Mount Sinai, to the top of the mountain, and Adonai called Moses to the top of the mountain, and Moses went up.
\verse And Adonai said to Moses, “Go down, warn the people, lest they break through to Adonai to see and many from them fall.
\verse And even the priests who come near Adonai must consecrate themselves, lest Adonai break out against them.”
\verse And Moses said to Adonai, “The people are not able to go up to Mount Sinai, because you yourself warned us, saying, ‘Set limits around the mountain and consecrate it.’ ”
\verse And Adonai said to him, “Go, go down, and come up, you and Aaron with you and the priests, but the people must not break through to go up to Adonai, lest he break out against them.”
\verse And Moses went down to the people, and he told them.
\end{biblechapter}

\begin{biblechapter} % Exodus 20
\verseWithHeading{Ten Commandments} And God spoke all these words, saying,
\verse “I am Adonai, your God, who brought you out from the land of Egypt, from the house of slaves.
\verse “There shall be for you no other gods before me.
\verse “You shall not make for yourself a divine image with any form that is in the heavens above or that is in the earth below or that is in the water below the earth.
\verse You will not bow down to them, and you will not serve them, because I am Adonai your God, a jealous God, punishing the guilt of the parents on the children on the third and on the fourth generations of those hating me,
\verse and showing loyal love to thousands of generations of those loving me and of those keeping my commandments.
\verse “You shall not \textit{misuse the name of Adonai your God}, because Adonai will not leave unpunished anyone who \textit{misuses his name}.
\verse “Remember the day of the Sabbath, to consecrate it.
\verse Six days you will work, and you will do all your work.
\verse But the seventh day is a Sabbath for Adonai your God; you will not do any work—you or your son or your daughter, your male slave or your female slave, or your animal, or your alien who is in your gates—
\verse because in six days Adonai made the heavens and the earth, the sea and all that is in them, and on the seventh day he rested. Therefore Adonai blessed the seventh day and consecrated it.
\verse “Honor your father and your mother, so that your days can be long on the land that Adonai your God is giving you.
\verse “You shall not murder.
\verse “You shall not commit adultery.
\verse “You shall not steal.
\verse “You shall not testify against your neighbor with a false witness.
\verse “You shall not covet the house of your neighbor; you will not covet the wife of your neighbor or his male servant or his female servant or his ox or his donkey or anything that is your neighbor’s.”
\verse And all the people were seeing the thunder and the lightning and the sound of the ram’s horn and the mountain smoking, and the people saw, and they trembled, and they stood at a distance.
\verse And they said to Moses, “You speak with us, and we will listen, but let not God speak with us, lest we die.”
\verse And Moses said to the people, “Do not be afraid. God has come to test you so that his fear will be before you so that you do not sin.”
\verse And the people stood at a distance, and Moses approached the very thick cloud where God was.
\verseWithHeading{Instructions for Building Altars} And Adonai said to Moses, “Thus you will say to the \textit{Israelites}, ‘You yourselves have seen that I have spoken to you from the heavens.
\verse You will not make alongside me gods of silver, and gods of gold you will not make for yourselves.
\verse An altar of earth you will make for me, and you will sacrifice on it your burnt offerings and your fellowship offerings, your sheep and your cattle. In every place where I cause my name to be remembered, I will come to you, and I will bless you.
\verse And if you make an altar of stones for me, you will not build them as hewn stone, because if you use your chisel on it, you have defiled it.
\verse You will not go up with steps onto my altar, that your nakedness not be exposed on it.’
\end{biblechapter}

\begin{biblechapter} % Exodus 21
\verseWithHeading{Regulations Regarding Hebrew Slaves} “And these are the regulations that you will set before them.
\verse ‘If you buy a Hebrew slave, he will serve six years, and in the seventh he will go out as free for nothing.
\verse If he comes in single, he will go out single. If he is the husband of a wife, his wife will go out with him.
\verse If his master gives him a wife and she bears for him sons or daughters, the wife and her children will belong to her master, and the slave will go out single.
\verse But if the slave explicitly says, “I love my master, my wife, and my children; I will not go out free,”
\verse his master will present him to God and bring him to the door or to the doorpost, and his master will pierce his ear with an awl, and he will serve him forever.
\verse “ ‘And if a man sells his daughter as a slave woman, she will not go out as male slaves go out.
\verse If \textit{she does not please her master} who selected her, he will allow her to be redeemed; he has no authority to sell her to foreign people, since he has dealt treacherously with her.
\verse And if he selects her for his son, he shall do for her according to the regulations for daughters.
\verse If he takes for himself another, he will not reduce her food, her clothing, or her right of cohabitation.
\verse And if he does not do for her these three, she shall go out for nothing; there will not be silver paid for her.
\verseWithHeading{Regulations Regarding Murder, Manslaughter, and Various Injuries} “ ‘\textit{Whoever strikes someone} and he dies will surely be put to death.
\verse But if he did not lie in wait \textit{and it was an accident}, I will appoint for you a place to which he may flee.
\verse But if a man schemes against his neighbor to kill him by treachery, you will take him from my altar to die.
\verse And \textit{whoever strikes} his father or his mother will surely be put to death.
\verse “ ‘And \textit{whoever kidnaps someone} and sells him, or he is found in his possession, he will surely be put to death.
\verse “ ‘And one who curses his father or his mother will surely be put to death.
\verse “ ‘And if men quarrel and a man strikes his neighbor with a stone or with a fist and he does not die, but \textit{he is confined to bed},
\verse if he stands and walks about in the outside on his staff, the striker will be unpunished; he will only pay for his inactivity \textit{toward his full recovery}.”
\verse And if a man strikes his male slave or his female slave with the rod and he dies under his hand, he will surely be avenged.
\verse Yet if he survives a day or two days, he will not be avenged, because he is his money.
\verse “ ‘And if men fight and they injure a pregnant woman, and her children go out and there is not serious injury, he will surely be fined as the woman’s husband demands concerning him \textit{and as the judges determine}.
\verse And if there is serious injury, you will give life in place of life,
\verse eye in place of eye, tooth in place of tooth, hand in place of hand, foot in place of foot,
\verse burn in place of burn, wound in place of wound, bruise in place of bruise.
\verse “ ‘And if a man strikes the eye of his male slave or the eye of his female slave and destroys it, he shall release him as free in place of his eye.
\verse And if he causes the tooth of his male slave or the tooth of his female slave to fall out, he will release him as free in place of his tooth.
\verse “ ‘And if an ox gores a man or a woman and he dies, the ox will surely be stoned, and its meat will not be eaten, and the owner of the ox is innocent.
\verse But if it was a goring ox \textit{before} and its owner was warned and did not restrain it and it kills a man or a woman, the ox will be stoned, and the owner also will be put to death.
\verse If a ransom is set on him, he will pay the redemption money for his life according to all that is set on him.
\verse If it gores a son or it gores a daughter, according to this regulation it shall be done to him.
\verse If the ox gores a male slave or a female slave, he will give thirty shekels of silver to his master, and the ox will be stoned.
\verse “ ‘If a man opens a pit or if a man digs a pit and he does not cover it and an ox or a donkey falls into it,
\verse the owner of the pit will pay restitution; he will pay silver to its owner, but the dead animal will be for him.
\verse And if a man’s ox injures the ox of his neighbor and it dies, they will sell the living ox and divide the \textit{money}, and they will also divide the dead one.
\verse Or if it was known that it was a goring ox \textit{before} and its owner did not restrain it, he will surely make restitution, an ox in place of the ox, and the dead one will be for him.
\end{biblechapter}

\begin{biblechapter} % Exodus 22
\verseWithHeading{Regulations Regarding Theft, Borrowing, and Accidental Damage or Loss}  “ ‘If a man steals an ox or small livestock and slaughters it or sells it, he will make restitution with five cattle in place of the ox and with four sheep or goats in place of the small livestock.
\verse “ ‘If a thief is found in the act of breaking in and he is struck and he dies, there is not bloodguilt for him.
\verse (If the sun has risen over him, there is bloodguilt for him. He will make full restitution. If \textit{he does not have enough}, he will be sold for his theft.
\verse If indeed the stolen item is found \textit{in his possession} alive, from ox to donkey to small livestock, he will make double restitution.
\verse “ ‘If a man grazes his livestock in a field or a vineyard and he releases his livestock and it grazes in the field of another, he will make restitution from the best of his field and the best of his vineyard.
\verse “ ‘If a fire is started and finds thorn bushes and a stack of sheaves or the standing grain or the field is consumed, the one who started the fire will surely make restitution.
\verse “ ‘If a man gives to his neighbor money or objects to watch over and it is stolen from the house of the man, if the thief is found, he will make double restitution.
\verse If the thief is not found, the owner of the house will be brought \textit{to the sanctuary} to learn whether or not he reached out his hand to his neighbor’s possession.
\verse Concerning every account of transgression—concerning an ox, concerning a donkey, concerning small livestock, concerning clothing, concerning all lost property—where someone says, “This belongs to me,” the matter of the two of them will come to God; whomever God declares guilty will make double restitution to his neighbor.
\verse “ ‘If a man gives to his neighbor a donkey or an ox or small livestock or any beast to watch over and it dies or is injured or is captured when there is no one who sees,
\verse the oath of Adonai will be between the two of them concerning whether or not he has reached out his hand to his neighbor’s possession, and its owner will accept this, and he will not make restitution.
\verse But if indeed it was stolen from him, he will make restitution to its owner.
\verse If indeed it was torn to pieces, he will bring it as evidence—the mangled carcass; he will not make restitution.
\verse “ ‘If a man borrows from his neighbor and it is injured or dies while its owner is not with it, he will make restitution.
\verse If its owner was with it, he will not make restitution; if it was hired, it came with its hiring fee.
\verseWithHeading{Regulations Regarding Various Offences} “ ‘If a man seduces a virgin who is not engaged and he lies with her, he surely will give her bride price \textit{to have her as his wife}.
\verse If her father absolutely refuses to give her to him, he will weigh out money according to the bride price for the virgin.
\verse “ ‘You will not let a witch live.
\verse “ ‘Anyone lying with an animal will surely be put to death.
\verse “ ‘Whoever sacrifices to the gods—not to Adonai, to him alone—will be destroyed.
\verseWithHeading{Regulations Regarding Foreigners and the Poor} “ ‘You will not mistreat an alien, and you will not oppress him, because you were aliens in the land of Egypt.
\verse “ ‘You will not afflict any widow or orphan.
\verse If you indeed afflict him, yes, if he cries out at all to me, I will certainly hear his cry of distress.
\verse And \textit{I will become angry}, and I will kill you with the sword, and your wives will be widows and your children orphans.
\verse “ ‘If you lend money to my people, to the needy with you, you will not be to him as a creditor; you will not \textit{charge him interest}.
\verse If indeed you require the cloak of your neighbor as a pledge, you will return it to him at sundown,
\verse because it is his only garment; it is his cloak for his skin. In what will he sleep? \textit{And} when he cries out to me, I will hear, because I am gracious.
\verseWithHeading{Regulations Regarding Tribute and Holiness} “ ‘You will not curse God, and you will not curse a leader among your people.
\verse “ ‘You will not delay the fullness of your harvest and the juice from your press; you will give me the firstborn of your sons.
\verse You will do likewise for your ox and for your sheep and goats; seven days it will be with its mother; on the eighth, you will give it to me.
\verse And you will be men of holiness for me; and you will not eat meat from a carcass mangled in the field; you will throw it to the dog.
\end{biblechapter}

\begin{biblechapter} % Exodus 23
\verseWithHeading{Regulations Regarding Justice} “ ‘You will not \textit{spread} a false report. Do not lift your hand with the wicked to be a malicious witness.
\verse You will not \textit{follow} a majority for evil, and you will not testify concerning a legal dispute to turn aside after a majority to pervert justice.
\verse You will not be partial to a powerless person in his legal dispute.
\verse “ ‘If you come upon the ox of your enemy or his donkey going astray, you will certainly bring it back to him.
\verse If you see the donkey of your enemy lying down under its burden, you will refrain from abandoning him. You will surely arrange it with him.
\verse “ ‘You will not pervert the justice of your poor in his legal dispute.
\verse You will stay far from a \textit{false charge}, and do not kill the innocent and the righteous, because I will not declare the wicked righteous.
\verse And you will not take a bribe, because the bribe makes the sighted blind and ruins the words of the righteous.
\verse And you will not oppress an alien; you yourselves know the feelings of the alien, because you were aliens in the land of Egypt.
\verseWithHeading{Regulations Regarding Work and Festivals} “ ‘And six years you will sow your land and gather its yield.
\verse But the seventh you will let it rest and leave it fallow, and the poor of your people will eat, and their remainder the animals of the field will eat. You will do likewise for your vineyard and for your olive trees.
\verse “ ‘Six days you will do your work, but on the seventh day you will stop so that your ox and your donkey will rest and the son of your slave woman and the alien will be refreshed.
\verse “ ‘And you will be attentive to all that I have said to you, and you will not \textit{profess} the name of other gods; it will not be heard in your mouth.
\verse “ ‘Three times in the year you will hold a festival for me.
\verse You will keep the Feast of Unleavened Bread; for seven days you will eat unleavened bread, as I commanded you at the appointed time, the month of Abib, because in it you came out from Egypt, and \textit{no one will} appear before me empty-handed.
\verse And you will keep the Feast of Harvest, with the firstfruits of your work, what you sow in the field. And you will keep the Feast of Harvest Gathering when the year goes out, when you gather your work from the field.
\verse Three times in the year all your men will appear before the Lord Adonai.
\verse “ ‘You will not sacrifice the blood of my sacrifice together with food with yeast, and you will not leave the fat of my feast overnight until morning.
\verse “ ‘The best of the firstfruits of your land you will bring to the house of Adonai your God. “ ‘You will not boil a young goat in its mother’s milk.
\verseWithHeading{Reasons for Loyal Obedience} “ ‘Look, I am about to send an angel before you to guard you on the way and to bring you to the place that I have prepared.
\verse Be attentive to him and listen to his voice; do not rebel against him, because he will not forgive your transgression, for my name is in him.
\verse But if you listen attentively to his voice and do all that I say, I will be an enemy to your enemies and a foe to your foes.
\verse When my angel goes before you and brings you to the Amorites and the Hittites and the Perizzites and the Canaanites and the Hivites and the Jebusites, I will wipe them out.
\verse “ ‘You will not bow to their gods, and you will not serve them, and you will not act according to their actions, because you will utterly demolish them, and you will utterly break their stone pillars.
\verse And you will serve Adonai your God, and he will bless your bread and your water, and I will remove sickness from among you.
\verse There will be no one suffering miscarriage or infertile in your land. I will make full the number of your days.
\verse “ ‘I will release my terror before you, and I will throw into confusion all the people against whom you come, and I will \textit{make all your enemies turn their back to you}.
\verse And I will send the hornet before you, and it will drive out the Hivites, the Canaanites, and the Hittites from before you.
\verse I will not drive them out from before you in one year, lest the land become a desolation and \textit{the wild animals} multiply against you.
\verse Little by little I will drive them out from before you until you are fruitful and take possession of the land.
\verse “ ‘And I will set your boundary from the \textit{Red Sea} and up to the sea of the Philistines and from the desert up to the river, because I will give the inhabitants of the land into your hand, and you will drive them out from before you.
\verse You will not make a covenant with them and with their gods.
\verse They will not live in your land, lest they cause you to sin against me when you serve their gods, for it will be a snare to you.’ ”
\end{biblechapter}

\begin{biblechapter} % Exodus 24
\verseWithHeading{Confirming the Covenant} And to Moses he said, “Go up to Adonai—you and Aaron, Nadab and Abihu, and seventy from the elders of Israel—and you will worship at a distance.
\verse And Moses \textit{alone} will come near to Adonai, and they will not come near, and the people will not go up with him.”
\verse And Moses came, and he told the people all the words of Adonai and all the regulations. And all the people answered with one voice, and they said, “All the words that Adonai has spoken we will do.”
\verse And Moses wrote all the words of Adonai, and he rose early in the morning, and he built an altar at the base of the mountain and set up twelve memorial stones for the twelve tribes of Israel.
\verse And he sent young men from the \textit{Israelites}, and they offered burnt offerings, and they sacrificed sacrifices as fellowship offerings to Adonai using bulls.
\verse And Moses took half of the blood, and he put it in bowls, and half of the blood he sprinkled on the altar.
\verse And he took the scroll of the covenant and read it in the hearing of the people, and they said, “All that Adonai has spoken we will do, and we will listen.”
\verse And Moses took the blood and sprinkled it on the people, and he said, “Look, the blood of the covenant that Adonai has made with you in accordance with all these words.”
\verse And Moses and Aaron, Nadab and Abihu, and seventy from the elders of Israel went up.
\verse And they saw the God of Israel, and what was under his feet was like sapphire tile work and like the very heavens for clearness.
\verse And toward the leaders of the \textit{Israelites} he did not stretch out his hand, and they beheld God, and they ate, and they drank.
\verseWithHeading{The Start of Forty Days and Nights on Mount Sinai} And Adonai said to Moses, “Come up to me on the mountain, and be there, and I will give you the tablets of stone and the law and the commandments that I have written to instruct them.”
\verse And Moses got up, and Joshua, his assistant, and Moses went up to the mountain of God.
\verse And to the elders he said, “Wait for us here until we return to you. And look, Aaron and Hur are with you. Whoever \textit{has a dispute} will bring it to you.”
\verse And Moses went up to the mountain, and the cloud covered the mountain.
\verse And the glory of Adonai settled on Mount Sinai, and the cloud covered it for six days, and he called to Moses on the seventh day from the midst of the cloud.
\verse And the appearance of the glory of Adonai was like a consuming fire on the top of the mountain to the eyes of the \textit{Israelites}.
\verse And Moses went into the midst of the cloud, and he went up the mountain, and Moses was on the mountain forty days and forty nights.
\end{biblechapter}

\begin{biblechapter} % Exodus 25
\verseWithHeading{Instruction to Collect Materials} And Adonai spoke to Moses, saying,
\verse “Speak to the \textit{Israelites}, and let them bring to me a contribution. You will receive my contribution from every man whose heart prompts him.
\verse And this is the contribution that you will receive from them—gold and silver and bronze,
\verse blue, purple, and crimson yarns, and fine linen and goat hair,
\verse and red-dyed ram skins, and fine leather, and acacia wood,
\verse oil for the lamp, balsam oils for the anointing oil and for the fragrant incense,
\verse onyx stones and stones for mountings on the ephod and the breast piece.
\verse And make a sanctuary for me, and I will dwell in the midst of them,
\verse according to all that I show you—the pattern of the tabernacle and the pattern of all its equipment—and so you will do.
\verseWithHeading{Instructions for Making the Ark of the Covenant} “And they will make an ark of acacia wood, two and a half cubits its length and a cubit and a half its width and a cubit and a half its height.
\verse And you will overlay it with pure gold, inside and outside you will overlay it, and you will make on it a gold molding all around.
\verse And you will cast for it four gold rings, and you will put them on its four feet, with two rings on its one side and two rings on its second side.
\verse And you will make poles of acacia wood, and you will overlay them with gold.
\verse And you will put the poles into the rings on the sides of the ark to carry the ark with them.
\verse In the rings of the ark will be the poles; they will not be removed from it.
\verse And you will put into the ark the testimony that I will give to you.
\verse “And you will make an atonement cover of pure gold, two and a half cubits its length and a cubit and a half its width.
\verse And you will make two cherubim of gold; you will make them of hammered work at the two ends of the atonement cover.
\verse And make one cherub \textit{at one end} and one cherub \textit{at the other end} of the atonement cover; you will make the cherubim on its two ends.
\verse And the cherubim will be with outspread wings above, covering with their wings over the atonement cover \textit{and facing each other}; the faces of the cherubim will be toward the atonement cover.
\verse And you will put the atonement cover above onto the ark, and into the ark you will put the testimony that I will give you.
\verse And I will meet you there, and I will speak with you from over the atonement cover, from between the two cherubim that are to be on the ark of the testimony—all that I will command you to the \textit{Israelites}.
\verseWithHeading{Instructions for Making a Table and a Lampstand} “And you will make a table of acacia wood, two cubits its length and a cubit its width and a cubit and a half its height.
\verse And you will overlay it with pure gold, and you will make for it a gold molding all around.
\verse And you will make for it a handbreadth rim all around, and you will make a gold molding for its rim all around.
\verse And you will make four gold rings for it, and you will put the rings on the four corners where its four \textit{legs} are.
\verse The rings will be near the rim as \textit{holders} for poles to carry the table.
\verse And you will make the poles of acacia wood and overlay them with gold, and the table will be carried with them.
\verse And you will make its plates and its ladles and its pitchers and its bowls with which libations will be poured; of pure gold you will make them.
\verse And you will put on the table the bread of presence to be before me continually.
\verse “And you will make a lampstand of pure gold; the lampstand will be made of hammered work—its base and its branch, its cups, its buds, and its blossoms will be from it.
\verse And six branches will be going out from its sides, three branches of the lampstand from its one side and three branches of the lampstand from its second side.
\verse Three almond-flower cups will be on the one branch with a bud and a blossom, and three almond-flower cups will be on the one branch with a bud and a blossom—likewise for the six branches going out from the lampstand.
\verse And on the lampstand will be four almond-flower cups, with its buds and its blossoms.
\verse And a bud will be under the two branches that come from it, and a bud under the two branches from it, and a bud under the two branches from it, likewise for the six branches coming out from the lampstand.
\verse Their buds and their branches will be from it, all of it one piece of pure gold hammered work.
\verse And you will make its seven lamps, and its lamps will be set up, and it will give light \textit{in the space in front of it}.
\verse And its snuffers and its fire pans will be pure gold.
\verse It will be made from a talent of pure gold, with all these pieces of equipment.
\verse And see and make all according to their pattern, which you were shown in the mountain.
\end{biblechapter}

\begin{biblechapter} % Exodus 26
\verseWithHeading{Instructions for Making the Tabernacle} “And the tabernacle you will make with ten curtains; you will make them of finely twisted linen and blue and purple and crimson yarns, with cherubim, the work of a skilled craftsman.
\verse The length of the one curtain will be twenty-eight cubits, and the width will be four cubits for the one curtain; one measure will be for all the curtains.
\verse Five curtains will be joined \textit{to one another}, and five curtains joined \textit{to one another}.
\verse And you will make loops of blue on the edge of the one curtain, at the end in the set; and you will do so on the edge of the end curtain in the second set.
\verse You will make fifty loops on the one curtain, and you will make fifty loops on the end of the curtain that is in the second set; the loops are to be opposite \textit{to one another}.
\verse And you will make fifty gold clasps and join the curtains \textit{to one another} with the clasps, so that the tabernacle will be one.
\verse “And you will make curtains of goat hair for a tent over the tabernacle; you will make them eleven curtains.
\verse The length of the one curtain will be thirty cubits, and the width will be four cubits for the one curtain; one measure will be for the eleven curtains.
\verse And you will join five curtains together and six curtains together, and you will fold double the sixth curtain at the front of the tent.
\verse And you will make fifty loops on the edge of the one curtain at the end of the first set and fifty loops on the edge of the curtain in the second set.
\verse And you will make fifty bronze clasps, and you will put the clasps in the loops and join the tent, so that it will be one.
\verse “And the surplus in the curtains of the tent will be an overhang; the surplus half curtain will hang over the back of the tabernacle.
\verse And a cubit \textit{from one side} and a cubit \textit{from the other side} in the surplus in the length of the curtains of the tent will be hung over the sides of the tabernacle \textit{equally} to cover it.
\verse “And you will make a covering for the tent of red-dyed ram skins and a covering of fine leather to go above.
\verse “And you will make the frames for the tabernacle of acacia wood as uprights.
\verse The length of the frame will be ten cubits, and the width of the one frame will be one and a half cubits.
\verse You will make two \textit{pegs} for the one frame for joining \textit{each to another} and likewise for all the frames of the tabernacle.
\verse And you will make the frames for the tabernacle with twenty frames for the \textit{south} side.
\verse And you will make forty silver bases under the twenty frames, with two bases under the one frame for its two \textit{pegs} and two bases under the \textit{next} frame for its two \textit{pegs}.
\verse And for the second side of the tabernacle, the north side, there will be twenty frames
\verse and their forty silver bases, with two bases under the one frame and two bases under the \textit{next} frame.
\verse “And for the rear of the tabernacle \textit{on the west} you will make six frames.
\verse And you will make two frames for the tabernacle corners at the rear.
\verse They will be double at the bottom, and they will be completely together on its top to the one ring; it will be likewise for the two of them; they will be for the two corners.
\verse And there will be eight frames and their silver bases, sixteen bases, with two bases under the one frame and two bases under the \textit{next} frame.
\verse “You will make five bars of acacia wood for the frames on the one side of the tabernacle,
\verse and five bars for the frames on the second side of the tabernacle, and five bars for the frames on the side of the tabernacle at the rear \textit{on the west}.
\verse And the bar in the middle, in the midst of the frames will run from end to end.
\verse And you will overlay the frames with gold, and you will make their rings of gold as \textit{holders} for the bars, and you will overlay the bars with gold.
\verse And you will erect the tabernacle according to its plan, which you have been shown on the mountain.
\verse “And you will make a curtain of blue and purple and crimson yarns and finely twisted linen, the work of a skilled craftsman; he will make it with cherubim.
\verse And you will put it on four acacia pillars overlaid with gold with their gold hooks on four silver bases.
\verse And you will put the curtain under the clasps, and you will bring the ark of the testimony there inside the curtain, and the curtain will separate for you between the holy and the \textit{most holy place}.
\verse And you will put the atonement cover on the ark of the testimony in the \textit{most holy place}.
\verse And you will place the table outside the curtain and the lampstand opposite the table on the south side of the tabernacle, and you will put the table on the north side.
\verse “And you will make for the entrance of the tent a screen of blue and purple and crimson yarns and finely twisted linen, the work of an embroiderer.
\verse And you will make for the screen five acacia pillars, and you will overlay them with gold with their gold hooks, and you will cast for them five bronze bases.
\end{biblechapter}

\begin{biblechapter} % Exodus 27
\verseWithHeading{Instructions for Making the Bronze Altar} “And you will make the altar of acacia wood, five cubits long and five cubits wide; the altar will be square, and its height will be three cubits.
\verse And you will make its horns on its four corners; its horns will be \textit{of one piece with it}, and you will overlay it with bronze.
\verse And you will make its pots for removing its fat-soaked ashes and its shovels and its sprinkling bowls and its forks and its fire pans; you will make all its equipment with bronze.
\verse And you will make for it a grating, a work of bronze network, and you will make on the network four bronze rings on its four ends.
\verse And you will put it under the ledge of the altar, below, and the network will be up to the middle of the altar.
\verse And you will make poles for the altar, poles of acacia wood, and you will overlay them with bronze.
\verse And the poles will be put into the rings, and the poles will be on the two sides of the altar when carrying it.
\verse You will make it hollow with boards. As it was shown you on the mountain, so they will do.
\verseWithHeading{Instructions for Making the Courtyard} “You will make the courtyard of the tabernacle; for the \textit{south} side will be hangings for the courtyard of finely twisted linen, one hundred cubits long for the one side.
\verse And its twenty pillars and their twenty bases will be bronze; the hooks of the pillars and their bands will be silver.
\verse And likewise for the north side along the length will be hangings one hundred cubits long; and its twenty pillars and their bases will be bronze; the hooks of the pillars and their bands will be silver.
\verse And the width of the courtyard for the west side will be hangings of fifty cubits, their ten pillars and their ten bases.
\verse And the width of the courtyard for the east side, toward sunrise, will be fifty cubits.
\verse And hangings for the shoulder will be fifteen cubits with their three pillars and their three bases.
\verse And fifteen cubits of hangings will be for the second shoulder with their three pillars and their three bases.
\verse And for the gate of the courtyard there will be a screen of twenty cubits of blue and purple and crimson yarns and finely twisted linen, the work of an embroiderer; with their four pillars and their four bases.
\verse “All the pillars of the courtyard all around will be banded with silver, and their hooks will be silver, and their bases will be bronze.
\verse The length of the courtyard will be one hundred cubits and the width fifty cubits and the height five cubits, of finely twisted linen, with their bronze bases.
\verse Bronze will be for all the equipment of the tabernacle in all its service and all its \textit{pegs} and all the \textit{pegs} of the courtyard.
\verseWithHeading{Instructions for Making Oil for the Lampstand} “And you will command the \textit{Israelites}, and they will bring to you pure, beaten olive oil for the light, to cause a lamp to burn continually.
\verse In the tent of assembly outside the curtain that is before the testimony, Aaron and his sons will arrange it, from evening until morning, before Adonai as a lasting statute throughout their generations from the \textit{Israelites}.
\end{biblechapter}

\begin{biblechapter} % Exodus 28
\verseWithHeading{Instructions for Making Garments for Priests} “And bring near to you Aaron, your brother, and his sons with him from the midst of the \textit{Israelites} to serve as priests for me—Aaron, Nadab and Abihu, Eleazar and Ithamar, the sons of Aaron.
\verse And you will make holy garments for Aaron, your brother, for glory and for splendor.
\verse And you will speak to all the skilled of heart, whom I have given \textit{a gift of skill}, and they will make the garments of Aaron to consecrate him for his serving as my priest.
\verse And these are the garments that they will make: A breast piece and an ephod and a robe and a tunic of specially woven fabric, a turban and a sash. And they will make holy garments for Aaron your brother, and for his sons to serve as priests for me.
\verse “And they will take the gold and the blue and the purple and the crimson yarns and the fine linen,
\verse and they will make the ephod of gold, blue and purple, and crimson yarns, and finely twisted linen, the work of a skilled craftsman.
\verse It will have two joining shoulder pieces at its two edges, so that it can be fastened.
\verse And the waistband of his ephod, which is on it, will be of like work to it—gold, blue, and purple and crimson yarns and finely twisted linen.
\verse “And you will take two onyx stones and engrave on them the names of the \textit{Israelites},
\verse with six of their names on the one stone and the remaining six on the second, according to their genealogies.
\verse As the work of a skilled stone craftsman, with seal engravings you will engrave on the two stones the names of the \textit{Israelites}; you will make them mounted in gold filigree settings.
\verse And you will set the two stones on the ephod’s shoulder pieces as stones of remembrance for the \textit{Israelites}, and Aaron will bear their names before Adonai on his two shoulder pieces for remembrance.
\verse “And you will make gold filigree settings.
\verse And you will make two braided chains of pure gold ornamental cord work, and you will put the chains of the ornamental cords on the filigree settings.
\verse And you will make a breast piece of judgment, a work of a skilled craftsman; you will make it like the work of the ephod; you will make it of gold, blue and purple and crimson yarns, and finely twisted linen.
\verse It will be squared, doubled, a span its length and a span its width.
\verse And you will fill it with stone mounting, four rows of stone, a row of carnelian, topaz, and emerald is the first row;
\verse and the second row is a malachite, a sapphire, and a moonstone;
\verse and the third row is a jacinth, an agate, and an amethyst;
\verse and the fourth row is a turquoise and an onyx and a jasper. Their settings will be woven with gold.
\verse The stones will be according to the names of the \textit{Israelites}, twelve according to their names, with seal engravings, each according to its name they will be for the twelve tribes.
\verse “And you will make on the breast piece braided chains, a work of pure gold ornamental cord.
\verse And you will make on the breast piece two gold rings, and you will put the two rings on the two edges of the breast piece.
\verse And you will put the two gold ornamental cords on the two rings on the edges of the breast piece.
\verse And you will put the two ends of the two ornamental cords on the two filigree settings, and you will put them on the shoulder pieces of the ephod at the front of it.
\verse And you will make two gold rings, and you will place them on the two ends of the breast piece, on its edge that is \textit{on the other side} of the ephod, \textit{to the inside}.
\verse And you will make two rings and put them on the two shoulder pieces of the ephod below at its front near its seam above the waistband of the ephod.
\verse And they will tie the breast piece by its rings to the rings of the ephod with a blue cord to be on the waistband of the ephod, and the breast piece will not come loose from the ephod.
\verse And Aaron will bear the names of the \textit{Israelites} in the breast piece of judgment on his heart, when he comes to the sanctuary, for a remembrance before Adonai continually.
\verse And you will put the Urim and the Thummim on the breast piece of judgment, and they will be on the heart of Aaron when he comes before Adonai, and Aaron will bear the judgment of the \textit{Israelites} on his heart before Adonai continually.
\verse “And you will make the robe of the ephod totally of blue yarn.
\verse And the opening for his head will be in the middle of it; its opening will have an edge all around, the work of a weaver; it will be like the opening of a sturdy garment for it, so that it will not be torn.
\verse And you will make on its hem pomegranates of blue and purple and crimson yarns on its hem all around and bells of gold in the midst of them all around,
\verse a gold bell and a pomegranate, a gold bell and a pomegranate, on the hem of the robe all around.
\verse And it will be on Aaron for serving, and its sound will be heard at his coming into the sanctuary before Adonai and at his going out, so that he will not die.
\verse “And you will make a pure gold rosette, and you will engrave on it with seal engravings: “A holy object for Adonai.”
\verse And you will place it on a blue cord, and it will be on the turban, at the front of the turban it will be.
\verse And it will be on the forehead of Aaron, and Aaron will bear the guilt of the holy objects that the \textit{Israelites} will consecrate for all their holy gifts, and it will be on his forehead continually for acceptance for them before Adonai.
\verse “And you will weave the tunic of fine linen, and you will make a turban of fine linen, and you will make a sash, the work of an embroiderer.
\verse And for the sons of Aaron you will make tunics, and you will make for them sashes and headdresses; you will make them for glory and for splendor.
\verse And you will clothe them—Aaron, your brother, and his sons with him—and you will anoint them and \textit{ordain them} and consecrate them, and they will serve as priests for me.
\verse And make for them undergarments of linen to cover naked flesh; they will be from loins to thigh.
\verse And they will be on Aaron and on his sons when they come to the tent of assembly or when they approach the altar to serve in the sanctuary, so that they will not bear guilt and die. It is a lasting statute for him and for his offspring after him.
\end{biblechapter}

\begin{biblechapter} % Exodus 29
\verseWithHeading{Instructions for Consecrating Aaron and His Sons} “And this is the thing that you will do for them to consecrate them to serve as a priest for me: Take \textit{one young bull} and two rams without defect
\verse and unleavened bread and unleavened, ring-shaped bread cakes mixed with oil, and wafers of unleavened breads smeared with oil. You will make them with finely milled wheat flour,
\verse and you will put them on one basket, and you will bring them on the basket and bring the bull and the two rams.
\verse And you will bring Aaron and his sons to the entrance of the tent of assembly, and you will wash them with water.
\verse And you will take the garments and clothe Aaron with the tunic and the robe of the ephod, and you will fasten to him the ephod and the breast piece with the waistband of the ephod.
\verse And you will set the turban on his head, and you will put the holy diadem on the turban.
\verse And you will take the anointing oil and pour it on his head and anoint him.
\verse And you will bring his sons and clothe them with tunics.
\verse And you will gird Aaron and his sons with sashes and wrap headdresses on them. And priesthood will be theirs as a lasting rule, and \textit{you will ordain Aaron and his sons.}
\verse “And you will bring the bull before the tent of assembly, and Aaron and his sons will lay their hands on the head of the bull.
\verse And you will slaughter the bull before Adonai at the entrance of the tent of assembly.
\verse And you will take some of the blood of the bull and with your finger put it on the horns of the altar, and you will pour out all the blood at the base of the altar.
\verse And you will take and turn into smoke on the altar all the fat covering the inner parts and the lobe on the liver and the two kidneys and the fat that is on them.
\verse And the flesh of the bull and its skin and its offal you will burn with fire outside the camp; it is a sin offering.
\verse “And you will take the one ram, and Aaron and his sons will lay their hands on the head of the ram.
\verse And you will slaughter the ram and take its blood and sprinkle it on the altar all around.
\verse And you will cut the ram into pieces and wash its inner parts, and you will put its legs with its pieces and with its head.
\verse And you will turn into smoke on the altar all of the ram; it is a burnt offering for Adonai; it is a smell of appeasement, an offering by fire for Adonai.
\verse “And you will take the second ram, and Aaron and his sons will lay their hands on the head of the ram.
\verse And you will slaughter the ram and take some of its blood and put it on Aaron’s right earlobe and on the right earlobe of his sons and on the thumb of their right hand and on \textit{the big toe} of their right foot, and you will sprinkle the blood at the base of the altar all around.
\verse And you will take some of the blood that is on the altar and some of the anointing oil, and you will spatter it on Aaron and on his garments and on his sons and on his sons’ garments with him, and he will be sacred, and his garments and his sons and his sons’ garments with him.
\verse And you will take from the ram the fat and the fat tail and the fat covering the inner parts and the lobe of the liver and the two kidneys and the fat that is on them and the right thigh, because it is the ram of ordination.
\verse “And one loaf of bread and one ring-shaped bread cake of oiled bread and one wafer from the basket of unleavened bread that is before Adonai—
\verse you will put them all on the palms of Aaron and on the palms of his sons, and you will wave them as a wave offering before Adonai.
\verse And you will take them from their hand and turn them to smoke on the altar beside the burnt offering as a fragrance of appeasement before Adonai; it is an offering made by fire before Adonai.
\verse “And you will take the breast section from the ram of ordination that is for Aaron, and you will wave it as a wave offering before Adonai. It will be your portion.
\verse And you will consecrate the wave offering breast section and the thigh of the contribution that was waved and that was presented from the ram of the ordination that is for Aaron and for his sons.
\verse And it will be for Aaron and for his sons as a lasting rule from the \textit{Israelites}, because it is a contribution, and it will be a contribution from the \textit{Israelites} from their sacrifices of fellowship, their contribution to Adonai.
\verse “And the holy garments that are for Aaron will be for his sons after him in which to anoint them and \textit{to ordain them}.
\verse Seven days the priest who replaces him from among his sons will wear them, who comes to the tent of assembly to serve in the sanctuary.
\verse And you will take the ram of ordination and boil its meat in a holy place.
\verse And Aaron and his sons will eat the meat of the ram and the bread that is in the basket at the entrance of the tent of assembly.
\verse And they will eat them—the things by which atonement was made for them to ordain them to consecrate them—and a stranger will not eat them because they are holy objects.
\verse If any remains until morning from the ordination meat or from the bread, you will burn the remainder in fire; it will not be eaten, because it is a holy object.
\verse And you will do so for Aaron and for his sons, according to all that I have commanded you; seven days you will ordain them.
\verseWithHeading{Instructions for Regular Offerings at the Tabernacle} “And you will offer a bull for a sin offering every day for the atonement; and you will offer a sin offering on the altar when you make atonement for it, and you will anoint it to consecrate it.
\verse Seven days you will make atonement for the altar, and you will consecrate it, and the altar will be \textit{a most holy thing}. \textit{Anyone who} touches the altar will be holy.
\verse “And this is what you shall offer on the altar: Two \textit{one-year-old} male lambs \textit{every day} continually.
\verse The first lamb you will offer in the morning, and the second lamb you will offer \textit{at twilight}.
\verse And a tenth of finely milled flour mixed with a fourth of a hin of beaten oil, and a fourth of a hin of wine as a libation with the first lamb.
\verse And the second lamb you will offer \textit{at twilight}; you will offer a grain offering and its libation like that of the morning for a fragrance of appeasement, an offering made by fire for Adonai.
\verse It will be a burnt offering of continuity throughout your generations at the entrance of the tent of assembly before Adonai, where I will meet with you to speak to you there.
\verse “And I will meet with the \textit{Israelites} there, and it will be consecrated by my glory.
\verse And I will consecrate the tent of assembly and the altar, and Aaron and his sons I will consecrate to serve as priests for me.
\verse And I will dwell in the midst of the \textit{Israelites}, and I will be their God.
\verse And they will know that I am Adonai, their God, who brought them out from the land of Egypt in order to dwell in their midst. I am Adonai their God.
\end{biblechapter}

\begin{biblechapter} % Exodus 30
\verseWithHeading{Instructions for Making the Incense Altar} “And you will make an altar for burning incense; you will make it of acacia wood,
\verse a cubit its length and a cubit its width—it will be square—and two cubits its height, its horns \textit{of one piece with it}.
\verse And you will overlay it with pure gold, its top and its sides all around and its horns, and you will make for it a gold molding all around.
\verse And you will make two gold rings for it; under its molding \textit{on two opposite sides} you will make them as \textit{holders} for poles to carry it with them.
\verse You will make the poles of acacia wood, and you will overlay them with gold.
\verse And you will put it before the curtain that is upon the ark of the testimony, before the atonement cover, which is on the testimony, there where I will meet with you.
\verse “And on it Aaron will turn fragrant incense into smoke; \textit{each morning} when he tends the lamps, he will turn it into smoke.
\verse And when Aaron sets up the lamps \textit{at twilight}, he will turn it into smoke—incense of continuity—before Adonai throughout your generations.
\verse You will not offer on it strange incense or a burnt offering or a grain offering, and you will not pour a libation on it.
\verse And Aaron will make atonement on its horns one time in the year from the blood of the sin offering of the atonement; one time in the year he will make atonement on it throughout your generations; it is a most holy thing for Adonai.”
\verseWithHeading{Instructions for Numbering the People} And Adonai spoke to Moses, saying,
\verse “\textit{When you take a census of} the \textit{Israelites} to count them, they will each give the ransom of his life for Adonai when counting them, and a plague will not be among them when counting them.
\verse This they will give, \textit{everyone who is counted}, the half shekel, according to the sanctuary shekel, which is twenty gerahs per shekel. The half shekel is a contribution for Adonai.
\verse \textit{Everyone who is counted} from \textit{twenty years old} and above will give the contribution of Adonai.
\verse The rich will not give more, and the poor will not give less than the half shekel to give the contribution of Adonai to make atonement for their lives.
\verse And you will take the atonement money from the \textit{Israelites} and give it to the service of the tent of assembly, and it will be as a memorial for the \textit{Israelites} before Adonai to make atonement for your lives.”
\verseWithHeading{Instructions for Making the Basin} And Adonai spoke to Moses, saying,
\verse “And you will make a basin of bronze and its bronze stand for washing, and you will put it between the tent of assembly and the altar, and you will put water there.
\verse And Aaron and his sons will wash their hands and their feet with it.
\verse When they come to the tent of assembly, they will wash with water so that they do not die, or when they approach the altar to serve by turning to smoke an offering made by fire to Adonai.
\verse And they will wash their hands and their feet so that they do not die, and it will be a lasting rule for them—to him and to his offspring throughout their generations.
\verseWithHeading{Instructions for Making Anointing Oil and Incense} And Adonai spoke to Moses, saying,
\verse “And take for yourself top quality balsam oils, five hundred shekels of flowing myrrh, half as much—two hundred and fifty shekels of fragrant cinnamon, and two hundred and fifty shekels of fragrant reed,
\verse and five hundred shekels of cassia, according to the sanctuary shekel, and a hin of olive oil.
\verse And you will make it into holy anointing oil, a spice blend of a fragrant ointment the work of a perfumer; it will be holy anointing oil.
\verse And you will anoint with it the tent of assembly and the ark of the testimony,
\verse and the table and all its equipment and the lampstand and its equipment and the incense altar,
\verse and the altar of burnt offering and all its equipment and the basin and its stand.
\verse And you will consecrate them, and they will be most holy things; \textit{anyone who} touches them will be holy.
\verse And you will anoint Aaron and his sons, and you will consecrate them to serve as priests for me.
\verse “And you will speak to the \textit{Israelites}, saying, ‘This will be my holy anointing oil throughout your generations.
\verse It will not be poured on human flesh, and with its measurements you will not make any like it; it is holy; it will be holy to you.
\verse Anyone who compounds perfume like it and who puts it on a stranger will be cut off from his people.’ ”
\verse And Adonai said to Moses, “Take for yourself fragrant perfumes—stacte resin and onycha and galbanum—fragrant perfumes and pure frankincense, \textit{an equal part of each},
\verse and make it into a compound of incense, the work of a perfumer, salted, pure, holy.
\verse And you will grind part of it to powder, and you will put part of it before the testimony in the tent of assembly where I will meet with you; it will be a most holy thing to you.
\verse And the incense that you will make with its measurements you will not make for yourselves; it will be holy to you for Adonai.
\verse Anyone who makes any like it to smell it will be cut off from his people.”
\end{biblechapter}

\begin{biblechapter} % Exodus 31
\verseWithHeading{Provision of Skilled People} And Adonai spoke to Moses, saying,
\verse “See, I have called by name Bezalel the son of Uri the son of Hur, from the tribe of Judah.
\verse And I have filled him with the Spirit of God, with wisdom and with skill and with knowledge and with every kind of craftsmanship,
\verse to devise designs, to work with gold and with silver and with bronze,
\verse and in stonecutting for setting and in cutting wood, for doing every kind of craftsmanship.
\verse And, look, I have given with him Oholiab the son of Ahisamach, from the tribe of Dan, and I have put skill in the heart of all the skilled of heart, and they will make all that I have commanded you—
\verse the tent of assembly and the ark of the testimony and the atonement cover that is on it and all the equipment of the tent,
\verse and the table and all its equipment, and the pure gold lampstand and all its equipment, and the incense altar,
\verse and the altar of burnt offering and all its equipment, and the basin and its stand,
\verse and the garments of woven material, and \textit{the garments of the sanctuary} for Aaron the priest and the garments of his sons to serve as priests,
\verse and the anointing oil and the fragrant incense for the sanctuary. According to all that I have commanded you, they will make it.
\verseWithHeading{Provision of Rest from Work} And Adonai spoke to Moses and said,
\verse “And you, speak to the \textit{Israelites}, saying, ‘Surely you must keep my Sabbaths, because it is a sign between me and you throughout your generations, in order to know that I am Adonai, who consecrates you.
\verse And you must keep the Sabbath, because it is holy for you; defilers of it will surely be put to death, because anyone who does work on it—that person will be cut off from among his people.
\verse On six days work can be done, and on the seventh \textit{is a Sabbath of complete rest}, \textit{a holy day} for Adonai; anyone doing work on the Sabbath day will surely be put to death.
\verse The \textit{Israelites} will pay attention to the Sabbath in order to fulfill the Sabbath throughout their generations as a lasting covenant.
\verse It is a sign between me and the \textit{Israelites} forever, because in six days Adonai made the heavens and the earth, and on the seventh he ceased and recovered.”
\verse And when he finished speaking with him on Mount Sinai, he gave to Moses the two tablets of the testimony, stone tablets, written with the finger of God.
\end{biblechapter}

\begin{biblechapter} % Exodus 32
\verseWithHeading{The Golden Calf} And the people saw that Moses delayed to come down from the mountain, and the people gathered opposite Aaron, and they said to him, “Come, make for us gods who will go before us, because this Moses, the man who brought us up from the land of Egypt, we do not know what has become of him.”
\verse And Aaron said to them, “Take off the rings of gold that are on the ears of your wives, your sons, and your daughters, and bring it to me.”
\verse And all the people took off the rings of gold that were on their ears and brought it to Aaron.
\verse And he took from their hand, and he shaped it with a tool, and he made it a cast-image bull calf, and they said, “These are your gods, Israel, who brought you up from the land of Egypt.”
\verse And Aaron saw, and he built an altar before it, and Aaron called, and he said, “A feast for Adonai tomorrow.”
\verse And they started early the next day, and they offered burnt offerings, and they presented fellowship offerings, and the people sat to eat and drink, and they rose up to revel.
\verse And Adonai spoke to Moses, “Go, go down because your people behave corruptly, whom you brought up from the land of Egypt.
\verse They have turned aside quickly from the way that I commanded them; they have made for themselves a cast-image bull calf, and they bowed to it, and they sacrificed to it, and they said, ‘These are your gods, Israel, who brought you up from the land of Egypt.’ ”
\verse And Adonai said to Moses, “I have seen this people, and, indeed, they are a stiff-necked people.
\verse And now leave me alone so that \textit{my anger may blaze} against them, and let me destroy them, and I will make you into a great nation.”
\verse And Moses \textit{implored Adonai} his God, and he said, “Why, Adonai, should \textit{your anger blaze} against your people whom you brought up from the land of Egypt with great power and with a strong hand?
\verse Why should the Egyptians \textit{say}, ‘With evil intent he brought them out to kill them in the mountains and wipe them from the face of the earth’? Turn from \textit{your fierce anger} and relent concerning the disaster for your people.
\verse Remember Abraham, Isaac, and Israel, your servants, to whom you swore by yourself, and you told them, ‘I will multiply your offspring like the stars of the heavens, and all this land that I promised I will give to your offspring, and they will inherit it forever.’ ”
\verse And Adonai relented concerning the disaster that he had \textit{threatened} to do to his people.
\verse And Moses turned and went down from the mountain, and the two tablets of the testimony were in his hand, tablets written on their two sides; \textit{on the front and on the back} they were written.
\verse And the tablets, they were the work of God; and the writing, it was the writing of God engraved on the tablets.
\verse And Joshua heard the sound of the people in their shouting, and he said to Moses, “A sound of war is in the camp.”
\verse But he said, “There is not a sound of shouting of victory, and there is not a sound of shouting of defeat. I hear a sound of singing.”
\verse \textit{And} as he came near to the camp, he saw the bull calf and dancing, and \textit{Moses became angry}, and he threw the tablets from his hand, and he broke them under the mountain.
\verse And he took the bull calf that they had made, and he burned it with the fire, and he crushed it until it became fine, and he scattered it on the surface of the water, and he made the \textit{Israelites} drink.
\verse And Moses said to Aaron, “What did this people do to you that you brought on them such a great sin?”
\verse And Aaron said, “\textit{Let not my lord become angry}. You yourself know the people, that \textit{they are intent on evil}.
\verse And they said to me, ‘Make for us gods who will go before us, because this Moses, the man who brought us up from the land of Egypt, we do not know what has become of him.’
\verse And I said to them, ‘Whoever has gold, take it off.’ And they gave it to me, and I threw it in the fire, and out came this bull calf.”
\verse And Moses saw the people, that they were running wild because Aaron had allowed them to run wild, for a laughingstock among \textit{their enemies}.
\verse And Moses stood at the entrance of the camp, and he said, “Whoever is for Adonai, to me.” And all the sons of Levi were gathered to him.
\verse And he said to them, “Thus says Adonai, the God of Israel, ‘Put each his sword on his side. Go \textit{back and forth} from gate to gate in the camp, and kill, each his brother and each his friend and each his close relative.’ ”
\verse And the sons of Levi did according to the word of Moses, and from the people on that day about three thousand persons fell.
\verse And Moses said, “\textit{You are ordained} today for Adonai, because each has been against his son and against his brother and so bringing on you today a blessing.”
\verse \textit{And} the next day Moses said to the people, “You have sinned a great sin. And now I will go up to Adonai. Perhaps I can make atonement for your sin.”
\verse And Moses returned to Adonai, and he said, “Alas, this people has sinned a great sin and made for themselves gods of gold.
\verse And now if you will forgive their sin—and if not, please blot me from your scroll that you have written.”
\verse And Adonai said to Moses, “Whoever has sinned against me I will blot him from my scroll.
\verse And now go, lead the people to where I spoke to you. Look, my angel will go before you, and on the day when I punish I will punish them for their sin.”
\verse And Adonai afflicted the people because they had made the bull calf that Aaron had made.
\end{biblechapter}

\begin{biblechapter} % Exodus 33
\verseWithHeading{Command to Resume Travel} And Adonai spoke to Moses, “Go, go up from here, you and the people whom you have brought up from the land of Egypt, to the land that I swore to Abraham, to Isaac, and to Jacob, saying, ‘I will give it to your offspring.’
\verse And I will send an angel before you, and I will drive out the Canaanites, the Amorites, and the Hittites and the Perizzites, the Hivites, and the Jebusites,
\verse Go to a land flowing with milk and honey, but I will not go up among you, because you are a stiff-necked people, lest I destroy you on the way.”
\verse And the people heard this troubling word, and they mourned, and they each did not put their ornaments on themselves.
\verse And Adonai said to Moses, “Say to the \textit{Israelites}, ‘You are a stiff-necked people; if one moment I were to go up among you, I would destroy you. And now take down your ornaments from on you, and I will \textit{decide} what I will do to you.”
\verse And the \textit{Israelites} stripped themselves of their ornaments from Mount Horeb onward.
\verseWithHeading{The Tent outside the Camp} And Moses took the tent and pitched it outside the camp far from the camp, and he called it the tent of assembly, \textit{and} all seeking Adonai would go out to the tent of assembly, which was outside the camp.
\verse \textit{And} at the going out of Moses to the tent, all the people would rise and stand, each at the opening of his tent, and gaze after Moses until his entering the tent.
\verse \textit{And} at the entering of Moses into the tent the column of cloud would descend and stand at the opening of the tent, and he would speak with Moses.
\verse And all the people would see the column of cloud standing at the opening of the tent, and all the people would rise and bow in worship, each at the opening of his tent.
\verse And Adonai would speak to Moses face to face, as a man speaks to his neighbor. And he would return to the camp, and his assistant Joshua the son of Nun, a young man, did not leave the middle of the tent.
\verseWithHeading{Presence and Glory} And Moses said to Adonai, “See, you are saying to me, ‘Take this people up.’ But you have not let me know whom you will send with me, and you yourself have said, ‘I know you by name, and you also have found favor in my eyes.’
\verse And now if I have found favor in your eyes, make known to me, please, your way, and so I may know you so that I can find favor in your eyes. And see that this nation is your people.”
\verse And he said, “My presence will go, and I will give you rest.”
\verse And he said to him, “If your presence is not going, do not bring us up from here.
\verse And by what will it be known then that I have found favor in your eyes, I and your people? Is it not by your going with us? And so we will be distinguished, I and your people, from all the people who are on the face of the ground.”
\verse And Adonai said to Moses, “Also I will do this thing that you have spoken, because you have found favor in my eyes and I have known you by name.”
\verse And he said, “Please show me your glory.”
\verse And he said, “I myself will cause all my goodness to pass over before you, and I will proclaim the name of Adonai before you, and I will be gracious to whom I will be gracious, and I will show compassion to whom I will show compassion.”
\verse But he said, “You are not able to see my face, because a human will not see me and live.”
\verse And Adonai said, “There is a place with me, and you will stand on the rock.
\verse \textit{And} when my glory passes over, I will put you in the rock’s crevice, and I will cover you with my hand until I pass over.
\verse And I will remove my hand, and you will see my back, but my face will not be visible.”
\end{biblechapter}

\begin{biblechapter} % Exodus 34
\verseWithHeading{Adonai’s Description of Himself} And Adonai said to Moses, “Cut for yourself two stone tablets like the first ones, and I will write on the tablets the words that were on the first tablets, which you broke.
\verse And be ready for the morning, and go up in the morning to Mount Sinai and present yourself to me there on the top of the mountain.
\verse And no one will go up with you, and neither let anyone be seen on all the mountain, nor let the sheep and goats and the cattle graze \textit{opposite} that mountain.”
\verse And Moses cut two stone tablets like the first ones, and he started early in the morning, and he went up to Mount Sinai, as Adonai had commanded him, and he took in his hand the two stone tablets.
\verse And Adonai descended in the cloud, and he stood with him there, and he proclaimed the name of Adonai.
\verse And Adonai passed over before him, and he proclaimed, “Adonai, Adonai, God, who is compassionate and gracious, \textit{slow to anger}, and abounding with loyal love and faithfulness,
\verse keeping loyal love to the thousands, forgiving iniquity and transgression and sin, and he does not leave utterly unpunished, punishing the guilt of fathers on sons and on sons of sons on third and fourth generations.”
\verse And Moses hurried and knelt down to the earth and worshiped.
\verse And he said, “Please, if I have found favor in your eyes, Lord, let my Lord, please, go among us—indeed it is a stiff-necked people—and forgive our iniquity and our sin and \textit{take us as your possession}.”
\verseWithHeading{Covenant Stipulations} And he said, “Look, I am about to make a covenant. In front of all your people I will do wonders that have not been created on all the earth and among all the nations, and all the people among whom you are will see Adonai’s work, because what I am about to do with you will be awesome.
\verse “Keep for yourself what I myself have commanded you today. Look, I am about to drive from before you the Amorites and the Canaanites and the Hittites and the Perizzites and the Hivites and the Jebusites.
\verse Be careful for yourself, lest you make a covenant with the inhabitants of the land to which you are going, lest it be a snare among you.
\verse Rather, you will tear down their altars, and you will break their stone pillars, and you will cut off their Asherah poles.
\verse For you will not bow in worship to another god, for ‘Adonai Is Jealous’ is his name, he is a jealous God,
\verse lest you make a covenant with the inhabitants of the land, and they prostitute themselves after their gods, and they sacrifice to their gods, and they invite you, and you eat their sacrifice,
\verse and you take from their daughters for your sons, and their daughters prostitute themselves after their gods, and they cause your sons to prostitute themselves after their gods.
\verse You will not make gods of cast metal for yourself.
\verse “You will keep the Feast of Unleavened Bread. Seven days you will eat unleavened bread, which I commanded you, at the appointed time of the month of Abib, for in the month of Abib you came out from Egypt.
\verse Every first offspring of a womb is for me—all of your male livestock, the first offspring of cattle and small livestock.
\verse But the first offspring of a donkey you will redeem with small livestock, and if you will not redeem it, you will break its neck. Every firstborn of your sons you will redeem, and you will not appear before me empty-handed.
\verse Six days you will work, and on the seventh day you will rest; in the time of plowing and in the time of harvest you will rest.
\verse And \textit{you yourself} will observe the Feast of Weeks—the firstfruits of the wheat harvest—and the Feast of Harvest Gathering at the turn of the year.
\verse Three times in the year all your males will appear before the Lord, Adonai, the God of Israel,
\verse because I will evict nations before you, and I will enlarge your territory, and no one will covet your land when you go up to appear before Adonai your God three times in the year.
\verse “You will not slaughter the blood of my sacrifice on food with yeast, and the sacrifice of the Feast of the Passover will not stay overnight to the morning.
\verse The beginning of the firstfruits of your land you will bring to the house of Adonai your God. You will not boil a young goat in its mother’s milk.”
\verse And Adonai said to Moses, “Write for yourself these words, because \textit{according to} these words I have made a covenant with you and with Israel.”
\verseWithHeading{The Shining Face of Moses} And he was there with Adonai forty days and forty nights. He ate no food and drank no water. And he wrote on the tablets the words of the covenant, the ten words.
\verse \textit{And} when Moses came down from Mount Sinai, the two tablets of the testimony were in the hand of Moses at his coming down from the mountain; and Moses did not know that the skin of his face shone because of his speaking with him.
\verse And Aaron and all the \textit{Israelites} saw Moses, and, \textit{to their amazement}, the skin of his face shone, and they were afraid of coming near to him.
\verse And Moses called to them, and Aaron and all the leaders of the community returned to him, and Moses spoke to them.
\verse And afterward all the \textit{Israelites} came near, and he commanded them all that Adonai had spoken with him on Mount Sinai.
\verse And Moses finished speaking with them, and he put a veil on his face.
\verse And when Moses came before Adonai to speak with him, he would remove the veil until he went out, and he would go out and would speak to the \textit{Israelites} what he had been commanded.
\verse And the \textit{Israelites} would see the face of Moses, that the skin of the face of Moses shone, and Moses would put back the veil on his face until his coming to speak with him.
\end{biblechapter}

\begin{biblechapter} % Exodus 35
\verseWithHeading{Preparation of Materials and People for Building} And Moses assembled all the community of the \textit{Israelites}, and he said to them, “These are the words that Adonai has commanded for us to do them.
\verse On six days work can be done, and on the seventh there will be for you a holy day, a \textit{Sabbath of complete rest}, for Adonai; anyone doing work on it will be put to death.
\verse You will not kindle a fire in any of your dwellings on the day of the Sabbath.”
\verse And Moses said to all the community of the \textit{Israelites}, saying, “This is the word that Adonai has commanded, saying,
\verse ‘Take from among you a contribution for Adonai, anyone willing of heart, let him bring Adonai’s contribution—gold and silver and bronze,
\verse and blue and purple and crimson yarns, and fine linen and goat hair,
\verse and red-dyed ram skins, and fine leather, and acacia wood,
\verse and oil for the lamp, balsam oils for the anointing oil and for the fragrant incense,
\verse onyx stones and stones for mountings on the ephod and the breast piece.
\verse “And let all the skilled of heart among you come and make all that Adonai has commanded:
\verse The tabernacle, its tent, and its covering; its clasps and its frames; its bars, its pillars, and its bases;
\verse the ark and its poles; the atonement cover and the curtain of the screen;
\verse the table and its poles and all its equipment; and the bread of the presence;
\verse and lampstand of the light and its equipment and its lamps and the oil for the light;
\verse and the altar of incense and its poles; and the anointing oil and the fragrant incense and the entrance curtain for the entrance of the tabernacle;
\verse the altar of the burnt offering and the bronze grating that is for it, its poles and all its equipment; the basin and its stand;
\verse the hangings of the courtyard, its pillars, and its bases, and the screen for the courtyard gate;
\verse the \textit{pegs} of the tabernacle and the \textit{pegs} of the courtyard and their cords;
\verse the woven garments for serving in the sanctuary—the holy garments for Aaron the priest and the garments of his sons to serve as priests.”
\verse And all the community of the \textit{Israelites} went out from before Moses.
\verse And they came—every man whose heart lifted him and every man whose spirit impelled him—they brought Adonai’s contribution for the work of the tent of assembly and for all its service and for the holy garments.
\verse And they came, the men in addition to the women, all who were willing of heart; they brought brooches and jewelry rings and signet rings and ornaments—every variety of gold object—every man who waved a wave offering of gold for Adonai,
\verse and every man with whom was found blue and purple and crimson yarns and fine linen and goat hair and red-dyed ram skins and fine leather brought it.
\verse All who were presenting a contribution of silver and bronze brought Adonai’s contribution, and all with whom was found acacia wood for all the work of service brought it.
\verse And every woman who was skilled of heart with her hands they spun, and they brought yarn—the blue and the purple, the crimson and the fine linen.
\verse And all the women whose heart lifted them with skill spun the goat hair.
\verse And the leaders brought the onyx stones and stones for mountings for the ephod and for the breast piece
\verse and the balsam oils and the oil for light and for the anointing oil and for the fragrant incense.
\verse Every man and woman whose heart impelled them to bring for all the work to be done that Adonai had commanded \textit{by the agency of} Moses—the \textit{Israelites} brought freely to Adonai.
\verse And Moses said to the \textit{Israelites}, “See, Adonai has called by name Bezalel the son of Uri the son of Hur, from the tribe of Judah.
\verse And he has filled him with the Spirit of God, with wisdom and with skill and with knowledge and with every kind of craftsmanship,
\verse and to devise designs, to work with the gold and with the silver and with the bronze,
\verse and in stonecutting for setting and in cutting wood, for doing every kind of design craftsmanship.
\verse And he has put it in his heart to teach—he and Oholiab the son of Ahisamach, from the tribe of Dan.
\verse He has filled them with skill of heart to do every work of a craftsman and a designer and an embroiderer with the blue and with the purple, with the crimson yarns and with the fine linen and a weaver; they are doers of every kind of craftsmanship and devisers of designs.
\end{biblechapter}

\begin{biblechapter} % Exodus 36
\verse And Bezalel and Oholiab and everyone who is skilled of heart in whom Adonai has put wisdom and skill to know and to do all the work for the service of the sanctuary—they will do it, according to all that Adonai has commanded.”
\verseWithHeading{Making the Tabernacle} And Moses called Bezalel and Oholiab and everyone skilled of heart, in whose heart Adonai had put skill, all whose heart lifted him to come near to the work in order to do it.
\verse And they took from Moses all the contributions that the \textit{Israelites} had brought for the work of the service for the sanctuary in order to do it, and they still brought to him voluntary offerings \textit{every morning}.
\verse And all the skilled workers who were doing all the work for the sanctuary came, \textit{each} from his work that they were doing.
\verse And they said to Moses, saying, “The people are \textit{bringing more} than enough for the service of the work that Adonai has commanded \textit{to be done}.”
\verse And Moses commanded, and they \textit{proclaimed the message} in the camp, saying, “Let no man or woman again make anything for the sanctuary contribution.” And so the people were restrained from bringing.
\verse And the material was enough for doing all the work, and it was left over.
\verse And all who were skilled of heart among the doers of the work made the tabernacle with ten curtains of finely twisted linen and blue and purple and crimson yarns, with cherubim; he made them, the work of a skilled craftsman.
\verse The length of the one curtain was twenty-eight cubits, and the width was four cubits for the one curtain; one measurement was for all the curtains.
\verse And he joined five of the curtains \textit{one to another}, and five curtains he joined \textit{one to another}.
\verse And he made loops of blue on the edge of the one curtain, at the end in the set; so he did on the edge of the end curtain in the second set.
\verse He made fifty loops on the one curtain, and he made fifty loops on the end of the curtain that was in the second set; the loops were opposite \textit{one to another}.
\verse And he made fifty gold clasps and joined the curtains \textit{one to another} with the clasps, so that the tabernacle was one.
\verse And he made curtains of goat hair for a tent over the tabernacle; he made them eleven curtains.
\verse The length of the one curtain was thirty cubits, and the width was four cubits for the one curtain; one measure was for the eleven curtains.
\verse And he joined five curtains together and six curtains together.
\verse And he made fifty loops on the edge of the end curtain in the set, and he made fifty loops on the edge of the curtain in the second set.
\verse And he made fifty bronze clasps for joining the tent to become one.
\verse And he made a covering for the tent of red-dyed ram skin and a covering of fine leather to go above.
\verse And he made the frames for the tabernacle of acacia wood as uprights.
\verse The length of the frame was ten cubits, and the width of the one frame was one and a half cubits.
\verse He made two \textit{pegs} for the one frame for joining \textit{one to another} and likewise for all the frames of the tabernacle.
\verse And he made the frames for the tabernacle with twenty frames for the south side.
\verse And he made forty silver bases under the twenty frames, with two bases under the one frame for its two \textit{pegs} and two bases under the \textit{next} frame for its two \textit{pegs}.
\verse And for the second side of the tabernacle, the north side, he made twenty frames
\verse and their forty silver bases, with two bases under the one frame and two bases under the \textit{next} frame.
\verse And for the rear of the tabernacle \textit{on the west} he made six frames.
\verse And he made two frames for the tabernacle corners at the rear.
\verse And they were double at the bottom, and they were completely together on its top to the one ring; he did likewise for the two of them, for the two corners.
\verse And there were eight frames and their sixteen silver bases, two bases, two bases under the one frame.
\verse And he made five bars of acacia wood for the frames on the one side of the tabernacle,
\verse and five bars for the frames on the second side of the tabernacle, and five bars for the frames at the rear \textit{on the west}.
\verse And he made the middle bar to run in the midst of the frames from end to end.
\verse And he overlaid the frames with gold, and he made their rings of gold as \textit{holders} for the bars, and he overlaid the bars with gold.
\verse And he made the curtain of blue and purple and crimson yarns and finely twisted linen, the work of a craftsman; he made it with cherubim.
\verse And he made for it four acacia pillars, and he overlaid them with gold, with their gold hooks, and he cast for them four silver bases.
\verse And he made for the entrance of the tent a screen of blue and purple and crimson yarns and finely twisted linen, the work of an embroiderer,
\verse and the five pillars and their hooks, and he overlaid their tops and their connections with gold, and their five bases were bronze.
\end{biblechapter}

\begin{biblechapter} % Exodus 37
\verseWithHeading{Making the Ark of the Covenant} And Bezalel made the ark of acacia wood, two and a half cubits its length and a cubit and a half its width and a cubit and a half its height.
\verse And he overlaid it with pure gold inside and outside, and he made for it a gold molding all around.
\verse And he cast for it four gold rings on its four feet; and two rings were on its one side, and two rings were on its second side.
\verse And he made poles of acacia wood, and he overlaid them with gold.
\verse And he put the poles into the rings on the sides of the ark to carry the ark.
\verse And he made an atonement cover of pure gold, two and a half cubits its length and a cubit and a half its width.
\verse And he made two cherubim of gold; he made them of hammered work at the two ends of the atonement cover.
\verse One cherub was \textit{at one end}, and one cherub was \textit{at the other end} of the atonement cover; he made the cherubim at its two ends.
\verse And the cherubim were with outspread wings above, covering with their wings over the atonement cover \textit{and facing each other}; the faces of the cherubim were toward the atonement cover.
\verseWithHeading{Making the Table and the Lampstand} And he made the table of acacia wood, two cubits its length and a cubit its width and a cubit and a half its height.
\verse And he overlaid it with pure gold, and he made for it a gold molding all around.
\verse And he made for it a handbreadth rim all around, and he made a gold molding for its rim all around.
\verse And he cast for it four gold rings, and he put the rings on the four corners where its four \textit{legs} were.
\verse The rings were near the rim as \textit{holders} for the poles to carry the table.
\verse And he made the poles of acacia wood, and he overlaid them with gold to carry the table.
\verse And he made the vessels that were on the table—its plates and its ladles, and its bowls and its pitchers with which libations were poured—of pure gold.
\verse And he made the lampstand of pure gold; he made the lampstand of hammered work—its base and its branch, its cups, its buds, and its blossoms were \textit{all part of it}.
\verse And six branches were going out from its sides, three branches of the lampstand from its one side and three branches of the lampstand from its second side.
\verse Three almond-flower cups were on the one branch with a bud and a blossom, and three almond-flower cups were on the one branch with a bud and a blossom—likewise for the six branches going out from the lampstand.
\verse And on the lampstand were four almond-flower cups, with its buds and its blossoms.
\verse And a bud was under the two branches that came from it, and a bud under the two branches from it, and a bud under the two branches from it, likewise for the six branches coming out from the lampstand.
\verse Their buds and their branches were from it, all of it one piece of pure gold, hammered work.
\verse And he made its seven lamps and its snuffers and its fire pans of pure gold.
\verse He made it from a talent of pure gold and all its pieces of equipment.
\verseWithHeading{Making the Incense Altar and Anointing Oil} And he made the incense altar of acacia wood, a cubit its length and a cubit its width—a square—and two cubits its height; its horns were \textit{of one piece with it}.
\verse And he overlaid it with pure gold—its top and its sides all around and its horns—and he made for it a gold molding all around.
\verse And he made for it two gold rings under its molding \textit{on two opposite sides} as \textit{holders} for poles to carry it with them.
\verse And he made the poles of acacia wood, and he overlaid them with gold.
\verse And he made the holy anointing oil and the pure fragrant incense—work of a perfumer.
\end{biblechapter}

\begin{biblechapter} % Exodus 38
\verseWithHeading{Making the Bronze Altar and the Basin} And he made the burnt-offering altar of acacia wood; its length was five cubits, and its width was five cubits—it was square—and its height was three cubits.
\verse And he made its horns on its four corners; its horns were \textit{of one piece with it}; and he overlaid it with bronze.
\verse And he made all the equipment of the altar—the pots and the shovels and the sprinkling bowls and the forks and the fire pans—all its equipment he made with bronze.
\verse And he made for the altar a grating, a work of bronze network under its ledge, below, up to its middle.
\verse And he cast four rings on the four ends of the bronze grating as \textit{holders} for the poles.
\verse And he made the poles of acacia wood, and he overlaid them with bronze.
\verse And he put the poles into the rings on the sides of the altar to carry it with them. He made it hollow with boards.
\verse And he made the basin of bronze and its stand of bronze from the mirrors of the serving women who served at the entrance of the tent of assembly.
\verseWithHeading{Making the Courtyard} And he made the courtyard; for the south side were the hangings of the court of finely twisted linen, one hundred cubits,
\verse with their twenty pillars and their twenty bases of bronze and with the hooks of the pillars and their bands of silver.
\verse And for the north side the hangings were one hundred cubits with their twenty pillars and their twenty bases of bronze and with the hooks of the pillars and their bands of silver.
\verse And for the \textit{west} side fifty cubits of hangings with their ten pillars and their ten bases and with the hooks of the pillars and their bands of silver.
\verse And for the eastward side, toward sunrise, fifty cubits;
\verse fifteen cubits of hangings were to the shoulder, with their three pillars and their three bases,
\verse and for the second shoulder \textit{on each side} of the gate of the courtyard were fifteen cubits of hangings, with their three pillars and their three bases.
\verse All the hangings of the courtyard all around were finely twisted linen,
\verse and the bases for the pillars were bronze, the hooks of the pillars and their bands were silver, and the overlay of their tops was silver, and all the pillars of the courtyard were banded with silver.
\verse And the screen of the gate of the courtyard was the work of an embroiderer, with blue and purple and crimson yarns and finely twisted linen; it was twenty cubits long and five cubits \textit{high}, like the hangings of the courtyard,
\verse and with their four pillars and their four bases of bronze, with their silver hooks and with their tops and their bands of silver.
\verse And all the \textit{pegs} for the tabernacle and for the courtyard all around were bronze.
\verseWithHeading{Amounts of Gold, Silver, and Bronze Used} These are the records of the tabernacle, the tabernacle of the testimony, which were recorded at the \textit{command of} Moses, the work of the Levites, in the hand of Ithamar the son of Aaron the priest.
\verse And Bezalel the son of Uri the son of Hur, of the tribe of Judah, did all that Adonai commanded Moses.
\verse And with him was Oholiab the son of Ahisamach of the tribe of Dan, a skilled craftsman and a designer and an embroiderer with the blue and with the purple and with the crimson yarns and with the linen.
\verse And all the gold used for the work, in the work of the sanctuary, it was the gold of the wave offering—twenty-nine talents and seven hundred and thirty shekels, according to the sanctuary shekel.
\verse And the silver recorded from the community was a hundred talents and a thousand seven hundred and seventy-five shekels, according to the sanctuary shekel.
\verse It was a bekah for the individual, the half shekel according to the sanctuary shekel, for \textit{everyone who was counted}, from \textit{twenty years old} and above, for six hundred and three thousand five hundred and fifty.
\verse And it was one hundred talents of the silver to cast the bases of the sanctuary and the bases of the curtain—one hundred bases for one hundred talents of silver, a talent for each base.
\verse And from the thousand seven hundred and seventy-five shekels he made hooks for the pillars, and he overlaid their tops, and he made bands for them.
\verse And the bronze of the wave offering was seventy talents and two thousand four hundred shekels.
\verse And he made with it the bases of the entrance of the tent of assembly and the bronze altar and the bronze grating that belonged to it and all the equipment of the altar
\verse and the bases of the courtyard all around and the bases of the gate of the courtyard and all the \textit{pegs} of the tabernacle and all the \textit{pegs} of the courtyard all around.
\end{biblechapter}

\begin{biblechapter} % Exodus 39
\verseWithHeading{Making Garments for Priests} And from the blue and the purple and the crimson yarns they made woven garments for serving in the sanctuary, and they made the holy garments that were for Aaron, as Adonai had commanded Moses.
\verse And he made the ephod of gold, blue, and purple and crimson yarns, and finely twisted linen.
\verse And he hammered out the leaves of gold, and he cut off cords to weave in the midst of the blue and in the midst of the purple and in the midst of the crimson and in the midst of the linen—the work of a skilled craftsman.
\verse They made joined shoulder pieces for it; it was joined on its two edges.
\verse And the waistband of his ephod, which \textit{was of one piece with it}, was of like work, gold, blue, and purple and crimson yarns, and finely twisted linen, as Adonai had commanded Moses.
\verse And they made onyx stones mounted in gold filigree settings, engraved with seal engravings according to the names of the \textit{Israelites}.
\verse And he set them on the ephod’s shoulder pieces as stones of remembrance for the \textit{Israelites}, as Adonai had commanded Moses.
\verse And he made the breast piece, the work of a skilled craftsman, like the work of the ephod, of gold, blue, and purple and crimson yarns, and finely twisted linen.
\verse It was squared; they made the breast piece doubled; its length was a span, and its width was a span when doubled.
\verse And they filled it with four rows of stone; a row of carnelian, topaz, and emerald was the first row;
\verse and the second row was a malachite, a sapphire, and a moonstone;
\verse and the third row was a jacinth, an agate, and an amethyst;
\verse and the fourth row was a turquoise, an onyx, and a jasper. They were set with gold filigree settings in their mountings.
\verse And the stones were according to the names of the \textit{Israelites}; they were twelve according to their names, with seal engravings, each according to its name for the twelve tribes.
\verse And they made on the breast piece braided chains, a work of pure gold ornamental cord.
\verse And they made two gold filigree settings and two gold rings, and they put the two rings on the two edges of the breast piece.
\verse And they put the two gold ornamental cords on the two rings on the edges of the breast piece.
\verse And they put the two ends of the two ornamental cords on the two filigree settings, and they put them on the shoulder pieces of the ephod at the front of it.
\verse And they made two gold rings, and they placed them on the two edges of the breast piece, on its lip that is on \textit{the other side} of the ephod, \textit{to the inside}.
\verse And they made two gold rings and put them on the ephod’s two shoulder pieces below, at its front near its seam above the waistband of the ephod.
\verse And they tied the breast piece by its rings to the rings of the ephod with a blue cord so that the breast piece would be on the waistband of the ephod and not come loose from the ephod, as Adonai had commanded Moses.
\verse And he made the robe of the ephod, weaver’s work, totally of blue yarn.
\verse And the opening of the robe in the middle of it was like the opening of a sturdy garment, with an edge for its opening all around so that it would not be torn.
\verse And they made on the hem of the robe pomegranates of finely twisted blue and purple and crimson.
\verse And they made pure gold bells and put the bells in the midst of the pomegranates on the hem of the robe all around in the midst of the pomegranates,
\verse a bell and a pomegranate, a bell and a pomegranate on the hem of the robe all around for serving, as Adonai had commanded Moses.
\verse And they made the tunics of fine linen, a weaver’s work, for Aaron and for his sons,
\verse and the turban of fine linen and the headdresses of the headbands of fine linen and undergarments of the linen cloth, finely twisted,
\verse and the sash of finely twisted linen and blue and purple and crimson yarns, the work of an embroiderer, as Adonai had commanded Moses.
\verse And they made the rosette of the holy diadem of pure gold, and they wrote on it with the writing of seal engravings: “A holy object for Adonai.”
\verse And they put a blue cord on it to put it above on the turban, as Adonai had commanded Moses.
\verseWithHeading{Preparations Complete} And all the work of the tabernacle of the tent of assembly was finished, and the \textit{Israelites} had done according to all that Adonai had commanded Moses; so they did.
\verse And they brought the tabernacle to Moses, the tent and all its equipment, its hooks, its frames, its bars, and its pillars and its bases;
\verse and the covering of the red-dyed ram skins and the covering of fine leather and the curtain of the screen;
\verse the ark of the testimony and its poles and the atonement cover;
\verse the table, all its equipment, and the bread of the presence;
\verse the pure gold lampstand, its lamps—the lamps of the row—and all its equipment and the oil of the light;
\verse the gold altar, the anointing oil, the fragrant incense, and the screen of the entrance of the tent;
\verse the bronze altar, and the bronze grating that is for it, its poles, and all its equipment, the basin and its stand,
\verse the hangings of the courtyard, its pillars, and its bases; and the screen for the courtyard gate, its tent cords and its \textit{pegs}; and all the equipment of the service of the tabernacle for the tent of assembly,
\verse the woven garments for serving in the sanctuary—the holy garments for Aaron the priest and the garments for his sons to serve as priests.
\verse According to all that Adonai had commanded Moses, so the \textit{Israelites} did all the work.
\verse And Moses saw all the work, and, indeed, they had done it as Adonai had commanded; so they did, and Moses blessed them.
\end{biblechapter}

\begin{biblechapter} % Exodus 40
\verseWithHeading{Completion of the Tabernacle} And Adonai spoke to Moses, saying,
\verse “On the first day of the month, you will set up the tabernacle of the tent of assembly.
\verse And you will put there the ark of the testimony, and you will cover over the ark with the curtain.
\verse And you will bring the table, and you will arrange its setting, and you will bring the lampstand, and you will set up its lamps.
\verse And you will put the gold altar for incense before the ark of the testimony, and you will set up the entrance screen for the tabernacle.
\verse And you will put the altar of the burnt offering before the entrance of the tabernacle of the tent of assembly.
\verse And you will put the basin between the tent of assembly and the altar, and you will put water \textit{in it}.
\verse And you will set up the courtyard all around, and you will put up the screen of the gate of the courtyard.
\verse “And you will take the anointing oil, and you will anoint the tabernacle and all that is in it, and you will consecrate all of its equipment, and it will be holy.
\verse And you will anoint the altar of the burnt offering and all of its equipment, and you will consecrate the altar, and the altar will be a most holy thing.
\verse And you will anoint the basin and its stand, and you will consecrate it.
\verse And you will bring Aaron and his sons to the entrance of the tent of assembly, and you will wash them with the water.
\verse And you will clothe Aaron with the holy garments, and you will anoint him, and you will consecrate him, and he will serve as a priest for me.
\verse And you will bring his sons, and you will clothe them with tunics.
\verse And you will anoint them as you anointed their father, and they will serve as priests for me. And their anointing will be for them to be a lasting priesthood throughout their generations.”
\verse And Moses did according to all that Adonai had commanded him; so he did.
\verse \textit{In} the first month of the second year, on the first of the month, the tabernacle was set up.
\verse And Moses raised the tabernacle, and he placed its bases, and he set up its frames, and he placed its bars, and he raised its pillars.
\verse And he spread the tent over the tabernacle; he placed the covering of the tent over it, above it, as Adonai had commanded Moses.
\verse And he took and he put the testimony into the ark, and he placed the poles on the ark, and he put the atonement cover on the ark, above it.
\verse And he brought the ark into the tabernacle, and he set up the curtain of the screening, and he shielded the ark of the testimony, as Adonai had commanded Moses.
\verse And he put the table in the tent of assembly on the north side of the tabernacle outside the curtain.
\verse And he arranged on it an arrangement of bread before Adonai, as Adonai had commanded Moses.
\verse And he placed the lampstand in the tent of assembly opposite the table on the south side of the tabernacle.
\verse And he set up the lamps before Adonai, as Adonai had commanded Moses.
\verse And he placed the gold altar in the tent of assembly before the curtain.
\verse And he turned fragrant incense into smoke on it, as Adonai had commanded Moses.
\verse And he set up the entrance screen for the tabernacle.
\verse And the altar of burnt offering he placed at the entrance of the tabernacle of the tent of assembly, and he offered on it the burnt offering and the grain offering, as Adonai had commanded Moses.
\verse And he placed the basin between the tent of assembly and the altar, and he put there water for washing.
\verse And Moses and Aaron and his sons washed their hands and their feet from it.
\verse At their going into the tent of assembly and at their approaching the altar, they washed, as Adonai had commanded Moses.
\verse And he set up the courtyard all around the tabernacle and the altar, and he put up the screen of the gate of the courtyard, and Moses completed the work.
\verse And the cloud covered the tent of assembly, and the glory of Adonai filled the tabernacle.
\verse And Moses was unable to go into the tent of assembly because the cloud settled on it and the glory of Adonai filled the tabernacle.
\verse And when the cloud was lifted from the tabernacle, the \textit{Israelites} set out on all their journeys.
\verse But if the cloud was not lifted, they did not set out until the day of its being lifted.
\verse For the cloud of Adonai was on the tabernacle by day, and fire was on it by night before the eyes of all the house of Israel throughout all their journeys.
\end{biblechapter}

\flushcolsend
\biblebook{Leviticus}

\begin{biblechapter} % Leviticus 1
\verseWithHeading{The burnt offering} And the \LORD called unto Moses, and spake unto him out of the tabernacle of the congregation, saying,
\verse Speak unto the children of Israel, and say unto them, If any man of you bring an offering unto the \LORD, ye shall bring your offering of the cattle, even of the herd, and of the flock.
\verse If his offering be a burnt sacrifice of the herd, let him offer a male without blemish: he shall offer it of his own voluntary will at the door of the tabernacle of the congregation before the \LORD.
\verse And he shall put his hand upon the head of the burnt offering; and it shall be accepted for him to make atonement for him.
\verse And he shall kill the bullock before the \LORD: and the priests, Aaron's sons, shall bring the blood, and sprinkle the blood round about upon the altar that is by the door of the tabernacle of the congregation.
\verse And he shall flay the burnt offering, and cut it into his pieces.
\verse And the sons of Aaron the priest shall put fire upon the altar, and lay the wood in order upon the fire:
\verse And the priests, Aaron's sons, shall lay the parts, the head, and the fat, in order upon the wood that is on the fire which is upon the altar:
\verse But his inwards and his legs shall he wash in water: and the priest shall burn all on the altar, to be a burnt sacrifice, an offering made by fire, of a sweet savour unto the \LORD.
\verse And if his offering be of the flocks, namely, of the sheep, or of the goats, for a burnt sacrifice; he shall bring it a male without blemish.
\verse And he shall kill it on the side of the altar northward before the \LORD: and the priests, Aaron's sons, shall sprinkle his blood round about upon the altar.
\verse And he shall cut it into his pieces, with his head and his fat: and the priest shall lay them in order on the wood that is on the fire which is upon the altar:
\verse But he shall wash the inwards and the legs with water: and the priest shall bring it all, and burn it upon the altar: it is a burnt sacrifice, an offering made by fire, of a sweet savour unto the \LORD.
\verse And if the burnt sacrifice for his offering to the \LORD be of fowls, then he shall bring his offering of turtledoves, or of young pigeons.
\verse And the priest shall bring it unto the altar, and wring off his head, and burn it on the altar; and the blood thereof shall be wrung out at the side of the altar:
\verse And he shall pluck away his crop with his feathers, and cast it beside the altar on the east part, by the place of the ashes:
\verse And he shall cleave it with the wings thereof, but shall not divide it asunder: and the priest shall burn it upon the altar, upon the wood that is upon the fire: it is a burnt sacrifice, an offering made by fire, of a sweet savour unto the \LORD.
\end{biblechapter}

\begin{biblechapter} % Leviticus 2
\verseWithHeading{The meat offering} And when any will offer a meat offering unto the \LORD, his offering shall be of fine flour; and he shall pour oil upon it, and put frankincense thereon:
\verse And he shall bring it to Aaron's sons the priests: and he shall take thereout his handful of the flour thereof, and of the oil thereof, with all the frankincense thereof; and the priest shall burn the memorial of it upon the altar, to be an offering made by fire, of a sweet savour unto the \LORD:
\verse And the remnant of the meat offering shall be Aaron's and his sons': it is a thing most holy of the offerings of the \LORD made by fire.
\verse And if thou bring an oblation of a meat offering baken in the oven, it shall be unleavened cakes of fine flour mingled with oil, or unleavened wafers anointed with oil.
\verse And if thy oblation be a meat offering baken in a pan, it shall be of fine flour unleavened, mingled with oil.
\verse Thou shalt part it in pieces, and pour oil thereon: it is a meat offering.
\verse And if thy oblation be a meat offering baken in the fryingpan, it shall be made of fine flour with oil.
\verse And thou shalt bring the meat offering that is made of these things unto the \LORD: and when it is presented unto the priest, he shall bring it unto the altar.
\verse And the priest shall take from the meat offering a memorial thereof, and shall burn it upon the altar: it is an offering made by fire, of a sweet savour unto the \LORD.
\verse And that which is left of the meat offering shall be Aaron's and his sons': it is a thing most holy of the offerings of the \LORD made by fire.
\verse No meat offering, which ye shall bring unto the \LORD, shall be made with leaven: for ye shall burn no leaven, nor any honey, in any offering of the \LORD made by fire.
\verse As for the oblation of the firstfruits, ye shall offer them unto the \LORD: but they shall not be burnt on the altar for a sweet savour.
\verse And every oblation of thy meat offering shalt thou season with salt; neither shalt thou suffer the salt of the covenant of thy God to be lacking from thy meat offering: with all thine offerings thou shalt offer salt.
\verse And if thou offer a meat offering of thy firstfruits unto the \LORD, thou shalt offer for the meat offering of thy firstfruits green ears of corn dried by the fire, even corn beaten out of full ears.
\verse And thou shalt put oil upon it, and lay frankincense thereon: it is a meat offering.
\verse And the priest shall burn the memorial of it, part of the beaten corn thereof, and part of the oil thereof, with all the frankincense thereof: it is an offering made by fire unto the \LORD.
\end{biblechapter}

\begin{biblechapter} % Leviticus 3
\verseWithHeading{The peace offering} And if his oblation be a sacrifice of peace offering, if he offer it of the herd; whether it be a male or female, he shall offer it without blemish before the \LORD.
\verse And he shall lay his hand upon the head of his offering, and kill it at the door of the tabernacle of the congregation: and Aaron's sons the priests shall sprinkle the blood upon the altar round about.
\verse And he shall offer of the sacrifice of the peace offering an offering made by fire unto the \LORD; the fat that covereth the inwards, and all the fat that is upon the inwards,
\verse And the two kidneys, and the fat that is on them, which is by the flanks, and the caul above the liver, with the kidneys, it shall he take away.
\verse And Aaron's sons shall burn it on the altar upon the burnt sacrifice, which is upon the wood that is on the fire: it is an offering made by fire, of a sweet savour unto the \LORD.
\verse And if his offering for a sacrifice of peace offering unto the \LORD be of the flock; male or female, he shall offer it without blemish.
\verse If he offer a lamb for his offering, then shall he offer it before the \LORD.
\verse And he shall lay his hand upon the head of his offering, and kill it before the tabernacle of the congregation: and Aaron's sons shall sprinkle the blood thereof round about upon the altar.
\verse And he shall offer of the sacrifice of the peace offering an offering made by fire unto the \LORD; the fat thereof, and the whole rump, it shall he take off hard by the backbone; and the fat that covereth the inwards, and all the fat that is upon the inwards,
\verse And the two kidneys, and the fat that is upon them, which is by the flanks, and the caul above the liver, with the kidneys, it shall he take away.
\verse And the priest shall burn it upon the altar: it is the food of the offering made by fire unto the \LORD.
\verse And if his offering be a goat, then he shall offer it before the \LORD.
\verse And he shall lay his hand upon the head of it, and kill it before the tabernacle of the congregation: and the sons of Aaron shall sprinkle the blood thereof upon the altar round about.
\verse And he shall offer thereof his offering, even an offering made by fire unto the \LORD; the fat that covereth the inwards, and all the fat that is upon the inwards,
\verse And the two kidneys, and the fat that is upon them, which is by the flanks, and the caul above the liver, with the kidneys, it shall he take away.
\verse And the priest shall burn them upon the altar: it is the food of the offering made by fire for a sweet savour: all the fat is the \LORDs.
\verse It shall be a perpetual statute for your generations throughout all your dwellings, that ye eat neither fat nor blood.
\end{biblechapter}

\begin{biblechapter} % Leviticus 4
\verseWithHeading{The sin offering} And the \LORD spake unto Moses, saying,
\verse Speak unto the children of Israel, saying, If a soul shall sin through ignorance against any of the commandments of the \LORD concerning things which ought not to be done, and shall do against any of them:
\verse If the priest that is anointed do sin according to the sin of the people; then let him bring for his sin, which he hath sinned, a young bullock without blemish unto the \LORD for a sin offering.
\verse And he shall bring the bullock unto the door of the tabernacle of the congregation before the \LORD; and shall lay his hand upon the bullock's head, and kill the bullock before the \LORD.
\verse And the priest that is anointed shall take of the bullock's blood, and bring it to the tabernacle of the congregation:
\verse And the priest shall dip his finger in the blood, and sprinkle of the blood seven times before the \LORD, before the vail of the sanctuary.
\verse And the priest shall put some of the blood upon the horns of the altar of sweet incense before the \LORD, which is in the tabernacle of the congregation; and shall pour all the blood of the bullock at the bottom of the altar of the burnt offering, which is at the door of the tabernacle of the congregation.
\verse And he shall take off from it all the fat of the bullock for the sin offering; the fat that covereth the inwards, and all the fat that is upon the inwards,
\verse And the two kidneys, and the fat that is upon them, which is by the flanks, and the caul above the liver, with the kidneys, it shall he take away,
\verse As it was taken off from the bullock of the sacrifice of peace offerings: and the priest shall burn them upon the altar of the burnt offering.
\verse And the skin of the bullock, and all his flesh, with his head, and with his legs, and his inwards, and his dung,
\verse Even the whole bullock shall he carry forth without the camp unto a clean place, where the ashes are poured out, and burn him on the wood with fire: where the ashes are poured out shall he be burnt.
\verse And if the whole congregation of Israel sin through ignorance, and the thing be hid from the eyes of the assembly, and they have done somewhat against any of the commandments of the \LORD concerning things which should not be done, and are guilty;
\verse When the sin, which they have sinned against it, is known, then the congregation shall offer a young bullock for the sin, and bring him before the tabernacle of the congregation.
\verse And the elders of the congregation shall lay their hands upon the head of the bullock before the \LORD: and the bullock shall be killed before the \LORD.
\verse And the priest that is anointed shall bring of the bullock's blood to the tabernacle of the congregation:
\verse And the priest shall dip his finger in some of the blood, and sprinkle it seven times before the \LORD, even before the vail.
\verse And he shall put some of the blood upon the horns of the altar which is before the \LORD, that is in the tabernacle of the congregation, and shall pour out all the blood at the bottom of the altar of the burnt offering, which is at the door of the tabernacle of the congregation.
\verse And he shall take all his fat from him, and burn it upon the altar.
\verse And he shall do with the bullock as he did with the bullock for a sin offering, so shall he do with this: and the priest shall make an atonement for them, and it shall be forgiven them.
\verse And he shall carry forth the bullock without the camp, and burn him as he burned the first bullock: it is a sin offering for the congregation.
\verse When a ruler hath sinned, and done somewhat through ignorance against any of the commandments of the \LORD his God concerning things which should not be done, and is guilty;
\verse Or if his sin, wherein he hath sinned, come to his knowledge; he shall bring his offering, a kid of the goats, a male without blemish:
\verse And he shall lay his hand upon the head of the goat, and kill it in the place where they kill the burnt offering before the \LORD: it is a sin offering.
\verse And the priest shall take of the blood of the sin offering with his finger, and put it upon the horns of the altar of burnt offering, and shall pour out his blood at the bottom of the altar of burnt offering.
\verse And he shall burn all his fat upon the altar, as the fat of the sacrifice of peace offerings: and the priest shall make an atonement for him as concerning his sin, and it shall be forgiven him.
\verse And if any one of the common people sin through ignorance, while he doeth somewhat against any of the commandments of the \LORD concerning things which ought not to be done, and be guilty;
\verse Or if his sin, which he hath sinned, come to his knowledge: then he shall bring his offering, a kid of the goats, a female without blemish, for his sin which he hath sinned.
\verse And he shall lay his hand upon the head of the sin offering, and slay the sin offering in the place of the burnt offering.
\verse And the priest shall take of the blood thereof with his finger, and put it upon the horns of the altar of burnt offering, and shall pour out all the blood thereof at the bottom of the altar.
\verse And he shall take away all the fat thereof, as the fat is taken away from off the sacrifice of peace offerings; and the priest shall burn it upon the altar for a sweet savour unto the \LORD; and the priest shall make an atonement for him, and it shall be forgiven him.
\verse And if he bring a lamb for a sin offering, he shall bring it a female without blemish.
\verse And he shall lay his hand upon the head of the sin offering, and slay it for a sin offering in the place where they kill the burnt offering.
\verse And the priest shall take of the blood of the sin offering with his finger, and put it upon the horns of the altar of burnt offering, and shall pour out all the blood thereof at the bottom of the altar:
\verse And he shall take away all the fat thereof, as the fat of the lamb is taken away from the sacrifice of the peace offerings; and the priest shall burn them upon the altar, according to the offerings made by fire unto the \LORD: and the priest shall make an atonement for his sin that he hath committed, and it shall be forgiven him.
\end{biblechapter}

\flushcolsend\columnbreak % layout hack

\begin{biblechapter} % Leviticus 5
\verse And if a soul sin, and hear the voice of swearing, and is a witness, whether he hath seen or known of it; if he do not utter it, then he shall bear his iniquity.
\verse Or if a soul touch any unclean thing, whether it be a carcase of an unclean beast, or a carcase of unclean cattle, or the carcase of unclean creeping things, and if it be hidden from him; he also shall be unclean, and guilty.
\verse Or if he touch the uncleanness of man, whatsoever uncleanness it be that a man shall be defiled withal, and it be hid from him; when he knoweth of it, then he shall be guilty.
\verse Or if a soul swear, pronouncing with his lips to do evil, or to do good, whatsoever it be that a man shall pronounce with an oath, and it be hid from him; when he knoweth of it, then he shall be guilty in one of these.
\verse And it shall be, when he shall be guilty in one of these things, that he shall confess that he hath sinned in that thing:
\verse And he shall bring his trespass offering unto the \LORD for his sin which he hath sinned, a female from the flock, a lamb or a kid of the goats, for a sin offering; and the priest shall make an atonement for him concerning his sin.
\verse And if he be not able to bring a lamb, then he shall bring for his trespass, which he hath committed, two turtledoves, or two young pigeons, unto the \LORD; one for a sin offering, and the other for a burnt offering.
\verse And he shall bring them unto the priest, who shall offer that which is for the sin offering first, and wring off his head from his neck, but shall not divide it asunder:
\verse And he shall sprinkle of the blood of the sin offering upon the side of the altar; and the rest of the blood shall be wrung out at the bottom of the altar: it is a sin offering.
\verse And he shall offer the second for a burnt offering, according to the manner: and the priest shall make an atonement for him for his sin which he hath sinned, and it shall be forgiven him.
\verse But if he be not able to bring two turtledoves, or two young pigeons, then he that sinned shall bring for his offering the tenth part of an ephah of fine flour for a sin offering; he shall put no oil upon it, neither shall he put any frankincense thereon: for it is a sin offering.
\verse Then shall he bring it to the priest, and the priest shall take his handful of it, even a memorial thereof, and burn it on the altar, according to the offerings made by fire unto the \LORD: it is a sin offering.
\verse And the priest shall make an atonement for him as touching his sin that he hath sinned in one of these, and it shall be forgiven him: and the remnant shall be the priest's, as a meat offering.
\vfill\columnbreak % layout hack
\verseWithHeading{The trespass offering} And the \LORD spake unto Moses, saying,
\verse If a soul commit a trespass, and sin through ignorance, in the holy things of the \LORD; then he shall bring for his trespass unto the \LORD a ram without blemish out of the flocks, with thy estimation by shekels of silver, after the shekel of the sanctuary, for a trespass offering:
\verse And he shall make amends for the harm that he hath done in the holy thing, and shall add the fifth part thereto, and give it unto the priest: and the priest shall make an atonement for him with the ram of the trespass offering, and it shall be forgiven him.
\verse And if a soul sin, and commit any of these things which are forbidden to be done by the commandments of the \LORD; though he wist it not, yet is he guilty, and shall bear his iniquity.
\verse And he shall bring a ram without blemish out of the flock, with thy estimation, for a trespass offering, unto the priest: and the priest shall make an atonement for him concerning his ignorance wherein he erred and wist it not, and it shall be forgiven him.
\verse It is a trespass offering: he hath certainly trespassed against the \LORD.
\end{biblechapter}

\begin{biblechapter} % Leviticus 6
\verse And the \LORD spake unto Moses, saying,
\verse If a soul sin, and commit a trespass against the \LORD, and lie unto his neighbour in that which was delivered him to keep, or in fellowship, or in a thing taken away by violence, or hath deceived his neighbour;
\verse Or have found that which was lost, and lieth concerning it, and sweareth falsely; in any of all these that a man doeth, sinning therein:
\verse Then it shall be, because he hath sinned, and is guilty, that he shall restore that which he took violently away, or the thing which he hath deceitfully gotten, or that which was delivered him to keep, or the lost thing which he found,
\verse Or all that about which he hath sworn falsely; he shall even restore it in the principal, and shall add the fifth part more thereto, and give it unto him to whom it appertaineth, in the day of his trespass offering.
\verse And he shall bring his trespass offering unto the \LORD, a ram without blemish out of the flock, with thy estimation, for a trespass offering, unto the priest:
\verse And the priest shall make an atonement for him before the \LORD: and it shall be forgiven him for any thing of all that he hath done in trespassing therein.
\verseWithHeading{The burnt offering} And the \LORD spake unto Moses, saying,
\verse Command Aaron and his sons, saying, This is the law of the burnt offering: It is the burnt offering, because of the burning upon the altar all night unto the morning, and the fire of the altar shall be burning in it.
\verse And the priest shall put on his linen garment, and his linen breeches shall he put upon his flesh, and take up the ashes which the fire hath consumed with the burnt offering on the altar, and he shall put them beside the altar.
\verse And he shall put off his garments, and put on other garments, and carry forth the ashes without the camp unto a clean place.
\verse And the fire upon the altar shall be burning in it; it shall not be put out: and the priest shall burn wood on it every morning, and lay the burnt offering in order upon it; and he shall burn thereon the fat of the peace offerings.
\verse The fire shall ever be burning upon the altar; it shall never go out.
\verseWithHeading{The meat offering} And this is the law of the meat offering: the sons of Aaron shall offer it before the \LORD, before the altar.
\verse And he shall take of it his handful, of the flour of the meat offering, and of the oil thereof, and all the frankincense which is upon the meat offering, and shall burn it upon the altar for a sweet savour, even the memorial of it, unto the \LORD.
\verse And the remainder thereof shall Aaron and his sons eat: with unleavened bread shall it be eaten in the holy place; in the court of the tabernacle of the congregation they shall eat it.
\verse It shall not be baken with leaven. I have given it unto them for their portion of my offerings made by fire; it is most holy, as is the sin offering, and as the trespass offering.
\verse All the males among the children of Aaron shall eat of it. It shall be a statute for ever in your generations concerning the offerings of the \LORD made by fire: every one that toucheth them shall be holy.
\verse And the \LORD spake unto Moses, saying,
\verse This is the offering of Aaron and of his sons, which they shall offer unto the \LORD in the day when he is anointed; the tenth part of an ephah of fine flour for a meat offering perpetual, half of it in the morning, and half thereof at night.
\verse In a pan it shall be made with oil; and when it is baken, thou shalt bring it in: and the baken pieces of the meat offering shalt thou offer for a sweet savour unto the \LORD.
\verse And the priest of his sons that is anointed in his stead shall offer it: it is a statute for ever unto the \LORD; it shall be wholly burnt.
\verse For every meat offering for the priest shall be wholly burnt: it shall not be eaten.
\verseWithHeading{The sin offering} And the \LORD spake unto Moses, saying,
\verse Speak unto Aaron and to his sons, saying, This is the law of the sin offering: In the place where the burnt offering is killed shall the sin offering be killed before the \LORD: it is most holy.
\verse The priest that offereth it for sin shall eat it: in the holy place shall it be eaten, in the court of the tabernacle of the congregation.
\verse Whatsoever shall touch the flesh thereof shall be holy: and when there is sprinkled of the blood thereof upon any garment, thou shalt wash that whereon it was sprinkled in the holy place.
\verse But the earthen vessel wherein it is sodden shall be broken: and if it be sodden in a brasen pot, it shall be both scoured, and rinsed in water.
\verse All the males among the priests shall eat thereof: it is most holy.
\verse And no sin offering, whereof any of the blood is brought into the tabernacle of the congregation to reconcile withal in the holy place, shall be eaten: it shall be burnt in the fire.
\end{biblechapter}

\begin{biblechapter} % Leviticus 7
\verseWithHeading{The trespass offering} Likewise this is the law of the trespass offering: it is most holy.
\verse In the place where they kill the burnt offering shall they kill the trespass offering: and the blood thereof shall he sprinkle round about upon the altar.
\verse And he shall offer of it all the fat thereof; the rump, and the fat that covereth the inwards,
\verse And the two kidneys, and the fat that is on them, which is by the flanks, and the caul that is above the liver, with the kidneys, it shall he take away:
\verse And the priest shall burn them upon the altar for an offering made by fire unto the \LORD: it is a trespass offering.
\verse Every male among the priests shall eat thereof: it shall be eaten in the holy place: it is most holy.
\verse As the sin offering is, so is the trespass offering: there is one law for them: the priest that maketh atonement therewith shall have it.
\verse And the priest that offereth any man's burnt offering, even the priest shall have to himself the skin of the burnt offering which he hath offered.
\verse And all the meat offering that is baken in the oven, and all that is dressed in the fryingpan, and in the pan, shall be the priest's that offereth it.
\verse And every meat offering, mingled with oil, and dry, shall all the sons of Aaron have, one as much as another.
\verseWithHeading{The peace offering} And this is the law of the sacrifice of peace offerings, which he shall offer unto the \LORD.
\verse If he offer it for a thanksgiving, then he shall offer with the sacrifice of thanksgiving unleavened cakes mingled with oil, and unleavened wafers anointed with oil, and cakes mingled with oil, of fine flour, fried.
\verse Besides the cakes, he shall offer for his offering leavened bread with the sacrifice of thanksgiving of his peace offerings.
\verse And of it he shall offer one out of the whole oblation for an heave offering unto the \LORD, and it shall be the priest's that sprinkleth the blood of the peace offerings.
\verse And the flesh of the sacrifice of his peace offerings for thanksgiving shall be eaten the same day that it is offered; he shall not leave any of it until the morning.
\verse But if the sacrifice of his offering be a vow, or a voluntary offering, it shall be eaten the same day that he offereth his sacrifice: and on the morrow also the remainder of it shall be eaten:
\verse But the remainder of the flesh of the sacrifice on the third day shall be burnt with fire.
\verse And if any of the flesh of the sacrifice of his peace offerings be eaten at all on the third day, it shall not be accepted, neither shall it be imputed unto him that offereth it: it shall be an abomination, and the soul that eateth of it shall bear his iniquity.
\verse And the flesh that toucheth any unclean thing shall not be eaten; it shall be burnt with fire: and as for the flesh, all that be clean shall eat thereof.
\verse But the soul that eateth of the flesh of the sacrifice of peace offerings, that pertain unto the \LORD, having his uncleanness upon him, even that soul shall be cut off from his people.
\verse Moreover the soul that shall touch any unclean thing, as the uncleanness of man, or any unclean beast, or any abominable unclean thing, and eat of the flesh of the sacrifice of peace offerings, which pertain unto the \LORD, even that soul shall be cut off from his people.
\verseWithHeading{Eating fat and blood forbidden} And the \LORD spake unto Moses, saying,
\verse Speak unto the children of Israel, saying, Ye shall eat no manner of fat, of ox, or of sheep, or of goat.
\verse And the fat of the beast that dieth of itself, and the fat of that which is torn with beasts, may be used in any other use: but ye shall in no wise eat of it.
\verse For whosoever eateth the fat of the beast, of which men offer an offering made by fire unto the \LORD, even the soul that eateth it shall be cut off from his people.
\verse Moreover ye shall eat no manner of blood, whether it be of fowl or of beast, in any of your dwellings.
\verse Whatsoever soul it be that eateth any manner of blood, even that soul shall be cut off from his people.
\verseWithHeading{The priests' share} And the \LORD spake unto Moses, saying,
\verse Speak unto the children of Israel, saying, He that offereth the sacrifice of his peace offerings unto the \LORD shall bring his oblation unto the \LORD of the sacrifice of his peace offerings.
\verse His own hands shall bring the offerings of the \LORD made by fire, the fat with the breast, it shall he bring, that the breast may be waved for a wave offering before the \LORD.
\verse And the priest shall burn the fat upon the altar: but the breast shall be Aaron's and his sons'.
\verse And the right shoulder shall ye give unto the priest for an heave offering of the sacrifices of your peace offerings.
\verse He among the sons of Aaron, that offereth the blood of the peace offerings, and the fat, shall have the right shoulder for his part.
\verse For the wave breast and the heave shoulder have I taken of the children of Israel from off the sacrifices of their peace offerings, and have given them unto Aaron the priest and unto his sons by a statute for ever from among the children of Israel.
\verse This is the portion of the anointing of Aaron, and of the anointing of his sons, out of the offerings of the \LORD made by fire, in the day when he presented them to minister unto the \LORD in the priest's office;
\verse Which the \LORD commanded to be given them of the children of Israel, in the day that he anointed them, by a statute for ever throughout their generations.
\verse This is the law of the burnt offering, of the meat offering, and of the sin offering, and of the trespass offering, and of the consecrations, and of the sacrifice of the peace offerings;
\verse Which the \LORD commanded Moses in mount Sinai, in the day that he commanded the children of Israel to offer their oblations unto the \LORD, in the wilderness of Sinai.
\end{biblechapter}

\begin{biblechapter} % Leviticus 8
\verseWithHeading{The ordination of Aaron and his \newline sons} And the \LORD spake unto Moses, saying,
\verse Take Aaron and his sons with him, and the garments, and the anointing oil, and a bullock for the sin offering, and two rams, and a basket of unleavened bread;
\verse And gather thou all the congregation together unto the door of the tabernacle of the congregation.
\verse And Moses did as the \LORD commanded him; and the assembly was gathered together unto the door of the tabernacle of the congregation.
\verse And Moses said unto the congregation, This is the thing which the \LORD commanded to be done.
\verse And Moses brought Aaron and his sons, and washed them with water.
\verse And he put upon him the coat, and girded him with the girdle, and clothed him with the robe, and put the ephod upon him, and he girded him with the curious girdle of the ephod, and bound it unto him therewith.
\verse And he put the breastplate upon him: also he put in the breastplate the Urim and the Thummim.
\verse And he put the mitre upon his head; also upon the mitre, even upon his forefront, did he put the golden plate, the holy crown; as the \LORD commanded Moses.
\verse And Moses took the anointing oil, and anointed the tabernacle and all that was therein, and sanctified them.
\verse And he sprinkled thereof upon the altar seven times, and anointed the altar and all his vessels, both the laver and his foot, to sanctify them.
\verse And he poured of the anointing oil upon Aaron's head, and anointed him, to sanctify him.
\verse And Moses brought Aaron's sons, and put coats upon them, and girded them with girdles, and put bonnets upon them; as the \LORD commanded Moses.
\verse And he brought the bullock for the sin offering: and Aaron and his sons laid their hands upon the head of the bullock for the sin offering.
\verse And he slew it; and Moses took the blood, and put it upon the horns of the altar round about with his finger, and purified the altar, and poured the blood at the bottom of the altar, and sanctified it, to make reconciliation upon it.
\verse And he took all the fat that was upon the inwards, and the caul above the liver, and the two kidneys, and their fat, and Moses burned it upon the altar.
\verse But the bullock, and his hide, his flesh, and his dung, he burnt with fire without the camp; as the \LORD commanded Moses.
\verse And he brought the ram for the burnt offering: and Aaron and his sons laid their hands upon the head of the ram.
\verse And he killed it; and Moses sprinkled the blood upon the altar round about.
\verse And he cut the ram into pieces; and Moses burnt the head, and the pieces, and the fat.
\verse And he washed the inwards and the legs in water; and Moses burnt the whole ram upon the altar: it was a burnt sacrifice for a sweet savour, and an offering made by fire unto the \LORD; as the \LORD commanded Moses.
\verse And he brought the other ram, the ram of consecration: and Aaron and his sons laid their hands upon the head of the ram.
\verse And he slew it; and Moses took of the blood of it, and put it upon the tip of Aaron's right ear, and upon the thumb of his right hand, and upon the great toe of his right foot.
\verse And he brought Aaron's sons, and Moses put of the blood upon the tip of their right ear, and upon the thumbs of their right hands, and upon the great toes of their right feet: and Moses sprinkled the blood upon the altar round about.
\verse And he took the fat, and the rump, and all the fat that was upon the inwards, and the caul above the liver, and the two kidneys, and their fat, and the right shoulder:
\verse And out of the basket of unleavened bread, that was before the \LORD, he took one unleavened cake, and a cake of oiled bread, and one wafer, and put them on the fat, and upon the right shoulder:
\verse And he put all upon Aaron's hands, and upon his sons' hands, and waved them for a wave offering before the \LORD.
\verse And Moses took them from off their hands, and burnt them on the altar upon the burnt offering: they were consecrations for a sweet savour: it is an offering made by fire unto the \LORD.
\verse And Moses took the breast, and waved it for a wave offering before the \LORD: for of the ram of consecration it was Moses' part; as the \LORD commanded Moses.
\verse And Moses took of the anointing oil, and of the blood which was upon the altar, and sprinkled it upon Aaron, and upon his garments, and upon his sons, and upon his sons' garments with him; and sanctified Aaron, and his garments, and his sons, and his sons' garments with him.
\verse And Moses said unto Aaron and to his sons, Boil the flesh at the door of the tabernacle of the congregation: and there eat it with the bread that is in the basket of consecrations, as I commanded, saying, Aaron and his sons shall eat it.
\verse And that which remaineth of the flesh and of the bread shall ye burn with fire.
\verse And ye shall not go out of the door of the tabernacle of the congregation in seven days, until the days of your consecration be at an end: for seven days shall he consecrate you.
\verse As he hath done this day, so the \LORD hath commanded to do, to make an atonement for you.
\verse Therefore shall ye abide at the door of the tabernacle of the congregation day and night seven days, and keep the charge of the \LORD, that ye die not: for so I am commanded.
\verse So Aaron and his sons did all things which the \LORD commanded by the hand of Moses.
\end{biblechapter}

\begin{biblechapter} % Leviticus 9
\verseWithHeading{The priests begin their ministry} And it came to pass on the eighth day, that Moses called Aaron and his sons, and the elders of Israel;
\verse And he said unto Aaron, Take thee a young calf for a sin offering, and a ram for a burnt offering, without blemish, and offer them before the \LORD.
\verse And unto the children of Israel thou shalt speak, saying, Take ye a kid of the goats for a sin offering; and a calf and a lamb, both of the first year, without blemish, for a burnt offering;
\verse Also a bullock and a ram for peace offerings, to sacrifice before the \LORD; and a meat offering mingled with oil: for to day the \LORD will appear unto you.
\verse And they brought that which Moses commanded before the tabernacle of the congregation: and all the congregation drew near and stood before the \LORD.
\verse And Moses said, This is the thing which the \LORD commanded that ye should do: and the glory of the \LORD shall appear unto you.
\verse And Moses said unto Aaron, Go unto the altar, and offer thy sin offering, and thy burnt offering, and make an atonement for thyself, and for the people: and offer the offering of the people, and make an atonement for them; as the \LORD commanded.
\verse Aaron therefore went unto the altar, and slew the calf of the sin offering, which was for himself.
\verse And the sons of Aaron brought the blood unto him: and he dipped his finger in the blood, and put it upon the horns of the altar, and poured out the blood at the bottom of the altar:
\verse But the fat, and the kidneys, and the caul above the liver of the sin offering, he burnt upon the altar; as the \LORD commanded Moses.
\verse And the flesh and the hide he burnt with fire without the camp.
\verse And he slew the burnt offering; and Aaron's sons presented unto him the blood, which he sprinkled round about upon the altar.
\verse And they presented the burnt offering unto him, with the pieces thereof, and the head: and he burnt them upon the altar.
\verse And he did wash the inwards and the legs, and burnt them upon the burnt offering on the altar.
\verse And he brought the people's offering, and took the goat, which was the sin offering for the people, and slew it, and offered it for sin, as the first.
\verse And he brought the burnt offering, and offered it according to the manner.
\verse And he brought the meat offering, and took an handful thereof, and burnt it upon the altar, beside the burnt sacrifice of the morning.
\verse He slew also the bullock and the ram for a sacrifice of peace offerings, which was for the people: and Aaron's sons presented unto him the blood, which he sprinkled upon the altar round about,
\verse And the fat of the bullock and of the ram, the rump, and that which covereth the inwards, and the kidneys, and the caul above the liver:
\verse And they put the fat upon the breasts, and he burnt the fat upon the altar:
\verse And the breasts and the right shoulder Aaron waved for a wave offering before the \LORD; as Moses commanded.
\verse And Aaron lifted up his hand toward the people, and blessed them, and came down from offering of the sin offering, and the burnt offering, and peace offerings.
\verse And Moses and Aaron went into the tabernacle of the congregation, and came out, and blessed the people: and the glory of the \LORD appeared unto all the people.
\verse And there came a fire out from before the \LORD, and consumed upon the altar the burnt offering and the fat: which when all the people saw, they shouted, and fell on their faces.
\end{biblechapter}

\begin{biblechapter} % Leviticus 10
\verseWithHeading{The death of Nadab and \newline Abihu} And Nadab and Abihu, the sons of Aaron, took either of them his censer, and put fire therein, and put incense thereon, and offered strange fire before the \LORD, which he commanded them not.
\verse And there went out fire from the \LORD, and devoured them, and they died before the \LORD.
\verse Then Moses said unto Aaron, This is it that the \LORD spake, saying, I will be sanctified in them that come nigh me, and before all the people I will be glorified. And Aaron held his peace.
\verse And Moses called Mishael and Elzaphan, the sons of Uzziel the uncle of Aaron, and said unto them, Come near, carry your brethren from before the sanctuary out of the camp.
\verse So they went near, and carried them in their coats out of the camp; as Moses had said.
\verse And Moses said unto Aaron, and unto Eleazar and unto Ithamar, his sons, Uncover not your heads, neither rend your clothes; lest ye die, and lest wrath come upon all the people: but let your brethren, the whole house of Israel, bewail the burning which the \LORD hath kindled.
\verse And ye shall not go out from the door of the tabernacle of the congregation, lest ye die: for the anointing oil of the \LORD is upon you. And they did according to the word of Moses.
\verse And the \LORD spake unto Aaron, saying,
\verse Do not drink wine nor strong drink, thou, nor thy sons with thee, when ye go into the tabernacle of the congregation, lest ye die: it shall be a statute for ever throughout your generations:
\verse And that ye may put difference between holy and unholy, and between unclean and clean;
\verse And that ye may teach the children of Israel all the statutes which the \LORD hath spoken unto them by the hand of Moses.
\verse And Moses spake unto Aaron, and unto Eleazar and unto Ithamar, his sons that were left, Take the meat offering that remaineth of the offerings of the \LORD made by fire, and eat it without leaven beside the altar: for it is most holy:
\verse And ye shall eat it in the holy place, because it is thy due, and thy sons' due, of the sacrifices of the \LORD made by fire: for so I am commanded.
\verse And the wave breast and heave shoulder shall ye eat in a clean place; thou, and thy sons, and thy daughters with thee: for they be thy due, and thy sons' due, which are given out of the sacrifices of peace offerings of the children of Israel.
\verse The heave shoulder and the wave breast shall they bring with the offerings made by fire of the fat, to wave it for a wave offering before the \LORD; and it shall be thine, and thy sons' with thee, by a statute for ever; as the \LORD hath commanded.
\verse And Moses diligently sought the goat of the sin offering, and, behold, it was burnt: and he was angry with Eleazar and Ithamar, the sons of Aaron which were left alive, saying,
\verse Wherefore have ye not eaten the sin offering in the holy place, seeing it is most holy, and God hath given it you to bear the iniquity of the congregation, to make atonement for them before the \LORD?
\verse Behold, the blood of it was not brought in within the holy place: ye should indeed have eaten it in the holy place, as I commanded.
\verse And Aaron said unto Moses, Behold, this day have they offered their sin offering and their burnt offering before the \LORD; and such things have befallen me: and if I had eaten the sin offering to day, should it have been accepted in the sight of the \LORD?
\verse And when Moses heard that, he was content.
\end{biblechapter}

\begin{biblechapter} % Leviticus 11
\verseWithHeading{Clean and unclean food} And the \LORD spake unto Moses and to Aaron, saying unto them,
\verse Speak unto the children of Israel, saying, These are the beasts which ye shall eat among all the beasts that are on the earth.
\verse Whatsoever parteth the hoof, and is clovenfooted, and cheweth the cud, among the beasts, that shall ye eat.
\verse Nevertheless these shall ye not eat of them that chew the cud, or of them that divide the hoof: as the camel, because he cheweth the cud, but divideth not the hoof; he is unclean unto you.
\verse And the coney, because he cheweth the cud, but divideth not the hoof; he is unclean unto you.
\verse And the hare, because he cheweth the cud, but divideth not the hoof; he is unclean unto you.
\verse And the swine, though he divide the hoof, and be clovenfooted, yet he cheweth not the cud; he is unclean to you.
\verse Of their flesh shall ye not eat, and their carcase shall ye not touch; they are unclean to you.
\verse These shall ye eat of all that are in the waters: whatsoever hath fins and scales in the waters, in the seas, and in the rivers, them shall ye eat.
\verse And all that have not fins and scales in the seas, and in the rivers, of all that move in the waters, and of any living thing which is in the waters, they shall be an abomination unto you:
\verse They shall be even an abomination unto you; ye shall not eat of their flesh, but ye shall have their carcases in abomination.
\verse Whatsoever hath no fins nor scales in the waters, that shall be an abomination unto you.
\verse And these are they which ye shall have in abomination among the fowls; they shall not be eaten, they are an abomination: the eagle, and the ossifrage, and the ospray,
\verse And the vulture, and the kite after his kind;
\verse Every raven after his kind;
\verse And the owl, and the night hawk, and the cuckow, and the hawk after his kind,
\verse And the little owl, and the cormorant, and the great owl,
\verse And the swan, and the pelican, and the gier eagle,
\verse And the stork, the heron after her kind, and the lapwing, and the bat.
\verse All fowls that creep, going upon all four, shall be an abomination unto you.
\verse Yet these may ye eat of every flying creeping thing that goeth upon all four, which have legs above their feet, to leap withal upon the earth;
\verse Even these of them ye may eat; the locust after his kind, and the bald locust after his kind, and the beetle after his kind, and the grasshopper after his kind.
\verse But all other flying creeping things, which have four feet, shall be an abomination unto you.
\verse And for these ye shall be unclean: whosoever toucheth the carcase of them shall be unclean until the even.
\verse And whosoever beareth ought of the carcase of them shall wash his clothes, and be unclean until the even.
\verse The carcases of every beast which divideth the hoof, and is not clovenfooted, nor cheweth the cud, are unclean unto you: every one that toucheth them shall be unclean.
\verse And whatsoever goeth upon his paws, among all manner of beasts that go on all four, those are unclean unto you: whoso toucheth their carcase shall be unclean until the even.
\verse And he that beareth the carcase of them shall wash his clothes, and be unclean until the even: they are unclean unto you.
\verse These also shall be unclean unto you among the creeping things that creep upon the earth; the weasel, and the mouse, and the tortoise after his kind,
\verse And the ferret, and the chameleon, and the lizard, and the snail, and the mole.
\verse These are unclean to you among all that creep: whosoever doth touch them, when they be dead, shall be unclean until the even.
\verse And upon whatsoever any of them, when they are dead, doth fall, it shall be unclean; whether it be any vessel of wood, or raiment, or skin, or sack, whatsoever vessel it be, wherein any work is done, it must be put into water, and it shall be unclean until the even; so it shall be cleansed.
\verse And every earthen vessel, whereinto any of them falleth, whatsoever is in it shall be unclean; and ye shall break it.
\verse Of all meat which may be eaten, that on which such water cometh shall be unclean: and all drink that may be drunk in every such vessel shall be unclean.
\verse And every thing whereupon any part of their carcase falleth shall be unclean; whether it be oven, or ranges for pots, they shall be broken down: for they are unclean, and shall be unclean unto you.
\verse Nevertheless a fountain or pit, wherein there is plenty of water, shall be clean: but that which toucheth their carcase shall be unclean.
\verse And if any part of their carcase fall upon any sowing seed which is to be sown, it shall be clean.
\verse But if any water be put upon the seed, and any part of their carcase fall thereon, it shall be unclean unto you.
\verse And if any beast, of which ye may eat, die; he that toucheth the carcase thereof shall be unclean until the even.
\verse And he that eateth of the carcase of it shall wash his clothes, and be unclean until the even: he also that beareth the carcase of it shall wash his clothes, and be unclean until the even.
\verse And every creeping thing that creepeth upon the earth shall be an abomination; it shall not be eaten.
\verse Whatsoever goeth upon the belly, and whatsoever goeth upon all four, or whatsoever hath more feet among all creeping things that creep upon the earth, them ye shall not eat; for they are an abomination.
\verse Ye shall not make yourselves abominable with any creeping thing that creepeth, neither shall ye make yourselves unclean with them, that ye should be defiled thereby.
\verse For I am the \LORD your God: ye shall therefore sanctify yourselves, and ye shall be holy; for I am holy: neither shall ye defile yourselves with any manner of creeping thing that creepeth upon the earth.
\verse For I am the \LORD that bringeth you up out of the land of Egypt, to be your God: ye shall therefore be holy, for I am holy.
\verse This is the law of the beasts, and of the fowl, and of every living creature that moveth in the waters, and of every creature that creepeth upon the earth:
\verse To make a difference between the unclean and the clean, and between the beast that may be eaten and the beast that may not be eaten.
\end{biblechapter}

\begin{biblechapter} % Leviticus 12
\verseWithHeading{Purification after childbirth} And the \LORD spake unto Moses, saying,
\verse Speak unto the children of Israel, saying, If a woman have conceived seed, and born a man child: then she shall be unclean seven days; according to the days of the separation for her infirmity shall she be unclean.
\verse And in the eighth day the flesh of his foreskin shall be circumcised.
\verse And she shall then continue in the blood of her purifying three and thirty days; she shall touch no hallowed thing, nor come into the sanctuary, until the days of her purifying be fulfilled.
\verse But if she bear a maid child, then she shall be unclean two weeks, as in her separation: and she shall continue in the blood of her purifying threescore and six days.
\verse And when the days of her purifying are fulfilled, for a son, or for a daughter, she shall bring a lamb of the first year for a burnt offering, and a young pigeon, or a turtledove, for a sin offering, unto the door of the tabernacle of the congregation, unto the priest:
\verse Who shall offer it before the \LORD, and make an atonement for her; and she shall be cleansed from the issue of her blood. This is the law for her that hath born a male or a female.
\verse And if she be not able to bring a lamb, then she shall bring two turtles, or two young pigeons; the one for the burnt offering, and the other for a sin offering: and the priest shall make an atonement for her, and she shall be clean.
\end{biblechapter}

\begin{biblechapter} % Leviticus 13
\verseWithHeading{Leprosy of the flesh} And the \LORD spake unto Moses and Aaron, saying,
\verse When a man shall have in the skin of his flesh a rising, a scab, or bright spot, and it be in the skin of his flesh like the plague of leprosy; then he shall be brought unto Aaron the priest, or unto one of his sons the priests:
\verse And the priest shall look on the plague in the skin of the flesh: and when the hair in the plague is turned white, and the plague in sight be deeper than the skin of his flesh, it is a plague of leprosy: and the priest shall look on him, and pronounce him unclean.
\verse If the bright spot be white in the skin of his flesh, and in sight be not deeper than the skin, and the hair thereof be not turned white; then the priest shall shut up him that hath the plague seven days:
\verse And the priest shall look on him the seventh day: and, behold, if the plague in his sight be at a stay, and the plague spread not in the skin; then the priest shall shut him up seven days more:
\verse And the priest shall look on him again the seventh day: and, behold, if the plague be somewhat dark, and the plague spread not in the skin, the priest shall pronounce him clean: it is but a scab: and he shall wash his clothes, and be clean.
\verse But if the scab spread much abroad in the skin, after that he hath been seen of the priest for his cleansing, he shall be seen of the priest again:
\verse And if the priest see that, behold, the scab spreadeth in the skin, then the priest shall pronounce him unclean: it is a leprosy.
\verse When the plague of leprosy is in a man, then he shall be brought unto the priest;
\verse And the priest shall see him: and, behold, if the rising be white in the skin, and it have turned the hair white, and there be quick raw flesh in the rising;
\verse It is an old leprosy in the skin of his flesh, and the priest shall pronounce him unclean, and shall not shut him up: for he is unclean.
\verse And if a leprosy break out abroad in the skin, and the leprosy cover all the skin of him that hath the plague from his head even to his foot, wheresoever the priest looketh;
\verse Then the priest shall consider: and, behold, if the leprosy have covered all his flesh, he shall pronounce him clean that hath the plague: it is all turned white: he is clean.
\verse But when raw flesh appeareth in him, he shall be unclean.
\verse And the priest shall see the raw flesh, and pronounce him to be unclean: for the raw flesh is unclean: it is a leprosy.
\verse Or if the raw flesh turn again, and be changed unto white, he shall come unto the priest;
\verse And the priest shall see him: and, behold, if the plague be turned into white; then the priest shall pronounce him clean that hath the plague: he is clean.
\verse The flesh also, in which, even in the skin thereof, was a boil, and is healed,
\verse And in the place of the boil there be a white rising, or a bright spot, white, and somewhat reddish, and it be shewed to the priest;
\verse And if, when the priest seeth it, behold, it be in sight lower than the skin, and the hair thereof be turned white; the priest shall pronounce him unclean: it is a plague of leprosy broken out of the boil.
\verse But if the priest look on it, and, behold, there be no white hairs therein, and if it be not lower than the skin, but be somewhat dark; then the priest shall shut him up seven days:
\verse And if it spread much abroad in the skin, then the priest shall pronounce him unclean: it is a plague.
\verse But if the bright spot stay in his place, and spread not, it is a burning boil; and the priest shall pronounce him clean.
\verse Or if there be any flesh, in the skin whereof there is a hot burning, and the quick flesh that burneth have a white bright spot, somewhat reddish, or white;
\verse Then the priest shall look upon it: and, behold, if the hair in the bright spot be turned white, and it be in sight deeper than the skin; it is a leprosy broken out of the burning: wherefore the priest shall pronounce him unclean: it is the plague of leprosy.
\verse But if the priest look on it, and, behold, there be no white hair in the bright spot, and it be no lower than the other skin, but be somewhat dark; then the priest shall shut him up seven days:
\verse And the priest shall look upon him the seventh day: and if it be spread much abroad in the skin, then the priest shall pronounce him unclean: it is the plague of leprosy.
\verse And if the bright spot stay in his place, and spread not in the skin, but it be somewhat dark; it is a rising of the burning, and the priest shall pronounce him clean: for it is an inflammation of the burning.
\verse If a man or woman have a plague upon the head or the beard;
\verse Then the priest shall see the plague: and, behold, if it be in sight deeper than the skin; and there be in it a yellow thin hair; then the priest shall pronounce him unclean: it is a dry scall, even a leprosy upon the head or beard.
\verse And if the priest look on the plague of the scall, and, behold, it be not in sight deeper than the skin, and that there is no black hair in it; then the priest shall shut up him that hath the plague of the scall seven days:
\verse And in the seventh day the priest shall look on the plague: and, behold, if the scall spread not, and there be in it no yellow hair, and the scall be not in sight deeper than the skin;
\verse He shall be shaven, but the scall shall he not shave; and the priest shall shut up him that hath the scall seven days more:
\verse And in the seventh day the priest shall look on the scall: and, behold, if the scall be not spread in the skin, nor be in sight deeper than the skin; then the priest shall pronounce him clean: and he shall wash his clothes, and be clean.
\verse But if the scall spread much in the skin after his cleansing;
\verse Then the priest shall look on him: and, behold, if the scall be spread in the skin, the priest shall not seek for yellow hair; he is unclean.
\verse But if the scall be in his sight at a stay, and that there is black hair grown up therein; the scall is healed, he is clean: and the priest shall pronounce him clean.
\verse If a man also or a woman have in the skin of their flesh bright spots, even white bright spots;
\verse Then the priest shall look: and, behold, if the bright spots in the skin of their flesh be darkish white; it is a freckled spot that groweth in the skin; he is clean.
\verse And the man whose hair is fallen off his head, he is bald; yet is he clean.
\verse And he that hath his hair fallen off from the part of his head toward his face, he is forehead bald: yet is he clean.
\verse And if there be in the bald head, or bald forehead, a white reddish sore; it is a leprosy sprung up in his bald head, or his bald forehead.
\verse Then the priest shall look upon it: and, behold, if the rising of the sore be white reddish in his bald head, or in his bald forehead, as the leprosy appeareth in the skin of the flesh;
\verse He is a leprous man, he is unclean: the priest shall pronounce him utterly unclean; his plague is in his head.
\verse And the leper in whom the plague is, his clothes shall be rent, and his head bare, and he shall put a covering upon his upper lip, and shall cry, Unclean, unclean.
\verse All the days wherein the plague shall be in him he shall be defiled; he is unclean: he shall dwell alone; without the camp shall his habitation be.
\verseWithHeading{Leprosy of the garments} The garment also that the plague of leprosy is in, whether it be a woollen garment, or a linen garment;
\verse Whether it be in the warp, or woof; of linen, or of woollen; whether in a skin, or in any thing made of skin;
\verse And if the plague be greenish or reddish in the garment, or in the skin, either in the warp, or in the woof, or in any thing of skin; it is a plague of leprosy, and shall be shewed unto the priest:
\verse And the priest shall look upon the plague, and shut up it that hath the plague seven days:
\verse And he shall look on the plague on the seventh day: if the plague be spread in the garment, either in the warp, or in the woof, or in a skin, or in any work that is made of skin; the plague is a fretting leprosy; it is unclean.
\verse He shall therefore burn that garment, whether warp or woof, in woollen or in linen, or any thing of skin, wherein the plague is: for it is a fretting leprosy; it shall be burnt in the fire.
\verse And if the priest shall look, and, behold, the plague be not spread in the garment, either in the warp, or in the woof, or in any thing of skin;
\verse Then the priest shall command that they wash the thing wherein the plague is, and he shall shut it up seven days more:
\verse And the priest shall look on the plague, after that it is washed: and, behold, if the plague have not changed his colour, and the plague be not spread; it is unclean; thou shalt burn it in the fire; it is fret inward, whether it be bare within or without.
\verse And if the priest look, and, behold, the plague be somewhat dark after the washing of it; then he shall rend it out of the garment, or out of the skin, or out of the warp, or out of the woof:
\verse And if it appear still in the garment, either in the warp, or in the woof, or in any thing of skin; it is a spreading plague: thou shalt burn that wherein the plague is with fire.
\verse And the garment, either warp, or woof, or whatsoever thing of skin it be, which thou shalt wash, if the plague be departed from them, then it shall be washed the second time, and shall be clean.
\verse This is the law of the plague of leprosy in a garment of woollen or linen, either in the warp, or woof, or any thing of skins, to pronounce it clean, or to pronounce it unclean.
\end{biblechapter}

\begin{biblechapter} % Leviticus 14
\verseWithHeading{Cleansing leprosy of the flesh} And the \LORD spake unto Moses, saying,
\verse This shall be the law of the leper in the day of his cleansing: He shall be brought unto the priest:
\verse And the priest shall go forth out of the camp; and the priest shall look, and, behold, if the plague of leprosy be healed in the leper;
\verse Then shall the priest command to take for him that is to be cleansed two birds alive and clean, and cedar wood, and scarlet, and hyssop:
\verse And the priest shall command that one of the birds be killed in an earthen vessel over running water:
\verse As for the living bird, he shall take it, and the cedar wood, and the scarlet, and the hyssop, and shall dip them and the living bird in the blood of the bird that was killed over the running water:
\verse And he shall sprinkle upon him that is to be cleansed from the leprosy seven times, and shall pronounce him clean, and shall let the living bird loose into the open field.
\verse And he that is to be cleansed shall wash his clothes, and shave off all his hair, and wash himself in water, that he may be clean: and after that he shall come into the camp, and shall tarry abroad out of his tent seven days.
\verse But it shall be on the seventh day, that he shall shave all his hair off his head and his beard and his eyebrows, even all his hair he shall shave off: and he shall wash his clothes, also he shall wash his flesh in water, and he shall be clean.
\verse And on the eighth day he shall take two he lambs without blemish, and one ewe lamb of the first year without blemish, and three tenth deals of fine flour for a meat offering, mingled with oil, and one log of oil.
\verse And the priest that maketh him clean shall present the man that is to be made clean, and those things, before the \LORD, at the door of the tabernacle of the congregation:
\verse And the priest shall take one he lamb, and offer him for a trespass offering, and the log of oil, and wave them for a wave offering before the \LORD:
\verse And he shall slay the lamb in the place where he shall kill the sin offering and the burnt offering, in the holy place: for as the sin offering is the priest's, so is the trespass offering: it is most holy:
\verse And the priest shall take some of the blood of the trespass offering, and the priest shall put it upon the tip of the right ear of him that is to be cleansed, and upon the thumb of his right hand, and upon the great toe of his right foot:
\verse And the priest shall take some of the log of oil, and pour it into the palm of his own left hand:
\verse And the priest shall dip his right finger in the oil that is in his left hand, and shall sprinkle of the oil with his finger seven times before the \LORD:
\verse And of the rest of the oil that is in his hand shall the priest put upon the tip of the right ear of him that is to be cleansed, and upon the thumb of his right hand, and upon the great toe of his right foot, upon the blood of the trespass offering:
\verse And the remnant of the oil that is in the priest's hand he shall pour upon the head of him that is to be cleansed: and the priest shall make an atonement for him before the \LORD.
\verse And the priest shall offer the sin offering, and make an atonement for him that is to be cleansed from his uncleanness; and afterward he shall kill the burnt offering:
\verse And the priest shall offer the burnt offering and the meat offering upon the altar: and the priest shall make an atonement for him, and he shall be clean.
\verse And if he be poor, and cannot get so much; then he shall take one lamb for a trespass offering to be waved, to make an atonement for him, and one tenth deal of fine flour mingled with oil for a meat offering, and a log of oil;
\verse And two turtledoves, or two young pigeons, such as he is able to get; and the one shall be a sin offering, and the other a burnt offering.
\verse And he shall bring them on the eighth day for his cleansing unto the priest, unto the door of the tabernacle of the congregation, before the \LORD.
\verse And the priest shall take the lamb of the trespass offering, and the log of oil, and the priest shall wave them for a wave offering before the \LORD:
\verse And he shall kill the lamb of the trespass offering, and the priest shall take some of the blood of the trespass offering, and put it upon the tip of the right ear of him that is to be cleansed, and upon the thumb of his right hand, and upon the great toe of his right foot:
\verse And the priest shall pour of the oil into the palm of his own left hand:
\verse And the priest shall sprinkle with his right finger some of the oil that is in his left hand seven times before the \LORD:
\verse And the priest shall put of the oil that is in his hand upon the tip of the right ear of him that is to be cleansed, and upon the thumb of his right hand, and upon the great toe of his right foot, upon the place of the blood of the trespass offering:
\verse And the rest of the oil that is in the priest's hand he shall put upon the head of him that is to be cleansed, to make an atonement for him before the \LORD.
\verse And he shall offer the one of the turtledoves, or of the young pigeons, such as he can get;
\verse Even such as he is able to get, the one for a sin offering, and the other for a burnt offering, with the meat offering: and the priest shall make an atonement for him that is to be cleansed before the \LORD.
\verse This is the law of him in whom is the plague of leprosy, whose hand is not able to get that which pertaineth to his cleansing.
\verseWithHeading{Cleansing leprosy of the garments} And the \LORD spake unto Moses and unto Aaron, saying,
\verse When ye be come into the land of Canaan, which I give to you for a possession, and I put the plague of leprosy in a house of the land of your possession;
\verse And he that owneth the house shall come and tell the priest, saying, It seemeth to me there is as it were a plague in the house:
\verse Then the priest shall command that they empty the house, before the priest go into it to see the plague, that all that is in the house be not made unclean: and afterward the priest shall go in to see the house:
\verse And he shall look on the plague, and, behold, if the plague be in the walls of the house with hollow strakes, greenish or reddish, which in sight are lower than the wall;
\verse Then the priest shall go out of the house to the door of the house, and shut up the house seven days:
\verse And the priest shall come again the seventh day, and shall look: and, behold, if the plague be spread in the walls of the house;
\verse Then the priest shall command that they take away the stones in which the plague is, and they shall cast them into an unclean place without the city:
\verse And he shall cause the house to be scraped within round about, and they shall pour out the dust that they scrape off without the city into an unclean place:
\verse And they shall take other stones, and put them in the place of those stones; and he shall take other morter, and shall plaister the house.
\verse And if the plague come again, and break out in the house, after that he hath taken away the stones, and after he hath scraped the house, and after it is plaistered;
\verse Then the priest shall come and look, and, behold, if the plague be spread in the house, it is a fretting leprosy in the house: it is unclean.
\verse And he shall break down the house, the stones of it, and the timber thereof, and all the morter of the house; and he shall carry them forth out of the city into an unclean place.
\verse Moreover he that goeth into the house all the while that it is shut up shall be unclean until the even.
\verse And he that lieth in the house shall wash his clothes; and he that eateth in the house shall wash his clothes.
\verse And if the priest shall come in, and look upon it, and, behold, the plague hath not spread in the house, after the house was plaistered: then the priest shall pronounce the house clean, because the plague is healed.
\verse And he shall take to cleanse the house two birds, and cedar wood, and scarlet, and hyssop:
\verse And he shall kill the one of the birds in an earthen vessel over running water:
\verse And he shall take the cedar wood, and the hyssop, and the scarlet, and the living bird, and dip them in the blood of the slain bird, and in the running water, and sprinkle the house seven times:
\verse And he shall cleanse the house with the blood of the bird, and with the running water, and with the living bird, and with the cedar wood, and with the hyssop, and with the scarlet:
\verse But he shall let go the living bird out of the city into the open fields, and make an atonement for the house: and it shall be clean.
\verse This is the law for all manner of plague of leprosy, and scall,
\verse And for the leprosy of a garment, and of a house,
\verse And for a rising, and for a scab, and for a bright spot:
\verse To teach when it is unclean, and when it is clean: this is the law of leprosy.
\end{biblechapter}

\begin{biblechapter} % Leviticus 15
\verseWithHeading{Uncleanliness of issue} And the \LORD spake unto Moses and to Aaron, saying,
\verse Speak unto the children of Israel, and say unto them, When any man hath a running issue out of his flesh, because of his issue he is unclean.
\verse And this shall be his uncleanness in his issue: whether his flesh run with his issue, or his flesh be stopped from his issue, it is his uncleanness.
\verse Every bed, whereon he lieth that hath the issue, is unclean: and every thing, whereon he sitteth, shall be unclean.
\verse And whosoever toucheth his bed shall wash his clothes, and bathe himself in water, and be unclean until the even.
\verse And he that sitteth on any thing whereon he sat that hath the issue shall wash his clothes, and bathe himself in water, and be unclean until the even.
\verse And he that toucheth the flesh of him that hath the issue shall wash his clothes, and bathe himself in water, and be unclean until the even.
\verse And if he that hath the issue spit upon him that is clean; then he shall wash his clothes, and bathe himself in water, and be unclean until the even.
\verse And what saddle soever he rideth upon that hath the issue shall be unclean.
\verse And whosoever toucheth any thing that was under him shall be unclean until the even: and he that beareth any of those things shall wash his clothes, and bathe himself in water, and be unclean until the even.
\verse And whomsoever he toucheth that hath the issue, and hath not rinsed his hands in water, he shall wash his clothes, and bathe himself in water, and be unclean until the even.
\verse And the vessel of earth, that he toucheth which hath the issue, shall be broken: and every vessel of wood shall be rinsed in water.
\verse And when he that hath an issue is cleansed of his issue; then he shall number to himself seven days for his cleansing, and wash his clothes, and bathe his flesh in running water, and shall be clean.
\verse And on the eighth day he shall take to him two turtledoves, or two young pigeons, and come before the \LORD unto the door of the tabernacle of the congregation, and give them unto the priest:
\verse And the priest shall offer them, the one for a sin offering, and the other for a burnt offering; and the priest shall make an atonement for him before the \LORD for his issue.
\verse And if any man's seed of copulation go out from him, then he shall wash all his flesh in water, and be unclean until the even.
\verse And every garment, and every skin, whereon is the seed of copulation, shall be washed with water, and be unclean until the even.
\verse The woman also with whom man shall lie with seed of copulation, they shall both bathe themselves in water, and be unclean until the even.
\verse And if a woman have an issue, and her issue in her flesh be blood, she shall be put apart seven days: and whosoever toucheth her shall be unclean until the even.
\verse And every thing that she lieth upon in her separation shall be unclean: every thing also that she sitteth upon shall be unclean.
\verse And whosoever toucheth her bed shall wash his clothes, and bathe himself in water, and be unclean until the even.
\verse And whosoever toucheth any thing that she sat upon shall wash his clothes, and bathe himself in water, and be unclean until the even.
\verse And if it be on her bed, or on any thing whereon she sitteth, when he toucheth it, he shall be unclean until the even.
\verse And if any man lie with her at all, and her flowers be upon him, he shall be unclean seven days; and all the bed whereon he lieth shall be unclean.
\verse And if a woman have an issue of her blood many days out of the time of her separation, or if it run beyond the time of her separation; all the days of the issue of her uncleanness shall be as the days of her separation: she shall be unclean.
\verse Every bed whereon she lieth all the days of her issue shall be unto her as the bed of her separation: and whatsoever she sitteth upon shall be unclean, as the uncleanness of her separation.
\verse And whosoever toucheth those things shall be unclean, and shall wash his clothes, and bathe himself in water, and be unclean until the even.
\verse But if she be cleansed of her issue, then she shall number to herself seven days, and after that she shall be clean.
\verse And on the eighth day she shall take unto her two turtles, or two young pigeons, and bring them unto the priest, to the door of the tabernacle of the congregation.
\verse And the priest shall offer the one for a sin offering, and the other for a burnt offering; and the priest shall make an atonement for her before the \LORD for the issue of her uncleanness.
\verse Thus shall ye separate the children of Israel from their uncleanness; that they die not in their uncleanness, when they defile my tabernacle that is among them.
\verse This is the law of him that hath an issue, and of him whose seed goeth from him, and is defiled therewith;
\verse And of her that is sick of her flowers, and of him that hath an issue, of the man, and of the woman, and of him that lieth with her that is unclean.
\end{biblechapter}

\begin{biblechapter} % Leviticus 16
\verseWithHeading{The Day of Atonement} And the \LORD spake unto Moses after the death of the two sons of Aaron, when they offered before the \LORD, and died;
\verse And the \LORD said unto Moses, Speak unto Aaron thy brother, that he come not at all times into the holy place within the vail before the mercy seat, which is upon the ark; that he die not: for I will appear in the cloud upon the mercy seat.
\verse Thus shall Aaron come into the holy place: with a young bullock for a sin offering, and a ram for a burnt offering.
\verse He shall put on the holy linen coat, and he shall have the linen breeches upon his flesh, and shall be girded with a linen girdle, and with the linen mitre shall he be attired: these are holy garments; therefore shall he wash his flesh in water, and so put them on.
\verse And he shall take of the congregation of the children of Israel two kids of the goats for a sin offering, and one ram for a burnt offering.
\verse And Aaron shall offer his bullock of the sin offering, which is for himself, and make an atonement for himself, and for his house.
\verse And he shall take the two goats, and present them before the \LORD at the door of the tabernacle of the congregation.
\verse And Aaron shall cast lots upon the two goats; one lot for the \LORD, and the other lot for the scapegoat.
\verse And Aaron shall bring the goat upon which the \LORDs lot fell, and offer him for a sin offering.
\verse But the goat, on which the lot fell to be the scapegoat, shall be presented alive before the \LORD, to make an atonement with him, and to let him go for a scapegoat into the wilderness.
\verse And Aaron shall bring the bullock of the sin offering, which is for himself, and shall make an atonement for himself, and for his house, and shall kill the bullock of the sin offering which is for himself:
\verse And he shall take a censer full of burning coals of fire from off the altar before the \LORD, and his hands full of sweet incense beaten small, and bring it within the vail:
\verse And he shall put the incense upon the fire before the \LORD, that the cloud of the incense may cover the mercy seat that is upon the testimony, that he die not:
\verse And he shall take of the blood of the bullock, and sprinkle it with his finger upon the mercy seat eastward; and before the mercy seat shall he sprinkle of the blood with his finger seven times.
\verse Then shall he kill the goat of the sin offering, that is for the people, and bring his blood within the vail, and do with that blood as he did with the blood of the bullock, and sprinkle it upon the mercy seat, and before the mercy seat:
\verse And he shall make an atonement for the holy place, because of the uncleanness of the children of Israel, and because of their transgressions in all their sins: and so shall he do for the tabernacle of the congregation, that remaineth among them in the midst of their uncleanness.
\verse And there shall be no man in the tabernacle of the congregation when he goeth in to make an atonement in the holy place, until he come out, and have made an atonement for himself, and for his household, and for all the congregation of Israel.
\verse And he shall go out unto the altar that is before the \LORD, and make an atonement for it; and shall take of the blood of the bullock, and of the blood of the goat, and put it upon the horns of the altar round about.
\verse And he shall sprinkle of the blood upon it with his finger seven times, and cleanse it, and hallow it from the uncleanness of the children of Israel.
\verse And when he hath made an end of reconciling the holy place, and the tabernacle of the congregation, and the altar, he shall bring the live goat:
\verse And Aaron shall lay both his hands upon the head of the live goat, and confess over him all the iniquities of the children of Israel, and all their transgressions in all their sins, putting them upon the head of the goat, and shall send him away by the hand of a fit man into the wilderness:
\verse And the goat shall bear upon him all their iniquities unto a land not inhabited: and he shall let go the goat in the wilderness.
\verse And Aaron shall come into the tabernacle of the congregation, and shall put off the linen garments, which he put on when he went into the holy place, and shall leave them there:
\verse And he shall wash his flesh with water in the holy place, and put on his garments, and come forth, and offer his burnt offering, and the burnt offering of the people, and make an atonement for himself, and for the people.
\verse And the fat of the sin offering shall he burn upon the altar.
\verse And he that let go the goat for the scapegoat shall wash his clothes, and bathe his flesh in water, and afterward come into the camp.
\verse And the bullock for the sin offering, and the goat for the sin offering, whose blood was brought in to make atonement in the holy place, shall one carry forth without the camp; and they shall burn in the fire their skins, and their flesh, and their dung.
\verse And he that burneth them shall wash his clothes, and bathe his flesh in water, and afterward he shall come into the camp.
\verse And this shall be a statute for ever unto you: that in the seventh month, on the tenth day of the month, ye shall afflict your souls, and do no work at all, whether it be one of your own country, or a stranger that sojourneth among you:
\verse For on that day shall the priest make an atonement for you, to cleanse you, that ye may be clean from all your sins before the \LORD.
\verse It shall be a Sabbath of rest unto you, and ye shall afflict your souls, by a statute for ever.
\verse And the priest, whom he shall anoint, and whom he shall consecrate to minister in the priest's office in his father's stead, shall make the atonement, and shall put on the linen clothes, even the holy garments:
\verse And he shall make an atonement for the holy sanctuary, and he shall make an atonement for the tabernacle of the congregation, and for the altar, and he shall make an atonement for the priests, and for all the people of the congregation.
\verse And this shall be an everlasting statute unto you, to make an atonement for the children of Israel for all their sins once a year. And he did as the \LORD commanded Moses.
\end{biblechapter}

\begin{biblechapter} % Leviticus 17
\verseWithHeading{Eating blood forbidden} And the \LORD spake unto Moses, saying,
\verse Speak unto Aaron, and unto his sons, and unto all the children of Israel, and say unto them; This is the thing which the \LORD hath commanded, saying,
\verse What man soever there be of the house of Israel, that killeth an ox, or lamb, or goat, in the camp, or that killeth it out of the camp,
\verse And bringeth it not unto the door of the tabernacle of the congregation, to offer an offering unto the \LORD before the tabernacle of the \LORD; blood shall be imputed unto that man; he hath shed blood; and that man shall be cut off from among his people:
\verse To the end that the children of Israel may bring their sacrifices, which they offer in the open field, even that they may bring them unto the \LORD, unto the door of the tabernacle of the congregation, unto the priest, and offer them for peace offerings unto the \LORD.
\verse And the priest shall sprinkle the blood upon the altar of the \LORD at the door of the tabernacle of the congregation, and burn the fat for a sweet savour unto the \LORD.
\verse And they shall no more offer their sacrifices unto devils, after whom they have gone a whoring. This shall be a statute for ever unto them throughout their generations.
\verse And thou shalt say unto them, Whatsoever man there be of the house of Israel, or of the strangers which sojourn among you, that offereth a burnt offering or sacrifice,
\verse And bringeth it not unto the door of the tabernacle of the congregation, to offer it unto the \LORD; even that man shall be cut off from among his people.
\verse And whatsoever man there be of the house of Israel, or of the strangers that sojourn among you, that eateth any manner of blood; I will even set my face against that soul that eateth blood, and will cut him off from among his people.
\verse For the life of the flesh is in the blood: and I have given it to you upon the altar to make an atonement for your souls: for it is the blood that maketh an atonement for the soul.
\verse Therefore I said unto the children of Israel, No soul of you shall eat blood, neither shall any stranger that sojourneth among you eat blood.
\verse And whatsoever man there be of the children of Israel, or of the strangers that sojourn among you, which hunteth and catcheth any beast or fowl that may be eaten; he shall even pour out the blood thereof, and cover it with dust.
\verse For it is the life of all flesh; the blood of it is for the life thereof: therefore I said unto the children of Israel, Ye shall eat the blood of no manner of flesh: for the life of all flesh is the blood thereof: whosoever eateth it shall be cut off.
\verse And every soul that eateth that which died of itself, or that which was torn with beasts, whether it be one of your own country, or a stranger, he shall both wash his clothes, and bathe himself in water, and be unclean until the even: then shall he be clean.
\verse But if he wash them not, nor bathe his flesh; then he shall bear his iniquity.
\end{biblechapter}

\begin{biblechapter} % Leviticus 18
\verseWithHeading{Unlawful nakedness} And the \LORD spake unto Moses, saying,
\verse Speak unto the children of Israel, and say unto them, I am the \LORD your God.
\verse After the doings of the land of Egypt, wherein ye dwelt, shall ye not do: and after the doings of the land of Canaan, whither I bring you, shall ye not do: neither shall ye walk in their ordinances.
\verse Ye shall do my judgments, and keep mine ordinances, to walk therein: I am the \LORD your God.
\verse Ye shall therefore keep my statutes, and my judgments: which if a man do, he shall live in them: I am the \LORD.
\verse None of you shall approach to any that is near of kin to him, to uncover their nakedness: I am the \LORD.
\verse The nakedness of thy father, or the nakedness of thy mother, shalt thou not uncover: she is thy mother; thou shalt not uncover her nakedness.
\verse The nakedness of thy father's wife shalt thou not uncover: it is thy father's nakedness.
\verse The nakedness of thy sister, the daughter of thy father, or daughter of thy mother, whether she be born at home, or born abroad, even their nakedness thou shalt not uncover.
\verse The nakedness of thy son's daughter, or of thy daughter's daughter, even their nakedness thou shalt not uncover: for theirs is thine own nakedness.
\verse The nakedness of thy father's wife's daughter, begotten of thy father, she is thy sister, thou shalt not uncover her nakedness.
\verse Thou shalt not uncover the nakedness of thy father's sister: she is thy father's near kinswoman.
\verse Thou shalt not uncover the nakedness of thy mother's sister: for she is thy mother's near kinswoman.
\verse Thou shalt not uncover the nakedness of thy father's brother, thou shalt not approach to his wife: she is thine aunt.
\verse Thou shalt not uncover the nakedness of thy daughter in law: she is thy son's wife; thou shalt not uncover her nakedness.
\verse Thou shalt not uncover the nakedness of thy brother's wife: it is thy brother's nakedness.
\verse Thou shalt not uncover the nakedness of a woman and her daughter, neither shalt thou take her son's daughter, or her daughter's daughter, to uncover her nakedness; for they are her near kinswomen: it is wickedness.
\verse Neither shalt thou take a wife to her sister, to vex her, to uncover her nakedness, beside the other in her life time.
\verse Also thou shalt not approach unto a woman to uncover her nakedness, as long as she is put apart for her uncleanness.
\verse Moreover thou shalt not lie carnally with thy neighbour's wife, to defile thyself with her.
\verse And thou shalt not let any of thy seed pass through the fire to Molech, neither shalt thou profane the name of thy God: I am the \LORD.
\verse Thou shalt not lie with mankind, as with womankind: it is abomination.
\verse Neither shalt thou lie with any beast to defile thyself therewith: neither shall any woman stand before a beast to lie down thereto: it is confusion.
\verse Defile not ye yourselves in any of these things: for in all these the nations are defiled which I cast out before you:
\verse And the land is defiled: therefore I do visit the iniquity thereof upon it, and the land itself vomiteth out her inhabitants.
\verse Ye shall therefore keep my statutes and my judgments, and shall not commit any of these abominations; neither any of your own nation, nor any stranger that sojourneth among you:
\verse (For all these abominations have the men of the land done, which were before you, and the land is defiled;)
\verse That the land spue not you out also, when ye defile it, as it spued out the nations that were before you.
\verse For whosoever shall commit any of these abominations, even the souls that commit them shall be cut off from among their people.
\verse Therefore shall ye keep mine ordinance, that ye commit not any one of these abominable customs, which were committed before you, and that ye defile not yourselves therein: I am the \LORD your God.
\end{biblechapter}

\columnbreak % layout hack

\begin{biblechapter} % Leviticus 19
\verseWithHeading{Various laws} And the \LORD spake unto Moses, saying,
\verse Speak unto all the congregation of the children of Israel, and say unto them, Ye shall be holy: for I the \LORD your God am holy.
\verse Ye shall fear every man his mother, and his father, and keep my Sabbaths: I am the \LORD your God.
\verse Turn ye not unto idols, nor make to yourselves molten gods: I am the \LORD your God.
\verse And if ye offer a sacrifice of peace offerings unto the \LORD, ye shall offer it at your own will.
\verse It shall be eaten the same day ye offer it, and on the morrow: and if ought remain until the third day, it shall be burnt in the fire.
\verse And if it be eaten at all on the third day, it is abominable; it shall not be accepted.
\verse Therefore every one that eateth it shall bear his iniquity, because he hath profaned the hallowed thing of the \LORD: and that soul shall be cut off from among his people.
\verse And when ye reap the harvest of your land, thou shalt not wholly reap the corners of thy field, neither shalt thou gather the gleanings of thy harvest.
\verse And thou shalt not glean thy vineyard, neither shalt thou gather every grape of thy vineyard; thou shalt leave them for the poor and stranger: I am the \LORD your God.
\verse Ye shall not steal, neither deal falsely, neither lie one to another.
\verse And ye shall not swear by my name falsely, neither shalt thou profane the name of thy God: I am the \LORD.
\verse Thou shalt not defraud thy neighbour, neither rob him: the wages of him that is hired shall not abide with thee all night until the morning.
\verse Thou shalt not curse the deaf, nor put a stumblingblock before the blind, but shalt fear thy God: I am the \LORD.
\verse Ye shall do no unrighteousness in judgment: thou shalt not respect the person of the poor, nor honour the person of the mighty: but in righteousness shalt thou judge thy neighbour.
\verse Thou shalt not go up and down as a talebearer among thy people: neither shalt thou stand against the blood of thy neighbour: I am the \LORD.
\verse Thou shalt not hate thy brother in thine heart: thou shalt in any wise rebuke thy neighbour, and not suffer sin upon him.
\verse Thou shalt not avenge, nor bear any grudge against the children of thy people, but thou shalt love thy neighbour as thyself: I am the \LORD.
\verse Ye shall keep my statutes. Thou shalt not let thy cattle gender with a diverse kind: thou shalt not sow thy field with mingled seed: neither shall a garment mingled of linen and woollen come upon thee.
\verse And whosoever lieth carnally with a woman, that is a bondmaid, betrothed to an husband, and not at all redeemed, nor freedom given her; she shall be scourged; they shall not be put to death, because she was not free.
\verse And he shall bring his trespass offering unto the \LORD, unto the door of the tabernacle of the congregation, even a ram for a trespass offering.
\verse And the priest shall make an atonement for him with the ram of the trespass offering before the \LORD for his sin which he hath done: and the sin which he hath done shall be forgiven him.
\verse And when ye shall come into the land, and shall have planted all manner of trees for food, then ye shall count the fruit thereof as uncircumcised: three years shall it be as uncircumcised unto you: it shall not be eaten of.
\verse But in the fourth year all the fruit thereof shall be holy to praise the \LORD withal.
\verse And in the fifth year shall ye eat of the fruit thereof, that it may yield unto you the increase thereof: I am the \LORD your God.
\verse Ye shall not eat any thing with the blood: neither shall ye use enchantment, nor observe times.
\verse Ye shall not round the corners of your heads, neither shalt thou mar the corners of thy beard.
\verse Ye shall not make any cuttings in your flesh for the dead, nor print any marks upon you: I am the \LORD.
\verse Do not prostitute thy daughter, to cause her to be a whore; lest the land fall to whoredom, and the land become full of wickedness.
\verse Ye shall keep my Sabbaths, and reverence my sanctuary: I am the \LORD.
\verse Regard not them that have familiar spirits, neither seek after wizards, to be defiled by them: I am the \LORD your God.
\verse Thou shalt rise up before the hoary head, and honour the face of the old man, and fear thy God: I am the \LORD.
\verse And if a stranger sojourn with thee in your land, ye shall not vex him.
\verse But the stranger that dwelleth with you shall be unto you as one born among you, and thou shalt love him as thyself; for ye were strangers in the land of Egypt: I am the \LORD your God.
\verse Ye shall do no unrighteousness in judgment, in meteyard, in weight, or in measure.
\verse Just balances, just weights, a just ephah, and a just hin, shall ye have: I am the \LORD your God, which brought you out of the land of Egypt.
\verse Therefore shall ye observe all my statutes, and all my judgments, and do them: I am the \LORD.
\end{biblechapter}

\begin{biblechapter} % Leviticus 20
\verseWithHeading{Punishments for sin} And the \LORD spake unto Moses, saying,
\verse Again, thou shalt say to the children of Israel, Whosoever he be of the children of Israel, or of the strangers that sojourn in Israel, that giveth any of his seed unto Molech; he shall surely be put to death: the people of the land shall stone him with stones.
\verse And I will set my face against that man, and will cut him off from among his people; because he hath given of his seed unto Molech, to defile my sanctuary, and to profane my holy name.
\verse And if the people of the land do any ways hide their eyes from the man, when he giveth of his seed unto Molech, and kill him not:
\verse Then I will set my face against that man, and against his family, and will cut him off, and all that go a whoring after him, to commit whoredom with Molech, from among their people.
\verse And the soul that turneth after such as have familiar spirits, and after wizards, to go a whoring after them, I will even set my face against that soul, and will cut him off from among his people.
\verse Sanctify yourselves therefore, and be ye holy: for I am the \LORD your God.
\verse And ye shall keep my statutes, and do them: I am the \LORD which sanctify you.
\verse For every one that curseth his father or his mother shall be surely put to death: he hath cursed his father or his mother; his blood shall be upon him.
\verse And the man that committeth adultery with another man's wife, even he that committeth adultery with his neighbour's wife, the adulterer and the adulteress shall surely be put to death.
\verse And the man that lieth with his father's wife hath uncovered his father's nakedness: both of them shall surely be put to death; their blood shall be upon them.
\verse And if a man lie with his daughter in law, both of them shall surely be put to death: they have wrought confusion; their blood shall be upon them.
\verse If a man also lie with mankind, as he lieth with a woman, both of them have committed an abomination: they shall surely be put to death; their blood shall be upon them.
\verse And if a man take a wife and her mother, it is wickedness: they shall be burnt with fire, both he and they; that there be no wickedness among you.
\verse And if a man lie with a beast, he shall surely be put to death: and ye shall slay the beast.
\verse And if a woman approach unto any beast, and lie down thereto, thou shalt kill the woman, and the beast: they shall surely be put to death; their blood shall be upon them.
\verse And if a man shall take his sister, his father's daughter, or his mother's daughter, and see her nakedness, and she see his nakedness; it is a wicked thing; and they shall be cut off in the sight of their people: he hath uncovered his sister's nakedness; he shall bear his iniquity.
\verse And if a man shall lie with a woman having her sickness, and shall uncover her nakedness; he hath discovered her fountain, and she hath uncovered the fountain of her blood: and both of them shall be cut off from among their people.
\verse And thou shalt not uncover the nakedness of thy mother's sister, nor of thy father's sister: for he uncovereth his near kin: they shall bear their iniquity.
\verse And if a man shall lie with his uncle's wife, he hath uncovered his uncle's nakedness: they shall bear their sin; they shall die childless.
\verse And if a man shall take his brother's wife, it is an unclean thing: he hath uncovered his brother's nakedness; they shall be childless.
\verse Ye shall therefore keep all my statutes, and all my judgments, and do them: that the land, whither I bring you to dwell therein, spue you not out.
\verse And ye shall not walk in the manners of the nation, which I cast out before you: for they committed all these things, and therefore I abhorred them.
\verse But I have said unto you, Ye shall inherit their land, and I will give it unto you to possess it, a land that floweth with milk and honey: I am the \LORD your God, which have separated you from other people.
\verse Ye shall therefore put difference between clean beasts and unclean, and between unclean fowls and clean: and ye shall not make your souls abominable by beast, or by fowl, or by any manner of living thing that creepeth on the ground, which I have separated from you as unclean.
\verse And ye shall be holy unto me: for I the \LORD am holy, and have severed you from other people, that ye should be mine.
\verse A man also or woman that hath a familiar spirit, or that is a wizard, shall surely be put to death: they shall stone them with stones: their blood shall be upon them.
\end{biblechapter}

\begin{biblechapter} % Leviticus 21
\verseWithHeading{Rules for priests} And the \LORD said unto Moses, Speak unto the priests the sons of Aaron, and say unto them, There shall none be defiled for the dead among his people:
\verse But for his kin, that is near unto him, that is, for his mother, and for his father, and for his son, and for his daughter, and for his brother,
\verse And for his sister a virgin, that is nigh unto him, which hath had no husband; for her may he be defiled.
\verse But he shall not defile himself, being a chief man among his people, to profane himself.
\verse They shall not make baldness upon their head, neither shall they shave off the corner of their beard, nor make any cuttings in their flesh.
\verse They shall be holy unto their God, and not profane the name of their God: for the offerings of the \LORD made by fire, and the bread of their God, they do offer: therefore they shall be holy.
\verse They shall not take a wife that is a whore, or profane; neither shall they take a woman put away from her husband: for he is holy unto his God.
\verse Thou shalt sanctify him therefore; for he offereth the bread of thy God: he shall be holy unto thee: for I the \LORD, which sanctify you, am holy.
\verse And the daughter of any priest, if she profane herself by playing the whore, she profaneth her father: she shall be burnt with fire.
\verse And he that is the high priest among his brethren, upon whose head the anointing oil was poured, and that is consecrated to put on the garments, shall not uncover his head, nor rend his clothes;
\verse Neither shall he go in to any dead body, nor defile himself for his father, or for his mother;
\verse Neither shall he go out of the sanctuary, nor profane the sanctuary of his God; for the crown of the anointing oil of his God is upon him: I am the \LORD.
\verse And he shall take a wife in her virginity.
\verse A widow, or a divorced woman, or profane, or an harlot, these shall he not take: but he shall take a virgin of his own people to wife.
\verse Neither shall he profane his seed among his people: for I the \LORD do sanctify him.
\verse And the \LORD spake unto Moses, saying,
\verse Speak unto Aaron, saying, Whosoever he be of thy seed in their generations that hath any blemish, let him not approach to offer the bread of his God.
\verse For whatsoever man he be that hath a blemish, he shall not approach: a blind man, or a lame, or he that hath a flat nose, or any thing superfluous,
\verse Or a man that is brokenfooted, or brokenhanded,
\verse Or crookbackt, or a dwarf, or that hath a blemish in his eye, or be scurvy, or scabbed, or hath his stones broken;
\verse No man that hath a blemish of the seed of Aaron the priest shall come nigh to offer the offerings of the \LORD made by fire: he hath a blemish; he shall not come nigh to offer the bread of his God.
\verse He shall eat the bread of his God, both of the most holy, and of the holy.
\verse Only he shall not go in unto the vail, nor come nigh unto the altar, because he hath a blemish; that he profane not my sanctuaries: for I the \LORD do sanctify them.
\verse And Moses told it unto Aaron, and to his sons, and unto all the children of Israel.
\end{biblechapter}

\begin{biblechapter} % Leviticus 22
\verse And the \LORD spake unto Moses, saying,
\verse Speak unto Aaron and to his sons, that they separate themselves from the holy things of the children of Israel, and that they profane not my holy name in those things which they hallow unto me: I am the \LORD.
\verse Say unto them, Whosoever he be of all your seed among your generations, that goeth unto the holy things, which the children of Israel hallow unto the \LORD, having his uncleanness upon him, that soul shall be cut off from my presence: I am the \LORD.
\verse What man soever of the seed of Aaron is a leper, or hath a running issue; he shall not eat of the holy things, until he be clean. And whoso toucheth any thing that is unclean by the dead, or a man whose seed goeth from him;
\verse Or whosoever toucheth any creeping thing, whereby he may be made unclean, or a man of whom he may take uncleanness, whatsoever uncleanness he hath;
\verse The soul which hath touched any such shall be unclean until even, and shall not eat of the holy things, unless he wash his flesh with water.
\verse And when the sun is down, he shall be clean, and shall afterward eat of the holy things; because it is his food.
\verse That which dieth of itself, or is torn with beasts, he shall not eat to defile himself therewith: I am the \LORD.
\verse They shall therefore keep mine ordinance, lest they bear sin for it, and die therefore, if they profane it: I the \LORD do sanctify them.
\verse There shall no stranger eat of the holy thing: a sojourner of the priest, or an hired servant, shall not eat of the holy thing.
\verse But if the priest buy any soul with his money, he shall eat of it, and he that is born in his house: they shall eat of his meat.
\verse If the priest's daughter also be married unto a stranger, she may not eat of an offering of the holy things.
\verse But if the priest's daughter be a widow, or divorced, and have no child, and is returned unto her father's house, as in her youth, she shall eat of her father's meat: but there shall no stranger eat thereof.
\verse And if a man eat of the holy thing unwittingly, then he shall put the fifth part thereof unto it, and shall give it unto the priest with the holy thing.
\verse And they shall not profane the holy things of the children of Israel, which they offer unto the \LORD;
\verse Or suffer them to bear the iniquity of trespass, when they eat their holy things: for I the \LORD do sanctify them.
\verseWithHeading{Unacceptable sacrifices} And the \LORD spake unto Moses, saying,
\verse Speak unto Aaron, and to his sons, and unto all the children of Israel, and say unto them, Whatsoever he be of the house of Israel, or of the strangers in Israel, that will offer his oblation for all his vows, and for all his freewill offerings, which they will offer unto the \LORD for a burnt offering;
\verse Ye shall offer at your own will a male without blemish, of the beeves, of the sheep, or of the goats.
\verse But whatsoever hath a blemish, that shall ye not offer: for it shall not be acceptable for you.
\verse And whosoever offereth a sacrifice of peace offerings unto the \LORD to accomplish his vow, or a freewill offering in beeves or sheep, it shall be perfect to be accepted; there shall be no blemish therein.
\verse Blind, or broken, or maimed, or having a wen, or scurvy, or scabbed, ye shall not offer these unto the \LORD, nor make an offering by fire of them upon the altar unto the \LORD.
\verse Either a bullock or a lamb that hath any thing superfluous or lacking in his parts, that mayest thou offer for a freewill offering; but for a vow it shall not be accepted.
\verse Ye shall not offer unto the \LORD that which is bruised, or crushed, or broken, or cut; neither shall ye make any offering thereof in your land.
\verse Neither from a stranger's hand shall ye offer the bread of your God of any of these; because their corruption is in them, and blemishes be in them: they shall not be accepted for you.
\verse And the \LORD spake unto Moses, saying,
\verse When a bullock, or a sheep, or a goat, is brought forth, then it shall be seven days under the dam; and from the eighth day and thenceforth it shall be accepted for an offering made by fire unto the \LORD.
\verse And whether it be cow or ewe, ye shall not kill it and her young both in one day.
\verse And when ye will offer a sacrifice of thanksgiving unto the \LORD, offer it at your own will.
\verse On the same day it shall be eaten up; ye shall leave none of it until the morrow: I am the \LORD.
\verse Therefore shall ye keep my commandments, and do them: I am the \LORD.
\verse Neither shall ye profane my holy name; but I will be hallowed among the children of Israel: I am the \LORD which hallow you,
\verse That brought you out of the land of Egypt, to be your God: I am the \LORD.
\end{biblechapter}

\begin{biblechapter} % Leviticus 23
\verseWithHeading{The appointed feasts} And the \LORD spake unto Moses, saying,
\verse Speak unto the children of Israel, and say unto them, Concerning the feasts of the \LORD, which ye shall proclaim to be holy convocations, even these are my feasts.
\verseWithSubheading{The Sabbath} Six days shall work be done: but the seventh day is the Sabbath of rest, an holy convocation; ye shall do no work therein: it is the Sabbath of the \LORD in all your dwellings.
\verseWithSubheading{The Passover and Unleavened Bread} These are the feasts of the \LORD, even holy convocations, which ye shall proclaim in their seasons.
\verse In the fourteenth day of the first month at even is the \LORDs Passover.
\verse And on the fifteenth day of the same month is the feast of unleavened bread unto the \LORD: seven days ye must eat unleavened bread.
\verse In the first day ye shall have an holy convocation: ye shall do no servile work therein.
\verse But ye shall offer an offering made by fire unto the \LORD seven days: in the seventh day is an holy convocation: ye shall do no servile work therein.
\verseWithSubheading{Offering the Firstfruits} And the \LORD spake unto Moses, saying,
\verse Speak unto the children of Israel, and say unto them, When ye be come into the land which I give unto you, and shall reap the harvest thereof, then ye shall bring a sheaf of the firstfruits of your harvest unto the priest:
\verse And he shall wave the sheaf before the \LORD, to be accepted for you: on the morrow after the Sabbath the priest shall wave it.
\verse And ye shall offer that day when ye wave the sheaf an he lamb without blemish of the first year for a burnt offering unto the \LORD.
\verse And the meat offering thereof shall be two tenth deals of fine flour mingled with oil, an offering made by fire unto the \LORD for a sweet savour: and the drink offering thereof shall be of wine, the fourth part of an hin.
\verse And ye shall eat neither bread, nor parched corn, nor green ears, until the selfsame day that ye have brought an offering unto your God: it shall be a statute for ever throughout your generations in all your dwellings.
\verseWithSubheading{The Feast of Weeks} And ye shall count unto you from the morrow after the Sabbath, from the day that ye brought the sheaf of the wave offering; seven Sabbaths shall be complete:
\verse Even unto the morrow after the seventh Sabbath shall ye number fifty days; and ye shall offer a new meat offering unto the \LORD.
\verse Ye shall bring out of your habitations two wave loaves of two tenth deals: they shall be of fine flour; they shall be baken with leaven; they are the firstfruits unto the \LORD.
\verse And ye shall offer with the bread seven lambs without blemish of the first year, and one young bullock, and two rams: they shall be for a burnt offering unto the \LORD, with their meat offering, and their drink offerings, even an offering made by fire, of sweet savour unto the \LORD.
\verse Then ye shall sacrifice one kid of the goats for a sin offering, and two lambs of the first year for a sacrifice of peace offerings.
\verse And the priest shall wave them with the bread of the firstfruits for a wave offering before the \LORD, with the two lambs: they shall be holy to the \LORD for the priest.
\verse And ye shall proclaim on the selfsame day, that it may be an holy convocation unto you: ye shall do no servile work therein: it shall be a statute for ever in all your dwellings throughout your generations.
\verse And when ye reap the harvest of your land, thou shalt not make clean riddance of the corners of thy field when thou reapest, neither shalt thou gather any gleaning of thy harvest: thou shalt leave them unto the poor, and to the stranger: I am the \LORD your God.
\verseWithSubheading{The Feast of Trumpets} And the \LORD spake unto Moses, saying,
\verse Speak unto the children of Israel, saying, In the seventh month, in the first day of the month, shall ye have a Sabbath, a memorial of blowing of trumpets, an holy convocation.
\verse Ye shall do no servile work therein: but ye shall offer an offering made by fire unto the \LORD.
\verseWithSubheading{The Day of Atonement} And the \LORD spake unto Moses, saying,
\verse Also on the tenth day of this seventh month there shall be a day of atonement: it shall be an holy convocation unto you; and ye shall afflict your souls, and offer an offering made by fire unto the \LORD.
\verse And ye shall do no work in that same day: for it is a day of atonement, to make an atonement for you before the \LORD your God.
\verse For whatsoever soul it be that shall not be afflicted in that same day, he shall be cut off from among his people.
\verse And whatsoever soul it be that doeth any work in that same day, the same soul will I destroy from among his people.
\verse Ye shall do no manner of work: it shall be a statute for ever throughout your generations in all your dwellings.
\verse It shall be unto you a Sabbath of rest, and ye shall afflict your souls: in the ninth day of the month at even, from even unto even, shall ye celebrate your Sabbath.
\verseWithSubheading{The Feast of Tabernacles} And the \LORD spake unto Moses, saying,
\verse Speak unto the children of Israel, saying, The fifteenth day of this seventh month shall be the feast of tabernacles for seven days unto the \LORD.
\verse On the first day shall be an holy convocation: ye shall do no servile work therein.
\verse Seven days ye shall offer an offering made by fire unto the \LORD: on the eighth day shall be an holy convocation unto you; and ye shall offer an offering made by fire unto the \LORD: it is a solemn assembly; and ye shall do no servile work therein.
\verse These are the feasts of the \LORD, which ye shall proclaim to be holy convocations, to offer an offering made by fire unto the \LORD, a burnt offering, and a meat offering, a sacrifice, and drink offerings, every thing upon his day:
\verse Beside the Sabbaths of the \LORD, and beside your gifts, and beside all your vows, and beside all your freewill offerings, which ye give unto the \LORD.
\verse Also in the fifteenth day of the seventh month, when ye have gathered in the fruit of the land, ye shall keep a feast unto the \LORD seven days: on the first day shall be a Sabbath, and on the eighth day shall be a Sabbath.
\verse And ye shall take you on the first day the boughs of goodly trees, branches of palm trees, and the boughs of thick trees, and willows of the brook; and ye shall rejoice before the \LORD your God seven days.
\verse And ye shall keep it a feast unto the \LORD seven days in the year. It shall be a statute for ever in your generations: ye shall celebrate it in the seventh month.
\verse Ye shall dwell in booths seven days; all that are Israelites born shall dwell in booths:
\verse That your generations may know that I made the children of Israel to dwell in booths, when I brought them out of the land of Egypt: I am the \LORD your God.
\verse And Moses declared unto the children of Israel the feasts of the \LORD.
\end{biblechapter}

\begin{biblechapter} % Leviticus 24
\verseWithHeading{Olive oil and bread set \newline before the \LORD} And the \LORD spake unto Moses, saying,
\verse Command the children of Israel, that they bring unto thee pure oil olive beaten for the light, to cause the lamps to burn continually.
\verse Without the vail of the testimony, in the tabernacle of the congregation, shall Aaron order it from the evening unto the morning before the \LORD continually: it shall be a statute for ever in your generations.
\verse He shall order the lamps upon the pure candlestick before the \LORD continually.
\verse And thou shalt take fine flour, and bake twelve cakes thereof: two tenth deals shall be in one cake.
\verse And thou shalt set them in two rows, six on a row, upon the pure table before the \LORD.
\verse And thou shalt put pure frankincense upon each row, that it may be on the bread for a memorial, even an offering made by fire unto the \LORD.
\verse Every Sabbath he shall set it in order before the \LORD continually, being taken from the children of Israel by an everlasting covenant.
\verse And it shall be Aaron's and his sons'; and they shall eat it in the holy place: for it is most holy unto him of the offerings of the \LORD made by fire by a perpetual statute.
\verseWithHeading{A blasphemer put to death} And the son of an Israelitish woman, whose father was an Egyptian, went out among the children of Israel: and this son of the Israelitish woman and a man of Israel strove together in the camp;
\verse And the Israelitish woman's son blasphemed the name of the \LORD, and cursed. And they brought him unto Moses: (and his mother's name was Shelomith, the daughter of Dibri, of the tribe of Dan:)
\verse And they put him in ward, that the mind of the \LORD might be shewed them.
\verse And the \LORD spake unto Moses, saying,
\verse Bring forth him that hath cursed without the camp; and let all that heard him lay their hands upon his head, and let all the congregation stone him.
\verse And thou shalt speak unto the children of Israel, saying, Whosoever curseth his God shall bear his sin.
\verse And he that blasphemeth the name of the \LORD, he shall surely be put to death, and all the congregation shall certainly stone him: as well the stranger, as he that is born in the land, when he blasphemeth the name of the \LORD, shall be put to death.
\verse And he that killeth any man shall surely be put to death.
\verse And he that killeth a beast shall make it good; beast for beast.
\verse And if a man cause a blemish in his neighbour; as he hath done, so shall it be done to him;
\verse Breach for breach, eye for eye, tooth for tooth: as he hath caused a blemish in a man, so shall it be done to him again.
\verse And he that killeth a beast, he shall restore it: and he that killeth a man, he shall be put to death.
\verse Ye shall have one manner of law, as well for the stranger, as for one of your own country: for I am the \LORD your God.
\verse And Moses spake to the children of Israel, that they should bring forth him that had cursed out of the camp, and stone him with stones. And the children of Israel did as the \LORD commanded Moses.
\end{biblechapter}

\begin{biblechapter} % Leviticus 25
\verseWithHeading{The Sabbath Year} And the \LORD spake unto Moses in mount Sinai, saying,
\verse Speak unto the children of Israel, and say unto them, When ye come into the land which I give you, then shall the land keep a Sabbath unto the \LORD.
\verse Six years thou shalt sow thy field, and six years thou shalt prune thy vineyard, and gather in the fruit thereof;
\verse But in the seventh year shall be a Sabbath of rest unto the land, a Sabbath for the \LORD: thou shalt neither sow thy field, nor prune thy vineyard.
\verse That which groweth of its own accord of thy harvest thou shalt not reap, neither gather the grapes of thy vine undressed: for it is a year of rest unto the land.
\verse And the Sabbath of the land shall be meat for you; for thee, and for thy servant, and for thy maid, and for thy hired servant, and for thy stranger that sojourneth with thee,
\verse And for thy cattle, and for the beast that are in thy land, shall all the increase thereof be meat.
\flushcolsend\columnbreak % layout hack
\verseWithHeading{The Year of Jubilee} And thou shalt number seven Sabbaths of years unto thee, seven times seven years; and the space of the seven Sabbaths of years shall be unto thee forty and nine years.
\verse Then shalt thou cause the trumpet of the jubile to sound on the tenth day of the seventh month, in the day of atonement shall ye make the trumpet sound throughout all your land.
\verse And ye shall hallow the fiftieth year, and proclaim liberty throughout all the land unto all the inhabitants thereof: it shall be a jubile unto you; and ye shall return every man unto his possession, and ye shall return every man unto his family.
\verse A jubile shall that fiftieth year be unto you: ye shall not sow, neither reap that which groweth of itself in it, nor gather the grapes in it of thy vine undressed.
\verse For it is the jubile; it shall be holy unto you: ye shall eat the increase thereof out of the field.
\verse In the year of this jubile ye shall return every man unto his possession.
\verse And if thou sell ought unto thy neighbour, or buyest ought of thy neighbour's hand, ye shall not oppress one another:
\verse According to the number of years after the jubile thou shalt buy of thy neighbour, and according unto the number of years of the fruits he shall sell unto thee:
\verse According to the multitude of years thou shalt increase the price thereof, and according to the fewness of years thou shalt diminish the price of it: for according to the number of the years of the fruits doth he sell unto thee.
\verse Ye shall not therefore oppress one another; but thou shalt fear thy God: for I am the \LORD your God.
\verse Wherefore ye shall do my statutes, and keep my judgments, and do them; and ye shall dwell in the land in safety.
\verse And the land shall yield her fruit, and ye shall eat your fill, and dwell therein in safety.
\verse And if ye shall say, What shall we eat the seventh year? behold, we shall not sow, nor gather in our increase:
\verse Then I will command my blessing upon you in the sixth year, and it shall bring forth fruit for three years.
\verse And ye shall sow the eighth year, and eat yet of old fruit until the ninth year; until her fruits come in ye shall eat of the old store.
\verse The land shall not be sold for ever: for the land is mine; for ye are strangers and sojourners with me.
\verse And in all the land of your possession ye shall grant a redemption for the land.
\verse If thy brother be waxen poor, and hath sold away some of his possession, and if any of his kin come to redeem it, then shall he redeem that which his brother sold.
\verse And if the man have none to redeem it, and himself be able to redeem it;
\verse Then let him count the years of the sale thereof, and restore the overplus unto the man to whom he sold it; that he may return unto his possession.
\verse But if he be not able to restore it to him, then that which is sold shall remain in the hand of him that hath bought it until the year of jubile: and in the jubile it shall go out, and he shall return unto his possession.
\verse And if a man sell a dwelling house in a walled city, then he may redeem it within a whole year after it is sold; within a full year may he redeem it.
\verse And if it be not redeemed within the space of a full year, then the house that is in the walled city shall be established for ever to him that bought it throughout his generations: it shall not go out in the jubile.
\verse But the houses of the villages which have no wall round about them shall be counted as the fields of the country: they may be redeemed, and they shall go out in the jubile.
\verse Notwithstanding the cities of the Levites, and the houses of the cities of their possession, may the Levites redeem at any time.
\verse And if a man purchase of the Levites, then the house that was sold, and the city of his possession, shall go out in the year of jubile: for the houses of the cities of the Levites are their possession among the children of Israel.
\verse But the field of the suburbs of their cities may not be sold; for it is their perpetual possession.
\verse And if thy brother be waxen poor, and fallen in decay with thee; then thou shalt relieve him: yea, though he be a stranger, or a sojourner; that he may live with thee.
\verse Take thou no usury of him, or increase: but fear thy God; that thy brother may live with thee.
\verse Thou shalt not give him thy money upon usury, nor lend him thy victuals for increase.
\verse I am the \LORD your God, which brought you forth out of the land of Egypt, to give you the land of Canaan, and to be your God.
\verse And if thy brother that dwelleth by thee be waxen poor, and be sold unto thee; thou shalt not compel him to serve as a bondservant:
\verse But as an hired servant, and as a sojourner, he shall be with thee, and shall serve thee unto the year of jubile:
\verse And then shall he depart from thee, both he and his children with him, and shall return unto his own family, and unto the possession of his fathers shall he return.
\verse For they are my servants, which I brought forth out of the land of Egypt: they shall not be sold as bondmen.
\verse Thou shalt not rule over him with rigour; but shalt fear thy God.
\verse Both thy bondmen, and thy bondmaids, which thou shalt have, shall be of the heathen that are round about you; of them shall ye buy bondmen and bondmaids.
\verse Moreover of the children of the strangers that do sojourn among you, of them shall ye buy, and of their families that are with you, which they begat in your land: and they shall be your possession.
\verse And ye shall take them as an inheritance for your children after you, to inherit them for a possession; they shall be your bondmen for ever: but over your brethren the children of Israel, ye shall not rule one over another with rigour.
\verse And if a sojourner or stranger wax rich by thee, and thy brother that dwelleth by him wax poor, and sell himself unto the stranger or sojourner by thee, or to the stock of the stranger's family:
\verse After that he is sold he may be redeemed again; one of his brethren may redeem him:
\verse Either his uncle, or his uncle's son, may redeem him, or any that is nigh of kin unto him of his family may redeem him; or if he be able, he may redeem himself.
\verse And he shall reckon with him that bought him from the year that he was sold to him unto the year of jubile: and the price of his sale shall be according unto the number of years, according to the time of an hired servant shall it be with him.
\verse If there be yet many years behind, according unto them he shall give again the price of his redemption out of the money that he was bought for.
\verse And if there remain but few years unto the year of jubile, then he shall count with him, and according unto his years shall he give him again the price of his redemption.
\verse And as a yearly hired servant shall he be with him: and the other shall not rule with rigour over him in thy sight.
\verse And if he be not redeemed in these years, then he shall go out in the year of jubile, both he, and his children with him.
\verse For unto me the children of Israel are servants; they are my servants whom I brought forth out of the land of Egypt: I am the \LORD your God.
\end{biblechapter}

\begin{biblechapter} % Leviticus 26
\verseWithHeading{Reward for obedience} Ye shall make you no idols nor graven image, neither rear you up a standing image, neither shall ye set up any image of stone in your land, to bow down unto it: for I am the \LORD your God.
\verse Ye shall keep my Sabbaths, and reverence my sanctuary: I am the \LORD.
\verse If ye walk in my statutes, and keep my commandments, and do them;
\verse Then I will give you rain in due season, and the land shall yield her increase, and the trees of the field shall yield their fruit.
\verse And your threshing shall reach unto the vintage, and the vintage shall reach unto the sowing time: and ye shall eat your bread to the full, and dwell in your land safely.
\verse And I will give peace in the land, and ye shall lie down, and none shall make you afraid: and I will rid evil beasts out of the land, neither shall the sword go through your land.
\verse And ye shall chase your enemies, and they shall fall before you by the sword.
\verse And five of you shall chase an hundred, and an hundred of you shall put ten thousand to flight: and your enemies shall fall before you by the sword.
\verse For I will have respect unto you, and make you fruitful, and multiply you, and establish my covenant with you.
\verse And ye shall eat old store, and bring forth the old because of the new.
\verse And I will set my tabernacle among you: and my soul shall not abhor you.
\verse And I will walk among you, and will be your God, and ye shall be my people.
\verse I am the \LORD your God, which brought you forth out of the land of Egypt, that ye should not be their bondmen; and I have broken the bands of your yoke, and made you go upright.
\verseWithHeading{Punishment for disobedience} But if ye will not hearken unto me, and will not do all these commandments;
\verse And if ye shall despise my statutes, or if your soul abhor my judgments, so that ye will not do all my commandments, but that ye break my covenant:
\verse I also will do this unto you; I will even appoint over you terror, consumption, and the burning ague, that shall consume the eyes, and cause sorrow of heart: and ye shall sow your seed in vain, for your enemies shall eat it.
\verse And I will set my face against you, and ye shall be slain before your enemies: they that hate you shall reign over you; and ye shall flee when none pursueth you.
\verse And if ye will not yet for all this hearken unto me, then I will punish you seven times more for your sins.
\verse And I will break the pride of your power; and I will make your heaven as iron, and your earth as brass:
\verse And your strength shall be spent in vain: for your land shall not yield her increase, neither shall the trees of the land yield their fruits.
\verse And if ye walk contrary unto me, and will not hearken unto me; I will bring seven times more plagues upon you according to your sins.
\verse I will also send wild beasts among you, which shall rob you of your children, and destroy your cattle, and make you few in number; and your high ways shall be desolate.
\verse And if ye will not be reformed by me by these things, but will walk contrary unto me;
\verse Then will I also walk contrary unto you, and will punish you yet seven times for your sins.
\verse And I will bring a sword upon you, that shall avenge the quarrel of my covenant: and when ye are gathered together within your cities, I will send the pestilence among you; and ye shall be delivered into the hand of the enemy.
\verse And when I have broken the staff of your bread, ten women shall bake your bread in one oven, and they shall deliver you your bread again by weight: and ye shall eat, and not be satisfied.
\verse And if ye will not for all this hearken unto me, but walk contrary unto me;
\verse Then I will walk contrary unto you also in fury; and I, even I, will chastise you seven times for your sins.
\verse And ye shall eat the flesh of your sons, and the flesh of your daughters shall ye eat.
\verse And I will destroy your high places, and cut down your images, and cast your carcases upon the carcases of your idols, and my soul shall abhor you.
\verse And I will make your cities waste, and bring your sanctuaries unto desolation, and I will not smell the savour of your sweet odours.
\verse And I will bring the land into desolation: and your enemies which dwell therein shall be astonished at it.
\verse And I will scatter you among the heathen, and will draw out a sword after you: and your land shall be desolate, and your cities waste.
\verse Then shall the land enjoy her Sabbaths, as long as it lieth desolate, and ye be in your enemies' land; even then shall the land rest, and enjoy her Sabbaths.
\verse As long as it lieth desolate it shall rest; because it did not rest in your Sabbaths, when ye dwelt upon it.
\verse And upon them that are left alive of you I will send a faintness into their hearts in the lands of their enemies; and the sound of a shaken leaf shall chase them; and they shall flee, as fleeing from a sword; and they shall fall when none pursueth.
\verse And they shall fall one upon another, as it were before a sword, when none pursueth: and ye shall have no power to stand before your enemies.
\verse And ye shall perish among the heathen, and the land of your enemies shall eat you up.
\verse And they that are left of you shall pine away in their iniquity in your enemies' lands; and also in the iniquities of their fathers shall they pine away with them.
\verse If they shall confess their iniquity, and the iniquity of their fathers, with their trespass which they trespassed against me, and that also they have walked contrary unto me;
\verse And that I also have walked contrary unto them, and have brought them into the land of their enemies; if then their uncircumcised hearts be humbled, and they then accept of the punishment of their iniquity:
\verse Then will I remember my covenant with Jacob, and also my covenant with Isaac, and also my covenant with Abraham will I remember; and I will remember the land.
\verse The land also shall be left of them, and shall enjoy her Sabbaths, while she lieth desolate without them: and they shall accept of the punishment of their iniquity: because, even because they despised my judgments, and because their soul abhorred my statutes.
\verse And yet for all that, when they be in the land of their enemies, I will not cast them away, neither will I abhor them, to destroy them utterly, and to break my covenant with them: for I am the \LORD their God.
\verse But I will for their sakes remember the covenant of their ancestors, whom I brought forth out of the land of Egypt in the sight of the heathen, that I might be their God: I am the \LORD.
\verse These are the statutes and judgments and laws, which the \LORD made between him and the children of Israel in mount Sinai by the hand of Moses.
\end{biblechapter}

\begin{biblechapter} % Leviticus 27
\verseWithHeading{Redeeming what is the \newline \LORDs} And the \LORD spake unto Moses, saying,
\verse Speak unto the children of Israel, and say unto them, When a man shall make a singular vow, the persons shall be for the \LORD by thy estimation.
\verse And thy estimation shall be of the male from twenty years old even unto sixty years old, even thy estimation shall be fifty shekels of silver, after the shekel of the sanctuary.
\verse And if it be a female, then thy estimation shall be thirty shekels.
\verse And if it be from five years old even unto twenty years old, then thy estimation shall be of the male twenty shekels, and for the female ten shekels.
\verse And if it be from a month old even unto five years old, then thy estimation shall be of the male five shekels of silver, and for the female thy estimation shall be three shekels of silver.
\verse And if it be from sixty years old and above; if it be a male, then thy estimation shall be fifteen shekels, and for the female ten shekels.
\verse But if he be poorer than thy estimation, then he shall present himself before the priest, and the priest shall value him; according to his ability that vowed shall the priest value him.
\verse And if it be a beast, whereof men bring an offering unto the \LORD, all that any man giveth of such unto the \LORD shall be holy.
\verse He shall not alter it, nor change it, a good for a bad, or a bad for a good: and if he shall at all change beast for beast, then it and the exchange thereof shall be holy.
\verse And if it be any unclean beast, of which they do not offer a sacrifice unto the \LORD, then he shall present the beast before the priest:
\verse And the priest shall value it, whether it be good or bad: as thou valuest it, who art the priest, so shall it be.
\verse But if he will at all redeem it, then he shall add a fifth part thereof unto thy estimation.
\verse And when a man shall sanctify his house to be holy unto the \LORD, then the priest shall estimate it, whether it be good or bad: as the priest shall estimate it, so shall it stand.
\verse And if he that sanctified it will redeem his house, then he shall add the fifth part of the money of thy estimation unto it, and it shall be his.
\verse And if a man shall sanctify unto the \LORD some part of a field of his possession, then thy estimation shall be according to the seed thereof: an homer of barley seed shall be valued at fifty shekels of silver.
\verse If he sanctify his field from the year of jubile, according to thy estimation it shall stand.
\verse But if he sanctify his field after the jubile, then the priest shall reckon unto him the money according to the years that remain, even unto the year of the jubile, and it shall be abated from thy estimation.
\verse And if he that sanctified the field will in any wise redeem it, then he shall add the fifth part of the money of thy estimation unto it, and it shall be assured to him.
\verse And if he will not redeem the field, or if he have sold the field to another man, it shall not be redeemed any more.
\verse But the field, when it goeth out in the jubile, shall be holy unto the \LORD, as a field devoted; the possession thereof shall be the priest's.
\verse And if a man sanctify unto the \LORD a field which he hath bought, which is not of the fields of his possession;
\verse Then the priest shall reckon unto him the worth of thy estimation, even unto the year of the jubile: and he shall give thine estimation in that day, as a holy thing unto the \LORD.
\verse In the year of the jubile the field shall return unto him of whom it was bought, even to him to whom the possession of the land did belong.
\verse And all thy estimations shall be according to the shekel of the sanctuary: twenty gerahs shall be the shekel.
\verse Only the firstling of the beasts, which should be the \LORDs firstling, no man shall sanctify it; whether it be ox, or sheep: it is the \LORDs.
\verse And if it be of an unclean beast, then he shall redeem it according to thine estimation, and shall add a fifth part of it thereto: or if it be not redeemed, then it shall be sold according to thy estimation.
\verse Notwithstanding no devoted thing, that a man shall devote unto the \LORD of all that he hath, both of man and beast, and of the field of his possession, shall be sold or redeemed: every devoted thing is most holy unto the \LORD.
\verse None devoted, which shall be devoted of men, shall be redeemed; but shall surely be put to death.
\verse And all the tithe of the land, whether of the seed of the land, or of the fruit of the tree, is the \LORDs: it is holy unto the \LORD.
\verse And if a man will at all redeem ought of his tithes, he shall add thereto the fifth part thereof.
\verse And concerning the tithe of the herd, or of the flock, even of whatsoever passeth under the rod, the tenth shall be holy unto the \LORD.
\verse He shall not search whether it be good or bad, neither shall he change it: and if he change it at all, then both it and the change thereof shall be holy; it shall not be redeemed.
\verse These are the commandments, which the \LORD commanded Moses for the children of Israel in mount Sinai.
\end{biblechapter}
\flushcolsend
\biblebook{Numbers}

\section*{The census}
\begin{biblechapter} % Numbers 1
\verse And the \LORD spake unto Moses in the wilderness of Sinai, in the tabernacle of the congregation, on the first day of the second month, in the second year after they were come out of the land of Egypt, saying,
\verse Take ye the sum of all the congregation of the children of Israel, after their families, by the house of their fathers, with the number of their names, every male by their polls;
\verse From twenty years old and upward, all that are able to go forth to war in Israel: thou and Aaron shall number them by their armies.
\verse And with you there shall be a man of every tribe; every one head of the house of his fathers.
\verse And these are the names of the men that shall stand with you: of the tribe of Reuben; Elizur the son of Shedeur.
\verse Of Simeon; Shelumiel the son of Zurishaddai.
\verse Of Judah; Nahshon the son of Amminadab.
\verse Of Issachar; Nethaneel the son of Zuar.
\verse Of Zebulun; Eliab the son of Helon.
\verse Of the children of Joseph: of Ephraim; Elishama the son of Ammihud: of Manasseh; Gamaliel the son of Pedahzur.
\verse Of Benjamin; Abidan the son of Gideoni.
\verse Of Dan; Ahiezer the son of Ammishaddai.
\verse Of Asher; Pagiel the son of Ocran.
\verse Of Gad; Eliasaph the son of Deuel.
\verse Of Naphtali; Ahira the son of Enan.
\verse These were the renowned of the congregation, princes of the tribes of their fathers, heads of thousands in Israel.
\verse And Moses and Aaron took these men which are expressed by their names:
\verse And they assembled all the congregation together on the first day of the second month, and they declared their pedigrees after their families, by the house of their fathers, according to the number of the names, from twenty years old and upward, by their polls.
\verse As the \LORD commanded Moses, so he numbered them in the wilderness of Sinai.
\verse And the children of Reuben, Israel's eldest son, by their generations, after their families, by the house of their fathers, according to the number of the names, by their polls, every male from twenty years old and upward, all that were able to go forth to war;
\verse Those that were numbered of them, even of the tribe of Reuben, were forty and six thousand and five hundred.
\verse Of the children of Simeon, by their generations, after their families, by the house of their fathers, those that were numbered of them, according to the number of the names, by their polls, every male from twenty years old and upward, all that were able to go forth to war;
\verse Those that were numbered of them, even of the tribe of Simeon, were fifty and nine thousand and three hundred.
\verse Of the children of Gad, by their generations, after their families, by the house of their fathers, according to the number of the names, from twenty years old and upward, all that were able to go forth to war;
\verse Those that were numbered of them, even of the tribe of Gad, were forty and five thousand six hundred and fifty.
\verse Of the children of Judah, by their generations, after their families, by the house of their fathers, according to the number of the names, from twenty years old and upward, all that were able to go forth to war;
\verse Those that were numbered of them, even of the tribe of Judah, were threescore and fourteen thousand and six hundred.
\verse Of the children of Issachar, by their generations, after their families, by the house of their fathers, according to the number of the names, from twenty years old and upward, all that were able to go forth to war;
\verse Those that were numbered of them, even of the tribe of Issachar, were fifty and four thousand and four hundred.
\verse Of the children of Zebulun, by their generations, after their families, by the house of their fathers, according to the number of the names, from twenty years old and upward, all that were able to go forth to war;
\verse Those that were numbered of them, even of the tribe of Zebulun, were fifty and seven thousand and four hundred.
\verse Of the children of Joseph, namely, of the children of Ephraim, by their generations, after their families, by the house of their fathers, according to the number of the names, from twenty years old and upward, all that were able to go forth to war;
\verse Those that were numbered of them, even of the tribe of Ephraim, were forty thousand and five hundred.
\verse Of the children of Manasseh, by their generations, after their families, by the house of their fathers, according to the number of the names, from twenty years old and upward, all that were able to go forth to war;
\verse Those that were numbered of them, even of the tribe of Manasseh, were thirty and two thousand and two hundred.
\verse Of the children of Benjamin, by their generations, after their families, by the house of their fathers, according to the number of the names, from twenty years old and upward, all that were able to go forth to war;
\verse Those that were numbered of them, even of the tribe of Benjamin, were thirty and five thousand and four hundred.
\verse Of the children of Dan, by their generations, after their families, by the house of their fathers, according to the number of the names, from twenty years old and upward, all that were able to go forth to war;
\verse Those that were numbered of them, even of the tribe of Dan, were threescore and two thousand and seven hundred.
\verse Of the children of Asher, by their generations, after their families, by the house of their fathers, according to the number of the names, from twenty years old and upward, all that were able to go forth to war;
\verse Those that were numbered of them, even of the tribe of Asher, were forty and one thousand and five hundred.
\verse Of the children of Naphtali, throughout their generations, after their families, by the house of their fathers, according to the number of the names, from twenty years old and upward, all that were able to go forth to war;
\verse Those that were numbered of them, even of the tribe of Naphtali, were fifty and three thousand and four hundred.
\verse These are those that were numbered, which Moses and Aaron numbered, and the princes of Israel, being twelve men: each one was for the house of his fathers.
\verse So were all those that were numbered of the children of Israel, by the house of their fathers, from twenty years old and upward, all that were able to go forth to war in Israel;
\verse Even all they that were numbered were six hundred thousand and three thousand and five hundred and fifty.
\verse But the Levites after the tribe of their fathers were not numbered among them.
\verse For the \LORD had spoken unto Moses, saying,
\verse Only thou shalt not number the tribe of Levi, neither take the sum of them among the children of Israel:
\verse But thou shalt appoint the Levites over the tabernacle of testimony, and over all the vessels thereof, and over all things that belong to it: they shall bear the tabernacle, and all the vessels thereof; and they shall minister unto it, and shall encamp round about the tabernacle.
\verse And when the tabernacle setteth forward, the Levites shall take it down: and when the tabernacle is to be pitched, the Levites shall set it up: and the stranger that cometh nigh shall be put to death.
\verse And the children of Israel shall pitch their tents, every man by his own camp, and every man by his own standard, throughout their hosts.
\verse But the Levites shall pitch round about the tabernacle of testimony, that there be no wrath upon the congregation of the children of Israel: and the Levites shall keep the charge of the tabernacle of testimony.
\verse And the children of Israel did according to all that the \LORD commanded Moses, so did they.
\end{biblechapter}

\section*{The arrangement of the tribal camps}
\begin{biblechapter} % Numbers 2
\verse And the \LORD spake unto Moses and unto Aaron, saying,
\verse Every man of the children of Israel shall pitch by his own standard, with the ensign of their father's house: far off about the tabernacle of the congregation shall they pitch.
\verse And on the east side toward the rising of the sun shall they of the standard of the camp of Judah pitch throughout their armies: and Nahshon the son of Amminadab shall be captain of the children of Judah.
\verse And his host, and those that were numbered of them, were threescore and fourteen thousand and six hundred.
\verse And those that do pitch next unto him shall be the tribe of Issachar: and Nethaneel the son of Zuar shall be captain of the children of Issachar.
\verse And his host, and those that were numbered thereof, were fifty and four thousand and four hundred.
\verse Then the tribe of Zebulun: and Eliab the son of Helon shall be captain of the children of Zebulun.
\verse And his host, and those that were numbered thereof, were fifty and seven thousand and four hundred.
\verse All that were numbered in the camp of Judah were an hundred thousand and fourscore thousand and six thousand and four hundred, throughout their armies. These shall first set forth.
\verse On the south side shall be the standard of the camp of Reuben according to their armies: and the captain of the children of Reuben shall be Elizur the son of Shedeur.
\verse And his host, and those that were numbered thereof, were forty and six thousand and five hundred.
\verse And those which pitch by him shall be the tribe of Simeon: and the captain of the children of Simeon shall be Shelumiel the son of Zurishaddai.
\verse And his host, and those that were numbered of them, were fifty and nine thousand and three hundred.
\verse Then the tribe of Gad: and the captain of the sons of Gad shall be Eliasaph the son of Reuel.
\verse And his host, and those that were numbered of them, were forty and five thousand and six hundred and fifty.
\verse All that were numbered in the camp of Reuben were an hundred thousand and fifty and one thousand and four hundred and fifty, throughout their armies. And they shall set forth in the second rank.
\verse Then the tabernacle of the congregation shall set forward with the camp of the Levites in the midst of the camp: as they encamp, so shall they set forward, every man in his place by their standards.
\verse On the west side shall be the standard of the camp of Ephraim according to their armies: and the captain of the sons of Ephraim shall be Elishama the son of Ammihud.
\verse And his host, and those that were numbered of them, were forty thousand and five hundred.
\verse And by him shall be the tribe of Manasseh: and the captain of the children of Manasseh shall be Gamaliel the son of Pedahzur.
\verse And his host, and those that were numbered of them, were thirty and two thousand and two hundred.
\verse Then the tribe of Benjamin: and the captain of the sons of Benjamin shall be Abidan the son of Gideoni.
\verse And his host, and those that were numbered of them, were thirty and five thousand and four hundred.
\verse All that were numbered of the camp of Ephraim were an hundred thousand and eight thousand and an hundred, throughout their armies. And they shall go forward in the third rank.
\verse The standard of the camp of Dan shall be on the north side by their armies: and the captain of the children of Dan shall be Ahiezer the son of Ammishaddai.
\verse And his host, and those that were numbered of them, were threescore and two thousand and seven hundred.
\verse And those that encamp by him shall be the tribe of Asher: and the captain of the children of Asher shall be Pagiel the son of Ocran.
\verse And his host, and those that were numbered of them, were forty and one thousand and five hundred.
\verse Then the tribe of Naphtali: and the captain of the children of Naphtali shall be Ahira the son of Enan.
\verse And his host, and those that were numbered of them, were fifty and three thousand and four hundred.
\verse All they that were numbered in the camp of Dan were an hundred thousand and fifty and seven thousand and six hundred. They shall go hindmost with their standards.
\verse These are those which were numbered of the children of Israel by the house of their fathers: all those that were numbered of the camps throughout their hosts were six hundred thousand and three thousand and five hundred and fifty.
\verse But the Levites were not numbered among the children of Israel; as the \LORD commanded Moses.
\verse And the children of Israel did according to all that the \LORD commanded Moses: so they pitched by their standards, and so they set forward, every one after their families, according to the house of their fathers.
\end{biblechapter}

\section*{The Levites}
\begin{biblechapter} % Numbers 3
\verse These also are the generations of Aaron and Moses in the day that the \LORD spake with Moses in mount Sinai.
\verse And these are the names of the sons of Aaron; Nadab the firstborn, and Abihu, Eleazar, and Ithamar.
\verse These are the names of the sons of Aaron, the priests which were anointed, whom he consecrated to minister in the priest's office.
\verse And Nadab and Abihu died before the \LORD, when they offered strange fire before the \LORD, in the wilderness of Sinai, and they had no children: and Eleazar and Ithamar ministered in the priest's office in the sight of Aaron their father.
\verse And the \LORD spake unto Moses, saying,
\verse Bring the tribe of Levi near, and present them before Aaron the priest, that they may minister unto him.
\verse And they shall keep his charge, and the charge of the whole congregation before the tabernacle of the congregation, to do the service of the tabernacle.
\verse And they shall keep all the instruments of the tabernacle of the congregation, and the charge of the children of Israel, to do the service of the tabernacle.
\verse And thou shalt give the Levites unto Aaron and to his sons: they are wholly given unto him out of the children of Israel.
\verse And thou shalt appoint Aaron and his sons, and they shall wait on their priest's office: and the stranger that cometh nigh shall be put to death.
\verse And the \LORD spake unto Moses, saying,
\verse And I, behold, I have taken the Levites from among the children of Israel instead of all the firstborn that openeth the matrix among the children of Israel: therefore the Levites shall be mine;
\verse Because all the firstborn are mine; for on the day that I smote all the firstborn in the land of Egypt I hallowed unto me all the firstborn in Israel, both man and beast: mine shall they be: I am the \LORD.
\verse And the \LORD spake unto Moses in the wilderness of Sinai, saying,
\verse Number the children of Levi after the house of their fathers, by their families: every male from a month old and upward shalt thou number them.
\verse And Moses numbered them according to the word of the \LORD, as he was commanded.
\verse And these were the sons of Levi by their names; Gershon, and Kohath, and Merari.
\verse And these are the names of the sons of Gershon by their families; Libni, and Shimei.
\verse And the sons of Kohath by their families; Amram, and Izehar, Hebron, and Uzziel.
\verse And the sons of Merari by their families; Mahli, and Mushi. These are the families of the Levites according to the house of their fathers.
\verse Of Gershon was the family of the Libnites, and the family of the Shimites: these are the families of the Gershonites.
\verse Those that were numbered of them, according to the number of all the males, from a month old and upward, even those that were numbered of them were seven thousand and five hundred.
\verse The families of the Gershonites shall pitch behind the tabernacle westward.
\verse And the chief of the house of the father of the Gershonites shall be Eliasaph the son of Lael.
\verse And the charge of the sons of Gershon in the tabernacle of the congregation shall be the tabernacle, and the tent, the covering thereof, and the hanging for the door of the tabernacle of the congregation,
\verse And the hangings of the court, and the curtain for the door of the court, which is by the tabernacle, and by the altar round about, and the cords of it for all the service thereof.
\verse And of Kohath was the family of the Amramites, and the family of the Izeharites, and the family of the Hebronites, and the family of the Uzzielites: these are the families of the Kohathites.
\verse In the number of all the males, from a month old and upward, were eight thousand and six hundred, keeping the charge of the sanctuary.
\verse The families of the sons of Kohath shall pitch on the side of the tabernacle southward.
\verse And the chief of the house of the father of the families of the Kohathites shall be Elizaphan the son of Uzziel.
\verse And their charge shall be the ark, and the table, and the candlestick, and the altars, and the vessels of the sanctuary wherewith they minister, and the hanging, and all the service thereof.
\verse And Eleazar the son of Aaron the priest shall be chief over the chief of the Levites, and have the oversight of them that keep the charge of the sanctuary.
\verse Of Merari was the family of the Mahlites, and the family of the Mushites: these are the families of Merari.
\verse And those that were numbered of them, according to the number of all the males, from a month old and upward, were six thousand and two hundred.
\verse And the chief of the house of the father of the families of Merari was Zuriel the son of Abihail: these shall pitch on the side of the tabernacle northward.
\verse And under the custody and charge of the sons of Merari shall be the boards of the tabernacle, and the bars thereof, and the pillars thereof, and the sockets thereof, and all the vessels thereof, and all that serveth thereto,
\verse And the pillars of the court round about, and their sockets, and their pins, and their cords.
\verse But those that encamp before the tabernacle toward the east, even before the tabernacle of the congregation eastward, shall be Moses, and Aaron and his sons, keeping the charge of the sanctuary for the charge of the children of Israel; and the stranger that cometh nigh shall be put to death.
\verse All that were numbered of the Levites, which Moses and Aaron numbered at the commandment of the \LORD, throughout their families, all the males from a month old and upward, were twenty and two thousand.
\verse And the \LORD said unto Moses, Number all the firstborn of the males of the children of Israel from a month old and upward, and take the number of their names.
\verse And thou shalt take the Levites for me (I am the \LORD) instead of all the firstborn among the children of Israel; and the cattle of the Levites instead of all the firstlings among the cattle of the children of Israel.
\verse And Moses numbered, as the \LORD commanded him, all the firstborn among the children of Israel.
\verse And all the firstborn males by the number of names, from a month old and upward, of those that were numbered of them, were twenty and two thousand two hundred and threescore and thirteen.
\verse And the \LORD spake unto Moses, saying,
\verse Take the Levites instead of all the firstborn among the children of Israel, and the cattle of the Levites instead of their cattle; and the Levites shall be mine: I am the \LORD.
\verse And for those that are to be redeemed of the two hundred and threescore and thirteen of the firstborn of the children of Israel, which are more than the Levites;
\verse Thou shalt even take five shekels apiece by the poll, after the shekel of the sanctuary shalt thou take them: (the shekel is twenty gerahs:)
\verse And thou shalt give the money, wherewith the odd number of them is to be redeemed, unto Aaron and to his sons.
\verse And Moses took the redemption money of them that were over and above them that were redeemed by the Levites:
\verse Of the firstborn of the children of Israel took he the money; a thousand three hundred and threescore and five shekels, after the shekel of the sanctuary:
\verse And Moses gave the money of them that were redeemed unto Aaron and to his sons, according to the word of the \LORD, as the \LORD commanded Moses.
\end{biblechapter}

\section*{The Kohathites}
\begin{biblechapter} % Numbers 4
\verse And the \LORD spake unto Moses and unto Aaron, saying,
\verse Take the sum of the sons of Kohath from among the sons of Levi, after their families, by the house of their fathers,
\verse From thirty years old and upward even until fifty years old, all that enter into the host, to do the work in the tabernacle of the congregation.
\verse This shall be the service of the sons of Kohath in the tabernacle of the congregation, about the most holy things:
\verse And when the camp setteth forward, Aaron shall come, and his sons, and they shall take down the covering vail, and cover the ark of testimony with it:
\verse And shall put thereon the covering of badgers' skins, and shall spread over it a cloth wholly of blue, and shall put in the staves thereof.
\verse And upon the table of shewbread they shall spread a cloth of blue, and put thereon the dishes, and the spoons, and the bowls, and covers to cover withal: and the continual bread shall be thereon:
\verse And they shall spread upon them a cloth of scarlet, and cover the same with a covering of badgers' skins, and shall put in the staves thereof.
\verse And they shall take a cloth of blue, and cover the candlestick of the light, and his lamps, and his tongs, and his snuffdishes, and all the oil vessels thereof, wherewith they minister unto it:
\verse And they shall put it and all the vessels thereof within a covering of badgers' skins, and shall put it upon a bar.
\verse And upon the golden altar they shall spread a cloth of blue, and cover it with a covering of badgers' skins, and shall put to the staves thereof:
\verse And they shall take all the instruments of ministry, wherewith they minister in the sanctuary, and put them in a cloth of blue, and cover them with a covering of badgers' skins, and shall put them on a bar:
\verse And they shall take away the ashes from the altar, and spread a purple cloth thereon:
\verse And they shall put upon it all the vessels thereof, wherewith they minister about it, even the censers, the fleshhooks, and the shovels, and the basons, all the vessels of the altar; and they shall spread upon it a covering of badgers' skins, and put to the staves of it.
\verse And when Aaron and his sons have made an end of covering the sanctuary, and all the vessels of the sanctuary, as the camp is to set forward; after that, the sons of Kohath shall come to bear it: but they shall not touch any holy thing, lest they die. These things are the burden of the sons of Kohath in the tabernacle of the congregation.
\verse And to the office of Eleazar the son of Aaron the priest pertaineth the oil for the light, and the sweet incense, and the daily meat offering, and the anointing oil, and the oversight of all the tabernacle, and of all that therein is, in the sanctuary, and in the vessels thereof.
\verse And the \LORD spake unto Moses and unto Aaron, saying,
\verse Cut ye not off the tribe of the families of the Kohathites from among the Levites:
\verse But thus do unto them, that they may live, and not die, when they approach unto the most holy things: Aaron and his sons shall go in, and appoint them every one to his service and to his burden:
\verse But they shall not go in to see when the holy things are covered, lest they die.
\section*{The Gershonites}
\verse And the \LORD spake unto Moses, saying,
\verse Take also the sum of the sons of Gershon, throughout the houses of their fathers, by their families;
\verse From thirty years old and upward until fifty years old shalt thou number them; all that enter in to perform the service, to do the work in the tabernacle of the congregation.
\verse This is the service of the families of the Gershonites, to serve, and for burdens:
\verse And they shall bear the curtains of the tabernacle, and the tabernacle of the congregation, his covering, and the covering of the badgers' skins that is above upon it, and the hanging for the door of the tabernacle of the congregation,
\verse And the hangings of the court, and the hanging for the door of the gate of the court, which is by the tabernacle and by the altar round about, and their cords, and all the instruments of their service, and all that is made for them: so shall they serve.
\verse At the appointment of Aaron and his sons shall be all the service of the sons of the Gershonites, in all their burdens, and in all their service: and ye shall appoint unto them in charge all their burdens.
\verse This is the service of the families of the sons of Gershon in the tabernacle of the congregation: and their charge shall be under the hand of Ithamar the son of Aaron the priest.
\section*{The Merarites}
\verse As for the sons of Merari, thou shalt number them after their families, by the house of their fathers;
\verse From thirty years old and upward even unto fifty years old shalt thou number them, every one that entereth into the service, to do the work of the tabernacle of the congregation.
\verse And this is the charge of their burden, according to all their service in the tabernacle of the congregation; the boards of the tabernacle, and the bars thereof, and the pillars thereof, and sockets thereof,
\verse And the pillars of the court round about, and their sockets, and their pins, and their cords, with all their instruments, and with all their service: and by name ye shall reckon the instruments of the charge of their burden.
\verse This is the service of the families of the sons of Merari, according to all their service, in the tabernacle of the congregation, under the hand of Ithamar the son of Aaron the priest.
\section*{The numbering of the Levite clans}
\verse And Moses and Aaron and the chief of the congregation numbered the sons of the Kohathites after their families, and after the house of their fathers,
\verse From thirty years old and upward even unto fifty years old, every one that entereth into the service, for the work in the tabernacle of the congregation:
\verse And those that were numbered of them by their families were two thousand seven hundred and fifty.
\verse These were they that were numbered of the families of the Kohathites, all that might do service in the tabernacle of the congregation, which Moses and Aaron did number according to the commandment of the \LORD by the hand of Moses.
\verse And those that were numbered of the sons of Gershon, throughout their families, and by the house of their fathers,
\verse From thirty years old and upward even unto fifty years old, every one that entereth into the service, for the work in the tabernacle of the congregation,
\verse Even those that were numbered of them, throughout their families, by the house of their fathers, were two thousand and six hundred and thirty.
\verse These are they that were numbered of the families of the sons of Gershon, of all that might do service in the tabernacle of the congregation, whom Moses and Aaron did number according to the commandment of the \LORD.
\verse And those that were numbered of the families of the sons of Merari, throughout their families, by the house of their fathers,
\verse From thirty years old and upward even unto fifty years old, every one that entereth into the service, for the work in the tabernacle of the congregation,
\verse Even those that were numbered of them after their families, were three thousand and two hundred.
\verse These be those that were numbered of the families of the sons of Merari, whom Moses and Aaron numbered according to the word of the \LORD by the hand of Moses.
\verse All those that were numbered of the Levites, whom Moses and Aaron and the chief of Israel numbered, after their families, and after the house of their fathers,
\verse From thirty years old and upward even unto fifty years old, every one that came to do the service of the ministry, and the service of the burden in the tabernacle of the congregation,
\verse Even those that were numbered of them, were eight thousand and five hundred and fourscore.
\verse According to the commandment of the \LORD they were numbered by the hand of Moses, every one according to his service, and according to his burden: thus were they numbered of him, as the \LORD commanded Moses.
\end{biblechapter}

\section*{The purity of the camp}
\begin{biblechapter} % Numbers 5
\verse And the \LORD spake unto Moses, saying,
\verse Command the children of Israel, that they put out of the camp every leper, and every one that hath an issue, and whosoever is defiled by the dead:
\verse Both male and female shall ye put out, without the camp shall ye put them; that they defile not their camps, in the midst whereof I dwell.
\verse And the children of Israel did so, and put them out without the camp: as the \LORD spake unto Moses, so did the children of Israel.
\section*{Restitution for wrongs}
\verse And the \LORD spake unto Moses, saying,
\verse Speak unto the children of Israel, When a man or woman shall commit any sin that men commit, to do a trespass against the \LORD, and that person be guilty;
\verse Then they shall confess their sin which they have done: and he shall recompense his trespass with the principal thereof, and add unto it the fifth part thereof, and give it unto him against whom he hath trespassed.
\verse But if the man have no kinsman to recompense the trespass unto, let the trespass be recompensed unto the \LORD, even to the priest; beside the ram of the atonement, whereby an atonement shall be made for him.
\verse And every offering of all the holy things of the children of Israel, which they bring unto the priest, shall be his.
\verse And every man's hallowed things shall be his: whatsoever any man giveth the priest, it shall be his.
\section*{The test for an unfaithful wife}
\verse And the \LORD spake unto Moses, saying,
\verse Speak unto the children of Israel, and say unto them, If any man's wife go aside, and commit a trespass against him,
\verse And a man lie with her carnally, and it be hid from the eyes of her husband, and be kept close, and she be defiled, and there be no witness against her, neither she be taken with the manner;
\verse And the spirit of jealousy come upon him, and he be jealous of his wife, and she be defiled: or if the spirit of jealousy come upon him, and he be jealous of his wife, and she be not defiled:
\verse Then shall the man bring his wife unto the priest, and he shall bring her offering for her, the tenth part of an ephah of barley meal; he shall pour no oil upon it, nor put frankincense thereon; for it is an offering of jealousy, an offering of memorial, bringing iniquity to remembrance.
\verse And the priest shall bring her near, and set her before the \LORD:
\verse And the priest shall take holy water in an earthen vessel; and of the dust that is in the floor of the tabernacle the priest shall take, and put it into the water:
\verse And the priest shall set the woman before the \LORD, and uncover the woman's head, and put the offering of memorial in her hands, which is the jealousy offering: and the priest shall have in his hand the bitter water that causeth the curse:
\verse And the priest shall charge her by an oath, and say unto the woman, If no man have lain with thee, and if thou hast not gone aside to uncleanness with another instead of thy husband, be thou free from this bitter water that causeth the curse:
\verse But if thou hast gone aside to another instead of thy husband, and if thou be defiled, and some man have lain with thee beside thine husband:
\verse Then the priest shall charge the woman with an oath of cursing, and the priest shall say unto the woman, The \LORD make thee a curse and an oath among thy people, when the \LORD doth make thy thigh to rot, and thy belly to swell;
\verse And this water that causeth the curse shall go into thy bowels, to make thy belly to swell, and thy thigh to rot: And the woman shall say, Amen, amen.
\verse And the priest shall write these curses in a book, and he shall blot them out with the bitter water:
\verse And he shall cause the woman to drink the bitter water that causeth the curse: and the water that causeth the curse shall enter into her, and become bitter.
\verse Then the priest shall take the jealousy offering out of the woman's hand, and shall wave the offering before the \LORD, and offer it upon the altar:
\verse And the priest shall take an handful of the offering, even the memorial thereof, and burn it upon the altar, and afterward shall cause the woman to drink the water.
\verse And when he hath made her to drink the water, then it shall come to pass, that, if she be defiled, and have done trespass against her husband, that the water that causeth the curse shall enter into her, and become bitter, and her belly shall swell, and her thigh shall rot: and the woman shall be a curse among her people.
\verse And if the woman be not defiled, but be clean; then she shall be free, and shall conceive seed.
\verse This is the law of jealousies, when a wife goeth aside to another instead of her husband, and is defiled;
\verse Or when the spirit of jealousy cometh upon him, and he be jealous over his wife, and shall set the woman before the \LORD, and the priest shall execute upon her all this law.
\verse Then shall the man be guiltless from iniquity, and this woman shall bear her iniquity.
\end{biblechapter}

\section*{The Nazirite}
\begin{biblechapter} % Numbers 6
\verse And the \LORD spake unto Moses, saying,
\verse Speak unto the children of Israel, and say unto them, When either man or woman shall separate themselves to vow a vow of a Nazarite, to separate themselves unto the \LORD:
\verse He shall separate himself from wine and strong drink, and shall drink no vinegar of wine, or vinegar of strong drink, neither shall he drink any liquor of grapes, nor eat moist grapes, or dried.
\verse All the days of his separation shall he eat nothing that is made of the vine tree, from the kernels even to the husk.
\verse All the days of the vow of his separation there shall no razor come upon his head: until the days be fulfilled, in the which he separateth himself unto the \LORD, he shall be holy, and shall let the locks of the hair of his head grow.
\verse All the days that he separateth himself unto the \LORD he shall come at no dead body.
\verse He shall not make himself unclean for his father, or for his mother, for his brother, or for his sister, when they die: because the consecration of his God is upon his head.
\verse All the days of his separation he is holy unto the \LORD.
\verse And if any man die very suddenly by him, and he hath defiled the head of his consecration; then he shall shave his head in the day of his cleansing, on the seventh day shall he shave it.
\verse And on the eighth day he shall bring two turtles, or two young pigeons, to the priest, to the door of the tabernacle of the congregation:
\verse And the priest shall offer the one for a sin offering, and the other for a burnt offering, and make an atonement for him, for that he sinned by the dead, and shall hallow his head that same day.
\verse And he shall consecrate unto the \LORD the days of his separation, and shall bring a lamb of the first year for a trespass offering: but the days that were before shall be lost, because his separation was defiled.
\verse And this is the law of the Nazarite, when the days of his separation are fulfilled: he shall be brought unto the door of the tabernacle of the congregation:
\verse And he shall offer his offering unto the \LORD, one he lamb of the first year without blemish for a burnt offering, and one ewe lamb of the first year without blemish for a sin offering, and one ram without blemish for peace offerings,
\verse And a basket of unleavened bread, cakes of fine flour mingled with oil, and wafers of unleavened bread anointed with oil, and their meat offering, and their drink offerings.
\verse And the priest shall bring them before the \LORD, and shall offer his sin offering, and his burnt offering:
\verse And he shall offer the ram for a sacrifice of peace offerings unto the \LORD, with the basket of unleavened bread: the priest shall offer also his meat offering, and his drink offering.
\verse And the Nazarite shall shave the head of his separation at the door of the tabernacle of the congregation, and shall take the hair of the head of his separation, and put it in the fire which is under the sacrifice of the peace offerings.
\verse And the priest shall take the sodden shoulder of the ram, and one unleavened cake out of the basket, and one unleavened wafer, and shall put them upon the hands of the Nazarite, after the hair of his separation is shaven:
\verse And the priest shall wave them for a wave offering before the \LORD: this is holy for the priest, with the wave breast and heave shoulder: and after that the Nazarite may drink wine.
\verse This is the law of the Nazarite who hath vowed, and of his offering unto the \LORD for his separation, beside that that his hand shall get: according to the vow which he vowed, so he must do after the law of his separation.
\section*{The priestly blessing}
\verse And the \LORD spake unto Moses, saying,
\verse Speak unto Aaron and unto his sons, saying, On this wise ye shall bless the children of Israel, saying unto them,
\verse The \LORD bless thee, and keep thee:
\verse The \LORD make his face shine upon thee, and be gracious unto thee:
\verse The \LORD lift up his countenance upon thee, and give thee peace.
\verse And they shall put my name upon the children of Israel; and I will bless them.
\end{biblechapter}

\section*{Offerings at the dedication of the tabernacle}
\begin{biblechapter} % Numbers 7
\verse And it came to pass on the day that Moses had fully set up the tabernacle, and had anointed it, and sanctified it, and all the instruments thereof, both the altar and all the vessels thereof, and had anointed them, and sanctified them;
\verse That the princes of Israel, heads of the house of their fathers, who were the princes of the tribes, and were over them that were numbered, offered:
\verse And they brought their offering before the \LORD, six covered wagons, and twelve oxen; a wagon for two of the princes, and for each one an ox: and they brought them before the tabernacle.
\verse And the \LORD spake unto Moses, saying,
\verse Take it of them, that they may be to do the service of the tabernacle of the congregation; and thou shalt give them unto the Levites, to every man according to his service.
\verse And Moses took the wagons and the oxen, and gave them unto the Levites.
\verse Two wagons and four oxen he gave unto the sons of Gershon, according to their service:
\verse And four wagons and eight oxen he gave unto the sons of Merari, according unto their service, under the hand of Ithamar the son of Aaron the priest.
\verse But unto the sons of Kohath he gave none: because the service of the sanctuary belonging unto them was that they should bear upon their shoulders.
\verse And the princes offered for dedicating of the altar in the day that it was anointed, even the princes offered their offering before the altar.
\verse And the \LORD said unto Moses, They shall offer their offering, each prince on his day, for the dedicating of the altar.
\verse And he that offered his offering the first day was Nahshon the son of Amminadab, of the tribe of Judah:
\verse And his offering was one silver charger, the weight thereof was an hundred and thirty shekels, one silver bowl of seventy shekels, after the shekel of the sanctuary; both of them were full of fine flour mingled with oil for a meat offering:
\verse One spoon of ten shekels of gold, full of incense:
\verse One young bullock, one ram, one lamb of the first year, for a burnt offering:
\verse One kid of the goats for a sin offering:
\verse And for a sacrifice of peace offerings, two oxen, five rams, five he goats, five lambs of the first year: this was the offering of Nahshon the son of Amminadab.
\verse On the second day Nethaneel the son of Zuar, prince of Issachar, did offer:
\verse He offered for his offering one silver charger, the weight whereof was an hundred and thirty shekels, one silver bowl of seventy shekels, after the shekel of the sanctuary; both of them full of fine flour mingled with oil for a meat offering:
\verse One spoon of gold of ten shekels, full of incense:
\verse One young bullock, one ram, one lamb of the first year, for a burnt offering:
\verse One kid of the goats for a sin offering:
\verse And for a sacrifice of peace offerings, two oxen, five rams, five he goats, five lambs of the first year: this was the offering of Nethaneel the son of Zuar.
\verse On the third day Eliab the son of Helon, prince of the children of Zebulun, did offer:
\verse His offering was one silver charger, the weight whereof was an hundred and thirty shekels, one silver bowl of seventy shekels, after the shekel of the sanctuary; both of them full of fine flour mingled with oil for a meat offering:
\verse One golden spoon of ten shekels, full of incense:
\verse One young bullock, one ram, one lamb of the first year, for a burnt offering:
\verse One kid of the goats for a sin offering:
\verse And for a sacrifice of peace offerings, two oxen, five rams, five he goats, five lambs of the first year: this was the offering of Eliab the son of Helon.
\verse On the fourth day Elizur the son of Shedeur, prince of the children of Reuben, did offer:
\verse His offering was one silver charger of the weight of an hundred and thirty shekels, one silver bowl of seventy shekels, after the shekel of the sanctuary; both of them full of fine flour mingled with oil for a meat offering:
\verse One golden spoon of ten shekels, full of incense:
\verse One young bullock, one ram, one lamb of the first year, for a burnt offering:
\verse One kid of the goats for a sin offering:
\verse And for a sacrifice of peace offerings, two oxen, five rams, five he goats, five lambs of the first year: this was the offering of Elizur the son of Shedeur.
\verse On the fifth day Shelumiel the son of Zurishaddai, prince of the children of Simeon, did offer:
\verse His offering was one silver charger, the weight whereof was an hundred and thirty shekels, one silver bowl of seventy shekels, after the shekel of the sanctuary; both of them full of fine flour mingled with oil for a meat offering:
\verse One golden spoon of ten shekels, full of incense:
\verse One young bullock, one ram, one lamb of the first year, for a burnt offering:
\verse One kid of the goats for a sin offering:
\verse And for a sacrifice of peace offerings, two oxen, five rams, five he goats, five lambs of the first year: this was the offering of Shelumiel the son of Zurishaddai.
\verse On the sixth day Eliasaph the son of Deuel, prince of the children of Gad, offered:
\verse His offering was one silver charger of the weight of an hundred and thirty shekels, a silver bowl of seventy shekels, after the shekel of the sanctuary; both of them full of fine flour mingled with oil for a meat offering:
\verse One golden spoon of ten shekels, full of incense:
\verse One young bullock, one ram, one lamb of the first year, for a burnt offering:
\verse One kid of the goats for a sin offering:
\verse And for a sacrifice of peace offerings, two oxen, five rams, five he goats, five lambs of the first year: this was the offering of Eliasaph the son of Deuel.
\verse On the seventh day Elishama the son of Ammihud, prince of the children of Ephraim, offered:
\verse His offering was one silver charger, the weight whereof was an hundred and thirty shekels, one silver bowl of seventy shekels, after the shekel of the sanctuary; both of them full of fine flour mingled with oil for a meat offering:
\verse One golden spoon of ten shekels, full of incense:
\verse One young bullock, one ram, one lamb of the first year, for a burnt offering:
\verse One kid of the goats for a sin offering:
\verse And for a sacrifice of peace offerings, two oxen, five rams, five he goats, five lambs of the first year: this was the offering of Elishama the son of Ammihud.
\verse On the eighth day offered Gamaliel the son of Pedahzur, prince of the children of Manasseh:
\verse His offering was one silver charger of the weight of an hundred and thirty shekels, one silver bowl of seventy shekels, after the shekel of the sanctuary; both of them full of fine flour mingled with oil for a meat offering:
\verse One golden spoon of ten shekels, full of incense:
\verse One young bullock, one ram, one lamb of the first year, for a burnt offering:
\verse One kid of the goats for a sin offering:
\verse And for a sacrifice of peace offerings, two oxen, five rams, five he goats, five lambs of the first year: this was the offering of Gamaliel the son of Pedahzur.
\verse On the ninth day Abidan the son of Gideoni, prince of the children of Benjamin, offered:
\verse His offering was one silver charger, the weight whereof was an hundred and thirty shekels, one silver bowl of seventy shekels, after the shekel of the sanctuary; both of them full of fine flour mingled with oil for a meat offering:
\verse One golden spoon of ten shekels, full of incense:
\verse One young bullock, one ram, one lamb of the first year, for a burnt offering:
\verse One kid of the goats for a sin offering:
\verse And for a sacrifice of peace offerings, two oxen, five rams, five he goats, five lambs of the first year: this was the offering of Abidan the son of Gideoni.
\verse On the tenth day Ahiezer the son of Ammishaddai, prince of the children of Dan, offered:
\verse His offering was one silver charger, the weight whereof was an hundred and thirty shekels, one silver bowl of seventy shekels, after the shekel of the sanctuary; both of them full of fine flour mingled with oil for a meat offering:
\verse One golden spoon of ten shekels, full of incense:
\verse One young bullock, one ram, one lamb of the first year, for a burnt offering:
\verse One kid of the goats for a sin offering:
\verse And for a sacrifice of peace offerings, two oxen, five rams, five he goats, five lambs of the first year: this was the offering of Ahiezer the son of Ammishaddai.
\verse On the eleventh day Pagiel the son of Ocran, prince of the children of Asher, offered:
\verse His offering was one silver charger, the weight whereof was an hundred and thirty shekels, one silver bowl of seventy shekels, after the shekel of the sanctuary; both of them full of fine flour mingled with oil for a meat offering:
\verse One golden spoon of ten shekels, full of incense:
\verse One young bullock, one ram, one lamb of the first year, for a burnt offering:
\verse One kid of the goats for a sin offering:
\verse And for a sacrifice of peace offerings, two oxen, five rams, five he goats, five lambs of the first year: this was the offering of Pagiel the son of Ocran.
\verse On the twelfth day Ahira the son of Enan, prince of the children of Naphtali, offered:
\verse His offering was one silver charger, the weight whereof was an hundred and thirty shekels, one silver bowl of seventy shekels, after the shekel of the sanctuary; both of them full of fine flour mingled with oil for a meat offering:
\verse One golden spoon of ten shekels, full of incense:
\verse One young bullock, one ram, one lamb of the first year, for a burnt offering:
\verse One kid of the goats for a sin offering:
\verse And for a sacrifice of peace offerings, two oxen, five rams, five he goats, five lambs of the first year: this was the offering of Ahira the son of Enan.
\verse This was the dedication of the altar, in the day when it was anointed, by the princes of Israel: twelve chargers of silver, twelve silver bowls, twelve spoons of gold:
\verse Each charger of silver weighing an hundred and thirty shekels, each bowl seventy: all the silver vessels weighed two thousand and four hundred shekels, after the shekel of the sanctuary:
\verse The golden spoons were twelve, full of incense, weighing ten shekels apiece, after the shekel of the sanctuary: all the gold of the spoons was an hundred and twenty shekels.
\verse All the oxen for the burnt offering were twelve bullocks, the rams twelve, the lambs of the first year twelve, with their meat offering: and the kids of the goats for sin offering twelve.
\verse And all the oxen for the sacrifice of the peace offerings were twenty and four bullocks, the rams sixty, the he goats sixty, the lambs of the first year sixty. This was the dedication of the altar, after that it was anointed.
\verse And when Moses was gone into the tabernacle of the congregation to speak with him, then he heard the voice of one speaking unto him from off the mercy seat that was upon the ark of testimony, from between the two cherubims: and he spake unto him.
\end{biblechapter}

\section*{Lighting the lamps}
\begin{biblechapter} % Numbers 8
\verse And the \LORD spake unto Moses, saying,
\verse Speak unto Aaron, and say unto him, When thou lightest the lamps, the seven lamps shall give light over against the candlestick.
\verse And Aaron did so; he lighted the lamps thereof over against the candlestick, as the \LORD commanded Moses.
\verse And this work of the candlestick was of beaten gold, unto the shaft thereof, unto the flowers thereof, was beaten work: according unto the pattern which the \LORD had shewed Moses, so he made the candlestick.
\section*{The setting apart of the Levites}
\verse And the \LORD spake unto Moses, saying,
\verse Take the Levites from among the children of Israel, and cleanse them.
\verse And thus shalt thou do unto them, to cleanse them: Sprinkle water of purifying upon them, and let them shave all their flesh, and let them wash their clothes, and so make themselves clean.
\verse Then let them take a young bullock with his meat offering, even fine flour mingled with oil, and another young bullock shalt thou take for a sin offering.
\verse And thou shalt bring the Levites before the tabernacle of the congregation: and thou shalt gather the whole assembly of the children of Israel together:
\verse And thou shalt bring the Levites before the \LORD: and the children of Israel shall put their hands upon the Levites:
\verse And Aaron shall offer the Levites before the \LORD for an offering of the children of Israel, that they may execute the service of the \LORD.
\verse And the Levites shall lay their hands upon the heads of the bullocks: and thou shalt offer the one for a sin offering, and the other for a burnt offering, unto the \LORD, to make an atonement for the Levites.
\verse And thou shalt set the Levites before Aaron, and before his sons, and offer them for an offering unto the \LORD.
\verse Thus shalt thou separate the Levites from among the children of Israel: and the Levites shall be mine.
\verse And after that shall the Levites go in to do the service of the tabernacle of the congregation: and thou shalt cleanse them, and offer them for an offering.
\verse For they are wholly given unto me from among the children of Israel; instead of such as open every womb, even instead of the firstborn of all the children of Israel, have I taken them unto me.
\verse For all the firstborn of the children of Israel are mine, both man and beast: on the day that I smote every firstborn in the land of Egypt I sanctified them for myself.
\verse And I have taken the Levites for all the firstborn of the children of Israel.
\verse And I have given the Levites as a gift to Aaron and to his sons from among the children of Israel, to do the service of the children of Israel in the tabernacle of the congregation, and to make an atonement for the children of Israel: that there be no plague among the children of Israel, when the children of Israel come nigh unto the sanctuary.
\verse And Moses, and Aaron, and all the congregation of the children of Israel, did to the Levites according unto all that the \LORD commanded Moses concerning the Levites, so did the children of Israel unto them.
\verse And the Levites were purified, and they washed their clothes; and Aaron offered them as an offering before the \LORD; and Aaron made an atonement for them to cleanse them.
\verse And after that went the Levites in to do their service in the tabernacle of the congregation before Aaron, and before his sons: as the \LORD had commanded Moses concerning the Levites, so did they unto them.
\verse And the \LORD spake unto Moses, saying,
\verse This is it that belongeth unto the Levites: from twenty and five years old and upward they shall go in to wait upon the service of the tabernacle of the congregation:
\verse And from the age of fifty years they shall cease waiting upon the service thereof, and shall serve no more:
\verse But shall minister with their brethren in the tabernacle of the congregation, to keep the charge, and shall do no service. Thus shalt thou do unto the Levites touching their charge.
\end{biblechapter}

\section*{The Passover}
\begin{biblechapter} % Numbers 9
\verse And the \LORD spake unto Moses in the wilderness of Sinai, in the first month of the second year after they were come out of the land of Egypt, saying,
\verse Let the children of Israel also keep the passover at his appointed season.
\verse In the fourteenth day of this month, at even, ye shall keep it in his appointed season: according to all the rites of it, and according to all the ceremonies thereof, shall ye keep it.
\verse And Moses spake unto the children of Israel, that they should keep the passover.
\verse And they kept the passover on the fourteenth day of the first month at even in the wilderness of Sinai: according to all that the \LORD commanded Moses, so did the children of Israel.
\verse And there were certain men, who were defiled by the dead body of a man, that they could not keep the passover on that day: and they came before Moses and before Aaron on that day:
\verse And those men said unto him, We are defiled by the dead body of a man: wherefore are we kept back, that we may not offer an offering of the \LORD in his appointed season among the children of Israel?
\verse And Moses said unto them, Stand still, and I will hear what the \LORD will command concerning you.
\verse And the \LORD spake unto Moses, saying,
\verse Speak unto the children of Israel, saying, If any man of you or of your posterity shall be unclean by reason of a dead body, or be in a journey afar off, yet he shall keep the passover unto the \LORD.
\verse The fourteenth day of the second month at even they shall keep it, and eat it with unleavened bread and bitter herbs.
\verse They shall leave none of it unto the morning, nor break any bone of it: according to all the ordinances of the passover they shall keep it.
\verse But the man that is clean, and is not in a journey, and forbeareth to keep the passover, even the same soul shall be cut off from among his people: because he brought not the offering of the \LORD in his appointed season, that man shall bear his sin.
\verse And if a stranger shall sojourn among you, and will keep the passover unto the \LORD; according to the ordinance of the passover, and according to the manner thereof, so shall he do: ye shall have one ordinance, both for the stranger, and for him that was born in the land.
\section*{The cloud above the tabernacle}
\verse And on the day that the tabernacle was reared up the cloud covered the tabernacle, namely, the tent of the testimony: and at even there was upon the tabernacle as it were the appearance of fire, until the morning.
\verse So it was alway: the cloud covered it by day, and the appearance of fire by night.
\verse And when the cloud was taken up from the tabernacle, then after that the children of Israel journeyed: and in the place where the cloud abode, there the children of Israel pitched their tents.
\verse At the commandment of the \LORD the children of Israel journeyed, and at the commandment of the \LORD they pitched: as long as the cloud abode upon the tabernacle they rested in their tents.
\verse And when the cloud tarried long upon the tabernacle many days, then the children of Israel kept the charge of the \LORD, and journeyed not.
\verse And so it was, when the cloud was a few days upon the tabernacle; according to the commandment of the \LORD they abode in their tents, and according to the commandment of the \LORD they journeyed.
\verse And so it was, when the cloud abode from even unto the morning, and that the cloud was taken up in the morning, then they journeyed: whether it was by day or by night that the cloud was taken up, they journeyed.
\verse Or whether it were two days, or a month, or a year, that the cloud tarried upon the tabernacle, remaining thereon, the children of Israel abode in their tents, and journeyed not: but when it was taken up, they journeyed.
\verse At the commandment of the \LORD they rested in the tents, and at the commandment of the \LORD they journeyed: they kept the charge of the \LORD, at the commandment of the \LORD by the hand of Moses.
\end{biblechapter}

\section*{The silver trumpets}
\begin{biblechapter} % Numbers 10
\verse And the \LORD spake unto Moses, saying,
\verse Make thee two trumpets of silver; of a whole piece shalt thou make them: that thou mayest use them for the calling of the assembly, and for the journeying of the camps.
\verse And when they shall blow with them, all the assembly shall assemble themselves to thee at the door of the tabernacle of the congregation.
\verse And if they blow but with one trumpet, then the princes, which are heads of the thousands of Israel, shall gather themselves unto thee.
\verse When ye blow an alarm, then the camps that lie on the east parts shall go forward.
\verse When ye blow an alarm the second time, then the camps that lie on the south side shall take their journey: they shall blow an alarm for their journeys.
\verse But when the congregation is to be gathered together, ye shall blow, but ye shall not sound an alarm.
\verse And the sons of Aaron, the priests, shall blow with the trumpets; and they shall be to you for an ordinance for ever throughout your generations.
\verse And if ye go to war in your land against the enemy that oppresseth you, then ye shall blow an alarm with the trumpets; and ye shall be remembered before the \LORD your God, and ye shall be saved from your enemies.
\verse Also in the day of your gladness, and in your solemn days, and in the beginnings of your months, ye shall blow with the trumpets over your burnt offerings, and over the sacrifices of your peace offerings; that they may be to you for a memorial before your God: I am the \LORD your God.
\section*{The Israelites leave Sinai}
\verse And it came to pass on the twentieth day of the second month, in the second year, that the cloud was taken up from off the tabernacle of the testimony.
\verse And the children of Israel took their journeys out of the wilderness of Sinai; and the cloud rested in the wilderness of Paran.
\verse And they first took their journey according to the commandment of the \LORD by the hand of Moses.
\verse In the first place went the standard of the camp of the children of Judah according to their armies: and over his host was Nahshon the son of Amminadab.
\verse And over the host of the tribe of the children of Issachar was Nethaneel the son of Zuar.
\verse And over the host of the tribe of the children of Zebulun was Eliab the son of Helon.
\verse And the tabernacle was taken down; and the sons of Gershon and the sons of Merari set forward, bearing the tabernacle.
\verse And the standard of the camp of Reuben set forward according to their armies: and over his host was Elizur the son of Shedeur.
\verse And over the host of the tribe of the children of Simeon was Shelumiel the son of Zurishaddai.
\verse And over the host of the tribe of the children of Gad was Eliasaph the son of Deuel.
\verse And the Kohathites set forward, bearing the sanctuary: and the other did set up the tabernacle against they came.
\verse And the standard of the camp of the children of Ephraim set forward according to their armies: and over his host was Elishama the son of Ammihud.
\verse And over the host of the tribe of the children of Manasseh was Gamaliel the son of Pedahzur.
\verse And over the host of the tribe of the children of Benjamin was Abidan the son of Gideoni.
\verse And the standard of the camp of the children of Dan set forward, which was the rereward of all the camps throughout their hosts: and over his host was Ahiezer the son of Ammishaddai.
\verse And over the host of the tribe of the children of Asher was Pagiel the son of Ocran.
\verse And over the host of the tribe of the children of Naphtali was Ahira the son of Enan.
\verse Thus were the journeyings of the children of Israel according to their armies, when they set forward.
\verse And Moses said unto Hobab, the son of Raguel the Midianite, Moses' father in law, We are journeying unto the place of which the \LORD said, I will give it you: come thou with us, and we will do thee good: for the \LORD hath spoken good concerning Israel.
\verse And he said unto him, I will not go; but I will depart to mine own land, and to my kindred.
\verse And he said, Leave us not, I pray thee; forasmuch as thou knowest how we are to encamp in the wilderness, and thou mayest be to us instead of eyes.
\verse And it shall be, if thou go with us, yea, it shall be, that what goodness the \LORD shall do unto us, the same will we do unto thee.
\verse And they departed from the mount of the \LORD three days' journey: and the ark of the covenant of the \LORD went before them in the three days' journey, to search out a resting place for them.
\verse And the cloud of the \LORD was upon them by day, when they went out of the camp.
\verse And it came to pass, when the ark set forward, that Moses said, Rise up, \LORD, and let thine enemies be scattered; and let them that hate thee flee before thee.
\verse And when it rested, he said, Return, O \LORD, unto the many thousands of Israel.
\end{biblechapter}

\section*{Fire from the \LORD}
\begin{biblechapter} % Numbers 11
\verse And when the people complained, it displeased the \LORD: and the \LORD heard it; and his anger was kindled; and the fire of the \LORD burnt among them, and consumed them that were in the uttermost parts of the camp.
\verse And the people cried unto Moses; and when Moses prayed unto the \LORD, the fire was quenched.
\verse And he called the name of the place Taberah: because the fire of the \LORD burnt among them.
\section*{Quail from the \LORD}
\verse And the mixt multitude that was among them fell a lusting: and the children of Israel also wept again, and said, Who shall give us flesh to eat?
\verse We remember the fish, which we did eat in Egypt freely; the cucumbers, and the melons, and the leeks, and the onions, and the garlick:
\verse But now our soul is dried away: there is nothing at all, beside this manna, before our eyes.
\verse And the manna was as coriander seed, and the colour thereof as the colour of bdellium.
\verse And the people went about, and gathered it, and ground it in mills, or beat it in a mortar, and baked it in pans, and made cakes of it: and the taste of it was as the taste of fresh oil.
\verse And when the dew fell upon the camp in the night, the manna fell upon it.
\verse Then Moses heard the people weep throughout their families, every man in the door of his tent: and the anger of the \LORD was kindled greatly; Moses also was displeased.
\verse And Moses said unto the \LORD, Wherefore hast thou afflicted thy servant? and wherefore have I not found favour in thy sight, that thou layest the burden of all this people upon me?
\verse Have I conceived all this people? have I begotten them, that thou shouldest say unto me, Carry them in thy bosom, as a nursing father beareth the sucking child, unto the land which thou swarest unto their fathers?
\verse Whence should I have flesh to give unto all this people? for they weep unto me, saying, Give us flesh, that we may eat.
\verse I am not able to bear all this people alone, because it is too heavy for me.
\verse And if thou deal thus with me, kill me, I pray thee, out of hand, if I have found favour in thy sight; and let me not see my wretchedness.
\verse And the \LORD said unto Moses, Gather unto me seventy men of the elders of Israel, whom thou knowest to be the elders of the people, and officers over them; and bring them unto the tabernacle of the congregation, that they may stand there with thee.
\verse And I will come down and talk with thee there: and I will take of the spirit which is upon thee, and will put it upon them; and they shall bear the burden of the people with thee, that thou bear it not thyself alone.
\verse And say thou unto the people, Sanctify yourselves against to morrow, and ye shall eat flesh: for ye have wept in the ears of the \LORD, saying, Who shall give us flesh to eat? for it was well with us in Egypt: therefore the \LORD will give you flesh, and ye shall eat.
\verse Ye shall not eat one day, nor two days, nor five days, neither ten days, nor twenty days;
\verse But even a whole month, until it come out at your nostrils, and it be loathsome unto you: because that ye have despised the \LORD which is among you, and have wept before him, saying, Why came we forth out of Egypt?
\verse And Moses said, The people, among whom I am, are six hundred thousand footmen; and thou hast said, I will give them flesh, that they may eat a whole month.
\verse Shall the flocks and the herds be slain for them, to suffice them? or shall all the fish of the sea be gathered together for them, to suffice them?
\verse And the \LORD said unto Moses, Is the \LORDs hand waxed short? thou shalt see now whether my word shall come to pass unto thee or not.
\verse And Moses went out, and told the people the words of the \LORD, and gathered the seventy men of the elders of the people, and set them round about the tabernacle.
\verse And the \LORD came down in a cloud, and spake unto him, and took of the spirit that was upon him, and gave it unto the seventy elders: and it came to pass, that, when the spirit rested upon them, they prophesied, and did not cease.
\verse But there remained two of the men in the camp, the name of the one was Eldad, and the name of the other Medad: and the spirit rested upon them; and they were of them that were written, but went not out unto the tabernacle: and they prophesied in the camp.
\verse And there ran a young man, and told Moses, and said, Eldad and Medad do prophesy in the camp.
\verse And Joshua the son of Nun, the servant of Moses, one of his young men, answered and said, My lord Moses, forbid them.
\verse And Moses said unto him, Enviest thou for my sake? would God that all the \LORDs people were prophets, and that the \LORD would put his spirit upon them!
\verse And Moses gat him into the camp, he and the elders of Israel.
\verse And there went forth a wind from the \LORD, and brought quails from the sea, and let them fall by the camp, as it were a day's journey on this side, and as it were a day's journey on the other side, round about the camp, and as it were two cubits high upon the face of the earth.
\verse And the people stood up all that day, and all that night, and all the next day, and they gathered the quails: he that gathered least gathered ten homers: and they spread them all abroad for themselves round about the camp.
\verse And while the flesh was yet between their teeth, ere it was chewed, the wrath of the \LORD was kindled against the people, and the \LORD smote the people with a very great plague.
\verse And he called the name of that place Kibrothhattaavah: because there they buried the people that lusted.
\verse And the people journeyed from Kibrothhattaavah unto Hazeroth; and abode at Hazeroth.
\end{biblechapter}

\section*{Miriam and Aaron oppose Moses}
\begin{biblechapter} % Numbers 12
\verse And Miriam and Aaron spake against Moses because of the Ethiopian woman whom he had married: for he had married an Ethiopian woman.
\verse And they said, Hath the \LORD indeed spoken only by Moses? hath he not spoken also by us? And the \LORD heard it.
\verse (Now the man Moses was very meek, above all the men which were upon the face of the earth.)
\verse And the \LORD spake suddenly unto Moses, and unto Aaron, and unto Miriam, Come out ye three unto the tabernacle of the congregation. And they three came out.
\verse And the \LORD came down in the pillar of the cloud, and stood in the door of the tabernacle, and called Aaron and Miriam: and they both came forth.
\verse And he said, Hear now my words: If there be a prophet among you, I the \LORD will make myself known unto him in a vision, and will speak unto him in a dream.
\verse My servant Moses is not so, who is faithful in all mine house.
\verse With him will I speak mouth to mouth, even apparently, and not in dark speeches; and the similitude of the \LORD shall he behold: wherefore then were ye not afraid to speak against my servant Moses?
\verse And the anger of the \LORD was kindled against them; and he departed.
\verse And the cloud departed from off the tabernacle; and, behold, Miriam became leprous, white as snow: and Aaron looked upon Miriam, and, behold, she was leprous.
\verse And Aaron said unto Moses, Alas, my lord, I beseech thee, lay not the sin upon us, wherein we have done foolishly, and wherein we have sinned.
\verse Let her not be as one dead, of whom the flesh is half consumed when he cometh out of his mother's womb.
\verse And Moses cried unto the \LORD, saying, Heal her now, O God, I beseech thee.
\verse And the \LORD said unto Moses, If her father had but spit in her face, should she not be ashamed seven days? let her be shut out from the camp seven days, and after that let her be received in again.
\verse And Miriam was shut out from the camp seven days: and the people journeyed not till Miriam was brought in again.
\verse And afterward the people removed from Hazeroth, and pitched in the wilderness of Paran.
\end{biblechapter}

\section*{Exploring Canaan}
\begin{biblechapter} % Numbers 13
\verse And the \LORD spake unto Moses, saying,
\verse Send thou men, that they may search the land of Canaan, which I give unto the children of Israel: of every tribe of their fathers shall ye send a man, every one a ruler among them.
\verse And Moses by the commandment of the \LORD sent them from the wilderness of Paran: all those men were heads of the children of Israel.
\verse And these were their names: of the tribe of Reuben, Shammua the son of Zaccur.
\verse Of the tribe of Simeon, Shaphat the son of Hori.
\verse Of the tribe of Judah, Caleb the son of Jephunneh.
\verse Of the tribe of Issachar, Igal the son of Joseph.
\verse Of the tribe of Ephraim, Oshea the son of Nun.
\verse Of the tribe of Benjamin, Palti the son of Raphu.
\verse Of the tribe of Zebulun, Gaddiel the son of Sodi.
\verse Of the tribe of Joseph, namely, of the tribe of Manasseh, Gaddi the son of Susi.
\verse Of the tribe of Dan, Ammiel the son of Gemalli.
\verse Of the tribe of Asher, Sethur the son of Michael.
\verse Of the tribe of Naphtali, Nahbi the son of Vophsi.
\verse Of the tribe of Gad, Geuel the son of Machi.
\verse These are the names of the men which Moses sent to spy out the land. And Moses called Oshea the son of Nun Jehoshua.
\verse And Moses sent them to spy out the land of Canaan, and said unto them, Get you up this way southward, and go up into the mountain:
\verse And see the land, what it is; and the people that dwelleth therein, whether they be strong or weak, few or many;
\verse And what the land is that they dwell in, whether it be good or bad; and what cities they be that they dwell in, whether in tents, or in strong holds;
\verse And what the land is, whether it be fat or lean, whether there be wood therein, or not. And be ye of good courage, and bring of the fruit of the land. Now the time was the time of the firstripe grapes.
\verse So they went up, and searched the land from the wilderness of Zin unto Rehob, as men come to Hamath.
\verse And they ascended by the south, and came unto Hebron; where Ahiman, Sheshai, and Talmai, the children of Anak, were. (Now Hebron was built seven years before Zoan in Egypt.)
\verse And they came unto the brook of Eshcol, and cut down from thence a branch with one cluster of grapes, and they bare it between two upon a staff; and they brought of the pomegranates, and of the figs.
\verse The place was called the brook Eshcol, because of the cluster of grapes which the children of Israel cut down from thence.
\verse And they returned from searching of the land after forty days.
\section*{Report on the exploration}
\verse And they went and came to Moses, and to Aaron, and to all the congregation of the children of Israel, unto the wilderness of Paran, to Kadesh; and brought back word unto them, and unto all the congregation, and shewed them the fruit of the land.
\verse And they told him, and said, We came unto the land whither thou sentest us, and surely it floweth with milk and honey; and this is the fruit of it.
\verse Nevertheless the people be strong that dwell in the land, and the cities are walled, and very great: and moreover we saw the children of Anak there.
\verse The Amalekites dwell in the land of the south: and the Hittites, and the Jebusites, and the Amorites, dwell in the mountains: and the Canaanites dwell by the sea, and by the coast of Jordan.
\verse And Caleb stilled the people before Moses, and said, Let us go up at once, and possess it; for we are well able to overcome it.
\verse But the men that went up with him said, We be not able to go up against the people; for they are stronger than we.
\verse And they brought up an evil report of the land which they had searched unto the children of Israel, saying, The land, through which we have gone to search it, is a land that eateth up the inhabitants thereof; and all the people that we saw in it are men of a great stature.
\verse And there we saw the giants, the sons of Anak, which come of the giants: and we were in our own sight as grasshoppers, and so we were in their sight.
\end{biblechapter}

\section*{The people rebel}
\begin{biblechapter} % Numbers 14
\verse And all the congregation lifted up their voice, and cried; and the people wept that night.
\verse And all the children of Israel murmured against Moses and against Aaron: and the whole congregation said unto them, Would God that we had died in the land of Egypt! or would God we had died in this wilderness!
\verse And wherefore hath the \LORD brought us unto this land, to fall by the sword, that our wives and our children should be a prey? were it not better for us to return into Egypt?
\verse And they said one to another, Let us make a captain, and let us return into Egypt.
\verse Then Moses and Aaron fell on their faces before all the assembly of the congregation of the children of Israel.
\verse And Joshua the son of Nun, and Caleb the son of Jephunneh, which were of them that searched the land, rent their clothes:
\verse And they spake unto all the company of the children of Israel, saying, The land, which we passed through to search it, is an exceeding good land.
\verse If the \LORD delight in us, then he will bring us into this land, and give it us; a land which floweth with milk and honey.
\verse Only rebel not ye against the \LORD, neither fear ye the people of the land; for they are bread for us: their defence is departed from them, and the \LORD is with us: fear them not.
\verse But all the congregation bade stone them with stones. And the glory of the \LORD appeared in the tabernacle of the congregation before all the children of Israel.
\verse And the \LORD said unto Moses, How long will this people provoke me? and how long will it be ere they believe me, for all the signs which I have shewed among them?
\verse I will smite them with the pestilence, and disinherit them, and will make of thee a greater nation and mightier than they.
\verse And Moses said unto the \LORD, Then the Egyptians shall hear it, (for thou broughtest up this people in thy might from among them;)
\verse And they will tell it to the inhabitants of this land: for they have heard that thou \LORD art among this people, that thou \LORD art seen face to face, and that thy cloud standeth over them, and that thou goest before them, by day time in a pillar of a cloud, and in a pillar of fire by night.
\verse Now if thou shalt kill all this people as one man, then the nations which have heard the fame of thee will speak, saying,
\verse Because the \LORD was not able to bring this people into the land which he sware unto them, therefore he hath slain them in the wilderness.
\verse And now, I beseech thee, let the power of my Lord be great, according as thou hast spoken, saying,
\verse The \LORD is longsuffering, and of great mercy, forgiving iniquity and transgression, and by no means clearing the guilty, visiting the iniquity of the fathers upon the children unto the third and fourth generation.
\verse Pardon, I beseech thee, the iniquity of this people according unto the greatness of thy mercy, and as thou hast forgiven this people, from Egypt even until now.
\verse And the \LORD said, I have pardoned according to thy word:
\verse But as truly as I live, all the earth shall be filled with the glory of the \LORD.
\verse Because all those men which have seen my glory, and my miracles, which I did in Egypt and in the wilderness, and have tempted me now these ten times, and have not hearkened to my voice;
\verse Surely they shall not see the land which I sware unto their fathers, neither shall any of them that provoked me see it:
\verse But my servant Caleb, because he had another spirit with him, and hath followed me fully, him will I bring into the land whereinto he went; and his seed shall possess it.
\verse (Now the Amalekites and the Canaanites dwelt in the valley.) To morrow turn you, and get you into the wilderness by the way of the Red sea.
\verse And the \LORD spake unto Moses and unto Aaron, saying,
\verse How long shall I bear with this evil congregation, which murmur against me? I have heard the murmurings of the children of Israel, which they murmur against me.
\verse Say unto them, As truly as I live, saith the \LORD, as ye have spoken in mine ears, so will I do to you:
\verse Your carcases shall fall in this wilderness; and all that were numbered of you, according to your whole number, from twenty years old and upward, which have murmured against me,
\verse Doubtless ye shall not come into the land, concerning which I sware to make you dwell therein, save Caleb the son of Jephunneh, and Joshua the son of Nun.
\verse But your little ones, which ye said should be a prey, them will I bring in, and they shall know the land which ye have despised.
\verse But as for you, your carcases, they shall fall in this wilderness.
\verse And your children shall wander in the wilderness forty years, and bear your whoredoms, until your carcases be wasted in the wilderness.
\verse After the number of the days in which ye searched the land, even forty days, each day for a year, shall ye bear your iniquities, even forty years, and ye shall know my breach of promise.
\verse I the \LORD have said, I will surely do it unto all this evil congregation, that are gathered together against me: in this wilderness they shall be consumed, and there they shall die.
\verse And the men, which Moses sent to search the land, who returned, and made all the congregation to murmur against him, by bringing up a slander upon the land,
\verse Even those men that did bring up the evil report upon the land, died by the plague before the \LORD.
\verse But Joshua the son of Nun, and Caleb the son of Jephunneh, which were of the men that went to search the land, lived still.
\verse And Moses told these sayings unto all the children of Israel: and the people mourned greatly.
\verse And they rose up early in the morning, and gat them up into the top of the mountain, saying, Lo, we be here, and will go up unto the place which the \LORD hath promised: for we have sinned.
\verse And Moses said, Wherefore now do ye transgress the commandment of the \LORD? but it shall not prosper.
\verse Go not up, for the \LORD is not among you; that ye be not smitten before your enemies.
\verse For the Amalekites and the Canaanites are there before you, and ye shall fall by the sword: because ye are turned away from the \LORD, therefore the \LORD will not be with you.
\verse But they presumed to go up unto the hill top: nevertheless the ark of the covenant of the \LORD, and Moses, departed not out of the camp.
\verse Then the Amalekites came down, and the Canaanites which dwelt in that hill, and smote them, and discomfited them, even unto Hormah.
\end{biblechapter}

\section*{Supplementary offerings}
\begin{biblechapter} % Numbers 15
\verse And the \LORD spake unto Moses, saying,
\verse Speak unto the children of Israel, and say unto them, When ye be come into the land of your habitations, which I give unto you,
\verse And will make an offering by fire unto the \LORD, a burnt offering, or a sacrifice in performing a vow, or in a freewill offering, or in your solemn feasts, to make a sweet savour unto the \LORD, of the herd, or of the flock:
\verse Then shall he that offereth his offering unto the \LORD bring a meat offering of a tenth deal of flour mingled with the fourth part of an hin of oil.
\verse And the fourth part of an hin of wine for a drink offering shalt thou prepare with the burnt offering or sacrifice, for one lamb.
\verse Or for a ram, thou shalt prepare for a meat offering two tenth deals of flour mingled with the third part of an hin of oil.
\verse And for a drink offering thou shalt offer the third part of an hin of wine, for a sweet savour unto the \LORD.
\verse And when thou preparest a bullock for a burnt offering, or for a sacrifice in performing a vow, or peace offerings unto the \LORD:
\verse Then shall he bring with a bullock a meat offering of three tenth deals of flour mingled with half an hin of oil.
\verse And thou shalt bring for a drink offering half an hin of wine, for an offering made by fire, of a sweet savour unto the \LORD.
\verse Thus shall it be done for one bullock, or for one ram, or for a lamb, or a kid.
\verse According to the number that ye shall prepare, so shall ye do to every one according to their number.
\verse All that are born of the country shall do these things after this manner, in offering an offering made by fire, of a sweet savour unto the \LORD.
\verse And if a stranger sojourn with you, or whosoever be among you in your generations, and will offer an offering made by fire, of a sweet savour unto the \LORD; as ye do, so he shall do.
\verse One ordinance shall be both for you of the congregation, and also for the stranger that sojourneth with you, an ordinance for ever in your generations: as ye are, so shall the stranger be before the \LORD.
\verse One law and one manner shall be for you, and for the stranger that sojourneth with you.
\verse And the \LORD spake unto Moses, saying,
\verse Speak unto the children of Israel, and say unto them, When ye come into the land whither I bring you,
\verse Then it shall be, that, when ye eat of the bread of the land, ye shall offer up an heave offering unto the \LORD.
\verse Ye shall offer up a cake of the first of your dough for an heave offering: as ye do the heave offering of the threshingfloor, so shall ye heave it.
\verse Of the first of your dough ye shall give unto the \LORD an heave offering in your generations.
\section*{Offerings for unintentional sins}
\verse And if ye have erred, and not observed all these commandments, which the \LORD hath spoken unto Moses,
\verse Even all that the \LORD hath commanded you by the hand of Moses, from the day that the \LORD commanded Moses, and henceforward among your generations;
\verse Then it shall be, if ought be committed by ignorance without the knowledge of the congregation, that all the congregation shall offer one young bullock for a burnt offering, for a sweet savour unto the \LORD, with his meat offering, and his drink offering, according to the manner, and one kid of the goats for a sin offering.
\verse And the priest shall make an atonement for all the congregation of the children of Israel, and it shall be forgiven them; for it is ignorance: and they shall bring their offering, a sacrifice made by fire unto the \LORD, and their sin offering before the \LORD, for their ignorance:
\verse And it shall be forgiven all the congregation of the children of Israel, and the stranger that sojourneth among them; seeing all the people were in ignorance.
\verse And if any soul sin through ignorance, then he shall bring a she goat of the first year for a sin offering.
\verse And the priest shall make an atonement for the soul that sinneth ignorantly, when he sinneth by ignorance before the \LORD, to make an atonement for him; and it shall be forgiven him.
\verse Ye shall have one law for him that sinneth through ignorance, both for him that is born among the children of Israel, and for the stranger that sojourneth among them.
\verse But the soul that doeth ought presumptuously, whether he be born in the land, or a stranger, the same reproacheth the \LORD; and that soul shall be cut off from among his people.
\verse Because he hath despised the word of the \LORD, and hath broken his commandment, that soul shall utterly be cut off; his iniquity shall be upon him.
\section*{The Sabbath-breaker put to death}
\verse And while the children of Israel were in the wilderness, they found a man that gathered sticks upon the sabbath day.
\verse And they that found him gathering sticks brought him unto Moses and Aaron, and unto all the congregation.
\verse And they put him in ward, because it was not declared what should be done to him.
\verse And the \LORD said unto Moses, The man shall be surely put to death: all the congregation shall stone him with stones without the camp.
\verse And all the congregation brought him without the camp, and stoned him with stones, and he died; as the \LORD commanded Moses.
\section*{Fringes on garments}
\verse And the \LORD spake unto Moses, saying,
\verse Speak unto the children of Israel, and bid them that they make them fringes in the borders of their garments throughout their generations, and that they put upon the fringe of the borders a ribband of blue:
\verse And it shall be unto you for a fringe, that ye may look upon it, and remember all the commandments of the \LORD, and do them; and that ye seek not after your own heart and your own eyes, after which ye use to go a whoring:
\verse That ye may remember, and do all my commandments, and be holy unto your God.
\verse I am the \LORD your God, which brought you out of the land of Egypt, to be your God: I am the \LORD your God.
\end{biblechapter}

\section*{Korah, Dathan, and Abiram}
\begin{biblechapter} % Numbers 16
\verse Now Korah, the son of Izhar, the son of Kohath, the son of Levi, and Dathan and Abiram, the sons of Eliab, and On, the son of Peleth, sons of Reuben, took men:
\verse And they rose up before Moses, with certain of the children of Israel, two hundred and fifty princes of the assembly, famous in the congregation, men of renown:
\verse And they gathered themselves together against Moses and against Aaron, and said unto them, Ye take too much upon you, seeing all the congregation are holy, every one of them, and the \LORD is among them: wherefore then lift ye up yourselves above the congregation of the \LORD?
\verse And when Moses heard it, he fell upon his face:
\verse And he spake unto Korah and unto all his company, saying, Even to morrow the \LORD will shew who are his, and who is holy; and will cause him to come near unto him: even him whom he hath chosen will he cause to come near unto him.
\verse This do; Take you censers, Korah, and all his company;
\verse And put fire therein, and put incense in them before the \LORD to morrow: and it shall be that the man whom the \LORD doth choose, he shall be holy: ye take too much upon you, ye sons of Levi.
\verse And Moses said unto Korah, Hear, I pray you, ye sons of Levi:
\verse Seemeth it but a small thing unto you, that the God of Israel hath separated you from the congregation of Israel, to bring you near to himself to do the service of the tabernacle of the \LORD, and to stand before the congregation to minister unto them?
\verse And he hath brought thee near to him, and all thy brethren the sons of Levi with thee: and seek ye the priesthood also?
\verse For which cause both thou and all thy company are gathered together against the \LORD: and what is Aaron, that ye murmur against him?
\verse And Moses sent to call Dathan and Abiram, the sons of Eliab: which said, We will not come up:
\verse Is it a small thing that thou hast brought us up out of a land that floweth with milk and honey, to kill us in the wilderness, except thou make thyself altogether a prince over us?
\verse Moreover thou hast not brought us into a land that floweth with milk and honey, or given us inheritance of fields and vineyards: wilt thou put out the eyes of these men? we will not come up.
\verse And Moses was very wroth, and said unto the \LORD, Respect not thou their offering: I have not taken one ass from them, neither have I hurt one of them.
\verse And Moses said unto Korah, Be thou and all thy company before the \LORD, thou, and they, and Aaron, to morrow:
\verse And take every man his censer, and put incense in them, and bring ye before the \LORD every man his censer, two hundred and fifty censers; thou also, and Aaron, each of you his censer.
\verse And they took every man his censer, and put fire in them, and laid incense thereon, and stood in the door of the tabernacle of the congregation with Moses and Aaron.
\verse And Korah gathered all the congregation against them unto the door of the tabernacle of the congregation: and the glory of the \LORD appeared unto all the congregation.
\verse And the \LORD spake unto Moses and unto Aaron, saying,
\verse Separate yourselves from among this congregation, that I may consume them in a moment.
\verse And they fell upon their faces, and said, O God, the God of the spirits of all flesh, shall one man sin, and wilt thou be wroth with all the congregation?
\verse And the \LORD spake unto Moses, saying,
\verse Speak unto the congregation, saying, Get you up from about the tabernacle of Korah, Dathan, and Abiram.
\verse And Moses rose up and went unto Dathan and Abiram; and the elders of Israel followed him.
\verse And he spake unto the congregation, saying, Depart, I pray you, from the tents of these wicked men, and touch nothing of theirs, lest ye be consumed in all their sins.
\verse So they gat up from the tabernacle of Korah, Dathan, and Abiram, on every side: and Dathan and Abiram came out, and stood in the door of their tents, and their wives, and their sons, and their little children.
\verse And Moses said, Hereby ye shall know that the \LORD hath sent me to do all these works; for I have not done them of mine own mind.
\verse If these men die the common death of all men, or if they be visited after the visitation of all men; then the \LORD hath not sent me.
\verse But if the \LORD make a new thing, and the earth open her mouth, and swallow them up, with all that appertain unto them, and they go down quick into the pit; then ye shall understand that these men have provoked the \LORD.
\verse And it came to pass, as he had made an end of speaking all these words, that the ground clave asunder that was under them:
\verse And the earth opened her mouth, and swallowed them up, and their houses, and all the men that appertained unto Korah, and all their goods.
\verse They, and all that appertained to them, went down alive into the pit, and the earth closed upon them: and they perished from among the congregation.
\verse And all Israel that were round about them fled at the cry of them: for they said, Lest the earth swallow us up also.
\verse And there came out a fire from the \LORD, and consumed the two hundred and fifty men that offered incense.
\verse And the \LORD spake unto Moses, saying,
\verse Speak unto Eleazar the son of Aaron the priest, that he take up the censers out of the burning, and scatter thou the fire yonder; for they are hallowed.
\verse The censers of these sinners against their own souls, let them make them broad plates for a covering of the altar: for they offered them before the \LORD, therefore they are hallowed: and they shall be a sign unto the children of Israel.
\verse And Eleazar the priest took the brasen censers, wherewith they that were burnt had offered; and they were made broad plates for a covering of the altar:
\verse To be a memorial unto the children of Israel, that no stranger, which is not of the seed of Aaron, come near to offer incense before the \LORD; that he be not as Korah, and as his company: as the \LORD said to him by the hand of Moses.
\verse But on the morrow all the congregation of the children of Israel murmured against Moses and against Aaron, saying, Ye have killed the people of the \LORD.
\verse And it came to pass, when the congregation was gathered against Moses and against Aaron, that they looked toward the tabernacle of the congregation: and, behold, the cloud covered it, and the glory of the \LORD appeared.
\verse And Moses and Aaron came before the tabernacle of the congregation.
\verse And the \LORD spake unto Moses, saying,
\verse Get you up from among this congregation, that I may consume them as in a moment. And they fell upon their faces.
\verse And Moses said unto Aaron, Take a censer, and put fire therein from off the altar, and put on incense, and go quickly unto the congregation, and make an atonement for them: for there is wrath gone out from the \LORD; the plague is begun.
\verse And Aaron took as Moses commanded, and ran into the midst of the congregation; and, behold, the plague was begun among the people: and he put on incense, and made an atonement for the people.
\verse And he stood between the dead and the living; and the plague was stayed.
\verse Now they that died in the plague were fourteen thousand and seven hundred, beside them that died about the matter of Korah.
\verse And Aaron returned unto Moses unto the door of the tabernacle of the congregation: and the plague was stayed.
\end{biblechapter}

\section*{The budding of Aaron's staff}
\begin{biblechapter} % Numbers 17
\verse And the \LORD spake unto Moses, saying,
\verse Speak unto the children of Israel, and take of every one of them a rod according to the house of their fathers, of all their princes according to the house of their fathers twelve rods: write thou every man's name upon his rod.
\verse And thou shalt write Aaron's name upon the rod of Levi: for one rod shall be for the head of the house of their fathers.
\verse And thou shalt lay them up in the tabernacle of the congregation before the testimony, where I will meet with you.
\verse And it shall come to pass, that the man's rod, whom I shall choose, shall blossom: and I will make to cease from me the murmurings of the children of Israel, whereby they murmur against you.
\verse And Moses spake unto the children of Israel, and every one of their princes gave him a rod apiece, for each prince one, according to their fathers' houses, even twelve rods: and the rod of Aaron was among their rods.
\verse And Moses laid up the rods before the \LORD in the tabernacle of witness.
\verse And it came to pass, that on the morrow Moses went into the tabernacle of witness; and, behold, the rod of Aaron for the house of Levi was budded, and brought forth buds, and bloomed blossoms, and yielded almonds.
\verse And Moses brought out all the rods from before the \LORD unto all the children of Israel: and they looked, and took every man his rod.
\verse And the \LORD said unto Moses, Bring Aaron's rod again before the testimony, to be kept for a token against the rebels; and thou shalt quite take away their murmurings from me, that they die not.
\verse And Moses did so: as the \LORD commanded him, so did he.
\verse And the children of Israel spake unto Moses, saying, Behold, we die, we perish, we all perish.
\verse Whosoever cometh any thing near unto the tabernacle of the \LORD shall die: shall we be consumed with dying?
\end{biblechapter}

\section*{Duties of priests and Levites}
\begin{biblechapter} % Numbers 18
\verse And the \LORD said unto Aaron, Thou and thy sons and thy father's house with thee shall bear the iniquity of the sanctuary: and thou and thy sons with thee shall bear the iniquity of your priesthood.
\verse And thy brethren also of the tribe of Levi, the tribe of thy father, bring thou with thee, that they may be joined unto thee, and minister unto thee: but thou and thy sons with thee shall minister before the tabernacle of witness.
\verse And they shall keep thy charge, and the charge of all the tabernacle: only they shall not come nigh the vessels of the sanctuary and the altar, that neither they, nor ye also, die.
\verse And they shall be joined unto thee, and keep the charge of the tabernacle of the congregation, for all the service of the tabernacle: and a stranger shall not come nigh unto you.
\verse And ye shall keep the charge of the sanctuary, and the charge of the altar: that there be no wrath any more upon the children of Israel.
\verse And I, behold, I have taken your brethren the Levites from among the children of Israel: to you they are given as a gift for the \LORD, to do the service of the tabernacle of the congregation.
\verse Therefore thou and thy sons with thee shall keep your priest's office for every thing of the altar, and within the vail; and ye shall serve: I have given your priest's office unto you as a service of gift: and the stranger that cometh nigh shall be put to death.
\section*{Offerings for priests and Levites}
\verse And the \LORD spake unto Aaron, Behold, I also have given thee the charge of mine heave offerings of all the hallowed things of the children of Israel; unto thee have I given them by reason of the anointing, and to thy sons, by an ordinance for ever.
\verse This shall be thine of the most holy things, reserved from the fire: every oblation of theirs, every meat offering of theirs, and every sin offering of theirs, and every trespass offering of theirs, which they shall render unto me, shall be most holy for thee and for thy sons.
\verse In the most holy place shalt thou eat it; every male shall eat it: it shall be holy unto thee.
\verse And this is thine; the heave offering of their gift, with all the wave offerings of the children of Israel: I have given them unto thee, and to thy sons and to thy daughters with thee, by a statute for ever: every one that is clean in thy house shall eat of it.
\verse All the best of the oil, and all the best of the wine, and of the wheat, the firstfruits of them which they shall offer unto the \LORD, them have I given thee.
\verse And whatsoever is first ripe in the land, which they shall bring unto the \LORD, shall be thine; every one that is clean in thine house shall eat of it.
\verse Every thing devoted in Israel shall be thine.
\verse Every thing that openeth the matrix in all flesh, which they bring unto the \LORD, whether it be of men or beasts, shall be thine: nevertheless the firstborn of man shalt thou surely redeem, and the firstling of unclean beasts shalt thou redeem.
\verse And those that are to be redeemed from a month old shalt thou redeem, according to thine estimation, for the money of five shekels, after the shekel of the sanctuary, which is twenty gerahs.
\verse But the firstling of a cow, or the firstling of a sheep, or the firstling of a goat, thou shalt not redeem; they are holy: thou shalt sprinkle their blood upon the altar, and shalt burn their fat for an offering made by fire, for a sweet savour unto the \LORD.
\verse And the flesh of them shall be thine, as the wave breast and as the right shoulder are thine.
\verse All the heave offerings of the holy things, which the children of Israel offer unto the \LORD, have I given thee, and thy sons and thy daughters with thee, by a statute for ever: it is a covenant of salt for ever before the \LORD unto thee and to thy seed with thee.
\verse And the \LORD spake unto Aaron, Thou shalt have no inheritance in their land, neither shalt thou have any part among them: I am thy part and thine inheritance among the children of Israel.
\verse And, behold, I have given the children of Levi all the tenth in Israel for an inheritance, for their service which they serve, even the service of the tabernacle of the congregation.
\verse Neither must the children of Israel henceforth come nigh the tabernacle of the congregation, lest they bear sin, and die.
\verse But the Levites shall do the service of the tabernacle of the congregation, and they shall bear their iniquity: it shall be a statute for ever throughout your generations, that among the children of Israel they have no inheritance.
\verse But the tithes of the children of Israel, which they offer as an heave offering unto the \LORD, I have given to the Levites to inherit: therefore I have said unto them, Among the children of Israel they shall have no inheritance.
\verse And the \LORD spake unto Moses, saying,
\verse Thus speak unto the Levites, and say unto them, When ye take of the children of Israel the tithes which I have given you from them for your inheritance, then ye shall offer up an heave offering of it for the \LORD, even a tenth part of the tithe.
\verse And this your heave offering shall be reckoned unto you, as though it were the corn of the threshingfloor, and as the fulness of the winepress.
\verse Thus ye also shall offer an heave offering unto the \LORD of all your tithes, which ye receive of the children of Israel; and ye shall give thereof the \LORDs heave offering to Aaron the priest.
\verse Out of all your gifts ye shall offer every heave offering of the \LORD, of all the best thereof, even the hallowed part thereof out of it.
\verse Therefore thou shalt say unto them, When ye have heaved the best thereof from it, then it shall be counted unto the Levites as the increase of the threshingfloor, and as the increase of the winepress.
\verse And ye shall eat it in every place, ye and your households: for it is your reward for your service in the tabernacle of the congregation.
\verse And ye shall bear no sin by reason of it, when ye have heaved from it the best of it: neither shall ye pollute the holy things of the children of Israel, lest ye die.
\end{biblechapter}

\section*{The water of cleansing}
\begin{biblechapter} % Numbers 19
\verse And the \LORD spake unto Moses and unto Aaron, saying,
\verse This is the ordinance of the law which the \LORD hath commanded, saying, Speak unto the children of Israel, that they bring thee a red heifer without spot, wherein is no blemish, and upon which never came yoke:
\verse And ye shall give her unto Eleazar the priest, that he may bring her forth without the camp, and one shall slay her before his face:
\verse And Eleazar the priest shall take of her blood with his finger, and sprinkle of her blood directly before the tabernacle of the congregation seven times:
\verse And one shall burn the heifer in his sight; her skin, and her flesh, and her blood, with her dung, shall he burn:
\verse And the priest shall take cedar wood, and hyssop, and scarlet, and cast it into the midst of the burning of the heifer.
\verse Then the priest shall wash his clothes, and he shall bathe his flesh in water, and afterward he shall come into the camp, and the priest shall be unclean until the even.
\verse And he that burneth her shall wash his clothes in water, and bathe his flesh in water, and shall be unclean until the even.
\verse And a man that is clean shall gather up the ashes of the heifer, and lay them up without the camp in a clean place, and it shall be kept for the congregation of the children of Israel for a water of separation: it is a purification for sin.
\verse And he that gathereth the ashes of the heifer shall wash his clothes, and be unclean until the even: and it shall be unto the children of Israel, and unto the stranger that sojourneth among them, for a statute for ever.
\verse He that toucheth the dead body of any man shall be unclean seven days.
\verse He shall purify himself with it on the third day, and on the seventh day he shall be clean: but if he purify not himself the third day, then the seventh day he shall not be clean.
\verse Whosoever toucheth the dead body of any man that is dead, and purifieth not himself, defileth the tabernacle of the \LORD; and that soul shall be cut off from Israel: because the water of separation was not sprinkled upon him, he shall be unclean; his uncleanness is yet upon him.
\verse This is the law, when a man dieth in a tent: all that come into the tent, and all that is in the tent, shall be unclean seven days.
\verse And every open vessel, which hath no covering bound upon it, is unclean.
\verse And whosoever toucheth one that is slain with a sword in the open fields, or a dead body, or a bone of a man, or a grave, shall be unclean seven days.
\verse And for an unclean person they shall take of the ashes of the burnt heifer of purification for sin, and running water shall be put thereto in a vessel:
\verse And a clean person shall take hyssop, and dip it in the water, and sprinkle it upon the tent, and upon all the vessels, and upon the persons that were there, and upon him that touched a bone, or one slain, or one dead, or a grave:
\verse And the clean person shall sprinkle upon the unclean on the third day, and on the seventh day: and on the seventh day he shall purify himself, and wash his clothes, and bathe himself in water, and shall be clean at even.
\verse But the man that shall be unclean, and shall not purify himself, that soul shall be cut off from among the congregation, because he hath defiled the sanctuary of the \LORD: the water of separation hath not been sprinkled upon him; he is unclean.
\verse And it shall be a perpetual statute unto them, that he that sprinkleth the water of separation shall wash his clothes; and he that toucheth the water of separation shall be unclean until even.
\verse And whatsoever the unclean person toucheth shall be unclean; and the soul that toucheth it shall be unclean until even.
\end{biblechapter}

\section*{Water from the rock}
\begin{biblechapter} % Numbers 20
\verse Then came the children of Israel, even the whole congregation, into the desert of Zin in the first month: and the people abode in Kadesh; and Miriam died there, and was buried there.
\verse And there was no water for the congregation: and they gathered themselves together against Moses and against Aaron.
\verse And the people chode with Moses, and spake, saying, Would God that we had died when our brethren died before the \LORD!
\verse And why have ye brought up the congregation of the \LORD into this wilderness, that we and our cattle should die there?
\verse And wherefore have ye made us to come up out of Egypt, to bring us in unto this evil place? it is no place of seed, or of figs, or of vines, or of pomegranates; neither is there any water to drink.
\verse And Moses and Aaron went from the presence of the assembly unto the door of the tabernacle of the congregation, and they fell upon their faces: and the glory of the \LORD appeared unto them.
\verse And the \LORD spake unto Moses, saying,
\verse Take the rod, and gather thou the assembly together, thou, and Aaron thy brother, and speak ye unto the rock before their eyes; and it shall give forth his water, and thou shalt bring forth to them water out of the rock: so thou shalt give the congregation and their beasts drink.
\verse And Moses took the rod from before the \LORD, as he commanded him.
\verse And Moses and Aaron gathered the congregation together before the rock, and he said unto them, Hear now, ye rebels; must we fetch you water out of this rock?
\verse And Moses lifted up his hand, and with his rod he smote the rock twice: and the water came out abundantly, and the congregation drank, and their beasts also.
\verse And the \LORD spake unto Moses and Aaron, Because ye believed me not, to sanctify me in the eyes of the children of Israel, therefore ye shall not bring this congregation into the land which I have given them.
\verse This is the water of Meribah; because the children of Israel strove with the \LORD, and he was sanctified in them.
\section*{Edom denies Israel passage}
\verse And Moses sent messengers from Kadesh unto the king of Edom, Thus saith thy brother Israel, Thou knowest all the travail that hath befallen us:
\verse How our fathers went down into Egypt, and we have dwelt in Egypt a long time; and the Egyptians vexed us, and our fathers:
\verse And when we cried unto the \LORD, he heard our voice, and sent an angel, and hath brought us forth out of Egypt: and, behold, we are in Kadesh, a city in the uttermost of thy border:
\verse Let us pass, I pray thee, through thy country: we will not pass through the fields, or through the vineyards, neither will we drink of the water of the wells: we will go by the king's high way, we will not turn to the right hand nor to the left, until we have passed thy borders.
\verse And Edom said unto him, Thou shalt not pass by me, lest I come out against thee with the sword.
\verse And the children of Israel said unto him, We will go by the high way: and if I and my cattle drink of thy water, then I will pay for it: I will only, without doing any thing else, go through on my feet.
\verse And he said, Thou shalt not go through. And Edom came out against him with much people, and with a strong hand.
\verse Thus Edom refused to give Israel passage through his border: wherefore Israel turned away from him.
\section*{The death of Aaron}
\verse And the children of Israel, even the whole congregation, journeyed from Kadesh, and came unto mount Hor.
\verse And the \LORD spake unto Moses and Aaron in mount Hor, by the coast of the land of Edom, saying,
\verse Aaron shall be gathered unto his people: for he shall not enter into the land which I have given unto the children of Israel, because ye rebelled against my word at the water of Meribah.
\verse Take Aaron and Eleazar his son, and bring them up unto mount Hor:
\verse And strip Aaron of his garments, and put them upon Eleazar his son: and Aaron shall be gathered unto his people, and shall die there.
\verse And Moses did as the \LORD commanded: and they went up into mount Hor in the sight of all the congregation.
\verse And Moses stripped Aaron of his garments, and put them upon Eleazar his son; and Aaron died there in the top of the mount: and Moses and Eleazar came down from the mount.
\verse And when all the congregation saw that Aaron was dead, they mourned for Aaron thirty days, even all the house of Israel.
\end{biblechapter}

\section*{Arad destroyed}
\begin{biblechapter} % Numbers 21
\verse And when king Arad the Canaanite, which dwelt in the south, heard tell that Israel came by the way of the spies; then he fought against Israel, and took some of them prisoners.
\verse And Israel vowed a vow unto the \LORD, and said, If thou wilt indeed deliver this people into my hand, then I will utterly destroy their cities.
\verse And the \LORD hearkened to the voice of Israel, and delivered up the Canaanites; and they utterly destroyed them and their cities: and he called the name of the place Hormah.
\verse And they journeyed from mount Hor by the way of the Red sea, to compass the land of Edom: and the soul of the people was much discouraged because of the way.
\verse And the people spake against God, and against Moses, Wherefore have ye brought us up out of Egypt to die in the wilderness? for there is no bread, neither is there any water; and our soul loatheth this light bread.
\verse And the \LORD sent fiery serpents among the people, and they bit the people; and much people of Israel died.
\verse Therefore the people came to Moses, and said, We have sinned, for we have spoken against the \LORD, and against thee; pray unto the \LORD, that he take away the serpents from us. And Moses prayed for the people.
\verse And the \LORD said unto Moses, Make thee a fiery serpent, and set it upon a pole: and it shall come to pass, that every one that is bitten, when he looketh upon it, shall live.
\verse And Moses made a serpent of brass, and put it upon a pole, and it came to pass, that if a serpent had bitten any man, when he beheld the serpent of brass, he lived.
\section*{The journey to Moab}
\verse And the children of Israel set forward, and pitched in Oboth.
\verse And they journeyed from Oboth, and pitched at Ijeabarim, in the wilderness which is before Moab, toward the sunrising.
\verse From thence they removed, and pitched in the valley of Zared.
\verse From thence they removed, and pitched on the other side of Arnon, which is in the wilderness that cometh out of the coasts of the Amorites: for Arnon is the border of Moab, between Moab and the Amorites.
\verse Wherefore it is said in the book of the wars of the \LORD, What he did in the Red sea, and in the brooks of Arnon,
\verse And at the stream of the brooks that goeth down to the dwelling of Ar, and lieth upon the border of Moab.
\verse And from thence they went to Beer: that is the well whereof the \LORD spake unto Moses, Gather the people together, and I will give them water.
\verse Then Israel sang this song, Spring up, O well; sing ye unto it:
\verse The princes digged the well, the nobles of the people digged it, by the direction of the lawgiver, with their staves. And from the wilderness they went to Mattanah:
\verse And from Mattanah to Nahaliel: and from Nahaliel to Bamoth:
\verse And from Bamoth in the valley, that is in the country of Moab, to the top of Pisgah, which looketh toward Jeshimon.
\section*{Defeat of Sihon and Og}
\verse And Israel sent messengers unto Sihon king of the Amorites, saying,
\verse Let me pass through thy land: we will not turn into the fields, or into the vineyards; we will not drink of the waters of the well: but we will go along by the king's high way, until we be past thy borders.
\verse And Sihon would not suffer Israel to pass through his border: but Sihon gathered all his people together, and went out against Israel into the wilderness: and he came to Jahaz, and fought against Israel.
\verse And Israel smote him with the edge of the sword, and possessed his land from Arnon unto Jabbok, even unto the children of Ammon: for the border of the children of Ammon was strong.
\verse And Israel took all these cities: and Israel dwelt in all the cities of the Amorites, in Heshbon, and in all the villages thereof.
\verse For Heshbon was the city of Sihon the king of the Amorites, who had fought against the former king of Moab, and taken all his land out of his hand, even unto Arnon.
\verse Wherefore they that speak in proverbs say, Come into Heshbon, let the city of Sihon be built and prepared:
\verse For there is a fire gone out of Heshbon, a flame from the city of Sihon: it hath consumed Ar of Moab, and the lords of the high places of Arnon.
\verse Woe to thee, Moab! thou art undone, O people of Chemosh: he hath given his sons that escaped, and his daughters, into captivity unto Sihon king of the Amorites.
\verse We have shot at them; Heshbon is perished even unto Dibon, and we have laid them waste even unto Nophah, which reacheth unto Medeba.
\verse Thus Israel dwelt in the land of the Amorites.
\verse And Moses sent to spy out Jaazer, and they took the villages thereof, and drove out the Amorites that were there.
\verse And they turned and went up by the way of Bashan: and Og the king of Bashan went out against them, he, and all his people, to the battle at Edrei.
\verse And the \LORD said unto Moses, Fear him not: for I have delivered him into thy hand, and all his people, and his land; and thou shalt do to him as thou didst unto Sihon king of the Amorites, which dwelt at Heshbon.
\verse So they smote him, and his sons, and all his people, until there was none left him alive: and they possessed his land.
\end{biblechapter}

\section*{Balak summons Balaam}
\begin{biblechapter} % Numbers 22
\verse And the children of Israel set forward, and pitched in the plains of Moab on this side Jordan by Jericho.
\verse And Balak the son of Zippor saw all that Israel had done to the Amorites.
\verse And Moab was sore afraid of the people, because they were many: and Moab was distressed because of the children of Israel.
\verse And Moab said unto the elders of Midian, Now shall this company lick up all that are round about us, as the ox licketh up the grass of the field. And Balak the son of Zippor was king of the Moabites at that time.
\verse He sent messengers therefore unto Balaam the son of Beor to Pethor, which is by the river of the land of the children of his people, to call him, saying, Behold, there is a people come out from Egypt: behold, they cover the face of the earth, and they abide over against me:
\verse Come now therefore, I pray thee, curse me this people; for they are too mighty for me: peradventure I shall prevail, that we may smite them, and that I may drive them out of the land: for I wot that he whom thou blessest is blessed, and he whom thou cursest is cursed.
\verse And the elders of Moab and the elders of Midian departed with the rewards of divination in their hand; and they came unto Balaam, and spake unto him the words of Balak.
\verse And he said unto them, Lodge here this night, and I will bring you word again, as the \LORD shall speak unto me: and the princes of Moab abode with Balaam.
\verse And God came unto Balaam, and said, What men are these with thee?
\verse And Balaam said unto God, Balak the son of Zippor, king of Moab, hath sent unto me, saying,
\verse Behold, there is a people come out of Egypt, which covereth the face of the earth: come now, curse me them; peradventure I shall be able to overcome them, and drive them out.
\verse And God said unto Balaam, Thou shalt not go with them; thou shalt not curse the people: for they are blessed.
\verse And Balaam rose up in the morning, and said unto the princes of Balak, Get you into your land: for the \LORD refuseth to give me leave to go with you.
\verse And the princes of Moab rose up, and they went unto Balak, and said, Balaam refuseth to come with us.
\verse And Balak sent yet again princes, more, and more honourable than they.
\verse And they came to Balaam, and said to him, Thus saith Balak the son of Zippor, Let nothing, I pray thee, hinder thee from coming unto me:
\verse For I will promote thee unto very great honour, and I will do whatsoever thou sayest unto me: come therefore, I pray thee, curse me this people.
\verse And Balaam answered and said unto the servants of Balak, If Balak would give me his house full of silver and gold, I cannot go beyond the word of the \LORD my God, to do less or more.
\verse Now therefore, I pray you, tarry ye also here this night, that I may know what the \LORD will say unto me more.
\verse And God came unto Balaam at night, and said unto him, If the men come to call thee, rise up, and go with them; but yet the word which I shall say unto thee, that shalt thou do.
\section*{Balaam's donkey}
\verse And Balaam rose up in the morning, and saddled his ass, and went with the princes of Moab.
\verse And God's anger was kindled because he went: and the angel of the \LORD stood in the way for an adversary against him. Now he was riding upon his ass, and his two servants were with him.
\verse And the ass saw the angel of the \LORD standing in the way, and his sword drawn in his hand: and the ass turned aside out of the way, and went into the field: and Balaam smote the ass, to turn her into the way.
\verse But the angel of the \LORD stood in a path of the vineyards, a wall being on this side, and a wall on that side.
\verse And when the ass saw the angel of the \LORD, she thrust herself unto the wall, and crushed Balaam's foot against the wall: and he smote her again.
\verse And the angel of the \LORD went further, and stood in a narrow place, where was no way to turn either to the right hand or to the left.
\verse And when the ass saw the angel of the \LORD, she fell down under Balaam: and Balaam's anger was kindled, and he smote the ass with a staff.
\verse And the \LORD opened the mouth of the ass, and she said unto Balaam, What have I done unto thee, that thou hast smitten me these three times?
\verse And Balaam said unto the ass, Because thou hast mocked me: I would there were a sword in mine hand, for now would I kill thee.
\verse And the ass said unto Balaam, Am not I thine ass, upon which thou hast ridden ever since I was thine unto this day? was I ever wont to do so unto thee? And he said, Nay.
\verse Then the \LORD opened the eyes of Balaam, and he saw the angel of the \LORD standing in the way, and his sword drawn in his hand: and he bowed down his head, and fell flat on his face.
\verse And the angel of the \LORD said unto him, Wherefore hast thou smitten thine ass these three times? behold, I went out to withstand thee, because thy way is perverse before me:
\verse And the ass saw me, and turned from me these three times: unless she had turned from me, surely now also I had slain thee, and saved her alive.
\verse And Balaam said unto the angel of the \LORD, I have sinned; for I knew not that thou stoodest in the way against me: now therefore, if it displease thee, I will get me back again.
\verse And the angel of the \LORD said unto Balaam, Go with the men: but only the word that I shall speak unto thee, that thou shalt speak. So Balaam went with the princes of Balak.
\verse And when Balak heard that Balaam was come, he went out to meet him unto a city of Moab, which is in the border of Arnon, which is in the utmost coast.
\verse And Balak said unto Balaam, Did I not earnestly send unto thee to call thee? wherefore camest thou not unto me? am I not able indeed to promote thee to honour?
\verse And Balaam said unto Balak, Lo, I am come unto thee: have I now any power at all to say any thing? the word that God putteth in my mouth, that shall I speak.
\verse And Balaam went with Balak, and they came unto Kirjathhuzoth.
\verse And Balak offered oxen and sheep, and sent to Balaam, and to the princes that were with him.
\verse And it came to pass on the morrow, that Balak took Balaam, and brought him up into the high places of Baal, that thence he might see the utmost part of the people.
\end{biblechapter}

\section*{Balaam's first message}
\begin{biblechapter} % Numbers 23
\verse And Balaam said unto Balak, Build me here seven altars, and prepare me here seven oxen and seven rams.
\verse And Balak did as Balaam had spoken; and Balak and Balaam offered on every altar a bullock and a ram.
\verse And Balaam said unto Balak, Stand by thy burnt offering, and I will go: peradventure the \LORD will come to meet me: and whatsoever he sheweth me I will tell thee. And he went to an high place.
\verse And God met Balaam: and he said unto him, I have prepared seven altars, and I have offered upon every altar a bullock and a ram.
\verse And the \LORD put a word in Balaam's mouth, and said, Return unto Balak, and thus thou shalt speak.
\verse And he returned unto him, and, lo, he stood by his burnt sacrifice, he, and all the princes of Moab.
\verse And he took up his parable, and said, Balak the king of Moab hath brought me from Aram, out of the mountains of the east, saying, Come, curse me Jacob, and come, defy Israel.
\verse How shall I curse, whom God hath not cursed? or how shall I defy, whom the \LORD hath not defied?
\verse For from the top of the rocks I see him, and from the hills I behold him: lo, the people shall dwell alone, and shall not be reckoned among the nations.
\verse Who can count the dust of Jacob, and the number of the fourth part of Israel? Let me die the death of the righteous, and let my last end be like his!
\verse And Balak said unto Balaam, What hast thou done unto me? I took thee to curse mine enemies, and, behold, thou hast blessed them altogether.
\verse And he answered and said, Must I not take heed to speak that which the \LORD hath put in my mouth?
\section*{Balaam's second message}
\verse And Balak said unto him, Come, I pray thee, with me unto another place, from whence thou mayest see them: thou shalt see but the utmost part of them, and shalt not see them all: and curse me them from thence.
\verse And he brought him into the field of Zophim, to the top of Pisgah, and built seven altars, and offered a bullock and a ram on every altar.
\verse And he said unto Balak, Stand here by thy burnt offering, while I meet the \LORD yonder.
\verse And the \LORD met Balaam, and put a word in his mouth, and said, Go again unto Balak, and say thus.
\verse And when he came to him, behold, he stood by his burnt offering, and the princes of Moab with him. And Balak said unto him, What hath the \LORD spoken?
\verse And he took up his parable, and said, Rise up, Balak, and hear; hearken unto me, thou son of Zippor:
\verse God is not a man, that he should lie; neither the son of man, that he should repent: hath he said, and shall he not do it? or hath he spoken, and shall he not make it good?
\verse Behold, I have received commandment to bless: and he hath blessed; and I cannot reverse it.
\verse He hath not beheld iniquity in Jacob, neither hath he seen perverseness in Israel: the \LORD his God is with him, and the shout of a king is among them.
\verse God brought them out of Egypt; he hath as it were the strength of an unicorn.
\verse Surely there is no enchantment against Jacob, neither is there any divination against Israel: according to this time it shall be said of Jacob and of Israel, What hath God wrought!
\verse Behold, the people shall rise up as a great lion, and lift up himself as a young lion: he shall not lie down until he eat of the prey, and drink the blood of the slain.
\verse And Balak said unto Balaam, Neither curse them at all, nor bless them at all.
\verse But Balaam answered and said unto Balak, Told not I thee, saying, All that the \LORD speaketh, that I must do?
\section*{Balaam's third message}
\verse And Balak said unto Balaam, Come, I pray thee, I will bring thee unto another place; peradventure it will please God that thou mayest curse me them from thence.
\verse And Balak brought Balaam unto the top of Peor, that looketh toward Jeshimon.
\verse And Balaam said unto Balak, Build me here seven altars, and prepare me here seven bullocks and seven rams.
\verse And Balak did as Balaam had said, and offered a bullock and a ram on every altar.
\end{biblechapter}

\begin{biblechapter} % Numbers 24
\verse And when Balaam saw that it pleased the \LORD to bless Israel, he went not, as at other times, to seek for enchantments, but he set his face toward the wilderness.
\verse And Balaam lifted up his eyes, and he saw Israel abiding in his tents according to their tribes; and the spirit of God came upon him.
\verse And he took up his parable, and said, Balaam the son of Beor hath said, and the man whose eyes are open hath said:
\verse He hath said, which heard the words of God, which saw the vision of the Almighty, falling into a trance, but having his eyes open:
\verse How goodly are thy tents, O Jacob, and thy tabernacles, O Israel!
\verse As the valleys are they spread forth, as gardens by the river's side, as the trees of lign aloes which the \LORD hath planted, and as cedar trees beside the waters.
\verse He shall pour the water out of his buckets, and his seed shall be in many waters, and his king shall be higher than Agag, and his kingdom shall be exalted.
\verse God brought him forth out of Egypt; he hath as it were the strength of an unicorn: he shall eat up the nations his enemies, and shall break their bones, and pierce them through with his arrows.
\verse He couched, he lay down as a lion, and as a great lion: who shall stir him up? Blessed is he that blesseth thee, and cursed is he that curseth thee.
\verse And Balak's anger was kindled against Balaam, and he smote his hands together: and Balak said unto Balaam, I called thee to curse mine enemies, and, behold, thou hast altogether blessed them these three times.
\verse Therefore now flee thou to thy place: I thought to promote thee unto great honour; but, lo, the \LORD hath kept thee back from honour.
\verse And Balaam said unto Balak, Spake I not also to thy messengers which thou sentest unto me, saying,
\verse If Balak would give me his house full of silver and gold, I cannot go beyond the commandment of the \LORD, to do either good or bad of mine own mind; but what the \LORD saith, that will I speak?
\verse And now, behold, I go unto my people: come therefore, and I will advertise thee what this people shall do to thy people in the latter days.
\section*{Balaam's fourth message}
\verse And he took up his parable, and said, Balaam the son of Beor hath said, and the man whose eyes are open hath said:
\verse He hath said, which heard the words of God, and knew the knowledge of the most High, which saw the vision of the Almighty, falling into a trance, but having his eyes open:
\verse I shall see him, but not now: I shall behold him, but not nigh: there shall come a Star out of Jacob, and a Sceptre shall rise out of Israel, and shall smite the corners of Moab, and destroy all the children of Sheth.
\verse And Edom shall be a possession, Seir also shall be a possession for his enemies; and Israel shall do valiantly.
\verse Out of Jacob shall come he that shall have dominion, and shall destroy him that remaineth of the city.
\section*{Balaam's fifth message}
\verse And when he looked on Amalek, he took up his parable, and said, Amalek was the first of the nations; but his latter end shall be that he perish for ever.
\section*{Balaam's sixth message}
\verse And he looked on the Kenites, and took up his parable, and said, Strong is thy dwellingplace, and thou puttest thy nest in a rock.
\verse Nevertheless the Kenite shall be wasted, until Asshur shall carry thee away captive.
\section*{Balaam's seventh message}
\verse And he took up his parable, and said, Alas, who shall live when God doeth this!
\verse And ships shall come from the coast of Chittim, and shall afflict Asshur, and shall afflict Eber, and he also shall perish for ever.
\verse And Balaam rose up, and went and returned to his place: and Balak also went his way.
\end{biblechapter}

\section*{Moab seduces Israel}
\begin{biblechapter} % Numbers 25
\verse And Israel abode in Shittim, and the people began to commit whoredom with the daughters of Moab.
\verse And they called the people unto the sacrifices of their gods: and the people did eat, and bowed down to their gods.
\verse And Israel joined himself unto Baalpeor: and the anger of the \LORD was kindled against Israel.
\verse And the \LORD said unto Moses, Take all the heads of the people, and hang them up before the \LORD against the sun, that the fierce anger of the \LORD may be turned away from Israel.
\verse And Moses said unto the judges of Israel, Slay ye every one his men that were joined unto Baalpeor.
\verse And, behold, one of the children of Israel came and brought unto his brethren a Midianitish woman in the sight of Moses, and in the sight of all the congregation of the children of Israel, who were weeping before the door of the tabernacle of the congregation.
\verse And when Phinehas, the son of Eleazar, the son of Aaron the priest, saw it, he rose up from among the congregation, and took a javelin in his hand;
\verse And he went after the man of Israel into the tent, and thrust both of them through, the man of Israel, and the woman through her belly. So the plague was stayed from the children of Israel.
\verse And those that died in the plague were twenty and four thousand.
\verse And the \LORD spake unto Moses, saying,
\verse Phinehas, the son of Eleazar, the son of Aaron the priest, hath turned my wrath away from the children of Israel, while he was zealous for my sake among them, that I consumed not the children of Israel in my jealousy.
\verse Wherefore say, Behold, I give unto him my covenant of peace:
\verse And he shall have it, and his seed after him, even the covenant of an everlasting priesthood; because he was zealous for his God, and made an atonement for the children of Israel.
\verse Now the name of the Israelite that was slain, even that was slain with the Midianitish woman, was Zimri, the son of Salu, a prince of a chief house among the Simeonites.
\verse And the name of the Midianitish woman that was slain was Cozbi, the daughter of Zur; he was head over a people, and of a chief house in Midian.
\verse And the \LORD spake unto Moses, saying,
\verse Vex the Midianites, and smite them:
\verse For they vex you with their wiles, wherewith they have beguiled you in the matter of Peor, and in the matter of Cozbi, the daughter of a prince of Midian, their sister, which was slain in the day of the plague for Peor's sake.
\end{biblechapter}

\section*{The second census}
\begin{biblechapter} % Numbers 26
\verse And it came to pass after the plague, that the \LORD spake unto Moses and unto Eleazar the son of Aaron the priest, saying,
\verse Take the sum of all the congregation of the children of Israel, from twenty years old and upward, throughout their fathers' house, all that are able to go to war in Israel.
\verse And Moses and Eleazar the priest spake with them in the plains of Moab by Jordan near Jericho, saying,
\verse Take the sum of the people, from twenty years old and upward; as the \LORD commanded Moses and the children of Israel, which went forth out of the land of Egypt.
\verse Reuben, the eldest son of Israel: the children of Reuben; Hanoch, of whom cometh the family of the Hanochites: of Pallu, the family of the Palluites:
\verse Of Hezron, the family of the Hezronites: of Carmi, the family of the Carmites.
\verse These are the families of the Reubenites: and they that were numbered of them were forty and three thousand and seven hundred and thirty.
\verse And the sons of Pallu; Eliab.
\verse And the sons of Eliab; Nemuel, and Dathan, and Abiram. This is that Dathan and Abiram, which were famous in the congregation, who strove against Moses and against Aaron in the company of Korah, when they strove against the \LORD:
\verse And the earth opened her mouth, and swallowed them up together with Korah, when that company died, what time the fire devoured two hundred and fifty men: and they became a sign.
\verse Notwithstanding the children of Korah died not.
\verse The sons of Simeon after their families: of Nemuel, the family of the Nemuelites: of Jamin, the family of the Jaminites: of Jachin, the family of the Jachinites:
\verse Of Zerah, the family of the Zarhites: of Shaul, the family of the Shaulites.
\verse These are the families of the Simeonites, twenty and two thousand and two hundred.
\verse The children of Gad after their families: of Zephon, the family of the Zephonites: of Haggi, the family of the Haggites: of Shuni, the family of the Shunites:
\verse Of Ozni, the family of the Oznites: of Eri, the family of the Erites:
\verse Of Arod, the family of the Arodites: of Areli, the family of the Arelites.
\verse These are the families of the children of Gad according to those that were numbered of them, forty thousand and five hundred.
\verse The sons of Judah were Er and Onan: and Er and Onan died in the land of Canaan.
\verse And the sons of Judah after their families were; of Shelah, the family of the Shelanites: of Pharez, the family of the Pharzites: of Zerah, the family of the Zarhites.
\verse And the sons of Pharez were; of Hezron, the family of the Hezronites: of Hamul, the family of the Hamulites.
\verse These are the families of Judah according to those that were numbered of them, threescore and sixteen thousand and five hundred.
\verse Of the sons of Issachar after their families: of Tola, the family of the Tolaites: of Pua, the family of the Punites:
\verse Of Jashub, the family of the Jashubites: of Shimron, the family of the Shimronites.
\verse These are the families of Issachar according to those that were numbered of them, threescore and four thousand and three hundred.
\verse Of the sons of Zebulun after their families: of Sered, the family of the Sardites: of Elon, the family of the Elonites: of Jahleel, the family of the Jahleelites.
\verse These are the families of the Zebulunites according to those that were numbered of them, threescore thousand and five hundred.
\verse The sons of Joseph after their families were Manasseh and Ephraim.
\verse Of the sons of Manasseh: of Machir, the family of the Machirites: and Machir begat Gilead: of Gilead come the family of the Gileadites.
\verse These are the sons of Gilead: of Jeezer, the family of the Jeezerites: of Helek, the family of the Helekites:
\verse And of Asriel, the family of the Asrielites: and of Shechem, the family of the Shechemites:
\verse And of Shemida, the family of the Shemidaites: and of Hepher, the family of the Hepherites.
\verse And Zelophehad the son of Hepher had no sons, but daughters: and the names of the daughters of Zelophehad were Mahlah, and Noah, Hoglah, Milcah, and Tirzah.
\verse These are the families of Manasseh, and those that were numbered of them, fifty and two thousand and seven hundred.
\verse These are the sons of Ephraim after their families: of Shuthelah, the family of the Shuthalhites: of Becher, the family of the Bachrites: of Tahan, the family of the Tahanites.
\verse And these are the sons of Shuthelah: of Eran, the family of the Eranites.
\verse These are the families of the sons of Ephraim according to those that were numbered of them, thirty and two thousand and five hundred. These are the sons of Joseph after their families.
\verse The sons of Benjamin after their families: of Bela, the family of the Belaites: of Ashbel, the family of the Ashbelites: of Ahiram, the family of the Ahiramites:
\verse Of Shupham, the family of the Shuphamites: of Hupham, the family of the Huphamites.
\verse And the sons of Bela were Ard and Naaman: of Ard, the family of the Ardites: and of Naaman, the family of the Naamites.
\verse These are the sons of Benjamin after their families: and they that were numbered of them were forty and five thousand and six hundred.
\verse These are the sons of Dan after their families: of Shuham, the family of the Shuhamites. These are the families of Dan after their families.
\verse All the families of the Shuhamites, according to those that were numbered of them, were threescore and four thousand and four hundred.
\verse Of the children of Asher after their families: of Jimna, the family of the Jimnites: of Jesui, the family of the Jesuites: of Beriah, the family of the Beriites.
\verse Of the sons of Beriah: of Heber, the family of the Heberites: of Malchiel, the family of the Malchielites.
\verse And the name of the daughter of Asher was Sarah.
\verse These are the families of the sons of Asher according to those that were numbered of them; who were fifty and three thousand and four hundred.
\verse Of the sons of Naphtali after their families: of Jahzeel, the family of the Jahzeelites: of Guni, the family of the Gunites:
\verse Of Jezer, the family of the Jezerites: of Shillem, the family of the Shillemites.
\verse These are the families of Naphtali according to their families: and they that were numbered of them were forty and five thousand and four hundred.
\verse These were the numbered of the children of Israel, six hundred thousand and a thousand seven hundred and thirty.
\verse And the \LORD spake unto Moses, saying,
\verse Unto these the land shall be divided for an inheritance according to the number of names.
\verse To many thou shalt give the more inheritance, and to few thou shalt give the less inheritance: to every one shall his inheritance be given according to those that were numbered of him.
\verse Notwithstanding the land shall be divided by lot: according to the names of the tribes of their fathers they shall inherit.
\verse According to the lot shall the possession thereof be divided between many and few.
\verse And these are they that were numbered of the Levites after their families: of Gershon, the family of the Gershonites: of Kohath, the family of the Kohathites: of Merari, the family of the Merarites.
\verse These are the families of the Levites: the family of the Libnites, the family of the Hebronites, the family of the Mahlites, the family of the Mushites, the family of the Korathites. And Kohath begat Amram.
\verse And the name of Amram's wife was Jochebed, the daughter of Levi, whom her mother bare to Levi in Egypt: and she bare unto Amram Aaron and Moses, and Miriam their sister.
\verse And unto Aaron was born Nadab, and Abihu, Eleazar, and Ithamar.
\verse And Nadab and Abihu died, when they offered strange fire before the \LORD.
\verse And those that were numbered of them were twenty and three thousand, all males from a month old and upward: for they were not numbered among the children of Israel, because there was no inheritance given them among the children of Israel.
\verse These are they that were numbered by Moses and Eleazar the priest, who numbered the children of Israel in the plains of Moab by Jordan near Jericho.
\verse But among these there was not a man of them whom Moses and Aaron the priest numbered, when they numbered the children of Israel in the wilderness of Sinai.
\verse For the \LORD had said of them, They shall surely die in the wilderness. And there was not left a man of them, save Caleb the son of Jephunneh, and Joshua the son of Nun.
\end{biblechapter}

\section*{Zelophehad's daughters}
\begin{biblechapter} % Numbers 27
\verse Then came the daughters of Zelophehad, the son of Hepher, the son of Gilead, the son of Machir, the son of Manasseh, of the families of Manasseh the son of Joseph: and these are the names of his daughters; Mahlah, Noah, and Hoglah, and Milcah, and Tirzah.
\verse And they stood before Moses, and before Eleazar the priest, and before the princes and all the congregation, by the door of the tabernacle of the congregation, saying,
\verse Our father died in the wilderness, and he was not in the company of them that gathered themselves together against the \LORD in the company of Korah; but died in his own sin, and had no sons.
\verse Why should the name of our father be done away from among his family, because he hath no son? Give unto us therefore a possession among the brethren of our father.
\verse And Moses brought their cause before the \LORD.
\verse And the \LORD spake unto Moses, saying,
\verse The daughters of Zelophehad speak right: thou shalt surely give them a possession of an inheritance among their father's brethren; and thou shalt cause the inheritance of their father to pass unto them.
\verse And thou shalt speak unto the children of Israel, saying, If a man die, and have no son, then ye shall cause his inheritance to pass unto his daughter.
\verse And if he have no daughter, then ye shall give his inheritance unto his brethren.
\verse And if he have no brethren, then ye shall give his inheritance unto his father's brethren.
\verse And if his father have no brethren, then ye shall give his inheritance unto his kinsman that is next to him of his family, and he shall possess it: and it shall be unto the children of Israel a statute of judgment, as the \LORD commanded Moses.
\section*{Joshua to succeed Moses}
\verse And the \LORD said unto Moses, Get thee up into this mount Abarim, and see the land which I have given unto the children of Israel.
\verse And when thou hast seen it, thou also shalt be gathered unto thy people, as Aaron thy brother was gathered.
\verse For ye rebelled against my commandment in the desert of Zin, in the strife of the congregation, to sanctify me at the water before their eyes: that is the water of Meribah in Kadesh in the wilderness of Zin.
\verse And Moses spake unto the \LORD, saying,
\verse Let the \LORD, the God of the spirits of all flesh, set a man over the congregation,
\verse Which may go out before them, and which may go in before them, and which may lead them out, and which may bring them in; that the congregation of the \LORD be not as sheep which have no shepherd.
\verse And the \LORD said unto Moses, Take thee Joshua the son of Nun, a man in whom is the spirit, and lay thine hand upon him;
\verse And set him before Eleazar the priest, and before all the congregation; and give him a charge in their sight.
\verse And thou shalt put some of thine honour upon him, that all the congregation of the children of Israel may be obedient.
\verse And he shall stand before Eleazar the priest, who shall ask counsel for him after the judgment of Urim before the \LORD: at his word shall they go out, and at his word they shall come in, both he, and all the children of Israel with him, even all the congregation.
\verse And Moses did as the \LORD commanded him: and he took Joshua, and set him before Eleazar the priest, and before all the congregation:
\verse And he laid his hands upon him, and gave him a charge, as the \LORD commanded by the hand of Moses.
\end{biblechapter}

\section*{Daily offerings}
\begin{biblechapter} % Numbers 28
\verse And the \LORD spake unto Moses, saying,
\verse Command the children of Israel, and say unto them, My offering, and my bread for my sacrifices made by fire, for a sweet savour unto me, shall ye observe to offer unto me in their due season.
\verse And thou shalt say unto them, This is the offering made by fire which ye shall offer unto the \LORD; two lambs of the first year without spot day by day, for a continual burnt offering.
\verse The one lamb shalt thou offer in the morning, and the other lamb shalt thou offer at even;
\verse And a tenth part of an ephah of flour for a meat offering, mingled with the fourth part of an hin of beaten oil.
\verse It is a continual burnt offering, which was ordained in mount Sinai for a sweet savour, a sacrifice made by fire unto the \LORD.
\verse And the drink offering thereof shall be the fourth part of an hin for the one lamb: in the holy place shalt thou cause the strong wine to be poured unto the \LORD for a drink offering.
\verse And the other lamb shalt thou offer at even: as the meat offering of the morning, and as the drink offering thereof, thou shalt offer it, a sacrifice made by fire, of a sweet savour unto the \LORD.
\section*{Sabbath offerings}
\verse And on the sabbath day two lambs of the first year without spot, and two tenth deals of flour for a meat offering, mingled with oil, and the drink offering thereof:
\verse This is the burnt offering of every sabbath, beside the continual burnt offering, and his drink offering.
\section*{Monthly offerings}
\verse And in the beginnings of your months ye shall offer a burnt offering unto the \LORD; two young bullocks, and one ram, seven lambs of the first year without spot;
\verse And three tenth deals of flour for a meat offering, mingled with oil, for one bullock; and two tenth deals of flour for a meat offering, mingled with oil, for one ram;
\verse And a several tenth deal of flour mingled with oil for a meat offering unto one lamb; for a burnt offering of a sweet savour, a sacrifice made by fire unto the \LORD.
\verse And their drink offerings shall be half an hin of wine unto a bullock, and the third part of an hin unto a ram, and a fourth part of an hin unto a lamb: this is the burnt offering of every month throughout the months of the year.
\verse And one kid of the goats for a sin offering unto the \LORD shall be offered, beside the continual burnt offering, and his drink offering.
\section*{Passover offerings}
\verse And in the fourteenth day of the first month is the passover of the \LORD.
\verse And in the fifteenth day of this month is the feast: seven days shall unleavened bread be eaten.
\verse In the first day shall be an holy convocation; ye shall do no manner of servile work therein:
\verse But ye shall offer a sacrifice made by fire for a burnt offering unto the \LORD; two young bullocks, and one ram, and seven lambs of the first year: they shall be unto you without blemish:
\verse And their meat offering shall be of flour mingled with oil: three tenth deals shall ye offer for a bullock, and two tenth deals for a ram;
\verse A several tenth deal shalt thou offer for every lamb, throughout the seven lambs:
\verse And one goat for a sin offering, to make an atonement for you.
\verse Ye shall offer these beside the burnt offering in the morning, which is for a continual burnt offering.
\verse After this manner ye shall offer daily, throughout the seven days, the meat of the sacrifice made by fire, of a sweet savour unto the \LORD: it shall be offered beside the continual burnt offering, and his drink offering.
\verse And on the seventh day ye shall have an holy convocation; ye shall do no servile work.
\section*{The Feast of Weeks}
\verse Also in the day of the firstfruits, when ye bring a new meat offering unto the \LORD, after your weeks be out, ye shall have an holy convocation; ye shall do no servile work:
\verse But ye shall offer the burnt offering for a sweet savour unto the \LORD; two young bullocks, one ram, seven lambs of the first year;
\verse And their meat offering of flour mingled with oil, three tenth deals unto one bullock, two tenth deals unto one ram,
\verse A several tenth deal unto one lamb, throughout the seven lambs;
\verse And one kid of the goats, to make an atonement for you.
\verse Ye shall offer them beside the continual burnt offering, and his meat offering, (they shall be unto you without blemish) and their drink offerings.
\end{biblechapter}

\section*{The Feast of Trumpets}
\begin{biblechapter} % Numbers 29
\verse And in the seventh month, on the first day of the month, ye shall have an holy convocation; ye shall do no servile work: it is a day of blowing the trumpets unto you.
\verse And ye shall offer a burnt offering for a sweet savour unto the \LORD; one young bullock, one ram, and seven lambs of the first year without blemish:
\verse And their meat offering shall be of flour mingled with oil, three tenth deals for a bullock, and two tenth deals for a ram,
\verse And one tenth deal for one lamb, throughout the seven lambs:
\verse And one kid of the goats for a sin offering, to make an atonement for you:
\verse Beside the burnt offering of the month, and his meat offering, and the daily burnt offering, and his meat offering, and their drink offerings, according unto their manner, for a sweet savour, a sacrifice made by fire unto the \LORD.
\section*{The Day of Atonement}
\verse And ye shall have on the tenth day of this seventh month an holy convocation; and ye shall afflict your souls: ye shall not do any work therein:
\verse But ye shall offer a burnt offering unto the \LORD for a sweet savour; one young bullock, one ram, and seven lambs of the first year; they shall be unto you without blemish:
\verse And their meat offering shall be of flour mingled with oil, three tenth deals to a bullock, and two tenth deals to one ram,
\verse A several tenth deal for one lamb, throughout the seven lambs:
\verse One kid of the goats for a sin offering; beside the sin offering of atonement, and the continual burnt offering, and the meat offering of it, and their drink offerings.
\section*{The Feast of Tabernacles}
\verse And on the fifteenth day of the seventh month ye shall have an holy convocation; ye shall do no servile work, and ye shall keep a feast unto the \LORD seven days:
\verse And ye shall offer a burnt offering, a sacrifice made by fire, of a sweet savour unto the \LORD; thirteen young bullocks, two rams, and fourteen lambs of the first year; they shall be without blemish:
\verse And their meat offering shall be of flour mingled with oil, three tenth deals unto every bullock of the thirteen bullocks, two tenth deals to each ram of the two rams,
\verse And a several tenth deal to each lamb of the fourteen lambs:
\verse And one kid of the goats for a sin offering; beside the continual burnt offering, his meat offering, and his drink offering.
\verse And on the second day ye shall offer twelve young bullocks, two rams, fourteen lambs of the first year without spot:
\verse And their meat offering and their drink offerings for the bullocks, for the rams, and for the lambs, shall be according to their number, after the manner:
\verse And one kid of the goats for a sin offering; beside the continual burnt offering, and the meat offering thereof, and their drink offerings.
\verse And on the third day eleven bullocks, two rams, fourteen lambs of the first year without blemish;
\verse And their meat offering and their drink offerings for the bullocks, for the rams, and for the lambs, shall be according to their number, after the manner:
\verse And one goat for a sin offering; beside the continual burnt offering, and his meat offering, and his drink offering.
\verse And on the fourth day ten bullocks, two rams, and fourteen lambs of the first year without blemish:
\verse Their meat offering and their drink offerings for the bullocks, for the rams, and for the lambs, shall be according to their number, after the manner:
\verse And one kid of the goats for a sin offering; beside the continual burnt offering, his meat offering, and his drink offering.
\verse And on the fifth day nine bullocks, two rams, and fourteen lambs of the first year without spot:
\verse And their meat offering and their drink offerings for the bullocks, for the rams, and for the lambs, shall be according to their number, after the manner:
\verse And one goat for a sin offering; beside the continual burnt offering, and his meat offering, and his drink offering.
\verse And on the sixth day eight bullocks, two rams, and fourteen lambs of the first year without blemish:
\verse And their meat offering and their drink offerings for the bullocks, for the rams, and for the lambs, shall be according to their number, after the manner:
\verse And one goat for a sin offering; beside the continual burnt offering, his meat offering, and his drink offering.
\verse And on the seventh day seven bullocks, two rams, and fourteen lambs of the first year without blemish:
\verse And their meat offering and their drink offerings for the bullocks, for the rams, and for the lambs, shall be according to their number, after the manner:
\verse And one goat for a sin offering; beside the continual burnt offering, his meat offering, and his drink offering.
\verse On the eighth day ye shall have a solemn assembly: ye shall do no servile work therein:
\verse But ye shall offer a burnt offering, a sacrifice made by fire, of a sweet savour unto the \LORD: one bullock, one ram, seven lambs of the first year without blemish:
\verse Their meat offering and their drink offerings for the bullock, for the ram, and for the lambs, shall be according to their number, after the manner:
\verse And one goat for a sin offering; beside the continual burnt offering, and his meat offering, and his drink offering.
\verse These things ye shall do unto the \LORD in your set feasts, beside your vows, and your freewill offerings, for your burnt offerings, and for your meat offerings, and for your drink offerings, and for your peace offerings.
\verse And Moses told the children of Israel according to all that the \LORD commanded Moses.
\end{biblechapter}

\section*{Vows}
\begin{biblechapter} % Numbers 30
\verse And Moses spake unto the heads of the tribes concerning the children of Israel, saying, This is the thing which the \LORD hath commanded.
\verse If a man vow a vow unto the \LORD, or swear an oath to bind his soul with a bond; he shall not break his word, he shall do according to all that proceedeth out of his mouth.
\verse If a woman also vow a vow unto the \LORD, and bind herself by a bond, being in her father's house in her youth;
\verse And her father hear her vow, and her bond wherewith she hath bound her soul, and her father shall hold his peace at her: then all her vows shall stand, and every bond wherewith she hath bound her soul shall stand.
\verse But if her father disallow her in the day that he heareth; not any of her vows, or of her bonds wherewith she hath bound her soul, shall stand: and the \LORD shall forgive her, because her father disallowed her.
\verse And if she had at all an husband, when she vowed, or uttered ought out of her lips, wherewith she bound her soul;
\verse And her husband heard it, and held his peace at her in the day that he heard it: then her vows shall stand, and her bonds wherewith she bound her soul shall stand.
\verse But if her husband disallowed her on the day that he heard it; then he shall make her vow which she vowed, and that which she uttered with her lips, wherewith she bound her soul, of none effect: and the \LORD shall forgive her.
\verse But every vow of a widow, and of her that is divorced, wherewith they have bound their souls, shall stand against her.
\verse And if she vowed in her husband's house, or bound her soul by a bond with an oath;
\verse And her husband heard it, and held his peace at her, and disallowed her not: then all her vows shall stand, and every bond wherewith she bound her soul shall stand.
\verse But if her husband hath utterly made them void on the day he heard them; then whatsoever proceeded out of her lips concerning her vows, or concerning the bond of her soul, shall not stand: her husband hath made them void; and the \LORD shall forgive her.
\verse Every vow, and every binding oath to afflict the soul, her husband may establish it, or her husband may make it void.
\verse But if her husband altogether hold his peace at her from day to day; then he establisheth all her vows, or all her bonds, which are upon her: he confirmeth them, because he held his peace at her in the day that he heard them.
\verse But if he shall any ways make them void after that he hath heard them; then he shall bear her iniquity.
\verse These are the statutes, which the \LORD commanded Moses, between a man and his wife, between the father and his daughter, being yet in her youth in her father's house.
\end{biblechapter}

\section*{Vengeance on the Midianites}
\begin{biblechapter} % Numbers 31
\verse And the \LORD spake unto Moses, saying,
\verse Avenge the children of Israel of the Midianites: afterward shalt thou be gathered unto thy people.
\verse And Moses spake unto the people, saying, Arm some of yourselves unto the war, and let them go against the Midianites, and avenge the \LORD of Midian.
\verse Of every tribe a thousand, throughout all the tribes of Israel, shall ye send to the war.
\verse So there were delivered out of the thousands of Israel, a thousand of every tribe, twelve thousand armed for war.
\verse And Moses sent them to the war, a thousand of every tribe, them and Phinehas the son of Eleazar the priest, to the war, with the holy instruments, and the trumpets to blow in his hand.
\verse And they warred against the Midianites, as the \LORD commanded Moses; and they slew all the males.
\verse And they slew the kings of Midian, beside the rest of them that were slain; namely, Evi, and Rekem, and Zur, and Hur, and Reba, five kings of Midian: Balaam also the son of Beor they slew with the sword.
\verse And the children of Israel took all the women of Midian captives, and their little ones, and took the spoil of all their cattle, and all their flocks, and all their goods.
\verse And they burnt all their cities wherein they dwelt, and all their goodly castles, with fire.
\verse And they took all the spoil, and all the prey, both of men and of beasts.
\verse And they brought the captives, and the prey, and the spoil, unto Moses, and Eleazar the priest, and unto the congregation of the children of Israel, unto the camp at the plains of Moab, which are by Jordan near Jericho.
\verse And Moses, and Eleazar the priest, and all the princes of the congregation, went forth to meet them without the camp.
\verse And Moses was wroth with the officers of the host, with the captains over thousands, and captains over hundreds, which came from the battle.
\verse And Moses said unto them, Have ye saved all the women alive?
\verse Behold, these caused the children of Israel, through the counsel of Balaam, to commit trespass against the \LORD in the matter of Peor, and there was a plague among the congregation of the \LORD.
\verse Now therefore kill every male among the little ones, and kill every woman that hath known man by lying with him.
\verse But all the women children, that have not known a man by lying with him, keep alive for yourselves.
\verse And do ye abide without the camp seven days: whosoever hath killed any person, and whosoever hath touched any slain, purify both yourselves and your captives on the third day, and on the seventh day.
\verse And purify all your raiment, and all that is made of skins, and all work of goats' hair, and all things made of wood.
\verse And Eleazar the priest said unto the men of war which went to the battle, This is the ordinance of the law which the \LORD commanded Moses;
\verse Only the gold, and the silver, the brass, the iron, the tin, and the lead,
\verse Every thing that may abide the fire, ye shall make it go through the fire, and it shall be clean: nevertheless it shall be purified with the water of separation: and all that abideth not the fire ye shall make go through the water.
\verse And ye shall wash your clothes on the seventh day, and ye shall be clean, and afterward ye shall come into the camp.
\section*{Dividing the spoils}
\verse And the \LORD spake unto Moses, saying,
\verse Take the sum of the prey that was taken, both of man and of beast, thou, and Eleazar the priest, and the chief fathers of the congregation:
\verse And divide the prey into two parts; between them that took the war upon them, who went out to battle, and between all the congregation:
\verse And levy a tribute unto the \LORD of the men of war which went out to battle: one soul of five hundred, both of the persons, and of the beeves, and of the asses, and of the sheep:
\verse Take it of their half, and give it unto Eleazar the priest, for an heave offering of the \LORD.
\verse And of the children of Israel's half, thou shalt take one portion of fifty, of the persons, of the beeves, of the asses, and of the flocks, of all manner of beasts, and give them unto the Levites, which keep the charge of the tabernacle of the \LORD.
\verse And Moses and Eleazar the priest did as the \LORD commanded Moses.
\verse And the booty, being the rest of the prey which the men of war had caught, was six hundred thousand and seventy thousand and five thousand sheep,
\verse And threescore and twelve thousand beeves,
\verse And threescore and one thousand asses,
\verse And thirty and two thousand persons in all, of women that had not known man by lying with him.
\verse And the half, which was the portion of them that went out to war, was in number three hundred thousand and seven and thirty thousand and five hundred sheep:
\verse And the \LORDs tribute of the sheep was six hundred and threescore and fifteen.
\verse And the beeves were thirty and six thousand; of which the \LORDs tribute was threescore and twelve.
\verse And the asses were thirty thousand and five hundred; of which the \LORDs tribute was threescore and one.
\verse And the persons were sixteen thousand; of which the \LORDs tribute was thirty and two persons.
\verse And Moses gave the tribute, which was the \LORDs heave offering, unto Eleazar the priest, as the \LORD commanded Moses.
\verse And of the children of Israel's half, which Moses divided from the men that warred,
\verse (Now the half that pertained unto the congregation was three hundred thousand and thirty thousand and seven thousand and five hundred sheep,
\verse And thirty and six thousand beeves,
\verse And thirty thousand asses and five hundred,
\verse And sixteen thousand persons;)
\verse Even of the children of Israel's half, Moses took one portion of fifty, both of man and of beast, and gave them unto the Levites, which kept the charge of the tabernacle of the \LORD; as the \LORD commanded Moses.
\verse And the officers which were over thousands of the host, the captains of thousands, and captains of hundreds, came near unto Moses:
\verse And they said unto Moses, Thy servants have taken the sum of the men of war which are under our charge, and there lacketh not one man of us.
\verse We have therefore brought an oblation for the \LORD, what every man hath gotten, of jewels of gold, chains, and bracelets, rings, earrings, and tablets, to make an atonement for our souls before the \LORD.
\verse And Moses and Eleazar the priest took the gold of them, even all wrought jewels.
\verse And all the gold of the offering that they offered up to the \LORD, of the captains of thousands, and of the captains of hundreds, was sixteen thousand seven hundred and fifty shekels.
\verse (For the men of war had taken spoil, every man for himself.)
\verse And Moses and Eleazar the priest took the gold of the captains of thousands and of hundreds, and brought it into the tabernacle of the congregation, for a memorial for the children of Israel before the \LORD.
\end{biblechapter}

\section*{The Transjordan tribes}
\begin{biblechapter} % Numbers 32
\verse Now the children of Reuben and the children of Gad had a very great multitude of cattle: and when they saw the land of Jazer, and the land of Gilead, that, behold, the place was a place for cattle;
\verse The children of Gad and the children of Reuben came and spake unto Moses, and to Eleazar the priest, and unto the princes of the congregation, saying,
\verse Ataroth, and Dibon, and Jazer, and Nimrah, and Heshbon, and Elealeh, and Shebam, and Nebo, and Beon,
\verse Even the country which the \LORD smote before the congregation of Israel, is a land for cattle, and thy servants have cattle:
\verse Wherefore, said they, if we have found grace in thy sight, let this land be given unto thy servants for a possession, and bring us not over Jordan.
\verse And Moses said unto the children of Gad and to the children of Reuben, Shall your brethren go to war, and shall ye sit here?
\verse And wherefore discourage ye the heart of the children of Israel from going over into the land which the \LORD hath given them?
\verse Thus did your fathers, when I sent them from Kadeshbarnea to see the land.
\verse For when they went up unto the valley of Eshcol, and saw the land, they discouraged the heart of the children of Israel, that they should not go into the land which the \LORD had given them.
\verse And the \LORDs anger was kindled the same time, and he sware, saying,
\verse Surely none of the men that came up out of Egypt, from twenty years old and upward, shall see the land which I sware unto Abraham, unto Isaac, and unto Jacob; because they have not wholly followed me:
\verse Save Caleb the son of Jephunneh the Kenezite, and Joshua the son of Nun: for they have wholly followed the \LORD.
\verse And the \LORDs anger was kindled against Israel, and he made them wander in the wilderness forty years, until all the generation, that had done evil in the sight of the \LORD, was consumed.
\verse And, behold, ye are risen up in your fathers' stead, an increase of sinful men, to augment yet the fierce anger of the \LORD toward Israel.
\verse For if ye turn away from after him, he will yet again leave them in the wilderness; and ye shall destroy all this people.
\verse And they came near unto him, and said, We will build sheepfolds here for our cattle, and cities for our little ones:
\verse But we ourselves will go ready armed before the children of Israel, until we have brought them unto their place: and our little ones shall dwell in the fenced cities because of the inhabitants of the land.
\verse We will not return unto our houses, until the children of Israel have inherited every man his inheritance.
\verse For we will not inherit with them on yonder side Jordan, or forward; because our inheritance is fallen to us on this side Jordan eastward.
\verse And Moses said unto them, If ye will do this thing, if ye will go armed before the \LORD to war,
\verse And will go all of you armed over Jordan before the \LORD, until he hath driven out his enemies from before him,
\verse And the land be subdued before the \LORD: then afterward ye shall return, and be guiltless before the \LORD, and before Israel; and this land shall be your possession before the \LORD.
\verse But if ye will not do so, behold, ye have sinned against the \LORD: and be sure your sin will find you out.
\verse Build you cities for your little ones, and folds for your sheep; and do that which hath proceeded out of your mouth.
\verse And the children of Gad and the children of Reuben spake unto Moses, saying, Thy servants will do as my lord commandeth.
\verse Our little ones, our wives, our flocks, and all our cattle, shall be there in the cities of Gilead:
\verse But thy servants will pass over, every man armed for war, before the \LORD to battle, as my lord saith.
\verse So concerning them Moses commanded Eleazar the priest, and Joshua the son of Nun, and the chief fathers of the tribes of the children of Israel:
\verse And Moses said unto them, If the children of Gad and the children of Reuben will pass with you over Jordan, every man armed to battle, before the \LORD, and the land shall be subdued before you; then ye shall give them the land of Gilead for a possession:
\verse But if they will not pass over with you armed, they shall have possessions among you in the land of Canaan.
\verse And the children of Gad and the children of Reuben answered, saying, As the \LORD hath said unto thy servants, so will we do.
\verse We will pass over armed before the \LORD into the land of Canaan, that the possession of our inheritance on this side Jordan may be ours.
\verse And Moses gave unto them, even to the children of Gad, and to the children of Reuben, and unto half the tribe of Manasseh the son of Joseph, the kingdom of Sihon king of the Amorites, and the kingdom of Og king of Bashan, the land, with the cities thereof in the coasts, even the cities of the country round about.
\verse And the children of Gad built Dibon, and Ataroth, and Aroer,
\verse And Atroth, Shophan, and Jaazer, and Jogbehah,
\verse And Bethnimrah, and Bethharan, fenced cities: and folds for sheep.
\verse And the children of Reuben built Heshbon, and Elealeh, and Kirjathaim,
\verse And Nebo, and Baalmeon, (their names being changed,) and Shibmah: and gave other names unto the cities which they builded.
\verse And the children of Machir the son of Manasseh went to Gilead, and took it, and dispossessed the Amorite which was in it.
\verse And Moses gave Gilead unto Machir the son of Manasseh; and he dwelt therein.
\verse And Jair the son of Manasseh went and took the small towns thereof, and called them Havothjair.
\verse And Nobah went and took Kenath, and the villages thereof, and called it Nobah, after his own name.
\end{biblechapter}

\section*{Stages on Israel's journey}
\begin{biblechapter} % Numbers 33
\verse These are the journeys of the children of Israel, which went forth out of the land of Egypt with their armies under the hand of Moses and Aaron.
\verse And Moses wrote their goings out according to their journeys by the commandment of the \LORD: and these are their journeys according to their goings out.
\verse And they departed from Rameses in the first month, on the fifteenth day of the first month; on the morrow after the passover the children of Israel went out with an high hand in the sight of all the Egyptians.
\verse For the Egyptians buried all their firstborn, which the \LORD had smitten among them: upon their gods also the \LORD executed judgments.
\verse And the children of Israel removed from Rameses, and pitched in Succoth.
\verse And they departed from Succoth, and pitched in Etham, which is in the edge of the wilderness.
\verse And they removed from Etham, and turned again unto Pihahiroth, which is before Baalzephon: and they pitched before Migdol.
\verse And they departed from before Pihahiroth, and passed through the midst of the sea into the wilderness, and went three days' journey in the wilderness of Etham, and pitched in Marah.
\verse And they removed from Marah, and came unto Elim: and in Elim were twelve fountains of water, and threescore and ten palm trees; and they pitched there.
\verse And they removed from Elim, and encamped by the Red sea.
\verse And they removed from the Red sea, and encamped in the wilderness of Sin.
\verse And they took their journey out of the wilderness of Sin, and encamped in Dophkah.
\verse And they departed from Dophkah, and encamped in Alush.
\verse And they removed from Alush, and encamped at Rephidim, where was no water for the people to drink.
\verse And they departed from Rephidim, and pitched in the wilderness of Sinai.
\verse And they removed from the desert of Sinai, and pitched at Kibrothhattaavah.
\verse And they departed from Kibrothhattaavah, and encamped at Hazeroth.
\verse And they departed from Hazeroth, and pitched in Rithmah.
\verse And they departed from Rithmah, and pitched at Rimmonparez.
\verse And they departed from Rimmonparez, and pitched in Libnah.
\verse And they removed from Libnah, and pitched at Rissah.
\verse And they journeyed from Rissah, and pitched in Kehelathah.
\verse And they went from Kehelathah, and pitched in mount Shapher.
\verse And they removed from mount Shapher, and encamped in Haradah.
\verse And they removed from Haradah, and pitched in Makheloth.
\verse And they removed from Makheloth, and encamped at Tahath.
\verse And they departed from Tahath, and pitched at Tarah.
\verse And they removed from Tarah, and pitched in Mithcah.
\verse And they went from Mithcah, and pitched in Hashmonah.
\verse And they departed from Hashmonah, and encamped at Moseroth.
\verse And they departed from Moseroth, and pitched in Benejaakan.
\verse And they removed from Benejaakan, and encamped at Horhagidgad.
\verse And they went from Horhagidgad, and pitched in Jotbathah.
\verse And they removed from Jotbathah, and encamped at Ebronah.
\verse And they departed from Ebronah, and encamped at Eziongaber.
\verse And they removed from Eziongaber, and pitched in the wilderness of Zin, which is Kadesh.
\verse And they removed from Kadesh, and pitched in mount Hor, in the edge of the land of Edom.
\verse And Aaron the priest went up into mount Hor at the commandment of the \LORD, and died there, in the fortieth year after the children of Israel were come out of the land of Egypt, in the first day of the fifth month.
\verse And Aaron was an hundred and twenty and three years old when he died in mount Hor.
\verse And king Arad the Canaanite, which dwelt in the south in the land of Canaan, heard of the coming of the children of Israel.
\verse And they departed from mount Hor, and pitched in Zalmonah.
\verse And they departed from Zalmonah, and pitched in Punon.
\verse And they departed from Punon, and pitched in Oboth.
\verse And they departed from Oboth, and pitched in Ijeabarim, in the border of Moab.
\verse And they departed from Iim, and pitched in Dibongad.
\verse And they removed from Dibongad, and encamped in Almondiblathaim.
\verse And they removed from Almondiblathaim, and pitched in the mountains of Abarim, before Nebo.
\verse And they departed from the mountains of Abarim, and pitched in the plains of Moab by Jordan near Jericho.
\verse And they pitched by Jordan, from Bethjesimoth even unto Abelshittim in the plains of Moab.
\verse And the \LORD spake unto Moses in the plains of Moab by Jordan near Jericho, saying,
\verse Speak unto the children of Israel, and say unto them, When ye are passed over Jordan into the land of Canaan;
\verse Then ye shall drive out all the inhabitants of the land from before you, and destroy all their pictures, and destroy all their molten images, and quite pluck down all their high places:
\verse And ye shall dispossess the inhabitants of the land, and dwell therein: for I have given you the land to possess it.
\verse And ye shall divide the land by lot for an inheritance among your families: and to the more ye shall give the more inheritance, and to the fewer ye shall give the less inheritance: every man's inheritance shall be in the place where his lot falleth; according to the tribes of your fathers ye shall inherit.
\verse But if ye will not drive out the inhabitants of the land from before you; then it shall come to pass, that those which ye let remain of them shall be pricks in your eyes, and thorns in your sides, and shall vex you in the land wherein ye dwell.
\verse Moreover it shall come to pass, that I shall do unto you, as I thought to do unto them.
\end{biblechapter}

\section*{Boundaries of Canaan}
\begin{biblechapter} % Numbers 34
\verse And the \LORD spake unto Moses, saying,
\verse Command the children of Israel, and say unto them, When ye come into the land of Canaan; (this is the land that shall fall unto you for an inheritance, even the land of Canaan with the coasts thereof:)
\verse Then your south quarter shall be from the wilderness of Zin along by the coast of Edom, and your south border shall be the outmost coast of the salt sea eastward:
\verse And your border shall turn from the south to the ascent of Akrabbim, and pass on to Zin: and the going forth thereof shall be from the south to Kadeshbarnea, and shall go on to Hazaraddar, and pass on to Azmon:
\verse And the border shall fetch a compass from Azmon unto the river of Egypt, and the goings out of it shall be at the sea.
\verse And as for the western border, ye shall even have the great sea for a border: this shall be your west border.
\verse And this shall be your north border: from the great sea ye shall point out for you mount Hor:
\verse From mount Hor ye shall point out your border unto the entrance of Hamath; and the goings forth of the border shall be to Zedad:
\verse And the border shall go on to Ziphron, and the goings out of it shall be at Hazarenan: this shall be your north border.
\verse And ye shall point out your east border from Hazarenan to Shepham:
\verse And the coast shall go down from Shepham to Riblah, on the east side of Ain; and the border shall descend, and shall reach unto the side of the sea of Chinnereth eastward:
\verse And the border shall go down to Jordan, and the goings out of it shall be at the salt sea: this shall be your land with the coasts thereof round about.
\verse And Moses commanded the children of Israel, saying, This is the land which ye shall inherit by lot, which the \LORD commanded to give unto the nine tribes, and to the half tribe:
\verse For the tribe of the children of Reuben according to the house of their fathers, and the tribe of the children of Gad according to the house of their fathers, have received their inheritance; and half the tribe of Manasseh have received their inheritance:
\verse The two tribes and the half tribe have received their inheritance on this side Jordan near Jericho eastward, toward the sunrising.
\verse And the \LORD spake unto Moses, saying,
\verse These are the names of the men which shall divide the land unto you: Eleazar the priest, and Joshua the son of Nun.
\verse And ye shall take one prince of every tribe, to divide the land by inheritance.
\verse And the names of the men are these: Of the tribe of Judah, Caleb the son of Jephunneh.
\verse And of the tribe of the children of Simeon, Shemuel the son of Ammihud.
\verse Of the tribe of Benjamin, Elidad the son of Chislon.
\verse And the prince of the tribe of the children of Dan, Bukki the son of Jogli.
\verse The prince of the children of Joseph, for the tribe of the children of Manasseh, Hanniel the son of Ephod.
\verse And the prince of the tribe of the children of Ephraim, Kemuel the son of Shiphtan.
\verse And the prince of the tribe of the children of Zebulun, Elizaphan the son of Parnach.
\verse And the prince of the tribe of the children of Issachar, Paltiel the son of Azzan.
\verse And the prince of the tribe of the children of Asher, Ahihud the son of Shelomi.
\verse And the prince of the tribe of the children of Naphtali, Pedahel the son of Ammihud.
\verse These are they whom the \LORD commanded to divide the inheritance unto the children of Israel in the land of Canaan.
\end{biblechapter}

\section*{Towns for the Levites}
\begin{biblechapter} % Numbers 35
\verse And the \LORD spake unto Moses in the plains of Moab by Jordan near Jericho, saying,
\verse Command the children of Israel, that they give unto the Levites of the inheritance of their possession cities to dwell in; and ye shall give also unto the Levites suburbs for the cities round about them.
\verse And the cities shall they have to dwell in; and the suburbs of them shall be for their cattle, and for their goods, and for all their beasts.
\verse And the suburbs of the cities, which ye shall give unto the Levites, shall reach from the wall of the city and outward a thousand cubits round about.
\verse And ye shall measure from without the city on the east side two thousand cubits, and on the south side two thousand cubits, and on the west side two thousand cubits, and on the north side two thousand cubits; and the city shall be in the midst: this shall be to them the suburbs of the cities.
\section*{Cities of refuge}
\verse And among the cities which ye shall give unto the Levites there shall be six cities for refuge, which ye shall appoint for the manslayer, that he may flee thither: and to them ye shall add forty and two cities.
\verse So all the cities which ye shall give to the Levites shall be forty and eight cities: them shall ye give with their suburbs.
\verse And the cities which ye shall give shall be of the possession of the children of Israel: from them that have many ye shall give many; but from them that have few ye shall give few: every one shall give of his cities unto the Levites according to his inheritance which he inheriteth.
\verse And the \LORD spake unto Moses, saying,
\verse Speak unto the children of Israel, and say unto them, When ye be come over Jordan into the land of Canaan;
\verse Then ye shall appoint you cities to be cities of refuge for you; that the slayer may flee thither, which killeth any person at unawares.
\verse And they shall be unto you cities for refuge from the avenger; that the manslayer die not, until he stand before the congregation in judgment.
\verse And of these cities which ye shall give six cities shall ye have for refuge.
\verse Ye shall give three cities on this side Jordan, and three cities shall ye give in the land of Canaan, which shall be cities of refuge.
\verse These six cities shall be a refuge, both for the children of Israel, and for the stranger, and for the sojourner among them: that every one that killeth any person unawares may flee thither.
\verse And if he smite him with an instrument of iron, so that he die, he is a murderer: the murderer shall surely be put to death.
\verse And if he smite him with throwing a stone, wherewith he may die, and he die, he is a murderer: the murderer shall surely be put to death.
\verse Or if he smite him with an hand weapon of wood, wherewith he may die, and he die, he is a murderer: the murderer shall surely be put to death.
\verse The revenger of blood himself shall slay the murderer: when he meeteth him, he shall slay him.
\verse But if he thrust him of hatred, or hurl at him by laying of wait, that he die;
\verse Or in enmity smite him with his hand, that he die: he that smote him shall surely be put to death; for he is a murderer: the revenger of blood shall slay the murderer, when he meeteth him.
\verse But if he thrust him suddenly without enmity, or have cast upon him any thing without laying of wait,
\verse Or with any stone, wherewith a man may die, seeing him not, and cast it upon him, that he die, and was not his enemy, neither sought his harm:
\verse Then the congregation shall judge between the slayer and the revenger of blood according to these judgments:
\verse And the congregation shall deliver the slayer out of the hand of the revenger of blood, and the congregation shall restore him to the city of his refuge, whither he was fled: and he shall abide in it unto the death of the high priest, which was anointed with the holy oil.
\verse But if the slayer shall at any time come without the border of the city of his refuge, whither he was fled;
\verse And the revenger of blood find him without the borders of the city of his refuge, and the revenger of blood kill the slayer; he shall not be guilty of blood:
\verse Because he should have remained in the city of his refuge until the death of the high priest: but after the death of the high priest the slayer shall return into the land of his possession.
\verse So these things shall be for a statute of judgment unto you throughout your generations in all your dwellings.
\verse Whoso killeth any person, the murderer shall be put to death by the mouth of witnesses: but one witness shall not testify against any person to cause him to die.
\verse Moreover ye shall take no satisfaction for the life of a murderer, which is guilty of death: but he shall be surely put to death.
\verse And ye shall take no satisfaction for him that is fled to the city of his refuge, that he should come again to dwell in the land, until the death of the priest.
\verse So ye shall not pollute the land wherein ye are: for blood it defileth the land: and the land cannot be cleansed of the blood that is shed therein, but by the blood of him that shed it.
\verse Defile not therefore the land which ye shall inhabit, wherein I dwell: for I the \LORD dwell among the children of Israel.
\end{biblechapter}

\section*{Inheritance of Zelophehad's daughters}
\begin{biblechapter} % Numbers 36
\verse And the chief fathers of the families of the children of Gilead, the son of Machir, the son of Manasseh, of the families of the sons of Joseph, came near, and spake before Moses, and before the princes, the chief fathers of the children of Israel:
\verse And they said, The \LORD commanded my lord to give the land for an inheritance by lot to the children of Israel: and my lord was commanded by the \LORD to give the inheritance of Zelophehad our brother unto his daughters.
\verse And if they be married to any of the sons of the other tribes of the children of Israel, then shall their inheritance be taken from the inheritance of our fathers, and shall be put to the inheritance of the tribe whereunto they are received: so shall it be taken from the lot of our inheritance.
\verse And when the jubile of the children of Israel shall be, then shall their inheritance be put unto the inheritance of the tribe whereunto they are received: so shall their inheritance be taken away from the inheritance of the tribe of our fathers.
\verse And Moses commanded the children of Israel according to the word of the \LORD, saying, The tribe of the sons of Joseph hath said well.
\verse This is the thing which the \LORD doth command concerning the daughters of Zelophehad, saying, Let them marry to whom they think best; only to the family of the tribe of their father shall they marry.
\verse So shall not the inheritance of the children of Israel remove from tribe to tribe: for every one of the children of Israel shall keep himself to the inheritance of the tribe of his fathers.
\verse And every daughter, that possesseth an inheritance in any tribe of the children of Israel, shall be wife unto one of the family of the tribe of her father, that the children of Israel may enjoy every man the inheritance of his fathers.
\verse Neither shall the inheritance remove from one tribe to another tribe; but every one of the tribes of the children of Israel shall keep himself to his own inheritance.
\verse Even as the \LORD commanded Moses, so did the daughters of Zelophehad:
\verse For Mahlah, Tirzah, and Hoglah, and Milcah, and Noah, the daughters of Zelophehad, were married unto their father's brothers' sons:
\verse And they were married into the families of the sons of Manasseh the son of Joseph, and their inheritance remained in the tribe of the family of their father.
\verse These are the commandments and the judgments, which the \LORD commanded by the hand of Moses unto the children of Israel in the plains of Moab by Jordan near Jericho.
\end{biblechapter}
\flushcolsend
\biblebook{Deuteronomy}

\begin{biblechapter} % Deuteronomy 1
\verseWithHeading{Preamble} These are the words that Moses spoke to all of Israel \textit{on the other side of}\lebnote{Literally “in the beyond of”} the Jordan in the desert,\lebnote{Or “the wilderness”} in the desert plateau\lebnote{Or “desert plain”; others render this as a named location, “Arabah”} opposite Suph, between Paran and between Tophel and Laban and Hazeroth and Dizahab.
\verse It is a journey of \textit{eleven days}\lebnote{Literally “one and ten days”} from Herob \textit{by the way of Mount Seir}\lebnote{Literally “by the mountain of Seir”} up to Kadesh Barnea.
\verse \textit{And it was}\lebnote{Literally “and it happened” or “and then”} in the fortieth year, on the eleventh month, on the first day of the month, Moses spoke to the \textit{Israelites}\lebnote{Literally “sons/children of Israel”} according to all that Adonai had instructed him to speak to them.
\verse This happened \textit{after defeating}\lebnote{Literally “after he had struck down”} Sihon king of the Amorites, who was reigning\lebnote{Or “dwelling,” and in this context “reigning”} in Heshbon, and Og the king of Bashan, who was reigning\lebnote{Or “dwelling,” and in this context “reigning”} in Ashtaroth in Edrei.
\verse On the other side of \lebnote{Literally “in the beyond of”} the Jordan in the land of Moab Moses began to explain this law,\lebnote{Hebrew \textit{torah|i|\lebnote{: \textit{saying}:\lebnote{Literally “to say”}
\verseWithHeading{Historical Prologue} “Adonai our God spoke to us at Horeb, \textit{saying},\lebnote{Literally “to say”} ‘You have stayed \textit{long enough}\lebnote{Literally “much time”} at this mountain.
\verse Turn now and \textit{move on},\lebnote{Literally “journey on with respect to yourselves” or “move yourselves out”} and go into the hill of the Amorites\lebnote{Hebrew “Amorite”} and to all of the neighboring regions\lebnote{Or “peoples”} in the Jordan \textit{Valley}\lebnote{Literally “Arabah,” arid stretches of land.} in the hill country and in the Negev and in the coastal area along the sea, into the land of the Canaanites\lebnote{Hebrew “Canaanite”} and into the Lebanon, as far as the great river Euphrates.
\verse Look, I have set the land \textit{before you};\lebnote{Literally “before your face”} go and take possession of the land that Adonai swore to your ancestors,\lebnote{Or “fathers”} to Abraham, to Isaac, and to Jacob, to give it to them and to their offspring after them.’
\verse “And I spoke to them at that time, \textit{saying},\lebnote{Literally “to say”} ‘I am not able to bear you alone.
\verse Adonai your God has multiplied you, and look; you are today as the stars of the heaven \textit{in number}.\lebnote{Literally “with respect to multitude/abundance”}
\verse Adonai, the God of your ancestors,\lebnote{Or “fathers”} may he add to you as you are now a thousand times, and may he bless you just as he \textit{promised you}.\lebnote{Literally “spoke to you”}
\verse How can I bear you \textit{by myself},\lebnote{Literally “to me alone”} your burden and your load and your strife?
\verse Choose for yourselves \textit{wise and discerning and knowledgeable men}\lebnote{Literally “men wise and discerning and knowledgeable”} for each of your tribes, and I will appoint them as your leaders.’
\verse “And you answered me, and you said, ‘The thing you have said to do is good.’
\verse And so I took the leaders of your tribes, wise and knowledgeable men, and then I appointed them as leaders over you as commanders of groups of thousands and commanders of groups of hundreds and commanders of groups of fifties and commanders of groups of tens as officials for your tribes.
\verse And at that time I instructed your judges, saying, ‘\textit{hear out your fellow men},\lebnote{Literally “hear between your brothers,” with the idea of listening carefully in this context} and then judge fairly\lebnote{Or “righteously”} between a man and between his brother and between \textit{his opponent who is a resident alien}.\lebnote{Literally “between his resident alien/non-Israelite who dwells temporarily among Israel”}
\verse You must not \textit{be partial}\lebnote{Literally “recognize faces”} \textit{in your judgment}; \lebnote{Literally “in the process of rendering a judgment”} hear out the small person as also the great person; \textit{do not be intimidated by any person},\lebnote{Literally “do not fear before the faces of any man”} because the judgment is God’s; and the case that is too difficult\lebnote{Or “hard”} for you, bring it to me, and I will hear it out.’
\verse And so I instructed you at that time concerning all of the things that you should do.
\verse “Then we set out from Horeb, and we went through the whole of that great and terrible desert that you saw on the way to the hill country of the Amorites\lebnote{Hebrew “Amorite”} as Adonai our God had commanded us, and so we came up to Kadesh Barnea.
\verse I said to you, ‘\textit{You have reached}\lebnote{Literally “You have come up to”} the hill country of the Amorites\lebnote{Hebrew “Amorite”} that Adonai our God is giving to us.
\verse See, Adonai your God has set before you the land; go up and possess it as Adonai the God of your ancestors\lebnote{Or “fathers”} said to you; do not fear and do not be dismayed.’
\verse “Then all of you approached me, and you said, ‘Let us send men \textit{before us},\lebnote{Literally “before our faces” or “ahead of us”} and let them explore the land for us, and let them bring back \textit{a report}\lebnote{Literally “a word”} to us concerning the way that \textit{we should take}\lebnote{Literally “we should go up by it”} and concerning the cities that we shall come to.’
\verse The plan was good \textit{in my opinion},\lebnote{Literally “in my eyes”} and so I took from among you twelve men, \textit{one from each tribe}.\lebnote{Literally “man one from the tribe”}
\verse And they \textit{set out}\lebnote{Literally “turned”} and \textit{went up into the hill country},\lebnote{Literally “went up toward the hill country mountain”} and they went up to the wadi\lebnote{A valley that is dry most of the year, but contains a stream during the rainy season} of Eschol, and they spied out the land.
\verse They took in their hands\lebnote{Hebrew “hand”} \textit{some of the fruit}\lebnote{Literally “from the fruit”} of the land, and they brought it down to us, and they brought to us back \textit{a report},\lebnote{Literally “a word”} and they said, ‘The land that Adonai our God is giving to us is good.’
\verse But you were not willing to go up, and you rebelled against the \textit{command}\lebnote{Literally “mouth”} of Adonai your God.
\verse And you grumbled in your tents, and you said, ‘Because of the hatred of Adonai toward us he has brought us out from the land of Egypt to give us into the hand of the Amorites\lebnote{Hebrew “Amorite”} to destroy us.
\verse Where can we go up? Our brothers have \textit{made our hearts melt},\lebnote{Literally “caused to melt our hearts”} \textit{saying},\lebnote{Literally “to say”} “The people are greater\lebnote{Or “bigger”} and taller than we are,\lebnote{Hebrew “than us”} and there are great fortified cities reaching up to heaven, and we saw the sons of the Anakites living there.” ’
\verse “And so I said to you, ‘Do not be terrified, and do not fear them.
\verse Adonai your God, who is going \textit{before you},\lebnote{Literally “to your faces”} will himself\lebnote{The Hebrew pronoun indicates emphasis} fight for you, \textit{just as}\lebnote{Literally “like all that”} he did for you in Egypt before your eyes,
\verse and just as he did in the wilderness\lebnote{Or “desert”} when\lebnote{Or “where”} you saw that\lebnote{Or “how”} Adonai your God carried you, just as someone\lebnote{Or “a man”} carries his son, all along the way that you traveled until \textit{you reached}\lebnote{Literally “you came”} this place.’
\verse But through all of this you did not trust in Adonai your God,
\verse \textit{who goes}\lebnote{Literally “is the one going”} \textit{before you}\lebnote{Literally “before your faces”} on your\lebnote{Hebrew “the”} way, seeking a place for your encampment, in fire at night and in a cloud by day, to show you the way that \textit{you should go}.\lebnote{Literally “you should go in it”}
\verse “Then Adonai heard the sound of your words, and he was angry, and he swore, \textit{saying},\lebnote{Literally “to say”}
\verse ‘No one of these men\lebnote{The beginning of an oath in Hebrew; the text emphatically denies anyone of that evil generation the privilege of seeing the promised land} of this evil generation will see the good land that I swore to give to your ancestors,\lebnote{Or “fathers”}
\verse except Caleb, the son of Jephunneh; he himself\lebnote{Emphatic use of the pronoun} shall see it, and to him I will give the land upon which he has trodden and to his sons\lebnote{Or “descendants”} because \textit{he followed Adonai unreservedly}.’\lebnote{Literally “he filled his hands after Adonai”}
\verse Even with me Adonai was angry because of you, saying, ‘Not even you shall enter there.
\verse Joshua, the son of Nun, \textit{your assistant},\lebnote{Literally “the one standing before you”} will go there; encourage him because he will cause Israel to inherit it.
\verse And your little children, who you thought shall become plunder, and your sons, who do not today know good or bad, shall themselves\lebnote{The Hebrew pronoun is used for emphasis} go there, and I will give it to them, and they shall take possession of it.
\verse But you turn and set out in the direction of the wilderness by way of the \textit{Red Sea}.’\lebnote{Literally “sea of reeds”}
\verse “You replied and said to me, ‘We have sinned against Adonai, and now we will go up and fight according to all that Adonai our God commanded us’; and so each man fastened on \textit{his battle gear},\lebnote{Literally “his vessels of battle”} and you regarded it as easy to go up into the hill country.
\verse So Adonai said to me, ‘Say to them, “You shall not go up, and you shall not fight because I am not in your midst; you will be defeated \textit{before}\lebnote{Literally “in the faces of your enemies”} your enemies.” ’
\verse So I spoke to you, but you did not listen; you rebelled against the \textit{command of Adonai};\lebnote{Literally “mouth of Adonai”} you behaved presumptuously, and you went up into the hill country.
\verse The Amorites living in the hill country went out \textit{to oppose you}\lebnote{Literally “to meet you”} and chased you as a swarm of wild honey bees do; and so they \textit{beat}\lebnote{Literally “cut”} you down in Seir as far as Hormah.
\verse So you returned and wept \textit{before Adonai};\lebnote{Literally “before the faces of Adonai”} but Adonai did not listen to your voice and did not pay any attention to you.
\verse You stayed in Kadesh many days; such were the days that you stayed there.
\end{biblechapter}

\begin{biblechapter} % Deuteronomy 2
\verse “Then we turned and set out toward the wilderness\lebnote{Or “desert”} in the direction of the \textit{Red Sea},\lebnote{Literally “sea of reeds”} as Adonai told me, and we went around Mount Seir for many days.
\verse Adonai spoke to me, \textit{saying},\lebnote{Literally “to say”}
\verse ‘Long enough you have been skirting this mountain; turn yourselves north,
\verse and instruct\lebnote{Or “command”} the people, \textit{saying},\lebnote{Literally “to say”} “You are about to cross through the territory of your brothers, the descendants\lebnote{Or “sons”} of Esau, who are living in Seir; they will be afraid of you, and so be very careful.
\verse Do not get involved in battle\lebnote{Or “provoke”} with them, for I will not give you any of their land, not even \textit{a foot’s breadth}\lebnote{Literally “a sole’s foot of space”} of it; since I have given Mount Seir as a possession for Esau.
\verse You shall buy food from them so that you may eat; and also you shall purchase water from them with money so that you may drink.
\verse The fact of the matter is, Adonai your God has blessed you in \textit{all the work you have done};\lebnote{Literally “the work of your hand”} he knows \textit{your travels}\lebnote{Literally “your journeying”} with respect to this great wilderness; forty years Adonai your God has been with you; you have not lacked a thing.” ’
\verse And so we passed by our brothers, the descendants\lebnote{Or “sons”} of Esau, who live in Seir, past the road of the Arabah,\lebnote{Or “Jordan Valley” since the Arabah is an extension of it} from Elath and Ezion Geber, and we turned and traveled along the route of the desert\lebnote{Or “wilderness”} of Moab.
\verse And Adonai said to me, ‘You shall not attack Moab, and you shall not engage in war\lebnote{Or “battle”} with them, for I will not give you any of his land as a possession; I have given Ar to the descendants\lebnote{Or “sons”} of Lot as a possession.’
\verse (The Emim previously lived in it, a people large,\lebnote{Or “great” in the sense of influential} numerous, and tall, like the Anakites.
\verse They were reckoned also as Rephaim as the Anakites were; but the Moabites called them Emim.
\verse The Horites previously lived in Seir, but the descendants\lebnote{Or “sons”} of Esau dispossessed them and destroyed them \textit{from among themselves},\lebnote{Literally “from their presence”} as Israel did with respect to the land of their\lebnote{Literally “its/his”} possession that Adonai gave to them.)
\verse So now arise and cross over the wadi\lebnote{A valley that is dry most of the year, but contains a stream during the rainy season} of Zered yourselves; and so we crossed the wadi\lebnote{A valley that is dry most of the year, but contains a stream during the rainy season} of Zered.
\verse Now the \textit{length of time}\lebnote{Literally “days”} that we had traveled from Kadesh Barnea until the time when we crossed the wadi\lebnote{A valley that is dry most of the year, but contains a stream during the rainy season} of Zered was thirty-eight years, until the perishing of all of that generation; that is, the men of war from the midst of the camp as Adonai had sworn to them.
\verse The hand of Adonai was against them to root them out from the midst of the camp until they perished completely.
\verse “\textit{And then}\lebnote{Literally “And it happened”} when all the men of war\lebnote{Or “battle”} \textit{had died}\lebnote{Literally “had finished/completed to die”} from among the people,
\verse Adonai spoke to me, \textit{saying},\lebnote{Literally “to say”}
\verse ‘You are about to cross over the boundary of Moab \textit{today}\lebnote{Literally “the day”} at Ar.
\verse When you approach \textit{the border of}\lebnote{Literally “opposite”} the \textit{Ammonites},\lebnote{Literally “the sons/children of Ammon”} you shall not harass them, and you shall not get involved in battle with them, for I have not given the land of \textit{the Ammonites}\lebnote{Literally “the sons/children of Ammon”} to you as a possession; because I have given it to the descendants\lebnote{Or “sons”} of Lot as a possession.
\verse (It is also considered the land of Rephaim; Rephaim lived in it \textit{previously},\lebnote{Literally “before”} and the Ammonites called them Zamzummim,
\verse a people great and numerous and as tall as the Anakites; Adonai destroyed them from before them,\lebnote{That is, the Ammonites} and they dispossessed them and settled in place of them,
\verse just as he did for the descendants\lebnote{Or “sons”} of Esau, who live in Seir, when he destroyed the Horites from \textit{before them}\lebnote{Literally “from the face/presence of them”} and dispossessed them, and then they settled in their\lebnote{Hebrew “them”} place up to this day.
\verse And also the Avvites, who lived in villages as far as Gaza, and the Caphtorim, who came out from Caphtor, destroyed them and then settled in their place.
\verse Arise,\lebnote{Or “Get up”} set out and cross over the wadi\lebnote{A valley that is dry most of the year, but contains a stream during the rainy season} of Arnon. Look! I have given Sihon the Amorite, the king of Heshbon, and his land into your hand; begin to take possession of it, and engage with him in battle.
\verse This day I will begin to place \textit{the dread of you}\lebnote{Literally “your dread”} and the fear of you \textit{before}\lebnote{Literally “upon the faces of”} the peoples under all the heavens. They will hear \textit{the report about you},\lebnote{Literally “your report”} and so they will shake and tremble \textit{because of you}.’ \lebnote{Literally “from before you”}
\verse “So I sent messengers from the wilderness\lebnote{Or “desert”} of Kedemoth to Sihon king of Heshbon; I sent terms of peace, \textit{saying},\lebnote{Literally “to say”}
\verse ‘Let me cross through your land and \textit{only along the road}\lebnote{Literally “in the road, in the road”} I will go; I will not turn aside to the right or to the left.
\verse Food for money you shall sell me, so that I may eat, and water for money you will give to me, so that I may drink; just let me cross on foot.
\verse Just as the descendants\lebnote{Or “the children/sons of”} of Esau did for me, who live in Seir, and the Moabites, who live in Ar, until I cross the Jordan into the land that Adonai our God is giving to us.’
\verse But Sihon king of Heshbon was not willing to let us cross through his territory because Adonai your God hardened his spirit and \textit{made him obstinate}\lebnote{Literally “made firm his heart”} \textit{in order to give him}\lebnote{Literally “to give him,” indicating purpose} into your hand, \textit{just as he has now done}.\lebnote{Literally “as it is this day”}
\verse Adonai said to me, ‘Look! I have begun to give \textit{over to you}\lebnote{Literally “to the face of you”} Sihon and his land; begin \textit{to take possession of his land}.’\lebnote{Literally, “take possession in order to possess his land”}
\verse Then Sihon and all his people came out to meet us for battle at Jahaz.
\verse And so Adonai our God gave him over \textit{to us},\lebnote{Literally “before us”} and we struck him down, and his sons\lebnote{Or “descendants”} and all of his people.
\verse So we captured all of his cities at that time, and we destroyed each town of males and the women and the children; we did not leave behind a survivor.
\verse We took only the livestock as spoil for ourselves,\lebnote{Hebrew “us”} and also the booty of the cities that we had captured.
\verse From Aroer, which is on the edge of the wadi\lebnote{A valley that is dry most of the year, but contains a stream during the rainy season} of Arnon and the city that was in the wadi\lebnote{A valley that is dry most of the year, but contains a stream during the rainy season} on up to Gilead, there was not a city that was inaccessible to us; Adonai our God gave \textit{everything}\lebnote{Literally “the whole”} \textit{to us}.\lebnote{Literally “before us”}
\verse Only the land of \textit{the Ammonites}\lebnote{Literally “children/sons of Ammon”} you did not approach, all along the whole upper region of the Jabbok \textit{River}\lebnote{Literally “wadi,” which here refers to a flowing river} and the towns of the hill country, according to all that Adonai our God had instructed.
\end{biblechapter}

\begin{biblechapter} % Deuteronomy 3
\verse “Then we turned, and we went up the road to Bashan, and Og the king of Bashan came out to meet us, he and all of his army for the battle at Edrei.
\verse And Adonai said to me, ‘You should not fear him, for I have given him and all of his army\lebnote{Or “people”} and his land into your hand. And so you will do to him as you did to Sihon the king of the Amorites,\lebnote{Hebrew “Amorite”} who was reigning in Heshbon.’
\verse And so Adonai our God also gave Og the king of Bashan, and all of his army\lebnote{Or “people”} into our hand, and we struck him down until not a survivor remained to\lebnote{Or “for”} him.
\verse And we captured all of his towns\lebnote{Or “villages/cities small”} at that time; there was not a city that we did not take from them.
\verse All of these were fortified towns with high walls, gates, and bars,\lebnote{Hebrew “bar”} \textit{apart from}\lebnote{Literally “alone from”} very many of the villages of the open country.
\verse And so we destroyed them just as we had done to Sihon the king of Heshbon; we destroyed utterly each town of males, the women, and the little children.
\verse But all of the livestock and the booty of the towns we kept as spoil for ourselves.\lebnote{Hebrew “us”}
\verse “And so we took at that time the land from \textit{the control of}\lebnote{Literally “the hand of”} the two kings of the Amorites\lebnote{Hebrew “Amorite”} who were \textit{on the other side of the Jordan},\lebnote{Literally “in the beyond of the Jordan”} from the wadi\lebnote{A valley that is dry most of the year, but contains a stream during the rainy season} of Arnon up to \textit{Mount Hermon}.\lebnote{Literally “the mountain of Hermon”}
\verse (The Sidonians called Hermon ‘Sirion,’ and the Amorites called it ‘Senir.’)
\verse All of the towns of the plateau and the whole of Gilead and all of Bashan up to Salecah and Edrei, the towns of the kingdom of Og in Bashan.
\verse (For only Og, king of Bashan, was left from the remnant of the Rephaim. Indeed, his bedstead—it was a bedstead of iron. It is in Rabbah of the \textit{Ammonites}.\lebnote{Literally “sons/children of Ammon”} Nine cubits is its length, and four cubits is its width according to the cubit of a man.)
\verse And so we took possession of this land at that time, from Aroer, which is on the edge of the wadi\lebnote{A valley that is dry most of the year, but contains a stream during the rainy season} of Arnon, and also half of the hill country of Gilead and its towns I gave to the Reubenites\lebnote{Hebrew “Reubenite”} and to the Gadites.\lebnote{Hebrew “Gadite”}
\verse And the remainder of Gilead and all of Bashan, the kingdom of Og, I gave to the half-tribe of Manasseh, the whole region of Argo. All of that area of Bashan was called the land of the Rephaim.
\verse Jair the descendant\lebnote{Or “son”} of Manasseh acquired the whole region of Argob, up to the boundary of the Geshurites and the Maacathites, and he called it,\lebnote{Hebrew “them”} that is Bashan, after his own name, Havvoth Jair, \textit{as it still is today}.\lebnote{Literally “up to this day”}
\verse And also I gave Gilead to Makir.
\verse And to the Reubenites\lebnote{Hebrew “Reubenite”} and to the Gadites\lebnote{Hebrew “Gadite”} I gave, from Gilead up to the wadi\lebnote{A valley that is dry most of the year, but contains a stream during the rainy season} of Arnon, the middle of the wadi\lebnote{A valley that is dry most of the year, but contains a stream during the rainy season} as a boundary and up to the Jabbok \textit{River},\lebnote{Literally “wadi,” which here refers to a flowing river} the boundary of the \textit{Ammonites}.\lebnote{Literally “sons/children of Ammon”}
\verse And the \textit{Jordan Valley}\lebnote{Literally “Arabah”} with the Jordan River as its boundary, from Kinnereth\lebnote{Or “Chinnereth,” the Sea of Galilee} up to the Sea of the Arabah, the Salt Sea,\lebnote{Or the Dead Sea} with the slopes of Pisgah toward the east.
\verse “And I charged you all at that time when I said, “Adonai has given you—to all of you—this land to possess. All the \textit{warriors}\lebnote{Literally “men of valor”} shall cross over, ready to fight, before your brothers, the \textit{Israelites}.\lebnote{Literally “sons/children of Israel”}
\verse Only your wives and your little children and your livestock (I know that \textit{you have much livestock})\lebnote{Literally “livestock much there is to you”} must stay in your towns that I have given you,
\verse until Adonai shall give rest to your brothers as he did to you, and also they take possession of the land that Adonai your God is giving to them beyond the Jordan; then they may return, each one to his possession that I have given to them.
\verse And I commanded Joshua at that time, saying, ‘Your eyes see all that Adonai your God has done to these two kings; so Adonai will do to all of the kingdoms where you are about to cross over to.
\verse You shall not fear them, for Adonai your God is the one fighting for you.
\verse “And I pleaded with Adonai at that time, saying,
\verse ‘Lord Adonai, you have begun to show your servant your greatness and your strong hand, for what\lebnote{Hebrew “who”} god is there in the heaven or on the earth who can do according to your works and according to your mighty deeds?
\verse Let me cross over, please, and let me see the good land \textit{that is beyond the Jordan},\lebnote{Literally “in the beyond of the Jordan”} this good hill country and Lebanon.’
\verse But Adonai was very angry with me because of you, and he would not listen to me, and Adonai said, ‘\textit{Enough of that from you}!\lebnote{Literally “Much to you”} You shall not speak to me any longer about this matter!
\verse Go up to the top of Pisgah and \textit{look around you}\lebnote{Literally “lift up your eyes”} toward the west, toward the north, and toward the east, and \textit{view}\lebnote{Literally “look”} the land with your eyes, for you will not cross this Jordan.
\verse Now instruct Joshua and support him and encourage him because he himself\lebnote{Emphatic use of the pronoun} will cross over before this people and enable them to inherit the land that you will see.’
\verse So we remained in the valley opposite Beth Peor.
\end{biblechapter}

\begin{biblechapter} % Deuteronomy 4
\verseWithHeading{Introduction to the Stipulations} “Now, Israel, listen to the rules and to the regulations that I am teaching you to do, in order that you may live and you may go in and you may take possession of the land that Adonai, the God of your ancestors,\lebnote{Or “fathers”} is giving to you.
\verse You must not add to the word that I am commanding you, and you shall not take away from it in order to keep\lebnote{Or “observe”} the commands of Adonai your God that I am commanding you to observe.
\verse Your eyes have seen\lebnote{Literally “your eyes the seeing what”} what Adonai did with\lebnote{Or “in”} the case of Baal Peor, for \textit{each}\lebnote{Literally “every one of”} man that followed after Baal Peor Adonai your God destroyed from your midst.
\verse But you, the ones holding fast to Adonai your God, are all alive \textit{today}.\lebnote{Literally “the day”}
\verse See, I now teach\lebnote{Or “I have taught” (compare NASB, NEB)} you rules and regulations just as Adonai my God has commanded me, to observe them just so in the midst of the land where you are going, to take possession of it.
\verse And \textit{you must observe them diligently},\lebnote{Literally “you must observe and you must do”} for that is your wisdom and your insight before the eyes of the people, who will hear all of these rules, and they will say, ‘Surely this great nation is a wise and discerning people.’
\verse For what great nation has for it a god near to it as Adonai our God, whenever\lebnote{Literally “in every matter/every time we call ”} we call upon him?
\verse And what other great nation has for it\lebnote{Literally “which/that is for him it”} just rules and regulations just like\lebnote{Literally “as”} \textit{this whole}\lebnote{Literally “all of”} law that I am setting \textit{before}\lebnote{Literally “to the face of”} you \textit{today}?\lebnote{Literally “the day”}
\verse “However, \textit{take care}\lebnote{Literally “watch”} for yourself and watch your inner self\lebnote{Or “soul”} closely, so that you do not forget the things that your eyes have seen, so that they do not slip from your mind all the days of your life; and you shall make them known to your children and to \textit{your grandchildren}.\lebnote{Literally “the children of your children”}
\verse Remember the day that you stood \textit{before}\lebnote{Literally “to the face of”} Adonai your God at Horeb \textit{when Adonai said to me},\lebnote{Literally “when to say Adonai”} ‘Summon for me the people so that I can tell them my words, that they may learn to fear me all of the days they are alive on the earth and so that they may teach their children.’
\verse And so you came near, and you stood under the mountain, and the mountain was burning with fire up to the heart of the heaven, \textit{dark with a very thick cloud}.\lebnote{Literally “darkness, cloud, and very thick darkness”}
\verse And Adonai spoke to you from the midst of the fire; \textit{you heard}\lebnote{Literally “you were hearing”} a sound of words, but \textit{you did not see}\lebnote{Literally “you were not seeing”} a form—only a voice.
\verse And he declared to you his covenant, \textit{the Ten Commandments},\lebnote{Literally “the ten words”} which he charged you \textit{to observe},\lebnote{Literally “to do”} and he wrote them on the two tablets of stone.
\verse And Adonai charged me at that time to teach you rules and regulations \textit{for your observation of them}\lebnote{Literally “for your doing them”} in the land that you are \textit{about to cross into}\lebnote{Literally “about to cross into there”} to take possession of it.
\verse “So \textit{you must be very careful for yourselves},\lebnote{Literally “watch yourselves diligently with respect to your souls”} because you did not see\lebnote{Or “have not seen”} any form on the day Adonai spoke to you at Horeb from the midst of the fire,
\verse so that you do not \textit{ruin yourselves}\lebnote{Literally “corrupt yourselves”} and make for yourselves a divine image in a form of any image, a replica of male or female,
\verse a replica of any animal that is upon the earth, a replica of any winged bird that flies in the air,
\verse a replica of any creeping thing on the ground, a replica of any fish that is in the water \textit{below}\lebnote{Literally “under”} the earth.
\verse \textit{And do this so that you do not lift}\lebnote{Literally “And lest you lift up”} your eyes toward heaven and \textit{observe}\lebnote{Literally “see”} the sun and the moon and the stars, all the host of the heaven, and be led astray and bow down to them and serve them, things that Adonai your God has allotted to all of the peoples under all of the heaven.
\verse But Adonai has taken you and brought you out from the furnace of iron, from Egypt, to be a people of inheritance to\lebnote{Or “for”; a people of inheritance = a people who is his very own possession} him, \textit{as it is this day}.\lebnote{Literally “as the day the this” = as you are now}
\verse “And Adonai was angry with me \textit{because of you},\lebnote{Literally “because of your matter”} and he swore \textit{that I would not cross the Jordan}\lebnote{Literally “not I to cross the Jordan”} and \textit{that I would not go to the good land}\lebnote{Literally “not I going to the land”} that Adonai your God is giving you as an inheritance.
\verse For I am going to die in this land; I am not going to cross the Jordan, but you are going to cross, and you are going to take possession of this good land.
\verse Watch out for yourselves so that you do not forget the covenant of Adonai your God that he had \textit{made}\lebnote{Literally “cut”} with you and make for yourselves a divine image of the form of anything that Adonai your God \textit{has forbidden},\lebnote{Literally “has commanded you about”}
\verse for Adonai your God is a devouring fire, a jealous\lebnote{Or “zealous” or “impassioned”} God.
\verse “When you have had children and \textit{grandchildren}\lebnote{Literally “children of children”} and you have grown old in the land and you act corruptly and you make a divine image of the form of anything and you do evil in the eyes of Adonai your God, thus provoking him to anger,
\verse I call\lebnote{Or “I shall call to witness”} to witness against you today the heaven and the earth, that you will perish soon and completely from the land that you are crossing the Jordan into it to take possession of it; \textit{you will not live long on it},\lebnote{Literally “you will not extend days”} but you will be completely destroyed.
\verse And Adonai will scatter you among the peoples, and you will be left \textit{few in number}\lebnote{Literally “people of number,” as opposed to people without number} among the nations \textit{to where Adonai will lead you}.\lebnote{Literally “he will lead Adonai you there”}
\verse And you will there serve gods \textit{made by human hands},\lebnote{Literally “the work of the hands of human”} of wood and stone, gods that cannot see and cannot hear and cannot eat and cannot smell.
\verse But from there you shall seek Adonai your God and will find him, if you seek him with all your heart and with all your soul.\lebnote{Or “inner self”}
\verse \textit{In your distress}\lebnote{Literally “In the distress for you”} when\lebnote{Hebrew “and”} all these things have found you in the \textit{latter days},\lebnote{Literally “in the last of the days”} then\lebnote{Hebrew “and”} you will return to Adonai your God, and you will listen to his voice.
\verse For Adonai your God is a compassionate God; he \textit{will not abandon you},\lebnote{Literally “will not leave you alone”} and he will not destroy you, and he will not forget the covenant of your ancestors\lebnote{Or “fathers”} that he swore to them.
\verse “Yes, ask, please, about former days that \textit{preceded you}\lebnote{Literally “that they were to the face of you”} from the day that God created humankind on the earth; ask even from one end of the heaven up to the other end of heaven \textit{whether anything ever happened}\lebnote{Literally “was it ever”} like this great thing or \textit{whether anything like it was ever heard}.\lebnote{Literally “was it ever heard as it”}
\verse Has a people ever heard the voice of God speaking from the midst of the fire, just as you heard it, and lived?
\verse Or has a god ever attempted to go to take for himself\lebnote{Hebrew “for him”} a nation from the midst of a nation, using trials and signs and wonders and war, with an outstretched arm and with great and awesome deeds, like all that Adonai your God did for you in Egypt before your eyes?
\verse You yourselves\lebnote{Emphatic use of pronoun; plural meaning implied} were shown this wonder in order for you to acknowledge that Adonai is the God;\lebnote{The definite article indicates that Israel’s God is alone the true God and the one who revealed himself to them} there is no other God \textit{besides him}.\lebnote{Literally “except him” or “to him alone”}
\verse From heaven he made you hear his voice to teach you, and on the earth he showed you his great fire, and you heard his words from the midst of the fire.
\verse And because he loved your ancestors\lebnote{Or “fathers”} he chose their \textit{descendants}\lebnote{Literally “seed”} after them. And he brought you forth from Egypt \textit{with his own presence},\lebnote{Literally “with his faces”} by his great strength,
\verse to drive out nations greater and more numerous than you \textit{from before you},\lebnote{Literally “from your face”} to bring you and to give to you their land as an inheritance, as it is this day.
\verse So you shall acknowledge \textit{today},\lebnote{Literally “the day”} and \textit{you must call to mind}\lebnote{Literally “you shall bring back to your heart”} that Adonai is God in heaven above and on the earth beneath. There is no other God.
\verse And you shall keep his rules and his commandments that I am commanding you \textit{today},\lebnote{Literally “the day”} so that \textit{it may go well}\lebnote{Literally “he/it is good”} for you and for your children\lebnote{Or “descendants”} after you, and so that \textit{you may remain a long time}\lebnote{Literally “you may make long/prolong days”} on the land that Adonai your God is giving to you during all of those days.”
\verse Then Moses set apart three cities \textit{on the other side of the Jordan},\lebnote{Literally “in the beyond of the Jordan”} \textit{toward the east},\lebnote{Literally “toward rising of the sun”}
\verse in order for \textit{a manslayer}\lebnote{Literally “a killer of a man”} to flee there who has killed his neighbor \textit{without intent}\lebnote{Literally “without previous knowledge”} and was not hating him \textit{previously},\lebnote{Literally “the day before yesterday”} and so he could flee to one of these cities \textit{and be safe}.\lebnote{Literally “and live”}
\verse He set apart Bezer in the wilderness\lebnote{Or “desert”} in the land of the plateau of the Reubenites;\lebnote{Hebrew “Reubenite”} Ramoth in Gilead of the Gadites,\lebnote{Hebrew “Gadite”} and Golan in Bashan of the Manassites.\lebnote{Hebrew “Manassite”}
\verse Now this is the law\lebnote{Hebrew “the \textit{torah}” = teaching, instruction, law} that Moses set \textit{before}\lebnote{Literally “before the faces of”} the \textit{Israelites};\lebnote{Literally “sons/children of Israel”}
\verse these are the legal provisions and the rules and the regulations that Moses spoke to the \textit{Israelites}\lebnote{Literally “sons/children of Israel”} \textit{when they left Egypt},\lebnote{Literally “at their going out from Egypt”}
\verse beyond the Jordan in the valley opposite Beth Peor in the land of Sihon the king of the Amorites,\lebnote{Hebrew “Amorite”} who was reigning in Heshbon and whom Moses and the \textit{Israelites}\lebnote{Literally “sons/children of Israel”} defeated \textit{when they came out of Egypt}.\lebnote{Literally “at their going out from Egypt”}
\verse And so they took possession of his land and the land of Og king of Bashan, the two kings of the Amorites\lebnote{Hebrew “Amorite”} who were beyond the Jordan, \textit{eastward},\lebnote{Literally “toward the rising of the sun”}
\verse from Aroer, which is on the bank of the wadi\lebnote{A valley that is dry most of the year, but contains a stream during the rainy season} of Arnon and as far as Mount Sirion; that is, Hermon,
\verse and all of the Arabah\lebnote{Or “valley” in this instance} beyond the Jordan, eastward, and as far as the Sea of the Arabah\lebnote{Commonly known today as the Dead Sea} under the slopes of Pisgah.
\end{biblechapter}

\begin{biblechapter} % Deuteronomy 5
\verseWithHeading{Basic Stipulations} And then Moses summoned all of Israel and said to them, “Hear, Israel, the rules and the regulations that I am speaking in your ears \textit{today},\lebnote{Literally “the day”} and you shall learn them, and \textit{you must observe them diligently}.\lebnote{Literally “you shall observe them carefully to do them”}
\verse Adonai our God made a covenant with us at Horeb.
\verse It was not with our ancestors\lebnote{Or “fathers”} that Adonai made this covenant, but with these of us who are here alive today.
\verse \textit{Face to face}\lebnote{Literally “Faces to faces”} Adonai spoke with you at the mountain from the midst of the fire.
\verse I was standing \textit{between Adonai and you}\lebnote{Literally “between Adonai and between you”} at that time to report\lebnote{Or “declare”} to you the word of Adonai, for you were afraid because of \textit{the presence of}\lebnote{Literally “the faces of”} the fire, and so you did not go up the mountain. \textit{He said},\lebnote{Literally “To say”}
\verse ‘I am Adonai your God, who brought you out from the land of Egypt, from the house of slavery.
\verse There shall not be for you other gods \textit{besides me}.\lebnote{Literally “before my face”}
\verse ‘You shall not make for yourself a divine image of any type of form that is in the heaven above or that is on the earth beneath or that is in the water under the earth.
\verse ‘You shall not bow down to them, and you shall not serve them, for I, Adonai your God, am a jealous God, punishing the guilt of fathers upon their children and upon the third and upon the fourth generation of those hating me,
\verse but showing loyal love to thousands of those who love me and of those who keep my commandments.
\verse ‘You shall not take up the name of Adonai your God for a worthless purpose, for Adonai will not leave unpunished anyone who uses his name for a worthless purpose.
\verse ‘Observe the \textit{Sabbath day}\lebnote{Literally “the day of the Sabbath”} to make it holy,\lebnote{Or “to sanctify it”} just as Adonai your God has commanded you.
\verse Six days you shall work, and you shall do all of your work,
\verse but the seventh day is a Sabbath unto Adonai your God; you shall not do any work, or your son, or your daughter, or your slave, or your slave woman, or your ox, or your donkey, or any of your domestic animals, or your resident alien who is in your \textit{towns},\lebnote{Literally “gates”} so that your slave and your slave woman may rest as you rest.
\verse And you shall remember that you were a slave in the land of Egypt, and Adonai your God brought you out with a strong hand and with an outstretched arm; therefore, Adonai your God commanded you to keep \textit{the Sabbath}.\lebnote{Literally “the day of the Sabbath”}
\verse ‘Honor your father and your mother, as Adonai your God commanded you, so that it will be good for you\lebnote{Or “it may go well for you”} \textit{in the land}\lebnote{Literally “on the ground”} that Adonai your God is giving to you.
\verse ‘You shall not murder.
\verse ‘And you shall not commit adultery.
\verse ‘And you shall not steal.
\verse ‘And you shall not falsely bear evidence against your neighbor.
\verse ‘And you shall not covet the wife of your neighbor, and you shall not crave the house of your neighbor, his field or his slave or his slave woman or his ox and his donkey or anything \textit{that belongs to your neighbor}.’\lebnote{Literally “that is for your neighbor”}
\verse “These words Adonai spoke to your whole assembly at the mountain from the midst of the fire and the very thick cloud with a loud voice, and \textit{he did not add anything},\lebnote{Literally “and not he added”} and then he wrote them on two tablets of stone and gave them to me.
\verse \textit{And then}\lebnote{Literally “And it happened”} \textit{when you heard}\lebnote{Literally “and as/when you heard” or “at the moment of your hearing”} the voice from the midst of the darkness, and as the mountain was burning with fire, and and all the heads of your tribes and your elders approached me,
\verse you said, ‘Look, Adonai our God has shown us his glory and his greatness, and we have heard his voice from the midst of the fire; this day we have seen\lebnote{Or “saw”} that God can speak with a human being, \textit{yet he remains alive}.\lebnote{Literally “and he lives”}
\verse And so then why shall we die, for this great fire will consume us if \textit{we continue}\lebnote{Literally “we do again to hear”; or “we are doing again to hear”} to hear the voice of Adonai our God any longer, and so we shall die?
\verse For who is there of all flesh who has heard the voice of the living God speaking from the midst of the fire as we have heard it \textit{and remained alive}?\lebnote{Literally “and lives”}
\verse You go near and hear \textit{everything}\lebnote{Literally “all of that which”} that Adonai our God will say; and then you tell us all that Adonai our God tells you, and we will listen, and we will do it.’
\verse “And Adonai heard the sound of your words \textit{when you spoke to me},\lebnote{Literally “in/at you to speak to me”} and Adonai said to me, ‘I have heard the sound of the words of this people that they have spoken to you; they are right with respect to all that they have spoken.
\verse \textit{If only}\lebnote{Literally “Who gives/will give”} \textit{they had such a mind}’;\lebnote{Literally “it would be their heart this to them”} that is, to fear me and to keep all my commandments \textit{at all times},\lebnote{Literally “all the days”} so that \textit{it will go well}\lebnote{Literally “he/it is good”} for them and for their children \textit{forever}.\lebnote{Literally “to eternity”}
\verse Go! Say to them, “Return to your tents.”
\verse But you stand here with me, and let me tell you all of the commandments and the rules and the regulations that you shall teach them, so that they may do them in the land that I am giving to them to take possession of it.’
\verse “So you must be careful to do just as Adonai your God commanded you; you shall not turn to the right or to the left.
\verse \textit{In exactly the path}\lebnote{Literally “In all the way/path”} that Adonai your God has commanded, you must go, so that you may live and \textit{it will go well}\lebnote{Literally “and good it is”} for you and \textit{you may live long}\lebnote{Literally “you may make long”} in the land that you will take possession of.”
\end{biblechapter}

\begin{biblechapter} % Deuteronomy 6
\verseWithHeading{Detailed Stipulations} “Now this is the commandment, the rules and the regulations, that Adonai your God charged to teach to you for you \textit{to observe}\lebnote{Literally “to do”} in the land that you are about to cross over into\lebnote{Or “there”} to take possession of it,
\verse so that you may revere Adonai your God by keeping all his statutes and his commandments that I am commanding you, you and your children and \textit{grandchildren},\lebnote{Literally “the children of your children”} all the days of your life and so you may live long\lebnote{Literally “they may be long your days”} lives.
\verse And you shall hear, Israel, and be careful to observe these instructions, so that \textit{it may go well for you}\lebnote{Literally “it/he may be good”} and that you may multiply greatly, just as Adonai, the God of your ancestors,\lebnote{Or “fathers”} \textit{promised}\lebnote{Literally “spoke”} you, in a land with milk and honey.
\verse “Hear, Israel, Adonai our God, Adonai is unique.\lebnote{Or “one”; or possibly “one Lord”}
\verse And you shall love Adonai your God with all of your heart\lebnote{Or “mind”} and with all of your soul\lebnote{Or “inner self”} and with all of your might.
\verse And these words that I am commanding you \textit{today}\lebnote{Literally “the day”} shall be on\lebnote{Or “in”} your heart.\lebnote{Or “mind”}
\verse And you shall recite them to your children, and you shall talk about them at the time of your living in your house and at the time of your going on the road and at the time of your lying down and at the time of your rising up.
\verse And you shall bind them as a sign on your hand, and they shall be as an emblem between your eyes.
\verse And you shall write them on the doorframe of your house and on your gates.
\verse “And then it will happen that when Adonai your God will bring you to the land that he swore to your ancestors,\lebnote{Or “fathers”} to Abraham, to Isaac, and to Jacob, to give to you large and fine cities that you did not build,
\verse and houses full of all sorts of good things\lebnote{Hebrew “thing”} that you did not fill, and hewn cisterns that you did not hew, vineyards and olive groves that you did not plant, and \textit{you have eaten your fill},\lebnote{Literally “you have eaten and you are satisfied”}
\verse then take care for yourself, so that you do not forget Adonai, who brought you out from the land of Egypt from the house of slavery.
\verse “You shall fear Adonai your God, and you shall serve him, and by his name you shall swear.
\verse You shall not go after other gods from the gods of the peoples who are all around you,
\verse for Adonai your God is a jealous god in your midst, \textit{so that the anger of Adonai your God would be kindled},\lebnote{Literally “so that not/lest the nose of Adonai your God would become hot”} and he would destroy you from the face of the earth.
\verse You shall not put Adonai your God to the test, as you tested him at Massah.
\verse You shall diligently keep the commandments of Adonai your God and his legal provisions and his rules that he has commanded you.
\verse And you shall do what is right and good in the eyes of Adonai, so that \textit{it shall go well}\lebnote{Literally “he/it shall be good”} for you and so that you may go and you may take possession of the good land that Adonai swore for your ancestors,\lebnote{Or “your fathers”}
\verse by driving out all of your enemies \textit{before you},\lebnote{Literally “the face of you”} just as Adonai \textit{has promised}.\lebnote{Literally “had spoken”}
\verse “When your child\lebnote{Or “son”} asks you in the future, \textit{saying},\lebnote{Literally “to say”} ‘What is the meaning of the legal provisions and the rules and the regulations that Adonai our God commanded for you?’
\verse Then you shall say to your child,\lebnote{Or “son”} ‘We were slaves of Pharaoh in Egypt, and Adonai brought us out from Egypt with a strong hand.
\verse And Adonai gave great and awesome signs and wonders in Egypt against Pharaoh and against his entire household \textit{in our presence}.\lebnote{Literally “before our eyes”}
\verse But he brought us out from there in order to bring us here to give us the land that he swore to our ancestors.\lebnote{Or “fathers”}
\verse And so Adonai commanded us to observe all of these rules and to revere Adonai our God \textit{for our benefit}\lebnote{Literally “for good for us”} \textit{all the days that we live},\lebnote{Literally “all the days for our living”} \textit{as it is today}.\lebnote{Literally “as the day the this”}
\verse And it shall be righteousness for us if we diligently observe and do all of this commandment \textit{before}\lebnote{Literally “in the faces of”} Adonai our God, as he has commanded us.’
\end{biblechapter}

\begin{biblechapter} % Deuteronomy 7
\verse “When Adonai your God brings you into the land that you are about to enter \textit{into it}\lebnote{Literally “there”} to take possession of it, and he drives\lebnote{Or “he will drive out”} out many nations \textit{before you},\lebnote{Literally “from your faces”} the Hittites\lebnote{Hebrew “Hittite”} and the Girgashites\lebnote{Hebrew “Girgashite”} and the Amorites\lebnote{Hebrew “Amorite”} and the Canaanites\lebnote{Hebrew “Canaanite”} and the Hivites\lebnote{Hebrew “Hivite”} and the Jebusites,\lebnote{Hebrew “Jebusite”} seven nations mightier and more numerous than you,
\verse and Adonai your God will give them \textit{over to you}\lebnote{Literally “to your faces” or “before you”} and you defeat them, you must \textit{utterly destroy them};\lebnote{Literally “set them aside for destruction”} you shall not make a covenant with them, and you shall not show mercy to them.
\verse And you shall not intermarry with them; you shall not give your daughter to their son; and you shall not take his daughter for your son.\lebnote{Grammatically singular pronominal suffixes here can also be rendered “their”}
\verse For their\lebnote{Or “his”} sons and daughters will cause your son\lebnote{Or “sons”} to turn away \textit{from following me},\lebnote{Literally from “behind/after me”} and so they will serve other gods, and \textit{the anger of Adonai would be kindled}\lebnote{Literally “and it would become hot the nose of Adonai”} against you, and he would quickly destroy you.
\verse But this is what you must do to them: you shall break down their altars, and their stone pillars you shall smash, and their Asherah poles you shall hew down, and you shall burn their idols with fire.
\verse For you are a holy people for Adonai your God; Adonai your God has chosen you to be for him a people, a treasured possession from among all the peoples that are on the face\lebnote{Or “surface”} of the earth.
\verse “Adonai loved you and chose you not \textit{because of your great number}\lebnote{Literally “from your multitude/abundance”} exceeding all other peoples, for you are fewer than all of the peoples,
\verse but\lebnote{Or “for”} \textit{because of}\lebnote{Literally “from”} the love of Adonai for you and because of his keeping of the sworn oath that he swore to your ancestors,\lebnote{Or “fathers”} Adonai brought you out with a strong hand and redeemed you from the house of slavery, from the hand of Pharaoh, the king of Egypt.
\verse So know that Adonai your God, he is God, the trustworthy God, maintaining his\lebnote{Hebrew “the”} covenant and his\lebnote{Hebrew “the”} loyal love with those who love him and with those who keep his commandments to a thousand generations,\lebnote{Hebrew “generation”}
\verse but repaying those\lebnote{Hebrew “the one”} who hate him \textit{in their own person}\lebnote{Literally “to their faces”} to destroy them;\lebnote{Hebrew “him”; “them” in sense} he is not slow with those who hate him \textit{in their own person};\lebnote{Literally “to their faces”} he repays them.\lebnote{Hebrew “him”}
\verse And so you shall keep the commandment and the rules and the regulations that I am commanding you \textit{today}\lebnote{Literally “the day”} to observe them.
\verse “\textit{And then}\lebnote{Literally “And it will happen”} because you listen to these regulations and you diligently keep and you do them, then Adonai your God will maintain his\lebnote{Hebrew “the”} covenant and his\lebnote{Or “fathers”} loyal love that he swore to your ancestors.\lebnote{Or “fathers”}
\verse And he will love you, and he will bless you, and he will multiply you, and he will bless the fruit of your womb and the fruit of your soil, your grain, your wine, and your olive oil, and newborn calves of your cattle, and the newborn lambs\lebnote{Or “increase of”} of your flocks in the land that he swore to your ancestors\lebnote{Or “fathers”} to give you.
\verse You shall be blessed more than all of the peoples; among you there shall not be sterility and bareness, even\lebnote{Hebrew “and”} among domestic animals.\lebnote{Hebrew “animal”}
\verse And Adonai will turn away from you all the illness and all the harmful diseases of Egypt that you experienced; he will not lay them on you, but he will lay them on all of those who hate you.
\verse And you shall devour all of the peoples that Adonai your God is giving to you; \textit{you shall not pity them},\lebnote{Literally “your eye shall not take pity on them”} and you shall not serve their gods, which will be a snare for you.
\verse “If you think in your heart, ‘These nations are more numerous than I, so how can I dispossess them?’
\verse then remember you must not be afraid of them; you must well remember what Adonai your God did to Pharaoh and to all of Egypt:
\verse the great trials that your eyes saw and the signs and the wonders and the workings of the strong hand and the outstretched arm by which Adonai your God brought you out; so Adonai your God will do to all of the peoples \textit{because of whom}\lebnote{Literally “from them”} you are in fear \textit{before them}.\lebnote{Literally “from their presence”}
\verse And, moreover, Adonai your God will send the hornets\lebnote{Hebrew “hornet”} among them until both the survivors and the fugitives \textit{are destroyed}\lebnote{Literally “the destroying of”} \textit{before you}.\lebnote{Literally “from your faces”}
\verse You must not be in dread from the presence of them, because Adonai your God, who is in your midst, is a great and awesome God.
\verse And Adonai your God will clear away these nations \textit{from before you}\lebnote{Literally “from your faces”} little by little; you will not be able to finish them off quickly, \textit{lest}\lebnote{Literally “so that not”} the \textit{wild animals}\lebnote{Literally “animals of the field”} \textit{multiply}\lebnote{Literally “will become numerous”} \textit{against you}.\lebnote{Literally “on you”}
\verse But Adonai your God will \textit{give them to you},\lebnote{Literally “give them to your face”} and he will throw them into great panic \textit{until they are destroyed}.\lebnote{Literally “until their destroying”}
\verse And he will give their kings into your hand, and you shall blot out their names\lebnote{Hebrew “name”} from under the heaven; anyone will not be able to stand \textit{against you}\lebnote{Literally “in your faces”} \textit{until you destroy them}.\lebnote{Literally “until your destroying them”}
\verse You shall burn the images of their gods with fire; you shall not covet the silver or gold that is on them, and so you take it for yourself, so that you are not ensnared by it, for it is a detestable thing to Adonai your God.
\verse And you must not bring a detestable thing into your house, or you will become a thing devoted to destruction like it; you must utterly detest it, and you must utterly abhor it, for it is an object devoted to destruction.
\end{biblechapter}

\begin{biblechapter} % Deuteronomy 8
\verse “All of the commandments\lebnote{Or “every commandment”} that I am commanding you today you must diligently observe, so that you may live and multiply, and you may go and take possession of the land that Adonai swore to your ancestors.\lebnote{Or “your fathers”}
\verse And you shall remember all of the way that Adonai your God led you during these forty years in the desert in order to humble you and to test you to know what is in your heart, whether you would diligently keep his commandments or not.
\verse And he humbled you and let you go hungry, and then he fed you with that which you did not know nor did your ancestors\lebnote{Or “fathers”} know, in order to make you know that not by bread alone but by all that goes out\lebnote{Or “by all of the going out of”} of the mouth of Adonai humankind shall live.
\verse Your clothing did not wear out \textit{on you},\lebnote{Literally “from on you”} and your feet did not swell during these forty years.
\verse And you should know with your heart that as a man disciplines his son, so Adonai your God is disciplining you.\lebnote{Or “has disciplined you”}
\verse So you must keep the commandments of Adonai your God by walking in his ways and by fearing him.
\verse For Adonai your God is bringing you to a good land with streams of water, springs and underground water, welling up in the valleys\lebnote{Hebrew “valley”} and in the hills,\lebnote{Hebrew “hill”}
\verse to a land of wheat and barley and vines\lebnote{Hebrew “vine”} and fig trees\lebnote{Hebrew “fig tree”} and pomegranate trees,\lebnote{Hebrew “pomegranate tree”} a land of olive trees,\lebnote{Hebrew “olive tree”} olive oil and honey;
\verse to a land where you may eat food in it \textit{without scarcity};\lebnote{Literally “not in scarcity”} you will not find anything lacking in it, a land where its stones are iron and from its mountains you can mine copper.
\verse And you will eat, and \textit{you will be satisfied},\lebnote{Literally “you will eat your fill”} and you will bless Adonai your God because of the good land that he has given to you.
\verse “Take care for yourself so that you not forget Adonai your God by not keeping his commandments and his regulations and his statutes that I am commanding you \textit{today},\lebnote{Literally “the day”}
\verse lest when you have eaten and you are satisfied and you have built good houses and you live in them,
\verse and your herds and your flocks have multiplied, and \textit{you have accumulated silver and gold},\lebnote{Literally “and silver and gold has multiplied for you”} and all that \textit{you have}\lebnote{Literally “is for you”} has multiplied,
\verse then your heart \textit{becomes proud}\lebnote{Literally “raises up”} and you forget Adonai your God, \textit{the one who brought you out}\lebnote{Literally “the one bringing you out”} from the land of Egypt, from the house of slavery,
\verse the one leading you in the great and terrible desert infested with \textit{dangerous}\lebnote{Literally “burning”} snakes\lebnote{Hebrew “snake”} and scorpions\lebnote{Hebrew “scorpion”} and parched ground, where there is\lebnote{Or “was”} no water, and the one bringing out water for you from flint rock,
\verse the one \textit{feeding you}\lebnote{Literally “causing you to eat”} manna in the desert, food that your ancestors\lebnote{Or “fathers”} did not know, in order to humble you and in order to test you so that he could do good to you \textit{in the future}.\lebnote{Literally “in the end/later”}
\verse And you may think\lebnote{Or “say”} in your heart, ‘My strength and the might of my hand \textit{acquired this wealth for}\lebnote{Literally “gave this wealth to”} me.’
\verse But you must remember Adonai your God, for he is the one giving you strength to acquire wealth in order to confirm his covenant that he swore to your ancestors\lebnote{Or “fathers”} \textit{as it is today}.\lebnote{Literally “as the day the this”}
\verse And it will happen that if you indeed forget Adonai your God and you go after other gods and you serve them and you bow down before them, I warn you today that you will surely perish.
\verse As with the nations that Adonai is destroying \textit{before you},\lebnote{Literally “from your faces”} so you shall perish because you would not obey the voice of Adonai your God.
\end{biblechapter}

\begin{biblechapter} % Deuteronomy 9
\verse “Hear, Israel, you are about to cross the Jordan today to go to dispossess nations larger and more numerous than you, great cities fortified \textit{with high walls},\lebnote{Literally “in heaven/the sky”}
\verse a great and tall people, the sons of the Anakites, whom you know and of whom you have heard it said, ‘Who could stand before the sons of Anak?’
\verse You should know \textit{today}\lebnote{Literally “the day”} that Adonai your God is the one crossing \textit{ahead of you}\lebnote{Literally “to your faces”} as a devouring fire; he will destroy them, and he will subdue them before you; so you will dispossess them, and you will destroy them quickly, \textit{just as}\lebnote{Literally “as that”} Adonai \textit{promised}\lebnote{Literally “spoke”} you.
\verse “You shall not say \textit{to yourself}\lebnote{Literally “in your heart”} \textit{when Adonai your God is driving them out}\lebnote{Literally “at the driving out of Adonai, your God, them”} \textit{before you},\lebnote{Literally “from before your faces”} \textit{saying},\lebnote{Literally “to say”} ‘Because of my righteousness Adonai brought me to take possession of this land’; but because of the wickedness of these nations Adonai is driving them out \textit{before you}.\lebnote{Literally “from your faces”}
\verse It is not because of your righteousness and because of the uprightness of your heart that you are coming to take possession of their land, but because of the wickedness of these nations Adonai your God is driving them \textit{before you},\lebnote{Literally “from your faces”} and in order to confirm the \textit{promise}\lebnote{Literally “word”} that Adonai swore to your ancestors,\lebnote{Or “fathers”} to Abraham, to Isaac, and to Jacob.
\verse “So you should understand that it is not because of your righteousness that Adonai your God is giving you this good land to take possession of it, because \textit{you are a stubborn people}. \lebnote{Literally “a people stiff of neck you”}
\verse Remember, \textit{do not forget},\lebnote{Literally “not you shall forget”} that you provoked Adonai your God in the desert, and from the day that you went out from the land of Egypt until \textit{you came to this place}\lebnote{Literally “your coming up to this place the this” = until arrived at this location} you were rebelling against Adonai.
\verse “And remember at Horeb you provoked Adonai, and Adonai became angry enough to destroy you.
\verse \textit{When I went up the mountain}\lebnote{Literally “In/At my going up the mountain”} to receive \textit{the stone tablets},\lebnote{Literally “the tablets of stones”} the tablets of the covenant that Adonai \textit{made}\lebnote{Literally “cut”} with you, and remained on the mountain forty days\lebnote{Hebrew “day”} and forty nights,\lebnote{Hebrew “night”} I did not eat food and I did not drink water.
\verse And Adonai gave me the two tablets of stone written with the finger of God, and on them was writing according to all the words that Adonai spoke with you at the mountain, from the midst of the fire on the day of the assembly.
\verse \textit{And then}\lebnote{Literally “And it happened”} at the end of forty days\lebnote{Hebrew “day”} and forty nights,\lebnote{Hebrew “night”} Adonai gave me the two tablets of stone, the tablets of the covenant.
\verse And Adonai said to me, ‘Come now, go down quickly from this mountain because your people behave corruptly whom you brought out from Egypt, for they turned quickly from the way that I commanded them to follow; they have made for themselves a cast image.’
\verse And Adonai spoke to me, \textit{saying},\lebnote{Literally “to say”} ‘I have seen this people, and look! \textit{They are a stubborn people}.\lebnote{Literally “It is a people stiff of neck”}
\verse Leave me alone, and let me destroy them, and let me blot out their name from under heaven, and let me make you into a nation mightier and more numerous than they!’
\verse “And I turned, and I went down the mountain, as the mountain was burning with fire, and the two tablets of the covenant were in my two hands.
\verse And I looked,\lebnote{Or “saw”} and indeed you had sinned against Adonai your God; you had made for yourselves an image of a calf of cast metal; \lebnote{Or “molten calf”} you had turned quickly from the way that Adonai had commanded for you.
\verse And I took hold of the two tablets, and I threw them out \textit{of}\lebnote{Literally “from”} my two hands and smashed them before your eyes.
\verse And then I lay prostrate \textit{before}\lebnote{Literally “to the face of”} Adonai, as earlier, forty days\lebnote{Hebrew “day”} and forty nights;\lebnote{Hebrew “night”} I did not eat food and I did not drink water because of all your sins that you committed, by doing evil in the eyes of Adonai and so provoking him.
\verse For \textit{I was in dread}\lebnote{Literally “I dreaded”} from being in the presence of the anger and the wrath with which Adonai was angry with you so as to destroy you, but Adonai listened to me also \textit{at that time}.\lebnote{Literally “the occurrence the that”}
\verse And with Aaron Adonai was \textit{angry enough}\lebnote{Literally “very angry”} to destroy him, and I prayed also for Aaron at that time.
\verse And your sinful thing that you had made, the molten calf, I took and I burned it with fire, and I crushed it, grinding it thoroughly until it was crushed to dust, and I threw its dust into \textit{the stream that flowed down the mountain}.\lebnote{Literally “the stream the one going down from the mountain”}
\verse “And also at Taberah and at Massah and at Kibroth Hattaavah you provoked Adonai to anger.
\verse And when Adonai sent you out from Kadesh Barnea, \textit{saying},\lebnote{Literally “to say”} ‘Go up and take possession of the land that I have given you,’ you rebelled against the command of Adonai your God, and you did not believe him, and you did not listen to his voice.
\verse You have been rebellious toward Adonai \textit{from the day I have known you}.\lebnote{Literally “from the day of my knowing you”}
\verse “And I lay prostrate before Adonai through forty days,\lebnote{Hebrew “day”} and through forty nights\lebnote{Hebrew “night”} I prostrated myself, because Adonai intended to kill you.
\verse And I prayed to Adonai, and I said, ‘Lord Adonai, you must not destroy your people and your inheritance whom you redeemed in your greatness,\lebnote{Or “your great act”} whom you brought out from Egypt with a strong hand.
\verse Remember your servants, Abraham, Isaac, and Jacob; you must not \textit{pay attention to}\lebnote{Literally “turn toward”} the stubbornness of this people, to their wickedness and to their sin,
\verse lest the people of the land from which you brought us out from there say, “Because Adonai was not able to bring them to the land that he \textit{promised}\lebnote{Literally “spoken”} to them and because of his hatred toward them, he has brought them out to kill them in the desert.”
\verse For they are your people and your inheritance whom you brought with your great power and with your outstretched arm.’
\end{biblechapter}

\begin{biblechapter} % Deuteronomy 10
\verse “At that time Adonai said to me, ‘Carve for yourself two tablets of stone just as the former ones, and come up the mountain to me, and you shall make for yourself an ark of wood.
\verse And I will write on the tablets the words that were on the former tablets, which you smashed, and you must put them in the ark.’
\verse And so I made an ark \textit{of acacia wood},\lebnote{Literally “of wood of acacia trees”} and I carved two tablets of stone like the former ones, and I went up the mountain with the two tablets in my hand.
\verse And he wrote upon the tablets \textit{according to the first writing},\lebnote{Literally “as the writing the first”} the ten words that Adonai spoke to you on the mountain from the midst of the fire on the day of the assembly, and Adonai gave them to me.
\verse And I turned, and I came down from the mountain, and I put the tablets in the ark that I had made, and they are there, just as Adonai commanded me.
\verse “And the \textit{Israelites}\lebnote{Literally “sons/children of Israel”} journeyed from the wells of Bene-Yaqan to Moserah; there Aaron died and was buried, and Eleazar, his son, served as a priest in place of him.
\verse From there they journeyed to Gudgodah, and from Gudgodah to Jotbathah, a land flowing with streams of water.
\verse At that time Adonai set apart the tribe of Levi to carry the ark of the covenant of Adonai to stand \textit{before}\lebnote{Literally “the faces of”} Adonai, to serve him and to bless the people in his name until this day.
\verse Therefore \textit{there was not}\lebnote{Literally “it/he was not”} for Levi an allotment or an inheritance along with his brothers; rather Adonai is his inheritance just as Adonai your God \textit{promised}\lebnote{Literally “spoke”} to him.
\verse And I stayed on the mountain just as during the former forty days\lebnote{Hebrew “day”} and forty nights,\lebnote{Hebrew “night”} and Adonai listened to me also on that occasion;\lebnote{Or “occurrence” or “that time”} Adonai was not willing to destroy you.
\verse And Adonai said to me, ‘\textit{Come, continue}\lebnote{Literally “Arise, go”} your journey \textit{before the people}, \lebnote{Literally “in the faces of the people”} so that you may go and take possession of the land that I swore to their ancestors\lebnote{Or “fathers”} to give to them.’
\verse And now, Israel, what is Adonai your God asking\lebnote{Or “requiring”} from you, except\lebnote{Or “but only”} to revere Adonai your God, to go\lebnote{Or “to walk”} in all his ways and to love him and to serve Adonai your God with all your heart and with all your soul,\lebnote{Or “inner self”}
\verse to keep the commandments of Adonai and his statutes that I am commanding you today \textit{for your own good}.\lebnote{Literally “for good to you”}
\verse Look! For to Adonai your God \textit{belong}\lebnote{Literally “is”} heaven and the \textit{highest heavens},\lebnote{Literally “heavens of the heavens”} the earth and all that is in it.
\verse \textit{Yet}\lebnote{Literally “Only”} to your ancestors\lebnote{Or “fathers”} Adonai was very attached, so as to love them, and so he chose their offspring after them, namely you, from all the peoples, as it is \textit{today}.\lebnote{Literally “the day the this”}
\verse So you shall circumcise the foreskin of your heart, and \textit{do not be stubborn}.\lebnote{Literally “your neck you shall not make stiff any longer”}
\verse For Adonai your God, he is God of the gods and Lord of the lords, the great and mighty God, the awesome one who \textit{is not partial},\lebnote{Literally “does not lift up faces”} and he does not take bribes.
\verse And he executes justice for the orphan and widow, and he is one who loves the alien, to give to them food and clothing.
\verse And you shall love the alien, for you were aliens in the land of Egypt.
\verse Adonai your God, you shall revere him, you shall serve him, and to him you shall cling, and by his name you shall swear.
\verse He is your praise, and he is your God, who has done with you these great and awesome things that your eyes have seen.
\verse With only seventy persons your ancestors\lebnote{Or “fathers”} went down to Egypt, but now Adonai your God has made you as the stars of the heaven \textit{with respect to multitude}.\lebnote{Literally “as far as number”}
\end{biblechapter}

\begin{biblechapter} % Deuteronomy 11
\verse “And you shall love Adonai your God, and you shall keep his obligations and his statutes and his regulations and his commandments \textit{always}.\lebnote{Literally “all of the days”}
\verse And you shall realize \textit{today}\lebnote{Literally “the day”} that it is not with your children who have not known and who have not seen the discipline of Adonai your God—his greatness, his strong hand, and his outstretched arm,
\verse and his signs and his deeds that he did in the midst of Egypt to Pharaoh, the king of Egypt, and to all of his land,
\verse and what he did to the army of Egypt and to their horses and to their chariots, and how he made the water of the \textit{Red Sea}\lebnote{Literally “sea of reed”} flow over them, \textit{when they pursued after them},\lebnote{Literally “in/at their pursuing after them”} and so Adonai has destroyed them, \textit{as is the case today},\lebnote{Literally “until the day the this”}
\verse and what he did to you in the desert until you came up to this place,
\verse and what he did to Dathan and to Abiram, the sons of Eliab, the son of Reuben, how the earth opened its mouth wide and swallowed them, their households and their tents, and all of the living creatures\lebnote{Hebrew “creature”} that were in their possession and that were \textit{following along with them}\lebnote{Literally “in their feet” = “living things traveling along with them”} in the midst of all of Israel.
\verse The fact of the matter is, \textit{your own eyes have seen}\lebnote{Literally “your eyes that are seeing”} all of the great deeds\lebnote{Hebrew “deed”} of Adonai that he did.
\verse “And you must keep all of the commandments\lebnote{Hebrew “commandment”} that I am commanding you \textit{today},\lebnote{Literally “the day”} so that you may have strength and you may go and you may take possession of that land \textit{to which you are crossing}\lebnote{Literally “you are crossing there”} to take possession of it,
\verse so that \textit{you may live long}\lebnote{Literally “you may make long days”} on the land that Adonai swore to your ancestors,\lebnote{Or “fathers”} to give it to them and to their offspring, a land flowing with milk and honey.
\verse For the fact is that the land \textit{that you are about to go into}\lebnote{Literally “you are about to go there”} to take possession of it is not like the land of Egypt, from which you have \textit{come out of},\lebnote{Literally “come out from there”} where you sow your seed and you give water by your foot, \textit{as in a vegetable garden}.\lebnote{Literally “as the garden of the vegetables”}
\verse But the land that you are about to cross into to take possession of it is a land of hills and valleys, and by the rain of the heaven it drinks water,
\verse a land that Adonai your God is caring for it; continually the eyes of Adonai your God are on it, from the beginning of the year up to the end of the year.
\verse “And it will happen that if you listen carefully to my commandments that I am commanding you \textit{today},\lebnote{Literally “the day”} to love Adonai your God, and to serve him with all of your heart and with all of your soul,\lebnote{Or “inner self”}
\verse then ‘I will send the rain for your land in its season, early rain and later rain, and you will gather your grain and your wine and your olive oil.
\verse And I will give vegetation in your field for your livestock, and \textit{you will eat your fill}.’\lebnote{Literally “you will eat and you will be satisfied”}
\verse \textit{Take care}\lebnote{Literally “Watch for yourself”} so that your heart is not easily deceived, and you turn away, and you serve other gods, and you bow down to them.
\verse And then \textit{the anger of Adonai will be kindled against you},\lebnote{Literally “the nose of Adonai become hot against you”} and he will shut up the heavens,\lebnote{Or “sky”} and there shall not be rain, and so the ground will not give its produce, and you will perish quickly from the good land that Adonai is giving to you.
\verse “And you shall put these, my words, on your heart and on your inner self,\lebnote{Or “soul”} and you shall bind them as a sign on your hand and let them be as an emblem between your eyes.
\verse And you shall teach them to your children\lebnote{Or “sons”} by talking about them when you sit in your house and when you travel on the road and when you lie down and when you get up.
\verse And you shall write them on the doorframes of your house and on your gates,
\verse so that they may multiply your days and the days of your children on the land that Adonai swore to your ancestors\lebnote{Or “fathers”} to give it to them \textit{as long as heaven endures over the earth}.\lebnote{Literally “like the days of heaven above the earth”}
\verse Yes, if you diligently keep all\lebnote{Or “every one of”} this commandment that I am commanding you \textit{to observe it}, \lebnote{Literally “to do it”} by loving Adonai your God \textit{by walking in}\lebnote{Literally “to walk in”} all his ways and by holding fast to him,
\verse then Adonai will drive out all of these nations \textit{before you},\lebnote{Literally “to the face of you”} and you will dispossess nations larger and more numerous than you.
\verse Every place on which the sole of your foot treads, it shall be yours; your boundary shall be from the desert and Lebanon from the river, the river Euphrates, on up to the western sea.\lebnote{That is, the Mediterranean Sea}
\verse No one can take a stand \textit{against you};\lebnote{Literally “to your faces”} your dread and your fear Adonai your God will put on the \textit{surface}\lebnote{Literally “faces”} of all the land \textit{where you tread},\lebnote{Literally “which you tread in it”} just as he \textit{promised}\lebnote{Literally “spoke”} to you.
\verse “See, I am setting \textit{before you}\lebnote{Literally “to the face of you”} \textit{today}\lebnote{Literally “the day”} a blessing and a curse:
\verse the blessing, if you listen to the commandments of Adonai your God that I am commanding you \textit{today},\lebnote{Literally “the day”}
\verse and the curse, if you do not listen to the commandments of Adonai your God, but rather you turn from the way that I am commanding you \textit{today}\lebnote{Literally “the day”} to go after other gods that you have not known.
\verse “And it will happen that when Adonai your God has brought you to \textit{the land that you are going to},\lebnote{Literally “the land that you are going there”} to take possession of it, then\lebnote{Hebrew “and”} you shall pronounce the blessing on Mount Gerizim and the curse on Mount Ebal.
\verse (Are they not beyond the Jordan, \textit{toward the west},\lebnote{Literally “after the way of the descent of the sun”} in the land of the Canaanites living in the Jordan Valley,\lebnote{Hebrew “in the Arabah”} opposite Gilgal beside the terebinth\lebnote{Or “oaks”} of Moreh?)
\verse For you are now about to cross the Jordan to go to take possession of the land that Adonai, your God, is giving to you, and you will take possession of it and live in it,
\verse and you must diligently \textit{observe}\lebnote{Literally “observe to do”} all the rules and the regulations that I am setting \textit{before you}\lebnote{Literally “before your face”} \textit{today}.”\lebnote{Literally “the day”}
\end{biblechapter}

\begin{biblechapter} % Deuteronomy 12
\verseWithHeading{Detailed Stipulations: Purity and Unity} “These are the rules and the regulations \textit{that you must diligently observe}\lebnote{Literally “that you must observe to do”} in the land that Adonai, the God of your ancestors,\lebnote{Or “fathers”} has given to you to take possession of it, during all the days that you are living on the land.
\verse You must completely demolish all of the places there where they served their gods, that is, the nations whom you are about to dispossess, on the high mountains, and on the hills and under each leafy green tree.\lebnote{Or “spreading tree”}
\verse And you shall break down their altars, and you shall smash their stone pillars, and their Asherah poles you must burn with fire, and the images of their gods you shall hew down, and you shall blot out their names from that place.
\verse You shall not worship Adonai your God like this.
\verse \textit{But only}\lebnote{Literally “But if”} to the place that Adonai your God will choose from all of your tribes to place his name there as his dwelling shall you seek, and there you shall go.
\verse And you shall bring there your burnt offerings and your sacrifices and your tithes and \textit{your donations}\lebnote{Literally “the contributions of your hand”} and your votive gifts and your freewill offerings and the firstling of your herd and your flock.
\verse And you shall eat there \textit{before}\lebnote{Literally “before the faces of”} Adonai your God, and you shall rejoice \textit{in all your endeavors},\lebnote{Literally “in all of the sending of your hand”} you and your family in which Adonai your God has blessed you.
\verse “You must not do just as we are doing here \textit{today},\lebnote{Literally “the day”} \textit{each according to all that is right in his eyes}.\lebnote{Literally “each man all of the right in his eyes”}
\verse For you have not come up to now to the resting place and to the inheritance that Adonai your God is giving to you.
\verse But you will cross the Jordan, and you will settle in the land that Adonai your God is giving you as an inheritance, and he will give rest to you from all your enemies from all around, and you will live securely,
\verse \textit{and then}\lebnote{Literally “and it will happen”} at the place that Adonai your God will choose, to let his name dwell there, there you shall bring all the things I am commanding you, your burnt offerings and your sacrifices, your tithes and \textit{your donations},\lebnote{Literally “the contribution of your hand”} and all of the choice things, your votive gifts that you vow to Adonai.
\verse And you shall rejoice \textit{before}\lebnote{Literally “the face of”} Adonai your God, you and your sons and your daughters and your slaves and your slave women and the Levite who is in your \textit{towns},\lebnote{Literally “gates”} because there is not for him a plot of ground and an inheritance with you.
\verse “Take care for yourself so that you do not offer your burnt offerings at just any place that you happen to see,
\verse \textit{but only}\lebnote{Literally “but only if”} at the place that Adonai will choose among one of your tribes; there you shall offer your burnt offerings, and there you shall do all the things that I am commanding you.
\verse “But \textit{whenever you desire}\lebnote{Literally “in all the wanting of your soul/inner self”} you may slaughter, and you may eat meat according to the blessing of Adonai your God that he has given to you in all of your \textit{towns};\lebnote{Literally “gates”} the unclean and the clean may eat it just as they would the gazelle and as the deer.
\verse Only the blood you must not eat, but on the ground you must pour it like water.
\verse You are not allowed to eat in your \textit{towns}\lebnote{Literally “gates”} the tithe of your grain and your wine and your olive oil and the firstborn of your herd and your flock and all of your votive gifts that you vowed and your freewill offering and \textit{your donations}.\lebnote{Literally “the contribution of your hand”}
\verse But only \textit{before}\lebnote{Literally “before the face of”} Adonai your God you shall eat it, at the place that Adonai your God will choose,\lebnote{Hebrew “choose it”} you and your son and your daughter and your slave and your slave woman and the Levite who is in your \textit{towns},\lebnote{Literally “gate”} and you must rejoice \textit{before}\lebnote{Literally “before the face of”} your God \textit{in all your undertakings}.\lebnote{Literally “in all of the sending of your hand”}
\verse \textit{Take care}\lebnote{Literally “Watch for yourself”} so that you do not neglect the Levite all of your days on your land.
\verse “When Adonai your God enlarges your territory just as he has \textit{promised}\lebnote{Literally “spoken”} to you, and you say ‘I want to eat some meat,’ \textit{because you want it},\lebnote{Literally “because your soul/inner self desires it”} \textit{whenever you desire}\lebnote{Literally “at/in all the wanting of your soul/inner self”} you may eat meat.
\verse If the place that\lebnote{Or “where”} Adonai your God will choose to put his name there is too far from you, and you slaughter any of your herd and any of your flock that Adonai has given to you just as I have commanded you, then you may eat whenever you desire in your \textit{towns}.\lebnote{Literally “gates”}
\verse Surely just as the gazelle and the deer is eaten, so both the unclean and the clean together may eat it.
\verse Only, be sure not to eat the blood, because the blood is the life, and you shall not eat the life with the meat.
\verse You shall not eat it, but on the ground you shall pour it out like water.
\verse You shall not eat it, so that \textit{it will go well}\lebnote{Literally “it/he is good”} for you and your children after you, because then you will be doing what is\lebnote{Hebrew “the”} right in the eyes of Adonai.
\verse Only your holy objects \textit{that are yours}\lebnote{Literally “that are for you”} and your votive gifts you must carry and you must bring to the place that Adonai will choose.
\verse And you shall offer your burnt offerings, the flesh and the blood on the altar of Adonai your God, and the blood of your sacrifices you shall pour out on the altar of Adonai your God, but the meat you may eat.
\verse \textit{Be careful to obey}\lebnote{Literally “Watch carefully and listen”} all these things that I am commanding you, so that \textit{it will go well}\lebnote{Literally “he/it is good”} for you and for your children after you \textit{forever},\lebnote{Literally “until ever”} because then you will be doing what is\lebnote{Hebrew “the”} good and right\lebnote{Hebrew “the right”} in the eyes of Adonai your God.
\verse “When Adonai your God has cut off the nations whom you are \textit{about to go to},\lebnote{Literally “about to go to there”} to dispossess them \textit{before you},\lebnote{Literally “from the face of you”} and you have dispossessed them, and you live in their land,
\verse \textit{take care}\lebnote{Literally “Watch for yourself”} so that you are not ensnared \textit{into imitating them}\lebnote{Literally “after them”} after their being destroyed from \textit{before you},\lebnote{Literally “from the face of you”} and so that you not enquire concerning their gods, \textit{saying},\lebnote{Literally “to say”} ‘How did these nations serve their gods, and thus \textit{I myself}\lebnote{Emphatic use of the pronoun} want to do also.’
\verse You must not do so toward Adonai your God, because of every detestable thing they have done for their gods Adonai hates, for even their sons and their daughters they would burn in the fire to their gods.
\verse \lebnote{Deuteronomy 12:32–13:18 in the English Bible is 13:1–19 in the Hebrew Bible} All of the things\lebnote{Hebrew “thing”} that I am commanding you, \textit{you must diligently observe};\lebnote{Literally “you must observe to do it”} you shall not add to it, and you shall not take away from it.”
\end{biblechapter}

\begin{biblechapter} % Deuteronomy 13
\verse “If a prophet stands up in your midst or a dreamer of dreams\lebnote{Hebrew “dream”} and he gives to you a sign or wonder,
\verse and the sign or\lebnote{Hebrew “and”} the wonder comes about that he spoke\lebnote{Or “promised/declared”} to you, \textit{saying},\lebnote{Literally “to say”} ‘Let us go after other gods (those whom you have not known), and let us serve them,’
\verse you must not listen to the words of that prophet or to that dreamer, for Adonai your God \textit{is testing you to know whether you love}\lebnote{Literally “a testing of Adonai your God concerning you in order to know whether there is you loving”} Adonai your God with all of your heart and with all of your inner self.\lebnote{Or “soul”}
\verse You shall go after Adonai your God, and him you shall revere, and his commandment you shall keep, and to his voice you shall listen, and him you shall serve, and to him you shall hold fast.
\verse But that prophet or the dreamer of that dream shall be executed, for he spoke falsely about Adonai your God, the one bringing you out from the land of Egypt and the one redeeming you from the house of slavery, in order to seduce you from the way that Adonai your God commanded you to go in it; so in this way you shall purge the evil from your midst.
\verse “If your brother, the son of your mother or your son or your daughter \textit{or your wife whom you embrace}\lebnote{Literally “or the wife of your embrace”} or \textit{your intimate friend}\lebnote{Literally “your friend who is like your soul/inner self”} in secrecy \textit{says},\lebnote{Literally “to say”} ‘Let us go and let us serve other gods!’ gods that you and your ancestors\lebnote{Or “fathers”} have not known,
\verse from among the gods of the people who are around you, \textit{those near you or those far from you},\lebnote{Literally “the near ones to you or the distant ones from you”} from one end of the earth and up to the other end of the earth,
\verse you must not give in to him, and you shall not listen to him, and your eye shall not take pity on him, and you shall not have compassion, and you shall not cover up\lebnote{Or “conceal” him} for him.
\verse But you shall certainly kill him; your hand shall be first against him to kill him and next the hand of all of the people.
\verse And you shall stone him with stones and let him die, for he tried to seduce you from Adonai your God, the one bringing you from the land of Egypt, from the house of slavery.
\verse And all of Israel shall hear, and they shall fear, and \textit{they shall not continue to act}\lebnote{Literally “they shall not add to do”} according to this evil thing in your midst.
\verse “If you hear in one of your towns which Adonai your God is giving to you to live in, someone \textit{saying that}\lebnote{Literally “to say”}
\verse \textit{worthless men}\lebnote{Literally “sons of Belial”} have gone out from your midst and have seduced the inhabitants of their town, \textit{saying},\lebnote{Literally “to say”} ‘Let us go and serve other gods!’ whom you have not known,
\verse then you shall inquire and examine and interrogate thoroughly, and, look! It is true; the thing has actually been done, this detestable thing in your midst,
\verse then you shall certainly strike down the inhabitants of that town with the \textit{edge}\lebnote{Literally “mouth”} of the sword; you shall destroy it and everything in it, its domestic animals with the \textit{edge}\lebnote{Literally “mouth”} of the sword.
\verse And then you shall gather all of its booty into the middle of its public square, and you shall burn the town and all of its war-booty totally for Adonai your God, and it shall be \textit{a pile of rubble forever};\lebnote{Literally “a tell/ruin forever”} it shall not be built again.
\verse And let not something cling to your hand from the things devoted to destruction, so that Adonai may turn back \textit{from his burning anger},\lebnote{Literally “from the burning of his nose”} and he may show compassion to you and he may continue to show compassion and so multiply you \textit{just as he swore}\lebnote{Literally “as that he swore”} to your ancestors,\lebnote{Or “fathers”}
\verse if you listen to the voice of Adonai your God, to keep all of his commandments that I am commanding to you \textit{today}\lebnote{Literally “the day”} so as to do the right thing in the eyes of Adonai your God.”
\end{biblechapter}

\begin{biblechapter} % Deuteronomy 14
\verse “You are children\lebnote{Or “sons”} of Adonai your God; therefore you must not gash\lebnote{Or “cut”} yourself, and \textit{you must not make your forehead bald}\lebnote{Literally “you shall not make/place baldness between your eyes”} for the dead.
\verse For you are a people holy to Adonai your God, and you Adonai has chosen to be a treasured possession from among all of the peoples that are on the surface\lebnote{Or “face”} of the earth.
\verse You shall not eat any detestable thing.
\verse These are the animals you may eat: ox, \textit{sheep, goats},\lebnote{Literally “small livestock of the sheep and a small livestock of the goats”}
\verse deer, gazelle, roebuck, wild goat, ibex, antelope, and mountain sheep.
\verse And any animal having a split hoof\lebnote{Or “divides the hoof”} and so \textit{a dividing of the hoof into two parts}\lebnote{Literally “a dividing cleft creating two hoofs”} \textit{and that chews the cud}\lebnote{Literally “and that brings up the cud”} among the animals\lebnote{Hebrew “animal”}—that animal you may eat.
\verse Only these you may not eat from \textit{those chewing the cud}\lebnote{Literally “bringing up the cud”} and from \textit{those having a division of the hoof}:\lebnote{Literally “those having a division of the hoof divided”} the camel and the hare and the coney, because \textit{they chew the cud},\lebnote{Literally “they bringing up of the cud”} but they do not divide the hoof; they are therefore unclean for you.
\verse And also the pig \textit{because it has a division of the hoof}\lebnote{Literally “because a division of the hoof”} \textit{but does not chew the cud};\lebnote{Literally “but not a chewing cud”} it is unclean for you; from their meat you shall not eat, and you shall not touch their \textit{carcasses}.\lebnote{Literally “dead body”}
\verse “This is what you shall eat from all that is in the water: \textit{everything}\lebnote{Literally “all of that”} \textit{that has fins and scales}\lebnote{Literally “for it is fins and scales,” showing possession of these features} you may eat.
\verse But \textit{anything that does not have}\lebnote{Literally “all of that there is not for it,” showing lack of possession of these features} fins and scales, you may not eat, for it is unclean for you.
\verse “All of the birds that are clean you may eat.
\verse Now these are the ones you shall not eat \textit{any of them}:\lebnote{Literally “from them”} the eagle and the vulture and the short-toed eagle,\lebnote{This list of birds is difficult to translate since the terms are not definitely known: e.g., some translations render the last bird as a “buzzard” (NASV); other translations give different names for all three: griffon vulture, black vulture, bearded vulture (NEB)}
\verse and the red kite\lebnote{Various options are available: large bird, kite, red kite, glede, buzzard} and the black kite or \textit{any kind of falcon},\lebnote{Literally “or the falcon according to its kind”; other options for falcon: bird, falcon, kite (others as carrion-bird), vulture, crow or raven, buzzard}
\verse and any kind of crow\lebnote{Or others translate as “raven”} according to its kind,
\verse and the \textit{ostrich}\lebnote{Literally “daughter of the ostrich”; others “desert owl”} and the short-eared owl and the seagull\lebnote{Or “long-eared owl”} and the hawk according to its kind,
\verse the little owl and the great owl and the barn owl,\lebnote{Or “white owl”}
\verse and the desert owl\lebnote{Or “large bird” or “horned-owl”} and the carrion vulture\lebnote{Or “large bird”} and the cormorant,\lebnote{Or “large bird,” or “fisher-owl”}
\verse and the stork and the heron according to its kind and the hoopoe and the bat.
\verse And also all of \textit{the winged insects};\lebnote{Literally “the swarmers of the flyers” or “all the swarms of things that fly”} they are unclean for you; you shall not eat them.
\verse You may eat any clean bird.
\verse “You shall not eat any carcass;\lebnote{Or “corpse”} you may give it to the alien who is in your \textit{towns},\lebnote{Literally “gates”} and he may eat it, or you may sell it to a foreigner, for you are a holy people for Adonai your God; you may not boil a kid in its mother’s milk.
\verse “Certainly you must give a tithe of all the yield of your seed, \textit{which comes forth from your field year after year}.\lebnote{Literally “the going forth of the field year by year”}
\verse And you shall eat \textit{before}\lebnote{Literally “before the faces of”} Adonai your God in the place that he will choose to make to dwell his name there the tithe of your grain, your wine and your olive oil and the firstling of your herd and your flock, so that you may learn to revere Adonai your God \textit{always}.\lebnote{Literally “all of the days”}
\verse But if \textit{the distance is too great for you},\lebnote{Literally “it is great from you the journey that”} so that you are not able to transport\lebnote{Or “carry”} it, because the place that Adonai your God will choose to set his name there, it is too far from you, when Adonai your God will bless you,
\verse then in that case \textit{you may exchange for money},\lebnote{Literally “you may give it in for the money/silver”} and you shall take\lebnote{Or “bind”} the money to your hand and go to the place that Adonai your God will choose.
\verse You may spend the money for anything \textit{that you desire},\lebnote{Literally “that your soul/inner self desires”} for oxen or for sheep or for wine or for strong drink or for anything \textit{that you desire},\lebnote{Literally “that your soul/inner self desires”} and you shall eat it there \textit{before}\lebnote{Literally “to the face of”} Adonai your God, and you shall rejoice, you and your household.
\verse And as to the Levite who is in your \textit{towns},\lebnote{Literally “gates”} you shall not neglect him, because there is not a plot of ground for him and an inheritance along with you.
\verse “At the end of three years you shall bring out all of the tithe of your yield for that year, and you shall store it in your \textit{towns}.\lebnote{Literally “gates”}
\verse And so the Levite may come, because there is no plot of ground for him or an inheritance with you, and the alien also may come and the orphan and the widow that are in your \textit{towns},\lebnote{Literally “gates”} and \textit{they may eat their fill},\lebnote{Literally “they may eat and they may be satisfied”} so that Adonai your God may bless you in all of the work of your hand that you undertake.”
\end{biblechapter}

\begin{biblechapter} % Deuteronomy 15
\verse “At the end of seven years you shall grant a remission of debt.
\verse And this is the manner of the remission of debt: every \textit{creditor}\lebnote{Literally “owner of the loan of his hand”} shall remit his claim that he holds against his neighbor, and he shall not exact payment from his brother because there\lebnote{Hebrew “it”} a remission of debt has been proclaimed unto\lebnote{Hebrew “for”} Adonai.
\verse With respect to the foreigner you may exact payment, but \textit{you must remit}\lebnote{Literally “your hand shall remit”} what shall be owed to you with respect to your brother.
\verse Nevertheless, there\lebnote{Hebrew “it”} shall not be among you a poor person, because Adonai will certainly bless you in the land that Adonai your God is giving to you as an inheritance, to take possession of it.
\verse If only you listen well to the voice of Adonai your God \textit{by observing diligently}\lebnote{Literally “to observe so as to do”} all of these commandments\lebnote{Hebrew “commandment”} that I am commanding you \textit{today}.\lebnote{Literally “the day”}
\verse When Adonai your God has blessed you, just as he \textit{promised}\lebnote{Literally “spoke”} to you, then you will lend to many nations, but you will not borrow from them, and you will rule over many nations, but they will not rule over you.
\verse If there is a poor person among you from among one of your brothers in one of your \textit{towns}\lebnote{Literally “gates”} that Adonai your God is giving to you, you shall not harden your heart, and you shall not shut your hand toward \textit{your brother who is poor}.\lebnote{Literally “from among your brothers, the poor one”}
\verse But you shall certainly open your hand for him, and \textit{you shall willingly lend}\lebnote{Literally “lending you shall lend”} to him enough to meet his need, \textit{whatever it is}.\lebnote{Literally “whatever is lacking for him”}
\verse \textit{Take care}\lebnote{Literally “Watch for yourself”} so that there\lebnote{Hebrew “it”} will not be \textit{a thought of wickedness}\lebnote{Literally “a thing in your heart wickedness”} in your heart, \textit{saying},\lebnote{Literally “to say”} ‘The seventh year, the year of the remission of debt is near,’ \textit{and you view your needy neighbor with hostility},\lebnote{Literally “is bad your eye against your brother who is needy”} and so you do not give to him, and he might cry out against you to Adonai, and \textit{you would incur guilt against yourself}.\lebnote{Literally “it will be against you a sin”}
\verse By all means you must give to him, and \textit{you must not be discontented}\lebnote{Literally “and not shall be bad/evil your heart at/when”} at your giving to him, because on account of this very thing, Adonai your God will bless you in all your work and \textit{in all that you undertake}.\lebnote{Literally “in all of the sending/putting forth of your hand”}
\verse For the poor\lebnote{Or “the needy person”} will not cease to be \textit{among you}\lebnote{Literally “from the midst of “} in the land; therefore I am commanding you, \textit{saying},\lebnote{Literally “to say”} ‘You shall willingly open your hand to your brother, to your needy and to your poor that are in your land.’
\verse If your relative\lebnote{Or “brother”} who is a Hebrew man or a Hebrew woman is sold to you, and he or she has served you six years, then in the seventh year you shall send that person out \textit{free}.\lebnote{Literally “free from with you”}
\verse And when you send him out free from you, you shall not send him away empty-handed.
\verse You shall generously supply him from among your flocks and from your threshing floor and from your press; according to that with which Adonai your God has blessed you, you shall give to him.
\verse And remember that you were a slave in the land of Egypt, and Adonai your God redeemed you; therefore I am commanding you thus \textit{today}.\lebnote{Literally “the day”}
\verse And then if it will happen that he says to you, ‘\textit{I do not want to go out}\lebnote{Literally “I will not go out”} from you,’ because he loves you and your family, because it is good for him to be with you;
\verse then you shall take an awl, and you shall thrust it through his earlobe and into the door, and he shall be to you \textit{a slave forever};\lebnote{Literally “a slave of eternity”} and you shall also do likewise for your slave woman.
\verse It shall not be hard in your eyes \textit{when you send him forth free},\lebnote{Literally “in/at you to send him forth free from being with you”} because for six years he has served you worth twice the wage of a hired worker; and Adonai your God will bless you \textit{in whatever you will do}.\lebnote{Literally “in all of that you will do”}
\verse “Every firstling male that is born of your herd and of your flock you shall consecrate to Adonai your God; you shall not do work with the firstling of your ox, and you shall not shear the firstling of your flock.
\verse Rather \textit{before Adonai}\lebnote{Literally “in the face of Adonai”} your God you shall eat it year by year at the place Adonai will choose, you and your household.
\verse But if there is a physical defect in it, such as lameness or blindness, any serious defect, you shall not sacrifice it to Adonai your God.
\verse In your \textit{towns}\lebnote{Literally “gates”} you shall eat it, the unclean and the clean together may eat it, just as they eat the gazelle and as they eat the deer.
\verse But you shall not eat its blood; you shall pour it on the ground like water.”
\end{biblechapter}

\begin{biblechapter} % Deuteronomy 16
\verse “Observe the month of Abib, and you shall keep the Passover to Adonai your God, for in the month of Abib Adonai your God brought you out from Egypt by night.
\verse And you shall offer the Passover sacrifice to Adonai your God from among your flock and herd at the place that Adonai will choose, to let his name dwell there.
\verse You shall not eat \textit{with it}\lebnote{Literally “in addition to” or “upon it”} anything leavened; seven days you shall eat \textit{with it}\lebnote{Literally “in addition to” or “upon it”} unleavened bread of affliction, because in haste you went out from the land of Egypt, so that you will remember the day of your going out from the land of Egypt all the days of your life.
\verse And leaven shall not be seen with\lebnote{Or “for”} you in any of your territory\lebnote{Or “all of” your territory} for seven days, and none of the meat that you will slaughter on the evening on the first day shall remain overnight until morning.
\verse You are not allowed to offer the Passover sacrifice in one of your \textit{towns}\lebnote{Literally “gates”} that Adonai your God is giving to you,
\verse but only at the place that Adonai your God will choose, to let his name dwell there; you shall offer the Passover sacrifice \textit{in the evening at sunset},\lebnote{Literally “in the evening as the sun goes/sets”} at the designated time\lebnote{The Hebrew word here indicates the specific time that God had chosen to bring Israel out of Egypt} of your going out from Egypt.
\verse And you shall cook, and you shall eat it at the place that Adonai your God will choose; and you may turn in the morning and go to your tents.
\verse Six days you shall eat unleavened bread, and on the seventh day there shall be an assembly for Adonai your God; you shall not do work.
\verse “You shall count off seven weeks for you; \textit{from the time you begin to harvest the standing grain}\lebnote{Literally “from the beginning of the sickle against the standing grain”} you shall begin to count seven weeks.
\verse And then you shall celebrate the Feast of Weeks for Adonai your God with the measure of the freewill offering of your hand that you shall give just as Adonai your God has blessed you.
\verse And you shall rejoice before Adonai your God, you and your son and your daughter and your slave and your slave woman and the Levite that is in your \textit{towns}\lebnote{Literally “gates”} and the alien and the orphan and the widow who are in your midst in the place that Adonai your God will choose to let his name dwell there.
\verse And you shall remember that you were a slave in Egypt, and so \textit{you shall diligently observe}\lebnote{Literally “you shall observe and do”} these rules.
\verse “You shall celebrate the Feast of Booths for yourselves\lebnote{Hebrew “for you”} seven days \textit{at the gathering in of the produce}\lebnote{Literally “at your gathering of the produce”} from your threshing floor and from your press;
\verse and you shall rejoice at your feast, you and your son and your daughter and your slave and your slave woman and the Levite and the orphan and the widow that are in your \textit{towns}.\lebnote{Literally “gates”}
\verse Seven days you shall celebrate your feast to Adonai your God at the place Adonai will choose, for Adonai your God shall bless you in all of your produce\lebnote{Or “increase”} and in all of the work of your hand, and you shall surely be rejoicing.\lebnote{Or “joyful”}
\verse Three times in the year all of your males shall appear \textit{before }\lebnote{Literally “with the face of”} Adonai your God at the place that he will choose, at the Feast of Unleavened Bread and at the Feast of Weeks and at the Feast of Booths, and they shall not appear \textit{before Adonai}\lebnote{Literally “with the face of Adonai”} empty-handed.
\verse Each person \textit{shall give as he is able},\lebnote{Literally “according to the gift of his hand”} that is, according to the blessing of Adonai your God that he has given to you.
\verse “You shall appoint judges and officials for you in all your \textit{towns}\lebnote{Literally “gates”} that Adonai your God is giving to you throughout your tribes, and you shall render for the people \textit{righteous judgments}.\lebnote{Literally “a judgment based on righteousness”}
\verse You shall not subvert\lebnote{Or “distort/pervert”} justice; you shall not \textit{show partiality};\lebnote{Literally “recognize faces”} and you shall not take a bribe, for the bribe makes blind the eyes of the wise and misrepresents the words of the righteous.
\verse \textit{Justice, only justice}\lebnote{Literally “justice justice”} you shall pursue, so that you may live, and you shall take possession of the land that Adonai your God is giving to you.
\verse You shall not plant for yourselves\lebnote{Hebrew “for/to you” but with collective meaning} \textit{an Asherah pole}\lebnote{Literally “an Asherah of any wood/tree”} beside the altar of Adonai your God that you make for yourselves.\lebnote{Hebrew “for/to you” but with collective meaning}
\verse And you shall not set up for yourselves a stone pillar, a thing that Adonai your God hates.
\end{biblechapter}

\begin{biblechapter} % Deuteronomy 17
\verse “You shall not sacrifice to Adonai your God an ox or sheep \textit{that has a physical defect}\lebnote{Literally “that is on it a physical defect”} \textit{of anything seriously wrong},\lebnote{Literally “of any thing bad/evil”} for that is a detestable thing to Adonai your God.
\verse If there is found in one of your \textit{towns}\lebnote{Literally “gates”} that Adonai your God is giving to you a man or a woman that does evil in the eyes of Adonai your God to transgress his covenant
\verse and by going and serving other gods and so he bows down to them and to the sun or to the moon or to any of the host of heaven \textit{which I have forbidden},\lebnote{Literally “which not I have commanded”}
\verse and it is reported to you or you hear about it and you enquire about it thoroughly and, indeed,\lebnote{Or “look!”} the trustworthiness of the deed\lebnote{Hebrew “the thing”} has been established, it \textit{has occurred},\lebnote{Literally “has been done”} this detestable thing, in Israel,
\verse then you shall bring out that man or that woman who has\lebnote{Hebrew “have”} done this evil thing to your gates; that is, the man or the woman, and you shall stone them with stones \textit{to death}.\lebnote{Literally “and so they die”}
\verse \textit{On the evidence of}\lebnote{Literally “By the mouth of”} two or three witnesses \textit{the person shall be put to death}.\lebnote{Literally “he/she shall be put to death, the dead person”} The person\lebnote{Hebrew “he/she”} shall not be put to death by the mouth of one witness.
\verse The hand of the witnesses shall be first against the person\lebnote{Hebrew “him”} to kill the person,\lebnote{Hebrew “him”} and afterward the hands\lebnote{Hebrew “hand”} of all the people, and so you shall purge the evil from your midst.
\verse “\textit{If a matter is too difficult for you},\lebnote{Literally “If is difficult/wonderful from you a thing for judgment”} for example disputes between blood and blood,\lebnote{Or “between one homicide and another”} between legal claim and legal claim\lebnote{Or “one kind of lawsuit and another”} and between assault and assault\lebnote{Or “one kind of abuse and another”} and between matters of discernment in your \textit{towns},\lebnote{Literally “gates”} then you shall get up and you shall go to the place that Adonai your God will choose;
\verse then you shall go to the priests and the Levites and to the judge who will be in office in those days, and you shall enquire, and they shall announce to you \textit{the verdict}.\lebnote{Literally “the word of the decision”}
\verse “And \textit{you shall carry out exactly the decision}\lebnote{Literally “and you shall do according to the mouth of the word”} that they announced to you from that place that Adonai will choose, and \textit{you shall diligently observe}\lebnote{Literally “and you shall observe and do”} according to all that they instruct you.
\verse And so according to \textit{the instruction of the law}\lebnote{Literally “mouth of the law”} that they teach you and according to the decisions that they say to you, you shall do; you shall not turn from the word that they tell you to the right or to the left.
\verse And the man who treats with contempt\lebnote{Or “acts presumptuously”} so as not to listen to the priest who is standing to minister on behalf of Adonai your God or to the judge, that man shall die; so you shall purge the evil from Israel.
\verse And all the people will hear and will be afraid, and they will not behave presumptuously again.
\verse “When you have come to that land that Adonai your God is giving to you and you have taken possession of it and you have settled in it, and you say, ‘I will set over me a king like all the nations that are around me,’
\verse indeed, you may set a king over you whom Adonai your God will choose, from the midst of your countrymen\lebnote{Or “brothers”} you must set a king over you; you are not allowed to appoint over you a man, a foreigner, who is not your countryman.\lebnote{Or “brother”}
\verse Except, he may \textit{not make numerous}\lebnote{Literally “not multiply”} for himself horses, and he may not allow the people to to go to Egypt \textit{in order to increase horses},\lebnote{Literally “in order to make numerous horse”} for Adonai has said to you that \textit{you may never return}.\lebnote{Literally “not you may do again to return”}
\verse And he must not \textit{acquire many}\lebnote{Or “make numerous”} wives for himself, so that his heart would turn aside; and \textit{he must not accumulate silver and gold for himself excessively}.\lebnote{Literally “gold and silver not he must make numerous for him very”}
\verse “\textit{And then}\lebnote{Literally “And it shall happen”} \textit{when he is sitting}\lebnote{Literally “as/when his sitting”} on the throne of his kingdom, then he shall write for himself a copy of this law on a scroll \textit{before}\lebnote{Literally “to the face of”} the Levitical priests.
\verse And it shall be with him, and he shall read it\lebnote{Hebrew “in it”} all the days of his life, so that he may learn to revere Adonai your God by \textit{diligently observing}\lebnote{Literally “by keeping ... to do them”} all the word of this law and these rules,
\verse so as not to exalt his heart above his countrymen\lebnote{Or “brothers”} and not to turn aside from the commandment to the right or to the left, so that \textit{he may reign long over his kingdom},\lebnote{Literally “he may make long his days over his kingdom”} he and his children in the midst of Israel.”
\end{biblechapter}

\begin{biblechapter} % Deuteronomy 18
\verse “And there shall not be for the Levitical priests, the whole tribe of Levi, a plot of ground and an inheritance with Israel, rather they may eat an offering made by fire as their inheritance,\lebnote{The meaning of the Hebrew text here is uncertain; possibly it reads: “rather an offering by fire then they will eat as their inheritance/patrimony”} for Adonai.
\verse And there shall not be for them\lebnote{Hebrew “him”} an inheritance of land in the midst of his brothers; rather Adonai is his inheritance, just as he \textit{promised}\lebnote{Literally “spoke”} to them.\lebnote{Hebrew “him”}
\verse Now this shall be the share of the priest from the people, from \textit{those who sacrifice the sacrifice},\lebnote{Literally “the sacrificers of the sacrifice”} whether it is an ox, sheep, or goat, and they shall give the priest the shoulder and the jawbones and the stomach.
\verse The firstfruits\lebnote{Hebrew “firstfruit”} of your grain, your wine, and your olive oil and the firstfruits\lebnote{Hebrew “firstfruit”} of the fleece of your sheep you shall give to him.
\verse For Adonai your God has chosen him from among all your tribes to stand to minister in the name of Adonai, he and his sons \textit{forever}.\lebnote{Literally “all of the days”}
\verse And if a\lebnote{Hebrew “the”} Levite comes from one of your \textit{towns}\lebnote{Literally “gates”} from \textit{anywhere in Israel}\lebnote{Literally “all of Israel”} where he is residing, \textit{he may come whenever he desires},\lebnote{Literally “he may come in all the desire of his soul/inner self”} to the place that Adonai will choose,
\verse and he may minister in the name of Adonai his God, just like all his brothers, \textit{the Levites who stand there}\lebnote{Literally “the Levites the ones standing there”} \textit{before}\lebnote{Literally “to the face of”} Adonai.
\verse They shall eat \textit{equal portions},\lebnote{Literally “a portion like a portion”} apart from what he may receive from the sale of his patrimony.\lebnote{Hebrew meaning of these phrases/words is not certain; also could translate as “except what they receive from the sale of their fathers’ estates” (NASB); or “what he may inherit from his father’s family”}
\verse “When you come to the land that Adonai your God is giving to you, you must not learn to do like the detestable practices of those nations.
\verse There shall not be found among you one who makes his son or his daughter go through the fire, or \textit{one who practices divination},\lebnote{Literally “a diviner of divination”} or an interpreter of signs,\lebnote{Or “soothsayer”} or an augur,\lebnote{Or “an interpreter of omens”} or sorcerer,
\verse or one who casts magic spells, or one who consults a spirit of the dead,\lebnote{Or “medium”} or spiritist, or one who inquires of the dead.\lebnote{Or “necromancer”}
\verse For everyone doing these things is detestable to Adonai, and because of these detestable things Adonai your God is driving them out from \textit{before you}.\lebnote{Literally “the face of you”}
\verse You must be blameless before Adonai your God.
\verse For these nations that you are about to dispossess listen to interpreters of signs\lebnote{Or “practice witchcraft”} and to diviners, but Adonai your God has not allowed you to do the same.
\verse “Adonai your God will raise up for you a prophet like me from your midst, from your countrymen,\lebnote{Or “your brothers”} and to him you shall listen.
\verse This is \textit{according to all that you asked}\lebnote{Literally “just as all of that you asked”} from Adonai your God at Horeb, on the day of the assembly, \textit{saying},\lebnote{Literally “to say”} ‘\textit{I do not want again to hear}\lebnote{Literally “not I want to do again to hear”} the voice of Adonai my God, and I do not want to see again this great fire, so that I may not die!’
\verse And Adonai said to me, ‘They are right in what they have said.
\verse I will raise up a prophet for them \textit{from among their countrymen}\lebnote{Literally “from the midst of their brothers”} like you, and I will place\lebnote{Or “give”} my words into his mouth, and he shall speak to them \textit{everything that I command him}.\lebnote{Literally “all of what I command him”}
\verse \textit{And then}\lebnote{Literally “And it will happen”} the man that will not listen to my words that he shall speak in my name, I will hold accountable.
\verse However, the prophet that behaves presumptuously by speaking a word in my name that I have not commanded him to speak, and who speaks in the name of other gods, then that prophet shall die.’
\verse And if you say \textit{to yourself},\lebnote{Literally “in your heart”} ‘How can we know the word that Adonai has not spoken to him?’
\verse Whenever what the prophet spoke in the name of Adonai, the thing does not take place and does not come about, that is the thing that Adonai has not spoken to him.\lebnote{Or “that Adonai has not spoken it”} Presumptuously the prophet spoke it; you shall not fear that\lebnote{Hebrew “the”} prophet.”
\end{biblechapter}

\begin{biblechapter} % Deuteronomy 19
\verse “When Adonai your God has \textit{exterminated}\lebnote{Literally “cut off”} the nations concerning whom Adonai your God is giving to you their land, and you have dispossessed them, and you have settled in their towns and in their houses,
\verse you shall set apart three cities for you in the midst of your land that Adonai your God is giving to you to take possession of it.
\verse You shall prepare the roads\lebnote{Hebrew “road”} for yourselves, and you shall divide the regions of your land into thirds that Adonai your God gives you as a possession, so that \textit{it will be available for any manslayer to flee there}.\lebnote{Literally “it shall be to flee there all/anyone killing”}
\verse “Now this is the case of the manslayer who may flee there and live there who has killed his neighbor \textit{unintentionally},\lebnote{Literally “with no knowledge”} and he did not hate him \textit{previously}.\lebnote{Literally “from yesterday and the day before”}
\verse \textit{For example},\lebnote{Literally “And as”} when somebody goes with his neighbor into the forest to cut wood, and the iron head slips from the handle of the tool and strikes his neighbor and he dies, then he may flee to one of these cities, and so he may live.
\verse He does this lest the avenger of blood might pursue after the killer, because \textit{he is hot with anger}\lebnote{Literally “is hot his heart”} and he overtakes him, because it is a long distance to the city of refuge, and so \textit{he kills him},\lebnote{Literally “he strikes him down as to his life/soul”} but \textit{he did not deserve a death sentence},\lebnote{Literally “for him there was not a judgment of death”} because he was not hating him \textit{before}.\lebnote{Literally “from yesterday and the day before”}
\verse Therefore I am commanding you, \textit{saying},\lebnote{Literally “to say”} ‘You shall set apart three cities.’
\verse Then if Adonai your God enlarges your territory just as he swore to your ancestors\lebnote{Or “fathers”} and gives\lebnote{Hebrew “will give”} to you all the land that he \textit{promised}\lebnote{Literally “spoke”} to give to your ancestors,\lebnote{Or “fathers”}
\verse then \textit{if you diligently observe this entire commandment}\lebnote{Literally “if you observe all of the commandment the this to do it”} that I am commanding you \textit{today}\lebnote{Literally “the day”} by loving Adonai your God and by going\lebnote{Or “walking”} in his ways \textit{at all times},\lebnote{Literally “all of the days”} then you shall add three more cities for yourselves to these three.
\verse Do this so that innocent blood will not be shed\lebnote{Hebrew “is not shed”} in the midst of your land that Adonai your God is giving to you as an inheritance and thereby bloodguilt would be on you.\lebnote{Hebrew “shall be on you,” but conditional sense of imperfect is clear}
\verse But if someone hates\lebnote{Hebrew “is hating”} his neighbor and lies in wait for him and rises up\lebnote{Or “gets up”} against him \textit{and murders him},\lebnote{Literally “and strikes him mortally with regard to his life and he dies”} and the murderer flees to one of these cities,
\verse then the elders of his city shall send and take him from there, and they shall give him into the hand of the avenger of blood, and he shall be put to death.
\verse Your eye shall not take pity on him, and you shall purge the guilt of innocent blood from Israel, \textit{so that good will be directed toward you}.\lebnote{Literally “and good shall be for you” or “it shall be good for you”}
\verse “You shall not move the boundary marker of your neighbor that \textit{former generations}\lebnote{Literally “the first settlers/ancestors”} set up on your property in the land that Adonai your God is giving to you to take possession of it.
\verse \textit{The testimony of a single witness may not be used to convict}\lebnote{Literally “Not shall get up only a single witness against a man”} with respect to any crime and for any wrongdoing in any offense that a person\lebnote{Hebrew “he”} committed; on the \textit{evidence}\lebnote{Literally “mouth”} of two witnesses or on the \textit{evidence}\lebnote{Literally “mouth”} of three witnesses \textit{a charge shall be sustained}.\lebnote{Literally “shall be established a case/charge”}
\verse If \textit{a malicious witness}\lebnote{Literally “a witness of violence”} gets up \textit{to accuse}\lebnote{Literally “against”} anyone to testify against him falsely,
\verse then the two men \textit{to whom the legal dispute pertains}\lebnote{Literally “who for them are the legal dispute”} shall stand \textit{before}\lebnote{Literally “to the face of”} Adonai, \textit{before}\lebnote{Literally “to the face of”} the priests and the judges who are in office in those days.
\verse Then judges shall make a thorough inquiry, and if it turns out\lebnote{Or “in fact”} that the witness is a false witness and he testified falsely against his brother,
\verse then you shall do to him as he meant\lebnote{Or “planned”} to do to his brother, and so you shall purge the evil from your midst.
\verse \textit{And the rest}\lebnote{Literally “and those remaining”} shall hear and shall fear, and \textit{they shall not continue to do such a thing again}\lebnote{Literally “and they shall not do again to do again”} as this evil thing in your midst.
\verse \textit{You must show no pity}:\lebnote{Literally “And not take pity your eye”} life for life, eye for eye, tooth for tooth, hand for hand, foot for foot.”
\end{biblechapter}

\begin{biblechapter} % Deuteronomy 20
\verse “If you go out to war against your enemies and you see a horse and a chariot, \textit{an army}\lebnote{Literally “a people”} larger that you, you shall not be afraid because of them; for Adonai your God is with you, the one who brought you from the land of Egypt.
\verse \textit{And then}\lebnote{Literally “And it will happen”} when you approach the battle, then the priest shall come near and speak to the troops.
\verse And he shall say to them, ‘Hear, Israel, you are near \textit{today}\lebnote{Literally “the day”} to the battle against your enemies; \textit{do not lose heart};\lebnote{Literally “do not be faint/tender-hearted”} you shall not be afraid, and you shall not panic, and you shall not be terrified \textit{because of them},\lebnote{Literally “from their faces”}
\verse for Adonai your God is going with you to fight for you against your enemies to help you.’
\verse And the officials shall speak to the troops, \textit{saying},\lebnote{Literally “to say”} ‘Who is the man who has built a new house and has not dedicated it? Let him go and return to this house, so that he does not die in battle and \textit{another man}\lebnote{Literally “a man other”} dedicates it.
\verse And who is the man that has planted a vineyard and has not enjoyed it? Let him go and let him return to his house, so that he does not die in battle and \textit{another man}\lebnote{Literally “a man other”} enjoys it.
\verse And who is the man who got engaged to a woman and has not married her? Let him go and let him return to his house, so that he does not die in battle and \textit{another man}\lebnote{Literally “a man other”} marries her.’
\verse And the officials shall continue to speak to the troops, and they shall say, ‘\textit{What man}\lebnote{Literally “Who is the man”} is afraid \textit{and disheartened}?\lebnote{Literally “and soft/weak of the heart”} Let him go, and let him return to his house, and let him not cause the heart of his brothers to melt\lebnote{The verb has a causal meaning here} like his.’
\verse And \textit{when the officials have finished speaking}\lebnote{Literally “it will happen as to finish the officials to speak”} to the army troops, then they shall appoint commanders of divisions at the head of the troops.
\verse “When you approach a city to fight against it, \textit{you must offer it peace}.\lebnote{Literally “you should call it for peace”}
\verse \textit{And then}\lebnote{Literally “And it will happen”} if \textit{they accept your terms of peace}\lebnote{Literally “if peace they reply to you”} and \textit{they surrender to you},\lebnote{Literally “they open to you”} \textit{and then}\lebnote{Literally “and it will happen”} all the people \textit{inhabiting it}\lebnote{Literally “being found in it”} shall be forced labor for you, and they shall serve you.
\verse But if they \textit{do not accept your terms of peace}\lebnote{Literally “not they make peace with you”} and they want to make war with you, then you shall lay siege against it.\lebnote{That is, the city}
\verse And Adonai your God will give it into your hand, and you shall kill all its males with the \textit{edge}\lebnote{Literally “mouth”} of the sword.
\verse Only the women and the little children and the domestic animals\lebnote{Hebrew “animal”} and all that shall be in the city, all of its spoil you may loot for yourselves, and you may enjoy the spoil of your enemies that Adonai you God has given to you.
\verse Thus you shall do to all the far\lebnote{Or “distant”} cities from you, which are not from the cities of these nations located \textit{nearby}.\lebnote{Literally “here”}
\verse But from the cities of these peoples that Adonai your God is giving to you as an inheritance, you shall not let anything live that breathes.\lebnote{Or “is alive”}
\verse Rather, you shall utterly destroy them, the Hittites\lebnote{Hebrew “Hittite”} and the Amorites,\lebnote{Hebrew “Amorite”} the Canaanites\lebnote{Hebrew “Canaanite”} and the Perizzites,\lebnote{Hebrew “Perizzite”} the Hivites,\lebnote{Hebrew “Hivite”} and the Jebusites,\lebnote{Hebrew “Jebusite”} just as Adonai your God has commanded you,
\verse so that they may not teach you to do like all their detestable things that they do for their gods and thereby you sin against Adonai your God.
\verse “If\lebnote{Or “when”} you besiege a town for many days to make war against it in order to seize it, you shall not destroy its trees by wielding an ax against them,\lebnote{Hebrew “it”} for you may eat from them,\lebnote{Hebrew “it”} and so you must not cut them\lebnote{Hebrew “it”} down. Are the trees of the field humans that they should come in siege \textit{against you}?\lebnote{Literally “from your face”}
\verse Only\lebnote{Or “But”} the trees\lebnote{Hebrew “tree”} that you know \textit{are not fruit trees}\lebnote{Literally “not a tree of food”} you may destroy and you may cut down, and you may build siege works against that city that is making war with you \textit{until it falls}.”\lebnote{Literally “it to fall”}
\end{biblechapter}

\begin{biblechapter} % Deuteronomy 21
\verse “If someone slain is found in the land that Adonai your God is giving to you to take possession of it and is lying in the field, and it is not known who \textit{killed him},\lebnote{Literally “struck/smote him”}
\verse then your elders and your judges shall go out and shall measure the distance to the cities that are around the slain one.
\verse \textit{And then}\lebnote{Literally “And it will happen”} the nearest city to the slain one, the elders of that city shall take a heifer of the herd that has not been worked with in the field, that has not pulled a yoke,
\verse and the elders of that city shall bring the heifer down to a \textit{wadi that flows with water all year}\lebnote{Literally “an ever-flowing wadi”} and that has not been plowed and has not been sown; then \textit{there they shall break the neck of the heifer in the wadi}.\lebnote{Literally “they shall break there the neck with respect to the heifer in the ever-flowing wadi”; the Hebrew verb carries the meaning “to break the neck of”}
\verse Then the priests, the descendants\lebnote{Or “sons”} of Levi, shall come near, for Adonai your God has chosen them to bless in the name of Adonai, and every legal dispute and every case of assault will be \textit{subject to their ruling}.\lebnote{Literally “on their mouth”}
\verse And all of the elders of that city nearest to the slain person shall wash their hands over the heifer with the broken neck in the wadi.\lebnote{A valley that is dry most of the year, but contains a stream during the rainy season}
\verse And they shall declare, and they shall say, ‘Our hands did not shed this blood, and our eyes did not see what was done.
\verse Forgive your people, Israel, whom you redeemed, Adonai, and do not \textit{allow}\lebnote{Literally “place/put”} the guilt of innocent blood in the midst of your people Israel, and let them be forgiven with regard to blood.’
\verse And so you shall purge the innocent blood from your midst, because you must do the right thing in the eyes of Adonai.
\verse “When you go out for battle against your enemies, and Adonai your God gives them into your hand, and you lead the captives\lebnote{Hebrew “his captive” but singular pronoun refers to the many captives taken with plural sense} away,
\verse and you see among the captives\lebnote{Hebrew “captive”} a woman beautiful in appearance, and you become attached to her and you want to take her as a wife,
\verse then you shall bring her into your household, and she shall shave her head, and she shall trim her nails.
\verse And she shall remove the clothing of her captivity from her, and she shall remain in your house, and she shall mourn her father and her mother \textit{a full month},\lebnote{Literally “a month of days”} and after this \textit{you may have sex with her},\lebnote{Literally “you may go into her”} and you may marry her, and she may \textit{become your wife}.\lebnote{Literally “become for you as wife”}
\verse And then if you do not take delight in her, then you shall let her go \textit{to do whatever she wants},\lebnote{Literally “according to her desire/soul”} but you shall not treat her as a slave, since you have dishonored\lebnote{Or “humbled”} her.
\verse “If a man has two wives, and the one is loved and the other one is disliked and the one loved and the one that is disliked have borne for him sons, if it happens that the firstborn son \textit{belongs to the one that is disliked},\lebnote{Literally “is to the wife who is hated”}
\verse nevertheless \textit{it will be the case that}\lebnote{Literally “it will happen”} on the day of bestowing his inheritance upon his sons, he will not be allowed to treat as the firstborn son the son of the beloved wife \textit{in preference to}\lebnote{Literally “over the faces of”} the son of the disliked wife, who is the firstborn son.
\verse But he shall acknowledge the firstborn son of the disliked wife \textit{by giving}\lebnote{Literally “to give”} him a double portion of \textit{all that he has},\lebnote{Literally “all that is found for him”} for he is the firstfruit of his vigor;\lebnote{Or “the beginning of his strength”} to him is the legal claim of the birthright.\lebnote{Or “the just claim of the firstborn”}
\verse “\textit{If a man has a stubborn and rebellious son}\lebnote{Literally “If shall be for a man, a son stubborn and rebellious”} who \textit{does not listen to}\lebnote{Literally “and there is no listening/obedience”} the voice of his father and to the voice of his mother, and they discipline him, and he does not obey\lebnote{Or “listen to”} them,
\verse then his father and his mother shall take hold of him, and they shall bring him out to the elders of his city and to the gate of his \textit{town},\lebnote{Literally “place”}
\verse and they shall say to the elders of his city, ‘This our son is stubborn and rebellious; \textit{he does not obey us},\lebnote{Literally “there is no listening to our voice”} and he is a glutton and a drunkard.’
\verse Then all the men of his city shall stone him with stones and let him die; and so you shall purge the evil from your midst, and all of Israel will hear, and they will fear.
\verse “And \textit{if a man commits a sin punishable by death},\lebnote{Literally “when shall be against a man a sin of judgment of death”} and so he is put to death and you hang him on a tree,
\verse his dead body shall not hang on the tree, but certainly you shall bury him on that day, for cursed by God is one that is being hung; so you shall not defile your land\lebnote{Or “ground”} that Adonai your God is giving to you as an inheritance.”
\end{biblechapter}

\begin{biblechapter} % Deuteronomy 22
\verse “You shall not watch the ox of your neighbor or his sheep or goat straying and ignore them; certainly you shall return them to your neighbor.
\verse And if your countryman\lebnote{Or “brother”} is not near you or you do not know \textit{who he is},\lebnote{Literally “him”} then you shall bring it \textit{to your household},\lebnote{Literally “to the midst of your house”} and it shall be\lebnote{Or “remain”} with you \textit{until your countryman seeks after it},\lebnote{Literally “the seeking of your countryman after it”} and you shall return it to him.
\verse And thus also you shall do regarding\lebnote{Or “to”} his donkey, and thus you shall do concerning\lebnote{Or “to”} his garment, and so you shall do with respect to\lebnote{Or “to”} all of the lost property of your countryman\lebnote{Or “brother”} that is lost from him and you find it; you are not allowed to withhold help.
\verse “You shall not see the donkey of your neighbor or his ox fallen on the road and you ignore them; certainly you must help them get up along with him.
\verse “The apparel of a man shall not be put on\lebnote{Literally “on”} a woman, and a man shall not wear the clothing of a woman, because everyone who does these things is detestable to Adonai your God.
\verse “If a bird’s nest is found \textit{before you}\lebnote{Literally “before your face”} on the road in any tree or on the ground, and there are chicks or eggs, and the mother is lying down on the chicks or the eggs, you shall not take the mother along with the young;
\verse you shall certainly let the mother go, but you may take the young for yourselves;\lebnote{Hebrew “for you”} do this \textit{so that it may go well}\lebnote{Literally “he/it is good”} for you and \textit{you may live long in the land}.\lebnote{Literally “and you may make long/extend days”}
\verse “When you build a new house then\lebnote{Hebrew “and”} you shall make a parapet wall\lebnote{Or “fence/rail”} for your roof, so that you will not bring bloodguilt on your house \textit{if anyone should fall from it}.\lebnote{Literally “if should fall the falling from it”}
\verse “You shall not sow your vineyard with differing kinds of seed, so that you shall not forfeit \textit{the whole harvest},\lebnote{Literally “the entire fullness”} both the seed that you sowed and the yield of the vineyard.
\verse “You shall not plow with an ox and with a donkey yoked together.
\verse “You shall not wear woven material made of wool and linen mixed together.
\verse “You shall make tassels for yourselves\lebnote{Hebrew “for you”} on the four corners of your clothing with which you cover yourself.
\verse “If a man takes a woman and \textit{he has sex with her},\lebnote{Literally “and he goes unto her”} but he then \textit{dislikes her},\lebnote{Literally “hates her”}
\verse and \textit{he accuses her falsely},\lebnote{Literally “he puts to her deeds of words”} and \textit{he defames her},\lebnote{Literally “he brings forth against her a bad name”} and he says ‘This woman I took and I lay with her and \textit{I discovered that she was not a virgin},’\lebnote{Literally “I found not with her virginity”}
\verse then in defense the father of the young woman shall take, along with her mother, and together they must bring out the evidence of the virginity of the young woman to display it to the elders of the city \textit{at the city gate}.\lebnote{Literally “toward the city gate”}
\verse And then the father of the young woman shall say to the elders, ‘I gave my daughter to this man as wife, but he now \textit{dislikes}\lebnote{Literally “hates”} her,
\verse and now look \textit{he has accused her falsely},\lebnote{Literally “he put deeds of words”} saying, “I did not find \textit{your daughter a virgin},”\lebnote{Literally “to your daughter virginity”} but here is evidence of the virginity of my daughter’; and they shall spread the cloth out \textit{before}\lebnote{Literally “to the face of”} the elders of the city.
\verse Then the elders of that city shall take the man, and they shall discipline him.
\verse Then they shall fine him a hundred shekels of silver, and they shall give them to the father of the young woman, for \textit{he defamed an Israelite young woman},\lebnote{Literally “for he brought out a bad name against the virgin of Israel”} and \textit{she shall become his wife};\lebnote{Literally “and for him she shall become as wife”} he will not be allowed \textit{to divorce her}\lebnote{Literally “to send her out”} all his days.
\verse “But if \textit{this charge}\lebnote{Literally “the thing”} was true, \textit{and the signs of virginity were not found}\lebnote{Literally “and they were not found evidence of virginity”} for the young woman,
\verse and then they shall bring out the young woman to the doorway of the house of her father, and the men of her city shall stone her with stones, and she shall die, because she did a disgraceful thing in Israel \textit{by playing the harlot}\lebnote{Literally “to play the harlot/to prostitute herself”} in the house of her father, and so you shall purge the evil from your midst.
\verse “If a man is found lying \textit{with a married woman},\lebnote{Literally “with a woman, a young woman of a husband”} then they shall both die; \textit{both of them},\lebnote{Literally “also/even the two of them”} the man who lay with the woman and the woman also, so you shall purge the evil from Israel.
\verse “If it happens that a young woman, a virgin, is engaged to a man, and a man finds her in the town and lies with her,
\verse then you shall bring out \textit{both of them}\lebnote{Literally “the two of them”} to the gate of that city, and you shall stone them with stones so that they shall die, the young woman because she did not cry out in the town, and the man because\lebnote{Literally “because of the fact that”} he violated his neighbor’s wife; and so you shall purge the evil from your midst.
\verse “But if the man finds the young engaged woman in the field and the man overpowers her and \textit{he has sex with her},\lebnote{Literally “he lies down with her”} then the man only\lebnote{Or “alone”} must die who lay\lebnote{Or “slept”} with her.
\verse But you shall not do anything\lebnote{Hebrew “a thing”} to the young woman, for there is not reckoned against\lebnote{Hebrew “to”} the young woman \textit{a sin deserving death};\lebnote{Literally “a sin of death”} \textit{it is similar to when}\lebnote{Literally “for as that”} a man rises up against his neighbor and murders him, \textit{a fellow human being},\lebnote{Literally “a soul/individual person”} just so is this \textit{case},\lebnote{Literally “thing/matter”}
\verse for he found her in the field, the engaged young woman cried out, but there was no\lebnote{Hebrew “there was not”} rescuer \textit{to help her}.\lebnote{Literally “for her”}
\verse “If a man finds a young woman, a virgin who is not engaged, and he seizes her and \textit{he has sex with her}\lebnote{Literally “he lies with her”} and they are caught,
\verse then \textit{the man who lay with her}\lebnote{Literally “the man the one lying with her”} shall give to the father of the young woman fifty shekels of silver, and she shall become \textit{his wife}\lebnote{Literally “for/as a wife”} \textit{because}\lebnote{Literally “in place of”} he violated her, and he is not allowed to divorce her \textit{during his lifetime}.\lebnote{Literally “all of his days”}
\verse \lebnote{Deuteronomy 22:30–23:25 in the English Bible is 23:1–26 in the Hebrew Bible} A man may not take the wife of his father, and so \textit{he may not dishonor his father}.\lebnote{Literally “not he may reveal the skirt of the garment of his father”}
\end{biblechapter}

\begin{biblechapter} % Deuteronomy 23
\verse “No man \textit{with crushed testicles}\lebnote{Literally “bruised of crushing”} or whose \textit{male organ is cut off}\lebnote{Literally “cut off of male organ”} may come into the assembly of Adonai.
\verse An illegitimate child may not come into the assembly of Adonai; even to the tenth generation none \textit{of his descendants}\lebnote{Literally “to him”} may come into the assembly of Adonai.
\verse An Ammonite or a Moabite may not come into the assembly of Adonai; even to the tenth generation none \textit{of his descendants}\lebnote{Literally “to him”} may come into the assembly of Adonai \textit{forever},\lebnote{Literally “until eternity”}
\verse \textit{because}\lebnote{Literally “because of the event when”} they did not come to meet you with food and with water \textit{when you came out of Egypt},\lebnote{Literally “at your going out of Egypt”} and also \textit{because}\lebnote{Literally “because of the event when”} they\lebnote{Hebrew “he,” but with plural meaning} hired Balaam, son of Beor, from Pethor, in Aram Naharaim\lebnote{Or “Mesopotamia” = “between the rivers”} to act against you to curse you.
\verse But Adonai your God was not willing to listen to Balaam, and Adonai your God turned the curse into a blessing for you, because Adonai your God loved you.
\verse You shall not promote their welfare or their prosperity all your days \textit{forever}.\lebnote{Literally “until eternity”}
\verse “You shall not abhor an Edomite, because he is your brother; you shall not abhor an Egyptian because you were an alien in his land.
\verse The children\lebnote{Or “sons”} that are born to them in the third generation may come \textit{representing them}\lebnote{Literally “for them” or possibly “become members of”} in the assembly of Adonai.
\verse “If\lebnote{Or “when”} you go out to encamp against your enemies, then you shall guard against doing anything evil.
\verse “If there is among you a man that is not clean because of a seminal emission \textit{during the night},\lebnote{Literally “of the night”} he shall go outside the camp; he shall not come within the camp.
\verse \textit{And then}\lebnote{Literally “And it will happen/come about that”} toward the \textit{coming}\lebnote{Literally “turning”} of the evening, he shall bathe with\lebnote{Hebrew “in” but this is instrumental use of this preposition} water, and at the going down\lebnote{The verb can mean “coming/going” and here it means “going”} of the sun, he may come to the midst of the camp.
\verse “And there shall be for you a designated place outside the camp; \textit{and you shall go there to relieve yourself},\lebnote{Literally “and you shall go out there, outside to relieve yourself”}
\verse and a digging tool\lebnote{Or “spade”} shall be included in addition to your other utensils\lebnote{Hebrew “utensil”} for yourself; \textit{and then}\lebnote{Literally “and it will happen”} \textit{when you relieve yourself}\lebnote{Literally “at your sitting”} outside the camp you shall dig with it, and then you shall turn, and you shall cover your excrement.
\verse For Adonai your God is walking about in the midst of your camp to deliver you and \textit{to hand your enemies over to you before you},\lebnote{Literally “to give your enemies to the face of you”} and so let your camp be holy, so that he shall not see in it \textit{anything indecent},\lebnote{Literally “nakedness of a thing”} and he shall turn away \textit{from going with you}.\lebnote{Literally “from behind/after you”}
\verse “And you shall not hand over a slave to his master who has escaped and fled to you from his master.
\verse He shall reside with you in your midst in the place that he chooses in one of \textit{your towns wherever he pleases};\lebnote{Literally “in your gates good for him”} you shall not oppress him.
\verse “No woman \textit{of Israel}\lebnote{Literally “from the daughters of Israel”} shall be a temple prostitute, and no man \textit{of Israel}\lebnote{Literally “from the sons of Israel”} shall be a male shrine prostitute.
\verse You may not bring the \textit{hire}\lebnote{Literally “gift”} of a prostitute or \textit{the earnings of a male prostitute}\lebnote{Literally “the earnings of a dog”} into the house of Adonai your God, for any vow offerings, because \textit{both}\lebnote{Literally “both, the two of them”} are a detestable thing to Adonai your God.
\verse “\textit{You shall not charge your brother interest on money},\lebnote{Literally “You shall not lend on interest to your brother, interest of money”} interest on\lebnote{Hebrew “of”} food, or interest on\lebnote{Hebrew “of”} anything that one could lend\lebnote{Or “can lend”} on interest.
\verse You may lend on interest to the foreigner, but to your countryman\lebnote{Or “brother”} you may not lend on interest, so that Adonai your God may bless you \textit{in all your undertakings}\lebnote{Literally “in all the sendings out of your hand”} in \textit{the land where you are going},\lebnote{Literally “the land that you are going there”} \textit{in order to take possession of it}.\lebnote{Literally “to take possession”}
\verse “\textit{When you make a vow}\lebnote{Literally “When/if you vow a vow”} to Adonai your God, you shall not postpone \textit{fulfillment of it},\lebnote{Literally “to fulfill it”} for certainly Adonai your God shall require it from you and if postponed \textit{you will incur guilt}.\lebnote{Literally “it shall become against you as sin”}
\verse And \textit{if you refrain from vowing},\lebnote{Literally “if you refrain to make a vow”} \textit{you shall not incur guilt}.\lebnote{Literally “it shall not become against you as sin”}
\verse The utterance of your lips \textit{you must perform diligently}\lebnote{Literally “you must observe and you shall do”} just as you have vowed freely to Adonai your God whatever it was that you promised\lebnote{Or “spoke”} with your mouth.
\verse “When\lebnote{Or “If”} you come into the vineyard of your neighbor, then you may eat grapes \textit{as you please}\lebnote{Literally “according to your desire”} and \textit{until you are full},\lebnote{Literally “to your satiation”} but you shall not put any into your container.
\verse “When\lebnote{Or “If”} you come into the standing grain of your neighbor, then you may pluck ears with your hand, but you may not \textit{swing}\lebnote{Literally “wave”} a sickle among the standing grain of your neighbor.”
\end{biblechapter}

\begin{biblechapter} % Deuteronomy 24
\verse “When\lebnote{Or “If”} a man takes a wife and he marries her \textit{and then}\lebnote{Literally “and it will happen”} \textit{she does not please him},\lebnote{Literally “if not she finds favor in his eyes”} because he found \textit{something objectionable}\lebnote{Literally “shameful/repulsive thing”} and writes her a letter of divorce and puts it in her hand and sends her away from his house,
\verse and she goes from his house, and she goes out and becomes a wife \textit{for another man},\lebnote{Literally “for a man other”}
\verse and then the second man dislikes her and he writes her a letter of divorce and places it into her hand and sends her from his house, or if the second man dies who took her \textit{to himself}\lebnote{Literally “to for him”} as a wife,
\verse her first husband who sent her away is not allowed \textit{to take her again}\lebnote{Literally “to return to take her”} to become a wife to him after she has \textit{been defiled},\lebnote{Literally “become unclean”} for that is a detestable thing \textit{before}\lebnote{Literally “to the face of”} Adonai, and so you shall not mislead into sin the land that Adonai your God is giving to you as an inheritance.
\verse “When\lebnote{Or “If”} a man takes a new wife he shall not go out with the army, and \textit{he shall not be obligated with anything};\lebnote{Literally “he shall not come across come over upon him to anything”} he shall be free from obligation, \textit{to stay at home}\lebnote{Literally “for his house”} for one year, and he shall bring joy to his wife that he took.
\verse “A person\lebnote{Hebrew “he”} shall not take\lebnote{Or “require”} a pair of millstones or an upper millstone, for \textit{he is taking necessities of life as a pledge}.\lebnote{Literally “for a life he is taking as a pledge”}
\verse “If a man is \textit{caught}\lebnote{Literally “found”} kidnapping somebody from among his countrymen,\lebnote{Or “brothers”} the \textit{Israelites},\lebnote{Literally “sons/children of Israel”} and he treats him as a slave or he sells him, then that kidnapper shall die, and so you shall purge the evil \textit{from among you}.\lebnote{Literally “from your midst”}
\verse Be watchful\lebnote{Or “Be careful”} \textit{with respect to}\lebnote{Literally “against”} an outbreak of any infectious skin disease, by being very careful and by acting\lebnote{Or “doing”} according to all that the priests and the Levites have instructed you, just as I have commanded them, \textit{so you shall diligently observe}.\lebnote{Literally “so you shall observe to do”}
\verse So remember what Adonai your God did to Miriam on the journey \textit{when you went out from Egypt}.\lebnote{Literally “in you to go out from Egypt”}
\verse “When you make a loan to your neighbor, a loan of any kind, you shall not go into his house \textit{to take his pledge}.\lebnote{Literally “to pledge with respect to his pledge”}
\verse You shall wait outside, and the man to whom you are lending, he shall bring the pledge outside to you.
\verse And if he is a needy man, you shall not sleep in his pledge.\lebnote{“His pledge” refers to “a garment given as pledge”}
\verse You shall certainly return the pledge to him \textit{as the sun sets},\lebnote{Literally “as/at the moment of the going out of the sun”} so that he may sleep in his cloak and may bless you, and it shall be considered righteousness \textit{on your behalf}\lebnote{Literally “to you”} \textit{before}\lebnote{Literally “to the face of”} Adonai your God.
\verse “You shall not exploit a hired worker, who is needy and poor, from among your fellow men or from among your aliens\lebnote{Hebrew “alien”} who are in your land and in your \textit{towns}.\lebnote{Literally “gates”}
\verse On his day you shall give his wage, and the sun shall not go \textit{down},\lebnote{Literally “over him”} because he is poor and \textit{his life depends on it};\lebnote{Literally “and to him it is a lifting up with respect to his life/soul”} do this so that he does not cry out against you to Adonai, \textit{and you incur guilt}.\lebnote{Literally “and it becomes against you as sin}
\verse “Fathers shall not be put to death because of their children, and children shall not be put to death because of their fathers; each one shall be put to death for his own sin.
\verse You shall not subvert the rights of an alien or an orphan, and you shall not take as pledge the garment of a widow.
\verse And you shall remember that you were a slave in Egypt and that Adonai your God redeemed you from there; therefore I am commanding you to do this commandment.
\verse “When\lebnote{Or “If”} you reap your harvest in your field and you forget a sheaf in the field, you shall not return to get it, for it shall be for the alien, for the orphan, and for the widow, so that Adonai your God may bless you in all the work of your hands.
\verse When you beat off the fruit of your olive trees you shall not search through the branches afterward, for it\lebnote{That is, what is left} shall be for the alien, for the orphan, and for the widow.
\verse When you harvest grapes, you shall not glean your vineyards \textit{again};\lebnote{Literally “behind/after you”} it\lebnote{That is, what is left} shall be for the alien, for the orphan, and for the widow.
\verse And you shall remember that you were a slave in the land of Egypt, therefore I am commanding you to do this thing.”
\end{biblechapter}

\begin{biblechapter} % Deuteronomy 25
\verse “When a legal dispute \textit{takes place}\lebnote{Literally “shall be”} between men and they come near to the court, and the judges judge with respect to them, then they shall declare the righteous to be in the right and they shall condemn the wicked,
\verse then it will happen if the guilty one \textit{deserves beating},\lebnote{Literally “is a son of beating”} then the judge shall make him lie, and he shall beat him \textit{before him},\lebnote{Literally “to the face of him”} \textit{according to}\lebnote{Literally “as”} \textit{the prescribed number of lashes proportionate to the offense}.\lebnote{Literally “as is sufficient/necessary, with respect to number, for his wickedness/wicked behavior”}
\verse He may beat him with forty lashes, and he shall not do more than these, so that he will not beat more in addition to these many blows,\lebnote{Hebrew “blow”} and your countryman\lebnote{Or “brother”} would be\lebnote{Hebrew “is”} degraded before your eyes.
\verse “You shall not muzzle an ox \textit{when he is threshing}.\lebnote{Literally “at his threshing”}
\verse “When brothers dwell together and one of them dies and has no son, the wife of the deceased shall not become the wife of a \textit{man of another family};\lebnote{Others: “strange man,” “man outside the family” (NEB)} her brother-in-law \textit{shall have sex with her},\lebnote{Literally “shall go to her”} and he shall take her \textit{to himself}\lebnote{Literally “to/for him”} as a wife, and he shall perform his duty as a brother-in-law with respect to her.
\verse And then the firstborn that she bears \textit{shall represent his dead brother},\lebnote{Literally “he shall stand upon the name of his brother the deceased”} so that his name is not blotted out from Israel.
\verse But if the man does not want to take his sister-in-law, then his sister-in-law shall go up to the gate, to the elders, and she shall say, ‘My brother-in-law refused \textit{to perpetuate his brother’s name}\lebnote{Literally “to cause a name to stand for his brother”} in Israel, for he is not willing \textit{to marry me}.’\lebnote{Literally “to consummate the marriage with the widow of a his brother”}
\verse Then the elders of his town shall summon him and speak to him, and if he persists and says, ‘\textit{I do not desire to}\lebnote{Literally “I do not delight in”} marry her’
\verse then his sister-in-law shall go near him before the eyes of the elders, and she shall pull off his sandal from his foot, and she shall spit in his face, and she shall \textit{declare}\lebnote{Literally “respond/answer”} and she shall say, ‘This is how it is done to the man who does not build the house of his brother.’
\verse And his \textit{family}\lebnote{Literally “name”} shall be called in Israel, ‘The house where the sandal was pulled off.’\lebnote{Or “the house of the man whose sandal was pulled off”}
\verse “If a man and his brother fight each other and the wife of the one man comes near to rescue her husband from the hand of his attacker and she stretches out her hand and she seizes his genitals,
\verse then you shall cut off her hand; your eye shall not take pity.
\verse “There shall not be \textit{for your use}\lebnote{Literally “to you”} in your bag \textit{two kinds of stone weights, a large one and a small one}.\lebnote{Literally “stone and stone large and small”}
\verse There shall not be in your house \textit{for your use}\lebnote{Literally “for you”} \textit{two kinds of measures}.\lebnote{Literally “ephah and ephah large and small”}
\verse Rather a full and honest weight shall be \textit{for your use};\lebnote{Literally “for you”} there shall be for you a full and honest \textit{measure},\lebnote{Literally “an ephah”} so that your days on the land that Adonai your God is giving to you may be long.
\verse For detestable to\lebnote{Or “for”} Adonai your God is everyone who is doing such things,\lebnote{Or “are all whoare doing”} everyone who is acting\lebnote{Or “all whoare acting”} dishonestly.
\verse “Remember what Amalek did to you on the journey when\lebnote{Or “at/in”} you went out from Egypt,
\verse that he met you on the journey and attacked you, all those lagging behind you and when you were weary and worn out, and he did not fear God.
\verse \textit{And when}\lebnote{Literally “And it will happen when”} Adonai your God gives rest to you from all your enemies from around about you in the land that Adonai your God is giving to you as an inheritance to take possession of it, you shall blot out the remembrance of Amalek from under the heavens; you shall not forget!”
\end{biblechapter}

\begin{biblechapter} % Deuteronomy 26
\verse “\textit{And then}\lebnote{Literally “And it shall happen”} when you come to the land that Adonai your God is giving to you as an inheritance, and you take possession of it and you settle in it,
\verse then you shall take from the firstfruit of all the fruit of the ground that you harvest from your land that Adonai your God is giving to you, and you shall put it in a basket, and you shall go to the place that Adonai your God will choose to make his name to dwell there.
\verse And you shall go to the priest who is in office in those days, and you shall say, ‘I declare \textit{today}\lebnote{Literally “the day”} to Adonai your God that I have come into the land that Adonai swore to our ancestors\lebnote{Or “fathers”} to give to us.’
\verse Then the priest takes the basket from your hand and places it \textit{before}\lebnote{Literally “to the face of”} the altar of Adonai your God.
\verse And \textit{you shall declare}\lebnote{Literally “answer”} and you shall say \textit{before}\lebnote{Literally “to the face of”} your God, ‘My ancestor\lebnote{Or “father”} was a wandering Aramean, and he went down to Egypt, and there he dwelt as an alien \textit{few in number},\lebnote{Literally “in people only a few”} and there he became a great nation, mighty and numerous.
\verse And the Egyptians treated us badly, and they oppressed us and imposed on us hard labor.
\verse And we cried to Adonai, the God of our ancestors,\lebnote{Or “fathers”} and Adonai heard our voice and saw our affliction and our toil and our oppression.
\verse And Adonai brought us out from Egypt with a strong hand and an outstretched arm and with great terror and with signs and with wonders.
\verse And he brought us to this place and gave to us this land, a land flowing with milk and honey.
\verse And now, look, I am bringing\lebnote{Hebrew “I bring”} the firstfruit of the fruit of the ground that you gave to me, Adonai,’ and you shall place it \textit{before}\lebnote{Literally “to the face of”} Adonai your God, and you shall bow down \textit{before}\lebnote{Literally “to the face of”} Adonai your God.
\verse And you shall celebrate with all of the bounty that Adonai your God gave to you and to your family, you and the Levite and the alien who is in your midst.
\verse “When you are finished \textit{giving a tithe},\lebnote{Literally “to give a tenth”} all of the tithe of your produce in the third year, the year of the tithe, then you shall give to the Levite, to the alien, to the orphan, and to the widow, so that they may eat in your towns and they may be satisfied.
\verse And you shall say \textit{before}\lebnote{Literally “to the face of”} Adonai your God, ‘I have removed the sacred portion from the\lebnote{Or “my” house since the definite article can show possession here} house and, moreover, I have given it to the Levite and to the alien and to the orphan and to the widow according to all your commandment that you commanded me; I have not transgressed any of your commandments, and I have not forgotten any of them.
\verse I have not eaten during my time of mourning, and I have not removed anything from it while being unclean, and I have not offered \textit{any of it}\lebnote{Literally “from it”} to someone who has died. I have listened to the voice of Adonai my God; I have done all that you commanded me to do.
\verse Look down from the dwelling place of your holiness, from heaven, and bless your people Israel, and the land that you have given to us, as you swore to our ancestors,\lebnote{Or “our fathers”} a land flowing with milk and honey.’
\verse “This day Adonai your God is commanding you to do these rules and regulations, \textit{and you must observe them diligently}\lebnote{Literally “and you shall observe and you shall do them”} with all your heart and with all your soul.\lebnote{Or “inner self”}
\verse Adonai you have declared \textit{today}\lebnote{Literally “the day”} to be for you as your God, and to go\lebnote{Or “to walk”} in his ways and to observe his rules and his commandments and his regulations and to listen to his voice.
\verse And Adonai has declared you \textit{today}\lebnote{Literally “the day”} to be for him as a people, a treasured possession, as he \textit{promised}\lebnote{Literally “spoke”} to you, and that you are to observe all his commandments,
\verse \textit{and that he then will set you}\lebnote{Literally “and to set you”} high above all the nations that he has made for his praise and \textit{for fame}\lebnote{Literally “for a name”} and for honor and \textit{for you to be a holy people}\lebnote{Literally “for you to be/your being a people holy”} to Adonai your God, as he \textit{promised}.”\lebnote{Literally “spoke”}
\end{biblechapter}

\begin{biblechapter} % Deuteronomy 27
\verseWithHeading{Ceremonies} Then Moses and the elders of Israel charged the people, \textit{saying},\lebnote{Literally “to say”} “Keep all of the commandment that I am commanding you \textit{today}.\lebnote{Literally “the day”}
\verse And then on the day that you cross the Jordan to the land that Adonai your God is giving to you, then you shall set up \textit{for yourselves}\lebnote{Literally “for you”} large stones, and you shall paint\lebnote{Or “coat/cover them”} them with lime,
\verse and you shall write on them all the words of this law at your crossing, so that you may come into the land that Adonai your God is giving to you, a land flowing with milk and honey, as Adonai, the God of your ancestors,\lebnote{Or “fathers”} \textit{promised}\lebnote{Literally “spoke”} to you.
\verse And \textit{when you cross the Jordan},\lebnote{Literally “at/in your crossing the Jordan”} you shall set up these stones that I am commanding you about \textit{today}\lebnote{Literally “the day”} on Mount Ebal, and you shall paint\lebnote{Or “coat/cover”} them with lime.
\verse And you shall build an altar there for Adonai your God, an altar of stone, but\lebnote{Or “and”} \textit{you shall not use an iron tool to shape the stones}.\lebnote{Literally “and you shall not wave to and fro over them an iron tool”}
\verse You must build the altar of your God with unhewn stones, and you shall sacrifice on it burnt offerings to Adonai your God.
\verse And you shall sacrifice fellowship offerings, and you shall eat them there, and you shall rejoice \textit{before}\lebnote{Literally “to the face of”} Adonai your God.
\verse You shall write on the stone all of the words of this law very clearly.”
\verse Then Moses and the priests, the Levites, spoke to all Israel, saying,\lebnote{Literally “to say”} “Be silent and hear, Israel, for this day you have become \textit{a people}\lebnote{Literally “for a people”} for Adonai your God.
\verse And listen to the voice of Adonai your God, and observe his commandments and his rules that I am commanding you \textit{today}.”\lebnote{Literally “the day”}
\verseWithHeading{Blessings and Curses} And Moses charged the people on that day, \textit{saying},\lebnote{Literally “to say”}
\verse “These tribes shall stand on Mount Gerizim to bless the people \textit{when you cross}\lebnote{Literally “at/in crossing your”} the Jordan: Simeon and Levi and Judah and Issachar and Joseph and Benjamin.
\verse And these shall stand on Mount Ebal for delivering the curse: Reuben, Gad and Asher and Zebulun, Dan and Naphtali.
\verse And \textit{the Levites shall declare},\lebnote{Literally “they shall answer the Levites”} and they shall say to each man of Israel with a loud voice,
\verse ‘Cursed be the man that makes a divine image or a cast image, which is a detestable thing for Adonai, the work of the hand of a skilled craftsman, and then sets it in a hiding place.’\lebnote{Or “a secret place”} And \textit{all the people shall respond},\lebnote{Literally “and they shall answer all of the people”} ‘Amen.’
\verse ‘Cursed be the one who dishonors his father or his mother.’ And all of the people shall say, ‘Amen.’
\verse ‘Cursed be the one who moves the boundary marker of his neighbor.’ And all the people shall say, ‘Amen.’
\verse ‘Cursed be the one who misleads a blind person on the road.’ And all the people shall say, ‘Amen.’
\verse ‘Cursed be the one who deprives the alien, the orphan, and the widow of justice.’ And all the people shall say, ‘Amen.’
\verse ‘Cursed be the one who lies with the wife of his father, because \textit{he has dishonored his father’s bed}.’\lebnote{Literally “because he uncovered the hem of his father”} And all the people shall say, ‘Amen.’
\verse ‘Cursed be the one who lies with any kind of animal.’ And all the people shall say, ‘Amen.’
\verse ‘Cursed be the one who lies with his sister, the daughter of his father or the daughter of his mother.’ And all the people shall say, ‘Amen.’
\verse ‘Cursed be the one who lies with his mother-in-law.’ And all the people shall say, ‘Amen.’
\verse ‘Cursed be the one who strikes down his neighbor in secret.’ And all the people shall say, ‘Amen.’
\verse ‘Cursed be the one who takes a bribe \textit{to murder an innocent person}.’\lebnote{Literally “to strike down an individual person in blood”} And all the people shall say, ‘Amen.’
\verse ‘Cursed be the one \textit{who does not keep}\lebnote{Literally “who not keeps”} the words of this law, to observe them.’ And all the people shall say, ‘Amen.’ ”
\end{biblechapter}

\begin{biblechapter} % Deuteronomy 28
\verse “And it will happen that if you indeed listen to the voice of Adonai your God, \textit{to diligently observe}\lebnote{Literally “to observe to do”} all his commandments that I am commanding you \textit{today},\lebnote{Literally “the day”} then Adonai your God will set you above all the nations of the earth.
\verse And all of these blessings shall come upon you, and \textit{they shall have an effect on you}\lebnote{Literally “they shall overtake you”} if you listen to the voice of Adonai your God:
\verse “You will be blessed in the city, and you will be blessed in the field.
\verse “Blessed will be the fruit of your womb and the fruit of your ground and the fruit of your livestock, \textit{the calf of your cattle and the lambs of your flock}.\lebnote{Literally “what is dropped of your cattle and the offspring of your flocks”}
\verse “Blessed will be your basket and your kneading trough.
\verse “Blessed will you be \textit{when you come in and blessed will you be when you go out}.\lebnote{Literally “in/at your coming and you will be blessed in/at your going out”}
\verse “Adonai will cause your enemies \textit{who rise up against you to be defeated before you};\lebnote{Literally “the ones rising against you being defeated to the face of you”} on one road\lebnote{Or “way”} they shall come out against you, but on seven roads\lebnote{Or “ways”} they shall flee \textit{before you}.\lebnote{Literally “to the face of you”}
\verse Adonai will command \textit{concerning you}\lebnote{Literally “with you”} the blessing to be in your barns and \textit{in all your endeavors};\lebnote{Literally “in all of the sending forth of your hand”} and he will bless you in the land that Adonai your God is giving to you.
\verse Adonai will establish you for \textit{himself}\lebnote{Literally “to/for him”} as a holy people as he has sworn to you, if you keep the commandments of Adonai your God and you walk in his ways.
\verse And all of the peoples of the earth shall see that \textit{by the name of Adonai you are called},\lebnote{Literally “that the name of Adonai is called/assigned upon you”} and \textit{they shall fear you}.\lebnote{Literally “they shall be afraid/fearful from you”}
\verse And \textit{Adonai will make you successful and prosperous},\lebnote{Literally “Adonai will cause for you goodness”} in the fruit of your womb and on the land that Adonai swore to your ancestors\lebnote{Or “fathers”} to give to you.
\verse Adonai shall open for you his \textit{rich}\lebnote{Literally “good”} storehouse, even the heavens, to give the rain for\lebnote{Hebrew “of”} your land in its time and to bless all of the work of your hand, and you will lend to many nations; you will not borrow from them.
\verse And Adonai shall make you\lebnote{Literally “place/set you up”} as head and not the tail, and you shall be only at the top of the nations, and you shall not be at the bottom, if you listen to the commandments of Adonai your God that I am commanding you \textit{today}\lebnote{Literally “the day”} \textit{and diligently observe them}.\lebnote{Literally “to observe them and to do them”}
\verse And you shall not turn aside from \textit{any of}\lebnote{Literally “all of”} the words that I am commanding you \textit{today}\lebnote{Literally “the day”} to the right or left by going after other gods to serve them.
\verse “\textit{And then}\lebnote{Literally “and it will happen”} if you do not listen to the voice of Adonai your God \textit{by diligently observing}\lebnote{Literally “to observe and to do”} all of his commandments and his statutes that I am commanding you \textit{today},\lebnote{Literally “the day”} then all of these curses shall come upon you, and they shall overtake you:
\verse “You shall\lebnote{All of these curses have a future orientation that points to Israel in the land of Canaan} be cursed in the city, and you shall be cursed in the field.
\verse “Your basket shall be cursed and your kneading trough.
\verse “The fruit of your womb shall be cursed and the fruit of you ground, the calves of your cattle and the lambs of your flock.
\verse “You shall be cursed \textit{when you come in},\lebnote{Literally “at/in your coming”} and you shall be cursed \textit{when you go out}.\lebnote{Literally “at/in your going out”}
\verse “Adonai will send upon you\lebnote{Others translate as a wish: “may Adonai send upon you …”} the curse,\lebnote{These words are translated variously, e.g., “starvation,” “thirst,” “rebuke/dysentery” (NEB) or “curses,” “confusion,” “frustration” (NLT)} the panic,\lebnote{Or “confusion”} and the threat\lebnote{Or “rebuke”} \textit{in everything that you undertake},\lebnote{Literally “in all the sending out of your hand that you do”} \textit{until you are destroyed and until you perish quickly}\lebnote{Literally “until your to be destroyed and until your to perish quickly”} \textit{because of}\lebnote{Literally “from the face of the evil of your deeds”} the evil of your deeds in that\lebnote{Literally “which”} you have forsaken me.
\verse Adonai will cause the plague to cling to you \textit{until it consumes you}\lebnote{Literally “it to consume to”} from the land that you are going to,\lebnote{Hebrew “to there”} to take possession of it.
\verse Adonai will afflict you with the wasting diseases and with the fever and with the inflammation and with the scorching heat and with the sword\lebnote{Others translate “drought” (NLT, NEB)} and with the blight and with the mildew, and they shall pursue you \textit{until you perish}.\lebnote{Literally “until your/you to perish”}
\verse And your heavens that are over your heads shall be like bronze, and the earth that is under you shall be like iron.
\verse Adonai will change the rain of your land to fine dust and to sand; from the heaven it shall come down upon you \textit{until you are destroyed}.\lebnote{Literally “until your being destroyed”}
\verse “Adonai shall cause you to be defeated \textit{before}\lebnote{Literally “to the face of”} your enemies; on one road you shall go \textit{against}\lebnote{Literally “to/toward”} them,\lebnote{Hebrew “him”} but you will flee on seven roads \textit{before}\lebnote{Literally “to the face of”} them,\lebnote{Hebrew “him”} and you shall become a thing of horror to all of the kingdoms of the earth.
\verse And your dead bodies shall be as food for all of the birds of the heaven and to the animals of the earth, and there shall not be \textit{anyone to frighten them away}.\lebnote{Literally “one causing fright to them”}
\verse “Adonai shall afflict you with the boils of Egypt and with tumors and with the scurvy and with \textit{the skin rash that cannot be healed}.\lebnote{Literally “the skin rash that not it is able it to be healed”}
\verse Adonai shall afflict you with madness and with blindness and with confusion of heart.
\verse And you shall be groping at noon just as the blind person gropes in the dark, and you shall not succeed in finding your way, and you shall only be abused and robbed \textit{all the time},\lebnote{Literally “all the days”} and there will not be \textit{anyone who will rescue you}. \lebnote{Literally “one who delivers/rescues you”}
\verse You shall become engaged to a woman, but another man shall sleep\lebnote{Or “ravish/violate”} with her; you shall build a house, but you shall not live in it; a vineyard you shall plant, but you shall not enjoy it.
\verse Your ox shall be slaughtered before your eyes, and you shall not eat it; your donkey shall be stolen right \textit{before you},\lebnote{Literally “to the face of you”} and it shall not be returned to you; your sheep and your goats shall be given to your enemies, and \textit{there shall not be anyone who rescues you}.\lebnote{Literally “there shall not be for you one who rescues”}
\verse Your sons and your daughters shall be given to other people, and \textit{you will be looking on}\lebnote{Literally “your eyes will see/look”} \textit{longingly}\lebnote{Literally “wearing out”} for\lebnote{Hebrew “to”} them all day, \textit{but you will be powerless to do anything}.\lebnote{Literally “but there will not be for power of your hand”}
\verse A people that you do not know shall consume the harvest of your land and all your labor, and you will be only oppressed and crushed \textit{for the rest of your lives}.\lebnote{Literally “all of the days”}
\verse You shall become mad \textit{because of what your eyes shall see}.\lebnote{Literally “from/by the sight of your eyes”}
\verse Adonai shall strike you with grievous boils on the knees and on the upper thighs from which \textit{you will not be able to be healed},\lebnote{Literally “which not you are able to heal”} from the sole of your foot and up to your crown.
\verse Adonai will bring you and your king whom you set up over you to a nation that you or your ancestors\lebnote{Or “fathers”} have not known, and there you will serve other gods of wood and stone.
\verse And you will become a horror and a proverb and ridicule among all the peoples where Adonai drives you there.
\verse “You shall carry out much seed to the field, but you shall gather little produce, for the locust shall devour it.
\verse You shall plant vineyards and you shall dress\lebnote{Or “cultivate”} them, but you shall not drink wine and you shall not gather grapes, for the worm shall eat it.\lebnote{That is, the produce}
\verse There shall be olive trees for you in all of your territory, but you shall not anoint yourself, for your olives\lebnote{Hebrew “olive”} shall drop off.
\verse You shall bear sons and daughters, but they shall not be \textit{yours},\lebnote{Literally “for you”} for they shall go into captivity.
\verse The cricket shall take possession of all your trees and the fruit of your ground.
\verse The alien that is in your midst shall ascend over you, higher and higher, but you shall go down lower and lower.
\verse He shall lend to you, but you shall not lend to him; he shall be the head, but you shall be the tail.
\verse “And all of these curses shall come over you, and they shall pursue you, and they shall overtake you \textit{until you are destroyed},\lebnote{Literally “until you to be destroyed”} because \textit{you did not listen}\lebnote{Literally “not you listened”} to the voice of Adonai your God, \textit{by observing}\lebnote{Literally “to observe”} his commandments and his statutes that he commanded you.
\verse And they shall be among you as a sign and as a wonder and among your offspring \textit{forever}.\lebnote{Literally “until eternity,” but not in a timeless, philosophic sense}
\verse “\textit{Because}\lebnote{Literally “Under”} of the fact that you did not serve Adonai your God with joy and with gladness of heart for the abundance of everything,
\verse then you shall serve your enemies, whom Adonai will send against you under conditions of famine, in thirst, in nakedness, and in a lack of everything; and he shall place\lebnote{Literally “give”} a yoke of iron on your neck \textit{until he has destroyed you}.\lebnote{Literally “until his destroying of you”}
\verse Adonai will raise up against you a nation from far off, from the end of the earth, attacking as the eagle swoops down, \textit{a nation whose language you will not understand},\lebnote{Literally “a nation who not you will understand its language”}
\verse \textit{a grim-faced nation}\lebnote{Literally “a nation fierce/determined of face”} \textit{who does not show respect}\lebnote{Literally “who not lifts up faces”} to the old and the young and does not show pity.
\verse And it\lebnote{That is, the invading nation} shall consume the fruit of your livestock and the fruit of your ground \textit{until you are destroyed},\lebnote{Literally “until your destroying”} and who will not leave for you any grain, wine, and olive oil, \textit{calves of your herds},\lebnote{Literally “what is dropped of your herds”} and \textit{lambs of}\lebnote{Literally “the offspring of small animals”} your flock \textit{until it has destroyed you}.\lebnote{Literally “until his to destroy you”}
\verse And it shall besiege you in all your towns \textit{until your high and fortified walls fall},\lebnote{Literally “until the coming down of your walls, high and fortified”} which you are trusting in\lebnote{Hebrew “in them,” which is redundant} \textit{throughout your land};\lebnote{Literally “in all of your land”} and it shall besiege you in all of your towns in all of your land that Adonai your God has given to you.
\verse And you shall eat the fruit of your womb, the flesh of your sons and your daughters, whom Adonai your God gave to you, \textit{during the siege and during the distress}\lebnote{Literally “in siege and in distress”} your enemy inflicts upon you.
\verse The most refined and the very sensitive\lebnote{Or “kindly disposed”} man among you \textit{shall be mean with his brother}\lebnote{Literally “shall be bad/evil his eye against his brother”} and \textit{against his beloved wife}\lebnote{Literally “against the wife of his lap”} and against the rest\lebnote{Or “remainder”} of his children that he has left over,
\verse \textit{by refraining from giving}\lebnote{Literally “from giving”} to even one of them any of the meat of his children that he eats, because there is not anything that is left over for him \textit{during the siege and distress}\lebnote{Literally “in the siege and distress”} that your enemy inflicts upon you.
\verse The most refined and the most delicate woman among you, who shall\lebnote{Or “would”} not venture to put the sole of her foot on the ground from being so delicate and from such gentleness, \textit{shall be mean to her beloved husband}\lebnote{Literally “shall be bad her eye against the husband of her lap”} and against her son and against her daughter,
\verse and even concerning her afterbirth \textit{that goes out}\lebnote{Literally “the going out”} from between her feet and also concerning her children that she bears, because she eats them for lack of anything in secret \textit{during the siege and during the distress}\lebnote{Literally “in the siege and in the distress”} that your enemy inflicts upon her in your \textit{towns}.\lebnote{Literally “gates”}
\verse “If \textit{you do not diligently observe}\lebnote{Literally “you observe to do”} all the words of this law written in this scroll by revering this glorious and awesome name, Adonai your God,
\verse then Adonai shall overwhelm you with your plagues and the plagues of your offspring, severe plagues and lasting illnesses, grievous and enduring.
\verse And he shall bring back upon you all the diseases of Egypt concerning which \textit{you were in dread}\lebnote{Literally “you dreaded”} \textit{because of them}.\lebnote{Literally “from their presence”}
\verse Also any illness and any plague\lebnote{Or “each illness and each plague”} that is not written in the scroll of this law, he shall bring them, Adonai, upon you until you are destroyed.
\verse And you shall remain \textit{only a few people}\lebnote{Literally “with people of few”} in place of the fact you were formerly as the stars of heaven as far as number is concerned, because you did not listen to the voice of Adonai your God.
\verse \textit{And then}\lebnote{Literally “And it will happen”} as Adonai delighted over you \textit{to make you prosperous}\lebnote{Literally “to do good with you”} to make you numerous, so Adonai shall delight over you to exterminate you\lebnote{Or “to cause you to perish”} and to destroy you, and so you shall be plucked from the land that you are going there to take possession of it.
\verse And Adonai shall scatter you among all the nations from one end of the earth up to the other end of the earth, and there you shall serve other gods that you have not known nor\lebnote{Hebrew “and”} your ancestors,\lebnote{Or “your fathers”} gods of wood and stone.
\verse And among these nations you shall not find rest, and there shall not be a resting place for the sole of your foot, and Adonai shall give you there an anxious heart and a weakening of eyes\lebnote{Or “a failure of eyes”} and a languishing of your inner self.\lebnote{Or “soul”}
\verse And \textit{your life shall hang in doubt before you},\lebnote{Literally “they shall be your life hanging for you in front”} and you shall be startled\lebnote{Or “in fear”} night and day, and you shall not be confident of your life.
\verse In the morning you shall say, ‘\textit{If only it was evening}!’\lebnote{Literally “Who shall give evening?”} and in the evening you shall say ‘\textit{If only it was morning}!’\lebnote{Literally “Who shall give morning?”} because of the dread of your heart that you shall feel, and because of the sight of your eyes that you shall see.
\verse And Adonai shall bring you back to Egypt in ships by the route that I \textit{promised}\lebnote{Literally “spoke”} to you that ‘\textit{You shall not see it again}!’\lebnote{Literally “you shall not do again further to see it/her”} And you shall sell yourself there to your enemies as slaves and as female slaves, but there will not be a buyer.”
\end{biblechapter}

\begin{biblechapter} % Deuteronomy 29
\verseWithHeading{Covenant Renewal, Oaths, Restoration, Charges to the Nation} \lebnote{Deuteronomy 29:1–29 in the English Bible is 28.69–29:28 in the Hebrew Bible} These are the words of the covenant that Adonai commanded Moses to make with the \textit{Israelites}\lebnote{Literally “sons/children of Israel”} in the land of Moab \textit{besides}\lebnote{Literally “from to alone”} the covenant that he made with them at Horeb.
\verse And Moses summoned all of Israel and said to them, “You saw all that Adonai did before your eyes in the land of Egypt and to Pharaoh and to all his servants and to all his land;
\verse that is, the great trials that your eyes saw, and those great signs and wonders.
\verse But Adonai has not given to you a heart to understand, or eyes to see, or ears to hear, even \textit{to this day}.\lebnote{Literally “until the day the this”}
\verse And I have led you forty years\lebnote{Hebrew “year”} in the desert; your clothes have not worn out \textit{on you},\lebnote{Literally “from on you”} and your sandal has not worn out \textit{on your foot}.\lebnote{Literally “from on your foot”}
\verse You have not eaten bread, and you have not drunk wine and strong drink, so that you may know that I am Adonai your God.
\verse And when you came to this place then Sihon the king of Heshbon, and Og the king of Bashan, came out to meet you for battle, and we defeated them.
\verse And we took their land and gave it as an inheritance to the Reubenites\lebnote{Hebrew “Reubenite”} and to the Gadites\lebnote{Hebrew “Gadite”} and to the half-tribe of Manasseh.
\verse And \textit{you must diligently observe the words of this covenant},\lebnote{Literally “you must keep the words of the covenant the this and you must do them”} so that you may succeed in all that you do.
\verse “You are standing \textit{today},\lebnote{Literally “the day”} all of you, \textit{before}\lebnote{Literally “to the face of”} Adonai your God, your leaders, your tribes, your elders, and your officials, all the men of Israel,
\verse your little children, your women and your aliens\lebnote{Hebrew “alien”} who are in the midst of your camp, from the choppers\lebnote{Hebrew “chopper”} of your wood to the drawers\lebnote{Hebrew “drawer”} of your water,
\verse \textit{in order for you to enter into the covenant of Adonai your God},\lebnote{Literally “for you to go over into the covenant of Adonai your God”} and into\lebnote{Or “with”} his oath that Adonai your God is \textit{making with you}\lebnote{Literally “cutting with you”} \textit{today},\lebnote{Literally “the day”}
\verse in order to establish you \textit{today}\lebnote{Literally “the day”} \textit{to himself}\lebnote{Literally “for him”} as a people and so that he may be for you as God, just as he \textit{promised}\lebnote{Literally “spoke”} to you and \textit{just as}\lebnote{Literally “according to that which”} he swore to your ancestors,\lebnote{Or “your fathers”} to Abraham, to Isaac, and to Jacob.
\verse “Now I am not \textit{making this covenant}\lebnote{Literally “cutting this covenant”} and this oath \textit{with you alone}.\lebnote{Literally “with you to alone you”}
\verse But with \textit{whoever is standing here}\lebnote{Literally “with who he is here with us standing”} with us \textit{today}\lebnote{Literally “the day”} \textit{before}\lebnote{Literally “to the face of”} Adonai our God, and with \textit{whoever is not standing here}\lebnote{Literally “with who he is not standing here with us”} with us \textit{today}.\lebnote{Literally “the day”}
\verse For you know how we lived in the land of Egypt and how we traveled through the midst of the nations that you traveled\lebnote{Or “journeyed”} through.
\verse And you have seen their detestable things and their idols of wood and stone, silver, and gold that were among them,
\verse so that \textit{there is not}\lebnote{Literally “lest there be/develop”} among you a man or a woman or a clan or a tribe \textit{whose heart}\lebnote{Literally “his heart”} turns \textit{today}\lebnote{Literally “the day”} from being with Adonai our God to go to serve the gods of these nations, so that there is not among you a root sprouting poison and wormwood.
\verse And then when he hears the words of this oath, then\lebnote{Or “and”} \textit{he will assure himself}\lebnote{Literally “he will bless himself”; HALOT 160 suggests “to consider oneself fortunate”} in his heart, \textit{saying},\lebnote{Literally “to say”} ‘\textit{Safety shall be mine even though I go in the stubbornness of my heart},’\lebnote{Literally “Peace shall happen/be for me, although/even if in the stubbornness of my heart I go”} thereby destroying the well-watered land along with the parched.\lebnote{Some translators prefer to include the last clause as a part of the words of the wicked man (NASV vs. NEB)}
\verse Adonai will not be willing to forgive him, for by then the anger of Adonai will smoke, and his passion against that man and all the curses written in this scroll will descend on him, and Adonai will blot out his name from under heaven.
\verse And Adonai will single him out for calamity out of all the tribes of Israel, according to all the curses of the covenant written in the scroll of this law.
\verse “And the next generation, that is, your children who will rise up after you, and the foreigner who will come from a distant land, when\lebnote{Hebrew “and”} they will see the plagues of that land and its diseases that Adonai has inflicted upon it, will say,
\verse ‘All its land is brimstone and salt left by fire, \textit{none of its land will be sown},\lebnote{Literally “all of its land will not be sown”} and it will not make plants sprout out and it will not grow any vegetation; it is as the destruction of Sodom and Gomorrah, Adman and Zeboiim, which Adonai overturned in his anger and in his wrath.’
\verse And all the nations will say, ‘\textit{Why}\lebnote{Literally “On what basis”} has Adonai done \textit{such a thing}\lebnote{Literally “so”} to this land? What caused the fierceness of this great anger?’
\verse And they will say,\lebnote{Or “answer/respond”} ‘It is because they abandoned the covenant of Adonai, the God of their ancestors,\lebnote{Or “fathers”} which he \textit{made}\lebnote{Literally “cut”} with them \textit{when he brought them out}\lebnote{Literally “at/in his to bring them out from the land of Egypt”} from the land of Egypt.
\verse And they went and served other gods and bowed down to them, gods whom they did not know them and he\lebnote{That is, Adonai} had not allotted to them.
\verse So \textit{the anger of Adonai was kindled}\lebnote{Literally “became hot the nose of Adonai”} against that land to bring upon it all the curses written in this scroll,
\verse and Adonai uprooted them from their land in anger and in wrath and in great fury, and he cast them into another land, \textit{just as it is today}.’\lebnote{Literally “as the day the this”}
\verse “The hidden things \textit{belong to Adonai}\lebnote{Literally “are for Adonai”} our God, but the revealed things \textit{belong to us}\lebnote{Literally “are for us”} to know and to our children \textit{forever},\lebnote{Literally “until eternity”} in order to do all the words of this law.”
\end{biblechapter}

\begin{biblechapter} % Deuteronomy 30
\verse “And then when all of these things come upon you, the blessing and the curse that I have set \textit{before you}\lebnote{Literally “to the face of you”} \textit{and you call them to mind}\lebnote{Literally “and you bring them back to your heart”} among the nations there where Adonai your God has scattered you,
\verse and you return to Adonai and you listen to his voice according to all that I am commanding you \textit{today},\lebnote{Literally “the day”} both you and your children,\lebnote{Or “sons”} with all your heart and with all your inner self,\lebnote{Or “soul”}
\verse and Adonai your God will restore your fortunes, and he will have compassion upon you, and \textit{he will again gather you together}\lebnote{Literally “he will do again and he will gather you together”} from all the peoples where Adonai your God scattered you there.
\verse “Even if \textit{you are outcasts}\lebnote{Literally “he/it shall be outcasting your”} at the end of the heavens, even from there Adonai your God shall gather you, and from there \textit{he shall bring you back}.\lebnote{Literally “he shall take/fetch you”}
\verse And Adonai your God will bring you to the land that your ancestors\lebnote{Or “fathers”} had taken possession of,\lebnote{Hebrew “of it”} and he will make you successful, and he will make you more numerous than your ancestors.\lebnote{Or “fathers”}
\verse “And Adonai your God will circumcise your heart and the heart of your offspring to love Adonai your God with all your heart and with all your inner self\lebnote{Or “soul”} \textit{so that you may live}.\lebnote{Literally “for the sake of your life”}
\verse And Adonai your God will put all these curses on your enemies and \textit{on those who hate you},\lebnote{Literally “the haters of you”} on \textit{those who harassed you}.\lebnote{Literally “who pursued after/persecuted you”}
\verse And \textit{you will again listen}\lebnote{Literally “you will return and you will listen”} to the voice of Adonai, and you will do all his commandments that I am commanding you \textit{today}.\lebnote{Literally “the day”}
\verse And Adonai your God will make you prosperous \textit{in all your undertakings},\lebnote{Literally “in all of the work of your hand”} and in the fruit of your livestock and in the fruit of your ground \textit{abundantly},\lebnote{Literally “to/for good”} for Adonai \textit{will again rejoice}\lebnote{Literally “he will return ... to rejoice”} over you, just as he rejoiced over your ancestors.\lebnote{Or “fathers”}
\verse He will do this if you listen to\lebnote{Or “if you obey/hearken to”} the voice of Adonai your God by keeping his commandment and his statutes written in the scroll of this law and if you return to Adonai your God with all your heart and with all your inner self.\lebnote{Or “soul”}
\verse “For this commandment that I am commanding you \textit{today}\lebnote{Literally “the day”} is \textit{not too wonderful for you},\lebnote{Literally “not is wonderful it from you”} and it is not too far from you.
\verse It is not in the heavens \textit{so that you might say},\lebnote{Literally “to say”} ‘Who will go up for us to the heavens and get it for us and cause us to hear it, so that we may do it?’
\verse And it is \textit{not beyond the sea},\lebnote{Literally “is not from beyond the sea”} \textit{so that you might say},\lebnote{Literally “to say”} ‘Who will cross for us to the other side of the sea and take it for us and cause us to hear it, so that we may do it?’
\verse But the word is very near you, even in your mouth and in your heart, \textit{so that you may do it}.\lebnote{Literally “to do it”}
\verse “See, I am setting \textit{before you}\lebnote{Literally “to the face of you”} \textit{today}\lebnote{Literally “the day”} life and prosperity and death and disaster;
\verse what I am commanding you \textit{today}\lebnote{Literally “the day”} is to love Adonai your God by going\lebnote{Or “walking”} in his ways and by keeping his commandments and his statutes and his regulations, and then you will live, and you will become numerous, and Adonai your God will bless you in the \textit{land where you are going}.\lebnote{Literally “the land where you are going there”}
\verse However, if your heart turns aside and you do not listen and you are lured away and you bow down to other gods and you serve them,
\verse I declare to you \textit{today}\lebnote{Literally “the day”} that you will certainly perish; \textit{you will not extend your time}\lebnote{Literally “you will not make long days”} on the land that you are crossing the Jordan to go there to take possession of it.
\verse I invoke as a witness against you \textit{today}\lebnote{Literally “the day”} the heaven and the earth: life and death I have set \textit{before you},\lebnote{Literally “to the face of you”} blessing and curse. So choose life, so that you may live, you and your offspring,\lebnote{Or “seed”}
\verse by loving Adonai your God by listening to his voice and by clinging to him, for he is your life and the length of your days in order for you to live on the land that Adonai swore to your ancestors,\lebnote{Or “fathers”} to Abraham, to Isaac, and to Jacob, to give to them.”
\end{biblechapter}

\begin{biblechapter} % Deuteronomy 31
\verseWithHeading{Succession, Deposition, and Recitation of Text} And Moses went and spoke these words to all Israel.
\verse And he said to them, “I am a hundred and twenty years old \textit{today};\lebnote{Literally “the day”} I am not able to go out and to come in any longer, and Adonai said to me, ‘You may not cross this Jordan.’
\verse Adonai your God is about to cross \textit{before you};\lebnote{Literally “to the face of you”} he will destroy these nations \textit{before you},\lebnote{Literally “to the face of you”} and you shall dispossess them. Joshua will be crossing \textit{before you},\lebnote{Literally “to the face of you”} just as Adonai \textit{promised}.\lebnote{Literally “spoke”}
\verse And Adonai will do to them just as he did to Sihon and to Og, kings of the Amorites, and to their land, which he destroyed with them.
\verse And Adonai \textit{will deliver them to you before you},\lebnote{Literally “will give them Adonai to the face of you”} and you shall do to them according to every commandment that I have commanded you.
\verse Be strong and be courageous; you should not be afraid, and you \textit{should not be in dread from their presence},\lebnote{Literally “not you should dread from their face”} for Adonai your God is the one going with you; he will not leave you alone and he will not forsake you.”
\verse Then Moses summoned Joshua, and he said to him \textit{in the presence of all Israel},\lebnote{Literally “before the eyes of all Israel”} “Be strong and be courageous, for you will go with this people into the land that Adonai swore to their ancestors\lebnote{Or “their fathers”} to give to them, and you will give it to them as an inheritance.
\verse Adonai is the one going \textit{before you};\lebnote{Literally “to the face of you”} he will be with you, and he will not leave you alone, and he will not forsake you; you shall not be afraid, and you shall not be discouraged.”
\verse So Moses wrote this law, and he gave it to the priests, the descendants\lebnote{Or “sons”} of Levi, the ones carrying the ark of the covenant of Adonai, and to all the elders of Israel.
\verse Then Moses commanded them, \textit{saying},\lebnote{Literally “to say”} “At the end of seven years, in the time of the year for canceling debts during the Feast of Booths,
\verse \textit{when all Israel comes to appear before}\lebnote{Literally “at/in the coming of all of Israel to see”} Adonai their God at the place that he will choose, you shall read this law before all Israel \textit{in their hearing}.\lebnote{Literally “in their ears”}
\verse Assemble the people, the men and the women and the little children and your aliens that are in your \textit{towns},\lebnote{Literally “gates”} so that they may hear and so that they may learn and they may revere Adonai your God, and \textit{they shall diligently observe}\lebnote{Literally “they shall keep to do”} all the words of this law.
\verse And then their children,\lebnote{Hebrew “child”} who have not known, they too may hear, and they may learn to revere Adonai their God all the days \textit{that you live}\lebnote{Literally “that you are alive”} on the land that you are crossing the Jordan \textit{to get there}\lebnote{Literally “there”} to take possession of it.”
\verse Then Adonai said to Moses, “Look, \textit{you are about to die};\lebnote{Literally “are near your days to die”} call Joshua and present yourselves in the tent of assembly,\lebnote{Or “meeting”} so that I may instruct him.” So Moses and Joshua went and presented themselves in\lebnote{Or “at”} the tent of assembly.\lebnote{Or “meeting”}
\verse And Adonai appeared in the tent in a column of cloud, and the column of the cloud stood at the entrance of the tent.
\verse And Adonai said to Moses, “Look, you are about to lie down with your ancestors,\lebnote{Or “fathers”} and this people will arise and they\lebnote{Hebrew “he”} will play the prostitute after \textit{the foreign gods}\lebnote{Literally “the gods of the foreigner/foreign person of the land”} of the land \textit{to which they are going}.\lebnote{Literally “which he is going to”}
\verse And \textit{my anger shall flare up against them}\lebnote{Literally “shall become my nose against it/him”} on that day, and I will forsake them, and I will hide my face from them, and they\lebnote{Hebrew “he”} shall become as prey, and disasters and troubles shall find\lebnote{Or “encounter”} them,\lebnote{Hebrew “him”} and they\lebnote{Hebrew “he”} shall say on that day, ‘Is it not because our\lebnote{Hebrew “my” but represents all the people} God is not in our\lebnote{Hebrew “my”} midst that these disasters have found\lebnote{Or “encountered”} us?’\lebnote{Hebrew “me”}
\verse But I will certainly hide my face on that day, because of all of the evil that they\lebnote{Hebrew “he”} did because they\lebnote{Hebrew “he”} turned to other gods.
\verse “And so then write this song for yourselves and teach it to the \textit{Israelites};\lebnote{Literally “sons/children of Israel”} put it in their mouth, so that this song may be for me as a witness against the \textit{Israelites}.\lebnote{Literally “sons/children of Israel”}
\verse For when I bring them\lebnote{Hebrew “him,” referring to the people} into the land that I swore to their\lebnote{Hebrew “his”} ancestors,\lebnote{Or “fathers”} a land flowing with milk and honey, \textit{they will eat their fill},\lebnote{Literally “he will eat and he will be satisfied”} and they\lebnote{Hebrew “he”} will become fat, and then they\lebnote{Hebrew “he”} will turn to other gods, and they will serve them, and they will spurn me, and they\lebnote{Hebrew “he”} will break my covenant.
\verse \textit{And then}\lebnote{Literally “And it shall happen”} many disasters and troubles will come upon them,\lebnote{Hebrew “him”} and this song will give evidence before them\lebnote{Hebrew “him”} as a witness, because it will not be forgotten from out of the mouth of their descendants,\lebnote{Hebrew “his offspring/descendant”} for I know their\lebnote{Hebrew “his”} inclination that they\lebnote{Hebrew “he”} are having \textit{today}\lebnote{Literally “the day”} before I have brought them\lebnote{Hebrew “him”} into the land that I swore.”
\verse And Moses wrote this song on that day and taught it to the \textit{Israelites}.\lebnote{Literally “sons/children of Israel”}
\verse Then he\lebnote{That is, Adonai} told Joshua the son of Nun, and said to him, “Be strong and be courageous, for you shall bring the \textit{Israelites}\lebnote{Literally “sons/children of Israel”} into the land that I swore to them, and I will be with you.”
\verse \textit{And then when Moses finished writing}\lebnote{Literally “And it happened as/when to finish Moses to write”} the words of this law on the scroll \textit{until they were complete},\lebnote{Literally “until their to be complete/”}
\verse then Moses commanded the Levites carrying the ark of the covenant of Adonai, \textit{saying},\lebnote{Literally “to say”}
\verse “Take the scroll of this law and put it at the side of the ark of the covenant of Adonai your God, and it will be there as a witness against you.
\verse For I know your rebellion and your stiff neck even now while I am still alive with you \textit{today},\lebnote{Literally “the day”} rebelling against Adonai, and \textit{how much more}\lebnote{Literally “and also for/indeed”} after my death.
\verse Assemble to me all the elders of your tribes and your officials, so that I may speak in their ears these words, and that I may call as witness against them heaven and earth.
\verse For I know that after my death you will certainly act corruptly, and you will turn aside from the way that I have commanded you, and the disaster in the future days will befall you, because you will do evil\lebnote{Hebrew “the evil”; the definite article indicates the general concept of evil} in the eyes of Adonai, \textit{provoking him with the work of your hands}.”\lebnote{Literally “to provoke him with/by the work of your hands”}
\verse So Moses spoke into the ears of the assembly of Israel the words of this song \textit{until they were complete}.\lebnote{Literally “until their to be complete”}
\end{biblechapter}

\begin{biblechapter} % Deuteronomy 32
\verseWithHeading{The Song of Moses} “Give ear, O heavens, and I will speak, 
and let the earth hear the words of my mouth.
\verse May my teaching trickle like the dew, 
my words like rain showers on tender grass, 
and like spring showers on new growth.
\verse For I will proclaim the name of Adonai; 
ascribe greatness to our God!
\verse The Rock, his work is perfect, 
for all his ways are just; 
he is a faithful God, and \textit{without injustice};\lebnote{Literally “there is not injustice”} 
righteous and upright is he.
\verse They have behaved corruptly toward\lebnote{Or “with”} him; 
they are not his children; this is their flaw, 
a generation crooked and perverse.
\verse Like this do you treat Adonai, 
foolish and \textit{unwise}\lebnote{Literally “not wise”} people? 
Has he not, your father, created you? 
He made you, and he established you.
\verse Remember the old days, \textit{the years long past};\lebnote{Literally “the years of from generation to generation”} 
ask your father, and he will inform you, 
your elders and they will tell you.\lebnote{Hebrew “to you”}
\verse \textit{When the Most High apportioned}\lebnote{Literally “In/at the apportioning of the Most High”} the nations, 
at his dividing up of the sons of humankind, 
he fixed the boundaries of the peoples, 
according to the number of the children of Israel.\lebnote{LXX reads “the number of the angels of God”; Dead Sea Scrolls reads “the number of the sons of God”}
\verse For Adonai’s portion was his people, 
Jacob the share of his inheritance.
\verse He found him in a desert land, 
and in a howling, desert wasteland; 
he \textit{encircled him},\lebnote{Literally “moved around him”} he cared for him, 
he protected him like the \textit{apple}\lebnote{Literally “pupil”} of his eye.
\verse As an eagle stirs up its nest, 
hovers over its young, 
spreads out its wings, takes them,\lebnote{Hebrew “it” but used poetically and with plural sense in context} 
carries them\lebnote{Hebrew “it” but used poetically and with plural sense in context} on its pinions,\lebnote{Hebrew “pinion”}
\verse so Adonai alone guided him,\lebnote{That is, Jacob, standing for Israel} 
and \textit{there was no foreign god accompanying him}.\lebnote{Literally “and there was not with him a god foreign/strange”}
\verse And he set him on the high places of the land, 
and he fed him the crops\lebnote{Hebrew “crop”} of the field, 
and he nursed him with honey from crags,\lebnote{Hebrew “crag”} 
and with oil from flinty rock,
\verse With curds\lebnote{Hebrew “curd”} from the herd, 
and with milk from the flock, 
with the fat of young rams, 
and rams, the offspring of Bashan, 
and with goats along with the finest kernels of wheat, 
and from the blood of grapes\lebnote{Hebrew “grape”} you drank fermented wine.\lebnote{Or “partially fermented wine”; others translate simply as wine (NASB, NEB); HALOT 330, “still fermenting wine”}
\verse And Jeshurun grew fat, and he kicked; 
you grew fat, you bloated, and you became obstinate; 
and he abandoned God, his maker, 
and he scoffed at the rock of his salvation.
\verse They made him jealous with strange gods; 
with detestable things they provoked him.
\verse They sacrificed to the demons, not God, 
to gods whom\lebnote{Hebrew “them”} they had not known, 
new gods who\lebnote{Hebrew “they,” understood in the verb form} came from recent times; 
\textit{their ancestors had not known them}.\lebnote{Literally “their fathers not knew them”}
\verse The rock who\lebnote{Hebrew “he,” understood in verb form} bore you, you neglected, 
and you forgot God, the one giving you birth.
\verse Then Adonai saw, and he spurned them, 
because of the provocation of his sons and his daughters.
\verse So he said, ‘I will hide my face from them; 
I will see what will be their end, 
for they are a generation of perversity, 
\textit{children in whom there is no faithfulness}.\lebnote{Literally “sons not faithfulness is in them”}
\verse They annoyed me with what is not a god; 
they provoked me with their idols. 
So I will make them jealous with those not a people, 
with a foolish nation I will provoke them.
\verse For a fire was kindled\lebnote{Or “ignited”} by my anger, 
and it burned \textit{up to the depths of Sheol},\lebnote{Literally “up to Sheol depths”} 
and it devoured the earth and its produce, 
and it set afire the foundation of the mountains.
\verse I will heap disasters upon them; 
my arrows I will spend on them.
\verse They will become weakened by famine, 
and consumed by plague and bitter pestilence; 
and the teeth of wild animals I will send against them, 
with the poison of the creeping things in the dust;
\verse From outside her boundaries the sword will make her childless, 
and from inside, terror; 
both for the young man and also the young woman, 
the infant along with the gray-headed man.
\verse I thought, “I will wipe them out; 
\textit{I will make people forget they ever existed}.”\lebnote{Literally “I will blot out from among human being their remembrance”}
\verse If I had not feared a provocation of the enemy, 
lest their foes might misunderstand, \lebnote{Or “so that they not make a false construal” of what has happened} 
lest they should say, \lebnote{Or “so that they might not”} “Our hand is 
\textit{triumphant},\lebnote{Literally “raised/held high”} and Adonai did not do all this.” ’
\verse For \textit{they are a nation void of sense},\lebnote{Literally “they a nation perishing of counsel”} 
and there is not any understanding in them.
\verse If only they were wise, they would understand this; 
they would discern \textit{for themselves their end}.\lebnote{Literally “they would discern for end their”}
\verse How could one chase a thousand 
and two could cause a myriad to flee, 
if their Rock had not sold them, 
and Adonai had not given them up?
\verse For the fact of the matter is, 
their rock is not like our Rock, 
and our enemies recognize\lebnote{The meaning of this expression is uncertain, but “discerns” or “judges” seem good choices; see HALOT 932, which allows for “judges” or “in the estimation of our enemies”} this.
\verse For their vine is from the vine of Sodom, 
and from the terraces of Gomorrah; 
their grapes are grapes of poison; 
\textit{their clusters are bitter}.\lebnote{Literally “clusters of bitter are for them”}
\verse Their wine is the poison of snakes, 
and the deadly poison of horned vipers.
\verse ‘Is not this stored up with me, 
sealed in my treasuries?’\lebnote{Or “storehouses}
\verse \textit{Vengeance belongs to me}\lebnote{Literally “To me is vengeance”} and also recompense, 
\textit{for at the time their foot slips},\lebnote{Literally “at the time when it shall slip foot their”} 
because the day of their disaster is near, 
\textit{and fate comes quickly for them}.’\lebnote{Literally “and comes quickly/hurries events to come to them”}
\verse For Adonai will judge on behalf of his people, 
and concerning his servants; 
he will change his mind when he sees that their power has disappeared, 
and there is no one left, confined\lebnote{Or “bond/bound”} or free.\lebnote{Or “freed”}
\verse And he will say, ‘Where are their gods, 
their rock in whom\lebnote{Hebrew “in him”} they took refuge?
\verse Who ate the fat of their sacrifices 
and drank the wine of their libations? 
Let them rise up, and let them help you; 
\textit{Let them be to you a refuge}.\lebnote{Literally “let there be/him/them unto you as a shelter”}
\verse See, now, that I, even I am he, 
and there is not a god besides me; 
I put to death and I give life; 
I wound and I heal; 
there is not one who delivers from my hand!
\verse For indeed I lift up my hand to heaven, 
And I promise \textit{as I live forever},\lebnote{Literally “live I to eternity”}
\verse When I sharpen\lebnote{Or “have sharpened”} \textit{my flashing sword},\lebnote{Literally “the flashing of my sword”} 
and my hand takes hold\lebnote{Or “seizes”} of it in judgment, 
\textit{I will take reprisals against my foes},\lebnote{Literally “I will let return vengeance to foes my”} 
and my haters I will repay.
\verse I will make my arrows drunk with blood, 
and my sword shall devour flesh with the blood of the slain, 
and captives\lebnote{Hebrew “captive”} from the heads\lebnote{Hebrew “head”} of the leaders of the enemy.’
\verse Call for songs of joy, O nations, concerning his people,\lebnote{Dead Sea Scrolls reads: “Rejoice, O heavenly ones, with him! Bow down, all you gods, before him!”} 
for the blood of his servants he will avenge, 
\textit{and he will take reprisals against his foes},\lebnote{Literally “vengeance he will let return to his foes”} 
and he will make atonement for his land, his people.”
\verse And Moses came, and he spoke\lebnote{Or “recited”} all the words of this song in the ears of the people; that is, he and Joshua the son of Nun.
\verse \textit{And when Moses finished speaking}\lebnote{Literally “And he finished Moses to speak”} all these words to all Israel,
\verse then he said to them, “\textit{Take to heart all the words}\lebnote{Literally “Put heart your to all the words”} that I am admonishing against you \textit{today}\lebnote{Literally “the day”} concerning which you should instruct them with respect to your children\lebnote{Or “sons”} \textit{so that they will observe diligently}\lebnote{Literally “to keep to do”} all the words of this law,
\verse for it is not a trifling matter among you, but it is your life, and through this word \textit{you will live long in the land}\lebnote{Literally “you will make long days on the land”} that you are about to cross the Jordan to get there to take possession of it.”
\verseWithHeading{Instructions Concerning Moses’ Death} And Adonai said to Moses on exactly this day, \textit{saying},\lebnote{Literally “to say”}
\verse “Go up to this mountain of the Abarim range, Mount Nebo, which is \textit{opposite Jericho},\lebnote{Literally “on the face of Jericho”} and see the land of Canaan that I am giving to the \textit{Israelites}\lebnote{Literally “sons/children of Israel”} as a possession.
\verse You shall die on that mountain that you are about to go up there, and you will be gathered to your people, just as your brother Aaron died on \textit{Mount Hor}\lebnote{Literally “on Hor the mountain”} and he was gathered to his people,
\verse because of the fact that you broke faith with me in the midst of Israel at the waters of Meribah Kadesh, in the desert of Zin, because \textit{you did not treat me as holy}\lebnote{Literally “that not you treated me as holy”} in the midst of the \textit{Israelites}.\lebnote{Literally “sons/children of Israel”}
\verse Yes, from afar you may view the land, but there you shall not enter there, that is, into the land that I am giving to the \textit{Israelites}.”\lebnote{Literally “sons/children of Israel”}
\end{biblechapter}

\begin{biblechapter} % Deuteronomy 33
\verseWithHeading{Blessings of Moses on Israel} Now\lebnote{Or “And”} this is the blessing with which Moses, the man of God, blessed the \textit{Israelites}\lebnote{Literally “sons/children of Israel”} \textit{before}\lebnote{Literally “to the face of”} his death.
\verse Then\lebnote{Or “And”} he said,
\verse “Adonai came from Sinai, 
and he dawned upon them from Seir; 
he shone forth from Mount Paran, 
and he came with myriads of holy ones, 
at his right hand a fiery law for them.\lebnote{This pointing of the MT is very difficult; here HALOT 93 suggests “fire a law for them” as a translation of the compound Hebrew word made up of “fire” + “law,” which I render as “a fiery law”}
\verse Moreover,\lebnote{Or “Indeed”} \textit{he loves his people},\lebnote{Literally “a lover of peoples”} 
all the holy ones were in your hand, 
and they bowed down to\lebnote{Or “at”} your feet, 
\textit{each one accepted directions from you}.\lebnote{Literally “he takes up from your words”}
\verse A law Moses instructed for us, 
as a possession for the assembly of Jacob.
\verse And then a king arose in Jeshurun, 
at the gathering of the leaders of the people, 
\textit{united were the tribes of Israel}.\lebnote{Literally “together the tribes of Israel”}
\verse “May Reuben live, and may he not die, 
\textit{and let his number not be few}.”\lebnote{Literally “let he be his people of number”}
\verse \textit{And he said this of Judah},\lebnote{Literally “and this concerning/to Judah, and he said”} 
“Hear, O Adonai, the voice of Judah, 
and bring him to his people; 
his own hands strive\lebnote{Or “is/are great” on his behalf} for him, 
and may you be a help \textit{against}\lebnote{Literally “from”} his foes.”
\verse And of Levi he said, 
“Your Thummim and your Urim 
are for \textit{your faithful one},\lebnote{Literally “man, your faithful one”} 
\textit{whom you tested at Massah};\lebnote{Literally “whom you tested him at Massah”} 
you contended with him 
at the waters of Meribah.
\verse The one saying of his father and of his mother, 
‘I have not regarded them,’ 
and his brothers he did not acknowledge, 
and his children\lebnote{Or “sons”} he did not know, 
but rather they observed your word, 
and your covenant they kept.
\verse They taught\lebnote{Or “teach/will teach”} your regulations to Jacob, 
and your law to Israel; 
they placed incense smoke \textit{before you},\lebnote{Literally “in your nose”} 
and whole burnt offerings on your altar.
\verse Bless, O Adonai, his substance, 
and with the work of his hands you must be pleased; 
smite the loins of those who attack him, 
and those hating him, \textit{so that they cannot arise}.”\lebnote{Literally “so that not they stand up”}
\verse Of Benjamin he said, 
“The beloved of Adonai dwells \textit{securely},\lebnote{Literally “in confidence”} 
the Most High\lebnote{This is the most likely reading in context (Most High God). Literally the Hebrew text reads “upon him,” so the NASV translates the first line as: dwell in security “by him”} shields all around him, 
all the day, 
and between his shoulders he dwells.”
\verse And of Joseph he said, 
“Blessed by Adonai is his land, 
with\lebnote{Hebrew “from”} the choice things of heaven, 
with\lebnote{Hebrew “from”} dew, and with\lebnote{Hebrew “from”} the deep lying down beneath,
\verse and with\lebnote{Hebrew “from”} the choice things of the fruits\lebnote{Or “produce of”} of the the sun, 
and with the choice things of the yield of the \textit{seasons},\lebnote{Literally “months”}
\verse and with\lebnote{Hebrew “from”} the finest things of the ancient mountains, 
and with\lebnote{Hebrew “from”} the choice things of the \textit{eternal hills},\lebnote{Literally “the hills of eternity never ending ages”}
\verse and with\lebnote{Hebrew “from”} the choice things of the earth and its fullness, 
and the favor of the one \textit{who dwelt}\lebnote{Literally “dwells” but context puts the event into the past; literally “dweller of the bush”} in the bush.\lebnote{Some scholars suggest that this expression must be read as: on Sinai} 
Let them come to the head of Joseph, 
and to the crown of the prince among his brothers.
\verse As the firstborn of his ox, majesty \textit{belongs to him},\lebnote{Literally “is for him”} 
and his horns are the horns of a wild ox; 
with them he drives people together,\lebnote{Or “all at once” (NASB)} 
and they are the myriads of Ephraim, 
and they are the thousands of Manasseh.”
\verse And of Zebulun he said, 
“Rejoice, Zebulun, in your going out, 
and rejoice, Issachar, in your tents;
\verse They summon people to the mountains;\lebnote{Hebrew “mountain”} 
there they sacrifice the sacrifices of righteousness, 
because the affluence of the seas they suck out, 
and \textit{the most hidden treasures of the sand}.”\lebnote{Literally “the covered of the hidden of the sand”}
\verse And of Gad he said, 
“Blessed be \textit{the one who enlarges Gad};\lebnote{Literally “the enlarger of Gad”} 
like a lion he dwells, 
and he tears an arm as well as a scalp.
\verse And he selected\lebnote{Or “provided”} the best part for himself, \lebnote{Hebrew “for him”} 
for there the portion of a ruler is included, 
and he came with the heads of the people;\lebnote{Or possibly, “the heads of the people came/assembled” (compare NEB)} 
he did\lebnote{Or “worked”} the righteousness of Adonai, 
and his regulations for Israel.”
\verse And of Dan he said, 
“Dan is a cub of a lion; 
he leaps from Bashan.”\lebnote{Hebrew “the Bashan,” referring to a well-known geographical and agricultural area}
\verse And of Naphtali, he said, 
“Oh, Naphtali, sated of favor, 
and full of the blessing of Adonai; 
take possession of the lake, 
and the land to the south.”
\verse And of Asher he said, 
“Blessed \textit{more than sons}\lebnote{Literally “blessed from/of sons is” features the comparative use the Hebrew preposition \textit{min}, “than”} is Asher; 
\textit{may he be the favorite}\lebnote{Literally “may he be favored”} of his brothers, 
\textit{dipping his feet in the oil}.\lebnote{Literally “and dipping in the oil his feet”}
\verse Your bars are\lebnote{Or “will be”} iron and bronze, 
and as your days, so is\lebnote{Or “so will be/may be”} your strength.”
\verse “There is no one like God, O, Jeshurun, 
who rides through the heavens to your help, 
and with his majesty through the skies.
\verse The God of \textit{ancient time}\lebnote{Literally “before/earlier times”} is a hiding place, 
and underneath are the arms of eternity,\lebnote{Or “everlasting ages” or “everlasting arms”} 
and he drove out \textit{from before}\lebnote{Literally “from your face”} you your enemy, 
and he said, ‘Destroy them!’
\verse So Israel dwells alone and carefree, 
the spring of Jacob in a land of grain and wine; 
his heavens even drip dew.
\end{biblechapter}

\begin{biblechapter} % Deuteronomy 34
\verseWithHeading{The Death of Moses and the Commissioning of Joshua} Then Moses went up from the desert plateau\lebnote{Or “plains/lowlands”} of Moab to Mount Nebo, to the top of Pisgah, \textit{which is opposite}\lebnote{Literally “which is on the face of”} Jericho, and Adonai showed him all of the land, Gilead all the way up to Dan,
\verse and all of Naphtali and the land of Ephraim and Manasseh and all of the land of Judah, up to the western sea,\lebnote{That is, the Mediterranean Sea}
\verse and the Negev and the plain of the valley of Jericho, the city of palms, on up to Zoar.
\verse And Adonai said to him, “This is the land that I swore to Abraham to Isaac and to Jacob, \textit{saying},\lebnote{Literally “to say”} ‘To your offspring I will give it.’ I have let you see it with your eyes, but you shall not cross \textit{into it}.”\lebnote{Literally “to there”}
\verse Then Moses, the servant of Adonai, died there in the land of Moab \textit{according to the command of Adonai}.\lebnote{Literally “at the mouth of Adonai”}
\verse And he\lebnote{That is, Adonai} buried him in the valley in the land of Moab opposite Beth Peor. But until this day no one knows his burial site.
\verse \textit{Now Moses was a hundred and twenty years old}\lebnote{Literally “and Moses was a son of a hundred and twenty year”} \textit{when he died};\lebnote{Literally “at/in his dying”} \textit{his sight was not impaired and his vigor had not abated}.\lebnote{Literally “not faded his eyes and not had fled away his vitality”}
\verse And the \textit{Israelites}\lebnote{Literally “sons/children of Israel”} wept concerning Moses thirty days; finally the days of weeping and mourning for Moses were completed.
\verse Now\lebnote{Or “And”} Joshua the son of Nun was full of the spirit of wisdom because Moses had placed his hands on him, and the \textit{Israelites}\lebnote{Literally “sons/children of Israel”} listened to him, and they did as Adonai had commanded Moses.
\verse And not again\lebnote{Or “since then”} has a prophet arisen in Israel like Moses, whom Adonai knew\lebnote{Hebrew “knew him”} face to face,
\verse as far as all the signs and the wonders Adonai sent him to do in the land of Egypt, against\lebnote{Hebrew “to”} Pharaoh and all of his servants and against\lebnote{Hebrew “to”} all of his land,
\verse and as far as all of \textit{the mighty deeds}\lebnote{Literally “the hand the strong”} and as far as \textit{the great awesome wonders}\lebnote{Literally “the terrifying the great actions”} Moses did before the eyes of all Israel.
\end{biblechapter}

\flushcolsend

\end{document}
