\documentclass[twoside,twocolumn,a4paper,10pt]{memoir}
\usepackage{lipsum}
\usepackage{fixltx2e}
\usepackage{xspace}
\usepackage[usenames,dvipsnames,svgnames,table]{xcolor}
\usepackage{lettrine}
\usepackage{flushend}
\usepackage{fancyhdr}
\usepackage[object=vectorian]{pgfornament}

\pagestyle{fancy}
\fancyhf{}
\fancyhead[RO,LE]{\rightmark}
\cfoot{\thepage}
\renewcommand{\headrulewidth}{0pt}

\usepackage{fontspec}
\setmainfont{equity}[
  % Files
  Path      = \string~/s/fonts/equity/ ,
  Extension = .otf ,
  % Fonts
  UprightFont     = Equity Text A Regular ,
  UprightFeatures = { SmallCapsFont = Equity Caps A Regular } ,
  BoldFont        = Equity Text A Bold ,
  BoldFeatures    = { SmallCapsFont = Equity Caps A Bold } ,
  ItalicFont      = Equity Text A Italic ,
  BoldItalicFont  = Equity Text A Bold Italic ,
  % Features
  Numbers = OldStyle ]

% old-style numbers don't look great as drop-caps
\newfontfamily{\lettrinefont}{lettrine}[
  % Files
  Path      = \string~/s/fonts/equity/ ,
  Extension = .otf ,
  % Fonts
  UprightFont     = Equity Text A Regular ,
  UprightFeatures = { SmallCapsFont = Equity Caps A Regular } ,
  BoldFont        = Equity Text A Bold ,
  BoldFeatures    = { SmallCapsFont = Equity Caps A Bold } ,
  ItalicFont      = Equity Text A Italic ,
  BoldItalicFont  = Equity Text A Bold Italic]

\setlength{\headwidth}{\textwidth}
\setlength{\columnseprule}{0pt}
\setlength{\columnsep}{30pt}
\setheadfoot{12pt}{12pt}
\setheaderspaces{*}{12pt}{*}
\clubpenalty10000
\widowpenalty10000

\usepackage{eso-pic}
\newcommand\AtPageUpperRight[1]{\AtPageUpperLeft{%
 \put(\LenToUnit{\paperwidth},\LenToUnit{0\paperheight}){#1}%
 }}%
\newcommand\AtPageLowerRight[1]{\AtPageLowerLeft{%
 \put(\LenToUnit{\paperwidth},\LenToUnit{0\paperheight}){#1}%
 }}%

\makeatletter
\newcommand\versenumcolor{red}
\newcommand\chapnumcolor{red}
\newlength{\biblechapskip}
  \setlength{\biblechapskip}{1em plus .33em minus .2em}
\newcounter{biblechapter}
\newcounter{bibleverse}[biblechapter]
\renewcommand\chaptername{Book}
\let\ltx@chapter=\chapter
\let\ltx@paragraph=\paragraph
\newcommand{\biblebook}[1]{%
  \gdef\currbook{#1}
  \ltx@chapter{#1}}
\newcount\biblechap@svdopt
\newenvironment{biblechapter}[1][\thebiblechapter]
  {\biblechap@svdopt=#1
  \ifnum\c@biblechapter=\biblechap@svdopt\else
    \advance\biblechap@svdopt by -1\fi
  \setcounter{biblechapter}{\the\biblechap@svdopt}
  \refstepcounter{biblechapter}
  \setbeforesecskip{1ex}\setbeforesubsecskip{1ex}
  \lettrine[lines=3,lhang=0,loversize=0.25]{\lettrinefont\color{\chapnumcolor}\thebiblechapter}{}\ignorespaces}
  {\par\vspace{\biblechapskip}\setbeforesecskip{0ex}\setbeforesubsecskip{0ex}}
\renewcommand{\verse}[1][\thebibleverse]{%
  \refstepcounter{bibleverse}%
  \markright{{\scshape\currbook} \thebiblechapter:\thebibleverse}%
  \ifnum\c@bibleverse=1\else{\color{\versenumcolor}\textbf{\thebibleverse}}\fi~}%
\makeatother

\makechapterstyle{dash-embiggened}{%
  \chapterstyle{default}
  \setlength{\beforechapskip}{5\onelineskip}
  \renewcommand*{\printchaptername}{}
  \renewcommand*{\chapternamenum}{}
  \renewcommand*{\chapnumfont}{\normalfont\Huge}
  \settoheight{\midchapskip}{\chapnumfont 1}
  \renewcommand*{\printchapternum}{\centering \chapnumfont
    \rule[0.5\midchapskip]{1em}{0.4pt} \thechapter\
    \rule[0.5\midchapskip]{1em}{0.4pt}}
  \renewcommand*{\afterchapternum}{\par\nobreak\vskip 0.5\onelineskip}
  \renewcommand*{\printchapternonum}{\centering
                 \vphantom{\chapnumfont 1}\afterchapternum}
  \renewcommand*{\chaptitlefont}{\normalfont\HUGE\scshape}
  \renewcommand*{\printchaptertitle}[1]{\centering \chaptitlefont ##1}
  \setlength{\afterchapskip}{2.5\onelineskip}}

\chapterstyle{dash-embiggened}

\setbeforesecskip{0ex}
\setaftersecskip{0.1ex}
\setsecheadstyle{\bfseries\raggedright\Large}

\setbeforesubsecskip{0ex}
\setaftersubsecskip{0.1ex}
\setsubsecheadstyle{\bfseries\itshape\raggedright\large}

\newcommand{\LORD}{\textsc{Lord}\xspace}

\title{The Holy Bible}
\date{}
\author{}

\begin{document}
\frontmatter

\begin{titlingpage}
\vspace*{\fill}

\newcommand{\framesize}{\textwidth}
\begin{tikzpicture}[color=Gold,
    transform shape,
    every node/.style={inner sep=0pt}]
  \node[minimum size=\framesize,fill=Beige!10](vecbox){};
  \node[anchor=north west] at (vecbox.north west){%
    \pgfornament[width=0.2*\framesize]{131}};
  \node[anchor=north east] at (vecbox.north east){%
    \pgfornament[width=0.2*\framesize,symmetry=v]{131}};
  \node[anchor=south west] at (vecbox.south west){%
    \pgfornament[width=0.2*\framesize,symmetry=h]{131}};
  \node[anchor=south east] at (vecbox.south east){%
    \pgfornament[width=0.2*\framesize,symmetry=c]{131}};
  \node[anchor=north] at (vecbox.north){%
    \pgfornament[width=0.6*\framesize,symmetry=h]{85}};
  \node[anchor=south] at (vecbox.south){%
    \pgfornament[width=0.6*\framesize]{85}};
  \node[anchor=north,rotate=90] at (vecbox.west){%
    \pgfornament[width=0.6*\framesize,symmetry=h]{85}};
  \node[anchor=north,rotate=-90] at (vecbox.east){%
    \pgfornament[width=0.6*\framesize,symmetry=h]{85}};
  \node[inner sep=6pt, color=black] (text) at (vecbox.center){%
    \HUGE \textsc{The Holy Bible}};
  \node[anchor=north, color=Goldenrod] (base) at (text.south){%
    \pgfornament[width=0.5*\framesize]{71}};
  \node[anchor=south, color=Goldenrod] at (text.north){%
    \pgfornament[width=0.5*\framesize,symmetry=h]{71}};
\end{tikzpicture}

\vspace*{\fill}
\end{titlingpage}

\AddToShipoutPictureBG{%
   \AtPageUpperLeft{\put(10,-35){\pgfornament[width=1.75cm,symmetry=h]{194}}}
   \AtPageUpperRight{\put(-60,-35){\pgfornament[width=1.75cm,symmetry=h]{195}}}
   \AtPageLowerLeft{\put(10,35){\pgfornament[width=1.75cm,symmetry=v]{195}}}
   \AtPageLowerRight{\put(-60,35){\pgfornament[width=1.75cm,symmetry=v]{194}}}
   }

\tableofcontents

\mainmatter
\part*{The Old Testament}

\biblebook{Genesis}

\begin{biblechapter} % Genesis 1
\verseWithHeading{The Creation} In the beginning, God created the heavens and the earth—
\verse Now the earth was formless and empty, and darkness \textit{was} over the face of the deep. And the Spirit of God \textit{was} hovering over the surface of the waters.
\verse And God said, “Let there be light!” And there was light.
\verse And God saw the light, that \textit{it was} good, and God caused \textit{there to be} a separation between the light and between the darkness.
\verse And God called the light Day, and the darkness he called Night. And there was evening and there was morning, \textit{the} first day.
\verse And God said, “Let there be a vaulted dome in the midst of the waters, and \textit{let it cause a separation between the waters}.”
\verse So God made the vaulted dome, and he caused a separation between the waters which \textit{were} under the vaulted dome and between the waters which were over the vaulted dome. And it was so.
\verse And God called the vaulted dome “heaven.” And there was evening, and there was morning, a second day.
\verse And God said, “Let the waters under heaven be gathered to one place, and let the dry ground appear.” And it was so.
\verse And God called the dry ground “earth,” and he called the collection of the waters “seas.” And God saw that \textit{it was} good.
\verse And God said, “Let the earth produce green plants \textit{that will} bear seed—fruit trees bearing fruit \textit{in which there is seed}—according to its kind, on the earth.” And it was so.
\verse And the earth brought forth green plants bearing seed according to its kind, and trees bearing fruit \textit{in which there was seed} according to its kind. And God saw that \textit{it was} good.
\verse And there was evening and there was morning, a third day.
\verse And God said, “Let there be lights in the vaulted dome of heaven \textit{to separate day from night}, and let them be as signs and for appointed times, and for days and years,
\verse and they shall be as lights in the vaulted dome of heaven to give light on the earth.” And it \textit{was} so.
\verse And God made two lights, the greater light to rule the day and the smaller light to rule the night, and the stars.
\verse And God placed them in the vaulted dome of heaven to give light on the earth
\verse and to rule over the day and over the night, and to \textit{separate light from darkness}. And God saw that \textit{it was} good.
\verse And there was evening and there was morning, a fourth day.
\verse And God said, “Let the waters swarm \textit{with} swarms of living creatures, and let birds fly over the earth across the face of the vaulted dome of heaven.
\verse So God created the great sea creatures and every living creature \textit{that} moves, \textit{with} which the waters swarm, according to their kind, and every bird \textit{with} wings according to its kind. And God saw that \textit{it was} good.
\verse And God blessed them, saying, “Be fruitful and multiply, and fill the waters in the seas, and let the birds multiply on the earth.”
\verse And there was evening, and there was morning, a fifth day.
\verse And God said, “Let the earth bring forth living creatures according to their kind: cattle and moving things, and wild animals according to their kind.” And it was so.
\verse So God made wild animals according to their kind and the cattle according to their kind, and every creeping thing of the earth according to its kind. And God saw that \textit{it was} good.
\verse And God said, “Let us make humankind in our image and according to our likeness, and let them rule over the fish of the sea, and over the birds of heaven, and over the cattle, and over all the earth, and over every moving thing that moves upon the earth.”
\verse So God created humankind in his image, in the likeness of God he created him, male and female he created them.
\verse And God blessed them, and God said to them, “Be fruitful and multiply, and fill the earth and subdue it, and rule over the fish of the sea and the birds of heaven, and over every animal that moves upon the earth.”
\verse And God said, “Look—I am giving to you every plant \textit{that} bears seed which \textit{is} on the face of the whole earth, and every kind of tree \textit{that bears fruit}. They shall be yours as food.”
\verse And to every kind of animal of the earth and to every bird of heaven, and to everything that moves upon the earth in which \textit{there is} life \textit{I am giving} every green plant as food.” And it was so.
\verse And God saw everything that he had made and, behold, \textit{it was} very good. And there was evening, and there was morning, a sixth day.
\end{biblechapter}

\begin{biblechapter} % Genesis 2
\verse And heaven and earth and all their array were finished.
\verse And on the seventh day God finished his work that he had done, and he rested on the seventh day from all his work that he had done.
\verse And God blessed the seventh day, and he sanctified it, because on it he rested from all his work \textit{of creating that \textit{there was} to do}.
\verseWithHeading{The Generations of Heaven and Earth} These are the generations of heaven and earth when they were created, in the day \textit{that} Adonai made earth and heaven—
\verse \textit{before any plant of the field was} on earth, and \textit{before} \textit{any plant of the field} had sprung up, because Adonai had not caused it to rain upon the earth, and there was no human being to cultivate the ground,
\verse but a stream \textit{would} rise from the earth and water the whole face of the ground—
\verse when Adonai formed the man \textit{of} dust from the ground, and he blew into his nostrils the breath of life, and the man became a living creature.
\verse And Adonai planted a garden in Eden in the east, and there he put the man whom he had formed.
\verse And Adonai caused to grow every tree \textit{that} was pleasing to the sight and good for food. And the tree of life \textit{was} in the midst of the garden, \textit{along with} the tree of the knowledge of good and evil.
\verse Now a river flowed out from Eden that watered the garden, and from there it diverged and became four branches.
\verse The name of the first \textit{is} the Pishon. It went around all the land of Havilah, where \textit{there is} gold.
\verse (The gold of that land \textit{is} good; bdellium and onyx stones \textit{are} there.)
\verse And the name of the second \textit{is} Gihon. It went around all the land of Cush.
\verse And the name of the third \textit{is} Tigris. It flows east of Assyria. And the fourth river \textit{is} the Euphrates.
\verse And Adonai took the man and set him in the garden of Eden to cultivate it and to keep it.
\verse And Adonai commanded the man, saying, “From every tree of the garden \textit{you may freely eat},
\verse but from the tree of the knowledge of good and evil you shall not eat, for in the day \textit{that you eat} from it \textit{you shall surely die}.”
\verse Then Adonai said, “\textit{it is} not good \textit{that} the man is alone. I will make for him a helper \textit{as his counterpart}.”
\verse And out of the ground Adonai formed every beast of the field and every bird of the sky, and he brought \textit{each} to the man to see what he would call it. And whatever the man called that living creature \textit{was} its name.
\verse And the man \textit{gave names} to every domesticated animal and to the birds of heaven and to all the wild animals. But for \textit{the} man there was not found a helper \textit{as his counterpart}.
\verse And Adonai caused a deep sleep to fall upon the man. While he slept, he took one of his ribs, and closed up \textit{the flesh where it had been}.
\verse And Adonai fashioned the rib which he had taken from the man into a woman and brought her to the man.
\verse And the man said, “\textit{She is now} bone from my bones 
and flesh from my flesh; 
\textit{she} shall be called ‘Woman,’ 
for \textit{she was taken} from man.”
\verse Therefore a man shall leave his father and his mother and shall cling to his wife, and they shall be as one flesh.
\verse And the man and his wife, both of them, were naked, and they were not ashamed.
\end{biblechapter}

\begin{biblechapter} % Genesis 3
\verseWithHeading{The Fall} Now the serpent was more crafty than any other \textit{wild animal} which Adonai had made. He said to the woman, “Did God indeed say, ‘You shall not eat from any tree in the garden’?”
\verse The woman said to the serpent, “From the fruit of the trees of the garden we may eat,
\verse but from the tree that is in the midst of the garden, God said, ‘You shall not eat from it, nor shall you touch it, lest you die’.”
\verse But the serpent said to the woman, “You shall not surely die.
\verse For God knows that on the day you \textit{both} eat from it, then your eyes will be opened and you \textit{both} shall be like gods, knowing good and evil.”
\verse When the woman saw that the tree \textit{was} good for food and that it \textit{was} a delight to the eyes, and the tree was desirable to make \textit{one} wise, then she took from its fruit and she ate. And she gave \textit{it} also to her husband with her, and he ate.
\verse Then the eyes of both of them were opened, and they knew that they \textit{were} naked. And they sewed together fig leaves and they made for themselves coverings.
\verse Then they heard the sound of Adonai walking in the garden \textit{at the windy time of day}. And the man and his wife hid themselves from the presence of Adonai among the trees of the garden.
\verse And Adonai called to the man and said to him, “Where \textit{are} you?”
\verse And he replied, “I heard the sound of you in the garden, and I was afraid because I \textit{am} naked, so I hid myself.”
\verse Then he asked, “Who told you that you \textit{were} naked? Have you eaten from \textit{the tree from which I forbade you to eat}?”
\verse And the man replied, “The woman whom you gave \textit{to be} with me—she gave to me from the tree and I ate.”
\verse Then Adonai said to the woman, “What \textit{is} this you have done?” And the woman said, “The serpent deceived me, and I ate.”
\verse Then Adonai said to the serpent,
\verse “Because you have done this, 
you \textit{will be} cursed 
more than any domesticated animal 
and more than any wild animal. 
On your belly you shall go 
and dust you shall eat 
all the days of your life.
\verse To the woman he said, “I will greatly increase 
\textit{your pain in childbearing}; 
in pain you shall bear children. 
And to your husband \textit{shall be} your desire. 
And he shall rule over you.”
\verse And to Adam he said, “Because you listened to the voice of your wife and you ate from the tree \textit{from which I forbade you to eat},
\verse the ground \textit{shall be} cursed on your account. 
In pain you shall eat \textit{from} it 
all the days of your life.
\verse And thorns and thistles shall sprout for you, 
and you shall eat the plants of the field.
\verse And the man \textit{named} his wife Eve, because she was the mother of all life.
\verse And Adonai made for Adam and for his wife garments of skin, and he clothed them.
\verse And Adonai said, “Look—the man has become as one of us, to know good and evil. \textit{What if} he stretches out his hand and takes also from the tree of life and eats, and lives forever?”
\verse And Adonai sent him out from the garden of Eden, to till the ground from which he was taken.
\verse So he drove the man out, and placed cherubim east of the garden of Eden, and \textit{a flaming, turning sword} to guard the way to the tree of life.
\end{biblechapter}

\begin{biblechapter} % Genesis 4
\verseWithHeading{Cain and Abel} Now Adam knew Eve his wife, and she conceived and bore Cain. And she said, “I have given birth to a man with \textit{the help of} Adonai.”
\verse \textit{Then she bore} his brother Abel. And Abel became a keeper of sheep, and Cain became a tiller of the ground.
\verse And \textit{in the course of time} Cain brought an offering from the fruit of the ground to Adonai,
\verse and Abel also brought \textit{an offering} from \textit{the choicest firstlings of his flock}. And Adonai looked with favor to Abel and to his offering,
\verse but to Cain and to his offering he did not look with favor. And Cain became very angry, and his face fell.
\verse And Adonai said to Cain, “Why are you angry, and why is your face fallen?
\verse If you do well \textit{will I not accept you}? But if you do not do well, sin is crouching at the door. And its desire \textit{is} for you, but you must rule over it.”
\verse Then Cain said to his brother Abel, \textit{“Let us go out into the field.”} And when they were in the field, Cain rose up against his brother Abel and killed him.
\verse Then Adonai said to Cain, “Where \textit{is} Abel your brother?” And he said, “I do not know; am I my brother’s keeper?”
\verse And he said, “What have you done? The voice of your brother’s blood is crying out to me from the ground.
\verse So now you are cursed from the ground, which has opened its mouth to receive the blood of your brother from your hand.
\verse When you till the ground \textit{it shall no longer yield its strength to you}. You shall be a wanderer and a fugitive on the earth.”
\verse And Cain said to Adonai, “My punishment \textit{is} greater than \textit{I can} bear.
\verse Look, you have driven me out today from the face of the ground, and from your face I must hide. I will be a wanderer and a fugitive on the earth, and it will happen that whoever finds me will kill me.”
\verse Then Adonai said to him, “Therefore, whoever kills Cain will be avenged sevenfold.” Then Adonai put a sign on Cain so that whoever found him would not kill him.
\verse And Cain went out from the presence of Adonai, and he settled in the land of Nod, east of Eden.
\verse And Cain knew his wife, and she conceived and gave birth to Enoch. And when he built a city \textit{he named the city after his son, Enoch}.
\verse And to Enoch was born Irad, and Irad fathered Mehujael, and Mehujael fathered Methushael, and Methushael fathered Lamech.
\verse And Lamech took to himself two wives. The name of the first \textit{was} Adah, and the name of the second \textit{was} Zillah.
\verse And Adah gave birth to Jabal; he was the father of those who live in tents and \textit{those who have} livestock.
\verse And the name of his brother \textit{was} Jubal; he was the father of all who play stringed instruments and wind instruments.
\verse Then Zillah also gave birth to Tubal-Cain who forged all \textit{kinds of} tools of bronze and iron. And the sister of Tubal-Cain \textit{was} Naamah.
\verse Then Lamech said to his wives,
\verse “Adah and Zillah, listen to my voice; 
O wives of Lamech, hear my words. 
I have killed a man for wounding me, 
Even a young man for injuring me.
\verse Then Adam knew his wife again, and she gave birth to a son. And she called his name Seth, for \textit{she said} “God has appointed to me another child in the place of Abel, because Cain killed him.”
\verse And as for Seth, he also fathered a son, and he called his name Enosh. At that time he began to call on the name of Adonai.
\end{biblechapter}

\begin{biblechapter} % Genesis 5
\verseWithHeading{Adam’s Descendants to Noah} This is the record of the generations of Adam. When God created Adam, he made him in the likeness of God.
\verse Male and female he created them. And he blessed them. And he called their name “Humankind” when they were created.
\verse And when Adam had lived one hundred and thirty years, he fathered a child in his likeness, according to his image. And he called his name Seth.
\verse And the days of Adam after he fathered Seth were eight hundred years. And he fathered sons and daughters.
\verse And all the days of Adam which he lived were nine hundred and thirty years, and he died.
\verse When Seth had lived one hundred and five years, he fathered Enosh.
\verse And after Seth had fathered Enosh he lived eight hundred and seven years, and fathered sons and daughters.
\verse And all the days of Seth were nine hundred and twelve years, and he died.
\verse When Enosh lived ninety years, he fathered Kenan.
\verse And after Enosh fathered Kenan he lived eight hundred and fifteen years, and fathered sons and daughters.
\verse And all the days of Enosh were nine hundred and five years, and he died.
\verse When Kenan had lived seventy years, he fathered Mahalalel.
\verse And after Kenan had fathered Mahalalel, he lived eight hundred and forty years, and fathered sons and daughters.
\verse And all the days of Kenan were nine hundred and ten years, and he died.
\verse When Mahalalel had lived sixty-five years, he fathered Jared.
\verse And after Mahalalel had fathered Jared, he lived eight hundred and thirty years, and fathered sons and daughters.
\verse And all the days of Mahalalel were eight hundred and ninety-five years, and he died.
\verse When Jared had lived one hundred and sixty-two years, he fathered Enoch.
\verse And after Jared had fathered Enoch, he lived eight hundred years, and fathered sons and daughters.
\verse And all the days of Jared were nine hundred and sixty-two years, and he died.
\verse When Enoch had lived sixty-five years, he fathered Methuselah.
\verse And Enoch walked with God after he fathered Methuselah three hundred years, and fathered sons and daughters.
\verse And all the days of Enoch were three hundred and sixty-five years.
\verse And Enoch walked with God, and he was no more, for God took him.
\verse When Methuselah had lived one hundred and eighty-seven years, he fathered Lamech.
\verse And after Methuselah had fathered Lamech, he lived seven hundred and eighty-two years, and fathered sons and daughters.
\verse And all the days of Methuselah were nine hundred and sixty-nine years, and he died.
\verse When Lamech had lived one hundred and eighty-two years, he fathered a son.
\verse And he called his name Noah, saying, “This one \textit{shall relieve us} from our work, and from the hard labor of our hands, from the ground which Adonai had cursed.
\verse And after Lamech had fathered Noah he lived five hundred and ninety-five years, and he fathered sons and daughters.
\verse All the days of Lamech were seven hundred and seventy-seven years, and he died.
\verse When Noah \textit{was five hundred years old}, Noah fathered Shem, Ham, and Japheth.
\end{biblechapter}

\begin{biblechapter} % Genesis 6
\verseWithHeading{Prelude to the Flood} And it happened \textit{that}, when humankind began to multiply on the face of the ground, daughters were born to them.
\verse Then the sons of God saw the daughters of humankind, that they \textit{were} beautiful. And they took for themselves wives from all that they chose.
\verse And Adonai said, “My Spirit shall not abide with humankind forever in that he \textit{is} also flesh. And his days \textit{shall be} one hundred and twenty years.”
\verse The Nephilim \textit{were} upon the earth in those days, and also afterward, when the sons of God went into the daughters of humankind, and they bore children to them.
\verse And Adonai saw that the evil of humankind \textit{was} great upon the earth, and every inclination of the thoughts of his heart \textit{was} always only evil.
\verse And Adonai regretted that he had made humankind on the earth, and \textit{he was grieved in his heart}.
\verse And Adonai said, “I will destroy humankind whom I created from upon the face of the earth, from humankind, to animals, to creeping things, and to the birds of heaven, for I regret that I have made them.”
\verse But Noah found favor in the eyes of Adonai.
\verse These \textit{are} the generations of Noah. Noah \textit{was} a righteous man, without defect in his generations. Noah walked with God.
\verse And Noah fathered three sons: Shem, Ham, and Japheth.
\verse And the earth \textit{was} corrupted before God, and the earth was filled \textit{with} violence.
\verse And God saw the earth, and behold, it was corrupt, for all flesh had corrupted its way upon the earth.
\verse And God said to Noah, “The end of all flesh \textit{has} come before me, for the earth was filled \textit{with} violence because of them. Now, look, I \textit{am going} to destroy them \textit{along} with the earth.
\verse Make for yourself an ark of cypress wood; you must make the ark \textit{with} rooms, then you must cover it with pitch, inside and outside.
\verse And this \textit{is} how you must make it: the length of the ark, three hundred cubits; its width fifty cubits; its height, thirty cubits.
\verse You must make a roof for the ark, and \textit{finish it to a cubit above}. And \textit{as for} the door of the ark, you must put \textit{it} in its side. You must make it \textit{with} a lower, second, and a third \textit{deck}.
\verse And I, behold, I \textit{am} about to bring the flood waters over the earth to destroy all flesh in which \textit{is} the breath of life from under the heaven; everything that \textit{is} on the earth shall perish.
\verse And I will establish my covenant with you, and you must go into the ark—you, and your sons, and your wife, and the wives of your sons with you.
\verse And of every living thing, from all flesh, you must bring two from every \textit{kind} into the ark to keep \textit{them} alive with you; they shall be male and female.
\verse From the birds according to their kind, and from the animals according to their kind, from every creeping thing \textit{on} the ground according to its kind—two from every kind shall come to you to keep \textit{them} alive.
\verse And \textit{as for} you, take for yourself from every kind of food that is eaten. And you must gather \textit{it} to yourself. And it shall be for you and for them for food.”
\verse And Noah did according to all that God commanded him; thus he did.
\end{biblechapter}

\begin{biblechapter} % Genesis 7
\verse Then Adonai said to Noah, “Go—you and all your household—into the ark, for I have seen you \textit{are} righteous before me in this generation.
\verse From all the clean animals you must take for yourself \textit{seven pairs}, a male and its mate. And from the animals that \textit{are} not clean \textit{you must take} two, a male and its mate,
\verse as well as from the birds of heaven \textit{seven pairs}, male and female, \textit{to keep their kind alive} on the face of the earth.
\verse For \textit{within seven days} I will send rain upon the earth \textit{for} forty days and forty nights. And I will blot out all the living creatures that I have made from upon the face of the ground.”
\verse And Noah did according to all that Adonai commanded him.
\verseWithHeading{The Flood} Noah \textit{was six hundred years old} when the flood waters came upon the earth.
\verse And Noah and his sons and his wife, and the wives of his sons with him, went into the ark because of the waters of the flood.
\verse Of clean animals, and of animals which \textit{are} not clean, and of the birds, and everything that creeps upon the ground,
\verse \textit{two of each} went to Noah, into the ark, male and female, as God had commanded Noah.
\verse And it happened \textit{that} after seven days the waters of the flood came over the earth.
\verse In the six hundredth year of the life of Noah, in the second month, on the seventeenth day of the month—on that day all the springs of the great deep were split open, and the windows of heaven were opened.
\verse And the rain came upon the earth forty days and forty nights.
\verse On this same day, Noah, Shem, Ham, and Japheth, the sons of Noah, and the wife of Noah and the three wives of his sons with them, went into the ark,
\verse they and all the living creatures according to their kind, and all the domesticated animals according to their kind, and all the creatures that creep upon the earth according to their kind, all the birds according to their kind, every winged creature.
\verse And they came to Noah to the ark, \textit{two of each}, from every living thing in which \textit{was} the breath of life.
\verse And those that came, male and female, of every living thing, came as God had commanded him. And Adonai shut the door behind him.
\verse And the flood came forty days and forty nights upon the earth. And the waters increased, and lifted the ark, and it rose up from the earth.
\verse And the waters prevailed and increased greatly upon the earth. And the ark went upon the surface of the waters.
\verse And the waters prevailed \textit{overwhelmingly} upon the earth, and they covered all the high mountains which were under the entire heaven.
\verse \textit{The waters swelled fifteen cubits above the mountains, covering them}.
\verse And every living thing that moved on the earth perished—the birds, and the domesticated animals, and the wild animals, and everything that swarmed on the earth, and all humankind.
\verse Everything in whose nostrils \textit{was} \textit{the breath of life}, among all that \textit{was} on dry land, died.
\verse And he blotted out every living thing upon the surface of the ground, from humankind, to animals, to creeping things, and to the birds of heaven; they were blotted out from the earth. Only Noah and those who \textit{were} with him in the ark remained.
\verse And the waters prevailed over the earth one hundred and fifty days.
\end{biblechapter}

\begin{biblechapter} % Genesis 8
\verseWithHeading{The Flood Subsides} And God remembered Noah and all the wild animals, and all the domesticated animals that \textit{were} with him in the ark. And God caused a wind to blow over the earth, and the waters subsided.
\verse And the fountains of the deep and the windows of the heavens were closed, and the rain from the heavens was restrained.
\verse And the waters receded from the earth \textit{gradually}, and the waters abated at the end of one hundred and fifty days.
\verse And the ark came to rest in the seventh month, on the seventeenth day of the month, on the mountains of Ararat.
\verse And the waters \textit{continued to recede} to the tenth month; in the tenth month, on the first of the month, the tops of the mountains appeared.
\verse And it happened \textit{that} at the end of forty days Noah opened the window of the ark that he had made.
\verse And he sent out a raven; \textit{it went to and fro} until the waters were dried up from upon the earth.
\verse And \textit{he sent out a dove} to see \textit{whether} the waters had subsided from upon the ground.
\verse But the dove did not find a resting place for the sole of her foot, and she returned to him into the ark, for the waters \textit{were still} on the face of the earth. And he stretched out his hand and took her, and brought her to himself into the ark.
\verse And he waited another seven days, and \textit{again he sent out} the dove from the ark.
\verse And the dove came to him \textit{in the evening}, and behold, a freshly-picked olive tree leaf \textit{was} in her mouth. And Noah knew that the waters had subsided from upon the earth.
\verse And he waited \textit{seven more days}, and he sent out the dove. But it did not return again to him.
\verse And it happened that, in the six hundred and first year, in the first \textit{month}, on the first \textit{day} of the month, the waters dried up from upon the earth. And Noah removed the covering of the ark and looked. And behold, the face of the ground was dried up.
\verse And in the second month, on the twenty-seventh day of the month, the earth was dry.
\verse And God spoke to Noah, saying:
\verse “Go out from the ark, you and your wife, and your sons, and your sons’ wives with you.
\verse Bring out with you all the living things which \textit{are} with you, from all the living creatures—birds, and animals, and everything that creeps on the earth, and let them swarm on the earth and be fruitful and multiply on the earth.”
\verse So Noah went out, with his sons and his wife, and the wives of his sons with him.
\verse Every animal, every creeping thing, and every bird, \textit{and} everything \textit{that} moves upon the earth, according to its families, went out from the ark.
\verse And Noah built an altar to Adonai, and he took from all the clean animals and from all the clean birds, and offered burnt offerings on the altar.
\verse And Adonai smelled the soothing fragrance, and Adonai said \textit{to himself}, “\textit{Never again will I curse} the ground for the sake of humankind, because the inclination of the heart of humankind \textit{is} evil from his youth. \textit{Nor will I ever again destroy} all life as I have done.
\verse \textit{As long as the earth endures}, seed and harvest, cold and heat, summer and winter, day and night will not cease.
\end{biblechapter}

\begin{biblechapter} % Genesis 9
\verseWithHeading{God’s Covenant with Noah and Humankind} And God blessed Noah and his sons, and said to them, “Be fruitful and multiply, and fill the earth.
\verse And fear of you and dread of you shall be upon every animal of the earth, and on every bird of heaven, \textit{and} on everything that moves upon the ground, and on all the fish of the sea. Into your hand they shall be given.
\verse Every moving thing that lives shall be for you as food. As \textit{I gave} the green plants to you, I have \textit{now} given you everything.
\verse Only you shall not eat \textit{raw flesh with blood in it}.
\verse And \textit{your lifeblood} I will require; from \textit{every animal} I will require it. And from the hand of humankind, from the hand of \textit{each} man to his brother I will require the life of humankind.
\verse “\textit{As for} the one shedding the blood of humankind, 
by humankind his blood shall be shed, 
for God made humankind in his own image.
\verse “And you, be fruitful and multiply, swarm on the earth and multiply in it.”
\verse And God said to Noah and to his sons with him,
\verse “As for me, behold, I am establishing my covenant with you and with your seed after you,
\verse and with every living creature that \textit{is} with you—the birds, the animals, and every animal of the earth with you, from all \textit{that} came out of the ark to all the animals of the earth.
\verse I am establishing my covenant with you, that never again will all flesh be cut off by the waters of a flood, nor will there ever be a flood that destroys the earth.”
\verse And God said, “This \textit{is} the sign of the covenant that I am making between me and you, and between every living creature that \textit{is} with you for future generations.
\verse My bow I have set in the clouds, and it shall be for a sign of \textit{the} covenant between me and between the earth.
\verse And when I make clouds appear over the earth the bow shall be seen in the clouds.
\verse Then I will remember my covenant that \textit{is} between me and you, and between every living creature, with all flesh. And the waters of a flood will never again \textit{cause the destruction} of all flesh.
\verse The bow shall be in the clouds, and I will see it, so as to remember \textit{the} everlasting covenant between God and between every living creature, with all flesh that \textit{is} upon the earth.”
\verse And God said to Noah, “This \textit{is} the sign of the covenant which I am establishing between me and all flesh that \textit{is} upon the earth.
\verseWithHeading{Noah’s Descendants} Now the sons of Noah who came out of the ark \textit{were} Shem, Ham, and Japheth. (Ham \textit{was} the father of Canaan.)
\verse These three \textit{were} the sons of Noah, and from these \textit{the whole earth was populated}.
\verse And Noah began \textit{to be} a man of the ground, and he planted a vineyard.
\verse And he drank some of the wine and became drunk, and he exposed himself in the midst of his tent.
\verse And Ham, the father of Canaan, saw the nakedness of his father, and he told his two brothers outside.
\verse Then Shem and Japheth took a garment, and the two of them put \textit{it} on \textit{their} shoulders and, walking backward, they covered the nakedness of their father. And their faces \textit{were turned} backward, so that they did not see the nakedness of their father.
\verse Then Noah awoke from his drunkenness, and he knew what his youngest son had done to him.
\verse And he said, “Cursed \textit{be} Canaan, 
a slave of slaves he shall be to his brothers.”
\verse Then he said,
\verse “Blessed \textit{be} Adonai, the God of Shem, 
and let Canaan be a slave to them.
\verse And Noah lived three hundred and fifty years after the flood.
\verse And all the days of Noah were nine hundred and fifty years, and he died.
\end{biblechapter}

\begin{biblechapter} % Genesis 10
\verseWithHeading{The Descendants of the Sons of Noah} These \textit{are} the generations of the sons of Noah—Shem, Ham, and Japheth. Children were born to them after the flood.
\verse The sons of Japheth: Gomer, Magog, Madai, Javan, Tubal, Meshech, and Tiras.
\verse And the sons of Gomer: Ashkenaz, Riphath, and Togarmah.
\verse And the sons of Javan: Elishah, Tarshish, Kittim, and Dodanim.
\verse From these the coastland peoples spread out through their lands, each according to his own language by their own families, in their nations.
\verse And the sons of Ham: Cush, Egypt, Put, and Canaan.
\verse And the sons of Cush: Seba, Havilah, Sabtah, Raamah, and Sabteca. The sons of Raamah: Sheba and Dedan.
\verse And Cush fathered Nimrod. \textit{He was the first on earth to be a mighty warrior}.
\verse He was a mighty hunter before Adonai. Therefore it was said, “Like Nimrod a mighty hunter before Adonai.”
\verse Now, the beginning of his kingdom \textit{was} Babel, Erech, Akkad, and Calneh, in the land of Shinar.
\verse From that land he went out \textit{to} Assyria, and he built Nineveh, Rehoboth-Ir, Calah,
\verse Resen between Nineveh and Calah; that \textit{is} the great city.
\verse And Egypt fathered Ludim, Anamim, Lehabim, Naphtuhim,
\verse Pathrusim, and Casluhim (from whom the Philistines came), and Caphtorim.
\verse Canaan fathered Sidon, his firstborn, and Heth,
\verse and the Jebusites, the Amorites, the Girgashites,
\verse the Hivites, the Arkites, the Sinites,
\verse the Arvadites, the Zemarites, and the Hamathites. Afterward the families of the Canaanites were spread abroad.
\verse And the territory of the Canaanites \textit{was} from Sidon \textit{in the direction of} Gerar as far as Gaza, and \textit{in the direction of} Sodom, Gomorrah, Admah, and Zeboiim, as far as Lasha.
\verse These \textit{are} the descendants of Ham, according to their families and their languages, in their lands, and in their nations.
\verse And to Shem, the father of all the children of Eber, the older brother of Japheth, \textit{children} were also born.
\verse The sons of Shem: Elam, Asshur, Arphaxad, Lud, and Aram.
\verse And the sons of Aram: Uz, Hul, Gether, and Mash.
\verse And Arphaxad fathered Shelah, and Shelah fathered Eber.
\verse And to Eber two sons were born. The name of the one was Peleg, for in his days the earth was divided, and the name of his brother \textit{was} Joktan.
\verse And Joktan fathered Almodad, Sheleph, Hazarmaveth, Jerah,
\verse Hadoram, Uzal, Diklah,
\verse Obal, Abimael, Sheba,
\verse Ophir, Havilah, and Jobab. All these \textit{were} the sons of Joktan.
\verse And their dwelling \textit{place} \textit{extended from} Mesha \textit{in the direction of} Sephar \textit{to} the hill country of the east.
\verse These \textit{are} the sons of Shem, according to their families, according to their languages, in their lands, and according to their nations.
\verse These are the families of the sons of Noah, according to their generations \textit{and} in their nations. And from these the nations spread abroad on the earth after the flood.
\end{biblechapter}

\begin{biblechapter} % Genesis 11
\verseWithHeading{The Tower of Babel} Now the whole earth \textit{had} one language and the same words.
\verse And as people migrated from the east they found a plain in the land of Shinar and settled there.
\verse And they said \textit{to each other}, “Come, let us make bricks and \textit{burn them thoroughly}.” And they had brick for stone and they had tar for mortar.
\verse And they said, “Come, let us build ourselves a city and a tower whose top \textit{reaches to} the heavens. And let us make a name for ourselves, lest we be scattered over the face of the whole earth.”
\verse Then Adonai came down to see the city and the tower that \textit{humankind} was building.
\verse And Adonai said, “Behold, \textit{they are one people with one language}, and \textit{this is only the beginning of what they will do}. So now nothing that they intend to do will be impossible for them.
\verse Come, let us go down and confuse their language there, so that they will not understand \textit{each other’s language}.”
\verse So Adonai scattered them from there over the face of the whole earth, and they stopped building the city.
\verse Therefore its name was called Babel, for there Adonai confused the language of the whole earth, and there Adonai scattered them over the face of the whole earth.
\verseWithHeading{The Descendants of Shem} These are the generations of Shem. When Shem \textit{was one hundred years old}, he fathered Arphaxad, two years after the flood.
\verse And Shem lived five hundred years after he fathered Arphaxad, and he fathered \textit{other} sons and daughters.
\verse When Arphaxad had lived thirty-five years, he fathered Shelah.
\verse And Arphaxad lived four hundred and three years after he fathered Shelah, and he fathered \textit{other} sons and daughters.
\verse When Shelah had lived thirty years, he fathered Eber.
\verse And Shelah lived four hundred and three years after he fathered Eber, and he fathered \textit{other} sons and daughters.
\verse When Eber had lived thirty-four years, he fathered Peleg.
\verse And Eber lived four hundred and thirty years after he fathered Peleg, and he fathered \textit{other} sons and daughters.
\verse When Peleg had lived thirty years, he fathered Reu.
\verse And Peleg lived two hundred and nine years after he fathered Reu, and he fathered \textit{other} sons and daughters.
\verse When Reu had lived thirty-two years, he fathered Serug.
\verse And Reu lived two hundred and seven years after he fathered Serug, and he fathered \textit{other} sons and daughters.
\verse When Serug had lived thirty years, he fathered Nahor.
\verse And Serug lived two hundred years after he fathered Nahor, and he fathered \textit{other} sons and daughters.
\verse When Nahor had lived twenty-nine years, he fathered Terah.
\verse And Nahor lived one hundred and nineteen years after he fathered Terah, and he fathered \textit{other} sons and daughters.
\verse When Terah had lived seventy years, he fathered Abram, Nahor, and Haran.
\verseWithHeading{The Descendants of Terah} Now these are the generations of Terah. Terah fathered Abram, Nahor, and Haran, and Haran fathered Lot.
\verse And Haran died in the presence of Terah his father in the land of his birth, in Ur of the Chaldeans.
\verse And Abram and Nahor took wives for themselves. The name of the wife of Abram \textit{was} Sarai, and the name of the wife of Nahor \textit{was} Milcah, the daughter of Haran, the father of Milcah and Iscah.
\verse And Sarai was barren; she had no child.
\verse And Terah took Abram his son, and Lot, the son of Haran, \textit{his grandson}, and Sarai his daughter-in-law, the wife of Abram his son, and went out with them from Ur of the Chaldeans to go to the land of Canaan. And they went to Haran, and they settled there.
\verse And the days of Terah \textit{were} two hundred and five years, and Terah died in Haran.
\end{biblechapter}

\begin{biblechapter} % Genesis 12
\verseWithHeading{The Call of Abram} And Adonai said to Abram, “Go out from your land and from your relatives, and from the house of your father, to the land that I will show you.
\verse And I will make you a great nation, and I will bless you, and I will make your name great. And you will be a blessing.
\verse And I will bless those who bless you, and those who curse you I will curse. And all families of the earth will be blessed in you.”
\verseWithHeading{Abram’s Journey} And Abram went \textit{out} as Adonai had told him, and Lot went with him. Now Abram \textit{was seventy-five years old} when he went out from Haran.
\verse And Abram took Sarai his wife, and Lot \textit{his nephew}, and all their possessions that they had gathered, and all the persons that they had acquired in Haran, and they went out to go to the land of Canaan. And they went to the land of Canaan.
\verse And Abram traveled through the land up to the place of Shechem, to the Oak of Moreh. Now the Canaanites \textit{were} in the land at that time.
\verse And Adonai appeared to Abram and said, “To your offspring I will give this land.” And he built an altar there to Adonai, who had appeared to him.
\verse And he moved on from there to the hill country, east of Bethel. And he pitched his tent at Bethel on the west, and at Ai on the east. And he built an altar there to Adonai. And he called on the name of Adonai.
\verse \textit{And Abram kept moving on}, toward the Negev.
\verse And there was a famine in the land. And Abram went down to Egypt to dwell as an alien there, for the famine was severe in the land.
\verse And it happened \textit{that} as he drew near to enter into Egypt, he said to Sarai his wife, “Look now, I know that you are a woman beautiful of appearance,
\verse and it shall happen \textit{that}, if the Egyptians see you, then they will say, ‘This \textit{is} his wife,’ then they will kill me but let you live.
\verse Please say you are my sister so that it will go well for me on your account. \textit{Then I will live} on account of you.”
\verse And it happened \textit{that} as Abram came into Egypt, the Egyptians saw the woman, that she \textit{was} very beautiful.
\verse And the officials of Pharaoh saw her, and they praised her \textit{beauty} to Pharaoh. And the woman was taken to the house of Pharaoh.
\verse And he dealt well with Abram on account of her, and he had sheep, cattle, male donkeys, male servants, female servants, female donkeys, and camels.
\verse Then Adonai afflicted Pharaoh and his household with severe plagues on account of the matter of Sarai the wife of Abram.
\verse Then Pharaoh called for Abram and said, “What \textit{is} this you have done to me? Why did you not tell me that she \textit{was} your wife?
\verse Why did you say ‘She \textit{is} my sister,’ so that I took her to myself as a wife? Now then, here \textit{is} your wife. Take her and go.”
\verse And Pharaoh commanded his men concerning him, and then sent him and his wife and all that \textit{was} with him away.
\end{biblechapter}

\begin{biblechapter} % Genesis 13
\verseWithHeading{The Parting of Abram and Lot} Then Abram went up from Egypt, he and his wife and all that \textit{was} with him. And Lot \textit{went} with him to the Negev.
\verse Now Abram \textit{was} very wealthy in livestock, in silver, and in gold.
\verse And he went according to his journey from the Negev, then to Bethel, to the place where his tent was at the beginning, between Bethel and Ai,
\verse to the place where he had made an altar at the beginning. And Abram called on the name of Adonai there.
\verse And Lot, who went with Abram, also had herds and tents.
\verse And the land could not \textit{support them} \textit{so as} to live together, because their possessions were \textit{so} many that they were not able to live together.
\verse And there was a quarrel between the herdsmen of the livestock of Abram and the herdsmen of the livestock of Lot. Now at that time the Canaanites and the Perizzites were living in the land.
\verse Then Abram said to Lot, “Please, let there not be quarreling between me and you, and between my shepherds and your shepherds, for we men \textit{are} brothers.
\verse Is not the whole land before you? Separate yourself from me. If \textit{you want what is on} the left, then I will go right; if \textit{you want what is on} the right, I will go left.”
\verse And Lot lifted up his eyes and saw the whole plain of the Jordan, that all of it \textit{was} well-watered land—\textit{this was} before Adonai destroyed Sodom and Gomorrah—like the garden of Adonai, like the land of Egypt \textit{in the direction of} Zoar.
\verse So Lot chose for himself all the plain of the Jordan. And Lot journeyed from the east, and so they separated \textit{from each other}.
\verse Abram settled in the land of Canaan, and Lot settled in the cities of the plain. And he pitched his tent toward Sodom.
\verse Now the men of Sodom \textit{were extremely wicked sinners against Adonai}.
\verse And Adonai said to Abram after Lot had separated from him, “Now, lift up your eyes and look from the place where you \textit{are} to the north, and to the south, and to the east and to the west,
\verse for all the land which you see I will give to you, and to your descendants, forever.
\verse I will make your descendants like the dust of the earth which, if anyone were able to count the dust of the earth, your descendants would be \textit{so} counted.
\verse Arise, go through the length of the land and through its breadth, for I will give it to you.”
\verse So Abram pitched his tent, and he came and settled at the oaks of Mamre, which \textit{were} at Hebron. And there he built an altar to Adonai.
\end{biblechapter}

\begin{biblechapter} % Genesis 14
\verseWithHeading{Abram Rescues Lot} And it happened \textit{that} in the days of Amraphel, the king of Shinar, Arioch, the king of Ellasar, Kedorlaomer, the king of Elam, and Tidal, the king of Goiim,
\verse made war with Bera, the king of Sodom, and Birsha, the king of Gomorrah, Shinab, the king of Admah, and Shemeber, the king of Zeboiim, and the king of Bela (that \textit{is}, Zoar).
\verse All these joined forces at the valley of Siddim (that \textit{is}, the sea of the salt).
\verse Twelve years they had served Kedorlaomer, but in the thirteenth year they rebelled.
\verse In the fourteenth year Kedorlaomer and the kings who \textit{were} with him came and defeated the Rephaim in Ashteroth-Karnaim, and the Zuzim in Ham, and the Emim in Shaveh-Kiriathaim,
\verse And the Horites in their hill country of Seir, as far as El-Paran, which is at the wilderness.
\verse Then they turned back and came to En-Mishpat (that \textit{is}, Kadesh). And they defeated the whole territory of the Amalekites, and also the Amorites who were living in Hazazon-Tamar.
\verse Then the king of Sodom, the king of Gomorrah, the king of Admah, the king of Zeboiim, and the king of Bela (that \textit{is}, Zoar) went out, and \textit{they took up battle position} in the Valley of Siddim
\verse with Kedorlaomer, king of Elam, and Tidal, king of Goiim, and Amraphel, king of Shinar, and Arioch, king of Ellasar, four kings against five.
\verse Now the Valley of Siddim \textit{was full of tar pits}. And the kings of Sodom and Gomorrah fled and \textit{fell into them}, but the rest fled to the mountains.
\verse So they took all the possessions of Sodom and Gomorrah and all their provisions, and they left.
\verse And they took Lot, the son of the brother of Abram, and his possessions, and left. (Now he had been living in Sodom.)
\verse Then one who escaped came and told Abram the Hebrew. And he was living at the oaks of Mamre the Amorite, brother of Eshcol and brother of Aner. \textit{They were allies with Abram}.
\verse When Abram heard that his \textit{relative} was taken captive, he summoned his trained men, born in his house, three hundred and eighteen \textit{of them}, and he went in pursuit up to Dan.
\verse And he divided \textit{his trained men} against them at night, he and his servants. And he defeated them and pursued them to Hobah, which \textit{is} north of Damascus.
\verse And he brought back all the possessions. And he also brought back Lot, his \textit{relative}, and his possessions, and the women and the people as well.
\verseWithHeading{Abram Meets Melchizedek} After his return from defeating Kedorlaomer and the kings who \textit{were} with him, the king of Sodom went out to meet him at the Valley of Shaveh (that \textit{is}, the Valley of the King).
\verse And Melchizedek, the king of Salem, brought out bread and wine. (He was the priest of God Most High).
\verse And he blessed him and said,
\verse “Blessed \textit{be} Abram by God Most High, 
Maker of heaven and earth.
\verse And he gave to him a tenth of everything.
\verse And the king of Sodom said to Abram, “Give me the people, but the possessions take for yourself.”
\verse And Abram said to the king of Sodom, “I have raised my hand to Adonai, God Most High, Maker of heaven and earth,
\verse \textit{that neither a thread nor} a thong of a sandal would I take from all that \textit{belongs} to you, that you might not say, ‘I made Abram rich.’
\end{biblechapter}

\begin{biblechapter} % Genesis 15
\verseWithHeading{Adonai’s Covenant with Abram} After these things the word of Adonai came to Abram in a vision, saying: “Do not be afraid, Abram; I \textit{am} your shield, \textit{and} your reward \textit{shall be} very great.”
\verse Then Abram said, “O Adonai, my Lord, what will you give me? \textit{I continue to be} childless, and \textit{my heir} is Eliezer of Damascus.”
\verse And Abram said, “Look, you have not given me a descendant, and here, \textit{a member of my household} \textit{is} \textit{my heir}.”
\verse And behold, the word of Adonai \textit{came} to him saying, “This \textit{person} will not \textit{be your heir}, but \textit{your own son will be your heir}.”
\verse And he brought him outside and said, “Look toward the heavens and count the stars if you are able to count them.” And he said to him, “So shall your offspring be.”
\verse And he believed in Adonai, and he reckoned it to him \textit{as} righteousness.
\verse And he said to him, “I \textit{am} Adonai, who brought you out from Ur of the Chaldeans to give this land to you, to possess it.”
\verse And he said, “O Adonai, how shall I know \textit{that} I will possess it?”
\verse And he said to him, “Take for me a three-year-old heifer, and a three-year-old female goat, and a three-year-old ram, and a turtledove and a young pigeon.”
\verse And he took for him all these and cut them in pieces down the middle. And he put each piece opposite \textit{the other}, but the birds he did not cut.
\verse And the birds of prey came down on the carcasses, but Abram drove them away.
\verse And it happened, as the sun \textit{went down}, then a deep sleep fell upon Abram and, behold, a great terrifying darkness fell upon him.
\verse And he said to Abram, “\textit{You must surely know} that your descendants shall be \textit{as} aliens in a land \textit{not their own}. And they shall serve them and they shall oppress them four hundred years.
\verse And also the nation that they serve I will judge. Then afterward they shall go out with great possessions.
\verse And \textit{as for} you, you shall go to your ancestors in peace; you shall be buried in a good old age.
\verse And the fourth generation shall return here, for the guilt of the Amorites \textit{is not yet complete}.”
\verse And after the sun had gone down and it \textit{was} dusk, behold, a smoking firepot and a flaming torch passed between those half pieces.
\verse On that day Adonai \textit{made} a covenant with Abram saying, “To your offspring I will give this land, from the river of Egypt to the great river, the Euphrates river,
\verse \textit{the land of} the Kenites, the Kenizzites, the Kadmonites,
\verse the Hittites, the Perizzites, the Rephaim,
\verse the Amorites, the Canaanites, the Girgashites, and the Jebusites.”
\end{biblechapter}

\begin{biblechapter} % Genesis 16
\verseWithHeading{Sarai and Hagar} Now Sarai, the wife of Abram, had borne him no children. And she had a female Egyptian servant, and her name \textit{was} Hagar.
\verse And Sarai said to Abram, “Look, please, Adonai has prevented me from bearing children. Please go in to my servant; perhaps \textit{I will have children by her}.” And Abram listened to the voice of Sarai.
\verse Then Sarai, the wife of Abram, took Hagar, her Egyptian servant, after Abram had lived ten years in the land of Canaan, and gave her to Abram her husband as his wife.
\verse And he went in to Hagar, and she conceived. And \textit{when} she saw that she had conceived, then her mistress grew small in her eyes.
\verse And Sarai said to Abram, “may my harm \textit{be} upon you. \textit{I had my servant sleep with you}, and \textit{when} she saw that she had conceived, \textit{she no longer respected me}. May Adonai judge between me and you!”
\verse And Abram said to Sarai, “Look, your servant \textit{is} \textit{under your authority}. Do to her that which \textit{is} good in your eyes.” And Sarai mistreated her, and she fled from her presence.
\verseWithHeading{Hagar and the Angel of Adonai} And the angel of Adonai found her at a spring of water in the wilderness, at the spring by the road of Shur.
\verse And he said to Hagar, the servant of Sarai, “\textit{From where} have you come, and where are you going?” And she said, “I am fleeing from the presence of Sarai my mistress.”
\verse Then the angel of Adonai said to her, “Return to your mistress and submit yourself under \textit{her authority}.”
\verse And the angel of Adonai said to her, “\textit{I will greatly multiply} your offspring, so that they cannot be counted for \textit{their} abundance.”
\verse And the angel of Adonai said to her:
\verse “Behold, you are pregnant 
and shall have a son. 
And you shall call his name Ishmael, 
for Adonai has listened to your suffering.
\verse So she called the name of Adonai who spoke to her, “You \textit{are} El-Roi,” for she said, “Here I have seen after he who sees me.”
\verse Therefore the well was called Beer-Lahai-Roi; behold, it \textit{is} between Kadesh and Bered.
\verse And Hagar had a child for Abram, a son. And Abram called the name of his son whom Hagar bore to him, Ishmael.
\verse And Abram \textit{was} eighty-six years old when Hagar bore Ishmael to Abram.
\end{biblechapter}

\begin{biblechapter} % Genesis 17
\verseWithHeading{Abram and Circumcision, the Sign of the Covenant} When Abram \textit{was} ninety-nine years old Adonai appeared to Abram. And he said to him, “I \textit{am} El-Shaddai; walk before me and be blameless
\verse so that I may make my covenant between me and you, and may multiply you \textit{exceedingly}.”
\verse Then Abram fell upon his face and God spoke with him, saying,
\verse “\textit{As for} me, behold, my covenant \textit{shall be} with you, and you shall be the father of a multitude of nations.
\verse Your name shall no longer be called Abram, but your name shall be Abraham, for I will make you the father of a multitude of nations.
\verse And I will make you \textit{exceedingly} fruitful. I will make you a nation, and kings shall go out from you.
\verse And I will establish my covenant between me and you, and between your offspring after you, throughout their generations as an everlasting covenant to be as God for you and to your offspring after you.
\verse And I will give to you and to your offspring after you \textit{the land in which you are living as an alien}, all the land of Canaan, as an everlasting property. And I will be to them as God.”
\verse And God said to Abraham, “Now \textit{as for} you, you must keep my covenant, you and your offspring after you, throughout their generations.
\verse This \textit{is} my covenant which you shall keep, between me and you, and \textit{also} with your offspring after you: Every male among you shall be circumcised.
\verse And you shall circumcise the flesh of your foreskin, and it shall be a sign of the covenant between me and you.
\verse And \textit{at eight days of age} you shall yourselves circumcise every male \textit{belonging} to your generations \textit{and} \textit{the servant born in your house and the one bought from any foreigner} who is not from your offspring.
\verse \textit{You must certainly circumcise} \textit{the servant born in your house and the one bought from any foreigner}. And my covenant shall be with your flesh as an everlasting covenant.
\verse And \textit{as for any} uncircumcised male who has not circumcised the flesh of his foreskin, that person shall be cut off from his people. He has broken my covenant.
\verse And God said to Abraham, “\textit{as for} Sarai your wife, you shall not call her name Sarai, for Sarah \textit{shall be} her name.
\verse And I will bless her; moreover, I give to you from her a son. And I will bless her, and \textit{she shall give rise to nations}. Kings of peoples shall come from her.”
\verse And Abraham fell upon his face and laughed. And he said in his heart, “\textit{Can a child be born to a man a hundred years old}, or \textit{can Sarah bear a child at ninety}?”
\verse And Abraham said to God, “Oh that Ishmael might live before you!”
\verse And God said, “No, but Sarah your wife shall bear a son for you, and you shall call his name Isaac. And I will establish my covenant with him as an everlasting covenant to his offspring after him.
\verse And \textit{as for} Ishmael, I have heard you. Behold, I will bless him and I will make him fruitful, and I will multiply him \textit{exceedingly}. He shall father twelve princes, and I will make him a great nation.
\verse But my covenant I will establish with Isaac, whom Sarah shall bear to you at this appointed time next year.”
\verse When he finished speaking with him, God went up from Abraham.
\verse And Abraham took Ishmael his son and all who were born of his house, and all \textit{those} acquired by his money, every male among the men of Abraham’s house, and he circumcised the flesh of their foreskin on the same day that God spoke with him.
\verse Abraham \textit{was} ninety-nine years old when he circumcised the flesh of his foreskin.
\verse And Ishmael his son \textit{was} thirteen years old when he circumcised the flesh of his foreskin.
\verse Abraham and his son Ishmael \textit{were} circumcised on the same day.
\verse And all the men of his house, \textit{those born in the house, and those acquired by money from a foreigner}, were circumcised with him.
\end{biblechapter}

\begin{biblechapter} % Genesis 18
\verseWithHeading{Adonai Appears to Abraham as a Man} And Adonai appeared to him by the oaks of Mamre. And he was sitting in the doorway of the tent at the heat of the day.
\verse And he lifted up his eyes and saw, and behold, three men were standing near him. And he saw \textit{them} and ran from the doorway of the tent to meet them. And he bowed down to the ground.
\verse And he said, “My lord, if I have found favor in your eyes do not pass by your servant.
\verse Let a little water be brought and wash your feet, and rest under the tree.
\verse And let me bring a piece of bread, then refresh \textit{yourselves}. Afterward you can pass on, \textit{once} you have passed by with your servant.” Then they said, “Do so as you have said.”
\verse Then Abraham hastened into the tent to Sarah, and he said, “Quickly—make three seahs of fine flour for kneading and make bread cakes!”
\verse And Abraham ran to the cattle and took a \textit{calf}, tender and good, and gave it to the servant, and he made haste to prepare it.
\verse Then he took curds and milk, and the calf which he prepared, and set \textit{it} before them. And he was standing by them under the tree while they ate.
\verse And they said to him, “Where \textit{is} Sarah your wife?” And he said, “Here, in the tent.”
\verse And he said, “I will certainly return to you \textit{in the spring}, and look, Sarah your wife \textit{will have} a son.” Now Sarah \textit{was} listening at the doorway of the tent, and which \textit{was} behind him.
\verse Now Abraham and Sarah \textit{were} old, \textit{advanced in age}; \textit{the way of women} had ceased to be for Sarah.
\verse So Sarah laughed to herself saying, “After I am worn out and my husband is old, shall \textit{this} pleasure be to me?”
\verse Then Adonai said to Abraham, “What \textit{is} this \textit{that} Sarah laughed, saying, ‘Is it indeed true \textit{that} I will bear a child, now \textit{that} I have grown old?’
\verse Is anything too difficult for Adonai? At the appointed time I will return to you \textit{in the spring} and Sarah \textit{shall have} a son.”
\verse But Sarah denied \textit{it}, saying, “I did not laugh,” because she was afraid. He said, “No, but you did laugh.”
\verse Then the men set out from there, and they looked down upon Sodom. And Abraham went with them \textit{to send them on their way}.
\verse Then Adonai said, “Shall I conceal from Abraham what I \textit{am going} to do?
\verse Abraham will surely become a great and strong nation, and all the nations of the earth will be blessed on account of him.
\verse For I have chosen him, that he will command his children and his household after him that they will keep the way of Adonai, to do righteousness and justice, so that Adonai may bring upon Abraham that which he said to him.”
\verse Then Adonai said, “Because the outcry of Sodom and Gomorrah \textit{is} great and because their sin \textit{is} very \textit{serious},
\verse I will go down and I will see. Have they done altogether according to its cry of distress \textit{which} has come to me? If not, I will know.”
\verseWithHeading{Abraham Intercedes for Sodom} And the men turned from there and went toward Sodom. And Abraham \textit{was} still standing before Adonai.
\verse And Abraham drew near \textit{to Adonai} and said, “Will you also sweep away the righteous with the wicked?
\verse If perhaps there are fifty righteous in the midst of the city, will you also sweep \textit{them} away and not forgive the place on account of the fifty righteous in her midst?
\verse Far be it from you to do such a thing as this, to kill \textit{the} righteous with \textit{the} wicked, that the righteous would be as the wicked! Far be it from you! Will not the Judge of all the earth do justice?”
\verse And Adonai said, “If I find fifty righteous in Sodom, in the midst of the city, then I will forgive the whole place for their sake.”
\verse Then Abraham answered and said, “Look, please, I was bold to speak to my Lord, but I \textit{am} dust and ashes.
\verse Perhaps the fifty righteous are lacking five—will you destroy the whole city on account of the five?” And he answered, “I will not destroy \textit{it} if I find forty-five there.”
\verse And \textit{once again he spoke} to him and said, “What if forty are found there?” And he answered, “I will not do \textit{it} on account of the forty.”
\verse And he said, “Please, let not my Lord be angry, and I will speak. What if thirty be found there?” And he answered, “I will not do \textit{it} if I find thirty there.”
\verse And he said, “Please, now, I was bold to speak to my Lord. What if twenty be found there?” And he answered, “I will not destroy \textit{it} for the sake of the twenty.”
\verse And he said, “Please, let not my Lord be angry, and I will speak only once more. What if ten are found there?” And he answered, “I will not destroy \textit{it} for the sake of the ten.”
\verse Then Adonai left, as he finished speaking to Abraham, and Abraham returned to his place.
\end{biblechapter}

\begin{biblechapter} % Genesis 19
\verseWithHeading{The Rescue of Lot from Sodom} And the two angels came to Sodom in the evening. And Lot was sitting in the gateway of Sodom. Then Lot saw \textit{them} and stood up to meet them. And he bowed down \textit{with his} face to the ground.
\verse And he said, “Behold, my lords, please turn aside into the house of your servant and spend the night and wash your feet. Then you can rise early and go on your way.” And they said, “No, but we will spend the night in the square.”
\verse But \textit{he urged them strongly}, and they turned aside with him and came into his house. And he made a meal for them and baked unleavened bread, and they ate.
\verse Before they laid down, the men of the city, the men of Sodom, both young and old, all the people \textit{to the last man}, surrounded the house.
\verse And they called to Lot and said to him, “Where \textit{are} the men who came to you tonight? Bring them out to us so that we may know them.”
\verse But Lot went out to them at the entrance, and he shut the door behind him.
\verse And he said, “No, my brothers, please do not do \textit{such a} wrong \textit{thing}.
\verse Behold, I have two daughters who have not known a man. Please, let me bring them out to you; then do to them as \textit{it seems} good in your eyes. Only to these men do not do \textit{this} thing, since they came under\textit{ my roof} for protection.”
\verse But they said, “Step aside!” Then they said, “\textit{This fellow} came to dwell as a foreigner and he acts as a judge! Now we shall do worse to you than them!” And they pressed very hard against the man, against Lot, and they drew near to break the door.
\verse Then the men reached out \textit{with} their hands and brought Lot in to them, into the house, and they shut the door.
\verse And the men who \textit{were} at the entrance of the house they struck with blindness, both small and great, and they were unable to find the entrance.
\verse Then the men said to Lot, “Who \textit{is} here with you? Bring out from the place \textit{your} sons-in-law, and your sons and your daughters, and all who \textit{are} with you in the city.
\verse For we are \textit{about to} destroy this place, because their cry has become great before Adonai. Adonai sent us to destroy it.”
\verse Then Lot went out and spoke to his sons-in-law \textit{who were} taking his daughters and said, “Get up! Go out from this place, because Adonai \textit{is going} to destroy the city!” But \textit{it seemed like a joke} in the eyes of his sons-in-law.
\verse And as the dawn came up the angels urged Lot saying, “Get up, take your wife and your two daughters \textit{who are staying with you}, lest you be destroyed with the punishment of the city.”
\verse But \textit{when} he lingered, the men seized him by his hand and his wife’s hand, and his two daughters by hand, on account of the mercy of Adonai upon him. And they brought him out and set him outside of the city.
\verse And after bringing them outside \textit{one} said, “Flee for your life; do not look behind you, and do not stand anywhere in the plain. Flee to the mountains lest you be destroyed.”
\verse And Lot said to them, “No, please, my lords.
\verse Behold, your servant has found favor in your eyes and \textit{you have shown me great kindness} in saving my life. But I cannot flee to the mountains, lest the disaster overtake me and I die.
\verse Behold, this city \textit{is} near \textit{enough} to flee there, and it \textit{is a} little \textit{one}. Please, let me flee there. Is it not a little \textit{one}? Then my life shall be saved.”
\verse And he said to him, “Behold, \textit{I will grant this favor as well}; that I will not overthrow the city of which you have spoken.
\verse Escape there quickly, for I cannot do \textit{this} thing until you get there.” Therefore, there name of the city was called Zoar.
\verseWithHeading{The Destruction of Sodom} \textit{After} the sun \textit{had risen} upon the earth and Lot had entered Zoar,
\verse Adonai rained down from heaven upon Sodom and Gomorrah brimstone and fire from Adonai.
\verse And he overthrew those cities and the whole plain, and the inhabitants of the cities and the vegetation of the ground.
\verse But his wife looked \textit{back}, and she became a pillar of salt.
\verse And Abraham arose early in the morning \textit{and went} to the place where he had stood before Adonai.
\verse And he looked down upon the surface of Sodom and Gomorrah, and upon the whole surface of the land, the plain. And he saw that, behold, the smoke of the land went up like the smoke of a smelting furnace.
\verse So it was, when God destroyed the cities of the plain that God remembered Abraham and sent Lot out from the midst of the overthrow, when he overthrew the cities in which Lot lived.
\verseWithHeading{Lot and His Daughters} And Lot went out from Zoar and settled in the hill country with his two daughters, for he was afraid to stay in Zoar. So he lived in a cave, he and his two daughters.
\verse And the firstborn \textit{daughter} said to the younger one, “Our father is old, and there is no man in the land to come in to us according to the manner of all the land.
\verse Come, let us give our father wine to drink and let us lie with him that \textit{we may secure descendants through our father}.”
\verse And they gave their father wine to drink that night, and the firstborn went and lay with her father, but he did not know when she lay down or when she got up.
\verse And it happened \textit{that}, the next day the firstborn said to the younger one, “Look, I laid with my father last night. Let us give him wine to drink also tonight, then go and lie with him that \textit{we may secure descendants through our father}.”
\verse And they gave their father wine to drink again that night, and the younger got up and lay with him, but he did not know when she lay down or when she got up.
\verse And the two daughters of Lot became pregnant by their father.
\verse The firstborn gave birth to a son, and she called his name Moab. He \textit{is} the father of Moab until this day.
\verse And the younger, she also gave birth to a son, and she called his name Ben-Ammi. He \textit{is} the father of the \textit{Ammonites} until this day.
\end{biblechapter}

\begin{biblechapter} % Genesis 20
\verseWithHeading{Abraham and Abimelech} And Abraham journeyed from there to the land of the Negev, and he settled between Kadesh and Shur. And he dwelled as an alien in Gerar.
\verse And Abraham said about Sarah his wife, “She \textit{is} my sister.” And Abimelech king of Gerar sent and took Sarah.
\verse And God came to Abimelech in a dream at night. And he said to him, “Look, you \textit{are} a dead man on account of the woman you have taken. For she \textit{is} \textit{a married woman}.”
\verse Now Abimelech had not approached her, so he said, “my Lord, will you even kill a righteous people?”
\verse Did not he himself say to me, ‘She \textit{is} my sister’? And she herself said, ‘He \textit{is} my brother.’ With integrity of my heart and with cleanness of my hands I did this.”
\verse Then God said to him in the dream, “Yes, I know that in the integrity of your heart you did this, and I also \textit{kept you from sinning} against me. Therefore, I did not allow you to touch her.
\verse So now, return the wife of the man, for he \textit{is} a prophet, so that he will pray for you and you will live. And \textit{if you do not return her}, know that you will certainly die, and all that \textit{are} yours.”
\verse So Abimelech rose early in the morning. And he called all his servants and \textit{told them all these things}, and the men were very afraid.
\verse And Abimelech called for Abraham and said to him, “What have you done to us? And how have I sinned against you that you brought upon me and upon my kingdom a great sin? You have done things to me that should not be done.”
\verse And Abimelech said to Abraham, “\textit{What were you thinking} that you did this thing?”
\verse And Abraham said, “Because I thought, surely there is no fear of God in this place; they will kill me on account of the matter of my wife.
\verse \textit{Besides}, she \textit{is} my sister, the daughter of my father, but not the daughter of my mother. And she became my wife.
\verse And it happened \textit{that} as God caused me to wander from the house of my father I said to her, ‘This \textit{is} your loyal kindness that you must do for me at every place where we come: say concerning me, “He \textit{is} my brother.” ’ ”
\verse And Abimelech took sheep and cattle and male slaves and female slaves, and he gave \textit{them} to Abraham. And he returned Sarah his wife to him.
\verse And Abimelech said, “Here \textit{is} my land before you; settle \textit{where it pleases you}.”
\verse And to Sarah he said, “Look, I have given a thousand \textit{pieces of} silver to your brother. It \textit{shall be} \textit{an exoneration}. \textit{You are vindicated before all who are with you}.”
\verse And Abraham prayed to God, and God healed Abimelech and his wife and his female servants so that they \textit{could} bear children \textit{again}.
\verse For Adonai had completely closed up all the wombs of the house of Abimelech because of the matter of Sarah, the wife of Abraham.
\end{biblechapter}

\begin{biblechapter} % Genesis 21
\verseWithHeading{The Birth of Isaac} And Adonai visited Sarah as he had said. And Adonai did to Sarah as he had promised.
\verse And she conceived, and Sarah bore to Abraham a son in his old age at the appointed time that God had told him.
\verse And Abraham called the name of his son who was born to him, whom Sarah bore to him, Isaac.
\verse And Abraham circumcised Isaac his son \textit{when he was} eight days old, as God had commanded him.
\verse And Abraham \textit{was} one hundred years old when Isaac his son was born to him.
\verse And Sarah said, “God has made laughter for me; all who hear will laugh for me.”
\verse And she said, “Who would announce to Abraham \textit{that} Sarah would nurse children? Yet I have borne a son \textit{to Abraham} in his old age.”
\verseWithHeading{Hagar and Ishmael} And the child grew and was weaned. And Abraham made a great feast on the day Isaac was weaned.
\verse And Sarah saw the son of Hagar the Egyptian, whom she had borne Abraham, mocking.
\verse Then she said to Abraham, “Drive out this slave woman and her son, for the son of this slave woman will not be heir with my son, with Isaac.”
\verse And the matter \textit{displeased Abraham very much} on account of his son.
\verse Then God said to Abraham, “\textit{Do not be displeased} on account of the boy and on account of the slave woman. \textit{Listen to everything that Sarah said to you}, for through Isaac \textit{your} offspring will be named.
\verse And I will also make the son of the slave woman into a nation, for he is your offspring.”
\verse Then Abraham rose up early in the morning and took bread and a skin of water and gave \textit{it} to Hagar, putting \textit{it} on her shoulder. And he sent her away with the child, and she went, wandering about in the wilderness, in Beersheba.
\verse And when the water was finished from the skin, she put the child under one of the bushes.
\verse And she went and \textit{she sat a good distance away}, for she said, “Let me not see the child’s death.” So she sat away from him and lifted up her voice and wept.
\verse And God heard the cry of the boy and the angel of God called to Hagar from the heavens and said to her, “\textit{What is the matter} Hagar? Do not be afraid, for God has heard the cry of the boy \textit{from where he is}.
\verse Get up, take up the boy and take him with your hand, for I will make him a great nation.”
\verse And God opened her eyes, and she saw a well of water. And she went and filled the skin with water and gave a drink to the boy.
\verse And God was with the boy, and he grew and lived in the wilderness. And he became \textit{an expert with a bow}.
\verse And he lived in the wilderness of Paran. And his mother took a wife for him from the land of Egypt.
\verseWithHeading{The Covenant Between Abraham and Abimelech} And it happened \textit{that} at that time, Abimelech and Phicol, the commander of his army, said to Abraham, “God \textit{is} with you, in all that you do.
\verse So now, swear to me here by God \textit{that} you will not deal with me falsely, or with my descendants, or my posterity. According to the kindness that I have done to you, you shall \textit{pledge} to do with me and with the land where you have dwelled as an alien.”
\verse And Abraham said, “I swear.”
\verse Then Abraham complained to Abimelech on account of the well of water that servants of Abimelech had seized.
\verse And Abimelech said, “I do not know who did this thing, neither did you tell me, nor have I heard \textit{of it} except for today.”
\verse And Abraham took sheep and cattle and gave \textit{them} to Abimelech. And the two of them \textit{made} a covenant.
\verse Then Abraham set \textit{off} seven ewe-lambs of the flock by themselves.
\verse And Abimelech said to Abraham, “What \textit{is the meaning of} these seven ewe-lambs that you have set \textit{off} by themselves?”
\verse And he said, “You shall take the seven ewe-lambs from my hand \textit{as proof on my behalf} that I dug this well.”
\verse Therefore that place is called Beersheba, because there the two of them swore an oath.
\verse And they \textit{made} a covenant at Beersheba. And Abimelech, and Phicol his army commander stood and returned to the land of the Philistines.
\verse And he planted a tamarisk tree in Beersheba, and there he called on the name of Adonai, \textit{the everlasting God}.
\verse And Abraham dwelled as an alien in the land of the Philistines many days.
\end{biblechapter}

\begin{biblechapter} % Genesis 22
\verseWithHeading{God Tests Abraham} And it happened \textit{that} after these things, God tested Abraham. And he said to him, “Abraham!” And he said, “Here I \textit{am}.”
\verse And he said, “Take your son, your only child, Isaac, whom you love, and go to the land of Moriah, and offer him there as a burnt offering on one of the mountains where I will tell you.”
\verse And Abraham rose up early in the morning and saddled his donkey. And he took two of his servants with him, and Isaac his son. And he chopped wood for a burnt offering. And he got up and went to the place which God had told him.
\verse On the third day Abraham lifted up his eyes, and he saw the place at a distance.
\verse And Abraham said to his servants, “You stay here with the donkey, and I and the boy will go up there. We will worship, then we will return to you.”
\verse And Abraham took the wood of the burnt offering and placed \textit{it} on Isaac his son. And he took the fire in his hand and the knife, and the two of them went together.
\verse And Isaac said to Abraham his father, “My father!” And he said, “Here I \textit{am}, my son.” And he said, “Here is the fire and the wood, but where is the lamb for a burnt offering?”
\verse And Abraham said, “\textit{God will provide} the lamb for a burnt offering, my son.” And the two of them went together.
\verse And they came to the place that God had told him. And Abraham built an altar there and arranged the wood. Then he bound Isaac his son and placed him on the altar atop the wood.
\verse And Abraham stretched out his hand and took the knife to slaughter his son.
\verse And the angel of Adonai called to him from heaven and said, “Abraham! Abraham!” And he said, “Here I \textit{am}.”
\verse And he said, “Do not stretch out your hand against the boy; do not do anything to him. For now I know that you are \textit{one who fears} God, since you have not withheld your son, your only child, from me.”
\verse And Abraham lifted up his eyes and looked. And behold, a ram was caught in the thicket by his horns. And Abraham went and took the ram, and offered it as a burnt offering in place of his son.
\verse And Abraham called the name of that place “Adonai \textit{will provide},” \textit{for which reason} it is said today, “on the mountain of Adonai \textit{it shall be provided}.”
\verse And the angel of Adonai called to Abraham a second time from heaven.
\verse And he said, “I swear by myself, declares Adonai, that because you have done this thing and have not withheld your son, your only child,
\verse that I will certainly bless you and greatly multiply your offspring as the stars of heaven, and as the sand that is by the shore of the sea. And your offspring will take possession of the gate of his enemies.
\verse All the nations of the earth will be blessed through your offspring, because you have listened to my voice.”
\verse And Abraham returned to his servants, and they got up and went together to Beersheba. And Abraham lived in Beersheba.
\verse And it happened \textit{that} after these things, it was told to Abraham, “Look, Milcah has also borne children to your brother Nahor:
\verse Uz his firstborn and Buz his brother, and Kemuel the father of Aram,
\verse and Kesed, Hazo, Pildash, Jidlaph, and Bethuel.”
\verse (Now, Bethuel fathered Rebekah). These eight Milcah bore to Nahor, the brother of Abraham.
\verse And his concubine, whose name was Reumah, also bore Tebah, Gaham, Tahash, and Maacah.
\end{biblechapter}

\begin{biblechapter} % Genesis 23
\verseWithHeading{Sarah’s Death and Burial} And \textit{Sarah lived} one hundred and twenty-seven years; \textit{these were} the years of the life of Sarah.
\verse And Sarah died in Kiriath Arba; that \textit{is} Hebron, in the land of Canaan.
\verse And Abraham went to mourn for Sarah and to weep for her. And Abraham rose up from his dead, and he spoke to the Hittites \textit{and} said,
\verse “I \textit{am} a stranger and an alien among you; give to me \textit{my own burial site} among you so that I may bury my dead from before me.”
\verse And the Hittites answered Abraham \textit{and} said to him,
\verse “Hear us, my lord, you \textit{are} a mighty prince in our midst. Bury your dead in the choicest of our burial sites. None of us \textit{will withhold his burial site} from you \textit{for} burying your dead.”
\verse And Abraham rose up and bowed to the people of the land, to the Hittites.
\verse And he spoke with them, saying, “\textit{If you are willing} \textit{that} I bury my dead from before me, hear me and intercede for me with Ephron the son of Zohar,
\verse that he may sell to me the cave of Machpelah which \textit{belongs to him}, which \textit{is} at the end of his field. At full value let him sell \textit{it} to me in your midst as \textit{a burial site}.”
\verse Now Ephron was sitting among the Hittites. And Ephron the Hittite answered Abraham in the hearing of the Hittites with respect to all \textit{who were} entering the gate of his city, \textit{and} said,
\verse “No, my lord, hear me. I give you the field and the cave which \textit{is} in it, I \textit{also} give it to you in the sight of the children of my people I give it to you. Bury your dead.”
\verse And Abraham bowed before the people of the land.
\verse And he spoke to Ephron in the hearing of the people of the land, saying, “\textit{If only you will hear me}—I give the price of the field. Take \textit{it} from me that I may bury my dead there.”
\verse And Ephron answered Abraham, saying to him,
\verse “My lord, hear me. A \textit{piece of} land \textit{worth} four hundred shekels of silver—what \textit{is} that between me and you? Bury your dead.”
\verse Then Abraham listened to Ephron, and Abraham weighed for Ephron the silver that he had named in the hearing of the Hittites: four hundred shekels of silver \textit{at the merchants’ current rate}.
\verse So the field of Ephron which \textit{was} in the Machpelah, which \textit{was} near Mamre—the field and the cave which \textit{was} in it, with all the trees that \textit{were} in the field, which \textit{were} within all its surrounding boundaries—\textit{passed}
\verse to Abraham as a property in the presence of the Hittites, with respect to all \textit{who were} entering the gate of his city.
\verse And thus afterward Abraham buried Sarah his wife in the cave of the field of Machpelah near Mamre; that \textit{is} Hebron, in the land of Canaan.
\verse And the field and the cave which \textit{was} in it \textit{passed} to Abraham as \textit{a burial site} from the Hittites.
\end{biblechapter}

\begin{biblechapter} % Genesis 24
\verseWithHeading{Isaac and Rebekah} Now Abraham \textit{was} old, \textit{advanced in age}, and Adonai had blessed Abraham in everything.
\verse And Abraham said to his servant, the oldest of his house, who had charge of all he had, “Please put your hand under my thigh
\verse that I may make you swear by Adonai, the God of heaven and the God of earth, that you will not take a wife for my son from the daughters of the Canaanites in whose midst I am dwelling,
\verse but that you will go to my land and to my family, and take a wife for my son, for Isaac.”
\verse And the servant said to him, “Perhaps the woman will not be willing \textit{to follow} me to this land—must I then return your son to the land from whence you came?”
\verse Abraham said to him, “\textit{You must take care} that you do not return my son there.
\verse Adonai, the God of heaven who took me from the house of my father and from the land of my family, and who spoke to me and swore to me, saying, ‘to your offspring I will give this land,’ he will send his angel before you, and you shall take a wife for my son from there.
\verse And if the woman is not willing \textit{to follow} you, then you shall be released from this oath of mine—only you must not return my son there.”
\verse Then the servant put his hand under the thigh of Abraham his master, and he swore to him concerning this matter.
\verse And the servant took ten camels from his master’s camels, and he went with all \textit{kinds of} his master’s good things in his hand. And he arose and went to Aram-Naharaim, to the city of Nahor.
\verse And he made the camels kneel outside the city at the well of water, at the time of evening, toward the time \textit{the women} went out to draw water.
\verse And he said, “O Adonai, God of my master Abraham, please grant me success today and show loyal love to my master Abraham.
\verse Behold, I am standing by the spring of water, and the daughters of the men of the city are going out to draw water.
\verse And let it be \textit{that} the girl to whom I shall say, ‘Please, offer your jar that I may drink’ and \textit{who} says, ‘Drink—and I will also water your camels,’ she \textit{is the one} you have chosen for your servant, for Isaac. By her I will know that you have shown loyal love to my master.”
\verse And it happened \textit{that} before he had finished speaking, behold, Rebekah—who was born to Bethuel, son of Milcah, the wife of Nahor, the brother of Abraham—came out, and her jar \textit{was} on her shoulder.
\verse Now the girl \textit{was} very pleasing in appearance. \textit{She was} a virgin; no man had known her. And she went down to the spring, filled her jar, and came up.
\verse And the servant ran to meet her. And he said, “Please, let me drink a little of the water from your jar.”
\verse And she said, “Drink, my lord.” And she quickly lowered her jar in her hand and gave him a drink.
\verse When she finished giving him a drink she said, “I will also draw water for your camels until they finish drinking.”
\verse And she quickly emptied her jar into the trough and ran again to the well to draw water. And she drew water for all his camels.
\verse And the man \textit{was} gazing at her silently to know \textit{if} Adonai had made his journey successful or not.
\verse And it happened \textit{that} as the camels finished drinking the man took a gold ring of a half shekel in weight and two bracelets for her arms, ten shekels in weight,
\verse and said, “Please tell me, whose daughter \textit{are} you? Is there a place \textit{at} the house of your father for us to spend the night?”
\verse And she said to him, “I \textit{am} the daughter of Bethuel, son of Milcah, whom she bore to Nahor.”
\verse Then she said to him, “We have both straw and fodder in abundance, as well as a place to spend the night.”
\verse And the man knelt down and worshiped Adonai.
\verse And he said, “Blessed \textit{be} Adonai, God of my master Abraham, who has not withheld his loyal love and his faithfulness from my master. I \textit{was} on the way \textit{and} Adonai led me \textit{to} the house of my master’s brother.”
\verse Then the girl ran and reported these things to the household of her mother.
\verse Now Rebekah had a brother, and his name \textit{was} Laban. And Laban ran out to the man toward the spring.
\verse And when he saw the ring and the bracelets on the arms of his sister and heard the words of Rebekah his sister, \textit{who} said, “Thus the man spoke to me,” he went to the man. And behold, \textit{he was} standing with the camels at the spring.
\verse And he said, “Come, O blessed \textit{one} of Adonai. Why do you stand outside? Now I have prepared the house and a place for the camels.”
\verse And the man came to the house and unloaded the camels. And he gave straw and fodder to the camels, and water to wash his feet and the feet of the men who \textit{were} with him.
\verse \textit{And food was placed before him} to eat. And he said, “I will not eat until \textit{I have told my errand}.” And he said, “Speak.”
\verse And he said, “I \textit{am} the servant of Abraham.
\verse Now Adonai has blessed my master exceedingly, and he has become great. He has given to him sheep and cattle, silver and gold, male slaves and female slaves, and camels and donkeys.
\verse And Sarah, the wife of my master, has borne a son to my master after her old age. And he has given to him all that he has.
\verse And my master made me swear, saying, ‘Do not take a wife for my son from the daughters of the Canaanites in whose land I am living.
\verse But you shall go to the house of my father, and to my family, and you shall take a wife for my son.’
\verse And I said to my master, ‘Perhaps the woman will not \textit{follow} me.’
\verse And he said to me, ‘Adonai, before whom I have walked, shall send his angel with you and will make your journey successful. And you shall take a wife for my son from my family, and from the house of my father.
\verse Then you shall be released from my oath, when you come to my family. And if they will not give \textit{a woman} to you, then you will be released from my oath.’
\verse Then today I came to the spring, and I said, ‘O Adonai, God of my master Abraham, \textit{if you would please make my journey successful}, upon which I am going.
\verse Behold, I am standing by the spring of water. Let it be \textit{that} the young woman who comes out to draw water and to whom I say, “Please give me a little water to drink from your jar,”
\verse let her say to me, “Drink; I will also draw water for your camels,” she \textit{is} the woman whom Adonai has appointed for the son of my master.’
\verse I had not yet finished speaking to myself when, behold, Rebekah \textit{was} coming out with her jar on her shoulder. And she went down to the spring and drew water. And I said to her, ‘Please give me a drink.’
\verse And she hastened and let down her jar \textit{from her shoulder} and said, ‘Drink, and I will give a drink to your camels also.’ Then I drank and she gave a drink to the camels also.
\verse Then I asked her and said, ‘Whose daughter \textit{are} you?’ And she said, ‘The daughter of Bethuel, son of Nahor, whom Milcah bore to him.’ And I put the ring on her nose and the bracelets on her arms.
\verse And I knelt down and worshiped Adonai, and I praised Adonai, the God of my master Abraham, who led me on the right way, to take the daughter of the brother of my master for his son.
\verse So now, \textit{if you are going to deal loyally and truly} with my master, tell me. And if not, tell me, so that I may turn to \textit{the} right or to \textit{the} left.”
\verse Then Laban and Bethuel answered, and they said, “The matter has gone out from Adonai; we are not able to speak bad or good to you.
\verse Here \textit{is} Rebekah before you. Take \textit{her} and go; let her be a wife for the son of your master as Adonai has spoken.”
\verse And it happened \textit{that} when the servant of Abraham heard their words he bowed down to the ground to Adonai.
\verse And the servant brought out silver jewelry and gold jewelry, and garments, and he gave \textit{them} to Rebekah. And he gave precious gifts to her brother and to her mother.
\verse And he and the men who \textit{were} with him ate and drank, and they spent the night. And they got up in the morning, and he said, “Let me go to my master.”
\verse And her brother and her mother said, “Let the girl remain with us ten days \textit{or so}; after \textit{that} she may go.”
\verse And he said to them, “Do not delay me. Now, Adonai has made my journey successful. Let me go. I must go to my master.”
\verse And they said, “Let us call the girl and ask \textit{her opinion}.”
\verse And they called Rebekah and said to her, “Will you go with this man?” And she said, “I will go.”
\verse So they sent away Rebekah their sister, and her nurse, and the servant of Abraham and his men.
\verse And they blessed Rebekah and said to her, “You \textit{are} our sister; may you become countless thousands; and may your offspring take possession of the gate of his enemies.”
\verse And Rebekah and her maidservants arose, and they mounted the camels and \textit{followed} the man. And the servant took Rebekah and left.
\verse Now Isaac \textit{was} coming from the direction of Beer-Lahai-Roi. And he \textit{was} living in the land of the Negev.
\verse And Isaac went out to meditate in the field \textit{early in the evening}, and he lifted up his eyes and saw—behold, camels were coming.
\verse And Rebekah lifted up her eyes and saw Isaac. And she got down from the camel.
\verse And she said to the servant, “Who \textit{is} this man walking around in the field to meet us?” And the servant said, “That \textit{is} my master.” And she took her veil and covered herself.
\verse And the servant told Isaac all the things that he had done.
\verse And Isaac brought her to the tent of Sarah his mother. And he took Rebekah, and she became his wife. And Isaac loved her and was comforted after \textit{the death of} his mother.
\end{biblechapter}

\begin{biblechapter} % Genesis 25
\verseWithHeading{The Death and Descendants of Abraham} Now Abraham again took a wife, and her name \textit{was} Keturah.
\verse And she bore to him Zimran, Jokshan, Medan, Midian, Ishbak, and Shuah.
\verse And Jokshan fathered Sheba and Dedan. And the sons of Dedan were Asshurim and Letushim and Leummim.
\verse And the sons of Midian \textit{were} Ephah, Epher, Hanoch, Abidah, and Eldaah. All of these \textit{were} the children of Keturah.
\verse And Abraham gave all he had to Isaac.
\verse But to the sons of Abraham’s concubines Abraham gave gifts. And while he \textit{was} still living he sent them away eastward, \textit{away} from his son Isaac, to the land of the east.
\verse Now these \textit{are} the days of the years of \textit{the life of Abraham}: one hundred and seventy-five years.
\verse And Abraham passed away and died in a good old age, old and full of years. And he was gathered to his people.
\verse And Isaac and Ishmael his sons buried him in the cave of Machpelah, in the field of Ephron, son of Zohar the Hittite, that \textit{was} east of Mamre,
\verse the field that Abraham had bought from the Hittites. There Abraham was buried and Sarah his wife.
\verse And it happened \textit{that} after the death of Abraham God blessed Isaac his son, and Isaac settled at Beer-Lahai-Roi.
\verse Now these \textit{are} the generations of Ishmael, the son of Abraham, that Hagar the Egyptian, the maidservant of Sarah, bore to Abraham.
\verse And these are the names of the sons of Ishmael, by their names according to their family records. The firstborn of Ishmael \textit{was} Nebaioth, then Kedar, Adbeel, Mibsam,
\verse Mishma, Dumah, Massa,
\verse Hadad, Tema, Jetur, Naphish, and Kedemah.
\verse These \textit{are} the sons of Ishmael, and these \textit{are} their names by their villages and by their encampments—12 leaders according to their tribes.
\verse Now these \textit{are} the years of the life of Ishmael: 137 years. And he passed away and died, and was gathered to his people.
\verse They settled from Havilah to Shur, which \textit{was} opposite Egypt, going toward Asshur, opposite; he \textit{settled} opposite all his brothers.
\verseWithHeading{Jacob and Esau} Now these \textit{are} the generations of Isaac, the son of Abraham. Abraham fathered Isaac,
\verse And Isaac was \textit{forty years old} when he took Rebekah, the daughter of Bethuel the Aramean of Paddan-Aram, the sister of Laban the Aramean, as his wife.
\verse And Isaac prayed to Adonai on behalf of his wife, for she \textit{was} barren. And Adonai responded to his prayer, and Rebekah his wife conceived.
\verse And the children in her womb jostled each other, and she said, “\textit{If it is going to be like this, why be pregnant}?” And she went to inquire of Adonai.
\verse And Adonai said to her, “Two nations \textit{are} in your womb, and two peoples \textit{from birth} shall be divided. And \textit{one people shall be stronger than the other}. And \textit{the} elder shall serve \textit{the} younger.”
\verse And when her days to give birth were completed, then—behold—twins \textit{were} in her womb.
\verse And the first came out red, all \textit{his body} \textit{was} like a hairy coat, so they called his name Esau.
\verse And afterward his brother came out, and his hand grasped the heel of Esau, so his name was called Jacob. And Isaac \textit{was sixty years old} at their birth.
\verse And the boys grew up. And Esau \textit{was} a skilled hunter, a man of the field, but Jacob \textit{was} a peaceful man, living \textit{in} tents.
\verse And Isaac loved Esau because \textit{he could eat of his game}, but Rebekah loved Jacob.
\verse Once Jacob cooked a thick stew, and Esau came in from the field, and he was exhausted.
\verse And Esau said to Jacob, “Give me \textit{some of that red stuff} to gulp down, for I am exhausted!” (Therefore his name was called Edom).
\verse Then Jacob said, “Sell me your birthright \textit{first}.”
\verse And Esau said, “Look, I am going to die; now what \textit{is} this birthright to me?”
\verse Then Jacob said, “Swear to me \textit{first}.” And he swore to him, and sold his birthright to Jacob.
\verse Then Jacob gave Esau bread, and thick lentil stew, and he ate and drank. Then he got up and went away. So Esau despised his birthright.
\end{biblechapter}

\begin{biblechapter} % Genesis 26
\verseWithHeading{Isaac and Abimelech} And there was a famine in the land, besides the former famine which was in the days of Abraham. And Isaac went to Abimelech, king of the Philistines, to Gerar.
\verse And Adonai appeared to him and said, “Do not go down to Egypt; dwell in the land which I will show to you.
\verse Dwell as an alien in this land, and I will be with you, and will bless you, for I will give all these lands to you and to your descendants, and I will establish the oath that I swore to Abraham you father.
\verse And I will multiply your descendants like the stars of heaven, and I will give to your descendants all these lands. And all nations of the earth will be blessed through your offspring,
\verse because Abraham listened to my voice and kept my charge: my commandments, my statutes, and my laws.”
\verse So Isaac settled in Gerar.
\verse When the men of the place asked concerning his wife, he said, “She \textit{is} my sister,” for he was afraid to say, “my wife,” thinking “the men of the place will kill me on account of Rebekah, for \textit{she was beautiful}.”
\verse And it happened \textit{that}, \textit{when he had been there a long time}, Abimelech the king of the Philistines looked through the window, and saw—behold—Isaac \textit{was} fondling Rebekah his wife.
\verse And Abimelech called Isaac and said, “Surely she \textit{is} your wife. Now why did you say ‘She \textit{is} my sister’?” And Isaac said to him, “Because I thought I would die on account of her.”
\verse And Abimelech said, “What \textit{is} this you have done to us? One of the people might easily have slept with your wife! Then you would have brought guilt upon us!”
\verse Then Abimelech instructed all the people, saying, “The \textit{one who} touches this man or his wife shall certainly die.”
\verse And Isaac sowed in that land and reaped in that \textit{same} year a hundredfold, and Adonai blessed him.
\verse And the man \textit{became wealthier and wealthier} until he was exceedingly wealthy.
\verse And he possessed sheep and cattle and many servants, so that the Philistines envied him.
\verse And the Philistines stopped up all the wells that the servants of his father had dug in the days of Abraham his father. They filled them with earth.
\verse And Abimelech said to Isaac, “Go \textit{away} from us, for you have become much too powerful for us.”
\verse So Isaac departed from there and camped in the valley of Gerar, and settled there.
\verse And Isaac dug again the wells of water which they had dug in the days of his father Abraham, which the Philistines had stopped up after the death of Abraham. And he gave to them \textit{the same names} which his father had given them.
\verse And when the servants of Isaac dug in the valley, they found a well of fresh water there.
\verse Then the herdsmen of Gerar quarreled with the herdsmen of Isaac, saying, “The water is ours.” And he called the name of the well Esek, because they contended with him.
\verse And they dug another well, and they quarreled over it also. And he called its name Sitnah.
\verse Then he moved from there and dug another well, and they did not quarrel over it. And he called its name Rehoboth, and said, “Now Adonai has made room for us, and we shall be fruitful in the land.”
\verse And from there he went up to Beersheba.
\verse And Adonai appeared to him that night and said, “I \textit{am} the God of your father Abraham. Do not be afraid, for I \textit{am} with you, and I will bless you and make your descendants numerous for the sake of my servant Abraham.”
\verse And he built an altar there and called on the name of Adonai. And he pitched his tent there, and the servants of Isaac dug a well there.
\verse Then Abimelech went to him from Gerar with Ahuzzath his friend and Phicol his army commander.
\verse And Isaac said to them, “Why have you come to me? You hate me and sent me away from you.”
\verse And they said, “We see clearly that Adonai has been with you, so we thought let there be an oath between us—between us and you—and let us \textit{make} a covenant with you
\verse that you may not do us harm just as we have not touched you, but have only done good to you and sent you away in peace. You \textit{are} now blessed by Adonai.”
\verse So he made a meal for them, and they ate and drank.
\verse And they arose early in the morning and each one swore to the other, and Isaac sent them away. And they left him in peace.
\verse And it happened \textit{that} on that same day the servants of Isaac came and told him about the well that they had dug. And they said, “We have found water!”
\verse And he called it Sheba. Therefore the name of the city \textit{is} Beersheba unto this day.
\verse And \textit{when} Esau was forty years old he took as wife Judith, daughter of Beeri the Hittite, and Basemath, daughter of Elon the Hittite.
\verse And \textit{they made life bitter} for Isaac and Rebekah.
\end{biblechapter}

\begin{biblechapter} % Genesis 27
\verseWithHeading{Jacob Steals Esau’s Blessing} And it happened \textit{that} when Isaac \textit{was} old and \textit{his eyesight was weak}, he called Esau his older son and said to him, “My son.” And he said to him, “Here I \textit{am}.”
\verse And he said, “Look, I \textit{am} old; I do not know the day of my death.
\verse So now, take your weapons, your quiver and your bow, and go out to the field and hunt food for me.
\verse Then make for me tasty food like I love, and bring \textit{it} to me. And I will eat \textit{it} so that I can bless you before I die.
\verse Now Rebekah \textit{was} listening as Isaac spoke to Esau his son, and \textit{when} Esau went to the field to hunt wild game to bring \textit{back},
\verse Rebekah said to Jacob her son, “Look, I heard your father speaking to Esau your brother saying,
\verse ‘Bring wild game to me and prepare tasty food so I can eat \textit{it} and bless you before Adonai before my death.’
\verse So now, my son, listen to my voice, to what I command you.
\verse Go to the flock and take two good young goats from it for me, and I will prepare them \textit{as} tasty food for your father, just as he likes.
\verse Then you must take it to your father and he will eat \textit{it} so that he may bless you before his death.”
\verse Then Jacob said to his mother, “Behold, Esau my brother \textit{is} a hairy man, but I \textit{am} a smooth man.
\verse Perhaps my father will feel me and I will be in his eyes \textit{as} a mocker, and he will bring upon me a curse and not a blessing.”
\verse Then his mother said to him, “Your curse be upon me, my son, only listen to my voice—go and get \textit{them} for me.”
\verse So he went and took \textit{them}, and brought \textit{them} to his mother, and his mother prepared tasty food as his father liked.
\verse Then Rebekah took \textit{some of} her older son Esau’s best garments that \textit{were} with her in the house, and she put \textit{them} on Jacob her younger son.
\verse And she put the skins of the young goats over his hands and over the smooth \textit{part of} his neck.
\verse And she put the tasty food and the bread that she had made into the hand of Jacob, her son.
\verse And he went to his father and said, “My father.” And he said, “Here I \textit{am}. Who \textit{are} you, my son?”
\verse And Jacob said to his father, “I \textit{am} Esau, your firstborn. I have done as you told me. Please get up, sit up and eat from my wild game so that you may bless me.”
\verse Then Isaac said to his son, “\textit{How} did you find \textit{it} so quickly, my son?” And he said, “Because Adonai your God \textit{caused me to find it}.”
\verse Then Isaac said to Jacob, “Please, come near and let me feel you, my son. \textit{Are you really} my son Esau or not?”
\verse And Jacob drew near to Isaac his father. And he felt him and said, “The voice \textit{is} the voice of Jacob, but the hands \textit{are} the hands of Esau.”
\verse And he did not recognize him because his hands were hairy like the hands of Esau his brother. And he blessed him.
\verse And he said, “\textit{Are you really} my son Esau?” And he said, “I \textit{am}.”
\verse Then he said, “Bring \textit{it} near to me that I may eat from the game of my son, so that I may bless you.” And he brought \textit{it} to him, and he ate. And he brought wine to him, and he drank.
\verse Then his father Isaac said to him, “Come near and kiss me, my son.”
\verse And he drew near and kissed him. And he smelled the smell of his garments, and he blessed him and said,
\verse “Look, the smell of my son \textit{is} like the smell of a field that Adonai has blessed!
\verse May God give you of the dew of heaven 
and of the fatness of the earth, 
and abundance of grain and new wine.
\verse And as soon as Isaac had finished blessing Jacob, \textit{immediately after} Jacob had gone out from the presence of Isaac his father, Esau his brother came \textit{back} from his hunting.
\verse He too prepared tasty food and brought \textit{it} to his father. And he said to his father, “Let my father arise and eat from the wild game of his son, that you may bless me.”
\verse And Isaac his father said to him, “Who \textit{are} you?” And he said, “I \textit{am} your son, your firstborn, Esau.”
\verse Then Isaac \textit{trembled violently}. Then he said, “Who then \textit{was} he that hunted wild game and brought \textit{it} to me, and I ate \textit{it} all before you came, and I blessed him? Moreover, he will be blessed!”
\verse When Esau heard the words of his father he cried out \textit{with} a great and exceedingly bitter cry of distress. And he said to his father, “Bless me as well, my father!”
\verse And he said, “Your brother came in deceit and took your blessing.”
\verse Then he said, “\textit{Isn’t that why he is named Jacob}? He has deceived me these two times. He took my birthright and, look, now he has taken my blessing!” Then he said, “Have you not reserved a blessing for me?”
\verse Then Isaac answered and said to Esau, “Behold, I have made him lord over you and I have given him all his brothers as servants, and \textit{with} grain and wine I have sustained him. Now what can I do for you, my son?”
\verse And Esau said to his father, “Have you only one blessing, my father? Bless me also, my father!” And Esau lifted up his voice and wept.
\verse Then Isaac his father answered and said to him,
\verse “Your home shall be from the fatness of the land, 
and from the dew of heaven above.
\verse Then Esau held a grudge against Jacob on account of the blessing with which his father had blessed him. And Esau said in his heart, “The days of mourning for my father are coming, then I will kill Jacob my brother.”
\verse But the words of Esau her older son were told to Rebekah. And she sent and called for her younger son Jacob. And she said to him, “Look, Esau your brother \textit{is} consoling himself concerning you, \textit{intending} to kill you.
\verse Now then, my son, listen to my voice; arise and flee to Haran to Laban my brother.
\verse Stay with him a few days until the wrath of your brother has turned—
\verse until the anger of your brother turns from you and he has forgotten what you have done to him. Then I will send and bring you from there. Why should I lose the two of you in one day?”
\verse Then Rebekah said to Isaac, “I loathe my life because of the Hittite women. If Jacob takes a wife from Hittite women like these, from the \textit{native women}, \textit{what am I living for}?”
\end{biblechapter}

\begin{biblechapter} % Genesis 28
\verseWithHeading{Jacob Flees to Haran} Then Isaac called Jacob and blessed him. And he instructed him and said to him, “You must not take a wife from the daughters of Canaan.
\verse Arise, go to Paddan-Aram, to the house of Bethuel, your mother’s father, and take for yourself a wife from there, from the daughters of Laban your mother’s brother.
\verse Now, may El-Shaddai bless you, and make you fruitful, and multiply you, so that you become an assembly of peoples.
\verse And may he give you the blessing of Abraham, to you and to your descendants with you, that you may take possession of the land of your sojourning, which God gave to Abraham.”
\verse Then Isaac sent Jacob away, and he went to Paddan-Aram, to Laban the son of Bethuel the Aramean, the brother of Rebekah, the mother of Jacob and Esau.
\verse Now Esau saw that Isaac had blessed Jacob and sent him away to Paddan-Aram, to take for himself a wife from there, and he blessed him and instructed him, saying, “You must not take a wife from the daughters of Canaan,”
\verse and \textit{that} Jacob listened to his father and to his mother and went to Paddan-Aram.
\verse Then Esau saw that the daughters of Canaan \textit{were} evil in the eyes of Isaac his father,
\verse then Esau went to Ishmael and took Mahalath, the daughter of Ishmael, son of Abraham, sister of Nebaioth, as a wife, in addition to the wives he had.
\verseWithHeading{Jacob’s Dream} Then Jacob went out from Beersheba and went to Haran.
\verse And he arrived at a \textit{certain} place and spent the night there, because the sun had set. And he took \textit{one} of the stones of the place and put \textit{it} under his head and slept at that place.
\verse And he dreamed, and behold, a stairway was set on the earth, and its top touched the heavens. And behold, angels of God \textit{were} going up and going down on it.
\verse And behold, Adonai \textit{was} standing beside him, and he said, “I \textit{am} Adonai, the God of Abraham your father, and the God of Isaac. The ground on which you \textit{were} sleeping I will give to you and to your descendants.
\verse Your descendants shall be like the dust of the earth, and you will spread out to the west, and to the east, and to the north and to the south. And all the families of the earth will be blessed through you and through your descendants.
\verse Now behold, I \textit{am} with you, and I will keep you wherever you go. And I will bring you to this land, for I will not leave you until I have done what I have promised to you.”
\verse Then Jacob awoke from his sleep and said, “Surely Adonai \textit{is indeed} in this place and I did not know!”
\verse Then he was afraid and said, “How awesome \textit{is} this place! \textit{This is nothing else than the house of God}, and this is the gate of heaven!”
\verse And Jacob rose early in the morning, and he took the stone that he had put under his head and set it up \textit{as} a stone pillar, and poured oil on top of it.
\verse And he called the name of that place Bethel; however, the name of the city \textit{was} formerly Luz.
\verse And Jacob made a vow saying, “If God will be with me and protect me on this way that I am going, and gives me food to eat and clothing to wear,
\verse and \textit{if} I return in peace to the house of my father, then Adonai will become my God.
\verse And this stone that I have set up \textit{as} a pillar shall be the house of God, and \textit{of} all that you give to me I will certainly give a tenth to you.”
\end{biblechapter}

\begin{biblechapter} % Genesis 29
\verseWithHeading{Jacob Flees to Haran} And Jacob \textit{continued his journey} and went to the land of the Easterners.
\verse And he looked, and behold, \textit{there was} a well in the field, and behold, there \textit{were} three flocks of sheep lying beside it, for out of that well the flocks were watered. And the stone on the mouth of the well \textit{was} large.
\verse And \textit{when} all the flocks were gathered there, they rolled away the stone from the mouth of the well. And they watered the sheep and returned the stone upon the mouth of the well to its place.
\verse And Jacob said to them, “My brothers, where \textit{are} you from?” And they said, “We \textit{are} from Haran.”
\verse And he said to them, “Do you know Laban, son of Nahor?” And they said, “We know \textit{him}.”
\verse And he said to them, “\textit{Is he well}?” And they said, “\textit{He is} well. Now look, Rachel his daughter is coming with the sheep.”
\verse And he said, “Look, \textit{it is} still \textit{broad daylight}; it is not the time \textit{for} the livestock to be gathered. Give water to the sheep and go, pasture them.”
\verse And they said, “We are not able, until all the flocks are gathered. Then the stone is rolled away from the mouth of the well, and we water the sheep.”
\verse While he was speaking with them, Rachel came with the sheep which belonged to her father, for she was pasturing \textit{them}.
\verse And it happened \textit{that}, when Jacob saw Rachel, the daughter of Laban, his mother’s brother, and the sheep of Laban, his mother’s brother, Jacob drew near and rolled away the stone from the mouth of the well and watered the sheep of Laban, his mother’s brother.
\verse And Jacob kissed Rachel, and lifted up his voice and wept.
\verse And Jacob told Rachel that he \textit{was} the relative of her father, and that he \textit{was} the son of Rebekah. And she ran and told her father.
\verse And it happened \textit{that} when Laban heard the message about Jacob, the son of his sister, he ran to meet him. And he embraced him and kissed him, and brought him to his house. And he told Laban all these things.
\verse And Laban said to him, “Surely you \textit{are} my flesh and my bone!” And he stayed with him a month.
\verseWithHeading{Jacob’s Marriages} Then Laban said to Jacob, “\textit{Just} because you \textit{are} my brother should you work for me for nothing? Tell me what your wage \textit{should be}.”
\verse Now Laban had two daughters. The name of the older \textit{was} Leah, and the name of the younger \textit{was} Rachel.
\verse Now the eyes of Leah \textit{were} dull, but Rachel was beautiful in form and appearance.
\verse And Jacob loved Rachel and said, “I will serve you seven years for Rachel your younger daughter.”
\verse Then Laban said, “Better \textit{that} I give her to you than I give her to another man. Stay with me.”
\verse And Jacob worked for Rachel seven years, but they were as a few days in his eyes because he loved her.
\verse And Jacob said to Laban, “Give \textit{me} my wife, that I may go in to her, for \textit{my time} is completed.”
\verse So Laban gathered all the men of the place and prepared a feast.
\verse And it happened \textit{that} in the evening he took Leah his daughter and brought her to him, and he went in to her.
\verse And Laban gave Zilpah his female servant to her, to Leah his daughter \textit{as} a female servant.
\verse And it happened \textit{that} in the morning, behold, it \textit{was} Leah! And he said to Laban, “What \textit{is} this you have done to me? Did I not serve with you for Rachel? Now why did you deceive me?”
\verse Then Laban said, “\textit{It is not the custom} in our country to give the younger before the firstborn.
\verse Complete the week of this one, then I will also give you the other, \textit{on the condition that you will work for me} another seven years.”
\verse And Jacob did so. So he completed the week of this \textit{one}, then he gave Rachel his daughter to him as a wife.
\verse And Laban gave Bilhah his female servant to Rachel his daughter as a female servant.
\verse Then he also went in to Rachel, and he loved Rachel more than Leah. And he served with him yet another seven years.
\verseWithHeading{Jacob’s Children} When Adonai saw that Leah \textit{was} unloved he opened her womb, but Rachel \textit{was} barren.
\verse Then Leah conceived and gave birth to a son, and she called his name Reuben, for she said, “Because Adonai has noticed my misery, that I \textit{am} unloved. Now my husband will love me.”
\verse And she conceived again and gave birth to a son. And she said, “\textit{It is} because Adonai has heard that I \textit{am} unloved that he gave me this \textit{son} also.” And she called his name Simeon.
\verse And she conceived again and gave birth to a son. Then she said, “Now this time my husband will be joined to me, for I have borne him three sons.” Therefore, she called his name Levi.
\verse And she conceived again and gave birth to a son. And she said, “This time I will praise Adonai.” Therefore she called his name Judah. And she ceased bearing children.
\end{biblechapter}

\begin{biblechapter} % Genesis 30
\verseWithHeading{Jacob’s Children} When Rachel saw that she could not bear children to Jacob, Rachel envied her sister. And she said to Jacob, “Give me children—if not, I will die!”
\verse And Jacob \textit{became angry} with Rachel. And he said, “\textit{Am} I in the place of God, who has withheld from you the fruit of the womb?”
\verse Then she said, “Here \textit{is} my servant girl Bilhah; go in to her that she may bear children \textit{as my surrogate}. Then I will even \textit{have children} by her.”
\verse Then she gave him Bilhah, her female servant, as a wife, and Jacob went in to her
\verse And Bilhah conceived and gave birth to a son for Jacob.
\verse Then Rachel said, “God has judged me, and has also heard my voice, and has given me a son.” Therefore she called his name Dan.
\verse And Bilhah, Rachel’s servant, conceived again and bore a second son to Jacob.
\verse And Rachel said, “I have struggled a mighty struggle with my sister and have prevailed.” And she called his name Naphtali.
\verse When Leah saw that she had ceased bearing children, she took Zilpah her female servant and gave her to Jacob as a wife.
\verse And Zilpah, the female slave of Leah, bore a son to Jacob.
\verse Then Leah said, “Good fortune!” And she called his name Gad.
\verse And Zilpah, Leah’s female servant, bore a second son to Jacob.
\verse Then Leah said, “How happy \textit{am} I! For women have called me happy.” So she called his name Asher.
\verse And in the days of the wheat harvest, Reuben went and found mandrakes in the field and he brought them to Leah his mother. And Rachel said to Leah, “Please give me some of your son’s mandrakes.”
\verse And she said to her, “\textit{Is} your taking my husband \textit{such} a small \textit{thing} that you will also take the mandrakes of my son?” Then Rachel said, “Then he may sleep with you tonight in exchange for your son’s mandrakes.”
\verse When Jacob came in from the field in the evening, Leah went out to meet him. And she said, “Come in to me, for \textit{I have hired} you with my son’s mandrakes.” And he slept with her that night.
\verse And God listened to Leah and she conceived and gave birth to a fifth son for Jacob.
\verse Then Leah said, “God has given \textit{me} my wage since I gave my servant girl to my husband.” And she called his name Issachar.
\verse And Leah conceived again and gave birth to a sixth son for Jacob.
\verse And Leah said, “God has endowed me with a good gift. This time my husband will acknowledge me, because I bore him six sons.” And she called his name Zebulun.
\verse And afterward she gave birth to a daughter. And she called her name Dinah.
\verse Then God remembered Rachel and listened to her, and God opened her womb.
\verse And she conceived and gave birth to a son. And she said, “God has taken away my disgrace.”
\verse And she called his name Joseph, saying, “Adonai has added to me another son.”
\verseWithHeading{Jacob’s Prosperity} And it happened \textit{that} as soon as Rachel had given birth to Joseph, Jacob said to Laban, “Send me away that I may go to my place and my land.
\verse Give \textit{me} my wives and my children for which I have served you, and let me go. For you yourself know my service that I have rendered to you.”
\verse But Laban said to him, “Please, if I have found favor in your eyes, I have learned by divination that Adonai has blessed me because of you.”
\verse And he said, “Name your wage to me and I will give \textit{it}.”
\verse Then he said to him, “You yourself know how I have served you and how your livestock have been with me.
\verse For you had little before me, and it has increased abundantly. And Adonai has blessed you \textit{wherever I turned}. So then, when shall I provide for my own family also?”
\verse And he said, “What shall I give you?” And Jacob said, “Do not give me anything. If you will do this thing for me, I will again feed your flocks and keep \textit{them}.
\verse Let me pass through all your flocks today, removing all the speckled and spotted sheep from them, along with every dark-colored sheep among the sheep, and the spotted and speckled among the goats. That shall be my wages.
\verse And my righteousness will answer for me \textit{later} when you come concerning my wages before you. Every \textit{one} that \textit{is} not speckled or spotted among the goats, or dark-colored among the sheep shall be stolen \textit{if it is} with me.”
\verse Then Laban said, “Look! Very well. It shall be according to your word.”
\verse But that day he removed the streaked and spotted male goats and all the speckled and spotted female goats, all that \textit{had} white on it, and every dark-colored ram, and \textit{put them in the charge of his sons}.
\verse And he put a journey of three days between him and Jacob, and Jacob pastured the remainder of Laban’s flock.
\verse Then Jacob took fresh branches of poplar, almond, and plane trees and peeled white strips on them, exposing the white which \textit{was} on the branches.
\verse And he set the branches that he had peeled in front of the flocks, in the troughs \textit{and} in the water containers. And they were in heat when they came to drink.
\verse And the flocks mated by the branches, so the flocks bore streaked, speckled, and spotted.
\verse And Jacob separated the lambs and turned the faces of the flocks toward the streaked and all the dark-colored in Laban’s flocks. And he put his own herds apart, and did not put them with the flocks of Laban.
\verse And whenever any of the stronger of the flocks were in heat, Jacob put the branches \textit{in full view} of the flock in the troughs that they might mate among the branches.
\verse But with the more feeble of the flock he would not put \textit{them there}. So the feebler were Laban’s and the stronger \textit{were} Jacob’s.
\verse And the man became \textit{exceedingly} rich and had large flocks, female slaves, male slaves, camels, and donkeys.
\end{biblechapter}

\begin{biblechapter} % Genesis 31
\verseWithHeading{Jacob Flees from Laban} Now he heard the words of the sons of Laban, saying, “Jacob has taken all that our father has,” and “From that which \textit{was} our father’s he has gained all this wealth.”
\verse Then Jacob saw the face of Laban and, behold, \textit{it was not like it had been in the past}.
\verse And Adonai said to Jacob, “Return to the land of your ancestors and to your family, and I will be with you.”
\verse So Jacob sent and called Rachel and Leah to the field, to his flocks,
\verse and he said to them, “Look, I see the face of your father, that \textit{it is not like it has been toward me in the past}. But the God of my father is with me.
\verse Now you yourselves know that I have served your father with all my strength,
\verse and your father has cheated me and changed my wages ten times, but God has not allowed him to harm me.
\verse If thus he said, ‘Speckled shall be your wage,’ then all the flock bore speckled. And \textit{if} he said, ‘Streaked shall be your wage,’ then all the flock bore streaked.
\verse God has taken away your father’s livestock and given \textit{them} to me.
\verse Now it happened \textit{that} at the time of the mating of the flock I lifted up my eyes and saw in a dream, and behold, the rams mounting the flock \textit{were} streaked, speckled, and dappled.
\verse Then the angel of God said to me in the dream, ‘Jacob,’ and I said, ‘Here I \textit{am}.’
\verse And he said, ‘Lift up your eyes and see—all the rams mounting the flock \textit{are} streaked, speckled, and dappled, for I have seen all that Laban is doing to you.
\verse I \textit{am} the God of Bethel where you anointed a stone pillar, where you made a vow to me. Now get up, go out from this land and return to the land of your birth.’ ”
\verse Then Rachel and Leah answered and said to him, “\textit{Is there} yet a portion for us, and an inheritance in the house of our father?
\verse Are we not regarded \textit{as} foreigners by him, because he has sold us and completely consumed our money?
\verse For all the wealth that God has taken away from our father, it belongs to us and to our sons. So now, all that God has said to you, do.”
\verse Then Jacob got up and put his children and his wives on the camels.
\verse And he drove all his livestock and his possessions that he had acquired, the livestock of his possession that he had acquired in Paddan-Aram, in order to go to Isaac his father, to the land of Canaan.
\verse Now Laban had gone to shear his sheep, and Rachel stole the idols that belonged to her father.
\verse And Jacob \textit{tricked} Laban the Aramean by not telling him that he \textit{intended to} flee.
\verse Then he fled with all that he had, and arose and crossed the Euphrates and set his face toward the hill country of Gilead.
\verse And on the third day it was told to Laban that Jacob had fled.
\verse Then he took his kinsmen with him and pursued after him, a seven-day journey, and he caught up with him in the hill country of Gilead.
\verse And God came to Laban the Aramean in a dream at night and said to him, “\textit{Take care} that you not speak with Jacob, whether good or evil.”
\verse And Laban overtook Jacob. Now Jacob had pitched his tent in the hill country, and Laban and his kinsmen pitched \textit{their tents} in the hill country of Gilead.
\verse Then Laban said to Jacob, “What have you done that you \textit{tricked me} and have carried off my daughters like captives of the sword?
\verse Why did you hide \textit{your intention} to flee and \textit{trick me}, and did not tell me so that I would have sent you away with joy and song and tambourine and lyre?
\verse And \textit{why} did you not give me opportunity to kiss my grandsons and my daughters \textit{goodbye}? Now you have behaved foolishly \textit{by} doing \textit{this}.
\verse \textit{It is in my power} to do harm to you, but the God of your father spoke to me last night saying, ‘\textit{Take care} from speaking with Jacob, whether good or evil.’
\verse Now, you have surely gone because you desperately longed for the house of your father, \textit{but} why did you steal my gods?”
\verse Then Jacob answered and said to Laban, “Because I \textit{was} afraid, for I thought, ‘Lest you take your daughters from me by force.’
\verse \textit{But} with whomever you find your gods, he shall not live. In the presence of your kinsmen \textit{now} identify what \textit{is} with me \textit{that is} yours and take it.” Now Jacob did not know that Rachel had stolen them.
\verse Then Laban went into Jacob’s tent and Leah’s tent and the tent of the two female servants and did not find \textit{his gods}. And he came out of Leah’s tent and went into Rachel’s tent.
\verse Now Rachel had taken the idols and put them in the saddle bag of the camel and sat on them. And Jacob searched the whole tent thoroughly but did not find them.
\verse And she said to her father, “Let there not be anger in the eyes of my lord, for I am not able to rise before you, for the way of women \textit{is} with me. And he searched carefully and did not find the idols.
\verse Then Jacob became angry and quarreled with Laban. Jacob answered and said to Laban, “What \textit{is} my offense? What \textit{is} my sin that you pursued after me?
\verse For you have searched all my possessions and what did you find among all the possessions of my household? Set it before my kinsmen and your kinsmen that they may decide between the two of us!
\verse These twenty years I \textit{was} with you; your ewes and your female goats did not miscarry, and the rams of your flocks I did not eat.
\verse I brought no mangled carcass to you—I bore its loss. From my hand you sought it, whether stolen by day or stolen by night.
\verse \textit{There} I was, during the day the heat consumed me, and the cold by night, and my sleep fled from my eyes.
\verse These twenty years \textit{I have been} in your house. I served you fourteen years for your two daughters and six years for your flock, and you have changed my wages ten times.
\verse If the God of my father, the God of Abraham and the Fear of Isaac had not been with me, indeed now you would have sent me away empty-handed. God saw my misery and the labor of my hands and rebuked you last night.”
\verse Then Laban answered and said to Jacob, “The daughters \textit{are} my daughters and the grandsons \textit{are} my grandsons, and the flocks \textit{are} my flocks, and all that you see, it \textit{is} mine. Now, what can I do for these my daughters today, or for their children whom they have borne?
\verse So now, come, let us \textit{make} a covenant, you and I, and let it be a witness between me and you.”
\verse And Jacob took a stone and set it up \textit{as} a stone pillar.
\verse And Jacob said to his kinsmen, “Gather stones.” And they took stones and made a pile of stones, and they ate there by the pile of stones.
\verse And Laban called it Jegar Sahadutha, but Jacob called it Galeed.
\verse Then Laban said, “This pile of stones \textit{is} a witness between me and you today.” Therefore its name is called Galeed,
\verse and Mizpah, because he said, “Adonai watch between me and you when \textit{we are out of sight of each other}.
\verse If you mistreat my daughters, and if you take wives besides my daughters, \textit{when} there is no man with us, see—God \textit{is} a witness between me and you.”
\verse And Laban said to Jacob, “See, this pile of stones, and see the pillar that I have set up between me and you.
\verse This pile of stones \textit{is} a witness, and the pillar \textit{is} a witness, that I will not pass beyond this pile of stones to you, and that you will not pass beyond this pile of stones and this pillar to me intending harm.
\verse May the God of Abraham and the God of Nahor, the God of their father judge between us.” Then Jacob swore by the Fear of his father Isaac.
\verse And Jacob sacrificed a sacrifice on the hill, and he called his kinsmen to eat the meal. And they ate the meal and spent the night on the hill.
\verse  And Laban arose early in the morning and kissed his grandsons and his daughters, and blessed them. Then Laban departed and returned to his homeland.
\end{biblechapter}

\begin{biblechapter} % Genesis 32
\verseWithHeading{Jacob Fears Esau} And Jacob went on his way, and angels of God met him.
\verse And when he saw them, Jacob said, “This \textit{is} the camp of God!” And he called the name of that place Mahanaim.
\verse Then Jacob sent messengers before him to Esau his brother, to the land of Seir, the territory of Edom.
\verse And he instructed them, saying, “Thus you must say to my lord, to Esau, ‘Thus says your servant Jacob, I have dwelled as an alien with Laban, and I have remained \textit{there} until now.
\verse And I have acquired cattle, male donkeys, flocks, and male and female slaves, and I have sent to tell my lord, to find favor in your eyes.’ ”
\verse And the messengers returned to Jacob \textit{and} said, “We came to your brother, to Esau, and he is coming to meet you, and four hundred men \textit{are} with him.”
\verse Then Jacob was very frightened and distressed. So he divided the people, flocks, cattle, and camels that \textit{were} with him into two companies.
\verse And he thought, “If Esau comes to one company and destroys it, the remaining company will be \textit{able} to escape.”
\verse Then Jacob said, “O God of my father Abraham, and God of my father Isaac, O Adonai, who said to me, ‘Return to your land and to your family, and I will deal well with you.’
\verse \textit{I am not worthy} of all the loyal love and all the faithfulness that you have shown your servant, for with \textit{only} my staff I crossed this Jordan, and now I have become two camps.
\verse Please rescue me from the hand of my brother, from the hand of Esau, for I fear him, lest he come and attack mother and children \textit{alike}.
\verse Now you yourself said, ‘I will surely deal well with you and make your offspring as the sand of the sea that cannot be counted for abundance.’ ”
\verse And he lodged there that night. Then he took \textit{from what he had with him} a gift for Esau his brother:
\verse two hundred female goats, twenty male goats, two hundred ewes, twenty rams,
\verse thirty milk camels with their young, forty cows, ten bulls, twenty female donkeys, and ten male donkeys.
\verse And he put \textit{them} under the hand of his servants, \textit{herd by herd}, and said to his servants, “Cross on ahead before me, and put some distance \textit{between herds}.
\verse And he instructed the foremost, saying, “When Esau my brother comes upon you and asks you, saying, ‘Whose \textit{are} you and where are you going? To whom do these \textit{animals} belong ahead of you?’
\verse Then you must say, ‘To your servant, to Jacob. It \textit{is} a gift sent to my lord, to Esau. Now behold, he \textit{is} also \textit{coming} after us.’ ”
\verse And he also instructed the second \textit{servant} and the third, and everyone \textit{else} who \textit{was} behind the herds, saying, “You must speak to Esau according to this word when you find him.
\verse And moreover, you shall say, ‘Look, your servant Jacob \textit{is} behind us.’ ” For he thought, “\textit{Let me appease him} with the gift going before me, and afterward I will see his face. Perhaps he will \textit{show me favor}.”
\verse So the gift passed on before him, but he himself spent that night in the camp.
\verseWithHeading{Jacob Wrestles with God} That night he arose and took his two wives, his two female servants, and his eleven children and crossed the ford of the Jabbok.
\verse And he took them and sent them across the stream. Then he sent across all his possessions.
\verse And Jacob remained alone, and a man wrestled with him until the breaking of the dawn.
\verse And when he saw that he could not prevail against him, he struck his hip socket, so that Jacob’s hip socket was sprained as he wrestled with him.
\verse Then he said, “Let me go, for dawn is breaking.” But he answered, “I will not let you go unless you bless me.”
\verse Then he said to him, “What \textit{is} your name?” And he said, “Jacob.”
\verse And he said, “Your name shall no longer be called Jacob, but Israel, for you have struggled with God and with men and have prevailed.”
\verse Then Jacob asked and said, “Please tell me your name.” And he said, “Why do you ask this—for my name?” And he blessed him there.
\verse Then Jacob called the name of the place Peniel \textit{which means} “I have seen God face to face and my life was spared.”
\verse Then the sun rose upon him as he passed Penuel, and he was limping because of his hip.
\verse Therefore the \textit{Israelites} do not eat the sinew of the sciatic nerve that \textit{is} upon the socket of the hip unto this day, because he struck the socket of the thigh of Jacob at the sinew of the sciatic nerve.
\end{biblechapter}

\begin{biblechapter} % Genesis 33
\verseWithHeading{Jacob Meets Esau and Settles at Shechem} And Jacob lifted up his eyes and looked. And behold, Esau \textit{was} coming and four hundred men \textit{were} with him. And he divided the children among Leah and among Rachel, and among the two of his female servants.
\verse And he put the female slaves and their children first, then Leah and her children next, then Rachel with Joseph last.
\verse And he himself passed on before them and bowed down to the ground seven times until he came to his brother.
\verse But Esau ran to meet him, and embraced him, and fell upon his neck and kissed him, and they wept.
\verse Then Esau lifted up his eyes and saw the women and the children and said, “Who \textit{are} these with you?” And he said, “The children whom God has graciously given your servant.”
\verse Then the female servants drew near, they and their children, and they bowed down.
\verse Then Leah and her children drew near and bowed down, and afterward Joseph and Rachel drew near and they bowed down.
\verse And he said, “\textit{What do you mean by} all this company that I have met?” Then he said, “To find favor in the eyes of my lord.”
\verse Then Esau said, “\textit{I have enough} my brother; \textit{keep what you have}.”
\verse And Jacob said, “No, please, if I have found favor in your eyes, you must take my gift from my hand, for then I have seen your face \textit{which is} like seeing the face of God, and you have received me.
\verse Please take my gift which has been brought to you, for God has dealt graciously with me, and because \textit{I have enough}.” And he urged him, so he took \textit{it}.
\verse Then he said, “Let us journey and go \textit{on}, and I will go ahead of you.”
\verse But he said to him, “My lord knows that the children \textit{are} frail, and the flocks and the cattle \textit{which are} nursing \textit{are a concern} to me. Now \textit{if} they drove them hard for a day all the flocks would die.
\verse Let my lord pass on before his servant and I will move along slowly at the pace of the livestock that are ahead of me, and at the pace of the children until I come to my lord in Seir.”
\verse And Esau said, “Let me leave some of my people with you.” But he said, “\textit{What need is there}? Let me find favor in the eyes of my lord.”
\verse So Esau turned that day on his way to Seir.
\verse But Jacob traveled on to Succoth, and he built for himself a house, and he made shelters for his livestock. Therefore he called the name of the place Succoth.
\verse And Jacob came safely to the city of Shechem which \textit{is} in the land of Canaan, \textit{on his way} from Paddan-Aram. And he camped before the city.
\verse And he bought a piece of land where he pitched his tent for one hundred pieces of money from the hand of the sons of Hamor, father of Shechem.
\verse And there he erected an altar and called it “El Elohe Israel.”
\end{biblechapter}

\begin{biblechapter} % Genesis 34
\verseWithHeading{The Rape of Dinah and the Massacre at Shechem} Now Dinah the daughter of Leah, whom she had borne to Jacob, went out to see the daughters of the land.
\verse And Shechem, the son of Hamor the Hivite, the prince of the land, saw her. And he took her and lay with her and raped her.
\verse And his soul clung to Dinah, the daughter of Jacob, and he loved the girl and spoke \textit{tenderly} to the girl.
\verse So Shechem said to Hamor his father, saying, “Get this girl for me as a wife.”
\verse And Jacob heard that Dinah his daughter had been defiled, but his sons were with his flocks in the field. And Jacob kept silent until they came.
\verse And Hamor, father of Shechem, went out to Jacob to speak with him.
\verse And the sons of Jacob came in from the field when they heard \textit{it}. And the men were distressed and very angry because he had done a disgraceful thing in Israel by having sexual relations with the daughter of Jacob—\textit{something that} should not be done.
\verse And Hamor spoke with them saying, “Shechem my son \textit{is in love with} your daughter. Please give her to him for a wife.
\verse Make marriages with us. Give us your daughters and take our daughters for yourselves.
\verse You shall dwell with us and the land shall be before you; settle and trade in it, and acquire \textit{property} in it.”
\verse Then Shechem said to her father and to her brothers, “Let me find favor in your eyes, and whatever you say to me I will do.
\verse \textit{Make the bride price and gift as high as you like}; I will give what you say to me. But give me the girl as a wife.”
\verse Then the sons of Jacob answered Shechem and his father Hamor speaking deceitfully, because he had defiled Dinah their sister.
\verse And they said to them, “We cannot do this thing, to give our sister to a man who \textit{is} uncircumcised, for that \textit{is} a disgrace for us.
\verse Only on this \textit{condition} will we give consent to you; if you will become like us—every male among you to be circumcised.
\verse Then we will give our daughters to you, and we will take for ourselves your daughters, and we will live with you and become one family.
\verse But if you will not listen to us, to be circumcised, then we will take our daughters and we will go.”
\verse And their words were good in the eyes of Hamor and in the eyes of Shechem, the son of Hamor.
\verse And the young man did not delay to do the thing, for he wanted the daughter of Jacob. Now he \textit{was} the most honored of his father’s house.
\verse Then Hamor and his son Shechem came to the gate of their city, and they spoke to the men of their city, saying,
\verse “These men \textit{are} at peace with us. Let them dwell in the land and let them trade in it. Now, behold, the land is \textit{broad enough for them}. Let us take their daughters as wives, and let us give our daughters to them.
\verse Only on this \textit{condition} will they give consent to us, to live with us \textit{and} to become one family—when every male among us \textit{is} circumcised as they are circumcised.
\verse Will not their livestock and their property and all their animals \textit{be} ours? Only let us give consent to them so they will live among us.”
\verse And all those who went out of the gate of his city listened to Hamor and Shechem. Every male was circumcised, all those who went out of the gate of his city.
\verse And it happened \textit{that} on the third day, while they were in pain, two of the sons of Jacob, Simeon and Levi, the brothers of Dinah, each took his sword and came against the unsuspecting city and killed all the males.
\verse They killed Hamor and his son Shechem with the edge of the sword, and they took Dinah from the house of Shechem and went out.
\verse The \textit{other} sons of Jacob came upon the slain and plundered the city, because they had defiled their sister.
\verse They took their flocks and their cattle and their donkeys, and whatever \textit{was} in the field.
\verse They captured and plundered all that \textit{was} in the houses—all their wealth, their little ones, and their women.
\verse Then Jacob said to Simeon and Levi, “You have brought trouble on me, making me stink among the inhabitants of the land, among the Canaanites and the Perizzites! I \textit{am} few in number! If they gather against me and attack me, I will be destroyed—I and my household!”
\verse But they said, “Shall he treat our sister like a prostitute?”
\end{biblechapter}

\begin{biblechapter} % Genesis 35
\verseWithHeading{Jacob Goes Back to Bethel} And God said to Jacob, “Arise, go up to Bethel and dwell there, and make an altar to the God who appeared to you when you fled from before Esau your brother.”
\verse Then Jacob said to his household and to all who \textit{were} with him, “Get rid of the foreign gods that \textit{are} in your midst and purify yourselves and change your garments.
\verse Then let us make ready and let us go up to Bethel, so that I can make an altar there to the God who answered me in the day of my trouble, and who has been with me on the way that I have gone.”
\verse So they gave to Jacob all the foreign gods that \textit{were} in their hands, and the ornamental rings that \textit{were} in their ears. And Jacob buried them under the oak which \textit{was} near Shechem.
\verse Then they set out on their journey, and the terror of God was upon the cities that \textit{were} all around them, so that they did not pursue after the sons of Jacob.
\verse And Jacob came to Luz which \textit{was} in the land of Canaan (that \textit{is} Bethel), he and all the people that \textit{were} with him.
\verse And he built an altar there and called the place El-Bethel, for there God had appeared to him when he fled before his brother.
\verse And Deborah, the nurse of Rebekah, died. And she was buried below Bethel, under the oak. And its name was called Allon-Bacuth.
\verse And God appeared to Jacob again when he came from Paddan-Aram, and he blessed him.
\verse And God said to him, “Your name \textit{is} Jacob. Your name shall no longer be called Jacob, but Israel shall be your name.” Then his name was called Israel.
\verse And God said to him, “I \textit{am} El-Shaddai. Be fruitful and multiply. A nation and an assemblage of nations shall be from you, and kings shall go out from your loins.
\verse And \textit{as for} the land that I gave to Abraham and to Isaac, I will give it to you. And to your descendants after you I will give the land.
\verse And God went up from him at the place where he spoke with him.
\verse And Jacob set up a pillar at the place where God had spoken with him, a pillar of stone. And he poured out a drink offering upon it, and poured oil on it.
\verse And Jacob called the name of the place where God had spoken with him Bethel.
\verseWithHeading{The Death of Rachel} Then they journeyed from Bethel. And \textit{when they were still some distance} from Ephrath, Rachel went into labor. And she had hard labor.
\verse And \textit{when her labor was the most difficult} the midwife said to her, “Do not be afraid \textit{for you have another son}.”
\verse And it happened \textit{that} when her life was departing (for she was dying), she called his name Ben-Oni. But his father called him Benjamin.
\verse And Rachel died and she was buried on the way to Ephrath (that \textit{is}, Bethlehem).
\verse And Jacob erected a pillar at her burial site. That \textit{is} the pillar of the burial site of Rachel unto this day.
\verse And Israel journeyed \textit{on} and pitched his tent beyond the tower of Eder.
\verse And while Israel was living in that land Reuben went and had sexual relations with Bilhah, his father’s concubine. And Israel heard \textit{about it}.
\verseWithHeading{The Twelve Sons of Jacob} Now the sons of Jacob \textit{were} twelve.
\verse The sons of Leah: The firstborn of Jacob \textit{was} Reuben. Then Simeon, Levi, Judah, Issachar, and Zebulun.
\verse The sons of Rachel: Joseph and Benjamin.
\verse The sons of Bilhah, the female servant of Rachel: Dan and Naphtali.
\verseWithHeading{The Death of Isaac} And Jacob came to Isaac his father \textit{at} Mamre, \textit{or} Kiriath-Arba (that \textit{is}, Hebron), where Abraham and Isaac dwelled as aliens.
\verse Now the days of Isaac were one hundred and eighty years.
\verse And Isaac passed away and died, and was gathered to his people, old and full of days. And his sons Esau and Jacob buried him.
\end{biblechapter}

\begin{biblechapter} % Genesis 36
\verseWithHeading{The Descendants of Esau} Now these \textit{are} the descendants of Esau (that \textit{is}, Edom).
\verse Esau took his wives from the daughters of Canaan: Adah, daughter of Elon, the Hittite, and Oholibamah, daughter of Anah, the daughter of Zibeon, the Hivite,
\verse and Basemath, the daughter of Ishmael, the sister of Nebaioth.
\verse And Adah bore to Esau Eliphaz; and Basemath bore Reuel;
\verse and Oholibamah bore Jeush and Jalam, and Korah. These \textit{are} the sons of Esau who were born to him in the land of Canaan.
\verse And Esau took his wives and his sons and his daughters, and all the persons of his household, and his sheep and goats, and all his cattle, and all the goods that he had acquired in the land of Canaan, and went to a land away from his brother Jacob.
\verse For their possessions were \textit{too many to live together}, so that the land of their sojourning was not able to support them on account of their livestock.
\verse So Esau dwelled in the hill country of Seir (Esau, that \textit{is} Edom).
\verse Now these \textit{are} the descendants of Esau, the father of Edom, in the hill country of Seir.
\verse These \textit{are} the names of the sons of Esau: Eliphaz, the son of Adah, the wife of Esau; Reuel, the son of Basemath, the wife of Esau.
\verse The sons of Eliphaz were Teman, Omar, Zepho, Gatam, and Kenaz.
\verse (Now Timnah was the concubine of Eliphaz, the son of Esau. And she bore Amalek to Eliphaz.) These \textit{are} the sons of Adah, the wife of Esau.
\verse Now these \textit{are} the sons of Reuel: Nahath, Zerah, Shammah, and Mizzah. These \textit{are} the sons of Basemath, the wife of Esau.
\verse Now these \textit{are} the sons of Oholibamah, the daughter of Anah, daughter of Zibeon, the wife of Esau: She bore to Esau Jeush, Jalam, and Korah.
\verse These \textit{are} the chiefs of the sons of Esau. The sons of Eliphaz, the firstborn of Esau: the chiefs of Teman, Omar, Zepho, Kenaz,
\verse Korah, Gatam, and Amalek. These \textit{are} the chiefs of Eliphaz in the land of Edom. These \textit{are} the sons of Adah.
\verse Now these \textit{are} the sons Reuel, the son of Esau: the chiefs Nahath, Zerah, Shammah, and Mizzah. These \textit{are} the chiefs of Reuel in the land of Edom. These \textit{are} the sons of Basemath, the wife of Esau.
\verse Now these \textit{are} the sons of Oholibamah, the wife of Esau: the chiefs Jeush, Jalam, and Korah. These \textit{are} the chiefs born of Oholibamah, the daughter of Anah, the wife of Esau.
\verse These \textit{are} the sons of Esau, and these \textit{are} their chiefs (that \textit{is}, Edom).
\verse These \textit{are} the sons of Seir, the Horite, the inhabitants of the land: Lotan, Shobal, Zibeon, Anah,
\verse Dishon, Ezer, and Dishan. These \textit{are} the chiefs of the Horites, the sons of Seir in the land of Edom.
\verse And the sons of Lotan were Hori and Hemam. And Lotan’s sister \textit{was} Timna.
\verse Now these \textit{are} the sons of Shobal: Alvan, Manahath, Ebal, Shepho, and Onam.
\verse Now these \textit{are} the sons of Zibeon: Aiah and Anah—he \textit{is} Anah who found the hot springs in the desert while he pastured the donkeys of Zibeon his father.
\verse Now these \textit{are} the sons of Anah: Dishon and Oholibamah, the daughter of Anah.
\verse Now these \textit{are} the sons of Dishon: Hemdan, Eshban, Ithran, and Keran.
\verse These \textit{are} the sons of Ezer: Bilhan, Zaavan, and Akan.
\verse These \textit{are} the sons of Dishan: Uz and Aran.
\verse These \textit{are} the chiefs of the Horites: the chiefs Lotan, Shobal, Zibeon, Anah,
\verse Dishon, Ezer, and Dishan. These \textit{are} the chiefs of the Horites, according to their chiefs in the land of Seir.
\verseWithHeading{The Kings of Edom} Now these \textit{are} the kings who reigned in the land of Edom before any king ruled over the \textit{Israelites}.
\verse Bela the son of Beor reigned in Edom. And the name of his city \textit{was} Dinhabah.
\verse And Bela died, and Jobab, the son of Zerah from Bozrah, reigned in his place.
\verse And Jobab died, and Husham from the land of the Temanites reigned in his place.
\verse And Husham died, and Hadad, son of Bedad, who defeated Midian in the field of Moab reigned in his place. And the name of his city \textit{was} Avith.
\verse And Hadad died, and Samlah from Masrekah reigned in his place.
\verse And Samlah died, and Shaul from Rehoboth \textit{on} the Euphrates reigned in his place.
\verse And Shaul died, and Baal-Hanan, the son of Acbor, reigned in his place.
\verse And Baal-Hanan the son of Acbor died, and Hadar reigned in his place. And the name of his city \textit{was} Pau, and the name of his wife \textit{was} Mehetabel, the daughter of Matred, daughter of Mezahab.
\verse Now these \textit{are} the names of the chiefs of Esau according to their families, according to their dwelling places, by their names: the chiefs Timna, Alvah, Jetheth,
\verse Oholibamah, Elah, Pinon,
\verse Kenaz, Teman, Mibzar,
\verse Magdiel, and Iram. These \textit{are} the chiefs of Edom (that \textit{is}, Esau, the father of Edom) according to their settlements in the land of their possession.
\end{biblechapter}

\begin{biblechapter} % Genesis 37
\verseWithHeading{The Dreams of Joseph} And Jacob settled in the land of the sojourning of his father, in the land of Canaan.
\verse These \textit{are} the generations of Jacob. Joseph, \textit{being} seventeen years old, was shepherding the flock with his brothers. Now he \textit{was} a helper with the sons of Bilhah and the sons of Zilpah, the wives of his father. And Joseph brought a bad report of them to his father.
\verse Now Israel loved Joseph more than all his sons, for he \textit{was} a son of his old age. And he made a robe with long sleeves for him.
\verse When his brothers saw that their father loved him more than all his brothers, they hated him and were not able to speak peaceably to him.
\verse And Joseph dreamed a dream, and he told \textit{it} to his brothers. And \textit{they hated him even more}.
\verse And he said to them, “Listen now to this dream that I dreamed.
\verse Now behold, we were binding sheaves in the midst of the field and, behold, my sheaf stood up and it remained standing. Then behold, your sheaves gathered around and bowed down to my sheaf.”
\verse Then his brothers said to him, “Will you really rule over us?” And \textit{they hated him even more} on account of his dream and because of his words.
\verse Then he dreamed yet another dream and told it to his brothers. And he said, “Behold, I dreamed a dream again, and behold, the sun and the moon and eleven stars were bowing down to me.”
\verse And he told \textit{it} to his father and to his brothers. And his father rebuked him and said to him, “What \textit{is} this dream that you have dreamed? Will I and your mother and your brothers indeed come to bow down to the ground to you?”
\verse And his brothers were jealous of him, but his father kept the matter \textit{in mind}.
\verseWithHeading{Joseph Sold Into Slavery by his Brothers} Now his brothers went to pasture the flock of their father in Shechem.
\verse And Israel said to Joseph, “Are not your brothers pasturing in Shechem? Come, let me send you to them.” And he said, “Here I \textit{am}.”
\verse Then he said to him, “Go now, see \textit{if it goes well for your brothers and for the flock}, then return word to me.” And he sent him from the valley of Hebron, and he arrived at Shechem.
\verse And a man found him, and behold, he was wandering about in a field. And the man asked him, “What do you seek?”
\verse And he said, “I am seeking my brothers. Tell me, please, where they are pasturing.”
\verse And the man said, “They have moved on from here, for I heard \textit{them} saying, ‘Let us go to Dothan.’ ” Then Joseph went after his brothers and found them in Dothan.
\verse And they saw him from a distance. And before he drew near to them, they conspired against him to kill him.
\verse And each said to his brothers, “Look, this master of dreams is coming.
\verse Now then, come, let us kill him and throw him in one of the pits. Then we will say a wild animal devoured him. Then we will see what his dreams become.”
\verse And Reuben heard \textit{it} and delivered him from their hand and said, “We must not take his life.”
\verse And Reuben said to them, “You must not shed blood. Throw him into this pit that \textit{is} in the desert, but do not lay a hand on him”—so that he might rescue him from their hand to return him to his father.
\verse And it happened \textit{that} as Joseph came to his brothers they stripped Joseph of his robe, the robe with long sleeves, that \textit{was} upon him.
\verse And they took him and threw him into the pit (the pit \textit{was} empty; there was no water in it).
\verse Then they sat down to eat \textit{some} food. And they lifted up their eyes and looked, and behold, a caravan of Ishmaelites was coming from Gilead. And their camels were carrying aromatic gum and balm and spices \textit{on the way} to Egypt.
\verse Then Judah said to his brothers, “What profit \textit{is there} if we kill our brother and conceal his blood?
\verse Come, let us sell him to the Ishmaelites, but our hand shall not be against him, for he \textit{is} our brother, our own flesh.” And his brothers agreed.
\verse Then Midianite traders passed by. And they drew Joseph up and brought \textit{him} up from the pit, and they sold Joseph to the Ishmaelites for twenty \textit{pieces of} silver. And they brought Joseph to Egypt.
\verse Then Reuben returned to the pit and, behold, Joseph was not in the pit. And he tore his clothes.
\verse And he returned to his brothers and said, “The boy \textit{is gone}! Now I, \textit{what can I do}?”
\verse Then they took the robe of Joseph and slaughtered a goat, and dipped the robe in the blood.
\verse Then they sent the robe with long sleeves and they brought \textit{it} to their father and said, “We found this; please examine \textit{it}. \textit{Is} it the robe of your son or not?”
\verse And he recognized it and said, “The robe of my son! A wild animal has devoured him! Joseph \textit{is} surely torn to pieces!”
\verse And Jacob tore his clothes and put sackcloth on his loins and mourned for his son many days.
\verse And all his sons and daughters tried to console him, but he refused to be consoled. And he said, “No, I shall go down to my son, to Sheol, mourning.” And his father wept for him.
\verse And the Midianites sold him in Egypt to Potiphar, a court official of Pharaoh, a commander of the imperial guard.
\end{biblechapter}

\begin{biblechapter} % Genesis 38
\verseWithHeading{Judah and Tamar} And it happened \textit{that} at that time Judah went down from his brothers and pitched his tent near a certain Adullamite, whose name \textit{was} Hirah.
\verse And Judah saw the daughter of a certain Canaanite there whose name \textit{was} Shua. And he took her and went in to her.
\verse And she conceived and bore a son, and he called his name Er.
\verse And she conceived again and bore a son, and he called his name Onan.
\verse And once again she bore a son, and she called his name Shelah. And he was in Chezib when she bore him.
\verse And Judah took a wife for Er his firstborn, and her name \textit{was} Tamar.
\verse And Er, the firstborn of Judah, was evil in the eyes of Adonai, and Adonai killed him.
\verse Then Judah said to Onan, “Go in to the wife of your brother and perform the duty of a brother-in-law to her, and raise up offspring for your brother.”
\verse But Onan knew that the offspring would not be for him, so whenever he went in to the wife of his brother he would waste \textit{it} on the ground so as not to give offspring to his brother.
\verse And what he did was evil in the sight of Adonai, so he killed him also.
\verse Then Judah said to Tamar, his daughter-in-law, “Stay a widow in your father’s house until Shelah my son grows up,” for \textit{he feared he would also die} like his brother. So Tamar went and stayed in the house of her father.
\verse \textit{And in the course of time} the daughter of Shua, the wife of Judah, died. When Judah was consoled he went up to his sheepshearers, he and his friend Hirah the Adullamite, to Timnah.
\verse And it was told to Tamar, saying, “Look, your father-in-law is going up to Timnah to shear his sheep.”
\verse So she removed the clothes of her widowhood and covered \textit{herself} with the veil and disguised herself. And she sat at the entrance to Eynayim, which \textit{is} on the way to Timnah, for she saw that Shelah was grown but she had not been given to him as a wife.
\verse And Judah saw her and reckoned her to \textit{be} a prostitute, for she had covered her face.
\verse And he turned aside to her at the roadside and said, “Please come, let me come in to you,” for he did not know that she \textit{was} his daughter-in-law. And she said, “What will you give to me that you may come in to me?”
\verse And he said, “I will send a kid from the goats of the flock.” And she said, “\textit{Only} if you give a pledge until you send \textit{it}.”
\verse And he said, “What \textit{is} the pledge that I must give to you?” And she said, “your seal, your cord, and your staff that \textit{is} in your hand.” And he gave \textit{them} to her and went in to her. And she conceived by him.
\verse And she arose and left, and she removed her veil from herself and put on the garments of her widowhood.
\verse And Judah sent the kid from the goats by the hand of his friend the Adullamite to take \textit{back} the pledge from the hand of the woman, but he could not find her.
\verse So he asked the men of her place, saying, “Where \textit{is} that cult prostitute \textit{that was} at Eynayim by the roadside?” And they said, “There is no cult prostitute here.”
\verse Then he returned to Judah and said, “I could not find her. Morever, the men of the place said, ‘There is no cult prostitute here.’ ”
\verse And Judah said, “Let her take \textit{them} for herself, lest we be \textit{laughed at}. Behold, I sent this kid, but you could not find her.”
\verse And \textit{about three months later} it was told to Judah, “Tamar your daughter-in-law has played the whore, and now, behold, she has conceived by prostitution.” And Judah said, “Bring her out and let her be burned.”
\verse She was brought out, but she sent to her father-in-law saying, “By the man to whom these \textit{belong} I have conceived.” And she said, “Now discern to whom these \textit{belong}: the seal and cord and the staff.”
\verse Then Judah recognized \textit{them} and said, “She is more righteous than I, since I did not give her to my son Shelah.” And he did not know her again.
\verse And it happened \textit{that} at the time she gave birth that, behold, twins \textit{were} in her womb.
\verse And it happened \textit{that} at her labor one \textit{child} put out a hand. And the midwife took \textit{it} and tied a crimson thread on his hand saying, “This \textit{one} came out first.”
\verse Then his hand drew back and, behold, his brother came out, and she said, “What a breach you have made for yourself!” And she called his name Perez.
\verse And afterward his brother who \textit{had} the crimson thread on his hand came out. And his name was called Zerah.
\end{biblechapter}

\begin{biblechapter} % Genesis 39
\verseWithHeading{Joseph in Potiphar’s House} Now Joseph had been brought down to Egypt, and Potiphar, a court official of Pharaoh, commander of the guard, an Egyptian, bought him from the hand of the Ishmaelites who had brought him down there.
\verse And Adonai was with Joseph, and he became a successful man. And he was in the house of his master, the Egyptian.
\verse And his master observed that Adonai \textit{was} with him, and everything that \textit{was} in his hand to do Adonai made successful.
\verse And Joseph found favor in his eyes and he served him. Then he appointed him over his house and all that he owned he put into his hand.
\verse And it happened \textit{that} from the time he appointed him over his house and over all that he had, Adonai blessed the house of the Egyptian on account of Joseph. And the blessing of Adonai was upon all that he had in the house and in the field.
\verse And he left all that he had in the hand of Joseph, and \textit{he did not worry about anything} except the food that he ate. Now Joseph was \textit{well built and handsome}.
\verse And it happened \textit{that} after these things his master’s wife cast her eyes on Joseph, and she said, “Lie with me.”
\verse But he refused and said to his master’s wife, “Look, my master \textit{does not worry about} what \textit{is} in the house, and everything he owns he has put in my hand.
\verse He has no greater \textit{authority} in this house than me, and he has not withheld anything from me except you, since you \textit{are} his wife. Now how could I do this great wickedness and sin against God?”
\verse And it happened \textit{that} as she spoke to Joseph \textit{day after day}, he did not heed her to lie beside her or to be with her.
\verse \textit{But one particular day} he came into the house to do his work and none of the men of the house were there in the house,
\verse she seized him by his garment \textit{and} said, “Lie with me!” And he left his garment in her hand and fled, and he went outside.
\verse And it happened \textit{that} when she saw that he left his garment in her hand and fled outside,
\verse she called to the men of her house and said to them, “Look! He brought a Hebrew man to us to mock us! He came to me to lie with me, and I cried out with a loud voice.
\verse And when he heard \textit{me}, that I raised my voice and called out, he left his garment beside me and fled, and he went outside.”
\verse Then she put his garment beside her until his master came to his house.
\verse Then she spoke to him according to these words, saying, “The Hebrew slave that you brought to us came to me to make fun of me.
\verse And it happened \textit{that} as I raised my voice and called out, he left his garment beside me and fled outside.”
\verse And when his master heard the words of his wife that she spoke to him, “\textit{This is what your servant did to me},” \textit{he became very angry}.
\verse And Joseph’s master took him and put him into prison, the place that the king’s prisoners were confined. And he was there in prison.
\verse And Adonai was with Joseph, and showed loyal love to him, and gave him favor in the eyes of the chief of the prison.
\verse And the chief of the prison put all the prisoners that \textit{were} in the prison into the hand of Joseph. And everything that was done there, he \textit{was} the one who did \textit{it}.
\verse The chief of the prison \textit{did not worry about} anything in his hand, since Adonai \textit{was} with him. And whatever he did Adonai made \textit{it} successful.
\end{biblechapter}

\begin{biblechapter} % Genesis 40
\verseWithHeading{Joseph Interprets Dreams in Prison} And it happened \textit{that} after these things the cupbearer of the king of Egypt and \textit{his} baker did wrong against their lord, against the king of Egypt.
\verse And Pharaoh was angry with his two officials, with the chief cupbearer and chief baker.
\verse And he put them in custody in the house of the chief of the guard, into the prison where Joseph was confined.
\verse And the chief of the guard appointed Joseph \textit{to be} with them, and he attended them. And they were in custody \textit{many days}.
\verse And the two of them, the cupbearer and the baker of the king of Egypt, who \textit{were} confined in the prison, dreamed a dream, each his own dream, with its own interpretation.
\verse When Joseph came to them in the morning he looked at them, and behold, they were troubled.
\verse And he asked the court officials of Pharaoh that \textit{were} with him in the custody of his master’s house, “Why \textit{are} your faces sad today?”
\verse And they said to him, “We \textit{each} dreamed a dream, but there is no one to interpret it.” And Joseph said to them, “Do not interpretations belong to God? Please tell \textit{them} to me.”
\verse Then the chief cupbearer told his dream to Joseph, and he said to him, “In my dream, now behold, \textit{there was} a vine before me,
\verse and on the vine \textit{were} three branches. And as it budded, its blossoms came up, \textit{and} its clusters of grapes grew ripe.
\verse And the cup of Pharaoh \textit{was} in my hand, and I took the grapes and squeezed them into the cup of Pharaoh. Then I placed the cup into the hand of Pharaoh.”
\verse Then Joseph said to him, “This \textit{is} its interpretation: The three branches, they \textit{are} three days.
\verse In three days Pharaoh will lift up your head and will restore you to your office. And you shall put the cup of Pharaoh into his hand as \textit{was} formerly the custom, when you were his cupbearer.
\verse But remember me when it goes well with you, and please may you show kindness with respect to me, and mention me to Pharaoh, and bring me out of this house.
\verse For I was surely kidnapped from the land of the Hebrews, and here also I have done nothing that they should put me in this pit.”
\verse And when the chief baker saw that the interpretation \textit{was} good he said to Joseph, “I also \textit{dreamed}. In my dream, now behold, \textit{there were} three baskets of bread upon my head.
\verse And in the upper basket \textit{were} all sorts of baked foods for Pharaoh, but the birds were eating them out of the basket upon my head.”
\verse Then Joseph answered and said, “This \textit{is} its interpretation: The three baskets, they \textit{are} three days.
\verse In three days Pharaoh will lift your head from you and hang you on a pole, and the birds will eat your flesh from you.”
\verse And it happened \textit{that} on the third day, \textit{which was} Pharaoh’s birthday, he made a feast for all his servants. And he lifted up the head of the chief cupbearer and the head of the chief baker in the midst of his servants.
\verse And he restored the chief cupbearer to his cupbearing \textit{position}. And he placed the cup in the hand of Pharaoh.
\verse But the chief baker he hanged as Joseph had interpreted to them.
\verse But the chief cupbearer did not remember Joseph, but forgot him.
\end{biblechapter}

\begin{biblechapter} % Genesis 41
\verseWithHeading{Joseph Interprets Pharaoh’s Dreams} And it happened \textit{that} after \textit{two full years} Pharaoh dreamed, and behold, he was standing by the Nile.
\verse And behold, seven cows, \textit{well built and fat}, were coming up from the Nile, and they grazed among the reeds.
\verse And behold, seven other cows came up after them from the Nile, \textit{ugly and gaunt}, and they stood beside those cows on the bank of the Nile.
\verse And the \textit{ugly and gaunt} cows ate the seven \textit{well built and fat} cows. Then Pharaoh awoke.
\verse And he fell asleep and dreamed a second time, and behold, seven ears of grain, plump and good, were coming out of one stalk.
\verse And behold, seven thin ears of grain, scorched by the east wind, sprouted up after them.
\verse And the thin ears of grain swallowed up the seven plump and full ears of grain. Then Pharaoh awoke, and behold, \textit{it was} a dream.
\verse And it happened \textit{that} in the morning his spirit was troubled, and he sent and called all of the magicians of Egypt, and all its wise men, and Pharaoh told his dream to them. But \textit{they had no interpretation} for Pharaoh.
\verse Then the chief of the cupbearers spoke with Pharaoh, saying, “I remember my sins today.
\verse Pharaoh was angry with his servants, and he put me and the chief baker in the custody of the house of the chief of the guard.
\verse And we dreamed a dream one night, I and he, \textit{each with a dream that had a meaning}.
\verse And there with us \textit{was} a young man, a Hebrew servant of the chief of the guard, and we told him \textit{the dream}, and he interpreted our dreams for us, each according to his dream he interpreted.
\verse And it happened just as he interpreted to us, so it was. He restored me to my office, and him he hanged.”
\verse Then Pharaoh sent and called \textit{for} Joseph, and they brought him quickly from the prison. And he shaved and changed his clothing, and came to Pharaoh.
\verse Then Pharaoh said to Joseph, “I dreamed a dream, but there is none to interpret it. Now, I have heard concerning you \textit{that when} you hear a dream \textit{you can} interpret it.”
\verse Then Joseph answered Pharaoh saying, “\textit{It is not in my power}; God will answer \textit{concerning} the well-being of Pharaoh.”
\verse And Pharaoh said to Joseph, “\textit{Now} in my dream, behold, I was standing on the bank of the Nile,
\verse and behold, seven cows, \textit{well built and fat}, were coming up from the Nile, and they grazed among the reeds.
\verse And behold, seven other cows came up after them from the Nile, very \textit{ugly and gaunt}—never have I seen \textit{any} as them in all the land of Egypt for ugliness.
\verse And the thin and ugly cows ate the former seven healthy cows.
\verse But \textit{when} they went into their bellies it could not be known that they went into their bellies, for their appearance \textit{was} as ugly as at the beginning. Then I awoke.
\verse Then I saw in my dream and behold, seven ears of grain were coming out of one stalk, full and good.
\verse And behold, seven withered ears of grain, thin \textit{and} scorched by the east wind, sprouted up after them.
\verse And the thin ears of grain swallowed up the seven good ears of grain. And I told the magicians, but there was none to explain \textit{it} to me.”
\verse Then Joseph said to Pharaoh, “The dreams of Pharaoh \textit{are} one. God has revealed to Pharaoh what he is about to do.
\verse The seven good cows, they are seven years, and the seven good ears of grain, they \textit{are} seven years. The dreams \textit{are} one.
\verse And the seven thin and ugly cows coming up after them, they \textit{are} seven years, and the seven empty ears of grain, scorched by the east wind, they are \textit{also} seven years of famine.
\verse This \textit{is} the word that I have spoken to Pharaoh; God has shown Pharaoh what he is about to do.
\verse Behold, seven years of great abundance are coming throughout the whole land of Egypt.
\verse Then seven years of famine will arise after them, and all the abundance in the land of Egypt will be forgotten. The famine will consume the land.
\verse Abundance in the land will not be known because of the famine \textit{that follows}, for it will be very heavy.
\verse Now concerning the repetition of the dream twice to Pharaoh, \textit{it is} because the matter \textit{is} established by God, and God will do \textit{it} quickly.
\verse Now then, let Pharaoh select a man \textit{who is} discerning and wise, and let him set him over the land of Egypt.
\verse Let Pharaoh do \textit{this}, and let him appoint supervisors over the land, and let him take one-fifth from the land of Egypt in the seven years of abundance.
\verse Then let them gather all the food of these coming good years and let them pile up grain under the hand of Pharaoh \textit{for} food in the cities, and let them keep \textit{it}.
\verse Then the food shall be as a deposit for the land for the seven years of the famine that will be in the land of Egypt, that the land will not perish on account of the famine.”
\verseWithHeading{Joseph Rises to Power} And the plan was good in the eyes of Pharaoh and in the eyes of all his servants.
\verse Then Pharaoh said to his servants, “Can we find a man like this in whom is the spirit of God?”
\verse Then Pharaoh said to Joseph, “Since God has made all of this known to you there is no one as discerning and wise as you.
\verse You shall be over my house, and to your word all my people shall submit. Only \textit{with respect to} the throne will I be greater than you.”
\verse Then Pharaoh said to Joseph, “See, I have set you over all the land of Egypt.”
\verse Then Pharaoh removed his signet ring from his finger and put it on the finger of Joseph. And he clothed him with garments of fine linen, and he put a chain of gold around his neck.
\verse And he had him ride in his second chariot. And they cried out before him, “Kneel!” And Pharaoh set him over all the land of Egypt.
\verse Then Pharaoh said to Joseph, “I \textit{am} Pharaoh, but without your consent no one will lift his hand or his foot in all the land of Egypt.”
\verse And Pharaoh called the name of Joseph Zaphenath-paneah and gave him Asenath, the daughter of Potiphera, priest of On, as a wife. And Joseph went out over the land of Egypt.
\verse Now Joseph \textit{was thirty years old} when he stood before Pharaoh, the king of Egypt. And Joseph went out from the presence of Pharaoh and traveled through the whole land of Egypt.
\verse And the land produced a plenty in the seven years of abundance.
\verse And he gathered all the food of the seven years which \textit{occurred} in the land of Egypt. And he stored the food in the cities. The food of the field that surrounded \textit{each} city he stored in its midst.
\verse And Joseph piled up grain like the sand of the sea in great abundance until he stopped counting \textit{it}, for \textit{it could not be counted}.
\verse Before the years of famine came, Asenath, daughter of Potiphera priest of On, bore two sons to him.
\verse And Joseph called the name of the firstborn Manasseh, for \textit{he said}, “God has caused me to forget all my hardship and all my father’s house.”
\verse And the name of the second he called Ephraim, for \textit{he said}, “God has made me fruitful in the land of my misfortune.”
\verse And the seven years of abundance which \textit{were} in the land of Egypt came to an end.
\verse And the seven years of famine began to come as Joseph had said. And there was famine in all of the countries, but in the land of Egypt there was food.
\verse And when all the land of Egypt was hungry the people cried out to Pharaoh for food. And Pharaoh said to all the land of Egypt, “Go to Joseph; what he says to you, you must do.”
\verse And the famine was over the whole land, and Joseph opened all the storehouses and sold \textit{food} to the Egyptians. And the famine was severe in the land of Egypt.
\verse And every land came to Egypt to Joseph to buy grain, for the famine was severe in every land.
\end{biblechapter}

\begin{biblechapter} % Genesis 42
\verseWithHeading{Joseph’s Brothers Go to Egypt for Food} When Jacob realized that there was grain in Egypt, Jacob said to his sons, “Why do you look at one another?”
\verse Then he said, “Look, I have heard that there is grain in Egypt. Go down there and buy grain for us there that we may live and not die.”
\verse And the ten brothers of Joseph went down to buy grain from Egypt.
\verse But Jacob did not send Benjamin, the brother of Joseph, for \textit{he feared harm would come to him}.
\verse Then the sons of Israel went to buy grain amid those \textit{other people} who went \textit{as well}, for there was famine in the land of Canaan.
\verse Now Joseph was the governor over the land. He \textit{was} the one who sold \textit{food} to all the people of the land. And the brothers of Joseph came and bowed down to him with their faces to the ground.
\verse And Joseph saw his brothers and recognized them, but he pretended to be a stranger to them. And he spoke with them harshly and said to them, “From where have you come?” And they said, “From the land of Canaan to buy food.”
\verse And Joseph recognized his brothers, but they did not recognize him.
\verse And Joseph remembered the dreams which he had dreamed concerning them, and he said to them, “You are spies! You have come to see the nakedness of the land!”
\verse And they said to him, “No, my lord, but your servants have come to buy food.
\verse We all are sons of one man. We \textit{are} honest \textit{men}. We, your servants, are not spies.”
\verse Then he said to them, “No, but you have come to see the nakedness of the land.”
\verse Then they said, “We, your servants, \textit{are} twelve brothers, the sons of one man in the land of Canaan, but behold, the youngest \textit{is} with our father today, and one is no more.”
\verse But Joseph said to them, “It \textit{is} what I said to you—you \textit{are} spies.
\verse By this you shall be tested. By the life of Pharaoh you will not go out from here unless your youngest brother comes here.
\verse Send one of you, and let him bring your brother, but you will be kept in prison so that your words might be tested \textit{to see} if \textit{there is} truth with you. And if not, by the life of Pharaoh surely you \textit{are} spies.”
\verse Then he gathered them into the prison for three days.
\verse On the third day Joseph said to them, “Do this and you will live; I fear God.
\verse If you \textit{are} honest, let one of your brothers be kept in prison \textit{where you are now being kept}, but \textit{the rest of} you go, carry grain for the famine for your households.
\verse You must bring your youngest brother to me, and then your words will be confirmed and you will not die.” And they did so.
\verse Then each said to his brother, “Surely we \textit{are} guilty on account of our brother when we saw the anguish of his soul when he pleaded for mercy to us and we would not listen. Therefore this trouble has come to us.”
\verse Then Reuben answered them, saying, “Did I not say to you, do not sin against the boy? But you did not listen, and now, behold, his blood has been sought.”
\verse Now they did not know that Joseph understood, for the interpreter \textit{was} between them.
\verse And he turned away from them and wept. Then he returned to them and spoke to them, and took Simeon from them and tied him up in front of them.
\verse Then Joseph gave orders to fill their bags with grain and to return their money to each sack, and to give them provisions for the journey. Thus he did for them.
\verse Then they loaded their grain upon their donkeys and went \textit{away} from there.
\verse And one \textit{of them later} opened his sack to give fodder to his donkey at the lodging place and saw his money—behold, it \textit{was} in the mouth of his sack.
\verse And he said to his brothers, “My money was returned and moreover, behold, \textit{it is} in my sack!” Then \textit{their hearts failed them} and each of them trembled \textit{and} said, “What \textit{is} this God has done to us?”
\verse And when they came to Jacob their father in the land of Canaan they told him everything \textit{that} had happened to them, saying,
\verse “The man, the lord of the land, spoke harshly to us and treated us as \textit{if we were} spying out the land.
\verse But we said to him, ‘We \textit{are} honest; we are not spies.
\verse We \textit{are} twelve brothers, the sons of our father. One is no more and the youngest \textit{is} with our father now in the land of Canaan.’
\verse Then the man, the lord of the land, said to us, ‘By this I will know that you \textit{are} honest. Leave one \textit{brother} with me, and take \textit{food for} the famine in your households and go.
\verse And bring your youngest brother to me. Then I will know that you \textit{are} not spies but you \textit{are} honest. And I will give your brother \textit{back} to you, and you will trade in the land.’ ”
\verse And it happened \textit{that when} they emptied their sacks, behold, each one’s pouch of money \textit{was} in his sack. And when they and their father saw the pouches of their money, they were greatly distressed.
\verse And Jacob their father said to them, “You have bereaved me—Joseph is no more and Simeon is no more, and Benjamin you would take! All of this \textit{is} against me!
\verse Then Reuben said to his father, “You may kill my two sons if I do not bring him back to you. Put him in my hand and I myself will return him to you.”
\verse But he said, “My son shall not go down with you, for his brother is dead and he alone remains. \textit{If} harm meets him on the journey that you would take, you would bring down my gray head in sorrow to Sheol.”
\end{biblechapter}

\begin{biblechapter} % Genesis 43
\verseWithHeading{Joseph’s Brothers Return to Egypt} Now the famine in the land \textit{was} severe.
\verse And it happened \textit{that} as they finished eating the grain which they had brought from Egypt their father said to them, “Return and buy a little food for us.”
\verse Then Judah said to him, “The man solemnly admonished us, saying, ‘You shall not see my face unless your brother \textit{is} with you.’
\verse \textit{If you will send} our brother with us, we will go down and buy food for you,
\verse but \textit{if you will not send} \textit{him}, we will not go down, for the man said to us, ‘You shall not see my face unless your brother \textit{is} with you.’ ”
\verse Then Israel said, “Why did you bring trouble to me by telling the man you still had a brother?”
\verse And they said, “The man asked explicitly about us and about our family, saying, ‘Is your father still alive? Do you have a brother?’ And we answered him according to these words. How could we know that he would say, ‘Bring down your brother’?”
\verse Then Judah said to his father Israel, “Send the boy with me, and let us arise and go, so that we will live and not die—you, we, and our children.
\verse I myself will be surety for him. You may seek him from my hand. If I do not bring him back to you and present him before you, then I will stand guilty before you forever.
\verse Surely if we had not hesitated by this \textit{time} we would have returned twice.”
\verse Then their father Israel said to them, “If \textit{it must be} so then do this. Take some of the best products of the land in your bags and take them down to the man as a gift—a little balm and honey, aromatic gum and myrrh, and pistachios and almonds.
\verse And take double \textit{the} money in your hands. Take back the money that was returned in the mouth of your sacks. Perhaps it \textit{was} a mistake.
\verse And take your brother. Now arise and return to the man.
\verse And may El-Shaddai grant you compassion before the man that he may release your other brother to you and Benjamin. As for me, if I am bereaved, I am bereaved.”
\verse So the men took this gift, and they took double money in their hands, and Benjamin, and they rose up and went down to Egypt and stood before Joseph.
\verse When Joseph saw Benjamin with them he said to the one who \textit{was} over his household, “Bring the men into the house and slaughter and prepare \textit{an animal}, for the men shall eat with me at noon.”
\verse And the man did as Joseph had said, and the man brought the men into the house of Joseph.
\verse And the men were afraid when they were brought into the house of Joseph. And they said “We were brought \textit{here} on account of the money that was returned to our sacks the first time, that he might attack us and fall upon us to take us as slaves with our donkeys.”
\verse So they approached the man who \textit{was} over Joseph’s house and spoke to him at the doorway of the house.
\verse And they said, “Please, my lord, we surely came down once before to buy food,
\verse but when we came to the place of lodging and we opened our sacks, then behold, each one’s money \textit{was} in the mouth of his sack—our money in its \textit{full} weight—so we have returned \textit{with} it in our hands.
\verse Now, other money we have brought down in our hand to buy food. We do not know who put our money in our sacks.”
\verse And he said, “Peace to you; do not be afraid. Your God and the God of your father must have given you a treasure in your sacks; your money came to me.” And he brought Simeon out to them.
\verse Then the man brought the men into Joseph’s house and he gave them water and washed their feet, and gave fodder to their donkeys.
\verse Then they laid out the gift until Joseph came at noon, for they had heard that they were to eat food there.
\verse And when Joseph came into the house they brought the gift that \textit{was} in their hand into the house to him, and they bowed down before him to the ground.
\verse And \textit{he greeted them} and said, “Is your father well, the old man of whom you spoke? Is he still alive?”
\verse And they said, “Your servant our father \textit{is} well; he is still alive.” And they knelt and bowed down.
\verse Then he lifted up his eyes and saw Benjamin his brother, the son of his mother, and said, “Is this your youngest brother of whom you told me?” And he continued, “God be gracious to you, my son.”
\verse Then Joseph \textit{hurried away}, \textit{being overcome with emotion} toward his brother, and sought for \textit{a place} to cry. Then he went into a room and wept there.
\verse Then he washed his face and went out, now controlling himself, and said, “Serve the food.”
\verse And they served him by himself, and them by themselves, and the Egyptians who were eating with him by themselves, for Egyptians \textit{could not dine} with Hebrews, because that \textit{was} a detestable thing to Egyptians.
\verse And they were seated before him \textit{from} the firstborn according to his birthright \textit{to} the youngest according to his youth. And the men \textit{looked at one another} amazed.
\verse And portions were served to them from \textit{his table}, and the portion of Benjamin was five times greater than the portion of any of them. And they drank and became drunk with him.
\end{biblechapter}

\begin{biblechapter} % Genesis 44
\verseWithHeading{Joseph Tests His Brothers} Then he commanded \textit{the one} who \textit{was} over his household, saying, “Fill the sacks of the men \textit{with} food as much as they are able to carry, and put each one’s money in the mouth of his sack.
\verse And my cup—the cup of silver—you shall put into the mouth of the sack of the youngest, and the money for his grain. And he did according to the word of Joseph that he had commanded.
\verse \textit{When} the morning light \textit{came} the men were sent away, they and their donkeys.
\verse They went out of the city, \textit{and} had not gone far when Joseph said to \textit{the one} who \textit{was} over his house, “Arise! Pursue after the men and overtake them. Then you shall say to them, ‘Why have you repaid evil for good?
\verse Is this not that from which my master drinks? Now he himself certainly practices divination with it. You have done evil \textit{in} what you have done.’ ”
\verse When he overtook them he spoke these words to them.
\verse And they said to him, “Why has my lord spoken according to these words? Far be it from your servants to do such a thing!
\verse Behold, the money that we found in the mouth of our sacks we returned to you from the land of Canaan. Now why would we steal silver or gold from the house of my lord?
\verse Whoever is found with it from among your servants shall die. And moreover, we will become slaves to my lord.”
\verse Then he said, “Now also according to your words, thus will it be. He who is found with it shall be my slave, but you shall be innocent.”
\verse Then each man quickly brought down his sack to the ground, and each one opened his sack.
\verse And he searched, beginning with the oldest and finishing with the youngest. And the cup was found in the sack of Benjamin.
\verse Then they tore their clothes, and each one loaded his donkey and they returned to the city.
\verse And Judah and his brothers came to the house of Joseph—now he \textit{was} still there—they fell before him to the ground.
\verse Then Joseph said to them, “What is this deed that you have done? Did you not know that a man who \textit{is} like me surely practices divination?”
\verse And Judah said, “What can we say to my lord? What can we speak? Now how can we show ourselves innocent? God has found the guilt of your servants! Behold, we \textit{are} slaves to my lord, both we and also he in whose hand the cup was found.”
\verse But he said, “Far be it from me to do this! The man in whose hand the cup was found, he will become my slave. But as for you, go up in peace to your father.”
\verse But Judah drew near to him and said, “Please my lord, let your servant speak a word in the ears of my lord, and \textit{let not your anger burn} against your servant, for \textit{you are like Pharaoh himself}.
\verse My lord had asked his servants, saying, ‘Do you have a father or a brother?’
\verse And we said to my lord, ‘We have an aged father, and a younger \textit{brother}, the child of his old age, and his brother died, and he alone remains from his mother, and his father loves him.’
\verse Then you said to your servants, ‘Bring him down to me that I may set my eyes upon him.’
\verse Then we said to my lord, ‘The boy cannot leave his father; if he should leave his father, then he would die.’
\verse Then you said to your servants, ‘Unless your youngest brother comes down with you, you shall not again see my face.’
\verse And it happened \textit{that} we went up to your servant, my father, and told him the words of my lord.
\verse And when our father said, ‘Buy a little food for us,’
\verse then we said, ‘We cannot go down. If our youngest brother \textit{is} with us, then we shall go down. For we will not be able to see the face of the man unless our youngest brother \textit{is} with us.’
\verse Then your servant, my father, said to us, ‘You yourselves know that my wife bore two sons to me.
\verse One went out from me, and I said, “Surely he must have been torn to pieces,” and I have never seen him since.
\verse And if you take this one also from me, and he encounters harm, you will bring down my gray head in sorrow to Sheol.’
\verse So now, when I come to your servant, my father, and the boy is not with us—now his life is bound up with his life—
\verse it shall happen \textit{that} when he sees that the boy is gone, he will die. And your servants will bring down the gray head of your servant, our father, to Sheol with sorrow.
\verse For your servant is pledged as surety for the boy by my father, saying, If I do not bring him to you, then I shall be culpable to my father forever.
\verse So then, please let your servant remain in place of the boy as a slave to my lord, and let the boy go up with his brothers.
\verse For how can I go up to my father if the boy is not with me? \textit{I do not want to see} the misery which will find my father.”
\end{biblechapter}

\begin{biblechapter} % Genesis 45
\verseWithHeading{Joseph Reveals His Identity} Then Joseph was not able to control himself before all who were standing by him. And he cried out, “Make every man go out from me!” So no one stood with him when Joseph made himself known to his brothers.
\verse And \textit{he wept loudly}, so that the Egyptians heard \textit{it} and the household of Pharaoh heard \textit{it}.
\verse Then Joseph said to his brothers, “I \textit{am} Joseph! Is my father still alive?” And his brothers were unable to answer him, for they were dismayed at his presence.
\verse So Joseph said to his brothers, “Come near to me, please.” And they drew near. And he said, “I \textit{am} Joseph, your brother, whom you sold into Egypt.
\verse So now, do not be distressed and do not be angry \textit{with yourselves} that you sold me here, for God sent me as deliverance before you.
\verse For these two years the famine \textit{has been} in the midst of the land, but \textit{there will be} five more years where there is no plowing or harvest.
\verse And God sent me before you \textit{all} to preserve for you a remnant in the land and to keep alive among you many survivors.
\verse So now, you yourselves did not send me here, but God put me here as father to Pharaoh and as master of all his household, and a ruler over all the land of Egypt.
\verse Hurry, and go up to my father and say to him, ‘Thus says your son Joseph, God has made me lord of all Egypt. Come down to me and do not delay.
\verse You shall settle in the land of Goshen so that you will be near me, you and your children and your grandchildren, and your flocks and your herds and all that you have.
\verse And I will provide for you there, because \textit{there are} still five years of famine—lest you and your household and all that you have become destitute.’
\verse Now behold, your eyes see, and the eyes of my brother Benjamin see, that \textit{it is I} who am speaking to you.
\verse And you must tell my father of all my honor in Egypt and all that you have seen. Now hurry and bring my father here.”
\verse Then he fell upon the neck of his brother Benjamin and wept, and Benjamin wept upon his neck.
\verse And he kissed all his brothers and wept upon them. And afterward his brothers spoke with him.
\verse Then the report was heard \textit{in} the house of Pharaoh, saying, “Joseph’s brothers have come.” And it pleased Pharaoh and his servants.
\verse Then Pharaoh said to Joseph, “Say to your brothers: ‘Do this—load your donkeys and go back to the land of Canaan,
\verse and take your father and your households and come to me, and I will give you the best of the land of Egypt, and you shall eat the fat of the land.’
\verse And you \textit{Joseph}, are commanded \textit{to say} this: ‘Do this! Take wagons from the land of Egypt for your little ones and your wives, and bring your father and come!
\verse \textit{Do not worry} about your possessions, for the best of all the land of Egypt is yours.’ ”
\verse And the sons of Israel did so. And Joseph gave them wagons at the word of Pharaoh, and gave them provisions for the journey.
\verse To each and to all of them he gave sets of clothing, but to Benjamin he gave three hundred pieces of silver and five sets of clothing.
\verse And to his father he sent \textit{as follows}: ten donkeys carrying the best of Egypt, and ten donkeys carrying grain and food and provisions for his father for the journey.
\verse Then he sent his brothers away, and when they departed he said to them, “Do not be agitated on the journey.”
\verse So they went up from Egypt and came to the land of Canaan to Jacob their father.
\verse And they spoke to him, saying, “Joseph \textit{is} still alive, and he \textit{is} ruler over all the land of Egypt.” And his heart \textit{went numb}, because he did not believe him.
\verse Then they told him all the words of Joseph that he had spoken to them. And when he saw the wagons that Joseph had sent to carry him, then the spirit of Jacob their father revived.
\verse And Israel said, “\textit{It is} enough. Joseph my son \textit{is} still alive. I will go and see him before I die.”
\end{biblechapter}

\begin{biblechapter} % Genesis 46
\verseWithHeading{Jacob and His Offspring Go to Egypt} So Israel journeyed with all that he had, and he came to Beersheba and offered sacrifices to the God of his father, Isaac.
\verse And God spoke to Israel in visions of the night and said, “Jacob, Jacob.” And he said, “Here I \textit{am}.”
\verse Then he said, “I \textit{am} the God of your father. Do not be afraid to go down to Egypt, for I will make you a great nation there.
\verse I myself will go down with you to Egypt, and I myself will also bring you up. And Joseph will place his hand over your eyes.”
\verse So Jacob arose from Beersheba. And the sons of Israel carried their father Jacob, and their little ones and their wives in the wagons Pharaoh had sent to transport him.
\verse And they took their livestock and their possessions that they had acquired in the land of Canaan. And they came to Egypt, Jacob and all his offspring with him,
\verse his sons and his sons’ sons with him, his daughters and his daughters’ daughters with him, into Egypt.
\verse Now these \textit{are} the names of the sons of Israel, who came into Egypt, Jacob and his sons. Reuben, the firstborn of Jacob
\verse and the sons of Reuben: Enoch, Pallu, Hezron, and Carmi.
\verse The sons of Simeon: Jemuel, Jamin, Ohad, Jakin, Zohar, and Shaul, the son of a Canaanite woman.
\verse The sons of Levi: Gershon, Kohath, and Merari.
\verse The sons of Judah: Er, Onan, Shelah, Perez, and Zerah (but Er and Onan died in the land of Canaan). And the sons of Perez were Hezron and Hamul.
\verse The sons of Issachar: Tolah, Puvah, Iob, and Shimron.
\verse The sons of Zebulun: Sered, Elon, and Jahleel.
\verse These \textit{are} the sons of Leah that she bore to Jacob in Paddan-Aram, and Dinah his daughter. His sons and daughters \textit{were} thirty-three persons in all.
\verse The sons of Gad: Ziphion, Haggi, Shuni, Ezbon, Eri, Arodi, and Areli.
\verse The sons of Asher: Imnah, Ishvah, Ishvi, and Beriah, and their sister Serah. And the sons of Beriah: Heber and Malkiel.
\verse There \textit{are} the sons of Zilpah, whom Laban gave to Leah his daughter, and she bore these to Jacob—sixteen persons.
\verse The sons of Rachel, Jacob’s wife: Joseph and Benjamin.
\verse And Ephraim and Manasseh, whom Asenath, daughter of Potiphera, priest of On bore to him, were born to Joseph in the land of Egypt.
\verse The sons of Benjamin: Bela, Beker, Ashbel, Gera, Naaman, Ehi, Rosh, Muppim, Huppim, and Ard.
\verse These \textit{are} the sons of Rachel who were born to Jacob—fourteen persons in all.
\verse The sons of Dan: Hushim.
\verse The sons of Naphtali: Jahzeel, Guni, Jezer, and Shillem.
\verse These \textit{are} the sons of Bilhah whom Laban gave to Rachel his daughter, and she bore these to Jacob—seven persons in all.
\verse All the persons belonging to Jacob who came to Egypt \textit{who were his descendants}, not including the wives of the sons of Jacob \textit{were} sixty-six persons in all.
\verse And the sons of Joseph who were born to him in Egypt \textit{were} two persons. All the persons of the house of Jacob who came to Egypt \textit{were} seventy.
\verse He had sent Judah ahead of him to Joseph to appear before him in Goshen. And they came to the land of Goshen.
\verse Then Joseph harnessed his chariot and went up to meet Israel his father in Goshen. He presented himself to him and fell upon his neck and wept upon his neck a long time.
\verse Then Israel said to Joseph, “Now let me die since I have seen your face, for you are still alive.”
\verse Then Joseph said to his brothers and to his father’s household, “I will go up and report to Pharaoh, and I will say to him, ‘My brothers and my father’s household who \textit{were} in the land of Canaan have come to me.
\verse And the men \textit{are} shepherds, for they are men of livestock, and they have brought their flocks and their cattle and all that they have.’
\verse And it shall be \textit{that} when Pharaoh calls you he will say, ‘What \textit{is} your occupation?’
\verse Then you must say, ‘You servants \textit{are} men of livestock from our childhood until now, both we and also our ancestors,’ so that you may dwell in the land of Goshen, for every shepherd \textit{is} a detestable thing to Egyptians.”
\end{biblechapter}

\begin{biblechapter} % Genesis 47
\verseWithHeading{Jacob Settles in Goshen} So Joseph went and reported to Pharaoh. And he said, “My father and my brothers, with their flocks and their herds, and all that they have, have come from the land of Canaan. Now \textit{they are} here in the land of Goshen.”
\verse And from among his brothers he took five men and presented them before Pharaoh.
\verse And Pharaoh said to his brothers, “What \textit{is} your occupation?” And they said to Pharaoh, “Your servants \textit{are} keepers of sheep, both we and also our ancestors.”
\verse And they said to Pharaoh, “We have come to sojourn in the land, for there is no pasture for your servant’s flocks, for the famine \textit{is} severe in the land of Canaan. So now, please let your servants dwell in the land of Goshen.”
\verse Then Pharaoh said to Joseph, “Your father and your brothers have come to you.
\verse The land of Egypt \textit{is} before you. Settle your father and your brothers in the best of the land. Let them live in the land of Goshen, and if you know there is among them men of ability, then appoint them overseers of my own livestock.”
\verse Then Joseph brought his father Jacob and presented him before Pharaoh. And Jacob blessed Pharaoh.
\verse Then Pharaoh said to Jacob, “\textit{How old are you}?”
\verse And Jacob said to Pharaoh, “The days of the years of my sojourning \textit{are} one hundred and thirty years. Few and hard have been the days of the years of my life, and they have not reached the days of the years of the lives of my ancestors in the days of their sojourning.”
\verse And Jacob blessed Pharaoh, and he went out from the presence of Pharaoh.
\verse And Joseph settled his father and his brothers, and he gave them property in the land of Egypt in the best part of the land, in the land of Rameses, as Pharaoh had instructed.
\verse And Joseph provided his father and his brothers and all the household of his father with food, according to the number of their children.
\verseWithHeading{The Famine in Egypt Continues} Now there was no food in all the land, for the famine \textit{was} very severe. And the land of Egypt languished, with the land of Canaan, on account of the famine.
\verse And Joseph collected all the money found in the land of Egypt and in the land of Canaan in exchange for the grain that they were buying. And Joseph brought the money into the house of Pharaoh.
\verse And when the money was spent in the land of Egypt and from the land of Canaan, all of Egypt came to Joseph, saying, “Give us food! Why should we die before you? For the money is used up.”
\verse And Joseph said, “Give your livestock and I will give you \textit{food} in exchange for your livestock if \textit{your} money is used up.”
\verse So they brought their herds to Joseph, and Joseph gave food to them in exchange for horses, their flocks, and their cattle and donkeys. And he provided them with food in exchange for all their livestock that year.
\verse When that year ended, they came to him in the following year and said to him, “We cannot hide from my lord that \textit{our} money and livestock belong to my lord. Nothing remains before my lord except our bodies and our land.
\verse Why should we die in front of you, both we and our land? Buy us and our land in exchange for food, then we and our land will be servants to Pharaoh. Then give us seed and we shall live and not die, and the land will not become desolate.”
\verse So Joseph bought all the land of Egypt for Pharaoh, for each Egyptian sold his field, for the famine \textit{was} severe upon them. And the land became Pharaoh’s.
\verse As for the people, he transferred them to the cities, from one end of the territory of Egypt to the other.
\verse Only the land of the priests he did not buy, for \textit{there was} an allotment for the priests from Pharaoh, and they \textit{lived on} the allotment that Pharaoh gave to them. Therefore they did not sell their land.
\verse And Joseph said to the people, “Look, I have bought you and your land this day for Pharaoh. Here \textit{is} seed for you so you can sow the land.
\verse And it shall happen \textit{that} at the harvest, you must give a fifth to Pharaoh and four-fifths shall be yours, as seed for the field and for your food and for those who \textit{are} in your households, and as food for your little ones.”
\verse And they said, “You have saved our lives. \textit{If} we have found favor in the eyes of my lord, we will be servants to Pharaoh.”
\verse So Joseph made it a statute unto this day concerning the land of Egypt: one fifth to Pharaoh. Only the land of the priests alone did not belong to Pharaoh.
\verse So Israel settled in the land of Egypt, in the land of Goshen. And they acquired possessions in it and were fruitful and multiplied greatly.
\verse And Jacob lived in the land of Egypt seventeen years. And the days of Jacob, the years of his life, were one hundred and forty-seven years.
\verse When \textit{the time of Israel’s death drew near}, he called to his son, to Joseph. And he said to him, “If I have found favor in your eyes, please put your hand under my thigh, that you might \textit{vow} to deal kindly and faithfully with me. Please do not bury me in Egypt,
\verse but let me lie with my ancestors. Carry me out of Egypt and bury me in their burial site.” And he said, “I will do according to your word.”
\verse Then he said, “Swear to me.” And he swore to him. Then Israel bowed himself on the head of the bed.
\end{biblechapter}

\begin{biblechapter} % Genesis 48
\verseWithHeading{Jacob Blesses Ephraim and Manasseh} And it happened \textit{that} after these things, it was said to Joseph, “Behold, your father \textit{is} ill.” And he took his two sons with him, Ephraim and Manasseh.
\verse And it was told to Jacob, “Behold, your son Joseph has come to you.” Then Israel strengthened himself and he sat up in the bed.
\verse Then Jacob said to Joseph, “El-Shaddai appeared to me in Luz, in the land of Canaan, and blessed me,
\verse and said to me, ‘Behold, I will make you fruitful and make you numerous, and will make you a company of nations. And I will give this land to your offspring after you \textit{as} an everlasting possession.’
\verse And now, your two sons who were born to you in the land of Egypt before my coming to you in Egypt, are mine. Ephraim and Manasseh shall be mine as Reuben and Simeon \textit{are}.
\verse And your children whom you father after them shall be yours. By the name of their brothers they shall be called, with respect to their inheritance.
\verse As for me, when I came to Paddan-Aram Rachel died \textit{to my sorrow} in the land of Canaan on the way when \textit{there was} still some distance to go to Ephrath. And I buried her there on the way to Ephrath (that \textit{is}, Bethlehem).”
\verse When Israel saw the sons of Joseph he said, “Who \textit{are} these?”
\verse Then Joseph said to his father, “They \textit{are} my sons whom God has given me here.” And he said, “Please bring them to me that I may bless them.”
\verse Now the eyes of Israel were dim on account of old age; he was not able to see. So he brought them near to him, and he kissed them and embraced them.
\verse And Israel said to Joseph, “I did not expect to see your face and behold, God has also shown me your offspring.”
\verse Then Joseph removed them from his knees and bowed down with his face to the ground.
\verse And Joseph took the two of them, Ephraim at his right \textit{to} the left of Israel, and Manasseh at his left \textit{to} the right of Israel. And he brought them near to him.
\verse And Israel stretched out his right hand and put \textit{it} on the head of Ephraim (now he was the younger), and his left hand on the head of Manasseh, crossing his hands, for Manasseh \textit{was} the firstborn.
\verse And he blessed Joseph and said,
\verse “The God before whom my fathers, Abraham and Isaac, walked, 
The God who shepherded me \textit{all my life} unto this day,
\verse When Joseph saw that his father put his right hand on the head of Ephraim, he was displeased. And he took hold of his father’s hand to remove it from the head of Ephraim \textit{over} to the head of Manasseh.
\verse And Joseph said to his father, “Not so, my father; because this one \textit{is} the firstborn. Put your right \textit{hand} upon his head.”
\verse But his father refused and said, “I know, my son; I know. He also shall become a people, and he also shall be great, but his younger brother shall be greater than him, and his offspring shall become a multitude of nations.”
\verse So he blessed them that day, saying, Through you Israel shall pronounce blessing, saying, 
‘May God make you like Ephraim and like Manasseh.’ ”
\verse So he put Ephraim before Manasseh.
\verse And Israel said to Joseph, “Behold, I \textit{am about} to die, but God will be with you and will bring you back to the land of your ancestors.
\end{biblechapter}

\begin{biblechapter} % Genesis 49
\verseWithHeading{Jacob Blesses His Twelve Sons} Then Jacob called his sons and said, “Gather together so that I can tell you what will happen with you in \textit{days to come}.
\verse Assemble and hear, O sons of Jacob! 
Listen to Israel your father!
\verse Reuben, you \textit{are} my firstborn, 
my strength, and the firstfruit of my vigor, 
excelling in rank and excelling in power.
\verse Unstable as water, you shall not excel \textit{any longer}, 
for you went up upon the bed of your father, 
then defiled \textit{it}. You went up upon my couch!
\verse Simeon and Levi \textit{are} brothers; 
weapons of violence \textit{are} their swords. 
Let me not come into their council.
\verse Let not my person be joined to their company. 
For in their anger they killed men, 
and at their pleasure they hamstrung cattle.
\verse Cursed be their anger, for \textit{it is} fierce, 
and their wrath, for \textit{it is} cruel. 
I will divide them in Jacob, 
and I will scatter them in Israel.
\verse Judah, \textit{as for} you, your brothers shall praise you. 
Your hand \textit{shall be} on the neck of your enemies. 
The sons of your father shall bow down to you.
\verse Judah \textit{is} a lion’s cub. 
From the prey, my son, you have gone up. 
He bowed down; he crouched like a lion and as a lioness. 
Who shall rouse him?
\verse The scepter shall not depart from Judah, 
nor the ruler’s staff between his feet, 
until Shiloh comes. 
And to him shall be the obedience of nations.
\verse Binding his donkey to the vine 
and his donkey’s colt to the choice vine, 
he washes his clothing in the wine 
and his garment in the blood of grapes.
\verse The eyes \textit{are} darker than wine, 
and the teeth whiter than milk.
\verse Zebulun shall settle by the shore of the sea. 
He \textit{shall become} a haven for ships, 
and his border \textit{shall be} at Sidon.
\verse Issachar \textit{is} a strong donkey, 
crouching between the sheepfolds.
\verse He saw a resting place that \textit{was} good, 
and land that \textit{was} pleasant. 
So he bowed his shoulder to the burden 
and became a servant of forced labor.
\verse Dan shall judge his people 
as one of the tribes of Israel.
\verse Dan shall be a serpent on the way, 
a viper on the road 
that bites the heels of a horse, 
so that its rider falls backward.
\verse I wait for your salvation, O Adonai.
\verse Bandits shall attack Gad, 
but he shall attack \textit{their} heels.
\verse Asher’s food \textit{is} delicious, 
and he shall provide from the king’s delicacies.
\verse Naphtali \textit{is} a doe running free 
that puts forth beautiful words.
\verse Joseph \textit{is} the bough of a fruitful vine, 
a fruitful bough by a spring. 
His branches climb over the wall.
\verse \textit{The archers} fiercely attacked him. 
They shot arrows \textit{at him} and were hostile to him.
\verse But his bow remained in a steady position; 
\textit{his arms} were made agile 
by the hands of the Mighty One of Jacob. 
From there \textit{is} the Shepherd, the Rock of Israel.
\verse Because of the God of your father he will help you 
and \textit{by} Shaddai he will bless you 
with the blessings of heaven above, 
blessings of the deep that crouches beneath, 
blessings of the breasts and the womb.
\verse The blessings of your father 
are superior to the blessings of my ancestors, 
to the bounty of the everlasting hills. 
May they be on the head of Joseph, 
and on the forehead of the prince of his brothers.
\verse Benjamin \textit{is} a devouring wolf, 
devouring the prey in the morning, 
and dividing the plunder in the evening.
\verseWithHeading{The Death and Burial of Jacob} All these \textit{are} the twelve tribes of Israel, and this \textit{is} what their father said to them when he blessed them, each according to their blessing.
\verse Then he instructed them and said to them, “I am \textit{about to be} gathered to my people. Bury me among my ancestors in the cave that \textit{is} in the field of Ephron the Hittite,
\verse in the cave that \textit{is} in the field of Machpelah that \textit{is} before Mamre in the land of Canaan, which Abraham bought with the field from Ephron the Hittite as a burial site.
\verse There they buried Abraham and Sarah his wife. There they buried Isaac and Rebekah his wife. And there I buried Leah—
\verse the purchase of the field and the cave which \textit{was} in it from the Hittites.”
\verse When Jacob finished instructing his sons he drew his feet up to the bed. Then he took his last breath and was gathered to his people.
\end{biblechapter}

\begin{biblechapter} % Genesis 50
\verseWithHeading{Jacob’s Funeral and Joseph’s Remaining Time in Egypt} Then Joseph fell on the face of his father and wept upon him and kissed him.
\verse And Joseph instructed his servants the physicians to embalm his father. So the physicians embalmed Israel.
\verse Forty days \textit{were required for it}, for thus \textit{are} the days \textit{required for} embalming. And the Egyptians wept for him seventy days.
\verse When the days of his weeping had passed, Joseph spoke to the household of Pharaoh, saying, “If I have found favor in your eyes, please speak in the hearing of Pharaoh, saying,
\verse ‘My father made me swear, saying, “Behold, I \textit{am about} to die. In the tomb that I have hewed out for myself in the land of Canaan—there you must bury me.” So then, please let me go up and let me bury my father; then I will return.’ ”
\verse Then Pharaoh said, “Go up and bury your father as he made you swear.”
\verse So Joseph went up to bury his father. And all the servants of Pharaoh, the elders of his household, and all the elders of the land of Egypt, went up with him,
\verse with all the household of Joseph, his brothers, and the household of his father. They left only their little children and their flocks and their herds in the land of Goshen.
\verse And there also went up with him chariots and horsemen. The company \textit{was} very great.
\verse When they came to the threshing floor of Atad, which \textit{was} beyond the Jordan, they lamented there with a very great and sorrowful wailing. And he made a mourning ceremony for his father seven days.
\verse And when the Canaanites, the inhabitants of the land, saw the mourning ceremony at the threshing floor of Atad they said, “This \textit{is} a severe mourning for the Egyptians.” Therefore its name was called Abel-Mizraim, which \textit{is} beyond the Jordan.
\verse Thus his sons did to him just as he had instructed them.
\verse And his sons carried him to the land of Canaan and buried him in the cave of the field of Machpelah, which field Abraham had bought as a burial site from Ephron the Hittite before Mamre.
\verse And after burying his father, Joseph returned to Egypt, he and his brothers and all who had gone up with him to bury his father.
\verse And when the brothers of Joseph saw that their father \textit{was} dead, they said, “It may be \textit{that} Joseph will hold a grudge against us and pay us back dearly for all the evil that we did to him.”
\verse So they sent \textit{word} to Joseph saying, “Your father commanded \textit{us} before his death, saying,
\verse “Thus you must say to Joseph, ‘O, please now forgive the transgression of your brothers and their sin, for they did evil to you.’ So now, please forgive the transgression of the servants of the God of your father.” And Joseph wept when they spoke to him.
\verse Then his brothers went also and fell before him and said, “Behold, we \textit{are} your servants.”
\verse Then Joseph said to them, “Do not be afraid, for \textit{am} I in the place of God?
\verse As for you, you planned evil against me, \textit{but} God planned it for good, in order to do this—to keep many people alive—as \textit{it is} today.
\verse So then, do not be afraid. I myself will provide for you and your little ones. And he consoled them and \textit{spoke kindly} to them.
\verseWithHeading{The Death of Joseph} So Joseph remained in Egypt, he and the house of his father. And Joseph lived one hundred and ten years.
\verse And Joseph saw Ephraim’s children to the third generation. Moreover, the children of Makir, son of Manasseh, were born on the knees of Joseph.
\verse And Joseph said to his brothers, “I \textit{am about} to die, but God will certainly visit you and bring you up from this land to the land that he swore to Abraham, to Isaac, and to Jacob.”
\verse Then Joseph made the sons of Israel swear an oath, saying, “God will surely visit you, and you shall bring up my bones from here.”
\verse So Joseph died, \textit{being} one hundred and ten years old. They embalmed him and he was placed in a coffin in Egypt.
\end{biblechapter}


\end{document}
