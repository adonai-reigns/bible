\documentclass[twopage,twocolumn,showtrims]{memoir}


\setstocksize{228.6mm}{152.4mm}% 6in x 9in
\settrimmedsize{8.5in}{5.5in}{*}
\settrims{.25in}{.25in}

\setcolsepandrule{4.5mm}{0pt}

\setulmarginsandblock{14mm}{9mm}{1}

%\setlrmargins{16mm}{12mm}{*}
\setlrmarginsandblock{14.5mm}{7.5mm}{*}

%\setheadfoot{26mm}{12mm}
%\setheaderspaces{*}{24mm}{*}

\checkandfixthelayout 

%\setmarginnotes{1bp}{1bp}{1bp}

\quarkmarks

%\usepackage{atbegshi}
%\AtBeginShipout{\special{pdf: put @thispage <</TrimBox [36.0 36.0 612 810]>>}}
%\special{pdf: put @thispage <</TrimBox [36.0 36.0 612 810]>>}


%\usepackage[pass,b5paper,inner=16mm,outer=12mm,top=24mm,bottom=8mm]{geometry}
\usepackage{lipsum}
\usepackage{fixltx2e}
\usepackage{xspace}
\usepackage[usenames,dvipsnames,svgnames,table]{xcolor}
\usepackage{lettrine}
\usepackage{flushend}
\usepackage{fancyhdr}
\usepackage{hyperref}
\usepackage{microtype}
\usepackage[object=vectorian]{pgfornament}
\usepackage{fontspec}
\usepackage{ifpdf}
\usepackage{eso-pic}
\usepackage[british]{babel}
\usepackage{titlesec}
\usepackage{blindtext}
\usepackage{hyperref}



\setmainfont{notoserif}[
  % Files
  Path      = fonts/noto-serif/ ,
  % Fonts
  UprightFont     = NotoSerif-Regular.ttf ,
  UprightFeatures = { SmallCapsFont = NotoSerif-Regular.ttf, } ,
  BoldFont        = NotoSerif-Bold.ttf,
  BoldFeatures    = { SmallCapsFont = NotoSerif-Bold.ttf } ,
  ItalicFont      = NotoSerif-Italic.ttf ,
  BoldItalicFont  = NotoSerif-BoldItalic.ttf ,
  % Features
  Numbers = OldStyle ]


% old-style numbers don't look great as drop-caps
\newfontfamily{\lettrinefont}{AlegreyaSC}[
  % Files
  Path      = fonts/alegreya-sc/ ,
  % Fonts
  UprightFont     = *-Regular.ttf ,
  UprightFeatures = { SmallCapsFont = *-Regular.ttf } ,
  BoldFont        = *-Bold.ttf ,
  BoldFeatures    = { SmallCapsFont = *-Bold.ttf } ,
  ItalicFont      = *-Italic.ttf ,
  BoldItalicFont  = *-BoldItalic.ttf]



% old-style numbers don't look great as drop-caps
\newfontfamily{\headings}{AlegreyaSC}[
  % Files
  Path      = fonts/alegreya-sc/ ,
  % Fonts
  UprightFont     = *-Regular.ttf ,
  UprightFeatures = { SmallCapsFont = *-Regular.ttf } ,
  BoldFont        = *-Bold.ttf ,
  BoldFeatures    = { SmallCapsFont = *-Bold.ttf } ,
  ItalicFont      = *-Italic.ttf ,
  BoldItalicFont  = *-BoldItalic.ttf]



% book names font family and huge
\titleformat{\chapter}[display]{\Huge\headings\centering}{\chaptertitlename\ \thechapter}{20pt}{\Huge}
\titlespacing*{\chapter}{0pt}{0pt}{30pt}

\pagestyle{fancy}
\fancyhf{}

% the book name, chapter and verse number at the top of the page
\fancyhead[RO,LE]{\textbf{\headings{\Large \rightmark}}}

% the page number in the bottom centre of each page
\cfoot{\thepage}

\renewcommand{\headrulewidth}{0pt}


\hypersetup{colorlinks=true, linkcolor=black}

\newcommand{\framesize}{\textwidth}
\setlength{\headwidth}{\textwidth}
\setlength{\columnseprule}{0pt}

% space below the page header
\setheaderspaces{*}{0mm}{*}
\clubpenalty10000
\widowpenalty10000

\newlength{\versespacing}
\setlength{\versespacing}{.16667em plus .08333em}
\newcommand{\versespace}{\hspace{\versespacing}}
\frenchspacing

\newcommand\AtPageUpperRight[1]{\AtPageUpperLeft{%
 \put(\LenToUnit{\paperwidth},\LenToUnit{0\paperheight}){#1}%
 }}%
\newcommand\AtPageLowerRight[1]{\AtPageLowerLeft{%
 \put(\LenToUnit{\paperwidth},\LenToUnit{0\paperheight}){#1}%
 }}%

\newcommand{\sectionstyle}{\bfseries\raggedright\Large}
\setsecheadstyle{\sectionstyle}
\setbeforesecskip{0ex}
\setaftersecskip{0.1ex}

\newcommand{\subsectionstyle}{\bfseries\itshape\raggedright\large}
\setsubsecheadstyle{\subsectionstyle}
\setbeforesubsecskip{0ex}
\setaftersubsecskip{0.1ex}

\makeatletter
\newcommand\versenumcolor{red}
\newcommand\chapnumcolor{red}
\newlength{\biblechapskip}
  \setlength{\biblechapskip}{1em plus .33em minus .2em}
\newcounter{biblechapter}
\newcounter{bibleverse}[biblechapter]
\renewcommand\chaptername{Book}
\newcommand{\biblebook}[1]{%
  \setcounter{biblechapter}{0}
  \gdef\currbook{#1}
  \chapter*{#1}
  \addcontentsline{toc}{chapter}{#1}}
\newcount\biblechap@svdopt
\newenvironment{biblechapter}[1][\thebiblechapter]
  {\biblechap@svdopt=#1
  \ifnum\c@biblechapter=\biblechap@svdopt\else
    \advance\biblechap@svdopt by -1\fi
  \setcounter{biblechapter}{\the\biblechap@svdopt}
  \stepcounter{biblechapter}
  \setbeforesecskip{2mm}\setbeforesubsecskip{2mm}
  \lettrine[lines=3,lhang=0,findent=0pt,nindent=0pt,loversize=0.25]{\lettrinefont\color{\chapnumcolor}\,\thebiblechapter\,}{}\ignorespaces}
  {\par\vspace{\biblechapskip}\setbeforesecskip{0ex}\setbeforesubsecskip{0ex}}
\newcommand{\@verse}{\stepcounter{bibleverse}\markright{{\scshape\currbook} \thebiblechapter:\thebibleverse}}
\newcommand{\@showversenum}{\ifnum\c@bibleverse=1\else{\color{\versenumcolor}\textbf{\thebibleverse}~}\fi\ignorespaces}
\renewcommand{\verse}{\@verse\@showversenum}
\newcommand{\verseWithHeading}[1]{%
  \@verse%
  \ifnum\c@bibleverse=1{\headings{\sectionstyle#1}\newline}\else\vspace{\baselineskip}\newline\begin{headings}{\sectionstyle#1}\end{headings}\newline\fi\@showversenum}
\newcommand{\verseWithSubheading}[1]{%
  \@verse
  \ifnum\c@bibleverse=1{\headings{\subsectionstyle#1}\newline}\else\vspace{\baselineskip}\newline\begin{headings}{\subsectionstyle#1}\end{headings}\newline\fi\@showversenum}
\makeatother

%\newcommand{\startornaments}{\AddToShipoutPictureBG{%
%  \checkoddpage%
%  \ifoddpage%
%   \AtPageUpperRight{\put(-100,-55){\pgfornament[width=1.75cm,symmetry=h,color=black]{195}}}%
%   \AtPageLowerRight{\put(-100,55){\pgfornament[width=1.75cm,symmetry=v,color=black]{194}}}%
% \else%
%   \AtPageUpperLeft{\put(10,-55){\pgfornament[width=1.75cm,symmetry=h,color=black]{194}}}%
%   \AtPageLowerLeft{\put(10,55){\pgfornament[width=1.75cm,symmetry=v,color=black]{195}}}
% \fi}}

\newcommand{\stopornaments}{\ClearShipoutPictureBG}

\renewcommand{\printparttitle}[1]{%
  \thispagestyle{empty}%
  \addcontentsline{toc}{part}{#1}%
  \vspace*{\fill}%
  \begin{tikzpicture}[transform shape,every node/.style={inner sep=0pt}]%
    \node[minimum size=\framesize](vecbox){};%
  \node[inner sep=6pt, color=black] (text) at (vecbox.center){%
    \HUGE \headings{\textsc{#1}}};%
  \node[anchor=north, color=Goldenrod] (base) at (text.south){%
    \pgfornament[width=0.5*\framesize]{88}};%
  \end{tikzpicture}%
  \vspace*{\fill}}

\makechapterstyle{dash-embiggened}{%
  \chapterstyle{default}
  \setlength{\beforechapskip}{5\onelineskip}
  \renewcommand*{\printchaptername}{}
  \renewcommand*{\chapternamenum}{}
  \renewcommand*{\chapnumfont}{\normalfont\Huge}
  \settoheight{\midchapskip}{\chapnumfont 1}
  \renewcommand*{\printchapternum}{\centering \chapnumfont
    \rule[0.5\midchapskip]{1em}{0.4pt} \thechapter\
    \rule[0.5\midchapskip]{1em}{0.4pt}}
  \renewcommand*{\afterchapternum}{\par\nobreak\vskip 0.5\onelineskip}
  \renewcommand*{\printchapternonum}{\centering
                 \vphantom{\chapnumfont 1}\afterchapternum}
  \renewcommand*{\chaptitlefont}{\normalfont\HUGE\scshape}
  \renewcommand*{\printchaptertitle}[1]{\centering \chaptitlefont ##1}
  \setlength{\afterchapskip}{2.5\onelineskip}}

\chapterstyle{dash-embiggened}

\newcommand{\columnbreak}{\pagebreak}

\newcommand{\LORD}{\textsc{\headings{Lord}}\xspace}
\newcommand{\LORDs}{\textsc{\headings{Lord's}}\xspace}

\begin{document}

\title{\Huge \headings{The Holy Bible}}
\date{}
\author{}

\frontmatter

\begin{titlingpage}
\vspace*{\fill}

\begin{tikzpicture}[color=Gold,
    transform shape,
    every node/.style={inner sep=0pt}]
  \node[minimum size=\framesize,fill=Beige!10](vecbox){};
  \node[anchor=north west] at (vecbox.north west){%
    \pgfornament[width=0.2*\framesize]{131}};
  \node[anchor=north east] at (vecbox.north east){%
    \pgfornament[width=0.2*\framesize,symmetry=v]{131}};
  \node[anchor=south west] at (vecbox.south west){%
    \pgfornament[width=0.2*\framesize,symmetry=h]{131}};
  \node[anchor=south east] at (vecbox.south east){%
    \pgfornament[width=0.2*\framesize,symmetry=c]{131}};
  \node[anchor=north] at (vecbox.north){%
    \pgfornament[width=0.6*\framesize,symmetry=h]{85}};
  \node[anchor=south] at (vecbox.south){%
    \pgfornament[width=0.6*\framesize]{85}};
  \node[anchor=north,rotate=90] at (vecbox.west){%
    \pgfornament[width=0.6*\framesize,symmetry=h]{85}};
  \node[anchor=north,rotate=-90] at (vecbox.east){%
    \pgfornament[width=0.6*\framesize,symmetry=h]{85}};
  \node[inner sep=6pt, color=black] (text) at (vecbox.center){%
    \HUGE \textsc{The Holy Bible}};
  \node[anchor=north, color=Goldenrod] (base) at (text.south){%
    \pgfornament[width=0.5*\framesize]{71}};
  \node[anchor=south, color=Goldenrod] at (text.north){%
    \pgfornament[width=0.5*\framesize,symmetry=h]{71}};
\end{tikzpicture}

\vspace*{\fill}
\end{titlingpage}



\renewcommand{\contentsname}{\headings{Table of Contents}}
\hypersetup{colorlinks=true ,urlcolor=blue,urlbordercolor={0 1 1}}
\begin{headings}
\tableofcontents*
\end{headings}


\mainmatter
\part*{The Old Testament}

\startornaments
\input{kjv/old_testament/genesis.tex}\flushcolsend
\input{kjv/old_testament/exodus.tex}\flushcolsend
\biblebook{Leviticus}

\begin{biblechapter} % Leviticus 1
\verseWithHeading{Laws for Burnt Offerings}{Then\lebnote{Or “And”}Adonai called to Moses and spoke to him from the tent of assembly, saying,}%
\verse{“Speak to the \textit{Israelites},\lebnote{Literally “sons/children of Israel”}and say to them, ‘When a person\lebnote{Or “man”—the singular noun is generic, thus the “you” and “your” in the remainder of the verse are all plural}from you presents an offering to Adonai, you shall present your offering from domestic animals, from the cattle\lebnote{Or “herd”}or from the flock.\lebnote{The Hebrew term refers collectively to both sheep and goats (small livestock animals)}}%
\verse{If his offering is a burnt offering from the cattle,\lebnote{Or “herd”}then he must present\lebnote{Hebrew “present it”}an unblemished male; he must present it at the door of the tent of assembly for his acceptance \textit{before}\lebnote{Literally “to the face of”}Adonai.}%
\verse{“ ‘He\lebnote{Or “And he”}must lay his hand on the head of the burnt offering and it will be accepted\lebnote{Same root word as “acceptance” in v. 3}for him in order to make atonement for him.}%
\verse{He\lebnote{Or “And he”}must slaughter \textit{the young bull}\lebnote{Literally “the son of the herd” or “the son of the cattle”}\textit{before}\lebnote{Literally “to the face of”}Adonai, then\lebnote{Or “and”}Aaron’s sons, the priests, will present the blood and sprinkle the blood all around the altar that is at the door of the tent of assembly.}%
\verse{Then\lebnote{Or “And”}he must remove the skin of the burnt offering and cut it into its pieces.}%
\verse{The sons of\lebnote{Or “And the sons of”}Aaron the priest will put fire on the altar and arrange the wood on the fire.}%
\verse{Then\lebnote{Or “And”}Aaron’s sons, the priests, will arrange the pieces of meat,\lebnote{Verse 6 in the context uses the same word with reference to the meat}the head, and the suet on the wood that is on the fire that is on the altar.}%
\verse{Then\lebnote{Or “And”}he must wash its inner parts\lebnote{Or “entrails”}and its lower leg bones with water, and the priest will turn into smoke the whole animal on the altar as a burnt offering by fire, as an appeasing fragrance for Adonai.}%
\verse{“ ‘But if his offering is from the flock,\lebnote{The Hebrew term refers collectively to both sheep and goats (small livestock animals)}from the sheep or from the goats for a burnt offering, he must present\lebnote{Hebrew “present it”}an unblemished\lebnote{Same word as in v. 3}male.}%
\verse{He\lebnote{Or “And he”}must slaughter it on the north side of the altar \textit{before}\lebnote{Literally “to the face of”}Adonai; then\lebnote{Or “and”}Aaron’s sons, the priests, will sprinkle its blood all around the altar.}%
\verse{Then\lebnote{Or “And”}he must cut it into pieces along with its head and its suet; and the priest will arrange them on the wood that is on the fire that is on the altar.}%
\verse{Then\lebnote{Or “And”}he must wash the inner parts\lebnote{Or “entrails”}and the lower leg bones with water and the priest shall present the whole animal and will turn it\lebnote{Understood direct object}into smoke on the altar; it is a burnt offering by fire as an appeasing fragrance for Adonai.}%
\verse{“ ‘But if his offering for Adonai is a burnt offering from the birds, then he must present his offering from the turtledoves or from \textit{the young doves}.\lebnote{Literally “the sons of the dove”}}%
\verse{The priest\lebnote{Or “And the priest”}will present it at the altar and must wring off its head and turn it into smoke on the altar, and its blood will be drained out on the wall\lebnote{Or “side”}of the altar.}%
\verse{He\lebnote{Or “And he”}must remove its crop with its plumage and throw it to the east beside the altar on the place of the fatty ashes.}%
\verse{Then\lebnote{Or “And”}he must tear it apart by its wings but must not sever it; then\lebnote{Or “and”}the priest will turn it into smoke on the altar. It is a burnt offering by fire as an appeasing fragrance for Adonai.’ ”}%
\end{biblechapter}

\begin{biblechapter} % Leviticus 2
\verseWithHeading{Laws for Grain Offerings}{“ ‘When a person\lebnote{Or “a soul”}brings a grain offering to Adonai, his offering must be finely milled flour, and he must pour out oil on it and place frankincense on it.}%
\verse{And he shall bring it to the sons of Aaron, the priests, and he\lebnote{That is, the priest—see v. 9}shall take \textit{his handful from its finely milled flour}\lebnote{Literally “from there his handful from its finely milled flour”}and from its oil in addition to all its frankincense. The priest\lebnote{Or “And the priest”}shall turn its token portion into smoke on the altar as an offering made by fire, as an appeasing fragrance for Adonai.}%
\verse{The remainder\lebnote{Or “And the remainder”}of the grain offering \textit{belongs to}\lebnote{Literally “for”}Aaron and to his sons—\textit{it is a most holy thing}\lebnote{Literally “a holiness of holinesses”}from the offerings made by fire for\lebnote{Hebrew “of”}Adonai.}%
\verse{“ ‘But if you bring a grain offering of something oven-baked, it must be of finely milled flour as ring-shaped unleavened bread mixed with oil or wafers of unleavened bread smeared with oil.}%
\verse{If your offering is a grain offering baked on a\lebnote{Hebrew “the”}flat baking pan, it must be finely milled flour, unleavened bread mixed with oil;}%
\verse{break it into pieces and pour out oil on it; it is a grain offering.}%
\verse{“ ‘If\lebnote{Or “And if”}your offering is a grain offering prepared in a cooking pan, it must be with finely milled flour in oil.}%
\verse{And you shall bring the grain offering that is made from these things to Adonai, and the offerer\lebnote{Changing from 2ms in the first verb to 3ms in the second verb apparently moves from general to specific; NET takes the second verb as an imperative (“Present it”), and NJPS translates it as an indefinite 3ms, making it passive (“it shall be brought”)}shall bring it to the priest, and he shall bring it to the altar.}%
\verse{And the priest shall take away from the grain offering its token portion, and he shall turn it into smoke on the altar as an offering made by fire, as an appeasing fragrance for Adonai.}%
\verse{And the remainder of the grain offering \textit{belongs to}\lebnote{Literally “for”}Aaron and to his sons—\textit{it is a most holy thing}\lebnote{Literally “a holiness of holinesses”}from the offerings made by fire for\lebnote{Hebrew “of”}Adonai.}%
\verse{“ ‘Every grain offering you\lebnote{The first time a plural 2m verb has been employed since 1:2}bring to Adonai must not be made of yeasted food, because you must not turn into smoke any yeast or any honey from\lebnote{Hebrew “from it”}an offering made by fire for Adonai.}%
\verse{As an offering of the choicest portion, you\lebnote{Another occurrence of the 2mp}may bring them to Adonai, but they must not be offered on the altar as an appeasing fragrance.}%
\verse{Also all\lebnote{Or “And all”}of your grain offerings you must season with salt; you\lebnote{Or “and you”}must not omit the salt of your God’s covenant from your offering.}%
\verse{“ ‘And if you bring to Adonai a grain offering of firstfruits, you must bring an ear of new grain roasted by fire, coarsely crushed ripe grain, as the grain offering of your firstfruits.}%
\verse{And you shall put oil on it and place frankincense on it; it is a grain offering.}%
\verse{The\lebnote{Or “And the”}priest shall turn into smoke its token portion from its coarsely crushed grain together with all of its frankincense—it is an offering made by fire for Adonai.’ ”}%
\end{biblechapter}

\begin{biblechapter} % Leviticus 3
\verseWithHeading{Laws for Fellowship Offerings}{“ ‘Now if\lebnote{Or “And if”}his offering is a sacrifice of fellowship offering, if he brings it from the cattle,\lebnote{Or “the herd”}whether male or female, he must bring it without defect before Adonai.}%
\verse{He\lebnote{Or “And he”}must lay his hand on the head of his offering and slaughter it at the entrance of the tent of assembly, and Aaron’s sons shall sprinkle the blood on the altar all around.}%
\verse{He shall present\lebnote{Or “And he shall present”}from the sacrifice of the fellowship offering an offering made by fire for Adonai consisting of\lebnote{The Hebrew accusative implies this transition}the fat covering the inner parts\lebnote{Or “entrails”}and all the fat that is on the inner parts,\lebnote{Or “entrails”}}%
\verse{the two kidneys,\lebnote{Or “and the two kidneys”}and the fat that is on them, which is on the loins, and he must remove the lobe on the liver in addition to the kidneys.}%
\verse{Aaron’s sons shall turn it to smoke on the altar in addition to the burnt offering that is on the wood, which is on the fire; it is an offering made by fire as an appeasing fragrance for Adonai.}%
\verse{“ ‘But if his offering for a sacrifice of fellowship offering for Adonai is from the flock,\lebnote{The Hebrew term refers collectively to both sheep and goats (small livestock animals)}he must bring a male or a female without defect.}%
\verse{If he brings a sheep as his offering, then\lebnote{Or “and”}he shall present it before Adonai,}%
\verse{and he shall lay his hand on the head of his offering, and he shall slaughter it before the tent of assembly, and Aaron’s sons shall sprinkle its blood on the altar all around.}%
\verse{He shall present\lebnote{Or “And he shall present”}from the sacrifice of the fellowship offering an offering made by fire for Adonai: he must remove its fat, the entire fat tail near the tailbone, and the fat covering the inner parts\lebnote{Or “entrails”}and all the fat that is on the inner parts,\lebnote{Or “entrails”}}%
\verse{the two kidneys,\lebnote{Or “and the two kidneys”}and the fat that is on them, which is on the loins, and he must remove the lobe on the liver in addition to the kidneys.}%
\verse{The priest\lebnote{Or “And the priest”}shall turn it into smoke on the altar as a food offering made by fire for Adonai.}%
\verse{“ ‘And\lebnote{Or “But”}if his offering is a goat, then\lebnote{Or “and”}he shall bring it before Adonai,}%
\verse{and he shall lay his hand on the head of his offering, and he shall slaughter it before the tent of assembly, and Aaron’s sons shall sprinkle its blood on the altar all around.}%
\verse{He shall present\lebnote{Or “And he shall present”}his offering from it as an offering made by fire for Adonai: the fat covering the inner parts\lebnote{Or “entrails”}and all the fat that is on the inner parts,\lebnote{Or “entrails”}}%
\verse{the two kidneys,\lebnote{Or “and the two kidneys”}and the fat that is on them, which is on the loins, and he must remove the lobe on the liver in addition to the kidneys.}%
\verse{The priest\lebnote{Or “And the priest”}shall turn them into smoke on the altar as a food offering;\lebnote{Compare v. 11}all the fat is an offering made by fire as an appeasing fragrance for Adonai.}%
\verse{This is\lebnote{Understood by context}a lasting statute for your generations in all your dwellings: you must not eat any fat or any blood.’ ”}%
\end{biblechapter}

\begin{biblechapter} % Leviticus 4
\verseWithHeading{Laws for Sin Offerings}{Then\lebnote{Or “And”}Adonai spoke to Moses, saying,}%
\verse{“Speak to the \textit{Israelites},\lebnote{Literally “sons/children of Israel”}saying, ‘If a person\lebnote{Or “a soul”}sins by an unintentional wrong from any of Adonai’s commands that should not be \textit{violated},\lebnote{Literally “done”}and he \textit{violates}\lebnote{Literally “does”}\textit{any of them}\lebnote{Literally “from one from them”; see NET}—}%
\verse{if the anointed priest sins, \textit{bringing guilt on the people},\lebnote{Literally “to the guilt of the people”}then,\lebnote{Or “and”}concerning the sin that he has \textit{committed},\lebnote{Literally “sinned”}he shall bring \textit{a young bull}\lebnote{Literally “a bull, a son of cattle”}without defect for Adonai as a sin offering.}%
\verse{He shall bring\lebnote{Or “And he shall bring”}the bull to the tent of assembly’s entrance before Adonai, place\lebnote{Or “And place”}his hand on the bull’s head, and slaughter the bull before Adonai.}%
\verse{The anointed priest\lebnote{Or “And the anointed priest”}shall take \textit{some of}\lebnote{Literally “from”}the bull’s blood and shall bring it to the tent of assembly,}%
\verse{and the priest shall dip his finger in the blood and shall spatter \textit{some of}\lebnote{Literally “from”}the blood seven times before Adonai in front of the sanctuary’s curtain.}%
\verse{The priest\lebnote{Or “And the priest”}shall put \textit{some of}\lebnote{Literally “from”}the blood on the horns of the altar of fragrant incense before Adonai, which is in the tent of assembly, and all the rest\lebnote{Implied by the context}of the bull’s blood he must pour out on the base of the altar of the burnt offering, which is at the entrance of the tent of assembly.}%
\verse{“ ‘He must remove\lebnote{Or “And he must remove”}all the fat from the bull of the sin offering: the fat that covers the inner parts\lebnote{Or “entrails”}and all of the fat that is on the inner parts,\lebnote{Or “entrails”}}%
\verse{the two\lebnote{Or “and the two”}kidneys and the fat that is on them, and the liver’s lobe that he must remove in addition to the kidneys—}%
\verse{just as it is removed from the ox of the fellowship offerings’ sacrifice—and the priest shall turn them into smoke on the altar of the burnt offering.}%
\verse{But\lebnote{Or “And”}the bull’s skin and its meat, in addition to its head, \textit{its lower leg bones},\lebnote{Literally “and in addition to its lower leg bones”}its\lebnote{Or “and its”}inner parts,\lebnote{Or “entrails”}and its offal—}%
\verse{he shall carry\lebnote{Or “and he shall carry out”}all of the bull \textit{outside the camp}\lebnote{Literally “to from an outside place of the camp”}to a ceremonially clean place, to the fatty ashes’ dump, and he shall burn it on wood in the fire; it must be burned up on the fatty ashes’ dump.}%
\verse{“ ‘If\lebnote{Or “And if”}Israel’s whole assembly did wrong unintentionally and the matter\lebnote{Or “thing”}was concealed from the assembly’s eyes, and they acted\lebnote{Or “did”}against \textit{any of}\lebnote{Literally “one from all of”}Adonai’s commands that should not be \textit{violated},\lebnote{Literally “done”}so that\lebnote{Or “and”}they incur guilt,}%
\verse{when\lebnote{Or “and”}the sin that they have \textit{committed}\lebnote{Literally “sinned”}against that command\lebnote{The 3fs pronominal suffix may refer to “one of Adonai’s commands”—“one” is also fs}becomes known, the assembly\lebnote{Or “and the assembly”}shall present \textit{a young bull}\lebnote{Literally “a bull, a son of cattle”}as a sin offering, and they shall bring it before the tent of assembly.}%
\verse{And the elders of the community shall place their hands on the bull’s head before Adonai, and he\lebnote{Or “someone”; see NET—the 3ms refers to either one of the elders or the priest}shall slaughter the bull before Adonai.}%
\verse{Then the anointed priest shall bring \textit{some of}\lebnote{Literally “from”}the bull’s blood to the tent of assembly,}%
\verse{and the priest shall dip his finger in\lebnote{Or “from”}the blood and shall spatter it seven times before Adonai in front of the curtain.}%
\verse{He must put\lebnote{Or “And he must put”}\textit{some of}\lebnote{Literally “from”}the blood on the altar’s horns before Adonai \textit{in the tent of assembly},\lebnote{Literally “which is in the tent of assembly”—the reference is to the altar}and all the rest\lebnote{Indicated by the context}of the blood he must pour out on the base of the altar of the burnt offering, which is at the tent of assembly’s entrance.}%
\verse{He must remove\lebnote{Or “And he must remove”}all its fat from it, and he shall turn it into smoke on the altar.}%
\verse{He shall do\lebnote{Or “And he shall do”}to the bull \textit{just as}\lebnote{Literally “as that”}he did to the sin offering’s bull, so he must do to it. The priest\lebnote{Or “And the priest”}shall make atonement for them, and \textit{they will be forgiven}.\lebnote{Literally “it shall be forgiven to them”}}%
\verse{He shall bring\lebnote{Or “And he shall bring out”}the rest of\lebnote{Indicated by the context}the bull \textit{outside the camp},\lebnote{Literally “to from an outside place of the camp”}and he shall burn it \textit{just as}\lebnote{Literally “as that”}he burned the first bull; it is the sin offering for the assembly.}%
\verse{“ ‘When a leader sins and commits\lebnote{Or “does”}an unintentional wrong against \textit{any of}\lebnote{Literally “one from all of”}the commands of Adonai his God that should not be \textit{violated},\lebnote{Literally “done”}so that\lebnote{Or “and”}he incurs guilt,}%
\verse{or his sin he has \textit{committed}\lebnote{Literally “sinned”}is made known to him, he shall bring\lebnote{Or “and he shall bring” or “then he shall bring”}\textit{a male goat}\lebnote{Literally “a he-goat of goats a male”}without defect as his offering.}%
\verse{He shall place\lebnote{Or “And he shall place”}his hand on the he-goat’s head and slaughter it in the place where he slaughtered the burnt offering before Adonai; it is a sin offering.}%
\verse{The priest\lebnote{Or “And the priest”}shall take \textit{some of}\lebnote{Literally “from”}the sin offering’s blood with his finger, and he shall put it on the horns of the altar of the burnt offering, and he must pour out the rest of\lebnote{Indicated by context}its blood on the base of the altar of the burnt offering.}%
\verse{He\lebnote{Or “And he”; the antecedent is the priest (cp. vv. 10, 31)}must turn all of its fat into smoke on the altar like the fat of the fellowship offerings’ sacrifice, and the priest shall make atonement for him \textit{because of}\lebnote{Literally “from”}his sin, and \textit{he will be forgiven}.\lebnote{Literally “it shall be forgiven to him”}}%
\verse{“ ‘If\lebnote{Or “And if”}\textit{anyone}\lebnote{Literally “a soul one”}of the people of the land sins by an unintentional wrong by \textit{violating}\lebnote{Literally “doing”}one of Adonai’s commands that should not be \textit{violated},\lebnote{Literally “done”}so that\lebnote{Or “and”}he incurs guilt,}%
\verse{or his sin he has \textit{committed}\lebnote{Literally “sinned”}is made known to him, he shall bring\lebnote{Or “and he shall bring” or “then he shall bring”}as his offering \textit{a female goat without defect}\lebnote{Literally “a she-goat of goats without defect a female”}as his offering for his sin that he \textit{committed}.\lebnote{Literally “sinned”}}%
\verse{He shall place\lebnote{Or “And he shall place”}his hand on the sin offering’s head and slaughter the sin offering in the place of the burnt offering.}%
\verse{The priest\lebnote{Or “And the priest”}shall take \textit{some of}\lebnote{Literally “from”}its blood with his finger, and he shall put it on the horns of the altar of the burnt offering, and he must pour out all the rest of\lebnote{Indicated by context}its blood on the altar’s base.}%
\verse{He\lebnote{Or “And he”}must remove all of its fat \textit{just as}\lebnote{Literally “as that”}the fat was removed from\lebnote{Or “from on” or “from upon”}the fellowship offerings’ sacrifice, and the priest shall turn it into smoke on the altar as an appeasing fragrance for Adonai. The priest\lebnote{Or “And the priest”}shall make atonement for him, and \textit{he will be forgiven}.\lebnote{Literally “it shall be forgiven to him”}}%
\verse{“ ‘But\lebnote{Or “And”}if he brings a lamb as his offering for a sin offering, he must bring a female without defect.}%
\verse{He shall place\lebnote{Or “And he shall place”}his hand on the sin offering’s head, and he shall slaughter it as a sin offering in the place where he slaughtered the burnt offering.}%
\verse{The priest\lebnote{Or “And the priest”}shall take \textit{some of}\lebnote{Literally “from”}the sin offering’s blood with his finger, and he shall put it on the horns of the altar of the burnt offering, and he must pour out all the rest\lebnote{Indicated by context}of its blood on the altar’s base.}%
\verse{He must remove\lebnote{Or “And he must remove”}all of its fat \textit{just as}\lebnote{Literally “as that”}the lamb’s fat from the fellowship offerings’ sacrifice was removed, and the priest shall turn them into smoke on the altar upon Adonai’s offerings made by fire; and the priest shall make atonement for him because of\lebnote{Or “for”}his sin that he \textit{committed},\lebnote{Literally “sinned”}and \textit{he will be forgiven}.’ ”\lebnote{Literally “it will be forgiven to him”}}%
\end{biblechapter}

\begin{biblechapter} % Leviticus 5
\verseWithHeading{More Laws Regarding Sin Offerings}{“ ‘When a person\lebnote{Or “a soul”}sins in that\lebnote{Or “and”}he hears the utterance of a curse and he is a witness or he sees or he knows, if he does not make it known, then\lebnote{Or “and”}he shall bear his guilt.}%
\verse{Or if a person\lebnote{Or “a soul”}touches anything unclean, \textit{whether}\lebnote{Literally “or”}an unclean wild\lebnote{Implied by the following phrase specifying domestic animals}animal’s dead body or an unclean domestic animal’s dead body or an unclean swarmer’s dead body, but\lebnote{Or “and”}\textit{he is unaware of it},\lebnote{Literally “it is concealed from him”}he is unclean and he is guilty.}%
\verse{Or when he touches human uncleanness, \textit{namely}\lebnote{Literally “for” or “to” (see HALOT 510)}any uncleanness of his by which he might become unclean, but\lebnote{Or “and”}\textit{he is unaware of it},\lebnote{Literally “it is concealed from him”}and he himself finds out, then\lebnote{Or “and”}he will be guilty.}%
\verse{Or when a person\lebnote{Or “a soul”}swears, speaking thoughtlessly with his lips, to do evil or to do good \textit{with regard to}\lebnote{Literally “for” or “to” (see HALOT 510)}anything that \textit{the person}\lebnote{Literally “the man”}in a sworn oath speaks thoughtlessly, but\lebnote{Or “and”}\textit{he is unaware of it},\lebnote{Literally “it is concealed from him”}he will be guilty \textit{in any of}\lebnote{Literally “for one of”}these.}%
\verse{When he becomes guilty \textit{in any of}\lebnote{Literally “for one of”}these, he shall confess\lebnote{Or “and he shall confess”}what he has sinned \textit{regarding}\lebnote{Literally “upon” or “against”}it,}%
\verse{and he shall bring his guilt offering to\lebnote{Or “for”}Adonai for his sin that he has \textit{committed}:\lebnote{Literally “sinned”}a female from the flock,\lebnote{The Hebrew term refers collectively to both sheep and goats (small livestock animals)}a ewe-lamb or \textit{a she-goat},\lebnote{Literally “a she-goat of goats”}as a sin offering, and the priest shall make atonement for him for\lebnote{Hebrew “from”}his sin.}%
\verse{“ ‘If\lebnote{Or “And if”}\textit{he cannot afford a sheep},\lebnote{Literally “his hand does not touch enough of small livestock”}he shall bring as his guilt offering for what he sinned two turtledoves or two \textit{young doves}\lebnote{Literally “sons of dove” or “children of dove”}for Adonai, one for a sin offering and one for a burnt offering.}%
\verse{He shall bring\lebnote{Or “And he shall bring”}them to the priest, and he shall present that which is for the sin offering first, and the priest\lebnote{Required by the previous action (see NET)}shall wring its head off \textit{at the back of its neck},\lebnote{Literally “from opposite its neck”}but\lebnote{Or “and”}he must not sever it,}%
\verse{and he shall spatter \textit{some of}\lebnote{Literally “from”}the sin offering’s blood on the altar’s side, and the leftover blood must be drained out on the altar’s base; it is a sin offering.}%
\verse{The second\lebnote{Or “And the second”}bird he must prepare as a burnt offering according to the regulation, and the priest shall make atonement for him for his sin that he has \textit{committed},\lebnote{Literally “sinned”}and he shall be forgiven.}%
\verse{“ ‘But\lebnote{Or “And”}if \textit{he cannot afford}\lebnote{Literally “his hand cannot produce for”}two turtledoves or two \textit{young doves},\lebnote{Literally “sons of dove” or “children of dove”}then,\lebnote{Or “and”}because he has sinned, he shall bring as his offering a tenth of an\lebnote{Hebrew “the”}ephah of finely milled flour as a sin offering. He must not put\lebnote{Or “And he must not put”}oil on it, nor should he put frankincense on it, because it is a sin offering.}%
\verse{He shall bring\lebnote{Or “And he shall bring”}it to the priest, and the priest \textit{shall take a handful of it}\lebnote{Literally “shall scoop up from it the fullness of his handful”}for its token portion, and he shall turn it to smoke on the altar \textit{in addition to}\lebnote{Literally “on” or “upon”}the offerings made by fire to Adonai; it is a sin offering.}%
\verse{Thus\lebnote{Or “And”}the priest shall make atonement for him because of the sin that he has \textit{committed}\lebnote{Literally “sinned”}\textit{in any of these},\lebnote{Literally “from one from these”}and he shall be forgiven. It shall be\lebnote{Or “And it shall be”}for the priest, like the grain offering.’ ”}%
\verseWithHeading{Laws for Guilt Offerings}{Then\lebnote{Or “And”}Adonai spoke to Moses, saying,}%
\verse{“When a person\lebnote{Or “a soul”}\textit{displays infidelity}\lebnote{Literally “acts unfaithfully infidelity” or “is unfaithful in unfaithfulness”}and he sins in an unintentional wrong \textit{in any of}\lebnote{Literally “from”}Adonai’s holy things, then\lebnote{Or “and”}he shall bring his guilt offering to\lebnote{Or “for”}Adonai: a ram without defect from the flock\lebnote{The Hebrew term refers collectively to both sheep and goats (small livestock animals)}as a guilt offering by your valuation in silver shekels\lebnote{Or “convertible into silver shekels” (NET, NRSV, TNK), “of the proper value in silver” (NIV; cp. ESV “valued in silver shekels,” CSB), or “or you may buy one of equal value with silver” (NLT)}according to the sanctuary shekel.}%
\verse{And he shall make restitution for what he sinned \textit{because of}\lebnote{Literally “from” (see HALOT 598)}a holy thing and shall add one-fifth of \textit{its value}\lebnote{Literally “it”}onto it and shall give it to the priest. The priest\lebnote{Or “And the priest”}shall make atonement for him with the ram of the guilt offering, and he will be forgiven.}%
\verse{“If\lebnote{Or “And if”}a person\lebnote{Or “a soul”}when he sins \textit{violates}\lebnote{Literally “and she/it does”}one from all of Adonai’s commands that should not \textit{be violated},\lebnote{Literally “they are to be done”}but\lebnote{Or “and”}he did not know, then\lebnote{Or “and”}he is guilty and he shall bear his guilt.}%
\verse{He shall bring\lebnote{Or “And he shall bring”}to the priest a ram without defect from the flock\lebnote{The Hebrew term refers collectively to both sheep and goats (small livestock animals)}as a guilt offering by your valuation, and the priest shall make atonement for him because of his unintentional wrong (although\lebnote{Or “and”}he himself did not know), and he will be forgiven.}%
\verse{It is a guilt offering; he certainly was guilty before Adonai.”}%
\end{biblechapter}

\begin{biblechapter} % Leviticus 6
\verseWithHeading{Additional Laws for Burnt Offerings}{\lebnote{Leviticus 6:1–30 in the English Bible is 5:20–6:23 in the Hebrew Bible} Then\lebnote{Or “And”}Adonai spoke to Moses, saying,}%
\verse{“When a person\lebnote{Or “a soul”}\textit{displays infidelity}\lebnote{Literally “acts unfaithfully infidelity” or “is unfaithful in unfaithfulness”}against Adonai and he deceives his fellow citizen regarding\lebnote{Literally “in”}something entrusted or \textit{a pledge}\lebnote{Literally “a pledge of a hand”}or stealing or he exploits his fellow citizen,}%
\verse{or he finds lost property and lies about it and swears \textit{falsely}\lebnote{Literally “in accordance with deception”}\textit{in regard to}\lebnote{Literally “on” or “upon”}any one of these things by which \textit{a person}\lebnote{Literally “the man”}might commit sin,}%
\verse{\textit{and when}\lebnote{Literally “and it will be when”}he sins and is guilty, then\lebnote{Or “and”}he shall bring back \textit{the things he had stolen}\lebnote{Literally “the robbed things that he had stolen”}or \textit{what he had extorted}\lebnote{Literally “the extortion that he had extorted”}or \textit{something with which he had been entrusted}\lebnote{Literally “something entrusted that had been entrusted to him”}or the lost property that he had found,}%
\verse{or \textit{regarding}\lebnote{Literally “from”}anything about which he has sworn \textit{falsely},\lebnote{Literally “in accordance with deception”}then\lebnote{Or “and”}he shall repay it \textit{according to}\lebnote{Literally “in”}its value and shall add one-fifth of its value to it—he must give it \textit{to whom it belongs}\lebnote{Literally “to whom it is to him”}on the day of his guilt offering.}%
\verse{And he must bring as his guilt offering to Adonai a ram without defect from the flock\lebnote{The Hebrew term refers collectively to both sheep and goats (small livestock animals)}by your valuation\lebnote{See 5:15 and 18}as a guilt offering to the priest,}%
\verse{and the priest shall make atonement for him \textit{before}\lebnote{Literally “to the faces of”}Adonai, and he shall be forgiven \textit{anything}\lebnote{Literally “one”}from all that he might do \textit{by which he might incur guilt}.”\lebnote{Literally “which he might do for incurring guilt by it”}}%
\verse{Then\lebnote{Or “And”}Adonai spoke to Moses, saying,}%
\verse{“Command Aaron and his sons, saying, ‘This is the regulation of the burnt offering: \textit{The burnt offering must remain on the hearth}\lebnote{Literally “It shall be the burnt offering on a hearth”}on the altar all night until the morning, and the altar’s fire must be kept burning on it.}%
\verse{And the priest shall put on his linen robe, and he must put his linen undergarments on his body, and he shall take away the fatty ashes of the burnt offering that the fire has consumed on the altar, and he shall place them\lebnote{Hebrew “him/it”—plural required by the English “fatty ashes”}beside the altar.}%
\verse{And he shall take off his garments and put on other garments, and he shall bring out the fatty ashes \textit{outside the camp}\lebnote{Literally “to from an outside place of the camp”}to a ceremonially clean place,}%
\verse{but\lebnote{Or “and”}the fire on the altar must be kept burning on it; it must not be quenched. And the priest must burn wood \textit{every morning}\lebnote{Literally “in the morning in the morning”}on it,\lebnote{Antecedent for this 3fs suffix is “fire” (“altar” is ms)}and he shall arrange the burnt offering on it,\lebnote{Antecedent for this 3fs suffix is “fire” (“altar” is ms)}and he shall turn into smoke the fat portions of the fellowship offerings on it.\lebnote{Antecedent for this 3fs suffix is “fire” (“altar” is ms)}}%
\verse{A perpetual fire must be kept burning on the altar; it must not be quenched.}%
\verseWithHeading{Additional Laws for Grain Offerings}{“ ‘And this is the regulation of the grain offering. Aaron’s sons shall present it \textit{before}\lebnote{Literally “to the faces of”}Adonai \textit{in front of}\lebnote{Literally “to the faces of”}the altar,}%
\verse{and he\lebnote{That is, the priest; understood by context and 3ms verb}in his fist shall take away from it \textit{some of}\lebnote{Literally “from”}the grain offering’s finely milled flour, and \textit{some of}\lebnote{Literally “from”}its oil and all of the frankincense that is on the grain offering, and he shall turn into smoke its token portion on the altar as an appeasing fragrance to Adonai.}%
\verse{And Aaron and his sons must eat the remainder of it; they must eat it as unleavened bread in a holy place—in the tent of assembly’s courtyard they must eat it.}%
\verse{It must not be baked with yeast. I have given it as their share from my offerings made by fire. It is \textit{a most holy thing},\lebnote{Literally “a holiness of holinesses”}like the sin offering and like the guilt offering.}%
\verse{Every male among Aaron’s sons may eat it as a lasting rule among your generations from the offerings made by fire \textit{belonging to}\lebnote{Literally “of”}Adonai. Anything that\lebnote{Or “Everyone who”}touches them will become holy.’ ”}%
\verse{Then\lebnote{Or “And”}Adonai spoke to Moses, saying,}%
\verse{“This is the offering of Aaron and his sons that they shall present to Adonai on the day of his being anointed: a tenth of an\lebnote{Hebrew “the”}ephah of finely milled flour as a perpetual grain offering, half of it in the morning and half of it in the evening.}%
\verse{It must be made\lebnote{Or “prepared”}in\lebnote{Or “with”}oil on a flat baking pan; you\lebnote{Singular masculine}must bring it well-mixed; you must present pieces of a grain offering’s baked goods\lebnote{Or “broken bits” (JPS, NET, NIV)}as an appeasing fragrance to Adonai.}%
\verse{And the anointed priest taking his place from among his sons must do it. As a lasting rule, it must be turned into smoke totally for Adonai.}%
\verse{And every grain offering of a priest must be a whole burnt offering; it must not be eaten.”}%
\verseWithHeading{Additional Laws for Sin Offerings}{Then\lebnote{Or “And”}Adonai spoke to Moses, saying,}%
\verse{“Speak to Aaron and his sons, saying, ‘This is the regulation of the sin offering: In the place where the sin offering is slaughtered, the sin offering must be slaughtered \textit{before}\lebnote{Literally “to the faces of”}Adonai; it is \textit{a most holy thing}.\lebnote{Literally “a holiness of holinesses”}}%
\verse{The priest \textit{who offers the sin offering}\lebnote{Literally “one who offers it a sin offering”}must eat it in a holy place—in the tent of assembly’s courtyard.}%
\verse{Anything that\lebnote{Or “Everyone who”}touches its flesh will become holy, and when \textit{some of}\lebnote{Literally “from”}its blood spatters on a\lebnote{Hebrew “the”}garment, what was spattered on it you\lebnote{Singular masculine}shall wash in a holy place.}%
\verse{And a clay vessel in which it was boiled must be broken, but\lebnote{Or “and”}if it was boiled in a bronze vessel, then\lebnote{Or “and”}it shall be thoroughly scoured and rinsed with water.}%
\verse{Any male among the priests may eat it; it is \textit{a most holy thing}.\lebnote{Literally “a holiness of holinesses”}}%
\verse{But\lebnote{Or “And”}any sin offering from which \textit{some of}\lebnote{Literally “from”}its blood is brought to the tent of assembly to make atonement in the sanctuary must not be eaten; it must be burned in the fire.’ ”}%
\end{biblechapter}

\begin{biblechapter} % Leviticus 7
\verseWithHeading{Additional Laws for Guilt Offerings}{“ ‘And this is the regulation of the guilt offering; it is \textit{a most holy thing}.\lebnote{Literally “a holiness of holinesses”}}%
\verse{In the place where they slaughter the burnt offering,\lebnote{Or, taking the verb as an indefinite (thus passive) imperfect 3mp, “the sin offering is slaughtered”}they must slaughter the guilt offering,\lebnote{Or, taking the verb as an indefinite (thus passive) imperfect 3mp, “the guilt offering must be slaughtered”}and he\lebnote{That is, the priest; understood by context and 3ms verb}must sprinkle its blood upon the altar all around.}%
\verse{And he must present all of its fat:\lebnote{Hebrew “all of its fat from it”}the fat tail and the fat that covers the inner parts,\lebnote{Or “entrails”}}%
\verse{and the two kidneys, and the fat that is on them, which is on the loins, and he must remove the lobe on the liver in addition to the kidneys.}%
\verse{And the priest shall turn it into smoke it on the altar as a food offering made by fire for Adonai; it is a guilt offering.}%
\verse{Every male among the priests may eat it; it must be eaten in a holy place; it is \textit{a most holy thing}.\lebnote{Literally “a holiness of holinesses”}}%
\verse{The instruction is \textit{the same for the guilt offering as for the sin offering};\lebnote{Literally “as the sin offering as the guilt offering one for them”}\textit{it belongs to}\lebnote{Literally “for him it shall be” or “it will become his”}the priest, who makes atonement with it.}%
\verseWithHeading{Portions for the Priests}{“ ‘And\lebnote{Or “As for”}the priest who presents \textit{a person’s}\lebnote{Literally “of a man”}burnt offering, to that\lebnote{Hebrew “the”}priest \textit{belongs}\lebnote{Literally “to/for him it shall be”}the skin of the burnt offering that he presented.}%
\verse{And every grain offering that is baked in the oven and all that is prepared in a\lebnote{Hebrew “the”}cooking pan or\lebnote{Or “and”}on a flat baking pan \textit{belongs to}\lebnote{Literally “for him it shall be” or “it will become his”}the priest who presented it.}%
\verse{And every grain offering, whether mixed with oil or dry, shall be for all of Aaron’s sons \textit{equally}.\lebnote{Literally “each as his brother”}}%
\verseWithHeading{Additional Laws for Fellowship Offerings}{“ ‘And this is the regulation of the fellowship offerings that he must present to Adonai:}%
\verse{If he presents it for thanksgiving, in addition to the thanksgiving sacrifice he shall present ring-shaped unleavened bread mixed with oil and unleavened bread wafers smeared with oil and well-mixed ring-shaped bread cakes of finely milled flour mixed with oil.}%
\verse{In addition to ring-shaped cakes of bread with yeast, he must present his grain\lebnote{Implied by v. 12}offering together with\lebnote{Or “in addition to”}his sacrifice of thanksgiving peace offerings.}%
\verse{And he shall present one of each kind of grain\lebnote{Implied by v. 12}offering as a contribution for Adonai; \textit{it belongs to}\lebnote{Literally “for him it shall be” or “it will become his”}the priest who sprinkles the fellowship offerings’ blood.}%
\verse{And the meat of the sacrifice of his thanksgiving fellowship offerings must be eaten on the day of his offering; he must not leave it until morning.}%
\verse{“ ‘But\lebnote{Or “And”}if his sacrifice is for a vow or as a freewill offering, it must be eaten on the day of his presenting his sacrifice, and on the next day the remainder\lebnote{Or “and the remainder”}from it may be eaten,}%
\verse{but\lebnote{Or “and”}the remainder from the sacrifice’s meat must be burned up in the fire on the third day.}%
\verse{And if indeed some of the meat of his fellowship offerings’ sacrifice is eaten on the third day, it will not be accepted; it will not be considered of benefit for the one who presented it—it shall be unclean meat, and the person\lebnote{Or “soul”}who eats it shall bear his guilt.}%
\verse{And the meat that touches anything unclean must not be eaten; it must be burned with fire, and as for the clean\lebnote{Understood by context}meat, anyone who is clean may eat the meat.}%
\verse{And the person\lebnote{Or “the soul”}who eats meat from the fellowship offerings’ sacrifice, which is for Adonai, and whose uncleanness is upon him—that person\lebnote{Or “the soul”}shall be cut off from his people.}%
\verse{And when a person\lebnote{Or “a soul”}touches anything unclean, whether human uncleanness or an unclean animal or any unclean detestable thing, and he eats from the meat of the fellowship offerings’ sacrifice, which is for Adonai, then\lebnote{Or “and”}that person\lebnote{Or “the soul”}shall be cut off from his people.’ ”}%
\verseWithHeading{Instructions for the People}{Then\lebnote{Or “And”}Adonai spoke to Moses, saying,}%
\verse{“Speak to the \textit{Israelites},\lebnote{Literally “sons/children of Israel”}saying, ‘You\lebnote{Plural}must not eat any fat of ox, or\lebnote{Or “and”}sheep, or\lebnote{Or “and”}goat;}%
\verse{and a dead body’s fat or\lebnote{Or “and”}mangled carcass’s fat may be used for any \textit{purpose},\lebnote{Literally “work”}but\lebnote{Or “and”}you\lebnote{Plural}certainly must not eat it.}%
\verse{When anyone eats fat from the domestic animal from which he presented an offering made by fire for Adonai, then\lebnote{Or “and”}that person\lebnote{Or “the soul”}who ate shall be cut off from his people.}%
\verse{And in any of your\lebnote{Plural}dwellings, you must not eat any blood \textit{belonging to}\lebnote{Literally “to/of”}birds\lebnote{Hebrew “the bird”; generic article with a collective noun}or\lebnote{Or “and”}domestic animals.\lebnote{Hebrew “the domestic animal”; generic article with a collective noun}}%
\verse{Any person\lebnote{Or “any soul” or “all soul(s)”}who eats any blood, that person\lebnote{Or “and that soul”}shall be cut off from his people.’ ”}%
\verseWithHeading{Portions of Fellowship Offerings for Priests}{Then\lebnote{Or “And”}Adonai spoke to Moses, saying,}%
\verse{“Speak to the \textit{Israelites},\lebnote{Literally “sons/children of Israel”}saying, ‘The one who presents his fellowship offerings’ sacrifice for Adonai shall bring his offering to Adonai from his fellowship offerings’ sacrifice.}%
\verse{His own hands must bring Adonai’s offerings made by fire. He must bring the fat in addition to the breast section to wave the breast section as a wave offering before Adonai,}%
\verse{and the priest shall turn the fat into smoke on the altar, and the breast section shall be for Aaron and his\lebnote{Hebrew “for his”}sons.}%
\verse{And the right upper thigh you\lebnote{Plural}must give as a contribution for the priest from your\lebnote{Plural}fellowship offerings’ sacrifice.}%
\verse{As for the one from Aaron’s sons who presents the blood of the fellowship offerings and the fat, the right upper thigh \textit{shall belong to him}\lebnote{Literally “for him it shall be”}as his share,}%
\verse{because I have taken the wave offering’s breast section and the contribution offering’s upper thigh from the \textit{Israelites}\lebnote{Literally “sons/children of Israel”}out of their fellowship offerings’ sacrifices, and I have given them to Aaron the priest and his\lebnote{Hebrew “to his”}sons from the \textit{Israelites}\lebnote{Literally “sons/children of Israel”}as a lasting rule.’ ”}%
\verse{This is Aaron’s allotted portion and his sons’ allotted portion from Adonai’s offerings made by fire \textit{when}\lebnote{Literally “in a day”}he brought them forward to serve as priests for Adonai.}%
\verse{This is what Adonai commanded to give them from the \textit{Israelites}\lebnote{Literally “sons/children of Israel”}on the day of his anointing them; it is a lasting statute for their generations.}%
\verseWithHeading{Concluding Summary Concerning the Offerings}{This is the regulation for the burnt offering, for the grain offering and for the sin offering and for the guilt offering and for the consecration offering and for the fellowship offerings’ sacrifice,}%
\verse{which Adonai commanded Moses on \textit{Mount Sinai}\lebnote{Literally “the mountain of Sinai”}on the day of his commanding the \textit{Israelites}\lebnote{Literally “sons/children of Israel”}to present their offerings to Adonai in the desert of Sinai.}%
\end{biblechapter}

\begin{biblechapter} % Leviticus 8
\verseWithHeading{Installing the Priests}{Then\lebnote{Or “And”}Adonai spoke to Moses, saying,}%
\verse{“Take Aaron and his sons with him, and the garments and the anointing oil and the bull of the sin offering and the two rams and the basket of the unleavened bread,}%
\verse{and summon all of the community to the entrance to\lebnote{Or “of”}the tent of assembly.”}%
\verse{So\lebnote{Or “And”}Moses did just as Adonai commanded him, and the community gathered by the entrance to\lebnote{Or “of”}the tent of assembly.}%
\verse{Then\lebnote{Or “And”}Moses said to the community, “This is the word that Adonai has commanded to be done.”\lebnote{Or “to do” or “me to do”}}%
\verse{So\lebnote{Or “And”}Moses brought Aaron and his sons near, and he washed them with water.}%
\verse{Then\lebnote{Or “And”}he put the tunic on him and tied\lebnote{Or “he tied”}the sash around him; then\lebnote{Or “and”}he clothed him with the robe and put\lebnote{Or “he put”}the ephod on him; then\lebnote{Or “and”}he tied the ephod’s waistband around him and fastened\lebnote{Or “he fastened”}the ephod\lebnote{Understood by context}to him with it.}%
\verse{Then\lebnote{Or “And”}he placed the breastpiece on him and put\lebnote{Or “he put”}the Urim and the Thummim into the breastpiece;}%
\verse{and he placed the turban on his head, and on \textit{the front of}\lebnote{Literally “to the front of his faces”}the turban he placed the gold rosette, the holy diadem, just as Adonai had commanded Moses.}%
\verse{Then\lebnote{Or “And”}Moses took the anointing oil and anointed\lebnote{Or “he anointed”}the tabernacle and all that was in it, and he consecrated them.}%
\verse{And he spattered \textit{part of}\lebnote{Literally “from”}it on the altar seven times—thus\lebnote{Or “and”}he anointed the altar and all of its utensils, and the basin and its stand, to consecrate them.}%
\verse{Then\lebnote{Or “And”}he poured out \textit{part of}\lebnote{Literally “from”}the anointing oil on Aaron’s head—thus\lebnote{Or “and”}he anointed him in order to consecrate him.}%
\verse{Then\lebnote{Or “And”}Moses brought Aaron’s sons near and clothed\lebnote{Or “he clothed”}them with tunics and tied\lebnote{Or “he tied”}a sash around \textit{each one},\lebnote{Literally “them”}and he bound headbands on\lebnote{Or “to”}them, just as Adonai had commanded Moses.}%
\verseWithHeading{The Offerings for Consecration}{Then\lebnote{Or “And”}he brought forth the bull of the sin offering, and Aaron and his sons placed their hands on the head of the bull of the sin offering,}%
\verse{and he slaughtered it, and Moses took the blood and put\lebnote{Or “he put”}it with his finger on the altar’s horns all around and purified the altar; then\lebnote{Or “and”}he poured the blood out on the altar’s base—thus\lebnote{Or “and”}he consecrated it in order to make atonement for it.\lebnote{Or “to make atonement upon it” (see NET, Tanakh)}}%
\verse{Then\lebnote{Or “And”}he took all the fat that was on the inner parts\lebnote{Or “entrails”}and the lobe on the liver and the two kidneys and their fat, and Moses turned them into smoke on the altar,}%
\verse{but\lebnote{Or “and”}he burned the bull and its skin and its meat and its offal in the fire \textit{outside the camp},\lebnote{Literally “from an outside place of the camp”}just as Adonai had commanded Moses.}%
\verse{Then\lebnote{Or “And”}he brought the ram of the burnt offering near, and Aaron and his sons placed their hands on the ram’s head,}%
\verse{and he slaughtered it. Then\lebnote{Or “And”}Moses sprinkled the blood on the altar all around.}%
\verse{Then\lebnote{Or “And”}he cut the ram into pieces, and Moses turned into smoke the head and the pieces and the suet,}%
\verse{but\lebnote{Or “and”}he washed the inner parts\lebnote{Or “entrails”}and the lower leg bones with water, and Moses turned into smoke all of the ram on the altar; it was a burnt offering as an appeasing fragrance, an offering made by fire for\lebnote{Or “to”}Adonai, just as Adonai had commanded Moses.}%
\verse{Then\lebnote{Or “And”}he brought the second ram near, the ram of the consecration, and Aaron and his sons placed their hands on the ram’s head,}%
\verse{and he slaughtered it. Then\lebnote{Or “And”}Moses took \textit{some of}\lebnote{Literally “from”}its blood and put\lebnote{Or “he put”}it on Aaron’s right ear lobe and on his right hand’s thumb and on his right foot’s big toe.}%
\verse{Then\lebnote{Or “And”}he brought Aaron’s sons near, and Moses put \textit{some of}\lebnote{Literally “from”}the blood on their right ear lobe and on their right hand’s thumb and on their right foot’s big toe, and Moses sprinkled the blood on the altar all around.}%
\verse{Then\lebnote{Or “And”}he took the fat and the fat tail and all of the fat that was on the inner parts\lebnote{Or “entrails”}and the lobe of the liver and the two kidneys and their fat and the right upper thigh;}%
\verse{and from the basket of the unleavened bread that was before Adonai he took one ring-shaped unleavened bread and one ring-shaped bread with\lebnote{Or “of”}oil and one wafer, and he placed them on the \textit{fat parts}\lebnote{Literally “fats”}and on the right upper thigh.}%
\verse{Then\lebnote{Or “And”}he put \textit{all of these}\lebnote{Literally “the all”}on Aaron’s palms and on his sons’ palms, and he waved them as a wave offering \textit{before}\lebnote{Literally “to the faces of”}Adonai.}%
\verse{Then\lebnote{Or “And”}Moses took them from upon their palms, and he turned them into smoke upon the burnt offering on the altar; they were a consecration offering as an appeasing fragrance—it was an offering made by fire for\lebnote{Or “to”}Adonai.}%
\verse{Then\lebnote{Or “And”}Moses took the breast section, and he waved it as a wave offering \textit{before}\lebnote{Literally “to the faces of”}Adonai from the ram of the consecration offering; it was Moses’ share, just as Adonai had commanded Moses.}%
\verseWithHeading{Anointing the Priests and Their Garments}{Then\lebnote{Or “And”}Moses took \textit{some of}\lebnote{Literally “from”}the anointing oil and \textit{some of}\lebnote{Literally “from”}the blood that was on the altar, and he spattered them on Aaron, on his garments, and on Aaron’s sons and on his sons’ garments with him—thus\lebnote{Or “and”}he consecrated Aaron, his garments, and his sons and his sons’ garments with him.}%
\verse{Then\lebnote{Or “And”}Moses said to Aaron and to his sons, “Boil the meat in the entrance to\lebnote{Or “of”}the tent of assembly, and there you must eat it and the bread that is in the basket of the consecration offering, just as I have commanded, saying, ‘Aaron and his sons must eat it,’}%
\verse{but\lebnote{Or “and”}the remainder of the meat and the bread you\lebnote{Plural}must burn in the fire.}%
\verse{And you must not go out from the entrance to\lebnote{Or “of”}the tent of assembly for seven days, until the day of fulfilling the days of your\lebnote{Plural}consecration, because \textit{it will take seven days to ordain you}.\lebnote{Literally “seven of days it will fill your (plural) hand”}}%
\verse{Just as was done\lebnote{Or “he did”}on this day, Adonai commanded to be done\lebnote{Or “to do”}in order to make atonement for you.\lebnote{Plural}}%
\verse{And you\lebnote{Plural}must stay at the entrance to\lebnote{Or “of”}the tent of assembly day and night for seven days, and you\lebnote{Plural}shall keep the obligation from\lebnote{Or “of”}Adonai, so you\lebnote{Plural}might not die, for thus I have been commanded.”}%
\verse{So\lebnote{Or “And”}Aaron and his sons did all the things that Adonai had commanded \textit{through}\lebnote{Literally “by the hand of”}Moses.}%
\end{biblechapter}

\begin{biblechapter} % Leviticus 9
\verseWithHeading{Worship at the Tent of Assembly}{Then\lebnote{Or “And it happened”}on the eighth day Moses summoned Aaron and his sons and Israel’s elders,}%
\verse{and he said to Aaron, “Take for yourself \textit{a bull calf}\lebnote{Literally “a bull-calf a son of cattle” or “a bull-calf a son of the herd”}as a sin offering and a ram as a burnt offering, without defect,\lebnote{An adjective in masculine plural to modify both animals; see NET}and present them \textit{before}\lebnote{Literally “to the faces of”}Adonai.}%
\verse{Then\lebnote{Or “And”}you\lebnote{Singular}must speak to the \textit{Israelites},\lebnote{Literally “sons/children of Israel”}saying, ‘Take \textit{a he-goat}\lebnote{Literally “a he-goat of goats”}as a sin offering and a bull calf and a male sheep, \textit{yearlings}\lebnote{Literally “sons of a year”}without defect,\lebnote{An adjective in masculine plural to modify both animals; see NET}as a burnt offering,}%
\verse{and an ox and a ram as fellowship offerings to sacrifice \textit{before}\lebnote{Literally “to the faces of”}Adonai, and a grain offering mixed with oil, because today Adonai will appear to you.”\lebnote{Plural}}%
\verse{So\lebnote{Or “And”}they took what Moses had commanded to the \textit{front of}\lebnote{Literally “to the faces of”}the tent of assembly, and the whole community presented themselves, and they stood \textit{before}\lebnote{Literally “to the faces of”}Adonai.}%
\verse{Then\lebnote{Or “And”}Moses said, “This is the word that Adonai commanded you\lebnote{Plural}to do \textit{so that}\lebnote{Or “and”}the glory of Adonai might appear to you.”\lebnote{Plural}}%
\verse{Then\lebnote{Or “And”}Moses said to Aaron, “Approach\lebnote{Or “Draw near unto”}the altar and \textit{sacrifice}\lebnote{Literally “do” or “make”}your sin offering and your burnt offering, and make atonement for yourself and for the people. And \textit{sacrifice}\lebnote{Literally “do” or “make”}the people’s offering and make atonement for them, just as Adonai has commanded.”}%
\verse{Then\lebnote{Or “And”}Aaron approached\lebnote{Or “drew near unto”}the altar, and he slaughtered the bull calf of the sin offering, which was for himself.}%
\verse{Then\lebnote{Or “And”}Aaron’s sons presented the blood to him, and he dipped his finger in the blood, and he put it on the altar’s horns, and he poured out the blood on the altar’s base.}%
\verse{And the fat and the kidneys and the lobe \textit{from the sin offering’s liver}\lebnote{Literally “from the liver from the sin offering”}he turned into smoke on the altar, just as Adonai had commanded Moses,}%
\verse{but\lebnote{Or “and”}the meat and the skin he burned \textit{with fire}\lebnote{Literally “in the fire”}\textit{outside the camp}.\lebnote{Literally “from an outside place to/of the camp”}}%
\verse{Then\lebnote{Or “And”}he slaughtered the burnt offering, and Aaron’s sons brought the blood to him, and he sprinkled it on the altar all around;}%
\verse{and they brought the burnt offering to him by its pieces, as well as\lebnote{Or “and”}the head, and he turned them into smoke on the altar;}%
\verse{and he washed the inner parts\lebnote{Or “entrails”}and the lower leg bones, then\lebnote{Or “and”}he turned them into smoke upon the burnt offering on the altar.}%
\verse{Then\lebnote{Or “And”}he presented the people’s offering, and he took the goat of the sin offering, which was for the people, and he slaughtered it and offered\lebnote{Or “he offered”}it as a sin offering like the first one.}%
\verse{Then\lebnote{Or “And”}he presented the burnt offering, and he \textit{sacrificed}\lebnote{Literally “did” or “made”}it according to the regulation.}%
\verse{Then\lebnote{Or “And”}he presented the grain offering, and he filled his palm \textit{with some of}\lebnote{Literally “from”}it, and he turned it into smoke on the altar besides the morning’s burnt offering.}%
\verse{Then\lebnote{Or “And”}he slaughtered the ox and the ram, the fellowship offerings that are for the people, and Aaron’s sons brought the blood to him, and he sprinkled it on the altar all around.}%
\verse{And as for the fat portions from the ox and from the ram (the fat tail and the layer of fat and the kidneys and the lobe of the liver),}%
\verse{they placed the fat portions on the breast sections, and he turned the fat portions into smoke on the altar.}%
\verse{Then\lebnote{Or “And”}Aaron waved the breast sections and the right upper thigh as a wave offering \textit{before}\lebnote{Literally “to the faces of”}Adonai, just as Moses had commanded.}%
\verse{Then\lebnote{Or “And”}Aaron lifted his hand toward\lebnote{Or “to”}the people, and he blessed them, and he came down \textit{after}\lebnote{Literally “from”}\textit{sacrificing}\lebnote{Literally “doing” or “making”}the sin offering and the burnt offering and the fellowship offerings.}%
\verse{Then\lebnote{Or “And”}Moses and Aaron entered the tent of assembly. When\lebnote{Or “And”}they came out, they blessed\lebnote{Or “and they blessed”}the people, and Adonai’s glory appeared to all the people.}%
\verse{Then\lebnote{Or “And”}a fire went out \textit{from before}\lebnote{Literally “from to the faces of”}Adonai, and it consumed the burnt offering and the fat portions on the altar. And all the people saw it, so\lebnote{Or “and”}they shouted for joy, and they fell on their faces.}%
\end{biblechapter}

\begin{biblechapter} % Leviticus 10
\verseWithHeading{The Deaths of Nadab and Abihu}{And Aaron’s sons Nadab and Abihu each took his censer, and they put fire in them and placed incense on it;\lebnote{That is, the fire}then\lebnote{Or “and”}they presented \textit{before}\lebnote{Literally “to the faces of”}Adonai illegitimate fire, which he had not commanded them.}%
\verse{So\lebnote{Or “And”}fire went out \textit{from before}\lebnote{Literally “from to the faces of”}Adonai, and it consumed them so that\lebnote{Or “and”}they died \textit{before}\lebnote{Literally “to the faces of”}Adonai.}%
\verse{Therefore\lebnote{Or “And”}Moses said to Aaron, “This is what Adonai spoke, saying, ‘Among those who are close to me I will show myself holy, and \textit{in the presence of}\lebnote{Literally “upon the faces of”}all the people I will display my glory.’ ”\lebnote{Or “I will be glorified”}So\lebnote{Or “And”}Aaron was silent.}%
\verse{Then\lebnote{Or “And”}Moses summoned Mishael and Elzaphan the sons of Uzziel, Aaron’s uncle, and he said to them, “Come forward.\lebnote{Or “Come near” or “Approach”}Carry your brothers from \textit{the front of}\lebnote{Literally “the faces of”}the sanctuary to \textit{outside the camp}.”\lebnote{Literally “from an outside place of the camp”}}%
\verse{So\lebnote{Or “And”}they came forward,\lebnote{Or “came near” or “approached”}and they carried them \textit{outside the camp}\lebnote{Literally “to from an outside place of the camp”}in their tunics, just as Moses had ordered.}%
\verse{Then\lebnote{Or “And”}Moses said to Aaron and to his sons Eleazar and Ithamar, “You must not let your hair hang loosely, and you must not tear your garments, so that\lebnote{Or “and”}you will not die and he\lebnote{That is, God}will be angry with all the community. But\lebnote{Or “And”}your brothers, all the house of Israel, may weep because of \textit{the burning that Adonai caused},\lebnote{Literally “the burning that Adonai burned”}}%
\verse{but\lebnote{Or “and”}you must not go out from the entrance to\lebnote{Or “of”}the tent of assembly lest you die, because Adonai’s anointing oil is on you.” So\lebnote{Or “And”}they did according to Moses’ word.}%
\verseWithHeading{Lasting Statutes}{Then\lebnote{Or “And”}Adonai spoke to Aaron, saying,}%
\verse{“You and your sons with you may not drink wine or\lebnote{Or “and”}strong drink when you come to the tent of assembly, so that\lebnote{Or “and”}you will not die—it is a lasting statute for your\lebnote{Plural}generations—}%
\verse{and to distinguish between the holy and the unholy,\lebnote{Or “the common”}as well as\lebnote{Or “and”}between the unclean and the clean,}%
\verse{and to teach the \textit{Israelites}\lebnote{Literally “sons/children of Israel”}all the rules that Adonai has spoken to them \textit{through}\lebnote{Literally “by the hand of”}Moses.”}%
\verse{Then\lebnote{Or “And”}Moses spoke to Aaron and to his sons Eleazar and Ithamar, “As for the remaining parts,\lebnote{Implied by plural form of noun and the immediate context}take the remainder of the grain offering from Adonai’s offerings made by fire and eat it, the unleavened bread, beside the altar, because it is \textit{a most holy thing}.\lebnote{Literally “a holiness of holinesses”}}%
\verse{And you shall eat it in a holy place, because it is your allotted portion and the allotted portion of your sons from Adonai’s offerings made by fire, for so I have been commanded.}%
\verse{And the wave offering’s breast section and the upper thigh of the contribution offering you\lebnote{Plural}must eat in a clean place, you and your sons and your daughters with you, because they are given as your allotted portion and your sons’ allotted portion from the sacrifices of the \textit{Israelites}’\lebnote{Literally “sons/children of Israel”}fellowship offerings.}%
\verse{They must bring the thigh of the contribution offering and the breast section of the wave offering in addition to the offerings made by fire, consisting of the fat portions, to wave as a wave offering \textit{before}\lebnote{Literally “to the faces of”}Adonai; and it will be for you and for your sons with you as a lasting rule, just as Adonai had commanded.”}%
\verseWithHeading{The Problem of the Uneaten Sin Offering}{Then\lebnote{Or “And”}Moses sought all over for the goat of the sin offering and behold, it was burned up. So\lebnote{Or “And”}he was angry with Aaron’s remaining sons Eleazar and Ithamar, saying,}%
\verse{“Why did you not eat the sin offering on the sanctuary’s site, because it is \textit{a most holy thing}?\lebnote{Literally “a holiness of holinesses”}And he gave it to you to remove the community’s guilt, to make atonement for them \textit{before}\lebnote{Literally “to the faces of”}Adonai.}%
\verse{Look, its blood was not brought \textit{inside the sanctuary}.\lebnote{Or “to the sanctuary inside” or “to the sanctuary within”}Certainly you should have eaten it in the sanctuary, as I commanded.”}%
\verse{So\lebnote{Or “And”}Aaron said to Moses, “Look, today they presented their sin offering and their burnt offering \textit{before}\lebnote{Literally “to the faces of”}Adonai, and things such as these have happened to me, and if I were to eat a sin offering today, would it have been good in Adonai’s eyes?”}%
\verse{When\lebnote{Or “And”}Moses heard, it\lebnote{Or “and it”}was good in his eyes.}%
\end{biblechapter}

\begin{biblechapter} % Leviticus 11
\verseWithHeading{Clean and Unclean Animals}{Then\lebnote{Or “And”}Adonai spoke to Moses and to Aaron, saying to them,}%
\verse{“Speak to the \textit{Israelites},\lebnote{Literally “sons/children of Israel”}saying, ‘These are the animals that you may eat from all the animals that are on the land:}%
\verse{Any among the animals that has a divided hoof and has a split cleft in\lebnote{Hebrew “of”}the hoof, such\lebnote{Hebrew “her/it”}you may eat.}%
\verse{However,\lebnote{Or “Only”}these\lebnote{By context; Hebrew “this”}you may not eat from those that chew the cud and from those that have a\lebnote{Hebrew “the”}divided hoof: the camel, because it is a chewer of cud but it does not have a hoof that is divided—it is unclean for you;}%
\verse{and the coney, because it is a chewer of cud but it does not have a hoof that is divided—it is unclean for you;}%
\verse{and the hare, because it is a chewer of cud but it does not have a hoof that is divided—it is unclean for you;}%
\verse{and the pig, because it has a divided hoof and has a split cleft in\lebnote{Hebrew “of”}the hoof but it does not chew cud—it is unclean for you.}%
\verse{You must not eat from their meat, and you must not touch their dead body—they are unclean for you.}%
\verse{“ ‘These\lebnote{By context; Hebrew “This”}you may eat from all that are in the water: any in the water that has a fin and scales, whether in the seas or\lebnote{Or “and”}in the streams—such\lebnote{Hebrew “them”}you may eat.}%
\verse{But any that does not have a fin and scales, whether in the seas or\lebnote{Or “and”}in the streams, \textit{among}\lebnote{Literally “from”}all the water’s swarmers among all the living creatures that are in the water—they are a detestable thing to you.}%
\verse{And they shall be detestable to you; you must not eat from their meat, and you must detest their dead body.}%
\verse{Any that does not have a fin and scales in the water—it is a detestable thing to you.}%
\verse{“ ‘And these you must detest from the birds; they must not be eaten—they are detestable: the eagle and the vulture and the short-toed eagle,}%
\verse{and the red kite and the black kite according to its kind,}%
\verse{every crow according to its kind,}%
\verse{and \textit{the ostrich}\lebnote{Literally “the daughter of the ostrich”}and the short-eared owl and the seagull and the hawk according to its kind,}%
\verse{and the little owl and the cormorant and the great owl,}%
\verse{and the barn owl and the desert owl and the carrion vulture,}%
\verse{and the stork, the heron according to its kind and the hoopoe and the bat.}%
\verse{“ ‘Any \textit{winged insect}\lebnote{Literally “swarmer of the wing”}that walks on all fours is detestable to you.}%
\verse{Only this may you eat from any of \textit{the winged insects}\lebnote{Literally “the swarmer of the wing”}that walk on all fours—that which has jointed legs above its feet for leaping upon the land.}%
\verse{\textit{From these}\lebnote{Literally “These from them”}you may eat the locust according to its kind and the bald locust according to its kind and the cricket according to its kind and the grasshopper according to its kind.}%
\verse{But\lebnote{Or “And”}any other\lebnote{Implied by context}\textit{winged insect}\lebnote{Literally “swarmer of wing”}that has four legs is detestable to you.}%
\verse{And by these you shall become unclean—anyone who touches their dead body shall become unclean until the evening,}%
\verse{and anyone who carries their dead body must wash his garments, and he shall be unclean until the evening.}%
\verse{“ ‘With regard to any animal that has a divided hoof but does not split the hoof, or\lebnote{Or “and”}does not have a cud for chewing\lebnote{So HALOT 830 s.v. 4}—they are unclean for you; anyone who touches them shall become unclean.}%
\verse{And anything that walks upon its paws among any of the animals\lebnote{Collective singular = plural by context}that walks on all fours—they are unclean for you; anyone who touches their dead body shall become unclean until the evening,}%
\verse{and the one who carries their dead body must wash his garments, and he shall be unclean until the evening—they are unclean for you.}%
\verse{“ ‘And these\lebnote{By context; Hebrew “this”}are the unclean for you among the swarmers\lebnote{Collective singular = plural}that swarm on the land: the weasel and the mouse and the thorn-tailed lizard according to its kind,}%
\verse{and the gecko and the land crocodile and the lizard and the sand lizard and the chameleon.}%
\verse{These are the unclean for you among all the swarmers; anyone who touches them at their death shall become unclean until the evening.}%
\verse{And anything on which \textit{one of them}\lebnote{Literally “from them”}falls at their death shall become unclean: any object of wood or garment or skin or sackcloth—any object that has performed work—must be placed in water, and it shall be unclean until the evening, and then it shall be clean.}%
\verse{And any clay vessel\lebnote{Context indicates a vessel as distinguished from a tool or utensil}into which it falls shall become unclean, and you must break it.}%
\verse{Any of the food that could be eaten on which water from such a vessel comes shall become unclean, and any liquid that could be drunk in any such vessel shall become unclean.}%
\verse{And anything on which \textit{one of their dead bodies}\lebnote{Literally “from their dead body”}falls shall become unclean: an oven or\lebnote{Or “and”}a stove must be broken\lebnote{Or “smashed” (NASB, HCSB, NET, NJPS) or “broken in pieces” (ASV, ESV, NRSV)}—they are unclean and shall be unclean for you.}%
\verse{Surely\lebnote{So HALOT 45}a spring or\lebnote{Or “and”}a cistern collecting water shall be clean, but that which touches their dead body shall become unclean.}%
\verse{And when \textit{one of their dead bodies}\lebnote{Literally “from their dead body”}falls on any \textit{seed for sowing},\lebnote{Literally “seed plant that is to be sown”}it is clean.}%
\verse{But\lebnote{Or “And”}when water is put on the seed and \textit{one of their dead bodies}\lebnote{Literally “from their dead body”}falls on it, it is unclean for you.}%
\verse{“ ‘And when \textit{one of the animals}\lebnote{Literally “from the animal”}dies that is for you to eat, the one who touches its dead body shall become unclean until the evening.}%
\verse{And the one who eats \textit{some of}\lebnote{Literally “from”}its dead body must wash his garments, and he shall be unclean until the evening; and the one who carries its dead body must wash his garments, and he shall be unclean until the evening.}%
\verse{“ ‘And any swarmer that swarms on the land is detestable; it must not be eaten.}%
\verse{You must not eat\lebnote{Hebrew “eat them”}anything that moves upon its belly or\lebnote{Or “and”}that walks on all fours, even any with numerous feet belonging to any swarmer that swarms on the land, because they are detestable.}%
\verse{You must not defile yourselves with any swarmer that swarms, and you must not make yourselves unclean by them and so be made unclean by them,}%
\verse{because I am Adonai your God, and you must keep yourselves sanctified, so that\lebnote{Or “and”}you shall be holy, because I am holy. And you must not make yourselves unclean with any swarmer that moves along on the land,}%
\verse{because I am Adonai, who brought you up from the land of Egypt to be for you as God. Thus\lebnote{Or “And”}you shall be holy, because I am holy.}%
\verse{“ ‘This is the regulation of the animals\lebnote{Collective singulars in this verse are plural by context}and the birds and all living creatures that move along in the water and \textit{concerning}\lebnote{Literally “of”}all the creatures that swarm on the land,}%
\verse{to distinguish between the unclean and the clean and between the animal that is to be eaten and the animal that must not be eaten.’ ”}%
\end{biblechapter}

\begin{biblechapter} % Leviticus 12
\verseWithHeading{Purification After Childbirth}{Then\lebnote{Or “And”}Adonai spoke to Moses, saying,}%
\verse{“Speak to the \textit{Israelites},\lebnote{Literally “sons/children of Israel”}saying, ‘When a woman becomes pregnant and she gives birth to a male,\lebnote{Or “son”}then\lebnote{Or “and”}she shall be unclean seven days—as in the time of her menstrual bleeding, she shall become unclean.}%
\verse{And on the eighth day his foreskin’s flesh shall be circumcised.}%
\verse{And for thirty-three days she shall stay in the blood of her cleansing; she must not touch any holy object, and she may not come to the sanctuary until the fulfilling of the days of her cleansing.}%
\verse{But\lebnote{Or “And”}if she gives birth to a female,\lebnote{Or “daughter”}then\lebnote{Or “and”}she shall be unclean for two weeks as in her menstruation, and for sixty-six days she shall stay \textit{through}\lebnote{Or “at” or “on”}the blood of her cleansing.}%
\verse{And at the fulfilling of the days of her cleansing, whether for a son or for a daughter, she must bring to the priest at the tent of assembly’s entrance a \textit{yearling}\lebnote{Literally “a son of his year”}male lamb as a burnt offering and \textit{young dove}\lebnote{Literally “son of a dove”}or a turtledove as a sin offering.}%
\verse{And the priest shall present it \textit{before}\lebnote{Literally “to the faces of”}Adonai, and he shall make atonement for her, so that\lebnote{Or “and”}she shall be clean from the flow of her blood. This is the regulation of childbearing\lebnote{Literally “the childbearing”}for the male\lebnote{Or “son”}or for the female.\lebnote{Or “daughter”}}%
\verse{And if \textit{she cannot afford}\lebnote{Literally “her hand does not find enough”}a sheep,\lebnote{Or “small livestock”}then\lebnote{Or “and”}she shall take two turtledoves or two \textit{young doves}\lebnote{Literally “sons of a dove”}—one as a burnt offering and one as a sin offering—and the priest shall make atonement for her, so that\lebnote{Or “and”}she shall be clean.’ ”}%
\end{biblechapter}

\begin{biblechapter} % Leviticus 13
\verseWithHeading{Regulations About Defiling Skin Diseases}{Then\lebnote{Or “And”}Adonai spoke to Moses and to Aaron, saying,}%
\verse{“When a \textit{person}\lebnote{Literally “man”}has on his body’s skin a swelling or an epidermal eruption or a spot and it becomes\lebnote{Perfect of הָיָה followed by לְ; see HALOT 244 s.v. 7.c}an infectious skin disease on his body’s skin, then\lebnote{Or “and”}he shall be brought to Aaron the priest or to one of his sons the priests.}%
\verse{And the priest shall examine the infection on his body’s skin, and if the hair in the infection turns white and the appearance of the infection is deeper than his body’s skin, it is an infectious skin disease, and the priest shall examine it, and he shall declare him unclean.}%
\verse{But\lebnote{Or “And”}if a spot is white on his body’s skin and its appearance is not deeper than the skin and its hair does not turn white, then\lebnote{Or “and”}the priest shall confine the afflicted person for seven days.}%
\verse{And the priest shall examine it on the seventh day, and \textit{if},\lebnote{Literally “look” or “behold”}in his eyes, the infection has stayed unchanged, the infection has not spread on the skin, then\lebnote{Or “and”}the priest shall confine him for seven days a second time.}%
\verse{And the priest shall examine him on the seventh day for a second time, and \textit{if}\lebnote{Literally “look” or “behold”}the infection has faded and the infection has not spread on the skin, then\lebnote{Or “and”}the priest shall declare him clean—it is an epidermal eruption; and he shall wash his garments, and so he shall be clean.}%
\verse{But\lebnote{Or “And”}if the epidermal eruption spreads further on the skin after showing himself to the priest for his cleansing, then\lebnote{Or “and”}he shall appear a second time to the priest.}%
\verse{And the priest shall examine it,\lebnote{The direct object is supplied from context in the English translation}and \textit{if}\lebnote{Literally “look” or “behold”}the epidermal eruption has spread on the skin, then\lebnote{Or “and”}the priest shall declare him unclean—it is an infectious skin disease.}%
\verse{“When an infectious skin disease is on a person and he is brought to the priest,}%
\verse{the priest shall examine it,\lebnote{The direct object is supplied from context in the English translation}and \textit{if}\lebnote{Literally “look” or “behold”}a white swelling is on the skin and it turns the hair white and \textit{raw flesh}\lebnote{Literally “living of living flesh”}is in the swelling,}%
\verse{it is a chronic infectious skin disease on his body’s skin, and the priest shall declare him unclean; he shall not confine him, because he is unclean.}%
\verse{And if the infectious skin disease breaks out all over on the skin and the infectious skin disease covers all of the afflicted person’s skin from his head to his feed, \textit{so far as the priest can see},\lebnote{Literally “for all the sight of the eyes of the priest”}}%
\verse{then\lebnote{Or “and”}the priest shall examine it,\lebnote{The direct object is supplied from context in the English translation}and \textit{if}\lebnote{Literally “look” or “behold”}the infectious skin disease covers his whole body, then\lebnote{Or “and”}he shall pronounce the afflicted person clean—all of it has turned white; he is clean.}%
\verse{But\lebnote{Or “And”}\textit{whenever}\lebnote{Literally “on a day”}\textit{raw flesh}\lebnote{Literally “living flesh”}appears on him, he shall become unclean.}%
\verse{And the priest shall examine the \textit{raw flesh},\lebnote{Literally “living flesh”}and he shall pronounce him unclean—the \textit{raw flesh}\lebnote{Literally “living flesh”}is unclean; it is an infectious skin disease.}%
\verse{Or, when the \textit{raw flesh}\lebnote{Literally “living flesh”}returns and it has changed to white, then\lebnote{Or “and”}he shall come to the priest,}%
\verse{and the priest shall examine him, and \textit{if}\lebnote{Literally “look” or “behold”}the infection has changed to white, then\lebnote{Or “and”}the priest shall pronounce the afflicted person clean—he is clean.}%
\verse{“And when someone’s body \textit{has}\lebnote{Literally “becomes on it”}a skin sore on his skin and it is healed}%
\verse{and a white swelling or a \textit{pinkish}\lebnote{Literally “white red”}spot appears in the skin sore’s place, then\lebnote{Or “and”}he shall show himself to the priest.}%
\verse{And the priest shall examine it,\lebnote{The direct object is supplied from context in the English translation}and \textit{if}\lebnote{Literally “look” or “behold”}its appearance is deeper than the skin and its hair has changed to white, then\lebnote{Or “and”}the priest shall declare him unclean—it is an infectious skin disease; it has broken out in the skin sore.}%
\verse{And if the priest examines it and \textit{if}\lebnote{Literally “look” or “behold”}there is no white hair in it and it is not deeper than the skin and it is faded, then\lebnote{Or “and”}the priest shall confine him for seven days.}%
\verse{But\lebnote{Or “And”}if it has spread further on the skin, then\lebnote{Or “and”}the priest shall declare him unclean—it is an infection.}%
\verse{But\lebnote{Or “And”}if the spot has stayed unchanged, it has not spread, it is the skin sore’s scar, so\lebnote{Or “and”}the priest shall declare him clean.}%
\verse{“Or when a body \textit{has}\lebnote{Literally “it becomes”}a burn-spot \textit{from}\lebnote{Literally “of”}fire on its skin and the \textit{raw flesh}\lebnote{Literally “living”}of the burn-spot is \textit{pinkish}\lebnote{Literally “white red”}or white,}%
\verse{then\lebnote{Or “and”}the priest shall examine it, and \textit{if}\lebnote{Literally “look” or “behold”}the hair turns white in the spot and its appearance is deeper than the skin, it is an infectious skin disease—it has broken out in the burn-spot; so\lebnote{Or “and”}the priest shall declare him unclean—it is an infectious skin disease.}%
\verse{But\lebnote{Or “And”}if the priest examines it and \textit{if}\lebnote{Literally “look” or “behold”}there is not white hair in the spot and it is not deeper than the skin and it is faded, then\lebnote{Or “and”}the priest shall confine him for seven days.}%
\verse{And the priest shall examine him on the seventh day; if it has spread further on the skin, then\lebnote{Or “and”}the priest shall declare him unclean—it is an infectious skin disease.}%
\verse{But\lebnote{Or “And”}if it the spot has stayed unchanged in its place, it has not spread on the skin and it is faded, then it is the burn-spot’s swelling, so\lebnote{Or “and”}the priest shall declare him clean, because it is the burn-spot’s scar.}%
\verse{“And when a man or a woman \textit{has}\lebnote{Literally “becomes on him”}an infection on the head or in the beard,}%
\verse{then\lebnote{Or “and”}the priest shall examine the infection, and \textit{if}\lebnote{Literally “look” or “behold”}its appearance is deeper than the skin and in it is thin bright red hair, then\lebnote{Or “and”}the priest shall declare it unclean—it is a diseased area of skin; it is an infectious skin disease of the head or the beard.}%
\verse{But\lebnote{Or “And”}if the priest examines the diseased area of the skin’s infection and \textit{if}\lebnote{Literally “look” or “behold”}its appearance is not deeper than the skin and there is no black hair in it, then\lebnote{Or “and”}the priest shall confine the afflicted person with the diseased area of skin for seven days.}%
\verse{And the priest shall examine the infection on the seventh day, and \textit{if}\lebnote{Literally “look” or “behold”}the diseased area of skin has not spread and it does not have bright red hair in it and the diseased area of the skin’s appearance is not deeper than the skin,}%
\verse{then\lebnote{Or “and”}he shall shave himself, but\lebnote{Or “and”}he shall not shave the diseased area of skin, and the priest shall confine the person with\lebnote{The direct object is supplied from context in the English translation}the diseased area of skin a second time for seven days.}%
\verse{And the priest shall examine the diseased area of skin on the seventh day, and \textit{if}\lebnote{Literally “look” or “behold”}the diseased area has not spread on the skin and its appearance is not deeper than the skin, then\lebnote{Or “and”}the priest shall pronounce him clean, and he shall wash his garments, and he shall be clean.}%
\verse{But\lebnote{Or “And”}if the diseased area of skin has not spread further on the skin after his cleansing,}%
\verse{then\lebnote{Or “and”}the priest shall examine him, and \textit{if}\lebnote{Literally “look” or “behold”}the diseased area of skin has spread on the skin, the priest shall not inspect for bright\lebnote{Hebrew “the bright”}red hair—he is unclean.}%
\verse{But\lebnote{Or “And”}if, in his eyes, the diseased area of skin has stayed unchanged and black hair has grown in it, the diseased area of skin is healed—he is clean, and the priest shall pronounce him clean.}%
\verse{“And when a man or a woman \textit{has}\lebnote{Literally “becomes”}spots on their body’s skin, white spots,}%
\verse{then\lebnote{Or “and”}the priest shall examine them,\lebnote{The direct object is supplied from context in the English translation}and \textit{if}\lebnote{Literally “look” or “behold”}the spots on their body’s skin are a faded white, it is a skin rash; it has broken out on the skin—it is clean.}%
\verse{“And if a man becomes bald, his head is bald, he is clean.}%
\verse{And if he becomes bald \textit{from his forehead},\lebnote{Literally “from the forehead of his face”}his head is bald, he is clean.}%
\verse{But\lebnote{Or “And”}if a \textit{pinkish}\lebnote{Literally “white red”}infection occurs\lebnote{Or “becomes” or “happens”}on the bald spot or on the bald forehead, it is an infectious skin disease that sprouts on his bald spot or on his bald forehead.}%
\verse{So\lebnote{Or “And”}the priest shall examine him, and \textit{if}\lebnote{Literally “look” or “behold”}the infection’s swelling is \textit{pinkish}\lebnote{Literally “white red”}on his bald spot or on his bald forehead, like the appearance of an infectious skin disease on\lebnote{Literally “of”}the body,}%
\verse{he is a man afflicted with a skin disease—he is unclean; the priest certainly shall declare him unclean—his infection is on his head.}%
\verse{“As for\lebnote{Or “And”}the person who is afflicted with a skin disease, his garments must be torn and his \textit{hair}\lebnote{Literally “head”}must be allowed to hang loosely, and he must cover his upper lip, and he must call out, ‘Unclean! Unclean!’}%
\verse{For all the days during which the infection is on him, he shall be unclean; he must live alone; his dwelling must be \textit{outside the camp}.”\lebnote{Literally “from an outside place of the camp”}}%
\verseWithHeading{Regulations About Contaminated Fabrics}{“And when the garment \textit{has}\lebnote{Literally “becomes”}an infectious skin disease\lebnote{Perhaps better translated “mold” rather than “skin disease”}on it, on a wool garment\lebnote{Literally “a garment of wool”}or on a linen garment,\lebnote{Literally “a garment of linen”}}%
\verse{or on woven material or on a linen fabric, or\lebnote{Or “and”}on wool or on leather or on any work of leather,}%
\verse{and if the infection is yellowish green or reddish on the garment or on the leather or on the woven material or on the fabric or on any leather object,\lebnote{Literally “an object of leather”}it is an infectious skin disease\lebnote{Perhaps better translated “mold” rather than “skin disease”}and it shall be shown to the priest.}%
\verse{And the priest shall examine the infection, and he shall confine the infected article for seven days.}%
\verse{And he shall examine the infection on the seventh day; if the infection has spread on the garment or on the woven material or on the fabric or on the leather, for any work for which the leather is used, the infection is a destructive skin disease\lebnote{Perhaps better translated “mold” rather than “skin disease”}—it is unclean.}%
\verse{And he shall burn the garment or the woven material or the fabric, \textit{whether wool or linen},\lebnote{Literally “in/on the wool or in/on the linen”}or any leather object that \textit{has}\lebnote{Literally “becomes on/in it”}the infection, because it is an infectious skin disease,\lebnote{Perhaps better translated “mold” rather than “skin disease”}which is destructive—it must be burned in the fire.}%
\verse{“But\lebnote{Or “And”}if the priest examines it\lebnote{The direct object is supplied from context in the English translation}and \textit{if}\lebnote{Literally “look” or “behold”}the infection has not spread on the garment or on the woven material or on the fabric or on any leather object,}%
\verse{then\lebnote{Or “and”}the priest shall command, and \textit{someone}\lebnote{Literally “he”}shall wash that on which the infection is, and he shall confine it a second time for seven days.}%
\verse{And the priest shall examine it\lebnote{The direct object is supplied from context in the English translation}after the infection has been washed off, and \textit{if}\lebnote{Literally “look” or “behold”}the infection has not changed its outward appearance and the infection has not spread, it is unclean; he must burn it in the fire; it is a fungus on its back or on its front.}%
\verse{But\lebnote{Or “And”}if the priest examines it\lebnote{The direct object is supplied from context in the English translation}and \textit{if}\lebnote{Literally “look” or “behold”}the infection is faded after it has been washed off, then\lebnote{Or “and”}he shall tear it from the garment or from the leather or from the woven material or from the fabric.}%
\verse{And if it appears again on the garment or on the woven material or on the fabric or on any leather object, it is spreading; you\lebnote{Singular}must burn in the fire that which has the infection in it.}%
\verse{And the garment or the woven material or the fabric or any leather object that he might wash and the infection is removed from them then\lebnote{Or “and”}shall be washed a second time, and it shall be clean.”}%
\verse{This is the regulation of the infectious skin disease\lebnote{Perhaps better translated “mold” rather than “skin disease”}in\lebnote{Hebrew “of”}the wool garment or the linen or the woven material or the fabric or any leather object to declare it clean or to declare it unclean.}%
\end{biblechapter}

\begin{biblechapter} % Leviticus 14
\verseWithHeading{Instructions for Cleansing Infectious Skin Diseases}{Then\lebnote{Or “And”}Adonai spoke to Moses, saying,}%
\verse{“This is the regulation of the person afflicted with a skin disease \textit{at the time of}\lebnote{Or “on the day of”}his cleansing. And he shall be brought to the priest,}%
\verse{and the priest shall go \textit{outside the camp},\lebnote{Literally “to from an outside place of the camp”}and the priest shall examine him,\lebnote{The direct object is supplied from context in the English translation}and \textit{if}\lebnote{Literally “look” or “behold”}the skin disease’s infection is healed on\lebnote{Hebrew “from”}the afflicted person,}%
\verse{then\lebnote{Or “and”}the priest shall command, and he shall take two living, clean birds and \textit{cedar wood}\lebnote{Literally “wood of cedar”}and \textit{a crimson thread}\lebnote{Literally “crimson thread of the worm”}and hyssop for the one who presents himself for cleansing.}%
\verse{Then\lebnote{Or “And”}the priest shall command someone\lebnote{The direct object is supplied from context in the English translation}to slaughter one bird over fresh water in a clay vessel.}%
\verse{He must take the living bird and the \textit{cedar wood}\lebnote{Literally “wood of cedar”}and \textit{the crimson thread}\lebnote{Literally “the crimson thread of the worm”}and the hyssop, and he shall dip them and the living bird in the bird’s blood slaughtered over the fresh water.}%
\verse{And he shall spatter the blood\lebnote{The direct object is supplied from context in the English translation}seven times on the one who presents himself for cleansing from the infectious skin disease, and he shall declare him clean, and he shall send the living bird \textit{into the open field}.\lebnote{Literally “toward the faces of the field”}}%
\verse{Then\lebnote{Or “And”}the one who presents himself for cleansing shall wash his garments, and he shall shave off all his hair, and he shall wash himself in the water; thus\lebnote{Or “and”}he shall be clean, and afterward he shall enter the camp, but\lebnote{Or “and”}he shall stay \textit{outside his tent}\lebnote{Literally “from an outside place of his tent”}for seven days.}%
\verse{\textit{And then}\lebnote{Literally “And it shall be”}on the seventh day he must shave off all his hair—he must shave his head and his beard and \textit{his eyebrows}\lebnote{Literally “the rims of his eyes”}and all the rest\lebnote{Implied by context}of his hair—and he shall wash his garments, and he shall wash his body in the water; thus\lebnote{Or “and”}he shall be clean.}%
\verse{“And on the eighth day he must take two male lambs without defect and one ewe-lamb \textit{in its first year}\lebnote{Literally “a daughter of her year”}without defect and three-tenths of an ephah\lebnote{Implied by context}of finely milled flour mixed with oil as a grain offering and one log of oil.}%
\verse{And the priest who cleanses him\lebnote{The direct object is supplied from context in the English translation}shall present the man who presents himself for cleansing and \textit{these things}\lebnote{Literally “with them”}\textit{before}\lebnote{Literally “to the faces of”}the tent of assembly’s entrance.}%
\verse{Then\lebnote{Or “And”}the priest shall take the one male lamb, and he shall present it as a guilt offering, and the log of oil, and he shall wave them as a wave offering \textit{before}\lebnote{Literally “to the faces of”}Adonai.}%
\verse{And he shall slaughter the male lamb in the place where he slaughters the sin offering and the burnt offering in the sanctuary’s space,\lebnote{Or “place” or “area”}because as the sin offering belongs to the priest, so also the guilt offering—\textit{it is a most holy thing}.\lebnote{Literally “a holiness of holinesses is it”}}%
\verse{And the priest shall take \textit{some of}\lebnote{Literally “from”}the guilt offering’s blood, and the priest shall put it\lebnote{The direct object is supplied from context in the English translation}on the right ear’s lobe of the one who presents himself for cleansing and on his right hand’s thumb and on his right foot’s big toe.}%
\verse{And the priest shall take \textit{some of}\lebnote{Literally “from”}the log of oil, and he shall pour it\lebnote{The direct object is supplied from context in the English translation}on his\lebnote{Hebrew “the priest’s”}left palm;}%
\verse{and the priest shall dip his right finger in the oil that is on his left palm, and he shall spatter \textit{some of}\lebnote{Literally “from”}the oil with his finger seven times \textit{before}\lebnote{Literally “to the faces of”}Adonai.}%
\verse{Then\lebnote{Or “And”}the priest shall put \textit{some of}\lebnote{Literally “from”}the remaining oil, which is on his palm, on the right ear’s lobe of the one to be cleansed and on his right hand’s thumb and on his right foot’s big toe, on top of the guilt offering’s blood.\lebnote{See v. 14}}%
\verse{And the remaining oil that is on the priest’s palm he shall put on the head of the one who presents himself for cleansing, and the priest shall make atonement for him \textit{before}\lebnote{Literally “to the faces of”}Adonai.}%
\verse{Thus\lebnote{Or “And”}the priest shall \textit{sacrifice}\lebnote{Literally “do” or “make”}the sin offering, and he shall make atonement for the one who presents himself for cleansing from his uncleanness, and afterward he shall slaughter the burnt offering.}%
\verse{Then\lebnote{Or “And”}the priest shall offer the burnt offering and the grain offering on the altar, and the priest shall make atonement for him, and so he shall be clean.}%
\verse{“But\lebnote{Or “And”}if he is poor and \textit{he cannot afford}\lebnote{Literally “his hand is not producing”}it,\lebnote{The direct object is supplied from context in the English translation}then\lebnote{Or “and”}he shall take one male lamb for a guilt offering as a wave offering to make atonement for himself and one-tenth of an ephah\lebnote{Implied by context}of finely milled flour mixed with oil for a grain offering, and a log of oil,}%
\verse{and two turtledoves or two \textit{young doves}\lebnote{Literally “sons of dove”}that \textit{he can afford},\lebnote{Literally “his hand can produce”}and one shall be a sin offering and the \textit{other}\lebnote{Literally “one”}a burnt offering.}%
\verse{And he shall bring them to the priest at the tent of assembly’s entrance \textit{before}\lebnote{Literally “to the faces of”}Adonai on the eighth day for his cleansing.}%
\verse{And the priest shall take the male lamb for the guilt offering and the log of oil, and the priest shall wave them as a wave offering \textit{before}\lebnote{Literally “to the faces of”}Adonai;}%
\verse{and he shall slaughter the male lamb of the guilt offering, and the priest shall take \textit{some of}\lebnote{Literally “from”}the guilt offering’s blood, and he shall put it\lebnote{The direct object is supplied from context in the English translation}on the right ear’s lobe of the one who presents himself for cleansing and on his right hand’s thumb and on his right foot’s big toe.}%
\verse{Then\lebnote{Or “And”}the priest shall pour out \textit{some of}\lebnote{Literally “from”}the oil on his own\lebnote{Hebrew “the priest’s”}left palm,}%
\verse{and with his right finger the priest shall spatter \textit{some of}\lebnote{Literally “from”}the oil that is on his left palm seven times \textit{before}\lebnote{Literally “to the faces of”}Adonai.}%
\verse{Then\lebnote{Or “And”}the priest shall put \textit{some of}\lebnote{Literally “from”}the oil that is on his palm on the right ear’s lobe of the one who presents himself for cleansing and on his right hand’s thumb and on his right foot’s big toe on the place of the guilt offering’s blood.}%
\verse{And the remaining oil that is on the priest’s palm he shall put on the head of the one who presents himself for cleansing to make atonement for him \textit{before}\lebnote{Literally “to the faces of”}Adonai.}%
\verse{And he shall \textit{sacrifice}\lebnote{Literally “do” or “make”}one of the turtledoves or the \textit{young doves}\lebnote{Literally “sons of dove”}that \textit{he can afford},\lebnote{Literally “his hand can produce”}}%
\verse{even\lebnote{Implied by context}what \textit{he can afford},\lebnote{Literally “his hand can produce”}the one as a sin offering and the \textit{other}\lebnote{Literally “one”}as a burnt offering in addition to the grain offering, and the priest shall make atonement for the one who presents himself for cleansing \textit{before}\lebnote{Literally “to the faces of”}Adonai.}%
\verse{This is the regulation of the one on whom is an infectious skin disease who \textit{cannot afford}\lebnote{Literally “his hand cannot produce”}the cost\lebnote{The direct object is supplied from context in the English translation}for his cleansing.”}%
\verseWithHeading{Instructions for Cleansing Houses}{Then\lebnote{Or “And”}Adonai spoke to Moses and to Aaron, saying,}%
\verse{“When you come into the land of Canaan, which I am about to give to you as your possession, and I put \textit{mildew}\lebnote{Literally “an infection of skin disease”}in a house in the land of your possession,}%
\verse{then\lebnote{Or “and”}the one who \textit{owns the house}\lebnote{Literally “the house is for him”}shall come and tell the priest, saying, ‘It appears to me that an infection is in the house.’}%
\verse{And the priest shall issue a command, and they shall clear out the house before the priest comes to examine the infection, so that\lebnote{Or “and”}all that is in the house might not become unclean; and \textit{afterward}\lebnote{Literally “after thus”}the priest shall go to examine the house.}%
\verse{And he shall examine the infection, and \textit{if}\lebnote{Literally “look” or “behold”}the infection on the house’s wall has yellowish-green or reddish spots and its appearance is deeper than the surface of\lebnote{Implied by the context}the wall,}%
\verse{then\lebnote{Or “and”}the priest shall go out from the house to the house’s entrance, and he shall confine\lebnote{Or “close up”}the house for seven days.}%
\verse{And the priest shall return on the seventh day, and he shall examine the infection,\lebnote{The direct object is supplied from context in the English translation}and \textit{if}\lebnote{Literally “look” or “behold”}the infection has spread on the house’s wall,}%
\verse{the priest shall issue a command and they shall remove the stones on which is the infection, and they shall throw them \textit{outside the city}\lebnote{Literally “to from an outside place of the city”}on an unclean place.}%
\verse{Then\lebnote{Or “And”}they\lebnote{Hebrew “he”}shall scrape off the plaster\lebnote{The direct object is supplied from context in the English translation}from all around the house, and they shall pour out the plaster,\lebnote{See HALOT 862 s.v. 3.b}which they scraped off, \textit{outside the city}\lebnote{Literally “to from an outside place of the city”}on an unclean place.}%
\verse{And they shall take other stones, and they shall \textit{put}\lebnote{Literally “bring”}them in place of those stones, and they\lebnote{Hebrew “he”}shall take other plaster,\lebnote{See HALOT 862 s.v. 3.b}and they\lebnote{Hebrew “he”}shall replaster\lebnote{Implied by the context; Hebrew “plaster”}the house.}%
\verse{“But\lebnote{Or “And”}if the infection should return and it breaks out in the house after they\lebnote{Hebrew “he”}have removed the stones and after scraping off the plaster\lebnote{The direct object is supplied from context in the English translation}of the house and after it has been replastered,\lebnote{Implied by the context; or “plaster”}}%
\verse{then\lebnote{Or “and”}the priest shall come, and he shall examine the infection,\lebnote{The direct object is supplied from context in the English translation}and \textit{if}\lebnote{Literally “look” or “behold”}the infection has spread in the house, it is a destructive \textit{mildew}\lebnote{Literally “infectious disease”}in the house—it is unclean.}%
\verse{So\lebnote{Or “And”}he shall break down the house, its stones and its wood and all of the house’s plaster,\lebnote{See HALOT 862 s.v. 3.b}and he shall bring it all\lebnote{The direct object is supplied from context in the English translation}\textit{outside the city}\lebnote{Literally “to from an outside place of the city”}to an unclean place.}%
\verse{And the person who enters into the house during all the days that he\lebnote{That is, the priest}confined\lebnote{Or “closed up”}it shall become unclean until the evening.}%
\verse{And the person who sleeps in the house must wash his garments, and the person who eats in the house shall wash his garments.}%
\verse{“And if the priest comes again and examines the house\lebnote{The direct object is supplied from context in the English translation}and \textit{if}\lebnote{Literally “look” or “behold”}the infection has not spread in the house after being replastered,\lebnote{Implied by the context; or “plaster”}then\lebnote{Or “and”}the priest shall pronounce the house clean, because the infection is healed.}%
\verse{And he shall take two birds and \textit{cedar wood}\lebnote{Literally “wood of cedar”}and a \textit{crimson thread}\lebnote{Literally “crimson thread of the worm”}and hyssop to cleanse the house;}%
\verse{and he shall slaughter the first bird over fresh water on a clay vessel.}%
\verse{Then\lebnote{Or “And”}he shall take the \textit{cedar wood}\lebnote{Literally “wood of cedar”}and the hyssop and \textit{the crimson thread}\lebnote{Literally “the crimson thread of the worm”}and the living bird, and he shall dip them in the slaughtered bird’s blood and in the fresh water, and he shall spatter them\lebnote{The direct object is supplied from context in the English translation}on the house seven times.}%
\verse{Thus\lebnote{Or “And”}he shall purify the house with the bird’s blood and with the fresh water and with the living bird and with the \textit{cedar wood}\lebnote{Literally “wood of cedar”}and with the hyssop and with \textit{the crimson thread}.\lebnote{Literally “the crimson thread of the worm”}}%
\verse{And he shall send the living bird \textit{outside the city}\lebnote{Literally “to from an outside place of the city”}\textit{into the open field};\lebnote{Literally “to the faces of the field”}and so he shall make atonement for the house, and it shall be clean.}%
\verse{“This is the instruction for any infectious skin disease and for the diseased area of skin,}%
\verse{and for a \textit{mildew}\lebnote{Literally “infectious skin disease”}of the garment and for the house,}%
\verse{and for the swelling and for the epidermal eruption and for the spot,}%
\verse{to teach \textit{when something is unclean and when something is clean}.\lebnote{Literally “on the day of the unclean and on the day of the clean”}This is the regulation of the infectious skin disease.”}%
\end{biblechapter}

\begin{biblechapter} % Leviticus 15
\verseWithHeading{Instructions About Bodily Discharges}{Then\lebnote{Or “And”}Adonai spoke to Moses and to Aaron, saying,}%
\verse{“Speak to the \textit{Israelites},\lebnote{Literally “sons/children of Israel”}and you shall say to them, ‘\textit{Any man}\lebnote{Literally “a man a man”}when a fluid discharge occurs\lebnote{Or “becomes” or “shall be”}from his body, his fluid discharge is unclean.}%
\verse{And this becomes his uncleanness in his fluid discharge: whether his body secretes his fluid discharge or his body blocks his fluid discharge, it is his uncleanness.}%
\verse{Any bed upon which the person who discharges lies down becomes unclean, and any object upon which he sits becomes unclean.}%
\verse{And anyone who touches his bed must wash his garments and shall wash\lebnote{Or “bathe”}himself\lebnote{The direct object is supplied from context in the English translation}with water, and he shall be unclean until evening.}%
\verse{And the person who sits on the object upon which the person who discharges has sat must wash his garments, and he shall wash\lebnote{Or “bathe”}himself\lebnote{The direct object is supplied from context in the English translation}with water, and he shall be unclean until the evening.}%
\verse{And the person who touches the body of the person who discharges must wash his garments, and he shall wash\lebnote{Or “bathe”}himself\lebnote{The direct object is supplied from context in the English translation}with water, and he shall be unclean until the evening.}%
\verse{And if the person who discharges spits on one who is\lebnote{The generic article designating a class of persons or things}clean, then\lebnote{Or “and”}that one\lebnote{Hebrew “he”}shall wash is garments, and he shall wash himself\lebnote{The direct object is supplied from context in the English translation}with water, and he shall be unclean until the evening.}%
\verse{And any saddle upon which the person who discharges rides becomes unclean.}%
\verse{And any person who touches anything that happened to be under him becomes unclean until the evening, and the person who carries them must wash his garments, and he shall wash himself\lebnote{The direct object is supplied from context in the English translation}with water, and he shall be unclean until the evening.}%
\verse{And anyone whom the person who discharges might touch \textit{without}\lebnote{Literally “and not”}rinsing off his hands with water shall wash his garments, and he shall wash himself\lebnote{The direct object is supplied from context in the English translation}with water, and he shall be unclean until the evening.}%
\verse{But\lebnote{Or “And”}a clay vessel that the person who discharges touches must be broken, and any \textit{wood object}\lebnote{Literally “object of wood”}must be rinsed with water.}%
\verse{“ ‘And when the person who discharges becomes clean from his body fluid discharge, he shall count\lebnote{Literally “and he shall count”}for himself seven days for his cleansing; then\lebnote{Or “and”}he shall wash his garments, and he shall wash his body with \textit{fresh}\lebnote{Or “running”; literally “living”}water, and he shall be clean.}%
\verse{Then\lebnote{Or “And”}on the eighth day he shall take for himself two turtledoves or two \textit{young doves},\lebnote{Literally “sons of dove”}and he shall come \textit{before}\lebnote{Literally “to the faces of”}Adonai at the tent of assembly’s entrance, and he shall give them to the priest.}%
\verse{And the priest shall \textit{sacrifice}\lebnote{Literally “do” or “make”}one as a sin offering and \textit{the other}\lebnote{Literally “the one”}as a burnt offering, and so the priest shall make atonement for him \textit{before}\lebnote{Literally “to the faces of”}Adonai from his body fluid discharge.}%
\verse{“ ‘And if an emission of semen goes out from anyone, then\lebnote{Or “and”}he shall wash all of his body with water, and he shall be unclean until the evening.}%
\verse{And any garment and anything leather on which is an emission of semen shall be washed with water, and it shall be unclean until the evening.}%
\verse{If\lebnote{Or “And”}there is a woman with whom a man lies down and there is an emission of semen, then\lebnote{Or “and”}they shall wash themselves\lebnote{The direct object is supplied from context in the English translation}with water, and they shall be unclean until the evening.}%
\verse{“ ‘And when a woman \textit{is menstruating},\lebnote{Literally “is discharging blood”}her body fluid discharge occurs in\lebnote{Or “from”}her body; for seven days she shall be in her menstruation, and any person who touches her shall become unclean until the evening.}%
\verse{And anything upon which she lies down during her menstruation shall become unclean, and anything upon which she sits shall become unclean.}%
\verse{And any person who touches her bed must wash his garments, and he shall wash himself\lebnote{The direct object is supplied from context in the English translation}with water, and he shall be unclean until the evening.}%
\verse{And any person who touches any object on which she sat must wash his garments, and he shall wash himself\lebnote{The direct object is supplied from context in the English translation}with water, and he shall be unclean until the evening.}%
\verse{And if it is on the bed or on the object on which she sits, at his touching it he becomes unclean until the evening.}%
\verse{And if a man indeed lies with her and her menstruation occurs on him, then\lebnote{Or “and”}he shall be unclean for seven days, and any bed on which he lies down becomes unclean.}%
\verse{“ ‘And when a woman discharges a body fluid consisting of her blood for many days, but not at the time of her menstruation, or when she discharges in addition to\lebnote{Or “beyond”}her menstruation, all the days of her unclean body fluid discharge she shall become unclean as in the days of her menstruation.}%
\verse{Any bed on which she lies during all the days of her body fluid discharge shall become for her as her bed of menstruation, and any object on which she sits becomes unclean as her menstruation’s uncleanness.}%
\verse{And any person who touches them becomes unclean, and he shall wash his garments, and he shall wash himself\lebnote{The direct object is supplied from context in the English translation}with water, and he shall be unclean until the evening.}%
\verse{“ ‘And if she is clean from her body fluid discharge, then\lebnote{Or “and”}she shall count for herself seven days, and afterward she becomes clean.}%
\verse{And on the eighth day she shall take for herself two turtledoves or two \textit{young doves},\lebnote{Literally “sons of dove”}and she shall bring them to the priest at the tent of assembly’s entrance.}%
\verse{And the priest shall \textit{sacrifice}\lebnote{Literally “do” or “make”}the one as a sin offering and the \textit{other}\lebnote{Literally “one”}as a burnt offering, and so the priest shall make atonement for her \textit{before}\lebnote{Literally “the the faces of”}Adonai from her unclean body fluid discharge.’}%
\verse{“And you shall keep the \textit{Israelites}\lebnote{Literally “sons/children of Israel”}separate from their uncleanness so that they might not die because of\lebnote{Or “by” or “in”}their uncleanness by their making my tabernacle, which is in their midst, unclean.}%
\verse{“This is the regulation of the one with the body fluid discharge and the one from whom an emission of semen goes out so that he becomes unclean by it}%
\verse{and \textit{concerning}\lebnote{Literally “of”}the menstruating woman in her bleeding\lebnote{See similar idiom at 12:2}and the person who discharges his body fluid, for the male and for the female and for a man who lies with an unclean woman.”}%
\end{biblechapter}

\begin{biblechapter} % Leviticus 16
\verseWithHeading{The Day of Atonement}{Then\lebnote{Or “And”}Adonai spoke to Moses after the death of Aaron’s two sons, when they had come near \textit{before}\lebnote{Literally “before faces of”}Adonai and they died.}%
\verse{And Adonai said to Moses, “Tell\lebnote{Or “Speak to”}your brother Aaron that he should not enter at any time into the sanctuary \textit{behind}\lebnote{Literally “from the interior/inside to”}the curtain \textit{in front of}\lebnote{Literally “to the faces of”}the atonement cover that is on the ark, so that he might not die, because I appear in the cloud over the atonement cover.}%
\verse{“Aaron must enter the sanctuary with this: \textit{a young bull}\lebnote{Literally “a bull a son of cattle”}as a sin offering and a ram as a burnt offering.}%
\verse{He must put on \textit{a holy linen tunic},\lebnote{Literally “a tunic of linen of holiness”}and linen undergarments must be on his body, and he must fasten himself\lebnote{The direct object is supplied from context in the English translation}with a linen sash, and he must wrap a linen turban around his head\lebnote{Supplied from context in the English translation}—they are holy garments, and he shall wash his body with water, then\lebnote{Or “and”}he shall put them on.}%
\verse{And he must take from the \textit{Israelites}’\lebnote{Literally “sons/children of Israel”}community two he-goats as a sin offering and one ram as a burnt offering.}%
\verse{“And Aaron shall present the sin offering’s bull, which is for himself, and so he shall make atonement for himself and for his family.}%
\verse{And he shall take the two goats, and he shall present them \textit{before}\lebnote{Literally “to the faces of”}Adonai at the tent of assembly’s entrance.}%
\verse{Then\lebnote{Or “And”}Aaron shall cast lots for the two goats: one lot for Adonai and one for Azazel.}%
\verse{And Aaron shall present the goat on which the lot for Adonai fell, and he shall \textit{sacrifice}\lebnote{Literally “do” or “make”}it as a sin offering.}%
\verse{But\lebnote{Or “And”}he must present alive \textit{before}\lebnote{Literally “to the faces of”}Adonai the goat on which the lot for Azazel fell to make atonement for himself, to send it away into the desert to Azazel.}%
\verse{“And Aaron shall present the sin offering’s bull, which is for himself, and so he shall make atonement for himself and for his family; then\lebnote{Or “and”}he shall slaughter the sin offering’s bull, which is for himself.}%
\verse{And he shall take a\lebnote{Hebrew “the”}censer full of \textit{burning charcoal}\lebnote{Literally “burning charcoal of fire”}from upon the altar from \textit{before}\lebnote{Literally “to the faces of”}Adonai and \textit{two handfuls}\lebnote{Literally “the fullness of the hollow of his hands”}of incense of powdered fragrant perfumes, and he shall bring it \textit{from behind}\lebnote{Literally “from the interior/inside of”}the curtain,}%
\verse{and he shall put the incense on the fire \textit{before}\lebnote{Literally “to the faces of”}Adonai so that the cloud of incense might cover the atonement cover, which is on the \textit{covenant text},\lebnote{Literally “testimony”}so that he might not die.}%
\verse{And he shall take \textit{some of}\lebnote{Literally “from”}the bull’s blood, and he shall spatter it\lebnote{The direct object is supplied from context in the English translation}with his finger on the atonement cover’s surface on the eastern side, and \textit{before}\lebnote{Literally “to the faces of”}the atonement cover he shall spatter \textit{some of}\lebnote{Literally “from”}the blood with his finger seven times.}%
\verse{“And he shall slaughter the sin offering’s goat, which is for the people, and he shall bring its blood \textit{from behind}\lebnote{Literally “from the interior/inside of”}the curtain, and he shall do with its blood as that which he did with\lebnote{Hebrew “to”}the bull’s blood, and he shall spatter it on the atonement cover and \textit{before}\lebnote{Literally “to the faces of”}the atonement cover.}%
\verse{Thus\lebnote{Or “And”}he shall make atonement for the sanctuary from the \textit{Israelites}’\lebnote{Literally “sons/children of Israel”}impurities and from their transgressions for all their sins; and so he must do for the tent of assembly, which dwells with them in the midst of their impurities.}%
\verse{And \textit{no person}\lebnote{Literally “any man not”}shall be in the tent of assembly when he enters to make atonement in the sanctuary until he comes out, and so he shall make atonement for himself and for his family and for all of Israel’s assembly.}%
\verse{“Then\lebnote{Or “And”}he shall go out to the altar that is \textit{before}\lebnote{Literally “to the faces of”}Adonai, and he shall make atonement for it; and he shall take \textit{some of}\lebnote{Literally “from”}the bull’s blood and \textit{some of}\lebnote{Literally “from”}the goat’s blood, and he shall put it\lebnote{The direct object is supplied from context in the English translation}all around on the altar’s horns.}%
\verse{And he shall spatter \textit{some of}\lebnote{Literally “from”}the blood on it seven times with his finger, and he shall cleanse it and consecrate\lebnote{Literally “he shall consecrate”}it from the \textit{Israelites}’\lebnote{Literally “of the sons/children of Israel”}impurities.}%
\verse{“And he shall finish making atonement for the sanctuary and the tent of assembly and the altar; then\lebnote{Or “and”}he shall present the living goat.}%
\verse{And Aaron shall place his two hands on the living goat’s head, and he shall confess over it all the \textit{Israelites}’\lebnote{Literally “of the sons/of the children of Israel”}iniquities and all their transgressions for all their sins, and he shall put them on the goat’s head, and he shall send it\lebnote{The direct object is supplied from context in the English translation}away into the desert \textit{with}\lebnote{Literally “by the hand of”}a man standing ready.\lebnote{Or “at hand”}}%
\verse{Thus\lebnote{Or “And”}the goat shall bear on it to \textit{a barren region}\lebnote{Literally “a land of separation” or “a land of infertility”}all their guilt, and he shall send the goat away into the desert.}%
\verse{“And Aaron shall enter the tent of assembly, and he shall take off the linen garments that he put on at his coming to the sanctuary, and he shall leave them there.}%
\verse{And he shall wash his body with water in a holy place, and he shall put on his garments and go out and \textit{sacrifice}\lebnote{Literally “he shall do” or “he shall make”}his burnt offering and the people’s burnt offering, and so he shall make atonement for himself and for people.}%
\verse{And he must turn into smoke the sin offering’s fat on the altar.}%
\verse{“And the person who sends out the goat for Azazel shall wash his garments, and he shall wash his body with water, and \textit{afterward}\lebnote{Literally “after thus”}he shall come to the camp.}%
\verse{And the sin offering’s bull and the sin offering’s goat, whose blood was brought to make atonement in the sanctuary, shall be brought \textit{outside the camp},\lebnote{Literally “to from an outside place of the camp”}and they shall burn their hide and their flesh and their offal in the fire.}%
\verse{And the person who burns them shall wash his garments, and he shall wash his body with water, and \textit{afterward}\lebnote{Literally “after thus”}he must come to the camp.}%
\verse{“And this shall be \textit{a lasting statute}\lebnote{Literally “a statute of eternity” or “a statute of long duration”}for you: in the seventh month, on the tenth of the month, you must deny yourselves and you must not do any work, whether the native or\lebnote{Or “and”}the alien who is dwelling in your midst,}%
\verse{because on this day he shall make atonement for you to cleanse you; you must be clean from all your sins \textit{before}\lebnote{Literally “to the faces of”}Adonai.}%
\verse{It is \textit{a Sabbath of complete rest}\lebnote{Literally “a Sabbath of ‘Sabbathation.’ ” “Sabbathation” is not a real word, but it is devised as an attempt to convey the sounds of the related nouns in the Hebrew phrase}for you, and you shall deny yourselves—it is \textit{a lasting statute}.\lebnote{Literally “a statute of eternity” or “a statute of long duration”}}%
\verse{And the priest who is anointed and who is \textit{ordained}\lebnote{Literally “filled his hand”}to serve as a priest in place of his father shall make atonement; thus\lebnote{Or “and”}he shall put on the linen garments, the holy garments,}%
\verse{and he shall make atonement for the sanctuary’s holy place, and he shall make atonement for the tent of assembly and the altar, and he shall make atonement for the priests and for all of the assembly’s people.}%
\verse{And this shall be \textit{a lasting statute}\lebnote{Literally “a statute of eternity” or “a statute of long duration”}for you to make atonement for the \textit{Israelites}\lebnote{Literally “sons/children of Israel”}one time in a year from all their sins.”}%
\end{biblechapter}

\begin{biblechapter} % Leviticus 17
\verseWithHeading{The Place of Sacrifice}{Then\lebnote{Or “And”}Adonai spoke to Moses, saying,}%
\verse{“Speak to Aaron and to his sons and to all the \textit{Israelites},\lebnote{Literally “sons/children of Israel”}and you shall say to them, ‘This is the word that Adonai has commanded, saying,}%
\verse{“\textit{Any man}\lebnote{Literally “A man a man”}from the house of Israel who slaughters an ox or a sheep or a goat in the camp or who slaughters it \textit{outside the camp}\lebnote{Literally “from an outside place of the camp”}}%
\verse{and he does not bring it to the tent of assembly’s entrance to present an offering to Adonai \textit{before}\lebnote{Literally “to the faces of”}Adonai’s tabernacle, then\lebnote{Or “and”}that man shall be accounted bloodguilty—he has poured out blood, and that man shall be cut off from the midst of his people.}%
\verse{This is so that the \textit{Israelites}\lebnote{Literally “sons/children of Israel”}may bring their sacrifices that they are sacrificing \textit{in the open field}\lebnote{Literally “on the faces of the field”}and bring\lebnote{Or “they shall bring”}them for Adonai to the tent of assembly’s entrance to the priest, and they shall sacrifice fellowship offerings for Adonai with them.}%
\verse{And the priest shall sprinkle the blood on Adonai’s altar at the tent of assembly’s entrance, and he shall \textit{burn}\lebnote{Literally “turn into smoke”}the fat as an appeasing fragrance for Adonai.}%
\verse{And \textit{they may no longer sacrifice}\lebnote{Literally “not they may sacrifice again”}their sacrifices to the goat-idols after which they were prostituting. This is \textit{a lasting statute}\lebnote{Literally “a statute of eternity” or “a statute of long duration”}for them \textit{throughout}\lebnote{Literally “to”}their generations.” ’}%
\verseWithHeading{Instructions Against Eating Blood}{“And you shall say to them, ‘if there is \textit{anyone}\lebnote{Literally “A man a man”}from the house of Israel or\lebnote{Or “and”}from the alien who dwells in their midst who offers a burnt offering or a sacrifice}%
\verse{and he does not bring it to the tent of assembly’s entrance to \textit{sacrifice}\lebnote{Literally “do” or “make”}it for Adonai, then\lebnote{Or “and”}that man shall be cut off from his people.}%
\verse{And if there is \textit{anyone}\lebnote{Literally “a man a man”}from the house of Israel or\lebnote{Or “and”}from the alien who is dwelling in their midst who eats any blood, then\lebnote{Or “and”}I will set my face against the person who eats the blood, and I will cut him off from among his people.}%
\verse{Indeed\lebnote{Or “Because”}the flesh’s life is in the blood, and I have given it to you on the altar to make atonement for your lives, because it is the blood with the life that makes atonement.}%
\verse{\textit{Therefore}\lebnote{Literally “Unto thus”}I said to the \textit{Israelites},\lebnote{Literally “sons/children of Israel”}‘\textit{None of you}\lebnote{Literally “all of individual self from you not”}may eat blood, nor\lebnote{Or “and not”}may the alien who is dwelling in your midst eat blood.’}%
\verse{“And if there is \textit{anyone}\lebnote{Literally “a man a man”}from the \textit{Israelites}\lebnote{Literally “sons/children of Israel”}or\lebnote{Or “and”}from the alien who is dwelling in their midst who hunts a wild game animal or a bird that may be eaten, then\lebnote{Or “and”}he shall pour out its blood, and he shall cover it with the soil.}%
\verse{Indeed,\lebnote{Or “Because”}the life of all flesh, its blood, is in its life, so\lebnote{Or “and”}I said to the \textit{Israelites},\lebnote{Literally “sons/children of Israel”}‘You may not eat the blood of any flesh, because the life of all flesh is its blood; anyone who eats it must be cut off.’}%
\verse{“And if there is any person who eats a dead body or\lebnote{Or “and”}a mangled carcass, whether among the native or\lebnote{Or “and”}among the alien, then\lebnote{Or “and”}he shall wash his garments, and he shall wash himself\lebnote{The direct object is supplied from context in the English translation}with water, and he shall be unclean until the evening, and he shall be clean.}%
\verse{But\lebnote{Or “And”}if he does not wash his garments\lebnote{The direct object is supplied from context in the English translation}and he does not wash his body, then\lebnote{Or “and”}he shall bear his guilt.”}%
\end{biblechapter}

\begin{biblechapter} % Leviticus 18
\verseWithHeading{Unlawful Sexual Relations}{Then\lebnote{Or “And”}Adonai spoke to Moses, saying,}%
\verse{“Speak to the \textit{Israelites},\lebnote{Literally “sons/children of Israel”}and say to them, ‘I am Adonai your God.}%
\verse{You must not \textit{carry out}\lebnote{Literally “do”}the practices of the land of Egypt, in which you lived, and you must not \textit{carry out}\lebnote{Literally “do”}the practices of the land of Canaan, to which I am bringing you; and you must not follow their statutes.}%
\verse{You must \textit{carry out}\lebnote{Literally “do”}my regulations, and you must observe\lebnote{Or “keep”}my statutes by following them; I am Adonai your God.}%
\verse{And you shall observe\lebnote{Or “keep”}my statutes and my regulations \textit{by which the person doing them shall live};\lebnote{Literally “which the man does them and he shall live in them”}I am Adonai.}%
\verse{“ ‘\textit{None}\lebnote{Literally “a man a man … not”}of you shall approach anyone who is \textit{his close relative}\lebnote{Literally “the flesh of his body”}to expose nakedness; I am Adonai.}%
\verse{You must not expose your father’s nakedness or\lebnote{Or “and”}your mother’s nakedness—she is your mother; you must not expose her nakedness.}%
\verse{You must not expose the nakedness of your father’s wife—it is your father’s nakedness.}%
\verse{As for your sister’s nakedness, whether your father’s daughter or your mother’s daughter, whether \textit{born at home}\lebnote{Literally “a relative of house”}or \textit{born abroad},\lebnote{Literally “a relative of an outside place”}you must not expose their nakedness.}%
\verse{As for the nakedness of your son’s daughter or your daughter’s daughter, you must not expose their nakedness, because they are your nakedness.}%
\verse{As for the nakedness of the daughter of your father’s wife, she is your sister, a relative of your father; you must not expose her nakedness.}%
\verse{You must not expose the nakedness of your father’s sister; she is \textit{your father’s close relative}.\lebnote{Literally “the flesh of your father”}}%
\verse{You must not expose the nakedness of your mother’s sister, because she is \textit{your mother’s close relative}.\lebnote{Literally “the flesh of your mother”}}%
\verse{You must not expose the nakedness of your father’s brother; you must not \textit{have sex with}\lebnote{Literally “approach”}his wife—she is your aunt.}%
\verse{You must not expose your daughter-in-law’s nakedness; she is your son’s wife; you must not expose her nakedness.}%
\verse{You must not expose the nakedness of your brother’s wife; she is your brother’s nakedness.}%
\verse{You must not expose the nakedness of a woman and her daughter, or her son’s daughter, or her daughter’s daughter; you must not take her as wife\lebnote{The verb לקח is used to speak of marrying (“taking a wife”)}to expose her nakedness; they are \textit{close relatives}\lebnote{Literally “flesh”; see vv. 6, 12, 13}—that is wickedness.}%
\verse{And you must not take as wife\lebnote{The verb לקח is used to speak of marrying (“taking a wife”)}a woman with her sister, to be a rival-wife, to expose her nakedness before her \textit{during}\lebnote{Literally “in”}her life.}%
\verse{“ ‘And you must not \textit{have sex with}\lebnote{Literally “approach”}a woman to expose her nakedness \textit{during}\lebnote{Literally “in”}her menstrual uncleanness.}%
\verse{And \textit{you must not have sex}\lebnote{Literally “you shall not give your lying down for semen”}with your fellow citizen’s\lebnote{Or “neighbor’s”}wife, becoming unclean with her.}%
\verse{“ ‘And you shall not give \textit{any of}\lebnote{Literally “from”}your offspring\lebnote{Or “descendants”}in order to sacrifice them to Molech, nor\lebnote{Or “and not”}shall you profane the name of your God; I am Adonai.}%
\verse{And you shall not lie with a male as lying \textit{with}\lebnote{Literally “of”}a woman; that is a detestable thing.}%
\verse{And you shall not \textit{have sexual relations}\lebnote{Literally “give your lying down”}with any animal, becoming unclean with it; and a woman shall not stand \textit{before}\lebnote{Literally “to the faces of”}an animal to copulate with it—that is a perversion.}%
\verse{“ ‘You must not make yourself unclean in any of these things, because the nations whom I am driving out from your presence were made unclean by all of these.}%
\verse{So\lebnote{Or “And”}the land became unclean, and \textit{I have brought the punishment of}\lebnote{Literally “I have visited”}its guilt upon it, and the land has vomited out its inhabitants.}%
\verse{But\lebnote{Or “And”}you (neither the native nor\lebnote{Or “and”}the alien who is dwelling in your midst) shall keep my statutes and my regulations, and you shall not \textit{practice}\lebnote{Literally “do”}any of these detestable things}%
\verse{(because the people of the land, who were \textit{before you},\lebnote{Literally “to the faces of you”}did all these detestable things, so the land became unclean),}%
\verse{so that the land will not vomit you out when you make it unclean just as it vomited out the nation that was \textit{before you}.\lebnote{Literally “to the faces of you”}}%
\verse{Indeed, anyone who does any of these detestable things, even\lebnote{Or “and”}those persons who do so shall be cut off from the midst of their people.}%
\verse{Thus\lebnote{Or “And”}you shall keep my obligation to not do any of the statutes \textit{regarding}\lebnote{Literally “of”}the detestable things that they did \textit{before you},\lebnote{Literally “to the faces of you”}so that you will not make yourselves unclean by them; I am Adonai your God.’ ”}%
\end{biblechapter}

\begin{biblechapter} % Leviticus 19
\verseWithHeading{Adonai Is Holy}{Then\lebnote{Or “And”}Adonai spoke to Moses, saying,}%
\verse{“Speak to all the community of the \textit{Israelites},\lebnote{Literally “sons/children of Israel”}and say to them, ‘You\lebnote{Plural}must be holy, because I, Adonai your\lebnote{Plural}God, am holy.}%
\verse{Each of you must revere your mother\lebnote{Hebrew “his mother”}and your father,\lebnote{Hebrew “his father”}and you\lebnote{Plural}must keep my Sabbaths; I am Adonai your\lebnote{Plural}God.}%
\verse{You\lebnote{Plural}must not turn to idols, and you\lebnote{Plural}must not make for yourselves gods of cast metal; I am Adonai your\lebnote{Plural}God.}%
\verse{“ ‘And when you\lebnote{Plural}sacrifice a sacrifice of fellowship offerings to Adonai, you\lebnote{Plural}must sacrifice it for your\lebnote{Plural}acceptance.}%
\verse{It must be eaten on the day of your\lebnote{Plural}sacrifice and the next day; but\lebnote{Or “and”}the remainder must be burned up in the fire by the third day.}%
\verse{And if it is indeed eaten on the third day, it is unclean meat; it shall not be regarded as accepted.}%
\verse{And the one who eats it shall bear his guilt, because he has profaned Adonai’s holiness, and that person shall be cut off from his people.}%
\verseWithHeading{Love Your Neighbor as Yourself}{“ ‘And at your\lebnote{Plural}reaping the harvest of your\lebnote{Plural}land you\lebnote{Singular}must not finish reaping the edge of your\lebnote{Singular}field, and you\lebnote{Singular}must not glean the remnants of your\lebnote{Singular}harvest.}%
\verse{And you\lebnote{Singular}must not glean your\lebnote{Singular}vineyard, and you\lebnote{Singular}must not gather your\lebnote{Singular}vineyard’s fallen grapes; you\lebnote{Singular}must leave them behind for the needy and for the alien; I am Adonai your\lebnote{Plural}God.}%
\verse{“ ‘You\lebnote{Plural}shall not steal, and you\lebnote{Plural}shall not deceive, and you\lebnote{Plural}shall not lie \textit{to one another};\lebnote{Literally “a man to his fellow citizen”}}%
\verse{and you\lebnote{Plural}shall not swear \textit{falsely}\lebnote{Literally “to the deception”}in my name, and so one of you\lebnote{Singular}profane\lebnote{Or “you shall profane”}the name of your\lebnote{Singular}God; I am Adonai.}%
\verse{“ ‘You\lebnote{Singular}shall not exploit your\lebnote{Singular}neighbor, and you\lebnote{Singular}shall not rob him; a hired worker’s wage you\lebnote{Singular}shall not \textit{withhold}\lebnote{Literally “leave with you”}overnight until morning.}%
\verse{You\lebnote{Singular}shall not curse the deaf, and you\lebnote{Singular}shall not put a stumbling block \textit{before}\lebnote{Literally “to the faces of”}a blind person, but\lebnote{Or “and”}you\lebnote{Singular}shall revere your\lebnote{Singular}God; I am Adonai.}%
\verse{“ ‘You\lebnote{Plural}shall not do injustice in judgment; \textit{you\lebnote{Singular}shall not show partiality to the powerless};\lebnote{Literally “you shall not lift up the faces of the poor/powerless”}you\lebnote{Singular}shall not give preference \textit{to the powerful};\lebnote{Literally “faces of the great”}you\lebnote{Singular}shall judge your\lebnote{Singular}fellow citizen with justice.}%
\verse{You\lebnote{Singular}shall not go about with slander among your\lebnote{Singular}people; \textit{you\lebnote{Singular}shall not endanger your\lebnote{Singular}neighbor’s life};\lebnote{Literally “you shall not stand on the blood of your neighbor”}I am Adonai.}%
\verse{“ ‘You\lebnote{Singular}shall not hate your\lebnote{Singular}brother in your\lebnote{Singular}heart; you\lebnote{Singular}shall surely rebuke your\lebnote{Singular}fellow citizen, so that you\lebnote{Singular}do not incur sin \textit{along with}\lebnote{Or “in addition to him”; literally “upon him”}him.}%
\verse{You\lebnote{Singular}shall not seek vengeance, and you\lebnote{Singular}shall not harbor a grudge against \textit{your\lebnote{Singular}fellow citizens};\lebnote{Literally “the sons of your people”}and you\lebnote{Singular}shall love your\lebnote{Singular}neighbor like yourself; \lebnote{Singular}I am Adonai.}%
\verseWithHeading{You Shall Keep My Statutes}{“ ‘You\lebnote{Plural}must keep my statutes: as for your\lebnote{Singular}domestic animals, you\lebnote{Singular}shall not cause two differing kinds to breed; as for your\lebnote{Singular}field, you\lebnote{Singular}shall not sow two differing kinds of seed; and, a garment of two differing kinds of woven material should not be worn on you.\lebnote{Singular}}%
\verse{“ ‘And when a man lies with a woman and there is an emission of semen and she is a female slave promised to a man, but\lebnote{Or “and”}she indeed has not been ransomed or freedom has not be given to her, there shall be an obligation to compensate; they shall not be put to death, because she has not been freed.}%
\verse{And he shall bring his guilt offering to Adonai at the tent of assembly’s entrance: a ram for a guilt offering.}%
\verse{And the priest shall make atonement for him \textit{before}\lebnote{Literally “to the faces of”}Adonai with the ram of the guilt offering for his sin that he \textit{committed},\lebnote{Literally “sinned”}and so his sin that he \textit{committed}\lebnote{Literally “sinned”}shall be forgiven him.}%
\verse{“ ‘And when you\lebnote{Plural}have come into the land and you\lebnote{Plural}plant any tree for\lebnote{Hebrew “of”}food, \textit{you\lebnote{Plural}shall regard its fruit as unharvestable};\lebnote{Literally “and you shall regard its foreskin as uncircumcised its fruit”}for three years it shall be forbidden for you; \lebnote{Plural}it shall not be eaten.}%
\verse{But\lebnote{Or “And”}in the fourth year all its fruit shall be holy, offerings of praise for Adonai.}%
\verse{And in the fifth year you\lebnote{Plural}shall eat its fruit to increase its yield for you; \lebnote{Plural}I am Adonai your\lebnote{Plural}God.}%
\verse{“ ‘You\lebnote{Plural}must not eat anything with the blood; you\lebnote{Plural}shall not practice divination, nor shall you\lebnote{Plural}interpret signs.}%
\verse{You\lebnote{Plural}shall not round off the corner hair of your\lebnote{Plural}head, and you\lebnote{Singular}shall not trim the corner of your\lebnote{Singular}beard.}%
\verse{And you\lebnote{Plural}shall not make a slash in your\lebnote{Plural}body for a dead person, nor shall you\lebnote{Plural}make on yourselves a tattoo’s mark; I am Adonai.}%
\verse{“ ‘You\lebnote{Singular}shall not profane your\lebnote{Singular}daughter by making her a prostitute, \textit{lest the land be prostituted and the land fill up with depravity}.\lebnote{Literally “and the land does not prostitute and the land fills up depravity” or “so that the land does not prostitute and the land fills up depravity”}}%
\verse{You\lebnote{Plural}shall keep my Sabbaths, and you\lebnote{Plural}shall revere my sanctuary; I am Adonai.}%
\verse{“ ‘You\lebnote{Plural}shall not turn to the mediums and to the soothsayers; you\lebnote{Plural}shall not seek them to become unclean with them; I am Adonai your\lebnote{Plural}God.}%
\verse{“ ‘\textit{Before}\lebnote{Literally “from the faces of”}old age you\lebnote{Singular}shall get up, and you\lebnote{Singular}shall show respect for an old person; and you\lebnote{Singular}shall revere your God; I am Adonai.}%
\verse{“ ‘And when an alien dwells with you\lebnote{Singular}in your\lebnote{Plural}land, you\lebnote{Plural}shall not oppress him.}%
\verse{The alien who is dwelling with you\lebnote{Plural}shall be like a native among you,\lebnote{Plural}and you\lebnote{Singular}shall love him like yourself, \lebnote{Singular}because you\lebnote{Plural}were aliens in the land of Egypt; I am Adonai your\lebnote{Plural}God.}%
\verse{“ ‘You\lebnote{Plural}shall not commit injustice in regulation, in measurement, in weight, or\lebnote{Or “and”}volume.}%
\verse{You\lebnote{Plural}must have honest balances, honest weights, an honest ephah, and an honest hin; I am Adonai your\lebnote{Plural}God who brought you\lebnote{Plural}out from the land of Egypt.}%
\verse{“ ‘Thus you\lebnote{Plural}shall keep all my statutes and all my regulations, and you\lebnote{Plural}shall do them; I am Adonai.’ ”}%
\end{biblechapter}

\begin{biblechapter} % Leviticus 20
\verseWithHeading{Molech Worship and Spiritism}{Then\lebnote{Or “And”}Adonai spoke to Moses, saying,}%
\verse{“And to the \textit{Israelites}\lebnote{Literally “sons/children of Israel”}you shall say, ‘If there is \textit{anyone}\lebnote{Literally “a man a man”}from the \textit{Israelites}\lebnote{Literally “sons/children of Israel”}or\lebnote{Or “and”}from the alien who is dwelling in Israel, who gives \textit{any of}\lebnote{Literally “from”}his offspring to Molech, he must surely be put to death; the people of the land must stone him with stones.\lebnote{Hebrew “the stone”}}%
\verse{And I myself will set my face against that man, and I will cut him off from the midst of his people, because he has given \textit{some of}\lebnote{Literally “from”}his offspring to Molech, so that \textit{he makes my sanctuary unclean}\lebnote{Literally “to make unclean my sanctuary”}and profanes\lebnote{Hebrew “to profane”}\textit{my holy name}.\lebnote{Literally “the name of my holiness”}}%
\verse{And if the people of the land ever shut their eyes from\lebnote{Or “disregard”}that man at his giving \textit{some of}\lebnote{Literally “from”}his offspring to Molech, not putting him to death,}%
\verse{then\lebnote{Or “and”}I myself will set my face against that man and against his clan, and I will cut him off and all those from the midst of their people who prostitute after Molech.}%
\verse{As for\lebnote{Or “And”}the person who turns to the mediums and the soothsayers to prostitute after them, I will set\lebnote{Hebrew “and I will set”}my face against that person, and I will cut him off from the midst of his people.}%
\verse{“ ‘And you shall consecrate yourselves, and you shall be holy, because I am Adonai your God.}%
\verse{And you shall keep my statutes, and you shall do them; I am Adonai who consecrates you.}%
\verseWithHeading{Family and Sexual Offenses}{“ ‘If there is \textit{anyone}\lebnote{Literally “a man a man”}who curses his father or his mother, he shall surely be put to death; he has cursed his father and his mother—his blood is upon him.}%
\verse{“ ‘As for\lebnote{Or “And”}a man who commits adultery with a man’s wife, who commits adultery with his neighbor’s wife, both the man who commits adultery and the woman who commits adultery shall surely be put to death.}%
\verse{As for\lebnote{Or “And”}a man who lies with his father’s wife, he has exposed his father’s nakedness; both of them shall be put to death—their blood is on them.}%
\verse{As for\lebnote{Or “And”}a man who lies with his daughter-in-law, both of them shall be put to death; they have committed a perversion—their blood is on them.}%
\verse{“ ‘As for\lebnote{Or “And”}the man who lies with a male as lying with a woman, they have committed a detestable thing; they shall surely be put to death—their blood is on them.}%
\verse{“ ‘As for\lebnote{Or “And”}a man who marries a woman and her mother, that is depravity; they shall burn him and them, so that it shall not become depravity in the midst of you all.}%
\verse{“ ‘As for\lebnote{Or “And”}a man who \textit{has sexual relations}\lebnote{Literally “he gives his lying down”}with an animal, he shall surely be put to death, and you must kill the animal.}%
\verse{As for\lebnote{Or “And”}a woman who approaches any animal to copulate with it, you shall kill\lebnote{Or “and you shall kill”}the woman and the animal; they shall surely be put to death—their blood is on them.}%
\verse{“ ‘As for\lebnote{Or “And”}a man who takes his sister, his father’s daughter or his mother’s daughter, and he sees her nakedness and she herself sees his nakedness, it is a disgrace, and they shall be cut off \textit{before}\lebnote{Literally “in”}the eyes of \textit{their people};\lebnote{Literally “the sons/children of their people”}he has exposed his sister’s nakedness—he must bear his guilt.}%
\verse{As for\lebnote{Or “And”}a man who lies with a menstruating woman, he exposes\lebnote{Or “and he exposes”}her nakedness—her source he exposes and she herself reveals her blood’s source—both of them shall be cut off from the midst of their people.}%
\verse{And you shall not expose the nakedness of your mother’s sister, and you shall not expose your father’s sister, because such a person\lebnote{Hebrew “he”}has dishonored his close relative—they must bear their guilt.}%
\verse{As for\lebnote{Or “And”}a man who lies with his aunt, he has exposed his uncle’s nakedness—they shall bear their sin; they shall die childless.}%
\verse{As for\lebnote{Or “And”}a man who marries his brother’s wife, it is an abomination; he has exposed his brother’s nakedness—they shall be childless.}%
\verseWithHeading{You Shall Be Holy}{“ ‘And you shall keep all my statutes and all my regulations, and you shall do them, so that\lebnote{Or “and”}the land, to which I am bringing you to inhabit it, shall not vomit you out.}%
\verse{And you shall not follow the statutes of the nation that I am driving out \textit{from before you},\lebnote{Literally “from your faces”}because they did all these things, and I detested them.}%
\verse{So\lebnote{Or “And”}I said to you, “You yourselves shall take possession of their land, and I myself shall give it to you to possess it—a land flowing with milk and honey”; I am Adonai your God, who \textit{has set you apart}\lebnote{Literally “I have set you apart”}from the nations.}%
\verse{And you shall distinguish between the clean and the unclean animal and between the unclean and the clean bird; and you shall not defile yourselves with the animal or\lebnote{Or “and”}with the bird or\lebnote{Or “and”}with anything that moves along the ground that I have set apart for you \textit{as unclean}.\lebnote{Literally “to make unclean”}}%
\verse{And you shall be holy for me, because I, Adonai, am holy, and I have singled you out from the nations to be mine.}%
\verse{“ ‘And a man or a woman, if a spirit of the dead or a spirit of divination is in them, they shall surely be put to death; they shall stone them with stones\lebnote{Hebrew “the stone”}—their blood is on them.’ ”}%
\end{biblechapter}

\begin{biblechapter} % Leviticus 21
\verseWithHeading{Regulations Concerning Priests}{Then\lebnote{Or “And”}Adonai said to Moses, “Speak to the priests, Aaron’s sons, and say to them, ‘One must not make himself unclean for a dead person among his own people,}%
\verse{except for his direct relative closest to him: his mother and his father, and his son and his daughter, and his brother,}%
\verse{and for his sister, a virgin, who is closest to him, \textit{who has not had a husband}\lebnote{Literally “who is not for a man/husband”}—for her he may defile himself.}%
\verse{He must not make himself unclean as \textit{a kinsman by marriage},\lebnote{Literally “a husband among his people”}defiling himself.}%
\verse{“ ‘And they shall not shave bald patches on their head, and they shall not shave off the corner of their beard, and they shall not make a cut in their body.}%
\verse{They shall be holy to their God, and they shall not profane the name of their God, because they are bringing near the offerings made by fire to\lebnote{Hebrew “of”}Adonai—their God’s food—and they shall be holy.}%
\verse{“ ‘They shall not \textit{marry}\lebnote{Literally “take”}a woman who is a prostitute and defiled, nor shall they \textit{marry}\lebnote{Literally “take”}a woman divorced from her husband, because each priest\lebnote{Hebrew “he”; singular antecedent specified from the context}is holy for his God.}%
\verse{And you shall consecrate him, because he is bringing near your God’s food; he shall be holy to you, since I, Adonai, who consecrates you, am holy.}%
\verse{“ ‘As for\lebnote{Or “And”}the daughter of any priest, if she is defiled by prostituting, she is disgracing her father—she shall be burned in the fire.}%
\verse{“ ‘As for\lebnote{Or “And”}the priest who is higher than his brothers, on whose head the oil of anointment is poured and \textit{he was ordained}\lebnote{Literally “he has filled up his hand”}to wear the garments, he shall not dishevel his head, and he shall not tear his garments.}%
\verse{And he shall not go near any dead person, nor shall he make himself unclean for his father or\lebnote{Or “and”}for his mother.}%
\verse{And he shall not go out from the sanctuary, and he shall not profane his God’s sanctuary, because the dedication of his God’s oil of anointment is on him; I am Adonai.}%
\verse{“ ‘And he himself must take a wife in her virginity.}%
\verse{A widow or\lebnote{Or “and”}a divorced woman or\lebnote{Or “and”}a defiled woman, a prostitute—these he must not take; he shall take only\lebnote{Or “instead”}a virgin from his people as wife.}%
\verse{And he shall not profane his offspring among his people, because I am Adonai, who consecrates him.’ ”}%
\verse{Then\lebnote{Or “And”}Adonai spoke to Moses, saying,}%
\verse{“Speak to Aaron, saying, ‘A man from your offspring throughout their generations, in whom is a physical defect, shall not come near to present your God’s food.}%
\verse{Indeed,\lebnote{Or “For”}any man in whom is a physical defect shall not come near: a blind man or lame or disfigured or deformed,}%
\verse{or a man in whom is a broken foot or a broken hand,}%
\verse{or a hunchback or a dwarf, or a spot in his eye or a skin disorder or a skin eruption or a crushed testicle.}%
\verse{Any man from Aaron the priest’s offspring in whom is a physical defect shall not come near to present offerings made by fire to\lebnote{Hebrew “of”}Adonai; a physical defect is in him; he shall not come near to present his God’s food.}%
\verse{He may eat his God’s food, from \textit{the most holy things}\lebnote{Literally “the holy things of the holy things”}and from the holy things.}%
\verse{But he must not enter the curtain, and he must not come near to the altar, because a physical defect is in him, and he must not profane my sanctuary, because I am Adonai, who consecrates them.’ ”}%
\verse{Thus\lebnote{Or “And”}Moses spoke to Aaron and to his sons and to all the \textit{Israelites}.\lebnote{Literally “sons/children of Israel”}}%
\end{biblechapter}

\begin{biblechapter} % Leviticus 22
\verseWithHeading{Priests and Their Food}{Then\lebnote{Or “And”}Adonai spoke to Moses, saying,}%
\verse{“Tell\lebnote{Or “Speak to”}Aaron and his sons\lebnote{Literally “to Aaron and to his sons”}that\lebnote{Or “and”}they must deal respectfully with the \textit{Israelites}’\lebnote{Literally “sons/children of Israel”}votive offerings, and they must not profane my holy name, which they are consecrating to me; I am Adonai.}%
\verse{“Say to them, ‘Throughout your generations, any man from any of your offspring who comes near the votive offerings that the \textit{Israelites}\lebnote{Literally “sons/children of Israel”}consecrate to Adonai with\lebnote{Or “and”}his uncleanness on him, that person shall be cut off\lebnote{Or “and that person shall be cut off”}from \textit{before me};\lebnote{Literally “to the faces of me”}I am Adonai.}%
\verse{“ ‘\textit{Anyone}\lebnote{Literally “A man a man”}from Aaron’s offspring, if\lebnote{Or “and”}he is afflicted with a skin disease or a fluid discharge, shall not eat in the sanctuary \textit{until}\lebnote{Literally “until which”}he is clean; and the one who touches any unclean person or a man from whom an emission of semen goes out,}%
\verse{or a man who touches any swarmer that is unclean for him or who touches a person who is unclean for him \textit{due to}\lebnote{Literally “for”}whatever his uncleanness,}%
\verse{a person who touches such a thing\lebnote{Hebrew “him” or “it”}shall be unclean until the evening, and he shall not eat from the votive offerings, except\lebnote{Or “but if” or “but rather”}when he washes\lebnote{Or “bathes”}his body with water}%
\verse{and \textit{the sun sets},\lebnote{Literally “the sun goes” or “the sun enters”}and he shall be clean; then\lebnote{Or “and”}afterward he may eat from the votive offerings, because it is his food.}%
\verse{He shall not eat a naturally dead body or\lebnote{Or “and”}a mangled carcass, \textit{so that he becomes unclean}\lebnote{Literally “to become unclean”}by it; I am Adonai.}%
\verse{“ ‘And they shall keep my obligation, and they shall not incur guilt because of it, so that\lebnote{Or “and”}they die through it, because they have profaned it; I am Adonai who consecrates them.}%
\verse{“ ‘\textit{No stranger shall eat}\lebnote{Literally “And any stranger shall not eat”}the votive offering; nor shall a temporary resident with\lebnote{Hebrew “of”}a priest or\lebnote{Or “and”}a hired worker eat the votive offering.}%
\verse{But\lebnote{Or “And”}a priest, if with his money he buys a person as \textit{his possession},\lebnote{Literally “property of”}that one may eat it, and the descendants of his house themselves may eat his food.}%
\verse{And a priest’s daughter, when \textit{she marries a layman},\lebnote{Literally “she becomes for a strange man”}she herself may not eat \textit{the votive offering}.\lebnote{Literally “the offering/lifting of the votive offering”}}%
\verse{But\lebnote{Or “And”}a priest’s daughter, when she becomes a widow or\lebnote{Or “and”}divorced or there is no offspring for her, and she returns to her father’s house as in her childhood, she may eat from her father’s food, but\lebnote{Or “and”}\textit{no layman may eat it}.\lebnote{Literally “any stranger shall not eat it”}}%
\verse{And if a man eats the votive offering unintentionally, then\lebnote{Or “and”}he shall add to it a fifth of it, and he shall give the votive offering to the priest.}%
\verse{And they shall not profane the \textit{Israelites}’\lebnote{Literally “sons/children of Israel”}votive offerings that they present to Adonai,}%
\verse{and so cause them, by their eating their votive offerings, to bear guilt requiring a guilt offering, because I am Adonai, who consecrates them.’ ”}%
\verseWithHeading{Acceptable Offerings}{Then\lebnote{Or “And”}Adonai spoke to Moses, saying,}%
\verse{“Speak to Aaron and to his sons and to all the \textit{Israelites},\lebnote{Literally “sons/children of Israel”}and say to them, ‘Anyone from the house of Israel or\lebnote{Or “and”}from the alien in Israel who presents his offering for any of their vows or\lebnote{Or “and”}for any of their freewill offerings that they present to Adonai as a burnt offering,}%
\verse{it must be without defect \textit{to be acceptable for you}:\lebnote{Literally “for your acceptance”}a male among the cattle, among the sheep, or\lebnote{Or “and”}among the goats.}%
\verse{You shall not present any animal in which is a physical defect, because it shall not be \textit{acceptable}\lebnote{Literally “for acceptance”}for you.}%
\verse{And if anyone brings a sacrifice of fellowship offerings for Adonai to fulfill a vow or as a freewill offering from\lebnote{Or “among”}the cattle or from\lebnote{Or “among”}the flock,\lebnote{The Hebrew term refers collectively to both sheep and goats (small livestock animals)}it must be without defect \textit{to be acceptable};\lebnote{Literally “for acceptance”}there must not be any physical defect in it.}%
\verse{The blind or the injured or the maimed or the seeping or one with a skin disorder or one with a skin eruption—these you shall not present to Adonai, nor shall you give from them an offering made by fire on the altar for Adonai.}%
\verse{As for\lebnote{Or “And”}an ox or sheep that is deformed or\lebnote{Or “and”}that is stunted, you may present it as a freewill offering, but\lebnote{Or “and”}for a vow it will not be accepted.}%
\verse{And you shall not present anything for Adonai with bruised or\lebnote{Or “and”}shattered or\lebnote{Or “and”}torn or\lebnote{Or “and”}cut-off testicles, and you shall not \textit{sacrifice}\lebnote{Literally “do” or “make”}such in your land.}%
\verse{And you shall not present your God’s food from any of these by the hand of \textit{a foreigner},\lebnote{Literally “a son of a foreign land”}because their deformity is in them; a physical defect is in them; they shall not be accepted for you.’ ”}%
\verse{Then\lebnote{Or “And”}Adonai spoke to Moses, saying,}%
\verse{“When an ox or a sheep or a goat is born, then\lebnote{Or “and”}it shall be under its mother for seven days, and from the eighth day and beyond it is acceptable as an offering made by fire for Adonai.}%
\verse{And you shall not slaughter an ox or a sheep and \textit{its young}\lebnote{Literally “his son” or “his child”}on \textit{the same day}.\lebnote{Literally “on day one” or “in one day”}}%
\verse{And when you sacrifice a sacrifice of thanksgiving to Adonai, you must sacrifice it to be acceptable for you.}%
\verse{It must be eaten on that day; you must not leave over anything from it until morning; I am Adonai.}%
\verse{“Thus\lebnote{Or “And”}you shall keep my commands, and you shall do them; I am Adonai.}%
\verse{“And you shall not profane my holy name, so that I may be consecrated in the midst of the \textit{Israelites};\lebnote{Literally “sons/children of Israel”}I am Adonai, who consecrates you,}%
\verse{the one who brought you out from the land of Egypt to be as God for you; I am Adonai.”}%
\end{biblechapter}

\begin{biblechapter} % Leviticus 23
\verseWithHeading{Adonai’s Feasts}{Then\lebnote{Or “And”}Adonai spoke to Moses, saying,}%
\verse{“Speak to the \textit{Israelites},\lebnote{Literally “sons/children of Israel”}and say to them, ‘The festivals of Adonai that you shall proclaim are holy assemblies; these are my appointed times.}%
\verseWithHeading{Adonai’s Sabbath}{“ ‘For six days work is to be done, and on the seventh day shall be \textit{a Sabbath of complete rest},\lebnote{Literally “a Sabbath of ‘Sabbathation.’ ” “Sabbathation” is not a real word, but it is devised as an attempt to convey the sounds of the related nouns in the Hebrew phrase}a holy assembly; you shall not do any work; it shall be a Sabbath for Adonai in all your dwellings.}%
\verseWithHeading{The Passover}{“ ‘These are Adonai’s appointed times, holy assemblies, which you shall proclaim at\lebnote{Or “on”}their appointed time.}%
\verse{In the first month, on the fourteenth of the month at the evening is Adonai’s Passover.}%
\verse{And on the fifteenth day of this month is Adonai’s Feast of Unleavened Bread; for seven days you shall eat unleavened bread.}%
\verse{On the first day there shall be a holy assembly for you; you shall not do \textit{any regular work}.\lebnote{Literally “all work of labor”}}%
\verse{And you shall present an offering for Adonai made by fire for seven days; on the seventh day there shall be a holy assembly; you shall not do \textit{any regular work}.’ ”\lebnote{Literally “all work of labor”}}%
\verseWithHeading{The Feast of Firstfruits}{Then\lebnote{Or “And”}Adonai spoke to Moses, saying,}%
\verse{“Speak to the \textit{Israelites},\lebnote{Literally “sons/children of Israel”}and say to them, ‘When you come to the land that I am about to give to you and you reap its harvest, then\lebnote{Or “and”}you shall bring a sheaf of the firstfruit of your harvest to the priest.}%
\verse{And he shall wave the sheaf \textit{before}\lebnote{Literally “to the faces of”}Adonai for your acceptance; the priest shall wave it \textit{on the day after}\lebnote{Literally “from the next day of”}the Sabbath.}%
\verse{And on the day of your waving the sheaf you shall \textit{offer}\lebnote{Or “sacrifice”; literally “do” or “make”}a \textit{yearling}\lebnote{Literally “a son of its year”}male lamb without defect as a burnt offering to Adonai.}%
\verse{And its grain offering shall be two-tenths of an ephah\lebnote{Supplied by context}of finely milled flour mixed with oil, an offering made by fire for Adonai, an appeasing fragrance; and its libation shall be a fourth of a hin\lebnote{Hebrew “the hin”}of wine.}%
\verse{And you shall not eat bread or\lebnote{Or “and”}roasted grain or\lebnote{Or “and”}ripe grain until \textit{this very same day},\lebnote{Literally “the exactly of this day”}until you present your God’s offering. This must be \textit{a lasting statute}\lebnote{Literally “a statute of eternity” or “a statute of long duration”}for your generations in all your dwellings.}%
\verseWithHeading{The Feast of Weeks}{“ ‘And you shall count for yourselves \textit{from the day after}\lebnote{Literally “from the next day of”}the Sabbath, from the day of your bringing the wave offering’s sheaf—there shall be seven full weeks.}%
\verse{Until \textit{the day after}\lebnote{Literally “from the next day of”}the seventh Sabbath you shall count fifty days; then\lebnote{Or “and”}you shall present a new grain offering for Adonai.}%
\verse{You shall bring from your dwellings for a wave offering two loaves of bread \textit{made with}\lebnote{Literally “they shall be”}two-tenths of an ephah\lebnote{Supplied by context}of finely milled flour; they must be baked with leaven—the firstfruits \textit{belonging to}\lebnote{Literally “to”}Adonai.}%
\verse{And, in addition to the bread, you shall present seven \textit{yearling}\lebnote{Literally “a son of its year”}male lambs without defects and one \textit{young bull}\lebnote{Literally “a bull a son of cattle”}and two rams—they shall be a burnt offering for Adonai with\lebnote{Or “and”}their grain offering and their libations, an offering made by fire, an appeasing fragrance for Adonai.}%
\verse{And you shall \textit{offer}\lebnote{Or “sacrifice”; literally “do” or “make”}one he-goat as a sin offering and two \textit{yearling}\lebnote{Literally “a son of its year”}male lambs as a sacrifice of fellowship offerings.}%
\verse{And the priest shall wave them with the bread of the firstfruits as a wave offering \textit{before}\lebnote{Literally “to the faces of”}Adonai; in addition to the two male lambs, they shall be holy for Adonai for the priest.}%
\verse{And \textit{you shall make a proclamation}\lebnote{Literally “you shall proclaim”}on \textit{this very same day};\lebnote{Literally “the exactly of this day”}it shall be a holy assembly for you; you shall not do \textit{any regular work};\lebnote{Literally “all work of labor”}this is a \textit{lasting statute}\lebnote{Literally “a statute of eternity” or “a statute of long duration”}in all your dwellings throughout\lebnote{Or “for”}your generations.}%
\verse{And when you\lebnote{Plural}reap the harvest of your\lebnote{Plural}land, you\lebnote{Singular}must not finish the edge of your\lebnote{Singular}field at your reaping, and you\lebnote{Singular}must not glean the remnants of your\lebnote{Singular}harvest—you\lebnote{Singular}shall leave them behind for the needy and for the alien; I am Adonai your\lebnote{Plural}God.’ ”}%
\verseWithHeading{The Feast of Trumpets}{Then\lebnote{Or “And”}Adonai spoke to Moses, saying,}%
\verse{“Speak to the \textit{Israelites},\lebnote{Literally “sons/children of Israel”}saying, ‘In the seventh month, on the first day of the month, \textit{you must have}\lebnote{Literally “it must be for yourselves”}a rest period, a remembrance of the trumpet blast, a holy assembly.}%
\verse{You must not do \textit{any regular work},\lebnote{Literally “all work of labor”}and you shall present an offering made by fire to Adonai.’ ”}%
\verseWithHeading{The Day of Atonement}{Then\lebnote{Or “And”}Adonai spoke to Moses, saying,}%
\verse{“Surely the Day of Atonement is on the tenth day of the seventh month; it shall be a holy assembly for you, and you shall deny yourselves, and you shall present an offering made by fire to Adonai.}%
\verse{And you must not do \textit{any regular work}\lebnote{Literally “all work of labor”}on \textit{this very same day},\lebnote{Literally “the exactly of this day”}because it is the Day of Atonement to make atonement for you \textit{before}\lebnote{Literally “to the faces of”}Adonai your God.}%
\verse{If there is any person who does not deny himself on \textit{this very same day},\lebnote{Literally “the exactly of this day”}then\lebnote{Or “and”}he shall be cut off from his people.}%
\verse{As for\lebnote{Or “And”}any person who does any work on \textit{this very same day},\lebnote{Literally “the exactly of this day”}I will exterminate that person from the midst of his people.}%
\verse{You must not do any work; it is a \textit{lasting statute}\lebnote{Literally “a statute of eternity” or “a statute of long duration”}throughout your generations in all your dwellings.}%
\verse{It is \textit{a Sabbath of complete rest}\lebnote{Literally “a Sabbath of ‘Sabbathation.’ ” “Sabbathation” is not a real word, but it is devised as an attempt to convey the sounds of the related nouns in the Hebrew phrase}for you, and you shall deny yourselves on the ninth day of the month in the evening—from evening to evening you must observe your extraordinary\lebnote{Supplied from the immediate context within this verse}Sabbath.”}%
\verseWithHeading{The Feast of Booths}{Then\lebnote{Or “And”}Adonai spoke to Moses, saying,}%
\verse{“Speak to the \textit{Israelites},\lebnote{Literally “sons/children of Israel”}saying, ‘On the fifteenth day of the seventh month, this shall be the Feast of Booths for seven days for Adonai.}%
\verse{On the first day there shall be a holy assembly; you must not do any \textit{regular work}.\lebnote{Literally “work of labor”}}%
\verse{For seven days you must present an offering made by fire to Adonai. On the eighth day it shall be a holy assembly for you, and you shall present an offering made by fire to Adonai; it is a celebration; you must not do \textit{any regular work}.\lebnote{Literally “all work of labor”}}%
\verseWithHeading{Summary}{“ ‘These are Adonai’s festivals, which you must proclaim, holy assemblies to present an offering made by fire to Adonai—burnt offering and grain offering, sacrifice and libations, \textit{each on its proper day}\lebnote{Literally “a thing of a day on its day”}—}%
\verse{\textit{besides}\lebnote{Literally “from to alone”}Adonai’s Sabbaths and \textit{besides}\lebnote{Literally “from to alone”}your gifts and \textit{besides}\lebnote{Literally “from to alone”}your vows and \textit{besides}\lebnote{Literally “from to alone”}all your freewill offerings that you give to Adonai.}%
\verse{“ ‘Surely on the fifteenth day of the seventh month, at your gathering the land’s produce, you shall hold Adonai’s festival for seven days; on the first day there shall be a rest period and on the eighth day a rest period.}%
\verse{And on the first day you shall take for yourselves the first fruit of majestic trees, branches of palm trees and branches\lebnote{Collective singular}of a leafy tree and of a brook’s poplar trees, and you shall rejoice \textit{before}\lebnote{Literally “to the faces of”}Adonai your God for seven days.}%
\verse{And you must hold it as a festival for Adonai for seven days in the year; it shall be a \textit{lasting statute}\lebnote{Literally “a statute of eternity” or “a statute of long duration”}throughout your generations; in the seventh month you must hold it.}%
\verse{You must live in the booths for seven days; all the natives in Israel must live in the booths,}%
\verse{so that your generations shall know that I made the \textit{Israelites}\lebnote{Literally “sons/children of Israel”}live in booths when I brought them from the land of Egypt; I am Adonai your God.’ ”}%
\verse{Thus\lebnote{Or “And”}Moses announced to the \textit{Israelites}\lebnote{Literally “sons/children of Israel”}Adonai’s appointed times.}%
\end{biblechapter}

\begin{biblechapter} % Leviticus 24
\verseWithHeading{The Sanctuary’s Lamp and Bread}{Then\lebnote{Or “And”}Adonai spoke to Moses, saying,}%
\verse{“Command the \textit{Israelites}\lebnote{Literally “sons/children of Israel”}that\lebnote{Or “and”}they should bring pure olive oil from beaten olives for the light to \textit{present}\lebnote{Literally “to cause raising up”; or “set up” (HALOT 830 s.v. 4.d)}a lamp continually.}%
\verse{Aaron shall arrange for it \textit{outside}\lebnote{Literally “from the outside of to”}the curtain of the testimony in the tent of assembly from evening until morning \textit{before}\lebnote{Literally “to the faces of”}Adonai continually; it shall be a \textit{lasting statute}\lebnote{Literally “a statute of eternity” or “a statute of long duration”}throughout your generations.}%
\verse{On the pure golden\lebnote{Supplied by context and the description of the lampstand in Exod 25:31}lampstand he shall arrange for the lamps \textit{before}\lebnote{Literally “to the faces of”}Adonai continually.}%
\verse{“And you shall take finely milled flour, and you shall bake with it twelve ring-shaped bread cakes: each one shall be two-tenths of an ephah.}%
\verse{And you shall place them in two rows, six to the row, on the pure gold\lebnote{Supplied by context and the description of the table in Exod 25:24}table \textit{before}\lebnote{Literally “to the faces of”}Adonai.}%
\verse{And you shall put pure frankincense on each\lebnote{Hebrew “the”}row so that\lebnote{Or “and”}it shall be for the bread as a memorial offering, an offering made by fire for Adonai.}%
\verse{\textit{On every Sabbath}\lebnote{Literally “On the day of the Sabbath on the day of the Sabbath”}he shall arrange it in rows \textit{before}\lebnote{Literally “to the faces of”}Adonai continually; they are from the \textit{Israelites}\lebnote{Literally “sons/children of Israel”}as an everlasting\lebnote{Or “eternal” or “enduring” or “perpetual”}covenant.}%
\verse{And it shall be for Aaron and for his sons, and they shall eat it in a holy place, because it is \textit{a most holy thing}\lebnote{Literally “a holy thing of holy things”}for him from Adonai’s offerings made by fire—a \textit{lasting rule}.”\lebnote{Literally “rule of eternity” or “rule of long duration”}}%
\verseWithHeading{Punishment for Blasphemy}{And an Israelite woman’s son, and he was an Egyptian man’s son, went out in the midst of the \textit{Israelites};\lebnote{Literally “sons/children of Israel”}and the Israelite woman’s son and an\lebnote{Hebrew “the”}Israelite man fought in the camp.}%
\verse{Then\lebnote{Or “And”}the Israelite woman’s son blasphemed the name, and he cursed, so\lebnote{Or “and”}they brought him to Moses—and the name of his mother was Shelomith the daughter of Dibri, of the tribe of Dan.}%
\verse{Then\lebnote{Or “And”}they put him in custody \textit{so that}\lebnote{Literally “to”}the matter might be made clear to them in accordance with the mouth of Adonai.}%
\verse{Then\lebnote{Or “And”}Adonai spoke to Moses, saying,}%
\verse{“Bring the curser \textit{outside the camp},\lebnote{Literally “to from an outside place of the camp”}and all the hearers shall place their hands on his head, and the whole community shall stone him.}%
\verse{And you shall speak to the \textit{Israelites},\lebnote{Literally “sons/children of Israel”}saying, ‘\textit{Any man}\lebnote{Literally “A man a man”}when he curses his God shall bear\lebnote{Or “and he shall bear”}his sin.}%
\verse{And he who blasphemes Adonai’s name certainly shall be put to death; the whole community certainly shall stone him. As the alien, so the native shall be put to death at blaspheming his name.}%
\verse{“ ‘And when a man kills any human being, he certainly shall be put to death.}%
\verse{And \textit{he who kills}\lebnote{Literally “he who kills the soul of” or “he who kills the life of”}a domestic animal must repay for it life in place of life.}%
\verse{And when a man \textit{causes}\lebnote{Literally “gives”}a physical defect in his fellow citizen \textit{according to}\lebnote{Literally “as that”}what he has done, so it shall be done to him:}%
\verse{fracture in place of fracture, eye in place of eye, tooth in place of tooth—\textit{according to}\lebnote{Literally “as that”}the physical defect he \textit{causes}\lebnote{Literally “gives”}to the person, likewise it shall be \textit{caused}\lebnote{Literally “given”}to him.}%
\verse{And a killer of a domestic animal must repay for it, and a killer of a human shall be put to death.}%
\verse{\textit{You must have}\lebnote{Literally “it shall be for you”}one norm; as for the alien, so\lebnote{Hebrew “as”}it must be for the native, because I am Adonai your God.’ ”}%
\verse{Thus\lebnote{Or “And”}Moses spoke to the \textit{Israelites},\lebnote{Literally “sons/children of Israel”}and they brought the curser \textit{outside the camp},\lebnote{Literally “to from an outside place of the camp”}and they stoned him with stones, and the \textit{Israelites}\lebnote{Literally “sons/children of Israel”}did just as Adonai had commanded Moses.}%
\end{biblechapter}

\begin{biblechapter} % Leviticus 25
\verseWithHeading{The Sabbath Year}{Then\lebnote{Or “And”}Adonai spoke to Moses on \textit{Mount Sinai},\lebnote{Literally “the mountain of Sinai”}saying,}%
\verse{“Speak to the \textit{Israelites},\lebnote{Literally “sons/children of Israel”}and say to them, ‘When you\lebnote{Plural}come into the land that I am about to give to you,\lebnote{Plural}then\lebnote{Or “and”}the land shall observe a Sabbath for Adonai.}%
\verse{Six years you\lebnote{Singular throughout this verse}shall sow your field, and six years you shall prune your vineyard, and you shall gather its yield.}%
\verse{But\lebnote{Or “And”}in the seventh year it shall be \textit{a Sabbath of complete rest}\lebnote{Literally “a Sabbath of ‘Sabbathation.’ ” “Sabbathation” is not a real word, but it is devised as an attempt to convey the sounds of the related nouns in the Hebrew phrase}for the land—a Sabbath for Adonai; you\lebnote{Singular throughout this verse}must not sow your field, and you must not prune your vineyard.}%
\verse{You\lebnote{Singular throughout this verse}must not reap your harvest’s aftergrowth, and you must not harvest the grapes of your unpruned vines—it shall be \textit{a year of complete rest}\lebnote{Literally “a year of a rest period”}for the land.}%
\verse{And a Sabbath of the land shall be for food for you:\lebnote{Plural}for you\lebnote{Singular here and through the rest of this verse}and for your slave and for your slave woman and for your hired worker and for your temporary residents\lebnote{Collective singular; Hebrew “temporary resident”}who are dwelling as aliens with you;}%
\verse{and all its yield shall be for your\lebnote{Singular throughout this verse}domestic animal and for the wild animal, which are in your land to eat.}%
\verseWithHeading{The Year of Jubilee}{“ ‘And you\lebnote{Singular throughout this verse}shall count for yourself seven Sabbaths of years, seven times seven years, and they shall be for you \textit{time periods of}\lebnote{Literally “days of”}years: \textit{forty-nine}\lebnote{Literally “nine and forty”}years.}%
\verse{And you\lebnote{Singular}shall cause \textit{a loud horn blast}\lebnote{Literally “a ram’s horn of a blast”}to be heard on the seventh month on the tenth of the month; on the Day of Atonement you\lebnote{Plural}shall cause a ram’s horn to be heard in all your\lebnote{Plural}land.}%
\verse{And you\lebnote{Plural throughout this verse}shall consecrate the fiftieth year, and you shall proclaim a release in the land for all its inhabitants. It is a Jubilee; it shall be for you, and you shall return. You must return—everyone to his property and everyone to his clan.\lebnote{Or “each of you must return to his property and to his clan”}}%
\verse{\textit{You\lebnote{Plural throughout this verse}shall have the fiftieth year as a Jubilee};\lebnote{Literally “a Jubilee it the year of the fiftieth year it shall be for you”}you must not reap its aftergrowth, and you must not harvest its unpruned vines.}%
\verse{Because it is a Jubilee, it shall be holy to you. You must eat its\lebnote{That is, the field’s}produce from the field.}%
\verse{“ ‘In this Year of Jubilee \textit{each of you\lebnote{Plural}shall return}\lebnote{Literally “you shall return a man” or “you shall return everyone”}to his property.}%
\verse{And when you\lebnote{Plural}sell something to your\lebnote{Singular}fellow citizen or you buy from your\lebnote{Singular}neighbor’s hand, you\lebnote{Plural}must not oppress \textit{one another}.\lebnote{Literally “a man his brother”}}%
\verse{You\lebnote{Singular throughout this verse}must buy from your fellow citizen according to the number of years after the Jubilee; he must sell to you according to the number of years of yield.}%
\verse{You\lebnote{Singular throughout this verse}must increase its price \textit{according to a greater number of years},\lebnote{Literally “to the mouth of many of the years”}but\lebnote{Or “and”}you must decrease its price \textit{according to a lesser number of years},\lebnote{Literally “to the mouth of few of the years”}because he is selling its yields to you.}%
\verse{And you\lebnote{Plural}must not oppress \textit{one another},\lebnote{Literally “a man his citizen”}but\lebnote{Or “and”}you\lebnote{Singular}shall revere your God, because I am Adonai, your\lebnote{Plural}God.}%
\verse{“ ‘And you\lebnote{Plural throughout this verse}shall do my statutes, and you must keep my regulations, and you shall do them, so that\lebnote{Or “and”}you shall live \textit{securely}\lebnote{Literally “with confidence”}on the land.}%
\verse{And the land shall give its fruit, and you\lebnote{Plural throughout this verse}shall eat \textit{your fill},\lebnote{Literally “to contentment”}and you shall live \textit{securely}\lebnote{Literally “with confidence”}on it.}%
\verse{And if you\lebnote{Plural}should say, “What shall we eat in the seventh year, \textit{if}\lebnote{Literally “look”}we do not sow and we do not gather its yield?”}%
\verse{then\lebnote{Or “and”}I will command my blessing for you\lebnote{Plural}in the sixth year, so that\lebnote{Or “and”}it will make\lebnote{Or “produce”}the yield for three years.}%
\verse{And you\lebnote{Plural throughout this verse}will sow in the eighth year, and you shall eat from the old yield;\lebnote{The verse divides here}until the ninth year, until the coming of its yield, you shall eat the old yield.}%
\verse{“ ‘But\lebnote{Or “And”}the land must not be sold in\lebnote{Or “for”}perpetuity, because the land is mine, because you\lebnote{Plural}are aliens and temporary residents with me.}%
\verse{And in all your\lebnote{Plural}property’s land you\lebnote{Plural}must provide redemption for the land.}%
\verse{“ ‘When your\lebnote{Singular}brother becomes poor and he sells part of his property, then\lebnote{Or “and”}\textit{his nearest redeemer}\lebnote{Literally “his redeemer the nearest to him”}shall come, and he shall redeem the thing sold by his brother.}%
\verse{But\lebnote{Or “And”}if\lebnote{Or “when”}a man \textit{does not have}\lebnote{Literally “it is not for him”}a redeemer, then\lebnote{Or “and”}\textit{he prospers}\lebnote{Literally “his hand produces”}and he finds enough for his redemption,}%
\verse{then\lebnote{Or “and”}he shall calculate the years of its selling, and he shall refund the balance to the man to whom he sold it, and he shall return to his property.}%
\verse{But\lebnote{Or “And”}if his hand does not find enough to refund to him, then\lebnote{Or “and”}\textit{what he has sold}\lebnote{Literally “his thing sold”}shall be in the buyer’s hand until the Year of Jubilee; and it shall go out of the buyer’s hand\lebnote{Meaning derived from context; an alternative translation could be “shall not be released” (cf. NKJV, NRSV, ESV, NJPS) or “shall not revert” (NASB, NET)}in the Jubilee, and he shall return to his property.}%
\verse{“ ‘And if a man sells \textit{a residential house in a walled city},\lebnote{Literally “a house of a dwelling of a city wall”}then\lebnote{Or “and”}it shall be his redemption until completing \textit{a year after his selling};\lebnote{Literally “a year of his selling”}its redemption \textit{shall last}\lebnote{Literally “shall be”}\textit{a year}.\lebnote{Literally “days”}}%
\verse{But\lebnote{Or “And”}if it is not redeemed \textit{before a full year has passed},\lebnote{Literally “until the fulfilling of for him/it an entire year”}then\lebnote{Or “and”}the house that is \textit{in the walled city}\lebnote{Literally “in the city that for it a wall”}shall belong to the buyer in\lebnote{Hebrew “for”}perpetuity throughout his generations; it shall not go out of the buyer’s hand\lebnote{Meaning derived from context; an alternative translation could be “shall not be released” (cf. NKJV, NRSV, ESV, NJPS) or “shall not revert” (NASB, NET)}in the Jubilee.}%
\verse{However,\lebnote{Or “And” or “But”}village houses that have no surrounding wall shall be considered \textit{open country};\lebnote{Literally “the field of the land”}there is redemption for it, and in the Jubilee it shall go out of the buyer’s hand.\lebnote{Meaning derived from context; an alternative translation could be “shall not be released” (cf. NKJV, NRSV, ESV, NJPS) or “shall not revert” (NASB, NET)}}%
\verse{“ ‘As for\lebnote{Or “And”}the cities of the Levites, that is, the houses in\lebnote{Hebrew “of”}their property’s cities, it shall be \textit{a lasting redemption}\lebnote{Or “a permanent redemption”; literally “redemption of eternity” or “redemption of long duration”}for the Levites.}%
\verse{And whatever anyone redeems from the Levites then\lebnote{Or “and”}must go out of the buyer’s hand\lebnote{Meaning derived from context; an alternative translation could be “shall not be released” (cf. NKJV, NRSV, ESV, NJPS) or “shall not revert” (NASB, NET)}in the Jubilee, including a house’s selling \textit{in his city’s property},\lebnote{Literally “the city of his property”}because the houses in\lebnote{Hebrew “of”}the cities of the Levites are their property in the midst of the \textit{Israelites}.\lebnote{Literally “sons/children of Israel”}}%
\verse{But\lebnote{Or “And”}a\lebnote{Hebrew “the”}field of their cities’ pastureland must not be sold, because \textit{it is their property for all time}.\lebnote{Literally “a property of eternity it is for them” or “a property of long duration it is for them”}}%
\verse{“ ‘And if your\lebnote{Singular throughout this verse}countryman\lebnote{Or “brother”}becomes poor and \textit{if he becomes dependent on you},\lebnote{Literally “and his hand is shaky with you”}then\lebnote{Or “and”}you shall support him like an alien and like a temporary resident, and he shall live with you.}%
\verse{You\lebnote{Singular throughout this verse}must not take interest or\lebnote{Or “and”}usury from him, but\lebnote{Or “and”}you shall revere your God, and your countryman\lebnote{Or “brother”}shall live with you.}%
\verse{You\lebnote{Singular throughout this verse}must not give your money to him with interest or\lebnote{Or “and”}give your food for\lebnote{Or “in” or “with”}profit.}%
\verse{I am Adonai your\lebnote{Plural throughout this verse}God, who brought you out from the land of Egypt to give \textit{you}\lebnote{Literally “to you”}the land of Canaan, to be as God for you.}%
\verse{“ ‘And if your\lebnote{Singular throughout this verse}countryman\lebnote{Or “brother”}who is with you becomes poor, and he is sold to you, \textit{you shall not treat him as a slave}.\lebnote{Literally “you shall not let him work the work of a slave” or “you shall not enslave him the work of a slave”}}%
\verse{He shall be with you like a hired worker, like a temporary resident; he shall work with you until the Year of Jubilee.}%
\verse{And he and his sons with him shall go out from you, and he shall return to his clan, and to the property of his ancestors\lebnote{Or “fathers”}he shall return.}%
\verse{Because they are my servants whom I brought out from the land of Egypt, they shall not be sold \textit{as a slave}.\lebnote{Literally “a selling of a slave”}}%
\verse{You\lebnote{Singular throughout this verse}shall not rule over him with ruthlessness, but\lebnote{Or “and”}you shall revere your God.}%
\verse{“ ‘As for\lebnote{Or “And”}your\lebnote{Singular}slave and your\lebnote{Singular}slave woman who are\lebnote{Or “may be” (permission; cf. NASU, ESV, NRSV) or “can be” (ability)}yours,\lebnote{Singular}from the nations that are all around you,\lebnote{Plural}from them you\lebnote{Plural}may buy a slave or\lebnote{Or “and”}a slave woman.}%
\verse{And you\lebnote{Plural throughout this verse}may buy also from the children\lebnote{Or “sons”}of the temporary residents who are dwelling with you as aliens and from their clan who are with you, who have children in your land; indeed,\lebnote{Or “and”}they may be as property for you.}%
\verse{And you\lebnote{Plural}may pass them on as an inheritance to your\lebnote{Plural}sons\lebnote{Or “children”}after you\lebnote{Plural}to take possession of as property \textit{for all time}\lebnote{Literally “to eternity” or “forever” or “for a long duration”}—you\lebnote{Plural}may let them work. But\lebnote{Or “And”}as for your\lebnote{Plural}countrymen,\lebnote{Or “brothers”}the \textit{Israelites},\lebnote{Literally “sons/children of Israel”}you\lebnote{Singular}shall not rule with ruthlessness over \textit{one another}.\lebnote{Literally “a man over his brother”}}%
\verse{“ ‘And if the alien or\lebnote{Or “and”}the temporary resident who are with you\lebnote{Singular throughout this verse}\textit{prosper},\lebnote{Literally “the hand produces”}but\lebnote{Or “and”}your countryman\lebnote{Or “brother”}who is with him becomes poor and he is sold to an alien, a temporary resident who is with you, or to a descendant of an alien’s clan,}%
\verse{after he is sold redemption shall be for him; one of his brothers may redeem him,}%
\verse{or his uncle or his uncle’s son may redeem him, or \textit{one of}\lebnote{Literally “from”}\textit{his close relatives}\lebnote{Literally “the direct relative of his flesh”}from his clan may redeem him; or if \textit{he prospers},\lebnote{Literally “his hand produces”}he may redeem himself.}%
\verse{And he shall calculate with his buyer from the year of \textit{his selling himself}\lebnote{Literally “his being sold for him”}until the Jubilee; and the value of his selling shall be according to the number of years—it shall be with him like\lebnote{Or “as”}a hired worker’s days.}%
\verse{If there are still many years, \textit{in keeping with them}\lebnote{Literally “to the mouth of them”}he shall restore his redemption \textit{in proportion to his purchase price}.\lebnote{Literally “from the price of his acquisition”}}%
\verse{And if there are a few years left until the Year of Jubilee, then\lebnote{Or “and”}he shall calculate for himself; he shall restore his redemption \textit{according to the number of his years}.\lebnote{Literally “according to the mouth of his years”}}%
\verse{He shall be with him \textit{as a yearly hired worker};\lebnote{Literally “as a hired worker of a year in a year”}he shall not rule over him with ruthlessness \textit{in your\lebnote{Singular}sight}.\lebnote{Literally “to your eyes” or “for your eyes”}}%
\verse{And if he is not redeemed by\lebnote{Or “in”}any of these ways,\lebnote{Bracketed words provided from the sense of the context}then\lebnote{Or “and”}he and his sons with him shall go out in the Year of Jubilee.}%
\verse{Indeed,\lebnote{Emphatic use of כִּי, since there is no indication of direct causation to only what precedes in vv. 47–54; or “Because”}the \textit{Israelites}\lebnote{Literally “sons/children of Israel”}are servants for me; they are my servants whom I brought out from the land of Egypt. I am Adonai your\lebnote{Plural}God.’ ”}%
\end{biblechapter}

\begin{biblechapter} % Leviticus 26
\verseWithHeading{Blessings for Obedience}{“ ‘You\lebnote{Plural throughout the entire chapter}shall not make for yourselves idols and divine images, and you shall not raise up stone pillars for yourselves, and you shall not put a sculptured stone in your land in order to\lebnote{Or “to”}worship before it, because I am Adonai your God.}%
\verse{“ ‘You shall keep my Sabbaths, and you shall revere my sanctuary; I am Adonai.}%
\verse{“ ‘If you walk in my statutes and you keep my commands and you do them,}%
\verse{then\lebnote{Or “and”}I will give you rains in their time, and the land shall give its produce, and the trees of the field shall give their fruit.}%
\verse{And for you the threshing season shall overtake the grape harvest, and the grape harvest shall overtake the sowing, and you shall eat your food \textit{to your fill}\lebnote{Literally “to contentment”}and you shall live \textit{securely}\lebnote{Literally “with confidence”}in your land.}%
\verse{And I will give peace in the land, and you shall lie down, and there shall not be \textit{anybody who makes you afraid},\lebnote{Literally “who makes afraid” or “making afraid”}and I will remove harmful animals from the land, and \textit{no sword shall pass through your land}.\lebnote{Literally “a sword shall not pass through in your land”}}%
\verse{And you shall pursue your enemies, and they shall fall by the sword \textit{before you}.\lebnote{Literally “to your faces”}}%
\verse{And five of\lebnote{Hebrew “from”}you shall pursue a hundred, and a hundred of\lebnote{Hebrew “from”}you shall pursue a myriad;\lebnote{Or “ten thousand”}and your enemies shall fall by the sword \textit{before you}.\lebnote{Literally “to your faces”}}%
\verse{And I will turn to you, and I will make you fruitful, and I will make you numerous; and I will keep my covenant with you.}%
\verse{And you shall eat \textit{old grain},\lebnote{Literally “old what is stale”}and \textit{you shall clear away the old before the new}.\lebnote{Literally “old from the faces of new you shall bring out”}}%
\verse{And I will put my dwelling place in your midst, and my inner self\lebnote{Or “soul”}shall not abhor you.}%
\verse{And I will walk about in your midst, and I shall be your God,\lebnote{Or “as a God for you”}and you shall be my people.\lebnote{Or “as a people for me”}}%
\verse{I am Adonai, your God who brought you out from the land of Egypt, from being their slaves; and I broke the bars of your yoke, and I caused you to walk erectly.}%
\verseWithHeading{Punishment for Disobedience}{“ ‘But\lebnote{Or “And”}if you do not listen to me and you do not carry out\lebnote{Or “you do not do” or “you do not observe” or “you do not perform”}all these commands,}%
\verse{and if you reject my statutes and if your inner self\lebnote{Or “soul”}abhors my regulations, to not carry out\lebnote{Or “do” or “observe” or “perform”}all my commands by your breaking my covenant,}%
\verse{I \textit{in turn}\lebnote{Literally “also” or “indeed”}will do this to you: then\lebnote{Or “and”}I will summon onto you horror, the wasting disease, and the fever that wastes eyes and that drains away life; and you shall sow your seed \textit{in vain},\lebnote{Literally “for the emptiness”}and your enemies shall eat it.}%
\verse{And I will set my face against you, and you shall be defeated \textit{before}\lebnote{Literally “to the faces of”}your enemies; and your haters shall rule over you, and you shall flee away, but\lebnote{Or “and”}there shall not be \textit{anybody who is pursuing}\lebnote{Literally “who pursues” or “pursuing”}you.}%
\verse{“ ‘And if in spite of these things you do not listen to me, then\lebnote{Or “and”}I will continue to discipline you seven times for your sins.}%
\verse{And I will break the pride of your strength; and I will make your heaven like iron\lebnote{Or “iron ore”}and your land like copper.}%
\verse{And your strength shall be consumed \textit{in vain};\lebnote{Literally “to the emptiness”}and your land shall not give its produce, and the land’s trees shall not give their fruit.}%
\verse{“ ‘And if you go against me in hostility and you are not willing to listen to me, then\lebnote{Or “and”}I will add a plague onto you seven times according to your sins.}%
\verse{And I will send \textit{wild animals}\lebnote{Literally “the animals of the field”}out among you, and they shall make you childless, and they shall cut down your domestic animals, and they shall make you fewer; and your roads shall be desolate.}%
\verse{“ ‘And if you do not accept correction from\lebnote{Hebrew “for” or “to”}me through these things, but\lebnote{Or “and”}you go against me in hostility,}%
\verse{then\lebnote{Or “and”}I myself\lebnote{Emphatic personal pronoun}will also go against you in hostility, and I myself\lebnote{Emphatic personal pronoun}also will strike you seven times for your sins.}%
\verse{And I will bring upon you a sword that seeks vengeance for the covenant, and you shall be gathered to your cities; and I will send a plague in your midst, and you shall be given into the hand of an enemy.}%
\verse{At my breaking the \textit{supply}\lebnote{Literally “staff”}of bread\lebnote{Or “food”}for you, then\lebnote{Or “and”}ten women shall bake your bread in one oven, and they shall return your bread by weight; and you shall eat it, and you shall not be satisfied.}%
\verse{“ ‘And if through this you do not listen to me and you go against me in hostility,}%
\verse{then\lebnote{Or “and”}I will go against you in hostile anger, and also\lebnote{Or “surely”}I myself\lebnote{Emphatic personal pronoun}will discipline you seven times for your sins.}%
\verse{And you shall eat the flesh of your sons; and the flesh of your daughters you shall eat.}%
\verse{And I will destroy your high places, and I will cut down your incense altars, and I will place your corpses on your idols’ corpses; and my inner self\lebnote{Or “soul”}shall abhor you.}%
\verse{And I will lay your cities in ruins, and I will lay waste your sanctuaries; and I shall not smell your sacrifices’\lebnote{Implied by the use of the same phraseology in regard to the sacrifices in the early chapters of the book}appeasing fragrance.}%
\verse{And I myself\lebnote{Emphatic personal pronoun}will lay waste the land, and your enemies who are living in it shall be appalled over it.}%
\verse{And I will scatter you among the nations, and I will draw a sword behind you; and your land shall be a desolation, and your cities shall be a ruin.}%
\verse{Then the land shall enjoy its Sabbaths all the days of its lying desolate, and you shall be in the land of your enemies; then the land shall rest, and it shall enjoy its Sabbaths.}%
\verse{All the days of its lying desolate it shall rest for the time\lebnote{Implied by context}that it had not rested during your Sabbaths while you were living on it.}%
\verse{As for\lebnote{Or “And”}the ones who remain among you, I will bring\lebnote{Or “and I will bring”}fearfulness in their hearts in the land of their enemies; and a sound of a windblown leaf shall pursue them, and they shall flee like flight \textit{before}\lebnote{Literally “of”}a sword, and they shall fall, but\lebnote{Or “and”}there shall not be a pursuer.}%
\verse{And they shall stumble over \textit{one another}\lebnote{Literally “a man on his brother”}as \textit{from before}\lebnote{Literally “from the faces of”}a sword, but\lebnote{Or “and”}there shall not be a pursuer; and \textit{you shall have no resistance}\lebnote{Literally “it shall not be for you resistance”}\textit{before}\lebnote{Literally “to the faces of”}your enemies.}%
\verse{And you shall perish among the nations, and the land of your enemies shall eat you.}%
\verse{And because of their guilt, the ones among you who remain shall decay in the land of their enemies; and also because of the iniquities of their ancestors,\lebnote{Or “fathers”}they shall decay with them.}%
\verse{“ ‘But\lebnote{Or “And”}when they confess their guilt and the guilt of their ancestors\lebnote{Or “fathers”}in their infidelity that they displayed against me, and moreover that they went against me in hostility—}%
\verse{I myself\lebnote{Emphatic personal pronoun}also\lebnote{Or “surely”}went against them in hostility, and I brought them into the land of their enemies—or if then their uncircumcised heart is humbled and then they pay for their guilt,}%
\verse{I will remember\lebnote{Or “and I will remember”}my covenant with Jacob; and I will remember also my covenant with Isaac and also my covenant with Abraham, and I will remember the land.}%
\verse{And the land shall be deserted by them, and it shall enjoy its Sabbaths in its being desolate from them, and they themselves\lebnote{Emphatic personal pronoun}must pay for their guilt, \textit{simply because}\lebnote{Literally “because and in because”}they rejected my regulations, and their inner self\lebnote{Or “soul”}abhorred my statutes.}%
\verse{And \textit{in spite of}\lebnote{Literally “also even” or “moreover also” or “moreover even”}this, \textit{when they are}\lebnote{Literally “in their being”}in the land of their enemies I will not reject them, and I will not abhor them to destroy them, to break my covenant with them, because I am Adonai their God.}%
\verse{And I will remember the first covenant for them\lebnote{Or “on behalf of them” or “on their behalf”}—whom I brought out from the land of Egypt \textit{in the sight of}\lebnote{Literally “to the eyes of” or “for the eyes of”}the nations \textit{to be their God}.\lebnote{Literally “to be for them for God” or “to be for them as a God”}I am Adonai.’ ”}%
\verse{These are the rules and the regulations and the laws that Adonai gave between himself and the \textit{Israelites}\lebnote{Literally “sons/children of Israel”}on \textit{Mount Sinai}\lebnote{Literally “the mountain of Sinai”}\textit{through}\lebnote{Literally “in/by the hand of”}Moses.}%
\end{biblechapter}

\begin{biblechapter} % Leviticus 27
\verseWithHeading{Instructions About Vows}{Then\lebnote{Or “And”}Adonai spoke to Moses, saying,}%
\verse{“Speak to the \textit{Israelites},\lebnote{Literally “sons/children of Israel”}and say\lebnote{Or “and you shall say”}to them, ‘When a man makes a vow according to your\lebnote{Singular when modifying “proper value” throughout the entire chapter}proper value of persons to Adonai,}%
\verse{if\lebnote{Or “and”}your proper value is for a male\lebnote{Hebrew “the male”}from \textit{twenty years of age}\lebnote{Literally “a son of twenty years”}up to\lebnote{Or “and up to”}\textit{sixty years of age},\lebnote{Literally “a son of sixty years”}then\lebnote{Or “and”}your proper value shall be fifty shekels of money according to the sanctuary’s shekel.}%
\verse{But\lebnote{Or “And”}if it is for a female, then\lebnote{Or “and”}your proper value shall be thirty shekels.}%
\verse{And if from \textit{five years of age}\lebnote{Literally “a son of five years”}up to\lebnote{Or “and up to”}\textit{twenty years of age},\lebnote{Literally “a son of twenty years”}then\lebnote{Or “and”}your proper value shall be twenty shekels for the male and ten shekels for the female.}%
\verse{And if from \textit{a month of age}\lebnote{Literally “a son of a month”}up to\lebnote{Or “and up to”}\textit{five years of age},\lebnote{Literally “a son of five years”}then\lebnote{Or “and”}your proper value shall be five shekels of money for the male, and your proper value for the female shall be three shekels of money.}%
\verse{And if from \textit{sixty years of age}\lebnote{Literally “a son of sixty years”}and above: if a male, then\lebnote{Or “and”}your proper value shall be fifteen shekels; and for the female, ten shekels.}%
\verse{But\lebnote{Or “And”}if he is poorer than your proper value, then\lebnote{Or “and”}he shall present himself \textit{before}\lebnote{Literally “to the faces of”}the priest, and the priest shall set a value on him; the priest shall value him \textit{according to}\lebnote{Literally “on a mouth of”}what the person who made a vow \textit{can afford}.\lebnote{Literally “his hand produces”}}%
\verse{“ ‘And if it is a domestic animal from which they present an offering for Adonai, all that he gives from it for Adonai shall be a holy object.}%
\verse{He shall not replace it, nor shall he exchange it, either good with bad or bad with good; and if he indeed exchanges a domestic animal with a domestic animal, then\lebnote{Or “and”}\textit{it and its substitution shall be a holy object}.\lebnote{Literally “it shall be and its substitution shall be holy”}}%
\verse{But\lebnote{Or “And”}if it is any unclean animal from which they may not present an offering for Adonai, then\lebnote{Or “and”}he shall present the animal \textit{before}\lebnote{Literally “to the faces of”}the priest.}%
\verse{And the priest shall set a value on it, \textit{either good or bad};\lebnote{Literally “between good and between bad”}as the priest sets your proper value, so it shall be.}%
\verse{And if he indeed wants to redeem it, then\lebnote{Or “and”}he shall add a fifth of it onto your proper value.}%
\verse{“ ‘And if a man consecrates his house as a holy object for Adonai, then\lebnote{Or “and”}the priest shall set a value on it, \textit{either good or bad};\lebnote{Literally “between good and between bad”}just as the priest sets a value on it, so it shall remain.}%
\verse{But\lebnote{Or “And”}if the one who consecrates it wants to redeem his house, then\lebnote{Or “and”}he shall add a fifth of your proper value’s money onto it, and it shall be his.}%
\verse{“ ‘And if a man consecrates \textit{some of}\lebnote{Literally “from”}his property’s fields\lebnote{Collective singular; Hebrew “field”}for Adonai, then\lebnote{Or “and”}your proper value shall be \textit{in accordance with its seed requirements}:\lebnote{Literally “to the number of its seed”}a homer of barley seed for fifty shekels of money.}%
\verse{If he consecrates his field from the Year of Jubilee, it shall stand as your proper value.}%
\verse{But\lebnote{Or “And”}if he consecrates his field after the Jubilee, then\lebnote{Or “and”}the priest shall calculate the money for him \textit{according to the number of years}\lebnote{Literally “on the mouth of the years”}that are left over until the Year of Jubilee; and it shall be deducted from your proper value.}%
\verse{And if he indeed redeems the field that is consecrated, then\lebnote{Or “and”}he shall add a fifth of your proper value’s money onto it, and it shall stand for\lebnote{Or “remain for” or “belong to”}him.}%
\verse{And if he does not redeem the field and if he sells the field to another man, it may not be redeemed again,}%
\verse{and the field shall be a holy object for Adonai when it goes out\lebnote{Or “is released” or “reverts”}in the Jubilee, like a devoted\lebnote{Or “permanently set apart”; a different Hebrew word than previously translated “consecrated” in this chapter}field; \textit{it shall be the priest’s property}.\lebnote{Literally “to/for the priest it shall be his property”}}%
\verse{“ ‘And if he consecrates for Adonai his acquired\lebnote{Or “purchased”}field that is not the field of his inherited possession,}%
\verse{then\lebnote{Or “and”}the priest shall calculate for him the \textit{amount}\lebnote{Literally “number”}of your proper value until the year of the Jubilee, and he shall give your proper value on that day as a holy object for Adonai.}%
\verse{In the Year of the Jubilee the field shall return to the one who bought it from him, to the one whose property the land is.}%
\verse{And every proper value of yours shall be in the sanctuary’s shekel—the shekel shall be twenty gerahs.}%
\verse{“ ‘However, a man shall not consecrate a firstborn among livestock, which belongs as firstborn to Adonai; whether an ox \textit{or}\lebnote{Literally “whether” or “if”}small livestock, it is for Adonai.}%
\verse{And if it is among the unclean animals, then\lebnote{Or “and”}he shall ransom it according to your proper value, and he shall add a fifth of its value onto it; and if it is not redeemed, then\lebnote{Or “and”}it shall be sold according to your proper value.}%
\verse{However, anything devoted\lebnote{Or “permanently set apart”; a different Hebrew word than previously translated “consecrated” in this chapter}that a man has devoted to Adonai \textit{from all that he has},\lebnote{Literally “from all that is for him”}from human or\lebnote{Or “and”}animal, or\lebnote{Or “and”}from the field of his property, may not be sold, and it may not be redeemed; anything devoted is \textit{a most holy thing}\lebnote{Literally “a holy thing of holy things”}for Adonai.}%
\verse{Anyone devoted who is devoted from \textit{human beings}\lebnote{Literally “the human”}cannot be ransomed—he shall surely be put to death.}%
\verse{“ ‘And any tithe of the land from the land’s seed or from the fruit of the trees is for Adonai; it is a holy object for Adonai.}%
\verse{And if a man indeed redeems from his tithe, he shall add a fifth of its value onto it.}%
\verse{As for\lebnote{Or “And”}every tithe of cattle or\lebnote{Or “and”}of the flock,\lebnote{The Hebrew term refers collectively to both sheep and goats (small livestock animals)}all which crosses under the rod, the tenth shall be a holy object for Adonai.}%
\verse{He shall not inspect between the good and the bad, and he shall not exchange it; but\lebnote{Or “and”}if he indeed exchanges it, then\lebnote{Or “and”}\textit{it and its substitution shall be}\lebnote{Literally “it shall be and its substitution shall be”}a holy object—it shall not be redeemed.’ ”}%
\verse{These are the commands that Adonai commanded Moses for the \textit{Israelites}\lebnote{Literally “sons/children of Israel”}on \textit{Mount Sinai}.\lebnote{Literally “the mountain of Sinai”}}%
\end{biblechapter}

\flushcolsend
\biblebook{Numbers}

\begin{biblechapter} % Numbers 1
\verseWithHeading{God Commands Moses to Take a Census} Adonai spoke to Moses in the desert of Sinai, in the tent of assembly, on the first of the month, in the second year \textit{after they came out}\lebnote{Literally “of their coming out”} of the land of Egypt, saying,
\verse “\textit{Take a census of}\lebnote{Literally “Lift up the head of”} the entire community of the \textit{Israelites}\lebnote{Literally “sons/children of Israel”} according to their clans and \textit{their families},\lebnote{Literally “the house of their fathers”} according to the number of names, every male individually
\verse from \textit{twenty years old}\lebnote{Literally “a son of twenty years”} and above, everyone in Israel who is able to go to war. You and Aaron must muster them\lebnote{Or “count them,” or “summon them,” or “enroll them”} for their wars.
\verse A man from each tribe will be with you, each man the head of \textit{his family}.\lebnote{Literally “the house of his father”}
\verse And these are the names of the men who will \textit{assist you}:\lebnote{Literally “stand with you”} from Reuben, Elizur son of Shedeur;
\verse from Simeon, Shelumiel son of Zurishaddai;
\verse from Judah, Nahshon son of Amminadab;
\verse from Issachar, Nethanel son of Zuar;
\verse from Zebulun, Eliab son of Helon.
\verse From the descendants of Joseph: from Ephraim, Elishama son of Ammihud; from Manasseh, Gamaliel son of Pedahzur.
\verse From Benjamin, Abidan son of Gideoni;
\verse from Dan, Ahiezer son of Ammishaddai;
\verse from Asher, Pagiel son of Ocran;
\verse from Gad, Eliasaph son of Deuel;
\verse and from Naphtali, Ahira son of Enan.”
\verse These are the ones summoned from the community, the leaders of their ancestors’\lebnote{Or “fathers’ ”} tribes; they are the heads of Israel’s clans.
\verse So Moses and Aaron took these men who had been designated by name,
\verse and they summoned the entire community on the first day of the second month. And they registered themselves among their clans according to \textit{their families},\lebnote{Literally “the house of their fathers”} according to the number of names from \textit{those twenty years old}\lebnote{Literally “a son of twenty years”} and above individually,
\verse just as Adonai commanded Moses. And he counted them in the desert of Sinai.
\verse The descendants of Reuben, the firstborn of Israel, their genealogies according to their clans, according to \textit{their families},\lebnote{Literally “the house of their fathers”} according to the number of names, every male individually from \textit{twenty years old}\lebnote{Literally “a son of twenty years”} and above, everyone who is able to go to war:
\verse those who were counted from the tribe of Reuben were forty-six thousand five hundred.
\verse From the descendants of Simeon, their genealogies according to their clans, according to \textit{their families},\lebnote{Literally “the house of their fathers”} those who were counted according to the number of their names, every individual male from \textit{twenty years old}\lebnote{Literally “a son of twenty years”} and above, everyone who is able to go to war:
\verse those who were counted from the tribe of Simeon were fifty-nine thousand three hundred.
\verse From the descendants of Gad, their genealogies according to their clans, according to \textit{their families},\lebnote{Literally “the house of their fathers”} according to the number of names, from \textit{those twenty years old}\lebnote{Literally “a son of twenty years”} and above, everyone who is able to go to war:
\verse those who were counted from the tribe of Gad were forty-five thousand six hundred and fifty.
\verse From the descendants of Judah, their genealogies according to their clans, according to \textit{their families},\lebnote{Literally “the house of their fathers”} according to the number of names, from \textit{those twenty years old}\lebnote{Literally “a son of twenty years”} and above, everyone who is able to go to war:
\verse those who were counted from the tribe of Judah were seventy-four thousand six hundred.
\verse From the descendants of Issachar, their genealogies according to their clans, according to \textit{their families},\lebnote{Literally “the house of their fathers”} according to the number of names, from \textit{those twenty years old}\lebnote{Literally “a son of twenty years”} and above, everyone who is able to go to war:
\verse those who were counted from the tribe of Issachar were fifty-four thousand four hundred.
\verse From the descendants of Zebulun, their genealogies according to their clans, according to \textit{their families},\lebnote{Literally “the house of their fathers”} according to the number of names, from \textit{those twenty years old}\lebnote{Literally “a son of twenty years”} and above, everyone who is able to go to war:
\verse those who were counted from the tribe of Zebulun were fifty-seven thousand four hundred.
\verse From the descendants of Joseph: from the descendants of Ephraim, their genealogies according to their clans, according to \textit{their families},\lebnote{Literally “the house of their fathers”} according to the number of names, from \textit{those twenty years old}\lebnote{Literally “a son of twenty years”} and above, everyone who is able to go to war:
\verse those who were counted from the tribe of Ephraim were forty thousand five hundred.
\verse From the descendants of Manasseh, their genealogies according to their clans, according to \textit{their families},\lebnote{Literally “the house of their fathers”} according to the number of names, from \textit{those twenty years old}\lebnote{Literally “a son of twenty years”} and above, everyone who is able to go to war:
\verse those who were counted from the tribe of Manasseh were thirty-two thousand two hundred.
\verse From the descendants of Benjamin, their genealogies according to their clans, according to \textit{their families},\lebnote{Literally “the house of their fathers”} according to the number of names, from \textit{those twenty years old}\lebnote{Literally “a son of twenty years”} and above, everyone who is able to go to war:
\verse those who were counted from the tribe of Benjamin were thirty-five thousand four hundred.
\verse From the descendants of Dan, their genealogies according to their clans, according to \textit{their families},\lebnote{Literally “the house of their fathers”} according to the number of names, from \textit{those twenty years old}\lebnote{Literally “a son of twenty years”} and above, everyone who is able to go to war:
\verse those who were counted from the tribe of Dan were sixty-two thousand seven hundred.
\verse From the descendants of Asher, their genealogies according to their clans, according to \textit{their families},\lebnote{Literally “the house of their fathers”} according to the number of names, from \textit{those twenty years old}\lebnote{Literally “a son of twenty years”} and above, everyone who is able to go to war:
\verse those who were counted from the tribe of Asher were forty-one thousand five hundred.
\verse From the descendants of Naphtali, their genealogies according to their clans, according to \textit{their families},\lebnote{Literally “their fathers”} according to the number of names, from \textit{those twenty years old}\lebnote{Literally “a son of twenty years”} and above, everyone who is able to go to war:
\verse those who were counted from the tribe of Naphtali were fifty-three thousand four hundred.
\verse These are the ones counted whom Moses and Aaron mustered,\lebnote{Or “counted,” or “summoned,” or “enrolled”} with the twelve leaders of Israel, each one from \textit{his family}.\lebnote{Literally “the house of his fathers”}
\verse So all those who were counted from the \textit{Israelites}\lebnote{Literally “sons/children of Israel”} according to \textit{their families},\lebnote{Literally “the house of their fathers”} from \textit{those twenty years old}\lebnote{Literally “a son of twenty years”} and above, everyone in Israel who is able to go to war.
\verse All of the ones counted were six hundred and three thousand, five hundred and fifty.
\verse The Levites from their ancestors’\lebnote{Or “fathers’ ”} tribe were not mustered\lebnote{Or “counted,” or “summoned,” or “enrolled”} in their midst.
\verse And Adonai spoke to Moses, saying,
\verse “You will not muster\lebnote{Or “count,” or “summon,” or “enroll”} the tribe of Levi, and you will not \textit{take a census of}\lebnote{Literally “Lift up the head of”} them in the midst of the \textit{Israelites}.\lebnote{Literally “sons/children of Israel”}
\verse You will \textit{appoint}\lebnote{Or “count,” or “summon,” or “enroll”} them over the tabernacle of the testimony,\lebnote{Other modern translations read “tabernacle of the covenant”} over all its vessels, and over all that belongs to it. They will carry the tabernacle and all its vessels, and they will care for it; and they will camp around the tabernacle.
\verse And when the tabernacle is set out, the Levites will \textit{take it down},\lebnote{Literally “lower it”} and when encamping the tabernacle the Levites will set it up; the stranger\lebnote{Or “outsider”} that approaches it will be put to death.
\verse The \textit{Israelites}\lebnote{Literally “sons/children of Israel”} will encamp, each in their own camp, and each by their\lebnote{Hebrew “his” or “its”} own banner according to their divisions.
\verse But the Levites will encamp around the tabernacle of the testimony,\lebnote{Other modern translations read “tabernacle of the covenant”} and there will not be wrath on the community of the \textit{Israelites};\lebnote{Literally “sons/children of Israel”} and the Levites will keep the requirements of the tabernacle of the testimony.”\lebnote{Other modern translations read “tabernacle of the covenant”}
\verse And the \textit{Israelites}\lebnote{Literally “sons/children of Israel”} did thus; they did everything that Adonai commanded Moses.
\end{biblechapter}

\begin{biblechapter} % Numbers 2
\verseWithHeading{The Arrangement of the Camps} Adonai spoke to Moses and Aaron, saying,
\verse “The \textit{Israelites}\lebnote{Literally “sons/children of Israel”} will encamp each with his standard, with a banner according to \textit{their families};\lebnote{Literally “the house of their fathers”} they will encamp around the tent of assembly.
\verse The ones who encamp on the eastern side, toward the sunrise, will be of the standard of the camp of Judah according to their divisions; and the leader of the descendants of Judah will be Nahshon son of Amminadab,
\verse and his division and \textit{the ones counted}\lebnote{Literally “the ones counted of them,” or “the ones mustered of them”} are seventy-four thousand six hundred.
\verse And the ones who encamp next to him will be the tribe of Issachar. And the leader of the descendants of Issachar will be Nethanel son of Zuar,
\verse and his division are fifty-four thousand four hundred.
\verse For the tribe of Zebulun: the leader of the descendants of Zebulun will be Eliab son of Helon,
\verse and his division and the \textit{ones counted}\lebnote{Literally “the ones counted of him,” or “the ones mustered of him”} are fifty-seven thousand four hundred.
\verse All those counted from the camp of Judah are one hundred and eighty-six thousand four hundred. They will set out first according to their divisions.
\verse “The standard of the camp of Reuben will be to the south according to their divisions. The leader of the descendants will be Elizur son of Shedeur.
\verse And his division and the \textit{ones counted}\lebnote{Literally “the ones counted of him,” or “the ones mustered of him”} are forty-six thousand five hundred.
\verse Those encamped next to him will be the tribe of Simeon. The leader of the descendants of Simeon will be Shelumiel son of Zurishaddai.
\verse And his division and the \textit{ones counted}\lebnote{Literally “the ones counted of them,” or “the ones mustered of them”} are fifty-nine thousand three hundred.
\verse For the tribe of Gad: the leader of the descendants of Gad will be Eliasaph son of Reuel.
\verse And his division and the \textit{ones counted}\lebnote{Literally “the ones counted of them,” or “the ones mustered of them”} are forty-five thousand six hundred and fifty.
\verse All \textit{those counted}\lebnote{Or “those mustered”} from the camp of Reuben are one hundred and fifty-one thousand four hundred and fifty. They will set out second according to their divisions.
\verse “The tent of assembly the camp of the Levites will set out in the midst of the camps; they will set out just as they encamped, \textit{each according to their standards}.\lebnote{Literally “each man on his hand according to their standards”}
\verse “The standard of the camp of Ephraim according to their divisions will be to the west. The leader of the descendants of Ephraim will be Elishama son of Ammihud.
\verse And his division and the \textit{ones counted}\lebnote{Literally “the ones counted of them,” or “the ones mustered of them”} are forty thousand five hundred.
\verse The tribe of Manasseh will be next to him. The leader of the descendants of the tribe of Manasseh will be Camaliel son of Pedahzur.
\verse And his division and the \textit{ones counted}\lebnote{Literally “the ones counted of them,” or “the ones mustered of them”} are thirty-two thousand two hundred.
\verse For the tribe of Benjamin: the leader of the descendants of Benjamin will be Abidan son of Gideoni.
\verse And his division and the \textit{ones counted}\lebnote{Literally “the ones counted of them,” or “the ones mustered of them”} are thirty-five thousand four hundred.
\verse All \textit{those counted}\lebnote{Or “those mustered”} from the camp of Ephraim are one hundred and eighty thousand one hundred. They will set out third according to their divisions.
\verse “The standard of the camp of Dan according to their divisions will be to the west. The leader of the descendants of Dan will be Ahiezer son of Ammishaddai.
\verse And his division and the \textit{ones counted}\lebnote{Literally “the ones counted of them,” or “the ones mustered of them”} are sixty-two thousand seven hundred.
\verse Those encamped next to him will be the tribe of Asher. The leader of the descendants of Asher will be Pagiel son of Ocran.
\verse And his division and the \textit{ones counted}\lebnote{Literally “the ones counted of them,” or “the ones mustered of them”} are forty-one thousand five hundred.
\verse For the tribe of Naphtali: the leader of the descendants of Naphtali will be Ahira son of Enan.
\verse And his division and the \textit{ones counted}\lebnote{Literally “the ones counted of them,” or “the ones mustered of them”} are fifty-three thousand four hundred.
\verse All the \textit{ones counted}\lebnote{Literally “the ones counted of them,” or “the ones mustered of them”} from the camp of Dan are one hundred and fifty-seven thousand six hundred. They will set out \textit{last}\lebnote{Literally “from behind”} according to their divisions.”
\verse These were the ones counted of the \textit{Israelites}\lebnote{Literally “sons/children of Israel”} according to \textit{their families};\lebnote{Literally “the house of their fathers”} all those counted from the camps according to their divisions were six hundred and three thousand five hundred.
\verse The Levites were not counted in the midst of the \textit{Israelites},\lebnote{Literally “sons/children of Israel”} just as Adonai commanded Moses.
\verse And the \textit{Israelites}\lebnote{Literally “sons/children of Israel”} did everything that Adonai commanded Moses. They encamped according to their standards, and they\lebnote{Hebrew “each one,” or “each man”} set out each one according to their clans\lebnote{Hebrew “his clans”} among \textit{their families}.\lebnote{Literally “the house of his fathers”}
\end{biblechapter}

\begin{biblechapter} % Numbers 3
\verseWithHeading{Aaron’s Sons} These are the genealogies of Aaron and Moses \textit{at the time}\lebnote{Literally “on a day”} when Adonai spoke to Moses on Mount Sinai.
\verse These are the names of the descendants of Aaron: Nadab the firstborn, Abihu, Eleazar, and Ithamar.
\verse These are the names of the descendants of Aaron, the priests, the anointed ones whom \textit{he consecrated as priests}.\lebnote{Literally “he filled their hands to serve as a priest”}
\verse Nadab and Abihu died \textit{before Adonai}\lebnote{Literally “before the face of Adonai”} when they presented a strange fire \textit{before Adonai}\lebnote{Literally “before the face of Adonai”} in the desert of Sinai, and \textit{they had no children}.\lebnote{Literally “sons were not for them”} Eleazar and Ithamar served as priest during the presence of Aaron their father.
\verse Adonai spoke to Moses, saying,
\verse “Bring near the tribe of Levi, and \textit{set the tribe}\lebnote{Literally “cause it to stand”} \textit{before Aaron}\lebnote{Literally “in the presence of Aaron”} the priest, and they will minister to him.
\verse They shall observe his duties and the duties of the entire community before the tent of assembly, to do the work of the tabernacle.
\verse And they will keep all the vessels of the tent of assembly and the responsibilities of the \textit{Israelites},\lebnote{Literally “sons/children of Israel”} to do the work\lebnote{Or “service”} of the tabernacle.
\verse You will give the Levites to Aaron and to his descendants; they are surely assigned to him from among the \textit{Israelites}.\lebnote{Literally “sons/children of Israel”}
\verse But you will count Aaron and his descendants; they will keep their priesthood, and the stranger\lebnote{Or “outsider”} who approaches will be put to death.”
\verse Adonai spoke to Moses saying,
\verse “I myself receive the Levites from the midst of the \textit{Israelites}\lebnote{Literally “sons/children of Israel”} in the place of all the firstborn of the offspring of the womb from the \textit{Israelites}.\lebnote{Literally “sons/children of Israel”} The Levites will be mine
\verse because all the firstborn are mine; on the day of my killing all the firstborn in the land of Egypt, I consecrated for myself all the firstborn in Israel, \textit{both humankind and animal};\lebnote{Literally “from humankind to animal”} they will be mine. I am Adonai.”
\verse Adonai spoke to Moses in the desert of Sinai, saying,
\verse “Muster\lebnote{Or “count,” or “summon,” or “enroll”} the descendants of Levi according to \textit{their families},\lebnote{Literally “the house of their fathers”} according to their clans. You will count every male from \textit{one month}\lebnote{Literally “the son of a month”} and above.”
\verse So Moses mustered\lebnote{Or “counted,” or “summoned,” or “enrolled”} them according to the \textit{command of Adonai},\lebnote{Literally “mouth of Adonai”} just as he commanded.
\verse These were the sons of Levi according to their names: Gershon, Kohath, and Merari.
\verse And these are the names of the sons of Gershon according to their clans: Libni and Shimei.
\verse And the sons of Kohath according to their clans: Amram, Izhar, Hebron, and Uzziel.
\verse The sons of Merari according to their tribes: Mahli and Mushi. These are the clans of the Levites according to \textit{their families}.\lebnote{Literally “the house of their fathers”}
\verse \textit{To Gershon belonged}\lebnote{Literally “To Gershon was”} the clan of the Libnites and the clan of the Shimeites; these are the clans of the Gershonites.
\verse The \textit{ones counted}\lebnote{Literally “the ones counted of them,” or “the ones mustered of them”} according to the number of every male from \textit{one month}\lebnote{Literally “the son of a month and above”} and above were seven thousand five hundred.
\verse The clans of the Gershonites will camp behind the tabernacle to the west,
\verse and the leader of \textit{the family}\lebnote{Literally “the house of his father”} of the Gershonites is Eliasaph son of Lael.
\verse And the responsibility of the descendants of Gershon in the tent of assembly is the tabernacle, and the tent covering it and the curtain of the doorway of the tent of the assembly,
\verse and the hangings of the courtyard and the curtain of the doorway of the courtyard that is around the tabernacle and the altar, and its ten cords, all of its use.
\verse \textit{To Kohath belonged}\lebnote{Literally “For Kohath was”} the clan of Amramites,\lebnote{Hebrew “Amramite”} the clan of the Izharites,\lebnote{Hebrew “Izharite”} the clan of the Hebronites,\lebnote{Hebrew “Hebronite”} and the clan of the Uzzielites;\lebnote{Hebrew “Uzzielite”} these were the clans of the Kohathites.\lebnote{Hebrew “Kohathite”}
\verse According to the number of every male from \textit{one month}\lebnote{Literally “the son of a month”} and above there were eight thousand six hundred keeping the responsibility of the sanctuary.
\verse The clan of the descendants of Kohath will encamp on the side of the tabernacle to the south.
\verse The leader of \textit{his family}\lebnote{Literally “the house of his father”} according to the clans of the Kohathites\lebnote{Hebrew “Kohathite”} is Elizaphan the son of Uzziel.
\verse Their responsibility was the ark, the table, the lampstand, the altar, and the vessels of the sanctuary, with which they ministered, and the curtain, and all of its use.
\verse The \textit{chief of the leaders}\lebnote{Literally “the leader of leaders”} of the Levites\lebnote{Hebrew “Levite”} was Eleazar son of Aaron the priest who had oversight of those keeping the responsibility of the sanctuary.
\verse \textit{To Merari belonged}\lebnote{Literally “for Merari was”} the clan of Mahlites\lebnote{Hebrew “Mahlite”} and the clan of the Mushites:\lebnote{Hebrew “Mushite”} these are the clans of Merari.
\verse The \textit{ones counted}\lebnote{Literally “the ones counted of them,” or “the ones mustered of them”} according to the number of every male from \textit{one month}\lebnote{Literally “the son of a month”} and above were six thousand two hundred.
\verse The leader of \textit{the family}\lebnote{Literally “the house of his father”} according to the clans of Merari is Zuriel son of Abihail; they will encamp of the side of the tabernacle to the north.
\verse The responsibility of the sons of Merari was the supervision of the frames of the tabernacle, its bars, pillars, bases, and all its vessels and all its service,
\verse and the pillars around the courtyard, and their bases, pegs, and cords.
\verse Those encamped before the tabernacle to the east—before the tent of assembly to the east—were Moses and Aaron and his sons; they will keep the responsibility of the sanctuary \textit{for the Israelites};\lebnote{Literally “for the responsibility of the children of Israel”} and the stranger\lebnote{Or “outsider”} who approaches will be put to death.
\verse All those counted from the Levites whom Moses and Aaron mustered\lebnote{Or “counted,” or “summoned,” or “enrolled”} according to \textit{the word of Adonai},\lebnote{Literally “the mouth of Adonai”} according to their clans, every male from \textit{one month}\lebnote{Literally “the son of a month”} and above were twenty-two thousand.
\verse And Adonai said to Moses, “Muster every firstborn male from the \textit{Israelites}\lebnote{Literally “sons/children of Israel”} from \textit{one month}\lebnote{Literally “the son of a month”} and above and \textit{count}\lebnote{Literally “and lift up the number of”} their names.
\verse And you will receive the Levites for me—I am Adonai—in the place of all the firstborn among the \textit{Israelites},\lebnote{Literally “sons/children of Israel”} and the animals\lebnote{Hebrew “animal”} of the Levites in the place of all the firstborn among the animals\lebnote{Hebrew “animal”} among the \textit{Israelites}.”\lebnote{Literally “sons/children of Israel”}
\verse So Moses mustered\lebnote{Or “counted” or “summoned” or “enrolled”} all the firstborn among the \textit{Israelites}\lebnote{Literally “sons/children of Israel”} just as Adonai commanded him.
\verse And all the firstborn males\lebnote{Hebrew “male”} among the number of names from \textit{one month}\lebnote{Literally “the son of a month”} and above, the \textit{ones counted},\lebnote{Literally “the ones counted of them,” or “the ones mustered of them”} were twenty-two thousand two hundred and seventy-three.
\verse Adonai spoke to Moses, saying,
\verse “Receive the Levites in the place of all the firstborn among the \textit{Israelites},\lebnote{Literally “sons/children of Israel”} and the animals of the Levites in the place of their animals; the Levites will be mine. I am Adonai.
\verse And the ransom of the two hundred and seventy-three of the firstborn of the \textit{Israelites}\lebnote{Literally “sons/children of Israel”} who are excessive over the Levites,
\verse you will receive five shekels a person, in the sanctuary shekel; you will collect twenty gerahs\lebnote{Hebrew “gerah”} per shekel.
\verse You will give the money to Aaron, and to his sons the ransom of the ones who are excessive among them.”
\verse And Moses received the money of the redemption from the ones who were excessive from those redeemed of the Levites.
\verse From the firstborn of the \textit{Israelites}\lebnote{Literally “sons/children of Israel”} he took the money, one thousand three hundred and sixty-five shekels, in the sanctuary shekel.
\verse And Moses gave the money of the ransom to Aaron and to his sons according to the \textit{word}\lebnote{Literally “mouth”} of Adonai just as Adonai commanded Moses.
\end{biblechapter}

\begin{biblechapter} % Numbers 4
\verseWithHeading{The Census of the Kohathites} Adonai spoke to Moses and Aaron, saying,
\verse “\textit{Take a census}\lebnote{Literally “and lift up the number of”} of the descendants\lebnote{Or “sons”} of Kohath from the midst of the descendants\lebnote{Or “sons”} of Levi, according to their clans and \textit{their families},\lebnote{Literally “the house of their fathers”}
\verse from \textit{thirty years old}\lebnote{Literally “a son of thirty years”} and above, up to \textit{fifty years old},\lebnote{Literally “a son of fifty years”} everyone who comes to the service to do the work in the tent of assembly.
\verse This is the work of the descendants\lebnote{Or “sons”} of Kohath in the tent of assembly, concerning the holiness of the sanctuary:
\verse When setting out the camp, Aaron and his sons will go and lower the curtain of the covering and cover with it the ark of the testimony.\lebnote{Or “the ark of the covenant”}
\verse They will put on it a covering of \textit{fine leather},\lebnote{Literally “the hide of a sea cow”} and they will spread a cloth of perfect blue over it, and they will place its poles.
\verse And over the table of the presence they will spread out a blue cloth and put on it the plates, dishes, and libation bowls, and the pitchers of the libation; and the bread of continuity will be on it.
\verse They will spread over it a scarlet cloth, and they will cover it with a covering of \textit{fine leather},\lebnote{Literally “the hide of a sea cow”} and they will place its poles.
\verse They will take a blue cloth and cover the lampstand for the light source, its lamps, a pair of its tongs, its small pans, and all the vessels of its oil with which they attend to it.
\verse They will put it and all its vessels inside a covering of \textit{fine leather}\lebnote{Literally “the hide of a sea cow”} and put it on the carrying frame.
\verse Over the altar of gold they will spread a blue cloth, and they will cover it with a covering of \textit{fine leather}\lebnote{Literally “the hide of a sea cow”} and place its poles.
\verse They will take all the vessels of the cultic service with which they serve in the sanctuary and put them on a blue cloth, and they will cover them with a covering of \textit{fine leather};\lebnote{Literally “the hide of a sea cow”} and they will put them on the carrying frame.
\verse They will remove the fat-soaked ashes from the altar and spread a purple cloth over it;
\verse they will put on it all the vessels with which they serve, the fire pans, forks, shovels, and bowls—all the vessels of the altar. They will spread on it a covering of \textit{fine leather};\lebnote{Literally “the hide of a sea cow”} and they will place its poles.
\verse And when Aaron and his sons have finished covering the sanctuary and all the vessels of the sanctuary when the camp sets out, the descendants\lebnote{Or “sons”} of Kohath will come after to carry these, but they must not touch the sanctuary, or they will die. These are the load of the descendants\lebnote{Or “sons”} of Kohath in the tent of assembly.
\verse “Eleazar son of Aaron the priest is to supervise the oil of the light source, the incense, \textit{the regular grain offering},\lebnote{Literally “the grain offering of continuity”} the oil of anointment, the supervision of all the tabernacle and all that is in it, in the sanctuary and in its vessels.”
\verse Adonai spoke to Moses and Aaron, saying,
\verse “You must not cut off the tribe of the clan of the Kohathites\lebnote{Hebrew “Kohathite”} from the midst of the Levites.
\verse Do this to them and they will live and not die when they come near the most holy things. Aaron and his sons will go and appoint them, each one to his task and burden.
\verse But they must not go and look \textit{for a moment}\lebnote{Literally “as devouring”} at the holy objects.”
\verse Adonai spoke to Moses and Aaron, saying,
\verse “\textit{Take a census}\lebnote{Literally “And lift up the number of”} of the descendants\lebnote{Or “sons”} of Gershon also, according to \textit{their families}\lebnote{Literally “the house of their fathers”} and clans,
\verse from \textit{those twenty years old}\lebnote{Literally “a son of twenty years”} and above until \textit{fifty years old};\lebnote{Literally “a son of fifty years”} you will muster\lebnote{Or “count,” or “summon,” or “enroll”} them, all who come to help to do the work of the tent of assembly.
\verse This is the work of the clans of the Gershonites:\lebnote{Hebrew “Gershonite”} to serve and to carry.
\verse They will carry the curtains of the tabernacle and the tent of assembly and its covering and the covering of \textit{fine leather},\lebnote{Literally “the hide of a sea cow”} which \textit{is on top of it},\lebnote{Literally “is above upon it”} and the curtain of the doorway of the tent of assembly,
\verse and the curtains of the courtyard, and the covering for the doorway of the gate of the courtyard, which is all around on the tabernacle and altar, and their cords and all the vessels of their work; and all that is done to them they will do.
\verse And all the work of the descendants\lebnote{Or “sons”} of the Gershonites\lebnote{Hebrew “Gershonite”} will be at \textit{the command}\lebnote{Literally “the mouth”} of Aaron and his sons, for all they are to carry and for all their work, and you will appoint to them responsibility for all they are to carry.
\verse This is the work of the clan of the descendants\lebnote{Or “sons”} of the Gershonites\lebnote{Hebrew “Gershonite”} in the tent of assembly, and their responsibility lies \textit{under the direction}\lebnote{Literally “in the hand”} of Ithamar son of Aaron the priest.
\verse “For the descendants\lebnote{Or “sons”} of Merari according to their clans, according to \textit{their families},\lebnote{Literally “the house of their fathers”} you will muster\lebnote{Or “count,” or “summon,” or “enroll”} them;
\verse from \textit{those thirty years old}\lebnote{Literally “a son of thirty years”} and above until \textit{fifty years old};\lebnote{Literally “a son of fifty years”} you will muster\lebnote{Or “count,” or “summon,” or “enroll”} them, all who come to do the work of the tent of assembly.
\verse And this is the responsibility of \textit{those who are to carry},\lebnote{Literally “them carrying”} all their work in the tent of assembly: the frames of the tabernacle and its bars, pillars, and bases,
\verse and the pillars of the courtyard all around, and their bases, pegs, and cords, with all their vessels and for all their work. You will appoint by name the vessels that they are responsible to carry.
\verse This is the work of the clan of the descendants\lebnote{Or “sons”} of Merari, for all their work in the tent of assembly \textit{under the direction}\lebnote{Literally “in the hand”} of Ithamar son of Aaron the priest.”
\verse And Moses and Aaron mustered the leaders of the community according to the house of \textit{their families},\lebnote{Literally “the house of their fathers”}
\verse from \textit{those thirty years old}\lebnote{Literally “a son of thirty years”} and above until \textit{fifty years old};\lebnote{Literally “a son of fifty years”} everyone who comes to the service to work in the tent of assembly,
\verse the \textit{ones counted}\lebnote{Literally “the ones counted of them,” or “the ones mustered of them”} were two thousand seven hundred and fifty.
\verse These were those counted of the clans of the Kohathites,\lebnote{Hebrew “Kohathite”} everyone who served in the tent of assembly whom Moses and Aaron mustered\lebnote{Or “counted,” or “summoned,” or “enrolled”} according to \textit{the command}\lebnote{Literally “the mouth”} of Adonai by the hand of Moses.\lebnote{Or “through Moses”}
\verse And the descendants\lebnote{Or “sons”} of Gershon counted according to their clans and according to \textit{their families},\lebnote{Literally “the house of their fathers”}
\verse from \textit{those thirty years old}\lebnote{Literally “a son of thirty years”} and above until \textit{fifty years old},\lebnote{Literally “a son of fifty years”} everyone who comes to the service to work in the tent of assembly;
\verse the \textit{ones counted},\lebnote{Literally “the ones counted of them,” or “the ones mustered of them”} according to their clans, according to \textit{their families},\lebnote{Literally “the house of their fathers”} were two thousand six hundred and thirty.
\verse These were those counted of the clans of the descendants\lebnote{Or “sons”} of Gershon, everyone who serves in the tent of assembly whom Moses and Aaron mustered\lebnote{Or “counted,” or “summoned,” or “enrolled”} according to \textit{the command}\lebnote{Literally “the mouth”} of Adonai.
\verse Those counted of the clans of the descendants\lebnote{Or “sons”} of Merari according to their clans, according to \textit{their families},\lebnote{Literally “the house of their fathers”}
\verse from \textit{those thirty years old}\lebnote{Literally “thirty years and above”} and above until \textit{fifty years old},\lebnote{Literally “a son of fifty years”} everyone who comes to the service to work in the tent of assembly,
\verse the \textit{ones counted},\lebnote{Literally “the ones counted of them,” or “the ones mustered of them”} according to their clans, were three thousand two hundred.
\verse These were those counted of the clans of the descendants\lebnote{Or “sons”} of Merari, whom Moses and Aaron mustered\lebnote{Or “counted,” or “summoned,” or “enrolled”} according to \textit{the command}\lebnote{Literally “the mouth”} of Adonai by the hand of Moses.\lebnote{Or “through Moses”}
\verse All those counted of the Levites whom Moses and Aaron and all the leaders of Israel mustered\lebnote{Or “counted,” or “summoned,” or “enrolled”} according to their clans, according to \textit{their families},\lebnote{Literally “the house of their fathers”}
\verse from \textit{those thirty years old}\lebnote{Literally “a son of thirty years”} and above until \textit{fifty years old},\lebnote{Literally “a son of fifty years”} everyone who comes to the service to do the work of the service carrying in the tent of assembly,
\verse the \textit{ones counted}\lebnote{Literally “the ones counted of them,” or “the ones mustered of them”} were eighty thousand five hundred and eighty.
\verse According to \textit{the command}\lebnote{Literally “the mouth”} of Adonai by the hand of Moses\lebnote{Or “through Moses”} \textit{they were mustered},\lebnote{Hebrew “he mustered”} each man according to his service and according to their\lebnote{Hebrew “his”} service and according to their\lebnote{Hebrew “his”} burden; and so they were counted by him just as Adonai commanded Moses.
\end{biblechapter}

\begin{biblechapter} % Numbers 5
\verseWithHeading{Rules Concerning Those Unclean} Adonai spoke to Moses, saying,
\verse “Command the \textit{Israelites}:\lebnote{Literally “sons/children of Israel”} they must send everyone from the camp who is afflicted with a rash,\lebnote{The precise meaning is uncertain; many modern translations suggest “leprosy”} everyone with a fluid discharge, and everyone unclean through contact with a corpse.
\verse You will send away \textit{both male and female};\lebnote{Literally “from male until female”} you will send them \textit{outside the camp}.\lebnote{Literally “to an outside place of the camp”} They must not make unclean their camps where I am dwelling in their midst.”
\verse So the \textit{Israelites}\lebnote{Literally “sons/children of Israel”} did so. They sent them away \textit{outside the camp};\lebnote{Literally “to an outside place of the camp”} just as Adonai spoke to Moses, so did the \textit{Israelites}.\lebnote{Literally “sons/children of Israel”}
\verseWithHeading{Rules of Restitution} Adonai spoke to Moses, saying,
\verse “Speak to the \textit{Israelites}:\lebnote{Literally “sons/children of Israel”} ‘When a man or woman \textit{commits}\lebnote{Literally “does”} any of the sins of humankind by acting unfaithfully, it is a sin against Adonai, and that person will be guilty;
\verse they will confess their sin that they did and will make restitution for their\lebnote{Hebrew “his”} guilt by adding a fifth \textit{to it}\lebnote{Literally “on top of it”} and giving it \textit{to whomever was wronged}.\lebnote{Literally “to whomever he was guilty”}
\verse But if the man does not have a redeemer to make restitution to him for the reparation, the reparation is to be given to Adonai for the priest, in addition to the ram of atonement by which atonement is made for him.
\verse And every contribution of all the holy objects of the \textit{Israelites}\lebnote{Literally “sons/children of Israel”} that they bring to the priest for him will be his.
\verse The holy objects of a man will be for him;\lebnote{That is, the priest} whatever he gives to the priest will be for him.’ ”
\verseWithHeading{Rules Concerning an Unfaithful Wife} Adonai spoke to Moses, saying,
\verse “Speak to the \textit{Israelites}\lebnote{Literally “sons/children of Israel”} and say to them, ‘If any man’s wife goes astray and acts unfaithfully to him,
\verse and a man sleeps with her and ejaculates and it is hidden from the eyes of her husband and she is concealed, although she is defiled, and there is no witness against her and she was not caught,
\verse if a spirit of jealousy comes over him, and he is jealous of his wife and she is defiled; or if a spirit of jealousy comes over him and he is jealous of his wife but she is not defiled,
\verse he will bring his wife to the priest. And he will bring her offering for her, one-tenth of an ephah of flour. He will not pour oil on it, and he will not put frankincense on it because it is a grain offering of jealousy, a grain offering of remembering,\lebnote{Or “memorial”} a reminding of guilt.
\verse “ ‘Then the priest will bring her near and present her \textit{before}\lebnote{Literally “before the face of”} Adonai;
\verse the priest will take holy waters in a clay vessel, and from the dust that is on the floor of the tabernacle, and the priest will put it into the waters.
\verse And the priest will present the woman \textit{before}\lebnote{Literally “before the face of”} Adonai, and he will uncover the head of the woman; he will then put in her hands the grain offering of the remembering—which is the grain offering of jealousy—and in the hand of the priest will be the waters of bitterness that brings a curse.
\verse Then the priest will make her swear an oath, and he will say to the woman, “If a man has not slept with you, and if you have not had an impurity affair under your husband, go unpunished from the waters of bitterness that brings this curse.
\verse But if you have had an affair under your husband, and if you are defiled and a man other than your husband had intercourse with you,”
\verse the priest will make the woman swear an oath of the sworn oath of the curse, the priest will say to the woman, “May Adonai give you a curse and a sworn oath in the midst of your people with Adonai making\lebnote{Hebrew “giving”} your hip fall away\lebnote{Or “waste away”} and your stomach swollen;
\verse and these waters that bring a curse will go into your intestines to cause your womb to swell and to make your hip fall away.”\lebnote{Or “waste away”} And the women will say, “Amen. Amen.”
\verse “ ‘And the priests will write these curses on the scroll, and he will wipe them off into the waters of the bitterness.
\verse He will make the woman drink the waters of the bitterness that brings\lebnote{Hebrew “bring”} a curse, and the waters of bitterness that bring a curse will go into her.
\verse The priest will take the grain offering of jealousy from the hand of the woman, and he will wave the grain offering \textit{before Adonai},\lebnote{Literally “before the face of Adonai”} and he will present it to the altar;
\verse the priest will grasp her memorial offering from the grain offering, and he will turn it into smoke on the altar, and afterward he will make the woman drink the waters.
\verse When he has made her drink the waters, it will come about, if she has defiled herself and acted unfaithfully to her husband and the waters of bitterness that bring a curse go into her and her stomach swells and her hip falls away,\lebnote{Or “wastes away”} the woman will be as a curse in the midst of her people.
\verse And if the woman is not defiled, and she is pure, she will go unpunished and be able to conceive children.
\verse “ ‘This is the regulation of jealousy, when a woman has an affair under her husband and she is defiled,
\verse or when a spirit of jealousy comes over a man and he is jealous of his wife, he will present the woman \textit{before Adonai},\lebnote{Literally “before the face of”}and the priest will do to her all of this law.
\verse The man will go unpunished from guilt, and the woman, she will bear her guilt.’ ”
\end{biblechapter}

\begin{biblechapter} % Numbers 6
\verseWithHeading{Rules Concerning Nazirites} Adonai spoke to Moses, saying,
\verse “Speak to the \textit{Israelites}\lebnote{Literally “sons/children of Israel”} and say to them, ‘When a man or a woman takes a special vow, a vow of a \textit{Nazirite},\lebnote{Literally “one separated”} to keep separate for Adonai,
\verse he will abstain from wine and fermented drink; he will not drink wine vinegar or vinegar of fermented drink; he will not drink the fruit juice of grapes or eat fresh or dry grapes.
\verse All the days of his separation\lebnote{That is, “the Nazirite”} you will not eat from anything that is made from the grapevine, from sour grapes to the skin of grapes.
\verse “ ‘All the days of the vow of his separation a razor will not pass over his head. Until fulfilling the days that he separated himself to Adonai he will be holy and grow long the locks of the hair of his head.
\verse “ ‘All the days of keeping himself separated for Adonai he will not go to a person who is dead;
\verse for even his father, mother, brother, or sister he will not make himself unclean \textit{by their death},\lebnote{Literally “with their dying”} because the separation to his God is on his head.
\verse He will be holy for Adonai all the days of his separation.\lebnote{That is, “the Nazirite”}
\verse “ ‘If someone dies suddenly and makes the head of his separation\lebnote{That is, “the Nazirite”} unclean, he will shave off the hair of his head on the day of his cleansing; on the seventh day he will shave it off.
\verse On the eighth day he will bring two turtledoves or two \textit{young pigeons}\lebnote{Literally “the sons of doves”} to the priest by the doorway of the tent of assembly,
\verse and the priest will offer one for a sin offering and one for a burnt offering, and he will make atonement for him because he sinned concerning the corpse. He will consecrate his head on that day.
\verse He will rededicate to Adonai the days of his separation\lebnote{That is, “the Nazirite”} and bring a ram-lamb \textit{in its first year}\lebnote{Literally “a son of his year”} as a guilt offering. The former days of his vow will fall away because his separation was defiled.
\verse “ ‘This is the regulation of the Nazirite for the day of the fulfilling of the days of his separation: one will bring him to the doorway of the tent of the assembly.
\verse He will present his offering to Adonai, one ram-lamb \textit{in its first year}\lebnote{Literally “a son of his year”} without defect as a burnt offering, and one ewe-lamb \textit{in its first year}\lebnote{Literally “a daughter of his year”} without defect as a sin offering, and one ram without defect as a fellowship offering;
\verse and a basket of unleavened bread, finely milled flour of ring-shaped bread cakes mixed with oil, and wafers of unleavened bread smeared with oil, and their grain offering and their libations.
\verse The priest will present \textit{before Adonai}\lebnote{Or “before the face of Adonai”} and offer his sin offering, his burnt offering;
\verse he will offer a ram as a sacrifice of a fellowship offering to Adonai, in addition to the basket of the unleavened bread; the priest will offer his grain offering and his libation.
\verse The Nazirite will shave off the hair of \textit{his consecrated head}\lebnote{Literally “the head of his separation”} at the doorway of the tent of assembly, and he will take the hair of \textit{his consecrated head},\lebnote{Literally “the head of his separation”} and he will put it on the fire that is beneath the sacrifice of the fellowship offering.
\verse The priest will take the shoulder from the ram that is boiled, and one ring-shaped bread cake of unleavened bread from the basket, and one wafer of unleavened bread, and he will put them on the hands of the Nazirite after he has shaved \textit{his consecrated head}.\lebnote{Literally “his separation”}
\verse The priest will wave them as a wave offering \textit{before the presence of}\lebnote{Literally “before the face of”} Adonai; they\lebnote{Hebrew “it”} are a holy object to the priest, in addition to the breast section of the wave offering, and in addition to the upper thigh of the contribution; and afterward the Nazirite may drink wine.
\verse “ ‘This is the regulation of the Nazirite who has made a vow of his offering to Adonai according to his separation, \textit{in addition to what he can afford}.\lebnote{Literally “except from that which overtakes his hand”} In accordance to the word of his vow that he vowed, he will do, concerning the instruction of his separation.’ ”
\verseWithHeading{The Priestly Blessing} Adonai spoke to Moses, saying,
\verse “Speak to Aaron and his sons, saying, ‘You will bless the \textit{Israelites}:\lebnote{Literally “sons/children of Israel”} You will say to them:
\verse Adonai will bless you 
and keep you;
\verse Adonai will make shine his face on you 
and be gracious to you;
\verse Adonai will lift up his face upon you, 
and he will give you peace.’
\verse And they will put my name on the \textit{Israelites},\lebnote{Literally “sons/children of Israel”} and I will bless them.”
\end{biblechapter}

\begin{biblechapter} % Numbers 7
\verseWithHeading{The Leaders Make Offerings} On the day Moses finished setting up the tabernacle and appointed and consecrated it and all its vessels, and the altar and its vessels, and he appointed them and consecrated them,
\verse the leaders of Israel, the heads of \textit{the families},\lebnote{Literally “the house of their fathers”} presented an offering; they were the leaders of the tribes and were the ones in charge of the counting.
\verse They brought their offering before the presence of\lebnote{Or “before the face of”} Adonai, six covered utility carts and twelve cattle, a utility cart for two of the leaders, and a bull for each; and they presented them \textit{before}\lebnote{Literally “before the face of”} the tabernacle.
\verse Adonai said to Moses, saying,
\verse “\textit{Take them},\lebnote{Literally “Take from them”} and they will be used to do the work of the tent of the assembly; and you will give them to the Levites, each \textit{according to his required service}.”\lebnote{Literally “according to the mouth of his work”}
\verse So Moses took the utility carts and the cattle, and he gave them to the Levites.
\verse Two utility carts and four cattle he gave to the descendants\lebnote{Or “sons”} of Gershon \textit{according to their required service};\lebnote{Literally “according to the mouth of their work”}
\verse and four utility carts and eight cattle he gave to the descendants\lebnote{Or “sons”} of Merari \textit{according to their required service},\lebnote{Literally “according to the mouth of their work”} \textit{under the authority of}\lebnote{Literally “in the hand of”} Ithamar son of Aaron the priest.
\verse But to the descendants\lebnote{Or “sons”} of Kohath he did not give anything because the work of the sanctuary they carried upon them on their shoulders.\lebnote{Hebrew “the shoulder”}
\verse The leaders presented offerings for the dedication of the altar on the day of its anointing, and the leaders presented their offerings \textit{before}\lebnote{Literally “in the presence of”} the altar.
\verse Adonai said to Moses, “\textit{One leader for each day}\lebnote{Literally “One leader for the day one leader for the day”} will present their offering for the dedication of the altar.”
\verse And it happened, the one who presented his offering on the first day was Nahshon son of Amminadab from the tribe of Judah.
\verse His offering was one plate of silver—its weight was one hundred and thirty shekels—and one silver bowl weighing seventy shekels according to the sanctuary shekel, \textit{both of them}\lebnote{Literally “the two of them”} filled with finely milled flour mixed with oil as a grain offering;
\verse one golden dish weighing ten shekels filled with incense;
\verse \textit{one young bull},\lebnote{Literally “one bull, a son of a cattle”} one ram, one male lamb \textit{in its first year}\lebnote{Literally “the son of its year”} as a burnt offering;
\verse one he-goat as a sin offering;
\verse and as a sacrifice of the fellowship offering, two cattle, five rams, five he-goats, and five male lambs \textit{a year old}.\lebnote{Literally “sons of a year”} This was the offering of Nahshon son of Amminadab.
\verse On the second day Nethanel son of Zuar, leader of Issachar, presented an offering.
\verse He presented as his offering one silver plate—its weight one hundred and thirty shekels—and one silver bowl for drinking wine weighing seventy shekels\lebnote{Hebrew “shekel”} according to the sanctuary shekel, \textit{both of them}\lebnote{Literally “the two of them”} filled with finely milled flour mixed with oil as a grain offering.
\verse One dish weighing ten shekels filled with incense;
\verse one \textit{young }\lebnote{Literally “a son of cattle”} bull, one ram, a male lamb \textit{in its first year}\lebnote{Literally “the son of its year”} as a burnt offering;
\verse one he-goat as a sin offering;
\verse and for the sacrifice of the fellowship offering, two cattle, five rams, five he-goats, and five male lambs \textit{in their first year}.\lebnote{Literally “sons of a year”} This was the offering of Nethanel son of Zuar.
\verse On the third day Eliab son of Helon, leader of the descendants\lebnote{Or “sons”} of Zebulun:
\verse his offering was one silver plate—its weight one hundred and thirty shekels—and one silver bowl for drinking wine weighing seventy shekels\lebnote{Hebrew “shekel”} according to the sanctuary shekel, \textit{both of them}\lebnote{Literally “the two of them”} filled with finely milled flour mixed with oil as a grain offering;
\verse \textit{one golden dish weighing ten shekels}\lebnote{Literally “one dish of ten gold”} filled with incense;
\verse one \textit{young}\lebnote{Literally “a son of cattle”} bull, one ram, a male lamb \textit{in its first year}\lebnote{Literally “the son of its year”} as a burnt offering;
\verse one he-goat as a sin offering;
\verse and for the sacrifice of the fellowship offering, two cattle, five rams, five he-goats, and five male lambs \textit{in their first year}.\lebnote{Literally “sons of a year”} This was the offering of Eliab son of Helon.
\verse On the fourth day Elizur son of Shedeur, leader of the descendants\lebnote{Or “sons”} of Reuben:
\verse his offering was one silver plate—its weight one hundred and thirty shekels—and one silver bowl for drinking weighing seventy shekels\lebnote{Hebrew “shekel”} according to the sanctuary shekel, \textit{both of them}\lebnote{Literally “the two of them”} filled with finely milled flour mixed with oil as a grain offering;
\verse \textit{one golden dish weighing ten shekels}\lebnote{Literally “one dish of ten gold”} filled with incense;
\verse one \textit{young}\lebnote{Literally “a son of cattle”} bull, one ram, a male lamb \textit{in its first year}\lebnote{Literally “the son of its year”} as a burnt offering;
\verse one he-goat as a sin offering;
\verse and for the sacrifice of the fellowship offering, two cattle, five rams, five he-goats, and five male lambs \textit{in their first year}.\lebnote{Literally “sons of a year”} This was the offering of Elizur son of Shedeur.
\verse On the fifth day Shelumiel son of Zurishaddai, leader of the descendants\lebnote{Or “sons”} of Simeon:
\verse his offering was one silver plate—its weight one hundred and thirty shekels—and one silver bowl for drinking weighing seventy shekels\lebnote{Hebrew “shekel”} according to the sanctuary shekel, \textit{both of them}\lebnote{Literally “the two of them”} filled with finely milled flour mixed with oil as a grain offering;
\verse \textit{one golden dish weighing ten shekels}\lebnote{Literally “one dish of ten gold”} filled with incense;
\verse one \textit{young}\lebnote{Literally “a son of cattle”} bull, one ram, a male lamb \textit{in its first year}\lebnote{Literally “the son of its year”} as a burnt offering;
\verse one he-goat as a sin offering;
\verse and for the sacrifice of the fellowship offering, two cattle, five rams, five he-goats, and five male lambs \textit{in their first year}.\lebnote{Literally “sons of a year”} This was the offering of Shelumiel son of Zurishaddai.
\verse On the sixth day Eliasaph son of Deuel, leader of the descendants\lebnote{Or “sons”} of Gad:
\verse his offering was one silver plate—its weight one hundred and thirty shekels—and one silver bowl for drinking weighing seventy shekels\lebnote{Hebrew “shekel”} according to the sanctuary shekel, \textit{both of them}\lebnote{Literally “the two of them”} filled with finely milled flour mixed with oil as a grain offering;
\verse \textit{one golden dish weighing ten shekels}\lebnote{Literally “one dish of ten gold”} filled with incense;
\verse one \textit{young}\lebnote{Literally “a son of cattle”} bull, one ram, a male lamb \textit{in its first year}\lebnote{Literally “the son of its year”} as a burnt offering;
\verse one he-goat as a sin offering;
\verse and for the sacrifice of the fellowship offering, two cattle, five rams, five he-goats, and five male lambs \textit{in their first year}.\lebnote{Literally “sons of a year”} This was the offering of Eliasaph son of Deuel.
\verse On the seventh day Elishama son of Ammihud, leader of the descendants\lebnote{Or “sons”} of Ephraim:
\verse his offering was one silver plate—its weight one hundred and thirty shekels—and one silver bowl for drinking weighing seventy shekels\lebnote{Hebrew “shekel”} according to the sanctuary shekel, \textit{both of them}\lebnote{Literally “the two of them”} filled with finely milled flour mixed with oil as a grain offering;
\verse \textit{one golden dish weighing ten shekels}\lebnote{Literally “one dish of ten gold”} filled with incense;
\verse one \textit{young}\lebnote{Literally “a son of cattle”} bull, one ram, a male lamb \textit{in its first year}\lebnote{Literally “the son of its year”} as a burnt offering;
\verse one he-goat as a sin offering;
\verse and for the sacrifice of the fellowship offering, two cattle, five rams, five he-goats, and five male lambs \textit{in their first year}.\lebnote{Literally “sons of a year”} This was the offering of Elishama son of Ammihud.
\verse On the eighth day Gamaliel son of Pedahzur, leader of the descendants\lebnote{Or “sons”} of Manasseh:
\verse his offering was one silver plate—its weight one hundred and thirty shekels—and one silver bowl for drinking weighing seventy shekels according to the sanctuary shekel, \textit{both of them}\lebnote{Literally “the two of them”} filled with finely milled flour mixed with oil as a grain offering;
\verse \textit{one golden dish weighing ten shekels}\lebnote{Literally “one dish of ten gold”} filled with incense;
\verse one \textit{young}\lebnote{Literally “a son of cattle”} bull, one ram, a male lamb \textit{in its first year}\lebnote{Literally “the son of its year”} as a burnt offering;
\verse one he-goat as a sin offering;
\verse and for the sacrifice of the fellowship offering, two cattle, five rams, five he-goats, and five male lambs \textit{in their first year}.\lebnote{Literally “sons of a year”} This was the offering of Gamaliel son of Pedahzur.
\verse On the ninth day Abidan son of Gideoni, leader of the descendants\lebnote{Or “sons”} of Benjamin:
\verse his offering was one silver plate—its weight one hundred and thirty shekels—and one silver bowl for drinking weighing seventy shekels\lebnote{Hebrew “shekel”} according to the sanctuary shekel, \textit{both of them}\lebnote{Literally “the two of them”} filled with finely milled flour mixed with oil as a grain offering;
\verse \textit{one golden dish weighing ten shekels}\lebnote{Literally “one dish of ten gold”} filled with incense;
\verse one \textit{young}\lebnote{Literally “a son of cattle”} bull, one ram, a male lamb \textit{in its first year}\lebnote{Literally “the son of its year”} as a burnt offering;
\verse one he-goat as a sin offering;
\verse and for the sacrifice of the fellowship offering, two cattle, five rams, five he-goats, and five male lambs \textit{in their first year}.\lebnote{Literally “sons of a year”} This was the offering of Abidan son of Gideoni.
\verse On the tenth day Ahiezer son of Ammishaddai, leader of the descendants\lebnote{Or “sons”} of Dan:
\verse his offering was one silver plate—its weight one hundred and thirty shekels—and one silver bowl for drinking weighing seventy shekels\lebnote{Hebrew “shekel”} according to the sanctuary shekel, \textit{both of them}\lebnote{Literally “the two of them”} filled with finely milled flour mixed with oil as a grain offering;
\verse \textit{one golden dish weighing ten shekels}\lebnote{Literally “one dish of ten gold”} filled with incense;
\verse one \textit{young}\lebnote{Literally “a son of cattle”} bull, one ram, a male lamb \textit{in its first year}\lebnote{Literally “the son of its year”} as a burnt offering;
\verse one he-goat as a sin offering;
\verse and for the sacrifice of the fellowship offering, two cattle, five rams, five he-goats, and five male lambs \textit{in their first year}.\lebnote{Literally “sons of a year”} This was the offering of Ahiezer son of Ammishaddai.
\verse On the eleventh day Pagiel son of Ocran, leader of the descendants\lebnote{Or “sons”} of Asher:
\verse his offering was one silver plate—its weight one hundred and thirty shekels—and one silver bowl for drinking weighing seventy shekels\lebnote{Hebrew “shekel”} according to the sanctuary shekel, \textit{both of them}\lebnote{Literally “the two of them”} filled with finely milled flour mixed with oil as a grain offering;
\verse \textit{one golden dish weighing ten shekels}\lebnote{Literally “one dish of ten gold”} filled with incense;
\verse one \textit{young}\lebnote{Literally “a son of cattle”} bull, one ram, a male lamb \textit{in its first year}\lebnote{Literally “the son of its year”} as a burnt offering;
\verse one he-goat as a sin offering;
\verse and for the sacrifice of the fellowship offering, two cattle, five rams, five he-goats, and five male lambs \textit{in their first year}.\lebnote{Literally “sons of a year”} This was the offering of Pagiel son of Ocran.
\verse On the twelfth day Ahira son of Enan, leader of the descendants\lebnote{Or “sons”} of Naphtali:
\verse his offering was one silver plate—its weight one hundred and thirty shekels—and one silver bowl for drinking weighing seventy shekels\lebnote{Hebrew “shekel”} according to the sanctuary shekel, \textit{both of them}\lebnote{Literally “the two of them”} filled with finely milled flour mixed with oil as a grain offering;
\verse \textit{one golden dish weighing ten shekels}\lebnote{Literally “one dish of ten gold”} filled with incense;
\verse one \textit{young}\lebnote{Literally “a son of cattle”} bull, one ram, a male lamb \textit{in its first year}\lebnote{Literally “the son of its year”} as a burnt offering;
\verse one he-goat as a sin offering;
\verse and for the sacrifice of the fellowship offering, two cattle, five rams, five he-goats, and five male lambs \textit{in their first year}.\lebnote{Literally “sons of a year”} This was the offering of Ahira son of Enan.
\verse This was the dedication of the altar on the day of anointing it, from the leaders of Israel: twelve silver plates, twelve silver bowls for drinking wine, twelve golden dishes;
\verse each plate of silver weighed one hundred and thirty shekels, and each bowl for drinking seventy, all the silver of the vessels two thousand four hundred shekels, according to the sanctuary shekel;
\verse the twelve golden dishes filled with incense, each dish weighing ten shekels according to the sanctuary shekel, all the gold of the dishes one hundred and twenty;
\verse all the cattle for the burnt offering twelve bulls, twelve rams, twelve male lambs \textit{in their first year},\lebnote{Literally “sons of a year”} and their grain offering; and twelve he-goats as a sin offering;
\verse and all the cattle of the sacrifice of the fellowship offering twenty-four bulls, sixty rams, sixty he-goats, sixty male lambs \textit{in their first year}.\lebnote{Literally “sons of a year”} These were the dedication of the altar after its anointing.
\verse And when Moses came to the tent of assembly to speak with him,\lebnote{That is, Adonai} he would hear the voice speaking to him from the atonement cover,\lebnote{Some modern translations have “mercy seat” (see, for example, the NRSV, NASB)} which is on the ark of the testimony, from between the two cherubim, and he would speak to him.
\end{biblechapter}

\begin{biblechapter} % Numbers 8
\verseWithHeading{The Seven Lamps} Adonai spoke to Moses, saying,
\verse “Speak to Aaron, and say to him: ‘When you are setting up the lamps, the seven lamps will give light in front of the face of the lampstand.’ ”
\verse And Aaron did so; he set up the lampstand in front of the face of its lamps, just as Adonai commanded Moses.
\verse And this is \textit{how the lampstand was made},\lebnote{Literally “the work of the lampstand”} a hammered-work of gold; from its base up to its blossom,\lebnote{Or “flower”} it was hammered-work according to the pattern that Adonai showed Moses; so he made the lampstand.
\verseWithHeading{Moses Consecrates the Levites} Adonai spoke to Moses, saying,
\verse “Take the Levites from the midst of the \textit{Israelites}\lebnote{Literally “sons/children of Israel”} and purify them.
\verse So you will do to them, to purify them: sprinkle on them waters of purification, and \textit{they will shave their whole body}\lebnote{Literally “they will send a razor on all their body”} and wash their garments.
\verse And they will take a \textit{young bull}\lebnote{Literally “a bull, a son of cattle”} and its grain offering of finely milled flour mixed with oil, and you will take a second \textit{young bull}\lebnote{Literally “a bull, a son of cattle”} as a sin offering.
\verse You will bring the Levites \textit{before}\lebnote{Literally “in the presence of”} the tent of assembly, and you will summon the entire community of the \textit{Israelites}.\lebnote{Literally “sons/children of Israel”}
\verse And you will bring the Levites \textit{before Adonai},\lebnote{Literally “in the presence of Adonai”} and the \textit{Israelites}\lebnote{Literally “sons/children of Israel”} will lay their hands on the Levites,
\verse and \textit{Aaron will offer}\lebnote{Literally “Aaron will wave”} the Levites as a wave offering \textit{before Adonai}\lebnote{Literally “in the presence of Adonai”} from the \textit{Israelites},\lebnote{Literally “sons/children of Israel”} and they will do the work of Adonai.
\verse And the Levites will lay their hands on the head of the one bull and offer it as a sin offering and the other one as a burnt offering to Adonai, to make atonement for the Levites.
\verse And you will present the Levites \textit{before}\lebnote{Literally “in the presence of”} Aaron and \textit{before}\lebnote{Literally “in the presence of”} his sons, and he \textit{will offer}\lebnote{Literally “Aaron will wave”} them as a wave offering to Adonai.
\verse “And you will separate the Levites from the midst of the \textit{Israelites},\lebnote{Literally “sons/children of Israel”} and the Levites will be for me.
\verse And after this the Levites will come to serve at the tent of assembly, and you will purify them, and you \textit{will offer}\lebnote{Literally “Aaron will wave”} them as a wave offering.
\verse For they are given to me exclusively from the midst of the \textit{Israelites}.\lebnote{Literally “sons/children of Israel”} I have taken them for myself in place of the firstborn of every womb, every firstborn from the \textit{Israelites}.\lebnote{Literally “sons/children of Israel”}
\verse For every firstborn among the \textit{Israelites}\lebnote{Literally “sons/children of Israel”} is mine, both humankind and animal. On the day I destroyed every firstborn in the land of Egypt I consecrated them to me,
\verse and I have taken the Levites in the place of every firstborn among the \textit{Israelites}.\lebnote{Literally “sons/children of Israel”}
\verse And I have given the Levites; they are given to Aaron and his sons from the midst of the \textit{Israelites}\lebnote{Literally “sons/children of Israel”} to do the work of the \textit{Israelites}\lebnote{Literally “sons/children of Israel”} in the tent of the assembly and to make atonement for the \textit{Israelites},\lebnote{Literally “sons/children of Israel”} so a plague will not be among the \textit{Israelites}\lebnote{Literally “sons/children of Israel”} when the \textit{Israelites}\lebnote{Literally “sons/children of Israel”} come near the sanctuary.”
\verse And Moses and Aaron and the entire community of the \textit{Israelites}\lebnote{Literally “sons/children of Israel”} did to the Levites; everything that Adonai commanded Moses concerning the Levites, the \textit{Israelites}\lebnote{Literally “sons/children of Israel”} did to them.
\verse And the Levites purified themselves, and they washed their garments, and \textit{Aaron offered them}\lebnote{Literally “Aaron waved them”} as a wave offering \textit{before Adonai};\lebnote{Literally “in the presence of Adonai”} and Aaron made atonement for them to purify them.
\verse After this the Levites came to do their work in the tent of assembly before Aaron and his sons. Just as Adonai commanded Moses concerning the Levities, so they did to them.
\verse Adonai spoke to Moses, saying,
\verse “This is what is for the Levites: \textit{those twenty-five years old}\lebnote{Literally “from a son of twenty-five years”} and above will\lebnote{Hebrew “and he will”} come to help with the service in the work of the tent of assembly;
\verse and \textit{those fifty years old}\lebnote{Literally “from a son of fifty years”} will\lebnote{Hebrew “he will”} return from the service of the work and will serve no longer.
\verse They\lebnote{Hebrew “he”} can attend\lebnote{Or “assist”} their brothers in the tent of assembly to keep their responsibilities, but they\lebnote{Hebrew “he”} will not do work. This is what you will do concerning the Levities and their responsibilities.”
\end{biblechapter}

\begin{biblechapter} % Numbers 9
\verseWithHeading{The Passover} Adonai spoke to Moses in the desert of Sinai, in the second year after they came out from the land of Egypt, in the first month, saying,
\verse “Let the \textit{Israelites}\lebnote{Literally “sons/children of Israel”} observe the Passover at its appointed time.
\verse On the fourteenth day of this month \textit{at twilight}\lebnote{Literally “between the two evenings”} you will perform it at its appointed time according to all its decrees; and according to all its stipulations you will observe it.”
\verse So Moses spoke to the \textit{Israelites}\lebnote{Literally “sons/children of Israel”} to observe the Passover.
\verse And they observed the Passover on the fourteenth day of the month \textit{at twilight}\lebnote{Literally “between the two evenings”} in the desert of Sinai. According to all that Adonai commanded Moses, thus the \textit{Israelites}\lebnote{Literally “sons/children of Israel”} did.
\verse And it happened, men who were unclean \textit{by a dead person}\lebnote{Literally “by a life of a person”} were not able to perform the Passover on that day. And they came \textit{before}\lebnote{Literally “in the presence of”} Moses and Aaron on that day.
\verse And those men said to him, “Although we are unclean \textit{by a dead person},\lebnote{Literally “by a life of a person”} why are we hindered from presenting the offering of Adonai at its appointed time in the midst of the \textit{Israelites}?”\lebnote{Literally “sons/children of Israel”}
\verse Moses said to them, “Stay. I will hear what Adonai commands to you.”
\verse And Adonai spoke to Moses, saying,
\verse “Speak to the \textit{Israelites},\lebnote{Literally “sons/children of Israel”} saying, ‘Each man that is unclean \textit{by a dead person}\lebnote{Literally “by a life of a person”} or is on a far journey, you or your \textit{descendants},\lebnote{Literally “generations”} he will observe the Passover of Adonai.
\verse On the second month on the fourteenth day \textit{at twilight}\lebnote{Literally “between the two evenings”} they will observe it; they will eat it with unleavened bread and bitter plants.
\verse They will leave none of it until morning, and they will not break a bone in it; they will observe it according to every decree of the Passover.
\verse But the man who is clean and not on a journey, and he fails to observe the Passover, that person will be cut off from the people because he did not present the offering of Adonai on its appointed time. That man will bear his guilt.
\verse If an alien dwells with you he will observe the Passover of Adonai according to the decree of the Passover and according to its stipulation; thus you will have one decree for you, for the alien and for the native of the land.’ ”
\verseWithHeading{The Cloud and the Fire} And on a day setting up the tabernacle, the cloud covered the tent of the tabernacle, the tent of the testimony; in the evening it was on the tabernacle as an appearance of fire until morning.
\verse So it was\lebnote{Hebrew “it will be”} continually; the cloud would cover it and the appearance of fire by night.
\verse Whenever the cloud lifted up from on the tent, after that the \textit{Israelites}\lebnote{Literally “sons/children of Israel”} would set out, and in the place where the cloud dwelled, there the \textit{Israelites}\lebnote{Literally “sons/children of Israel”} camped.
\verse On the \textit{command of Adonai}\lebnote{Literally “mouth of Adonai”} the \textit{Israelites}\lebnote{Literally “sons/children of Israel”} would set out, and on the \textit{command of Adonai}\lebnote{Literally “mouth of Adonai”} they encamped; all the days that the cloud dwelled on the tabernacle they encamped.
\verse And when the cloud prolonged on the tabernacle many days the \textit{Israelites}\lebnote{Literally “sons/children of Israel”} kept the kept requirement of Adonai and did not set out.
\verse When the cloud \textit{remained}\lebnote{Literally “it was there”} a number of days on the tabernacle, on the \textit{command of Adonai}\lebnote{Literally “mouth of Adonai”} they encamped; and on the \textit{command of Adonai}\lebnote{Literally “mouth of Adonai”} they set out.
\verse When the cloud \textit{remained}\lebnote{Literally “it was there”} from evening until morning, and the cloud lifted up in the morning, they would set out, or if it remained in the daytime and at night, when the cloud lifted up they set out.
\verse When it was two days, a month, \textit{or a year}\lebnote{Literally “or days”} that the cloud prolonged to dwell on the tabernacle, the \textit{Israelites}\lebnote{Literally “sons/children of Israel”} encamped, and they did not set out; when it lifted up they set out.
\verse On the \textit{command of Adonai}\lebnote{Literally “mouth of Adonai”} they encamped, and on the \textit{command of Adonai}\lebnote{Literally “mouth of Adonai”} they set out. They kept the requirement of Adonai, on the \textit{command of Adonai}\lebnote{Literally “mouth of Adonai”} in the hand of Moses.\lebnote{Or “through Moses”}
\end{biblechapter}

\begin{biblechapter} % Numbers 10
\verseWithHeading{The Silver Trumpets} Adonai spoke to Moses, saying,
\verse “Make yourself two silver trumpets; make them of hammered-work. \textit{You will use them}\lebnote{Literally “They will be for you”} for calling the community and for breaking the camp.
\verse You will blow them, and all the community will assemble to the doorway of the tent of assembly.
\verse But if they blow only one, the leaders, the heads of the thousands of Israel, will assemble to you.
\verse When you will blow a blast, the camps that are camping on the east will set out;
\verse when you blow a second blast, the camps that are camping on the south will set out; they will blow a blast for their journeys.
\verse But when summoning the assembly, you will blow, but you will not signal with a loud noise.
\verse The sons of Aaron, the priests, will blow on the trumpets; this will be an eternal decree for your generations.
\verse If you go to war in your land against the enemy who attacks you, you will signal with a loud noise on the trumpets. You will be remembered \textit{before}\lebnote{Literally “in the presence of”} Adonai your God, and you will be rescued from your enemies.
\verse “And on the day of your joy and in your appointed times, at the beginning of your months, you will blow on the trumpets in addition to your burnt offerings and in addition to the sacrifices of your fellowship offerings. And they will be as a memorial for you \textit{before}\lebnote{Literally “in the presence of”} your God; I am Adonai your God.”
\verseWithHeading{The Israelites Depart from Sinai} And it happened, in the second year, in the second month, on the twentieth of the month the cloud was lifted from upon the tabernacle of the testimony.\lebnote{Some modern translations (e.g., the NRSV) have “tabernacle of the covenant”}
\verse And the \textit{Israelites}\lebnote{Literally “sons/children of Israel”} set out for their journey\lebnote{Hebrew “journeys”} from the desert of Sinai, and the cloud dwelled in the desert of Paran.
\verse They set out \textit{for the first time}\lebnote{Literally “in the beginning”} on the command of Adonai in the hand of Moses.\lebnote{Or “through Moses”}
\verse The standard of the camp of the descendants\lebnote{Or “sons”} of Judah set out for the first time according to their divisions, with Nahshon son of Amminadab over its division.
\verse And Nathanel son of Zuar was over the division of the descendants\lebnote{Or “sons”} of Issachar;
\verse Eliab son of Helon was over the division of the tribe of the descendants\lebnote{Or “sons”} of Zebulun.
\verse The tabernacle was taken down, and the sons of Gershon and the sons of Merari, the bearers of the tabernacle, set out.
\verse And the standard of the camp of Reuben according to their divisions; Elizur son of Shedeur was over their division.
\verse Shelumiel son of Zurishaddai was over the division of the sons of the tribe of Simeon.
\verse Eliasaph son of Deuel was over the division of the tribe of the descendants\lebnote{Or “sons”} of Gad.
\verse The Kohathites, the bearers of the sanctuary, set out, and they set up the tabernacle before they arrived.
\verse And the stand of the camp of the descendants\lebnote{Or “sons”} of Ephraim set out according to their divisions; Elishama son of Ammihud was over its division.
\verse Gamaliel son of Pedahzur was over the division of the tribe of the descendants\lebnote{Or “sons”} of Manasseh.
\verse Abidan son of Gideoni was over the division of the tribe of the descendants\lebnote{Or “sons”} of Benjamin.
\verse Then the standard of the camp of the descendants\lebnote{Or “sons”} of Dan, who formed a rear guard for all the camps, set out according to their divisions; Ahiezer son of Ammishaddai was over its division.
\verse Pagiel son of Ocran was over the division of the tribe of the descendants\lebnote{Or “sons”} of Asher.
\verse Ahira son of Enan was over the division of the tribe of the descendants\lebnote{Or “sons”} of Naphtali.
\verse These were the departures of the \textit{Israelites}\lebnote{Literally “sons/children of Israel”} according to their divisions; and so they set out.
\verse Moses said to Hobab son of Reuel the Midianite, the father-in-law of Moses, “We are setting out to the place that Adonai said, ‘I will give it to you’; go with us, and we will \textit{treat you well}\lebnote{Literally “do good to you”} because \textit{Adonai promised}\lebnote{Literally “Adonai spoke”} good concerning Israel.”
\verse But he said to him, “I will not go. I will only go to my land and to my family.”
\verse He\lebnote{That is, Moses} said, “Please, do not abandon us because you know our encampment in the desert, and \textit{you should be our guide}.\lebnote{Literally “you should be our eyes”}
\verse Moreover, if you go with us, the good that Adonai will do to us we will do to you.”
\verse And so they set out from the mountain of Adonai a journey of three days, with the ark of the covenant of Adonai setting out ahead of them\lebnote{Or “before them”} three days’ journey to search out a resting place for them;
\verse and the cloud of Adonai was over them by day when they set out from the camp.
\verse And whenever the ark was setting out Moses would say,
\verse “Rise up, Adonai! 
May your enemies be scattered; 
may the ones that hate you flee from your presence.”
\end{biblechapter}

\begin{biblechapter} % Numbers 11
\verseWithHeading{The Israelites Complain} And it happened, the people were like those who \textit{complain of hardship}\lebnote{Literally “complain of bad”} \textit{in the hearing}\lebnote{Literally “in the ears”} of Adonai, and Adonai \textit{became angry},\lebnote{Literally “his nose became hot”} and the fire of Adonai burned among them, and it consumed the edge of the camp.
\verse Then the people cried out to Moses, and Moses prayed to Adonai, and the fire died down.
\verse And he called the name of that place Taberah\lebnote{This word is difficult, but some modern translations suggest the word in Hebrew means “burning” (see NRSV, NASB)} because the fire of Adonai burned among them.
\verse The riff-raff that were in their midst \textit{had a strong desire};\lebnote{Literally “desired a desire”} and the \textit{Israelites}\lebnote{Literally “sons/children of Israel”} turned back and also wept, and they said, “Who will feed us meat?
\verse We remember the fish that we ate in Egypt for nothing, the cucumber, melon, leek, the onions, and the garlic.
\verse But now \textit{our strength is dried up};\lebnote{Literally “our life is dry”} there is nothing whatsoever except \textit{for the manna before us}.”\lebnote{Literally “for the manna of our eyes”}
\verse Now the manna was like coriander seed, and its outward appearance was like that of bdellium-gum.
\verse The people went about and gathered it, and they ground it with mills or crushed it with mortar. Then they boiled it in a pot and made it into bread-cakes; and it tasted like olive oil cakes.
\verse When the dew came down on the camp at night, the manna came down with it.
\verse Moses heard the people weeping according to their\lebnote{Hebrew “its”} clans, each at the doorway of their tents. Then \textit{Adonai became very angry},\lebnote{Literally “the nose of Adonai became very hot”} and in the eyes of Moses it was bad.
\verse And Moses said to Adonai, “Why have you brought trouble to your servant? Why have I not found favor in your eyes, that the burdens of all these people have been placed on me?
\verse Did I conceive all these people? If I have fathered them,\lebnote{Hebrew “him/it”} that you could say to me, ‘Carry them\lebnote{Hebrew “him/it”} in your lap, just as a foster-father carries the suckling on the land that you swore an oath to their ancestors?’\lebnote{Or “fathers”}
\verse From where do I have meat to give all these people? They weep before me, saying, ‘Give us meat and let us eat!’
\verse I am not able to carry all these people along alone; they are too heavy for me.
\verse If this is how you are going to treat me, please kill me immediately if I find favor in your eyes, and do not let me see my misery.”
\verse And Adonai said to Moses, “Gather for me seventy men from the elders of Israel whom you know are elders of the people and their\lebnote{Hebrew “his/its”} officials; take them to the tent of assembly, and they will stand there with you.
\verse I will come down and speak with you there; I will take away from the spirit that is on you, and I will place it on them; and they will bear the burdens of the people with you; you will not bear it alone.
\verse And you will say to the people, ‘Sanctify yourselves tomorrow, for you will eat meat because you have wept in the ears of Adonai, saying, “Who will feed us good meat? It was good for us in Egypt.” Adonai will give to you meat, and you will eat.
\verse You will eat, not one day, or two days, or five days, or ten days, or twenty days,
\verse \textit{but for a whole month},\lebnote{Literally “until a period of one month”} until it comes out from your nose and becomes as nausea to you; because you have rejected Adonai, who is in your midst, and you wept before \textit{his presence},\lebnote{Literally “before his face”} saying, “Why did we ever leave Egypt?” ’ ”
\verse But Moses said, “There are six hundred thousand on foot, among whom I am in the midst, and you yourself said, ‘I will give meat to them, and they will eat for a whole month.’
\verse Should flocks and cattle be slaughtered for them? Should all the fish of the sea be gathered together for them, to be enough for them?”
\verse And Adonai said to Moses, “\textit{Is Adonai’s power limited}?\lebnote{Literally “Is Adonai’s hand short?”} Now you will see if my word will happen or not.”
\verse So Moses went out, and he spoke the words of Adonai to the people, and he gathered together seventy men from the elders of the people, and he \textit{made them stand}\lebnote{Literally “caused them to stand”} all around the tent.
\verse Then Adonai went down in the cloud and spoke to him, and he took away the spirit that was on him, and he \textit{put it}\lebnote{Or “gave it”} on the seventy elders. And as soon as the spirit was resting on them they prophesied, but they did not do it again.
\verse But two men were left in the camp; the name of one was Eldad, and the name of the second was Medad, and the spirit rested on them; they were among those who were written down, but they did not go out to the tent, so they prophesied in the camp.
\verse So a boy\lebnote{Hebrew “the boy”} ran and told Moses and said, “Eldad and Medad are prophesying in the camp.”
\verse And Joshua son of Nun, the assistant of Moses from time of his youth, answered, “Moses, my lord, stop them.”
\verse But Moses said to him, “Are you jealous for my sake? Would that he\lebnote{That is, Adonai} give all Adonai’s people prophets, that Adonai put his spirit on them!”
\verse Then Moses and the elders of Israel were gathered to the camp.
\verseWithHeading{The Quail} Then a wind set out from Adonai, and it drove quails from the west, and he spread them out on the camp about a day’s journey on one side and about a day’s journey on the other, all around the camp, about two cubits on the surface of the land.
\verse And so the people \textit{worked}\lebnote{Literally “arose”} all day and all night and all the next day, and they gathered the quail (the least of the ones collecting gathered ten homers).\lebnote{HALOT 330, “a dry measure”}
\verse While the meat was still between their teeth, before it was consumed, Adonai was angry with the people, and Adonai struck a very great plague among the people.
\verse And he called the name of that place Kibroth Hattaavah\lebnote{Hebrew “the graves of greediness”} because they buried the people that \textit{were greedy}.\lebnote{Literally “craved”}
\verse From Kibroth Hattaavah\lebnote{Hebrew “the graves of greediness”} the people set out to Hazeroth; and they stayed\lebnote{Hebrew “they were”} in Hazeroth.
\end{biblechapter}

\begin{biblechapter} % Numbers 12
\verseWithHeading{Aaron and Miriam Murmur Against Moses} And Miriam and Aaron spoke against Moses because of the Cushite woman whom he took (because he took a Cushite wife);
\verse and they said, “Has Adonai spoken only through Moses? Has not Adonai also spoken through us?” And Adonai heard it.
\verse Now the man, Moses, was more humble than any other person on the face of the earth,
\verse and Adonai said suddenly to Moses, Aaron, and Miriam, “Go out, you three, to the tent of assembly.” So the three of them when out.
\verse And Adonai went down in a column of cloud and stood at the doorway of the tent, and he called Aaron and Miriam, and the two of them went,
\verse and he said,
\verse “Please hear my words: 
If there is a prophet among you, I, Adonai, 
will make myself known to him in a vision. 
I will speak to him in a dream.\lebnote{Hebrew “the dream”}
\verse Not so with my servant Moses; 
in all my house he is faithful. Why were you not afraid to speak against my servant, against Moses?”
\verse \textit{And Adonai became very angry}\lebnote{Literally “And the nose of Adonai became hot”} with them, and he went away.
\verse And the cloud departed from on the tent, and behold, Miriam was infected with \textit{a skin disease}\lebnote{The precise meaning is uncertain; many modern translations suggest “leprosy”} white like snow; when Aaron turned toward Miriam, behold, she was afflicted with a skin disease.
\verse So Aaron said to Moses, “Please, my lord, please do not put on us this sin in which we were foolish and in which we have sinned.
\verse Please do not let her be like the dead, whose flesh is half consumed when coming out from the womb of its mother.”
\verse And Moses cried to Adonai, saying, “God, \textit{please heal her}!”\lebnote{Literally “Please heal please her”}
\verse But Adonai said to Moses, “If her father had surely spit in her face, would she not bear her shame for seven days? Let her be confined for seven days to an outside place of the camp, and afterward she may be gathered.”
\verse So Miriam was confined to the outside place of the camp seven days, and the people did not set out until Miriam was gathered.
\verse And afterward the people set out from Hazeroth, and they encamped in the desert of Paran.
\end{biblechapter}

\begin{biblechapter} % Numbers 13
\verseWithHeading{Spies Sent to Spy Out the Land of Canaan} And Adonai spoke to Moses, saying,
\verse “Send for yourself men, and let them explore the land of Canaan, which I am about to give to the \textit{Israelites};\lebnote{Literally “sons/children of Israel”} \textit{from each tribe of his father send one man},\lebnote{Literally “one man one man from the tribe of his father”} everyone a leader among them.”
\verse So Moses sent them from the desert of Paran on the command of Adonai; all of the men were \textit{leaders}\lebnote{Literally “heads”} of the \textit{Israelites}.\lebnote{Literally “sons/children of Israel”}
\verse And these are their names: from the tribe of Reuben, Shammua son of Zaccur;
\verse from the tribe of Simeon, Shaphat son of Hori;
\verse from the tribe of Judah, Caleb son of Jephunneh;
\verse from the tribe of Issachar, Igal son of Joseph;
\verse from the tribe of Ephraim, Hoshea son of Nun;
\verse from the tribe of Benjamin, Palti son of Raphu;
\verse from the tribe of Zebulun, Gaddiel son of Sodi;
\verse from the tribe of Joseph, from the tribe of Manasseh, Gaddi son of Susi;
\verse from the tribe of Dan, Ammiel son of Gemalli;
\verse from the tribe of Asher, Sethur son of Michael;
\verse from the tribe of Naphtali, Nahbi son of Vophsi;
\verse from the tribe of Gad, Geuel son of Maki.
\verse These are the names of the men whom Moses sent to explore the land. And Moses called Hoshea son of Nun Joshua.
\verse Moses sent them to explore the land of Canaan, and he said to them, “Go up like this to the Negev,\lebnote{An arid region south of the Judean hills} and go up into the hill country,
\verse and you will see what the land is like and if the people who inhabit it are strong or weak, or whether they are few or many,
\verse and whether the land that they are inhabiting is good or bad, and whether the cities they are inhabiting are camps or fortifications,
\verse and whether the land is fertile or lean, and whether there are trees on it or not. You will show yourself courageous, and you will take some of the fruit of the land.” It was the time of first ripe grapes.
\verse So they went up and explored the land from the desert of Zin until Rehob, at Lebo Hamath.\lebnote{Or “near Hamath”}
\verse They went up through the Negev\lebnote{An arid region south of the Judean hills} and came to Hebron, where\lebnote{Hebrew “and there”} Ahiman, Sheshai, and Talmai the descendants of the Anakites were. (Hebron was built seven years before Zoan in Egypt.)
\verse And they came up to the valley\lebnote{Or “wadi”} of Eshcol, and they cut off a vine branch and one cluster of grapes from there; they carried it on a pole between two men, with pomegranates and figs.
\verse That place he called the valley\lebnote{Or “wadi”} of Eshcol on account of the cluster of grapes that the \textit{Israelites}\lebnote{Literally “sons/children of Israel”} cut off from there.
\verseWithHeading{The Spies Return} They returned from exploring the land at the end of forty days.\lebnote{Hebrew “day”}
\verse And they came\lebnote{Hebrew “they went and came”} to Moses and Aaron and to the entire community of the \textit{Israelites}\lebnote{Literally “sons/children of Israel”} in the desert of Paran at Kadesh; they brought back word to them and to all the community, and they showed them the fruit of the land.
\verse And they told him,\lebnote{Hebrew “they told him and said”} “We came to the land that you sent us, and it is flowing of milk and honey; this is its fruit.
\verse Yet the people who are inhabiting it are strong and the cities are fortified and very large; moreover, we saw the descendants of the Anakites there.
\verse The Amalekites are living in the land of the Negev;\lebnote{An arid region south of the Judean hills} the Hittites, Jebusites, and the Amorites are living in the hill country; and the Canaanites are living at the sea and on the banks of the Jordan.”
\verse And Caleb silenced the people before Moses and said, “Surely, let us go up and let us take possession of it because surely we will be able to prevail over it.”
\verse And the men who went up with him said, “We are not able to go up to the people because they are stronger than us.”
\verse And they presented the report of the land that they explored to the \textit{Israelites},\lebnote{Literally “sons/children of Israel”} saying, “The land that we went through to explore is a land that eats its inhabitants, and all the people whom we saw in its midst are \textit{men of great size}.\lebnote{Literally “men of measurements”}
\verse There we saw the Nephilim (the descendants\lebnote{Or “sons”} of Anak came from the Nephilim), and we were like grasshoppers in our own sight, and so we were in their eyes.”
\end{biblechapter}

\begin{biblechapter} % Numbers 14
\verseWithHeading{The People Complain} Then all the community \textit{lifted up their voices},\lebnote{Hebrew “they lifted up and gave their voice”} and the people wept during that night.
\verse And all the \textit{Israelites}\lebnote{Literally “sons/children of Israel”} grumbled against Moses and Aaron, and all the community said to them, “If only we had died in the land of Egypt or in this desert!
\verse Why did Adonai bring us into this land to fall by the sword? Our wives and our little children will become plunder; would it not be better for us to return to Egypt?”
\verse \textit{They said to each other},\lebnote{Literally “A man said to his brother”} “Let us appoint a leader, and we will return to Egypt.”
\verse Then Moses and Aaron fell on their faces \textit{before}\lebnote{Literally “in the presence of”} the assembly of the community of the \textit{Israelites}.\lebnote{Literally “sons/children of Israel”}
\verse Joshua son of Nun and Caleb son of Jephunneh, from the explorers of the land, tore their garments.
\verse And they said to all the community of the \textit{Israelites},\lebnote{Literally “sons/children of Israel”} “The land that we went through to explore is an \textit{exceptionally good land}.\lebnote{Literally “the land is very very good”}
\verse If Adonai delights in us, then he will bring us into this land, and he will give it to us, a land that is flowing with milk and honey.
\verse Only do not rebel against Adonai, and you will not fear the people of the land, because they will be our food. \textit{Their protection}\lebnote{Literally “Their shadow”} has been turned from them; Adonai is with us. You should not fear them.”
\verse And all the community said to stone them with stones, but the glory of Adonai appeared in the tent of assembly among the \textit{Israelites}.\lebnote{Literally “sons/children of Israel”}
\verse And Adonai said to Moses, “How long until this people will despise me, and how long until they will not believe in me, and in all the signs that I have done in their\lebnote{Hebrew “his”} midst?
\verse I will strike them\lebnote{Hebrew “him”} with disease, and I will dispossess them;\lebnote{Hebrew “him”} I will make you into a greater and stronger nation than them.”\lebnote{Hebrew “him”}
\verse And Moses said to Adonai, “Then the Egyptians will hear that you brought up this people from their\lebnote{Hebrew “his”} midst in your power,
\verse and they will \textit{tell it}\lebnote{Literally “say”} to the inhabitants of this land. They heard that you, Adonai, are in the midst of this people, that you are seen eye to eye, and your cloud is standing over them, and in a column of cloud you go before them by day and in a column of fire at night.
\verse But if you destroy this people \textit{all at once},\lebnote{Literally “as one man”} the nations that will have heard your message will say,
\verse ‘Adonai was unable to bring this people in the land that he swore by an oath, and he slaughtered them in the desert.’
\verse But now, please, let the power of my Lord be great, just has you spoke,
\verse ‘Adonai is \textit{slow to anger}\lebnote{Literally “slow of noses”} 
and great of loyal love, 
\textit{forgiving}\lebnote{Literally “lifting up”} sin and rebellion; 
but surely he leaves nothing unpunished, 
visiting the sin of the fathers on the sons 
to the third and fourth generations.’
\verse Please forgive the sin of this people according to the greatness of your loyal love, just as you \textit{forgave}\lebnote{Literally “lifted up”} this people, from Egypt until now.”
\verse Adonai said, “I have forgiven them according to your word;
\verse but as I am alive, the glory of Adonai will fill all the earth.
\verse But because all the men who have seen my glory and my signs that I did in Egypt and in the desert yet tested me these ten times and did not listen to my voice,
\verse they will not see the land that I swore by oath to their ancestors,\lebnote{Or “fathers”} and all those who despised me will not see it.
\verse But my servant Caleb, because another spirit was with him, he remained true after me, and I will bring him into the land that \textit{he entered},\lebnote{Or “he went to”} and his offspring will take possession of it.
\verse And the Amalekites and the Canaanites live in the valleys; tomorrow turn and set out for the desert by way of the \textit{Red Sea}.”\lebnote{Literally “sea of reed”}
\verse And Adonai spoke to Moses and Aaron, saying,
\verse “How long will I bear this evil community who are grumbling against me? I have heard the grumbling of the \textit{Israelites}\lebnote{Literally “sons/children of Israel”} which \textit{they are making}\lebnote{Literally “they are grumbling”} against me.
\verse Say to them, ‘Surely as I live,’ \textit{declares}\lebnote{Literally “declaration of”} Adonai, ‘just as you spoke \textit{in my hearing},\lebnote{Literally “in my ears”} so I will do to you;
\verse in this desert your corpses will fall, and all your counted ones, according to all your number, from \textit{twenty years old}\lebnote{Literally “a son of twenty years”} and above who grumbled against me.
\verse You yourselves will not come into the land that \textit{I swore by oath}\lebnote{Literally “I lifted up my hand”} to make you to dwell in it, but Caleb son of Jephunneh and Joshua son of Nun.
\verse But your little children, whom you said would be plunder, I will bring them, and they will know the land that you rejected.
\verse But for you, all your corpses will fall in this desert.
\verse And your children will be shepherds in the desert forty years,\lebnote{Hebrew “year”} and you will bear your unfaithfulness until \textit{all your corpses have fallen}\lebnote{Literally “until to complete your corpses”} in the desert.
\verse According to the number of the days\lebnote{Hebrew “day”} that you explored the land, forty days,\lebnote{Hebrew “day”} \textit{a day for each year},\lebnote{Literally “a day for a year a day for a year”} you will bear your sins forty years,\lebnote{Hebrew “year”} and you will know my opposition.’
\verse I, Adonai, have spoken; I will surely do this to all this evil community who has banded together against me. In this desert they will come to an end, and there they will die.”
\verse As for the men whom Moses sent to explore the land, who returned and made the community grumble against him by spreading a report over the land,
\verse the men who spread the evil report of the land died by the plague \textit{before Adonai}.\lebnote{Literally “in the presence of Adonai”}
\verse But Joshua son of Nun and Caleb son of Jephunneh lived from among the men who went to explore the land.
\verse And Moses spoke words to all the \textit{Israelites},\lebnote{Literally “sons/children of Israel”} and the people mourned greatly.
\verse They rose early in the morning and went to the top of the mount, saying, “Here we are. We will go up to the place that Adonai said, because we have sinned.”
\verse But Moses said, “Why are you going against the command of Adonai? It will not succeed.
\verse You should not go up because Adonai is not in your midst; do not let yourselves be defeated in the presence of your enemies,
\verse because the Amalekites\lebnote{Hebrew “Amalekite”} and the Canaanites\lebnote{Hebrew “Canaanite”} are there \textit{before you},\lebnote{Literally “in your presence”} and you will fall by the sword; because you have turned back from Adonai, and Adonai will not be with you.”
\verse But they dared to go to the top of the mountain, and the ark of the covenant of Adonai and Moses did not depart from the midst of the camp.
\verse So the Amalekites\lebnote{Hebrew “Amalekite”} and the Canaanites\lebnote{Hebrew “Canaanite”} who were living on the mountain descended, and they beat them down, up to Hormah.
\end{biblechapter}

\begin{biblechapter} % Numbers 15
\verseWithHeading{Various Sacrifices and Offerings} Adonai spoke to Moses, saying,
\verse “Speak to the \textit{Israelites}\lebnote{Literally “sons/children of Israel”} and say to them, ‘When you come into the land of your dwellings that I am about to give to you,
\verse you will make an offering by fire for Adonai from the cattle or from the flock, a burnt offering or a sacrifice to fulfill a vow, or as a freewill offering or at your feasts, to make a fragrance of appeasement for Adonai.
\verse And the one who presents an offering\lebnote{Hebrew “his offering”} for Adonai, he will present a grain offering of finely milled flour; a tenth will be mixed with a fourth of the liquid measure of oil;
\verse and you will add a fourth of wine for the libation upon the burnt offering, or to the sacrifice for each ram-lamb.
\verse Or for the ram you will make a grain offering of two-tenths of finely milled flour mixed into a third of a liquid measure of oil.
\verse You will present a third of the liquid measure of wine for the libation, a fragrance of appeasement for Adonai.
\verse When you prepare \textit{a bull}\lebnote{Literally “a son of cattle”} as a burnt offering or a sacrifice to fulfill a vow or a fellowship offering for Adonai,
\verse you will present with \textit{the bull}\lebnote{Literally “the son of the cattle”} a grain offering of three-tenths of finely milled flour mixed with half a liquid measure of oil,
\verse and you will present half a liquid measure of wine as a libation, as an offering made by fire, a fragrance of appeasement for Adonai.
\verse “ ‘This is how it should be done for each bull, or for the each ram, or for the small four-footed mammal, or ram-lambs, or goats.
\verse According to the number that you prepare, so should you do to each according to their number.
\verse Every native must do these things to present an offering made by fire, a fragrance of appeasement for Adonai.
\verse If an alien dwells among you, or whoever is in your midst throughout your generations,\lebnote{Hebrew “for your generations”} and prepares an offering made by fire, a fragrance of appeasement for Adonai, he should do as you do.
\verse For the assembly, there will be one decree for you and for the alien who dwells among you; it is an eternal decree for all your generations. \textit{You as well as the alien}\lebnote{Literally “like you like the alien”} will be \textit{before Adonai}.\lebnote{Literally “in the presence of Adonai”}
\verse There will be one law and one stipulation for you and for the alien dwelling among you.’ ”
\verse Adonai spoke to Moses, saying,
\verse “Speak to the \textit{Israelites}\lebnote{Literally “sons/children of Israel”} and say to them, ‘When you come into the land to which I am about to bring you,
\verse whenever you eat from the food of the land, you will lift up a contribution to Adonai.
\verse You must lift up a contribution of the first batch of your ring-shaped dough bread; you must lift it up as a contribution of the threshing floor.
\verse You will give to Adonai a contribution from the first of your dough throughout your generations.\lebnote{Hebrew “for your generations”}
\verse “ ‘But if you go astray and you do not follow\lebnote{Or “do”} all these commandments that Adonai commanded to Moses,
\verse all that Adonai commanded you by the hand of Moses\lebnote{Or “through Moses”} from the day that Adonai commanded and beyond, throughout your generations,\lebnote{Hebrew “for your generations”}
\verse and if it was done unintentionally \textit{without the knowledge}\lebnote{Literally “from the eyes”} of the community, then the entire community must prepare one \textit{young bull}\lebnote{Literally “a bull a son of cattle”} as a burnt offering, as a fragrance of appeasement for Adonai, and its grain offering and its libation, according to the stipulation, and one male goat as a sin offering.
\verse The priest will make atonement for all of the community of the \textit{Israelites},\lebnote{Literally “sons/children of Israel”} and \textit{they will be forgiven}\lebnote{Literally “it will be forgiven to them”} because it was unintentional; they will bring their offering, an offering made by fire for Adonai, their sin offering \textit{before Adonai}\lebnote{Literally “in the presence of Adonai”} for their unintentional sin.
\verse All of the community of the \textit{Israelites}\lebnote{Literally “sons/children of Israel”} will be forgiven, as well as the alien that dwells in their midst, because the whole community was involved in the unintentional wrong.
\verse “ ‘If one person sins unintentionally, that person will present a female goat \textit{in its first year}\lebnote{Literally “a daughter of a year”} as a sin offering.
\verse And the priest will make atonement for the person who \textit{sinned unintentionally}\lebnote{Literally “sinned unintentionally when sinning an unintentional wrong”} \textit{before Adonai},\lebnote{Literally “in the presence of Adonai”} to make atonement for him, and he will be forgiven.
\verse For the native among the \textit{Israelites}\lebnote{Literally “sons/children of Israel”} and the alien that dwells in their midst, there will be one law for anyone who commits an unintentional wrong.
\verse But the one \textit{who acts presumptuously}\lebnote{Literally “who acts with a high hand”} from among the native or alien blasphemes against Adonai, and that person must be cut off from the midst of the people.
\verse Because he despised the word of Adonai and broke his command, that person will be surely cut off and bear the guilt.’ ”
\verseWithHeading{Violation of the Sabbath} When the \textit{Israelites}\lebnote{Literally “sons/children of Israel”} were in the desert, they found a man who was gathering wood on the day of the Sabbath.
\verse The ones who found him gathering wood brought him to Moses, Aaron, and to all the community.
\verse And they put him under watch because it was not made clear what should be done to him.
\verse And Adonai said to Moses, “Surely the man must be put to death by stoning him; all the community must stone him with stones from outside the camp.”
\verse So the entire community brought him out to a place outside the camp, and \textit{they stoned him to death}\lebnote{Literally “they stoned him with stones and he died”} just as Adonai commanded Moses.
\verseWithHeading{Garment Fringes} Adonai spoke to Moses, saying,
\verse “Speak to the \textit{Israelites},\lebnote{Literally “sons/children of Israel”} and tell them to make for themselves tassels\lebnote{Hebrew “tassel”} on the hems of their garments throughout their generations\lebnote{Hebrew “for their generations”} and to put a blue cord on the tassel of the hem.
\verse You will have a tassel \textit{for you to look at}\lebnote{Literally “and you will look at it”} and remember all the commands of Adonai and do them, and not follow \textit{after the unfaithfulness of your own heart and eyes},\lebnote{Literally “after your heart and after your eyes, which you are unfaithful after them”}
\verse so that you will remember and do all my commandments, and you will be holy for your God.
\verse I am Adonai your God, who brought you out of the land of Egypt, to be your God; I am Adonai your God.”
\end{biblechapter}

\begin{biblechapter} % Numbers 16
\verseWithHeading{Korah, Dathan, and Abiram Rebel} Now Korah son of Izhar, son of Kohath, son of Levi, and Dathan and Abiram sons of Eliab, and On son of Peleth, the descendants\lebnote{Or “sons”} of Reuben,
\verse took two hundred and fifty men from the \textit{Israelites},\lebnote{Literally “sons/children of Israel”} leaders of the community summoned from the assembly, \textit{renowned men},\lebnote{Literally “men of name”} and \textit{they confronted}\lebnote{Literally “they rose up before”} Moses.
\verse They were assembled in front of Moses and Aaron, and they said to them, “\textit{You take too much upon yourselves}!\lebnote{Literally “It is much for you”} All of the community is holy, every one of them, and Adonai is in their midst, so why do you raise yourselves over the assembly of Adonai?”
\verse When Moses heard this, he fell on his face.
\verse And he said to Korah and to his entire company, saying, “Tomorrow morning Adonai will make known who is his and who is holy, and he will bring him near to him, whomever he chooses he will bring near to him.
\verse Do this: take for yourselves censers, Korah and all of your company;\lebnote{Hebrew “his company”}
\verse tomorrow put fire in them and place incense on them \textit{before}\lebnote{Literally “in the presence of”} Adonai; the man whom Adonai chooses will be the holy one. You take too much upon yourselves, sons of Levi!”
\verse And Moses said to Korah, “Please listen, sons of Levi!
\verse Is it too little for you that the God of Israel set you apart from the community of Israel \textit{to allow you to approach him}\lebnote{Literally “to bring you near to him”} to do the work of the tabernacle of Adonai, to stand \textit{before}\lebnote{Literally “in the presence of”} the community to serve them?
\verse \textit{He has allowed you to approach him},\lebnote{Literally “He has brought you near”} you with all your brothers, the descendants\lebnote{Or “sons”} of Levi, but yet you also seek the priesthood.
\verse Therefore you and your company that has banded together against Adonai. What is Aaron that you grumble against him?”
\verse Moses sent to call for Dathan and Abiram son of Eliab, but they said, “We will not come!\lebnote{Hebrew “We will not come up!”}
\verse Is it too little that you have brought us from a land that flows with milk and honey to kill us in the desert, and that you also appoint yourself as a ruler over us?
\verse Surely, you have not brought us to a land that flows with milk and honey, and you have not given us the inheritance of fields and a vineyard. Will you gouge out the eyes of these men? We will not come!”\lebnote{Hebrew “We will not come up!”}
\verse Then Moses became angry, and he said to Adonai, “Do not notice their grain offering. I have not offered one donkey from them, and I have not mistreated one of them.”
\verse And Moses said to Korah, “You and your entire company will be \textit{before}\lebnote{Literally “in the presence of”} Adonai tomorrow, you and they and Aaron.
\verse Each one take his censer, and put incense on it\lebnote{Hebrew “on them”} and you will present it \textit{before}\lebnote{Literally “in the presence of”} Adonai, and each of you bring his censer, two hundred and fifty censers, you and Aaron, each his censer.”
\verse So each of them took his censer, and they put fire on them, and they placed incense on them; they stood at the doorway of the tent of the assembly of Moses and Aaron.
\verse And Korah summoned them, the entire community, by the doorway of the tent of assembly, and the glory of Adonai appeared to all the community.
\verse And Adonai spoke to Moses and Aaron, saying,
\verse “Separate yourselves from the midst of this community, that I can destroy them in a moment.”
\verse And they fell on their faces, and they said, “God, God of the spirits of all flesh, will one man sin and you become angry toward the entire community?”
\verse Adonai spoke to Moses, saying,
\verse “Speak to the community, saying, ‘Move away from the dwelling of Korah, Dathan, and Abiram.’ ”
\verse So Moses stood up and went to Dathan and Abiram; the elders of Israel followed after him.
\verse He said to the community, saying, “Please turn away from the tents of these wicked men, and do not touch anything \textit{that belongs to them},\lebnote{Literally “that is to them”} or you will be destroyed with all their sins.”
\verse And so they moved away from around the dwellings\lebnote{Hebrew “dwelling”} of Korah, Dathan, and Abiram; and Dathan and Abiram came out standing at the doorway of their tents, with their wives, sons, and little children.
\verse And Moses said, “In this you will know that Adonai has sent me to do all these works; it is not from my heart.
\verse \textit{If they die a natural death}\lebnote{Literally “If they die like the death of every human”} or \textit{if a natural fate is visited upon them},\lebnote{Literally “If the fate of every human is visited upon them”} Adonai has not sent me.
\verse But if Adonai creates something new, and the ground opens its mouth and swallows them up and \textit{all that belongs to them},\lebnote{Literally “All that is for them”} and they go down alive to Sheol, and you will know that these men have despised Adonai.”
\verse And it happened, as soon as he finished \textit{speaking}\lebnote{Literally “to speak”} all these words, the ground that was under them split open.
\verse The land opened its mouth and swallowed them up with their houses and every person \textit{that belonged to Korah}\lebnote{Literally “that was to Korah”} and all the property.
\verse They went down alive to Sheol, they and all that belonged to them, and the land covered over them, and they perished from the midst of the assembly.
\verse All Israel who were around them fled at their cry, because they said, “Lest the land swallow us up!”
\verse And fire went out from Adonai, and it consumed the two hundred and fifty men presenting the incense.
\verse \lebnote{Numbers 16:36–17:13 in the English Bible is 17:1–28 in the Hebrew Bible} And Adonai spoke to Moses, saying,
\verse “Say to Eleazar son of Aaron the priest, ‘\textit{Take out}\lebnote{Literally “raise up”} the censers from among the place of burning because they are sacred, and scatter the fire outside.
\verse The censers of these \textit{who have sinned}\lebnote{Literally “sinners”} at the cost of their lives, let them be made into gilded leafing plating for the altar; because they presented them \textit{before Adonai},\lebnote{Literally “in the presence of Adonai”} they are holy; and they will be a sign for the \textit{Israelites}.’ ”\lebnote{Literally “sons/children of Israel”}
\verse Eleazar the priest took the bronze censers that the ones who were burned presented, and they hammered them out thinly as plating for the altar;
\verse it was a memorial for the \textit{Israelites},\lebnote{Literally “sons/children of Israel”} so that \textit{no strange man}\lebnote{NASB translates “no layman”} who is not from \textit{the offspring}\lebnote{Literally “the seed”} of Aaron should approach the presence of Adonai to burn a smoke offering;\lebnote{That is, an incense offering} he will not be like Korah and his company, just as Adonai had spoken to him by the hand of Moses.\lebnote{Or “through Moses”}
\verseWithHeading{The Israelites Grumble} The next day all the community of the \textit{Israelites}\lebnote{Literally “sons/children of Israel”} grumbled against Moses and Aaron, saying, “You have killed the people of Adonai!”
\verse Then, when the community had gathered against Moses and Aaron, they turned to the tent of assembly, and behold, the cloud covered it, and the glory of Adonai appeared.
\verse And Moses and Aaron came to the front of the tent of assembly,
\verse and Adonai spoke to Moses, saying,
\verse “Get away from the midst of this community, and I will finish them in an instant,”\lebnote{Or “in a moment”} but they fell on their faces.
\verse And Moses and Aaron said, “Take the censer, and put fire on it from the altar. Place incense on it, and bring it quickly to the community, and make atonement for them, because wrath went out from the presence of Adonai, and a plague has begun.”
\verse And so Aaron took it just as Moses had spoken, and he ran into the midst of the assembly, for behold, the plague had begun among the people; so he gave the incense and made atonement for the people.
\verse He stood between the dead and between the living, and the plague was stopped.
\verse Those who died by the plague were fourteen thousand seven hundred, besides those who died on account of Korah.
\verse Then Aaron returned to Moses at the doorway of the tent of assembly, and the plague was stopped.
\end{biblechapter}

\begin{biblechapter} % Numbers 17
\verseWithHeading{Aaron’s Staff Is Chosen} Adonai spoke to Moses, saying,
\verse “Speak to the \textit{Israelites},\lebnote{Literally “sons/children of Israel”} and take from among them twelve staffs, \textit{a staff from each family}\lebnote{Literally “a staff from a house of a father”} from among all their leaders according to their families’ households. Write the name of each man on his staff,
\verse and the name of Aaron on the staff of Levi, because one staff is for the head of each of \textit{their families}.\lebnote{Literally “the house of their fathers”}
\verse You must then put them in the tent of assembly \textit{before}\lebnote{Literally “in the presence of”} the testimony\lebnote{Or “the statute”} where I meet with you.
\verse And it will happen, the man whom I will choose, his staff will blossom, and so I will rid from upon myself the grumblings of the \textit{Israelites},\lebnote{Literally “sons/children of Israel”} who are grumbling against you.”
\verse Moses spoke to the \textit{Israelites},\lebnote{Literally “sons/children of Israel”} and all their leaders gave him a staff for each leader, one from \textit{each of their families},\lebnote{Literally “each from the house of their fathers”} twelve staffs, and the staff of Aaron was in the midst of their tribes.
\verse And Moses put the staffs before Adonai in the tent of testimony.
\verse Then the next day, Moses went into the tent of the testimony, and behold the staff of Aaron for the house of Levi blossomed and put forth a flower and produced blossoms, and it produced almonds.
\verse Then Moses brought out to all the \textit{Israelites}\lebnote{Literally “sons/children of Israel”} all the staffs before the presence of Adonai, and they saw, and each man took his staff.
\verse And Adonai said to Moses, “Bring back the staff of Aaron before the testimony\lebnote{Or “the statute”} as a guard and sign for the children of rebellion, and let them finish their grumblings before me and not die.”
\verse So Moses did; just as Adonai commanded him, so he did.
\verse And the \textit{Israelites}\lebnote{Literally “sons/children of Israel”} said to Moses, saying, “Look! We will die! We will be destroyed! All of us will perish!
\verse Anyone who approaches the tabernacle of Adonai will die. \textit{Will we all die}?”\lebnote{Literally “Will we all die to perish?”}
\end{biblechapter}

\begin{biblechapter} % Numbers 18
\verseWithHeading{The Duties of the Priests and Levities} Adonai said to Aaron, “You, your sons, and your family with you will bear the guilt of the sanctuary, and you and your sons with you will bear the guilt of your priesthood.
\verse Moreover, bring your brothers with you, the tribe of Levi the tribe of your father, that they may be joined to you and minister to you, you and your sons with you before the tent of testimony.
\verse They will keep your responsibility and the responsibility of all the tent, only they may not come near the vessels of the sanctuary and the altar, so both you and they will not die.
\verse They will be joined to you, and they will keep the responsibility of the tent of assembly for the entire service of the tent; a stranger may not come near you.
\verse You will keep the responsibility of the sanctuary and the responsibility of the altar, and there will no longer be wrath on the \textit{Israelites}.\lebnote{Literally “sons/children of Israel”}
\verse Look, I myself have chosen your brothers the Levites from the midst of the children. They are a gift to you given from Adonai to perform the work of the tent of assembly.
\verse But you with your sons will keep your priesthood to perform your priestly duties for everything at the altar\lebnote{Or “for all the things of the altar”} and for \textit{the area behind the curtain}.\lebnote{Literally “the house of the curtain”} I give you the priesthood as a gift, but the stranger who approaches will be put to death.”
\verseWithHeading{Portions for the Priests} Adonai spoke to Aaron, “Behold, I myself have given to you the responsibility of my contributions for all the holy objects of the \textit{Israelites};\lebnote{Literally “sons/children of Israel”} I have given them as a portion to you and your sons as an eternal decree.
\verse This will be for you from the sanctuary of the holy things from the fire; all of their offerings, from every grain offering, from every sin offering, and from every guilt offering which they will bring to me is \textit{a most holy thing}\lebnote{Literally “a holy object of holiness”} for you and your sons.
\verse You will eat it in the most holy place;\lebnote{Alternatively “as a holy object of holiness”} every male will eat it. It will be a holy object to you.
\verse This is also for you: the contribution of their gift of the wave offerings of the children Israel. I have given them to you and your sons and your daughters with you as an eternal decree; whoever is clean in your house may eat it.
\verse All the finest olive oil and all the finest new wine and their best grain that they have given to Adonai, I have given them to you.
\verse The firstfruits of all that is in their land that they present to Adonai will be for you; whoever is clean in your house may eat it.
\verse All consecrated possessions\lebnote{Hebrew “possession”} in Israel will be for you.
\verse All the first offspring of a womb of any creature that they offer to Adonai, whether human or animal, will be yours; you will surely redeem the firstborn of the human and the unclean firstborn of the animal.
\verse As to their price of redemption, from \textit{a one-month-old}\lebnote{Literally “a son of a month”} you will redeem them according to your proper value, five shekels of silver according to the shekel of the sanctuary, which is twenty gerah.
\verse Only the firstborn of an ox or the firstborn of a sheep or the firstborn of a goat you will not redeem; they are holy. Their blood you will sprinkle over the altar, and their fat you will turn into smoke as an offering made by fire, a fragrance of appeasement for Adonai.
\verse But their flesh will be for you like the breast section of the wave offering, and it will be for you like the right upper thigh.
\verse All the contributions of holiness that the \textit{Israelites}\lebnote{Literally “sons/children of Israel”} offer to Adonai I have given to you and your sons and your daughters with you as an eternal decree; it is an eternal covenant of salt \textit{before}\lebnote{Literally “in the presence of”} Adonai to you and your offspring with you.”
\verse Then Adonai said to Aaron, “You will not receive an inheritance in their land, and there will not be a plot of ground for you in the midst of the \textit{Israelites}.\lebnote{Literally “sons/children of Israel”}
\verse “Behold, I have given to the descendants\lebnote{Or “sons”} of Levi every tithe in Israel as an inheritance in return for their service, which they are doing, the work of the tent of assembly.
\verse The \textit{Israelites}\lebnote{Literally “sons/children of Israel”} will not come near again to the tent of assembly, or \textit{they will bear sin}\lebnote{The NRSV translates “they will incur guilt”} and die.
\verse The Levites\lebnote{Hebrew “Levite”} will perform the service of the tent of assembly, and they will bear their guilt, an eternal decree for all your generations. But they will not receive an inheritance in the midst of the \textit{Israelites}\lebnote{Literally “sons/children of Israel”}
\verse because the tithes\lebnote{Hebrew “tithe”} of the \textit{Israelites}\lebnote{Literally “sons/children of Israel”} that are \textit{offered}\lebnote{Hebrew “raised up”} to Adonai as a contribution, I have given to the Levites as an inheritance; therefore I said to them, ‘They will not receive an inheritance in the midst of the \textit{Israelites}.’ ”\lebnote{Literally “sons/children of Israel”}
\verse Adonai spoke to Moses, saying,
\verse “You will speak to the Levites and say to them, ‘When you receive the tithe from the \textit{Israelites}\lebnote{Literally “sons/children of Israel”} that I have given to you from them for your inheritance, \textit{you will present}\lebnote{Literally “you will raise up”} a contribution from it to Adonai, a tithe from a tithe.
\verse Your contribution will be credited to you like the grain from the threshing floor and like the produce from the press.
\verse So \textit{you will present}\lebnote{Literally “you will raise up”} your own contribution to Adonai from all your tithes that you receive from the \textit{Israelites};\lebnote{Literally “sons/children of Israel”} from it you will give the contribution of Adonai to Aaron the priest.
\verse From all your gifts \textit{you will present}\lebnote{Literally “you will raise up”} every contribution of Adonai, from all its fat, the part that is sacred.’
\verse You will say to them, ‘When \textit{you are presenting}\lebnote{Literally “you are raising up”} its fat, the rest will be credited to the Levites like a yield of the threshing floor and like a yield from the press.
\verse You may eat it anywhere, you and your household, because it is a wage in return for your service in the tent of assembly.
\verse You will not bear any sin because \textit{you have presented}\lebnote{Literally “you have raised up”} its fat; you will not defile the holy objects of the \textit{Israelites},\lebnote{Literally “sons/children of Israel”} or you will die.’ ”
\end{biblechapter}

\begin{biblechapter} % Numbers 19
\verseWithHeading{Ashes of the Red Heifer} And Adonai spoke to Moses and Aaron, saying,
\verse “This is the decree of the law that Adonai has commanded, saying, ‘Speak to the \textit{Israelites}\lebnote{Literally “sons/children of Israel”} and let them take to you a red heifer without a physical defect, on which a yoke \textit{has not been placed}.\lebnote{Literally “has not gone up”}
\verse And you will give it to Eleazar the priest, and it will be brought\lebnote{Or “he will bring it out”} out to a place outside the camp, and it will be slaughtered\lebnote{Or “he will slaughter it”} in his presence.
\verse Then Eleazar the priest will take some of its blood on his finger and spatter it toward the mouth of the tent of assembly seven times.
\verse The heifer will be burned\lebnote{Hebrew “The heifer will burn”} in his sight; its skin, its meat, and its blood, in addition to its offal, will burn.
\verse The priest will take cedar wood, hyssop, and crimson thread, and he will throw them in the midst of \textit{the burning heifer}.\lebnote{Literally “the burning of the heifer”}
\verse The priest will wash his garments and his body in the water, and afterward he will come to the camp; the priest will be unclean until the evening.
\verse The one who burns it will wash his garments and his body in water; he will be unclean until the evening.
\verse A clean man will gather the ashes of the heifer, and he will put them in a clean \textit{place outside the camp};\lebnote{Literally “an outside place of the camp”} it will be for the community of the \textit{Israelites}\lebnote{Literally “sons/children of Israel”} as a requirement for waters of impurity; it is a purification offering.
\verse The one who gathers the ashes of the heifer will wash his garments; he will be unclean until evening. It will be an eternal decree for the \textit{Israelites}\lebnote{Literally “sons/children of Israel”} and for one who dwells as an alien in their midst.
\verse “ ‘The one who touches a corpse of \textit{any person}\lebnote{Literally “any human person”} will be unclean for seven days.
\verse He will purify himself on the third day, and on the seventh day he will be clean. If he does not purify himself on the third day, he will not be clean on the seventh day.
\verse Anyone who touches a corpse, the person of a human being who died, and does not purify himself, defiles the tabernacle of Adonai, and that person will be cut off from Israel because the waters of impurity were not sprinkled on him. He will still be unclean, and uncleanness is on him.
\verse “ ‘This is the law of a person who dies in a tent: everyone who comes into the tent and all who are in the tent will be unclean seven days.
\verse Every container that is opened that does not have a lid cord\lebnote{That is, “that does not have a lid tied shut”} on it is unclean.
\verse Anyone \textit{in the open field}\lebnote{Literally “upon the face of a field”} who touches \textit{one who has been slain},\lebnote{Literally “the dead of sword”} or a corpse, or a bone of a person, or a burial site, he will be unclean for seven days.
\verse For the unclean person they will take\lebnote{Hebrew “he will take”} from the powder of the \textit{burnt purification offering},\lebnote{Or “burning of the sin offering”} and they will put\lebnote{Hebrew “he will put”} running water into a container.
\verse A clean person will take hyssop and dip it into the water and sprinkle it on the tent and on all the objects and persons who were there, and on one who touched the bone, or the one slain, or the dead, or the burial site.
\verse The clean person will spatter the unclean on the third day and on the seventh day; and on the seventh day he will purify him, and he will wash his garments; he will bathe in the waters, and in the evening he will be clean.
\verse But the man who is unclean and does not purify himself, that person will be cut off from the midst of the assembly because he defiled the sanctuary of Adonai; the water of impurity was not sprinkled on him; he is unclean.
\verse “ ‘It will be an eternal decree for them. The one who spatters the waters of impurity will wash his garments, and the one who touches the waters of impurity will be unclean until the evening.
\verse Anything that the unclean person touches will be unclean, and the person who touches it will be unclean until the evening.’ ”
\end{biblechapter}

\begin{biblechapter} % Numbers 20
\verseWithHeading{Miriam Dies} Then the entire community of the \textit{Israelites}\lebnote{Literally “sons/children of Israel”} came to the desert of Zin on the first month, and the people stayed in Kadesh; Miriam died and was buried there.
\verse There was no water for the community, and they were gathered before Moses and Aaron.
\verse And the people quarreled with Moses and spoke, saying, “If only we died when our brothers were dying \textit{before}\lebnote{Literally “in the presence of”} Adonai!
\verse Why have you brought the assembly of Adonai, us and our livestock, into this desert to die here?
\verse Why have you brought us from Egypt to bring us to this bad place? It is not a place of seed or figs\lebnote{Hebrew “fig”} or vines\lebnote{Hebrew “vine”} or pomegranate trees,\lebnote{Hebrew “tree”} and there is not water to drink.”
\verse And Moses and Aaron went from the presence of the assembly to the doorway of the tent of assembly. They fell on their faces, and the glory of Adonai appeared to them.
\verse Adonai spoke to Moses, saying,
\verse “Take the staff and summon the community, you and Aaron your brother, and speak to the rock before their eyes, and it will give water. Bring out for them water from the rock, and let the community and their livestock drink.”
\verse So Moses took the staff from \textit{before}\lebnote{Literally “in the presence of”} Adonai just as he command him,
\verse and Moses and Aaron summoned the assembly to the presence of the rock, and he said to them, “Please listen, you rebels; can we bring out water for you from this rock?”
\verse Then Moses lifted up his hand and struck the rock with his staff twice. And abundant water went out, and the community and their livestock drank.
\verse But Adonai said to Moses and Aaron, “Because you have not trusted in me, to regard me as holy \textit{in the sight of}\lebnote{Or “before the eyes of”} the \textit{Israelites},\lebnote{Literally “sons/children of Israel”} you will not bring this assembly into the land that I have given to them.”
\verse Those were the waters of Meribah, where the \textit{Israelites}\lebnote{Literally “sons/children of Israel”} quarreled with Adonai, and he showed himself holy among them.
\verse From Kadesh Moses sent messengers to the king of Edom: “Thus your brother Israel has said, ‘You know all the hardship that has found us;
\verse our ancestors\lebnote{Or “fathers”} went down to Egypt, and we lived in Egypt \textit{a long time},\lebnote{Literally “many days”} and the Egyptians mistreated us and our ancestors.\lebnote{Or “fathers”}
\verse Then we cried to Adonai, and he heard our voice; he sent an angel and brought us out from Egypt. And look, we are in Kadesh, a city on the edge of your territory.
\verse Please let us go through your land. We will not go through a field or vineyard, and we will not drink water from a well. We will go along the road of the king; we will not turn aside right or left until we have gone through your territory.’ ”
\verse Then Edom said to him, “You will not pass through us\lebnote{Hebrew “me”} lest \textit{we will go out}\lebnote{Hebrew “I will go out”} to meet you with the sword.”
\verse The \textit{Israelites}\lebnote{Literally “sons/children of Israel”} said to him, “We will go up on the main road, and if we\lebnote{Hebrew “I”} and our livestock\lebnote{Hebrew “my livestock”} drink your water, we will pay for it.\lebnote{Hebrew “I will give their worth”} It is only a small matter; let us pass through on our feet.”\lebnote{Hebrew “I will go through on my feet”}
\verse But he said, “You will not go through.” And Edom went out to meet \textit{them}\lebnote{Hebrew “him”} with a large army and a strong hand.
\verse So Edom refused to give Israel passage through his territory, and Israel turned aside from him.
\verseWithHeading{Aaron Dies} And they set out from Kadesh. The \textit{Israelites},\lebnote{Literally “sons/children of Israel”} the whole community, came to Mount Hor.
\verse Adonai said to Moses and to Aaron on Mount Hor, on the boundary of the land of Edom, saying,
\verse “Let Aaron be gathered to his people; he will not come into the land that I have given to the \textit{Israelites}\lebnote{Literally “sons/children of Israel”} because you rebelled against \textit{my word}\lebnote{Literally “my mouth”} at the waters of Meribah.
\verse Take Aaron and Eleazar his son, and take them up Mount Hor.
\verse Strip off Aaron’s garments, and put them on Eleazar his son; Aaron will be gathered to his people, and he will die there.”
\verse So Moses did just as Adonai commanded, and they went up to Mount Hor before the eyes of all the community.
\verse And Moses stripped off Aaron’s garments and put them on Eleazar his son. Aaron died there on the top of the mountain; and Moses and Eleazar went down from the mountain.
\verse All the community saw that Aaron died; so all the house of Israel wept for Aaron thirty days.\lebnote{Hebrew “day”}
\end{biblechapter}

\begin{biblechapter} % Numbers 21
\verseWithHeading{Arad Captured} The Canaanite king of Arad, who was dwelling in the Negev,\lebnote{An arid region south of the Judean hills} heard that Israel came along the way of Atharim; he fought against Israel and took some of them captive.
\verse Israel made a vow to Adonai, and they said, “If you will surely give this people into our\lebnote{Hebrew “my”} hand, then we\lebnote{Hebrew “I”} \textit{will destroy}\lebnote{Literally “devote to God”} their cities.”
\verse Adonai heard the voice of Israel; he gave to them the Canaanites, and \textit{they destroyed them}\lebnote{Literally “they devoted to God”} and their cities. They called the name of the place Hormah.
\verse They set out from Mount Hor by the way of the \textit{Red Sea}\lebnote{Literally “sea of reed”} to go around the land of Edom; but \textit{the people became impatient}\lebnote{Literally “the life of the people became short”} along the way.
\verse The people spoke against God and against Moses, “Why have you brought us from Egypt to die in the desert? There is no food and no water, and our hearts detest this miserable food.”
\verseWithHeading{The Bronze Serpent} And Adonai sent among the people poisonous snakes; they bit the people, and many people from Israel died.
\verse The people came to Moses and said, “We have sinned because we have spoken against Adonai and against you. Pray to Adonai and let him remove the snakes\lebnote{Hebrew “snake”} from among us.” So Moses prayed for the people.
\verse And Adonai said to Moses, “Make for yourself a snake and place it on a pole. When\lebnote{Hebrew “And it will happen”} anyone is bitten and looks at it, that person will live.”
\verse So Moses made a snake of bronze, and he placed it on the pole; whenever\lebnote{Hebrew “And it will happen”} a snake bit someone, and that person looked at the snake of bronze, he lived.
\verse The \textit{Israelites}\lebnote{Literally “sons/children of Israel”} set out and encamped at Oboth.
\verse They set out from Oboth and encamped at Iye Abarim in the desert, which was in front of Moab \textit{toward the sunrise}.\lebnote{Literally “from the east of the sun”}
\verse From there they set out and encamped at the valley of Zered.
\verse From there they set out and encamped beyond Arnon, which is in the desert that goes out from the boundary of the Amorites,\lebnote{Hebrew “Amorite”} because Arnon is the boundary of Moab, between Moab and the Amorites.\lebnote{Hebrew “Amorite”}
\verse Therefore thus it is said in the scroll of the Wars of Adonai,
\verse “Waheb in Suphah, 
and the wadis of Arnon,
\verse From there they went to Beer, which is the water well where Adonai spoke to Moses, “Gather the people, that I may give them water.”
\verse Then Israel sang this song, “Arise, well water! Sing to it!
\verse Well water that the princes dug, that the leaders of the people dug, with a staff and with their rods.” And from the desert they continued to Mattanah,
\verse and from Mattanah to Nahaliel, and from Nahaliel to Bamoth;
\verse and from Bamoth to the valley that is in the territory of Moab, by the top of Pisgah, which overlooks the surface of the wasteland.
\verseWithHeading{Sihon and Og Defeated} Israel sent messengers to Sihon, the king of the Amorites,\lebnote{Hebrew “Amorite”} saying,
\verse “Let us go through your land; we will not turn aside into a field or vineyard; we will not drink well water along the way of the king until we have gone through your territory.”
\verse But Sihon did not allow Israel to go through his territory. Sihon gathered all his people and went out to meet Israel; he came to the desert, to Jahaz, and he fought against Israel.
\verse But Israel struck him with the edge of the sword, and they took possession of his land from Arnon to Jabbok, until the \textit{Ammonites},\lebnote{Literally “sons of Ammon”} because the boundary of the \textit{Ammonites}\lebnote{Literally “sons of Ammon”} was strong.
\verse Israel took all these cities, and Israel inhabited all the cities of the Amorites,\lebnote{Hebrew “Amorite”} in Heshbon, and in all its environs.\lebnote{Hebrew “her daughters;” other modern versions translate “its villages”}
\verse Because Heshbon was the city of Sihon king of the Amorites,\lebnote{Hebrew “Amorite”} who had fought against the former king of Moab and taken all his land from his hand until Arnon.
\verse Thus the ones who quote proverbs say,
\verse “Come to Heshbon! Let it be built! 
And let the city of Sihon be established.
\verse Because fire went out from Heshbon, 
a flame from the city of Sihon; 
it consumed Ar of Moab, 
the lords of the\lebnote{Or “the dominant”} high places of Arnon.
\verse Woe to you, Moab! 
You have perished, people of Chemosh. 
He has given his sons as fugitives, 
and his daughters into captivity, 
to the king of the Amorites,\lebnote{Hebrew “Amorite”} Sihon.
\verse Thus Israel lived in the land of the Amorites.\lebnote{Hebrew “Amorite”}
\verse Moses sent to explore Jaazer; they captured \textit{its environs}\lebnote{Hebrew “her daughters;” other modern versions translate “its villages”} and dispossessed the Amorites\lebnote{Hebrew “Amorite”} who were there.
\verse Then they turned and went up by the way of the Bashan, and Og king of the Bashan and all his people went out to meet them for battle at Edrei.
\verse And Adonai said to Moses, “Do not fear him because I will give him and all his people and all his land into your hand. You will do to him just as you did to Sihon king of the Amorites,\lebnote{Hebrew “Amorite”} who was living in Heshbon.”
\verse And so they destroyed him and his sons, and all his people until they had not spared a survivor; and they took possession of his land.
\end{biblechapter}

\begin{biblechapter} % Numbers 22
\verseWithHeading{Balak and Balaam} The \textit{Israelites}\lebnote{Literally “sons/children of Israel”} set out, and they encamped on the desert-plateau of Moab, across from Jericho beyond the Jordan.
\verse Balak son of Zippor saw all that Israel did to the Amorites,\lebnote{Hebrew “Amorite”}
\verse and Moab was very terrified in the presence of the people because \textit{they}\lebnote{Hebrew “he” or “it”} were numerous; and Moab dreaded the presence of the \textit{Israelites}.\lebnote{Literally “sons/children of Israel”}
\verse And Moab said to the elders of Midian, “Now the crowd will lick up all around us, like a bull devours the grass of the field.” And Balak son of Zippor was king of Moab at that time.
\verse He sent messengers to Balaam son of Beor at Pethor, which is by the river,\lebnote{That is, the Euphrates} in the land of the children of his people, to summon him, saying, “Look! A people went out from Egypt. Look! They cover \textit{the surface of the land};\lebnote{Literally “the eye of the land”} they are about to dwell opposite me.
\verse Now, please go, curse this people for me because they\lebnote{Hebrew “he”} are stronger than me; perhaps I will be able to strike them\lebnote{Hebrew “he”} and drive them\lebnote{Hebrew “he”} out from the land because I know whoever you bless is blessed, and whoever you cursed is cursed.”
\verse So the elders of Moab and the elders of Midian went with a fee for divination in their hand; they came to Balaam and spoke the words of Balak to him.
\verse He said to them, “Spend the night here, and I will return, and I will return word to you, just as Adonai speaks to me.” So the princes of Moab stayed with Balaam.
\verse And God came to Balaam and said, “Who are these men with you?”
\verse And Balaam said to God, “Balak son of Zippor, king of Moab, sent word to me,
\verse ‘Look! A people went out from Egypt. Look! They cover \textit{the surface of the land}.\lebnote{Literally “the eye of the land”} Now, go, curse them\lebnote{Hebrew “him”} for me. Perhaps I will be able to attack them\lebnote{Hebrew “him”} and drive them\lebnote{Hebrew “him”} out.”
\verse God said to Balaam, “You will not go with them; you will not curse the people, because they\lebnote{Hebrew “he”} are blessed.”
\verse Balaam got up in the morning, and he said to the princes of Balak, “Go to your land, because Adonai refused to allow me to go with you.”
\verse The princes of Moab got up and went to Balak, and they said, “Balaam refused to come with us.”
\verse Balak again sent many princes, who were more honored \textit{than the former}.\lebnote{Literally “than these”}
\verse They came to Balaam and said to him, “Thus says Balak son of Zippor, ‘Please, let nothing keep you from coming to me
\verse because I will surely honor you greatly, and all that you say to me I will do. Please, come; curse this people for me.’ ”
\verse Balaam answered and said to the servants of Balak, “Even though Balak gives to me his house full of silver and gold, I am not able to go beyond \textit{the command of Adonai}\lebnote{Literally “the mouth of Adonai”} my God to do a little or a lot.
\verse And now please, \textit{you also stay here}\lebnote{Literally “please stay in this”} the night, and \textit{let me find out}\lebnote{Literally “let me know”} again what Adonai will say with me.”
\verse And God came to Balaam at night, and he said to him, “If the men have come to call you, get up and go with them; but only the word that I will speak to you, you will do.”
\verse So Balaam got up in the morning and saddled his donkey, and he went with the princes of Moab.
\verseWithHeading{Balaam and the Angel} But \textit{God became angry}\lebnote{Literally “God’s nose became hot”} because he was going, and the angel of Adonai stood in the road as an adversary to him; he was riding on his donkey, and two servants were with him.
\verse The donkey saw the angel of Adonai standing in the road with his sword drawn in his hand, and the donkey turned aside from the road and went into the field. And Balaam struck the donkey to turn her back to the road.
\verse The angel of Adonai stood in the narrow path of the vineyards, with \textit{a wall on either side}.\lebnote{Literally “a wall from this and a wall from this”}
\verse When the donkey saw the angel of Adonai, she pressed herself into the wall, and she pressed the foot of Balaam into the wall, so he struck her again.
\verse Then the angel of Adonai went further ahead and stood in a narrow place where there was not a way to turn aside to the right or left.
\verse When the donkey saw the angel of Adonai, she lay down under Balaam, so \textit{Balaam became angry},\lebnote{Literally “Balaam’s nose became hot”} and he struck the donkey with his staff.
\verse Adonai opened the mouth of the donkey, and she said to Balaam, “What did I do to you that you struck me these three times?”
\verse Balaam said to the donkey, “Because you made a mockery of me! If only I had a sword in my hand, I would kill you right now!”
\verse The donkey said to Balaam, “Am I not your donkey on which you have ridden all your life until this day? Have I been in the habit of doing this to you?” He said, “No.”
\verse Then Adonai exposed the eyes of Balaam, and he saw the angel of Adonai standing in the road with his sword drawn in his hand, and he bowed down and worshiped to his face.
\verse The angel of Adonai said to him, “Why have you struck this donkey three times? Look, I have come out as an adversary because your conduct is perverse before me.
\verse The donkey saw me and turned aside from me these three times. If she had not turned aside from my face, then I would have killed you and kept her alive.”
\verse Balaam said to the angel of Adonai, “I have sinned because I did not know that you were standing to meet me in the road. Now, \textit{if it is displeasing to you},\lebnote{Literally “if it is evil in your eyes”} I will turn back.”
\verse The angel of Adonai said to Balaam, “Go with the men, but speak only the word that I will speak to you.” So Balaam went with the princes of Balak.
\verse When Balak heard that Balaam was coming, he went out to meet him by the city of Moab, which was on the boundary of Aaron at the end of the territory.
\verse And Balak said to Balaam, “Did I not urgently send to meet with you? Why did you not come to me? Am I really not able to honor you?”
\verse Balaam said to Balak, “Look, I came to you now. Am I really able to speak anything at all? I speak the word that God puts in my mouth.”
\verse Balaam went with Balak, and they came to Kiriath-Huzoth.
\verse And Balak sacrificed cattle and sheep, and he sent them to Balaam and to the princes who were with him.
\verse And it happened, in the morning Balak took Balaam and took him up to Bamoth-Baal, and he saw from there the end of the nation.
\end{biblechapter}

\begin{biblechapter} % Numbers 23
\verseWithHeading{Balaam’s Oracles} Balaam said to Balak, “Build for me this: seven altars. And prepare for me this: seven bulls and seven rams.”
\verse And Balak did just as Balaam spoke, and Balak offered Balaam a bull and a ram on the altar.
\verse And Balaam said to Balak, “Station yourself at your burnt offering, and I will go; perhaps Adonai will come to meet me, and whatever he shows me I will tell to you.” So he went to a barren height.
\verse And God met with Balaam, and he said to him, “I have arranged seven altars, and I have offered a bull and a ram on the altar.”
\verse Adonai put a word in the mouth of Balaam and said, “Return to Balak, and you must speak thus.”
\verse So he returned to him, and behold, he was standing beside his burnt offering, he and all the leaders of Moab.
\verse And he lifted up his oracle and said,
\verse “From Aram Balak lead me, 
from the mountains of the east the king of Moab, 
‘Go for me, curse Jacob, 
and go, denounce Israel.’
\verse How can I curse whom God has not cursed, 
and how can I denounce whom Adonai has not denounced?
\verse Because from the top of the rocks I see him, 
from hilltops I watch him. 
Behold, a people who dwell alone, 
they do not consider themselves among the nations.
\verse And Balak said to Balaam, “What have you done to me? I took you to curse my enemies, and look, you have surely blessed them!”
\verse He answered and said, “\textit{Should I not speak}\lebnote{Literally “Should I not observe to speak”} what Adonai puts in my mouth?”
\verse Then Balak said, “Please walk with me to another place where you will see them, but you will only see part of them and will not see all of them; and curse them for me from there.”
\verse So he took him to the field of Zophim to the top of Pisgah, and he built seven altars, and he offered a bull and a ram on each altar.
\verse Balaam\lebnote{Hebrew “He”} said to Balak, “Station yourself here at the burnt offering while I myself meet with Adonai there.”
\verse Then Adonai met with Balaam, and he put a word in his mouth, and he said, “Return to Balak, and you must speak thus.”
\verse He came to him, and behold, he was standing at his burnt offering, and the princes of Moab with him. And Balak said to him, “What has Adonai spoken?”
\verse Then \textit{he uttered}\lebnote{Literally “he lifted up”} his oracle, and said,
\verse “Stand up, Balak, and hear; 
listen to me, son of Zippor!
\verse God is not a man, that he should lie, 
nor a son of humankind, 
that he should change his mind. 
Has he said, and will he not do it? 
And has he spoken, and will he not fulfill it?
\verse Behold, I have received a command to bless; 
when he has blessed, I cannot cause it to return.
\verse He has no regard for evil in Jacob, 
and he does not see trouble in Israel; 
Adonai his God is with him, 
and a shout\lebnote{Or “a blast”} of a king is among them.\lebnote{Hebrew “him”}
\verse God, who brings them out from Egypt, 
is like the strength\lebnote{Or “like the horns”} of a wild ox for them.\lebnote{Hebrew “him”}
\verse Because there is no sorcery against Jacob, 
and there is no divination against Israel. 
Now\lebnote{Or “At the right time”} it will be said to Jacob and Israel, 
what God has done!
\verse Then Balak said to Balaam, “Do not curse them\lebnote{Hebrew “him”} at all, nor bless them\lebnote{Hebrew “him”} at all!”
\verse But Balaam answered and said to Balak, “Did I not speak to you, saying, ‘Whatever Adonai speaks I will do’?”
\verse Then Balak said to Balaam, “Please, come, I will take you to another place; perhaps \textit{it will be acceptable to}\lebnote{Literally “it will be right in the eyes of”} God, and you will curse for me from there.”
\verse So Balak took Balaam to the top of Peor, which looks down on the face of the Jeshimon.\lebnote{Or “the wasteland”}
\verse And Balaam said to Balak, “Build for me these seven altars, and prepare for me these seven bulls and seven rams.”
\verse Balak did just as Balaam said, and he offered a bull and a ram on each altar.
\end{biblechapter}

\begin{biblechapter} % Numbers 24
\verseWithHeading{Balaam Continues to Utter Oracles} And Balaam saw that \textit{it pleased}\lebnote{Literally “it was good in the eyes of Adonai”} Adonai to bless Israel, and he did not go \textit{as other times}\lebnote{Literally “as time on time”} \textit{to seek out}\lebnote{Literally “to meet”} sorcery; instead, he set his face toward the desert.
\verse Balaam lift up his eyes, and he saw Israel dwelling according to its tribes, and the spirit of God was upon it.\lebnote{That is, Israel}
\verse He \textit{uttered}\lebnote{Literally “lifted up”} his oracle and said,
\verse “The declaration of Balaam son Beor, 
the declaration of the man whose eyes are closed,
\verse the declaration of the hearer of God’s words,\lebnote{Or “God’s sayings”} 
who sees the revelation of Shaddai,\lebnote{Often translated “the Almighty”} 
falling down but whose eyes are uncovered.
\verse How good are your tents, O Jacob, 
your dwellings, O Israel!
\verse They are spread out like valleys, 
like gardens on a river, 
like aloes planted by Adonai, 
like cedars at the waters.
\verse He will pour water from his buckets, 
and his offspring will be like many waters; 
his king will be higher than Agag, 
and his kingdom will be exalted.
\verse God, who brings him out from Egypt, 
is like the strength\lebnote{Or “the horns”} of a wild ox for him. 
He will devour the nations who are his enemies; 
he will break their bones; 
he will pierce them with his arrows.
\verse Then \textit{Balak became angry with}\lebnote{Literally “the nose of Balak became hot against”} Balaam, and he clapped his hands and said to Balaam, “I called you to curse my enemies, but look, you have surely blessed them these three times.
\verse \textit{Flee}\lebnote{Literally “Flee for yourself”} to your place now. I said I would richly honor you, but look, Adonai has withheld honor from you.”
\verse Balaam said to Balak, “Did I not speak to your messengers whom you sent to me, saying,
\verse ‘If Balak gave to me the fullness of his house full of silver and gold, I am not able to go beyond \textit{the command of Adonai}\lebnote{Literally “the mouth of Adonai”} to do good or evil, from my heart; what Adonai speaks, I will speak’?\lebnote{Hebrew “I will speak it”}
\verse And now, look, I am about to go to my people; I will advise you what this people will do to your people \textit{in the following days}.”\lebnote{Literally “in the last of the days”}
\verse And he \textit{uttered}\lebnote{Literally “lifted up”} his oracle and said,
\verse “The declaration of Balaam son of Beor, 
and the declaration of the man whose eye is closed,
\verse the declaration of the hearer of God’s words,\lebnote{Or “God’s sayings”} 
and the knower of the knowledge of the Most High, 
who sees the vision of Shaddai,\lebnote{Often translated “the Almighty”} 
who is falling, and his eyes are revealed.
\verse I see him, but not now; 
I behold him, but not near; 
a star will go out from Jacob, 
and a scepter will rise from Israel; 
it will crush the foreheads of Moab 
and destroy all the children of Seth.
\verse Edom will be a captive; 
Seir, its enemies, will be a captive, 
and Israel will be acting \textit{courageously}.\lebnote{Literally “with physical strength”}
\verse Someone\lebnote{Hebrew “He”} from Jacob will rule 
and will destroy a remnant\lebnote{Or “survivor”} from the city.”
\verse And he looked at the Kenites,\lebnote{Hebrew “Kenite”} \textit{uttered}\lebnote{Literally “lifted up”} his oracle, and said,
\verse “Steady is your dwelling place; 
in the rock is your nest.
\verse Again he \textit{uttered}\lebnote{Literally “lifted up”} his oracle and said,
\verse “Woe, who will live when God establishes this?\lebnote{Hebrew “it”}
\verse Then Balaam got up and went and returned to his place, and Balak also went on his way.
\end{biblechapter}

\begin{biblechapter} % Numbers 25
\verseWithHeading{The Plague of Israel} When Israel dwelled in Shittim, the people began to prostitute\lebnote{Or “have sexual relations”} themselves with the daughters of Moab.
\verse And they invited the people to the sacrifices of their gods, and the people ate and worshiped their gods.
\verse So Israel was joined together to Baal Peor, and \textit{Adonai became angry}\lebnote{Literally “the nose of Adonai became hot”} with Israel.
\verse Adonai said to Moses, “Take all \textit{the leaders}\lebnote{Literally “the heads”} of the people and kill them before the sun, so the fierce anger of Adonai will turn from Israel.”
\verse So Moses said to the judges of Israel, “Each of you kill his men who are joined together with Baal Peor.”
\verse And behold, a man from the \textit{Israelites}\lebnote{Literally “sons/children of Israel”} came and brought to his brothers a Midianite woman before the eyes of Moses and before the eyes of all of the community of the \textit{Israelites},\lebnote{Literally “sons/children of Israel”} and they were weeping at the doorway of the tent of assembly.
\verse When Phinehas son of Eleazar son of Aaron the priest saw, he got up from the midst of the community and took a spear in his hand.
\verse He went after the man of Israel into the woman’s section of the tent, and he drove the two of them, the man of Israel and the woman, into her belly. And the plague among the \textit{Israelites}\lebnote{Literally “sons/children of Israel”} stopped.
\verse The ones who died in the plague were twenty-four thousand.
\verse Adonai spoke to Moses, saying,
\verse “Phinehas son of Eleazar, son of Aaron the priest, turned away my anger from among the \textit{Israelites}\lebnote{Literally “sons/children of Israel”} when he was jealous with my jealousy in their midst, and I did not destroy the \textit{Israelites}\lebnote{Literally “sons/children of Israel”} with my jealousy.
\verse Therefore say, ‘Behold, I am giving to him my covenant of peace,
\verse and it will be for him and his \textit{offspring}\lebnote{Literally “seed”} after him a covenant of an eternal priesthood because he was jealous for his God and made atonement for the \textit{Israelites}.’ ”\lebnote{Literally “sons/children of Israel”}
\verse The name of the man of Israel who was struck with the Midianite woman was Zimri son of Salu, a leader of \textit{the family}\lebnote{Or “the father’s house” or “the ancestor’s house”} of the Simeonites.\lebnote{Hebrew “Simeonite”}
\verse The name of the Midianite woman who was struck was Cozbi daughter of Zur, \textit{a leader}\lebnote{Literally “a head”} of a tribe of \textit{the family}\lebnote{Or “the father’s house” or “the ancestor’s house”} in Midian.
\verse Adonai spoke to Moses, saying,
\verse “Attack the Midianites and strike them
\verse because they were attacking you with their deception, with which they have deceived you on the matter of Peor and on the matter of Cozbi the daughter of the leader of Midian, their sister who was struck on the day of the plague because of the matter of Peor.”
\end{biblechapter}

\begin{biblechapter} % Numbers 26
\verseWithHeading{A New Census} \lebnote{Numbers 26:1a in the English Bible is 25:19b in the Hebrew Bible} And it happened after the plague, \lebnote{In the Hebrew Bible, Numbers 26:1 begins here}Adonai said to Moses and to Eleazar son of Aaron the priest, saying,
\verse “\textit{Take a census}\lebnote{Literally “Lift up the heads”} of the community of the \textit{Israelites}\lebnote{Literally “sons/children of Israel”} from \textit{those twenty years old}\lebnote{Literally “a son of twenty years”} and above, according to \textit{their families},\lebnote{Literally “the house of their fathers”} all who are able to go out to war in Israel.”
\verse So Moses and Eleazar the priest spoke with them on the desert-plateau of Moab by the Jordan across from Jericho, saying,
\verse “Take a census of the community from \textit{those twenty years old}\lebnote{Literally “a son of twenty years”} and above, just as Adonai commanded Moses.” The Israelites who went out from the land of Egypt were:
\verse Reuben, the firstborn of Israel, the descendants\lebnote{Or “sons”} of Reuben: of Hanoch, the clan of the Hanochites; of Pallu, the clan of the Palluites;\lebnote{Hebrew “Palluite”}
\verse of Hezron, the clan of the Hezronites;\lebnote{Hebrew “Hezronite”} of Carmi, the clan of the Carmites.\lebnote{Hebrew “Carmite”}
\verse These are the clans of the Reubenites,\lebnote{Hebrew “Reubenite”} and the ones counted of them were forty-three thousand seven hundred and thirty.
\verse The children of Pallu: Eliab.
\verse The children of Eliab: Nemuel, Dathan, and Abiram. These are the same Dathan and Abiram who were appointed of the community, who rebelled against Moses and Aaron in the company of Korah, when they rebelled against Adonai,
\verse and the land opened its mouth and swallowed them with Korah, when that company died, when the fire consumed two hundred and fifty men, and they were a sign.\lebnote{That is, a warning sign}
\verse The children\lebnote{Or “sons”} of Korah, however, did not die.
\verse The descendants\lebnote{Or “sons”} of Simeon, according to their clans: of Nemuel, the clan of the Nemuelites;\lebnote{Hebrew “Nemuelite”} of Jamin, the clan of the Jaminites;\lebnote{Hebrew “Jaminite”} of Jakin, the clan of the Jakinites;\lebnote{Hebrew “Jakinite”}
\verse of Zerah, the clan of the Zerahites;\lebnote{Hebrew “Zerahite”} of Shaul, the clan of the Shaulites.\lebnote{Hebrew “Shaulite”}
\verse These were the clans of the Simeonites,\lebnote{Hebrew “Simeonite”} twenty-two thousand two hundred.
\verse The descendants\lebnote{Or “sons”} of Gad according to their clans: of Zephon, the clan of the Zephonites;\lebnote{Hebrew “Zephonite”} of Haggi, the clan of the Haggites;\lebnote{Hebrew “Haggite”} of Shuni, the clan of the Shunites;\lebnote{Hebrew “Shunite”}
\verse of Ozni, the clan of the Oznites;\lebnote{Hebrew “Oznite”} of Eri, the clan of the Erites;\lebnote{Hebrew “Erite”}
\verse of Arod, the clan of the Arodites;\lebnote{Hebrew “Arodite”} of Areli, the clan of the Arelites;\lebnote{Hebrew “Arelite”}
\verse These were the clans of the descendants\lebnote{Or “sons”} of Gad according to the ones counted of them, forty thousand five hundred.
\verse The sons of Judah: Er and Onan; but Er and Onan died in the land of Canaan.
\verse The descendants\lebnote{Or “sons”} of Judah according to their clans were: of Shelah, the clan of the Shelanites;\lebnote{Hebrew “Shelanite”} of Perez, the clan of the Perezites;\lebnote{Hebrew “Perezite”} of Zerah, the clan of the Zerahites.\lebnote{Hebrew “Zerahite”}
\verse The children of Perez were: of Hezron, the clan of the Hezronites;\lebnote{Hebrew “Hezronite”} of Hamul, the clan of the Hamulites.\lebnote{Hebrew “Hamulite”}
\verse These were the clans of Judah according to the ones counted of them, seventy-six thousand five hundred.
\verse The descendants\lebnote{Or “sons”} of Issachar according to their clans: of Tola, the clan of the Tolaites;\lebnote{Hebrew “Tolaite”} of Puvah, the clan of the Punites;\lebnote{Hebrew “Punite”}
\verse of Jashub, the clans of the Jashubites;\lebnote{Hebrew “Jashubite”} of Shimron, the clan of the Shimronites.\lebnote{Hebrew “Shimronite”}
\verse These were the clans of Issachar according to the ones counted of them, sixty-four thousand three hundred.
\verse The descendants\lebnote{Or “sons”} of Zebulun according to their clans: of Sered, the clan of the Seredites;\lebnote{Hebrew “Seredite”} of Elon, the clan of the Elonites;\lebnote{Hebrew “Elonite”} of Jahleel, the clan of the Jahleelites.\lebnote{Hebrew “Jahleelite”}
\verse These were the clans of the Zebulunites\lebnote{Hebrew “Zebulunite”} according to the ones counted of them, sixty thousand five hundred.
\verse The descendants\lebnote{Or “sons”} of Joseph according to their clans: Manasseh and Ephraim.
\verse The descendants\lebnote{Or “sons”} of Manasseh: of Makir, the clan of the Makirites.\lebnote{Hebrew “Makirite”} And Makir fathered Gilead; of Gilead, the clan of the Gileadites.\lebnote{Hebrew “Gileadite”}
\verse These were the descendants\lebnote{Or “sons”} of Gilead: of Iezer, the clan of the Iezerites;\lebnote{Hebrew “Iezerite”} of Helek, the clan of the Helekites;\lebnote{Hebrew “Helekite”}
\verse and of Asriel, the clan of the Asrielites;\lebnote{Hebrew “Asrielite”} and of Shechem, the clan of the Shechemites;\lebnote{Hebrew “Shechemite”}
\verse and of Shemida, the clan of the Shemidaites;\lebnote{Hebrew “Shemidaite”} and of Hepher, the clan of the Hepherites.\lebnote{Hebrew “Hepherite”}
\verse Zelophehad son of Hepher did not have sons, but only daughters; and the names\lebnote{Hebrew “name”} of the daughters of Zelophehad were Mahlah, Noah, Hoglah, Milcah, and Tirzah.
\verse These were the clans of Manasseh, and the ones counted of them were fifty-two thousand seven hundred.
\verse These were the descendants\lebnote{Or “sons”} of Ephraim according to their clans: of Shuthelah, the clan of the Shuthelahites;\lebnote{Hebrew “Shuthelahite”} of Beker, the clan of the Bekerites;\lebnote{Hebrew “Bekerite”} of Tahan, the clan of the Tahanites.\lebnote{Hebrew “Tahanite”}
\verse And these were the descendants\lebnote{Or “sons”} of Shuthelah: of Eran, the family of the Eranites.\lebnote{Hebrew “Eranite”}
\verse These were the clans of the descendants\lebnote{Or “sons”} of Ephraim according to the ones counted of them, thirty-two thousand five hundred. These were the descendants\lebnote{Or “sons”} of Joseph according to their clans.
\verse The descendants\lebnote{Or “sons”} of Benjamin according to their clans: of Bela, the clan of the Belaites;\lebnote{Hebrew “Belaite”} of Ashbel, the clan of the Ashbelites;\lebnote{Hebrew “Ashbelite”} of Ahiram, the clan of the Ahiramites;\lebnote{Hebrew “Ahiramite”}
\verse of Shephupham, the clan of the Shuphamites;\lebnote{Hebrew “Shuphamite”} of Hupham, the clan of the Huphamites.\lebnote{Hebrew “Huphamite”}
\verse The sons of Bela were Ard and Naaman: of Ard, the clan of the Ardites;\lebnote{Hebrew “Ardite”} of Naaman, the clan of the Naamites.\lebnote{Hebrew “Naamite”}
\verse These were the descendants\lebnote{Or “sons”} of Benjamin according to their clans. And the ones counted of them were forty-five thousand six hundred.
\verse These were the descendants\lebnote{Or “sons”} of Dan according to their clans: of Shuham, the clan of the Shuhamites.\lebnote{Hebrew “Shuhamite”} These were the clans of Dan according to their clans.
\verse All the clans of the Shuhamites,\lebnote{Hebrew “Shumhamite”} according to the ones counted of them, were sixty-four thousand four hundred.
\verse The descendants\lebnote{Or “sons”} of Asher according to their clans: of Imnah, the clan of the Imnahites;\lebnote{Hebrew “Imnahite”} of Ishvi, the clan of the Ishvites;\lebnote{Hebrew “Ishvite”} of Beriah, the clan of the Beriahites.\lebnote{Hebrew “Beriahite”}
\verse The descendants\lebnote{Or “sons”} of Beriah: of Heber, the clan of the Heberites;\lebnote{Hebrew “Heberite”} of Malkiel, the clan of the Malkielites.\lebnote{Hebrew “Malkielite”}
\verse The name of the daughter of Asher was Serah.
\verse These were the clans of the descendants\lebnote{Or “sons”} of Asher according to the ones counted of them, fifty-three thousand four hundred.
\verse The descendants\lebnote{Or “sons”} of Naphtali according to their clans: of Jahzeel, the clan of the Jahzeelites;\lebnote{Hebrew “Jahzeelite”} of Guni, the clan of the Gunites;\lebnote{Hebrew “Gunite”}
\verse of Jezer, the clan of the Jezerites;\lebnote{Hebrew “Jezerite”} of Shillem, the clan of the Shillemites.\lebnote{Hebrew “Shillemite”}
\verse These were the clans of Naphtali according to their clans, the ones counted of them, forty-five thousand four hundred.
\verse These were the ones counted of the \textit{Israelites},\lebnote{Literally “sons/children of Israel”} six hundred and one thousand seven hundred and thirty.
\verse Then Adonai spoke to Moses, saying,
\verse “For these the land must be divided as an inheritance according to the number of names.
\verse For the larger group you must increase their inheritance, and for the smaller group you must make smaller their inheritance; each must be given \textit{their}\lebnote{Hebrew “his”} inheritance according to the number of the ones counted of them.
\verse Surely the land will be divided by lot. They will inherit according to the names of the tribes of their ancestors.\lebnote{Or “fathers”}
\verse \textit{Their}\lebnote{Hebrew “His”} inheritance must be divided according to the lot between the larger and smaller groups.”
\verse These are the ones counted of the Levites according to their clans: of Gershon, the clan of the Gershonites;\lebnote{Hebrew “Gershonite”} of Kohath, the clan of the Kohathites;\lebnote{Hebrew “Kohathite”} of Merari, the clan of the Merarites.\lebnote{Hebrew “Merarite”}
\verse These are the clans of Levi: the clan of the Libnites,\lebnote{Hebrew “Libnite”} the clan of the Hebronites,\lebnote{Hebrew “Hebronite”} the clan of the Mahlites,\lebnote{Hebrew “Mahlite”} the clan of the Mushites,\lebnote{Hebrew “Mushite”} the clan of the Korahites.\lebnote{Hebrew “Korahite”} Kohath fathered Amram.
\verse The name of the wife of Amram was Jochebed, the daughter of Levi, whose mother bore her for Levi in Egypt; she bore to Amram: Aaron and Moses and their sister Miriam.
\verse To Aaron were born Nadab and Abihu, Eleazar and Ithamar.
\verse But Nadab and Abihu died when they presented strange fire \textit{before}\lebnote{Literally “in the presence of”} Adonai.
\verse The ones counted were twenty-three thousand, every male from \textit{a month old}\lebnote{Literally “a son of one month”} and above, because they were not counted in the midst of the \textit{Israelites}\lebnote{Literally “sons/children of Israel”} since no inheritance was given to them in the midst of the \textit{Israelites}.\lebnote{Literally “sons/children of Israel”}
\verse These were the ones counted by Moses and Eleazar the priest, who counted the \textit{Israelites}\lebnote{Literally “sons/children of Israel”} on the desert-plateau of Moab on the Jordan across Jericho.
\verse And among these there was not a man of those counted by Moses and Aaron the priest, who counted the \textit{Israelites}\lebnote{Literally “sons/children of Israel”} in the desert of Sinai.
\verse For Adonai said to them, “They will surely die in the desert.” And not a man was left over from them, except Caleb son of Jephunneh and Joshua son of Nun.
\end{biblechapter}

\begin{biblechapter} % Numbers 27
\verseWithHeading{Laws of Inheritance} Then the daughters of Zelophehad, the son of Hepher, the son of Gilead, the son of Makir, the son of Manasseh, of the clan of Manasseh the son of Joseph, came near; and these were the names of his daughters: Mahlah, Noah, Hoglah, and Tirzah.
\verse They stood before Moses and before Eleazar the priest and before the leaders of the entire community at the doorway of the tent of assembly,\lebnote{Or “meeting”} saying,
\verse “Our father died in the desert; he was not in the midst of the company of those who banded together against Adonai in the company of Korah, but he died in his own sin, and he had no sons.
\verse Why should the name of our father disappear from the midst of his clan because he does not have a son? Give us property in the midst of the brothers of our father.”
\verse So Moses brought their case before Adonai.
\verse And Adonai said to Moses, saying,
\verse “\textit{The statements of the daughters of Zelophehad are right}.\lebnote{Literally “Such as the daughters of Zelophehad are speaking”} You must surely give them the property of an inheritance in the midst of their father’s brothers, and you must transfer the inheritance of their father to them.
\verse And you must speak to the \textit{Israelites},\lebnote{Literally “sons/children of Israel”} saying, ‘If a man dies and has no son, you must transfer his inheritance to his daughter.
\verse And if he has no daughter, you must give his inheritance to his brothers.
\verse If he has no brothers, then you must give his inheritance to his father’s brothers.
\verse If his father has no brothers, then you must give his inheritance to his nearest relative from his own clan, and he will take possession of it. It will be as a decree of stipulation for the \textit{Israelites},\lebnote{Literally “sons/children of Israel”} just as Adonai commanded Moses.’ ”
\verse Adonai said to Moses, “Go up to this mountain of Abarim, and see the land that I have given to the \textit{Israelites}.\lebnote{Literally “sons/children of Israel”}
\verse When you see it, you will be gathered to your people, just as Aaron your brother was gathered,
\verse because you rebelled against my word in the desert of Zin when the community quarreled regarding my holiness at the waters.” (These are the waters of Meribah-Kadesh in the desert of Zin.)
\verseWithHeading{Joshua Succeeds Moses} Adonai spoke to Moses, saying,
\verse “Let Adonai, the God of the spirits of all flesh, appoint a man over the community
\verse who will go out before them and will come in before them, and who will lead them out and bring them in, so the community of Adonai will not be like a flock that does not have a shepherd.”
\verse Then Adonai said to Moses, “Take Joshua son of Nun, a man in whom is the spirit, and place your hand on him.
\verse Have him stand before Eleazar the priest and before the entire community, and \textit{commission him}\lebnote{Literally “command him”} \textit{in their sight}.\lebnote{Literally “before their eyes”}
\verse You will give to him from your authority so that the entire community of Israel will \textit{obey him}.\lebnote{Literally “hear him”}
\verse He will stand before Eleazar the priest, who will ask for him by the decision\lebnote{Or “judgment”} of the Urim before Adonai. On \textit{his command}\lebnote{Literally “his word”} they will go out, and at \textit{his command}\lebnote{Literally “his word”} they will come in, both he and all of the \textit{Israelites}\lebnote{Literally “sons/children of Israel”} with him, the entire community.”
\verse Moses did just as Adonai commanded him, and he took Joshua and set him before Eleazar the priest and before the entire community.
\verse And he placed his hands on him and \textit{commissioned him}\lebnote{Literally “commanded him”} just as Adonai spoke by the hand of Moses.\lebnote{Or “through Moses”}
\end{biblechapter}

\begin{biblechapter} % Numbers 28
\verseWithHeading{Daily Sacrifice} Adonai spoke to Moses, saying,
\verse “Command the \textit{Israelites}\lebnote{Literally “sons/children of Israel”} and say to them, ‘\textit{You will be careful to present}\lebnote{Literally “you will observe to present”} my offering, my food of my offerings made by fire, of a fragrance of appeasement to me, at its appointed time.’
\verse You will say to them, ‘This is the offering made by fire that you will offer to Adonai: two male lambs without defect \textit{in their first year}\lebnote{Literally “sons of a year”} as a continual burnt offering each day.
\verse You will offer one male lamb in the morning, and the second male lamb you will offer \textit{at twilight},\lebnote{Literally “between the two evenings”}
\verse and a tenth of an ephah of finely milled flour as a grain offering, mixed with a fourth of a measure of beaten oil.
\verse It is a continual burnt offering that was \textit{ordained}\lebnote{Literally “done”} on Mount Sinai as a fragrance of appeasement, an offering made by fire for Adonai.
\verse The libation with it will be a fourth of a liquid measure for each male lamb; in the sanctuary you will pour out the libation of fermented drink for Adonai.
\verse And the second male lamb you will offer \textit{at twilight};\lebnote{Literally “between the two evenings”} as the grain offering of the morning and as its libation you will offer it, an offering made by fire, a fragrance of appeasement for Adonai.
\verse “ ‘On the day of the Sabbath, two male lambs without defect \textit{in their first year},\lebnote{Literally “sons of a year”} and two-tenths of finely milled flour mixed with oil for a grain offering and its libation.
\verse This is the burnt offering every Sabbath in addition to the continual burnt offering and its libation.
\verse “ ‘And at the beginning of each of your months, you will present a burnt offering for Adonai: two bulls and one ram, seven male lambs without defect \textit{in their first year};\lebnote{Literally “sons of a year”}
\verse and three-tenths of finely milled flour mixed with oil for a grain offering, for each bull; and two-tenths of finely milled flour mixed with oil for a grain offering for the one ram;
\verse and a tenth of finely milled flour mixed with oil as a grain offering for each male lamb, for a burnt offering of a fragrance of appeasement, an offering of fire for Adonai.
\verse Their libations will be half a liquid measure of wine for the bull and a third of a liquid measure of wine for the ram and a fourth of a liquid measure of wine for the male lamb; this is the burnt offering for every month for the months of the year.
\verse And one male goat as a sin offering for Adonai; it will be offered in addition to the continual burnt offering and its libation.
\verse “ ‘On the fourteenth day of the first month is the Passover for Adonai.
\verse On the fifteenth day of this month is a religious feast, unleavened bread must be eaten for seven days.
\verse On the first day there will be a holy convocation; \textit{you will not do any regular work}.\lebnote{Literally “you will not do work of labor”}
\verse You will present an offering by fire, a burnt offering for Adonai: two bulls and one ram and seven male lambs \textit{in their first year};\lebnote{Literally “sons of a year”} they will be for you without defect.
\verse For their grain offering, you will offer finely milled flour mixed with oil: three-tenths for the bull and two-tenths for the ram.
\verse You will offer a tenth for each of the seven male lambs;
\verse and a goat for one sin offering to make atonement for you.
\verse You will offer these besides the burnt offering of the morning, which is for the continual burnt offering.
\verse Like this you will offer daily, for seven days, the food of the offering made by fire, a fragrance of appeasement for Adonai; it will be offered in addition to the continual burnt offering and its libation.
\verse On the seventh day you will have a holy convocation; \textit{you will not do any regular work}.\lebnote{Literally “you will not do work of labor”}
\verseWithHeading{Offerings for the Festival of Weeks} “ ‘And on the day of firstfruits, when you are presenting a new offering for Adonai during your Festival of Weeks, you will have a holy convocation; \textit{you will not do any regular work}.\lebnote{Literally “you will not do work of labor”}
\verse You will present a burnt offering for a fragrance of appeasement for Adonai: two bulls, one ram, seven male lambs \textit{in their first year};\lebnote{Literally “sons of a year”}
\verse and their grain offering will be finely milled flour mixed with oil: three-tenths for each bull, two-tenths for one ram,
\verse a tenth for each of the male lambs;
\verse and one male goat to make atonement for you.
\verse In addition to the continual burnt offering and its grain offering, you will offer them without defect with their libation.
\end{biblechapter}

\begin{biblechapter} % Numbers 29
\verseWithHeading{Offers for the Seventh Month} “ ‘On the seventh month, on the first day of the month, you will have a holy convocation; \textit{you will not do any regular work}.\lebnote{Literally “you will not do work of labor”} It will be a day for you of blowing trumpets.
\verse You will offer a burnt offering as a fragrance of appeasement for Adonai: one bull, one ram, and seven male lambs \textit{in their first year};\lebnote{Literally “sons of a year”} they will be without defect.
\verse Their grain offering will be finely milled flour mixed with oil: three-tenths for the bull, two-tenths for the ram;
\verse and one-tenth for each of the seven male lambs;
\verse with one male goat for a sin offering, to make atonement for you,
\verse in addition to the burnt offering of the new moon and its grain offering, the continual burnt offering and its grain offering, and their libations, according to their stipulations, as a fragrance of appeasement by fire for Adonai.
\verseWithHeading{Offerings for the Day of Atonement} “ ‘And on the tenth of this seventh month you will have a holy convocation, and \textit{you will afflict yourselves};\lebnote{Literally “you will afflict your lives” or “you will afflict your souls”} you will not do any work.
\verse You will present a burnt offering for Adonai, a fragrance of appeasement: one bull, one ram, seven male lambs \textit{in their first year};\lebnote{Literally “sons of a year”} they will be without defect.
\verse And their grain offering will be of finely milled flour mixed with oil: three-tenths for the bull, two-tenths for the one ram,
\verse one-tenth for each of the seven male lambs;
\verse one male goat for a sin offering, in addition to the sin offering of atonement and the continual burnt offering and its grain offering, and their libations.
\verse “ ‘Then on the fifteenth day of the seventh month you will have a holy convocation; \textit{you will not do any regular work},\lebnote{Literally “you will not do work of labor”} and you will hold a religious feast for Adonai for seven days.
\verse You will present a burnt offering, an offering made by fire as a fragrance of appeasement for Adonai: thirteen bulls, two rams, fourteen male lambs \textit{in their first year};\lebnote{Literally “sons of a year”} they will be without defect.
\verse And their grain offering will be of finely milled flour mixed with oil: three-tenths for the bull, two-tenths for the one ram,
\verse one-tenth for each of the seven male lambs;
\verse and one male goat for a sin offering, in addition to the continual burnt offering, its grain offering, and its libation.
\verse “ ‘On the second day: twelve bulls, two rams, fourteen male lambs \textit{in their first year};\lebnote{Literally “sons of a year”} they will be without defect;
\verse and their grain offering and their libations for the bulls, for the rams, and for the male lambs, by their number according to the stipulation;
\verse and one male goat for a sin offering, in addition to the continual burnt offering and its grain offering, and their libations.
\verse “ ‘On the third day: eleven bulls, two rams, fourteen male lambs without defect \textit{in their first year};\lebnote{Literally “sons of a year”}
\verse and their grain offering and their libations for the bulls, for the rams, and for the male lambs, by their number according to the stipulation;
\verse and one male goat for a sin offering, in addition to the continual burnt offering, its grain offering, and its libation.
\verse “ ‘On the fourth day: ten bulls, two rams, fourteen male lambs without defect \textit{in their first year};\lebnote{Literally “sons of a year”}
\verse and their grain offering and their libations for the bulls, for the rams, and for the male lambs by their number according to the stipulation;
\verse and one male goat for a sin offering, in addition to the continual burnt offering, its grain offering, and its libation.
\verse “ ‘On the fifth day: nine bulls, two rams, fourteen male lambs without defect \textit{in their first year};\lebnote{Literally “sons of a year”}
\verse and their grain offering and their libations for the bulls, for the rams, and for the male lambs by their number according to the stipulation;
\verse and one male goat for a sin offering, in addition to the continual burnt offering, its grain offering, and its libation.
\verse “ ‘On the sixth day: eight bulls, two rams, fourteen male lambs without defect \textit{in their first year};\lebnote{Literally “sons of a year”} and their grain offering and their libations for the bulls, for the rams, and for the male lambs by their number according to the stipulation;
\verse and their grain offering and their libations for the bulls, for the rams, and for the male lambs by their number according to the stipulation;
\verse and one male goat for a sin offering, in addition to the continual burnt offering, its grain offering, and its libation.
\verse “ ‘On the seventh day: seven bulls, two rams, fourteen male lambs without defect \textit{in their first year};\lebnote{Literally “sons of a year”}
\verse and their grain offering and their libations for the bulls, for the rams, and for the male lambs by their number according to the stipulation;
\verse and one male goat for a sin offering, in addition to the continual burnt offering, its grain offering, and its libation.
\verse “ ‘On the eighth day you will have an assembly; \textit{you will not do any regular work}.\lebnote{Literally “you will not do work of labor”}
\verse You will present a burnt offering, an offering made by fire as a fragrance of appeasement for Adonai: one bull, one ram, seven male lambs without defect \textit{in their first year};\lebnote{Literally “sons of a year”}
\verse and their grain offering and their libations for the bulls, for the rams, and for the male lambs by their number according to the stipulation;
\verse and one male goat for a sin offering, in addition to the continual burnt offering, its grain offering, and its libation.
\verse “ ‘You will present these to Adonai at your appointed time, in addition to your vows and your freewill offerings, for your burnt offerings and for you grain offerings and for your libations and for your fellowship offerings.’ ”
\verse \lebnote{Numbers 29:40–30:16 in the English Bible is 30:1–17 in the English Bible} So Moses said to the \textit{Israelites}\lebnote{Literally “sons/children of Israel”} in accordance with all that Adonai commanded Moses.
\end{biblechapter}

\begin{biblechapter} % Numbers 30
\verseWithHeading{Laws for Vows} Then Moses spoke to \textit{the leaders}\lebnote{Literally “the heads”} of the tribes concerning the \textit{Israelites},\lebnote{Literally “sons/children of Israel”} saying, “This is the word that Adonai commanded:
\verse if a man makes a vow for Adonai or swears an oath with a binding pledge on himself, he must not render his word invalid; he must do all that went out from his mouth.
\verse “If a woman makes a vow to Adonai, and she binds a pledge on herself in her father’s house in your childhood,
\verse but if her father hears her vow or her pledge that she bound on herself and says nothing to her, then all her vows will stand, and every pledge that she binds on her life will stand.
\verse If her father forbids her on the day he hears of it, all her vows or her pledges that she bound on herself will not stand, and Adonai will forgive her because her father has forgiven her.
\verse “If \textit{she has a husband}\lebnote{Literally “she is to a man”} while bound by her vows or a rash promise of her lips,
\verse and her husband hears of it and is silent on the day he hears it, her vows will stand, and her pledge that she bound upon herself will stand.
\verse But if on the day her husband hears of it, he forbids her, then he will nullify her vow that she is under, and the rash promise of her lips that she bound on herself; and Adonai will forgive her.
\verse “But the vow of a widow or a woman who is divorced, all that she binds on herself will stand on her.
\verse But if she made a vow in her husband’s house, or bound herself on a pledge with a sworn oath,
\verse and her husband heard it but was silent to her, and he did not forbid her, all her vows will stand and every pledge that she bound on herself will stand.
\verse But if her husband nullified them on the day he hears them, all her vows going out of her lips concerning her vows or the pledge on herself will not stand; her husband has nullified them, and Adonai will forgive her.
\verse “Any vow and any sworn oath of a pledge to inflict on herself, her husband can confirm it or her husband can nullify it.
\verse But if her husband is completely silent from day to day, then he confirms all her vows or all her pledges that are on her; he confirms them because he was silent to her on the day he heard them.
\verse But if he indeed nullifies them after he hears them, then he will bear her guilt.”
\verse These are the decrees that Adonai commanded Moses, as between a husband and his wife, and between a father and his daughter, while her childhood is in her father’s house.
\end{biblechapter}

\begin{biblechapter} % Numbers 31
\verseWithHeading{War Against the Midianites} Adonai spoke to Moses, saying,
\verse “Seek vengeance for the \textit{Israelites}\lebnote{Literally “sons/children of Israel”} on the Midianites; afterward you will be gathered to your people.”
\verse Moses spoke to the people, saying, “Arm yourself from among your men for the battle, so that they will \textit{go}\lebnote{Literally “be”} against Midian to mete out the vengeance of Adonai on Midian.
\verse A thousand from each tribe of every tribe of Israel you will send to battle.”
\verse So theywere assigned from the thousands of Israel, a thousand from each tribe, twelve thousand equipped for battle.
\verse Moses sent them, a thousand from each tribe, to the battle, and Phinehas son of Eleazar the priest to the battle with them, and the vessels of the sanctuary and the trumpets of the blast were in his hand.
\verse And they fought against Midian just as Adonai commanded Moses, and they killed every male.
\verse They killed the kings of Midian in addition to the ones they had slain: Evi and Rekem and Zur and Hur and Reba, the five kings of Midian; they also killed Balaam son of Beor by the sword.
\verse The \textit{Israelites}\lebnote{Literally “sons/children of Israel”} took captive the women of Midian and their children, and they plundered all their domestic animals and all their livestock and all their wealth.
\verse They burned all their cities where they dwelled and all their camps with fire.
\verse They took all the plunder and all the war-booty with the humans\lebnote{Hebrew “human”} and domestic animals.\lebnote{Hebrew “animal”}
\verse They brought the captives, the war-booty, and the plunder to Moses, and to Eleazar the priest, and to the community of the \textit{Israelites},\lebnote{Literally “sons/children of Israel”} to the camp to the desert-plateau of Moab, which was on the Jordan across Jericho.
\verse And Moses and Eleazar the priest and all the leaders of the community went out to meet them outside the camp.
\verse But Moses was angry toward the leaders of the troops, the commanders of the thousands and the commanders of the hundreds, who came from the battle of the war.
\verse And Moses said to them, “You have kept alive every female?
\verse Behold, these women \textit{caused}\lebnote{Literally “were to”} the \textit{Israelites},\lebnote{Literally “sons/children of Israel”} by the word of Balaam, to be in apostasy against Adonai in the matter of Peor, so that the plague was among the community of Adonai.
\verse Now kill every male among the little children, and kill every woman who has \textit{had sexual intercourse with a man}.\lebnote{Literally “who has known the bed of a male”}
\verse But all the females who have not \textit{had sexual intercourse with a man},\lebnote{Literally “who has known the bed of a male”} keep alive for yourselves.
\verse And you, camp outside the camp seven days; all who killed a person and all who touched the slain purify yourselves on the third day and on the seventh day, you and your captives.
\verse You will purify yourselves and every garment and every object of hide and all the work of goats’ hair, and every object of wood.”
\verse Then Eleazar the priest said to the men of the battle who came from the war, “This is the decree of the law that Adonai commanded Moses.
\verse Only the gold and the silver, the bronze, the iron, the tin, and the lead—
\verse everything that will go through the fire—you will pass through the fire, and it will be clean, and only in waters of impurity will it be purified. Whatever does not go into the fire you will pass through the waters.
\verse And you will wash your garments on the seventh day and be clean, and afterward you will come into the camp.”
\verseWithHeading{Division of the War-Booty} Adonai said to Moses, saying,
\verse “You and Eleazar the priest and the leaders of the \textit{families}\lebnote{Literally “fathers”} of the community, \textit{take count}\lebnote{Literally “lift up the head”} of the war-booty that was captured, both humans\lebnote{Hebrew “human”} and the domestic animals;\lebnote{Hebrew “animal”}
\verse divide the war-booty between those who engaged in the war, who went out to the battle, and all the community.
\verse Exact a tribute for Adonai from the men of the war, those who went out to the battle, one from five hundred persons, and from the cattle and from the male donkeys and from the flock;
\verse take from their half and give it to Eleazar the priest as a contribution to Adonai.
\verse From half of the \textit{Israelites},\lebnote{Literally “sons/children of Israel”} take one share drawn by lot from the fifty from the humans,\lebnote{Hebrew “human”} from the cattle, from the male donkeys, from the flock, from all the domestic animals,\lebnote{Hebrew “animal”} and give them to the Levities who keep the responsibilities of the tabernacle of Adonai.”
\verse Moses and Eleazar the priest did just as Adonai commanded Moses.
\verse Thus the war-booty that remained of the plunder that the people of the battle plundered was six hundred and seventy-five thousand flocks of sheep,
\verse seventy-two thousand cattle,
\verse sixty-one thousand male donkeys,
\verse and the life of humankind, from the women who did not \textit{have sexual intercourse with a man},\lebnote{Literally “who has known the bed of a male”} all the persons\lebnote{Hebrew “person”} were thirty-two thousand.
\verse The half of the share that was going out to the battle: the number of the flock of sheep was thee hundred and thirty-seven thousand five hundred;
\verse the tribute to Adonai from the flock was six hundred and seventy-five;
\verse and the cattle were thirty-six thousand; and the tribute to Adonai was seventy-two.
\verse Of the male donkeys there were thirty thousand five hundred, and the tribute to Adonai was sixty-one;
\verse the humans\lebnote{Hebrew “human”} were sixteen thousand, and the tribute to Adonai was thirty-two persons.
\verse And Moses gave away the tribute of the contribution of Adonai to Eleazar the priest, just as Adonai commanded Moses.
\verse From the half of the \textit{Israelites},\lebnote{Literally “sons/children of Israel”} which Moses divided from the men who were fighting,
\verse the half that belonged to the community was three hundred and thirty-seven thousand five hundred from the flock,
\verse and thirty-six thousand cattle,
\verse and thirty thousand five hundred male donkeys,
\verse and sixteen thousand humans.\lebnote{Hebrew “human”}
\verse From the half that belonged to the \textit{Israelites},\lebnote{Literally “sons/children of Israel”} Moses took one share drawn by lot out of every fifty humans\lebnote{Hebrew “human”} and domestic animals, and he gave them to the Levites, who keep the responsibility of the tabernacle of Adonai, just as Adonai commanded Moses.
\verse Then the leaders of the thousands of the army, the commanders of the thousands and the commanders of the hundreds, approached Moses,
\verse and they said to Moses, “Your servants have \textit{taken count}\lebnote{Literally “lifted up the head”} of the men of war who were \textit{in our charge},\lebnote{Literally “in our hand”} and no man is missing from us.
\verse So we brought the offering of Adonai, what each man found, objects of gold, bangles,\lebnote{Hebrew “bangle”} bracelets,\lebnote{Hebrew “bracelet”} rings,\lebnote{Hebrew “ring”} earrings,\lebnote{Hebrew “earring”} and female ornaments,\lebnote{Hebrew “ornament”} to make atonement for ourselves \textit{before}\lebnote{Literally “in the presence of”} Adonai.”
\verse Moses and Eleazar the priest took the gold from them, all objects of work.
\verse All the gold of the contribution that they raised up to Adonai, from the commanders of the thousands and the commanders of the hundreds, was sixteen thousand seven hundred and fifty shekels.
\verse The men of battle plundered each for himself.
\verse So Moses and Eleazar the priest took the gold from the commanders of the thousands and hundreds, and they brought it to the tent of the assembly as a memorial for the \textit{Israelites}\lebnote{Literally “sons/children of Israel”} before Adonai.
\end{biblechapter}

\begin{biblechapter} % Numbers 32
\verseWithHeading{Gad and Reuben Inherit Gilead} The descendants\lebnote{Or “sons”} of Reuben and the descendants\lebnote{Or “sons”} of Gad had a very large number of livestock. And they saw the land of Jazer and the land of Gilead, and behold it was a place for livestock.
\verse The descendants\lebnote{Or “sons”} of Gad and the descendants\lebnote{Or “sons”} of Reuben came, and they said to Moses and to Eleazar the priest and to the leaders of the community, saying,
\verse “Ataroth, Dibon, Jazer, Nimrah, Heshbon, Elealeh, Sebam, Nebo, and Beon,
\verse the land that Adonai struck before the community of Israel, is a land of livestock, and your servants have livestock.”
\verse They said, “If we have found favor in \textit{your sight},\lebnote{Literally “your eyes”} let this land be given to your servants as property; do not lead us across the Jordan.”
\verse But Moses said to the descendants\lebnote{Or “sons”} of Gad and to the descendants\lebnote{Or “sons”} of Reuben, “Will your brothers go to war while you yourselves live here?
\verse Why are you discouraging the hearts of the \textit{Israelites}\lebnote{Literally “sons/children of Israel”} from crossing into the land that Adonai gave to them?
\verse This is what your fathers did when I sent them from Kadesh Barnea to see the land.
\verse When they went up to the valley\lebnote{Or “the wadi”} of Eshcol and saw the land, they discouraged the heart of the \textit{Israelites}\lebnote{Literally “sons/children of Israel”} so that they did not come to the land that Adonai gave to them.
\verse So \textit{Adonai’s anger burned}\lebnote{Literally “Adonai’s nose became hot”} on that day, and he swore an oath, saying,
\verse ‘The men who went up from Egypt, from \textit{those twenty years old}\lebnote{Literally “a son of twenty years”} and above, will not see the land that I swore with an oath to Abraham, Isaac, and Jacob because they have not wholly followed me,
\verse except Caleb son of Jephunneh the Kenizzite and Joshua son of Nun, because they followed Adonai wholly.’
\verse And \textit{Adonai became angry},\lebnote{Literally “Adonai’s nose became hot”} and he made them wander in the desert forty years until the entire generation who did evil in the \textit{sight of Adonai}\lebnote{Literally “eyes of Adonai”} \textit{had died}.\lebnote{Literally “had completed”}
\verse Behold, you stand in the place of your fathers, a brood of sinful men, to increase still more \textit{Adonai’s fierce anger}\lebnote{Literally “the fierce anger of Adonai’s nose”} against Israel.
\verse If you turn from following him, he will again abandon them\lebnote{Hebrew “him”} in the wilderness, and you would have destroyed all these people.”
\verse They came near to him and said, “We will build sheep pens here for the flock of our livestock and cities for our little children;
\verse but we ourselves will become armed and ready before the \textit{Israelites}\lebnote{Literally “sons/children of Israel”} until we have brought them to their place, and our little children will live in the fortified cities because of the inhabitants of the land.
\verse We will not return to our houses until the \textit{Israelites}\lebnote{Literally “sons/children of Israel”} each obtain their\lebnote{Hebrew “his”} inheritance for themselves.
\verse For we will not take possession with them from across the Jordan and beyond because our inheritance has come to us from across the Jordan to the east.”
\verse So Moses said to them, “If you do this thing, if you arm yourselves \textit{before}\lebnote{Literally “in the presence of”} Adonai for the war,
\verse and everyone of you armed cross the Jordan \textit{before}\lebnote{Literally “in the presence of”} Adonai until he has driven out his enemies from before him,
\verse and the land is subdued \textit{before}\lebnote{Literally “in the presence of”} Adonai, then afterward you will return and be free of obligation from Adonai and from Israel, and this land will be your property \textit{before}\lebnote{Literally “in the presence of”} Adonai.
\verse But if you do not do so, behold, you have sinned against Adonai, and know that your sin will find you.
\verse Build for yourselves cities for your little children and sheep pens for your flocks; what has gone out from your mouth you will do.”
\verse So the descendants\lebnote{Or “sons”} of Gad and the descendants\lebnote{Or “sons”} of Reuben said to Moses, saying, “Your servants will do just as my lord commands.
\verse Our little children, our wives, our livestock, and all of our animals\lebnote{Hebrew “animal”} \textit{will remain}\lebnote{Literally “will be there”} in the cities of Gilead,
\verse but your servants, everyone who is armed for battle, will cross over \textit{before}\lebnote{Literally “in the presence of”} Adonai to the war, just as my lord says.”
\verse So Moses commanded them, Eleazar the priest, Joshua son of Nun, and the heads of the \textit{families}\lebnote{Literally “fathers”} of the tribes of the \textit{Israelites}.\lebnote{Literally “sons/children of Israel”}
\verse Moses said to them, “If the descendants\lebnote{Or “sons”} of Gad and the descendants\lebnote{Or “sons”} of Reuben, everyone who is armed for the war, cross over the Jordan \textit{before}\lebnote{Literally “in the presence of”} Adonai, and the land is subdued before you, you will give them the land of Gilead as property.
\verse But if they will not cross over with you armed, they will acquire land in your midst in Canaan.”
\verse The descendants\lebnote{Or “sons”} of Gad and the descendants of Reuben answered and said, “What Adonai has commanded your servants, we will do.
\verse We ourselves will cross over armed \textit{before}\lebnote{Literally “in the presence of”} Adonai to the land of Canaan, and the property of our inheritance will remain with us beyond the Jordan.”
\verse So Moses gave to them, to the descendants\lebnote{Or “sons”} of Gad and the descendants\lebnote{Or “sons”} of Reuben, and to half of the tribe of Joseph’s son Manasseh, the kingdom of Sihon the king of Amorites and the kingdom of Og the king of the Bashan, the land with its cities and their territories, the cities of the surrounding land.
\verse The descendants\lebnote{Or “sons”} of Gad rebuilt Dibon, Ataroth, and Aroer,
\verse and Atroth Shophan, Jazer, and Jogbehah,
\verse and Beth Nimrah and Beth Haran, the cities of Mibzar, and the sheep pens for flocks.
\verse The descendants\lebnote{Or “sons”} of Reuben rebuilt Heshbon, Elealeh, and Kiriathaim,
\verse and Nebo, Baal Meon (their names\lebnote{Hebrew “name”} were changed), and Sibmah, and \textit{they renamed}\lebnote{Literally “they called to names the names”} the cities that they rebuilt.
\verse The descendants\lebnote{Or “sons”} of Makir son of Manasseh went to Gilead, and they captured it and drove out the Amorites\lebnote{Hebrew “Amorite”} who were in it.
\verse So Moses gave Gilead to Makir son of Manasseh, and he lived in it.
\verse And Jair son of Manasseh went and captured their unwalled villages, and he called them Havvoth Jair.
\verse Nobah went and captured Kenath and its villages, and he called it Nobah after his own name.
\end{biblechapter}

\begin{biblechapter} % Numbers 33
\verseWithHeading{The Travels Are Recounted} These were the journeys of the \textit{Israelites},\lebnote{Literally “sons/children of Israel”} who went out from the land of Egypt according to their divisions, by the hand of Moses and Aaron.
\verse Moses wrote down their movements according to their journeys on the command of Adonai, and these are their journeys according to their movements.
\verse They set out from Rameses on the first month, on the fifteenth day of the first month; on the next day after the Passover the \textit{Israelites}\lebnote{Literally “sons/children of Israel”} went out \textit{boldly}\lebnote{Literally “with a hand that was raised”} \textit{in the sight}\lebnote{Literally “for the eyes”} of all the Egyptians
\verse while the Egyptians were burying all the firstborn among them whom Adonai struck. Adonai also executed punishments among their gods.
\verse Then the \textit{Israelites}\lebnote{Literally “sons/children of Israel”} set out from Rameses, and they camped in Succoth.
\verse They journeyed from Succoth and camped in Etham, which is on the edge of the desert.
\verse Then they set out from Etham and returned to Pi-Hahiroth, which faces Baal Zephon, and they camped before Migdol.
\verse They set out from Pi-Hahiroth and went through the midst of the sea into the desert; and they went a journey of three days into the desert of Etham and camped at Marah.
\verse They set out from Marah and came to Elim, and in Elim there were twelve springs of water and seventy palm trees, and they camped there.
\verse They set out from Elim, and they camped at the \textit{Red Sea}.\lebnote{Literally “sea of reed”}
\verse They set out from the \textit{Red Sea}\lebnote{Literally “sea of reed”} and camped at the desert of Sin.
\verse They set out from the desert of Sin and camped at Dophkah.
\verse They set out from Dophkah and camped at Alush.
\verse They set out from Alush and encamped at Rephidim; and it was there that the people had no water to drink.
\verse They set out from Rephidim and camped in the desert of Sinai.
\verse The set out from the desert of Sinai and camped at Kibroth Hattaavah.
\verse They set out from Kibroth Hattaavah and camped at Hazeroth.
\verse They set out from Hazeroth and camped at Rithmah.
\verse They set out from Rithmah and camped at Rimmon Perez.
\verse They set out from Rimmon Perez and camped at Libnah.
\verse They set out from Libnah and camped at Rissah.
\verse They set out from Rissah and camped at Kehelathah.
\verse They set out from Kehelathah and camped at Mount Shapher.
\verse They set out from Mount Shapher and camped at Haradah.
\verse They set out from Haradah and camped at Makheloth.
\verse They set out from Makheloth and camped at Tahath.
\verse They set out from Tahath and camped at Terah.
\verse They set out from Terah and camped at Mithcah.
\verse They set out from Mithcah and camped at Hashmonah.
\verse They set out from Hashmonah and camped at Moserah.
\verse They set out from Moserah and camped at Bene-Jaakan.
\verse They set out from Bene-Jaakan and camped at Hor Haggidgad.
\verse They set out from Hor Haggidgad and camped at Jotbathah.
\verse They set out from Jotbathah and camped at Abronah.
\verse They set out from Abronah and camped at Ezion Geber.
\verse They set out from Ezion Geber and camped in the desert of Zin, that is, Kadesh.
\verse They set out from Kadesh and camped at Mount Hor, at the edge of the land of Edom.
\verse Aaron the priest went up to Mount Hor at the \textit{command}\lebnote{Literally “mouth”} of Adonai, and he died there in the fortieth year after the \textit{Israelites}\lebnote{Literally “sons/children of Israel”} had gone out from the land of Egypt, in the fifth month on the first day of the month.
\verse Aaron was one hundred and twenty-three years old when he died on Mount Hor.
\verse Now the Canaanite, the king of Arad, who was living in the Negev\lebnote{An arid region south of the Judean hills} in the land of Canaan, heard of the coming of the \textit{Israelites}.\lebnote{Literally “sons/children of Israel”}
\verse Then they set out from Mount Hor and camped at Zalmonah.
\verse They set out from Zalmonah and camped at Punon.
\verse They set out from Punon and camped at Oboth.
\verse They set out from Oboth and camped at Iye Abarim, the boundary of Moab.
\verse They set out from Iyim and camped at Dibon Gad.
\verse They set out from Dibon Gad and camped at Almon-Diblatayim.
\verse They set out from Almon-Diblatayim and camped in the mountains of Abarim, before Nebo.
\verse They set out from the mountains of Abarim and camped on the desert-plateau of Moab by the Jordan across Jericho.
\verse They camped by the Jordan, from Beth-Jeshimoth up to Abel Shittim, on the desert-plateau of Moab.
\verse Then Adonai spoke to Moses on the desert-plateau of Moab by the Jordan across Jericho, saying,
\verse “Speak to the \textit{Israelites}\lebnote{Literally “sons/children of Israel”} and say to them, ‘When you cross the Jordan into the land of Canaan,
\verse you will drive out the inhabitants of the land from your presence, and you will destroy all their idols and all the images of their molten idols, and you will demolish all their high places;
\verse you will dispossess the land and live in it because I have given the land to you to possess it.
\verse You will distribute the land by lot according to your clans; to the larger group you will give a larger inheritance, and to the smaller group you will give less inheritance. However the lot falls for him, there the lot will be. You will distribute it according to the tribes of your ancestors.\lebnote{Or “fathers”}
\verse But if you do not drive out the inhabitants of the land from your presence, then it will happen that whomever you let remain of them will be like irritants in your eyes and like thorns in your sides; they will be your enemies in the land in which you live.
\verse And just as I planned to do to them, I will do to you.’ ”
\end{biblechapter}

\begin{biblechapter} % Numbers 34
\verseWithHeading{The Land of Canaan Is Divided} Then Adonai spoke to Moses, saying,
\verse “Command the \textit{Israelites}\lebnote{Literally “sons/children of Israel”} and say to them, ‘When you come into the land of Canaan, this is the land that was allotted to you as an inheritance, the land of Canaan according to its boundaries.
\verse Your southern edge will be from the desert of Zin toward the side of Edom, and your southern border will be from the end of the Salt Sea\lebnote{That is, the Dead Sea} to the east;
\verse your boundary will turn from the south to the ascent of Akrabbim\lebnote{Or “Scorpions”} and will pass over to Zin, and its limits will be from the south of Kadesh Barnea; it will continue to Hazar Addar and pass over to Azmon.
\verse The boundary will turn from Azmon to the valley of Egypt, and its limits will be to the sea.
\verse “ ‘Your western boundary will be the Great Sea;\lebnote{That is, the Mediterranean} this will be your western boundary.
\verse Your northern border will be from the Great Sea;\lebnote{That is, the Mediterranean} you will make a boundary from the Great Sea\lebnote{That is, the Mediterranean} to Mount Hor.
\verse From Mount Hor you will make a boundary to reach Hamath; the limits of the territory will be at Zedad.
\verse The boundary will go out to Ziphron, and its limits will be at Hazar Enan. This will be your boundary to the north.
\verse “ ‘You will mark out your eastern boundary from Hazar Enan to Shepham;
\verse the boundary will go down from Shepham to Riblah from the east side of Ain, and the boundary will go down and border on the eastern side of the Sea of Kinnereth.
\verse The boundary will go down to the Jordan, and its limits will be at the Salt Sea.\lebnote{That is, the Dead Sea} This will be your land according to its boundaries all around.’ ”
\verse So Moses commanded the \textit{Israelites},\lebnote{Literally “sons/children of Israel”} saying, “This is the land that you will obtain as an inheritance for yourself by lot, which Adonai commanded to give to the nine and a half tribes.
\verse For the tribe of the children of the Reubenites,\lebnote{Hebrew “Reubenite”} the children of the Gadites,\lebnote{Hebrew “Gadite”} and the half-tribe of Manasseh took their inheritance according to the house of their families.\lebnote{Or “their fathers”}
\verse The two and a half tribes have taken their inheritance from beyond the Jordan across Jericho, east toward the sunrise.”
\verse Adonai spoke to Moses, saying,
\verse “These are the names of the men who divide up the land for your inheritance: Eleazar the priest and Joshua son of Nun.
\verse You will take one leader from every tribe to divide up the land for inheritance.
\verse These are the names of the men: of the tribe of Judah, Caleb son of Jephunneh;
\verse of the tribe of the descendants of Simeon, Samuel son of Ammihud;
\verse of the tribe of Benjamin, Elidad son of Chislon;
\verse of the tribe of the descendants\lebnote{Or “sons”} of Dan, the leader Bukki son of Jogli;
\verse of the descendants\lebnote{Or “sons”} of Joseph, the tribe of the descendants\lebnote{Or “sons”} of Manasseh, the leader Hanniel son of Ephod.
\verse Of the tribe of the descendants\lebnote{Or “sons”} of Ephraim, the leader Kemuel son of Shiphtan;
\verse of the tribe of the descendants\lebnote{Or “sons”} of Zebulum, the leader Elizaphan son of Parnach;
\verse of the tribe of the descendants\lebnote{Or “sons”} of Issachar, the leader Paltiel son of Azzan;
\verse of the tribe of the descendants\lebnote{Or “sons”} of Asher, the leader Ahihud son of Shelomi;
\verse of the tribe of the descendants\lebnote{Or “sons”} of Naphtali, the leader Pedahel son of Ammihud.”
\verse These are those whom Adonai commanded to allot to the \textit{Israelites}\lebnote{Literally “sons/children of Israel”} the land of Canaan.
\end{biblechapter}

\begin{biblechapter} % Numbers 35
\verseWithHeading{Cities for the Levites} Adonai spoke to Moses on the desert plains of Moab beyond the Jordan across Jericho, saying,
\verse “Command the \textit{Israelites}\lebnote{Literally “sons/children of Israel”} that they give to the Levites from the inheritance of their property cities to live in; and you will give to the Levites pastureland all around the cities.\lebnote{Hebrew “them”}
\verse The cities will be theirs to live in, and their pasturelands will be for their domestic animals, for their possessions, and their animals.\lebnote{Hebrew “animal”}
\verse “The pasturelands of the cities that you will give to the Levites will extend from the wall of the city to a distance of a thousand cubits all around.
\verse You will measure outside the city the eastern edge two thousand cubits, for the southern edge two thousand cubits, for the western edge two thousand cubits, and for the northern edge two thousand cubits, with the city in the middle; this will be for them the pasturelands of the cities.
\verse “All the cities that you will give the Levites will be six cities of refuge, to which the killer can flee; in addition to them you will give forty-two cities.
\verse All the cities that you will give to the Levites will be forty-eight cities, them with their pasturelands.
\verse And the cities that you will give from the property of the \textit{Israelites},\lebnote{Literally “sons/children of Israel”} you will take more from the larger group and less from the smaller group; each of them will give according to the portion of their inheritance according to the portion that he inherits.”
\verseWithHeading{Cities of Refuge} Adonai spoke to Moses, saying,
\verse “Speak to the \textit{Israelites}\lebnote{Literally “sons/children of Israel”} and say to them, ‘When you cross the Jordan into the land of Canaan,
\verse you will select for yourselves cities for your cities of refuge, that a killer who has killed a person unintentionally can flee there.
\verse The cities will be to you a refuge from a redeemer, so that the killer will not die until he stands before the community for judgment.
\verse The cities that you are to give will be your six cities of refuge.
\verse You will give three cities across the Jordan and three cities in the land of Canaan; they will be cities of refuge.
\verse To the \textit{Israelites},\lebnote{Literally “sons/children of Israel”} to the alien, and to the temporary resident in their midst there will be these six cities as a refuge to which anyone who unintentionally kills a person may flee.
\verse “ ‘But if he hit him with an object of iron, so that he dies, the killer must surely be put to death.
\verse And if he hit him with a stone in the hand, by which he will die, and he does die, he is a killer; the killer must surely be put to death.
\verse Or if he hit him with a wooden object, by which he will die, and he does die, he is a killer; the killer must surely be put to death.
\verse The blood avenger himself will put the killer to death; he must put him to death when meeting him.
\verse If he shoves him in hatred, or he throws something at him with intention, and he dies,
\verse or if he hits him in hostility with his hand, and he dies, the one that struck him will put to death the killer when meeting him.
\verse “ ‘Or if in an instant he shoved him, not in hostility, or threw something at him without intention,
\verse or with any stone, without seeing it dropped on him so that he dies, while he was not seeking his injury,
\verse then the community will judge between the striker and between the blood avenger according to these ordinances.
\verse The community will deliver the killer from the hand of the blood avenger, and the community will restore him to the city of his refuge to which he fled; and he will live there in it until the death of the high priest who was anointed with holy oil.
\verse But if the killer surely goes out of the territory of the city of his refuge to which he fled,
\verse and the blood avenger finds him outside the territory of the city of his refuge, and the blood avenger kills the killer, \textit{he will not be guilty of blood}\lebnote{Literally “there will not be blood for him”}
\verse because he must live in the city of his refuge until the death of the high priest. But after the death of the high priest the killer will return to the land of his property.
\verse These things will be as a decree of justice for you for your generations in all your dwellings.
\verse “ ‘If anyone kills a person, the killer will be put to death \textit{according to the testimony}\lebnote{Literally “according to the mouth”} of witnesses, but someone cannot die on testimony of one person.
\verse Also, you will not take a ransom payment for the life of a killer who is guilty of death; indeed, he must surely be put to death.
\verse You will not take a ransom payment for the one that flees to the city of his refuge, so that he may return to live in the land before the death of the priest.
\verse So you will not pollute the land in which you are; because blood pollutes the land, and no atonement can be made for the land for the blood that is poured out on it except with the blood of the one who poured it out.
\verse You will not defile the land on which you are living because I am living in the midst of it; I am Adonai; I am living in the midst of the \textit{Israelites}.’ ”\lebnote{Literally “sons/children of Israel”}
\end{biblechapter}

\begin{biblechapter} % Numbers 36
\verseWithHeading{Inheritance Through Marriage} \textit{The leaders}\lebnote{Literally “The heads”} of the families\lebnote{Or “fathers”} of the clans of descendants\lebnote{Or “sons”} of Gilead the son of Makir, the son of Manasseh, of the clans of the descendants\lebnote{Or “sons”} of Joseph came near and spoke \textit{before}\lebnote{Literally “before the face of”} Moses and \textit{before}\lebnote{Literally “before the face of”} \textit{the leaders}\lebnote{Literally “the heads”} of the families\lebnote{Hebrew “of the fathers”} of the \textit{Israelites}.\lebnote{Literally “sons/children of Israel”}
\verse And they said, “Adonai commanded my lord to give the land by lot as an inheritance to the \textit{Israelites},\lebnote{Literally “sons/children of Israel”} and my lord was commanded by Adonai to give the inheritance of Zelophehad our brother to his daughters.
\verse But if they become wives to one of the sons from another tribe of the \textit{Israelites},\lebnote{Literally “sons/children of Israel”} their inheritance will disappear from the inheritance of our ancestors,\lebnote{Or “fathers”} and it will be added to the inheritance of the tribe to which they belong; the lot of our inheritance would disappear.
\verse When the Jubilee of the \textit{Israelites}\lebnote{Literally “sons/children of Israel”} \textit{comes},\lebnote{Literally “will be”} it will be added to the inheritance of the tribe to which they belong; and their inheritance will disappear from the tribe of our father.”
\verse Then Moses commanded the \textit{Israelites}\lebnote{Literally “sons/children of Israel”} by the command of Adonai, saying, “The tribe of the descendants\lebnote{Or “sons”} of Joseph is right regarding what they are speaking.
\verse This is the word that Adonai commanded the daughters of Zelophehad, saying, ‘\textit{Let them marry}\lebnote{Literally “Let them be as wives”} \textit{whomever they like};\lebnote{Literally “they will be as the good in their eyes”} only \textit{they must marry}\lebnote{Literally “they must be as wives”} from within the clan of the tribe of their father.
\verse Thus an inheritance of the \textit{Israelites}\lebnote{Literally “sons/children of Israel”} will not go around from tribe to tribe. Rather, the inheritance of each tribe of his father will remain with the \textit{Israelites}.\lebnote{Literally “sons/children of Israel”}
\verse Every daughter who possesses an inheritance from the tribes of the \textit{Israelites}\lebnote{Literally “sons/children of Israel”} will \textit{marry}\lebnote{Literally “be a wife to”} one of the clan of the tribe of her father, so that the \textit{Israelites}\lebnote{Literally “sons/children of Israel”} will possess the inheritance of his ancestors.\lebnote{Or “fathers”}
\verse Therefore an inheritance will not go around from one tribe to another tribe because the tribes of the \textit{Israelites}\lebnote{Literally “sons/children of Israel”} will each hold to their own inheritance.’ ”
\verse Just as Adonai commanded to Moses, so the daughters of Zelophehad did:
\verse Mahlah, Tirzah, Hoglah, Milcah, and Noah, the daughters of Zelophehad, \textit{married}\lebnote{Literally “were as wives to”} the sons of their uncles.
\verse \textit{They married}\lebnote{Literally “They were as wives to”} those from the sons of Manasseh son of Joseph, and their inheritance \textit{remained}\lebnote{Literally “was”} among the tribe of the clan of their ancestors.\lebnote{Or “fathers”}
\verse These were the commands and the stipulations that Adonai commanded by the hand of Moses\lebnote{Or “through Moses”} to the \textit{Israelites}\lebnote{Literally “sons/children of Israel”} on the desert-plateaus of Moab by the Jordan across Jericho.
\end{biblechapter}

\flushcolsend
\biblebook{Deuteronomy}

\begin{biblechapter} % Deuteronomy 1
\verseWithHeading{Preamble} These \textit{are} the words that Moses spoke to all \textit{of} Israel \textit{on the other side of} the Jordan in the desert, in the desert plateau opposite Suph, between Paran and between Tophel and Laban and Hazeroth and Dizahab.
\verse It is \textit{a journey of} \textit{eleven days} from Herob \textit{by the way of Mount Seir} up to Kadesh Barnea.
\verse \textit{And it was} in the fortieth year, on the eleventh month, on \textit{the} first \textit{day} of the month, Moses spoke to the \textit{Israelites} according to all that Adonai had instructed him \textit{to speak} to them.
\verse \textit{This happened} \textit{after defeating} Sihon king of the Amorites, who \textit{was} reigning in Heshbon, and Og the king of Bashan, who \textit{was} reigning in Ashtaroth in Edrei.
\verse On the other side of  the Jordan in the land of Moab Moses began to explain this law, \textit{saying}:
\verseWithHeading{Historical Prologue} “Adonai our God spoke to us at Horeb, \textit{saying}, ‘You have stayed \textit{long enough} at this mountain.
\verse Turn \textit{now} and \textit{move on}, and go \textit{into} the hill of the Amorites and to all \textit{of} the neighboring regions in the Jordan \textit{Valley} in the hill country and in the Negev and in the coastal area along the sea, \textit{into} the land of the Canaanites and \textit{into} the Lebanon, as far as the great river Euphrates.
\verse Look, I have set the land \textit{before you}; go and take possession of the land that Adonai swore to your ancestors, to Abraham, to Isaac, and to Jacob, to give \textit{it} to them and to their offspring after them.’
\verse “And I spoke to them at that time, \textit{saying}, ‘I am not able to bear you alone.
\verse Adonai your God has multiplied you, and look; you \textit{are} today as the stars of the heaven \textit{in number}.
\verse Adonai, the God of your ancestors, may he add to you as you are \textit{now} a thousand times, and may he bless you just as he \textit{promised you}.
\verse How can I bear you \textit{by myself}, your burden and your load and your strife?
\verse Choose for yourselves \textit{wise and discerning and knowledgeable men} for \textit{each of} your tribes, and I will appoint them as your leaders.’
\verse “And you answered me, and you said, ‘The thing you have said to do is good.’
\verse And so I took the leaders of your tribes, wise and knowledgeable men, and \textit{then} I appointed them as leaders over you \textit{as} commanders of \textit{groups of} thousands and commanders of \textit{groups of} hundreds and commanders of \textit{groups of} fifties and commanders of \textit{groups of} tens \textit{as} officials for your tribes.
\verse And at that time I instructed your judges, saying, ‘\textit{hear out your fellow men}, and \textit{then} judge fairly between a man and between his brother and between \textit{his opponent who is a resident alien}.
\verse You must not \textit{be partial} \textit{in your judgment};  hear \textit{out} the small \textit{person} as \textit{also} the great \textit{person}; \textit{do not be intimidated by any person}, because the judgment \textit{is} God’s; and the case that is too difficult for you, bring \textit{it} to me, and I will hear it \textit{out}.’
\verse And so I instructed you at that time \textit{concerning} all \textit{of} the things that you should do.
\verse “Then we set out from Horeb, and we went \textit{through} the whole \textit{of} that great and terrible desert that you saw \textit{on} the way \textit{to} the hill country of the Amorites as Adonai our God had commanded us, and \textit{so} we came up to Kadesh Barnea.
\verse I said to you, ‘\textit{You have reached} the hill country of the Amorites that Adonai our God \textit{is} giving to us.
\verse See, Adonai your God has set before you the land; go up and possess \textit{it} as Adonai the God of your ancestors said to you; do not fear and do not be dismayed.’
\verse “Then all of you approached me, and \textit{you} said, ‘Let us send men \textit{before us}, and let them explore the land for us, and let them bring back \textit{a report} to us \textit{concerning} the way that \textit{we should take} \textit{and concerning} the cities that we shall come to.’
\verse The plan was good \textit{in my opinion}, and \textit{so} I took from \textit{among} you twelve men, \textit{one from each tribe}.
\verse And they \textit{set out} and \textit{went up into the hill country}, and they went up to the wadi of Eschol, and they spied out \textit{the land}.
\verse They took in their hands \textit{some of the fruit} of the land, and they brought \textit{it} down to us, and they brought to us back \textit{a report}, and they said, ‘The land that Adonai our God \textit{is} giving to us \textit{is} good.’
\verse But you were not willing to go up, and you rebelled against the \textit{command} of Adonai your God.
\verse And you grumbled in your tents, and you said, ‘Because of the hatred of Adonai \textit{toward} us he has brought us out from the land of Egypt to give us into the hand of the Amorites to destroy us.
\verse Where \textit{can} we go up? Our brothers have \textit{made our hearts melt}, \textit{saying}, “The people are greater and taller than we are, \textit{and there are} great fortified cities \textit{reaching} up to heaven, and we saw the sons of the Anakites \textit{living} there.” ’
\verse “And so I said to you, ‘Do not be terrified, and do not fear them.
\verse Adonai your God, who is going \textit{before you}, will himself fight for you, \textit{just as} he did for you in Egypt before your eyes,
\verse and \textit{just as he did} in the wilderness when you saw that Adonai your God carried you, just as someone carries his son, all \textit{along} the way \textit{that} you traveled until \textit{you reached} this place.’
\verse But through all of this you did not trust in Adonai your God,
\verse \textit{who goes} \textit{before you} on your way, seeking a place for your encampment, in fire at night and in a cloud by day, to show you the way that \textit{you should go}.
\verse “Then Adonai heard the sound of your words, and he was angry, and he swore, \textit{saying},
\verse ‘No one of these men of this evil generation will see the good land that I swore to give to your ancestors,
\verse except Caleb, the son of Jephunneh; he himself shall see it, and to him I will give the land upon which he has trodden and to his sons because \textit{he followed Adonai unreservedly}.’
\verse Even with me Adonai was angry because of you, saying, ‘Not even you shall enter there.
\verse Joshua, the son of Nun, \textit{your assistant}, will go there; encourage him because he will cause Israel to inherit it.
\verse And your little children, who you thought shall become plunder, and your sons, who do not today know good or bad, shall themselves go there, and I will give it to them, and they shall take possession of it.
\verse But you turn and set out \textit{in the direction of} the wilderness by way of the \textit{Red Sea}.’
\verse “You replied and said to me, ‘We have sinned against Adonai, \textit{and now} we will go up and fight according to all that Adonai our God commanded us’; and \textit{so} each man fastened on \textit{his battle gear}, and you regarded \textit{it} as easy to go up \textit{into} the hill country.
\verse So Adonai said to me, ‘Say to them, “You shall not go up, and you shall not fight because I am not in your midst; you will be defeated \textit{before} your enemies.” ’
\verse So I spoke to you, but you did not listen; you rebelled against the \textit{command of Adonai}; you behaved presumptuously, and you went up into the hill country.
\verse The Amorites living in the hill country went out \textit{to oppose you} and chased you as \textit{a swarm of} wild honey bees do; and \textit{so} they \textit{beat} you down in Seir as far as Hormah.
\verse So you returned and wept \textit{before Adonai}; but Adonai did not listen to your voice and did not pay \textit{any} attention to you.
\verse You stayed in Kadesh many days; such were the days that you stayed \textit{there}.
\end{biblechapter}

\begin{biblechapter} % Deuteronomy 2
\verse “\textit{Then} we turned and set out \textit{toward} the wilderness in the direction of the \textit{Red Sea}, as Adonai told me, and we went around Mount Seir \textit{for} many days.
\verse Adonai spoke to me, \textit{saying},
\verse ‘\textit{Long} enough you have been skirting this mountain; turn yourselves north,
\verse and instruct the people, \textit{saying}, “You \textit{are} about to cross \textit{through} the territory of your brothers, the descendants of Esau, who are living in Seir; they will be afraid of you, and \textit{so} be very careful.
\verse Do not get involved in battle with them, for I will not give you any of their land, not even \textit{a foot’s breadth} \textit{of it}; since I have given Mount Seir \textit{as} a possession for Esau.
\verse You shall buy food from them so that you may eat; and also you shall purchase water from them with money so that you may drink.
\verse The fact of the matter is, Adonai your God has blessed you in \textit{all the work you have done}; he knows \textit{your travels} \textit{with respect to} this great wilderness; forty years Adonai your God \textit{has been} with you; you have not lacked a thing.” ’
\verse And so we passed by our brothers, the descendants of Esau, who live in Seir, past the road of the Arabah, from Elath and Ezion Geber, and we turned and traveled along the route of the desert of Moab.
\verse And Adonai said to me, ‘You shall not attack Moab, and you shall not engage in war with them, for I will not give you any of his land \textit{as} a possession; I have given Ar to the descendants of Lot \textit{as} a possession.’
\verse (The Emim previously lived in it, a people large, numerous, and tall, like the Anakites.
\verse They were reckoned also \textit{as} Rephaim as the Anakites \textit{were}; but the Moabites called them Emim.
\verse The Horites previously lived in Seir, but the descendants of Esau dispossessed them and destroyed them \textit{from among themselves}, as Israel did with respect to the land of their possession that Adonai gave to them.)
\verse So \textit{now} arise and cross over the wadi of Zered yourselves; and \textit{so} we crossed the wadi of Zered.
\verse Now the \textit{length of time} that we had traveled from Kadesh Barnea until \textit{the time when} we crossed the wadi of Zered \textit{was} thirty-eight years, until the perishing of all of that generation; \textit{that is}, the men of war from the midst of the camp as Adonai had sworn to them.
\verse The hand of Adonai was against them to root them out from the midst of the camp until they perished completely.
\verse “\textit{And then} when all the men of war \textit{had died} from among the people,
\verse Adonai spoke to me, \textit{saying},
\verse ‘You \textit{are} about to cross over the boundary of Moab \textit{today} at Ar.
\verse When you approach \textit{the border of} the \textit{Ammonites}, you shall not harass them, and you shall not get involved in battle with them, for I have not given the land of \textit{the Ammonites} to you as a possession; because I have given it to the descendants of Lot \textit{as} a possession.
\verse (It is also considered the land of Rephaim; Rephaim lived in it \textit{previously}, and the Ammonites called them Zamzummim,
\verse a people great and numerous and as tall as the Anakites; Adonai destroyed them from before them, and they dispossessed them and settled in place of them,
\verse just as he did for the descendants of Esau, who live in Seir, when he destroyed the Horites from \textit{before them} and dispossessed them, and \textit{then} they settled in their place up to this day.
\verse And \textit{also} the Avvites, who lived in villages as far as Gaza, \textit{and the} Caphtorim, who came out from Caphtor, destroyed them and \textit{then} settled in their place.
\verse Arise, set out and cross \textit{over} the wadi of Arnon. Look! I have given Sihon the Amorite, the king of Heshbon, and his land into your hand; begin to take possession of \textit{it}, and engage with him in battle.
\verse This day I will begin to place \textit{the dread of you} and the fear of you \textit{before} the peoples under all the heavens. \textit{They} will hear \textit{the report about you}, and \textit{so} they will shake and tremble \textit{because of you}.’ 
\verse “So I sent messengers from the wilderness of Kedemoth to Sihon king of Heshbon; \textit{I sent} terms of peace, \textit{saying},
\verse ‘Let me cross through your land \textit{and} \textit{only along the road} I will go; I will not turn aside to the right or \textit{to the} left.
\verse Food for money you shall sell me, so that I may eat, and water for money you will give to me, so that I may drink; just let me cross on foot.
\verse \textit{Just as} the descendants of Esau did for me, who live in Seir, and the Moabites, who live in Ar, until I cross the Jordan into the land that Adonai our God is giving to us.’
\verse But Sihon king of Heshbon was not willing to let us cross through his \textit{territory} because Adonai your God hardened his spirit and \textit{made him obstinate} \textit{in order to give him} into your hand, \textit{just as he has now done}.
\verse Adonai said to me, ‘Look! I have begun to give \textit{over to you} Sihon and his land; begin \textit{to take possession of his land}.’
\verse Then Sihon and all his people came out to meet us for battle at Jahaz.
\verse And so Adonai our God gave him over \textit{to us}, and we struck him down, and his sons and all of his people.
\verse So we captured all of his cities at that time, and we destroyed each town \textit{of} males and the women and the children; we did not leave behind a survivor.
\verse We took only the livestock as spoil for ourselves, and \textit{also} the booty of the cities that we had captured.
\verse From Aroer, which \textit{is} on the edge of the wadi of Arnon and the city that \textit{was} in the wadi on up to Gilead, \textit{there} was not a city that was inaccessible to us; Adonai our God gave \textit{everything} \textit{to us}.
\verse Only the land of \textit{the Ammonites} you did not approach, \textit{all along} the whole upper region of the Jabbok \textit{River} and the towns of the hill country, \textit{according to all} that Adonai our God had instructed.
\end{biblechapter}

\begin{biblechapter} % Deuteronomy 3
\verse “Then we turned, and we went up the road to Bashan, and Og the king of Bashan came out to meet us, he and all \textit{of} his army for the battle \textit{at} Edrei.
\verse And Adonai said to me, ‘You should not fear him, for I have given him and all \textit{of} his army and his land into your hand. And \textit{so} you will do to him as you did to Sihon the king of the Amorites, who \textit{was} reigning in Heshbon.’
\verse And \textit{so} Adonai our God also gave Og the king of Bashan, and all of his army into our hand, and we struck him down until not a survivor remained to him.
\verse And we captured all \textit{of} his towns at that time; \textit{there} was not a city that we did not take from them.
\verse All of these \textit{were} fortified towns with high walls, gates, and bars, \textit{apart from} very many \textit{of} the villages of the open country.
\verse And \textit{so} we destroyed them just as we had done to Sihon the king of Heshbon; \textit{we destroyed} utterly each town of males, the women, and the little children.
\verse But all \textit{of} the livestock and the booty of the towns we kept as spoil for ourselves.
\verse “And so we took at that time the land from \textit{the control of} \textit{the} two kings of the Amorites who \textit{were} \textit{on the other side of the Jordan}, from the wadi of Arnon up to \textit{Mount Hermon}.
\verse (The Sidonians called Hermon ‘Sirion,’ and the Amorites called it ‘Senir.’)
\verse All of the towns of the plateau and the whole of Gilead and all of Bashan up to Salecah and Edrei, the towns of the kingdom of Og in Bashan.
\verse (For only Og, king of Bashan, was left from the remnant of the Rephaim. Indeed, his bedstead—it \textit{was} a bedstead of iron. It is in Rabbah of the \textit{Ammonites}. Nine cubits \textit{is} its length, and four cubits \textit{is} its width according to the cubit of a man.)
\verse And \textit{so} we took possession of this land at that time, from Aroer, which \textit{is} on the \textit{edge of the} wadi of Arnon, and \textit{also} half of the hill country of Gilead and its towns I gave to the Reubenites and to the Gadites.
\verse And the remainder of Gilead and all of Bashan, the kingdom of Og, I gave to the half-tribe of Manasseh, the whole region of Argo. All of that \textit{area of} Bashan was called \textit{the} land of the Rephaim.
\verse Jair the descendant of Manasseh acquired the whole region of Argob, up to the boundary of the Geshurites and the Maacathites, and he called it, \textit{that is} Bashan, after his \textit{own} name, Havvoth Jair, \textit{as it still is today}.
\verse And \textit{also} I gave Gilead to Makir.
\verse And to the Reubenites and to the Gadites I gave, from Gilead up to the wadi of Arnon, the middle of the wadi \textit{as a} boundary and up to the Jabbok \textit{River}, the boundary of the \textit{Ammonites}.
\verse And the \textit{Jordan Valley} \textit{with} the Jordan \textit{River as its} boundary, from Kinnereth up to the Sea of the Arabah, the Salt Sea, \textit{with} the slopes of Pisgah toward the east.
\verse “And I charged you \textit{all} at that time \textit{when I} said, “Adonai has given you—to \textit{all of} you—this land to possess. All the \textit{warriors} shall cross over, ready to fight, before your brothers, the \textit{Israelites}.
\verse Only your wives and your little children and your livestock (I know that \textit{you have much livestock}) must stay in your towns that I have given you,
\verse until Adonai shall give rest to your brothers as \textit{he did} to you, and also they take possession \textit{of} the land that Adonai your God \textit{is} giving to them beyond the Jordan; then they may return, each \textit{one} to his possession that I have given to them.
\verse And I commanded Joshua at that time, saying, ‘Your eyes see all that Adonai your God has done to these two kings; so Adonai will do to all \textit{of} the kingdoms where you are about to cross over to.
\verse You shall not fear them, for Adonai your God is the \textit{one} fighting for you.
\verse “And I pleaded with Adonai at that time, saying,
\verse ‘Lord Adonai, you have begun to show your servant your greatness and your strong hand, for what god \textit{is there} in the heaven or on the earth who can do according to your works and according to your mighty deeds?
\verse Let me cross over, please, and let me see the good land \textit{that is beyond the Jordan}, this good hill country and Lebanon.’
\verse But Adonai was very angry with me because of you, and he would not listen to me, and Adonai said, ‘\textit{Enough of that from you}! You shall not speak to me \textit{any} longer about this matter!
\verse Go up \textit{to} the top of Pisgah and \textit{look around you} toward the west, toward the north, and toward the east, and \textit{view} \textit{the land} with your eyes, for you will not cross this Jordan.
\verse Now instruct Joshua and support him and encourage him because he himself will cross \textit{over} before this people and enable them to inherit the land that you will see.’
\verse So we remained in the valley opposite Beth Peor.
\end{biblechapter}

\begin{biblechapter} % Deuteronomy 4
\verseWithHeading{Introduction to the Stipulations} “Now, Israel, listen to the rules and to the regulations that I \textit{am} teaching you to do, in order that you may live and you may go \textit{in} and you may take possession of the land that Adonai, the God of your ancestors, \textit{is} giving to you.
\verse You must not add to the word that I \textit{am} commanding you, and you shall not take away from it \textit{in order} to keep the commands of Adonai your God that I \textit{am} commanding you \textit{to observe}.
\verse Your eyes have seen what Adonai did with \textit{the case of} Baal Peor, for \textit{each} man that followed after Baal Peor Adonai your God destroyed from your midst.
\verse But you, the \textit{ones} holding fast to Adonai your God, \textit{are} all alive \textit{today}.
\verse See, I now teach you rules and regulations \textit{just} as Adonai my God has commanded me, to observe \textit{them} \textit{just} so in the midst of the land where you \textit{are} going, to take possession of it.
\verse And \textit{you must observe them diligently}, for that \textit{is} your wisdom and your insight before the eyes of the people, who will hear all \textit{of} these rules, and they will say, ‘Surely this great nation \textit{is} a wise and discerning people.’
\verse For what great nation \textit{has} for it a god near to it as Adonai our God, whenever we call \textit{upon} him?
\verse And what \textit{other} great nation has for it just rules and regulations just like \textit{this whole} law that I \textit{am} setting \textit{before} you \textit{today}?
\verse “However, \textit{take care} for yourself and watch your inner self closely, so that you do not forget the things that your eyes have seen, so that they do not slip from your mind all the days of your life; and you shall make them known to your children and to \textit{your grandchildren}.
\verse \textit{Remember} the day that you stood \textit{before} Adonai your God at Horeb \textit{when Adonai said to me}, ‘Summon for me the people so that I can tell them my words, that they may learn to fear me all \textit{of} the days they \textit{are} alive on the earth and so \textit{that} they may teach their children.’
\verse And \textit{so} you came near, and you stood under the mountain, and the mountain \textit{was} burning with fire up to the heart of the heaven, \textit{dark with a very thick cloud}.
\verse And Adonai spoke to you from the midst of the fire; \textit{you heard} a sound of words, but \textit{you did not see} a form—only a voice.
\verse And he declared to you his covenant, \textit{the Ten Commandments}, which he charged you \textit{to observe}, and he wrote them on \textit{the} two tablets of stone.
\verse And Adonai charged me at that time to teach you rules and regulations \textit{for your observation of them} in the land that you \textit{are} \textit{about to cross into} to take possession of it.
\verse “So \textit{you must be very careful for yourselves}, because you did not see any form on the day Adonai spoke to you at Horeb from the midst of the fire,
\verse so that you \textit{do} not \textit{ruin yourselves} and make for yourselves a divine image \textit{in} a form of any image, a replica of male or female,
\verse a replica of any animal that \textit{is} upon the earth, a replica of any winged bird that flies in the air,
\verse a replica of any creeping thing on the ground, a replica of any fish that \textit{is} in the water \textit{below} the earth.
\verse \textit{And do this so that you do not lift} your eyes \textit{toward} heaven and \textit{observe} the sun and the moon and the stars, all the host of the heaven, and be led astray and bow down to them and serve them, things that Adonai your God has allotted to all \textit{of} the peoples under all \textit{of} the heaven.
\verse But Adonai has taken you and brought you out from the furnace of iron, from Egypt, to be a people of inheritance to him, \textit{as it is this day}.
\verse “And Adonai was angry with me \textit{because of you}, and he swore \textit{that I would not cross the Jordan} and \textit{that I would not go to the good land} that Adonai your God is giving you as an inheritance.
\verse For I \textit{am} going to die in this land; I am not going to cross the Jordan, but you \textit{are} going to cross, and you are going to take possession of this good land.
\verse Watch \textit{out} for yourselves so that you do not forget the covenant of Adonai your God that he had \textit{made} with you and make for yourselves a divine image \textit{of} the form of anything that Adonai your God \textit{has forbidden},
\verse for Adonai your God \textit{is} a devouring fire, a jealous God.
\verse “When you have had children and \textit{grandchildren} and you have grown old in the land and you act corruptly and you make a divine image \textit{of} the form of anything and you do evil in the eyes of Adonai your God, \textit{thus} provoking him to anger,
\verse I call to witness against you today the heaven and the earth, that you will perish soon and completely from the land that you \textit{are} crossing the Jordan into it to take possession of it; \textit{you will not live long on it}, but you will be completely destroyed.
\verse And Adonai will scatter you among the peoples, and you will be left \textit{few in number} among the nations \textit{to where Adonai will lead you}.
\verse And you will there serve gods \textit{made by human hands}, \textit{of} wood and stone, \textit{gods} that cannot see and cannot hear and cannot eat and cannot smell.
\verse But from there you shall seek Adonai your God and will find him, if you seek him with all your heart and with all your soul.
\verse \textit{In your distress} when all these things have found you in the \textit{latter days}, then you will return to Adonai your God, and you will listen to his voice.
\verse For Adonai your God is a compassionate God; he \textit{will not abandon you}, and he will not destroy you, and he will not forget the covenant of your ancestors that he swore to them.
\verse “Yes, ask, please, about former days that \textit{preceded you} from the day that God created humankind on the earth; \textit{ask even} from \textit{one} end of the heaven up to the \textit{other} end of heaven \textit{whether anything ever happened} like this great thing or \textit{whether anything like it was ever heard}.
\verse Has a people \textit{ever} heard the voice of God speaking from the midst of the fire, \textit{just} as you heard \textit{it}, and lived?
\verse Or has a god \textit{ever} attempted to go to take for himself a nation from the midst of a nation, \textit{using} trials and signs and wonders and war, with an outstretched arm and with great and awesome deeds, like all that Adonai your God did for you in Egypt before your eyes?
\verse You \textit{yourselves} were shown \textit{this wonder} in order \textit{for you} to acknowledge that Adonai \textit{is} the God; there is no other \textit{God} \textit{besides him}.
\verse From heaven he made you hear his voice to teach you, and on the earth he showed you his great fire, and you heard his words from the midst of the fire.
\verse And because he loved your ancestors he chose their \textit{descendants} after them. And he brought you forth from Egypt \textit{with his own presence}, by his great strength,
\verse to drive out nations greater and more numerous than you \textit{from before you}, to bring you \textit{and} to give to you their land \textit{as} an inheritance, as \textit{it is} this day.
\verse So you shall acknowledge \textit{today}, and \textit{you must call to mind} that Adonai \textit{is} God in heaven above and on the earth beneath. There is no other \textit{God}.
\verse And you shall keep his rules and his commandments that I am commanding you \textit{today}, \textit{so} that \textit{it may go well} for you and for your children after you, and so that \textit{you may remain a long time} on the land that Adonai your God \textit{is} giving to you \textit{during} all \textit{of} those days.”
\verse Then Moses set apart three cities \textit{on the other side of the Jordan}, \textit{toward the east},
\verse \textit{in order for} \textit{a manslayer} to flee there \textit{who} has killed his neighbor \textit{without intent} and was not hating him \textit{previously}, and \textit{so} he could flee to one of these cities \textit{and be safe}.
\verse \textit{He set apart} Bezer in the wilderness in the land of the plateau of the Reubenites; Ramoth in Gilead of the Gadites, and Golan in Bashan of the Manassites.
\verse Now this \textit{is} the law that Moses set \textit{before} the \textit{Israelites};
\verse these are the legal provisions and the rules and the regulations that Moses spoke to the \textit{Israelites} \textit{when they left Egypt},
\verse beyond the Jordan in the valley opposite Beth Peor in the land of Sihon the king of the Amorites, who \textit{was} reigning in Heshbon \textit{and} whom Moses and the \textit{Israelites} defeated \textit{when they came out of Egypt}.
\verse And \textit{so} they took possession of his land and the land of Og king of Bashan, the two kings of the Amorites who \textit{were} beyond the Jordan, \textit{eastward},
\verse from Aroer, which \textit{is} on the bank of the wadi of Arnon and as far as Mount Sirion; that \textit{is}, Hermon,
\verse and all \textit{of} the Arabah beyond the Jordan, eastward, and as far as the Sea of the Arabah under the slopes of Pisgah.
\end{biblechapter}

\begin{biblechapter} % Deuteronomy 5
\verseWithHeading{Basic Stipulations} And \textit{then} Moses summoned all \textit{of} Israel and said to them, “Hear, Israel, the rules and the regulations that I \textit{am} speaking in your ears \textit{today}, and you shall learn them, and \textit{you must observe them diligently}.
\verse Adonai our God made a covenant with us at Horeb.
\verse \textit{It was} not with our ancestors \textit{that} Adonai made this covenant, \textit{but with} these \textit{of} us \textit{who are} here alive today.
\verse \textit{Face to face} Adonai spoke with you at the mountain from the midst of the fire.
\verse I was standing \textit{between Adonai and you} at that time to report to you the word of Adonai, for you were afraid because of \textit{the presence of} the fire, and \textit{so} you \textit{did} not go up the mountain. \textit{He said},
\verse ‘I \textit{am} Adonai your God, who brought you out from the land of Egypt, from the house of slavery.
\verse \textit{There} shall not be for you other gods \textit{besides me}.
\verse ‘You shall not make for yourself a divine image of any \textit{type} of form that \textit{is} in the heaven above or that \textit{is} on the earth beneath or that \textit{is} in the water under the earth.
\verse ‘You shall not bow down to them, and you shall not serve them, for I, Adonai your God, \textit{am} a jealous God, punishing the guilt of fathers upon \textit{their} children and upon \textit{the} third and upon \textit{the} fourth generation of \textit{those} hating me,
\verse but showing loyal love to thousands of \textit{those who} love me and of \textit{those who} keep my commandments.
\verse ‘You shall not take up the name of Adonai your God for \textit{a} worthless purpose, for Adonai will not leave unpunished \textit{anyone} who uses his name for \textit{a} worthless purpose.
\verse ‘Observe the \textit{Sabbath day} to make it holy, \textit{just} as Adonai your God has commanded you.
\verse Six days you shall work, and you shall do all \textit{of} your work,
\verse but the seventh day \textit{is} \textit{a} Sabbath unto Adonai your God; you shall not do any work, or your son, or your daughter, or your slave, or your slave woman, or your ox, or your donkey, or any of your domestic animals, or your \textit{resident} alien who \textit{is} in your \textit{towns}, so that your slave and your slave woman may rest as you \textit{rest}.
\verse And you shall remember that you were a slave in the land of Egypt, and Adonai your God brought you out with a strong hand and with an outstretched arm; therefore, Adonai your God commanded you to keep \textit{the Sabbath}.
\verse ‘Honor your father and your mother, as Adonai your God commanded you, so that it will be good for you \textit{in the land} that Adonai your God \textit{is} giving to you.
\verse ‘You shall not murder.
\verse ‘And you shall not commit adultery.
\verse ‘And you shall not steal.
\verse ‘And you shall not falsely bear evidence against your neighbor.
\verse ‘And you shall not covet the wife of your neighbor, and you shall not crave the house of your neighbor, his field or his slave or his slave woman or his ox and his donkey or anything \textit{that belongs to your neighbor}.’
\verse “These words Adonai spoke to your whole assembly at the mountain from the midst of the fire and the very thick cloud \textit{with} a loud voice, and \textit{he did not add anything}, and \textit{then} he wrote them on two tablets of stone and gave them to me.
\verse \textit{And then} \textit{when you heard} the voice from the midst of the darkness, and \textit{as} the mountain \textit{was} burning with fire, and \textit{and} all the heads of your tribes and your elders approached me,
\verse you said, ‘Look, Adonai our God has shown us his glory and his greatness, and we have heard his voice from the midst of the fire; this day we have seen that God can speak with a human being, \textit{yet he remains alive}.
\verse And so then why shall we die, for this great fire will consume us if \textit{we continue} to hear the voice of Adonai our God \textit{any} longer, and \textit{so} we shall die?
\verse For who \textit{is there of} all flesh who has heard the voice of the living God speaking from the midst of the fire as we \textit{have heard it} \textit{and remained alive}?
\verse \textit{You} go near and hear \textit{everything} that Adonai our God will say; and \textit{then} you tell us all that Adonai our God tells you, and we will listen, and we will do \textit{it}.’
\verse “And Adonai heard the sound of your words \textit{when you spoke to me}, and Adonai said to me, ‘I have heard the sound of the words of this people that they have spoken to you; \textit{they} are right \textit{with respect to} all that they have spoken.
\verse \textit{If only} \textit{they had such a mind}’; \textit{that is}, to fear me and to keep all my commandments \textit{at all times}, so that \textit{it will go well} for them and for their children \textit{forever}.
\verse Go! Say to them, “Return to your tents.”
\verse But you stand here with me, and let me tell you all \textit{of} the commandments and the rules and the regulations that you shall teach them, so that they may do \textit{them} in the land that I \textit{am} giving to them to take possession of it.’
\verse “\textit{So} you must be careful to do \textit{just} as Adonai your God commanded you; you shall not turn \textit{to the} right or \textit{to the} left.
\verse \textit{In exactly the path} that Adonai your God has commanded, you must go, so that you may live and \textit{it will go well} for you and \textit{you may live long} in the land that you will take possession of.”
\end{biblechapter}

\begin{biblechapter} % Deuteronomy 6
\verseWithHeading{Detailed Stipulations} “Now this \textit{is} the commandment, the rules and the regulations, that Adonai your God charged to teach \textit{to} you \textit{for you} \textit{to observe} in the land that you \textit{are} about to cross \textit{over} into to take possession of it,
\verse so that you may revere Adonai your God by keeping all his statutes and his commandments that I \textit{am} commanding you, you and your children and \textit{grandchildren}, all the days of your life and so you may live long \textit{lives}.
\verse And you shall hear, Israel, and be careful to observe \textit{these instructions}, so that \textit{it may go well for you} and that you may multiply greatly, \textit{just} as Adonai, the God of your ancestors, \textit{promised} you, \textit{in} a land with milk and honey.
\verse “Hear, Israel, Adonai our God, Adonai is unique.
\verse And you shall love Adonai your God with all of your heart and with all of your soul and with all of your might.
\verse And these words that I am commanding you \textit{today} shall be on your heart.
\verse And you shall recite them to your children, and you shall talk about them at \textit{the time of} your living in your house and at \textit{the time of} your going on the road and at \textit{the time of} your lying down and at \textit{the time of} your rising \textit{up}.
\verse And you shall bind them as a sign on your hand, and they shall be as an emblem between your eyes.
\verse And you shall write them on the doorframe of your house and on your gates.
\verse “And then it will happen \textit{that} when Adonai your God will bring you to the land that he swore to your ancestors, to Abraham, to Isaac, and to Jacob, to give to you large and fine cities that you did not build,
\verse and houses full of all \textit{sorts} of good things that you did not fill, and hewn cisterns that you did not hew, vineyards and olive groves that you did not plant, and \textit{you have eaten your fill},
\verse then take care for yourself, so that you \textit{do} not forget Adonai, who brought you out from the land of Egypt from the house of slavery.
\verse “You shall fear Adonai your God, and you shall serve him, and by his name you shall swear.
\verse You shall not go after other gods from the gods of the peoples who \textit{are} all around you,
\verse for Adonai your God \textit{is} a jealous god in your midst, \textit{so that the anger of Adonai your God would be kindled}, and he would destroy you from the face of the earth.
\verse You shall not put Adonai your God to \textit{the} test, as you tested \textit{him} at Massah.
\verse You shall diligently keep the commandments of Adonai your God and his legal provisions and his rules that he has commanded you.
\verse And you shall do \textit{what is} right and good in the eyes of Adonai, so that \textit{it shall go well} for you and \textit{so that} you may go and you may take possession of the good land that Adonai swore for your ancestors,
\verse by driving out all \textit{of} your enemies \textit{before you}, \textit{just} as Adonai \textit{has promised}.
\verse “When your child asks you in the future, \textit{saying}, ‘What \textit{is the meaning of} the legal provisions and the rules and the regulations that Adonai our God commanded for you?’
\verse Then you shall say to your child, ‘We were slaves of Pharaoh in Egypt, and Adonai brought us out from Egypt with a strong hand.
\verse And Adonai gave great and awesome signs and wonders in Egypt against Pharaoh and against his entire household \textit{in our presence}.
\verse But \textit{he} brought us out from there in order to bring us \textit{here} to give us the land that he swore to our ancestors.
\verse And \textit{so} Adonai commanded us to observe all \textit{of} these rules \textit{and} to revere Adonai our God \textit{for our benefit} \textit{all the days that we live}, \textit{as it is today}.
\verse And it shall be righteousness for us if we diligently observe \textit{and} do all \textit{of} this commandment \textit{before} Adonai our God, as he has commanded us.’
\end{biblechapter}

\begin{biblechapter} % Deuteronomy 7
\verse “When Adonai your God brings you into the land that you \textit{are} about to enter \textit{into it} to take possession of it, and he drives out many nations \textit{before you}, the Hittites and the Girgashites and the Amorites and the Canaanites and the Hivites and the Jebusites, seven nations mightier and more numerous than you,
\verse and Adonai your God will give them \textit{over to you} and you defeat them, you must \textit{utterly destroy them}; you shall not make a covenant with them, and you shall not show mercy \textit{to them}.
\verse And you shall not intermarry with them; you shall not give your daughter to their son; and you shall not take his daughter for your son.
\verse For their sons and daughters will cause your son to turn away \textit{from following me}, and \textit{so} they will serve other gods, and \textit{the anger of Adonai would be kindled} against you, and he would quickly destroy you.
\verse But this is what you must do to them: you shall break down their altars, and their stone pillars you shall smash, and their Asherah poles you shall hew down, and you shall burn their idols with fire.
\verse For you \textit{are} a holy people for Adonai your God; Adonai your God has chosen you to be for him a people, a treasured possession from \textit{among} all the peoples that \textit{are} on the face of the earth.
\verse “Adonai loved you and chose you not \textit{because of your great number} exceeding all \textit{other} peoples, for you \textit{are} fewer than all of the peoples,
\verse but \textit{because of} the love of Adonai \textit{for} you and because of his keeping \textit{of} the sworn oath that he swore to your ancestors, Adonai brought you \textit{out} with a strong hand and redeemed you from the house of slavery, from the hand of Pharaoh, the king of Egypt.
\verse \textit{So} know that Adonai your God, he \textit{is} God, the trustworthy God, maintaining his covenant and his loyal love with those who love him and with those who keep his commandments to a thousand generations,
\verse but repaying those who hate him \textit{in their own person} to destroy them; he is not slow with those who hate him \textit{in their own person}; he repays them.
\verse And \textit{so} you shall keep the commandment and the rules and the regulations that I \textit{am} commanding you \textit{today} to observe them.
\verse “\textit{And then} because you listen \textit{to} these regulations and you diligently keep and you do them, then Adonai your God will maintain his covenant and his loyal love that he swore to your ancestors.
\verse And he will love you, and he will bless you, and he will multiply you, and he will bless the fruit of your womb and the fruit of your soil, your grain, your wine, and your olive oil, and \textit{newborn} calves of your cattle, and the \textit{newborn} lambs of your flocks in the land that he swore to your ancestors to give you.
\verse You shall be blessed more than all of the peoples; among you there shall not be sterility and bareness, even among domestic animals.
\verse And Adonai will turn away from you all the illness and all the harmful diseases of Egypt that you experienced; he will not lay them on you, but he will lay them on all \textit{of} those who hate you.
\verse And you shall devour all of the peoples \textit{that} Adonai your God \textit{is} giving to you; \textit{you shall not pity them}, and you shall not serve their gods, \textit{which} will be a snare for you.
\verse “If you think in your heart, ‘These nations \textit{are} more numerous than I, so how can I dispossess them?’
\verse \textit{then remember} you must not be afraid of them; you must well remember what Adonai your God did to Pharaoh and to all \textit{of} Egypt:
\verse the great trials that your eyes saw and the signs and the wonders and the \textit{workings of} the strong hand and the outstretched arm \textit{by} which Adonai your God brought you out; so Adonai your God will do to all \textit{of} the peoples \textit{because of whom} you \textit{are} in fear \textit{before them}.
\verse And, moreover, Adonai your God will send the hornets among them until \textit{both} the survivors and the fugitives \textit{are destroyed} \textit{before you}.
\verse You must not \textit{be in dread} from the presence of them, because Adonai your God, \textit{who is} in your midst, \textit{is} a great and awesome God.
\verse And Adonai your God will clear away these nations \textit{from before you} little by little; you will not be able to finish them off quickly, \textit{lest} the \textit{wild animals} \textit{multiply} \textit{against you}.
\verse But Adonai your God will \textit{give them to you}, and he will throw them into great panic \textit{until they are destroyed}.
\verse And he will give their kings into your hand, and you shall blot out their names from under the heaven; anyone will not \textit{be able to} stand \textit{against you} \textit{until you destroy them}.
\verse You shall burn the images of their gods with fire; you shall not covet \textit{the} silver or gold \textit{that is} on them, and \textit{so} you take \textit{it} for yourself, so that you are not ensnared by it, for it \textit{is} a detestable thing to Adonai your God.
\verse And you must not bring a detestable thing into your house, or you will become a thing devoted to destruction like it; you must utterly detest it, and you must utterly abhor it, for it \textit{is} \textit{an} object devoted to destruction.
\end{biblechapter}

\begin{biblechapter} % Deuteronomy 8
\verse “All of the commandments that I \textit{am} commanding you today you must diligently observe, so that you may live and multiply, and you may go and take possession of the land that Adonai swore to your ancestors.
\verse And you shall remember all \textit{of} the way that Adonai your God led you \textit{during} these forty years in the desert in order to humble you and to test you to know what \textit{is} in your heart, whether you would \textit{diligently} keep his commandments or not.
\verse And he humbled you and let you go hungry, and \textit{then} he fed you with that which you did not know nor did your ancestors know, in order to make you know that not by bread alone but by all \textit{that} goes out of the mouth of Adonai humankind shall live.
\verse Your clothing did not wear out \textit{on you}, and your feet did not swell \textit{during} these forty years.
\verse And you should know with your heart that as a man disciplines his son, \textit{so} Adonai your God \textit{is} disciplining you.
\verse So you must keep the commandments of Adonai your God by walking in his ways and by fearing him.
\verse For Adonai your God \textit{is} bringing you to a good land \textit{with} streams of water, springs and underground water, welling up in the valleys and in the hills,
\verse \textit{to} a land of wheat and barley and vines and fig trees and pomegranate trees, a land of olive trees, olive oil and honey;
\verse \textit{to} a land where you may eat food in it \textit{without scarcity}; you will not find anything lacking in it, a land where its stones \textit{are} iron and from its mountains you can mine copper.
\verse And you will eat, and \textit{you will be satisfied}, and you will bless Adonai your God because \textit{of} the good land that he has given to you.
\verse “Take care for yourself so that you not forget Adonai your God \textit{by} not keeping his commandments and his regulations and his statutes that I \textit{am} commanding you \textit{today},
\verse lest \textit{when} you have eaten and you are satisfied and you have built good houses and you live \textit{in them},
\verse and your herds and your flocks have multiplied, and \textit{you have accumulated silver and gold}, and all that \textit{you have} has multiplied,
\verse then your heart \textit{becomes proud} and you forget Adonai your God, \textit{the one who brought you out} from the land of Egypt, from the house of slavery,
\verse the one leading you in the great and terrible desert \textit{infested} with \textit{dangerous} snakes and scorpions and parched ground, where there is no water, \textit{and} the one bringing out water for you from flint rock,
\verse the one \textit{feeding you} manna in the desert, \textit{food} that your ancestors did not know, in order to humble you and in order to test you \textit{so that he could} do good to you \textit{in the future}.
\verse And you may think in your heart, ‘My strength and the might of my hand \textit{acquired this wealth for} me.’
\verse But you must remember Adonai your God, for he \textit{is} the \textit{one} giving you strength to acquire wealth in order to confirm his covenant that he swore to your ancestors \textit{as it is today}.
\verse And it will happen \textit{that} if you indeed forget Adonai your God and you go after other gods and you serve them and you bow down before them, I warn you today that you will surely perish.
\verse As \textit{with} the nations that Adonai \textit{is} destroying \textit{before you}, so you shall perish because you would not obey the voice of Adonai your God.
\end{biblechapter}

\begin{biblechapter} % Deuteronomy 9
\verse “Hear, Israel, you \textit{are} about to cross the Jordan today to go to dispossess nations larger and more numerous than you, great cities fortified \textit{with high walls},
\verse a great and tall people, the sons of \textit{the} Anakites, whom you know and \textit{of whom you} have heard \textit{it said}, ‘Who could stand before the sons of Anak?’
\verse You should know \textit{today} that Adonai your God is the one crossing \textit{ahead of you} \textit{as} a devouring fire; he will destroy them, and he will subdue them before you; so you will dispossess them, and you will destroy them quickly, \textit{just as} Adonai \textit{promised} you.
\verse “You shall not say \textit{to yourself} \textit{when Adonai your God is driving them out} \textit{before you}, \textit{saying}, ‘Because of my righteousness Adonai brought me to take possession of this land’; but because of the wickedness of these nations Adonai \textit{is} driving them out \textit{before you}.
\verse \textit{It is} not because of your righteousness and because of the uprightness of your heart \textit{that} you \textit{are} coming to take possession of their land, but because of the wickedness of these nations Adonai your God \textit{is} driving them \textit{before you}, and in order to confirm the \textit{promise} that Adonai swore to your ancestors, to Abraham, to Isaac, and to Jacob.
\verse “So you should understand that \textit{it is} not because of your righteousness \textit{that} Adonai your God \textit{is} giving you this good land to take possession of it, because \textit{you are a stubborn people}. 
\verse Remember, \textit{do not forget}, that you provoked Adonai your God in the desert, \textit{and} from the day that you went out from the land of Egypt until \textit{you came to this place} you were rebelling against Adonai.
\verse “And \textit{remember} at Horeb you provoked Adonai, and Adonai became angry \textit{enough} to destroy you.
\verse \textit{When I went up the mountain} to receive \textit{the stone tablets}, the tablets of the covenant that Adonai \textit{made} with you, and remained on the mountain forty days and forty nights, I did not eat food and I did not drink water.
\verse And Adonai gave me the two tablets of stone written with the finger of God, and on them \textit{was writing} according to all the words that Adonai spoke with you at the mountain, from the midst of the fire on the day of the assembly.
\verse \textit{And then} at the end of forty days and forty nights, Adonai gave me the two tablets of stone, the tablets of the covenant.
\verse And Adonai said to me, ‘Come \textit{now}, go down quickly from this mountain because your people behave corruptly whom you brought out from Egypt, \textit{for} they turned quickly from the way that I commanded them \textit{to follow}; they have made for themselves a cast image.’
\verse And Adonai spoke to me, \textit{saying}, ‘I have seen this people, and look! \textit{They are a stubborn people}.
\verse Leave me alone, and let me destroy them, and let me blot out their name from under heaven, and let me make you into a nation mightier and more numerous than they!’
\verse “And I turned, and I went down the mountain, as the mountain was burning with fire, and the two tablets of the covenant \textit{were} in my two hands.
\verse And I looked, and indeed you had sinned against Adonai your God; you \textit{had} made for yourselves an image of a calf \textit{of} cast metal;  you \textit{had} turned quickly from the way that Adonai had commanded \textit{for} you.
\verse And I took hold of the two tablets, and I threw them out \textit{of} my two hands and smashed them before your eyes.
\verse And \textit{then} I lay prostrate \textit{before} Adonai, as earlier, forty days and forty nights; I did not eat food and I did not drink water because of all your sins that you committed, by doing evil in the eyes of Adonai \textit{and so} provoking him.
\verse For \textit{I was in dread} from \textit{being in} the presence of the anger and the wrath \textit{with} which Adonai was angry with you \textit{so as} to destroy you, but Adonai listened to me also \textit{at that time}.
\verse And with Aaron Adonai was \textit{angry enough} to destroy him, and I prayed also for Aaron at that time.
\verse And your sinful thing that you had made, the molten calf, I took and I burned it with fire, and I crushed it, grinding it thoroughly until it was crushed to dust, and I threw its dust into \textit{the stream that flowed down the mountain}.
\verse “And \textit{also} at Taberah and at Massah and at Kibroth Hattaavah you provoked Adonai to anger.
\verse And when Adonai sent you \textit{out} from Kadesh Barnea, \textit{saying}, ‘Go up and take possession of the land that I have given you,’ you rebelled against the command of Adonai your God, and you did not believe him, and you did not listen to his voice.
\verse You have been rebellious toward Adonai \textit{from the day I have known you}.
\verse “And I lay prostrate before Adonai through forty days, and through forty nights I prostrated myself, because Adonai intended to kill you.
\verse And I prayed to Adonai, and I said, ‘Lord Adonai, you must not destroy your people and your inheritance whom you redeemed in your greatness, whom you brought out from Egypt with a strong hand.
\verse Remember your servants, Abraham, Isaac, and Jacob; you must not \textit{pay attention to} the stubbornness of this people, to their wickedness and to their sin,
\verse lest \textit{the people of} the land from which you brought us out from there say, “Because Adonai was not able to bring them to the land that he \textit{promised} to them and because of his hatred \textit{toward} them, he has brought them out to kill them in the desert.”
\verse For they \textit{are} your people and your inheritance whom you brought with your great power and with your outstretched arm.’
\end{biblechapter}

\begin{biblechapter} % Deuteronomy 10
\verse “At that time Adonai said to me, ‘Carve for yourself two tablets of stone \textit{just} as the former \textit{ones}, and come up the mountain to me, and you shall make for yourself an ark of wood.
\verse And I will write on the tablets the words that were on the former \textit{tablets}, which you smashed, and you must put them in the ark.’
\verse And \textit{so} I made an ark \textit{of acacia wood}, and I carved two tablets of stone like the former \textit{ones}, and I went up the mountain \textit{with} the two tablets in my hand.
\verse And he wrote upon the tablets \textit{according to the first writing}, the ten words that Adonai spoke to you on the mountain from the midst of the fire on the day of the assembly, and Adonai gave them to me.
\verse And I turned, and I came down from the mountain, and I put the tablets in the ark that I had made, and they are there, \textit{just} as Adonai commanded me.
\verse “And the \textit{Israelites} journeyed from \textit{the} wells of Bene-Yaqan \textit{to} Moserah; there Aaron died and was buried, and Eleazar, his son, served as a priest in place of him.
\verse From there they journeyed to Gudgodah, and from Gudgodah to Jotbathah, a land \textit{flowing with} streams of water.
\verse At that time Adonai set apart the tribe of Levi to carry the ark of the covenant of Adonai to stand \textit{before} Adonai, to serve him and to bless \textit{the people} in his name until this day.
\verse Therefore \textit{there was not} for Levi an allotment or an inheritance \textit{along} with his brothers; \textit{rather} Adonai \textit{is} his inheritance \textit{just} as Adonai your God \textit{promised} to him.
\verse And I stayed on the mountain \textit{just} as \textit{during} the former forty days and forty nights, and Adonai listened to me also on that occasion; Adonai was not willing to destroy you.
\verse And Adonai said to me, ‘\textit{Come, continue} your journey \textit{before the people},  so that you may go and take possession of the land that I swore to their ancestors to give to them.’
\verse And now, Israel, what \textit{is} Adonai your God asking from you, except to revere Adonai your God, to go in all his ways and to love him and to serve Adonai your God with all your heart and with all your soul,
\verse to keep the commandments of Adonai and his statutes that I am commanding you today \textit{for your own good}.
\verse Look! For to Adonai your God \textit{belong} heaven and the \textit{highest heavens}, the earth and all that \textit{is} in it.
\verse \textit{Yet} to your ancestors Adonai was very attached, \textit{so as to} love them, and \textit{so} he chose their offspring after them, \textit{namely} you, from all the peoples, as it is \textit{today}.
\verse So you shall circumcise the foreskin of your heart, and \textit{do not be stubborn}.
\verse For Adonai your God, he \textit{is} God of the gods and Lord of the lords, the great and mighty God, the awesome \textit{one} who \textit{is not partial}, and he does not take bribes.
\verse \textit{And he} executes justice for the orphan and widow, and \textit{he is} one who loves \textit{the} alien, to give to them food and clothing.
\verse And you shall love the alien, for you were aliens in the land of Egypt.
\verse Adonai your God, you shall revere him, you shall serve him, and to him you shall cling, and by his name you shall swear.
\verse He \textit{is} your praise, and he \textit{is} your God, who has done with you these great and awesome \textit{things} that your eyes have seen.
\verse With \textit{only} seventy persons your ancestors went down to Egypt, but now Adonai your God has made you as the stars of the heaven \textit{with respect to multitude}.
\end{biblechapter}

\begin{biblechapter} % Deuteronomy 11
\verse “And you shall love Adonai your God, and you shall keep his obligations and his statutes and his regulations and his commandments \textit{always}.
\verse And you shall realize \textit{today} that \textit{it is} not \textit{with} your children who have not known and who have not seen the discipline of Adonai your God—his greatness, his strong hand, and his outstretched arm,
\verse and his signs and his deeds that he did in the midst of Egypt to Pharaoh, the king of Egypt, and to all \textit{of} his land,
\verse and what he did to the army of Egypt and to their horses and to their chariots, \textit{and} how he made the water of the \textit{Red Sea} flow over them, \textit{when they pursued after them}, and so Adonai has destroyed them, \textit{as is the case today},
\verse and what he did to you in the desert until you came up to this place,
\verse and what he did to Dathan and to Abiram, the sons of Eliab, the son of Reuben, how the earth opened its mouth wide and swallowed them, their households and their tents, and all \textit{of} the living creatures that \textit{were} in their possession \textit{and that were} \textit{following along with them} in the midst of all \textit{of} Israel.
\verse The fact of the matter is, \textit{your own eyes have seen} all \textit{of} the great deeds of Adonai that he did.
\verse “And you must keep all \textit{of} the commandments that I \textit{am} commanding you \textit{today}, so that you may have strength and \textit{you may} go and \textit{you may} take possession of that land \textit{to which you are crossing} to take possession of it,
\verse so that \textit{you may live long} on the land that Adonai swore to your ancestors, to give \textit{it} to them and to their offspring, a land flowing with milk and honey.
\verse For the fact is \textit{that} the land \textit{that you are about to go into} to take possession of it \textit{is} not like the land of Egypt, \textit{from} which you have \textit{come out of}, where you sow your seed and you give water by \textit{your} foot, \textit{as in a vegetable garden}.
\verse But the land that you \textit{are} about to cross into to take possession of it \textit{is} a land of hills and valleys, \textit{and} by the rain of the heaven it drinks water,
\verse a land that Adonai your God \textit{is} caring for it; continually the eyes of Adonai your God \textit{are} on it, from the beginning of the year up to the end of \textit{the} year.
\verse “And it will happen \textit{that} if you listen carefully to my commandments that I \textit{am} commanding you \textit{today}, to love Adonai your God, and to serve him with all \textit{of} your heart and with all \textit{of} your soul,
\verse then ‘I will send the rain for your land in its season, early rain and later rain, and you will gather your grain and your wine and your olive oil.
\verse And I will give vegetation in your field for your livestock, and \textit{you will eat your fill}.’
\verse \textit{Take care} so that your heart is not easily deceived, and you turn away, and you serve other gods, and you bow down to them.
\verse And \textit{then} \textit{the anger of Adonai will be kindled against you}, and he will shut up the heavens, and there shall not be rain, and \textit{so} the ground will not give its produce, and you will perish quickly from the good land that Adonai \textit{is} giving to you.
\verse “And you shall put these, my words, on your heart and on your inner self, and you shall bind them as a sign on your hand and let them be as an emblem between your eyes.
\verse And you shall teach them to your children by talking about them when you sit in your house and when you travel on the road and when you lie down and when you get up.
\verse And you shall write them on the doorframes of your house and on your gates,
\verse so that \textit{they} may multiply your days and the days of your children on the land that Adonai swore to your ancestors to give \textit{it} to them \textit{as long as heaven endures over the earth}.
\verse Yes, if you diligently keep all this commandment that I \textit{am} commanding you \textit{to observe it},  by loving Adonai your God \textit{by walking in} all his ways and by holding fast to him,
\verse then Adonai will drive out all \textit{of} these nations \textit{before you}, and you will dispossess nations larger and more numerous than you.
\verse Every place on which the sole of your foot treads, it shall be yours; your boundary shall be from the desert and Lebanon from the river, the river Euphrates, on up to the western sea.
\verse No one can take a stand \textit{against you}; your dread and your fear Adonai your God will put on the \textit{surface} of all the land \textit{where you tread}, \textit{just} as he \textit{promised} to you.
\verse “See, I am setting \textit{before you} \textit{today} a blessing and a curse:
\verse the blessing, if you listen to the commandments of Adonai your God that I \textit{am} commanding you \textit{today},
\verse and the curse, if you \textit{do} not listen to the commandments of Adonai your God, but \textit{rather} you turn from the way that I \textit{am} commanding you \textit{today} to go after other gods that you have not known.
\verse “And it will happen \textit{that} when Adonai your God has brought you to \textit{the land that you are going to}, to take possession of it, then you shall pronounce the blessing on Mount Gerizim and the curse on Mount Ebal.
\verse (Are they not beyond the Jordan, \textit{toward the west}, in the land of the Canaanites living in the Jordan Valley, opposite Gilgal beside the terebinth of Moreh?)
\verse For you are \textit{now} about to cross the Jordan to go to take possession of the land that Adonai, your God, \textit{is} giving to you, and you will take possession of it and live in it,
\verse and you must diligently \textit{observe} all the rules and the regulations that I \textit{am} setting \textit{before you} \textit{today}.”
\end{biblechapter}

\begin{biblechapter} % Deuteronomy 12
\verseWithHeading{Detailed Stipulations: Purity and Unity} “These \textit{are} the rules and the regulations \textit{that you must diligently observe} in the land that Adonai, \textit{the} God of your ancestors, has given to you to take possession of it, \textit{during} all the days that you \textit{are} living on the land.
\verse You must completely demolish all \textit{of} the places there where \textit{they} served their gods, \textit{that is}, the nations whom you \textit{are} about to dispossess, on the high mountains, and on the hills and under each leafy green tree.
\verse And you shall break down their altars, and you shall smash their stone pillars, and their Asherah poles you must burn with fire, and the images of their gods you shall hew down, and you shall blot out their names from that place.
\verse You shall not worship Adonai your God like this.
\verse \textit{But only} to the place that Adonai your God will choose from all \textit{of} your tribes to place his name there as his dwelling shall you seek, and there you shall go.
\verse And you shall bring there your burnt offerings and your sacrifices and your tithes and \textit{your donations} and your votive gifts and your freewill offerings and the firstling of your herd and your flock.
\verse And you shall eat there \textit{before} Adonai your God, and you shall rejoice \textit{in all your endeavors}, you and your family \textit{in} which Adonai your God has blessed you.
\verse “You must not do \textit{just} as we \textit{are} doing here \textit{today}, \textit{each according to all that is right in his eyes}.
\verse For you have not come up to now to the resting place and to the inheritance that Adonai your God \textit{is} giving to you.
\verse But you will cross the Jordan, and you will settle in the land that Adonai your God \textit{is} giving you as an inheritance, and he will give rest to you from all your enemies from all around, and you will live securely,
\verse \textit{and then} \textit{at} the place that Adonai your God will choose, to let his name dwell there, there you shall bring all \textit{the things} I \textit{am} commanding you, your burnt offerings and your sacrifices, your tithes and \textit{your donations}, and all \textit{of} the choice \textit{things}, your votive gifts that you vow to Adonai.
\verse And you shall rejoice \textit{before} Adonai your God, you and your sons and your daughters and your slaves and your slave women and the Levite who \textit{is} in your \textit{towns}, because there is not for him a plot of ground and an inheritance with you.
\verse “Take care for yourself so that you do not offer your burnt offerings at \textit{just} any place that you happen to see,
\verse \textit{but only} at the place that Adonai will choose among one of your tribes; there you shall offer your burnt offerings, and there you shall do all \textit{the things} that I \textit{am} commanding you.
\verse “But \textit{whenever you desire} you may slaughter, and you may eat meat according to the blessing of Adonai your God that he has given to you in all \textit{of} your \textit{towns}; the unclean and the clean may eat it \textit{just} as \textit{they would} the gazelle and as the deer.
\verse Only the blood you must not eat, \textit{but} on the ground you must pour it like water.
\verse You are not allowed to eat in your \textit{towns} the tithe of your grain and your wine and your olive oil and the firstborn of your herd and your flock and all \textit{of} your votive gifts that you vowed and your freewill offering and \textit{your donations}.
\verse But only \textit{before} Adonai your God you shall eat it, at the place that Adonai your God will choose, you and your son and your daughter and your slave and your slave woman and the Levite who \textit{is} in your \textit{towns}, and you must rejoice \textit{before} your God \textit{in all your undertakings}.
\verse \textit{Take care} so that you do not neglect the Levite all \textit{of} your days on your land.
\verse “When Adonai your God enlarges your territory \textit{just} as he has \textit{promised} to you, and you say ‘I want to eat \textit{some} meat,’ \textit{because you want it}, \textit{whenever you desire} you may eat meat.
\verse If the place that Adonai your God will choose to put his name there is \textit{too} far from you, and you slaughter any of your herd and any of your flock that Adonai has given to you \textit{just} as I have commanded you, then you may eat whenever you desire in your \textit{towns}.
\verse Surely \textit{just} as the gazelle and the deer is eaten, so \textit{both} the unclean and the clean together may eat it.
\verse Only, be sure not to eat the blood, because the blood is the life, and you shall not eat the life with the meat.
\verse You shall not eat it, \textit{but} on the ground you shall pour it out like water.
\verse You shall not eat it, so that \textit{it will go well} for you \textit{and} your children after you, because \textit{then} you will \textit{be} doing what is right in the eyes of Adonai.
\verse Only your holy objects \textit{that are yours} and your votive gifts you must carry and you must bring to the place that Adonai will choose.
\verse And you shall offer your burnt offerings, the flesh and the blood on the altar of Adonai your God, and the blood of your sacrifices you shall pour out on the altar of Adonai your God, but the meat you may eat.
\verse \textit{Be careful to obey} all these things that I \textit{am} commanding you, so that \textit{it will go well} for you and for your children after you \textit{forever}, because then you will \textit{be} doing what is good and right in the eyes of Adonai your God.
\verse “When Adonai your God has cut off the nations whom you \textit{are} \textit{about to go to}, to dispossess them \textit{before you}, and you have dispossessed them, and you live in their land,
\verse \textit{take care} so that you are not ensnared \textit{into imitating them} after their being destroyed from \textit{before you}, and so that you not enquire concerning their gods, \textit{saying}, ‘How did these nations serve their gods, and \textit{thus} \textit{I myself} want to do also.’
\verse You must not do so toward Adonai your God, because of every detestable thing they have done for their gods Adonai hates, for even their sons and their daughters they would burn in the fire to their gods.
\verse  All of the things that I \textit{am} commanding you, \textit{you must diligently observe}; you shall not add to it, and you shall not take away from it.”
\end{biblechapter}

\begin{biblechapter} % Deuteronomy 13
\verse “If a prophet stands up in your midst or a dreamer of dreams and he gives to you a sign or wonder,
\verse and the sign or the wonder comes \textit{about} that he spoke to you, \textit{saying}, ‘Let us go after other gods (those whom you have not known), and let us serve them,’
\verse you must not listen to the words of that prophet or to that dreamer, for Adonai your God \textit{is testing you to know whether you love} Adonai your God with all of your heart and with all \textit{of} your inner self.
\verse You shall go after Adonai your God, and him you shall revere, and his commandment you shall keep, and to his voice you shall listen, and him you shall serve, and to him you shall hold fast.
\verse But that prophet or the dreamer of that dream shall be executed, for he spoke falsely about Adonai your God, the \textit{one} bringing you out from the land of Egypt and the \textit{one} redeeming you from the house of slavery, in order to seduce you from the way that Adonai your God commanded you to go in it; so \textit{in this way} you shall purge the evil from your midst.
\verse “If your brother, the son of your mother or your son or your daughter \textit{or your wife whom you embrace} or \textit{your intimate friend} in secrecy \textit{says}, ‘Let us go and let us serve other gods!’ \textit{gods} that you and your ancestors have not known,
\verse from \textit{among} the gods of the people who are around you, \textit{those near you or those far from you}, from \textit{one} end of the earth and up to the \textit{other} end of the earth,
\verse you must not give in to him, and you shall not listen to him, and your eye shall not take pity on him, and you shall not have compassion, and you shall not cover \textit{up} for him.
\verse But you shall certainly kill him; your hand shall be first against him to kill him and next the hand of all \textit{of} the people.
\verse And you shall stone him with stones and let him die, for he tried to seduce you from Adonai your God, the \textit{one} bringing you from \textit{the} land of Egypt, from the house of slavery.
\verse And all \textit{of} Israel shall hear, and they shall fear, and \textit{they shall not continue to act} according to this evil thing in your midst.
\verse “If you hear in one of your towns which Adonai your God \textit{is} giving to you to live in, \textit{someone} \textit{saying that}
\verse \textit{worthless men} have gone out from your midst and have seduced the inhabitants of their town, \textit{saying}, ‘Let us go and serve other gods!’ whom you have not known,
\verse then you shall inquire and examine and interrogate thoroughly, and, look! \textit{It is} true; the thing has actually been done, this detestable thing in your midst,
\verse \textit{then} you shall certainly strike down the inhabitants of that town with the \textit{edge} of \textit{the} sword; you shall destroy it and everything in it, its domestic animals with \textit{the} \textit{edge} of \textit{the} sword.
\verse And \textit{then} you shall gather all of its booty into the middle of its public square, and you shall burn the town and all \textit{of} its war-booty totally for Adonai your God, and it shall be \textit{a pile of rubble forever}; it shall not be built again.
\verse And let not something cling to your hand from the things devoted to destruction, so that Adonai may turn back \textit{from his burning anger}, and he may show compassion to you and he may \textit{continue} to show compassion and \textit{so} multiply you \textit{just as he swore} to your ancestors,
\verse if you listen to the voice of Adonai your God, to keep all \textit{of} his commandments that I \textit{am} commanding \textit{to} you \textit{today} \textit{so as} to do the right \textit{thing} in the eyes of Adonai your God.”
\end{biblechapter}

\begin{biblechapter} % Deuteronomy 14
\verse “You \textit{are} children of Adonai your God; \textit{therefore} you must not gash yourself, and \textit{you must not make your forehead bald} for \textit{the} dead.
\verse For you \textit{are} a people holy to Adonai your God, and you Adonai has chosen to be a treasured possession from \textit{among} all \textit{of} the peoples that are on the surface of the earth.
\verse You shall not eat any detestable thing.
\verse These are the animals you may eat: ox, \textit{sheep, goats},
\verse deer, gazelle, roebuck, wild goat, ibex, antelope, and mountain sheep.
\verse And any animal having a split hoof and \textit{so} \textit{a dividing of the hoof into two parts} \textit{and that chews the cud} among the animals—that \textit{animal} you may eat.
\verse Only these you may not eat from \textit{those chewing the cud} and from \textit{those having a division of the hoof}: the camel and the hare and the coney, because \textit{they chew the cud}, but they \textit{do} not divide the hoof; they are \textit{therefore} unclean for you.
\verse And \textit{also} the pig \textit{because it has a division of the hoof} \textit{but does not chew the cud}; it \textit{is} unclean for you; from their meat you shall not eat, and you shall not touch their \textit{carcasses}.
\verse “This \textit{is what} you shall eat from all that \textit{is} in the water: \textit{everything} \textit{that has fins and scales} you may eat.
\verse But \textit{anything that does not have} fins and scales, you may not eat, \textit{for} it \textit{is} unclean for you.
\verse “All \textit{of} \textit{the} birds \textit{that} \textit{are} clean you may eat.
\verse Now these \textit{are} the ones you shall not eat \textit{any of them}: the eagle and the vulture and the short-toed eagle,
\verse and the red kite and the black kite or \textit{any kind of falcon},
\verse and any \textit{kind} of crow according to its kind,
\verse and the \textit{ostrich} and the short-eared owl and the seagull and the hawk according to its kind,
\verse the little owl and the great owl and the barn owl,
\verse and the desert owl and the carrion vulture and the cormorant,
\verse and the stork and the heron according to its kind and the hoopoe and the bat.
\verse And \textit{also} all \textit{of} \textit{the winged insects}; they \textit{are} unclean for you; you shall not eat \textit{them}.
\verse You may eat any clean bird.
\verse “You shall not eat any carcass; you may give it to the alien who \textit{is} in your \textit{towns}, and he may eat it, or you may sell \textit{it} to a foreigner, for you \textit{are} a holy people for Adonai your God; you may not boil a kid in its mother’s milk.
\verse “Certainly you must give a tithe \textit{of} all the yield of your seed, \textit{which comes forth from your field year after year}.
\verse And you shall eat \textit{before} Adonai your God in the place that he will choose to make to dwell his name there the tithe of your grain, your wine and your olive oil and the firstling of your herd and your flock, so that you may learn to revere Adonai your God \textit{always}.
\verse But if \textit{the distance is too great for you}, \textit{so} that you are not able to transport it, because the place that Adonai your God will choose to set his name there, it is \textit{too} far from you, when Adonai your God will bless you,
\verse then \textit{in that case} \textit{you may exchange for money}, and you shall take the money to your hand and go to the place that Adonai your God will choose.
\verse You may spend the money for anything \textit{that you desire}, for oxen or for sheep or for wine or for strong drink or for anything \textit{that you desire}, and you shall eat \textit{it} there \textit{before} Adonai your God, and you shall rejoice, you and your household.
\verse And \textit{as to} the Levite who \textit{is} in your \textit{towns}, you shall not neglect him, because there is not a plot of ground for him and an inheritance \textit{along} with you.
\verse “At the end of three years you shall bring out all \textit{of} the tithe of your yield for that year, and you shall store \textit{it} in your \textit{towns}.
\verse And \textit{so} the Levite may come, because there is no plot of ground for him or an inheritance with you, and the alien \textit{also may come} and the orphan and the widow that \textit{are} in your \textit{towns}, and \textit{they may eat their fill}, so that Adonai your God may bless you in all \textit{of} the work of your hand that you undertake.”
\end{biblechapter}

\begin{biblechapter} % Deuteronomy 15
\verse “At the end of seven years you shall grant \textit{a} remission of debt.
\verse And this \textit{is} the manner of the remission of debt: every \textit{creditor} shall remit his claim that he holds against his neighbor, and he shall not exact payment \textit{from} his brother because there \textit{a} remission of debt has been proclaimed unto Adonai.
\verse \textit{With respect to} the foreigner you may exact payment, but \textit{you must remit} what shall be \textit{owed} to you \textit{with respect to} your brother.
\verse Nevertheless, there shall not be among you a poor \textit{person}, because Adonai will certainly bless you in the land that Adonai your God \textit{is} giving to you \textit{as} an inheritance, to take possession of it.
\verse If only you listen well to the voice of Adonai your God \textit{by observing diligently} all of these commandments that I \textit{am} commanding you \textit{today}.
\verse When Adonai your God has blessed you, \textit{just} as he \textit{promised} to you, then you will lend to many nations, but you will not borrow \textit{from them}, and you will rule over many nations, but they will not rule over you.
\verse If \textit{there} is a poor \textit{person} among you from \textit{among} one of your brothers in one of your \textit{towns} that Adonai your God \textit{is} giving to you, you shall not harden your heart, and you shall not shut your hand toward \textit{your brother who is poor}.
\verse But you shall certainly open your hand for him, and \textit{you shall willingly lend} \textit{to} him enough to meet his need, \textit{whatever it is}.
\verse \textit{Take care} so that there will not be \textit{a thought of wickedness} in your heart, \textit{saying}, ‘The seventh year, the year of the remission of debt is near,’ \textit{and you view your needy neighbor with hostility}, and \textit{so} you \textit{do} not give to him, and he might cry \textit{out} against you to Adonai, and \textit{you would incur guilt against yourself}.
\verse By all means you must give to him, and \textit{you must not be discontented} at your giving to him, because on account of this \textit{very} thing, Adonai your God will bless you in all your work and \textit{in all that you undertake}.
\verse For the poor will not cease to be \textit{among you} \textit{in} the land; therefore I \textit{am} commanding you, \textit{saying}, ‘You shall willingly open your hand to your brother, to your needy and to your poor \textit{that are} in your land.’
\verse If your relative who is a Hebrew man or a Hebrew woman is sold to you, and \textit{he or she} has served you six years, then in the seventh year you shall send that person \textit{out} \textit{free}.
\verse And when you send him \textit{out} free from you, you shall not send him \textit{away} empty-handed.
\verse You shall generously supply him from \textit{among} your flocks and from your threshing floor and from your press; \textit{according to} that \textit{with which} Adonai your God has blessed you, you shall give to him.
\verse And remember that you were a slave in the land of Egypt, and Adonai your God redeemed you; therefore I \textit{am} commanding you thus \textit{today}.
\verse And then \textit{if} it will happen \textit{that} he says to you, ‘\textit{I do not want to go out} from you,’ because he loves you and your family, because it is good for him \textit{to be} with you;
\verse then you shall take an awl, and you shall thrust \textit{it} through his earlobe and into the door, and he shall be to you \textit{a slave forever}; and you shall also do likewise for your slave woman.
\verse It shall not be hard in your eyes \textit{when you send him forth free}, because for six years he has served you \textit{worth} twice the wage of a hired worker; and Adonai your God will bless you \textit{in whatever you will do}.
\verse “Every firstling male that is born of your herd and of your flock you shall consecrate to Adonai your God; you shall not do work with the firstling of your ox, and you shall not shear the firstling of your flock.
\verse \textit{Rather} \textit{before Adonai} your God you shall eat it year by year at the place Adonai will choose, you and your household.
\verse But if there is a physical defect in it, \textit{such as} lameness or blindness, any serious defect, you shall not sacrifice it to Adonai your God.
\verse In your \textit{towns} you shall eat it, the unclean and the clean together \textit{may eat it}, \textit{just} as \textit{they eat} the gazelle and as \textit{they eat} the deer.
\verse But you shall not eat its blood; you shall pour it on the ground like water.”
\end{biblechapter}

\begin{biblechapter} % Deuteronomy 16
\verse “Observe the month of Abib, and you shall keep \textit{the} Passover to Adonai your God, for in the month of Abib Adonai your God brought you out from Egypt \textit{by} night.
\verse And you shall offer the Passover sacrifice to Adonai your God \textit{from among} \textit{your} flock and herd at the place that Adonai will choose, to let his name dwell there.
\verse You shall not eat \textit{with it} anything leavened; seven days you shall eat \textit{with it} unleavened bread of affliction, because in haste you went out from the land of Egypt, so that you will remember the day of your going out from the land of Egypt all the days of your life.
\verse And leaven shall not be seen with you in any of your territory for seven days, and none of the meat that you will slaughter on the evening on the first day shall remain overnight until morning.
\verse You are not allowed to offer the Passover sacrifice in one of your \textit{towns} that Adonai your God is giving to you,
\verse but only at the place that Adonai your God will choose, to let his name dwell there; you shall offer the Passover sacrifice \textit{in the evening at sunset}, \textit{at the} designated time of your going out from Egypt.
\verse And you shall cook, and you shall eat \textit{it} at the place that Adonai your God will choose; and you may turn in the morning and go to your tents.
\verse Six days you shall eat unleavened bread, and on the seventh day \textit{there shall be} an assembly for Adonai your God; you shall not do work.
\verse “You shall count \textit{off} seven weeks for you; \textit{from the time you begin to harvest the standing grain} you shall begin to count seven weeks.
\verse And \textit{then} you shall celebrate the Feast of Weeks for Adonai your God with the measure of the freewill offering of your hand that you shall give \textit{just} as Adonai your God has blessed you.
\verse And you shall rejoice before Adonai your God, you and your son and your daughter and your slave and your slave woman and the Levite that \textit{is} in your \textit{towns} and the alien and the orphan and the widow who are in your midst in the place that Adonai your God will choose to let his name dwell there.
\verse And you shall remember that you were a slave in Egypt, and \textit{so} \textit{you shall diligently observe} these rules.
\verse “You shall celebrate the Feast of Booths for yourselves seven days \textit{at the gathering in of the produce} from your threshing floor and from your press;
\verse and you shall rejoice at your feast, you and your son and your daughter and your slave and your slave woman and the Levite and the orphan and the widow that \textit{are} in your \textit{towns}.
\verse Seven days you shall celebrate \textit{your} feast to Adonai your God at the place Adonai will choose, for Adonai your God shall bless you in all of your produce and in all \textit{of} the work of your hand, and you shall surely \textit{be} rejoicing.
\verse Three times in the year all \textit{of} your males shall appear \textit{before } Adonai your God at the place that he will choose, at the Feast of Unleavened Bread and at the Feast of Weeks and at the Feast of Booths, and they shall not appear \textit{before Adonai} empty-handed.
\verse Each \textit{person} \textit{shall give as he is able}, \textit{that is}, according to the blessing of Adonai your God that he has given to you.
\verse “You shall appoint judges and officials for you in all your \textit{towns} that Adonai your God \textit{is} giving to you throughout your tribes, and you shall render \textit{for} the people \textit{righteous judgments}.
\verse You shall not subvert justice; you shall not \textit{show partiality}; and you shall not take a bribe, for the bribe makes blind \textit{the} eyes of \textit{the} wise and misrepresents \textit{the} words of \textit{the} righteous.
\verse \textit{Justice, only justice} you shall pursue, so that you may live, and you shall take possession of the land that Adonai your God \textit{is} giving to you.
\verse You shall not plant for yourselves \textit{an Asherah pole} beside the altar of Adonai your God that you make for yourselves.
\verse And you shall not set up for yourselves a stone pillar, \textit{a thing} that Adonai your God hates.
\end{biblechapter}

\begin{biblechapter} % Deuteronomy 17
\verse “You shall not sacrifice to Adonai your God an ox or sheep \textit{that has a physical defect} \textit{of anything seriously wrong}, for that \textit{is} a detestable thing to Adonai your God.
\verse If \textit{there} is found in one of your \textit{towns} that Adonai your God \textit{is} giving to you a man or a woman that does evil in the eyes of Adonai your God to transgress his covenant
\verse and by going and serving other gods and \textit{so} he bows down to them and to the sun or to the moon or to any \textit{of} the host of heaven \textit{which I have forbidden},
\verse and it is reported to you or you hear \textit{about it} and you enquire \textit{about it} thoroughly and, indeed, \textit{the} trustworthiness of the deed \textit{has} been established, it \textit{has occurred}, this detestable thing, in Israel,
\verse then you shall bring out that man or that woman who has done this evil thing to your gates; that is, the man or the woman, and you shall stone them with stones \textit{to death}.
\verse \textit{On the evidence of} two or three witnesses \textit{the person shall be put to death}. The person shall not be put to death by the mouth of one witness.
\verse The hand of the witnesses shall be first against the person to kill the person, and afterward the hands \textit{of} all the people, and \textit{so} you shall purge the evil from your midst.
\verse “\textit{If a matter is too difficult for you}, \textit{for example disputes} between blood and blood, between legal claim and legal claim and between assault and assault \textit{and between} matters of discernment in your \textit{towns}, then you shall get up and you shall go to the place that Adonai your God will choose;
\verse then you shall go to the priests and the Levites and to the judge who will be \textit{in office} in those days, and you shall enquire, and they shall announce to you \textit{the verdict}.
\verse “And \textit{you shall carry out exactly the decision} that they announced to you from that place that Adonai will choose, and \textit{you shall diligently observe} according to all that they instruct you.
\verse \textit{And so} according to \textit{the instruction of the law} that they teach you and according to the decisions that they say to you, you shall do; you shall not turn from the word that they tell you to the right or \textit{to the} left.
\verse And the man who treats with contempt \textit{so as} not to listen to the priest \textit{who} \textit{is} standing to minister on behalf of Adonai your God or to the judge, that man shall die; so you shall purge the evil from Israel.
\verse And all the people will hear and will be afraid, and they will not behave presumptuously again.
\verse “When you have come to that land that Adonai your God \textit{is} giving to you and you have taken possession of it and you have settled in it, and you say, ‘I will set over me a king like all the nations that \textit{are} around me,’
\verse indeed, you may set a king over you whom Adonai your God will choose, from the midst of your countrymen you must set a king over you; you are not allowed to appoint over you a man, a foreigner, who \textit{is} not your countryman.
\verse Except, he may \textit{not make numerous} for himself horses, and he may not allow the people to \textit{to go} to Egypt \textit{in order to increase horses}, for Adonai has said to you that \textit{you may never return}.
\verse And he must not \textit{acquire many} wives for himself, so that his heart \textit{would} turn aside; and \textit{he must not accumulate silver and gold for himself excessively}.
\verse “\textit{And then} \textit{when he is sitting} on the throne of his kingdom, then he shall write for himself a copy of this law on a scroll \textit{before} the Levitical priests.
\verse And it shall be with him, and he shall read it all the days of his life, so that he may learn to revere Adonai your God by \textit{diligently observing} all the word of this law and these rules,
\verse \textit{so as} not to exalt his heart above his countrymen and not to turn aside from the commandment to the right or to the left, so that \textit{he may reign long over his kingdom}, he and his children in the midst of Israel.”
\end{biblechapter}

\begin{biblechapter} % Deuteronomy 18
\verse “And there shall not be for the Levitical priests, the whole tribe of Levi, a plot of ground and an inheritance with Israel, \textit{rather} they may eat an offering made by fire \textit{as} their inheritance, for Adonai.
\verse And there shall not be for them an inheritance \textit{of land} in the midst of his brothers; \textit{rather} Adonai is his inheritance, \textit{just} as he \textit{promised} to them.
\verse Now this shall be the share of the priest from the people, from \textit{those who sacrifice the sacrifice}, \textit{whether it is} an ox, sheep, or goat, and they shall give the priest the shoulder and the jawbones and the stomach.
\verse The firstfruits of your grain, your wine, and your olive oil and the firstfruits of the fleece of your sheep you shall give to him.
\verse For Adonai your God has chosen him from \textit{among} all your tribes to stand to minister in the name of Adonai, he and his sons \textit{forever}.
\verse And if a Levite comes from one of your \textit{towns} from \textit{anywhere in Israel} where he is residing, \textit{he may come whenever he desires}, to the place that Adonai will choose,
\verse and he may minister in the name of Adonai his God, \textit{just} like all his brothers, \textit{the Levites who stand there} \textit{before} Adonai.
\verse They shall eat \textit{equal portions}, apart from what he may receive from the sale of his patrimony.
\verse “When you come to the land that Adonai your God \textit{is} giving to you, you must not learn to do like the detestable practices of those nations.
\verse \textit{There} shall not be found among you one who makes his son or his daughter go through the fire, \textit{or} \textit{one who practices divination}, \textit{or} an interpreter of signs, or an augur, or sorcerer,
\verse or one who casts magic spells, or one who consults \textit{a} spirit of the dead, or spiritist, or one who inquires of the dead.
\verse For everyone doing these \textit{things} \textit{is} detestable to Adonai, and because of these detestable things Adonai your God \textit{is} driving them out from \textit{before you}.
\verse You must be blameless before Adonai your God.
\verse For these nations that you \textit{are} about to dispossess listen to interpreters of signs and to diviners, but Adonai your God has not allowed you to do the same.
\verse “Adonai your God will raise up for you a prophet like me from your midst, from your countrymen, \textit{and} to him you shall listen.
\verse \textit{This is} \textit{according to all that you asked} from Adonai your God at Horeb, on the day of the assembly, \textit{saying}, ‘\textit{I do not want again to hear} the voice of Adonai my God, \textit{and} I do not want to see again this great fire, so that I may not die!’
\verse And Adonai said to me, ‘They are right \textit{in} what they have said.
\verse I will raise up a prophet for them \textit{from among their countrymen} like you, and I will place my words into his mouth, and he shall speak to them \textit{everything that I command him}.
\verse \textit{And then} the man that will not listen to my words that he shall speak in my name, I will hold accountable.
\verse However, the prophet that behaves presumptuously by speaking a word in my name that I have not commanded him to speak, and who speaks in the name of other gods, then that prophet shall die.’
\verse And if you say \textit{to yourself}, ‘How can we know the word that Adonai has not spoken to him?’
\verse \textit{Whenever} what the prophet spoke in the name of Adonai, the thing does not take place and \textit{does} not come \textit{about}, that \textit{is} the thing that Adonai has not spoken to him. Presumptuously the prophet spoke it; you shall not fear that prophet.”
\end{biblechapter}

\begin{biblechapter} % Deuteronomy 19
\verse “When Adonai your God has \textit{exterminated} the nations \textit{concerning whom} Adonai your God \textit{is} giving to you their land, and you have dispossessed them, and you have settled in their towns and in their houses,
\verse you shall set apart three cities for you in the midst of your land that Adonai your God \textit{is} giving to you to take possession of it.
\verse You shall prepare the roads for yourselves, and you shall divide the regions of your land into thirds that Adonai your God gives you as a possession, so that \textit{it will be available for any manslayer to flee there}.
\verse “Now this \textit{is} the case of the manslayer who may flee there and live \textit{there} who has killed his neighbor \textit{unintentionally}, and he did not hate him \textit{previously}.
\verse \textit{For example}, when somebody goes with his neighbor into the forest to cut wood, and the iron \textit{head} slips from the handle \textit{of the tool} and strikes his neighbor and he dies, \textit{then} he may flee to one of these cities, and \textit{so} he may live.
\verse \textit{He does this} lest the avenger of blood might pursue after the killer, because \textit{he is hot with anger} and he overtakes him, because it is a long distance \textit{to the city of refuge}, and \textit{so} \textit{he kills him}, but \textit{he did not deserve a death sentence}, because he \textit{was} not hating him \textit{before}.
\verse Therefore I \textit{am} commanding you, \textit{saying}, ‘You shall set apart three cities.’
\verse Then if Adonai your God enlarges your territory \textit{just} as he swore to your ancestors and gives to you all the land that he \textit{promised} to give to your ancestors,
\verse \textit{then} \textit{if you diligently observe this entire commandment} that I \textit{am} commanding you \textit{today} by loving Adonai your God and by going in his ways \textit{at all times}, then you shall add three more cities for yourselves to these three.
\verse \textit{Do this} so that innocent blood will not be shed in the midst of your land that Adonai your God \textit{is} giving to you \textit{as an} inheritance and \textit{thereby} bloodguilt would be on you.
\verse But if someone hates his neighbor and lies in wait for him and rises up against him \textit{and murders him}, and \textit{the murderer} flees to one of these cities,
\verse then the elders of his city shall send and take him from there, and they shall give him into the hand of the avenger of blood, and he shall be put to death.
\verse Your eye shall not take pity on him, and you shall purge the guilt of innocent blood from Israel, \textit{so that good will be directed toward you}.
\verse “You shall not move the boundary \textit{marker} of your neighbor that \textit{former generations} set up on your property in the land that Adonai your God \textit{is} giving to you to take possession of it.
\verse \textit{The testimony of a single witness may not be used to convict} \textit{with respect to} any crime and for any wrongdoing in any offense that a person committed; on the \textit{evidence} of two witnesses or on the \textit{evidence} of three witnesses \textit{a charge shall be sustained}.
\verse If \textit{a malicious witness} gets up \textit{to accuse} anyone to testify against him falsely,
\verse then the two men \textit{to whom the legal dispute pertains} shall stand \textit{before} Adonai, \textit{before} the priests and the judges who are \textit{in office} in those days.
\verse Then judges shall make a thorough inquiry, and \textit{if} it turns out that the witness is a false witness \textit{and} he testified falsely against his brother,
\verse then you shall do to him as he meant to do to his brother, and so you shall purge the evil from your midst.
\verse \textit{And the rest} shall hear and shall fear, and \textit{they shall not continue to do such a thing again} as this evil thing in your midst.
\verse \textit{You must show no pity}: life for life, eye for eye, tooth for tooth, hand for hand, foot for foot.”
\end{biblechapter}

\begin{biblechapter} % Deuteronomy 20
\verse “If you go out to war against your enemies and you see a horse and a chariot, \textit{an army} larger that you, you shall not be afraid because of them; for Adonai your God \textit{is} with you, the one who brought you from the land of Egypt.
\verse \textit{And then} when you approach the battle, then the priest shall come near and speak to the troops.
\verse And he shall say to them, ‘Hear, Israel, you are near \textit{today} to the battle against your enemies; \textit{do not lose heart}; you shall not be afraid, and you shall not panic, and you shall not be terrified \textit{because of them},
\verse for Adonai your God \textit{is} going with you to fight for you against your enemies to help you.’
\verse And the officials shall speak to the troops, \textit{saying}, ‘Who \textit{is} the man who has built a new house and has not dedicated it? Let him go and return to this house, so that he does not die in battle and \textit{another man} dedicates it.
\verse And who \textit{is} the man that has planted a vineyard and has not enjoyed it? Let him go and let him return to his house, so that he does not die in battle and \textit{another man} enjoys it.
\verse And who \textit{is} the man who got engaged to a woman and \textit{has} not married her? Let him go and let him return to his house, so that he does not die in battle and \textit{another man} marries her.’
\verse And the officials shall continue to speak to the troops, and they shall say, ‘\textit{What man} is afraid \textit{and disheartened}? Let him go, and let him return to his house, and let him not cause the heart of his brothers to melt like his.’
\verse And \textit{when the officials have finished speaking} to the army troops, then they shall appoint commanders of divisions at the head of the troops.
\verse “When you approach a city to fight against it, \textit{you must offer it peace}.
\verse \textit{And then} if \textit{they accept your terms of peace} and \textit{they surrender to you}, \textit{and then} all the people \textit{inhabiting it} shall be forced labor for you, and they shall serve you.
\verse But if they \textit{do not accept your terms of peace} and they want to make war with you, then you shall lay siege against it.
\verse And Adonai your God will give it into your hand, and you shall kill all its males with the \textit{edge} of \textit{the} sword.
\verse Only the women and the little children and the domestic animals and all that shall be in the city, all of its spoil you may loot for yourselves, and you may enjoy the spoil of your enemies that Adonai you God has given to you.
\verse Thus you shall do to all the far cities from you, which \textit{are} not from the cities of these nations located \textit{nearby}.
\verse But from the cities of these peoples that Adonai your God \textit{is} giving to you as an inheritance, you shall not let anything live that breathes.
\verse Rather, you shall utterly destroy them, the Hittites and the Amorites, the Canaanites and the Perizzites, the Hivites, and the Jebusites, \textit{just} as Adonai your God has commanded you,
\verse so that they may not teach you to do like all their detestable things that they do for their gods and \textit{thereby} you sin against Adonai your God.
\verse “If you besiege a town \textit{for} many days to make war against it \textit{in order to} seize it, you shall not destroy its trees by wielding an ax against them, for you may eat from them, and \textit{so} you must not cut them down. Are the trees of the field humans that they should come in siege \textit{against you}?
\verse Only the trees that you know \textit{are not fruit trees} you may destroy and you may cut down, and you may build siege works against that city that is making war with you \textit{until it falls}.”
\end{biblechapter}

\begin{biblechapter} % Deuteronomy 21
\verse “If someone slain is found in the land that Adonai your God \textit{is} giving to you to take possession of it \textit{and} \textit{is} lying in the field, \textit{and} it is not known who \textit{killed him},
\verse then your elders and your judges shall go out and shall measure \textit{the distance} to the cities that \textit{are} around the slain one.
\verse \textit{And then} the nearest city to the slain one, the elders of that city shall take a heifer of the herd that has not been worked with \textit{in the field}, that has not pulled a yoke,
\verse and the elders of that city shall bring the heifer down to a \textit{wadi that flows with water all year} and \textit{that} has not been plowed and has not been sown; \textit{then} \textit{there they shall break the neck of the heifer in the wadi}.
\verse Then the priests, the descendants of Levi, shall come near, for Adonai your God has chosen them to bless in the name of Adonai, and every legal dispute and every \textit{case of} assault will be \textit{subject to their ruling}.
\verse And all of the elders of that city nearest to the slain person shall wash their hands over the heifer \textit{with} the broken neck in the wadi.
\verse And they shall declare, and they shall say, ‘Our hands did not shed this blood, and our eyes did not see \textit{what was done}.
\verse Forgive your people, Israel, whom you redeemed, Adonai, and \textit{do} not \textit{allow} the guilt of innocent blood in the midst of your people Israel, and let them be forgiven \textit{with regard to} blood.’
\verse And \textit{so} you shall purge the innocent blood from your midst, because you must do the right thing in the eyes of Adonai.
\verse “When you go out for battle against your enemies, and Adonai your God gives them into your hand, and you lead the captives away,
\verse and you see among the captives a woman beautiful in appearance, and you become attached to her and you want to take her as \textit{a} wife,
\verse then you shall bring her into your household, and she shall shave her head, and she shall trim her nails.
\verse And she shall remove the clothing of her captivity from her, and she shall remain in your house, and she shall mourn her father and her mother \textit{a full month}, and after this \textit{you may have sex with her}, and you may marry her, and she may \textit{become your wife}.
\verse And then if you do not take delight in her, then you shall let her go \textit{to do whatever she wants}, but you shall not treat her as a slave, since you have dishonored her.
\verse “If a man has two wives, \textit{and} the one \textit{is} loved and the \textit{other} one \textit{is} disliked and the one loved and the one that is disliked have borne for him sons, if it happens \textit{that} the firstborn son \textit{belongs to the one that is disliked},
\verse \textit{nevertheless} \textit{it will be the case that} on the day of bestowing his inheritance upon his sons, he will not be allowed to treat as \textit{the} firstborn son the son of the beloved \textit{wife} \textit{in preference to} the son of the disliked \textit{wife}, \textit{who is} the firstborn \textit{son}.
\verse But he shall acknowledge the firstborn son of the disliked \textit{wife} \textit{by giving} him a double portion of \textit{all that he has}, for he \textit{is} the firstfruit of his vigor; to him \textit{is} the legal claim of the birthright.
\verse “\textit{If a man has a stubborn and rebellious son} \textit{who} \textit{does not listen to} the voice of his father and to the voice of his mother, and they discipline him, and he does not obey them,
\verse then his father and his mother shall take hold of him, and they shall bring him out to the elders of his city and to the gate of his \textit{town},
\verse and they shall say to the elders of his city, ‘This our son \textit{is} stubborn and rebellious; \textit{he does not obey us}, \textit{and} he is a glutton and a drunkard.’
\verse Then all the men of his city shall stone him with stones and let him die; and so you shall purge the evil from your midst, and all of Israel will hear, and they will fear.
\verse “And \textit{if a man commits a sin punishable by death}, and \textit{so} he is put to death and you hang him on a tree,
\verse his dead body shall not hang on the tree, but certainly you shall bury him on that day, for cursed by God \textit{is} one that is \textit{being} hung; so you shall not defile your land that Adonai your God \textit{is} giving to you \textit{as an} inheritance.”
\end{biblechapter}

\begin{biblechapter} % Deuteronomy 22
\verse “You shall not watch the ox of your neighbor or his sheep or goat straying and ignore them; certainly you shall return them to your neighbor.
\verse And if your countryman \textit{is} not near you or you do not know \textit{who he is}, then you shall bring it \textit{to your household}, and it shall be with you \textit{until your countryman seeks after it}, and you shall return it to him.
\verse And thus \textit{also} you shall do regarding his donkey, and thus you shall do concerning his garment, and so you shall do with respect to all \textit{of} the lost property of your countryman that is lost from him and you find it; you are not allowed to withhold help.
\verse “You shall not see the donkey of your neighbor or his ox fallen on the road and you ignore them; certainly you must help them \textit{get} up \textit{along} with him.
\verse “The apparel of a man shall not be \textit{put} on a woman, and a man shall not wear the clothing of a woman, because everyone who does these things is detestable to Adonai your God.
\verse “If a bird’s nest is found \textit{before you} on the road in any tree or on the ground, \textit{and there are} chicks or eggs, and the mother \textit{is} lying down on the chicks or the eggs, you shall not take the mother along with the young;
\verse you shall certainly let the mother go, but you may take the young for yourselves; \textit{do this} \textit{so that it may go well} for you and \textit{you may live long in the land}.
\verse “When you build a new house then you shall make a parapet wall for your roof, so that you will not bring bloodguilt on your house \textit{if anyone should fall from it}.
\verse “You shall not sow your vineyard \textit{with} differing kinds \textit{of seed}, so that you shall not forfeit \textit{the whole harvest}, \textit{both} the seed that you sowed and the yield of the vineyard.
\verse “You shall not plow with an ox and with a donkey \textit{yoked} together.
\verse “You shall not wear woven material \textit{made of} wool and linen \textit{mixed} together.
\verse “You shall make tassels for yourselves on the four corners of your clothing with which you cover \textit{yourself}.
\verse “If a man takes a woman and \textit{he has sex with her}, but \textit{he} then \textit{dislikes her},
\verse and \textit{he accuses her falsely}, and \textit{he defames her}, and he says ‘This woman I took and I lay with her and \textit{I discovered that she was not a virgin},’
\verse then \textit{in defense} the father of the young woman shall take, along with her mother, and \textit{together} they must bring out the \textit{evidence of} the virginity of the young woman \textit{to display it} to the elders of the city \textit{at the city gate}.
\verse And \textit{then} the father of the young woman shall say to the elders, ‘I gave my daughter to this man as wife, but he \textit{now} \textit{dislikes} her,
\verse and now look \textit{he has accused her falsely}, saying, “I did not find \textit{your daughter a virgin},” but here \textit{is} \textit{evidence of} the virginity of my daughter’; and they shall spread the cloth \textit{out} \textit{before} the elders of the city.
\verse Then the elders of that city shall take the man, and they shall discipline him.
\verse Then they shall fine him \textit{a} hundred \textit{shekels of} silver, and they shall give \textit{them} to the father of the young woman, for \textit{he defamed an Israelite young woman}, and \textit{she shall become his wife}; he will not be allowed \textit{to divorce her} all his days.
\verse “But if \textit{this charge} was true, \textit{and the signs of virginity were not found} for the young woman,
\verse and \textit{then} they shall bring out the young woman to the doorway of the house of her father, and the men of her city shall stone her with stones, and she shall die, because she did a disgraceful thing in Israel \textit{by playing the harlot} \textit{in} the house of her father, and so you shall purge the evil from your midst.
\verse “If a man is found lying \textit{with a married woman}, then they shall both die; \textit{both of them}, the man who lay with the woman and the woman \textit{also}, so you shall purge the evil from Israel.
\verse “If it happens \textit{that} a young woman, a virgin, \textit{is} engaged to a man, \textit{and} a man finds her in the town and lies with her,
\verse then you shall bring out \textit{both of them} to the gate of that city, and you shall stone them with stones so that they shall die, the young woman because she did not cry out in the town, and the man because he violated his neighbor’s wife; and so you shall purge the evil from your midst.
\verse “But if the man finds the young engaged woman in the field and the man overpowers her and \textit{he has sex with her}, then the man only must die who lay with her.
\verse But you shall not do anything to the young woman, \textit{for} there is not \textit{reckoned} against the young woman \textit{a sin deserving death}; \textit{it is similar to when} a man rises up against his neighbor and murders him, \textit{a fellow human being}, \textit{just} so \textit{is} this \textit{case},
\verse for he found her in the field, the engaged young woman cried out, but there was no rescuer \textit{to help her}.
\verse “If a man finds a young woman, a virgin \textit{who} is not engaged, and he seizes her and \textit{he has sex with her} and they are caught,
\verse then \textit{the man who lay with her} shall give to the father of the young woman fifty \textit{shekels} of silver, and she shall become \textit{his wife} \textit{because} he violated her, and he is not allowed to divorce her \textit{during his lifetime}.
\verse  A man may not take the wife of his father, and \textit{so} \textit{he may not dishonor his father}.
\end{biblechapter}

\begin{biblechapter} % Deuteronomy 23
\verse “No man \textit{with crushed testicles} or \textit{whose} \textit{male organ is cut off} may come into the assembly of Adonai.
\verse An illegitimate child may not come into the assembly of Adonai; even \textit{to} the tenth generation none \textit{of his descendants} may come into the assembly of Adonai.
\verse An Ammonite or a Moabite may not come into the assembly of Adonai; even \textit{to} the tenth generation none \textit{of his descendants} may come into the assembly of Adonai \textit{forever},
\verse \textit{because} they did not come to meet you with food and with water \textit{when you came out of Egypt}, and \textit{also} \textit{because} they hired Balaam, son of Beor, from Pethor, in Aram Naharaim \textit{to act} against you to curse you.
\verse But Adonai your God was not willing to listen to Balaam, and Adonai your God turned the curse into a blessing for you, because Adonai your God loved you.
\verse You shall not promote their welfare or their prosperity all your days \textit{forever}.
\verse “You shall not abhor an Edomite, because he \textit{is} your brother; you shall not abhor an Egyptian because you were an alien in his land.
\verse The children \textit{that} are born to them \textit{in} the third generation may come \textit{representing them} in the assembly of Adonai.
\verse “If you go out to encamp against your enemies, then you shall guard against \textit{doing} anything evil.
\verse “If \textit{there} is among you a man that is not clean because of a seminal emission \textit{during the night}, he shall go outside the camp; he shall not come within the camp.
\verse \textit{And then} toward the \textit{coming} of the evening, he shall bathe with water, and at the going down of the sun, he may come to the midst of the camp.
\verse “And there shall be for you a designated place outside the camp; \textit{and you shall go there to relieve yourself},
\verse and a digging tool shall \textit{be included} in addition to your \textit{other} utensils for yourself; \textit{and then} \textit{when you relieve yourself} outside \textit{the camp} you shall dig with it, and \textit{then} you shall turn, and you shall cover your excrement.
\verse For Adonai your God \textit{is} walking about in the midst of your camp to deliver you and \textit{to hand your enemies over to you before you}, and so let your camp be holy, so that he shall not see in it \textit{anything indecent}, and he shall turn away \textit{from going with you}.
\verse “And you shall not hand over a slave to his master who has escaped \textit{and fled} to you from his master.
\verse He shall reside with you in your midst in the place that he chooses in one of \textit{your towns wherever he pleases}; you shall not oppress him.
\verse “No woman \textit{of Israel} shall be a temple prostitute, and no man \textit{of Israel} shall be a male shrine prostitute.
\verse You may not bring the \textit{hire} of a prostitute or \textit{the earnings of a male prostitute} \textit{into} the house of Adonai your God, for any vow offerings, because \textit{both} are a detestable thing to Adonai your God.
\verse “\textit{You shall not charge your brother interest on money}, interest on food, or interest on anything that one could lend on interest.
\verse You may lend on interest to the foreigner, but to your countryman you may not lend on interest, so that Adonai your God may bless you \textit{in all your undertakings} in \textit{the land where you are going}, \textit{in order to take possession of it}.
\verse “\textit{When you make a vow} to Adonai your God, you shall not postpone \textit{fulfillment of it}, \textit{for} certainly Adonai your God shall require it from you and \textit{if postponed} \textit{you will incur guilt}.
\verse And \textit{if you refrain from vowing}, \textit{you shall not incur guilt}.
\verse The utterance of your lips \textit{you must perform diligently} \textit{just} as you have vowed freely to Adonai your God whatever \textit{it was} that you promised with your mouth.
\verse “When you come into the vineyard of your neighbor, then you may eat grapes \textit{as you please} and \textit{until you are full}, but you shall not put \textit{any} into your container.
\verse “When you come into the standing grain of your neighbor, then you may pluck ears with your hand, but you may not \textit{swing} a sickle among the standing grain of your neighbor.”
\end{biblechapter}

\begin{biblechapter} % Deuteronomy 24
\verse “When a man takes a wife and he marries her \textit{and then} \textit{she does not please him}, because he found \textit{something objectionable} and writes her a letter of divorce and puts \textit{it} in her hand and sends her \textit{away} from his house,
\verse and she goes from his house, and she goes \textit{out} and becomes \textit{a wife} \textit{for another man},
\verse and \textit{then} the second man dislikes her and he writes her a letter of divorce and places \textit{it} into her hand and sends her from his house, or if the second man dies who took her \textit{to himself} as a wife,
\verse her first husband who sent her \textit{away} is not allowed \textit{to take her again} to become a wife to him after she has \textit{been defiled}, for that \textit{is} a detestable thing \textit{before} Adonai, and \textit{so} you shall not mislead into sin the land that Adonai your God \textit{is} giving to you \textit{as an} inheritance.
\verse “When a man takes a new wife he shall not go out with the army, and \textit{he shall not be obligated with anything}; he shall be free from obligation, \textit{to stay at home} for one year, and he shall bring joy \textit{to} his wife that he took.
\verse “A person shall not take a pair of millstones or an upper millstone, for \textit{he is taking necessities of life as a pledge}.
\verse “If a man is \textit{caught} kidnapping somebody from \textit{among} his countrymen, the \textit{Israelites}, and he treats him as a slave or he sells him, then that kidnapper shall die, and \textit{so} you shall purge the evil \textit{from among you}.
\verse Be watchful \textit{with respect to} \textit{an} outbreak of \textit{any} infectious skin disease, by being very careful and by acting according to all that the priests and the Levites have instructed you, \textit{just} as I have commanded them, \textit{so you shall diligently observe}.
\verse \textit{So} remember what Adonai your God did to Miriam on the journey \textit{when you went out from Egypt}.
\verse “When you make a loan to your neighbor, a loan of any kind, you shall not go into his house \textit{to take his pledge}.
\verse You shall wait outside, and the man \textit{to} whom you \textit{are} lending, he shall bring the pledge outside to you.
\verse And if \textit{he is} a needy man, you shall not sleep in his pledge.
\verse You shall certainly return the pledge to him \textit{as the sun sets}, so that he may sleep in his cloak and may bless you, and it shall be \textit{considered} righteousness \textit{on your behalf} \textit{before} Adonai your God.
\verse “You shall not exploit a hired worker, \textit{who is} needy and poor, from among your fellow men or from \textit{among} your aliens who are in your land \textit{and} in your \textit{towns}.
\verse On his day you shall give his wage, and the sun shall not go \textit{down}, because \textit{he is} poor and \textit{his life depends on it}; \textit{do this} so that he does not cry out against you to Adonai, \textit{and you incur guilt}.
\verse “Fathers shall not be put to death because of \textit{their} children, and children shall not be put to death because of \textit{their} fathers; each one shall be put to death for his \textit{own} sin.
\verse You shall not subvert the rights of an alien \textit{or} an orphan, and you shall not take as pledge \textit{the} garment of a widow.
\verse And you shall remember that you were a slave in Egypt and \textit{that} Adonai your God redeemed you from there; therefore I \textit{am} commanding you to do this commandment.
\verse “When you reap your harvest in your field and you forget a sheaf in the field, you shall not return to get it, for it shall be for the alien, for the orphan, and for the widow, so that Adonai your God may bless you in all the work of your hands.
\verse When you beat off the fruit of your olive trees you shall not search through the branches afterward, for it shall be for the alien, for the orphan, and for the widow.
\verse When you harvest \textit{grapes}, you shall not glean your vineyards \textit{again}; it shall be for the alien, for the orphan, and for the widow.
\verse And you shall remember that you were a slave in the land of Egypt, therefore I \textit{am} commanding you to do this thing.”
\end{biblechapter}

\begin{biblechapter} % Deuteronomy 25
\verse “When a legal dispute \textit{takes place} between men and they come near to the court, and \textit{the judges} judge \textit{with respect to} them, then they shall declare the righteous \textit{to be} in the right and they shall condemn the wicked,
\verse then it will happen if the guilty \textit{one} \textit{deserves beating}, then the judge shall make him lie, and he shall beat him \textit{before him}, \textit{according to} \textit{the prescribed number of lashes proportionate to the offense}.
\verse He may beat him \textit{with} forty lashes, and he shall not do more \textit{than these}, so that he \textit{will} not beat more in addition to these many blows, and your countryman would be degraded before your eyes.
\verse “You shall not muzzle an ox \textit{when he is threshing}.
\verse “When brothers dwell together and one of them dies and has no son, the wife of the deceased shall not become the wife of a \textit{man of another family}; her brother-in-law \textit{shall have sex with her}, and he shall take her \textit{to himself} as \textit{a} wife, and he shall perform his duty as \textit{a} brother-in-law \textit{with respect to} her.
\verse And then the firstborn that she bears \textit{shall represent his dead brother}, so that his name is not blotted out from Israel.
\verse But if the man \textit{does} not want to take his sister-in-law, then his sister-in-law shall go up to the gate, to the elders, and she shall say, ‘My brother-in-law refused \textit{to perpetuate his brother’s name} in Israel, \textit{for} he is not willing \textit{to marry me}.’
\verse Then the elders of his town shall summon him and speak to him, and \textit{if} he persists and says, ‘\textit{I do not desire to} marry her’
\verse \textit{then} his sister-in-law shall go near him before the eyes of the elders, and she shall pull off his sandal from his foot, and she shall spit in his face, and she shall \textit{declare} and she shall say, ‘This is how it is done to the man who does not build the house of his brother.’
\verse And his \textit{family} shall be called in Israel, ‘The house where the sandal was pulled off.’
\verse “If a man and his brother fight each other and the wife of the one \textit{man} comes near to rescue her husband from the hand of his attacker and she stretches \textit{out} her hand and she seizes his genitals,
\verse then you shall cut off her hand; your eye shall not take pity.
\verse “There shall not be \textit{for your use} in your bag \textit{two kinds of stone weights, a large one and a small one}.
\verse There shall not be in your house \textit{for your use} \textit{two kinds of measures}.
\verse \textit{Rather} a full and honest weight shall be \textit{for your use}; there shall be for you a full and honest \textit{measure}, so that your days on the land that Adonai your God \textit{is} giving to you may be long.
\verse For detestable to Adonai your God \textit{is} everyone who \textit{is} doing such things, everyone who \textit{is} acting dishonestly.
\verse “Remember what Amalek did to you on the journey when you went out from Egypt,
\verse that he met you on the journey and attacked you, all those lagging behind you and \textit{when} you were weary and worn out, and he did not fear God.
\verse \textit{And when} Adonai your God gives rest to you from all your enemies from around \textit{about you} in the land that Adonai your God \textit{is} giving to you \textit{as an} inheritance to take possession of it, you shall blot out the remembrance of Amalek from under the heavens; you shall not forget!”
\end{biblechapter}

\begin{biblechapter} % Deuteronomy 26
\verse “\textit{And then} when you come to the land that Adonai your God \textit{is} giving to you \textit{as an} inheritance, and you take possession of it and you settle in it,
\verse then you shall take from the firstfruit of all the fruit of the ground that you harvest from your land that Adonai your God \textit{is} giving to you, and you shall put \textit{it} in a basket, and you shall go to the place that Adonai your God will choose to make his name to dwell there.
\verse And you shall go to the priest who is \textit{in office} in those days, and you shall say, ‘I declare \textit{today} to Adonai your God that I have come into the land that Adonai swore to our ancestors to give to us.’
\verse Then the priest takes the basket from your hand and places it \textit{before} the altar of Adonai your God.
\verse And \textit{you shall declare} and you shall say \textit{before} your God, ‘My ancestor \textit{was} a wandering Aramean, and he went down to Egypt, and there he dwelt as an alien \textit{few in number}, and there he became a great nation, mighty and numerous.
\verse And the Egyptians treated us badly, and they oppressed us and imposed on us hard labor.
\verse And we cried to Adonai, the God of our ancestors, and Adonai heard our voice and saw our affliction and our toil and our oppression.
\verse And Adonai brought us \textit{out} from Egypt with a strong hand and an outstretched arm and with great terror and with signs and with wonders.
\verse And he brought us to this place and gave to us this land, a land flowing with milk and honey.
\verse And now, look, I am bringing the firstfruit of the fruit of the ground that you gave to me, Adonai,’ and you shall place it \textit{before} Adonai your God, and you shall bow down \textit{before} Adonai your God.
\verse And you shall celebrate with all of the bounty that Adonai your God gave to you and to your family, you and the Levite and the alien who \textit{is} in your midst.
\verse “When you are finished \textit{giving a tithe}, all of the tithe of your produce in the third year, the year of the tithe, then you shall give to the Levite, to the alien, to the orphan, and to the widow, so that they may eat in your towns and they may be satisfied.
\verse And you shall say \textit{before} Adonai your God, ‘I have removed the sacred portion from the house and, moreover, I have given it to the Levite and to the alien and to the orphan and to the widow according to all your commandment that you commanded me; I have not transgressed any of your commandments, and I have not forgotten \textit{any of them}.
\verse I have not eaten during my \textit{time of} mourning, and I have not removed \textit{anything} from it while \textit{being} unclean, and I have not offered \textit{any of it} to someone who \textit{has} died. I have listened to the voice of Adonai my God; I have done all that you commanded me \textit{to do}.
\verse Look down from the dwelling place of your holiness, from heaven, and bless your people Israel, and the land that you have given to us, as you swore to our ancestors, a land flowing with milk and honey.’
\verse “This day Adonai your God is commanding you to do these rules and regulations, \textit{and you must observe them diligently} with all your heart and with all your soul.
\verse Adonai you \textit{have} declared \textit{today} to be for you as \textit{your} God, and to go in his ways and to observe his rules and his commandments and his regulations and to listen to his voice.
\verse And Adonai \textit{has} declared you \textit{today} to be for him as a people, a treasured possession, as he \textit{promised} to you, and \textit{that you are} to observe all his commandments,
\verse \textit{and that he then will set you} high above all the nations that he has made for his praise and \textit{for fame} and for honor and \textit{for you to be a holy people} to Adonai your God, as he \textit{promised}.”
\end{biblechapter}

\begin{biblechapter} % Deuteronomy 27
\verseWithHeading{Ceremonies} Then Moses and the elders of Israel charged the people, \textit{saying}, “Keep all of the commandment that I \textit{am} commanding you \textit{today}.
\verse And then on the day that you cross the Jordan to the land that Adonai your God \textit{is} giving to you, then you shall set up \textit{for yourselves} large stones, and you shall paint them with lime,
\verse and you shall write on them all the words of this law at your crossing, so that you may come into the land that Adonai your God \textit{is} giving to you, a land flowing with milk and honey, as Adonai, the God of your ancestors, \textit{promised} to you.
\verse And \textit{when you cross the Jordan}, you shall set up these stones that I \textit{am} commanding you \textit{about} \textit{today} on Mount Ebal, and you shall paint them with lime.
\verse And you shall build an altar there for Adonai your God, an altar of stone, but \textit{you shall not use an iron tool to shape the stones}.
\verse You must build the altar of your God \textit{with} unhewn stones, and you shall sacrifice on it burnt offerings to Adonai your God.
\verse And you shall sacrifice fellowship offerings, and you shall eat \textit{them} there, and you shall rejoice \textit{before} Adonai your God.
\verse You shall write on the stone all of the words of this law very clearly.”
\verse Then Moses and the priests, the Levites, spoke to all Israel, saying, “Be silent and hear, Israel, \textit{for} this day you have become \textit{a people} for Adonai your God.
\verse And listen to the voice of Adonai your God, and observe his commandments and his rules that I \textit{am} commanding you \textit{today}.”
\verseWithHeading{Blessings and Curses} And Moses charged the people on that day, \textit{saying},
\verse “These \textit{tribes} shall stand on Mount Gerizim to bless the people \textit{when you cross} the Jordan: Simeon and Levi and Judah and Issachar and Joseph and Benjamin.
\verse And these shall stand on Mount Ebal for \textit{delivering} the curse: Reuben, Gad and Asher and Zebulun, Dan and Naphtali.
\verse And \textit{the Levites shall declare}, and they shall say to each man of Israel \textit{with} a loud voice,
\verse ‘Cursed be the man that makes a divine image or a cast image, \textit{which is} a detestable thing for Adonai, the work of the hand of a skilled craftsman, and \textit{then} sets \textit{it} in a hiding place.’ And \textit{all the people shall respond}, ‘Amen.’
\verse ‘Cursed be the one who dishonors his father or his mother.’ And all of the people shall say, ‘Amen.’
\verse ‘Cursed be the one who moves the boundary \textit{marker} of his neighbor.’ And all the people shall say, ‘Amen.’
\verse ‘Cursed be the one who misleads a blind person on the road.’ And all the people shall say, ‘Amen.’
\verse ‘Cursed be the one who deprives \textit{the} alien, \textit{the} orphan, and \textit{the} widow of justice.’ And all the people shall say, ‘Amen.’
\verse ‘Cursed be the one who lies with the wife of his father, because \textit{he has dishonored his father’s bed}.’ And all the people shall say, ‘Amen.’
\verse ‘Cursed be the one who lies with any \textit{kind} of animal.’ And all the people shall say, ‘Amen.’
\verse ‘Cursed be the one who lies with his sister, the daughter of his father or the daughter of his mother.’ And all the people shall say, ‘Amen.’
\verse ‘Cursed be the one who lies with his mother-in-law.’ And all the people shall say, ‘Amen.’
\verse ‘Cursed be the one who strikes down his neighbor in secret.’ And all the people shall say, ‘Amen.’
\verse ‘Cursed be the one who takes a bribe \textit{to murder an innocent person}.’ And all the people shall say, ‘Amen.’
\verse ‘Cursed be \textit{the one} \textit{who does not keep} the words of this law, to observe them.’ And all the people shall say, ‘Amen.’ ”
\end{biblechapter}

\begin{biblechapter} % Deuteronomy 28
\verse “And it will happen \textit{that} if you indeed listen to the voice of Adonai your God, \textit{to diligently observe} all his commandments that I \textit{am} commanding you \textit{today}, then Adonai your God will set you above all the nations of the earth.
\verse And all of these blessings shall come upon you, and \textit{they shall have an effect on you} if you listen to the voice of Adonai your God:
\verse “You will be blessed in the city, and you will be blessed in the field.
\verse “Blessed will be the fruit of your womb and the fruit of your ground and the fruit of your livestock, \textit{the calf of your cattle and the lambs of your flock}.
\verse “Blessed will be your basket and your kneading trough.
\verse “Blessed will you be \textit{when you come in and blessed will you be when you go out}.
\verse “Adonai will cause your enemies \textit{who rise up against you to be defeated before you}; on one road they shall come out \textit{against} you, but on seven roads they shall flee \textit{before you}.
\verse Adonai will command \textit{concerning you} the blessing \textit{to be} in your barns and \textit{in all your endeavors}; and he will bless you in the land that Adonai your God \textit{is} giving to you.
\verse Adonai will establish you for \textit{himself} as a holy people as he has sworn to you, if you keep the commandments of Adonai your God and you walk in his ways.
\verse And all of the peoples of the earth shall see that \textit{by the name of Adonai you are called}, and \textit{they shall fear you}.
\verse And \textit{Adonai will make you successful and prosperous}, in the fruit of your womb and on the land that Adonai swore to your ancestors to give to you.
\verse Adonai shall open for you his \textit{rich} storehouse, \textit{even} the heavens, to give the rain for your land in its time and to bless all of the work of your hand, and you will lend to many nations; you will not borrow \textit{from them}.
\verse And Adonai shall make you as head and not the tail, and you shall be only at the top \textit{of the nations}, and you shall not be at the bottom, if you listen to the commandments of Adonai your God that I \textit{am} commanding you \textit{today} \textit{and diligently observe them}.
\verse And you \textit{shall} not turn aside from \textit{any of} the words that I \textit{am} commanding you \textit{today} \textit{to the} right or left by going after other gods to serve them.
\verse “\textit{And then} if you do not listen to the voice of Adonai your God \textit{by diligently observing} all of his commandments and his statutes that I \textit{am} commanding you \textit{today}, then all of these curses shall come upon you, and they shall overtake you:
\verse “You \textit{shall} be cursed in the city, and \textit{you shall} be cursed in the field.
\verse “Your basket \textit{shall} be cursed and your kneading trough.
\verse “The fruit of your womb \textit{shall} be cursed and the fruit of you ground, the calves of your cattle and the lambs of your flock.
\verse “You shall be cursed \textit{when you come in}, and you \textit{shall} be cursed \textit{when you go out}.
\verse “Adonai will send upon you the curse, the panic, and the threat \textit{in everything that you undertake}, \textit{until you are destroyed and until you perish quickly} \textit{because of} the evil of your deeds in that you have forsaken me.
\verse Adonai will cause the plague to cling to you \textit{until it consumes you} from the land that you \textit{are} going to, to take possession of it.
\verse Adonai will afflict you with the wasting diseases and with the fever and with the inflammation and with the scorching heat and with the sword and with the blight and with the mildew, and they shall pursue you \textit{until you perish}.
\verse And your heavens that \textit{are} over your heads shall be \textit{like} bronze, and the earth that \textit{is} under you \textit{shall be like} iron.
\verse Adonai will change the rain of your land \textit{to} fine dust and \textit{to} sand; from the heaven it shall come down upon you \textit{until you are destroyed}.
\verse “Adonai shall cause you to be defeated \textit{before} your enemies; on one road you shall go \textit{against} them, but you will flee on seven roads \textit{before} them, and you shall become a thing of horror to all of the kingdoms of the earth.
\verse And your dead bodies shall be as food for all of the birds of the heaven and to the animals of the earth, and there shall not be \textit{anyone to frighten them away}.
\verse “Adonai shall afflict you with the boils of Egypt and with tumors and with the scurvy and with \textit{the skin rash that cannot be healed}.
\verse Adonai shall afflict you with madness and with blindness and with confusion of heart.
\verse And you shall be groping at noon just as the blind person gropes in the dark, and you shall not succeed \textit{in finding} your way, and you shall only be abused and robbed \textit{all the time}, and there will not be \textit{anyone who will rescue you}. 
\verse You shall become engaged to a woman, but another man shall sleep with her; you shall build a house, but you shall not live in it; a vineyard you shall plant, but you shall not enjoy it.
\verse Your ox \textit{shall} \textit{be} slaughtered before your eyes, and you shall not eat it; your donkey \textit{shall be} stolen \textit{right} \textit{before you}, and it shall not be returned to you; your sheep and your goats \textit{shall} be given to your enemies, and \textit{there shall not be anyone who rescues you}.
\verse Your sons and your daughters \textit{shall be} given to other people, and \textit{you will be looking on} \textit{longingly} for them all day, \textit{but you will be powerless to do anything}.
\verse A people that you do not know shall consume the harvest of your land and all your labor, and you will be only oppressed and crushed \textit{for the rest of your lives}.
\verse You shall become mad \textit{because of what your eyes shall see}.
\verse Adonai shall strike you with grievous boils on the knees and on the upper thighs \textit{from which} \textit{you will not be able to be healed}, from the sole of your foot and up to your crown.
\verse Adonai will bring you and your king whom you set up over you to a nation that you or your ancestors have not known, and there you will serve other gods \textit{of} wood and stone.
\verse And you will become a horror and a proverb and ridicule among all the peoples where Adonai drives you there.
\verse “You shall carry out much seed to the field, but you shall gather little \textit{produce}, for the locust shall devour it.
\verse You shall plant vineyards and you shall dress \textit{them}, but you shall not drink wine and you shall not gather grapes, for the worm shall eat it.
\verse There shall be olive trees for you in all of your territory, but you shall not anoint \textit{yourself}, for your olives shall drop off.
\verse You shall bear sons and daughters, but they shall not be \textit{yours}, for they shall go into captivity.
\verse The cricket shall take possession of all your trees and the fruit of your ground.
\verse The alien that \textit{is} in your midst shall ascend over you, higher \textit{and} higher, but you shall go down lower \textit{and} lower.
\verse He shall lend to you, but you shall not lend to him; he shall be \textit{the} head, but you shall be \textit{the} tail.
\verse “And all of these curses shall come over you, and they shall pursue you, and they shall overtake you \textit{until you are destroyed}, because \textit{you did not listen} to the voice of Adonai your God, \textit{by observing} his commandments and his statutes that he commanded you.
\verse And they shall be among you as a sign and as a wonder and among your offspring \textit{forever}.
\verse “\textit{Because} \textit{of the fact} that you did not serve Adonai your God with joy and with gladness of heart for the abundance of everything,
\verse then you shall serve your enemies, whom Adonai will send against you \textit{under conditions of} famine, in thirst, in nakedness, and in \textit{a} lack of everything; and he shall place a yoke of iron on your neck \textit{until he has destroyed you}.
\verse Adonai will raise \textit{up} against you a nation from far \textit{off}, from the end of the earth, \textit{attacking} as the eagle swoops \textit{down}, \textit{a nation whose language you will not understand},
\verse \textit{a grim-faced nation} \textit{who does not show respect} to \textit{the} old and \textit{the} young \textit{and} does not show pity.
\verse And it shall consume the fruit of your livestock and the fruit of your ground \textit{until you are destroyed}, \textit{and} who will not leave for you \textit{any} grain, wine, and olive oil, \textit{calves of your herds}, and \textit{lambs of} your flock \textit{until it has destroyed you}.
\verse And it shall besiege you in all your towns \textit{until your high and fortified walls fall}, which you \textit{are} trusting in \textit{throughout your land}; and it shall besiege you in all of your towns in all of your land that Adonai your God has given to you.
\verse And you shall eat the fruit of your womb, the flesh of your sons and your daughters, whom Adonai your God gave to you, \textit{during the siege and during the distress} your enemy inflicts upon you.
\verse The most refined and the very sensitive man among you \textit{shall be mean with his brother} and \textit{against his beloved wife} and against the rest of his children that he has left over,
\verse \textit{by refraining from giving} to \textit{even} one of them any of the meat of his children that he eats, because \textit{there is} not anything \textit{that is} left over for him \textit{during the siege and distress} that your enemy inflicts upon you.
\verse The most refined and the \textit{most} delicate \textit{woman} among you, who shall not venture to put the sole of her foot on the ground from being \textit{so} delicate and from \textit{such} gentleness, \textit{shall be mean to her beloved husband} and against her son and against her daughter,
\verse and \textit{even} concerning her afterbirth \textit{that goes out} from between her feet and \textit{also} concerning her children that she bears, because she eats them for lack of anything in secret \textit{during the siege and during the distress} that your enemy inflicts upon her in your \textit{towns}.
\verse “If \textit{you do not diligently observe} all the words of this law written in this scroll by revering this glorious and awesome name, Adonai your God,
\verse then Adonai shall overwhelm \textit{you} with your plagues and the plagues of your offspring, severe plagues and lasting illnesses, grievous and enduring.
\verse And he shall bring back upon you all the diseases of Egypt \textit{concerning} which \textit{you were in dread} \textit{because of them}.
\verse Also any illness and any plague that \textit{is} not written in the scroll of this law, he shall bring them, Adonai, upon you until you are destroyed.
\verse And you shall remain \textit{only a few people} in place of \textit{the fact} you were \textit{formerly} as the stars of heaven as far as number \textit{is concerned}, because you did not listen to the voice of Adonai your God.
\verse \textit{And then} as Adonai delighted over you \textit{to make you prosperous} to make you numerous, so Adonai shall delight over you to exterminate you and to destroy you, and \textit{so} you shall be plucked from the land that you \textit{are} going there to take possession of it.
\verse And Adonai shall scatter you among all the nations from \textit{one} end of the earth up to the \textit{other} end of the earth, and there you shall serve other gods that you have not known nor your ancestors, \textit{gods of} wood and stone.
\verse And among these nations you shall not find rest, and \textit{there} shall not be a resting place for the sole of your foot, and Adonai shall give you there an anxious heart and a weakening of eyes and a languishing of \textit{your} inner self.
\verse And \textit{your life shall hang in doubt before you}, and you shall be startled night and day, and you shall not be confident of your life.
\verse In the morning you shall say, ‘\textit{If only it was evening}!’ and in the evening you shall say ‘\textit{If only it was morning}!’ because of the dread of your heart that you shall feel, and because of the sight of your eyes that you shall see.
\verse And Adonai shall bring you back \textit{to} Egypt in ships by the route that I \textit{promised} to you \textit{that} ‘\textit{You shall not see it again}!’ And you shall sell yourself there to your enemies as slaves and as female slaves, but there will not be a buyer.”
\end{biblechapter}

\begin{biblechapter} % Deuteronomy 29
\verseWithHeading{Covenant Renewal, Oaths, Restoration, Charges to the Nation}  These \textit{are} the words of the covenant that Adonai commanded Moses to make with the \textit{Israelites} in the land of Moab \textit{besides} the covenant that he made with them at Horeb.
\verse And Moses summoned all of Israel and said to them, “You saw all that Adonai did before your eyes in the land of Egypt and to Pharaoh and to all his servants and to all his land;
\verse \textit{that is}, the great trials that your eyes saw, \textit{and} those great signs and wonders.
\verse But Adonai has not given to you a heart to understand, or eyes to see, or ears to hear, \textit{even} \textit{to this day}.
\verse And I have led you forty years in the desert; your clothes \textit{have} not worn out \textit{on you}, and your sandal has not worn out \textit{on your foot}.
\verse You have not eaten bread, and you have not drunk wine and strong drink, so that you may know that I \textit{am} Adonai your God.
\verse And \textit{when} you came to this place \textit{then} Sihon the king of Heshbon, and Og the king of Bashan, came out to meet you for battle, and we defeated them.
\verse And we took their land and gave it as \textit{an} inheritance to the Reubenites and to the Gadites and to \textit{the} half-tribe of Manasseh.
\verse And \textit{you must diligently observe the words of this covenant}, so that you may succeed in all that you do.
\verse “You \textit{are} standing \textit{today}, all of you, \textit{before} Adonai your God, your leaders, your tribes, your elders, and your officials, all the men of Israel,
\verse your little children, your women and your aliens who \textit{are} in the midst of your camp, from the choppers of your wood to the drawers of your water,
\verse \textit{in order for you to enter into the covenant of Adonai your God}, and into his oath that Adonai your God \textit{is} \textit{making with you} \textit{today},
\verse in order to establish you \textit{today} \textit{to himself} as a people and \textit{so that} he may be for you \textit{as} God, \textit{just} as he \textit{promised} to you and \textit{just as} he swore to your ancestors, to Abraham, to Isaac, and to Jacob.
\verse “Now I am not \textit{making this covenant} and this oath \textit{with you alone}.
\verse But with \textit{whoever is standing here} with us \textit{today} \textit{before} Adonai our God, and with \textit{whoever is not standing here} with us \textit{today}.
\verse For you know how we lived in the land of Egypt and how we traveled through the midst of the nations that you traveled through.
\verse And you have seen their detestable things and their idols of wood and stone, silver, and gold that \textit{were} among them,
\verse so that \textit{there is not} among you a man or a woman or a clan or a tribe \textit{whose heart} turns \textit{today} from \textit{being} with Adonai our God to go to serve the gods of these nations, so that there is not among you a root sprouting poison and wormwood.
\verse And then when he hears the words of this oath, then \textit{he will assure himself} in his heart, \textit{saying}, ‘\textit{Safety shall be mine even though I go in the stubbornness of my heart},’ thereby destroying the well-watered \textit{land} \textit{along} with the parched.
\verse Adonai will not be willing to forgive him, for \textit{by then} the anger of Adonai will smoke, and his passion against that man and all the curses written in this scroll will descend on him, and Adonai will blot out his name from under heaven.
\verse And Adonai will single him out for calamity out of all the tribes of Israel, according to all the curses of the covenant written in the scroll of this law.
\verse “And the next generation, \textit{that is}, your children who will rise \textit{up} after you, and the foreigner who will come from a distant land, when they will see the plagues of that land and its diseases that Adonai has inflicted upon it, will say,
\verse ‘All its land is brimstone and salt left by fire, \textit{none of its land will be sown}, and it will not make plants sprout out and it will not grow any vegetation; \textit{it is} as the destruction of Sodom and Gomorrah, Adman and Zeboiim, which Adonai overturned in his anger and in his wrath.’
\verse And all the nations will say, ‘\textit{Why} has Adonai done \textit{such a thing} to this land? What \textit{caused} the fierceness of this great anger?’
\verse And they will say, ‘\textit{It is} because they abandoned the covenant of Adonai, the God of their ancestors, which he \textit{made} with them \textit{when he brought them out} from the land of Egypt.
\verse And they went and served other gods and bowed down to them, gods whom they did not know them and he had not allotted to them.
\verse So \textit{the anger of Adonai was kindled} against that land to bring upon it all the curses written in this scroll,
\verse and Adonai uprooted them from their land in anger and in wrath and in great fury, and he cast them into another land, \textit{just as it is today}.’
\verse “The hidden \textit{things} \textit{belong to Adonai} our God, but the revealed \textit{things} \textit{belong to us} \textit{to know} and to our children \textit{forever}, \textit{in order} to do all the words of this law.”
\end{biblechapter}

\begin{biblechapter} % Deuteronomy 30
\verse “And then when all of these things come upon you, the blessing and the curse that I have set \textit{before you} \textit{and you call them to mind} among the nations there where Adonai your God \textit{has} scattered you,
\verse and you return to Adonai and you listen to his voice according to all that I am commanding you \textit{today}, \textit{both} you and your children, with all your heart and with all your inner self,
\verse and Adonai your God will restore your fortunes, and he will have compassion \textit{upon} you, and \textit{he will again gather you together} from all the peoples where Adonai your God scattered you there.
\verse “\textit{Even} if \textit{you are outcasts} at the end of the heavens, \textit{even} from there Adonai your God shall gather you, and from there \textit{he shall bring you back}.
\verse And Adonai your God will bring you to the land that your ancestors had taken possession of, and he will make you successful, and he will make you more numerous than your ancestors.
\verse “And Adonai your God will circumcise your heart and the heart of your offspring to love Adonai your God with all your heart and with all your inner self \textit{so that you may live}.
\verse And Adonai your God will put all these curses on your enemies and \textit{on those who hate you}, \textit{on} \textit{those who harassed you}.
\verse And \textit{you will again listen} to the voice of Adonai, and you will do all his commandments that I \textit{am} commanding you \textit{today}.
\verse And Adonai your God will make you prosperous \textit{in all your undertakings}, and in the fruit of your livestock and in the fruit of your ground \textit{abundantly}, for Adonai \textit{will again rejoice} over you, \textit{just} as he rejoiced over your ancestors.
\verse \textit{He will do this} if you listen to the voice of Adonai your God by keeping his commandment and his statutes written in the scroll of this law \textit{and} if you return to Adonai your God with all your heart and with all your inner self.
\verse “For this commandment that I \textit{am} commanding you \textit{today} \textit{is} \textit{not too wonderful for you}, and it \textit{is} not \textit{too} far \textit{from you}.
\verse It is not in the heavens \textit{so that you might say}, ‘Who will go up for us to the heavens and get it for us and cause us to hear it, so that we may do it?’
\verse And \textit{it is} \textit{not beyond the sea}, \textit{so that you might say}, ‘Who will cross for us to the other side of the sea and take it for us and cause us to hear it, so that we may do it?’
\verse But the word is very near you, \textit{even} in your mouth and in your heart, \textit{so that you may do it}.
\verse “See, I am setting \textit{before you} \textit{today} life and prosperity and death and disaster;
\verse what I am commanding you \textit{today} \textit{is} to love Adonai your God by going in his ways and by keeping his commandments and his statutes and his regulations, and then you will live, and you will become numerous, and Adonai your God will bless you in the \textit{land where you are going}.
\verse However, if your heart turns aside and you \textit{do} not listen and you are lured away and you bow down to other gods and you serve them,
\verse I declare to you \textit{today} that you will certainly perish; \textit{you will not extend your time} on the land that you are crossing the Jordan to go there to take possession of it.
\verse I invoke as a witness against you \textit{today} the heaven and the earth: life and death I have set \textit{before you}, blessing and curse. So choose life, so that you may live, you and your offspring,
\verse by loving Adonai your God by listening to his voice and by clinging to him, for he \textit{is} your life and the length of your days \textit{in order for you} to live on the land that Adonai swore to your ancestors, to Abraham, to Isaac, and to Jacob, to give to them.”
\end{biblechapter}

\begin{biblechapter} % Deuteronomy 31
\verseWithHeading{Succession, Deposition, and Recitation of Text} And Moses went and spoke these words to all Israel.
\verse And he said to them, “I am \textit{a} hundred and twenty years old \textit{today}; I am not able to go out and to come \textit{in} \textit{any} longer, and Adonai said to me, ‘You may not cross this Jordan.’
\verse Adonai your God \textit{is about to} cross \textit{before you}; he will destroy these nations \textit{before you}, and you shall dispossess them. Joshua \textit{will be} crossing \textit{before you}, \textit{just} as Adonai \textit{promised}.
\verse And Adonai will do to them \textit{just} as he did to Sihon and to Og, kings of the Amorites, and to their land, which he destroyed with them.
\verse And Adonai \textit{will deliver them to you before you}, and you shall do to them according to every commandment that I have commanded you.
\verse Be strong and be courageous; you should not be afraid, and you \textit{should not be in dread from their presence}, for Adonai your God \textit{is} the \textit{one} going with you; he will not leave you alone and he will not forsake you.”
\verse Then Moses summoned Joshua, and he said to him \textit{in the presence of all Israel}, “Be strong and be courageous, for you will go with this people into the land that Adonai swore to their ancestors to give to them, and you will give it to them as an inheritance.
\verse Adonai \textit{is} the \textit{one} going \textit{before you}; he will be with you, and he will not leave you alone, and he will not forsake you; you shall not be afraid, and you shall not be discouraged.”
\verse So Moses wrote this law, and he gave it to the priests, the descendants of Levi, the \textit{ones} carrying the ark of the covenant of Adonai, and to all the elders of Israel.
\verse Then Moses commanded them, \textit{saying}, “At the end of seven years, in the time of the year for canceling debts during the Feast of Booths,
\verse \textit{when all Israel comes to appear before} Adonai their God at the place that he will choose, you shall read this law before all Israel \textit{in their hearing}.
\verse Assemble the people, the men and the women and the little children and your aliens that are in your \textit{towns}, so that they may hear and so that they may learn and they may revere Adonai your God, and \textit{they shall diligently observe} all the words of this law.
\verse And \textit{then} their children, who have not known, they \textit{too} may hear, and they may learn to revere Adonai their God all the days \textit{that you live} on the land that you \textit{are} crossing the Jordan \textit{to get there} to take possession of it.”
\verse Then Adonai said to Moses, “Look, \textit{you are about to die}; call Joshua and present yourselves in \textit{the} tent of assembly, so that I may instruct him.” So Moses and Joshua went and presented themselves in \textit{the} tent of assembly.
\verse And Adonai appeared in the tent in a column of cloud, and the column of the cloud stood at the entrance of the tent.
\verse And Adonai said to Moses, “Look, you \textit{are} about to lie \textit{down} with your ancestors, and this people will arise and they will play the prostitute after \textit{the foreign gods} of the land \textit{to which they are going}.
\verse And \textit{my anger shall flare up against them} on that day, and I will forsake them, and I will hide my face from them, and they shall become as prey, and disasters and troubles shall find them, and they shall say on that day, ‘\textit{Is it} not because our God is not in our midst \textit{that} these disasters have found us?’
\verse But I will certainly hide my face on that day, because \textit{of} all of the evil that they did because they turned to other gods.
\verse “And so then write this song for yourselves and teach it to the \textit{Israelites}; put it in their mouth, so that this song may be for me as a witness against the \textit{Israelites}.
\verse For when I bring them into the land that I swore to their ancestors, a land flowing with milk and honey, \textit{they will eat their fill}, and they will become fat, and \textit{then} they will turn to other gods, and they will serve them, and they will spurn me, and they will break my covenant.
\verse \textit{And then} many disasters and troubles will come upon them, and this song will give evidence before them as a witness, because it will not be forgotten from \textit{out of} the mouth of their descendants, for I know their inclination that they \textit{are} having \textit{today} before I have brought them into the land that I swore.”
\verse And Moses wrote this song on that day and taught it \textit{to} the \textit{Israelites}.
\verse Then he told Joshua the son of Nun, and said \textit{to him}, “Be strong and be courageous, for you shall bring the \textit{Israelites} into the land that I swore to them, and I will be with you.”
\verse \textit{And then when Moses finished writing} the words of this law on the scroll \textit{until they were complete},
\verse then Moses commanded the Levites carrying the ark of the covenant of Adonai, \textit{saying},
\verse “Take the scroll of this law and put it at the side of the ark of the covenant of Adonai your God, and it will be there as a witness against you.
\verse For I know your rebellion and your stiff neck \textit{even now} while I \textit{am} still alive with you \textit{today}, rebelling against Adonai, and \textit{how much more} after my death.
\verse Assemble to me all the elders of your tribes and your officials, so that I may speak in their ears these words, and \textit{that} I may call as witness against them heaven and earth.
\verse For I know that after my death you will certainly act corruptly, and you will turn aside from the way that I have commanded you, and the disaster in the future days will befall you, because you will do evil in the eyes of Adonai, \textit{provoking him with the work of your hands}.”
\verse So Moses spoke into the ears of the assembly of Israel the words of this song \textit{until they were complete}.
\end{biblechapter}

\begin{biblechapter} % Deuteronomy 32
\verseWithHeading{The Song of Moses} “Give ear, O heavens, and I will speak, 
and let the earth hear the words of my mouth.
\verse May my teaching trickle like the dew, 
my words like rain showers on tender grass, 
and like spring showers on new growth.
\verse For I will proclaim the name of Adonai; 
ascribe greatness to our God!
\verse The Rock, his work \textit{is} perfect, 
for all his ways \textit{are} just; 
\textit{he is} a faithful God, and \textit{without injustice}; 
righteous and upright \textit{is} he.
\verse They \textit{have} behaved corruptly toward him; 
\textit{they are} not his children; \textit{this is} their flaw, 
a generation crooked and perverse.
\verse \textit{Like} this do you treat Adonai, 
foolish and \textit{unwise} people? 
\textit{Has} he not, your father, created you? 
He made you, and he established you.
\verse Remember \textit{the} old days, \textit{the years long past}; 
ask your father, and he will inform you, 
your elders and they will tell you.
\verse \textit{When the Most High apportioned} \textit{the} nations, 
at his dividing \textit{up} of the sons of humankind, 
he fixed the boundaries of \textit{the} peoples, 
according to the number of the children of Israel.
\verse For Adonai’s portion \textit{was} his people, 
Jacob the share of his inheritance.
\verse He found him in a desert land, 
and in a howling, desert wasteland; 
he \textit{encircled him}, he cared for him, 
he protected him like the \textit{apple} of his eye.
\verse As an eagle stirs up its nest, 
hovers over its young, 
spreads out its wings, takes them, 
carries them on its pinions,
\verse \textit{so} Adonai alone guided him, 
and \textit{there was no foreign god accompanying him}.
\verse And he set him on the high places of \textit{the} land, 
and he fed \textit{him} the crops of \textit{the} field, 
and he nursed him with honey from crags, 
and \textit{with} oil from flinty rock,
\verse \textit{With} curds from \textit{the} herd, 
and \textit{with} milk from \textit{the} flock, 
with \textit{the} fat of young rams, 
and rams, the offspring of Bashan, 
and with goats \textit{along} with the finest kernels of wheat, 
and \textit{from} the blood of grapes you drank fermented wine.
\verse And Jeshurun grew fat, and he kicked; 
you grew fat, you bloated, and you became obstinate; 
and he abandoned God, his maker, 
and he scoffed \textit{at} the rock of his salvation.
\verse They made him jealous with strange \textit{gods}; 
with detestable things they provoked him.
\verse They sacrificed to the demons, not God, 
\textit{to} gods whom they had not known, 
new \textit{gods} \textit{who} came from recent times; 
\textit{their ancestors had not known them}.
\verse \textit{The} rock who bore you, you neglected, 
and you forgot God, \textit{the one} giving you birth.
\verse Then Adonai saw, and he spurned \textit{them}, 
because of the provocation of his sons and his daughters.
\verse So he said, ‘I will hide my face from them; 
I will see what \textit{will be} their end, 
for they \textit{are} a generation of perversity, 
\textit{children in whom there is no faithfulness}.
\verse They annoyed me with \textit{what is} not a god; 
they provoked me with their idols. 
So I will make them jealous with \textit{those} not a people, 
with a foolish nation I will provoke them.
\verse For a fire was kindled by my anger, 
and it burned \textit{up to the depths of Sheol}, 
and it devoured \textit{the} earth and its produce, 
and it set afire the foundation of \textit{the} mountains.
\verse I will heap disasters upon them; 
my arrows I will spend on them.
\verse \textit{They will become} weakened by famine, 
and consumed by plague and bitter pestilence; 
and the teeth of wild animals I will send against them, 
with \textit{the} poison of \textit{the} creeping \textit{things} in \textit{the} dust;
\verse From outside \textit{her boundaries} the sword will make \textit{her} childless, 
and from inside, terror; 
both \textit{for} \textit{the} young man \textit{and} also the young woman, 
the infant \textit{along with} the gray-headed man.
\verse I thought, “I will wipe them out; 
\textit{I will make people forget they ever existed}.”
\verse If I had not feared a provocation of \textit{the} enemy, 
lest their foes might misunderstand,  
lest they should say,  “Our hand is 
\textit{triumphant}, and Adonai \textit{did} not do all this.” ’
\verse For \textit{they are a nation void of sense}, 
and there is not \textit{any} understanding in them.
\verse If only they were wise, they would understand this; 
they would discern \textit{for themselves their end}.
\verse How could one chase a thousand 
and two could cause a myriad to flee, 
if their Rock had not sold them, 
and Adonai had \textit{not} given them up?
\verse For the fact of the matter is, 
their rock \textit{is} not like our Rock, 
and our enemies recognize \textit{this}.
\verse For their vine \textit{is} from the vine of Sodom, 
and from the terraces of Gomorrah; 
their grapes \textit{are} grapes of poison; 
\textit{their clusters are bitter}.
\verse Their wine \textit{is} the poison of snakes, 
and \textit{the} deadly poison of horned vipers.
\verse ‘\textit{Is} not this stored up with me, 
sealed in my treasuries?’
\verse \textit{Vengeance belongs to me} and \textit{also} recompense, 
\textit{for at the time their foot slips}, 
because the day of their disaster \textit{is} near, 
\textit{and fate comes quickly for them}.’
\verse For Adonai will judge \textit{on behalf of} his people, 
and concerning his servants; 
he will change his mind when he sees that \textit{their} power has disappeared, 
and \textit{there is} no one left, confined or free.
\verse And he will say, ‘Where \textit{are} their gods, 
\textit{their} rock in whom they took refuge?
\verse Who ate the fat of their sacrifices 
\textit{and} drank the wine of their libations? 
Let them rise \textit{up}, and let them help you; 
\textit{Let them be to you a refuge}.
\verse See, now, that I, \textit{even} I \textit{am} he, 
and there is not a god besides me; 
I put to death and I give life; 
I wound and I heal; 
there is not \textit{one} who delivers from my hand!
\verse For \textit{indeed} I lift up my hand to heaven, 
And I promise \textit{as I live forever},
\verse When I sharpen \textit{my flashing sword}, 
and my hand takes hold \textit{of it} in judgment, 
\textit{I will take reprisals against my foes}, 
and my haters I will repay.
\verse I will make my arrows drunk with blood, 
and my sword shall devour flesh with the blood of the slain, 
and captives from the heads of the leaders of \textit{the} enemy.’
\verse Call for songs of joy, O nations, \textit{concerning} his people, 
for the blood of his servants he will avenge, 
\textit{and he will take reprisals against his foes}, 
and he will make atonement \textit{for} his land, his people.”
\verse And Moses came, and he spoke all the words of this song in the ears of the people; \textit{that is}, he and Joshua the son of Nun.
\verse \textit{And when Moses finished speaking} all these words to all Israel,
\verse then he said to them, “\textit{Take to heart all the words} that I \textit{am} admonishing \textit{against} you \textit{today} \textit{concerning} which you should instruct them \textit{with respect to} your children \textit{so that they will observe diligently} all the words of this law,
\verse for it \textit{is} not a trifling matter among you, but it \textit{is} your life, and through this word \textit{you will live long in the land} that you \textit{are about} to cross the Jordan \textit{to get} there to take possession of it.”
\verseWithHeading{Instructions Concerning Moses’ Death} And Adonai said to Moses on exactly this day, \textit{saying},
\verse “Go up to this mountain of the Abarim \textit{range}, Mount Nebo, which \textit{is} \textit{opposite Jericho}, and see the land of Canaan that I \textit{am} giving to the \textit{Israelites} as \textit{a} possession.
\verse You shall die on that mountain that you \textit{are} about to go up there, and \textit{you will} be gathered to your people, \textit{just} as your brother Aaron died on \textit{Mount Hor} and he was gathered to his people,
\verse because \textit{of the fact} that you broke faith with me in the midst of Israel at the waters of Meribah Kadesh, \textit{in} the desert of Zin, because \textit{you did not treat me as holy} in the midst of the \textit{Israelites}.
\verse Yes, from afar you may view the land, but there you shall not enter there, \textit{that is}, into the land that I \textit{am} giving to the \textit{Israelites}.”
\end{biblechapter}

\begin{biblechapter} % Deuteronomy 33
\verseWithHeading{Blessings of Moses on Israel} Now this \textit{is} the blessing \textit{with} which Moses, the man of God, blessed the \textit{Israelites} \textit{before} his death.
\verse Then he said,
\verse “Adonai came from Sinai, 
and he dawned upon them from Seir; 
he shone forth from Mount Paran, 
and he came with myriads of holy ones, 
at his right hand a fiery law for them.
\verse Moreover, \textit{he loves his people}, 
all the holy ones \textit{were} in your hand, 
and they bowed down to your feet, 
\textit{each one accepted directions from you}.
\verse A law Moses instructed for us, 
\textit{as a} possession \textit{for} the assembly of Jacob.
\verse And \textit{then} a king arose in Jeshurun, 
at the gathering of the leaders of \textit{the} people, 
\textit{united were the tribes of Israel}.
\verse “May Reuben live, and may he not die, 
\textit{and let his number not be few}.”
\verse \textit{And he said this of Judah}, 
“Hear, O Adonai, the voice of Judah, 
and bring him to his people; 
his own hands strive for him, 
and may you be a help \textit{against} his foes.”
\verse And of Levi he said, 
“Your Thummim and your Urim 
\textit{are} for \textit{your faithful one}, 
\textit{whom you tested at Massah}; 
you contended \textit{with} him 
at the waters of Meribah.
\verse The \textit{one} saying of his father and of his mother, 
‘I have not regarded them,’ 
and his brothers he \textit{did} not acknowledge, 
and his children \textit{he did} not know, 
but \textit{rather} they observed your word, 
and your covenant they kept.
\verse They taught your regulations to Jacob, 
and your law to Israel; 
they placed incense smoke \textit{before you}, 
and whole burnt offerings on your altar.
\verse Bless, O Adonai, his substance, 
and with the work of his hands you must be pleased; 
smite the loins of those who attack him, 
and those hating him, \textit{so that they cannot arise}.”
\verse Of Benjamin he said, 
“The beloved of Adonai dwells \textit{securely}, 
the Most High shields \textit{all} around him, 
all the day, 
and between his shoulders he dwells.”
\verse And of Joseph he said, 
“Blessed by Adonai \textit{is} his land, 
with the choice things of heaven, 
with dew, and with the deep lying down beneath,
\verse and with \textit{the} choice things of \textit{the} fruits of the \textit{the} sun, 
and with the choice things of \textit{the} yield of \textit{the} \textit{seasons},
\verse and with the finest things of \textit{the} ancient mountains, 
and with \textit{the} choice things of \textit{the} \textit{eternal hills},
\verse and with the choice things of \textit{the} earth and its fullness, 
and the favor of \textit{the one} \textit{who dwelt} in \textit{the} bush. 
Let them come to the head of Joseph, 
and to the crown of the prince among his brothers.
\verse \textit{As} the firstborn of his ox, majesty \textit{belongs to him}, 
and his horns \textit{are} \textit{the} horns of a wild ox; 
with them he drives people together, 
and they \textit{are} the myriads of Ephraim, 
and they \textit{are} the thousands of Manasseh.”
\verse And of Zebulun he said, 
“Rejoice, Zebulun, in your going out, 
and \textit{rejoice}, Issachar, in your tents;
\verse They summon people \textit{to the} mountains; 
there they sacrifice the sacrifices of righteousness, 
because the affluence of \textit{the} seas they suck \textit{out}, 
and \textit{the most hidden treasures of the sand}.”
\verse And of Gad he said, 
“Blessed \textit{be} \textit{the one who enlarges Gad}; 
like a lion he dwells, 
and he tears an arm as well as a scalp.
\verse And he selected \textit{the} best \textit{part} for himself,  
for there \textit{the} portion of a ruler \textit{is} included, 
and he came \textit{with the} heads of \textit{the} people; 
he did the righteousness of Adonai, 
and his regulations for Israel.”
\verse And of Dan he said, 
“Dan \textit{is} a cub of \textit{a} lion; 
he leaps from Bashan.”
\verse And of Naphtali, he said, 
“Oh, Naphtali, sated of favor, 
and full of the blessing of Adonai; 
take possession of \textit{the} lake, 
and \textit{the land to the} south.”
\verse And of Asher he said, 
“Blessed \textit{more than sons} \textit{is} Asher; 
\textit{may he be the favorite} of his brothers, 
\textit{dipping his feet in the oil}.
\verse Your bars \textit{are} iron and bronze, 
and as your days, \textit{so is} your strength.”
\verse “There is no \textit{one} like God, O, Jeshurun, 
who rides \textit{through} the heavens to your help, 
and with his majesty \textit{through the} skies.
\verse The God of \textit{ancient time} \textit{is} a hiding place, 
and underneath \textit{are} \textit{the} arms of eternity, 
and he drove out \textit{from before} you your enemy, 
and he said, ‘Destroy \textit{them}!’
\verse So Israel dwells alone \textit{and} carefree, 
the spring of Jacob in a land of grain and wine; 
his heavens even drip dew.
\end{biblechapter}

\begin{biblechapter} % Deuteronomy 34
\verseWithHeading{The Death of Moses and the Commissioning of Joshua} Then Moses went up from the desert plateau of Moab to Mount Nebo, to the top of Pisgah, \textit{which is opposite} Jericho, and Adonai showed him all of the land, Gilead \textit{all the way} up to Dan,
\verse and all of Naphtali and the land of Ephraim and Manasseh and all of the land of Judah, up to the western sea,
\verse and the Negev and the plain of the valley of Jericho, the city of palms, \textit{on} up to Zoar.
\verse And Adonai said to him, “This \textit{is} the land that I swore to Abraham to Isaac and to Jacob, \textit{saying}, ‘To your offspring I will give it.’ I have let you see \textit{it} with your eyes, but you shall not cross \textit{into it}.”
\verse Then Moses, the servant of Adonai, died there in the land of Moab \textit{according to the command of Adonai}.
\verse And he buried him in the valley in the land of Moab opposite Beth Peor. But until this day no one knows his burial site.
\verse \textit{Now Moses was a hundred and twenty years old} \textit{when he died}; \textit{his sight was not impaired and his vigor had not abated}.
\verse And the \textit{Israelites} wept \textit{concerning} Moses thirty days; \textit{finally} the days of weeping and mourning for Moses were completed.
\verse Now Joshua the son of Nun was full of the spirit of wisdom because Moses had placed his hands on him, and the \textit{Israelites} listened to him, and they did as Adonai had commanded Moses.
\verse And not again has a prophet arisen in Israel like Moses, whom Adonai knew face to face,
\verse as far as all the signs and the wonders Adonai sent him to do in the land of Egypt, against Pharaoh and all of his servants and against all of his land,
\verse and as far as all of \textit{the mighty deeds} and as far as \textit{the great awesome wonders} Moses did before the eyes of all Israel.
\end{biblechapter}

\flushcolsend
\biblebook{Joshua}

\begin{biblechapter} % Joshua 1
\verseWithHeading{Joshua Addresses the Israelites}{After the death of Moses the servant of Adonai, Adonai said to Joshua son of Nun, the assistant\lebnote{Or “servant”} of Moses, saying,}%
\verse{“My servant Moses is dead. Get up and cross the Jordan, you and all this people, into the land that I am giving to them, to the \textit{Israelites}.\lnAKM{}}%
\verse{Every place that the soles of your feet will tread, I have given it to you, as I promised\lebnote{Or “spoke”} to Moses.}%
\verse{From the wilderness \textit{and the Lebanon},\lebnote{“this Lebanon”; “Lebanon” in Hebrew means “white mountain”} up to the great river, the river Euphrates, all of the land of the Hittites, and up to \textit{the great sea in the west},\lebnote{“the great sea of the setting sun”} will be your territory.}%
\verse{No one will stand before you\lebnote{Or “in your presence”} all the days of your life. Just as I was with Moses, so will I be with you; I will not fail you, and I will not forsake you.}%
\verse{Be strong and courageous, for you will give the people this land as an inheritance that I swore to their ancestors\lnAGP{} to give them.}%
\verse{Only be strong and very courageous \textit{to observe diligently the whole law}\lebnote{“to keep and to act according to the whole law”} that Moses my servant commanded you. Do not turn aside from it, to the right or left, so that you may succeed \textit{wherever you go}.\lnEOV{}}%
\verse{The scroll of this law will not depart from your mouth; you will meditate on it day and night so that \textit{you may observe diligently all that is written}\lebnote{“to keep and to act according to all that is written”} in it. For then you will succeed in your ways and prosper.}%
\verse{Did I not command you? Be strong and courageous! Do not fear or be dismayed, for Adonai your God is with you \textit{wherever you go}.”\lnEOV{}}%
\verse{Then Joshua commanded the officers of the people, saying,}%
\verse{“Pass through the midst of the camp and command the people: ‘Prepare your provisions;\lebnote{Hebrew “provision”} \textit{in}\lebnote{“for in still”} three days you are to cross the Jordan to go possess the land that Adonai your God is giving you to possess.’ ”}%
\verse{To the Reubenites, the Gadites, and the half-tribe of Manasseh Joshua said,}%
\verse{“Remember the word that Moses Adonai’s servant commanded you, saying, ‘Adonai your God is giving rest to you, and he is giving\lebnote{Hebrew “he gave”} you this land.’}%
\verse{Your wives, your little children, and your livestock, they will remain in the land that Moses gave to you beyond the Jordan. \textit{All of the best fighting men}\lebnote{“All the mighty warriors of the troop”} will cross armed in front of your families; they will help you}%
\verse{until Adonai gives rest to your families as well as to you. They will take possession of the land that Adonai your God is giving to them. Then you will return to your own land and take possession of it, the land that Moses the servant of Adonai gave you beyond the Jordan \textit{to the east}.”\lebnote{“to the sunrise”}}%
\verse{And they answered Joshua, saying, “All that you have commanded us we will do, and wherever you will send us we will go.}%
\verse{Just as we obeyed Moses, so will we obey you. Only may Adonai your God be with you, as he was with Moses.}%
\verse{Whoever \textit{rebels against your orders}\lebnote{“defiles against your mouth”} and does not obey your words according to what you commanded us will be put to death. Only be strong and courageous.”}%
\end{biblechapter}

\begin{biblechapter} % Joshua 2
\verseWithHeading{Spies View the Land}{Then Joshua son of Nun secretly sent two men from Acacia Grove\lebnote{Hebrew “Shittim”} as spies, saying, “Go, view the land, especially Jericho.” So they went, and entered the house of a prostitute whose name was Rahab, and \textit{spent the night}\lebnote{“lay down”} there.}%
\verse{The king of Jericho was told, “Look, some men from the \textit{Israelites}\lnAKM{} have come here tonight to search out the land.”}%
\verse{And the king of Jericho sent for Rahab, saying, “Bring out the men who came to you, the ones who have entered your house, for they have come to search out the whole land.”}%
\verse{But the woman took the two men and hid them. And she said, “Yes, the men came to me, but I did not know where they were from.}%
\verse{And when it was time to shut the gate \textit{for the night},\lebnote{“in the darkness” or “at dark”} the men left, and I do not know where they went. Chase after them quickly, for you may catch up to them.”}%
\verse{(But she had taken them to the roof and had hidden them \textit{in the stalks of flax}\lebnote{“in the flaxes of the plant stalk”} that she \textit{had spread out}\lebnote{“had been arranged”} on the roof.)}%
\verse{So the men chased after them on \textit{the way to the Jordan}\lebnote{“the road of the Jordan”} at the fords; and they shut the gate behind the pursuers that had gone out after them.}%
\verse{Before \textit{they went to sleep},\lebnote{“they lay down”} she came up to them on the roof}%
\verse{and said to the men, “I know that Adonai has given you the land, and that dread of you has fallen on us, and that all the inhabitants of the land melt away in fear because of your presence.}%
\verse{For we have heard how Adonai dried up the waters of the \textit{Red Sea}\lebnote{“sea of reeds”} before you when you went out from Egypt, and what you did to the two kings of the Amorites that were beyond the Jordan, Sihon and Og, whom you utterly destroyed.}%
\verse{We heard this, and our hearts\lnEOY{} melted, and \textit{no courage was left in anyone}\lebnote{“a spirit no longer stood in anyone”} because of your presence. For Adonai your God is God in the heavens above and on the earth below.}%
\verse{So then please swear to me by Adonai, because \textit{I have shown loyalty to you},\lebnote{“I have done with you a loyal love”} and \textit{you will also show loyalty}\lebnote{“you will do … loyal love”} \textit{to my family}.\lebnote{“with the house of my father”} You must give me a sign of good faith,\lebnote{Or “faithfulness”}}%
\verse{and you will spare my father and mother, my brothers and sisters, and all that belongs to them; you will deliver our lives from death.”}%
\verse{And the men said to her, “\textit{Our lives for yours}.\lebnote{“Our lives in place of yours to die”} If you do not report this business of ours, \textit{we will show you loyalty and faithfulness}\lebnote{“we will do with you loyal love and faithfulness”} when Adonai gives us the land.”}%
\verse{Then she lowered them with a rope through the window, as her house was on the outer side of the wall, and she was residing in the wall.}%
\verse{And she said to them, “Go to the mountain, so that the pursuers will not find you, and hide yourselves there three days until the pursuers return, and afterward you may go \textit{on your way}.”\lebnote{“to your way”}}%
\verse{The men said to her, “We will be released from this oath of yours that you made us swear.}%
\verse{When we come to the land, you must tie this scarlet cord in the window through which you let us down, and you must gather your father and mother, and your brothers, and your whole family to your house.}%
\verse{If anyone goes outside the doors of your house, \textit{they will be responsible for their own death},\lebnote{“his blood will be on his head”} and we will be innocent. Anyone who will be with you in the house, \textit{we will be responsible for their death}\lebnote{“his blood will be on our head”} \textit{if a hand is laid on them}.\lebnote{“if a hand will be against him”}}%
\verse{But if you report this business of ours, we will be released from your oath that you made us swear.”}%
\verse{And she said, “According to your word it will be.” Then she sent them away, and they went, and she tied the scarlet cord in the window.}%
\verse{They departed and came to the mountain, and they stayed there three days until the pursuers returned. The pursuers searched \textit{all along the way}\lebnote{“on all of the road”} but did not find them.}%
\verse{The two men returned and went down from the mountain, and they crossed over and came to Joshua son of Nun, and they told him everything that happened to them.}%
\verse{They said to Joshua, “Surely Adonai has given all the land into our hand; also, all the inhabitants of the land melt away in fear because of our presence.”}%
\end{biblechapter}

\begin{biblechapter} % Joshua 3
\verseWithHeading{The Israelites Cross the Jordan}{Joshua rose early in the morning, and they set out from Acacia Grove.\lebnote{Hebrew “Shittim”} And they came up to the Jordan, he and all the \textit{Israelites},\lnAKM{} and they spent the night there before they crossed over.}%
\verse{At the end of the three days the officers passed through the midst of the camp,}%
\verse{and they commanded the people: “When you see the Levitical priests carrying the ark of the covenant of Adonai your God you must set out from your place and go after it.}%
\verse{But there will be a distance between you and it of about two thousand cubits in measurement.\lebnote{That is, about 1 km} Do not come near it, so that you may know the way that you must go, for you have not passed on this way \textit{before}.”\lebnote{“from yesterday three days before”}}%
\verse{And Joshua said to the people, “Sanctify yourselves, because tomorrow Adonai will do wonders in your midst.”}%
\verse{And Joshua said to the priests, “Take up the ark of the covenant and cross over \textit{ahead of the people}.”\lnEPA{} And they took up the ark of the covenant and went \textit{ahead of the people}.\lnEPA{}}%
\verse{Then Adonai said to Joshua, “This day I will begin exalting you \textit{in the sight}\lebnote{“in the eyes”} of all Israel, that they may know that I was with Moses, and I will be with you.}%
\verse{You will command the priests carrying the ark of the covenant, saying, ‘At the moment that you come to the edge of the waters of the Jordan, you will stand still in the Jordan.’ ”}%
\verse{And Joshua said to the \textit{Israelites},\lnAKM{} “Come here, and hear the words of Adonai your God.”}%
\verse{Joshua said, “By this you will know that the living God is in your midst, and he will certainly drive out the Canaanites \textit{from before you},\lebnote{Or “from your presence”} and the Hittites, Hivites, the Perizzites, the Girgashites, the Amorites, and the Jebusites.}%
\verse{Look! The ark of the covenant of the Lord of all the earth\lnEPD{} is about to cross over ahead of you into the Jordan.}%
\verse{So then, take twelve men from the tribes of Israel, \textit{one from each tribe}.\lebnote{“one man for the tribe”}}%
\verse{When the soles of the feet of the priests carrying the ark of Adonai, Lord of all the earth,\lnEPD{} rest in the waters of the Jordan, the waters of the Jordan will be cut off \textit{upstream},\lebnote{“going down from above”} and they will stand still in one heap.}%
\verse{And it happened, when the people set out from their tents to cross over the Jordan, the priests carrying the ark of the covenant were \textit{ahead of the people}.\lnEPA{}}%
\verse{When those carrying the ark came up to the Jordan, and the priests carrying the ark dipped their feet in the edge of the water (the Jordan was flowing over its banks during all the days of harvest),}%
\verse{the waters flowing down from above stood still; they stood up in one heap very far from Adam, the city that is beside Zarethan, while the waters flowing down to the sea of the Arabah, the Salt Sea,\lnDUH{} \textit{were completely cut off};\lebnote{“they were completed they were cut off”} and the people crossed over opposite Jericho.}%
\verse{And the priests carrying the ark of the covenant of Adonai stood firmly on the dry land in the middle of the Jordan while all Israel crossed on dry ground, until all the nation finished crossing the Jordan.}%
\end{biblechapter}

\begin{biblechapter} % Joshua 4
\verseWithHeading{The Israelites Make a Memorial}{After all the nation finished crossing the Jordan, Adonai said to Joshua,}%
\verse{“Take twelve men from the people, \textit{one man from each tribe},\lebnote{“man one man one from tribe”}}%
\verse{and command them, saying, ‘Take for yourselves twelve stones from the middle of the Jordan where the priests’ feet stood firmly, and bring them over with you, and set them up in \textit{the place where you will camp tonight}.’ ”\lebnote{“the lodging place where you will lodge tonight”}}%
\verse{So Joshua summoned the twelve men whom he had appointed from the \textit{Israelites},\lnAKM{} one from each tribe.}%
\verse{And Joshua said to them, “Cross over before\lebnote{Or “the presence”} the ark of Adonai your God to the middle of the Jordan, and each one of you lift up a stone on your\lnDAT{} shoulder, according to the number of the tribes of the \textit{Israelites},\lnAKM{}}%
\verse{so that this may be a reminder\lebnote{Or “sign”} among you. When your children ask in the future, saying, ‘\textit{What do these stones mean to you}?’\lebnote{“What are these stones for you?”}}%
\verse{you will say to them that the waters of the Jordan were cut off \textit{from before}\lebnote{Or “the presence of”} the ark of the covenant of Adonai. When it crossed the Jordan, the waters of the Jordan were cut off. These stones will be as a memorial for the \textit{Israelites}\lnAKM{} for eternity.”}%
\verse{Thus the \textit{Israelites}\lnAKM{} did as Joshua commanded. They took twelve stones from the middle of the Jordan as Adonai told Joshua, according to the number of the tribes of the \textit{Israelites},\lnAKM{} and they carried them over with them to \textit{the camp},\lebnote{“the place of overnight lodging”} and they put them there.}%
\verse{Then Joshua set up twelve stones in the middle of the Jordan, in the place where the feet of the priests carrying the ark of the covenant stood, and they are there to this day.}%
\verse{The priests carrying the ark remained standing in the middle of the Jordan until everything that Adonai commanded Joshua to tell the people was finished, according to all that Moses commanded Joshua. And the people hastily crossed over.}%
\verse{And it happened, when all the people had finished crossing, the ark of Adonai and the priests crossed over in front of\lebnote{Or “before the presence”} the people.}%
\verse{The children of Reuben, Gad, and the half-tribe of Manasseh crossed over armed before the \textit{Israelites},\lnAKM{} as Moses told them.}%
\verse{About forty thousand armed for fighting crossed over before the presence of Adonai to the plains of Jericho for battle.}%
\verse{On that day Adonai exalted Joshua \textit{in the sight}\lebnote{“in the eyes”} of all Israel, and they respected him\lebnote{Or “feared him”} as they respected\lebnote{Or “feared”} Moses all the days of his life.}%
\verse{Then Adonai said to Joshua, saying,}%
\verse{“Command the priests carrying the ark of the testimony\lebnote{That is, the ark of the covenant} \textit{to come up}\lebnote{“and let them come up”} from the Jordan.”}%
\verse{So Joshua commanded the priests, saying, “Come up from the Jordan.”}%
\verse{And it happened that when the priests carrying the ark came up from the middle of the Jordan, and the soles of the priests’ feet \textit{touched dry land},\lebnote{“were raised from the ground to dry land”} the waters of the Jordan returned to their place and flowed over all its banks as before.}%
\verse{And the people came up from the Jordan on the tenth day of the first month, and they camped in Gilgal on the eastern edge\lebnote{That is, border} of Jericho.}%
\verse{And those twelve stones that they took from the Jordan, Joshua set up in Gilgal.}%
\verse{And he said to the \textit{Israelites},\lnAKM{} “When your children ask in the future \textit{their parents},\lebnote{“their fathers”} ‘\textit{What is the meaning of these stones}?’\lebnote{“What are these stones?”}}%
\verse{you will let your children know by saying, ‘Israel crossed this Jordan on dry ground.’}%
\verse{For Adonai your God dried up the waters of the Jordan before you, until you had crossed, just as Adonai your God did to the \textit{Red Sea},\lebnote{“sea of reeds”} which he dried up before us until we had crossed over,}%
\verse{so that all the peoples of the earth may know that the hand of Adonai is strong, so that you may fear Adonai your God \textit{forever}.”\lnDZX{}}%
\end{biblechapter}

\begin{biblechapter} % Joshua 5
\verseWithHeading{The Israelites Are Circumcised}{And it happened, when all the kings of the Amorites who were beyond the Jordan to the west, and all the kings of the Canaanites\lnAMI{} who were by the sea heard that Adonai dried up the waters of the Jordan in front of the \textit{Israelites}\lnAKM{} until they crossed over, their hearts melted, and \textit{there was no courage left in them}\lebnote{“a spirit was no longer in them”} because of the presence of the \textit{Israelites}. \lnAKM{}}%
\verse{At that time Adonai said to Joshua, “Make for yourself knives of flint, and circumcise the \textit{Israelites}\lnAKM{} a second time.”}%
\verse{So Joshua made knives of flint, and he circumcised the \textit{Israelites}\lnAKM{} at the hill of the foreskins.\lebnote{Hebrew “Gibeath-haaraloth”}}%
\verse{This is the reason why Joshua circumcised all the people: all the males who went out from Egypt, all the warriors, died in the wilderness as they went out from Egypt \textit{on the journey}.\lnEPV{}}%
\verse{For all the people who left were circumcised, but all the people born in the wilderness \textit{on the journey}\lnEPV{} after they left from Egypt were not circumcised.}%
\verse{For forty years the \textit{Israelites}\lnAKM{} traveled in the wilderness until all the nation, the warriors that left Egypt, perished, because they did not listen to the voice of Adonai. To them Adonai swore that they would not see the land that he\lebnote{Hebrew “Adonai”} swore to their ancestors\lnAGP{} to give to us, a land flowing with milk and honey.}%
\verse{And it was their children whom he raised in their place that Joshua circumcised, for they were uncircumcised, because they had not been circumcised \textit{on the journey}.\lnEPV{}}%
\verseWithHeading{The Israelites Celebrate Passover in Canaan}{And it happened, when all the nation had finished circumcising, they remained where they were in the camp until they recovered.}%
\verse{And Adonai said to Joshua, “Today I have rolled away the disgrace of Egypt from you.” Therefore, the name of that place is called Gilgal\lebnote{Hebrew “rolling”} to this day.}%
\verse{And the \textit{Israelites}\lnAKM{} camped at Gilgal, and they kept the Passover on the fourteenth day of the month, in the evening, on the plains of Jericho.}%
\verse{On the next day after the Passover, on that\lnEQB{} very day, they ate from the produce of the land, unleavened cakes and roasted corn.}%
\verse{And the manna ceased the day after, when they started eating the produce of the land, and there was no longer manna for the \textit{Israelites}.\lnAKM{} They ate from the crop of the land of Canaan in that year.}%
\verseWithHeading{The Commander of Adonai’s Army Appears Before Joshua}{And it happened, when Joshua was by Jericho, \textit{he looked up},\lebnote{“he lifted up his eyes”} and he saw a man standing \textit{opposite him}\lebnote{“against him”} with his sword drawn in his hand. And Joshua went to him and said, “Are you with us, or with our adversaries?”}%
\verse{And he said, “Neither. I have come now as the commander of Adonai’s army.” And Joshua fell on his face to the earth, and \textit{he bowed down}\lebnote{Or “he worshiped”} and said to him, “What is my lord commanding his servant?”}%
\verse{The commander of Adonai’s army said to Joshua, “Take off your sandals\lebnote{Hebrew “sandal”} from your feet,\lebnote{Hebrew “foot”} for the place where you are standing is holy.” And Joshua did so.}%
\end{biblechapter}

\begin{biblechapter} % Joshua 6
\verseWithHeading{The Battle of Jericho}{Now Jericho was shut up inside and out because of the presence of the \textit{Israelites};\lnAKM{} no one was going out or coming in.}%
\verse{And Adonai said to Joshua, “Look, I am giving Jericho into your hand, its king and the soldiers of the army.}%
\verse{You will march around the city, all the warriors circling the city once; you will do so for six days.}%
\verse{And seven priests will bear seven trumpets of rams’ horns before the ark. On the seventh day you will march around the city seven times, and the priests will blow on the trumpets.}%
\verse{And when they blow long on the horn of the ram, when you hear the sound of the trumpet, all the people will shout with a great war cry, and the wall of the city will fall flat,\lebnote{Or “in its place”} and \textit{the people will charge, each one straight ahead}.”\lebnote{“the people will go up, each before him”}}%
\verse{So Joshua son of Nun summoned the priests and said to them, “Take up the ark of the covenant, and let seven priests carry the trumpets of the rams’ horns before the ark of Adonai.”}%
\verse{And he said\lebnote{Hebrew “they said”} to the people, “Go forward and march around the city, and let the armed men pass before the ark of Adonai.}%
\verse{\textit{And when Joshua spoke}\lebnote{“And it happened the moment of the saying of Joshua”} to the people, the seven priests carrying the seven trumpets of rams’ horns before the presence of Adonai went forward and they blew the trumpets; and the ark of the covenant of Adonai followed behind them.}%
\verse{And the armed men went before the priests who blew the trumpets, and the rear guard came after the ark, while they were blowing the trumpets.}%
\verse{But Joshua commanded the people, saying, “You will not shout, and you will not let your voice be heard; a word will not go out from your mouth until the day I say to you ‘Shout!’ Then you will shout.”}%
\verse{And the ark of Adonai went around the city, \textit{circling once},\lebnote{“circling one occurrence”} and they came into the camp and spent the night in the camp.}%
\verse{Then Joshua got up early in the morning, and the priests took up the ark of Adonai.}%
\verse{The seven priests carrying the seven trumpets of the rams’ horns before the ark of Adonai went on continually, and they blew on the trumpets. And the armed men went before them, and the rear guard came after the ark of Adonai, while the trumpets blew continually.}%
\verse{And they marched around the city once on the second day, and they returned to the camp. They did this for six days.}%
\verse{Then on the seventh day they rose early at dawn, and they marched around the city in this manner seven times. It was only on that day that they marched around the city seven times.}%
\verse{And at the seventh time the priests blew on the trumpets, and Joshua said to the people, “Shout! For Adonai has given you the city.}%
\verse{The city and all that is in it will be devoted to Adonai; only Rahab the prostitute and all who are with her in the house will live, because she hid the messengers whom we sent.}%
\verse{As for you, keep away from the things\lebnote{Hebrew “thing”} devoted to destruction so that you do not take them and bring about your own destruction, making the camp of Israel an object for destruction, bringing trouble upon it.}%
\verse{But all of the silver and gold, and the items of bronze and iron, are holy to Adonai, and they must go to Adonai’s treasury.”}%
\verse{So the people shouted, and they\lebnote{That is, the priests} blew on the trumpets. And when the people heard the sound of the trumpet, they raised a great shout, and the wall fell flat. \textit{The people charged, each one straight ahead into the city},\lebnote{“The people went up, each before him”} and they captured it.}%
\verse{And they utterly destroyed \textit{by the edge of the sword}\lebnote{“by the mouth of the sword”} all who were in the city, both men and women, young and old, ox, sheep, and donkey.}%
\verse{Then Joshua said to the two men who spied on the land, “Go to the prostitute’s house and bring out from there the woman and all who belong to her, just as you swore to her.”}%
\verse{So the young men who were spies went and brought Rahab and her father and mother, her brothers, and all who were with her. And they brought all her family out and set them outside the camp of Israel.}%
\verse{And they burned the city and all that was in it with fire; they put only the silver and gold, and the items of copper and iron, into the treasury of the house of Adonai.}%
\verse{But Joshua spared Rahab the prostitute \textit{and her family}\lebnote{“and the house of her father”} and all who were with her, and she has lived in the midst of Israel until this day, because she hid the messengers whom Joshua sent to spy out Jericho.}%
\verse{And Joshua swore at that time, saying, “Cursed is anyone before Adonai who gets up and builds Jericho, this city. At the cost of his firstborn he will lay its foundation, and at the cost of his youngest he will set up its gates.”}%
\verse{So Adonai was with Joshua, and his fame was in all the land.}%
\end{biblechapter}

\begin{biblechapter} % Joshua 7
\verseWithHeading{The Sin of Achan}{But the \textit{Israelites}\lnAKM{} broke faith concerning the devoted things.\lebnote{Or “consecrated possessions”} Achan son of Carmi son of Zabdi son of Zerah, of the tribe of Judah, took from the devoted things;\lnEQF{} and \textit{Adonai’s anger was kindled}\lebnote{“the nose of Adonai became hot”} against the \textit{Israelites}.\lnAKM{}}%
\verse{Now Joshua sent men from Jericho to Ai, which is near Beth Aven, east of Bethel, and he said to them, “Go up and spy out Ai.” And the men went up and spied out Ai.}%
\verse{And they returned to Joshua and said to him, “Do not let all the people go up and attack Ai; only two or three thousand men should go up because they\lebnote{That is, the people of Ai} are few. Do not make all the people weary up there.”}%
\verse{So about three thousand from the people went up there, and they fled before the men of Ai.}%
\verse{The men of Ai killed about thirty-six of them, and they chased them from the gate up to Shebarim and killed them on the slope. And the hearts\lnEOY{} of the people melted and became like water.}%
\verse{And Joshua tore his clothes and fell to the ground on his face before the ark of Adonai until the evening, he and the elders of Israel; and they put dust on their heads.\lnENT{}}%
\verse{And Joshua said, “Ah, my Lord! Why did you bring this people across the Jordan to give us into the hand of the Amorites to destroy us? If only we had been content and stayed beyond the Jordan.}%
\verse{Please, my Lord! What can I say after \textit{Israel has fled from its enemies}?\lebnote{“Israel has turned its neck before its enemies”}}%
\verse{The Canaanites and all the inhabitants of the land will hear of this, and they will surround us and cut off our name from the land. What will you do, for your great name?”}%
\verse{And Adonai said to Joshua, “Stand up! \textit{Why}\lebnote{“For what this”} have you fallen on your face?}%
\verse{Israel has sinned and \textit{transgressed my covenant}\lebnote{“crossed my covenant”} that I commanded them. They have taken from the devoted things;\lebnote{Hebrew “thing” or “from the consecrated possession”} they have stolen and acted deceitfully, and they have put them among their belongings.}%
\verse{The \textit{Israelites}\lnAKM{} were unable to stand before their enemies; \textit{they fled from their enemies}\lebnote{“they turned their neck before their enemies”} because they have become a thing devoted\lebnote{Or “consecrated”} for destruction. \textit{I will be with you no more}\lebnote{“I will not do again to be with you”} unless you destroy the devoted things\lnEQF{} from among you.}%
\verse{Get up, sanctify the people, and say, ‘Sanctify yourselves for tomorrow. Thus says Adonai the God of Israel: “There are devoted things\lnEQF{} in your midst, O Israel. You will be unable to stand before you enemies until you remove the devoted things\lnEQF{} from your midst.”}%
\verse{In the morning you will come forward, \textit{tribe by tribe},\lebnote{“according to your tribes”} and the tribe that Adonai will select by lot will come forward by clans, and the clan that Adonai selects by lot will come forward by families, and the family that Adonai selects by lot will come forward one by one.}%
\verse{The one caught with the devoted things\lnEQF{} will be burned with fire, he and all that belongs to him, because \textit{he transgressed the covenant}\lebnote{“he crossed the covenant”} of Adonai, and because he did a disgraceful thing in Israel.” ’ ”}%
\verse{So\lnAAA{} Joshua rose early in the morning and brought forward Israel, \textit{tribe by tribe},\lebnote{“according to its tribes”} and the tribe of Judah was selected by lot.}%
\verse{And he brought forward the clans of Judah and selected the clan of the Zerahites\lnDPA{} by lot. Then he brought forward the clan of the Zerahites,\lnDPA{} one by one, and Zabdi was selected by lot.}%
\verse{He brought forward his family, one by one, and Achan son of Carmi son of Zabdi son of Zerah, of the tribe of Judah, was selected by lot.}%
\verse{And Joshua said to Achan, “My son, please, give glory to Adonai the God of Israel, and give him a doxology in court.\lebnote{Some interpret “make a confession”} Tell me, please, what you have done; do not hide it from me.”}%
\verse{And Achan answered Joshua and said, “It is true. I have sinned against Adonai the God of Israel, and this is what I did:}%
\verse{I saw among the spoil a beautiful robe from Shinar,\lebnote{“Shinar” refers to Babylonia} two hundred shekels of silver, and one bar of gold that weighed fifty shekels; I coveted them and took them. They are hidden in the ground inside my tent, and the silver is under it.”}%
\verse{Joshua sent messengers, and they ran to the tent; and there they were, hidden in his tent, and the silver was under it.}%
\verse{And they took them from the tent and brought them to Joshua and all the \textit{Israelites}.\lnAKM{} And they spread them out before the presence of Adonai.}%
\verse{Then Joshua, and all Israel with him, took Achan son of Zerah, the silver, the robe, the bar of gold, his sons and daughters, his cattle and donkeys, his sheep, his tent, and everything that was his, and they brought them to the valley of Achor.\lnEQS{}}%
\verse{And Joshua said, “Why did you bring us trouble? Adonai will bring you trouble on this day.” And all Israel stoned them\lnDJA{} with stones;\lebnote{Hebrew “stone”} and they burned them with fire after they stoned them with stones.}%
\verse{Then \textit{they placed}\lebnote{“they raised up”} on top of him a great pile of stones that remains to this day. And Adonai \textit{turned from his burning anger},\lebnote{“turned from his burning nose”} and thus the name of that place to this day is called the valley of Achor.\lnEQS{}}%
\end{biblechapter}

\begin{biblechapter} % Joshua 8
\verseWithHeading{Ai Is Destroyed}{Then Adonai said to Joshua, “Do not fear or be dismayed. Take \textit{all the fighting men}\lnEQV{} with you \textit{and go up immediately to Ai}.\lebnote{“get up and go up to Ai”} Look, I am giving into your hand the king of Ai, his city, and his land.}%
\verse{You will do to Ai and its king that which you did to Jericho and its king; you may take only its spoils\lebnote{Hebrew “spoil”} and livestock as booty for yourself. Set for yourself an ambush against the city from behind it.”}%
\verse{So Joshua and \textit{all the fighting men}\lnEQV{} went up immediately to Ai. Joshua chose thirty thousand of the best fighting men and sent them by night.}%
\verse{And he commanded them, saying, “Look, you are to lay an ambush against the city from behind. Do not go very far from the city and be ready.}%
\verse{Then I and all of the people who are with me will approach the city. And when they go out to meet us \textit{as before},\lnEQX{} we will flee from them.\lnEQY{}}%
\verse{They will come out after us until we draw them away from the city, because they will think, ‘They are fleeing from us\lebnote{Or “before our presence”} \textit{as before}.’\lnEQX{} So we will flee from them.\lnEQY{}}%
\verse{Then you will rise up from the ambush and take possession of the city, for Adonai your God will give it into your hand.}%
\verse{And when you capture the city you will set it on fire as Adonai commanded. Look, I have commanded you.”}%
\verse{So Joshua sent them out, and they went to the place of the ambush, and they sat between Bethel and Ai, to the west of Ai; but Joshua spent the night \textit{with the people}.\lebnote{“in the middle of the people”}}%
\verse{Joshua rose early in the morning and mustered the people, and he went up with the elders of Israel before the people of Ai.\lebnote{Or “before the presence of the people of Ai”}}%
\verse{\textit{All the fighting men}\lebnote{“All the people of war”} who were with him \textit{went up and drew near before the city}\lebnote{“went up, drew near, and came before the city”} and camped north of Ai; there was a valley between him and Ai.}%
\verse{And he took about five thousand men and set them in ambush between Bethel and Ai, to the west of the city.}%
\verse{So they stationed the forces; all the army was north of the city while \textit{the rear guard was west}.\lebnote{“while its rear guard was west of the city”} But Joshua went that night to the middle of the valley.}%
\verse{When the king of Ai saw this, the men of the city hurried and rose early and went out to meet Israel for battle—he\lebnote{That is, the king} and all his army—to the meeting place before the Arabah.\lnERB{} He did not know that there was an ambush for him behind the city.}%
\verse{Then Joshua and all Israel acted like they were beaten before them, and they fled \textit{in the direction of the wilderness}.\lebnote{“the way of the wilderness”}}%
\verse{All of the people who were in the city were called to pursue after them. As they pursued after Joshua, they were drawn away from the city.}%
\verse{Not a man remained in Ai or Bethel who had not gone out after Israel; they left the city open and pursued after Israel.}%
\verse{And Adonai said to Joshua, “Stretch out the sword\lebnote{Or “spear”} that is in your hand to Ai, because I will give it into your hand.” And Joshua stretched out the sword that was in his hand to the city.}%
\verse{The moment he stretched out his hand, those in the ambush stood up quickly from their place and ran. And they went into the city and captured it, quickly setting the city ablaze with fire.}%
\verse{Then the men of Ai looked behind them, and they saw smoke from the city rising to the sky; \textit{they had no power to flee this way or that},\lebnote{“it was not in their hands to flee here and here”} and the people fleeing the wilderness turned around to the pursuers.\lebnote{Hebrew “pursuer”}}%
\verse{And Joshua and all Israel saw that the ambush had captured the city and that the smoke of the city was rising; they returned and struck down the men of Ai.}%
\verse{Then the others from the city came out to meet them, \textit{and they found themselves surrounded by Israel},\lebnote{“they were in the middle of Israel”} \textit{some on one side, and others on the other side}.\lebnote{“these from these, and these from this”} And they\lnDNX{} struck them down until no survivor or fugitive was left.}%
\verse{But they captured the king of Ai alive, and they brought him to Joshua.}%
\verse{When Israel finished slaughtering all the inhabitants of Ai in the open field, in the wilderness where they pursued them, and when all of them had fallen by \textit{the edge of the sword}\lnERD{} until they all had perished, all Israel returned to Ai and attacked it with \textit{the edge of the sword}.\lnERD{}}%
\verse{All the people that fell on that day, both men and women, were twelve thousand—all the inhabitants of Ai.}%
\verse{For Joshua did not draw back his hand that was stretched out with the sword until he had utterly destroyed all the inhabitants of Ai.}%
\verse{Only the livestock and the spoil of that city Israel took as booty for themselves, according to the word of Adonai that Joshua commanded.}%
\verse{So Joshua burned Ai and made it an everlasting heap of rubbish, a desolate place until this day.}%
\verse{The king of Ai he hanged on a tree until the time of evening, and as the sun went down Joshua commanded them, and they brought down his dead body from the tree. Then they threw it at the entrance of the gate of the city, and they raised over it a great heap of stones that remains to this day.}%
\verseWithHeading{Israel Renews the Covenant}{Then Joshua built an altar on Mount Ebal for Adonai the God of Israel,}%
\verse{as Moses Adonai’s servant commanded the \textit{Israelites},\lnAKM{} as it is written in the scroll of the law of Moses: “an altar of unhewn\lebnote{Or “whole”} stones on which no one has \textit{wielded}\lebnote{“waved over them”} an iron implement.”\lebnote{See Exod 20:25} And they offered burnt offerings on it and sacrificed fellowship offerings.}%
\verse{And there Joshua wrote on the stones a copy of the law of Moses, which he\lebnote{That is, Moses} had written, in the presence of the \textit{Israelites}.\lnAKM{}}%
\verse{Then all Israel, \textit{foreigner as well as native},\lebnote{“as the alien as the native”} with the elders, officials, and judges stood \textit{on either side}\lnATG{} of the ark before the priests and the Levites, who carried the ark of the covenant of Adonai. Half of them stood in front of Mount Gerizim, and the other half in front of Mount Ebal, as Moses Adonai’s servant had commanded before to bless the people of Israel.}%
\verse{And afterward he read all the words of the law, the blessings\lebnote{Hebrew “blessing”} and the curses,\lebnote{Hebrew “curse”} according to all that was written in the scroll of the law.}%
\verse{There was not a word from all that Moses commanded that Joshua did not read before the assembly of all Israel, and the women, the little children, and the traveling foreigners\lebnote{Hebrew “foreigner”} among them.}%
\end{biblechapter}

\begin{biblechapter} % Joshua 9
\verseWithHeading{The Gibeonites Act with Cunning}{Now\lebnote{Or “And it happened”} when all the kings who were beyond the Jordan in the hill country and in the Shephelah,\lnERI{} and on all the coast of the great sea toward Lebanon\lebnote{“white mountain”}—the Hittites,\lnAMJ{} the Amorites,\lnAMK{} the Canaanites,\lnAMI{} the Perizzites,\lnAML{} the Hivites,\lnAMM{} and the Jebusites\lebnote{Hebrew “Jebusite”}—heard of this,}%
\verse{they gathered themselves together to fight with one accord against\lebnote{Hebrew “with”} Joshua and Israel.}%
\verse{But the inhabitants of Gibeon heard what Joshua did to Jericho and Ai,}%
\verse{and they acted on their part with cunning: they went and prepared provisions,\lebnote{The Hebrew is difficult here. Some ancient manuscripts read, “they sent out a delegation/an envoy”} and took worn-out sacks\lebnote{Or “sackcloths”} for their donkeys and old wineskins that were torn and mended.}%
\verse{The sandals on their feet were patched and old, their clothes were old, and their food was dry and crumbled.}%
\verse{And they went to Joshua at the camp at Gilgal and said to him and to the men of Israel, “We have come from a far land; so then \textit{make a covenant with us}.”\lnERO{}}%
\verse{And the men of Israel said to the Hivites,\lnAMM{} “Perhaps you are living among us; how can we \textit{make a covenant}\lebnote{“cut a covenant”} with you?”}%
\verse{They said to Joshua, “We are your servants.” And Joshua said to them, “Who are you, and from where do you come?”}%
\verse{And they said to him, “Your servants have come from a very far land because of the name of Adonai your God; we have heard of his reputation, of all that he did in Egypt,}%
\verse{and of all that he did to the two kings of the Amorites who were beyond the Jordan—to Sihon king of Heshbon and to Og king of Bashan, who was in Ashtaroth.}%
\verse{So our elders said to us and all the inhabitants of our land, ‘Take in your hand provisions for the journey, and go to meet them, and say to them, “We are your servants; so then \textit{make a covenant with us}.” ’\lnERO{}}%
\verse{This is our bread; it was hot when we took it from our houses as provisions on the day we set out to come to you. But now, look, it is dry and crumbled.}%
\verse{These are the wineskins that we filled new, but look, they have burst; and these are our clothes and sandals that have worn out from the very long journey.”}%
\verse{So the leaders\lnERR{} took from their provisions, but \textit{they did not ask direction from Adonai}.\lebnote{“the mouth of Adonai they did not ask”}}%
\verse{And Joshua made peace with them, and \textit{he made a covenant with them}\lebnote{“he cut a covenant with them”} to allow them to live happily, and the leaders of the congregation swore an oath to them.}%
\verse{And it happened that at the end of three days, after they made a covenant with them, they heard that \textit{they were their neighbors}\lebnote{“they were near them”} and living among them.}%
\verse{And the \textit{Israelites}\lnAKM{} set out and went to their cities on the third day (their cities were Gibeon, Kephirah, Beeroth, and Kiriath Jearim).}%
\verse{But the \textit{Israelites}\lnAKM{} did not attack them, because the leaders of the congregation had sworn to them by Adonai the God of Israel. And all the congregation murmured \textit{against their leaders}.\lebnote{“against the leaders of the congregation”}}%
\verse{But all the leaders of the congregation said, “We have sworn to them by Adonai the God of Israel, and so we cannot touch them.}%
\verse{This we will do to them: we will let them live so that wrath will not be on us because of the oath we swore to them.”}%
\verse{And the leaders\lnERR{} said to them, “Let them live.” So they became woodcutters and water carriers for all the congregation, just as the leaders had said to them.}%
\verse{And Joshua summoned them and said, “Why have you deceived us saying, ‘We are very far from you’ when you are living among us?}%
\verse{Therefore you are cursed; some of you will always be slaves as woodcutters and water carriers for the house of my God.”}%
\verse{And they answered Joshua and said, “Because it was told with certainty to your servants that Adonai your God commanded Moses his servant to give to you all the land and to destroy all the inhabitants of the land before you, so we were very afraid for our lives because of you, and so we did this thing.}%
\verse{So then, look, we are in your hand; do with us whatever seems good and right in your eyes.”}%
\verse{So he did this to them: he saved them from the hand of the \textit{Israelites},\lnAKM{} and they did not kill them.}%
\verse{And that day Joshua made them woodcutters and water carriers for the congregation and for the altar of Adonai, to this day, in the place that he should choose.}%
\end{biblechapter}

\begin{biblechapter} % Joshua 10
\verseWithHeading{The Sun Stands Still at Gibeon}{And it happened that when Adoni-Zedek king of Jerusalem heard that Joshua captured Ai and had utterly destroyed it (just as he had done to Jericho and its king, so he did to Ai and its king) and that the inhabitants of Gibeon had made peace with Israel and were among them,}%
\verse{he\lebnote{Hebrew “they”} became very afraid because Gibeon was a very large city, \textit{like one of the royal cities},\lebnote{“like one of the cities of the kingship”} and because it was larger than Ai, and all its men were mighty warriors.}%
\verse{So Adoni-Zedek king of Jerusalem sent word to Hohman king of Hebron, to Piram king of Jarmuth, to Japhia king of Lachish, and to Debir king of Eglon, saying,}%
\verse{“Come up and help me, and let us attack Gibeon, because it has made peace with Joshua and the \textit{Israelites}.”\lnAKM{}}%
\verse{And the five kings of the Amorites,\lnAMK{} the king of Jerusalem, the king of Hebron, the king of Jarmuth, the king of Lachish, and the king of Eglon, gathered together and went up, they and all their forces, and \textit{they laid siege to Gibeon}\lebnote{“they camped against Gibeon”; see Josh 10:31} and made war against it.}%
\verse{And the men of Gibeon sent word to Joshua at the camp at Gilgal, saying, “\textit{Do not abandon}\lebnote{“do not release your hands from”} your servant. Come up to us quickly and save us! Help us, for all the kings of the Amorites\lnAMK{} who dwell in the hill country have gathered against us.”}%
\verse{So Joshua went up from Gilgal, he and \textit{all the fighting men}\lebnote{“all the people of the war”} with him, \textit{all the best warriors}.\lebnote{“all the mighty warriors of the army”}}%
\verse{And Adonai said to Joshua, “Do not be afraid of them, for I have given them into your hand; \textit{no one will withstand you}.\lebnote{“not a man of them will stand in your presence”}}%
\verse{Joshua came upon them suddenly \textit{by marching up}\lebnote{“he went up”} all night from Gilgal.}%
\verse{And Adonai threw them into panic before Israel, who\lnAQG{} struck them with a great blow at Gibeon and pursued them by the way of the ascent of Beth-horon and struck them as far as Azekah and Makkedah.}%
\verse{And as they were fleeing from Israel, they were on the slope of Beth-horon, and Adonai threw huge stones from the heavens on them as far as Azekah; and more died by the hail stones than those whom the \textit{Israelites}\lnAKM{} killed by the sword.}%
\verse{Then Joshua spoke to Adonai, on the day Adonai gave the Amorites\lnAMK{} over to the \textit{Israelites},\lnAKM{} and he said in the sight of Israel,}%
\verse{“Sun in Gibeon, stand still, 
and moon, in the valley of Aijalon.”}%
\verse{Is it not written in the scroll of Jashar? The sun stood still in the middle of the heaven and was not in haste to set for about a full day.}%
\verse{There has not been a day like this before it or after, that Adonai listened to the voice of a man; for Adonai fought for Israel.}%
\verseWithHeading{The Kings of the Amorites Are Killed}{But these five kings fled and hid themselves in the cave at Makkedah.}%
\verse{And it was told to Joshua, saying, “The five kings were found hidden in the cave at Makkedah.”}%
\verse{And Joshua said, “Roll large stones against the mouth of the cave, and set men in front of it to guard them.}%
\verse{But do not stay there; pursue after your enemies and attack them from the rear. Do not allow them to go into their cities, for Adonai your God has given them into your hand.”}%
\verse{When Joshua and the \textit{Israelites}\lnAKM{} had finished striking them with a very great blow, until they perished, \textit{those of them who survived}\lebnote{“the survivors who survived”} went into the fortified cities,}%
\verse{and all the people returned to the camp safely\lebnote{Or “in peace”} to Joshua at Makkedah. \textit{No one spoke}\lebnote{“No one moved his tongue”} against the \textit{Israelites}.\lnAKM{}}%
\verse{And Joshua said, “Open the mouth of the cave, and bring to me those five kings from the cave.”}%
\verse{And they did so, and brought him these five kings from the cave, the king of Jerusalem, the king of Hebron, the king of Jarmuth, the king of Lachish, and the king of Eglon.}%
\verse{And when they brought these kings to Joshua, Joshua called all the men of Israel and said to the commanders of \textit{the fighting men}\lebnote{“the men of war”} who had gone with him, “Come near, put your feet on the necks of these kings.” So they came near and put their feet on their necks.}%
\verse{And Joshua said to them, “Do not be afraid or dismayed! Be strong and bold, for thus Adonai will do to all your enemies whom you are about to fight.}%
\verse{And after this Joshua struck them down and killed them, and he hanged them on five trees. And they were hanging on the trees until the evening.}%
\verse{And it happened \textit{at the time of sunset},\lebnote{“at the time of the going of the sun”} Joshua commanded, and they took them down from the trees and threw them into the cave where they had hidden themselves, and they put large stones against the mouth of the cave, which are there to this very day.}%
\verse{Joshua captured Makkedah on that day, and he struck it and its king with \textit{the edge of the sword};\lnERD{} he utterly destroyed it and everyone that was in it. He did not leave behind a survivor. So he did to the king of Makkedah just as he did to the king of Jericho.}%
\verseWithHeading{Joshua’s Conquest of the South}{And Joshua passed on, and all of Israel with him, from Makkedah to Libnah, and he fought against Libnah.}%
\verse{And Adonai also gave it into the hand of Israel, and its king and all the people in it he struck with \textit{the edge of the sword}.\lnERD{} He left in it no survivor. He did to its king just as he did to the king of Jericho.}%
\verse{And Joshua passed on, and all of Israel with him, from Libnah to Lachish, and \textit{he laid siege to it}\lebnote{“he camped opposite it”} and fought against it.}%
\verse{And Adonai gave Lachish into the hand of Israel, and he captured it on the second day. He struck it with \textit{the edge of the sword},\lnERD{} and everyone in it, just as he did to Libnah.}%
\verse{Then Horam king of Gezer came up to help Lachish, and Joshua struck him and his people until he left no survivor behind.}%
\verse{And Joshua passed on, and all of Israel with him, from Lachish to Eglon, and \textit{they laid siege to it}\lebnote{“they camped opposite it”} and fought against it.}%
\verse{And they captured it on that day, and he struck it with \textit{the edge of the sword},\lnERD{} and all the people that were in it on that day he utterly destroyed as he had done to Lachish.}%
\verse{And Joshua went up, and all Israel with him, from Eglon to Hebron, and they fought against it}%
\verse{and captured it, and they struck it with \textit{the edge of the sword},\lnERD{} its king and all its cities, and all the people that were in it; he left behind no survivor, as he had done to Eglon, and he utterly destroyed it and all the people that were in it.}%
\verse{Then Joshua returned to Debir, and all of Israel with him, and they fought against it,}%
\verse{and he captured it and its king and all its cities, and they struck them with \textit{the the edge of the sword},\lnERD{} and they utterly destroyed all the people that were in it; he left behind no survivor, just as he had done to Hebron. Thus he did to Debir and its king what he had done to Libnah and its king.}%
\verse{So Joshua struck all the land—the hill country, the Negev,\lnESL{} the Shephelah,\lnERI{} and the slopes\lebnote{The slopes of the hills of the western Jordan to the Dead Sea region}—and all their kings; he left behind no survivor, and \textit{all that breathed}\lebnote{“all of the breath”} he utterly destroyed as Adonai the God of Israel commanded.}%
\verse{Joshua struck them from Kadesh Barnea to Gaza, and all the land of Goshen up to Gibeon;}%
\verse{all of these kings and their land Joshua captured at one time, because Adonai the God of Israel fought for Israel.}%
\verse{And Joshua returned, and all Israel with him, to the camp at Gilgal.}%
\end{biblechapter}

\begin{biblechapter} % Joshua 11
\verseWithHeading{Joshua’s Conquest of the North}{And it happened, when Jabin king of Hazor heard this, he sent to Jobab king of Madon, to the king of Shimron, to the king of Acshaph,}%
\verse{and to the kings who were in the north in the hill country, in the Arabah\lebnote{A dry region that runs south of the Sea of Galilee along the Jordan valley} south of Kinnereth,\lebnote{That is, the Sea of Galilee} in the Shephelah,\lnERI{} and in Naphoth Dor\lebnote{Or “the heights of Dor”} in the west,}%
\verse{to the Canaanites\lnAMI{} in the east and west, the Amorites,\lnAMK{} the Hittites,\lnAMJ{} the Perizzites,\lnAML{} and the Jebusites\lebnote{Hebrew “Jebusite”} in the hill country, and the Hivites\lnAMM{} \textit{at the foot of}\lebnote{“under”} Hermon in the land of Mizpah.}%
\verse{They came out, they and all their armies with them, as a great army like the sand on the seashore, with very many horses and chariots.}%
\verse{And all these kings joined forces, and they came and camped together by the waters of Merom to fight with Israel.}%
\verse{And Adonai said to Joshua, “Do not be afraid because of their presence, for tomorrow at this time I will hand them over slain to Israel; you will hamstring their horses and burn their chariots with fire.”}%
\verse{So Joshua, and \textit{all the fighting men}\lnEQV{} with him, came against them suddenly at the waters of Merom, and \textit{they attacked them}.\lebnote{“they fell upon them”}}%
\verse{And Adonai gave them into the hand of Israel, and they struck them and pursued them up to Great Sidon and Misrephoth Maim, and eastward up to the valley of Mizpeh. And they struck them until they left behind no survivor.}%
\verse{And Joshua did to them as Adonai commanded him; he hamstrung their horses and burned their chariots with fire.}%
\verse{Then Joshua turned back at that time, and he captured Hazor and struck its king with the sword, because Hazor formerly was the head of all these kingdoms.}%
\verse{He struck all the people that were in it with \textit{the edge of the sword},\lnERD{} utterly destroying them. \textit{There was no one left who breathed},\lebnote{“No one was left over of any breath”} and he burned Hazor with fire.}%
\verse{And Joshua captured all the cities of these kings, and all their kings, and he utterly destroyed them with \textit{the edge of the sword},\lnERD{} as Moses the servant of Adonai commanded.}%
\verse{Israel did not burn the cities standing on their mounds,\lebnote{Hebrew “mound”} except Hazor alone, which Joshua burned.}%
\verse{And all the spoil and livestock of these cities the \textit{Israelites} took as booty; they struck the people with \textit{the edge of the sword},\lnERD{} until they had destroyed them—they left behind no one who breathed.}%
\verse{Just as Adonai commanded Moses his servant, so Moses commanded Joshua, and Joshua did; he left nothing undone that Adonai had commanded Moses.}%
\verseWithHeading{A Review of Joshua’s Conquests}{So Joshua took all this land: the hill country, all the Negev,\lnESL{} all the land of Goshen,\lebnote{A southern region; the name means “on the mountains”} the Shephelah,\lnERI{} the Arabah,\lebnote{A dry region that runs south of the Sea of Galilee along the Jordan valley} and the hill country of Israel and its Shephelah,\lnERI{}}%
\verse{from Mount Halak that rises to Seir and to Baal Gad in the valley of Lebanon\lnETA{} \textit{at the foot of}\lebnote{“under”} Mount Hermon; he captured all their kings, struck them, and killed them.}%
\verse{For many days Joshua made war with all these kings.}%
\verse{There was not a city that made peace with the \textit{Israelites}\lnAKM{} besides the Hivites\lnAMM{} and the inhabitants of Gibeon—\textit{all were taken in battle}.\lebnote{“the all they took in the battle”}}%
\verse{For it was Adonai that \textit{hardened their hearts},\lebnote{“made their hearts strong”} to meet Israel in war in order to utterly destroy them without mercy, that they would destroy them just as Adonai commanded Moses.}%
\verse{At that time Joshua came and exterminated the Anakites from the hill country, from Hebron, Debir, Anab, and from all the hill country of Judah, and from all the hill country of Israel; Joshua utterly destroyed them with their cities.}%
\verse{None of the Anakites were left in the land of the \textit{Israelites};\lnAKM{} some remained only in Gaza, Gath, and Ashdod.}%
\verse{Joshua took all the land according to all that Adonai had spoken to Moses; and Joshua gave it as an inheritance to Israel, according to their tribal divisions, and the land rested from war.}%
\end{biblechapter}

\begin{biblechapter} % Joshua 12
\verseWithHeading{The Kings Conquered by Joshua}{These are the kings of the land whom the \textit{Israelites}\lnAKM{} defeated, and of whose land they took possession beyond the Jordon \textit{to the east},\lebnote{“to the rising of the sun”} from the wadi\lnDWS{} of Arnon up to Mount Hermon, and all the Arabah\lnERB{} to the east:}%
\verse{Sihon king of the Amorites, who lived in Heshbon, and ruled from Aroer, which is on the edge of the wadi\lnDWS{} of Arnon, from the middle of the valley and half of Gilead, up to the \textit{Jabbok River},\lebnote{“Jabbok the wadi”} which marks the border of the \textit{Ammonites};\lebnote{“sons of Ammon” or “children of Ammon”}}%
\verse{and the Arabah\lnERB{} up to the Kinnereth Sea\lebnote{That is, the Sea of Galilee} to the east, and as far as the sea of Arabah, the Salt Sea\lnDUH{} to the east, \textit{in the direction of}\lebnote{“the way of”} Beth Jeshimoth, and to the area southward, \textit{at the foot of}\lebnote{“under”} the slopes of Pisgah;\lebnote{Or Ashdoth Pisgah}}%
\verse{the territory of Og king of Bashan, one of the last of the Rephaites, who lived at Ashtaroth and Edrei}%
\verse{and ruled over Mount Hermon and Salecah and over all Bashan up to the border of the Geshurites\lnETK{} and the Maacathites,\lnETL{} and half of Gilead, as far as the border of Sihon king of Heshbon.}%
\verse{Moses Adonai’s servant and the \textit{Israelites}\lnAKM{} defeated them; and Moses Adonai’s servant gave it as a possession to the Reubenites,\lnDOX{} the Gadites,\lnDUO{} and the half-tribe of Manasseh.}%
\verseWithHeading{The Kings Conquered by Moses}{These are the kings of the land whom Joshua and the \textit{Israelites}\lnAKM{} defeated beyond to the Jordan to the west, from Baal Gad in the valley of Lebanon,\lnETA{} and up to Mount Halak, which rises to Seir. And Joshua gave it as a possession to the tribes of Israel according to their allotments,}%
\verse{in the hill country, the Shephelah,\lnERI{} the Arabah,\lnERB{} on the slopes, in the wilderness, and in the Negev;\lnESL{} the Hittites,\lnAMJ{} the Amorites,\lnAMK{} the Canaanites,\lnAMI{} the Perizzites,\lnAML{} the Hivites,\lnAMM{} and the Jebusites:\lebnote{Hebrew “Jebusite”}}%
\verse{the king of Jericho, one; the king of Ai, which is beside Bethel, one;}%
\verse{the king of Jerusalem, one; the king of Hebron, one;}%
\verse{the king of Jarmuth, one; the king of Lachish, one;}%
\verse{the king of Eglon, one; the king of Gezer, one;}%
\verse{the king of Debir, one; the king of Geder, one;}%
\verse{the king of Hormah, one; the king of Arad, one;}%
\verse{the king of Libnah, one; the king of Adullam, one;}%
\verse{the king of Makkedah, one; the king of Bethel, one;}%
\verse{the king of Tappuah, one; the king of Hepher, one;}%
\verse{the king of Aphek, one; the king of Lasharon, one;}%
\verse{the king of Madon, one; the king of Hazor, one;}%
\verse{the king of Shimron-meron, one; the king of Acshaph, one;}%
\verse{the king of Taanach, one; the king of Megiddo, one;}%
\verse{the king of Kedesh, one; the king of Jokneam in Carmel, one;}%
\verse{the king of Dor in Naphath Dor, one; the king of Goiim for Gilgal, one;}%
\verse{the king of Tirzah, one; all the kings, thirty-one.}%
\end{biblechapter}

\begin{biblechapter} % Joshua 13
\verseWithHeading{Land Still Remains to be Conquered}{Now Joshua was old and \textit{advanced in years},\lebnote{“he went in the days”} and Adonai said to him, “You are old and \textit{advanced in years},\lebnote{“you went in the days”} and very much of the land remains to be possessed.}%
\verse{This is the remaining land: all the regions of the Philistines, and all of the Geshurites,\lnETK{}}%
\verse{from the Shihor, which is \textit{east of Egypt},\lebnote{“on the face of Egypt”} up to the border of Ekron to the north, which is reckoned as Canaanite; there are five Philistine rulers: the Gazites,\lebnote{Hebrew “Gazite”} Ashdodites,\lebnote{Hebrew “Ashdodite”} Ashkelonites,\lebnote{Hebrew “Ashkelonite”} Gittites,\lebnote{Hebrew “Gittite”} Ekronites,\lebnote{Hebrew “Ekronite”} and the Avvim.}%
\verse{In the south; all the land of the Canaanites,\lnAMI{} and Mearah, which belongs to the Sidonians up to Aphek, to the border of the Amorites,\lnAMK{}}%
\verse{and the land of the Gebalites, and all the Lebanon,\lnETA{} \textit{toward the east},\lebnote{“the rise of the sun”} from Baal Gad \textit{at the foot of}\lebnote{Or “below” or “under”} Mount Hermon up to Lebo-Hamath;}%
\verse{all the inhabitants of the hill country, from the Lebanon\lnETA{} up to Misrephoth Maim, and all the Sidonians. I will drive them out from before the \textit{Israelites};\lnAKM{} only allocate it to Israel as an inheritance just as I have commanded you.}%
\verse{Therefore, divide this land as an inheritance to the nine tribes and the half-tribe of Manasseh.”}%
\verse{With it\lebnote{That is, the other half-tribe of Manasseh} the Reubenites,\lnDOX{} and the Gadites\lnDUO{} received their inheritance, which Moses gave them beyond the Jordan to the east, just as Moses Adonai’s servant gave to them:}%
\verse{from Aroer, which is on the edge of the wadi\lnDWS{} of Arnon, and the city which is in the middle of the wadi, and all the plateau from Medeba up to Dibon;}%
\verse{and all the cities of Sihon king of the Amorites,\lnAMK{} who reigned in Heshbon up to the border of the \textit{Ammonites};\lnAIA{}}%
\verse{and Gilead, and the border of the Geshurite\lnETK{} and the Maacathites,\lnETL{} all of \textit{Mount Hermon},\lebnote{Or “the hill country of Hermon”} and Bashan up to Salecah;}%
\verse{all the kingdom of Og in Bashan, who reigned in Ashtaroth and Edrei—he was left over from the survivors\lebnote{Hebrew “survivor”} of the Rephaim; these Moses had defeated and driven out.}%
\verse{But the \textit{Israelites}\lnAKM{} did not drive out the Geshurites\lnETK{} or the Maacathites;\lnETL{} Geshur and Maacah live among Israel to this day.}%
\verse{Only the tribe of Levites\lebnote{Hebrew “Levite”} Moses did not give an inheritance; the offerings made by fire to Adonai the God of Israel are their\lnDAT{} inheritance, just as he promised to them.\lebnote{Hebrew “he said to him”}}%
\verseWithHeading{Reuben’s Inheritance}{Moses gave an inheritance to the tribe of the descendants\lnAFK{} of Reuben according to their families.}%
\verse{Their territory was from Aroer, which was on the edge of the wadi\lnDWS{} of Arnon, and the city that is in the middle of the valley, and all the plateau by Medeba;}%
\verse{Heshbon and its cities that are on the plateau; Dibon, Bamoth Baal, Beth Baal Meon,}%
\verse{Jahaz, Kedemoth, Mephaath,}%
\verse{Kiriathaim, Sibmah, and Zereth Shahar on the hill of the valley;}%
\verse{Beth Peor, the slopes of Pisgah, and Beth Jeshimoth;}%
\verse{all of the cities of the plateau, and all the kingdom of Sihon king of the Amorites,\lnAMK{} who reigned in Heshbon and whom Moses defeated with the leaders of Midian, Evi, Rekem, Zur, Hur, and Reba, the princes of Sihon who dwelled in the land.}%
\verse{In addition to their slain, the \textit{Israelites}\lnAKM{} killed with the sword Balaam son of Beor, who practiced divination.}%
\verse{And the border of the descendants\lnAFK{} of Reuben was the Jordan and its banks.\lnEUV{} This was the inheritance of the descendants\lnAFK{} of Reuben according to their families, the cities, and their villages.}%
\verseWithHeading{Gad’s Inheritance}{Moses gave an inheritance to the tribe of Gad, to the descendants\lnAFK{} of Gad, according to their families.}%
\verse{Their territory was Jazer and all the cities of Gilead, and half the land of the \textit{Ammonites}\lnAIA{} up to Aroer, \textit{which is east of Rabbah};\lebnote{“which is before Rabbah”}}%
\verse{and from Heshbon up to Ramah-Mizpeh and Betonim, and from Mahanaim up to the territory to Debir;}%
\verse{in the valley of Beth Haram, Beth Nimrah, Succoth, Zaphon, and the rest of the kingdom of Sihon king of Heshbon, the Jordan and its banks,\lnEUV{} up to the lower end of the Kinnereth Sea\lebnote{That is, the Sea of Galilee} beyond the Jordan to the east.}%
\verse{This is the inheritance of the \textit{Gadites}\lebnote{“sons/children of Gad”} according to their families, the cities, and their villages.}%
\verseWithHeading{The Half-Tribe of Manasseh’s Inheritance}{Moses gave an inheritance to the half-tribe of Manasseh; it was for the half-tribe of the descendants\lnAFK{} of Manasseh according to their families.}%
\verse{Their territory was from Mahanaim, all Bashan, all the kingdom of Og king of Bashan, and all the settlements\lebnote{Or “tent villages”} of Jair, which are in Bashan, sixty cities,}%
\verse{and half of Gilead, with Ashtaroth, Edrei, and the cities of the kingdom of Og in Bashan; these were allotted to the children of Makir son of Manasseh, for half of the children of Makir according to their families.}%
\verse{These are the territories that Moses gave as an inheritance on the desert-plateau of Moab, beyond the Jordan, east of Jericho.}%
\verse{But to the tribe of Levi Moses did not give an inheritance; Adonai the God of Israel, he is their inheritance, just as \textit{he promised them}.\lebnote{“he said to them”}}%
\end{biblechapter}

\begin{biblechapter} % Joshua 14
\verseWithHeading{The Land Allotted West of the Jordan}{These are the territories that the \textit{Israelites}\lnAKM{} inherited in the land of Canaan, which Eleazar the priest, Joshua son of Nun, and the heads of the families of the tribes of the \textit{Israelites}\lnAKM{} gave as an inheritance to them.}%
\verse{Their inheritance was by lot, just as Adonai commanded through the hand of Moses, for the nine tribes and the half-tribe.}%
\verse{For Moses had given an inheritance of the two tribes and the half-tribe beyond the Jordan, but to the Levites he gave no inheritance among them.}%
\verse{For the descendants\lnAFK{} of Joseph were two tribes, Manasseh and Ephraim, and they did not give a plot of ground to the Levites in the land, only cities to live in, with their pastureland for their flocks and for their goods.}%
\verse{Just as Adonai commanded Moses, so the \textit{Israelites}\lnAKM{} did; and they allotted the land.}%
\verseWithHeading{Caleb Receives Hebron}{Then the descendants\lnAFK{} of Judah came to Joshua at Gilgal; and Caleb son of Jephunneh the Kenizzite said to him, “You know the word that Adonai said to Moses the man of God at Kadesh Barnea concerning you and me.}%
\verse{\textit{I was forty years old}\lebnote{“I was a son of forty years”} when Moses Adonai’s servant sent me from Kadesh Barnea to spy out the land, and I returned \textit{with an honest report}.\lebnote{“with a word just as was with my heart”}}%
\verse{My companions who went up with me made the hearts\lnEOY{} of the people melt, but I remained true to Adonai my God.}%
\verse{And Moses swore on that day, saying, ‘Surely the land that your foot has trodden on will be an inheritance to you and your sons forever, because you remained true to Adonai my God.’}%
\verse{So then, look, Adonai has kept me alive just as he promised these forty-five years,\lnDJK{} from the time that Adonai spoke this word to Moses while Israel \textit{wandered}\lebnote{“went”} in the wilderness. Now look, today \textit{I am eighty-five years old}.\lebnote{“I am a son of eighty-five years”}}%
\verse{Today I am still strong, just as on the day that Moses sent me; as my strength was then, so now also is my strength for war \textit{and for daily activities}.\lebnote{“for going out and for coming in”}}%
\verse{So now give me this hill country that Adonai spoke of on that day, for you heard on that day that the Anakites were there, with great and fortified cities. Perhaps Adonai is with me, and I will drive them out just as Adonai promised.”\lnACX{}}%
\verse{And Joshua blessed him and gave Hebron to Caleb son of Jephunneh as an inheritance.}%
\verse{Thus Hebron became the inheritance of Caleb son of Jephunneh the Kenizzite to this day, because he remained true to Adonai the God of Israel.}%
\verse{And the name of Hebron formerly was Kiriath Arba;\lebnote{Or “the city of Arba”} Arba was the greatest person among the Anakites. And the land rested from war.}%
\end{biblechapter}

\begin{biblechapter} % Joshua 15
\verseWithHeading{The Allotment of Judah}{The allotment for the tribe of the descendants\lnAFK{} of Judah according to their families reached to the border of Edom, to the wilderness of Zin, \textit{to the far south}.\lebnote{“to the south at the end of south”}}%
\verse{Their southern border was from the end of the Salt Sea,\lnDUH{} from the bay facing southward;}%
\verse{it continues\lnEVL{} to the south to the ascent of Akrabbim, passes along to Zin, it goes up south of Kadesh Barnea, passes along Hezron, goes up to Addar, and makes a turn to Karka;}%
\verse{\textit{it passes on}\lebnote{“it was to”} to Azmon, continues\lnEVL{} by the wadi of Egypt, and \textit{it ends}\lebnote{“the goings out of the border were”} at the sea. This will be your southern border.}%
\verse{The eastern border is the Salt Sea\lnDUH{} up to the mouth\lnEVO{} of the Jordan. The border on the northern side runs from the bay of the sea at the mouth\lnEVO{} of the Jordan;}%
\verse{the border goes up to Beth-hoglah and passes along north of Beth Arabah; and the border goes up the stone of Bohan son of Reuben;}%
\verse{and the border goes up to Debir from the valley of Achor, and to the north, turning to Gilgal, which is opposite the ascent of Adummim, which is south of the wadi;\lnDWS{} and the border passes on to the waters of En Shemesh, and it ends at En Rogel.}%
\verse{Then the border goes up by the Valley of Ben Hinnom\lebnote{Or “valley of the son of Hinnom”} to the slope of the Jebusites\lebnote{Hebrew “Jebusite”} from the south (that is, Jerusalem); and the border goes up to the top of the mountain that lies opposite the valley of Hinnom to the west, which is at the end of the valley of Rephaim to the north;}%
\verse{then the border turns from the top of the mountain to the spring of the waters of Nephtoah, and continues\lnEVL{} from there to the cities of Mount Ephron; the border then turns to Baalah (that is, Kiriath Jearim);}%
\verse{and the border goes around from Baalah to the west, to Mount Seir, and passes on to the slope of Mount Jearim from the north (that is, Kesalon), and goes down to Beth Shemesh, and passes along by Timnah.}%
\verse{The border continues\lebnote{“Hebrew “goes out”} to the slope of Ekron to the north, then bends around to Shikkeron, it passes on to Mount Baalah and continues to Jabneel; and \textit{the border ends}\lebnote{“the goings out of the border were”} at the sea.}%
\verse{And the western border is to the Great Sea\lnDUI{} and its coast. This is the border surrounding the descendants\lnAFK{} of Judah according to their families.}%
\verse{\textit{According to the commandment of Adonai to Joshua},\lebnote{“according to the mouth of Adonai to Joshua”} he gave to Caleb son of Jephunneh a plot of ground among the descendants\lnAFK{} of Judah, Kiriath Arba,\lebnote{Or “the city of Arba”} which is Hebron (Arba was Anak’s father).}%
\verse{Caleb drove out from there three of Anak’s sons, Sheshai, Ahiman, and Talmai, the descendants\lnCQN{} of Anak.}%
\verse{And from there he went up against the inhabitants of Debir (the former name of Debir was Kiriath Sepher).}%
\verse{And Caleb said, “Whoever attacks Kiriath Sepher and captures it, I will give to him my daughter Acsah as a wife.”}%
\verse{Othniel son of Kenaz, the brother of Caleb, captured it, and he gave to him Acsah his daughter as a wife.}%
\verse{When she came to him she urged him to ask her father for a field. So she dismounted from the donkey, and Caleb said to her, “\textit{What do you want}?”\lebnote{“What is for you?”}}%
\verse{And she said to him, “Give to me a gift;\lebnote{Or “blessing”} you have given me the land of the Negev,\lnESL{} and you must give to me a spring of water.” And he gave to her the upper and lower spring.\lebnote{Joshua 15:13–19 is almost identical to Judges 1:11–15}}%
\verseWithHeading{The Cities of Judah}{This is the inheritance of the tribe of the descendants\lnAFK{} of Judah according to their families:}%
\verse{the cities belonging to the tribe of the descendants\lnAFK{} of Judah to the far south, to the border of Edom to the south, were Kabzeel, Eder, Jagur,}%
\verse{Kinah, Dimonah, Adadah,}%
\verse{Kedesh, Hazor, Ithnan,}%
\verse{Ziph, Telem, Bealoth,}%
\verse{Hazor Hadattah, Kerioth Hezron (that is, Hazor),}%
\verse{Amam, Shema, Moladah,}%
\verse{Hazar Gaddah, Heshmon, Beth Pelet,}%
\verse{Hazar Shual, Beersheba, Biziothiah,}%
\verse{Baalah, Iim, Ezem,}%
\verse{Eltolad, Kesil, Hormah,}%
\verse{Ziklag, Madmannah, Sansannah,}%
\verse{Lebaoth, Shilhim, Ain, and Rimmon; in all, twenty-nine cities and their villages.}%
\verse{In the Shephelah:\lnERI{} Eshtaol, Zorah, Ashnah,}%
\verse{Zanoah, En Gannim, Tappuah, Enam,}%
\verse{Jarmuth, Adullam, Socoh, Azekah,}%
\verse{Shaaraim, Adithaim, Gederah, and Gederothaim; fourteen cities and their villages.}%
\verse{Zenan, Hadashah, Migdal Gad,}%
\verse{Dilean, Mizpah, Joktheel,}%
\verse{Lachish, Bozkath, Eglon,}%
\verse{Cabbon, Lahma, Kitlish,}%
\verse{Gederoth, Beth Dagon, Naamah, and Makkedah; sixteen cities and their villages.}%
\verse{Libnah, Ether, Ashan,}%
\verse{Jephthah, Ashnah, Nezib,}%
\verse{Keilah, Aczib, and Mareshah; nine cities and their villages.}%
\verse{Ekron, its towns and villages;}%
\verse{from Ekron to the sea, and all that \textit{were near}\lebnote{“were on the hand of”} Ashdod and their villages.}%
\verse{Ashdod, its towns and villages; Gaza, its towns and villages, up to the wadi\lnDWS{} of Egypt and the Great Sea\lnDUI{} and its coast.\lebnote{“border”}}%
\verse{And in the hill country: Shamir, Jattir, Socoh,}%
\verse{Dannah, Kiriath Sanna (that is, Debir),}%
\verse{Anab, Eshtemoh, Anim,}%
\verse{Goshen, Holon, and Giloh; eleven cities and their villages.}%
\verse{Arab, Dumah, Eshan,}%
\verse{Janim, Beth-tappuah, Aphekah,}%
\verse{Humtah, Kiriath Arba\lebnote{Or “the city of Arba”} (that is, Hebron), and Zior; nine cities and their villages.}%
\verse{Moan, Carmel, Ziph, Juttah,}%
\verse{Jezreel, Jokdeam, Zanoah,}%
\verse{Kain, Gibeah, and Timnah; ten cities and their villages.}%
\verse{Halhul, Beth Zur, Gedor,}%
\verse{Maarath, Beth Anoth, and Eltekon; six cities and their villages.}%
\verse{Kiriath Baal (that is, Kiriath Jearim) and Rabbah; two cities and their villages.}%
\verse{In the wilderness: Beth Arabah, Middin, Secacah,}%
\verse{Nibshan, the city of Salt, and En Gedi; six cities and their villages.}%
\verse{But the descendants\lnAFK{} of Judah were unable to drive out the Jebusites, the inhabitants of Jerusalem, so the Jebusites live with the descendants\lnAFK{} of Judah in Jerusalem to this day.}%
\end{biblechapter}

\begin{biblechapter} % Joshua 16
\verseWithHeading{The Allotment of Ephraim and Manasseh}{The allotment of the descendants\lnAFK{} of Joseph went from the Jordan by Jericho, at the waters of Jericho to the east, into the wilderness, going up from Jericho into the hill country to Bethel;}%
\verse{it continues from Bethel to Luz, and it passes along to the territory of the Arkites\lebnote{Hebrew “Arkite”} at Ataroth.}%
\verse{Then it goes down, to the west, to the territory of the Japhletites,\lebnote{Hebrew “Japhletite”} up to the territory of Lower Beth-horon, then to Gezer, and \textit{it ends}\lnEWF{} at the sea.}%
\verse{And the descendants\lnAFK{} of Joseph, Manasseh and Ephraim, received their inheritance.}%
\verse{This was the border of the descendants\lnAFK{} of Ephraim according to their families: the border of their inheritance to the east was Ataroth Addar, up to Upper Beth-horon.}%
\verse{The border continues to the sea; from Micmethath to the north, the border turns to the east to Taanath Shiloh, and it passes along it from the east to Janoah.}%
\verse{Then it goes down from Janoah to Ataroth and to Naarah; it touches Jericho and ends at the Jordan;}%
\verse{from Tappuah the border goes to the west, to the wadi\lnDWS{} of Kanah, and \textit{it ends}\lnEWF{} at the sea. This is the inheritance of the tribe of the descendants\lnAFK{} of Ephraim according to their families,}%
\verse{with the cities that were set apart for the descendants\lnAFK{} of Ephraim in the midst of the inheritance of the descendants\lnAFK{} of Manasseh, all the cities and their villages.}%
\verse{But they did not drive out the Canananites\lnAMI{} who were dwelling in Gezer, and so the Canaanites\lnAMI{} live in the midst of Ephraim to this day, but they became forced laborers.}%
\end{biblechapter}

\begin{biblechapter} % Joshua 17
\verseWithHeading{The Allotment of the Other Half-Tribe of Manasseh}{Then the allotment was made for the tribe of Manasseh, because he was the firstborn of Joseph. To Makir, the firstborn of Manasseh, the father of Gilead, \textit{were allotted}\lebnote{“and there was to him”} Gilead and Bashan, because he was a warrior.\lebnote{“a man of war”}}%
\verse{An allotment was made for the remaining descendants\lnAFK{} of Manasseh, according to their families: For the children of Abiezer, Helek, Asriel, Shechem, Hepher, and Shemida—these were the male descendants\lnAFK{} of Manasseh son of Joseph according to their families.}%
\verse{But Zelophehad son of Hepher, son of Gilead, son of Makir, son of Manasseh, had no sons, only daughters. These are the names of his daughters: Mahlah, Noah, Hoglah, Milcah, and Tirzah.}%
\verse{They came before Eleazar the priest, Joshua son of Nun, and the leaders, saying, “Adonai commanded Moses to give an inheritance to us among our kinsmen.”\lebnote{Or “among our brothers”} Therefore, according to the \textit{command of Adonai}\lnCXF{} he gave them an inheritance among the kinsmen\lnCQS{} of their father.}%
\verse{Thus ten shares fell to Manasseh, besides the land of Gilead and Bashan, which is beyond the Jordan,}%
\verse{because the daughters of Manasseh received an inheritance among his sons. And the land of Gilead was allotted to the remaining descendants\lnAFK{} of Manasseh.}%
\verse{The border of Manasseh was from Asher to Micmethath, which is opposite Shechem;\lebnote{Or “which faces Shechem”} then the border goes to the south, to the inhabitants of En Tappuah.}%
\verse{The land of Tappuah \textit{belonged to Manasseh},\lebnote{“was to Manasseh”} but Tuppuah on the border of Manasseh \textit{belonged to the descendants of Ephraim}.\lebnote{“was to the children of Ephraim”}}%
\verse{Then the border goes down to the wadi\lnDWS{} of Kanah to the south of the wadi. These cities belong to Ephraim among the cities of Manasseh. The border of Manasseh is north of the wadi, and \textit{it ends}\lnEWF{} at the sea.}%
\verse{The south is Ephraim’s, and the north is Manasseh’s; the sea is their\lnDIC{} border; Asher touches the north and on the east Issachar.}%
\verse{In Issachar and Asher, Manasseh had Beth-shean and its villages, Ibleam and its villages, the inhabitants of Dor and its villages, the inhabitants of En-dor and its villages, the inhabitants of Taanach and its villages, the inhabitants of Megiddo and its villages; the third is Napheth.}%
\verse{But the descendants\lnAFK{} of Manasseh were not able to take possession of these towns; the Canaanites\lnAMI{} were determined to live in this land.}%
\verse{And it happened, when the \textit{Israelites}\lnAKM{} grew strong, they put the Canaanites\lnAMI{} to forced labor but never drove them out completely.}%
\verseWithHeading{The Tribes of Joseph Object}{The descendants\lnAFK{} of Joseph spoke with Joshua, saying, “Why have you given us\lebnote{Hebrew “to me”} one allotment and one share as an inheritance? We are many people, which Adonai has blessed.”}%
\verse{And Joshua said to them, “If you are many people, go up to the forest and clear a place there for yourselves in the land of the Perizzites\lnAML{} and Rephaim, since the hill country of Ephraim is too narrow for you.”}%
\verse{And the descendants\lnAFK{} of Joseph said, “The hill country is not enough for us, and all of the Canaanites\lnAMI{} living in the land of the valley have chariots\lnAQY{} of iron, those in Beth-shean and its villages, and those in the Jezreel Valley.”}%
\verse{And Joshua said to the house of Joseph, to Ephraim and Manasseh, “You are many people and have great power; you will not have one allotment only;}%
\verse{the hill country will be yours. Even though it is a forest, you will clear it, and it will be yours to its farthest borders. You will drive out the Canaanites,\lnAMI{} even though they have iron chariots and are strong.”}%
\end{biblechapter}

\begin{biblechapter} % Joshua 18
\verseWithHeading{The Last of the Land Is Divided}{The entire congregation of the \textit{Israelites}\lnAKM{} assembled at Shiloh, and they set up there the tent of meeting, and the land was subdued before them.\lebnote{Or “in their presence”}}%
\verse{And seven tribes remained among the \textit{Israelites}\lnAKM{} who had not been apportioned their inheritance.}%
\verse{And Joshua said to the \textit{Israelites},\lnAKM{} “\textit{How long}\lebnote{“Until when”} will you be slack about going to take possession of the land that Adonai, the God of your ancestors,\lnAGP{} has given you?}%
\verse{Provide three men \textit{from each tribe},\lebnote{“according to each tribe”} and I will send them so that they may begin to go through the land and write a description of it \textit{according to their inheritance},\lebnote{“according to the mouth of their inheritance”} and let them come to me.}%
\verse{They will divide it among themselves into seven portions; \textit{Judah will maintain its border}\lebnote{“Judah will stand on its border”} from the south, and \textit{the house of Joseph will maintain its border}\lebnote{“the house of Joseph will stand on its border”} from the north.}%
\verse{Describe the land in seven divisions, and bring it to me here; I will cast lots for you here before Adonai our God.}%
\verse{The Levites among you have no portion, for their inheritance is the priesthood of Adonai; Gad, Reuben, and the half-tribe of Manasseh received their inheritance beyond the Jordan to the east, which Moses Adonai’s servant gave to them.”}%
\verse{And \textit{the men went immediately},\lebnote{“the men got up and went”} and Joshua commanded the ones going to describe the land, saying, “Go and walk about through the land, write a description, and return to me, and here I will cast a lot for you before\lnEXL{} Adonai at Shiloh.”}%
\verse{And the men went and passed through the land, and they \textit{described}\lebnote{“wrote it”} the cities in seven divisions in a book; and they came to Joshua to the camp at Shiloh,}%
\verse{and Joshua cast a lot for them at Shiloh before\lnEXL{} Adonai, and there he divided the land for the \textit{Israelites},\lnAKM{} \textit{to each a portion}.\lebnote{“according to their allotment”}}%
\verseWithHeading{The Allotment of Benjamin}{And the allotment of the tribe of Benjamin came up according to their families, and the border of their allotment fell\lebnote{“went out”} between the descendants\lnAFK{} of Judah and the descendants\lnAFK{} of Joseph.}%
\verse{Their northern border began at the Jordan and went up to the slope of Jericho on the north and continued into the hill country to the west; \textit{it ends}\lnEWF{} at the wilderness of Beth Aven.}%
\verse{The border passes on from there to Luz, to the slope of Luz to the south (that is, Bethel); then the border goes down to Ataroth Addar to the mountain that is south of Lower Beth-Horon.}%
\verse{Then the border changes direction and turns to the western side southward, from the mountain that \textit{is opposite}\lebnote{“against the face of”} Beth-Horon to the south. \textit{It ends}\lnEWF{} at Kiriath Baal (that is, Kiriath Jearim), a town belonging to the descendants\lnAFK{} of Judah. This is the western side.}%
\verse{The southern side begins on the outskirts of Kiriath Jearim, and the border continues to the west to the spring of the waters of Nephtoah;}%
\verse{the border goes down to the foot\lebnote{Or “to the edge”} of the mountain, which is opposite the Valley of Ben Hinnom,\lebnote{Or “valley of the son of-Hinnom”} which is in the valley of Rephaim to the north; then it does down the valley of Hinnom to the slope of the Jebusites\lebnote{Hebrew “Jebusite”} to the south, and then it goes down to En Rogel.}%
\verse{It changes direction from the north, and it continues to En Shemesh; it goes out to Geliloth, which is opposite the ascent of Adummim, and it goes down to the stone of Bohan, son of Reuben.}%
\verse{It passes on to the slope opposite the Arabah\lnERB{} to the north, and it goes down to the Arabah.\lnERB{}}%
\verse{The border passes on to the slope of Beth-hoglah to the north and \textit{ends}\lnEWF{} at the north bay of the Salt Sea\lnDUH{} at the south end of the Jordan. This is the southern border.}%
\verse{The Jordan forms its border on the eastern side. This is the inheritance of the tribe of Benjamin, its borders that surrounds them, according to their families.}%
\verse{Now the towns of the tribes of the descendants\lnAFK{} of Benjamin, according to their families, were Jericho, Beth-hoglah, Emek Keziz,}%
\verse{Beth Arabah, Zemaraim, Bethel,}%
\verse{Avvim, Parah, Ophrah,}%
\verse{Kephar Ammoni, Ophni, and Geba; twelve cities and their villages.}%
\verse{Gibeon, Ramah, Beeroth,}%
\verse{Mizpeh, Kephirah, Mozah,}%
\verse{Rekem, Irpeel, Taralah,}%
\verse{Zela, Haeleph, Jebus (that is, Jerusalem), Gibeah, and Kiriath; fourteen cities and their villages. This is the inheritance of the descendants\lnAFK{} of Benjamin according to their families.}%
\end{biblechapter}

\begin{biblechapter} % Joshua 19
\verseWithHeading{The Allotment of Simeon}{The second allotment \textit{fell}\lebnote{“went out”} for Simeon, for the tribe of the descendants\lnAFK{} of Simeon, according to their families. And their inheritance was in the midst of the inheritance of the descendants\lnAFK{} of Judah.}%
\verse{And they had as their inheritance Beersheba, Sheba, Moladah,}%
\verse{Hazar Shual, Balah, Ezem,}%
\verse{Eltolad, Bethul, Hormah,}%
\verse{Ziklag, Beth Marcaboth, Hazar Susah,}%
\verse{Beth Lebaoth, and Sharuhen; thirteen cities and their villages.}%
\verse{Ain, Rimmon, Ether, and Ashan; four cities and their villages,}%
\verse{and all the villages that were around these towns up to Baalat-Beor, Ramath of the Negev.\lnESL{} This was the inheritance of the tribe of the descendants\lnAFK{} of Simeon according to their families.}%
\verse{Part of the portion allotted to the descendants\lnAFK{} of Judah became the inheritance of the descendants\lnAFK{} of Simeon because the portion for the descendants\lnAFK{} of Judah was \textit{too large for them},\lebnote{“large from them”} so the descendants\lnAFK{} of Simeon inherited from their inheritance.}%
\verseWithHeading{The Allotment of Zebulun}{The third allotment came up for the descendants\lnAFK{} of Zebulun according to their families. The border of their inheritance went up to Sarid.}%
\verse{Their border goes up to the west, to Maralah; it touches\lebnote{Or “reaches to”} Dabbesheth, then the wadi\lnDWS{} that is opposite Jokneam.}%
\verse{It turns from Sarid to the east to the sunrise, to the border of Kislot-Tabor; it continues to Daberath and goes up to Japhia.}%
\verse{From there it passes along to the east toward the sunrise, to Gath Hepher and to Eth Kazin, and continuing to Rimmon, it turns to Neah;}%
\verse{it changes direction from the north of Hannathon, and \textit{it ends}\lebnote{“the goings out of it were” } at the valley of Yiptah-El;}%
\verse{Kattath, Nahalal, Shimron, Idalah, and Bethlehem; twelve cities and their villages.}%
\verse{This is the inheritance of the descendants\lnAFK{} of Zebulun according to their families, these cities and their villages.}%
\verseWithHeading{The Allotment of Issachar}{The fourth allotment \textit{fell}\lnEYK{} for Issachar, for the descendants\lnAFK{} of Issachar, according to their families.}%
\verse{Their border went to Jezreel, Chesulloth, Shunem,}%
\verse{Hapharaim, Shion, Anaharath,}%
\verse{Rabbith, Kishion, Ebez,}%
\verse{Remeth, En Gannim, En Haddah, and Beth Pazzez;}%
\verse{and the border touches Tabor, Shahazumah, and Beth Shemesh. \textit{Its border ends}\lebnote{“the goings out of their border were”} at the Jordan; sixteen cities and their villages.}%
\verse{This is the inheritance of the tribe of the descendants\lnAFK{} of Issachar according to their families, the cities and their villages.}%
\verseWithHeading{The Allotment of Asher}{The fifth allotment \textit{fell}\lebnote{“came out” } for the tribe of the descendants\lnAFK{} of Asher according to their families.}%
\verse{Their border was Helkath, Hali, Beten, Acshaph,}%
\verse{Allamelech, Amad, and Mishal; it touches Carmel to the west, and Shihor-Libnat.}%
\verse{Then it turns \textit{eastward}\lebnote{“to the rising of the sun”} to Beth-dagon and touches Zebulun and the valley of Yiptah-El to the north\lebnote{“left hand/side” } to Beth Emeck and Neiel; it continues to Cabul from the north,}%
\verse{and Ebron, Rehob, Hammon, and Kanah up to Great Sidon;}%
\verse{then the border turns to Ramah, and up to the fortified city of Tyre, where the border turns to Hosah; \textit{it ends}\lnEWF{} at the sea, from Hebel to Aczib.}%
\verse{Included were Ummah, Aphek, and Rehob; twenty-two cities and their villages.}%
\verse{This is the inheritance of the tribe of the descendants\lnAFK{} of Asher according to their families, these cities and their villages.}%
\verseWithHeading{The Allotment of Naphtali}{The sixth allotment \textit{fell}\lnEYK{} for the children of Naphtali, for the children of Naphtali according to their families.}%
\verse{Their border was from Heleph, from the oak in Zaanannim, Adami Nekeb, Jabneel, up to Lakkum; \textit{it ends}\lnEWF{} at the Jordan;}%
\verse{then the border turns to the west, to Aznoth Tabor, and continues from there to Hukok, and it touches\lebnote{Or “reaches to”} Zebulun on the south, Asher on the west, and Judah on \textit{the east}\lebnote{“the rising of the sun”} at the Jordan.}%
\verse{\textit{The fortified cities}\lebnote{“the cities of fortification”} are Ziddim, Zer, Hammath, Rakkath, Kinnereth,}%
\verse{Adamah, Ramah, Hazor,}%
\verse{Kedesh, Edrei, En Hazor,}%
\verse{Yiron, Migdal El, Horem, Beth-anath, Beth Shemesh; nineteen cities and their villages.}%
\verse{This is the inheritance of the tribe of the children of Naphtali according to their families, the cities and their villages.}%
\verseWithHeading{The Allotment of Dan}{The seventh lot \textit{fell}\lnEYK{} for the tribe of the descendants\lnAFK{} of Dan according to their families.}%
\verse{The border of their inheritance was Zorah, Eshtaol, Ir Shemesh,}%
\verse{Shaalabbin, Aijalon, Ithlah,}%
\verse{Elon, Timnah, Ekron,}%
\verse{Eltekeh, Gibbethon, Baalath,}%
\verse{Jehud, Bene Berak, Gath Rimmon,}%
\verse{Me Jarkon, Rakkon, with the border opposite Joppa.}%
\verse{The border of the descendants\lnAFK{} of Dan continued \textit{beyond them},\lebnote{“from them”} because the descendants\lnAFK{} of Dan went up and fought with Lesham, and they captured and struck it with \textit{the edge of the sword},\lnERD{} and they took possession of it and settled in it; and they called Leshem Dan, after the name of Dan their ancestor.\lnEJC{}}%
\verse{This is the inheritance of the tribe of the descendants\lnAFK{} of Dan according to their families, these cities and their villages.}%
\verseWithHeading{The Allotment Is Completed}{They finished assigning the land according to its borders, and the \textit{Israelites}\lnAKM{} gave an inheritance from among them to Joshua son of Nun.}%
\verse{\textit{According to the commandment of Adonai},\lebnote{“On the mouth of Adonai”} they gave him the city that he requested, Timnath Serah, in the hill country of Ephraim, and he rebuilt the city and settled\lebnote{Or “dwelt”} in it.}%
\verse{These are the inheritances that Eleazar the priest, Joshua son of Nun, and the heads of the families of the tribes, distributed by allotment to the \textit{Israelites},\lnAKM{} at Shiloh \textit{before Adonai}\lebnote{Or “in the presence of Adonai”} at the entrance of the tent of meeting. And they finished dividing the land.}%
\end{biblechapter}

\begin{biblechapter} % Joshua 20
\verseWithHeading{Cities of Refuge Are Established}{And Adonai spoke to Joshua, saying,}%
\verse{“Speak to the \textit{Israelites},\lnAKM{} saying, ‘Appoint for yourselves cities of refuge, of which I spoke to you through the hand of Moses.}%
\verse{Anyone who kills a person by accident or unintentionally\lebnote{Or “by not knowing”} may flee there; they will be for yourselves a refuge from the avenger of blood.}%
\verse{The killer will flee to one of these cities, stand at the entrance of the gate of the city, and \textit{state his case to the elders of that city};\lebnote{“he will speak his words in the ears of the elders of that city”} and they will take him\lebnote{Or “they will gather him”} into the city and give him a place, and he will dwell among them.}%
\verse{And if the avenger of blood pursues after him, they will not hand over the killer into his hand, because he killed his neighbor unintentionally, and \textit{he did not hate him previously}.\lebnote{“he did not hate him since yesterday and the day before that”}}%
\verse{The killer will stay in that city until he stands before the congregation for the trial, until the death of the one who is the high priest in those days. Then \textit{the killer will return}\lebnote{“the killer will return and go”} to his city and to his house, to the city from which he fled.’ ”}%
\verse{So \textit{they set apart}\lebnote{Or “consecrated”} Kedesh in Galilee in the hill country of Naphtali, Shechem in the hill country of Ephraim, and Kiriath Arba\lebnote{Or “the city of Arba”} (that is, Hebron) in the hill country of Judah.}%
\verse{Beyond the Jordan east of Jericho, they appointed Bezer in the wilderness on the plateau, from the tribe of Reuben, Ramoth in Gilead, from the tribe of Gad, and Golan in the Bashan, from the tribe of Manasseh.}%
\verse{These were the cities designated for all the \textit{Israelites},\lnAKM{} and for the foreigners\lebnote{Hebrew “foreigner”} dwelling among them, for anyone that kills a person unintentionally to flee there, and not die by the hand of the avenger of blood, \textit{until there is a trial}\lebnote{“until he stands”} before the congregation.}%
\end{biblechapter}

\begin{biblechapter} % Joshua 21
\verseWithHeading{The Allotment of the Levites}{Then the heads of the families of the Levites came to Eleazar the priest, to Joshua son of Nun, and to the heads of the families of the tribes of the \textit{Israelites}.\lnAKM{}}%
\verse{And they spoke to them at Shiloh in the land of Canaan, saying, “Adonai commanded through the hand of Moses to give us cities to live in, with their pasturelands for our livestock.”}%
\verse{So, \textit{by command of Adonai},\lebnote{“by the mouth of Adonai”} the \textit{Israelites}\lnAKM{} gave the Levites these cities and their pasturelands from their inheritance.}%
\verse{The allotment \textit{fell}\lnEYK{} for the families of the Kohathites.\lnCXI{} The descendants\lnAFK{} of Aaron the priest, who were of the Levites, \textit{received}\lebnote{“they had”} by lot thirteen towns from the tribes of Judah, Simeon, and Benjamin.}%
\verse{The remaining descendants\lnAFK{} of Kohath received by lot ten cities from the families of the tribes of Ephraim, Dan, and the half-tribe of Manasseh.}%
\verse{The descendants\lnAFK{} of Gershon received by lot thirteen cities from the families of the tribes of Issachar, Asher, and Naphtali and from the half-tribe of Manasseh in Bashan.}%
\verse{The descendants\lnAFK{} of the Merarites\lebnote{Hebrew “Merarite”} according to their families received twelve cities from the tribes of Reuben, Gad, and Zebulun.}%
\verse{The \textit{Israelites}\lnAKM{} gave to the Levites these cities and their pastureland by lot, just as Adonai commanded through the hand of Moses.}%
\verse{They gave these cities, which are here mentioned by name, from the tribe of the families of Judah and from the tribe of the families of Simeon;}%
\verse{and they were for the descendants\lnAFK{} of Aaron, from the families of the Kohathites,\lnCXI{} from the descendants\lnAFK{} of Levi, because the first lot was theirs.}%
\verse{And they gave to them Kiriath Arba,\lebnote{Or “the city of Arba”} Arba being the father of Anak (that is, Hebron), in the hill country of Judah and the pasturelands surrounding it.}%
\verse{But the field of the city and its villages they gave to Caleb son of Jephunneh as his property.}%
\verse{To the descendants\lnAFK{} of Aaron the priest they gave Hebron, the city of refuge for the killer, and its pasturelands, Libnah and its pasturelands,}%
\verse{Jattir and its pasturelands, Eshtemoa and its pasturelands,}%
\verse{Holon and its pasturelands, Debir and its pasturelands,}%
\verse{Ain and its pasturelands, Juttah and its pasturelands, and Beth Shemesh and its pasturelands; nine cities from these two tribes.}%
\verse{From the tribe of Benjamin, Gibeon and its pasturelands, Geba and its pasturelands,}%
\verse{Anathoth and its pasturelands, Almon and its pasturelands; four cities.}%
\verse{All the cities of the descendants\lnAFK{} of Aaron the priests, thirteen cities and their pasturelands.}%
\verse{For the families of the descendants\lnAFK{} of Kohath, the remaining Levites of the descendants\lnAFK{} of Kohath, they received the cities of their lot from the tribe of Ephraim.}%
\verse{They gave them Shechem, the city of refuge for the killer, and its pasturelands in the hill country of Ephraim, Gezer and its pasturelands,}%
\verse{Kibzaim and its pasturelands, and Beth-horon and its pasturelands; four cities.}%
\verse{From the tribe of Dan, Eltekeh and its pasturelands, Gibbethon and its pasturelands,}%
\verse{Aijalon and its pasturelands, and Gath Rimmon and its pasturelands; four cities.}%
\verse{From the half-tribe of Manasseh, Taanach and its pasturelands and Gath Rimmon with its pasturelands; two cities.}%
\verse{All the cities and their pasturelands for the remaining families of the descendants\lnAFK{} of Kohath were ten.}%
\verse{To the descendants\lnAFK{} of Gershon, one of the families of the Levites, from the half-tribe of Manasseh, Golan in Bashan, a city of refuge for the killer, and its pasturelands, and Eshtarah and its pasturelands; two cities.}%
\verse{From the tribe of Issachar, Kishion and its pasturelands, Daberath and its pasturelands,}%
\verse{Jarmuth and its pasturelands, En Gannim and its pasturelands; four cities.}%
\verse{From the tribe of Asher, Mishal and its pasturelands, Abdon and its pasturelands,}%
\verse{Helkath and its pasturelands, Rehob and its pasturelands; four cities.}%
\verse{From the tribe of Naphtali, Kedesh in Galilee, the city of refuge for the killer, and its pasturelands, Hammoth Dor and its pasturelands, and Kartan and its pasturelands; three cities.}%
\verse{All the cities of the Gershonites\lnCYX{} according to their families were thirteen cities and their pasturelands.}%
\verse{To the families of the descendants\lnAFK{} of Merarite, the remaining Levites, from the tribe of Zebulun, Jokneam and its pasturelands, Kartah and its pasturelands,}%
\verse{Dimnah and its pasturelands, and Nahalal and its pasturelands; four cities.}%
\verse{From the tribe of Reuben, Bezer and its pasturelands, Jahaz and its pasturelands,}%
\verse{Kedemoth and its pasturelands, and Mephaath and its pasturelands; four cities.}%
\verse{From the tribe of Gad, Ramoth in Gilead, the city of refuge for the killer, and its pasturelands, Mahanaim and its pasturelands,}%
\verse{Heshbon and its pasturelands, and Jazer and its pasturelands; four cities in all.}%
\verse{All these were the cities of the descendants\lnAFK{} of Merarite according to their families, the remaining families of the Levites; their allotment was twelve cities.}%
\verse{All the cities of the Levites among the property of the \textit{Israelites}\lnAKM{} were forty-eight cities and their pasturelands.}%
\verse{Each of these cities had pasturelands surrounding them; so it was for all of these cities.}%
\verse{And Adonai gave to Israel all the land that he swore to give to their ancestors,\lnAGP{} and they took possession of it and \textit{settled in it}.\lebnote{Or “dwelled in it”}}%
\verse{Adonai gave them rest on every side, according to all that he had sworn to their ancestors,\lnAGP{} and nobody from all their enemies withstood them, for Adonai had given all their enemies into their hand.}%
\verse{And \textit{nothing failed from}\lebnote{“Not a thing fell”} all the good things\lebnote{Hebrew “thing”} that Adonai promised to the house of Israel; \textit{everything came to pass}.\lebnote{“everything it came”}}%
\end{biblechapter}

\begin{biblechapter} % Joshua 22
\verseWithHeading{The Eastern Tribes Return}{Then Joshua summoned the Reubenites, the Gadites, and the half-tribe of Manasseh,}%
\verse{and he said to them, “You have observed all that Moses Adonai’s servant commanded you, and you have listened\lebnote{Or “you have obeyed”} to my voice in all that I have commanded you;}%
\verse{you have not forsaken your kinsmen\lnCQS{} these many days, up to this day, and you have observed the obligation of the command of Adonai your God.}%
\verse{So then, Adonai your God has given rest to your kinsmen,\lnCQS{} just as he promised them; so then, turn and go to your tents to the land of your possession, which Moses Adonai’s servant gave to you beyond the Jordan.}%
\verse{Only be very careful to observe the commandment and law that Moses Adonai’s servant commanded you, to love Adonai your God, to walk in all his ways, to keep his commandments, to hold fast to him, and to serve him with all your heart and with all your soul.”\lebnote{Or “inner self”}}%
\verse{And Joshua blessed them and sent them away, and they went to their tents.}%
\verse{And to the half-tribe of Manasseh Moses had given a possession in Bashan, but to the other half Joshua had given a possession with their kinsmen\lnCQS{} beyond the Jordan to the west; and when Joshua sent them to their tents and blessed them,}%
\verse{he said to them, “Return to your tents with much wealth, and with very much livestock, with silver, gold, copper, iron, and with very much clothing; divide the war-booty of your enemies with your kinsmen.”\lnCQS{}}%
\verse{So the descendants\lnAFK{} of Reuben, Gad, and the half-tribe of Manasseh returned home and departed with the \textit{Israelites}\lnAKM{} at Shiloh, which is in the land of Canaan, to go to the land of Gilead to the land of their possession, which they had acquired \textit{according to the command of Adonai}\lebnote{“on the mouth of Adonai”} through the hand of Moses.}%
\verse{And they came to the region of the Jordan that is in the land of Canaan, and the descendants\lnAFK{} of Reuben, Gad, and the half-tribe of Manasseh built there an altar on the Jordan, \textit{a large and imposing altar}.\lebnote{“a large altar to appearance”}}%
\verse{And the \textit{Israelites}\lnAKM{} heard it said that the descendants\lnAFK{} of Reuben, Gad, and the half-tribe of Manasseh had built an altar next to the land of Canaan, in the region of the Jordan, on the side belonging to the \textit{Israelites}.\lnAKM{}}%
\verse{When the \textit{Israelites}\lnAKM{} heard of it, the whole congregation of the \textit{Israelites}\lnAKM{} gathered at Shiloh to go up against them for battle.}%
\verse{And the \textit{Israelites}\lnAKM{} sent to the descendants\lnAFK{} of Reuben, Gad, and the half-tribe of Manasseh, to the land of Gilead, Phinehas the priest son of Eleazar,}%
\verse{and ten leaders with him, \textit{one leader for each}\lebnote{“one leader, one leader”} \textit{family}\lebnote{“of the house of father”} from each of the tribes of Israel; and each one was the head of \textit{his family}\lebnote{“house of their fathers”} among the clans of Israel.\lebnote{Or “thousands of Israel”}}%
\verse{They came to the descendants\lnAFK{} of Reuben, Gad, and the half-tribe of Manasseh, to the land of Gilead, and they spoke with them, saying,}%
\verse{“Thus says all the congregation of Adonai: ‘What is this treachery that you have committed against the God of Israel by turning away today from following Adonai, by building for yourselves an altar to rebel today against Adonai?}%
\verse{Is not the sin of Peor \textit{enough for us},\lebnote{“too little for us”} from which we have not cleansed ourselves today, and for which a plague came to the congregation of Adonai,}%
\verse{that you must turn today from following Adonai? If you rebel today against Adonai, tomorrow he will be angry with all of the congregation of Israel;}%
\verse{if, however, the land of your property is unclean, cross over to the land of Adonai’s property, where Adonai’s tabernacle resides,\lebnote{Or “stands”} and take possession among us. But you must not rebel against Adonai or against us by building for yourselves an altar other than the altar of Adonai our God.}%
\verse{Did not Achan son of Zerah commit treachery with devoted things,\lebnote{Or “consecrated possession”} and wrath fell on all the congregation of Israel? \textit{And he alone}\lebnote{“And he is one man that”} did not perish because of his iniquity.’ ”}%
\verse{And the descendants\lnAFK{} of Reuben, Gad, and the half-tribe of Manasseh spoke with the heads of the clans\lebnote{Or “thousands”} of Israel,}%
\verse{“Adonai, God of gods! Adonai, God of gods knows. And let Israel itself know, if it was in rebellion or treachery against Adonai, do not spare us this day}%
\verse{for building for ourselves an altar to turn away from Adonai, or if it was to offer burnt offerings,\lnFAR{} grain offerings,\lnFAR{} or fellowship offerings on it, may Adonai himself take vengeance.}%
\verse{But in fact, we have done this because of anxiety, because of a reason, saying, ‘In the future your children may say to our children, ‘\textit{What is the relationship between you and Adonai the God of Israel}?\lebnote{“What is to you and to Adonai the God of Israel”}}%
\verse{Adonai has made the Jordan a border between us and you, the descendants\lnAFK{} of Reuben and Gad; you have no portion in Adonai.’ So your children may put an end to our children worshiping\lebnote{Or “seeing”} Adonai.}%
\verse{So we said, ‘Let us build immediately for ourselves an altar, not for burnt offerings\lnFAR{} or for sacrifices;\lnFAV{}}%
\verse{instead, it is a witness between us and you, and between our generations after us for performing the serving of Adonai in his presence with our burnt offerings, sacrifices, and fellowship offerings; so that your children may not say in the future to our children, “You have no portion in Adonai.” ’}%
\verse{And we thought, if they say to us and to our children in the future, we can say, ‘Look at this replica of the altar of Adonai, which our ancestors\lnAGP{} made, not for burnt offerings\lnFAR{} or sacrifices;\lnFAV{} rather, it is a witness between us and you.’}%
\verse{Far be it from us to rebel against Adonai, to turn today from following Adonai, to build an altar for burnt offerings,\lnFAR{} grain offerings,\lnFAR{} or sacrifices,\lnFAV{} instead of the altar of Adonai our God that is before his tabernacle.”}%
\verse{Phinehas the priest, the leaders of the congregation, and the heads of the clans\lebnote{Or “thousands”} of Israel who were with him heard the words that the descendants\lnAFK{} of Reuben, Gad, and Manasseh spoke, and \textit{they were satisfied}.\lebnote{“it was good in their eyes”}}%
\verse{Phinehas the priest, son of Eleazar, said to the descendants\lnAFK{} of Reuben, Gad, and Manasseh, “Today we know that Adonai is among us, because you have not committed this treachery against Adonai. Therefore you have rescued the \textit{Israelites}\lnAKM{} from the hand of Adonai.”}%
\verse{And Phinehas the priest, son of Eleazar, and the leaders returned from the descendants\lnAFK{} of Reuben and Gad, from the land of Gilead, to the land of Canaan to the \textit{Israelites},\lnAKM{} and \textit{they gave them their report}.\lebnote{“they brought back a word to them”}}%
\verse{\textit{The report satisfied the Israelites};\lebnote{“The report was good in the eyes of the children of Israel”} they blessed God, and they did not speak of going up for battle against them to destroy the land in which the descendants\lnAFK{} of Reuben and Gad were living.\lebnote{Or “dwelling”}}%
\verse{The descendants\lnAFK{} of Reuben and Gad called the altar Witness, “Because,” they said, “it is a witness between us that Adonai is God.”}%
\end{biblechapter}

\begin{biblechapter} % Joshua 23
\verseWithHeading{Joshua’s Farewell Address}{And it happened, after \textit{a long time},\lebnote{“many days”} after Adonai had given rest to Israel from all their surrounding enemies, and after Joshua was old and \textit{well-advanced in years},\lebnote{“he went into the days”}}%
\verse{Joshua summoned all Israel, their elders, heads, judges, and officials, and he said to them, “I am old and \textit{well-advanced in years},\lebnote{“I went into the days”}}%
\verse{and you have seen all that Adonai your God has done to all these nations \textit{for your sake},\lebnote{“because of your presence”} for Adonai your God is fighting for you.}%
\verse{Look! I have allotted to you these remaining nations as an inheritance for your tribes, from the Jordan, with all the nations that I have cut off, to the Great Sea\lnDUI{} \textit{in the west}.\lebnote{“to the setting of the sun”}}%
\verse{And Adonai your God will push them \textit{back before you}\lebnote{“from your presence”} and drive them \textit{out of your sight},\lebnote{“from your face”} and you will possess their land, just as Adonai your God promised to you.}%
\verse{Be very strong to observe carefully all that is written in the scroll of the law of Moses so as not to turn aside from it, to the right or left,}%
\verse{so as not to go among these remaining nations with you; \textit{do not profess}\lebnote{“do not mention”} the name of their gods, and do not swear by them, serve them, or \textit{bow down to them}.\lebnote{“bow yourselves down to them”}}%
\verse{But hold fast to Adonai your God, just as you have done up to this day.}%
\verse{Adonai has driven out before you great and strong nations; and as for you, \textit{nobody has withstood}\lebnote{“a man has not stood in your presence”} you to this day.}%
\verse{One of your men put to flight a thousand, for Adonai your God is fighting for you, just as he promised you.}%
\verse{Take utmost care for the sake of your life to love Adonai your God,}%
\verse{for if indeed you turn back and join these remaining nations \textit{among you},\lebnote{“with you”} and you intermarry with them, \textit{marrying their women and they yours},\lebnote{“you go into them and they into you”}}%
\verse{know for certain that Adonai your God \textit{will not continue to drive out}\lebnote{“will not drive out again”} these nations from before you; they will be for you a snare and a trap, a whip on your sides and thorns in your eyes, until you perish from this good land that Adonai your God has given to you.}%
\verse{Look! \textit{I am about to die},\lebnote{“I am going today on the way of all the earth”} and you know in all your hearts\lnEOY{} and souls\lebnote{Hebrew “soul”; or “inner self”} that not one thing \textit{failed}\lebnote{“fell”} from all the good things that Adonai your God promised concerning you; everything \textit{has been fulfilled};\lebnote{“has come out”} \textit{not one thing failed}.\lebnote{“not one thing fell from it”}}%
\verse{But just as all the good things\lebnote{Hebrew “thing”} came to you that Adonai your God promised, so will Adonai bring to you all the bad things\lebnote{Hebrew “things”} until he has destroyed you from this good land that Adonai your God has given to you.}%
\verse{If you transgress the covenant of Adonai your God, which he commanded to you, and you go and serve other gods and bow down to them, \textit{Adonai’s anger will be kindled}\lebnote{“Adonai’s nose will become hot”} against you, and you will perish quickly from the good land that he has given to you.”}%
\end{biblechapter}

\begin{biblechapter} % Joshua 24
\verseWithHeading{Joshua Recounts Their History}{And Joshua gathered all the tribes of Israel to Shechem; he summoned the elders of Israel, their heads, their judges, and their officials, and they presented themselves before God.}%
\verse{And Joshua said to all the people, “Thus says Adonai the God of Israel: ‘\textit{Long ago}\lebnote{“from ancient”} your ancestors\lnAGP{}—Terah the father of Abraham and the father of Nahor—lived beyond the river,\lnDNH{} and they served other gods.}%
\verse{I took your ancestor\lnEJC{} Abraham from beyond the river\lnDNH{} and led him through all the land of Canaan, and I increased his offspring; I gave him Isaac,}%
\verse{and to Isaac I gave Jacob and Esau. To Esau I gave the hill country of Seir to possess, but Jacob and his children went down to Egypt.}%
\verse{And I sent Moses and Aaron, and I plagued Egypt with what I did in its midst; and afterward I brought you out.}%
\verse{When I brought out your ancestors\lnAGP{} from Egypt, you came to the sea, and the Egyptians pursued after your ancestors\lnAGP{} with chariots\lnAQY{} and horsemen to the \textit{Red Sea}.\lnAOZ{}}%
\verse{They cried out to Adonai, and he put darkness between you and the Egyptians, and he brought the sea over them\lebnote{That is, the Egyptians} and covered them; your own eyes saw what I did in Egypt. Then you lived in the wilderness for many days.}%
\verse{And I brought you to the land of the Amorites\lnAMK{} who lived beyond the Jordan; they fought you, and I gave them into your hand; you took possession of their land, and I destroyed them \textit{before you}.\lebnote{“in your presence”}}%
\verse{Then Balak son of Zippor, king of Moab, set out and fought against Israel, and he sent and summoned Balaam son of Beor to curse you,}%
\verse{but I was not willing to listen to Balaam, and he richly blessed you. So I rescued you from his hand,}%
\verse{and you crossed the Jordan and came to Jericho. And the citizens of Jericho, the Amorites,\lnAMK{} the Perizzites,\lnAML{} the Canaanites,\lnAMI{} the Hittites,\lnAMJ{} the Girgashites,\lebnote{Hebrew “Girgashite”} the Hivites,\lnAMM{} and the Jebusites,\lebnote{Hebrew “Jebusite”} fought against you, and I gave them into your hand.}%
\verse{I sent before you the hornet and they drove out before you two kings of the Amorites;\lnAMK{} but not by your sword or bow.}%
\verse{I gave to you a land that you have not labored on, and cities that you have not built, and you live\lnFCA{} in them; you eat from vineyards and olive groves that you have not planted.’}%
\verseWithHeading{The Israelites Promise to Serve Adonai}{“So now, revere Adonai and serve him in sincerity and faithfulness; remove the gods that your ancestors\lnAGP{} served beyond the river\lnDNH{} and in Egypt, and serve Adonai.}%
\verse{But if it is bad in your eyes to serve Adonai, choose for yourselves today whom you want to serve, whether it is the gods that your ancestors\lnAGP{} served beyond the river,\lnDNH{} or the gods of the Amorites\lnAMK{} in whose land you are living; but as for me and my household, we will serve Adonai.”}%
\verse{And the people answered and said, “Far be it from us that we would forsake Adonai to serve other gods,}%
\verse{for Adonai our God brought us and our ancestors\lnAGP{} from the land of Egypt, from the house of slavery, and did these great signs before our eyes. He protected us along the entire way that we went, and among all the peoples through whose midst we passed.}%
\verse{And Adonai drove out all the people before us, the Amorites\lebnote{“Hebrew “Amorite”} who live\lnFCA{} in the land. We will serve Adonai, for he is our God.”}%
\verse{But Joshua said to the people, “You cannot serve Adonai, for he is a holy and jealous God; he will not forgive your transgressions or your sins.}%
\verse{If you forsake Adonai and serve foreign gods, he will turn and bring disaster to you; he will destroy you after he has done good to you.”}%
\verse{And the people said to Joshua, “No, we will serve Adonai.”}%
\verse{And Joshua said to the people, “You are witnesses against yourselves that you have chosen for yourselves to serve Adonai.” And they said, “We are witnesses.”}%
\verse{He said, “Remove the foreign gods that are in your midst, and incline your hearts to Adonai the God of Israel.”}%
\verse{And the people said to Joshua, “We will serve Adonai our God, and we will listen to his voice.”}%
\verse{So Joshua \textit{made a covenant}\lebnote{“cut a covenant”} with the people on that day, and he established for them a statute and a judgment at Shechem.}%
\verse{Then Joshua wrote these words in a scroll of the law of God, and he took a large stone and set it up there under a large tree, which is at the shrine of Adonai.}%
\verse{And Joshua said to all the people, “Look, this stone will be a witness against us, for it has heard all the words of Adonai that he spoke with us. It will be as a witness against you, so that you do not deny your God.”}%
\verse{Then Joshua sent the people away to their inheritance.}%
\verse{After these things Joshua son of Nun servant of Adonai died; \textit{he was one hundred and ten years old}.\lebnote{“a son of one hundred and ten years”}}%
\verse{They buried him in the territory of his inheritance, at Timnath-Serah, which is in the hill country of Ephraim, north of Mount Gaash.}%
\verse{Israel served Adonai all the days of Joshua, and all the days of the elders who lived long after Joshua, and who had known all the work that Adonai did for Israel.\lebnote{Joshua 24:28–31 is repeated in Judges 2:6–10}}%
\verse{The bones of Jacob, which the \textit{Israelites}\lnAKM{} had brought out from Egypt, they buried at Shechem, in a piece of land that Jacob had bought from the children of Hamor, the father of Shechem, for one hundred pieces of money;\lebnote{Hebrew \textit{kesitah}} it became an inheritance for the descendants\lnAFK{} of Joseph.}%
\verse{And Eleazar son of Aaron died; and they buried him in Gibeah in the hill country of Ephraim, which had been given to his son Phinehas.}%
\end{biblechapter}

\flushcolsend
\biblebook{Judges}

\begin{biblechapter} % Judges 1
\verseWithHeading{Israel Continues Its Conquest}{After the death of Joshua, the \textit{Israelites}\lnALZ{}inquired of Adonai, saying, “Who will go up first for us against the Canaanites\lnAOY{}to fight against them?”}%
\verse{And Adonai said, “Judah will go up. I hereby give the land into his hand.”}%
\verse{And Judah said to Simeon his brother, “Go up with me into my allotment, and let us fight against the Canaanites; then I too will go with you into your allotment.” And Simeon went with him.}%
\verse{And Judah went up, and Adonai gave the Canaanites\lnAOY{}and the Perizzites\lnAPB{}into their hand, and they defeated ten thousand men at Bezek.}%
\verse{At Bezek they came upon Adoni-bezek, and they fought against him and defeated the Canaanites\lnAOY{}and the Perizzites.\lnAPB{}}%
\verse{And Adoni-bezek fled, but they pursued after him; they caught him and cut off \textit{his thumbs and big toes}.\lebnote{Literally “the thumbs of his hands and feet”}}%
\verse{Adoni-bezek said, “Seventy kings with \textit{their thumbs and big toes}\lebnote{Literally “the thumbs of their hands and feet”}cut off used to pick up scraps under my table; just as I have done, so God has repaid to me. And they brought him to Jerusalem, and he died there.}%
\verse{The descendants\lnGAW{}of Judah fought against Jerusalem, and they captured it, \textit{put it to the sword},\lebnote{Literally “they struck it with the mouth of the sword”}and \textit{set the city on fire}.\lebnote{Literally “the city they sent away with fire”}}%
\verse{Afterward the descendants\lnGAW{}of Judah pursued to fight against the Canaanites\lnAOY{}who were living in the hill country, the Negev,\lnDUD{}and the Shephelah.\lebnote{A geographical region associated with an area of low country on the western edge of the Judaean hills.}}%
\verse{And Judah went against the Canaanites\lnAOY{}living in Hebron (the former name of Hebron was Kiriath Arba). And they defeated Sheshai, Ahiman, and Talmai.}%
\verse{And from there they went to the inhabitants of Debir (the former name of Debir was Kiriath Sepher).}%
\verse{And Caleb said, “Whoever attacks Kiriath Sepher and captures it, I will give to him Acsah my daughter as a wife.”}%
\verse{Othniel son of Kenaz, the younger brother of Caleb, captured it, and he gave to him Acsah his daughter as a wife.}%
\verse{When she came to him, she urged him to ask her father for a field. As she dismounted from the donkey, Caleb said to her, “\textit{What do you want}?”\lnFSX{}}%
\verse{And she said to him, “Give me \textit{a gift};\lebnote{Literally “blessing”}you have given me the land of the Negev,\lnFOR{}and give me also a spring of water.” And Caleb gave to her the upper and lower spring.\lebnote{Judges 1:11–15 is almost identical to Joshua 15:13–19}}%
\verse{The descendants\lnGAW{}of Hobab the Kenite, Moses’ father-in-law, went up with the descendants\lnGAW{}of Judah from the city of palms into the wilderness of Judah, which is in the Negev\lnFOR{}near Arad. And they went\lebnote{Hebrew “he went”}and settled with the people.}%
\verse{And Judah went with his brother Simeon, and they defeated the Canaanites\lnAOY{}inhabiting Zephath; they utterly destroyed it, so he called the name of the city Hormah.}%
\verse{Judah captured Gaza and its territory, Ashkelon and its territory, and Ekron and its territory.}%
\verse{And Adonai was with Judah, and he took possession of the hill country, but they could not drive out the inhabitants of the plain because they had chariots of iron.}%
\verse{They gave Hebron to Caleb just as Moses said, and he drove out the three sons of Anak from there.}%
\verse{But the descendants\lnGAW{}of Benjamin did not drive out the Jebusites\lnAPD{}who lived in Jerusalem, so the Jebusites have lived among the descendants\lnGAW{}of Benjamin in Jerusalem to this day.}%
\verse{Likewise, the house of Joseph went up against Bethel, and Adonai was with them.}%
\verse{And the house of Joseph spied out Bethel (the former name of the city was Luz).}%
\verse{And when the spies saw a man leaving the city,\lebnote{Or “going out from the city”}they said to him, “Please show us the entrance of the city, and we will deal kindly\lebnote{Or “do a loyal love”}with you.”}%
\verse{So he showed them the entrance of the city, and they struck the city with \textit{the edge of the sword},\lnFNC{}but they let go the man and all his family.}%
\verse{And the man went to the land of the Hittites, and he built a city and named it Luz; this is its name to this day.}%
\verse{Manasseh did not drive out Beth-Sean and its towns, or Taanach and its towns, or the inhabitants\lnGBL{}of Dor and its towns, or the inhabitants of Ibleam and its towns, or the inhabitants of Megiddo and its towns; the Canaanites\lnAOY{}were determined to live in this land.}%
\verse{And it happened, when Israel grew strong, they put the Canaanites\lnAOY{}to forced labor, but they never totally drove them out.}%
\verse{Ephraim did not drive out the Canaanites\lnAOY{}living in Gezer, so the Canaanites\lnAOY{}lived in their midst in Gezer.}%
\verse{Zebulun did not drive out the inhabitants of Kitron or Nahalol, so the Canaanites\lnAOY{}lived in their midst and became subjected to forced labor.}%
\verse{Asher did not drive out the inhabitants of Acco, Sidon, Ahlab, Aczib, Helbah, Aphik, or Rehob,}%
\verse{so the Asherites\lebnote{Hebrew “Asherite”}lived in the midst of the Canaanites,\lnAOY{}the inhabitants of the land, for they did not drive them out.}%
\verse{Naphtali did not drive out the inhabitants of Beth Shemesh or Beth-anath, but lived in the midst of the Canaanites,\lnAOY{}the inhabitants of the land; the inhabitants of Beth Shemesh and Beth-anath became forced labor for them.}%
\verse{The Amorites pressed\lebnote{Or “forced”}the descendants\lnGAW{}of Dan to the hill country, and they did not allow them to come down to the plain;}%
\verse{the Amorites were determined to live in Har-heres,\lebnote{Or “Mount Heres”}in Aijalon, and in Shaalbim, but the hand of the house of Joseph was heavy on them, and they became subjected to forced labor.}%
\verse{The border of the Amorites\lnAPA{}ran from the ascent of Akrabbim from Sela and upward.}%
\end{biblechapter}

\begin{biblechapter} % Judges 2
\verseWithHeading{Israel Disobeys Adonai}{And the angel of Adonai went up from Gilgal to Bokim and said, “I brought you up from Egypt, and I brought you to the land that I had promised to your ancestors.\lnAHC{}I said, ‘I will never break my covenant with you.}%
\verse{And as for you, do not \textit{make a covenant}\lnFNU{}with the inhabitants of this land; break down their altars.’ But you did not listen to my voice. \textit{Why would you do such a thing}?\lebnote{Literally “What is this thing you have done?”}}%
\verse{Now I say, I will not drive them out from before you; they will \textit{become as thorns}\lebnote{Or “become snares;” some ancient manuscripts read “be adversaries”}for you, and their gods will be a trap for you.”}%
\verse{And as the angel of Adonai spoke these words to all the \textit{Israelites},\lnALZ{}the people \textit{wept bitterly}.\lebnote{Literally “lifted up their voices and they wept”}}%
\verse{And they called the name of this place Bokim,\lebnote{“Bokim” means “weepers”}and there they sacrificed to Adonai.}%
\verseWithHeading{Joshua Dies}{And Joshua sent the people away, and the \textit{Israelites}\lnALZ{}went each to their\lnDMD{}own inheritance to take possession of the land.}%
\verse{And the people served Adonai all the days of Joshua, and all the days of the elders who outlived Joshua, who saw all the great work Adonai had done for Israel.}%
\verse{And Joshua son of Nun, servant of Adonai, died \textit{at the age of one hundred and ten years}.\lnGAM{}}%
\verse{They buried him within the border of his inheritance in Timnah-heres, in the hill country of Ephraim north of Mount Gaash.}%
\verse{Moreover, that entire generation was gathered to their\lnDMD{}ancestors,\lnAHC{}and another generation grew up after them who did not know Adonai or the work he had done for Israel.\lebnote{Judges 2:6–10 is repeated in Joshua 24:28–31}}%
\verseWithHeading{Israel Worships the Baals}{The \textit{Israelites}\lnALZ{}did evil in the eyes of Adonai, and \textit{they served}\lebnote{Or “they worshiped”}the Baals.}%
\verse{They abandoned Adonai the God of their ancestors,\lnAHC{}who brought them out from the land of Egypt. They \textit{followed}\lnAJZ{}other gods from the gods of the people who were around them; and they bowed down to them, and they provoked the anger of Adonai.}%
\verse{They abandoned Adonai, and they served Baal and the Ashtaroth.}%
\verse{So \textit{the anger of Adonai was kindled}\lebnote{Literally “The nose of Adonai became hot”}against Israel, and he gave them into the hand of plunderers; and they plundered them, and he sold them into the hand of their enemies from all sides. They were unable to withstand their enemies any longer.}%
\verse{\textit{Whenever}\lebnote{Literally “At all that”}they went out, the hand of Adonai was against them to harm them, just as Adonai warned, and just as Adonai had sworn to them. And \textit{they were very distressed}.\lebnote{Literally “it was very cramped for them”}}%
\verse{Then Adonai raised up leaders,\lnGCG{}and they delivered them from the hand of their plunderers.}%
\verse{But they did not listen to their leaders,\lnGCG{}but lusted after other gods and bowed down to them. They turned away quickly from the way that their ancestors\lnAHC{}went, who had obeyed the commandment of Adonai; they did not do as their ancestors.}%
\verse{And when Adonai raised leaders\lnGCG{}for them, Adonai was with the leader,\lnGCK{}and he delivered them from the hand of their enemies all the days of the leader,\lnGCK{}for Adonai was moved by their groaning because of their persecutors and oppressors.}%
\verse{But when the leader\lnGCK{}died they relapsed and acted corruptly, more than their ancestors,\lnAHC{}following other gods, serving them,\lebnote{Or “worshiping them”}and bowing down to them. They would not give up their deeds or their stubborn ways.}%
\verse{\textit{So the anger of Adonai burned}\lebnote{Literally “So the nose of Adonai became hot”}against Israel, and he said, “Because this people transgressed my covenant that I commanded their ancestors,\lnAHC{}and have not obeyed my voice,}%
\verse{I will not again drive out anyone from before them from the nations that Joshua left when he died,}%
\verse{in order to test Israel whether or not they would observe the way of Adonai, to walk in it just as their ancestors\lnAHC{}did.”}%
\verse{So Adonai left those nations; he did not drive them out at once, and he did not give them into the hand of Joshua.}%
\end{biblechapter}

\begin{biblechapter} % Judges 3
\verseWithHeading{Some Nations Remain in the Land}{These are the nations that Adonai left, to test Israel by them (that is, to test all those who \textit{had not experienced}\lnGCQ{}any of the wars of Canaan,}%
\verse{in order that the generations of Israel would know war, to teach those \textit{who had not experienced it}\lebnote{Literally “who had not known it”}before):}%
\verse{the five rulers of the Philistines, all the Canaanites,\lnAOY{}the Sidonians,\lebnote{Hebrew “Sidonian”}and the Hivites\lnAPC{}living on Mount Lebanon, from Mount Baal Hermon up to Lebo-Hamath.}%
\verse{They were left for testing Israel, to know whether they would keep the commands of Adonai that he commanded their ancestors\lnAHC{}through the hand of Moses.}%
\verse{And the \textit{Israelites}\lnALZ{}lived in the midst of the Canaanites,\lnAOY{}the Hittites,\lnAOZ{}the Amorites,\lnAPA{}the Perizzites,\lnAPB{}the Hivites,\lnAPC{}and the Jebusites.\lnAPD{}}%
\verse{And they took their daughters as wives for themselves, and they gave their daughters to their sons, and they served their gods.\lebnote{Or “worshiped their gods”}}%
\verseWithHeading{Othniel}{The \textit{Israelites}\lnALZ{}did evil in the eyes of Adonai. They forgot Adonai their God, and they served the Baals and the Asheroth.\lebnote{Asheroth are cultic poles set up next to an altar symbolizing the goddess Asherah}}%
\verse{And \textit{the anger of Adonai was kindled}\lnEAW{}against Israel, and he sold them into the hand of Cushan-Rishathaim, the king of Aram Naharaim; and the \textit{Israelites}\lnALZ{}served Cushan-Rishathaim eight years.}%
\verse{The \textit{Israelites}\lnALZ{}cried out to Adonai, and Adonai raised up a deliverer for the \textit{Israelites}\lnALZ{}who delivered them, Othniel son of Kenaz, Caleb’s younger brother.}%
\verse{And the spirit of Adonai came upon him, and he judged Israel. He went out to war, and Adonai gave Cushan-Rishathaim king of Aram into his hand, and \textit{he prevailed over}\lebnote{Literally “his hand was strong over”}Cushan-Rishathaim.}%
\verse{So the land rested forty years. Then Othniel son of Kenaz died.}%
\verseWithHeading{Ehud}{And again the \textit{Israelites}\lnALZ{}did evil in the eyes of Adonai. So Adonai strengthened Eglon king of Moab against Israel, because they did evil in the eyes of Adonai.}%
\verse{He gathered to himself the \textit{Ammonites and Amalekites},\lebnote{Literally “sons of Ammon and Amalek” or “children of Ammon and Amalek”}and he went and defeated Israel, and they took possession of the city of palms.}%
\verse{And the \textit{Israelites}\lnALZ{}served Eglon king of Moab eighteen years.}%
\verse{And the \textit{Israelites}\lnALZ{}cried out to Adonai, and Adonai raised up for them a deliverer, Ehud son of Gera, a Benjaminite and \textit{a left-handed man}.\lebnote{Literally “a man bound by his right hand”}And the \textit{Israelites}\lnALZ{}sent a tribute to Eglon king of Moab \textit{through him}.\lnAVH{}}%
\verse{Ehud made for himself a short, \textit{two-edged}\lebnote{Literally “with two mouths”}sword (a cubit in length), and he fastened it under his clothes on his right thigh.}%
\verse{Then he presented the tribute to Eglon king of Moab. Now Eglon was a very fat man.}%
\verse{When Ehud\lebnote{That is, Eglon}had finished presenting the tribute, he sent away the people who carried the tribute.}%
\verse{But he turned back from the sculptured stones\lebnote{Or “Pesilim”; some translations translate the phrase as a proper name}that were near Gilgal, and he said, “I have \textit{a secret message}\lebnote{Literally “a word of secrecy”}for you, O king.” And he\lnATH{}said, “Silence!” So all those standing in his presence went out,}%
\verse{and Ehud came to him while he was sitting alone in his cool upper room. And Ehud said, “I have a \textit{message from God}\lebnote{Literally “word of God”}for you.” So he got up from his seat.}%
\verse{Then Ehud reached with his left hand for the sword on his right thigh, and he thrust it into his\lebnote{That is, Eglon’s}stomach.}%
\verse{And the handle also went in after the blade, and the fat closed over the blade because he did not draw back the sword from his stomach; and it went protruding out the back.\lebnote{The Hebrew is uncertain; some translations have “and the dirt/entrails came out”}}%
\verse{And Ehud went out the vestibule, and he closed the doors of the upper room and locked them behind him.}%
\verse{After he left, his servants returned. When they saw that the doors of the upper room were locked, \textit{they thought},\lnFLX{}“Surely he is \textit{relieving himself}\lebnote{Literally “covering his feet”}in the cool inner room.”}%
\verse{And they waited so long they became embarrassed because he did not open the doors of the upper room. So they took the key and opened the doors, and there their lord was lying on the ground dead.}%
\verse{And Ehud escaped while they delayed. He passed by the sculptured stones\lebnote{Or “Pesilim;” some translations to translate the phrase as a proper name}and escaped to Seirah.}%
\verse{And when he arrived he sounded the trumpet in the hill country of Ephraim, and the \textit{Israelites}\lnALZ{}went down from the hill country with him leading them.}%
\verse{And he said to them, “Follow after me! Adonai has given Moab your enemies into your hand.” So they went down after him, and they captured the fords of the Jordan toward Moab; and they did not allow anyone to cross over.}%
\verse{And they struck Moab at that time, about ten thousand men, \textit{all strong and able men};\lebnote{Literally “all fat and men of strength”}no one escaped.}%
\verse{And Moab was subdued on that day under the hand of Israel. And the land rested eighty years.\lnDVF{}}%
\verseWithHeading{Shamgar}{And Shamgar son of Anath came after him, and he killed six hundred Philistines with the goad of an ox; he also delivered Israel.}%
\end{biblechapter}

\begin{biblechapter} % Judges 4
\verseWithHeading{Deborah and Barak}{And again the \textit{Israelites}\lnALZ{}did evil in the eyes of Adonai, and Ehud died.}%
\verse{So Adonai sold them into the hand of Jabin king of Canaan, who reigned in Hazor. The commander of his army was Sisera, and he was living in Harosheth Haggoyim.}%
\verse{And the \textit{Israelites}\lnALZ{}cried to Adonai, as he had nine hundred iron chariots,\lnATZ{}and he oppressed the \textit{Israelites}\lnALZ{}\textit{cruelly}\lebnote{Literally “with force”}for twenty years.\lnDVF{}}%
\verse{Now at that time Deborah, a prophetess, the wife of Lappidoth, was judging\lebnote{Or “leading”}Israel.}%
\verse{And she used to sit under the palm tree of Deborah between Ramah and Bethel in the hill country of Ephraim; and the \textit{Israelites}\lnALZ{}went up to her for judgment.}%
\verse{She sent and called for Barak son of Abinoam from Kedesh Naphtali and said to him, “Has not Adonai the God of Israel commanded you? ‘Go, \textit{march to}\lebnote{Literally “take the lead at”}Mount Tabor, and take ten thousand men from the descendants\lnGAW{}of Naphtali and Zebulun.}%
\verse{\textit{I will draw out}\lebnote{Literally “I will draw to you”}Sisera, the commander of Jabin’s army, with his chariots and troops, to the wadi\lnEKD{}of Kishon, and I will give him into your hand.’ ”}%
\verse{Barak said to her, “If you go with me, I will go; but if you do not go with me, I will not go.”}%
\verse{She said, “Surely I will go with you; however, there will be no glory for you in \textit{the path you are taking},\lebnote{Or “the way you are going”}for Adonai will sell Sisera into the hand of a woman.” And Deborah stood up and went with Barak to Kedesh.}%
\verse{Barak summoned Zebulun and Naphtali to Kedesh; and \textit{they went up behind him},\lebnote{Literally “they went up in his feet”}ten thousand men, and Deborah went up with him.}%
\verse{And Heber the Kenite was separated from the other Kenites,\lnEAS{}that is, from the descendants\lnGAW{}of Hobab the father-in-law of Moses. And \textit{he was encamped}\lebnote{Literally “he spread his tent”}at Elon-bezaanannim, which is near Kedesh.}%
\verse{When they\lebnote{The subject is not specified in Hebrew}reported to Sisera that Barak son of Abinoam had gone up to Mount Tabor,}%
\verse{Sisera summoned all his chariots—all nine hundred chariots\lnATZ{}of iron—and the entire army that was with him from Harosheth Haggoyim to the wadi\lnEKD{}of Kishon.}%
\verse{And Deborah said to Barak, “Get up! This is the day that Adonai has given Sisera into your hand. Has Adonai not gone out before you?” So Barak went out from Mount Tabor with ten thousand men following him.}%
\verse{And Adonai threw Sisera and all his chariots and army into confusion\lebnote{Or “routed Sisera and all his chariots and army”}before \textit{the edge of Barak’s sword};\lebnote{Literally “the mouth of the sword before Barak”}and Sisera dismounted from his chariot and fled on foot.}%
\verse{But Barak pursued after the chariots\lnATZ{}and army as far as Harosheth Haggoyim, and all of Sisera’s army fell to \textit{the edge of the sword};\lnFNC{}no one was left.}%
\verse{Sisera fled on foot to the tent of Jael, the wife of Heber the Kenite, because there was peace between Jabin king of Hazor and the house of Heber the Kenite.}%
\verse{And Jael came out to meet Sisera, and she said to him, “Turn aside, my lord; turn aside to me and do not be afraid.” So he turned aside into her tent, and she covered him \textit{with a blanket}.\lebnote{Literally “with a covering/rug”}}%
\verse{And he said to her, “Please, give me a drink of water, because I am thirsty.” So she opened a skin vessel of milk and gave him a drink and covered him.}%
\verse{And he said to her, “Stand at the doorway of the tent, and if anyone comes and asks you, and says, ‘Is there anyone here?’ You must answer, ‘No.’ ”}%
\verse{But Jael, the wife of Heber the Kenite, took in her hand a tent peg and a hammer, and she went softly\lebnote{Or “secretly”}to him and drove the peg into his temple, and it went through into the ground; he was fast asleep since he was exhausted, and he died.}%
\verse{And behold, Barak was pursuing Sisera, and Jael went out to meet him, and she said to him, “Come, and I will show you the man whom you are seeking.” And he came with her and saw that Sisera was lying dead with the peg in his temple.}%
\verse{On that day God subdued Jabin king of Canaan before the \textit{Israelites}.\lnALZ{}}%
\verse{And the hand of the \textit{Israelites}\lnALZ{}\textit{pressed harder}\lebnote{Literally “went harder”}and harder on Jabin king of Canaan, until they destroyed Jabin king of Canaan.}%
\end{biblechapter}

\begin{biblechapter} % Judges 5
\verseWithHeading{The Song of Deborah and Barak}{And Deborah and Barak son of Abinoam sang on that day:}%
\verse{“When long hair hangs loosely in Israel, 
when the people willingly offer themselves, 
bless Adonai!}%
\verse{Hear, O kings! Give ear, O princes! 
I will sing to Adonai; 
I will sing praise to Adonai, 
the God of Israel.}%
\verse{Adonai, when you went down from Seir, 
when you marched from the region of Edom, 
the earth trembled, the heavens poured down, 
the clouds poured down water.}%
\verse{The mountains trembled\lebnote{Or “quaked”}before Adonai, 
this Sinai, at the presence of Adonai, the God of Israel.}%
\verse{“In the days of Shamgar son of Anath, 
in the days of Jael, the caravans had ceased, 
\textit{the travelers},\lebnote{Literally “the ones walking on the paths”}\textit{they kept to the byways}.\lebnote{Literally “they went on the crooked roads”}}%
\verse{The warriors\lebnote{Others interpret this word as referring to the “rural dwellers”}ceased; 
they failed to appear in Israel, 
until I,\lnGEF{}Deborah, arose; 
I\lnGEF{}arose as a mother in Israel.}%
\verse{God chose new leaders,\lebnote{ESV, NRSV translate “when new gods were chosen”}
then war was at the gates; 
a small shield or a spear was not seen 
among forty thousand in Israel.}%
\verse{My heart goes out to the commanders of Israel, 
those offering themselves willingly among the people; 
bless Adonai!}%
\verse{The riders of white female donkeys, 
those sitting on saddle blankets, 
and those going on the way, talk about it!}%
\verse{At the sound of those dividing\lebnote{Meaning uncertain; other translations have “archers” (Tanakh), “musicians” (ESV, NRSV) or “singers” (NIV, HCSB)}the sheep 
among the watering places, 
there they will recount the righteous deeds of Adonai, 
the righteous deeds for his warriors\lebnote{Hebrew “warrior”}in Israel. 
Then the people of Adonai went down to the gates.}%
\verse{“Wake up, wake up, Deborah! 
Wake up, wake up, sing a song! 
Get up, Barak! 
Take captive your captives, O son of Abinoam.}%
\verse{Then the remnant went down to the nobles; 
the people of Adonai went down for him\lnDYL{}against the mighty.}%
\verse{From Ephraim is their root into Amalek, 
after you, Benjamin, with your family; 
from Makir the commanders went down, 
and from Zebulun those carrying the scepter 
of the military commander.}%
\verse{And the chiefs\lebnote{Hebrew “my chiefs”; ancient translations read “the chiefs”}in Issachar were with Deborah; 
and Issachar likewise was with Barak; 
into the valley \textit{he was sent to get him from behind}.\lebnote{Literally “he was sent at his feet”}
Among the clans of Reuben 
were great \textit{decisions of the heart}.\lebnote{Or “thoughts of the heart”}}%
\verse{Why do you sit among the sheepfolds, 
to hear the calling sounds of the herds? 
For the clans of Reuben, 
there were great searchings of the heart.}%
\verse{Gilead has remained\lebnote{Or “stayed”}beyond the Jordan. 
Why did Dan dwell as a foreigner with ships? 
Asher sat at the coast of the waters, 
and by his coves he has been settling down.}%
\verse{Zebulun is a people who scorned death, 
and Naphtali, on the heights of the field.}%
\verse{“The kings came, they fought; 
then the kings of Canaan fought; 
at Taanach by the waters of Megiddo, 
they got no plunder in silver.}%
\verse{The stars fought from heaven; 
from their courses they fought against Sisera.}%
\verse{The wadi\lnEKD{}torrent of Kishon swept them away, 
the raging wadi torrent, 
the wadi torrent of Kishon. 
March on, my soul, with strength!}%
\verse{“Then the hooves of the horse beat loudly, 
because of galloping, galloping of his stallions.}%
\verse{‘Curse Meroz,’ says the angel of Adonai; 
‘curse bitterly its inhabitants, 
because they did not come to the help of Adonai, 
to the help of Adonai against the mighty.’}%
\verse{“Most blessed of women is Jael, 
the wife of Heber the Kenite; 
most blessed is she of women among tent dwellers.}%
\verse{He asked for water, and she gave milk; 
in a drinking bowl for nobles, she brought curds.}%
\verse{She reached out her hand to the peg, 
and her right hand for the workman’s hammer; 
and she struck Sisera, crushed his head, 
and she shattered and pierced his temple.}%
\verse{Between her feet he sank, he fell, he lay. 
Between her feet he sank down, he fell; 
Where he sank down, there he fell—\textit{dead}.\lebnote{Literally “devastated”}}%
\verse{“Through the window she looked down; 
the mother of Sisera cried out through the lattice, 
‘Why is his chariot delayed in coming? 
Why do the hoof beats\lebnote{Or “steps”}of his chariot tarry?’}%
\verse{The wisest of her ladies answer her; 
she also answers the question herself:}%
\verse{‘Are they not finding and dividing the plunder? 
\textit{A bedmate or two bedmates for every man};\lebnote{Literally “a womb, two wombs for head of every man”}
colorful garments for Sisera, 
plunder of colorful garments,\lnGEJ{}
beautifully finished colorful garments, 
on the neck of the plunderer?’}%
\verse{So may all your enemies perish, O Adonai, 
but those who love him are like the rising sun at its brightest.” And the land had rest for forty years.}%
\end{biblechapter}

\begin{biblechapter} % Judges 6
\verseWithHeading{The Midianites Oppresses Israel}{The \textit{Israelites}\lnALZ{}did evil in the eyes of Adonai, and Adonai gave them into the hand of the Midianites\lnGEL{}for seven years.}%
\verse{The hand of the Midianites\lnGEL{}prevailed over Israel; because of the presence of the Midianites,\lnGEL{}the \textit{Israelites}\lnALZ{}made for themselves hiding places that were in the mountains, caves, and strongholds.}%
\verse{And whenever Israel sowed seed, the Midianites,\lnGEL{}Amalekites,\lnDVL{}and the people of the east would come up against them.}%
\verse{They would camp against them and destroy the produce of the land \textit{as far as}\lebnote{Literally “until your coming to”}Gaza; they left no produce in Israel, or sheep, ox, or donkey.}%
\verse{For they, their livestock, and their tents would come up like a great number of locusts; they and their camels could not be counted; they came into the land and devoured it.}%
\verse{Israel was very poor because of the presence of the Midianites,\lnGEL{}and the \textit{Israelites}\lnALZ{}cried out to Adonai.}%
\verse{When the \textit{Israelites}\lnALZ{}cried out to Adonai on account of the Midianites,\lnGEL{}}%
\verse{Adonai sent a prophet to the \textit{Israelites},\lnALZ{}and he said to them, “Thus says Adonai the God of Israel: ‘I brought you up from Egypt; I brought you from the house of slavery.\lebnote{Hebrew “slaves”}}%
\verse{I delivered you from the hand of Egypt and from the hand of all your oppressors, and drove them out from \textit{before you};\lebnote{Literally “your face/presence”}and I gave you their land.}%
\verse{And I said to you, ‘I am Adonai your God; do not fear\lebnote{Or “revere”}the gods of the Amorites,\lnAPA{}in whose land you are living.’ But you have not listened to my voice.”}%
\verseWithHeading{The Angel of Adonai Calls Gideon}{The angel of Adonai came and sat under the oak\lnGEX{}that was at Ophrah that belonged to Jehoash the Abiezrite; and Gideon his son was threshing wheat in the winepress to hide it\lebnote{Or “keep it away”}from the Midianites.\lnGEL{}}%
\verse{The angel of Adonai appeared to him and said to him, “Adonai is with you, \textit{you mighty warrior}.”\lebnote{Literally “strong/mighty of power”}}%
\verse{Gideon said to him, “Excuse me, my lord. If Adonai is with us, why then has all this happened to us? Where are all his wonderful deeds that our ancestors\lnAHC{}recounted to us, saying, ‘Did not Adonai bring us up from Egypt?’ But now Adonai has forsaken us; he has given us into the palm of Midian.”}%
\verse{And Adonai turned to him and said, “Go in this your strength, and you will deliver Israel from the palm of Midian. Did I not send you?”}%
\verse{He said to him, “Excuse me, my lord. How will I deliver Israel? Look, my clan is the weakest in Manasseh, and I am the youngest in my father’s house.”}%
\verse{And Adonai said to him, “But I will be with you, and you will defeat Midian \textit{as if they are one man}.”\lnDUX{}}%
\verse{And he\lnGFB{}said to him, “Please, if I have found favor in your eyes, \textit{show me a sign}\lebnote{Literally “make a sign for me”}that you are speaking with me.}%
\verse{Please, do not depart from here until I come back to you and bring out my gift and set it out before you.”\lnFKD{}And he said, “I will stay until you return.”}%
\verse{And Gideon went and prepared \textit{a young goat}\lebnote{Literally “a kid of goat”}and unleavened cakes from an ephah of flour; he put meat in a basket, and the broth he put in a pot, and he brought them to him under the oak\lnGEX{}and presented them.}%
\verse{The angel of God said to him, “Take the meat and the unleavened cakes and put them on this rock; pour the broth over it.” And he did so.}%
\verse{Then the angel of Adonai reached out the tip of the staff that was in his hand, and he touched the meat and the unleavened cakes; and fire went up from the rock and consumed the meat and the unleavened cakes. And the angel of Adonai went \textit{from his sight}.\lebnote{Literally “from his eyes”}}%
\verse{And Gideon realized that he was the angel of Adonai; and Gideon said, “Oh, my lord Adonai! For now I have seen the angel of Adonai face to face.”}%
\verse{And Adonai said to him, “Peace be with you. Do not fear; you will not die.”}%
\verse{And Gideon built there an altar to Adonai, and he called it “Adonai is peace.” To this day it is still in Ophrah of the Abiezrites.\lnGFE{}}%
\verse{Now on that same night Adonai said to him, “Take the bull of the cattle that belongs to your father, and a second bull seven years old, and pull down the altar of Baal that belongs to your father, and cut down the Asherah\lnGFF{}that is beside it;}%
\verse{and build an altar to Adonai your God on the top of this stronghold in the proper arrangement, and take a second bull and offer it as a burnt offering with the wood of the Asherah\lnGFF{}that you will cut down.}%
\verse{Gideon took ten men from his servants, and he did just as Adonai told him;\lebnote{Or “spoke to him”}and because he was too afraid of his \textit{father’s family}\lebnote{Literally “father’s house”}and the men of the city to do it during the day, he did it during night.}%
\verseWithHeading{Gideon Destroys the Altar of Baal}{When the men of the city got up early in the morning, look, the altar of Baal and the Asherah\lnGFF{}that was beside it were cut down, and the second bull was offered on the altar that had been built.}%
\verse{\textit{And they said to one another},\lebnote{Literally “And each man to his neighbor said”}“Who did this thing?” So they searched and inquired, and they said, “Gideon son of Jehoash did this thing.”}%
\verse{And the men of the city said to Jehoash, “Bring out your son so that he may die, for he has pulled down the altar of Baal and cut down the Asherah that was beside it.”}%
\verse{But Jehoash said to all who stood against him,\lebnote{Or “who arrayed against him”}“Will you contend for Baal? Will you rescue him? Whoever contends for him will be put to death by the morning. If he is a god, let him contend for himself because \textit{his altar has been pulled down}.”\lebnote{Literally “because he has pulled down his altar”; the subject in Hebrew has not been specified}}%
\verse{Thus, on that day he\lnGFB{}was called Jerub-Baal, \textit{which means},\lnAQL{}“Let Baal contend against him,” because he had pulled down his altar.}%
\verse{Then all the Midianites,\lnGEL{}Amalekites,\lnDVL{}and the people of the east gathered together and crossed the Jordan; and they camped in the valley of Jezreel.}%
\verse{So the Spirit of Adonai \textit{took possession of}\lebnote{Literally “clothed”}Gideon, and he blew on the trumpet, and the Abiezrites\lebnote{Hebrew “Abiezerite”}were called to follow him.}%
\verse{He sent messengers throughout all Manasseh, and they were also called to follow him; and he sent messengers throughout Asher, Zebulun, and Naphtali, and they went up to meet them.}%
\verseWithHeading{Gideon Tests Adonai With the Fleece}{Then Gideon said to God, “In order to see that you will deliver Israel by my hand, just as you have said,}%
\verse{I will place a fleece of wool on the threshing floor. If there is dew on the fleece only, and all of the ground is dry, I will know that you will deliver Israel by my hand, just as you have said.”}%
\verse{And it was so. He arose early the next day and squeezed the fleece, and he wrung out dew from the fleece, a full drinking bowl of water.}%
\verse{And Gideon said to God, “\textit{Do not let your anger burn}\lebnote{Literally “Do not let your nose become hot”}against me; let me speak once more. Please let me test once more with the fleece; let the fleece be dry, and let there be dew on the ground.”}%
\verse{And God did so that night; only the fleece was dry, and dew was on all the ground.}%
\end{biblechapter}

\begin{biblechapter} % Judges 7
\verseWithHeading{Gideon’s Three Hundred Men}{Then Jerub-Baal (that is, Gideon) rose early, and all the army that was with him. They were camped beside the spring of Harod;\lebnote{Or “En-Harod”}the camp of Midian was north of the hill of Moreh, in the valley.}%
\verse{And Adonai said to Gideon, “The troops that are with you are too many for me to give Midian into their hands; Israel will boast, saying, ‘My hand has delivered me.’}%
\verse{So then, please proclaim in the \textit{hearing}\lnGFM{}of the troops, saying, ‘Whoever is fearful and trembling, let him return and depart from the Mount of Gilead.’ ” About twenty-two thousand troops returned, and ten thousand remained.}%
\verse{And Adonai said to Gideon, “There are still too many troops; bring them down to the water, and I will sift through them\lebnote{Or “test them”}for you there. For whomever I say to you, ‘This one will go with you,’ he will go with you; and for all whom I say to you, ‘This one will not go with you,’ he will not go.”}%
\verse{So he brought down the troops to the water, and Adonai said to Gideon, “You must separate everyone who laps up the water to drink with his tongue like a dog from those \textit{who kneel}.”\lebnote{Literally “who kneels to his knees”}}%
\verse{The number of those lapping up the water with their hand to their mouth was three hundred men; all the rest of the troops kneeled to drink the water.}%
\verse{And Adonai said to Gideon, “I will deliver you with the three hundred men lapping up the water; I will give Midian into your hand, so let the other troops go, each to his own place.}%
\verse{So they took their provisions and their trumpets into their hand, and he sent all the men of Israel, each one, to his tent; but three hundred of the men he kept; the camp of Midian was below him in the valley.}%
\verse{And that night Adonai said to him, “Get up; go down against the camp, for I have given it into your hand.}%
\verse{But if you are afraid, go down to the camp with Purah your servant,}%
\verse{and you will hear what they say; and afterward \textit{you will have courage},\lebnote{Literally “your hands will be strong”}and you will go down against the camp.” Then he went down with Purah his servant to the outpost of the armed men\lebnote{Or “edge of the men lined up in array”}that were in the camp.}%
\verse{Now the Midianites,\lnGEL{}Amalekites,\lnDVL{}and all the people of the east were lying in the valley, like a great multitude of locusts; their camels were without number, as numerous as the sand that is on the shore of the sea.}%
\verse{When Gideon came, a man was recounting a dream\lebnote{Or “telling a dream”}to his friend, and he said, “Behold, \textit{I had a dream};\lebnote{Literally “I dreamed a dream”}a round loaf of barley bread was tumbling into the camp of Midian, and it came up to the tent, it struck it, and it fell and turned it upside down so that the tent fell.”}%
\verse{His friend answered him and said, “This cannot be anything except the sword of Gideon son of Jehoash, a man of Israel; God has given Midian and the entire camp into his hand.”}%
\verse{When Gideon heard the recounting of the dream\lebnote{Or “telling of the dream”}and its interpretation, he bowed down and returned to the camp of Israel; and he said, “Get up, for Adonai has given the camp of Midian into your hand.”}%
\verse{He divided the three hundred men into three companies,\lnGFP{}and he put trumpets and empty jars in everyone’s hand, with torches inside the jars.}%
\verse{And he said to them, “Watch me and do the same. When I come to the edge of the camp, \textit{do just as I do}.\lebnote{Literally “just as I will do, you must do”}}%
\verse{When I and all who are with me blow on the trumpet, you must also blow on the trumpets and surround the camp, and you must say, ‘To Adonai and to Gideon!’ ”}%
\verse{So Gideon and the hundred men who were with him came to the edge of the camp at the beginning of the middle night-watch, when they had just finished setting up the guards, and they blew on the trumpets and smashed the jars that were in their hands.\lnEKE{}}%
\verse{When three companies\lnGFP{}blew on the trumpets and broke the jars, they held in their left hand the torches and in their right hand the trumpets for blowing, and they cried, “A sword for Adonai and for Gideon!”}%
\verse{And each stood \textit{in his place}\lnGFS{}all around the camp, and all the camp ran, and they cried out as they fled.}%
\verse{When they blew the three hundred trumpets, Adonai set the sword of each one against his neighbor throughout the whole camp, and the camp\lnGFT{}fled as far as Beth Shittah toward Zererah, up to Abel Meholah, the border by Tabbath.}%
\verse{And the men of Israel were called from Naphtali, from Asher, and from all of Manasseh, and they pursued after Midian.}%
\verse{And Gideon sent messengers throughout all the hill country of Ephraim, saying, “Come down to oppose\lebnote{Or “meet”}Midian, and capture from them the waters up to Beth Barah and the Jordan.” He called out all the men of Ephraim, and they captured the waters up to Beth Barah and the Jordan.}%
\verse{And they captured the two commanders of Midian, Oreb and Zeeb, and they killed Oreb at the rock of Oreb, and they killed Zeeb at the wine press of Zeeb, while they chased Midian; and they brought the heads\lnFIR{}of Oreb and Zeeb to Gideon from beyond the Jordan.}%
\end{biblechapter}

\begin{biblechapter} % Judges 8
\verseWithHeading{Gideon Pursues Zebah and Zalmunna}{The men of Ephraim said to him, “What is this thing you have done to us, not calling us when you went to fight against the Midianites?”\lnGEL{}And they quarreled with him severely.}%
\verse{And he said to them, “What I have done now in comparison with you? Are not the gleanings of Ephraim better than the grape harvest of Abiezer?}%
\verse{God has given into your hand the commanders of Midian, Oreb, and Zeeb. What have I been able to do in comparison with you?” And their \textit{anger}\lebnote{Literally “spirit”}against him subsided when \textit{he said that}.\lebnote{Literally “he said this thing/word”}}%
\verse{Then Gideon came to the Jordan, crossing it with the three hundred men who were with him, weary and pursuing.}%
\verse{He said to the men of Succoth, “Please give loaves of bread to the people who are \textit{following me},\lebnote{Literally “at my feet”}for they are weary, and I am pursuing Zebah and Zalmunna, the kings of Midian.”}%
\verse{The officials of Succoth said, “Is the hand\lebnote{Or “palm”}of Zebah and Zalmunna in your hand now, that we should give bread to your army?”}%
\verse{Gideon said, “Well then, when Adonai gives Zebah and Zalmunna into my hand, I will trample your flesh with the thorns and briers of the wilderness.”}%
\verse{He went from there to Penuel, and he spoke similarly to them; and the men of Penuel answered him just as the men of Succoth answered.}%
\verse{And he said also to the men of Penuel, saying, “When I return \textit{safely},\lnGFW{}I will tear down this tower.”}%
\verse{Now Zebah and Zalmunna were in Karkor, and their armies with them, about fifteen thousand men remained from the entire army\lnGFX{}of the people of the east; those that fell in battle were one hundred and twenty thousand \textit{swordsmen}.\lebnote{Literally “sword-drawing men”}}%
\verse{And Gideon went up the route of those who dwell in tents on the east of Nobah and Jogbehah, and he attacked the army\lnGFX{}when it was off its guard.\lebnote{Or “unsuspecting”}}%
\verse{And Zebah and Zalmunna fled, and he pursued them and captured the two kings of Midian, Zebah and Zalmunna, and he routed\lebnote{Or “he frightened” or “he threw into panic”}the entire army.\lnGFX{}}%
\verse{Then Gideon son of Jehoash returned from the battle \textit{by way of}\lnBHR{}the ascent of Heres.\lebnote{Or “pass of Heres”}}%
\verse{He captured a young man \textit{from Succoth}\lebnote{Literally “from the men of Succoth”}and questioned him. The young man listed\lebnote{Or “wrote down”}for him the commanders of Succoth and its elders, seventy-seven men.}%
\verse{He came to the men of Succoth, and he said, “Here is Zebah and Zalmunna, about whom you taunted me, saying, ‘Is the palm\lnGGB{}of Zebah and Zalmunna in your hand now, that we should give food to your weary men?’ ”}%
\verse{He took the elders of the city and the thorn bushes and briers of the wilderness, and \textit{he trampled}\lebnote{Literally “he taught,” but see v. 7}the men of Succoth with them.}%
\verse{He broke down the tower of Penuel, and he killed the men of the city.}%
\verse{And he said to Zebah and Zalmunna, “What type were the men whom you killed at Tabor?” And they said, “\textit{They were like you};\lebnote{Literally, “like you, like them”}each one of them had the appearance of the sons of the king.”}%
\verse{He said, “They were my brothers, the sons of my mother. As Adonai lives, if you had kept them alive I would not kill you.”}%
\verse{And he said to Jether, his firstborn, “Get up, kill them.” But the boy did not draw his sword, for he was afraid because he was still a boy.}%
\verse{Zebah and Zalmunna said, “Get up yourself, and strike us, for as is the man, so is his power.”\lnGGC{}So Gideon got up and killed Zebah and Zalmunna, and he took the crescent ornaments that were on the necks of their camels.}%
\verse{The men of Israel said to Gideon, “Rule over us, both you and your sons, and your sons’ son, for you have delivered us from the hand of Midian.}%
\verse{But Gideon said to them, “I will not rule over you, and my son will not rule over you; Adonai will rule over you.”}%
\verse{And Gideon said to them, “\textit{Let me make a request of you},\lebnote{Literally “Let me ask from you a request”}that each of you give to me an ornamental ring from his plunder.” (They had ornamental rings of gold, because they were Ishmaelites.)}%
\verse{They said, “We will gladly give them,” and they spread out a garment, and everyone threw there an ornamental ring of his plunder.}%
\verse{The weight of the ornamental rings of gold that he requested was one thousand seven hundred shekels of gold, apart from the crescents, pendants, and purple garments that were on the kings of Midian, and apart from the pendants that were on the necks of their camels.}%
\verse{Gideon made an ephod out of it, and he put it in his town in Ophrah, and all Israel prostituted themselves to it there, and it became a snare to Gideon and his family.}%
\verse{And Midian was subdued before the \textit{Israelites},\lnALZ{}and they did not again lift up their head, and the land rested for forty years in the days of Gideon.}%
\verseWithHeading{The Death of Gideon}{Jerub-Baal son of Joash lived in his own house.}%
\verse{Now Gideon had seventy sons, \textit{his own offspring},\lebnote{Literally “from his own loins”}for he had many wives.}%
\verse{His concubine who was in Shechem also bore for him a son, and \textit{he named him}\lebnote{Literally “he set his name”}Abimelech.}%
\verse{And Gideon son of Joash died at a good old age, and he was buried in the tomb of Jehoash his father, in Ophrah of the Abiezrites.\lnGFE{}}%
\verse{And it happened, as soon as Gideon died, the \textit{Israelites}\lnALZ{}returned and prostituted themselves after the Baals, and they made for themselves Baal-Berith as god.}%
\verse{The \textit{Israelites}\lnALZ{}did not remember Adonai their God, who had delivered them from the hand of their enemies from all around,}%
\verse{nor did they show favor\lebnote{Or “loyalty”}to the house of Jerub-Baal (that is, Gideon) in accordance with all the good that he did for Israel.}%
\end{biblechapter}

\begin{biblechapter} % Judges 9
\verseWithHeading{Abimelech Attempts to Become King}{And Abimelech son of Jerub-Baal went to Shechem, to the relatives of his mother, and he said to them and to the house of his mother’s father,}%
\verse{“\textit{Speak to}\lebnote{Literally “Speak into the ears”}the lords of Shechem, ‘What is better for you, that seventy men all from the sons of Jerub-Baal rule over you, or that one man rules over you?’ Remember that I am your bone and your flesh.”}%
\verse{And his mother’s relatives spoke all these words concerning him \textit{to}\lebnote{Literally “in the ears of”}all the lords of Shechem; and \textit{they supported Abimelech},\lebnote{Literally “it reached their heart after Abimelech”}for they said, “He is our relative.”\lnCZJ{}}%
\verse{And they gave to him seventy pieces of silver from the temple\lnGGI{}of Baal-Berith, and Abimelech hired with them\lebnote{That is, the pieces of silver}worthless and reckless men, and \textit{they followed him}.\lebnote{Literally “They went after him”}}%
\verse{And he went to his father’s house at Ophrah, and he killed his brothers, the sons of Jerub-Baal, seventy men, on one stone. But Jotham the youngest son of Jerub-Baal survived, because he hid himself.}%
\verse{All the lords of Shechem and Beth-Millo gathered, and they went and made Abimelech as king, near the oak\lnGEX{}of the pillar that is at Shechem.}%
\verse{And they told Jotham, and he went up and stood on the top of Mount Gerizim, and \textit{he cried out loud}\lebnote{Literally “he lifted his voice and called”}and said to them, “Listen to me, lords of Shechem, so that God may listen to you.}%
\verse{“The trees went certainly, 
to anoint a king over themselves. 
And they said to the olive tree, 
‘Rule over us.’}%
\verse{And the olive tree replied, 
‘Should I stop producing my oil, 
which by me gods and men are honored, 
to go sway over the trees?’}%
\verse{Then the trees said to the fig tree, 
‘You, come rule over us.’}%
\verse{But the fig tree said to them, 
‘Should I stop producing my sweetness, 
and my good crop, 
to go sway over the trees?’}%
\verse{And the trees said to the vine, 
‘You, come rule over us.’}%
\verse{But the vine said to them, 
‘Should I stop producing my wine 
that makes the gods and men happy, 
to go sway over the trees?’}%
\verse{So all the trees said to the thornbush, 
‘You, come rule over us.’}%
\verse{And the thornbush said to the trees, 
‘If in good faith\lnGGK{}you are anointing 
me as king over you, 
then come and take refuge in my shade; 
if not, may fire go out from the thornbush 
and devour the cedars of Lebanon.’}%
\verse{“So then, if you have acted in good faith\lnGGK{}and sincerity in making Abimelech king, and if you have dealt well with Jerub-Baal and his house, and have dealt with him \textit{according to his accomplishments}\lebnote{Literally “according to the dealings of his hand”}—}%
\verse{for my father fought and \textit{risked his life}\lebnote{Literally “he threw his life in front”}for you and delivered you from the hand of Midian;}%
\verse{but today you have risen against the house of my father and killed his sons, seventy men on one stone, and you have made Abimelech, the son of his slave woman, a king over the lords of Shechem, because he is your relative\lnCZJ{}—}%
\verse{if you have acted in good faith\lnGGK{}and sincerity with Jerub-Baal and his house this day, then rejoice in Abimelech, and let him also rejoice in you.}%
\verse{But if not, let a fire come out from Abimelech and let it devour the lords of Shechem and Beth-Millo; and let a fire come out from the lords of Shechem, and from Beth-Millo, and let it devour Abimelech.”}%
\verse{And Jotham escaped and fled, and went to Beer; he remained there because of Abimelech his brother.}%
\verseWithHeading{The Downfall of Shechem and Abimelech}{Abimelech ruled over Israel three years.}%
\verse{And God sent an evil spirit between Abimelech and the lords of Shechem, and the lords of Shechem dealt treacherously with Abimelech,}%
\verse{so that the violence done to the seventy brothers of Abimelech would be avenged and their blood be placed on Abimelech their brother, who killed them, and on the lords of Shechem, \textit{who helped}\lebnote{Literally “who strengthened his hands”}to kill his brothers.}%
\verse{And the lords of Shechem set for him ambushes on the top of the mountains, and they robbed all who passed by them along the road; and it was reported to Abimelech.}%
\verse{And Gaal son of Ebel and his relatives came, and they crossed over into Shechem, and the lords of Shechem \textit{gave him confidence}.\lebnote{Literally “put confidence into him”}}%
\verse{They went out into the field and harvested their vineyards and trod them, and they \textit{held a festival}.\lebnote{Literally “made a rejoicing”}And they went into the temple\lnGGI{}of their god, and they ate and drank and cursed Abimelech.}%
\verse{Then Gaal son of Ebed said, “Who is Abimelech, and who are we of Shechem that we should serve him? Is he not the son of Jerub-Baal, and is not Zebul his chief officer?\lnGGP{}Serve the men of Hamor the father of Shechem. Why should we serve him?\lebnote{That is, Abimelech}}%
\verse{If only this people \textit{were in my command}!\lebnote{Literally “were in my hand”}Then I would remove Abimelech, and I would have said,\lebnote{Hebrew “he said”; the translation “I would have said” follows the LXX}‘Increase your army and come out!’ ”}%
\verse{When Zebul the commander of the city heard the words of Gaal son of Ebed, \textit{he became angry},\lnDTQ{}}%
\verse{and he sent messengers to Abimelech in Tormah, saying, “Look, Gaal son of Ebed and his relatives are coming to Shechem, and they are stirring up the city against you.}%
\verse{So then, get up by night, you and the army that is with you, and lie in ambush in the field.}%
\verse{And in the morning at sunrise, get up and rush the city; and look, when he and the troops who are with him come out to you, \textit{you must act according to whatever opportunity offers itself}.\lebnote{Literally “you must do to him just as it will find your hand”}}%
\verse{So Abimelech and all the army that was with him got up by night, and they lay in ambush against Shechem in four divisions.\lnGGR{}}%
\verse{Gaal son of Ebed went out and stood at the entrance of the city gate, and Abimelech and the army that was with him got up from the ambush.}%
\verse{When Gaal saw the army, he said to Zebul, “Look, people are coming down from the top of the mountains!” And Zebul said to him, “\textit{The shadows of the mountains look like people to you}.”\lebnote{Literally “the shadows of the mountains you are seeing as men”}}%
\verse{\textit{And Gaal spoke again}\lebnote{Literally “And he did again again Gaal to speak”}and said, “Look, people are coming down from Tabbur-erez, and one division\lebnote{Or “company”}is coming from the direction of Elon-meonenim.”\lebnote{Or “the Sorcerer’s Oak”}}%
\verse{And Zebul said to him, “Where then \textit{is your boast},\lebnote{Literally “is your mouth”}you who said, ‘Who is Abimelech that we should serve him?’ Is this not the army that you rejected? Please, go out now and fight against them.”\lebnote{Hebrew “against him”}}%
\verse{So Gaal went out before the lords of Shechem and fought against Abimelech.}%
\verse{And Abimelech chased him, and he fled before him; many fell slain up to the entrance of the gate.}%
\verse{So Abimelech resided at Arumah, and Zebul drove out Gaal and his relatives\lnDAP{}from living in Shechem.}%
\verse{On the next day the people went out to the field; and it was reported to Abimelech,}%
\verse{so he took the army and divided them into three divisions,\lnGGR{}and he laid an ambush in the field. And he saw the people were coming out from the city, and he arose against them and killed them.}%
\verse{Then Abimelech and the divisions\lnGGR{}that were with him dashed out and stood at the entrance of the city gate, and the two divisions dashed out against all who were in the field, and they killed them.}%
\verse{Abimelech fought against the city all that day, and he captured the city and killed the people that were in it; then he broke down the city and sowed it with salt.}%
\verse{When all the lords of the tower of Shechem heard, they went to the vault\lnGGV{}of the temple\lnGGI{}of El-Berith.}%
\verse{It was told to Abimelech that all the lords of the tower of Shechem had gathered.}%
\verse{So Abimelech went up Mount Zalmon, he and all his army that were with him, and Abimelech took the ax\lebnote{Hebrew “axes”}in his hand and cut down a bundle of brushwood, and he lifted it and put it on his shoulder. And he said to the army that was with him, “What you have seen me do, quickly do also.”}%
\verse{So the whole army cut down each one branch for himself and followed Abimelech, and they put them against the vault\lnGGV{}and set the vault ablaze with fire on those inside, so that all the men of the tower of Shechem died, about a thousand men and women.}%
\verse{Then Abimelech went to Thebez, and he encamped against Thebez and captured it.}%
\verse{But there was a strong tower in the middle of the city, and all the men, women, and lords of the city fled there and shut themselves in; and they went up to the roof of the tower.}%
\verse{Abimelech came up to the tower and fought against it, and he came near the entrance of the tower to burn it with fire.}%
\verse{But a certain woman\lebnote{Or “But one woman”}threw an upper millstone on Abimelech’s head and cracked open his skull.}%
\verse{He called quickly to the young man carrying his weapons, and he said to him, “Draw your sword and kill me, so that they will not say of me, ‘A woman killed him.’ ” So the young man\lebnote{Hebrew “his young man”}stabbed him, and he died.}%
\verse{When the men of Israel saw that Abimelech was dead, each one went to his home.\lnGGY{}}%
\verse{So God repaid\lnGGZ{}the wickedness\lnGHA{}that Abimelech committed against his father by killing his seventy brothers.}%
\verse{And God also repaid all the wickedness\lnGHA{}of the men of Shechem on their heads, and the curse of Jotham son of Jerub-Baal fell on them.}%
\end{biblechapter}

\begin{biblechapter} % Judges 10
\verseWithHeading{The Philistines and Ammonites Afflict the Israelites}{After Abimelech, Tola son of Puah son of Dod, a man of Issachar, rose up to deliver Israel; and he was living at Shamir in the hill country of Ephraim.}%
\verse{And he judged Israel twenty-three years. And he died and was buried in Shamir.}%
\verse{After him Jair the Gileadite rose up, and he judged Israel twenty-two years.}%
\verse{And he had thirty sons who would ride on thirty donkeys, and they had thirty towns that are in the land of Gilead that they called Havvoth Jair until this day.}%
\verse{And Jair died and was buried in Kamon.}%
\verse{And again, the \textit{Israelites}\lnALZ{}did evil in the eyes of Adonai. They served the Baals, the Ashtaroth, the gods of Aram, Sidon, Moab, and the gods of the \textit{Ammonites}\lnAIT{}and Philistines; they abandoned Adonai and did not serve him.}%
\verse{And \textit{the anger of Adonai burned}\lnEAW{}against Israel, and he sold them into the hand of the Philistines and the \textit{Ammonites}.\lnAIT{}}%
\verse{They crushed and oppressed the \textit{Israelites}\lnALZ{}in that year; for eighteen years they crushed all the \textit{Israelites}\lnALZ{}who were beyond the Jordan, in the land of the Amorites,\lnAPA{}which is in Gilead.}%
\verse{The \textit{Ammonites}\lnAIT{}crossed the Jordan to fight also against Judah, Benjamin, and the house of Ephraim; and Israel was very distressed.}%
\verse{Then the \textit{Israelites}\lnALZ{}cried out to Adonai, saying, “We have sinned against you; we have abandoned our God and served the Baals.”}%
\verse{And Adonai said to the \textit{Israelites},\lnALZ{}“Did I not deliver you from the Egyptians,\lebnote{Hebrew “from Egypt”}the Amorites,\lnAPA{}from the \textit{Ammonites},\lnAIT{}and from the Philistines?\lnGHO{}}%
\verse{And when the Sidonians, the Amalekites,\lnDVL{}and the Maonites\lebnote{Hebrew “Maonite”}oppressed you, you cried out to me, and I delivered you from their hand.}%
\verse{Yet you have abandoned me and served other gods. Therefore I will no longer deliver you.}%
\verse{Go and cry out to the gods whom you have chosen; let them deliver you in the time of your trouble.”}%
\verse{And the \textit{Israelites}\lnALZ{}said to Adonai, “We have sinned; \textit{do to us accordingly as you see fit};\lebnote{Literally “do to us according to the good in your eyes”}only please deliver us this day.”}%
\verse{So they removed the foreign gods from their midst and served Adonai; and \textit{he could no longer bear}\lebnote{Literally “his soul was short with”}the misery of Israel.}%
\verse{And the \textit{Ammonites}\lnAIT{}were summoned,\lebnote{Or “were called to arms”}and they camped in Gilead. And the \textit{Israelites}\lnALZ{}gathered and camped at Mizpah.}%
\verse{The people, the ones commanding Gilead, \textit{said to each other},\lebnote{Literally “said each to his neighbor”}“Who is the man that will begin to fight against the \textit{Ammonites}?\lnAIT{}He will be as head over all the inhabitants of Gilead.”}%
\end{biblechapter}

\begin{biblechapter} % Judges 11
\verseWithHeading{Jephthah}{Jephthah the Gileadite was a mighty warrior; he was the son of a prostitute, and \textit{Gilead was his father}.\lebnote{Literally “Gilead fathered Jephthah”}}%
\verse{Gilead’s wife also bore for him sons; and the sons of his wife grew up and drove Jephthah away, and they said to him, “You will not inherit the house of our father because you are the son of another woman.”}%
\verse{So Jephthah fled from the presence of his brothers, and he lived in the land of Tob. And \textit{outlaws}\lebnote{Literally “unprincipled/worthless men”}gathered around Jephthah and went with him.}%
\verse{After a time the \textit{Ammonites}\lnAIT{}made war with Israel.}%
\verse{When the \textit{Ammonites}\lnAIT{}made war with Israel, the elders of Gilead went to bring Jephthah from the land of Tob.}%
\verse{And they said to Jephthah, “Come and be our commander, so that we may make war against the \textit{Ammonites}.”\lnAIT{}}%
\verse{Jephthah said to the elders, “Did you not shun me and drive me out from the house of my father? Why do you come to me now when you have trouble?”}%
\verse{And the elders of Gilead said to Jephthah, “That being so, we have now returned to you, that you may go with us \textit{to fight}\lnGHX{}against the \textit{Ammonites}\lnAIT{}and become for us as head of all the inhabitants of Gilead.”}%
\verse{So Jephthah said to the elders of Gilead, “If you bring me back \textit{to fight}\lnGHX{}against the \textit{Ammonites},\lnAIT{}and Adonai gives them \textit{over to me},\lnGIB{}will I be your head?”}%
\verse{And the elders of Gilead said to Jephthah, “Adonai will be a \textit{witness}\lebnote{Literally “hearer”}between us; we will act according to your word.”}%
\verse{So Jephthah went with the elders of Gilead, and the people made him head and commander over them. And Jephthah spoke all his words before Adonai at Mizpah.}%
\verse{And Jephthah sent messengers to the king of the \textit{Ammonites},\lnAIT{}saying, “What is between you and me that you have come to me to make war against my land?”}%
\verse{And the king of the \textit{Ammonites}\lnAIT{}said to Jephthah’s messengers, “Because Israel took my land from the Arnon up to the Jabbok and the Jordan when they came up from Egypt; so then, restore it peacefully.”}%
\verse{Once again Jephthah sent messengers to the king of the \textit{Ammonites},\lnAIT{}}%
\verse{and he said to him, “Thus says Jephthah, ‘Israel did not take the land of Moab or the land of the \textit{Ammonites},\lnAIT{}}%
\verse{because when they came up from Egypt, Israel went through the wilderness to the \textit{Red Sea}\lnARX{}and went to Kadesh.}%
\verse{Israel sent messengers to the king of Edom, saying, “Please let us cross through your land,” but the king of Edom would not listen. And they also sent messengers to the king of Moab, but he was not willing. So Israel stayed in Kadesh.}%
\verse{Then they traveled through the wilderness, went around the land of Edom and Moab, and came to \textit{the east}\lebnote{Literally “from rise of sun”}side of the land of Moab, and they\lnATH{}encamped beyond the Arnon; and they did not go into the territory of Moab because the Arnon was the border of Moab.}%
\verse{Israel sent messengers to Sihon king of the Amorites,\lnAPA{}king of Heshbon; and Israel said to him, ‘Please let us cross through your land \textit{to our country}.’\lebnote{Literally “to our place”}}%
\verse{But Sihon did not trust Israel to cross through his territory, so Sihon gathered all his people and then encamped at Jahaz; and he made war with Israel.}%
\verse{And Adonai, the God of Israel, gave Sihon and all his people into the hand of Israel, and they defeated them; and Israel occupied all the land of the Amorites\lnAPA{}inhabiting that land.}%
\verse{They occupied all the territory of the Amorites\lnAPA{}from the Arnon up to the Jabbok, and from the wilderness up to the Jordan.}%
\verse{So then Adonai, the God of Israel, has driven out the Amorites\lnAPA{}from before his people Israel, and you want to possess it?}%
\verse{Do you not possess what Chemosh your god gave you to possess? Whoever Adonai our God has driven out before us, we will possess it.}%
\verse{So then, are you any better than Balak son of Zippor, king of Moab? Did he ever quarrel with Israel, or did he ever make war against them?}%
\verse{When Israel lived in Heshbon and its villages, and in Aroer and its villages, and in all the towns that are \textit{along the Arnon},\lebnote{Literally “on the hands of the Arnon”}for three hundred years,\lnDVF{}why did you not recover them at that time?}%
\verse{I have not sinned against you; but you are the one who is doing wrong by making war against me. \textit{Let Adonai judge}\lebnote{Literally “Let Adonai the judge, judge”}today between the \textit{Israelites}\lnALZ{}and the \textit{Ammonites}.”\lnAIT{}}%
\verse{But the king of the \textit{Ammonites}\lnAIT{}did not listen to the message that Jephthah sent to him.}%
\verseWithHeading{Jephthah Makes a Vow}{And the Spirit of Adonai came upon Jephthah, and he passed through Gilead and Manasseh. He passed through Mizpah of Gilead, and from Mizpah of Gilead he passed through to the \textit{Ammonites}.\lnAIT{}}%
\verse{And Jephthah made a vow to Adonai, and he said, “If indeed you will give the \textit{Ammonites}\lnAIT{}into my hand,}%
\verse{whatever\lebnote{Or “whoever”}comes out from the doors of my house to meet me when I return safely from the \textit{Ammonites}\lnAIT{}will be Adonai’s, and I will offer it as a burnt offering.”}%
\verse{And Jephthah crossed over to the \textit{Ammonites}\lnAIT{}to make war against them; and Adonai gave them into his hand.}%
\verse{And he defeated them with a very great blow, from Aroer as far as Minnith, twenty towns,\lebnote{Hebrew “town”}up to Abel Keramim. And the \textit{Ammonites}\lnAIT{}were subdued before the \textit{Israelites}.\lnALZ{}}%
\verse{Jephthah came to Mizpah, to his house, and behold his daughter came out to meet him with tambourines and dancing. She was his only child; he did not have a son or daughter except her.}%
\verse{And the moment he saw her, he tore his clothes and said, “Ah! My daughter, you have caused me to bow down, and you have become my trouble. \textit{I made an oath}\lebnote{Literally “I have opened wide my mouth”}to Adonai, and I cannot take it back.”}%
\verse{She said to him, “My father, \textit{you made an oath}\lebnote{Literally “you have opened wide your mouth”}to Adonai. Do to me according to what has gone out from your mouth, since Adonai gave vengeance to you against your enemies, the \textit{Ammonites}.”\lnAIT{}}%
\verse{And she said to her father, “Let this thing be done for me: grant me two months so that I may go wander\lebnote{Hebrew “down”}on the mountains and lament over my virginity, I and my companions.}%
\verse{And he said, “Go.” He sent her away for two months, and she went with her friends, and she lamented over her virginity on the mountains.}%
\verse{At the end of the two months she returned to her father, and he did to her according to his vow; and \textit{she did not sleep with a man}.\lebnote{Literally “she did not know a man”}And it became an annual custom in Israel}%
\verse{for the daughters of Israel to go and lament the daughter of Jephthah the Gileadite for forty days of the year.}%
\end{biblechapter}

\begin{biblechapter} % Judges 12
\verseWithHeading{Tribal Conflict Between Gilead and Ephraim}{The men of Ephraim were called to arms, and they crossed over to Zaphon and said to Jephthah, “Why did you cross over and make war against the \textit{Ammonites},\lnAIT{}and why did you not call us to go with you? We will burn down your house over you with fire.”}%
\verse{And Jephthah said to them, “I and my people were engaged in great conflict with the \textit{Ammonites};\lnAIT{}I called you, but you did not deliver me from their hand.}%
\verse{I saw that you would not deliver us; \textit{I risked my own life},\lebnote{Literally “I put my life in my hand”}and I crossed over to the \textit{Ammonites},\lnAIT{}and Adonai gave them into my hand. Why have you come up to me this day to fight against me?”}%
\verse{Jephthah gathered all the men of Gilead, and he made war with Ephraim; and the men of Gilead defeated Ephraim because they said, “You are fugitives of Ephraim, you Gileadites, in the midst of Ephraim and Manasseh.”}%
\verse{Then Gilead captured the fords of the Jordan from Ephraim, and whenever a fugitive of Ephraim said, “Let me cross over,” the men of Gilead said to him, “Are you an Ephraimite?” When he said, “No,”}%
\verse{they said to him, “Please say Shibboleth,” and if he said, “Sibboleth”—because he could not \textit{pronounce it}\lebnote{Literally “speak it”}correctly—they grabbed him and executed him at the fords of Jordan. At that time forty-two thousand from Ephraim fell.}%
\verse{Jephthah judged Israel six years. Then Jephthah the Gileadite died, and he was buried in one of the cities of Gilead.}%
\verseWithHeading{Ibzan, Elon, and Abdon}{After him Ibzan from Bethlehem judged Israel.}%
\verse{He had thirty sons. He gave his thirty daughters away in marriage outside his clan and brought in from outside thirty young women for his sons. He judged Israel for seven years.}%
\verse{Then\lnAAA{}Ibzan died and was buried in Bethlehem.}%
\verse{After him Elon the Zebulunite judged Israel, and he judged Israel ten years.}%
\verse{Then\lnAAA{}Elon the Zebulunite died and was buried in Aijalon in the land of Zebulun.}%
\verse{After him Abdon the son of Hillel the Pirathonite judged Israel.}%
\verse{He had forty sons and thirty grandsons that rode on seventy male donkeys. He judged Israel for eight years.}%
\verse{Then\lnAAA{}Abdon the son of Hillel the Pirathonite died and was buried in Pirathon, in the land of Ephraim in the hill country of the Amalekites.\lnDVL{}}%
\end{biblechapter}

\begin{biblechapter} % Judges 13
\verseWithHeading{Samson’s Parents}{And again, the \textit{Israelites}\lnALZ{}did evil in the eyes of Adonai, and Adonai gave them into the hand of the Philistines forty years.}%
\verse{There was a certain man from Zorah, from the tribe of the Danites, and his name was Manoah; his wife was infertile and did not bear children.}%
\verse{And an angel of Adonai appeared to the woman, and he said to her, “Behold, you are infertile and have not borne children, but you will conceive and bear a son.}%
\verse{So then, be careful and do not drink wine or strong drink, and do not eat anything unclean,}%
\verse{because you will conceive and bear a son. No razor \textit{will touch}\lebnote{Literally “will go upon”}his head, because the boy will be a \textit{Nazirite of God}\lnGJF{}\textit{from birth}.\lebnote{Literally “from the womb”}And it is he who will begin to deliver Israel from the hand of the Philistines.”}%
\verse{And the woman came and told her husband, saying, “A man of God came to me, and his appearance was like the appearance of an angel of God, very awesome. I did not ask him from where he came, and he did not tell me his name.}%
\verse{And he said to me, ‘Look, you will conceive and bear a son, so then, do not drink wine or strong drink, and do not eat anything unclean, for the boy will be a Nazirite of God from birth\lebnote{Or “the womb”}until the day of his death.’ ”}%
\verse{Then Manoah prayed to Adonai and said, “Excuse me, my Lord, please let the man of God whom you sent again come to us and teach us what we should do concerning the boy who will be born.”}%
\verse{And God listened to the voice of Manoah, and an angel of God came again to the woman; she was sitting in the field, but Manoah her husband was not with her.}%
\verse{The woman quickly ran and told her husband, and she said to him, “Look! The man who came to me the other day appeared to me.”}%
\verse{So Manoah got up and went after his wife, and he came to the man and said to him, “Are you the man that spoke to the woman?” And he said, “I am.”}%
\verse{And Manoah said to him, “Now \textit{when your words come true},\lebnote{Literally “when your words will come”}what will be the boy’s \textit{manner of life}\lebnote{Literally “measure”}and work?”}%
\verse{And the angel of Adonai said to Manoah, “Let the woman be attentive to all that I said.}%
\verse{She should not eat of anything that comes from the vine, or drink wine or strong drink, or eat anything unclean; she should keep all that I commanded.”}%
\verse{And Manoah said to the angel of Adonai, “\textit{Please stay},\lebnote{Literally “Please, let us hold you back”}and let us prepare a \textit{young goat}\lnGJG{}for you.”}%
\verse{The angel of Adonai said to Manoah, “If you keep me, I will not eat your food, but if you prepare a burnt offering for Adonai, you can offer it (for Manoah did not know that he was an angel of Adonai).”}%
\verse{And Manoah said to the angel of Adonai, “What is your name so that when your words come true we may honor you?”}%
\verse{But the angel of Adonai said to him, “Why do you ask my name? It is too wonderful.”}%
\verse{And Manoah took the \textit{young goat}\lnGJG{}and the grain offering, and he offered it to Adonai on the rock, to the one who performs miracles.\lebnote{Or “the one who works wonders”}And Manoah and his wife were watching.}%
\verse{And when the flame went up toward the heaven from the altar, the angel of Adonai went up in the flame of the altar to heaven while Manoah and his wife were watching. And they fell on their faces to the ground.}%
\verse{The angel of Adonai did not appear again to Manoah and his wife, and then Manoah knew that he was a messenger\lebnote{Or “angel”}of Adonai.}%
\verse{And Manoah said to his wife, “We will certainly die because we have seen God.”}%
\verse{But his wife said to him, “If Adonai wanted to kill us he would not have taken\lebnote{Or “accepted”}from our hand the burnt offering and the grain offering, or shown us all these things, or now announced to us things such as these.”}%
\verse{The woman bore a son, and she called him Samson; the boy grew big, and Adonai blessed him.}%
\verse{And the Spirit of Adonai began to stir him in the camp of Dan, between Zorah and Eshtaol.}%
\end{biblechapter}

\begin{biblechapter} % Judges 14
\verseWithHeading{Samson Marries}{And Samson went down to Timnah, and he saw a woman in Timnah from the daughters of the Philistines.}%
\verse{He went up and told his father and mother, and he said, “I saw a woman in Timnah from the daughters of the Philistines; so then, take her for me as a wife.”}%
\verse{But his father and mother said to him, “Is there not a wife among the daughters of your relatives, or among all our\lnDYT{}people, that you must take a wife from the uncircumcised Philistines?” But Samson said to his father, “Take her for me because \textit{she pleases me}.”\lebnote{Literally “she is right in my eyes”}}%
\verse{His father and mother did not know that this was from Adonai; he was seeking for an occasion against the Philistines. Now at that time the Philistines were ruling\lebnote{Or “having control”}in Israel.}%
\verse{And Samson and his father and mother went down to Timnah, and they came to the vineyards of Timnah, and suddenly a young lion came roaring to meet him.}%
\verse{And the Spirit of Adonai rushed upon him, and he tore the lion apart as one might tear apart a male kid goat (\textit{he was bare-handed}).\lebnote{Literally “there was nothing in his hand”}But he did not tell his father and mother what he had done.}%
\verse{Then he went down and talked to the woman, and \textit{she pleased Samson}.\lebnote{Literally “she did what was right in the eyes of Samson”}}%
\verse{And he returned after awhile \textit{to marry her},\lebnote{Literally “to take her”}and he turned aside to see the carcass of the lion, and there was a swarm of wild honey bees in the body of the lion, and honey.}%
\verse{He scraped it out into his hands, and he went on, eating it as he went. And he went to his father and mother and gave some to them, and they ate it. But he did not tell them that he had scraped the honey from the body of the lion.}%
\verse{His father went down to the woman, and Samson prepared there a feast, as young men were accustomed to doing this.}%
\verse{When they saw him, they took thirty companions, and they were with him.}%
\verse{And Samson said to them, “Let me tell you a riddle. If you can fully explain it to me within the seven days of the feast, and find it out, I will give to you thirty linen garments and thirty festal garments.}%
\verse{But if you are unable to explain it to me, you must give me thirty linen garments and thirty festal garments.” So they said to him, “Tell your riddle; let us hear it.”}%
\verse{He said to them, “From the eater came out food, 
From the strong came out sweet.” But they were unable to explain the riddle for three days.}%
\verse{When it was the fourth\lebnote{Hebrew “seventh”; other ancient translations have “fourth”}day, they said to Samson’s wife, “Entice your husband and tell us the riddle, or we will burn you and your father’s house with fire. Have you invited us to rob us?”}%
\verse{And Samson’s wife wept before him, and she said, “You must hate me; you do not love me. You told the riddle to \textit{my people},\lebnote{Literally “the sons/children of my people”}but you have not explained it to me.” He said to her, “I have not explained it to my father and mother. Why should I explain it to you?”}%
\verse{She wept before him the seven days of their feast; and it happened, because she nagged him, on the seventh day he explained it to her, and she told the riddle to \textit{her people}.\lebnote{Literally “the sons/children of her people”}}%
\verse{The men of the city said to him on the seventh day before the sun went down, “What is sweeter than honey? 
What is stronger than a lion?” And he said to them, “If you had not plowed with my heifer, 
you would not have found out my riddle.”}%
\verse{And the Spirit of Adonai rushed on him, and he went down to Ashkelon. He killed thirty men from them, and he took their belongings, and he gave festal garments to the ones that explained the riddle. \textit{He was angry},\lebnote{Literally “his nose was hot”}and he went up to his father’s house.}%
\verse{And Samson’s wife was given to his companion who was his best man.\lebnote{Or “friend”}}%
\end{biblechapter}

\begin{biblechapter} % Judges 15
\verseWithHeading{Samson Defeats the Philistines}{After a while, at the time of the wheat harvest, Samson visited his wife with a \textit{young goat}.\lnGJG{}He said, “I want to go to my wife’s private room.” But her father would not allow him to enter.}%
\verse{Her father said, “I really thought that you hated her, so I gave her to your companion. Is not her younger sister more beautiful than she? Please, \textit{take her instead}.”\lebnote{Literally “let her be in the place of her”}}%
\verse{And Samson said to them, “This time, as far as the Philistines are concerned, when I do something evil I am without blame.”}%
\verse{And Samson went and captured three hundred foxes, and he took torches. He turned them tail to tail, and he put one torch between two tails.}%
\verse{He set fire to the torches and let the foxes go into the standing grain of the Philistines, and he burned both the stacks\lebnote{Hebrew “stack”}of sheaves and the standing grain, up to the vineyards\lebnote{Hebrew “vineyard”}of olive groves.}%
\verse{And the Philistines said, “Who has done this?” And they said, “Samson the son-in-law of the Timnite, because he took his wife and gave her to his companion.” And the Philistines went up and burned her and her father with fire.}%
\verse{Samson said to them, “If you want to behave like this, I swear I will not rest unless I have taken revenge on you.”}%
\verse{And \textit{he gave them a thorough beating},\lebnote{Literally “he struck them hip and thigh with a great blow”}and he went down and stayed in the cleft of the rock of Etam.}%
\verse{Then the Philistines came up and encamped in Judah, and they overran Lehi.}%
\verse{And the men of Judah said, “Why have you come up against us?” And they said, “To bind Samson; to do to him just as he did to us.”}%
\verse{Then three thousand men from Judah went down to the cleft of the rock of Etam, and they said to Samson, “Do you not know that the Philistines are ruling over us? What is this that you have done to us?” And he said to them, “Just as they did to me, so I have done to them.”}%
\verse{They said to him, “We have come down to bind you and give you over into the hand of the Philistines.” And Samson said to them, “Swear to me that you will not attack me yourselves.”}%
\verse{They said to him, “No, we will only bind you and give you into their hand; we will certainly not kill you.” So they bound him with two new ropes, and they brought him up from the rock.}%
\verse{As he came up to Lehi, the Philistines came shouting to meet him; and the Spirit of Adonai rushed on him, and the ropes that were on his arms became like flax that has burned with fire, and his bindings melted from his hands.}%
\verse{And he found a fresh jawbone of a donkey; he reached down and took it and killed one thousand men with it.}%
\verse{And Samson said, “With the jawbone of the donkey, 
heap upon heap; 
with the jawbone of the donkey, 
I struck dead one thousand men.”}%
\verse{And it happened, when he finished speaking he threw the jawbone from his hand; and he called that place Ramath Lehi.\lebnote{That is, “Hill of the Jawbone”}}%
\verse{And he was very thirsty, and he called to Adonai and said, “You gave this great victory into the hand of your servant, but now I must die of thirst and fall into the hand of the uncircumcised?”}%
\verse{So God split the hollow place that is at Lehi, and water came out from it; and he drank, and his spirit returned, and he was revived. Thus he called its name \textit{The Spring of Ha-Qore},\lebnote{Literally “the spring of the one who called” or “En-hakkore”}which is at Lehi to this day.}%
\verse{And he judged Israel in the days of the Philistines twenty years.}%
\end{biblechapter}

\begin{biblechapter} % Judges 16
\verseWithHeading{Samson and Delilah}{Samson went down to Gaza; there he saw a prostitute and \textit{had sex with her}.\lebnote{Literally “went into her”}}%
\verse{The Gazites were told, “Samson has come here,” so they surrounded the place and lay in ambush for him all night at the city gate. They kept silent all night, saying, “We will wait until the morning light, and then we will kill him.”}%
\verse{But Samson lay until the middle of the night; he got up in the middle of the night and took hold of the doors of the city gate and the two door posts, tore them loose with the bar, put them on his shoulders, and carried them up to the top of the hill that is \textit{in front of}\lnGJK{}Hebron.}%
\verse{After this he fell in love with a woman in the wadi\lnEKD{}of Sorek, and her name was Delilah.}%
\verse{And the rulers of the Philistines came up to her and said, “Entice him and find out what makes his strength so great, and how we can overpower him, so that we may bind him up in order to subdue him; each of us will give you eleven hundred pieces of silver.}%
\verse{So Delilah said to Samson, “Please tell me what makes your strength so great, and with what can you be tied up to subdue you?”}%
\verse{Samson said to her, “If you tie me up with seven fresh bowstrings that are not dried up, I will become weak like everyone else.”}%
\verse{So the rulers of the Philistines brought up to her seven fresh bowstrings that were not dried up, and she tied him up with them.}%
\verse{The ambush was sitting in wait for her in an inner room. And she said to him, “The Philistines are upon you Samson!” And he snapped the bowstrings just as flax fiber snaps when it comes close to fire. And the secret of his strength remained unknown.}%
\verse{Delilah said to Samson, “Look, you have mocked me and told me lies. Please tell me how you can be bound.”}%
\verse{He said to her, “If they tie me tightly with new ropes that have not been used, I will become weak and be like everyone else.”}%
\verse{So Delilah took new ropes and tied him up with them, and she said to him, “The Philistines are upon you, Samson!” (The ambush was sitting in an inner room.) But he snapped them from his arms like thread.}%
\verse{And Delilah said to Samson, “Until now you have mocked me and told lies to me. Tell me how you can be bound.” And he said to her, “If you weave seven locks of my head with warp-threads.”\lebnote{Many modern translations include an additional phrase found in the Greek translation: “and fasten it with a pin, then I will become weak and be like everyone else. So while he slept, Delilah took the seven locks of his head and wove them”}}%
\verse{She fastened it with the pin and said to him, “The Philistines are upon you, Samson!” And Samson woke up from his sleep and tore loose the loom pin of the web and the warp-threads.\lebnote{Hebrew “warp-thread”}}%
\verse{And she said to him, “How can you say, ‘I love you,’ when your heart is not with me? You have mocked me these three times, and you have not told me how your strength is so great.”}%
\verse{And because she nagged him day after day with her words, and pestered him, \textit{his soul grew impatient to the point of death}.\lebnote{Literally “his inner self grew tired/impatient to death”}}%
\verse{So \textit{he confided everything to her},\lebnote{Literally “he told her all his heart”}and he said to her, “A razor \textit{has never touched}\lebnote{Literally “has never gone up”}my head, for I am a \textit{Nazirite of God}\lnGJF{}\textit{from birth}.\lebnote{Literally “from the womb of my mother”}If I am shaved my strength will leave me, and I will become weak, like everyone else.}%
\verse{Delilah realized that \textit{he had confided in her},\lebnote{Literally “he had told her all his heart”}so she sent and called the rulers of the Philistines, saying, “Come up one more time, for \textit{he has confided in me}.”\lebnote{Literally “he has told me all his heart”}And the rulers of the Philistines came up, and they brought the money \textit{with them}.\lebnote{Literally “in their hand”}}%
\verse{And she put him to sleep on her lap; then she called the men\lnGJN{}and shaved off seven locks of his head. Then she began to subdue him,\lebnote{Or “humiliate him”}and his strength went away from him.}%
\verse{And she said to him, “The Philistines are upon you, Samson!” And he woke up from his sleep and said, “I will go out just like every other time and shake myself free,” but he did not know that Adonai had left him.}%
\verse{And the Philistines seized him, gouged his eyes, and brought him to Gaza. They tied him up with bronze shackles, and he became a grinder \textit{in the prison}.\lebnote{Literally “in the house of the prisoners”}}%
\verse{But the hair of his head began to grow back after it had been shaved off.}%
\verse{The rulers of the Philistines had gathered to sacrifice a great sacrifice to Dagon their god and to rejoice. And they said, “Our god has given Samson our enemy into our hand.”}%
\verse{And the people saw him, and they praised their god, for they said, “Our god has given into our hand those who hate us, devastate our land, and have \textit{killed many of us}.”\lebnote{Literally “made numerous our slain”}}%
\verse{After awhile, when their hearts\lnFKK{}were merry, they said, “Call Samson and let him entertain us.” And they called Samson \textit{from the prison},\lebnote{Literally “from the house of the prisoner”}and \textit{he entertained them}.\lebnote{Literally “he made sport before them”}And they made him stand between the pillars.}%
\verse{Then Samson said to the servant who was holding him by his hand, “Position me so that I can touch the pillars on which the house\lnGJP{}rests, so I can lean on them.”}%
\verse{And the house\lnGJP{}was full of men and women, and all of the rulers of the Philistines were there—about three thousand men and women were on the roof watching the performance of Samson.}%
\verseWithHeading{Samson’s Revenge}{And Samson called to Adonai and said, “My Lord Adonai, remember me! Please give me strength this one time, O God, so that I can repay with one act of revenge to the Philistines for my eyes.”}%
\verse{And Samson reached out and held two of the middle pillars on which the house\lnGJP{}was resting, and he leaned on them, one on his right and one on his left.}%
\verse{And Samson said, “Let me die with the Philistines.” And he pushed\lebnote{Or “he caused to bend”}with all his strength, and the house\lnGJP{}fell on the rulers and all of the people who were with him. And the dead whom he killed in his death were more than those he killed in his life.}%
\verse{His brothers and \textit{his whole family}\lebnote{Literally “all the house of his father”}came down and picked him up; and they brought him up and buried him between Zorah and Eshtaol in the tomb of Manoah his father; he judged Israel twenty years.}%
\end{biblechapter}

\begin{biblechapter} % Judges 17
\verseWithHeading{Micah’s Idolatry}{There was a man from the hill country of Ephraim; his name was Micah.}%
\verse{And he said to his mother, “The eleven hundred pieces of silver that were taken from you, and about which you also pronounced a curse \textit{in my hearing},\lebnote{Literally “In my ears”}are with me; I took it.” And his mother said, “Blessed be my son by Adonai.”}%
\verse{He returned the eleven hundred pieces\lnGJT{}of silver to his mother, and his mother thought, “I will certainly consecrate to Adonai the pieces\lnGJT{}of silver from my hand for my son to make an idol of cast metal; now then, I will return them\lnDNH{}to you.”}%
\verse{When he returned the pieces\lnGJT{}of silver to his mother, his mother took two hundred pieces\lnGJT{}of silver, and she gave it to the smith, and he made it into an idol of cast metal; and it was in the house of Micah.}%
\verse{The man Micah had for himself \textit{a shrine},\lebnote{Literally “a house of god”}and he made an ephod and teraphim, and \textit{he appointed one of his sons}\lebnote{Literally “he filled the hand of one of his sons”}who became a priest for him.}%
\verse{In those days there was no king in Israel, and each one did what was right in his own eyes.}%
\verse{There was a young man from Bethlehem in Judah, from the clan of Judah; he was a Levite and was dwelling as a foreigner\lnGJY{}there.}%
\verse{And the man went from the town of Bethlehem in Judah to live as a foreigner\lnGJY{}wherever he could find a place. And he came to the hill country of Ephraim, to the house of Micah, to continue his journey.}%
\verse{And Micah said to him, “From where do you come?” And he said to him, “I am a Levite from Bethlehem in Judah; I am going to dwell as a foreigner\lnGJY{}wherever I can find a place.”}%
\verse{And Micah said to him, “Stay with me and be to me a father and a priest, and I will give to you ten pieces\lnGJT{}of silver a year, a set of clothes, and your food.” So the Levite went with him.}%
\verse{The Levite agreed to stay with the man; and the young man became as one of his sons.}%
\verse{So Micah \textit{appointed the Levite},\lebnote{Literally “filled the hand of the Levite”}and the young man became a priest for him; and he was in the house of Micah.}%
\verse{And Micah said, “Now I know Adonai will make me prosperous, because the Levite has become my priest.”}%
\end{biblechapter}

\begin{biblechapter} % Judges 18
\verseWithHeading{The Tribe of Dan Seeks Territory}{In those days there was no king in Israel. And in those days the tribe of the Danites was seeking territory for itself to live in, because until that day it had not been allotted territory among the tribes of Israel.}%
\verse{The descendants\lnGAW{}of Dan sent from the whole number of their clan five \textit{capable men}\lebnote{Literally “sons of physical strength”}from Zorah and Eshtaol to spy out the land and to explore it. And they said to them, “Go, explore the land.” And they went to the hill country of Ephraim, to the house of Micah, and they spent the night there.\lebnote{Or “they lodged there”}}%
\verse{While they were with the house of Micah, they recognized the voice of the young Levite, and they turned aside there and said to him, “Who brought you here? What are you doing in this place, and \textit{what is your business here}?”\lebnote{Literally “What is for you here?”}}%
\verse{And he said to them, “Micah did such and such for me and hired me, and I became his priest.”}%
\verse{And they said to him, “Please inquire of God that we may know whether our journey that we are going on will be successful.”}%
\verse{And the priest said to them, “Go in peace. Adonai is in front of you on the journey you want to go on.”}%
\verse{And the five men went and came to Laish, and they observed the people who were living according to the customs of the Sidonians, quiet and unsuspecting, and lacking nothing in the land, and possessing restraint.\lebnote{Or perhaps “prosperity”; the Hebrew of this word is uncertain}And they were far from the Sidonians and had no word with anyone.}%
\verse{They came to their relatives\lnDAP{}at Zorah and Eshtaol, and their relatives\lnDAP{}said to them, “What do you report?”}%
\verse{And they said to them, “Come, let us go up against them; for we have seen the land, and it is very good. Will you do nothing? Do not hesitate to go, to enter, to possess the land.}%
\verse{When you go you will come to an unsuspecting people, and the land \textit{is spread out on all sides};\lebnote{Literally “is wide of hands”}God has given a place into your hands where there is no lack of anything that is on the earth.”}%
\verse{Six hundred men from the clan of the Danites from Zorah and Eshtaol, armed with weapons of war, set out from there.}%
\verse{They went up and encamped at Kiriath Jearim in Judah. Therefore they called this place Camp of Dan\lebnote{That is “Mahaneh-dan”}to this day; it is west of Kiriath Jearim.}%
\verse{From there they crossed over to the hill country of Ephraim, and they came to the house of Micah.}%
\verse{And the five men that went out to spy out the land (that is, Laish) responded and said to their relatives,\lnDAP{}“Do you know that there are in these houses an ephod, teraphim, and an idol of cast metal? So then, consider what you must do.”}%
\verse{So they turned to that direction, and they came to the house of the young Levite, the house of Micah, and \textit{they greeted him}.\lebnote{Literally “they asked for him in peace”}}%
\verse{And six hundred men from the descendants\lnGAW{}of Dan, armed with their weapons of war, were standing at the entrance of the gate.}%
\verse{And the five men that went to spy out the land went up, and they entered there and took the carved divine image, ephod, teraphim, and the molten image. The priest was standing at the entrance of the gate with the six hundred men armed with the weapons of war.}%
\verse{When these went to Micah’s house, they took the divine carved image, ephod, the teraphim, and the molten image, and the priest asked them, “What are you doing?”}%
\verse{And they said to him, “Keep quiet! Put your hand on your mouth and come with us and be for us a father and a priest. Is it better being a priest for a house of one man or being a priest for a tribe and clan in Israel?”}%
\verse{\textit{The priest accepted the offer},\lebnote{Literally “It was good in the heart of the priest”}and he took the ephod, teraphim, and molten image and went along with the people.}%
\verse{And they turned\lebnote{Or “resumed their journey”}and went and put the little children, the livestock, and the valuable property in front of them.}%
\verse{When they were at a distance from the house, Micah and the men who were in the houses that were near the house of Micah cried out, and they overtook the descendants\lnGAW{}of Dan.}%
\verse{And they called to the descendants\lnGAW{}of Dan, who turned around to face them, and they said to Micah, “What is the matter with you that you assembled together?”}%
\verse{He said, “You took away my gods that I had made, and the priest, and then you go away. What is now left for me? How can you say to me, ‘What is the matter?’ ”}%
\verse{And the descendants\lnGAW{}of Dan said to him, “You should not let your voice be heard among us, so that \textit{ill-tempered men}\lebnote{Literally “men of bitter soul”}\textit{will not attack you},\lebnote{Literally “will not fall on you”}\textit{and take your life}\lebnote{Literally “and you will lose your life”}and the lives of your household.”}%
\verse{And the descendants\lnGAW{}of Dan went their way. When Micah saw that they were stronger than him, he turned to return to his house.}%
\verse{And they took what Micah had made, and his priest, and they came to Laish, to a quiet and unsuspecting people, and \textit{they put them to the sword}\lebnote{Literally “they put them to the mouth of the sword”}and burned the city with fire.}%
\verse{There was no deliverer, because it was far from Sidon, and \textit{they had had no dealings with anyone}.\lebnote{Literally “a thing was not for them with anyone”}It was in the valley that belonged to Beth-rehob, and they rebuilt\lebnote{Hebrew “built”}the city and lived in it.}%
\verse{And they called the name of the city Dan, after Dan their ancestor,\lnFCQ{}who was born to Israel; but the former name of the city was Laish.}%
\verse{And the descendants\lnGAW{}of Dan set up for themselves the carved divine image, and Jonathan son of Gershom, son of Manasseh,\lebnote{Other ancient versions read “Moses”}he and his sons were priests for the tribe of the Danites until the time of the captivity of the land.}%
\verse{So they set up for themselves the carved divine image that Micah had made, all the days that the house of God was in Shiloh.}%
\end{biblechapter}

\begin{biblechapter} % Judges 19
\verseWithHeading{The Concubine and the Levite}{In those days there was no king in Israel; there was a man, a Levite, who dwelled as a foreigner\lnGJY{}in the remote areas of the hill country of Ephraim. And he took for himself a concubine from Bethlehem in Judah.}%
\verse{But his concubine felt repugnance toward him,\lebnote{Other modern translations read “his concubine played the harlot against him”}and she left him and went to her father’s house, to Bethlehem in Judah; she was there some four months.}%
\verse{So her husband set out, and he went after her to speak \textit{tenderly to her},\lebnote{Literally “to her heart”}to bring her back. He took with him his servant and a pair of donkeys. And she brought him to her father’s house, and the father of the young woman saw him and was glad to meet him.}%
\verse{His father-in-law, the young woman’s father, urged him to stay with him three days; and they ate and drank, and they spent the night there.}%
\verse{On the fourth day, they rose early in the morning, and he prepared to go, but the father of the young woman said to his son-in-law, “\textit{Refresh yourself}\lebnote{Literally “Refresh your heart”}with a bit of food, and afterward you may go.”}%
\verse{So the two of them sat and ate and drank together, and the father of the young woman said to the man, “Please, agree to spend the night and \textit{enjoy yourself}.”\lnGKO{}}%
\verse{The man got up to go, but his father-in-law urged him, and he returned and spent the night there.}%
\verse{On the fifth day he rose early in the morning to go, and the father of the young woman said, “Please, \textit{enjoy yourself},”\lnGKO{}and they lingered until the day declined, and the two of them ate.}%
\verse{And the man got up to go—he, his concubine, and his servant—but his father-in-law, the father of the young woman, said to him, “Please, the day has worn on to evening; please, spend the night, the day has drawn to a close. Spend the night here and \textit{enjoy yourself}.\lnGKO{}You can rise early tomorrow for your journey and \textit{go to your home}.”\lebnote{Literally “go to your tent”}}%
\verse{But the man was not willing to spend the night, and he got up and went; and he arrived opposite Jebus (that is, Jerusalem). He had with him a pair of saddled donkeys and his concubine.}%
\verse{They were near Jebus, and \textit{the day was far spent},\lebnote{Literally “the day went down very”}and the servant said to his master, “Please, come, let us turn aside to this city of the Jebusites,\lnAPD{}and let us spend the night in it.”}%
\verse{But his master said to him, “We will not turn aside to the city of foreigners,\lnFNJ{}who are not from the \textit{Israelites};\lnALZ{}we will cross over up to Gibeah.”}%
\verse{And he said to his servant, “Come, let us approach one of these places; we will spend the night in Gibeah or in Ramah.”}%
\verse{So they crossed over and went their way, and the sun went down on them beside Gibeah, which belongs to Benjamin.}%
\verse{And they turned aside there to enter and to spend the night at Gibeah. And they went and sat in the open square of the city, but no one \textit{took them in to spend the night}.\lebnote{Literally “was receiving them to their house to spend the night”}}%
\verse{Then behold, an old man was coming from his work from the field in the evening, and the man was from the hill country of Ephraim, and he was dwelling as a foreigner\lnGJY{}in Gibeah. (The people of the place were descendants\lnGAW{}of Benjamin.)}%
\verse{And the old man raised his eyes and saw the traveler in the open square of the city, and he said, “Where are you going, and from where do you come?”}%
\verse{And he said to him, “We are crossing over from Bethlehem in Judah up to the remote areas of the hill country of Ephraim, where I am from. I went to Bethlehem in Judah, but now I am going to Adonai’s house,\lebnote{Or “my house,” according to the LXX and some modern translations (NASB, NRSV)}but no one \textit{took me in to spend the night}.\lebnote{Literally “was receiving me to their house”}}%
\verse{There is both straw and fodder for our donkeys, and also bread and wine for me, for your servant,\lebnote{That is, the concubine}and for the young man who is with your servant; there is no lack of anything.”}%
\verse{And the old man said, “Peace to you. I will take care of your needs; however, you must not spend the night in the open square.”}%
\verse{So he brought him to his house, and he fed the donkeys; they washed their feet, ate, and drank.}%
\verse{While \textit{they were enjoying themselves},\lebnote{Literally “their hearts were being good”}behold, the men of the city, \textit{the perverse lot},\lnGKW{}surrounded the house, pounding on the door. And they said to the old man, the owner of the house, “Bring out the man who came to your house so that \textit{we may have sex with him}.”\lebnote{Literally “we may know him”}}%
\verse{So the man, the owner of the house, went out to them and said to them, “No, my brothers, do not act wickedly; since this man has come into my house, do not do this disgraceful thing.}%
\verse{Here is my virgin daughter and his concubine. Please, let me bring them out; do violence to them,\lebnote{Or “rape them”}and do to them \textit{whatever you please}.\lebnote{Literally “the good in your eyes”}Do not do this disgraceful thing to this man.”}%
\verse{But the men were not willing to listen to him, and the man seized his concubine and brought her out to them; and they had intercourse with her, and they abused her all night until the morning; they let her go at the approach of dawn.}%
\verse{And the woman came as the morning appeared, and she fell at the entrance of the man’s house where her master was, until daylight.\lebnote{Hebrew “light”}}%
\verse{In the morning her master got up, and he opened the doors of the house and went out to go on his journey, and behold, his concubine was falling\lebnote{Or “spread out”}at the entrance of the house, with her hand on the threshold.}%
\verse{And he said to her, “Get up, let us go,” but there was no answer. So he put her on the donkey, and the man got up and went to his place.}%
\verse{When he entered his house he took a knife, and he grasped his concubine and cut her into twelve pieces; and he sent her throughout the whole territory of Israel.}%
\verse{All who saw it said, “Nothing like this has ever been since the \textit{Israelites}\lnALZ{}went up from the land of Egypt until this day. Take note of it, consider it, and speak up.”}%
\end{biblechapter}

\begin{biblechapter} % Judges 20
\verseWithHeading{The Punishment of Benjamin}{All the \textit{Israelites}\lnALZ{}went out, from Dan to Beersheba, including the land of Gilead, and they gathered as one body\lnGJN{}to Adonai at Mizpah.}%
\verse{And the leaders of all the people, all the tribes of Israel, presented themselves in the assembly of the people of God, four hundred thousand sword-bearing infantry.}%
\verse{(The descendants\lnGAW{}of Benjamin heard that the \textit{Israelites}\lnALZ{}had gone up to Mizpah.) And the \textit{Israelites}\lebnote{Literally “sons of Israel” or “children of “Israel”}said, “Tell us, how did this evil act occur?”}%
\verse{The Levite, the husband of the murdered woman, answered and said, “I came to Gibeah, which belongs to Benjamin, I and my concubine, to spend the night.}%
\verse{The lords of Gibeah rose up against me and surrounded the house at night. They intended to kill me, and they raped my concubine, and she died.}%
\verse{Then I grabbed my concubine and cut her into pieces, and I sent her throughout all the territory of Israel’s inheritance; for they committed a shameful and disgraceful thing in Israel.}%
\verse{Look, all you \textit{Israelites}!\lnALZ{}\textit{Make your decision}\lebnote{Literally “give your word”}and advice here.”}%
\verse{All the people got up as one body,\lnGJN{}saying, “Not one of us will go to his tent, or will any of us return to his house.}%
\verse{So then, this is what we will do to Gibeah: we will go up against her by lot.}%
\verse{We will take ten men of one hundred from all the tribes of Israel, and one hundred of one thousand, and one thousand of ten thousand, to bring provisions to the troops, to repay them when they come to Gibeah\lnGLE{}of Benjamin for all the disgraceful things\lnEPR{}they did in Israel.”}%
\verse{And all the men of Israel were gathered to the city as one body\lnGJN{}united.}%
\verse{Then the tribes of Israel sent men throughout all the tribes of Benjamin, saying, “What is this wickedness that happened among you?}%
\verse{So then, hand over the men, \textit{the perverse lot},\lnGKW{}who are in Gibeah, so that we may kill them and purge this wickedness from Israel.” But the descendants\lnGAW{}of Benjamin were not willing to listen to the voice of their relatives,\lnDAP{}the \textit{Israelites}.\lnALZ{}}%
\verse{And the descendants\lnGAW{}of Benjamin were gathered from the cities to Gibeah to go out for battle against the \textit{Israelites}.\lnALZ{}}%
\verse{From the cities the descendants\lnGAW{}of Benjamin were counted on that day twenty-six thousand \textit{sword-wielding men},\lnGLO{}besides the inhabitants of Gibeah, who were counted seven hundred well-trained\lebnote{Or “chosen”}men.}%
\verse{From all these troops were seven hundred well-trained\lnGLP{}men \textit{who were left-handed};\lebnote{Literally “bound by his right hand”}each one could sling with a stone at a hair and not miss.}%
\verse{And the men of Israel besides Benjamin were counted four hundred thousand \textit{sword-wielding men};\lnGLO{}\textit{all were warriors}.\lebnote{Literally “all were men of war”}}%
\verse{Then the \textit{Israelites}\lnALZ{}got up and went up to Bethel, and they inquired of God, saying, “Who will go up first for the battle against the descendants\lnGAW{}of Benjamin?” And Adonai said, “Judah will go first.”}%
\verse{And the \textit{Israelites}\lnALZ{}got up in the morning, and they encamped against Gibeah.}%
\verse{Then the men of Israel marched out for the battle against Benjamin, and the men of Israel took up positions against them for battle at Gibeah.}%
\verse{The descendants\lnGAW{}of Benjamin went out from Gibeah, and they struck down on that day twenty-two thousand men of Israel to the ground.}%
\verse{But the troops, the men of Israel, \textit{encouraged themselves},\lebnote{Literally “took heart”}and again they arranged their battle lines in the place where they had arranged themselves the first day.}%
\verse{The \textit{Israelites}\lnALZ{}went up and wept before Adonai until evening and inquired of Adonai, saying, “Should we again draw near from the battle against the descendants\lnGAW{}of Benjamin, our relatives?”\lnGLX{}And Adonai said, “Go up against them.”\lnDUT{}}%
\verse{And the \textit{Israelites}\lnALZ{}drew near to the descendants\lnGAW{}of Benjamin on the second day.}%
\verse{And Benjamin went out from Gibeah to meet them on the second day, and they struck down the \textit{Israelites}\lnALZ{}again, eighteen thousand men to the ground; all of these were \textit{sword-wielding}.\lebnote{Literally “drawers of sword”}}%
\verse{And all the \textit{Israelites}\lnALZ{}and all the troops went up and came to Bethel and wept; and they sat there before Adonai and fasted on that day until evening; and they offered burnt offerings and fellowship offerings before Adonai.}%
\verse{And the \textit{Israelites}\lnALZ{}inquired of Adonai (In those days the ark of the covenant of God was there,}%
\verse{and Phinehas son of Eleazar, son of Aaron, was standing \textit{before it}\lebnote{Literally “before presence of it/him”}to minister in those days), saying, “Should we go out once more to battle against the descendants\lnGAW{}of Benjamin our relatives,\lnGLX{}or should we stop?” And Adonai said, “Go up tomorrow; I will give them\lnDUT{}into your hand.”}%
\verse{So Israel stationed an ambush all around Gibeah.}%
\verse{And the \textit{Israelites}\lnALZ{}went up against the descendants\lnGAW{}of Benjamin on the third day, and they took up positions against Gibeah as before.}%
\verse{The descendants\lnGAW{}of Benjamin went out to meet the troops, and they lured them away from the city and began to inflict casualties on the troops as before, on the main road, one of which goes up to Bethel, the other to Gibeah; and in the field there were about thirty men of Israel.}%
\verse{And the descendants\lnGAW{}of Benjamin thought,\lnACZ{}“They are being defeated before us \textit{as previously},”\lebnote{Literally “on the first”}and the \textit{Israelites}\lnALZ{}said, “Let us flee and lure them\lnDUT{}away from the city to the main roads.”}%
\verse{And all the men of Israel got up from their places\lnGMO{}and took up positions in Baal Tamar; and the ambush of Israel charged from their places,\lnGMO{}from the vicinity of Gibeah.\lnGLE{}}%
\verse{Then ten thousand well-trained\lnGLP{}men from all Israel came out against Gibeah, and the battle became fierce;\lebnote{Or “the battle became heavy”}they did not know that disaster was about to \textit{close in}\lnGMS{}on them.}%
\verse{And Adonai defeated Benjamin in the presence of Israel, and the \textit{Israelites}\lnALZ{}destroyed on that day twenty-five thousand one hundred men of Benjamin, all of them \textit{sword-wielding}.\lebnote{Literally “drawing sword”}}%
\verse{The descendants\lnGAW{}of Benjamin saw that they were defeated, and the men of Israel gave ground\lebnote{Or “a place” or “space”}to Benjamin because they trusted the ambush that they had set against Gibeah.}%
\verse{And the ambush came quickly and advanced against Gibeah, and it \textit{put the whole city to the sword}.\lebnote{Literally “struck the whole city with the mouth of the sword”}}%
\verse{Now the appointed time for the men of Israel with the ambush was that they sent up for them a great amount of smoke from the city.}%
\verse{And the men of Israel turned in the battle, and Benjamin began to inflict casualties on about thirty men of Israel because they thought,\lnACZ{}“They will be completely defeated before us as in the first battle.”}%
\verse{And the cloud of smoke began to go up from the city in a column of smoke, and Benjamin turned backward, and behold, the whole city was going up \textit{in smoke}!\lebnote{Literally “to heaven”}}%
\verse{And the men of Israel turned, and the men of Benjamin were dismayed because they saw that disaster was \textit{closing in}\lnGMS{}on them.}%
\verse{And they retreated from before the men of Israel to the way of the wilderness, but the battle caught up to them, and those who came from the cities\lebnote{Other ancient translations “city”}destroyed them in the midst of them.}%
\verse{They surrounded Benjamin; they pursued them\lnDUT{}without rest\lebnote{Or “at Nohah,” a location}and trod them\lnDUT{}down opposite Gibeah to the east.}%
\verse{And eighteen thousand men from Benjamin fell, all of them \textit{able men}.\lnGMZ{}}%
\verse{And they turned and fled toward the wilderness, to the rock of Rimmon, but they cut down on the main roads five thousand men; and they pursued after them up to Gidom, and they struck down two thousand men.}%
\verse{So all of Benjamin who fell on that day were twenty-five thousand \textit{sword-wielding men};\lnGLO{}all of these were \textit{able men}.\lnGMZ{}}%
\verse{But six hundred turned and fled toward the wilderness, to the rock of Rimmon, and they remained at the rock of Rimmon for four months.}%
\verse{And the men of Israel returned to the descendants\lnGAW{}of Benjamin, and they put them to \textit{the edge of the sword},\lebnote{Literally the mouth of the sword”}both the inhabitants of city and the animals that were found; they also set on fire all the cities that they found.}%
\end{biblechapter}

\begin{biblechapter} % Judges 21
\verseWithHeading{A Decision Is Made About the Tribe of Benjamin}{The men of Israel had sworn at Mizpah, saying, “None of us will give his daughter to Benjamin as a wife.”}%
\verse{And the people of Bethel came and sat there until evening before God, and they lifted their voices and \textit{wept bitterly}.\lebnote{Literally “they cried a great weeping”}}%
\verse{And they said, “Why, Adonai, God of Israel, has it happened today in Israel that one tribe is lacking from Israel?”}%
\verse{On the next day the people rose early, and they built there an altar and sacrificed burnt offerings and fellowship offerings.}%
\verse{And the \textit{Israelites}\lnALZ{}said, “Who in the assembly has not come up from all the tribes of Israel to Adonai?” For a solemn oath was taken concerning whoever did not come up to Adonai at Mizpah, saying, “He will certainly be put to death.”}%
\verse{But the \textit{Israelites}\lnALZ{}had compassion for Benjamin, their relatives,\lebnote{Hebrew “his brother”}and they said, “Today one tribe is cut off from Israel.}%
\verse{What will we do for them—for the ones being left over—for wives? For we have sworn by Adonai not to give to them our daughters as wives.”}%
\verse{They asked, “Which one is there from the tribes of Israel who did not come up to Adonai at Mizpah?” And behold, no one came from Jabesh-gilead to the camp, to the assembly.}%
\verse{The people were counted, and no one was there from the inhabitants of Jabesh-gilead.}%
\verse{And the congregation sent there twelve thousand men from the troops, and they commanded them, saying, “Go, strike the inhabitants of Jabesh-gilead with \textit{the edge of the sword},\lnFNC{}and the women and children.}%
\verse{This is the thing you will do: \textit{you will destroy}\lebnote{Literally “you will devote to destruction”}every man and \textit{every woman who had sex with a man}.”\lebnote{Literally “every woman who knows the bed of a man”}}%
\verse{And they found among the inhabitants of Jabesh-gilead four hundred young virgins\lebnote{Hebrew “virgin”}who had not \textit{had sex with a man},\lebnote{Literally “known a man, as far as the bed of a male”}and they brought them to the camp at Shiloh, which is in the land of Canaan.}%
\verse{Then the congregation sent and spoke all this to the descendants\lnGAW{}of Benjamin who were at the rock of Rimmon, and \textit{they assured them they would not be hurt}.\lebnote{Literally “they proclaimed peace to them”}}%
\verse{And Benjamin returned at that time, and they gave to them the women whom they kept alive from Jabesh-gilead; but they were not enough for them.}%
\verse{And the people felt sorry for Benjamin because \textit{Adonai weakened the tribes of Israel}.\lebnote{Literally “Adonai make a breach with the tribes of Israel”}}%
\verse{So the elders of the congregation said, “What should we do for the remaining ones for wives, since the women from Benjamin have been destroyed?”}%
\verse{And they said, “There must be a remnant for Benjamin, so that a tribe will not be blotted out from Israel.}%
\verse{But we cannot give them wives from our daughters.” (For the \textit{Israelites}\lnALZ{}swore, saying, “Cursed be anyone who gives a wife to Benjamin.”)}%
\verse{And they said, “Look, the annual feast of Adonai is in Shiloh, which is to the north of Bethel, \textit{east}\lebnote{Literally “rising of the sun”}of the main road that goes up from Bethel to Shechem, and south of Lebonah.}%
\verse{They instructed the descendants\lnGAW{}of Benjamin, saying, “Go, lie in ambush in the vineyards,}%
\verse{and watch and look; when the daughters of Shiloh dance in the dances, come out from the vineyards and seize for yourselves a wife from the daughters of Shiloh, and go to the land of Benjamin.}%
\verse{And if their fathers or their brothers complain to us, we will say to them, ‘Allow us to have them, because we did not capture a wife for each man in the battle, and because you did not give them to them, \textit{or else}\lebnote{Literally “as the time”}you would have been guilty.’ ”}%
\verse{The descendants\lnGAW{}of Benjamin did likewise, and they took wives for each of them from the dancers whom they seized, and they went and returned to their territory, and they rebuilt the cities and they lived in them.}%
\verse{So the \textit{Israelites}\lnALZ{}dispersed from there at that time according to tribe and family; and they went out from there, each one to their own territory.}%
\verse{In those days there was no king in Israel; each one did what was right in his own eyes.}%
\end{biblechapter}

\flushcolsend
\biblebook{Ruth}

\begin{biblechapter} % Ruth 1
\verseWithHeading{Elimelech Takes His Family to Live in Moab} And it happened in the days \textit{when} \textit{the judges ruled}, there was a famine in the land, and a man from Bethlehem \textit{of} Judah went \textit{to reside} in the countryside of Moab—he and his wife and his two sons.
\verse And the name of the man \textit{was} Elimelech, and the name of his wife \textit{was} Naomi, and the name\textit{s} of his two sons \textit{were} Mahlon and Kilion. \textit{They were} Ephraimites from Bethlehem \textit{in} Judah. And they went \textit{to} the countryside of Moab and remained there.
\verse But Elimelech the husband of Naomi died and she was left behind \textit{with} \textit{her two sons}.
\verse And \textit{they took} for themselves Moabite wives. The name of the one \textit{was} Orpah and the name of the other \textit{was} Ruth. And they lived there about ten years.
\verse But \textit{both} Mahlon and Kilion died, and the woman was left without her two sons and without her husband.
\verseWithHeading{Naomi Returns with Ruth} And she got up, she and her daughters-in-law, and returned from the countryside of Moab, because she had heard in the countryside of Moab that Adonai had \textit{come to the aid of} his people to give food to them.
\verse So she set out from the place \textit{where she was} and her two daughters-in-law with her, and they went on the way to return to the land of Judah.
\verse But Naomi said to her two daughters-in-law, “Go, each \textit{of you} return to her mother’s house. May Adonai \textit{show kindness to you} \textit{just} as you did with the dead and with me.
\verse May Adonai \textit{grant that you} find a resting place, each \textit{in} the house of her husband.” And she kissed them, and they lifted up their voice\textit{s} and cried.
\verse And they said to her, “\textit{No,} we want to return with you to your people.”
\verse And Naomi said, “Return, my daughters. \textit{Why do you still want to go with me}? \textit{Are there} sons \textit{in my womb} that may be husbands for you?
\verse Turn back, my daughters! Go, for \textit{I am too old to have a husband}. If I should think there is hope for me, even if I should have a husband \textit{this} night, and even if I should bear sons,
\verse would you therefore wait until they were grown? Would you therefore \textit{refrain from marrying}? No, my daughters, for it is far more bitter to me than \textit{to} you. For the hand of Adonai has gone out against me.”
\verse And they lifted up their voices and cried again. And Orpah kissed her mother-in-law \textit{goodbye}, but Ruth clung to her.
\verse And she said, “Look, your sister-in-law has returned to her people and to her gods. Return after your sister-in-law \textit{too}.”
\verse But Ruth said, “\textit{Do not urge} me to leave you \textit{or} to return from \textit{following you}! For where you go, I will go, and where you lodge, I will lodge. Your people \textit{will be} my people and your God \textit{will be} my God.
\verse Where you die I will die, and there I will be buried. So may Adonai do to me, \textit{and even more, unless} death \textit{separates you and me}!”
\verse \textit{When Naomi} saw that she was determined to go with her, \textit{she said no more}.
\verse \textit{So} the two of them went until they came to Bethlehem. \textit{And when they came} to Bethlehem, all of the town was stirred because of them. And they said, “\textit{Is} this Naomi?”
\verse And she said to them, “You should not call me Naomi; call me Mara, for Shaddai has \textit{caused me to be very bitter}.
\verse I went \textit{away} full, but Adonai brought me back empty-handed! Why call me Naomi \textit{when Adonai has testified against me} and Shaddai has brought calamity upon me?”
\verse So Naomi returned, and Ruth the Moabite her daughter-in-law with her, returning from the countryside of Moab. And they came \textit{to} Bethlehem at \textit{the} beginning of \textit{the} harvest of barley.
\end{biblechapter}

\begin{biblechapter} % Ruth 2
\verseWithHeading{Ruth Meets Boaz} \textit{Now} Naomi \textit{had a relative of her husband}, \textit{a prominent rich man} from the clan of Elimelech, \textit{whose} name was Boaz.
\verse And Ruth the Moabite said to Naomi, “Please let me go \textit{to} the field and glean among the ears of grain after \textit{someone} in whose eyes I \textit{may} find favor.” And she said to her, “Go, my daughter.”
\verse So she went and came and gleaned in the field behind the reapers. And she happened \textit{by} chance \textit{upon} the tract of field \textit{belonging to} Boaz, who \textit{was} from the clan of Elimelech.
\verse And look, Boaz came from Bethlehem and said to the reapers, “\textit{May} Adonai \textit{be} with you.” And they said to him, “\textit{May} Adonai bless you.”
\verse And Boaz said to his servant \textit{in charge of the reapers}, “To whom \textit{does} this young woman \textit{belong}?”
\verse And \textit{the servant in charge of the reapers} said, “She \textit{is} a Moabite girl returning with Naomi from the countryside of Moab.
\verse And she said, ‘Please let me glean and let me gather among the sheaves behind the reapers.’ So she came and remained from the morning up to now. \textit{She is sitting for a little while in the house}.”
\verse And Boaz said to Ruth, “\textit{Listen carefully}, my daughter, go no longer to glean in another field. Moreover, do not leave from this one, but \textit{stay close} with my young women.
\verse Keep your eyes on the field that they reap and go after them. Have I not ordered the servants not to bother you? And if you get thirsty, you shall go to the containers and drink from where the servants have drawn.”
\verse And she fell on her face and bowed down to the ground and said to him, “Why have I found favor in your eyes by recognizing me—for I \textit{am} a foreigner?”
\verse And Boaz answered and said to her, “All that you have done for your mother-in-law after the death of your husband was fully told to me. \textit{How} you left your father and mother and the land of your birth, and you went to a people that you did not know \textit{before}.
\verse May Adonai reward your work and may a full reward be \textit{given to} you from Adonai, the God of Israel, under whose wings you came to take refuge.”
\verse And she said, “May I find favor in your eyes, my lord, for you have comforted me and have spoken \textit{kindly to your servant}, and I am not one of your servants.”
\verse And Boaz said to her \textit{at mealtime}, “Come here and eat from the bread and dip your morsel in the wine vinegar.” So she sat beside the gleaners, and he offered to her roasted grain. And she ate and was satisfied, and she had some left over.
\verse And she got up to glean, and Boaz instructed his servants saying, “Let her also glean between the sheaves and do not reproach her.
\verse And also pull out for her from your bundles and leave \textit{it} so that she may glean—and do not rebuke her.”
\verse So she gleaned in the field until the evening and she beat out what she had gleaned, and it was about an ephah \textit{of} barley.
\verse And she picked \textit{it} up and went \textit{to} the town. Her mother-in-law saw how much she had gleaned. And she took \textit{it} out and gave to her what she had left over \textit{after being satisfied}.
\verse And her mother-in-law said to her, “Where did you glean \textit{today} and where did you work? May he \textit{who} took notice of you be blessed.” And she told her mother-in-law \textit{with whom she had worked} and said, “The name of the man who I worked with today \textit{is} Boaz.”
\verse And Naomi said to her daughter-in-law, “\textit{May} he be blessed by Adonai, whose loyal love has not forsaken the living or the dead.” And Naomi said to her, “The man is a close relative for us, he \textit{is} one of our redeemers.”
\verse And Ruth the Moabite said, “Also, he said to me, ‘You shall stay close with the servants which are mine until they have finished all of the harvest which is mine.’ ”
\verse And Naomi said to Ruth her daughter-in-law, “\textit{It is} good, my daughter, that you go out with his maidservants so that you will not \textit{be bothered} in another field.”
\verse So she stayed close with the maidservants \textit{of} Boaz to glean until the end of the barley harvest and wheat harvest. And she lived with her mother-in-law.
\end{biblechapter}

\begin{biblechapter} % Ruth 3
\verseWithHeading{Ruth Meets Boaz at the Threshing Floor} Now Naomi her mother-in-law said to her, “My daughter, should I not seek for you security that \textit{things} may be good for you?
\verse So then, \textit{is} not Boaz our kinsman whose maidservants you were with? Look, he \textit{is} winnowing the barley at the threshing floor tonight.
\verse Wash, anoint yourself, put your clothing on, and go down to the threshing floor. Do not make yourself known to the man until he finishes eating and drinking.
\verse And when he lies, take notice of the place where he lies. And you shall go and \textit{uncover} his feet and lie \textit{down}, and he shall tell you what to do.”
\verse And she said to her, “I will do all that you say.”
\verse And she went down to the threshing floor and did all that her mother-in-law had instructed her.
\verse And Boaz ate and drank \textit{until} his heart was \textit{merry} \textit{and then} he came to lie at the end of the grain heap. And she came in quietly and \textit{uncovered} his feet and lay down.
\verse And it happened in the middle of the night the man was startled and he reached out and behold, a woman \textit{was} lying at his feet.
\verse And he said, “Who \textit{are} you?” And she said, “I \textit{am} Ruth, your servant. Spread \textit{your garment} over your servant because you \textit{are} a redeemer.”
\verse And he said, “You \textit{are} blessed by Adonai my daughter. You did better \textit{in this} last kindness than the first by not going after young men, whether poor or rich.
\verse And so then my daughter, do not be afraid. All that you ask I will do for you, for the entire assembly of my people knows that you \textit{are} a worthy woman.
\verse Now truly I \textit{am} a redeemer, but there is also a redeemer \textit{of a} closer relationship than me.
\verse \textit{Stay tonight,} and in the morning, if he wants to redeem you, good; but if he is not willing to redeem, \textit{then as Adonai lives,} I will redeem you. Lie down until the morning.”
\verse So she lay at his feet until morning and got up before \textit{anyone could recognize each other}. And he said, “It must not be known that \textit{you} came \textit{to} the threshing floor.”
\verse And he said, “Bring the cloak that \textit{is} on you and \textit{hold it out}.” And she held it out and he measured six \textit{measures of} barley and put it on her \textit{cloak}. Then she went \textit{into} the city.
\verse And she came to her mother-in-law, and she said, “\textit{How did it go for you,} my daughter?” And she told her all that the man did for her.
\verse And she said, “These six measures of barley he gave to me, for he said, ‘You shall not go empty-handed to your mother-in-law.’ ”
\verse And she said, “Wait, my daughter, until you know how the matter turns out, for the man will not rest but will settle the matter today.”
\end{biblechapter}

\begin{biblechapter} % Ruth 4
\verseWithHeading{Boaz Redeems Ruth} And Boaz had gone up \textit{to} the \textit{city} gate and sat there. And look, the redeemer of whom Boaz had spoken \textit{was} passing by. And he said, “Come over here to sit, \textit{friend}.” And he came over and sat.
\verse And he took ten men from the elders of the city and said, “Sit here.” And they sat.
\verse And he said to the redeemer, “Naomi, who returned from the countryside \textit{of} Moab, is selling the tract of land which \textit{was} for our brother Elimelech.
\verse And I thought \textit{I would tell you} and say, ‘\textit{Buy it in the presence of} those sitting and before the elders of my people,’ if you want to redeem \textit{it}, redeem \textit{it}. But if you do not want to redeem, tell me so that I may know, for there is no one except you to redeem \textit{it}, and I \textit{am} after you.” And he said, “I want to redeem \textit{it}.”
\verse And Boaz said, “On the day of your acquiring the field from the hand of Naomi, you also acquire Ruth the Moabite, the wife of the dead \textit{man}, \textit{in order} to raise up \textit{for} the name of the dead his inheritance.”
\verse And the redeemer said, “I am not able to redeem for myself, lest I ruin my inheritance. You redeem for yourself my kinsman-redemption, for I am not able to redeem \textit{it}.”
\verse (Now this \textit{was the custom in former times} in Israel concerning the kinsman-redemption and transfer of property: to confirm the matter, a man removed his sandal and gave \textit{it} to his fellow countryman. This \textit{was} the manner of attesting in Israel.)
\verse So the redeemer said to Boaz, “Acquire \textit{it} for yourself,” and he removed his sandal.
\verse And Boaz said to the elders and all of the people, “You \textit{are} witnesses today that I have acquired all that \textit{was} for Elimelech and that \textit{was} for Kilion and Mahlon from the hand of Naomi.
\verse And also Ruth the Moabite, the wife of Mahlon, I have acquired as a wife, to raise up the name of the dead over his inheritance, so that the name of the dead may not be cut off from his relatives and from the gate of his \textit{birth} place. You \textit{are} witnesses today.”
\verse And all of the people who \textit{were} at the gate and the elders said, “\textit{We are} witnesses. May Adonai make the woman coming into your house as Rachel and as Leah, who \textit{together} built the house of Israel. May you have strength in Ephrathah and bestow a name in Bethlehem.
\verse And may your house be like the house of Perez, whom Tamar bore to Judah from the offspring that Adonai will give to you from this young woman.”
\verseWithHeading{The Lineage of King David} So Boaz took Ruth and she became his wife. And he went \textit{in} to her, and Adonai \textit{enabled her to conceive}, and she bore a son.
\verse And the women said to Naomi, “Blessed be Adonai who today \textit{did not leave you without a redeemer}! And may his name be renowned in Israel!
\verse He shall be for you a restorer of life and a sustainer in your old age, for your daughter-in-law who loves you, who \textit{is} better for you than seven sons, has borne him.”
\verse And Naomi took the child and she put him on her bosom and became his nurse.
\verse And the women of the neighborhood gave him a name, saying, “A son has been born to Naomi.” And they called his name Obed. He \textit{was} the father of Jesse, the father of David.
\verse Now these are the descendants of Perez: Perez fathered Hezron,
\verse and Hezron fathered Ram, and Ram fathered Amminadab,
\verse and Amminadab fathered Nahshon, and Nahshon fathered Salmon,
\verse and Salmon fathered Boaz, and Boaz fathered Obed,
\verse and Obed fathered Jesse, and Jesse fathered David.
\end{biblechapter}

\flushcolsend

\end{document}
