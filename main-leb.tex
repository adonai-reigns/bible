\documentclass[twopage,twocolumn,10pt,openany]{memoir}


\setstocksize{9in}{6in}% 6in x 9in
\settrimmedsize{9in}{6in}{*}
\settrims{0in}{0in}


\quarkmarks


\usepackage{lipsum}
\usepackage{fixltx2e}
\usepackage{xspace}
\usepackage[usenames,dvipsnames,svgnames,table]{xcolor}
\usepackage{lettrine}
\usepackage{flushend}
\usepackage{fancyhdr}
\usepackage{hyperref}
\usepackage{microtype}
\usepackage[object=vectorian]{pgfornament}
\usepackage{fontspec}
\usepackage{ifpdf}
\usepackage{eso-pic}
\usepackage[british]{babel}
\usepackage{titlesec}
\usepackage{blindtext}
\usepackage{hyperref}
\usepackage{multicol}
\usepackage{xifthen}
\usepackage{csquotes}
\usepackage{ragged2e}
\usepackage{titletoc}
\usepackage{soul}
\usepackage{pdfpages}
\usepackage{graphicx}
\usepackage{enumitem}
\usepackage{scrextend}
\usepackage[para]{footmisc}
\usepackage{fixfoot}
\usepackage{chngcntr}
\usepackage{setspace}
\usepackage{contour}
\usepackage{etex}
\usepackage{hyphenat}



\setmainfont{\frankrhuel}[
  % Files
  Path      = fonts/frankrhuel/ ,
  % Fonts
  UprightFont     = FrankRuehlCLM-Medium.ttf ,
  UprightFeatures = { SmallCapsFont = FrankRuehlCLM-Medium.ttf, } ,
  BoldFont        = FrankRuehlCLM-Bold.ttf,
  BoldFeatures    = { SmallCapsFont = FrankRuehlCLM-Bold.ttf } ,
  ItalicFont      = FrankRuehlCLM-MediumOblique.ttf ,
  BoldItalicFont  = FrankRuehlCLM-BoldOblique.ttf ,
  % Features
  Numbers         = OldStyle,
  WordSpace       = 0.65
 ]


% old-style numbers don't look great as drop-caps
\newfontfamily{\lettrinefont}{Bodoni12_ITC}[
  % Files
  Path      = fonts/numbers/ ,
  % Fonts
  UprightFont     = *-Regular.ttf ,
  UprightFeatures = { SmallCapsFont = *-Regular.ttf } ,
  BoldFont        = *-Regular.ttf ,
  BoldFeatures    = { SmallCapsFont = *-Regular.ttf } ,
  ItalicFont      = *-Regular.ttf ,
  BoldItalicFont  = *-Regular.ttf,
  % Features
  Numbers         = NewStyle,
  WordSpace       = 0.6]


\newfontfamily{\spoken}{Sans}[
  % Files
  Path      = fonts/sans/ ,
  % Fonts
  UprightFont     = *-Regular.ttf ,
  UprightFeatures = { SmallCapsFont = *-Regular.ttf } ,
  BoldFont        = *-Bold.ttf ,
  BoldFeatures    = { SmallCapsFont = *-Bold.ttf } ,
  ItalicFont      = *-Italic.ttf ,
  BoldItalicFont  = *-Italic.ttf ,
  % Features
  Numbers         = OldStyle,
  WordSpace       = 0.65,
  LetterSpace     = -2]

\newfontfamily{\versetext}{FrankRuehlCLM}[
  % Files
  Path      = fonts/frankrhuel/ ,
  % Fonts
  UprightFont     = *-Medium.ttf ,
  UprightFeatures = { SmallCapsFont = *-Medium.ttf, } ,
  BoldFont        = FrankRuehlCLM-Bold.ttf,
  BoldFeatures    = { SmallCapsFont = *-Bold.ttf } ,
  ItalicFont      = *-MediumOblique.ttf ,
  BoldItalicFont  = *-BoldOblique.ttf ,
  % Features
  Numbers         = OldStyle,
  WordSpace       = 0.6,
  LetterSpace     = -2
 ]


% old-style numbers don't look great as drop-caps
\newfontfamily{\headings}{AlegreyaSC}[
  % Files
  Path      = fonts/alegreya-sc/ ,
  % Fonts
  UprightFont     = *-Regular.ttf ,
  UprightFeatures = { SmallCapsFont = *-Regular.ttf } ,
  BoldFont        = *-Bold.ttf ,
  BoldFeatures    = { SmallCapsFont = *-Bold.ttf } ,
  ItalicFont      = *-Italic.ttf ,
  BoldItalicFont  = *-BoldItalic.ttf]

\setlength{\parskip}{\baselineskip}%

% book names font family and huge
\titleformat{\chapter}[display]{\Huge\bfseries\headings\centering}{\chaptertitlename\ \thechapter}{20pt}{\Huge}
\titlespacing*{\chapter}{0pt}{0pt}{30pt}
\titleformat{\section}[display]{\Huge\headings\centering}{\sectiontitlename\ \thesection}{20pt}{\Huge}
\titlespacing*{\section}{0pt}{0pt}{0pt}
\titleformat{\subsection}[display]{\Large\headings}{\subsectiontitlename\ \thesubsection}{20pt}{\Large}
\titlespacing*{\subsection}{0pt}{0pt}{0pt}
\pagestyle{fancy}
\fancyhf{}


% the page number in the bottom centre of each page
\cfoot{\thepage}

% set the footnote font style
\renewcommand{\footnotelayout}{\vfill\frankrhuel}
\renewcommand{\footnoterule}{\vfill\kern -3pt \hrule width 0.4\columnwidth \kern 2.6pt}
\newcommand*{\multfootsep}{\textsuperscript{\normalfont,}}


% cause the footnotes to be listed inline rather than on new lines
\footglue=.25em plus.15em minus.15em

\renewcommand{\headrule}{\hbox to\headwidth{%
    \color{white}\leaders\hrule height 0 \hfill}}


\hypersetup{colorlinks=true, linkcolor=black, urlcolor=blue, urlbordercolor={0 1 1}}

\newcommand{\framesize}{\textwidth}
\setlength{\headwidth}{\textwidth}
\setlength{\columnseprule}{0pt}

% space below the page header
\setheaderspaces{*}{0mm}{*}
\clubpenalty10000
\widowpenalty10000


%\newlength{\versespacing}
%\setlength{\versespacing}{0pt}
%\newcommand{\versespace}{\hspace{\versespacing}}
\frenchspacing

\AtBeginDocument{%
  \begingroup\tiny%
  \global\footnotesep=0.7\baselineskip%
  \global\footnotebaselineskip\baselineskip%
  \endgroup
}

\newcommand\AtPageUpperRight[1]{\AtPageUpperLeft{%
 \put(\LenToUnit{\paperwidth},\LenToUnit{0\paperheight}){#1}%
 }}%
\newcommand\AtPageLowerRight[1]{\AtPageLowerLeft{%
 \put(\LenToUnit{\paperwidth},\LenToUnit{0\paperheight}){#1}%
 }}%

\sodef\vh{}{-0.3pt}{2.5pt plus0pt minus 1pt}{1pt plus0.15pt minus0.1pt}
\sodef\vt{}{-.03em}{0.2em plus0.35em minus 0.09em}{1em plus.1em minus.1em}
\sodef\fnm{}{.1em}{0.1em plus0.1em minus 0.09em}{0.1em plus.1em minus.09em}
\sodef\fnt{}{-0.04em}{0pt}{0pt}
\sodef\soulVerseNum{}{.1em}{0.1em plus0.1em minus 0.09em}{0.1em plus.1em minus.09em}

\newcommand{\sectionstyle}{\bfseries\raggedright\fontsize{10pt}{9.75pt}\selectfont\headings}
\setsecheadstyle{\sectionstyle}
\setbeforesecskip{0pt}
\setaftersecskip{0pt}

\newcommand{\subsectionstyle}{\bfseries\itshape\raggedright\fontsize{10pt}{9.75pt}\selectfont\headings}
\setsubsecheadstyle{\subsectionstyle}
\setbeforesubsecskip{0pt}
\setaftersubsecskip{0pt}

\makeatletter
\patchcmd{\@footnotetext}{\footnotesize}{\tiny}{}{}

\iffalse
% color printing
\definecolor{JesusRed}{rgb}{0.68, 0.08, 0.08}
\definecolor{HighlightGray}{rgb}{0.94, 0.94, 0.94}
\definecolor{VerseUlColor}{rgb}{1, 1, 1}
\definecolor{VerseNumColor}{rgb}{0.75, 0.75, 0.75}
\definecolor{fnTextColor}{rgb}{0.53, 0.53, 0.53}
\definecolor{fnBorderColor}{rgb}{0.95, 0.95, 0.95}
\definecolor{fnBgColor}{rgb}{.98, .98, .98}
\definecolor{ChapterRed}{rgb}{1, 0, 0}
\fi

% b&w printing
\definecolor{JesusRed}{rgb}{0.28, 0.28, 0.28}
\definecolor{HighlightGray}{rgb}{0.94, 0.94, 0.94}
\definecolor{VerseUlColor}{rgb}{1, 1, 1}
\definecolor{VerseNumColor}{rgb}{0.75, 0.75, 0.75}
\definecolor{fnTextColor}{rgb}{0.53, 0.53, 0.53}
\definecolor{fnBorderColor}{rgb}{0.95, 0.95, 0.95}
\definecolor{fnBgColor}{rgb}{.98, .98, .98}
\definecolor{ChapterRed}{rgb}{.333, .333, .333}


%\definecolor{VerseNumColor}{rgb}{0.41, 0.41, 0.41}
%\definecolor{VerseUlColor}{rgb}{0.85, 0.85, 0.85}
\newcommand\versecolor{black}
\newcommand\chapnumcolor{ChapterRed}
\newlength{\biblechapskip}
  \setlength{\biblechapskip}{-10pt plus .33em minus .2em}
\newcounter{biblechapter}
\newcounter{bibleverse}[biblechapter]
\renewcommand\chaptername{Book}
\newcommand{\customsection}[2][]{%
  \ifthenelse{\isempty{#1}}%
    % use the section title name for the TOC contents line
    {\section*{#2}\addcontentsline{toc}{part}{\headings{#2}}}
    % permit an override value to be used in TOC
    {\section*{#2}\addcontentsline{toc}{part}{\headings{#1}}}
}
\newcommand{\biblebook}[2][]{%
  \setcounter{biblechapter}{0}
  \ifthenelse{\isempty{#1}}%
    % use the section title name for the TOC contents line
    {\gdef\currbook{#2}\chapter*{#2}\addcontentsline{toc}{chapter}{\headings{#2}}}
    % permit an override value to be used in TOC
    {\gdef\currbook{#2}\chapter*{#2}\addcontentsline{toc}{chapter}{\headings{#1}}}
}

% our custom macro for footnotes
\newcommand{\lebnote}[1]{%
  %\footnote{\label{bibleverse}#1}
  \in@{:|NP|:}{#1}% ignore notes with a setting of "no print"
  \ifin@{}\else{\footnote{#1}}%
  \fi
}






\newsavebox{\fnmarkboxbox}
\newcommand{\fnmarkbox}[1]{%
  \colorbox{fnBgColor}{%
    \sbox{\fnmarkboxbox}{#1}%
    \setlength{\fboxsep}{0.8pt}% don't add space
    \setlength{\fboxrule}{0.08pt}%
    \color{fnBorderColor}%
    \fbox{\usebox{\fnmarkboxbox}}%
  }%
}



\providecommand\lnsuperscript[1]{%
  \setlength{\fboxsep}{0pt}%
  \setlength{\fboxrule}{0pt}%
  \fontsize{6.5pt}{5pt}\selectfont%
  \raisebox{3pt\relax}{\fnmarkbox{\textcolor{fnTextColor}{#1}}}%
  \fontsize{10.45pt}{10.45pt}\selectfont%
}



% 20200217 - https://tex.stackexchange.com/questions/26693/change-the-color-of-footnote-marker-in-latex
\renewcommand\@makefnmark{\lnsuperscript{\@thefnmark}}%

\long\def\@makefntext#1{%\noindent\fontsize{7pt}{7pt}\selectfont\bfseries\fnm{[\@thefnmark] }\normalfont#1%
  \setlength{\fboxsep}{0pt}\setlength{\fboxrule}{0pt}\noindent\fontsize{7pt}{-7pt}\selectfont\fnmarkbox{\textcolor{fnTextColor}{\@thefnmark}}\hspace{0.8pt}#1%
}

\newcount\biblechap@svdopt
\newenvironment{biblechapter}[1][\thebiblechapter]
  {\biblechap@svdopt=#1
  \ifnum\c@biblechapter=\biblechap@svdopt\else
    \advance\biblechap@svdopt by -1\fi
  \setcounter{biblechapter}{\the\biblechap@svdopt}
  \stepcounter{biblechapter}
  \setbeforesecskip{0}\setbeforesubsecskip{0}
  \lettrine[lines=2,lhang=0,findent=-1pt,nindent=0pt,loversize=0,lraise=0.09]{\hspace{-4.5pt}\lettrinefont\fontsize{24pt}{10pt}\selectfont\color{\chapnumcolor}\,\thebiblechapter\,}{}\ignorespaces}
  {\par\vspace{\biblechapskip}\setbeforesecskip{0ex}\setbeforesubsecskip{0ex}}

\newcommand{\@showversenum}{\ifnum\c@bibleverse=1\else{%
  ~\color{VerseNumColor}{%
    \bfseries%
    \fontsize{10.4pt}{10.5pt}\selectfont%
    \sbox0{$\thebibleverse$}% 
    \raisebox{0pt\relax}{\thebibleverse}%
    \normalfont%
    \hspace{1.8pt}%
  }%
}\fi\ignorespaces}

\newcommand{\VerseText}{\versetext\fontsize{10.4pt}{10.7pt}\selectfont\color{black}}
\newcommand{\@verse}{\VerseText\stepcounter{bibleverse}\markright{{\scshape\currbook} \thebiblechapter:\thebibleverse}\hspace{0pt}}%
\renewcommand{\verse}{\@verse\@showversenum\hspace{0pt}}

\newcommand{\verseWithHeading}[1]{%
  \@verse%
  \ifnum\c@bibleverse=1{\sectionstyle\vh{#1}\newline}\else{\vspace{3.8pt}\newline\sectionstyle\vh{#1}\newline}\fi\@showversenum}
\newcommand{\verseWithSubheading}[1]{%
  \@verse%
  \ifnum\c@bibleverse=1{\subsectionstyle\vh{#1}\newline}\else\vspace{3.8pt}\newline{\subsectionstyle\nohyphens{\vh{#1}}}\newline\fi\@showversenum}
\newcommand{\innerVerseHeading}[1]{%
  \vspace{3.8pt}\newline{\subsectionstyle\vh{#1}}\newline}

\newcommand*{\JesusWords}[1]{%
% b&w printing
\spoken\fontsize{9.1pt}{10.7pt}\selectfont\color{JesusRed}\textit{#1}\VerseText%
%color printing
%\spoken\fontsize{9.1pt}{10.7pt}\selectfont\color{JesusRed}#1\VerseText%
}

\newcommand{\dictionaryentry}[3]{\setlength{\parindent}{0em}\setlength{\parskip}{0pt}\markboth{}{}\textbf{#1}\ {(#2)}:\ {#3}\smallbreak}

\counterwithin*{footnote}{biblechapter}

\makeatother


%\newcommand{\startornaments}{\AddToShipoutPictureBG{%
%  \checkoddpage%
%  \ifoddpage%
%   \AtPageUpperRight{\put(-100,-55){\pgfornament[width=1.75cm,symmetry=h,color=black]{195}}}%
%   \AtPageLowerRight{\put(-100,55){\pgfornament[width=1.75cm,symmetry=v,color=black]{194}}}%
% \else%
%   \AtPageUpperLeft{\put(10,-55){\pgfornament[width=1.75cm,symmetry=h,color=black]{194}}}%
%   \AtPageLowerLeft{\put(10,55){\pgfornament[width=1.75cm,symmetry=v,color=black]{195}}}
% \fi}}

\newcommand{\stopornaments}{\ClearShipoutPictureBG}

\newcommand{\LORD}{\textsc{\headings{Lord}}\xspace}
\newcommand{\LORDs}{\textsc{\headings{Lord's}}\xspace}



%\setlength{\footnotesep}{24pt}%
%\setlength{\skip\footins}{-24pt plus 3pt minus 3pt} % for example



\begin{document}

\frontmatter

\setlrmarginsandblock{0.5125in}{0.375in}{*}
\setulmarginsandblock{0.5in}{0.75in}{1}

\checkandfixthelayout 



\begin{titlingpage}

% a tribute to the cause for this publication.
\begin{center}
\includegraphics[width=126mm,height=192mm,keepaspectratio]{./leb/content/pictures/the_good_shepherd_by_poundcakery_d5hiu3j_deviantart.com.jpg}
%\includepdf{leb/content/pictures/the_good_shepherd_by_poundcakery_d5hiu3j_deviantart.com.jpg}
\end{center}
\end{titlingpage}








\begin{titlingpage}
\includepdf[noautoscale=true, scale=1]{leb/content/pictures/title-page-holy-bible.png}
\end{titlingpage}

\onecolumn
\renewcommand{\contentsname}{\headings{Books of The Holy Bible}}

% toc font styles
\titlecontents{part}[3mm]{\normalsize\bfseries}{PART\space\thecontentslabel:\space\MakeUppercase}{}{\normalfont\dotfill\makebox[12mm][l]{\thecontentspage}}
\titlecontents{chapter}[3mm]{\vspace{-12pt}\normalsize}{}{}{\dotfill\makebox[12mm][l]{\thecontentspage}}


% some help from % https://tex.stackexchange.com/questions/389319/making-the-second-column-of-the-table-of-contents-clear-the-page-header
\makeatletter  
\singlespacing
\begingroup % start a TeX group
\color{colortoc}% or whatever color you wish to use
\chapter*{\contentsname
  \@mkboth{%
    \MakeUppercase\contentsname}{\MakeUppercase\contentsname}}
% If you want to turn off columns being forced to equal height, use the starred version \begin{multicols*}{2}.
\begin{multicols}{2}
\@starttoc{toc}
\end{multicols}
\endgroup   % end of TeX group
\makeatother

\twocolumn

% a blank page on the back of the page - lext page on a new right-sided page
\newpage
\thispagestyle{empty}
\mbox{}
\biblebook[Preface]{The Lexham English Bible\\\small Fourth Edition}



\begin{multicols}{2}

\begin{itemize}
    \item Editors
    \item W. Hall Harris III
    \item Elliot Ritzema
    \item Rick Brannan
    \item Douglas Mangum
    \item John Dunham
    \item Jeffrey A. Reimer
    \item Micah Wierenga
    \item Translators
    \item W. Hall Harris III
    \item Michael S. Heiser
    \item Jeremy Penner
    \item David M. Fouts
    \item Eugene E. Carpenter
    \item Gordon H. Johnston
    \item H. Daniel Zacharias
    \item William D. Barrick
    \item Michael A. Grisanti
    \item Chip McDaniel
    \item Israel Loken
    \item Ken M. Penner
    \item Dorian G. Coover-Cox
    \item Amy L. Pfeister
    \item Logos Bible Software, 2012
\end{itemize}

\end{multicols}


The Lexham English Bible, Fourth Edition\\
Copyright 2010, 2012 Logos Bible Software\\
Logos Bible Software, 1313 Commercial St., Bellingham, WA 98225\\
http://www.logos.com












\clearpage
\mainmatter
\addcontentsline{toc}{part}{The Old Testament}%
\onecolumn
\begin{centering}
\includepdf[noautoscale=true, scale=1]{leb/content/pictures/title-page-old-testament.png}
\end{centering}
\twocolumn

% the book name, chapter and verse number at the top of the page
\fancyhead[RO,LE]{\textbf{\headings{\Large \rightmark}}}
\fancyhfoffset[RO]{1pt}

\setcolsepandrule{0.1875in}{0pt}
\setlength\textwidth{5.1125in}
\setlength\columnwidth{2.4625in}
\setlength\columnsep{0.1875in}
\checkandfixthelayout 



% a blank page on the back of the page - next page on a new right-sided page
\clearpage
\newpage
\thispagestyle{empty}
\mbox{}
\endpage
\newpage



\endpage
\newpage

% Declare our footnote values for use with the fixfoot package
\DeclareFixedFootnote{\lnAAA}{Or “And”}%
\DeclareFixedFootnote{\lnAAB}{Or “expanse”}%
\DeclareFixedFootnote{\lnAAH}{Or “the sky”}%
\DeclareFixedFootnote{\lnAAI}{“which its seed is in it”}%
\DeclareFixedFootnote{\lnAAJ}{Or “their”}%
\DeclareFixedFootnote{\lnAAN}{Or “light sources”}%
\DeclareFixedFootnote{\lnAAU}{Or “light source”}%
\DeclareFixedFootnote{\lnAAV}{“as the authority of”}%
\DeclareFixedFootnote{\lnABE}{“animals of the earth/land”}%
\DeclareFixedFootnote{\lnABI}{Or “as”}%
\DeclareFixedFootnote{\lnABR}{Or “the earth”}%
\DeclareFixedFootnote{\lnABS}{Or “family records”}%
\DeclareFixedFootnote{\lnABV}{Or “and”}%
\DeclareFixedFootnote{\lnABW}{“The man” indicates the noun is singular and occurs with the definite article}%
\DeclareFixedFootnote{\lnACE}{“as his opposite”}%
\DeclareFixedFootnote{\lnACX}{Or “said”}%
\DeclareFixedFootnote{\lnADE}{The noun lacks the definite article and is taken as a proper noun in this context}%
\DeclareFixedFootnote{\lnADQ}{“On the day”}%
\DeclareFixedFootnote{\lnADR}{Or “humankind”}%
\DeclareFixedFootnote{\lnAEG}{Or “its”}%
\DeclareFixedFootnote{\lnAEK}{“seven, seven”}%
\DeclareFixedFootnote{\lnAEO}{“two, two”}%
\DeclareFixedFootnote{\lnAEY}{Or “the”}%
\DeclareFixedFootnote{\lnAFK}{Or “sons”}%
\DeclareFixedFootnote{\lnAFM}{“in your going”}%
\DeclareFixedFootnote{\lnAGE}{In context, the enemy}%
\DeclareFixedFootnote{\lnAGI}{“brother”}%
\DeclareFixedFootnote{\lnAGK}{“was”}%
\DeclareFixedFootnote{\lnAGM}{Some nation hostile to Abram’s descendants (Egypt in later history)}%
\DeclareFixedFootnote{\lnAGP}{Or “fathers”}%
\DeclareFixedFootnote{\lnAGQ}{“cut”}%
\DeclareFixedFootnote{\lnAGT}{That is, Hagar}%
\DeclareFixedFootnote{\lnAGY}{“with very very”}%
\DeclareFixedFootnote{\lnAHA}{Or “between”}%
\DeclareFixedFootnote{\lnAHB}{“those born of house and acquisition of money from every son of a foreigner”}%
\DeclareFixedFootnote{\lnAHD}{Or “be”}%
\DeclareFixedFootnote{\lnAHI}{“a son of cattle”}%
\DeclareFixedFootnote{\lnAHK}{Adonai}%
\DeclareFixedFootnote{\lnAHL}{“at the time of life”}%
\DeclareFixedFootnote{\lnAHP}{“heavy”}%
\DeclareFixedFootnote{\lnAHR}{Or “Perhaps”}%
\DeclareFixedFootnote{\lnAHV}{Hebrew idiom for sexual intercourse (cp. Gen 4:1)}%
\DeclareFixedFootnote{\lnAHY}{“that we might preserve offspring from our father”}%
\DeclareFixedFootnote{\lnAIA}{“sons/children of Ammon”}%
\DeclareFixedFootnote{\lnAIB}{Or “voice”}%
\DeclareFixedFootnote{\lnAIF}{Or “that”}%
\DeclareFixedFootnote{\lnAIG}{Or “sons of Heth”}%
\DeclareFixedFootnote{\lnAIH}{“ownership of a grave”}%
\DeclareFixedFootnote{\lnAIK}{Or “give”}%
\DeclareFixedFootnote{\lnAIQ}{“went up”}%
\DeclareFixedFootnote{\lnAIV}{“to go after”}%
\DeclareFixedFootnote{\lnAIZ}{“went after”}%
\DeclareFixedFootnote{\lnAJD}{Or “upon the face of”}%
\DeclareFixedFootnote{\lnAJH}{“a son of sixty years”}%
\DeclareFixedFootnote{\lnAJJ}{“as the day”}%
\DeclareFixedFootnote{\lnAJL}{Or “called”}%
\DeclareFixedFootnote{\lnAJP}{That is, Esau}%
\DeclareFixedFootnote{\lnAJQ}{“are you this one?”}%
\DeclareFixedFootnote{\lnAJS}{Or “daughters of the Hittites”}%
\DeclareFixedFootnote{\lnAJV}{Or “it”}%
\DeclareFixedFootnote{\lnAJW}{“before the eyes of”}%
\DeclareFixedFootnote{\lnAJX}{That is, Jacob}%
\DeclareFixedFootnote{\lnAJZ}{“watch to yourself”}%
\DeclareFixedFootnote{\lnAKF}{Hebrew for “the heap of witness”}%
\DeclareFixedFootnote{\lnAKH}{Or “food, bread”}%
\DeclareFixedFootnote{\lnAKK}{That is, the man}%
\DeclareFixedFootnote{\lnAKM}{“sons/children of Israel”}%
\DeclareFixedFootnote{\lnAKP}{“There is to me abundance”}%
\DeclareFixedFootnote{\lnAKS}{Or “foot”}%
\DeclareFixedFootnote{\lnAKW}{“they added still to hate him”}%
\DeclareFixedFootnote{\lnAKY}{That is, Judah}%
\DeclareFixedFootnote{\lnAKZ}{“beautiful of appearance and healthy of flesh”}%
\DeclareFixedFootnote{\lnALA}{“Poor of appearance and thin of flesh”}%
\DeclareFixedFootnote{\lnALE}{That is, Pharaoh}%
\DeclareFixedFootnote{\lnALI}{Or “inner parts”}%
\DeclareFixedFootnote{\lnALK}{“a son of thirty years”}%
\DeclareFixedFootnote{\lnALU}{Or “son”}%
\DeclareFixedFootnote{\lnALX}{“house”}%
\DeclareFixedFootnote{\lnAMA}{“burdens” or “burdensome labor”}%
\DeclareFixedFootnote{\lnAMD}{“houses”}%
\DeclareFixedFootnote{\lnAMI}{Hebrew “Canaanite”}%
\DeclareFixedFootnote{\lnAMJ}{Hebrew “Hittite”}%
\DeclareFixedFootnote{\lnAMK}{Hebrew “Amorite”}%
\DeclareFixedFootnote{\lnAML}{Hebrew “Perizzite”}%
\DeclareFixedFootnote{\lnAMM}{Hebrew “Hivite”}%
\DeclareFixedFootnote{\lnANB}{“and he/it will be”}%
\DeclareFixedFootnote{\lnAND}{“look” or “behold”}%
\DeclareFixedFootnote{\lnANM}{“yesterday three days ago”}%
\DeclareFixedFootnote{\lnAOA}{“the house of their fathers”}%
\DeclareFixedFootnote{\lnAOB}{Or “generations”}%
\DeclareFixedFootnote{\lnAOC}{“seven and thirty and hundred of year”}%
\DeclareFixedFootnote{\lnAOF}{“fathers”}%
\DeclareFixedFootnote{\lnAOS}{“look”}%
\DeclareFixedFootnote{\lnAOZ}{“sea of reed”}%
\DeclareFixedFootnote{\lnAPE}{“a man for the mouth of his eating”}%
\DeclareFixedFootnote{\lnAPF}{“between the evenings”}%
\DeclareFixedFootnote{\lnAPG}{Or “summons,” “convocation”}%
\DeclareFixedFootnote{\lnAPI}{“and it will be”}%
\DeclareFixedFootnote{\lnAPJ}{“service” or “work”}%
\DeclareFixedFootnote{\lnAPW}{Hebrew “human”}%
\DeclareFixedFootnote{\lnAPX}{Hebrew “animal”}%
\DeclareFixedFootnote{\lnAQG}{Hebrew “he”}%
\DeclareFixedFootnote{\lnAQS}{Or “power”}%
\DeclareFixedFootnote{\lnAQW}{Hebrew “wonder”}%
\DeclareFixedFootnote{\lnAQX}{Hebrew “horse”}%
\DeclareFixedFootnote{\lnAQY}{Hebrew “chariot”}%
\DeclareFixedFootnote{\lnARF}{“at/in his hearing”}%
\DeclareFixedFootnote{\lnARR}{“men”}%
\DeclareFixedFootnote{\lnARS}{Plural}%
\DeclareFixedFootnote{\lnARW}{“and it would be”}%
\DeclareFixedFootnote{\lnARX}{“And it was”}%
\DeclareFixedFootnote{\lnASG}{Or “many” or “the many”}%
\DeclareFixedFootnote{\lnASW}{“from end from this”}%
\DeclareFixedFootnote{\lnATA}{“a woman to her sister”}%
\DeclareFixedFootnote{\lnATE}{“from this”}%
\DeclareFixedFootnote{\lnATG}{“from this and from this”}%
\DeclareFixedFootnote{\lnATH}{Or “boards” or “planks”}%
\DeclareFixedFootnote{\lnATI}{“standing”}%
\DeclareFixedFootnote{\lnATJ}{Or “board” or “plank”}%
\DeclareFixedFootnote{\lnATL}{“hands”}%
\DeclareFixedFootnote{\lnATU}{“one”}%
\DeclareFixedFootnote{\lnAUB}{Or “westward,” literally “seaward,” toward the Mediterranean Sea}%
\DeclareFixedFootnote{\lnAUP}{“holy place of the holiness”}%
\DeclareFixedFootnote{\lnAUR}{“from it”}%
\DeclareFixedFootnote{\lnAUS}{Or “side,” referring to the span on one side of the courtyard’s entry}%
\DeclareFixedFootnote{\lnAUZ}{Or “garments of holiness”}%
\DeclareFixedFootnote{\lnAVJ}{“lifted up”}%
\DeclareFixedFootnote{\lnAVM}{“sons of a year”}%
\DeclareFixedFootnote{\lnAVT}{“in the morning in the morning”}%
\DeclareFixedFootnote{\lnAVU}{“between the two evenings”}%
\DeclareFixedFootnote{\lnAVW}{“all of the going over to the being counted”}%
\DeclareFixedFootnote{\lnAVY}{“a son of twenty years”}%
\DeclareFixedFootnote{\lnAWC}{Or “recipe” or “formula”}%
\DeclareFixedFootnote{\lnAXK}{“by the hand of”}%
\DeclareFixedFootnote{\lnAXP}{“one to one”}%
\DeclareFixedFootnote{\lnAZW}{Or “vessels” or “utensils” or “contents”}%
\DeclareFixedFootnote{\lnBAC}{Or “herd”}%
\DeclareFixedFootnote{\lnBAD}{The Hebrew term refers collectively to both sheep and goats (small livestock animals)}%
\DeclareFixedFootnote{\lnBAF}{“to the face of”}%
\DeclareFixedFootnote{\lnBAG}{Or “And he”}%
\DeclareFixedFootnote{\lnBAN}{Or “entrails”}%
\DeclareFixedFootnote{\lnBAV}{Or “And the priest”}%
\DeclareFixedFootnote{\lnBAZ}{Or “a soul”}%
\DeclareFixedFootnote{\lnBBB}{“a holiness of holinesses”}%
\DeclareFixedFootnote{\lnBBC}{Hebrew “of”}%
\DeclareFixedFootnote{\lnBBD}{Hebrew “the”}%
\DeclareFixedFootnote{\lnBBH}{Or “And he shall present”}%
\DeclareFixedFootnote{\lnBBK}{Or “and the two kidneys”}%
\DeclareFixedFootnote{\lnBCC}{“sinned”}%
\DeclareFixedFootnote{\lnBCD}{“a bull, a son of cattle”}%
\DeclareFixedFootnote{\lnBCE}{Or “And he shall bring”}%
\DeclareFixedFootnote{\lnBCF}{“from”}%
\DeclareFixedFootnote{\lnBCS}{Or “from”}%
\DeclareFixedFootnote{\lnBCU}{Indicated by the context}%
\DeclareFixedFootnote{\lnBCV}{“as that”}%
\DeclareFixedFootnote{\lnBDB}{Or “And he shall place”}%
\DeclareFixedFootnote{\lnBDE}{Indicated by context}%
\DeclareFixedFootnote{\lnBDW}{Or “for”}%
\DeclareFixedFootnote{\lnBED}{“it is concealed from him”}%
\DeclareFixedFootnote{\lnBEE}{“for” or “to” (see HALOT 510)}%
\DeclareFixedFootnote{\lnBEM}{“for one of”}%
\DeclareFixedFootnote{\lnBER}{Hebrew “from”}%
\DeclareFixedFootnote{\lnBES}{“sons of dove” or “children of dove”}%
\DeclareFixedFootnote{\lnBFT}{“in”}%
\DeclareFixedFootnote{\lnBFU}{“in accordance with deception”}%
\DeclareFixedFootnote{\lnBGB}{“to the faces of”}%
\DeclareFixedFootnote{\lnBGG}{Antecedent for this 3fs suffix is “fire” (“altar” is ms)}%
\DeclareFixedFootnote{\lnBGQ}{Singular masculine}%
\DeclareFixedFootnote{\lnBHF}{“for him it shall be” or “it will become his”}%
\DeclareFixedFootnote{\lnBHK}{Implied by v. 12}%
\DeclareFixedFootnote{\lnBHP}{Or “soul”}%
\DeclareFixedFootnote{\lnBHQ}{Or “the soul”}%
\DeclareFixedFootnote{\lnBIQ}{Or “of”}%
\DeclareFixedFootnote{\lnBIW}{Or “he tied”}%
\DeclareFixedFootnote{\lnBIY}{Or “he put”}%
\DeclareFixedFootnote{\lnBJK}{Or “to”}%
\DeclareFixedFootnote{\lnBLF}{An adjective in masculine plural to modify both animals; see NET}%
\DeclareFixedFootnote{\lnBLI}{Singular}%
\DeclareFixedFootnote{\lnBLW}{“do” or “make”}%
\DeclareFixedFootnote{\lnBMS}{“from to the faces of”}%
\DeclareFixedFootnote{\lnBOM}{Implied by context}%
\DeclareFixedFootnote{\lnBOO}{“from their dead body”}%
\DeclareFixedFootnote{\lnBPD}{Or “daughter”}%
\DeclareFixedFootnote{\lnBPF}{“a son of his year”}%
\DeclareFixedFootnote{\lnBPN}{“man”}%
\DeclareFixedFootnote{\lnBPX}{The direct object is supplied from context in the English translation}%
\DeclareFixedFootnote{\lnBQH}{“living flesh”}%
\DeclareFixedFootnote{\lnBQO}{“white red”}%
\DeclareFixedFootnote{\lnBSI}{Perhaps better translated “mold” rather than “skin disease”}%
\DeclareFixedFootnote{\lnBSR}{“he”}%
\DeclareFixedFootnote{\lnBTH}{“wood of cedar”}%
\DeclareFixedFootnote{\lnBTL}{“the crimson thread of the worm”}%
\DeclareFixedFootnote{\lnBUN}{“his hand can produce”}%
\DeclareFixedFootnote{\lnBVK}{“after thus”}%
\DeclareFixedFootnote{\lnBVP}{“to from an outside place of the city”}%
\DeclareFixedFootnote{\lnBVT}{See HALOT 862 s.v. 3.b}%
\DeclareFixedFootnote{\lnBWB}{Implied by the context; or “plaster”}%
\DeclareFixedFootnote{\lnBWJ}{That is, the priest}%
\DeclareFixedFootnote{\lnBWZ}{“a man a man”}%
\DeclareFixedFootnote{\lnBXA}{Or “bathe”}%
\DeclareFixedFootnote{\lnBYR}{“from the interior/inside of”}%
\DeclareFixedFootnote{\lnBYY}{Hebrew “to”}%
\DeclareFixedFootnote{\lnBZN}{“a statute of eternity” or “a statute of long duration”}%
\DeclareFixedFootnote{\lnCAA}{“to”}%
\DeclareFixedFootnote{\lnCAH}{Or “Because”}%
\DeclareFixedFootnote{\lnCAI}{“Unto thus”}%
\DeclareFixedFootnote{\lnCBA}{“do”}%
\DeclareFixedFootnote{\lnCBD}{Or “keep”}%
\DeclareFixedFootnote{\lnCBG}{“approach”}%
\DeclareFixedFootnote{\lnCBH}{The verb לקח is used to speak of marrying (“taking a wife”)}%
\DeclareFixedFootnote{\lnCBS}{“to the faces of you”}%
\DeclareFixedFootnote{\lnCGS}{“the name of my holiness”}%
\DeclareFixedFootnote{\lnCHL}{“from your faces”}%
\DeclareFixedFootnote{\lnCHR}{“take”}%
\DeclareFixedFootnote{\lnCIY}{“for acceptance”}%
\DeclareFixedFootnote{\lnCIZ}{Or “among”}%
\DeclareFixedFootnote{\lnCJQ}{“all work of labor”}%
\DeclareFixedFootnote{\lnCJW}{“from the next day of”}%
\DeclareFixedFootnote{\lnCJX}{“a son of its year”}%
\DeclareFixedFootnote{\lnCJY}{Supplied by context}%
\DeclareFixedFootnote{\lnCKB}{“the exactly of this day”}%
\DeclareFixedFootnote{\lnCLN}{“from to alone”}%
\DeclareFixedFootnote{\lnCMP}{“gives”}%
\DeclareFixedFootnote{\lnCNB}{Singular throughout this verse}%
\DeclareFixedFootnote{\lnCNL}{Plural throughout this verse}%
\DeclareFixedFootnote{\lnCNS}{“a man his brother”}%
\DeclareFixedFootnote{\lnCOC}{“with confidence”}%
\DeclareFixedFootnote{\lnCOX}{Meaning derived from context; an alternative translation could be “shall not be released” (cf. NKJV, NRSV, ESV, NJPS) or “shall not revert” (NASB, NET)}%
\DeclareFixedFootnote{\lnCPM}{Or “brother”}%
\DeclareFixedFootnote{\lnCPV}{“to you”}%
\DeclareFixedFootnote{\lnCQN}{Or “children”}%
\DeclareFixedFootnote{\lnCQS}{Or “brothers”}%
\DeclareFixedFootnote{\lnCRJ}{“to your faces”}%
\DeclareFixedFootnote{\lnCRX}{Emphatic personal pronoun}%
\DeclareFixedFootnote{\lnCSU}{Or “and up to”}%
\DeclareFixedFootnote{\lnCSZ}{“a son of five years”}%
\DeclareFixedFootnote{\lnCTP}{“between good and between bad”}%
\DeclareFixedFootnote{\lnCUO}{Or “fathers’ ”}%
\DeclareFixedFootnote{\lnCVO}{Or “counted,” or “summoned,” or “enrolled”}%
\DeclareFixedFootnote{\lnCVU}{Or “count,” or “summon,” or “enroll”}%
\DeclareFixedFootnote{\lnCVX}{Other modern translations read “tabernacle of the covenant”}%
\DeclareFixedFootnote{\lnCWF}{“the ones counted of them,” or “the ones mustered of them”}%
\DeclareFixedFootnote{\lnCWG}{“the ones counted of him,” or “the ones mustered of him”}%
\DeclareFixedFootnote{\lnCWV}{“before the face of Adonai”}%
\DeclareFixedFootnote{\lnCXD}{“the son of a month”}%
\DeclareFixedFootnote{\lnCXF}{“mouth of Adonai”}%
\DeclareFixedFootnote{\lnCXI}{Hebrew “Kohathite”}%
\DeclareFixedFootnote{\lnCXO}{“the mouth of Adonai”}%
\DeclareFixedFootnote{\lnCYC}{“mouth”}%
\DeclareFixedFootnote{\lnCYH}{“a son of fifty years”}%
\DeclareFixedFootnote{\lnCYJ}{“the hide of a sea cow”}%
\DeclareFixedFootnote{\lnCYX}{Hebrew “Gershonite”}%
\DeclareFixedFootnote{\lnCZB}{“the mouth”}%
\DeclareFixedFootnote{\lnCZE}{“in the hand”}%
\DeclareFixedFootnote{\lnCZU}{Or “through Moses”}%
\DeclareFixedFootnote{\lnDAT}{Hebrew “his”}%
\DeclareFixedFootnote{\lnDAW}{“to an outside place of the camp”}%
\DeclareFixedFootnote{\lnDBF}{“before the face of”}%
\DeclareFixedFootnote{\lnDBH}{Or “waste away”}%
\DeclareFixedFootnote{\lnDBM}{That is, “the Nazirite”}%
\DeclareFixedFootnote{\lnDBS}{“the head of his separation”}%
\DeclareFixedFootnote{\lnDBV}{Hebrew “it”}%
\DeclareFixedFootnote{\lnDCB}{“according to the mouth of their work”}%
\DeclareFixedFootnote{\lnDCF}{“in the presence of”}%
\DeclareFixedFootnote{\lnDCG}{“the two of them”}%
\DeclareFixedFootnote{\lnDCH}{“the son of its year”}%
\DeclareFixedFootnote{\lnDCJ}{Hebrew “shekel”}%
\DeclareFixedFootnote{\lnDCR}{“one dish of ten gold”}%
\DeclareFixedFootnote{\lnDFM}{“in the presence of Adonai”}%
\DeclareFixedFootnote{\lnDFO}{“Aaron will wave”}%
\DeclareFixedFootnote{\lnDGP}{“by a life of a person”}%
\DeclareFixedFootnote{\lnDHC}{“it was there”}%
\DeclareFixedFootnote{\lnDIA}{“his nose became hot”}%
\DeclareFixedFootnote{\lnDIC}{Hebrew “its”}%
\DeclareFixedFootnote{\lnDID}{Hebrew “him/it”}%
\DeclareFixedFootnote{\lnDIG}{Hebrew “the graves of greediness”}%
\DeclareFixedFootnote{\lnDIK}{An arid region south of the Judean hills}%
\DeclareFixedFootnote{\lnDIM}{Or “wadi”}%
\DeclareFixedFootnote{\lnDIP}{Hebrew “day”}%
\DeclareFixedFootnote{\lnDJA}{Hebrew “him”}%
\DeclareFixedFootnote{\lnDJI}{“declaration of”}%
\DeclareFixedFootnote{\lnDJK}{Hebrew “year”}%
\DeclareFixedFootnote{\lnDJQ}{Hebrew “Amalekite”}%
\DeclareFixedFootnote{\lnDJW}{Hebrew “for your generations”}%
\DeclareFixedFootnote{\lnDKO}{Hebrew “We will not come up!”}%
\DeclareFixedFootnote{\lnDLA}{Or “the statute”}%
\DeclareFixedFootnote{\lnDLR}{“you will raise up”}%
\DeclareFixedFootnote{\lnDMG}{Hebrew “me”}%
\DeclareFixedFootnote{\lnDMO}{Hebrew “And it will happen”}%
\DeclareFixedFootnote{\lnDMU}{“sons of Ammon”}%
\DeclareFixedFootnote{\lnDMX}{Hebrew “her daughters;” other modern versions translate “its villages”}%
\DeclareFixedFootnote{\lnDNH}{That is, the Euphrates}%
\DeclareFixedFootnote{\lnDNI}{“the eye of the land”}%
\DeclareFixedFootnote{\lnDNS}{Hebrew “He”}%
\DeclareFixedFootnote{\lnDNX}{That is, Israel}%
\DeclareFixedFootnote{\lnDNZ}{Often translated “the Almighty”}%
\DeclareFixedFootnote{\lnADC}{“seed”}%
\DeclareFixedFootnote{\lnDOQ}{Or “the father’s house” or “the ancestor’s house”}%
\DeclareFixedFootnote{\lnDOX}{Hebrew “Reubenite”}%
\DeclareFixedFootnote{\lnDPA}{Hebrew “Zerahite”}%
\DeclareFixedFootnote{\lnDPY}{Hebrew “His”}%
\DeclareFixedFootnote{\lnDQG}{Or “meeting”}%
\DeclareFixedFootnote{\lnDQK}{“his word”}%
\DeclareFixedFootnote{\lnDQU}{“you will not do work of labor”}%
\DeclareFixedFootnote{\lnDRU}{“who has known the bed of a male”}%
\DeclareFixedFootnote{\lnDSD}{Hebrew “person”}%
\DeclareFixedFootnote{\lnDST}{“Adonai’s nose became hot”}%
\DeclareFixedFootnote{\lnDUH}{That is, the Dead Sea}%
\DeclareFixedFootnote{\lnDUI}{That is, the Mediterranean}%
\DeclareFixedFootnote{\lnDUO}{Hebrew “Gadite”}%
\DeclareFixedFootnote{\lnDWA}{“in the beyond of”}%
\DeclareFixedFootnote{\lnDWC}{Or “dwelling,” and in this context “reigning”}%
\DeclareFixedFootnote{\lnDWF}{Hebrew \textit{torah}}%
\DeclareFixedFootnote{\lnDWG}{“to say”}%
\DeclareFixedFootnote{\lnDWJ}{Or “peoples”}%
\DeclareFixedFootnote{\lnDWR}{“a word”}%
\DeclareFixedFootnote{\lnDWS}{A valley that is dry most of the year, but contains a stream during the rainy season}%
\DeclareFixedFootnote{\lnDXH}{Or “wilderness”}%
\DeclareFixedFootnote{\lnDXN}{“And it happened”}%
\DeclareFixedFootnote{\lnDXP}{“the day”}%
\DeclareFixedFootnote{\lnDXQ}{“the sons/children of Ammon”}%
\DeclareFixedFootnote{\lnDXW}{“to the face of you”}%
\DeclareFixedFootnote{\lnDXX}{“before us”}%
\DeclareFixedFootnote{\lnDYB}{Or “people”}%
\DeclareFixedFootnote{\lnDZA}{Hebrew “and”}%
\DeclareFixedFootnote{\lnDZL}{“at their going out from Egypt”}%
\DeclareFixedFootnote{\lnDZT}{“the day of the Sabbath”}%
\DeclareFixedFootnote{\lnDZU}{“gates”}%
\DeclareFixedFootnote{\lnDZX}{“all the days”}%
\DeclareFixedFootnote{\lnDZZ}{“spoke”}%
\DeclareFixedFootnote{\lnEAA}{Or “mind”}%
\DeclareFixedFootnote{\lnEAT}{“to their faces”}%
\DeclareFixedFootnote{\lnEAX}{“And it will happen”}%
\DeclareFixedFootnote{\lnEBV}{Hebrew “night”}%
\DeclareFixedFootnote{\lnECQ}{Hebrew “creature”}%
\DeclareFixedFootnote{\lnEDI}{“the face of”}%
\DeclareFixedFootnote{\lnEDR}{“from the face of you”}%
\DeclareFixedFootnote{\lnEEI}{“that your soul/inner self desires”}%
\DeclareFixedFootnote{\lnEEZ}{“in addition to” or “upon it”}%
\DeclareFixedFootnote{\lnEFE}{Hebrew “for you”}%
\DeclareFixedFootnote{\lnEFH}{Hebrew “for/to you” but with collective meaning}%
\DeclareFixedFootnote{\lnEFU}{Hebrew “firstfruit”}%
\DeclareFixedFootnote{\lnEGD}{“from yesterday and the day before”}%
\DeclareFixedFootnote{\lnEGS}{“a man other”}%
\DeclareFixedFootnote{\lnEHO}{“on”}%
\DeclareFixedFootnote{\lnEHV}{“to him”}%
\DeclareFixedFootnote{\lnEHX}{“until eternity”}%
\DeclareFixedFootnote{\lnEHY}{“because of the event when”}%
\DeclareFixedFootnote{\lnEIF}{Or “If”}%
\DeclareFixedFootnote{\lnEIR}{That is, what is left}%
\DeclareFixedFootnote{\lnEIV}{“for you”}%
\DeclareFixedFootnote{\lnEJC}{Or “father”}%
\DeclareFixedFootnote{\lnEJY}{Or “way”}%
\DeclareFixedFootnote{\lnENR}{Hebrew “it” but used poetically and with plural sense in context}%
\DeclareFixedFootnote{\lnENT}{Hebrew “head”}%
\DeclareFixedFootnote{\lnEOV}{“in all that you go”}%
\DeclareFixedFootnote{\lnEOY}{Hebrew “heart”}%
\DeclareFixedFootnote{\lnEPA}{“before the presence of the people”}%
\DeclareFixedFootnote{\lnEPD}{Or “world”}%
\DeclareFixedFootnote{\lnEPV}{“on the road”}%
\DeclareFixedFootnote{\lnEQB}{“this”}%
\DeclareFixedFootnote{\lnEQF}{Hebrew “thing” or “consecrated possession”}%
\DeclareFixedFootnote{\lnEQS}{Hebrew “trouble”; a valley in the Jericho region}%
\DeclareFixedFootnote{\lnEQV}{“all the people of war”}%
\DeclareFixedFootnote{\lnEQX}{“as that at the first occasion”}%
\DeclareFixedFootnote{\lnEQY}{Or “before their presence”}%
\DeclareFixedFootnote{\lnERB}{A dry region that runs south of the Sea of Galilee along the Jordan Valley}%
\DeclareFixedFootnote{\lnERD}{“the mouth of the sword”}%
\DeclareFixedFootnote{\lnERI}{Or “lowlands”; a geographical region on the western edge of the hills of Judea}%
\DeclareFixedFootnote{\lnERO}{“cut for us a covenant”}%
\DeclareFixedFootnote{\lnERR}{Or “men”}%
\DeclareFixedFootnote{\lnESL}{An arid region south of the Judaean hills}%
\DeclareFixedFootnote{\lnETA}{Or “white mountain”}%
\DeclareFixedFootnote{\lnETK}{Hebrew “Geshurite”}%
\DeclareFixedFootnote{\lnETL}{Hebrew “Maacathite”}%
\DeclareFixedFootnote{\lnEUV}{Hebrew “bank”}%
\DeclareFixedFootnote{\lnEVL}{Hebrew “goes out”}%
\DeclareFixedFootnote{\lnEVO}{Or “end”}%
\DeclareFixedFootnote{\lnEWF}{“the goings out of it were”}%
\DeclareFixedFootnote{\lnEXL}{Or “in the presence of”}%
\DeclareFixedFootnote{\lnEYK}{“came out”}%
\DeclareFixedFootnote{\lnFAR}{Hebrew “offering”}%
\DeclareFixedFootnote{\lnFAV}{Hebrew “sacrifice”}%
\DeclareFixedFootnote{\lnFCA}{Or “dwell”}%
\DeclareFixedFootnote{\lnFCQ}{Or “sons/children”}%
\DeclareFixedFootnote{\lnFDD}{Hebrew “inhabitant”}%
\DeclareFixedFootnote{\lnFDW}{Or “judges”}%
\DeclareFixedFootnote{\lnFEA}{Or “judge”}%
\DeclareFixedFootnote{\lnFFP}{Or “you”}%
\DeclareFixedFootnote{\lnFFU}{Hebrew “Midianite”}%
\DeclareFixedFootnote{\lnFGI}{An Asherah is a cultic pole set up next to an altar symbolizing the goddess Asherah}%
\DeclareFixedFootnote{\lnFGN}{“ears”}%
\DeclareFixedFootnote{\lnFGQ}{Or “divisions”}%
\DeclareFixedFootnote{\lnFGU}{Or “camp”}%
\DeclareFixedFootnote{\lnFGY}{Or “hand”}%
\DeclareFixedFootnote{\lnFHD}{Or “house”}%
\DeclareFixedFootnote{\lnFHE}{Or “honesty”}%
\DeclareFixedFootnote{\lnFHK}{Or “companies”}%
\DeclareFixedFootnote{\lnFHO}{Or “cellar”}%
\DeclareFixedFootnote{\lnFHR}{Or “evil”}%
\DeclareFixedFootnote{\lnFIM}{“to make war”}%
\DeclareFixedFootnote{\lnFJT}{“kid of goat”}%
\DeclareFixedFootnote{\lnFJX}{Hebrew “man”}%
\DeclareFixedFootnote{\lnFJZ}{Or “temple”}%
\DeclareFixedFootnote{\lnFKD}{Hebrew “piece”}%
\DeclareFixedFootnote{\lnFKI}{Or “alien”}%
\DeclareFixedFootnote{\lnFKY}{“and let your heart be good”}%
\DeclareFixedFootnote{\lnFLS}{“men drawing sword”}%
\DeclareFixedFootnote{\lnFMA}{Hebrew “my brother”}%
\DeclareFixedFootnote{\lnFMR}{Hebrew “place”}%
\DeclareFixedFootnote{\lnFMT}{“touching”}%
\DeclareFixedFootnote{\lnFNA}{“men of strength”}%
\DeclareFixedFootnote{\lnFNR}{“the life of Adonai”}%
\DeclareFixedFootnote{\lnFNX}{“the declaration of”}%
\DeclareFixedFootnote{\lnFOA}{“arm,” figurative of “strength” in the context of “descendants”}%
\DeclareFixedFootnote{\lnFOH}{“the”}%
\DeclareFixedFootnote{\lnFOJ}{Or “taken”}%
\DeclareFixedFootnote{\lnFOL}{“bowed down”}%
\DeclareFixedFootnote{\lnFOR}{The Masoretic Hebrew text (\textit{Kethib}) reads “boils”; the reading tradition (\textit{Qere}) has “tumors”}%
\DeclareFixedFootnote{\lnFOW}{Or perhaps “chest” or “bag”}%
\DeclareFixedFootnote{\lnFPF}{Or “to rule”}%
\DeclareFixedFootnote{\lnFPN}{“sons of wickedness”}%
\DeclareFixedFootnote{\lnFQC}{“the young man carrying his weapons”}%
\DeclareFixedFootnote{\lnFQD}{“from the beyond from this”}%
\DeclareFixedFootnote{\lnFQG}{“the one carrying his weapons”}%
\DeclareFixedFootnote{\lnFQM}{“and look”}%
\DeclareFixedFootnote{\lnFQQ}{Hebrew “And”}%
\DeclareFixedFootnote{\lnFQS}{“opposite one”}%
\DeclareFixedFootnote{\lnFQY}{“so that I can bow down to”}%
\DeclareFixedFootnote{\lnFQZ}{That is, Saul}%
\DeclareFixedFootnote{\lnFRG}{“was good in the eyes of”}%
\DeclareFixedFootnote{\lnFRT}{Or “boy”}%
\DeclareFixedFootnote{\lnFRU}{Or possibly “equipment” or “weapons”}%
\DeclareFixedFootnote{\lnFSA}{“from the side of the mountain from this”}%
\DeclareFixedFootnote{\lnFSM}{“and”}%
\DeclareFixedFootnote{\lnFSP}{“The life of Adonai”}%
\DeclareFixedFootnote{\lnFST}{Or “necromancers”}%
\DeclareFixedFootnote{\lnFSW}{“face”}%
\DeclareFixedFootnote{\lnFTC}{“but if”}%
\DeclareFixedFootnote{\lnFTF}{“Look”}%
\DeclareFixedFootnote{\lnFTG}{“the carrier of his weapons”}%
\DeclareFixedFootnote{\lnFTO}{Hebrew “pursued after”}%
\DeclareFixedFootnote{\lnFTT}{“from after”}%
\DeclareFixedFootnote{\lnFTY}{“more than”}%
\DeclareFixedFootnote{\lnFUL}{“arrayed”}%
\DeclareFixedFootnote{\lnFVD}{“every man”}%
\DeclareFixedFootnote{\lnFVI}{“at his feet”}%
\DeclareFixedFootnote{\lnFVN}{“going up and weeping”}%
\DeclareFixedFootnote{\lnFVP}{“son of ability”}%
\DeclareFixedFootnote{\lnFVR}{“so this and so this”}%
\DeclareFixedFootnote{\lnFVV}{“they will not set heart toward us”}%
\DeclareFixedFootnote{\lnFVX}{“And let it happen what”}%
\DeclareFixedFootnote{\lnFWC}{“as far as the many”}%
\DeclareFixedFootnote{\lnFWI}{“not turn my face”}%
\DeclareFixedFootnote{\lnFWL}{“where and where”}%
\DeclareFixedFootnote{\lnFWN}{“dying you will die”}%
\DeclareFixedFootnote{\lnFXU}{“on their face”}%
\DeclareFixedFootnote{\lnFZX}{“on the face of the Jordan”}%
\DeclareFixedFootnote{\lnGAG}{“What is for you here, Elijah”}%
\DeclareFixedFootnote{\lnGBA}{“returned and sent”}%
\DeclareFixedFootnote{\lnGBD}{“life of your soul”}%
\DeclareFixedFootnote{\lnGBE}{“from over your head”}%
\DeclareFixedFootnote{\lnGBK}{“here and here”}%
\DeclareFixedFootnote{\lnGBM}{“about that season as the time of life”}%
\DeclareFixedFootnote{\lnGBS}{“each to his friend”}%
\DeclareFixedFootnote{\lnGBV}{“from under the hand”}%
\DeclareFixedFootnote{\lnGBZ}{“What is for you and for peace”}%
\DeclareFixedFootnote{\lnGCE}{“the going out of the Sabbath”}%
\DeclareFixedFootnote{\lnGES}{Or “took”}%
\DeclareFixedFootnote{\lnGFA}{According to the reading tradition (\textit{Qere})}%
\DeclareFixedFootnote{\lnGFJ}{“from time to time”}%
\DeclareFixedFootnote{\lnGFK}{“in number”}%
\DeclareFixedFootnote{\lnGFS}{Or “name”}%
\DeclareFixedFootnote{\lnGGK}{Or “Beno”}%
\DeclareFixedFootnote{\lnGGN}{That is, under the direction}%
\DeclareFixedFootnote{\lnGGZ}{“for bowl and bowl”}%
\DeclareFixedFootnote{\lnGHK}{Or “Syria”}%
\DeclareFixedFootnote{\lnGHL}{This is the spelling in Hebrew, though many translations have “Hiram”}%
\DeclareFixedFootnote{\lnGHO}{“the house of the holy of the holies”}%
\DeclareFixedFootnote{\lnGID}{Or “made”}%
\DeclareFixedFootnote{\lnGIO}{“from this and from that”}%
\DeclareFixedFootnote{\lnGJB}{“in the evening, in the evening”}%
\DeclareFixedFootnote{\lnGKZ}{“coming the Sabbath”}%
\DeclareFixedFootnote{\lnGNJ}{“the house of your fathers”}%
\DeclareFixedFootnote{\lnGNK}{Or “writing”}%
\DeclareFixedFootnote{\lnGNN}{“sons of the people”}%
\DeclareFixedFootnote{\lnGTI}{“at his hand”}%
\DeclareFixedFootnote{\lnGTK}{“At their hand”}%
\DeclareFixedFootnote{\lnGTP}{“At his hand”}%
\DeclareFixedFootnote{\lnGYZ}{“its daughters”}%
\DeclareFixedFootnote{\lnGZE}{“their daughters”}%
\DeclareFixedFootnote{\lnGZV}{“tongue”}%
\DeclareFixedFootnote{\lnGZX}{“house of the women”}%
\DeclareFixedFootnote{\lnGZY}{“to the hand of”}%
\DeclareFixedFootnote{\lnGZZ}{Or “exiled”}%
\DeclareFixedFootnote{\lnHAA}{Hebrew “exile”}%
\DeclareFixedFootnote{\lnHAD}{“house of the king”}%
\DeclareFixedFootnote{\lnHAO}{Hebrew “cubit”}%
\DeclareFixedFootnote{\lnHAP}{Hebrew “word”}%
\DeclareFixedFootnote{\lnHAQ}{“how am I able”}%
\DeclareFixedFootnote{\lnHAV}{“send their hand to”}%
\DeclareFixedFootnote{\lnHAX}{“rested”}%
\DeclareFixedFootnote{\lnHBG}{Hebrew “the accuser,” or “the adversary”}%
\DeclareFixedFootnote{\lnHBN}{“set your heart”}%
\DeclareFixedFootnote{\lnHBQ}{“all that is his”}%
\DeclareFixedFootnote{\lnHBT}{“if not”}%
\DeclareFixedFootnote{\lnHBZ}{“faces of”}%
\DeclareFixedFootnote{\lnHCC}{“with the mouth of the sword”}%
\DeclareFixedFootnote{\lnHCO}{“said to”}%
\DeclareFixedFootnote{\lnHDR}{“answered”}%
\DeclareFixedFootnote{\lnHDX}{Most likely God}%
\DeclareFixedFootnote{\lnHIK}{“to his faces”}%
\DeclareFixedFootnote{\lnHKE}{“If not”}%
\DeclareFixedFootnote{\lnHKQ}{Hebrew “from above”}%
\DeclareFixedFootnote{\lnHMC}{“all of him”}%
\DeclareFixedFootnote{\lnHNE}{That is, the poor}%
\DeclareFixedFootnote{\lnHNG}{“from not”}%
\DeclareFixedFootnote{\lnHNK}{Or “soul,” or “inner self”}%
\DeclareFixedFootnote{\lnHNQ}{The probable antecedent is God}%
\DeclareFixedFootnote{\lnHOJ}{Or “concerning”}%
\DeclareFixedFootnote{\lnHOP}{“where this”}%
\DeclareFixedFootnote{\lnHPY}{“to the face of me”}%
\DeclareFixedFootnote{\lnHRG}{Or “Where exactly”}%
\DeclareFixedFootnote{\lnHRJ}{“not a man”}%
\DeclareFixedFootnote{\lnHSV}{“And it came to be for him,” or “And it was for him”}%
\DeclareFixedFootnote{\lnHTD}{Hebrew “lie”}%
\DeclareFixedFootnote{\lnHTM}{“soul”}%
\DeclareFixedFootnote{\lnHUH}{Or “victory”}%
\DeclareFixedFootnote{\lnHUQ}{“the king of the glory”}%
\DeclareFixedFootnote{\lnHUX}{“heart”}%
\DeclareFixedFootnote{\lnHVM}{Hebrew “mouth”}%
\DeclareFixedFootnote{\lnHWS}{“call”}%
\DeclareFixedFootnote{\lnHWW}{Or “captivity”}%
\DeclareFixedFootnote{\lnHXO}{Hebrew “in”}%
\DeclareFixedFootnote{\lnHYA}{A shortened form of “Adonai”}%
\DeclareFixedFootnote{\lnHZH}{Hebrew “bird”}%
\DeclareFixedFootnote{\lnIAE}{“To a generation and a generation”}%
\DeclareFixedFootnote{\lnIAH}{Hebrew “profane; treat as common”}%
\DeclareFixedFootnote{\lnIAJ}{Hebrew “be”}%
\DeclareFixedFootnote{\lnIBL}{Or “shows compassion to”}%
\DeclareFixedFootnote{\lnIBN}{Hebrew \textit{hallelujah}}%
\DeclareFixedFootnote{\lnIDF}{Or “promise”}%
\DeclareFixedFootnote{\lnIDG}{That is, “examine”}%
\DeclareFixedFootnote{\lnIDI}{Or “your promise”}%
\DeclareFixedFootnote{\lnIEG}{Hebrew “masters”}%
\DeclareFixedFootnote{\lnIEN}{Hebrew “sons”}%
\DeclareFixedFootnote{\lnIES}{Or “Greatly”}%
\DeclareFixedFootnote{\lnIEZ}{That is, the temple}%
\DeclareFixedFootnote{\lnIGK}{Or “fear, dread”}%
\DeclareFixedFootnote{\lnIGM}{Or “lord”}%
\DeclareFixedFootnote{\lnIGO}{Or “quarrels”}%
\DeclareFixedFootnote{\lnIGQ}{Or “wife”}%
\DeclareFixedFootnote{\lnIGU}{“at the place of”}%
\DeclareFixedFootnote{\lnIHP}{“the planners of”}%
\DeclareFixedFootnote{\lnIHY}{Or “life,” or “inner self”}%
\DeclareFixedFootnote{\lnIIG}{Or “him”}%
\DeclareFixedFootnote{\lnIIU}{Or “punishment”}%
\DeclareFixedFootnote{\lnIJD}{Or “land”}%
\DeclareFixedFootnote{\lnIJV}{“also the two of them”}%
\DeclareFixedFootnote{\lnIJY}{“parts of the inmost”}%
\DeclareFixedFootnote{\lnILT}{Hebrew “right”}%
\DeclareFixedFootnote{\lnILZ}{Hebrew “Qohelet”}%
\DeclareFixedFootnote{\lnIMB}{Or “not”}%
\DeclareFixedFootnote{\lnIME}{“the sons of the man”}%
\DeclareFixedFootnote{\lnIMH}{“in my heart”}%
\DeclareFixedFootnote{\lnIND}{“her”}%
\DeclareFixedFootnote{\lnINE}{“Behold!” Or “Look!”}%
\DeclareFixedFootnote{\lnINH}{“What is your beloved more than another beloved …?”}%
\DeclareFixedFootnote{\lnINK}{“she is one”}%
\DeclareFixedFootnote{\lnINN}{Or “Return, return …!”}%
\DeclareFixedFootnote{\lnINU}{“in his rising”}%
\DeclareFixedFootnote{\lnIOB}{Hebrew “brier”}%
\DeclareFixedFootnote{\lnIOC}{Hebrew “thornbush”}%
\DeclareFixedFootnote{\lnIOG}{Hebrew “noble”}%
\DeclareFixedFootnote{\lnIOM}{“unclean of lips”}%
\DeclareFixedFootnote{\lnIOQ}{The Hebrew is plural}%
\DeclareFixedFootnote{\lnIPF}{“in the way of Egypt”}%
\DeclareFixedFootnote{\lnIPH}{“dwell as an alien”}%
\DeclareFixedFootnote{\lnIPJ}{Hebrew “cloud”}%
\DeclareFixedFootnote{\lnIPK}{Hebrew “descendant”}%
\DeclareFixedFootnote{\lnIPV}{“Egypt”}%
\DeclareFixedFootnote{\lnIPY}{“engulf”}%
\DeclareFixedFootnote{\lnIQE}{“from the face of”}%
\DeclareFixedFootnote{\lnIQG}{“put”}%
\DeclareFixedFootnote{\lnIRC}{Hebrew “voice”}%
\DeclareFixedFootnote{\lnIRN}{“a valley of fat”}%
\DeclareFixedFootnote{\lnIRQ}{In this context, the Hebrew expressions \textit{tsaw-tsaw} and \textit{qaw-qaw} are likely meant to sound like baby talk, but they could mean “command upon command” and “rule upon rule”}%
\DeclareFixedFootnote{\lnISF}{“Call”}%
\DeclareFixedFootnote{\lnISG}{“know a document”}%
\DeclareFixedFootnote{\lnIUV}{Hebrew “plant”}%
\DeclareFixedFootnote{\lnIVS}{“said”}%
\DeclareFixedFootnote{\lnIVX}{“flesh”}%
\DeclareFixedFootnote{\lnIWI}{“the men of your strife”}%
\DeclareFixedFootnote{\lnIXI}{“caused to hear”}%
\DeclareFixedFootnote{\lnIXY}{Or “be startled”}%
\DeclareFixedFootnote{\lnIYF}{“one who forms”}%
\DeclareFixedFootnote{\lnIYI}{“then”}%
\DeclareFixedFootnote{\lnIYU}{“you shall not add they shall call to”}%
\DeclareFixedFootnote{\lnIZE}{“intestines”}%
\DeclareFixedFootnote{\lnJAU}{Or “stride”}%
\DeclareFixedFootnote{\lnJBZ}{“high”}%
\DeclareFixedFootnote{\lnJCB}{“from profaning”}%
\DeclareFixedFootnote{\lnJCI}{“crushed”}%
\DeclareFixedFootnote{\lnJDW}{“the spirit of his holiness”}%
\DeclareFixedFootnote{\lnJEW}{“a son of a hundred year”}%
\DeclareFixedFootnote{\lnJFQ}{“a declaration of”}%
\DeclareFixedFootnote{\lnJGA}{“To thus”}%
\DeclareFixedFootnote{\lnJGF}{“what for you”}%
\DeclareFixedFootnote{\lnJGR}{Or “unfaithful”}%
\DeclareFixedFootnote{\lnJGS}{Here the direct object is supplied from context in the English translation}%
\DeclareFixedFootnote{\lnJHS}{“until when”}%
\DeclareFixedFootnote{\lnJHW}{Or “ordinance”}%
\DeclareFixedFootnote{\lnJIV}{“the words of the deception”}%
\DeclareFixedFootnote{\lnJJA}{“my name over it”}%
\DeclareFixedFootnote{\lnJMG}{“at not yet”}%
\DeclareFixedFootnote{\lnJOK}{“I”}%
\DeclareFixedFootnote{\lnJRZ}{Or “oracle”}%
\DeclareFixedFootnote{\lnJTC}{Or “foreign”}%
\DeclareFixedFootnote{\lnJTK}{Or “nobles”}%
\DeclareFixedFootnote{\lnJTS}{“from there is not”}%
\DeclareFixedFootnote{\lnJVK}{Hebrew “fortune”}%
\DeclareFixedFootnote{\lnJYH}{Or “scroll”}%
\DeclareFixedFootnote{\lnKAJ}{“to the face of them”}%
\DeclareFixedFootnote{\lnKCG}{“a scroll of a scroll”}%
\DeclareFixedFootnote{\lnKCH}{Hebrew “disaster”}%
\DeclareFixedFootnote{\lnKDI}{Or “cistern”}%
\DeclareFixedFootnote{\lnKDV}{“a thing”}%
\DeclareFixedFootnote{\lnKDY}{“your life” or “your soul”}%
\DeclareFixedFootnote{\lnKEE}{Hebrew “guards”}%
\DeclareFixedFootnote{\lnKFQ}{The Hebrew verb is singular}%
\DeclareFixedFootnote{\lnKIV}{“not right”}%
\DeclareFixedFootnote{\lnKIZ}{Hebrew “exuberant shout”}%
\DeclareFixedFootnote{\lnKJD}{Hebrew “flutes”}%
\DeclareFixedFootnote{\lnKNY}{“river of Kebar”}%
\DeclareFixedFootnote{\lnKOB}{Or “middle”}%
\DeclareFixedFootnote{\lnKOE}{“they did not turn at their going”}%
\DeclareFixedFootnote{\lnKOF}{“of the four of them”}%
\DeclareFixedFootnote{\lnKOO}{“at their standing”}%
\DeclareFixedFootnote{\lnKOQ}{Or “like”}%
\DeclareFixedFootnote{\lnKOR}{Or “mortal,” or “son of humankind”}%
\DeclareFixedFootnote{\lnKPB}{“heavy/thick of tongue”}%
\DeclareFixedFootnote{\lnKPD}{“to correspond to”}%
\DeclareFixedFootnote{\lnKPM}{Or “soul,” or “self”}%
\DeclareFixedFootnote{\lnKPS}{“a house of rebellion”}%
\DeclareFixedFootnote{\lnKQA}{Hebrew “you”}%
\DeclareFixedFootnote{\lnKQZ}{“not any”}%
\DeclareFixedFootnote{\lnKRF}{“to the way of the north”}%
\DeclareFixedFootnote{\lnKRM}{“you will return, you will see”}%
\DeclareFixedFootnote{\lnKUP}{Or “souls”}%
\DeclareFixedFootnote{\lnKUW}{“they gave before their faces”}%
\DeclareFixedFootnote{\lnKUX}{Or “iniquity”}%
\DeclareFixedFootnote{\lnKVB}{“from behind me”}%
\DeclareFixedFootnote{\lnKVO}{“live I”}%
\DeclareFixedFootnote{\lnKVQ}{“they to alone them”}%
\DeclareFixedFootnote{\lnKWC}{Or “wood”}%
\DeclareFixedFootnote{\lnKWJ}{“on the day of being born you”}%
\DeclareFixedFootnote{\lnKXD}{Or “indeed”}%
\DeclareFixedFootnote{\lnKYT}{Or “devoured”}%
\DeclareFixedFootnote{\lnKZI}{“and I raised my hand”}%
\DeclareFixedFootnote{\lnLAU}{“I, I”}%
\DeclareFixedFootnote{\lnLBI}{“in you”}%
\DeclareFixedFootnote{\lnLCA}{“sons of Assyria”}%
\DeclareFixedFootnote{\lnLCF}{“sons of Babylon”}%
\DeclareFixedFootnote{\lnLCM}{“and remainder your”}%
\DeclareFixedFootnote{\lnLDR}{“they, they”}%
\DeclareFixedFootnote{\lnLEG}{Or “pit”}%
\DeclareFixedFootnote{\lnLEO}{“wise men”}%
\DeclareFixedFootnote{\lnLFK}{“the ones despising them”}%
\DeclareFixedFootnote{\lnLFN}{“two ten”}%
\DeclareFixedFootnote{\lnLGI}{That is, Thebes}%
\DeclareFixedFootnote{\lnLHV}{“the children of your people”}%
\DeclareFixedFootnote{\lnLKB}{Or “the rest of”}%
\DeclareFixedFootnote{\lnLKF}{“humans/humankind”}%
\DeclareFixedFootnote{\lnLKV}{“all around, all around”}%
\DeclareFixedFootnote{\lnLLL}{“to eternity” or “to unlimited ages”}%
\DeclareFixedFootnote{\lnLLQ}{Or “chief”}%
\DeclareFixedFootnote{\lnLLX}{Or “seize”; this word is chosen because of the alliteration in Hebrew}%
\DeclareFixedFootnote{\lnLMY}{That is, 10.5 feet}%
\DeclareFixedFootnote{\lnLMZ}{That is, about 10.5 feet}%
\DeclareFixedFootnote{\lnLNE}{“from the house”}%
\DeclareFixedFootnote{\lnLNK}{That is, 87.5 feet}%
\DeclareFixedFootnote{\lnLNP}{That is, 175 feet}%
\DeclareFixedFootnote{\lnLNT}{“to the way of the south”}%
\DeclareFixedFootnote{\lnLNX}{That is, 43.75 feet}%
\DeclareFixedFootnote{\lnLOA}{“the measurements the these”}%
\DeclareFixedFootnote{\lnLOK}{“from here and from here”}%
\DeclareFixedFootnote{\lnLOP}{That is, 21 inches}%
\DeclareFixedFootnote{\lnLOV}{That is, 35 feet}%
\DeclareFixedFootnote{\lnLPB}{That is, 7 feet}%
\DeclareFixedFootnote{\lnLPD}{“to above, to above”}%
\DeclareFixedFootnote{\lnLPI}{That is, 8.75 feet}%
\DeclareFixedFootnote{\lnLPL}{“was to”}%
\DeclareFixedFootnote{\lnLPY}{“from here”}%
\DeclareFixedFootnote{\lnLQR}{“holy objects of the holy objects”}%
\DeclareFixedFootnote{\lnLRB}{Or “corpses”}%
\DeclareFixedFootnote{\lnLRJ}{Or “trench”}%
\DeclareFixedFootnote{\lnLRP}{“four ten”}%
\DeclareFixedFootnote{\lnLRV}{“son of cattle,” or “son of the herd”}%
\DeclareFixedFootnote{\lnLSL}{“child of a foreign land”}%
\DeclareFixedFootnote{\lnLTG}{That is, 8.33 miles}%
\DeclareFixedFootnote{\lnLTJ}{“the contribution of the holiness”}%
\DeclareFixedFootnote{\lnLUP}{About 1,750 feet}%
\DeclareFixedFootnote{\lnLVD}{“the side of the east}%
\DeclareFixedFootnote{\lnLVE}{“the side of the sea/west”}%
\DeclareFixedFootnote{\lnLVF}{“the side of the east”}%
\DeclareFixedFootnote{\lnLVL}{“the side of the sea/west}%
\DeclareFixedFootnote{\lnLVW}{That is, 3.5 miles}%
\DeclareFixedFootnote{\lnLWB}{“to corresponding”}%
\DeclareFixedFootnote{\lnLWH}{That is, 1.5 miles}%
\DeclareFixedFootnote{\lnLWL}{That is, 150 yards}%
\DeclareFixedFootnote{\lnLWY}{“to in front of”}%
\DeclareFixedFootnote{\lnLXM}{“the wine of his drink”}%
\DeclareFixedFootnote{\lnLXV}{Or “soothsayer-priests”}%
\DeclareFixedFootnote{\lnLXW}{Or “enchanters”}%
\DeclareFixedFootnote{\lnLYA}{“Chaldeans”}%
\DeclareFixedFootnote{\lnLYE}{Or “interpretation”}%
\DeclareFixedFootnote{\lnLYK}{Aramaic “Jehud”}%
\DeclareFixedFootnote{\lnLYT}{“the nations and the languages”}%
\DeclareFixedFootnote{\lnLYU}{Aramaic “the gold”}%
\DeclareFixedFootnote{\lnLZK}{Aramaic “animal”}%
\DeclareFixedFootnote{\lnLZU}{“the humankind”}%
\DeclareFixedFootnote{\lnMAA}{“to over against”}%
\DeclareFixedFootnote{\lnMAB}{“father”}%
\DeclareFixedFootnote{\lnMBE}{“being corrupt”}%
\DeclareFixedFootnote{\lnMBG}{“pit of lions”}%
\DeclareFixedFootnote{\lnMBR}{“I was watching”}%
\DeclareFixedFootnote{\lnMBW}{“I Daniel”}%
\DeclareFixedFootnote{\lnMBX}{Or “saints”}%
\DeclareFixedFootnote{\lnMCS}{Or “messiah”}%
\DeclareFixedFootnote{\lnMCU}{Or “matter,” or “message”}%
\DeclareFixedFootnote{\lnMCX}{This is an emphatic use of the first person personal pronoun}%
\DeclareFixedFootnote{\lnMDD}{That is, Greece}%
\DeclareFixedFootnote{\lnMDQ}{Or “book”}%
\DeclareFixedFootnote{\lnMDT}{Or “river”}%
\DeclareFixedFootnote{\lnMDW}{“Call his name”}%
\DeclareFixedFootnote{\lnMDX}{Jezreel means “God sows”}%
\DeclareFixedFootnote{\lnMEA}{Lo-ruhamah means “Not pitied”}%
\DeclareFixedFootnote{\lnMEF}{Hebrew “festival”}%
\DeclareFixedFootnote{\lnMER}{Or “brides”}%
\DeclareFixedFootnote{\lnMEU}{Hebrew “her”}%
\DeclareFixedFootnote{\lnMFK}{“Sheol” is a Hebrew term for the place where the dead reside, i.e., the underworld}%
\DeclareFixedFootnote{\lnMFS}{“before the presence of it”}%
\DeclareFixedFootnote{\lnMGX}{“from to face of”}%
\DeclareFixedFootnote{\lnMGZ}{“going/growing and storming”}%
\DeclareFixedFootnote{\lnMHH}{Hebrew “Assyria”}%
\DeclareFixedFootnote{\lnMIG}{The Hebrew term is also the name of a Canaanite deity}%
\DeclareFixedFootnote{\lnMIM}{“a name”}%
\DeclareFixedFootnote{\lnMIO}{“through the hand of”}%
\DeclareFixedFootnote{\lnMIR}{“Set your heart on your ways”}%
\DeclareFixedFootnote{\lnMIW}{Or “be strong”}%
\DeclareFixedFootnote{\lnMJV}{“lifted up my eyes”}%
\DeclareFixedFootnote{\lnMKM}{Or “ephah,” a measure of grain}%
\DeclareFixedFootnote{\lnMKZ}{Or “wonderful”}%
\DeclareFixedFootnote{\lnMLU}{Hebrew “him” (or “it”), referring to the flock in v. 3}%
\DeclareFixedFootnote{\lnMMF}{Syriac reads “treasury,” followed by NAB, NRSV, TEV}%
\DeclareFixedFootnote{\lnMMI}{“the eye of his right”}%
\DeclareFixedFootnote{\lnMMS}{“clans, clans alone”}%
\DeclareFixedFootnote{\lnMNG}{“the mountain of the olive trees”}%
\DeclareFixedFootnote{\lnMOA}{Or “is there no wrong?”}%
\DeclareFixedFootnote{\lnMOG}{“You must keep watch on”}%
\DeclareFixedFootnote{\lnMOL}{Or “fear”}%
\DeclareFixedFootnote{\lnMPC}{Or “nations”; the same Greek word can be translated “nations” or “Gentiles” depending on the context}%
\DeclareFixedFootnote{\lnMPG}{“but if not”}%
\DeclareFixedFootnote{\lnMPH}{Traditionally “rust,” but more likely in this context along with “moth” the term refers to “eating” by other types of insects or vermin}%
\DeclareFixedFootnote{\lnMPK}{“the”; the Greek article is used here as a possessive pronoun}%
\DeclareFixedFootnote{\lnMQJ}{Or “every kind of”}%
\DeclareFixedFootnote{\lnMRR}{“fruit,” describing here the grain harvested from the healthy plants; in contemporary English this would more naturally be expressed by terms like “grain” or “crop”}%
\DeclareFixedFootnote{\lnMRT}{“because of what”}%
\DeclareFixedFootnote{\lnMWB}{Some manuscripts have “Have mercy on us, Lord”}%
\DeclareFixedFootnote{\lnMXT}{The phrase “by his oath” is not in the Greek text but is implied}%
\DeclareFixedFootnote{\lnMXZ}{Or “Gentiles”; the same Greek word can be translated “nations” or “Gentiles” depending on the context}%
\DeclareFixedFootnote{\lnMZR}{“surely I am not”; the negative construction in Greek anticipates a negative answer here, indicated in the translation by “am I”}%
\DeclareFixedFootnote{\lnNBB}{Although many manuscripts omit “Jesus” here, it is so hard to explain why a scribe would have added it that the reading is probably original}%
\DeclareFixedFootnote{\lnNBN}{“by name”}%
\DeclareFixedFootnote{\lnNCT}{“who were having badly”}%
\DeclareFixedFootnote{\lnNDN}{“for the age”}%
\DeclareFixedFootnote{\lnNKB}{Or “Messiah”}%
\DeclareFixedFootnote{\lnNRT}{“in the sight of you”}%
\DeclareFixedFootnote{\lnNVQ}{Or “council”}%
\DeclareFixedFootnote{\lnNYG}{This pronoun is neuter singular in Greek, but is collective}%
\DeclareFixedFootnote{\lnNYI}{“with one another”}%
\DeclareFixedFootnote{\lnNYX}{Some manuscripts explicitly state “me”}%
\DeclareFixedFootnote{\lnNZX}{“he has maturity”}%
\DeclareFixedFootnote{\lnOEE}{A reference to the Roman province of Asia (modern Asia Minor)}%
\DeclareFixedFootnote{\lnOIE}{“whom”}%
\DeclareFixedFootnote{\lnOJJ}{Or “with a certain Simon Berseus”; most modern English versions treat the word as Simon’s profession (“Simon the tanner”), but the word may actually be a surname (“Simon Berseus” or “Simon Tanner”)}%
\DeclareFixedFootnote{\lnOQB}{Or “headquarters”}%
\DeclareFixedFootnote{\lnOUC}{“in the open”}%
\DeclareFixedFootnote{\lnOUE}{Gen 15:6}%
\DeclareFixedFootnote{\lnOUG}{“in circumcision”}%
\DeclareFixedFootnote{\lnOUH}{“in uncircumcision”}%
\DeclareFixedFootnote{\lnOUU}{“has labored much”}%
\DeclareFixedFootnote{\lnOVF}{“for the sake of us”}%
\DeclareFixedFootnote{\lnOVL}{“questioning nothing”}%
\DeclareFixedFootnote{\lnOVQ}{Some manuscripts have “and to another”}%
\DeclareFixedFootnote{\lnOWE}{“on behalf of us”}%
\DeclareFixedFootnote{\lnOWH}{“according to the flesh”}%
\DeclareFixedFootnote{\lnOWM}{“to the things immeasurable”}%
\DeclareFixedFootnote{\lnOXB}{“the things concerning you”}%
\DeclareFixedFootnote{\lnOXL}{“to abound”}%
\DeclareFixedFootnote{\lnOXQ}{“man,” used here in a generic sense to refer to persons of either gender}%
\DeclareFixedFootnote{\lnOXW}{“if they will enter”}%
\DeclareFixedFootnote{\lnOXZ}{Ps 95:11}%
\DeclareFixedFootnote{\lnOYH}{“according to year”}%
\DeclareFixedFootnote{\lnOYM}{“a man,” but clearly in a generic sense here meaning “someone, a person”}%
\DeclareFixedFootnote{\lnOZD}{“for the ages of the ages”}%
\DeclareFixedFootnote{\lnPAS}{“from afar”}%

\biblebook{Genesis}
\begin{biblechapter}% Genesis 1
\verseWithHeading{The Creation}{In the beginning, God created the heavens and the earth —}%
\verse{Now\lnA{} the earth was formless and empty, and darkness was over the face of the deep. And the Spirit of God was hovering over the surface of the waters.}%
\verse{And God said, “Let there be light!” And there was light.}%
\verse{And God saw the light, that it was good, and God caused there to be a separation between the light and between the darkness.}%
\verse{And God called the light Day, and the darkness he called Night. And there was evening and there was morning, the first day.}%
\verse{And God said, “Let there be a vaulted dome\lnB{} in the midst of the waters, and let it cause a separation between the waters.”\lebnote{“Let there be a separation made between waters to waters”}}%
\verse{So\lnA{} God made the vaulted dome,\lnB{} and he caused a separation between the waters which were under the vaulted dome\lnB{} and between the waters which were over the vaulted dome.\lnB{} And it was so.}%
\verse{And God called the vaulted dome\lnB{} “heaven.”\lebnote{Or “sky”} And there was evening, and there was morning, a second day.}%
\verse{And God said, “Let the waters under heaven\lnC{} be gathered to one place, and let the dry ground appear.” And it was so.}%
\verse{And God called the dry ground “earth,” and he called the collection of the waters “seas.” And God saw that it was good.}%
\verse{And God said, “Let the earth produce green plants that will bear seed — fruit trees bearing fruit in which there is seed\lnD{} — according to its\lnE{} kind, on the earth.” And it was so.}%
\verse{And the earth brought forth green plants bearing seed according to its\lnE{} kind, and trees bearing fruit in which there was seed\lnD{} according to its\lnE{} kind. And God saw that it was good.}%
\verse{And there was evening and there was morning, a third day.}%
\verse{And God said, “Let there be lights\lnF{} in the vaulted dome\lnB{} of heaven\lnC{} to separate day from night,\lebnote{“to make a separation between the day and between the night”} and let them be as signs and for appointed times, and for days and years,}%
\verse{and they shall be as lights\lnF{} in the vaulted dome\lnB{} of heaven\lnC{} to give light on the earth.” And it was so.}%
\verse{And God made two lights,\lnF{} the greater light\lnG{} to rule\lnH{} the day and the smaller light\lnG{} to rule\lnH{} the night, and the stars.}%
\verse{And God placed them in the vaulted dome\lnB{} of heaven\lnC{} to give light on the earth}%
\verse{and to rule over the day and over the night, and to separate light from darkness.\lebnote{“make a separation between the light and between the darkness”} And God saw that it was good.}%
\verse{And there was evening and there was morning, a fourth day.}%
\verse{And God said, “Let the waters swarm with swarms of living creatures, and let birds fly over the earth across the face of the vaulted dome\lnB{} of heaven.\lnC{}}%
\verse{So\lnA{} God created the great sea creatures and every living creature that moves, with which the waters swarm, according to their kind, and every bird with wings according to its\lnE{} kind. And God saw that it was good.}%
\verse{And God blessed them, saying, “Be fruitful and multiply, and fill the waters in the seas, and let the birds multiply on the earth.”}%
\verse{And there was evening, and there was morning, a fifth day.}%
\verse{And God said, “Let the earth bring forth living creatures according to their kind: cattle and moving things, and wild animals\lnI{} according to their kind.” And it was so.}%
\verse{So\lnA{} God made wild animals\lnI{} according to their kind and the cattle according to their kind, and every creeping thing of the earth according to its\lnE{} kind. And God saw that it was good.}%
\verse{And God said, “Let us make humankind in\lnJ{} our image and according to\lnJ{} our likeness, and let them rule over the fish of the sea, and over the birds of heaven,\lnC{} and over the cattle, and over all the earth, and over every moving thing that moves upon the earth.”}%
\verse{So\lnA{} God created humankind in\lnJ{} his image, in\lnJ{} the likeness of God he created him,\lebnote{Or “it” \textit{collective: humankind}} male and female he created them.}%
\verse{And God blessed them, and God said to them, “Be fruitful and multiply, and fill the earth and subdue it, and rule over the fish of the sea and the birds of heaven,\lnC{} and over every animal that moves upon the earth.”}%
\verse{And God said, “Look — I am giving to you every plant that bears seed which is on the face of the whole earth, and every kind of tree that bears fruit.\lebnote{“which in it fruit of a tree bears seed”} They shall be yours as food.”}%
\verse{And to every kind of animal of the earth and to every bird of heaven,\lnC{} and to everything that moves upon the earth in which there is life I am giving every green plant as food.” And it was so.}%
\verse{And God saw everything that he had made and, behold, it was very good. And there was evening, and there was morning, a sixth day.}%
\end{biblechapter}%
\begin{biblechapter}% Genesis 2
\verse{And heaven\lnK{} and earth\lnL{} and all their array\lebnote{Or “host”} were finished.}%
\verse{And on the seventh day God finished his work that he had done, and he rested on the seventh day from all his work that he had done.}%
\verse{And God blessed the seventh day, and he sanctified\lebnote{Or “hallowed”} it, because on it he rested from all his work of creating that there was to do.\lebnote{“which God created to do”}}%
\verseWithHeading{The Generations of Heaven and Earth}{These are the generations\lebnote{Or “family records”} of heaven\lnK{} and earth\lnL{} when they were created, in the day that Adonai God made earth and heaven\lebnote{Or “sky”} —}%
\verse{before any plant of the field was\lebnote{“and every plant of the field not yet was”} on earth, and before any plant of the field\lebnote{“and every plant of the field not yet”} had sprung up, because Adonai God had not caused it to rain upon the earth, and there was no human being to cultivate the ground,}%
\verse{but a stream would rise from the earth and water the whole face of the ground —}%
\verse{when\lnM{} Adonai God formed the man\lnN{} of dust from the ground, and he blew into his nostrils the breath of life, and the man became a living creature.}%
\verse{And Adonai God planted a garden in Eden in the east, and there he put the man\lnN{} whom he had formed.}%
\verse{And Adonai God caused to grow every tree that was pleasing to the sight and good for food. And the tree of life was in the midst of the garden, along with\lnM{} the tree of the knowledge of good and evil.}%
\verse{Now\lnO{} a river flowed out from Eden that watered the garden, and from there it diverged and became four branches.}%
\verse{The name of the first is the Pishon. It went around all the land of Havilah, where there is gold.}%
\verse{(The gold of that land is good; bdellium and onyx stones are there.)}%
\verse{And the name of the second is Gihon. It went around all the land of Cush.}%
\verse{And the name of the third is Tigris. It flows east of Assyria. And the fourth river is the Euphrates.}%
\verse{And Adonai God took the man\lnN{} and set him in the garden of Eden to cultivate it and to keep it.}%
\verse{And Adonai God commanded the man,\lnN{} saying, “From every tree of the garden you may freely eat,\lebnote{“eating you may eat”}}%
\verse{but from the tree of the knowledge of good and evil you shall not eat, for in the day that you eat\lebnote{“of your eating”} from it you shall surely die.”\lebnote{“dying you shall die”}}%
\verse{Then\lnO{} Adonai God said, “it is not good that the man\lnN{} is alone. I will make for him a helper as his counterpart.”\lnP{}}%
\verse{And out of the ground Adonai God formed every beast of the field and every bird of the sky,\lebnote{Or “the heavens”} and he brought each to the man\lnN{} to see what he would call it. And whatever the man\lnN{} called that living creature was its name.}%
\verse{And the man\lnN{} gave names\lebnote{“called names”} to every domesticated animal and to the birds of heaven\lnK{} and to all the wild animals.\lebnote{“animals of the earth/land”} But for the man there was not found a helper as his counterpart.\lnP{}}%
\verse{And Adonai God caused a deep sleep to fall upon the man.\lnN{} While\lnO{} he slept, he took one of his ribs, and closed up the flesh where it had been.\lebnote{“the flesh in the place of it”}}%
\verse{And Adonai God fashioned the rib which he had taken from the man\lnN{} into a woman and brought her to the man.\lnN{}}%
\verse{And the man\lnN{} said, “She is now\lebnote{“this \textit{one} the time”} bone from my bones and flesh from my flesh; she\lebnote{“this \textit{one}”} shall be called ‘Woman,’ for she was taken\lebnote{“this \textit{one} was taken”} from man.”}%
\verse{Therefore a man shall leave his father and his mother and shall cling to his wife, and they shall be as one flesh.}%
\verse{And the man\lnN{} and his wife, both of them, were naked, and they were not ashamed.}%
\end{biblechapter}%
\begin{biblechapter}% Genesis 3
\verseWithHeading{The Fall}{Now the serpent was more crafty than any other wild animal\lebnote{“animal of the field”} which Adonai God had made. He said to the woman, “Did God indeed say, ‘You shall not eat from any tree in the garden’?”}%
\verse{The woman said to the serpent, “From the fruit of the trees of the garden we may eat,}%
\verse{but from the tree that is in the midst of the garden, God said, ‘You shall not eat from it, nor shall you touch it, lest you die’.”}%
\verse{But the serpent said to the woman, “You shall not surely die.}%
\verse{For God knows that on the day you both eat from it, then your eyes will be opened and you both shall be like gods,\lebnote{The plural is in the context of v. 22 \textit{“one of us”} and the plural suffix pronouns [[“you \textit{all}”]] throughout the verse} knowing good and evil.”}%
\verse{When\lnQ{} the woman saw that the tree was good for food and that it was a delight to the eyes, and the tree was desirable to make one wise, then\lebnote{Or “and”} she took from its fruit and she ate. And she gave it also to her husband with her, and he ate.}%
\verse{Then\lnQ{} the eyes of both of them were opened, and they knew that they were naked. And they sewed together fig leaves and they made for themselves coverings.}%
\verse{Then\lnQ{} they heard the sound of Adonai God walking in the garden at the windy time of day.\lebnote{“at the wind of the day”} And the man\lnR{} and his wife hid themselves from the presence of Adonai God among the trees of the garden.}%
\verse{And Adonai God called to the man\lnR{} and said to him, “Where are you?”}%
\verse{And he replied,\lebnote{“And he said”; “replied” distinguishes Adam as the speaker} “I heard the sound of you in the garden, and I was afraid because I am naked, so I hid myself.”}%
\verse{Then he\lebnote{That is, Adonai God} asked,\lnS{} “Who told you that you were naked? Have you eaten from the tree from which I forbade you to eat?”\lebnote{“the tree which I commanded to not eat from it”}}%
\verse{And the man\lnR{} replied,\lnS{} “The woman whom you gave to be with me — she gave to me from the tree and I ate.”}%
\verse{Then\lnQ{} Adonai God said to the woman, “What is this you have done?” And the woman said, “The serpent deceived me, and I ate.”}%
\verse{Then\lnQ{} Adonai God said to the serpent, “Because you have done this, you will be cursed more than any domesticated animal and more than any wild animal.\lebnote{“animal of the earth/land”} On your belly you shall go and dust you shall eat all the days of your life.}%
\verse{And I will put hostility between you and between the woman, and between your offspring\lnT{} and between her offspring;\lnT{} he will strike you on the head, and you will strike him on the heel.”}%
\verse{To the woman he said, “I will greatly increase your pain in childbearing;\lebnote{“your pain and your childbearing”} in pain you shall bear children. And to your husband shall be your desire. And he shall rule over you.”}%
\verse{And to Adam\lnU{} he said, “Because you listened to the voice of your wife and you ate from the tree from which I forbade you to eat,\lebnote{“from the tree which I commanded saying not to eat from it”} the ground shall be cursed on your account. In pain you shall eat from it all the days of your life.}%
\verse{And thorns and thistles shall sprout for you, and you shall eat the plants of the field.}%
\verse{By the sweat of your brow\lebnote{“your face”} you shall eat bread, until your return to the ground. For from it you were taken; for you are dust, and to dust you shall return.”}%
\verse{And the man\lnR{} named\lebnote{“called the name”} his wife Eve, because she was the mother of all life.}%
\verse{And Adonai God made for Adam\lnU{} and for his wife garments of skin, and he clothed them.}%
\verse{And Adonai God said, “Look — the man has become as one of us, to know good and evil. What if\lebnote{“And now lest”} he stretches out his hand and takes also from the tree of life and eats, and lives forever?”}%
\verse{And Adonai God sent him out from the garden of Eden, to till the ground from which he was taken.}%
\verse{So\lnQ{} he drove the man out, and placed cherubim east of the garden of Eden, and a flaming, turning sword\lebnote{“a flame of the sword which was turning”} to guard the way to the tree of life.}%
\end{biblechapter}%
\begin{biblechapter}% Genesis 4
\verseWithHeading{Cain and Abel}{Now Adam knew\lnV{} Eve his wife, and she conceived and bore Cain. And she said, “I have given birth to a man with the help of Adonai.”}%
\verse{Then she bore\lebnote{“And she added to bear”} his brother Abel. And Abel became a keeper of sheep, and Cain became a tiller of the ground.}%
\verse{And in the course of time\lebnote{“it came to pass at the end of days”} Cain brought an offering from the fruit of the ground to Adonai,}%
\verse{and Abel also brought an offering from the choicest firstlings of his flock.\lebnote{“from the firstlings of his flock and from their fat”} And Adonai looked with favor to Abel and to his offering,}%
\verse{but to Cain and to his offering he did not look with favor. And Cain became very angry, and his face fell.}%
\verse{And Adonai said to Cain, “Why are you angry, and why is your face fallen?}%
\verse{If you do well will I not accept you?\lebnote{“a lifting up”; this is an abbreviation of the Hebrew idiom “to lift up the face,” which means “to accept or regard with favor”} But if you do not do well, sin is crouching at the door. And its desire is for you, but you must rule over it.”}%
\verse{Then\lnW{} Cain said to his brother Abel, “Let us go out into the field.”\lebnote{This phrase is not present in the Hebrew; it is supplied in other versions \textit{Samaritan, Septuagint, Syriac, Vulgate}} And when they were in the field, Cain rose up against his brother Abel and killed him.}%
\verse{Then\lnW{} Adonai said to Cain, “Where is Abel your brother?” And he said, “I do not know; am I my brother’s keeper?”}%
\verse{And he said, “What have you done? The voice of your brother’s blood is crying out to me from the ground.}%
\verse{So now you are cursed from the ground, which has opened its mouth to receive the blood of your brother from your hand.}%
\verse{When you till the ground it shall no longer yield its strength to you.\lebnote{“it shall not add to give its strength to you”} You shall be a wanderer and a fugitive on the earth.”}%
\verse{And Cain said to Adonai, “My punishment is greater than I can bear.}%
\verse{Look, you have driven me out today from the face of the ground, and from your face I must hide. I will be a wanderer and a fugitive on the earth, and it will happen that whoever finds me will kill me.”}%
\verse{Then\lnW{} Adonai said to him, “Therefore, whoever kills Cain will be avenged sevenfold.” Then\lnW{} Adonai put a sign on Cain so that whoever found him would not kill him.}%
\verse{And Cain went out from the presence of Adonai, and he settled in the land of Nod, east of Eden.}%
\verse{And Cain knew\lnV{} his wife, and she conceived and gave birth to Enoch. And when he built a city he named the city after his son, Enoch.\lebnote{“he called the name of the city as the name of his son, Enoch”}}%
\verse{And to Enoch was born Irad, and Irad fathered Mehujael, and Mehujael fathered Methushael, and Methushael fathered Lamech.}%
\verse{And Lamech took to himself two wives.\lebnote{Or “women”} The name of the first was Adah, and the name of the second was Zillah.}%
\verse{And Adah gave birth to Jabal; he was the father of those who live in tents and those who have livestock.}%
\verse{And the name of his brother was Jubal; he was the father of all who play stringed instruments and wind instruments.}%
\verse{Then\lnW{} Zillah also gave birth to Tubal-Cain who forged all kinds of tools of bronze and iron. And the sister of Tubal-Cain was Naamah.}%
\verse{Then\lnW{} Lamech said to his wives, “Adah and Zillah, listen to my voice; O wives of Lamech, hear my words. I have killed a man for wounding me, Even a young man for injuring me.}%
\verse{If Cain is avenged sevenfold, Then Lamech will be avenged seventy and seven times.}%
\verse{Then\lnW{} Adam knew\lnV{} his wife again, and she gave birth to a son. And she called his name Seth, for she said “God has appointed to me another child in the place of Abel, because Cain killed him.”}%
\verse{And as for Seth, he also fathered a son, and he called his name Enosh. At that time he\lebnote{Or “humankind,” if the 3ms form is considered collective} began to call on the name of Adonai.}%
\end{biblechapter}%
\begin{biblechapter}% Genesis 5
\verseWithHeading{Adam’s Descendants to Noah}{This is the record of the generations\lebnote{Or “family records”} of Adam. When\lnX{} God created Adam,\lebnote{Or “humankind”} he made him in the likeness of God.}%
\verse{Male and female he created them. And he blessed them. And he called their name “Humankind” when\lebnote{“on the day”} they were created.}%
\verse{And when Adam had lived one hundred and thirty years, he fathered a child in his likeness, according to his image. And he called his name Seth.}%
\verse{And the days of Adam after he fathered Seth were eight hundred years. And he fathered sons and daughters.}%
\verse{And all the days of Adam which he lived were nine hundred and thirty years, and he died.}%
\verse{When\lnX{} Seth had lived one hundred and five years, he fathered Enosh.}%
\verse{And after Seth had fathered Enosh he lived eight hundred and seven years, and fathered sons and daughters.}%
\verse{And all the days of Seth were nine hundred and twelve years, and he died.}%
\verse{When\lnX{} Enosh lived ninety years, he fathered Kenan.}%
\verse{And after Enosh fathered Kenan he lived eight hundred and fifteen years, and fathered sons and daughters.}%
\verse{And all the days of Enosh were nine hundred and five years, and he died.}%
\verse{When\lnX{} Kenan had lived seventy years, he fathered Mahalalel.}%
\verse{And after Kenan had fathered Mahalalel, he lived eight hundred and forty years, and fathered sons and daughters.}%
\verse{And all the days of Kenan were nine hundred and ten years, and he died.}%
\verse{When\lnX{} Mahalalel had lived sixty-five years, he fathered Jared.}%
\verse{And after Mahalalel had fathered Jared, he lived eight hundred and thirty years, and fathered sons and daughters.}%
\verse{And all the days of Mahalalel were eight hundred and ninety-five years, and he died.}%
\verse{When\lnX{} Jared had lived one hundred and sixty-two years, he fathered Enoch.}%
\verse{And after Jared had fathered Enoch, he lived eight hundred years, and fathered sons and daughters.}%
\verse{And all the days of Jared were nine hundred and sixty-two years, and he died.}%
\verse{When\lnX{} Enoch had lived sixty-five years, he fathered Methuselah.}%
\verse{And Enoch walked with God after he fathered Methuselah three hundred years, and fathered sons and daughters.}%
\verse{And all the days of Enoch were three hundred and sixty-five years.}%
\verse{And Enoch walked with God, and he was no more, for God took him.}%
\verse{When\lnX{} Methuselah had lived one hundred and eighty-seven years, he fathered Lamech.}%
\verse{And after Methuselah had fathered Lamech, he lived seven hundred and eighty-two years, and fathered sons and daughters.}%
\verse{And all the days of Methuselah were nine hundred and sixty-nine years, and he died.}%
\verse{When\lnX{} Lamech had lived one hundred and eighty-two years, he fathered a son.}%
\verse{And he called his name Noah, saying, “This one shall relieve us\lebnote{“shall comfort us”} from our work, and from the hard labor of our hands, from the ground which Adonai had cursed.}%
\verse{And after Lamech had fathered Noah he lived five hundred and ninety-five years, and he fathered sons and daughters.}%
\verse{All the days of Lamech were seven hundred and seventy-seven years, and he died.}%
\verse{When\lnX{} Noah was five hundred years old,\lebnote{“was a son of five hundred years”} Noah fathered Shem, Ham, and Japheth.}%
\end{biblechapter}%
\begin{biblechapter}% Genesis 6
\verseWithHeading{Prelude to the Flood}{And it happened that, when humankind began to multiply on the face of the ground, daughters were born to them.}%
\verse{Then\lebnote{Or “And”} the sons of God saw the daughters of humankind, that they were beautiful. And they took for themselves wives from all that they chose.}%
\verse{And Adonai said, “My Spirit shall not abide with humankind forever in that he is also flesh. And his days shall be one hundred and twenty years.”}%
\verse{The Nephilim were upon the earth in those days, and also afterward, when the sons of God went into the daughters of humankind, and they bore children to them.}%
\verse{And Adonai saw that the evil of humankind was great upon the earth, and every inclination of the thoughts of his heart was always\lebnote{“every day”} only evil.}%
\verse{And Adonai regretted that he had made humankind on the earth, and he was grieved in his heart.\lebnote{“he was grieved to his heart”}}%
\verse{And Adonai said, “I will destroy humankind whom I created from upon the face of the earth, from humankind, to animals, to creeping things, and to the birds of heaven,\lnY{} for I regret that I have made them.”}%
\verse{But Noah found favor in the eyes of Adonai.}%
\verse{These are the generations\lebnote{Or “family records”} of Noah. Noah was a righteous man, without defect in his generations. Noah walked with God.}%
\verse{And Noah fathered three sons: Shem, Ham, and Japheth.}%
\verse{And the earth was corrupted before God, and the earth was filled with violence.}%
\verse{And God saw the earth, and behold, it was corrupt, for all flesh had corrupted its way upon the earth.}%
\verse{And God said to Noah, “The end of all flesh has come before me, for the earth was filled with violence because of them. Now, look, I am going to destroy them along with the earth.}%
\verse{Make for yourself an ark of cypress wood; you must make the ark with rooms, then\lebnote{Or “and”} you must cover it with pitch, inside and outside.}%
\verse{And this is how you must make it: the length of the ark, three hundred cubits; its width fifty cubits; its height, thirty cubits.}%
\verse{You must make a roof for the ark, and finish it to a cubit above.\lebnote{“to one cubit you must finish it from above”} And as for the door of the ark, you must put it in its side. You must make it with a lower, second, and a third deck.}%
\verse{And I, behold, I am about to bring the flood waters over the earth to destroy all flesh in which is the breath of life from under the heaven;\lnY{} everything that is on the earth shall perish.}%
\verse{And I will establish my covenant with you, and you must go into the ark — you, and your sons, and your wife, and the wives of your sons with you.}%
\verse{And of every living thing, from all flesh, you must bring two from every kind into the ark to keep them alive with you; they shall be male and female.}%
\verse{From the birds according to their\lnZ{} kind, and from the animals according to their\lnZ{} kind, from every creeping thing on the ground according to its\lebnote{Or “their”} kind — two from every kind shall come to you to keep them alive.}%
\verse{And as for you, take for yourself from every kind of food that is eaten. And you must gather it to yourself. And it shall be for you and for them for food.”}%
\verse{And Noah did according to all that God commanded him; thus he did.}%
\end{biblechapter}%
\begin{biblechapter}% Genesis 7
\verse{Then\lebnote{Or “And”} Adonai said to Noah, “Go — you and all your household — into the ark, for I have seen you are righteous before me in this generation.}%
\verse{From all the clean animals you must take for yourself seven pairs,\lnBA{} a male and its mate. And from the animals that are not clean you must take two, a male and its mate,}%
\verse{as well as from the birds of heaven\lnBB{} seven pairs,\lnBA{} male and female, to keep their kind alive\lebnote{“to keep seed alive”} on the face of the earth.}%
\verse{For within seven days\lebnote{“to seven days still”} I will send rain upon the earth for forty days and forty nights. And I will blot out all the living creatures that I have made from upon the face of the ground.”}%
\verse{And Noah did according to all that Adonai commanded him.}%
\verseWithHeading{The Flood}{Noah was six hundred years old\lebnote{“was a son of six hundred years”} when\lebnote{Or “and”} the flood waters came upon the earth.}%
\verse{And Noah and his sons and his wife, and the wives of his sons with him, went into the ark because of the waters of the flood.}%
\verse{Of clean animals, and of animals which are not clean, and of the birds, and everything that creeps upon the ground,}%
\verse{two of each\lnBC{} went to Noah, into the ark, male and female, as God had commanded Noah.}%
\verse{And it happened that after seven days the waters of the flood came over the earth.}%
\verse{In the six hundredth year of the life of Noah, in the second month, on the seventeenth day of the month — on that day all the springs of the great deep were split open, and the windows of heaven were opened.}%
\verse{And the rain came upon the earth forty days and forty nights.}%
\verse{On this same day, Noah, Shem, Ham, and Japheth, the sons of Noah, and the wife of Noah and the three wives of his sons with them, went into the ark,}%
\verse{they and all the living creatures according to their\lnBD{} kind, and all the domesticated animals according to their\lnBD{} kind, and all the creatures that creep upon the earth according to their\lnBD{} kind, all the birds according to their\lnBD{} kind, every winged creature.}%
\verse{And they came to Noah to the ark, two of each,\lnBC{} from every living thing in which was the breath of life.}%
\verse{And those that came, male and female, of every living thing, came as God had commanded him. And Adonai shut the door behind him.}%
\verse{And the flood came forty days and forty nights upon the earth. And the waters increased, and lifted the ark, and it rose up from the earth.}%
\verse{And the waters prevailed and increased greatly upon the earth. And the ark went upon the surface\lebnote{Hebrew “face”} of the waters.}%
\verse{And the waters prevailed overwhelmingly\lebnote{“very very”} upon the earth, and they covered all the high mountains which were under the entire heaven.\lnBB{}}%
\verse{The waters swelled fifteen cubits above the mountains, covering them.\lebnote{“Fifteen cubits above the waters swelled, and they covered the mountains”}}%
\verse{And every living thing that moved on the earth perished — the birds, and the domesticated animals, and the wild animals, and everything that swarmed on the earth, and all humankind.}%
\verse{Everything in whose nostrils was the breath of life,\lebnote{“a breath of spirit of life”} among all that was on dry land, died.}%
\verse{And he\lebnote{In context, God} blotted out every living thing upon the surface of the ground, from humankind, to animals, to creeping things, and to the birds of heaven;\lnBB{} they were blotted out from the earth. Only Noah and those who were with him in the ark remained.}%
\verse{And the waters prevailed over the earth one hundred and fifty days.}%
\end{biblechapter}%
\begin{biblechapter}% Genesis 8
\verseWithHeading{The Flood Subsides}{And God remembered Noah and all the wild animals, and all the domesticated animals that were with him in the ark. And God caused a wind to blow\lebnote{Or “go”} over the earth, and the waters subsided.}%
\verse{And the fountains of the deep and the windows of the heavens\lnBE{} were closed, and the rain from the heavens\lnBE{} was restrained.}%
\verse{And the waters receded from the earth gradually,\lebnote{“going and returning”} and the waters abated at the end of one hundred and fifty days.}%
\verse{And the ark came to rest in the seventh month, on the seventeenth day of the month, on the mountains of Ararat.}%
\verse{And the waters continued to recede\lebnote{“going and receding”} to the tenth month; in the tenth month, on the first of the month, the tops of the mountains appeared.}%
\verse{And it happened that at the end of forty days Noah opened the window of the ark that he had made.}%
\verse{And he sent out a\lnBF{} raven;\lebnote{Or “crow”} it went to and fro\lebnote{“it went out, going out and returning”} until the waters were dried up from upon the earth.}%
\verse{And he sent out a\lnBF{} dove\lebnote{“and he sent out a dove from him”} to see whether the waters had subsided from upon the ground.}%
\verse{But the dove did not find a resting place for the sole of her foot, and she returned to him into the ark, for the waters were still on the face of the earth. And he stretched out his hand and took her, and brought her to himself into the ark.}%
\verse{And he waited another seven days, and again he sent out\lebnote{“he added to send”} the dove from the ark.}%
\verse{And the dove came to him in the evening,\lebnote{“at the time of the evening”} and behold, a freshly-picked olive tree leaf was in her mouth. And Noah knew that the waters had subsided from upon the earth.}%
\verse{And he waited seven more days,\lebnote{“again another seven days”} and he sent out the dove. But it did not return again to him.}%
\verse{And it happened that, in the six hundred and first year, in the first month, on the first day of the month, the waters dried up from upon the earth. And Noah removed the covering of the ark and looked. And behold, the face of the ground was dried up.}%
\verse{And in the second month, on the twenty-seventh day of the month, the earth was dry.}%
\verse{And God spoke to Noah, saying:}%
\verse{“Go out from the ark, you and your wife, and your sons, and your sons’ wives with you.}%
\verse{Bring out with you all the living things which are with you, from all the living creatures — birds, and animals, and everything that creeps on the earth, and let them swarm on the earth and be fruitful and multiply on the earth.”}%
\verse{So\lebnote{Or “And”} Noah went out, with\lebnote{Or “and”} his sons and his wife, and the wives of his sons with him.}%
\verse{Every animal, every creeping thing, and every bird, and everything that moves upon the earth, according to its families, went out from the ark.}%
\verse{And Noah built an altar to Adonai, and he took from all the clean animals and from all the clean birds, and offered burnt offerings on the altar.}%
\verse{And Adonai smelled the soothing fragrance, and Adonai said to himself,\lebnote{“to his heart”} “Never again will I curse\lebnote{“I will not add to curse again”} the ground for the sake of humankind, because the inclination of the heart of humankind is evil from his youth. Nor will I ever again destroy\lebnote{“And I will not add again to destroy”} all life as I have done.}%
\verse{As long as the earth endures,\lebnote{“While all the days of the earth”} seed and harvest, cold and heat, summer and winter, day and night will not cease.}%
\end{biblechapter}%
\begin{biblechapter}% Genesis 9
\verseWithHeading{God’s Covenant with Noah and Humankind}{And God blessed Noah and his sons, and said to them, “Be fruitful and multiply, and fill the earth.}%
\verse{And fear of you and dread of you shall be upon every animal of the earth, and on every bird of heaven,\lebnote{Or “the sky”} and on everything that moves upon the ground, and on all the fish of the sea. Into your hand they shall be given.}%
\verse{Every moving thing that lives shall be for you as food. As I gave the green plants to you, I have now given you everything.}%
\verse{Only you shall not eat raw flesh with blood in it.\lebnote{“flesh with its life, its blood”}}%
\verse{And\lebnote{Or “only”; the same word that occurs at the beginning of v. 4} your lifeblood\lebnote{“your blood belonging to your life”} I will require; from every animal\lebnote{“from the hand of every animal”} I will require it. And from the hand of humankind, from the hand of each man to his brother I will require the life of humankind.}%
\verse{“As for the one shedding the blood of humankind, by humankind his blood shall be shed, for God made humankind in his own image.}%
\verse{“And you, be fruitful and multiply, swarm on the earth and multiply in it.”}%
\verse{And God said to Noah and to his sons with him,}%
\verse{“As for me, behold, I am establishing my covenant with you and with your seed after you,}%
\verse{and with every living creature that is with you — the birds, the animals, and every animal of the earth with you, from all that came out of the ark to all the animals of the earth.}%
\verse{I am establishing my covenant with you, that never again will all flesh be cut off by the waters of a flood, nor will there ever be a flood that destroys the earth.”}%
\verse{And God said, “This is the sign of the covenant that I am making between me and you, and between every living creature that is with you for future generations.}%
\verse{My bow I have set in the clouds, and it shall be for a sign of the covenant between me and between the earth.}%
\verse{And when I make clouds appear over the earth the bow shall be seen in the clouds.}%
\verse{Then\lnBG{} I will remember my covenant that is between me and you, and between every living creature, with all flesh. And the waters of a flood will never again cause the destruction\lebnote{“be to destroy”} of all flesh.}%
\verse{The bow shall be in the clouds, and I will see it, so as to remember the everlasting covenant between God and between every living creature, with all flesh that is upon the earth.”}%
\verse{And God said to Noah, “This is the sign of the covenant which I am establishing between me and all flesh that is upon the earth.}%
\verseWithHeading{Noah’s Descendants}{Now\lnBG{} the sons of Noah who came out of the ark were Shem, Ham, and Japheth. (Ham was the father of Canaan.)}%
\verse{These three were the sons of Noah, and from these the whole earth was populated.\lebnote{“all the earth branched out”}}%
\verse{And Noah began to be a man of the ground, and he planted a vineyard.}%
\verse{And he drank some of the wine and became drunk, and he exposed himself in the midst of his tent.}%
\verse{And Ham, the father of Canaan, saw the nakedness of his father, and he told his two brothers outside.}%
\verse{Then\lnBG{} Shem and Japheth took a garment, and the two of them put it on their shoulders and, walking backward, they covered the nakedness of their father. And their faces were turned backward, so that\lebnote{Or “and”} they did not see the nakedness of their father.}%
\verse{Then\lnBG{} Noah awoke from his drunkenness,\lebnote{Or “his wine”} and he knew what his youngest son had done to him.}%
\verse{And he said, “Cursed be Canaan, a slave of slaves he shall be to his brothers.”}%
\verse{Then\lnBG{} he said, “Blessed be Adonai, the God of Shem, and let Canaan be a slave to them.}%
\verse{May God make space for Japheth, and let him dwell in the tents of Shem, and let Canaan be a slave for him.”}%
\verse{And Noah lived three hundred and fifty years after the flood.}%
\verse{And all the days of Noah were nine hundred and fifty years, and he died.}%
\end{biblechapter}%
\begin{biblechapter}% Genesis 10
\verseWithHeading{The Descendants of the Sons of Noah}{These are the generations\lnBH{} of the sons of Noah — Shem, Ham, and Japheth. Children\lebnote{Or “sons”} were born to them after the flood.}%
\verse{The sons of Japheth: Gomer, Magog, Madai, Javan, Tubal, Meshech, and Tiras.}%
\verse{And the sons of Gomer: Ashkenaz, Riphath, and Togarmah.}%
\verse{And the sons of Javan: Elishah, Tarshish, Kittim, and Dodanim.}%
\verse{From these the coastland peoples spread out through their lands, each according to his own language by their own families, in their nations.}%
\verse{And the sons of Ham: Cush, Egypt, Put, and Canaan.}%
\verse{And the sons of Cush: Seba, Havilah, Sabtah, Raamah, and Sabteca. The sons of Raamah: Sheba and Dedan.}%
\verse{And Cush fathered Nimrod. He was the first on earth to be a mighty warrior.\lebnote{“He began to be a mighty warrior on the earth”}}%
\verse{He was a mighty hunter before Adonai. Therefore it was said, “Like Nimrod a mighty hunter before Adonai.”}%
\verse{Now,\lebnote{Or “And”} the beginning of his kingdom was Babel, Erech, Akkad, and Calneh, in the land of Shinar.}%
\verse{From that land he went out to Assyria, and he built Nineveh, Rehoboth-Ir, Calah,}%
\verse{Resen between Nineveh and Calah; that is the great city.}%
\verse{And Egypt fathered Ludim, Anamim, Lehabim, Naphtuhim,}%
\verse{Pathrusim, and Casluhim (from whom the Philistines came), and Caphtorim.}%
\verse{Canaan fathered Sidon, his firstborn, and Heth,}%
\verse{and the Jebusites, the Amorites, the Girgashites,}%
\verse{the Hivites, the Arkites, the Sinites,}%
\verse{the Arvadites, the Zemarites, and the Hamathites. Afterward the families of the Canaanites were spread abroad.}%
\verse{And the territory of the Canaanites was from Sidon in the direction of\lnBI{} Gerar as far as Gaza, and in the direction of\lnBI{} Sodom, Gomorrah, Admah, and Zeboiim, as far as Lasha.}%
\verse{These are the descendants\lebnote{Or “sons” (“descendants” was chosen since the proper names in this section are gentilic in form—that is, they indicate one’s place of origin—and have the definite article)} of Ham, according to their families and their languages, in their lands, and in their nations.}%
\verse{And to Shem, the father of all the children of Eber, the older brother of Japheth, children were also born.}%
\verse{The sons of Shem: Elam, Asshur, Arphaxad, Lud, and Aram.}%
\verse{And the sons of Aram: Uz, Hul, Gether, and Mash.}%
\verse{And Arphaxad fathered Shelah, and Shelah fathered Eber.}%
\verse{And to Eber two sons were born. The name of the one was Peleg, for in his days the earth was divided, and the name of his brother was Joktan.}%
\verse{And Joktan fathered Almodad, Sheleph, Hazarmaveth, Jerah,}%
\verse{Hadoram, Uzal, Diklah,}%
\verse{Obal, Abimael, Sheba,}%
\verse{Ophir, Havilah, and Jobab. All these were the sons of Joktan.}%
\verse{And their dwelling place extended from\lebnote{“was from”} Mesha in the direction of\lnBI{} Sephar to the hill country of the east.}%
\verse{These are the sons of Shem, according to their families, according to their languages, in their lands, and according to their nations.}%
\verse{These are the families of the sons of Noah, according to their generations\lnBH{} and in their nations. And from these the nations spread abroad on the earth after the flood.}%
\end{biblechapter}%
\begin{biblechapter}% Genesis 11
\verseWithHeading{The Tower of Babel}{Now\lnBJ{} the whole earth had one language and the same words.}%
\verse{And as people migrated\lebnote{Or “set out”} from the east they found a plain in the land of Shinar and settled there.}%
\verse{And they said to each other,\lebnote{“each to his companion”} “Come, let us make bricks and burn them thoroughly.”\lebnote{“burn to burning”} And they had brick for stone and they had tar for mortar.}%
\verse{And they said, “Come, let us build ourselves a city and a tower whose top reaches to the heavens. And let us make a name for ourselves, lest we be scattered over the face of the whole earth.”}%
\verse{Then\lnBJ{} Adonai came down to see the city and the tower that humankind\lebnote{“sons of Adam” or “sons of humankind”} was building.}%
\verse{And Adonai said, “Behold, they are one people with one language,\lebnote{“one people and one language to all of them”} and this is only the beginning of what they will do.\lebnote{“and this they begin to do”} So\lnBJ{} now nothing that they intend to do will be impossible for them.}%
\verse{Come, let us go down and confuse their language there, so that they will not understand each other’s language.”\lebnote{“each the language of his companion”}}%
\verse{So Adonai scattered them from there over the face of the whole earth, and they stopped building the city.}%
\verse{Therefore its name was called Babel, for there Adonai confused the language of the whole earth, and there Adonai scattered them over the face of the whole earth.}%
\verseWithHeading{The Descendants of Shem}{These are the generations\lnBK{} of Shem. When Shem was one hundred years old,\lebnote{“the son of one hundred years”} he fathered Arphaxad, two years after the flood.}%
\verse{And Shem lived five hundred years after he fathered Arphaxad, and he fathered other sons and daughters.}%
\verse{When Arphaxad had lived thirty-five years, he fathered Shelah.}%
\verse{And Arphaxad lived four hundred and three years after he fathered Shelah, and he fathered other sons and daughters.}%
\verse{When Shelah had lived thirty years, he fathered Eber.}%
\verse{And Shelah lived four hundred and three years after he fathered Eber, and he fathered other sons and daughters.}%
\verse{When Eber had lived thirty-four years, he fathered Peleg.}%
\verse{And Eber lived four hundred and thirty years after he fathered Peleg, and he fathered other sons and daughters.}%
\verse{When Peleg had lived thirty years, he fathered Reu.}%
\verse{And Peleg lived two hundred and nine years after he fathered Reu, and he fathered other sons and daughters.}%
\verse{When Reu had lived thirty-two years, he fathered Serug.}%
\verse{And Reu lived two hundred and seven years after he fathered Serug, and he fathered other sons and daughters.}%
\verse{When Serug had lived thirty years, he fathered Nahor.}%
\verse{And Serug lived two hundred years after he fathered Nahor, and he fathered other sons and daughters.}%
\verse{When Nahor had lived twenty-nine years, he fathered Terah.}%
\verse{And Nahor lived one hundred and nineteen years after he fathered Terah, and he fathered other sons and daughters.}%
\verse{When Terah had lived seventy years, he fathered Abram, Nahor, and Haran.}%
\verseWithHeading{The Descendants of Terah}{Now\lnBJ{} these are the generations\lnBK{} of Terah. Terah fathered Abram, Nahor, and Haran, and Haran fathered Lot.}%
\verse{And Haran died in the presence of Terah his father in the land of his birth, in Ur of the Chaldeans.}%
\verse{And Abram and Nahor took wives for themselves. The name of the wife of Abram was Sarai, and the name of the wife of Nahor was Milcah, the daughter of Haran, the father of Milcah and Iscah.}%
\verse{And Sarai was barren; she had no child.}%
\verse{And Terah took Abram his son, and Lot, the son of Haran, his grandson,\lebnote{“the son of his son”} and Sarai his daughter-in-law, the wife of Abram his son, and went out with them from Ur of the Chaldeans to go to the land of Canaan. And they went to Haran, and they settled there.}%
\verse{And the days of Terah were two hundred and five years, and Terah died in Haran.}%
\end{biblechapter}%
\begin{biblechapter}% Genesis 12
\verseWithHeading{The Call of Abram}{And Adonai said to Abram, “Go out from your land and from your relatives, and from the house of your father, to the land that I will show you.}%
\verse{And I will make you a great nation, and I will bless you, and I will make your name great. And you will be a blessing.}%
\verse{And I will bless those who bless you, and those who curse you I will curse. And all families of the earth will be blessed in you.”}%
\verseWithHeading{Abram’s Journey}{And Abram went out as Adonai had told him, and Lot went with him. Now Abram was seventy-five years old\lebnote{“the son of five years and seventy years”} when he went out from Haran.}%
\verse{And Abram took Sarai his wife, and Lot his nephew,\lebnote{“the son of his brother”} and all their possessions that they had gathered, and all the persons that they had acquired in Haran, and they went out to go to the land of Canaan. And they went to the land of Canaan.}%
\verse{And Abram traveled through the land up to the place of Shechem, to the Oak of Moreh. Now the Canaanites were in the land at that time.}%
\verse{And Adonai appeared to Abram and said, “To your offspring I will give this land.” And he built an altar there to Adonai, who had appeared to him.}%
\verse{And he moved on from there to the hill country, east of Bethel. And he pitched his tent at Bethel on the west, and at Ai on the east. And he built an altar there to Adonai. And he called on the name of Adonai.}%
\verse{And Abram kept moving on,\lebnote{“And Abram moved on, going and moving on”} toward the Negev.}%
\verse{And there was a famine in the land. And Abram went down to Egypt to dwell as an alien there, for the famine was severe in the land.}%
\verse{And it happened that as he drew near to enter into Egypt, he said to Sarai his wife, “Look now, I know that you are a woman beautiful of appearance,}%
\verse{and it shall happen that, if the Egyptians see you, then they will say, ‘This is his wife,’ then they will kill me but let you live.}%
\verse{Please say you are my sister so that it will go well for me on your account. Then I will live\lebnote{“And my soul will live”} on account of you.”}%
\verse{And it happened that as Abram came into Egypt, the Egyptians saw the woman, that she was very beautiful.}%
\verse{And the officials of Pharaoh saw her, and they praised her beauty to Pharaoh. And the woman was taken to the house of Pharaoh.}%
\verse{And he dealt well with Abram on account of her, and he had sheep, cattle, male donkeys, male servants, female servants, female donkeys, and camels.}%
\verse{Then\lebnote{Or “And”} Adonai afflicted Pharaoh and his household with severe plagues on account of the matter of Sarai the wife of Abram.}%
\verse{Then Pharaoh called for Abram and said, “What is this you have done to me? Why did you not tell me that she was your wife?}%
\verse{Why did you say ‘She is my sister,’ so that I took her to myself as a wife? Now then, here is your wife. Take her and go.”}%
\verse{And Pharaoh commanded his men concerning him, and then sent him and his wife and all that was with him away.}%
\end{biblechapter}%
\begin{biblechapter}% Genesis 13
\verseWithHeading{The Parting of Abram and Lot}{Then Abram went up from Egypt, he and his wife and all that was with him. And Lot went with him to the Negev.}%
\verse{Now Abram was very wealthy in livestock, in silver, and in gold.}%
\verse{And he went according to his journey from the Negev, then to Bethel, to the place where his tent was at the beginning, between Bethel and Ai,}%
\verse{to the place where he had made an altar at the beginning. And Abram called on the name of Adonai there.}%
\verse{And Lot, who went with Abram, also had herds and tents.}%
\verse{And the land could not support them\lebnote{“lift them up”} so as to live together, because their possessions were so many that\lebnote{Or “and”} they were not able to live together.}%
\verse{And there was a quarrel between the herdsmen of the livestock of Abram and the herdsmen of the livestock of Lot. Now at that time the Canaanites and the Perizzites were living in the land.}%
\verse{Then Abram said to Lot, “Please, let there not be quarreling between me and you, and between my shepherds and your shepherds, for we men are brothers.}%
\verse{Is not the whole land before you? Separate yourself from me. If you want what is on the left, then I will go right; if you want what is on the right, I will go left.”}%
\verse{And Lot lifted up his eyes and saw the whole plain of the Jordan, that all of it was well-watered land — this was before Adonai destroyed Sodom and Gomorrah — like the garden of Adonai, like the land of Egypt in the direction of\lebnote{“your going”} Zoar.}%
\verse{So\lnBL{} Lot chose for himself all the plain of the Jordan. And Lot journeyed from the east, and so they separated from each other.\lebnote{“each from his brother”}}%
\verse{Abram settled in the land of Canaan, and Lot settled in the cities of the plain. And he pitched his tent toward Sodom.}%
\verse{Now the men of Sodom were extremely wicked sinners against Adonai.\lebnote{“\textit{were} wicked and sinners against Adonai very”}}%
\verse{And Adonai said to Abram after Lot had separated from him, “Now, lift up your eyes and look from the place where you are to the north, and to the south, and to the east and to the west,}%
\verse{for all the land which you see I will give to you, and to your descendants, forever.}%
\verse{I will make your descendants like the dust of the earth which, if anyone were able to count the dust of the earth, your descendants would be so counted.}%
\verse{Arise, go through the length of the land and through its breadth, for I will give it to you.”}%
\verse{So\lnBL{} Abram pitched his tent, and he came and settled at the oaks of Mamre, which were at Hebron. And there he built an altar to Adonai.}%
\end{biblechapter}%
\begin{biblechapter}% Genesis 14
\verseWithHeading{Abram Rescues Lot}{And it happened that in the days of Amraphel, the king of Shinar, Arioch, the king of Ellasar, Kedorlaomer, the king of Elam, and Tidal, the king of Goiim,}%
\verse{made war with Bera, the king of Sodom, and Birsha, the king of Gomorrah, Shinab, the king of Admah, and Shemeber, the king of Zeboiim, and the king of Bela (that is, Zoar).}%
\verse{All these joined forces at the valley of Siddim (that is, the sea of the salt).}%
\verse{Twelve years they had served Kedorlaomer, but in the thirteenth year they rebelled.}%
\verse{In the fourteenth year Kedorlaomer and the kings who were with him came and defeated the Rephaim in Ashteroth-Karnaim, and the Zuzim in Ham, and the Emim in Shaveh-Kiriathaim,}%
\verse{And the Horites in their hill country of Seir, as far as El-Paran, which is at the wilderness.}%
\verse{Then\lnBM{} they turned back and came to En-Mishpat (that is, Kadesh). And they defeated the whole territory of the Amalekites, and also the Amorites who were living in Hazazon-Tamar.}%
\verse{Then\lnBM{} the king of Sodom, the king of Gomorrah, the king of Admah, the king of Zeboiim, and the king of Bela (that is, Zoar) went out, and they took up battle position\lebnote{“And they took up position with them for battle”} in the Valley of Siddim}%
\verse{with Kedorlaomer, king of Elam, and Tidal, king of Goiim, and Amraphel, king of Shinar, and Arioch, king of Ellasar, four kings against five.}%
\verse{Now\lnBM{} the Valley of Siddim was full of tar pits.\lebnote{“pits, pits of tar”} And the kings of Sodom and Gomorrah fled and fell into them,\lebnote{“they fell there”} but the rest fled to the mountains.}%
\verse{So\lnBM{} they\lnBN{} took all the possessions of Sodom and Gomorrah and all their provisions, and they left.}%
\verse{And they\lnBN{} took Lot, the son of the brother of Abram, and his possessions, and left. (Now he had been living in Sodom.)}%
\verse{Then\lnBM{} one who escaped came and told Abram the Hebrew. And he was living at the oaks of Mamre the Amorite, brother of Eshcol and brother of Aner. They were allies with Abram.\lebnote{“And they \textit{were} owners of a covenant of Abram”}}%
\verse{When\lnBM{} Abram heard that his relative\lnBO{} was taken captive, he summoned his trained men, born in his house, three hundred and eighteen of them, and he went in pursuit up to Dan.}%
\verse{And he divided his trained men against them at night, he and his servants. And he defeated them and pursued them to Hobah, which is north of Damascus.}%
\verse{And he brought back all the possessions. And he also brought back Lot, his relative,\lnBO{} and his possessions, and the women and the people as well.}%
\verseWithHeading{Abram Meets Melchizedek}{After his return from defeating Kedorlaomer and the kings who were with him, the king of Sodom went out to meet him at the Valley of Shaveh (that is, the Valley of the King).}%
\verse{And Melchizedek, the king of Salem, brought out bread and wine. (He was the priest of God Most High).}%
\verse{And he blessed him and said, “Blessed be Abram by God Most High, Maker of heaven and earth.}%
\verse{And blessed be God Most High who delivered your enemies into your hand.” And he\lebnote{That is, Abram} gave to him a tenth of everything.}%
\verse{And the king of Sodom said to Abram, “Give me the people, but the possessions take for yourself.”}%
\verse{And Abram said to the king of Sodom, “I have raised my hand to Adonai, God Most High, Maker of heaven and earth,}%
\verse{that neither a thread nor\lebnote{“that not from a thread and up to”} a thong of a sandal would I take from all that belongs to you, that you might not say, ‘I made Abram rich.’}%
\verse{Nothing besides what\lebnote{“besides me only which”} the servants have eaten and the share of the men who went out with me will I take. Let Aner, Eshcol, and Mamre take their share.”}%
\end{biblechapter}%
\begin{biblechapter}% Genesis 15
\verseWithHeading{Adonai’s Covenant with Abram}{After these things the word of Adonai came\lebnote{“was”} to Abram in a vision, saying: “Do not be afraid, Abram; I am your shield, and your reward shall be very great.”}%
\verse{Then\lnBP{} Abram said, “O Adonai, my Lord, what will you give me? I continue to be\lebnote{“I am going”} childless, and my heir\lebnote{“the son of the inheritance of my house”} is Eliezer of Damascus.”}%
\verse{And Abram said, “Look, you have not given me a descendant, and here, a member of my household\lebnote{“a son of my house”} is my heir.”\lebnote{“inheriting me”}}%
\verse{And behold, the word of Adonai came to him saying, “This person will not be your heir,\lebnote{“inherit you”} but your own son will be your heir.”\lebnote{“he who goes out from your body, he will inherit you”}}%
\verse{And he brought him outside and said, “Look toward the heavens and count the stars if you are able to count them.” And he said to him, “So shall your offspring be.”}%
\verse{And he believed in Adonai, and he reckoned it to him as righteousness.}%
\verse{And he said to him, “I am Adonai, who brought you out from Ur of the Chaldeans to give this land to you, to possess it.”}%
\verse{And he said, “O Adonai God, how shall I know that I will possess it?”}%
\verse{And he said to him, “Take for me a three-year-old heifer, and a three-year-old female goat, and a three-year-old ram, and a turtledove and a young pigeon.”}%
\verse{And he took for him all these and cut them in pieces down the middle. And he put each piece opposite the other,\lebnote{“its companion”} but the birds he did not cut.}%
\verse{And the birds of prey came down on the carcasses, but Abram drove them away.}%
\verse{And it happened, as the sun went down,\lebnote{“to go”} then a deep sleep fell upon Abram and, behold, a great terrifying darkness fell upon him.}%
\verse{And he said to Abram, “You must surely know\lebnote{“knowing you must know”} that your descendants shall be as aliens in a land not their own.\lebnote{“not to them”} And they shall serve them\lnBQ{} and they\lnBQ{} shall oppress them four hundred years.}%
\verse{And also the nation that they\lebnote{That is, Abram’s descendants} serve I will judge. Then\lnBP{} afterward they shall go out with great possessions.}%
\verse{And as for you, you shall go to your ancestors\lebnote{Or “fathers”} in peace; you shall be buried in a good old age.}%
\verse{And the fourth generation shall return here, for the guilt of the Amorites is not yet complete.”\lebnote{“is not complete up to now”}}%
\verse{And after the sun had gone down and it was dusk, behold, a smoking firepot and a flaming torch passed between those half pieces.}%
\verse{On that day Adonai made\lebnote{“cut”} a covenant with Abram saying, “To your offspring I will give this land, from the river of Egypt to the great river, the Euphrates river,}%
\verse{the land of the Kenites, the Kenizzites, the Kadmonites,}%
\verse{the Hittites, the Perizzites, the Rephaim,}%
\verse{the Amorites, the Canaanites, the Girgashites, and the Jebusites.”}%
\end{biblechapter}%
\begin{biblechapter}% Genesis 16
\verseWithHeading{Sarai and Hagar}{Now\lnBR{} Sarai, the wife of Abram, had borne him no children. And she had a female Egyptian servant, and her name was Hagar.}%
\verse{And Sarai said to Abram, “Look, please, Adonai has prevented me from bearing children. Please go in to my servant; perhaps I will have children by her.”\lebnote{“I will be built from her”} And Abram listened to the voice of Sarai.}%
\verse{Then\lnBR{} Sarai, the wife of Abram, took Hagar, her Egyptian servant, after Abram had lived ten years in the land of Canaan, and gave her to Abram her husband as his wife.}%
\verse{And he went in to Hagar, and she conceived. And when she\lnBS{} saw that she had conceived, then her mistress grew small in her eyes.}%
\verse{And Sarai said to Abram, “may my harm be upon you. I had my servant sleep with you,\lebnote{“I put my servant in your lap”} and when she saw that she had conceived, she no longer respected me.\lebnote{“I grew small in her eyes”} May Adonai judge between me and you!”}%
\verse{And Abram said to Sarai, “Look, your servant is under your authority.\lebnote{“in your hand”} Do to her that which is good in your eyes.” And Sarai mistreated her, and she\lnBS{} fled from her presence.}%
\verseWithHeading{Hagar and the Angel of Adonai}{And the angel of Adonai found her at a spring of water in the wilderness, at the spring by the road of Shur.}%
\verse{And he said to Hagar, the servant of Sarai, “From where\lebnote{“where from this”} have you come, and where are you going?” And she said, “I am fleeing from the presence of Sarai my mistress.”}%
\verse{Then\lnBR{} the angel of Adonai said to her, “Return to your mistress and submit yourself under her authority.”\lebnote{“her hand”}}%
\verse{And the angel of Adonai said to her, “I will greatly multiply\lebnote{“multiplying I will multiply”} your offspring, so that they cannot be counted for their abundance.”}%
\verse{And the angel of Adonai said to her: “Behold, you are pregnant and shall have a son. And you shall call his name Ishmael, for Adonai has listened to your suffering.}%
\verse{And he shall be a wild donkey of a man, his hand will be against everyone, and the hand of everyone will be against him, and he will live in hostility with all his brothers.”\lebnote{“against the face of all his brothers”}}%
\verse{So\lnBR{} she called the name of Adonai who spoke to her, “You are El-Roi,”\lebnote{“God of seeing”} for she said, “Here I have seen after he who sees me.”}%
\verse{Therefore the well was called Beer-Lahai-Roi; behold, it is between Kadesh and Bered.}%
\verse{And Hagar had a child for Abram, a son. And Abram called the name of his son whom Hagar bore to him, Ishmael.}%
\verse{And Abram was eighty-six years old when Hagar bore Ishmael to Abram.}%
\end{biblechapter}%
\begin{biblechapter}% Genesis 17
\verseWithHeading{Abram and Circumcision, the Sign of the Covenant}{When\lebnote{“And it happened that”} Abram was ninety-nine years old Adonai appeared to Abram. And he said to him, “I am El-Shaddai;\lebnote{Often translated, “God Almighty”} walk before me and be blameless}%
\verse{so that\lebnote{Or “and”} I may make my covenant between me and you, and may multiply you exceedingly.”\lnBT{}}%
\verse{Then Abram fell upon his face and God spoke with him, saying,}%
\verse{“As for me, behold, my covenant shall be with you, and you shall be the father of a multitude of nations.}%
\verse{Your name shall no longer be called Abram, but your name shall be Abraham, for I will make you the father of a multitude of nations.}%
\verse{And I will make you exceedingly\lnBT{} fruitful. I will make you a nation, and kings shall go out from you.}%
\verse{And I will establish my covenant between me and you, and between your offspring after you, throughout their generations as an everlasting covenant to be as God for you and to your offspring after you.}%
\verse{And I will give to you and to your offspring after you the land in which you are living as an alien,\lebnote{“a land of your sojournings”} all the land of Canaan, as an everlasting property. And I will be to them as God.”}%
\verse{And God said to Abraham, “Now as for you, you must keep my covenant, you and your offspring after you, throughout their generations.}%
\verse{This is my covenant which you shall keep, between me and you, and also with\lebnote{Or “between”} your offspring after you: Every male among you shall be circumcised.}%
\verse{And you shall circumcise the flesh of your foreskin, and it shall be a sign of the covenant between me and you.}%
\verse{And at eight days of age\lebnote{“a son of eight days”} you shall yourselves circumcise every male belonging to your generations and the servant born in your house and the one bought from any foreigner\lnBU{} who is not from your offspring.}%
\verse{You must certainly circumcise\lebnote{“circumcising you will circumcise”} the servant born in your house and the one bought from any foreigner.\lnBU{} And my covenant shall be with your flesh as an everlasting covenant.}%
\verse{And as for any uncircumcised male who has not circumcised the flesh of his foreskin, that person shall be cut off from his\lebnote{“her”; the reference is still to a male; the pronoun is feminine because “person” (\textit{nephesh}) is grammatically feminine} people. He has broken my covenant.}%
\verse{And God said to Abraham, “as for Sarai your wife, you shall not call her name Sarai, for Sarah shall be her name.}%
\verse{And I will bless her; moreover, I give to you from her a son. And I will bless her, and she shall give rise to nations.\lebnote{“she shall become to nations”} Kings of peoples shall come\lebnote{Or “be”} from her.”}%
\verse{And Abraham fell upon his face and laughed. And he said in his heart, “Can a child be born to a man a hundred years old,\lebnote{“to a man one hundred years old can he be born?”} or can Sarah bear a child at ninety?”\lebnote{“can ninety-year-old Sarah bear a child?”}}%
\verse{And Abraham said to God, “Oh that Ishmael might live before you!”}%
\verse{And God said, “No, but Sarah your wife shall bear a son for you, and you shall call his name Isaac. And I will establish my covenant with him as an everlasting covenant to his offspring after him.}%
\verse{And as for Ishmael, I have heard you. Behold, I will bless him and I will make him fruitful, and I will multiply him exceedingly.\lnBT{} He shall father twelve princes, and I will make him a great nation.}%
\verse{But my covenant I will establish with Isaac, whom Sarah shall bear to you at this appointed time next year.”}%
\verse{When\lebnote{Or “And”} he finished speaking with him, God went up from Abraham.}%
\verse{And Abraham took Ishmael his son and all who were born of his house, and all those acquired by his money, every male among the men of Abraham’s house, and he circumcised the flesh of their foreskin on the same day that God spoke with him.}%
\verse{Abraham was ninety-nine years old when he circumcised the flesh of his foreskin.}%
\verse{And Ishmael his son was thirteen years old when he circumcised the flesh of his foreskin.}%
\verse{Abraham and his son Ishmael were circumcised on the same day.}%
\verse{And all the men of his house, those born in the house, and those acquired by money from a foreigner,\lebnote{“born of house and acquisition of money from the son of a foreigner”} were circumcised with him.}%
\end{biblechapter}%
\begin{biblechapter}% Genesis 18
\verseWithHeading{Adonai Appears to Abraham as a Man}{And Adonai appeared to him\lebnote{Abraham} by the oaks of Mamre. And he was sitting in the doorway of the tent at the heat of the day.}%
\verse{And he lifted up his eyes and saw, and behold, three men were standing near\lebnote{Or “by”; the context requires some distance for Abraham to run to them} him. And he saw them and ran from the doorway of the tent to meet them. And he bowed down to the ground.}%
\verse{And he said, “My lord, if I have found favor in your eyes do not pass by your servant.}%
\verse{Let a little water be brought and wash your feet, and rest under the tree.}%
\verse{And let me bring a piece of bread, then\lnBV{} refresh yourselves.\lebnote{“your heart”} Afterward you can pass on, once\lebnote{“for therefore”} you have passed by with your servant.” Then they said, “Do so as you have said.”}%
\verse{Then\lnBW{} Abraham hastened into the tent to Sarah, and he said, “Quickly — make three seahs of fine flour for kneading and make bread cakes!”}%
\verse{And Abraham ran to the cattle and took a calf,\lebnote{“a son of cattle”} tender and good, and gave it to the servant, and he made haste to prepare it.}%
\verse{Then he took curds and milk, and the calf which he prepared, and set it before them. And he was standing by them under the tree while\lnBV{} they ate.}%
\verse{And they said to him, “Where is Sarah your wife?” And he said, “Here, in the tent.”}%
\verse{And he\lnBX{} said, “I will certainly return to you in the spring,\lnBY{} and look, Sarah your wife will have a son.” Now Sarah was listening at the doorway of the tent, and which was behind him.}%
\verse{Now Abraham and Sarah were old, advanced in age;\lebnote{“going in the days”} the way of women\lebnote{“the road according to women”} had ceased to be for Sarah.}%
\verse{So\lnBW{} Sarah laughed to herself saying, “After I am worn out and my husband is old, shall this pleasure be to me?”}%
\verse{Then Adonai said to Abraham, “What is this that Sarah laughed, saying, ‘Is it indeed true that I will bear a child, now that I have grown old?’}%
\verse{Is anything too difficult for Adonai? At the appointed time I will return to you in the spring\lnBY{} and Sarah shall have a son.”}%
\verse{But Sarah denied it, saying, “I did not laugh,” because she was afraid. He\lnBX{} said, “No, but you did laugh.”}%
\verse{Then the men set out from there, and they looked down upon Sodom. And Abraham went with them to send them on their way.\lebnote{“to send them away”}}%
\verse{Then Adonai said, “Shall I conceal from Abraham what I am going to do?}%
\verse{Abraham will surely become a great and strong nation, and all the nations of the earth will be blessed on account of him.}%
\verse{For I have chosen\lebnote{Or “have known”} him, that he will command his children and his household after him that they will keep the way of Adonai, to do righteousness and justice, so that Adonai may bring upon Abraham that which he said to him.”}%
\verse{Then Adonai said, “Because the outcry of Sodom and Gomorrah is great and because their sin is very serious,\lebnote{“heavy”}}%
\verse{I will go down and I will see. Have they done altogether according to its cry of distress which has come to me? If not, I will know.”}%
\verseWithHeading{Abraham Intercedes for Sodom}{And the men turned from there and went toward Sodom. And Abraham was still standing before Adonai.}%
\verse{And Abraham drew near to Adonai and said, “Will you also sweep away the righteous with the wicked?}%
\verse{If perhaps there are fifty righteous in the midst of the city, will you also sweep them away and not forgive the place on account of the fifty righteous in her midst?}%
\verse{Far be it from you to do such a thing as this, to kill the righteous with the wicked, that\lnBV{} the righteous would be as the wicked! Far be it from you! Will not the Judge of all the earth do justice?”}%
\verse{And Adonai said, “If I find fifty righteous in Sodom, in the midst of the city, then I will forgive the whole place for their sake.”}%
\verse{Then Abraham answered and said, “Look, please, I was bold to speak to my Lord, but I am dust and ashes.}%
\verse{Perhaps the fifty righteous are lacking five — will you destroy the whole city on account of the five?” And he answered, “I will not destroy it if I find forty-five there.”}%
\verse{And once again he spoke\lebnote{“And he added again to speak”} to him and said, “What if\lnBZ{} forty are found there?” And he answered, “I will not do it on account of the forty.”}%
\verse{And he said, “Please, let not my Lord be angry, and I will speak. What if\lnBZ{} thirty be found there?” And he answered, “I will not do it if I find thirty there.”}%
\verse{And he said, “Please, now, I was bold to speak to my Lord. What if\lnBZ{} twenty be found there?” And he answered, “I will not destroy it for the sake of the twenty.”}%
\verse{And he said, “Please, let not my Lord be angry, and I will speak only once more. What if\lnBZ{} ten are found there?” And he answered, “I will not destroy it for the sake of the ten.”}%
\verse{Then Adonai left,\lebnote{Or “went”} as he finished speaking to Abraham, and Abraham returned to his place.}%
\end{biblechapter}%
\begin{biblechapter}% Genesis 19
\verseWithHeading{The Rescue of Lot from Sodom}{And the two angels came to Sodom in the evening. And Lot was sitting in the gateway of Sodom. Then Lot saw them and stood up to meet them. And he bowed down with his face to the ground.}%
\verse{And he said, “Behold, my lords, please turn aside into the house of your servant and spend the night and wash your feet. Then you can rise early and go on your way.” And they said, “No, but we will spend the night in the square.”}%
\verse{But he urged them strongly,\lebnote{“he pressed upon them very”} and they turned aside with him and came into his house. And he made a meal for them and baked unleavened bread, and they ate.}%
\verse{Before they laid down, the men of the city, the men of Sodom, both young and old, all the people to the last man,\lebnote{“from the end”} surrounded the house.}%
\verse{And they called to Lot and said to him, “Where are the men who came to you tonight? Bring them out to us so that we may know\lnCA{} them.”}%
\verse{But Lot went out to them at the entrance, and he shut the door behind him.}%
\verse{And he said, “No, my brothers, please do not do such a wrong thing.}%
\verse{Behold, I have two daughters who have not known\lnCA{} a man. Please, let me bring them out to you; then do to them as it seems good in your eyes. Only to these men do not do this thing, since they came under my roof\lebnote{“my beam”} for protection.”}%
\verse{But they said, “Step aside!” Then they said, “This fellow\lebnote{“the one”} came to dwell as a foreigner and he acts as a judge! Now we shall do worse to you than them!” And they pressed very hard against the man, against Lot, and they drew near to break the door.}%
\verse{Then the men reached out with their hands and brought Lot in to them, into the house, and they shut the door.}%
\verse{And the men who were at the entrance of the house they struck with blindness, both small and great, and they were unable to find the entrance.}%
\verse{Then the men said to Lot, “Who is here with you? Bring out from the place your sons-in-law, and your sons and your daughters, and all who are with you in the city.}%
\verse{For we are about to destroy this place, because their cry has become great before Adonai. Adonai sent us to destroy it.”}%
\verse{Then Lot went out and spoke to his sons-in-law who were taking\lebnote{That is, marrying} his daughters and said, “Get up! Go out from this place, because Adonai is going to destroy the city!” But it seemed like a joke\lebnote{“it was as one making fun”} in the eyes of his sons-in-law.}%
\verse{And as the dawn came up the angels urged Lot saying, “Get up, take your wife and your two daughters who are staying with you,\lebnote{“who are found”} lest you be destroyed with the punishment of the city.”}%
\verse{But when he lingered, the men seized him by his hand and his wife’s hand, and his two daughters by hand, on account of the mercy of Adonai upon him. And they brought him out and set him outside of the city.}%
\verse{And after bringing them outside one said, “Flee for your life; do not look behind you, and do not stand anywhere in the plain. Flee to the mountains lest you be destroyed.”}%
\verse{And Lot said to them, “No, please, my lords.}%
\verse{Behold, your servant has found favor in your eyes and you have shown me great kindness\lebnote{“your kindness which you have done to me you have made great”} in saving my life. But I cannot flee to the mountains, lest the disaster overtake me and I die.}%
\verse{Behold, this city is near enough to flee there, and it is a little one. Please, let me flee there. Is it not a little one? Then my life shall be saved.”}%
\verse{And he said to him, “Behold, I will grant this favor as well;\lebnote{“I am lifting up your face also concerning this thing”} that I will not overthrow the city of which you have spoken.}%
\verse{Escape there quickly, for I cannot do this thing until you get there.” Therefore, there name of the city was called Zoar.}%
\verseWithHeading{The Destruction of Sodom}{After the sun had risen\lebnote{“had gone out”} upon the earth and Lot had entered Zoar,}%
\verse{Adonai rained down from heaven upon Sodom and Gomorrah brimstone and fire from Adonai.}%
\verse{And he overthrew those cities and the whole plain, and the inhabitants of the cities and the vegetation of the ground.}%
\verse{But his\lebnote{That is, Lot’s} wife looked back,\lebnote{“behind him”} and she became a pillar of salt.}%
\verse{And Abraham arose early in the morning and went to the place where he had stood before Adonai.}%
\verse{And he looked down upon the surface of Sodom and Gomorrah, and upon the whole surface of the land, the plain. And he saw that,\lebnote{Or “and”} behold, the smoke of the land went up like the smoke of a smelting furnace.}%
\verse{So it was, when God destroyed the cities of the plain that God remembered Abraham and sent Lot out from the midst of the overthrow, when he overthrew the cities in which Lot lived.}%
\verseWithHeading{Lot and His Daughters}{And Lot went out from Zoar and settled in the hill country\lebnote{Or “mountain”} with his two daughters, for he was afraid to stay in Zoar. So he lived in a cave, he and his two daughters.}%
\verse{And the firstborn daughter said to the younger one, “Our father is old, and there is no man in the land to come in to us according to the manner of all the land.}%
\verse{Come, let us give our father wine to drink and let us lie with him\lebnote{Idiom for “have sexual intercourse with him”} that we may secure descendants through our father.”\lnCB{}}%
\verse{And they gave their father wine to drink that night, and the firstborn went and lay with her father, but he did not know when she lay down or when she got up.}%
\verse{And it happened that, the next day the firstborn said to the younger one, “Look, I laid with my father last night. Let us give him wine to drink also tonight, then go and lie with him that we may secure descendants through our father.”\lnCB{}}%
\verse{And they gave their father wine to drink again that night, and the younger got up and lay with him, but he did not know when she lay down or when she got up.}%
\verse{And the two daughters of Lot became pregnant by their father.}%
\verse{The firstborn gave birth to a son, and she called his name Moab. He is the father of Moab until this day.}%
\verse{And the younger, she also gave birth to a son, and she called his name Ben-Ammi. He is the father of the Ammonites\lebnote{“sons/children of Ammon”} until this day.}%
\end{biblechapter}%
\begin{biblechapter}% Genesis 20
\verseWithHeading{Abraham and Abimelech}{And Abraham journeyed from there to the land of the Negev, and he settled between Kadesh and Shur. And he dwelled as an alien in Gerar.}%
\verse{And Abraham said about Sarah his wife, “She is my sister.” And Abimelech king of Gerar sent and took Sarah.}%
\verse{And God came to Abimelech in a dream at night. And he said to him, “Look, you are a dead man on account of the woman you have taken. For she is a married woman.”\lebnote{“married of a husband” or “a woman owned by a husband”}}%
\verse{Now Abimelech had not approached her, so he said, “my Lord, will you even kill a righteous people?”}%
\verse{Did not he himself say to me, ‘She is my sister’? And she herself said, ‘He is my brother.’ With integrity of my heart and with cleanness of my hands I did this.”}%
\verse{Then God said to him in the dream, “Yes,\lebnote{Or “Also”} I know that in the integrity of your heart you did this, and I also kept you from sinning\lebnote{“kept back from sinning”} against me. Therefore, I did not allow you to touch her.}%
\verse{So now, return the wife of the man, for he is a prophet, so that he will pray for you and you will live. And if you do not return her,\lebnote{“If there is not for you a returning”} know that you will certainly die, and all that are yours.”}%
\verse{So Abimelech rose early in the morning. And he called all his servants and told them all these things,\lebnote{“said all these things in their ears”} and the men were very afraid.}%
\verse{And Abimelech called for Abraham and said to him, “What have you done to us? And how have I sinned against you that you brought upon me and upon my kingdom a great sin? You have done things to me that should not be done.”}%
\verse{And Abimelech said to Abraham, “What were you thinking\lebnote{“What have you seen?”} that you did this thing?”}%
\verse{And Abraham said, “Because I thought, surely there is no fear of God in this place; they will kill me on account of the matter of my wife.}%
\verse{Besides,\lebnote{“And also moreover”} she is my sister, the daughter of my father, but not the daughter of my mother. And she became my wife.}%
\verse{And it happened that as God caused me to wander from the house of my father I said to her, ‘This is your loyal kindness that you must do for me at every place where we come: say concerning me, “He is my brother.”’”}%
\verse{And Abimelech took sheep and cattle and male slaves and female slaves, and he gave them to Abraham. And he returned Sarah his wife to him.}%
\verse{And Abimelech said, “Here is my land before you; settle where it pleases you.”\lebnote{“in the good in your eyes”}}%
\verse{And to Sarah he said, “Look, I have given a thousand pieces of silver to your brother. It shall be an exoneration.\lebnote{“a covering of the eyes”} You are vindicated before all who are with you.”\lebnote{“to all who are with you and with all you are vindicated”}}%
\verse{And Abraham prayed to God, and God healed Abimelech and his wife and his female servants so that they could bear children again.}%
\verse{For Adonai had completely closed up all the wombs of the house of Abimelech because of the matter of Sarah, the wife of Abraham.}%
\end{biblechapter}%
\begin{biblechapter}% Genesis 21
\verseWithHeading{The Birth of Isaac}{And Adonai visited Sarah as he had said. And Adonai did to Sarah as he had promised.}%
\verse{And she conceived, and Sarah bore to Abraham a son in his old age at the appointed time that God had told him.}%
\verse{And Abraham called the name of his son who was born to him, whom Sarah bore to him, Isaac.}%
\verse{And Abraham circumcised Isaac his son when he was eight days old, as God had commanded him.}%
\verse{And Abraham was one hundred years old when Isaac his son was born to him.}%
\verse{And Sarah said, “God has made laughter for me; all who hear will laugh for me.”}%
\verse{And she said, “Who would announce to Abraham that Sarah would nurse children? Yet I have borne a son to Abraham in his old age.”}%
\verseWithHeading{Hagar and Ishmael}{And the child grew and was weaned. And Abraham made a great feast on the day Isaac was weaned.}%
\verse{And Sarah saw the son of Hagar the Egyptian, whom she had borne Abraham, mocking.}%
\verse{Then she said to Abraham, “Drive out this slave woman and her son, for the son of this slave woman will not be heir with my son, with Isaac.”}%
\verse{And the matter displeased Abraham very much\lebnote{“was very bad in the eyes of Abraham”} on account of his son.}%
\verse{Then God said to Abraham, “Do not be displeased\lebnote{“Do not let it be bad”} on account of the boy and on account of the slave woman. Listen to everything that Sarah said to you,\lebnote{“Everything Sarah said to you, listen with respect to her voice”} for through Isaac your offspring will be named.}%
\verse{And I will also make the son of the slave woman into a nation, for he is your offspring.”}%
\verse{Then Abraham rose up early in the morning and took bread and a skin of water and gave it to Hagar, putting it on her shoulder. And he sent her away with the child, and she went, wandering about in the wilderness, in Beersheba.}%
\verse{And when the water was finished from the skin, she put the child under one of the bushes.}%
\verse{And she went and she sat a good distance away,\lebnote{“she sat with respect to her, opposite, far away, as the shooting of a bow”} for she said, “Let me not see the child’s death.” So she sat away from him and lifted up her voice and wept.}%
\verse{And God heard the cry\lnCC{} of the boy and the angel of God called to Hagar from the heavens and said to her, “What is the matter\lebnote{“What to you”} Hagar? Do not be afraid, for God has heard the cry\lnCC{} of the boy from where he is.\lebnote{“in which he \textit{is} there”}}%
\verse{Get up, take up the boy and take him with your hand, for I will make him a great nation.”}%
\verse{And God opened her eyes, and she saw a well of water. And she went and filled the skin with water and gave a drink to the boy.}%
\verse{And God was with the boy, and he grew and lived in the wilderness. And he became an expert with a bow.\lebnote{“an archer with the bow”}}%
\verse{And he lived in the wilderness of Paran. And his mother took a wife for him from the land of Egypt.}%
\verseWithHeading{The Covenant Between Abraham and Abimelech}{And it happened that at that time, Abimelech and Phicol, the commander of his army, said to Abraham, “God is with you, in all that you do.}%
\verse{So now, swear to me here by God that you will not deal with me falsely, or with my descendants, or my posterity. According to the kindness that I have done to you, you shall pledge to do with me and with the land where you have dwelled as an alien.”}%
\verse{And Abraham said, “I swear.”}%
\verse{Then Abraham complained to Abimelech on account of the well of water that servants of Abimelech had seized.}%
\verse{And Abimelech said, “I do not know who did this thing, neither did you tell me, nor have I heard of it except for today.”}%
\verse{And Abraham took sheep and cattle and gave them to Abimelech. And the two of them made\lnCD{} a covenant.}%
\verse{Then Abraham set off seven ewe-lambs of the flock by themselves.}%
\verse{And Abimelech said to Abraham, “What is the meaning of these seven ewe-lambs that you have set off by themselves?”}%
\verse{And he said, “You shall take the seven ewe-lambs from my hand as proof on my behalf\lebnote{“for the sake that it shall be a witness for me”} that I dug this well.”}%
\verse{Therefore that place is called Beersheba, because there the two of them swore an oath.}%
\verse{And they made\lnCD{} a covenant at Beersheba. And Abimelech, and Phicol his army commander stood and returned to the land of the Philistines.}%
\verse{And he\lebnote{That is, Abraham} planted a tamarisk tree in Beersheba, and there he called on the name of Adonai, the everlasting God.\lebnote{“El-Olam”}}%
\verse{And Abraham dwelled as an alien in the land of the Philistines many days.}%
\end{biblechapter}%
\begin{biblechapter}% Genesis 22
\verseWithHeading{God Tests Abraham}{And it happened that after these things, God tested Abraham. And he said to him, “Abraham!” And he said, “Here I am.”}%
\verse{And he said, “Take your son, your only child, Isaac, whom you love, and go to the land of Moriah, and offer him there as a burnt offering on one of the mountains where\lebnote{Or “that”} I will tell you.”}%
\verse{And Abraham rose up early in the morning and saddled his donkey. And he took two of his servants with him, and Isaac his son. And he chopped wood for a burnt offering. And he got up and went to the place which God had told him.}%
\verse{On the third day Abraham lifted up his eyes, and he saw the place at a distance.}%
\verse{And Abraham said to his servants, “You stay here with the donkey, and I and the boy will go up there. We will worship, then we will return to you.”}%
\verse{And Abraham took the wood of the burnt offering and placed it on Isaac his son. And he took the fire in his hand and the knife, and the two of them went together.}%
\verse{And Isaac said to Abraham his father, “My father!” And he said, “Here I am, my son.” And he said, “Here is the fire and the wood, but where is the lamb for a burnt offering?”}%
\verse{And Abraham said, “God will provide\lebnote{“God will see to it” or “God will see for himself”} the lamb for a burnt offering, my son.” And the two of them went together.}%
\verse{And they came to the place that God had told him. And Abraham built an altar there and arranged the wood. Then he bound Isaac his son and placed him on the altar atop the wood.}%
\verse{And Abraham stretched out his hand and took the knife to slaughter his son.}%
\verse{And the angel of Adonai called to him from heaven and said, “Abraham! Abraham!” And he said, “Here I am.”}%
\verse{And he said, “Do not stretch out your hand against the boy; do not do anything to him. For now I know that you are one who fears\lebnote{“a fearer”} God, since you have not withheld your son, your only child, from me.”}%
\verse{And Abraham lifted up his eyes and looked. And behold, a ram was caught in the thicket by his horns. And Abraham went and took the ram, and offered it as a burnt offering in place of his son.}%
\verse{And Abraham called the name of that place “Adonai will provide,”\lebnote{“will see”} for which reason\lebnote{“which”} it is said today, “on the mountain of Adonai it shall be provided.”\lebnote{“it/he shall be seen”}}%
\verse{And the angel of Adonai called to Abraham a second time from heaven.}%
\verse{And he said, “I swear by myself, declares Adonai, that because you have done this thing and have not withheld your son, your only child,}%
\verse{that I will certainly bless you and greatly multiply your offspring as the stars of heaven, and as the sand that is by the shore of the sea. And your offspring will take possession of the gate of his enemies.}%
\verse{All the nations of the earth will be blessed through your offspring, because you have listened to my voice.”}%
\verse{And Abraham returned to his servants, and they got up and went together to Beersheba. And Abraham lived in Beersheba.}%
\verse{And it happened that after these things, it was told to Abraham, “Look, Milcah has also borne children to your brother Nahor:}%
\verse{Uz his firstborn and Buz his brother, and Kemuel the father of Aram,}%
\verse{and Kesed, Hazo, Pildash, Jidlaph, and Bethuel.”}%
\verse{(Now, Bethuel fathered Rebekah). These eight Milcah bore to Nahor, the brother of Abraham.}%
\verse{And his concubine, whose name was Reumah, also bore Tebah, Gaham, Tahash, and Maacah.}%
\end{biblechapter}%
\begin{biblechapter}% Genesis 23
\verseWithHeading{Sarah’s Death and Burial}{And Sarah lived\lebnote{“the lives of Sarah were”} one hundred and twenty-seven years; these were the years of the life of Sarah.}%
\verse{And Sarah died in Kiriath Arba; that is Hebron, in the land of Canaan.}%
\verse{And Abraham went to mourn for Sarah and to weep for her. And Abraham rose up from his dead, and he spoke to the Hittites\lnCE{} and said,}%
\verse{“I am a stranger and an alien among you; give to me my own burial site\lnCF{} among you so that I may bury my dead from before me.”}%
\verse{And the Hittites\lnCE{} answered Abraham and said to him,}%
\verse{“Hear us, my lord, you are a mighty prince\lebnote{Or “prince of God”} in our midst. Bury your dead in the choicest of our burial sites. None of us will withhold his burial site\lebnote{“will withhold from himself his burial site} from you for burying your dead.”}%
\verse{And Abraham rose up and bowed to the people of the land, to the Hittites.\lnCE{}}%
\verse{And he spoke with them, saying, “If you are willing\lebnote{“if there is with your inner selves / souls”} that I bury my dead from before me, hear me and intercede for me with Ephron the son of Zohar,}%
\verse{that he may sell\lnCG{} to me the cave of Machpelah which belongs to him,\lebnote{“which \textit{is} to him”} which is at the end of his field. At full value let him sell\lnCG{} it to me in your midst as a burial site.”\lnCF{}}%
\verse{Now Ephron was sitting among the Hittites.\lnCE{} And Ephron the Hittite answered Abraham in the hearing of the Hittites\lnCE{} with respect to all who were entering the gate of his city, and said,}%
\verse{“No, my lord, hear me. I give you the field and the cave which is in it, I also give it to you in the sight of the children of my people I give it to you. Bury your dead.”}%
\verse{And Abraham bowed before the people of the land.}%
\verse{And he spoke to Ephron in the hearing of the people of the land, saying, “If only you will hear me\lebnote{“Only if you perhaps hear me”} — I give the price of the field. Take it from me that I may bury my dead there.”}%
\verse{And Ephron answered Abraham, saying to him,}%
\verse{“My lord, hear me. A piece of land worth four hundred shekels of silver — what is that between me and you? Bury your dead.”}%
\verse{Then Abraham listened to Ephron, and Abraham weighed for Ephron the silver that he had named\lebnote{Or “spoken”} in the hearing of the Hittites:\lnCE{} four hundred shekels of silver at the merchants’ current rate.\lebnote{“passing to the merchant”}}%
\verse{So the field of Ephron which was in the Machpelah, which was near Mamre — the field and the cave which was in it, with all the trees that were in the field, which were within all its surrounding boundaries — passed\lnCH{}}%
\verse{to Abraham as a property in the presence of the Hittites,\lnCE{} with respect to all who were entering the gate of his city.}%
\verse{And thus afterward Abraham buried Sarah his wife in the cave of the field of Machpelah near Mamre; that is Hebron, in the land of Canaan.}%
\verse{And the field and the cave which was in it passed\lnCH{} to Abraham as a burial site\lnCF{} from the Hittites.\lnCE{}}%
\end{biblechapter}%
\begin{biblechapter}% Genesis 24
\verseWithHeading{Isaac and Rebekah}{Now Abraham was old, advanced in age,\lebnote{“going in the days”} and Adonai had blessed Abraham in everything.}%
\verse{And Abraham said to his servant, the oldest of his house, who had charge of all he had, “Please put your hand under my thigh}%
\verse{that I may make you swear by Adonai, the God of heaven and the God of earth, that you will not take a wife for my son from the daughters of the Canaanites in whose midst I am dwelling,}%
\verse{but that you will go to my land and to my family, and take a wife for my son, for Isaac.”}%
\verse{And the servant said to him, “Perhaps the woman will not be willing to follow\lnCI{} me to this land — must I then return your son to the land from whence you came?”}%
\verse{Abraham said to him, “You must take care\lebnote{“let it be careful to you”} that you do not return my son there.}%
\verse{Adonai, the God of heaven who took me from the house of my father and from the land of my family, and who spoke to me and swore to me, saying, ‘to your offspring I will give this land,’ he will send his angel before you, and you shall take a wife for my son from there.}%
\verse{And if the woman is not willing to follow\lnCI{} you, then you shall be released from this oath of mine — only you must not return my son there.”}%
\verse{Then the servant put his hand under the thigh of Abraham his master, and he swore to him concerning this matter.}%
\verse{And the servant took ten camels from his master’s camels, and he went with all kinds of his master’s good things in his hand. And he arose and went to Aram-Naharaim, to the city of Nahor.}%
\verse{And he made the camels kneel outside the city at the well of water, at the time of evening, toward the time the women went out to draw water.}%
\verse{And he said, “O Adonai, God of my master Abraham, please grant me success today and show loyal love to my master Abraham.}%
\verse{Behold, I am standing by the spring of water, and the daughters of the men of the city are going out to draw water.}%
\verse{And let it be that the girl to whom I shall say, ‘Please, offer your jar that I may drink’ and who says, ‘Drink — and I will also water your camels,’ she is the one you have chosen for your servant, for Isaac. By her I will know that you have shown loyal love to my master.”}%
\verse{And it happened that before he had finished speaking, behold, Rebekah — who was born to Bethuel, son of Milcah, the wife of Nahor, the brother of Abraham — came out, and her jar was on her shoulder.}%
\verse{Now the girl was very pleasing in appearance. She was a virgin; no man had known her. And she went down to the spring, filled her jar, and came up.}%
\verse{And the servant ran to meet her. And he said, “Please, let me drink a little of the water from your jar.”}%
\verse{And she said, “Drink, my lord.” And she quickly lowered her jar in her hand and gave him a drink.}%
\verse{When\lebnote{Or “And”} she finished giving him a drink she said, “I will also draw water for your camels until they finish drinking.”}%
\verse{And she quickly emptied her jar into the trough and ran again to the well to draw water. And she drew water for all his camels.}%
\verse{And the man was gazing at her silently to know if Adonai had made his journey successful or not.}%
\verse{And it happened that as the camels finished drinking the man took a gold ring of a half shekel in weight and two bracelets for her arms, ten shekels in weight,}%
\verse{and said, “Please tell me, whose daughter are you? Is there a place at the house of your father for us to spend the night?”}%
\verse{And she said to him, “I am the daughter of Bethuel, son of Milcah, whom she bore to Nahor.”}%
\verse{Then she said to him, “We have both straw and fodder in abundance, as well as a place to spend the night.”}%
\verse{And the man knelt down and worshiped Adonai.}%
\verse{And he said, “Blessed be Adonai, God of my master Abraham, who has not withheld\lebnote{Or “abandoned”} his loyal love and his faithfulness from my master. I was on the way and Adonai led me to the house of my master’s brother.”}%
\verse{Then the girl ran and reported these things to the household of her mother.}%
\verse{Now Rebekah had a brother, and his name was Laban. And Laban ran out to the man toward the spring.}%
\verse{And when he saw the ring and the bracelets on the arms of his sister and heard the words of Rebekah his sister, who said, “Thus the man spoke to me,” he went to the man. And behold, he was standing with the camels at the spring.}%
\verse{And he said, “Come, O blessed one of Adonai. Why do you stand outside? Now I have prepared the house and a place for the camels.”}%
\verse{And the man came to the house and unloaded the camels. And he gave straw and fodder to the camels, and water to wash his feet and the feet of the men who were with him.}%
\verse{And food was placed before him\lebnote{“And it was placed before him”} to eat. And he said, “I will not eat until I have told my errand.”\lebnote{“I have spoken my word”} And he said, “Speak.”}%
\verse{And he said, “I am the servant of Abraham.}%
\verse{Now Adonai has blessed my master exceedingly, and he has become great. He has given to him sheep and cattle, silver and gold, male slaves and female slaves, and camels and donkeys.}%
\verse{And Sarah, the wife of my master, has borne a son to my master after her old age. And he\lebnote{That is, Abraham} has given to him all that he has.}%
\verse{And my master made me swear, saying, ‘Do not take a wife for my son from the daughters of the Canaanites in whose land I am living.}%
\verse{But you shall go to the house of my father, and to my family, and you shall take a wife for my son.’}%
\verse{And I said to my master, ‘Perhaps the woman will not follow\lebnote{“go after”} me.’}%
\verse{And he said to me, ‘Adonai, before whom I have walked, shall send his angel with you and will make your journey successful. And you shall take a wife for my son from my family, and from the house of my father.}%
\verse{Then you shall be released from my oath, when you come to my family. And if they will not give a woman to you, then you will be released from my oath.’}%
\verse{Then today I came to the spring, and I said, ‘O Adonai, God of my master Abraham, if you would please make my journey successful,\lebnote{“if there is you making successful my journey”} upon which I am going.}%
\verse{Behold, I am standing by the spring of water. Let it be that the young woman who comes out to draw water and to whom I say, “Please give me a little water to drink from your jar,”}%
\verse{let her say to me, “Drink; I will also draw water for your camels,” she is the woman whom Adonai has appointed for the son of my master.’}%
\verse{I had not yet finished speaking to myself\lebnote{Or “to my heart”} when, behold, Rebekah was coming out with\lebnote{Or “and”} her jar on her shoulder. And she went down to the spring and drew water. And I said to her, ‘Please give me a drink.’}%
\verse{And she hastened and let down her jar from her shoulder\lebnote{“from upon her”} and said, ‘Drink, and I will give a drink to your camels also.’ Then I drank and she gave a drink to the camels also.}%
\verse{Then I asked her and said, ‘Whose daughter are you?’ And she said, ‘The daughter of Bethuel, son of Nahor, whom Milcah bore to him.’ And I put the ring on her nose and the bracelets on her arms.}%
\verse{And I knelt down and worshiped Adonai, and I praised Adonai, the God of my master Abraham, who led me on the right way, to take the daughter of the brother of my master for his son.}%
\verse{So now, if you are going to deal loyally and truly\lebnote{“if there is you doing loyal love and faithfulness”} with my master, tell me. And if not, tell me, so that I may turn to the right or to the left.”\lebnote{An idiom for “so that I might know what to do”}}%
\verse{Then Laban and Bethuel answered, and they said, “The matter has gone out from Adonai; we are not able to speak bad or good to you.}%
\verse{Here is Rebekah before you. Take her and go; let her be a wife for the son of your master as Adonai has spoken.”}%
\verse{And it happened that when the servant of Abraham heard their words he bowed down to the ground to Adonai.}%
\verse{And the servant brought out silver jewelry and gold jewelry, and garments, and he gave them to Rebekah. And he gave precious gifts to her brother and to her mother.}%
\verse{And he and the men who were with him ate and drank, and they spent the night. And they got up in the morning, and he said, “Let me go to my master.”}%
\verse{And her brother and her mother said, “Let the girl remain with us ten days or so; after that she may go.”}%
\verse{And he said to them, “Do not delay me. Now, Adonai has made my journey successful. Let me go. I must go to my master.”}%
\verse{And they said, “Let us call the girl and ask her opinion.”\lebnote{“her mouth”}}%
\verse{And they called Rebekah and said to her, “Will you go with this man?” And she said, “I will go.”}%
\verse{So they sent away Rebekah their sister, and her nurse, and the servant of Abraham and his men.}%
\verse{And they blessed Rebekah and said to her, “You are our sister; may you become countless thousands; and may your offspring take possession of the gate of his enemies.”}%
\verse{And Rebekah and her maidservants arose, and they mounted the camels and followed\lebnote{“went after”} the man. And the servant took Rebekah and left.}%
\verse{Now Isaac was coming from the direction of Beer-Lahai-Roi. And he was living in the land of the Negev.}%
\verse{And Isaac went out to meditate in the field early in the evening,\lebnote{“at the turning of evening”} and he lifted up his eyes and saw — behold, camels were coming.}%
\verse{And Rebekah lifted up her eyes and saw Isaac. And she got down from the camel.}%
\verse{And she said to the servant, “Who is this man walking around in the field to meet us?” And the servant said, “That is my master.” And she took her\lebnote{Or “the”} veil and covered herself.}%
\verse{And the servant told Isaac all the things that he had done.}%
\verse{And Isaac brought her to the tent of Sarah his mother. And he took Rebekah, and she became his wife. And Isaac loved her and was comforted after the death of his mother.}%
\end{biblechapter}%
\begin{biblechapter}% Genesis 25
\verseWithHeading{The Death and Descendants of Abraham}{Now Abraham again took a wife, and her name was Keturah.}%
\verse{And she bore to him Zimran, Jokshan, Medan, Midian, Ishbak, and Shuah.}%
\verse{And Jokshan fathered Sheba and Dedan. And the sons of Dedan were Asshurim and Letushim and Leummim.}%
\verse{And the sons of Midian were Ephah, Epher, Hanoch, Abidah, and Eldaah. All of these were the children of Keturah.}%
\verse{And Abraham gave all he had to Isaac.}%
\verse{But to the sons of Abraham’s concubines Abraham gave gifts. And while he was still living he sent them away eastward, away from his son Isaac, to the land of the east.}%
\verse{Now these are the days of the years of the life of Abraham:\lebnote{“the years of the life of Abraham which he lived”} one hundred and seventy-five years.}%
\verse{And Abraham passed away and died in a good old age, old and full of years. And he was gathered to his people.}%
\verse{And Isaac and Ishmael his sons buried him in the cave of Machpelah, in the field of Ephron, son of Zohar the Hittite, that was east of Mamre,}%
\verse{the field that Abraham had bought from the Hittites.\lebnote{Or “sons of Heth”} There Abraham was buried and Sarah his wife.}%
\verse{And it happened that after the death of Abraham God blessed Isaac his son, and Isaac settled at Beer-Lahai-Roi.}%
\verse{Now these are the generations\lnCJ{} of Ishmael, the son of Abraham, that Hagar the Egyptian, the maidservant of Sarah, bore to Abraham.}%
\verse{And these are the names of the sons of Ishmael, by their names according to their family records. The firstborn of Ishmael was Nebaioth, then Kedar, Adbeel, Mibsam,}%
\verse{Mishma, Dumah, Massa,}%
\verse{Hadad, Tema, Jetur, Naphish, and Kedemah.}%
\verse{These are the sons of Ishmael, and these are their names by their villages and by their encampments — 12 leaders according to their tribes.}%
\verse{Now these are the years of the life of Ishmael: 137 years. And he passed away and died, and was gathered to his people.}%
\verse{They settled from Havilah to Shur, which was opposite\lnCK{} Egypt, going toward Asshur, opposite;\lnCK{} he settled\lebnote{“fell”} opposite\lnCK{} all his brothers.}%
\verseWithHeading{Jacob and Esau}{Now these are the generations\lnCJ{} of Isaac, the son of Abraham. Abraham fathered Isaac,}%
\verse{And Isaac was forty years old\lebnote{“a son of forty years”} when he took Rebekah, the daughter of Bethuel the Aramean of Paddan-Aram, the sister of Laban the Aramean, as his wife.}%
\verse{And Isaac prayed to Adonai on behalf of his wife, for she was barren. And Adonai responded to his prayer, and Rebekah his wife conceived.}%
\verse{And the children in her womb jostled each other, and she said, “If it is going to be like this, why be pregnant?”\lebnote{“if so, why this I?”} And she went to inquire of Adonai.}%
\verse{And Adonai said to her, “Two nations are in your womb, and two peoples from birth\lebnote{“from your bowels”} shall be divided. And one people shall be stronger than the other.\lebnote{“people than people shall be stronger”} And the elder shall serve the younger.”}%
\verse{And when her days to give birth were completed,\lebnote{Or “full”} then — behold — twins were in her womb.}%
\verse{And the first came out red, all his body\lebnote{“of him”} was like a hairy coat, so they called his name Esau.}%
\verse{And afterward his brother came out, and his hand grasped the heel of Esau, so his name was called Jacob. And Isaac was sixty years old\lebnote{“a son of sixty years”} at their birth.}%
\verse{And the boys grew up. And Esau was a skilled\lebnote{Or “knowing” (knowledgeable)} hunter, a man of the field, but Jacob was a peaceful man, living in tents.}%
\verse{And Isaac loved Esau because he could eat of his game,\lebnote{“game in his mouth”} but Rebekah loved Jacob.}%
\verse{Once\lebnote{Or “and”} Jacob cooked a thick stew, and Esau came in from the field, and he was exhausted.}%
\verse{And Esau said to Jacob, “Give me some of that red stuff\lebnote{“some of the red, this red”} to gulp down, for I am exhausted!” (Therefore his name was called Edom).}%
\verse{Then Jacob said, “Sell me your birthright first.”\lnCL{}}%
\verse{And Esau said, “Look, I am going to die; now what is this birthright to me?”}%
\verse{Then Jacob said, “Swear to me first.”\lnCL{} And he swore to him, and sold his birthright to Jacob.}%
\verse{Then Jacob gave Esau bread, and thick lentil stew, and he ate and drank. Then he got up and went away. So Esau despised his birthright.}%
\end{biblechapter}%
\begin{biblechapter}% Genesis 26
\verseWithHeading{Isaac and Abimelech}{And there was a famine in the land, besides the former famine which was in the days of Abraham. And Isaac went to Abimelech, king of the Philistines, to Gerar.}%
\verse{And Adonai appeared to him and said, “Do not go down to Egypt; dwell in the land which I will show to you.}%
\verse{Dwell as an alien in this land, and I will be with you, and will bless you, for I will give all these lands to you and to your descendants, and I will establish the oath that I swore to Abraham you father.}%
\verse{And I will multiply your descendants like the stars of heaven, and I will give to your descendants all these lands. And all nations of the earth will be blessed through your offspring,}%
\verse{because Abraham listened to my voice and kept my charge: my commandments, my statutes, and my laws.”}%
\verse{So Isaac settled in Gerar.}%
\verse{When the men of the place asked concerning his wife, he said, “She is my sister,” for he was afraid to say, “my wife,” thinking\lebnote{Or “lest”} “the men of the place will kill me on account of Rebekah, for she was beautiful.”\lebnote{“good of appearance”}}%
\verse{And it happened that, when he had been there a long time,\lebnote{“when the days there were long to him”} Abimelech the king of the Philistines looked through the window, and saw — behold — Isaac was fondling Rebekah his wife.}%
\verse{And Abimelech called Isaac\lebnote{Or “to Isaac”} and said, “Surely she is your wife. Now why did you say ‘She is my sister’?” And Isaac said to him, “Because I thought I would die on account of her.”}%
\verse{And Abimelech said, “What is this you have done to us? One of the people might easily have slept with your wife! Then you would have brought guilt upon us!”}%
\verse{Then Abimelech instructed all the people, saying, “The one who touches this man or his wife shall certainly die.”}%
\verse{And Isaac sowed in that land and reaped in that same year a hundredfold, and Adonai blessed him.}%
\verse{And the man became wealthier and wealthier\lebnote{“became great and went, going and became great”} until he was exceedingly wealthy.}%
\verse{And he possessed sheep and cattle and many servants, so that the Philistines envied him.}%
\verse{And the Philistines stopped up all the wells that the servants of his father had dug in the days of Abraham his father. They filled them with earth.}%
\verse{And Abimelech said to Isaac, “Go away from us, for you have become much too powerful for us.”}%
\verse{So Isaac departed from there and camped in the valley of Gerar, and settled there.}%
\verse{And Isaac dug again the wells of water which they had dug in the days of his father Abraham, which the Philistines had stopped up after the death of Abraham. And he gave\lnCM{} to them the same names\lebnote{“names as names”} which his father had given\lnCM{} them.}%
\verse{And when the servants of Isaac dug in the valley, they found a well of fresh water there.}%
\verse{Then the herdsmen of Gerar quarreled with the herdsmen of Isaac, saying, “The water is ours.” And he called the name of the well Esek, because they contended with him.}%
\verse{And they dug another well, and they quarreled over it also. And he called its name Sitnah.}%
\verse{Then he moved from there and dug another well, and they did not quarrel over it. And he called its name Rehoboth, and said, “Now Adonai has made room for us, and we shall be fruitful in the land.”}%
\verse{And from there he went up to Beersheba.}%
\verse{And Adonai appeared to him that night and said, “I am the God of your father Abraham. Do not be afraid, for I am with you, and I will bless you and make your descendants numerous for the sake of my servant Abraham.”}%
\verse{And he built an altar there and called on the name of Adonai. And he pitched his tent there, and the servants of Isaac dug a well there.}%
\verse{Then Abimelech went to him from Gerar with Ahuzzath his friend and Phicol his army commander.}%
\verse{And Isaac said to them, “Why have you come to me? You hate me and sent me away from you.”}%
\verse{And they said, “We see clearly that Adonai has been with you, so we thought\lebnote{Or “said”} let there be an oath between us — between us and you — and let us make\lebnote{“cut”} a covenant with you}%
\verse{that you may not do us harm just as we have not touched you, but have only done good to you and sent you away in peace. You are now blessed by Adonai.”}%
\verse{So he made a meal for them, and they ate and drank.}%
\verse{And they arose early in the morning and each one swore to the other,\lebnote{Or “to his brother”} and Isaac sent them away. And they left him in peace.}%
\verse{And it happened that on that same day the servants of Isaac came and told him about the well that they had dug. And they said, “We have found water!”}%
\verse{And he called it Sheba. Therefore the name of the city is Beersheba unto this day.}%
\verse{And when Esau was forty years old he took as wife Judith, daughter of Beeri the Hittite, and Basemath, daughter of Elon the Hittite.}%
\verse{And they made life bitter\lebnote{“they caused bitterness of spirit”} for Isaac and Rebekah.}%
\end{biblechapter}%
\begin{biblechapter}% Genesis 27
\verseWithHeading{Jacob Steals Esau’s Blessing}{And it happened that when Isaac was old and his eyesight was weak,\lebnote{“his eyes were weak from seeing”} he called Esau his older son and said to him, “My son.” And he\lebnote{That is, Esau} said to him, “Here I am.”}%
\verse{And he said, “Look, I am old; I do not know the day of my death.}%
\verse{So now, take your weapons, your quiver and your bow, and go out to the field and hunt food for me.}%
\verse{Then make for me tasty food like I love, and bring it to me. And I will eat it so that I\lnCN{} can bless you before I die.}%
\verse{Now Rebekah was listening as Isaac spoke to Esau his son, and when Esau went to the field to hunt wild game to bring back,}%
\verse{Rebekah said to Jacob her son, “Look, I heard your father speaking to Esau your brother saying,}%
\verse{‘Bring wild game to me and prepare tasty food so I can eat it and bless you before Adonai before my death.’}%
\verse{So now, my son, listen to my voice, to what I command you.}%
\verse{Go to the flock and take two good young goats from it for me, and I will prepare them as tasty food for your father, just as he likes.}%
\verse{Then you must take it to your father and he will eat it so that he may bless you before his death.”}%
\verse{Then Jacob said to his mother, “Behold, Esau my brother is a hairy man, but I am a smooth man.}%
\verse{Perhaps my father will feel me and I will be in his eyes as a mocker, and he will bring upon me a curse and not a blessing.”}%
\verse{Then his mother said to him, “Your curse be upon me, my son, only listen to my voice — go and get them for me.”}%
\verse{So he went and took them, and brought them to his mother, and his mother prepared tasty food as his father liked.}%
\verse{Then Rebekah took some of her older son Esau’s best garments that were with her in the house, and she put them on Jacob her younger son.}%
\verse{And she put the skins of the young goats over his hands and over the smooth part of his neck.}%
\verse{And she put the tasty food and the bread that she had made into the hand of Jacob, her son.}%
\verse{And he went to his father and said, “My father.” And he said, “Here I am. Who are you, my son?”}%
\verse{And Jacob said to his father, “I am Esau, your firstborn. I have done as you told me. Please get up, sit up and eat from my wild game so that you\lnCO{} may bless me.”}%
\verse{Then Isaac said to his son, “How\lebnote{“what is this?”} did you find it so quickly, my son?” And he said, “Because Adonai your God caused me to find it.”\lebnote{“made it to happen before me”}}%
\verse{Then Isaac said to Jacob, “Please, come near and let me feel you, my son. Are you really\lnCP{} my son Esau or not?”}%
\verse{And Jacob drew near to Isaac his father. And he felt him and said, “The voice is the voice of Jacob, but the hands are the hands of Esau.”}%
\verse{And he did not recognize him because his hands were hairy like the hands of Esau his brother. And he blessed him.}%
\verse{And he said, “Are you really\lnCP{} my son Esau?” And he said, “I am.”}%
\verse{Then he said, “Bring it near to me that I may eat from the game of my son, so that I\lnCN{} may bless you.” And he brought it to him, and he ate. And he brought wine to him, and he drank.}%
\verse{Then his father Isaac said to him, “Come near and kiss me, my son.”}%
\verse{And he drew near and kissed him. And he\lebnote{That is, Isaac} smelled the smell of his garments, and he blessed him and said, “Look, the smell of my son is like the smell of a field that Adonai has blessed!}%
\verse{May God give you of the dew of heaven and of the fatness of the earth, and abundance of grain and new wine.}%
\verse{Let peoples serve you, and nations bow down to you; Be lord of your brothers, and may the sons of your mother bow down to you. Cursed be those cursing you, and blessed be those blessing you.”}%
\verse{And as soon as Isaac had finished blessing Jacob, immediately after\lebnote{“it was only”} Jacob had gone out from the presence of Isaac his father, Esau his brother came back from his hunting.}%
\verse{He too prepared tasty food and brought it to his father. And he said to his father, “Let my father arise and eat from the wild game of his son, that you\lnCO{} may bless me.”}%
\verse{And Isaac his father said to him, “Who are you?” And he said, “I am your son, your firstborn, Esau.”}%
\verse{Then Isaac trembled violently.\lebnote{“trembled a great trembling unto exceeding”} Then he said, “Who then was he that hunted wild game and brought it to me, and I ate it all before you came, and I blessed him? Moreover, he will be blessed!”}%
\verse{When Esau heard the words of his father he cried out with a great and exceedingly bitter cry of distress. And he said to his father, “Bless me as well, my father!”}%
\verse{And he said, “Your brother came in deceit and took your blessing.”}%
\verse{Then he said, “Isn’t that why he is named Jacob?\lebnote{“that his name is called Jacob?”} He has deceived me these two times. He took my birthright and, look, now he has taken my blessing!” Then he said, “Have you not reserved a blessing for me?”}%
\verse{Then Isaac answered and said to Esau, “Behold, I have made him lord over you and I have given him all his brothers as servants, and with grain and wine I have sustained him. Now what can I do for you, my son?”}%
\verse{And Esau said to his father, “Have you only one blessing, my father? Bless me also, my father!” And Esau lifted up his voice and wept.}%
\verse{Then Isaac his father answered and said to him, “Your home shall be from\lebnote{Or “away from”} the fatness of the land, and from the dew of heaven above.}%
\verse{But by your sword you shall live, and you shall serve your brother. But it shall be that when free yourself you shall tear off his yoke from your neck.}%
\verse{Then Esau held a grudge against Jacob on account of the blessing with which his father had blessed him. And Esau said in his heart,\lebnote{That is, “to himself”} “The days of mourning for my father are coming, then I will kill Jacob my brother.”}%
\verse{But the words of Esau her older son were told to Rebekah. And she sent and called for her younger son Jacob. And she said to him, “Look, Esau your brother is consoling himself concerning you, intending to kill you.}%
\verse{Now then, my son, listen to my voice; arise and flee to Haran to Laban my brother.}%
\verse{Stay with him a few days until the wrath of your brother has turned —}%
\verse{until the anger of your brother turns from you and he has forgotten what you have done to him. Then I will send and bring you from there. Why should I lose the two of you in one day?”}%
\verse{Then Rebekah said to Isaac, “I loathe my life because of the Hittite women.\lnCQ{} If Jacob takes a wife from Hittite women\lnCQ{} like these, from the native women,\lebnote{“daughters of the land”} what am I living for?”\lebnote{“What is life to me?”}}%
\end{biblechapter}%
\begin{biblechapter}% Genesis 28
\verseWithHeading{Jacob Flees to Haran}{Then Isaac called Jacob and blessed him. And he instructed him and said to him, “You must not take a wife from the daughters of Canaan.}%
\verse{Arise, go to Paddan-Aram, to the house of Bethuel, your mother’s father, and take for yourself a wife from there, from the daughters of Laban your mother’s brother.}%
\verse{Now, may El-Shaddai\lebnote{Often translated “God Almighty”} bless you, and make you fruitful, and multiply you, so that you become an assembly of peoples.}%
\verse{And may he give you the blessing of Abraham, to you and to your descendants with you, that you may take possession of the land of your sojourning, which God gave to Abraham.”}%
\verse{Then Isaac sent Jacob away, and he went to Paddan-Aram, to Laban the son of Bethuel the Aramean, the brother of Rebekah, the mother of Jacob and Esau.}%
\verse{Now Esau saw that Isaac had blessed Jacob and sent him away to Paddan-Aram, to take for himself a wife from there, and\lebnote{Or “when”} he blessed him and instructed him, saying, “You must not take a wife from the daughters of Canaan,”}%
\verse{and that Jacob listened to his father and to his mother and went to Paddan-Aram.}%
\verse{Then Esau saw that the daughters of Canaan were evil in the eyes of Isaac his father,}%
\verse{then Esau went to Ishmael and took Mahalath, the daughter of Ishmael, son of Abraham, sister of Nebaioth, as a wife, in addition to the wives he had.}%
\verseWithHeading{Jacob’s Dream}{Then Jacob went out from Beersheba and went to Haran.}%
\verse{And he arrived at a certain place and spent the night there, because the sun had set. And he took one of the stones of the place and put it under his head and slept at that place.}%
\verse{And he dreamed, and behold, a stairway was set on the earth, and its top touched the heavens. And behold, angels of God were going up and going down on it.}%
\verse{And behold, Adonai was standing beside him,\lebnote{Or “above it”} and he said, “I am Adonai, the God of Abraham your father, and the God of Isaac. The ground on which you were sleeping I will give to you and to your descendants.}%
\verse{Your descendants shall be like the dust of the earth, and you will spread out to the west, and to the east, and to the north and to the south. And all the families of the earth will be blessed through you and through your descendants.}%
\verse{Now behold, I am with you, and I will keep you wherever you go. And I will bring you to this land, for I will not leave you until I have done what I have promised to you.”}%
\verse{Then Jacob awoke from his sleep and said, “Surely Adonai is indeed\lebnote{“there is”} in this place and I did not know!”}%
\verse{Then he was afraid and said, “How awesome is this place! This is nothing else than the house of God,\lebnote{“there is not this but the house of God”} and this is the gate of heaven!”}%
\verse{And Jacob rose early in the morning, and he took the stone that he had put under his head and set it up as a stone pillar, and poured oil on top of it.}%
\verse{And he called the name of that place Bethel; however, the name of the city was formerly Luz.}%
\verse{And Jacob made a vow\lebnote{Or “vowed a vow”} saying, “If God will be with me and protect me on this way that I am going, and gives me food to eat and clothing to wear,}%
\verse{and if I return in peace to the house of my father, then Adonai will become my God.}%
\verse{And this stone that I have set up as a pillar shall be the house of God, and of all that you give to me I will certainly give a tenth to you.”}%
\end{biblechapter}%
\begin{biblechapter}% Genesis 29
\verseWithHeading{Jacob Flees to Haran}{And Jacob continued his journey\lebnote{“lifted up his feet”} and went to the land of the Easterners.\lebnote{Or “children of the east”}}%
\verse{And he looked, and behold, there was a well in the field, and behold, there were three flocks of sheep lying beside it, for out of that well the flocks were watered. And the stone on the mouth of the well was large.}%
\verse{And when all the flocks were gathered there, they rolled away the stone from the mouth of the well. And they watered the sheep and returned the stone upon the mouth of the well to its place.}%
\verse{And Jacob said to them, “My brothers, where are you from?” And they said, “We are from Haran.”}%
\verse{And he said to them, “Do you know Laban, son of Nahor?” And they said, “We know him.”}%
\verse{And he said to them, “Is he well?”\lebnote{“Is it well for him?”} And they said, “He is well. Now look, Rachel his daughter is coming with the sheep.”}%
\verse{And he said, “Look, it is still broad daylight;\lebnote{“high day”} it is not the time for the livestock to be gathered. Give water to the sheep and go, pasture them.”}%
\verse{And they said, “We are not able, until all the flocks are gathered. Then the stone is rolled away from the mouth of the well, and we water the sheep.”}%
\verse{While he was speaking with them, Rachel came with the sheep which belonged to her father, for she was pasturing them.}%
\verse{And it happened that, when Jacob saw Rachel, the daughter of Laban, his mother’s brother, and the sheep of Laban, his mother’s brother, Jacob drew near and rolled away the stone from the mouth of the well and watered the sheep of Laban, his mother’s brother.}%
\verse{And Jacob kissed Rachel, and lifted up his voice and wept.}%
\verse{And Jacob told Rachel that he was the relative of her father, and that he was the son of Rebekah. And she ran and told her father.}%
\verse{And it happened that when Laban heard the message about Jacob, the son of his sister, he ran to meet him. And he embraced him and kissed him, and brought him to his house. And he told Laban all these things.}%
\verse{And Laban said to him, “Surely you are my flesh and my bone!” And he stayed with him a month.}%
\verseWithHeading{Jacob’s Marriages}{Then Laban said to Jacob, “Just because you are my brother should you work for me for nothing? Tell me what your wage should be.”}%
\verse{Now Laban had two daughters. The name of the older was Leah, and the name of the younger was Rachel.}%
\verse{Now the eyes of Leah were dull, but Rachel was beautiful in form and appearance.}%
\verse{And Jacob loved Rachel and said, “I will serve you seven years for Rachel your younger daughter.”}%
\verse{Then Laban said, “Better that I give her to you than I give her to another man. Stay with me.”}%
\verse{And Jacob worked for Rachel seven years, but they were as a few days in his eyes because he loved her.}%
\verse{And Jacob said to Laban, “Give me my wife, that I may go in to her, for my time\lebnote{“my days”} is completed.”}%
\verse{So Laban gathered all the men of the place and prepared a feast.}%
\verse{And it happened that in the evening he took Leah his daughter and brought her to him, and he went in to her.}%
\verse{And Laban gave Zilpah his female servant to her, to Leah his daughter as a female servant.}%
\verse{And it happened that in the morning, behold, it was Leah! And he said to Laban, “What is this you have done to me? Did I not serve with you for Rachel? Now why did you deceive me?”}%
\verse{Then Laban said, “It is not the custom\lebnote{“it is not thus done”} in our country to give the younger before the firstborn.}%
\verse{Complete the week of this one,\lebnote{Leah; the wedding feast would last one week} then I will also give you the other, on the condition that you will work for me\lebnote{“with respect to the work that you will work with me yet”} another seven years.”}%
\verse{And Jacob did so. So he completed the week of this one,\lebnote{That is, Leah} then he gave Rachel his daughter to him as a wife.}%
\verse{And Laban gave Bilhah his female servant to Rachel his daughter as a female servant.}%
\verse{Then he also went in to Rachel, and he loved Rachel more than Leah. And he served with him yet another seven years.}%
\verseWithHeading{Jacob’s Children}{When Adonai saw that Leah was unloved he opened her womb, but Rachel was barren.}%
\verse{Then Leah conceived and gave birth to a son, and she called his name Reuben, for she said, “Because Adonai has noticed my misery, that I am unloved. Now my husband will love me.”}%
\verse{And she conceived again and gave birth to a son. And she said, “It is because Adonai has heard that I am unloved that he gave me this son also.” And she called his name Simeon.}%
\verse{And she conceived again and gave birth to a son. Then she said, “Now this time my husband will be joined to me, for I have borne him three sons.” Therefore, she called his name Levi.}%
\verse{And she conceived again and gave birth to a son. And she said, “This time I will praise Adonai.” Therefore she called his name Judah. And she ceased bearing children.}%
\end{biblechapter}%
\begin{biblechapter}% Genesis 30
\verseWithHeading{Jacob’s Children}{When Rachel saw that she could not bear children to Jacob, Rachel envied her sister. And she said to Jacob, “Give me children — if not, I will die!”}%
\verse{And Jacob became angry\lebnote{“became hot of nose”} with Rachel. And he said, “Am I in the place of God, who has withheld from you the fruit of the womb?”}%
\verse{Then she said, “Here is my servant girl Bilhah; go in to her that she may bear children as my surrogate.\lebnote{“upon my knees”} Then I will even have children\lebnote{“be built up”} by her.”}%
\verse{Then she gave him Bilhah, her female servant, as a wife, and Jacob went in to her}%
\verse{And Bilhah conceived and gave birth to a son for Jacob.}%
\verse{Then Rachel said, “God has judged me, and has also heard my voice, and has given me a son.” Therefore she called his name Dan.}%
\verse{And Bilhah, Rachel’s servant, conceived again and bore a second son to Jacob.}%
\verse{And Rachel said, “I have struggled a mighty struggle with my sister and have prevailed.” And she called his name Naphtali.}%
\verse{When Leah saw that she had ceased bearing children, she took Zilpah her female servant and gave her to Jacob as a wife.}%
\verse{And Zilpah, the female slave of Leah, bore a son to Jacob.}%
\verse{Then Leah said, “Good fortune!” And she called his name Gad.}%
\verse{And Zilpah, Leah’s female servant, bore a second son to Jacob.}%
\verse{Then Leah said, “How happy am I! For women have called me happy.” So she called his name Asher.}%
\verse{And in the days of the wheat harvest, Reuben went and found mandrakes in the field and he brought them to Leah his mother. And Rachel said to Leah, “Please give me some of your son’s mandrakes.”}%
\verse{And she said to her, “Is your taking my husband such a small thing that you will also take the mandrakes of my son?” Then Rachel said, “Then he may sleep with you tonight in exchange for your son’s mandrakes.”}%
\verse{When Jacob came in from the field in the evening, Leah went out to meet him. And she said, “Come in to me, for I have hired\lebnote{“I have fully paid for”} you with my son’s mandrakes.” And he slept with her that night.}%
\verse{And God listened to Leah and she conceived and gave birth to a fifth son for Jacob.}%
\verse{Then Leah said, “God has given me my wage since I gave my servant girl to my husband.” And she called his name Issachar.}%
\verse{And Leah conceived again and gave birth to a sixth son for Jacob.}%
\verse{And Leah said, “God has endowed me with a good gift. This time my husband will acknowledge me, because I bore him six sons.” And she called his name Zebulun.}%
\verse{And afterward she gave birth to a daughter. And she called her name Dinah.}%
\verse{Then God remembered Rachel and listened to her, and God opened her womb.}%
\verse{And she conceived and gave birth to a son. And she said, “God has taken away my disgrace.”}%
\verse{And she called his name Joseph, saying, “Adonai has added to me another son.”}%
\verseWithHeading{Jacob’s Prosperity}{And it happened that as soon as Rachel had given birth to Joseph, Jacob said to Laban, “Send me away that I may go to my place and my land.}%
\verse{Give me my wives and my children for which I have served you, and let me go. For you yourself know my service that I have rendered to you.”}%
\verse{But Laban said to him, “Please, if I have found favor in your eyes, I have learned by divination that Adonai has blessed me because of you.”}%
\verse{And he said, “Name your wage to me and I will give it.”}%
\verse{Then he said to him, “You yourself know how I have served you and how your livestock have been with me.\lebnote{I.e., “How well they have done under my care”}}%
\verse{For you had little before me, and it has increased abundantly. And Adonai has blessed you wherever I turned.\lebnote{“to my foot”} So then, when shall I provide for my own family also?”}%
\verse{And he said, “What shall I give you?” And Jacob said, “Do not give me anything. If you will do this thing for me, I will again feed your flocks and keep them.}%
\verse{Let me pass through all your flocks today, removing all the speckled and spotted sheep from them, along with\lebnote{Or “and”} every dark-colored sheep among the sheep, and the spotted and speckled among the goats. That\lebnote{Or “it”} shall be my wages.}%
\verse{And my righteousness will answer for me later\lebnote{“on the day tomorrow”} when you come concerning my wages before you. Every one that is not speckled or spotted among the goats, or dark-colored among the sheep shall be stolen if it is with me.”}%
\verse{Then Laban said, “Look! Very well. It shall be according to your word.”}%
\verse{But that day he\lebnote{That is, Laban} removed the streaked and spotted male goats and all the speckled and spotted female goats, all that had white on it, and every dark-colored ram, and put them in the charge of his sons.\lebnote{“he gave into the hands of his sons”}}%
\verse{And he put a journey of three days between him and Jacob, and Jacob pastured the remainder of Laban’s flock.}%
\verse{Then Jacob took fresh branches of poplar, almond, and plane trees and peeled white strips on them, exposing the white which was on the branches.}%
\verse{And he set the branches that he had peeled in front of the flocks, in the troughs and in the water containers. And they were in heat when they came to drink.}%
\verse{And the flocks mated by the branches, so the flocks bore streaked, speckled, and spotted.}%
\verse{And Jacob separated the lambs and turned the faces of the flocks toward the streaked and all the dark-colored in Laban’s flocks. And he put his own herds apart, and did not put them with the flocks of Laban.}%
\verse{And whenever any of the stronger of the flocks were in heat, Jacob put the branches in full view\lebnote{“before the eyes of”} of the flock in the troughs that they might mate among the branches.}%
\verse{But with the more feeble of the flock he would not put them there. So the feebler were Laban’s and the stronger were Jacob’s.}%
\verse{And the man became exceedingly\lebnote{“very, very”} rich and had large flocks, female slaves, male slaves, camels, and donkeys.}%
\end{biblechapter}%
\begin{biblechapter}% Genesis 31
\verseWithHeading{Jacob Flees from Laban}{Now he\lebnote{That is, Jacob} heard the words of the sons of Laban, saying, “Jacob has taken all that our father has,” and “From that which was our father’s he has gained all this wealth.”}%
\verse{Then Jacob saw the face of Laban and, behold, it was not like it had been in the past.\lebnote{“there was not with him like yesterday or the day before”}}%
\verse{And Adonai said to Jacob, “Return to the land of your ancestors\lebnote{Or “fathers”} and to your family, and I will be with you.”}%
\verse{So Jacob sent and called Rachel and Leah to the field, to his flocks,}%
\verse{and he said to them, “Look, I see the face of your father, that it is not like it has been toward me in the past.\lebnote{“it is not to me like yesterday or the day before”} But the God of my father is with me.}%
\verse{Now you yourselves know that I have served your father with all my strength,}%
\verse{and your father has cheated me and changed my wages ten times, but God has not allowed him to harm me.}%
\verse{If thus he said, ‘Speckled shall be your wage,’ then all the flock bore speckled. And if he said, ‘Streaked shall be your wage,’ then all the flock bore streaked.}%
\verse{God has taken away your father’s livestock and given them to me.}%
\verse{Now it happened that at the time of the mating of the flock I lifted up my eyes and saw in a dream, and behold, the rams mounting the flock were streaked, speckled, and dappled.}%
\verse{Then the angel of God said to me in the dream, ‘Jacob,’ and I said, ‘Here I am.’}%
\verse{And he said, ‘Lift up your eyes and see — all the rams mounting the flock are streaked, speckled, and dappled, for I have seen all that Laban is doing to you.}%
\verse{I am the God of Bethel where you anointed a stone pillar, where you made a vow to me. Now get up, go out from this land and return to the land of your birth.’”}%
\verse{Then Rachel and Leah answered and said to him, “Is there yet a portion for us, and an inheritance in the house of our father?}%
\verse{Are we not regarded as foreigners by him, because he has sold us and completely consumed our money?}%
\verse{For all the wealth that God has taken away from our father, it belongs to us and to our sons. So now, all that God has said to you, do.”}%
\verse{Then Jacob got up and put his children and his wives on the camels.}%
\verse{And he drove all his livestock and his possessions that he had acquired, the livestock of his possession that he had acquired in Paddan-Aram, in order to go to Isaac his father, to the land of Canaan.}%
\verse{Now Laban had gone to shear his sheep, and Rachel stole the idols\lebnote{Hebrew \textit{teraphim}} that belonged to her father.}%
\verse{And Jacob tricked\lebnote{“stole the heart of”} Laban the Aramean by not telling him that he intended to flee.}%
\verse{Then he fled with all that he had, and arose and crossed the Euphrates\lebnote{Or “the river”} and set his face toward the hill country of Gilead.}%
\verse{And on the third day it was told to Laban that Jacob had fled.}%
\verse{Then he took his kinsmen with him and pursued after him, a seven-day journey, and he caught up with him in the hill country of Gilead.}%
\verse{And God came to Laban the Aramean in a dream at night and said to him, “Take care\lnCR{} that you not speak with Jacob, whether good or evil.”}%
\verse{And Laban overtook Jacob. Now Jacob had pitched his tent in the hill country, and Laban and his kinsmen pitched their tents in the hill country of Gilead.}%
\verse{Then Laban said to Jacob, “What have you done that you tricked me\lebnote{“stole my heart”} and have carried off my daughters like captives of the sword?}%
\verse{Why did you hide your intention to flee and trick me,\lebnote{“steal my heart”} and did not tell me so that I would have sent you away with joy and song and tambourine and lyre?}%
\verse{And why did you not give me opportunity to kiss my grandsons\lnCS{} and my daughters goodbye? Now you have behaved foolishly by doing this.}%
\verse{It is in my power\lebnote{“there is power in my hand”} to do harm to you, but the God of your father spoke to me last night saying, ‘Take care\lnCR{} from speaking with Jacob, whether good or evil.’}%
\verse{Now, you have surely gone because you desperately longed for the house of your father, but why did you steal my gods?”}%
\verse{Then Jacob answered and said to Laban, “Because I was afraid, for I thought, ‘Lest you take your daughters from me by force.’}%
\verse{But with whomever you find your gods, he shall not live. In the presence of your kinsmen now identify what is with me that is yours and take it.” Now Jacob did not know that Rachel had stolen them.}%
\verse{Then Laban went into Jacob’s tent and Leah’s tent and the tent of the two female servants and did not find his gods. And he came out of Leah’s tent and went into Rachel’s tent.}%
\verse{Now Rachel had taken the idols and put them in the saddle bag of the camel and sat on them. And Jacob searched the whole tent thoroughly but did not find them.}%
\verse{And she said to her father, “Let there not be anger in the eyes of my lord, for I am not able to rise before you, for the way of women is with me. And he searched carefully and did not find the idols.}%
\verse{Then Jacob became angry and quarreled with Laban. Jacob answered and said to Laban, “What is my offense? What is my sin that you pursued after me?}%
\verse{For you have searched all my possessions and what did you find among all the possessions of my household? Set it before my kinsmen and your kinsmen that they may decide between the two of us!}%
\verse{These twenty years I was with you; your ewes and your female goats did not miscarry, and the rams of your flocks I did not eat.}%
\verse{I brought no mangled carcass to you — I bore its loss. From my hand you sought it, whether stolen by day or stolen by night.}%
\verse{There I was, during the day the heat consumed me, and the cold by night, and my sleep fled from my eyes.}%
\verse{These twenty years I have been in your house. I served you fourteen years for your two daughters and six years for your flock, and you have changed my wages ten times.}%
\verse{If the God of my father, the God of Abraham and the Fear of Isaac had not been with me, indeed now you would have sent me away empty-handed. God saw my misery and the labor of my hands and rebuked you last night.”}%
\verse{Then Laban answered and said to Jacob, “The daughters are my daughters and the grandsons\lnCS{} are my grandsons,\lnCS{} and the flocks are my flocks, and all that you see, it is mine. Now, what can I do for these my daughters today, or for their children whom they have borne?}%
\verse{So now, come, let us make\lebnote{“cut”} a covenant, you and I, and let it be a witness between me and you.”}%
\verse{And Jacob took a stone and set it up as a stone pillar.}%
\verse{And Jacob said to his kinsmen, “Gather stones.” And they took stones and made a pile of stones, and they ate there by the pile of stones.}%
\verse{And Laban called it Jegar Sahadutha,\lebnote{Aramaic for “the heap of witness”} but Jacob called it Galeed.\lnCT{}}%
\verse{Then Laban said, “This pile of stones is a witness between me and you today.” Therefore its name is called Galeed,\lnCT{}}%
\verse{and Mizpah,\lebnote{Hebrew for “watchpost”} because he said, “Adonai watch between me and you when we are out of sight of each other.\lebnote{“each from his neighbor is concealed”}}%
\verse{If you mistreat my daughters, and if you take wives besides my daughters, when there is no man with us, see — God is a witness between me and you.”}%
\verse{And Laban said to Jacob, “See, this pile of stones, and see the pillar that I have set up between me and you.}%
\verse{This pile of stones is a witness, and the pillar is a witness, that I will not pass beyond this pile of stones to you, and that you will not pass beyond this pile of stones and this pillar to me intending harm.}%
\verse{May the God of Abraham and the God of Nahor, the God of their father\lebnote{That is, Terah} judge between us.” Then Jacob swore by the Fear of his father Isaac.}%
\verse{And Jacob sacrificed a sacrifice on the hill, and he called his kinsmen to eat the meal.\lnCU{} And they ate the meal\lnCU{} and spent the night on the hill.}%
\verse{\lebnote{Genesis 31:55–32:32 in the English Bible is 32:1–33 in the Hebrew Bible} And Laban arose early in the morning and kissed his grandsons\lnCS{} and his daughters, and blessed them. Then Laban departed and returned to his homeland.}%
\end{biblechapter}%
\begin{biblechapter}% Genesis 32
\verseWithHeading{Jacob Fears Esau}{And Jacob went on his way, and angels of God met him.}%
\verse{And when he saw them, Jacob said, “This is the camp of God!” And he called the name of that place Mahanaim.}%
\verse{Then Jacob sent messengers before him to Esau his brother, to the land of Seir, the territory of Edom.}%
\verse{And he instructed them, saying, “Thus you must say to my lord, to Esau, ‘Thus says your servant Jacob, I have dwelled as an alien with Laban, and I have remained there until now.}%
\verse{And I have acquired cattle, male donkeys, flocks, and male and female slaves, and I have sent to tell my lord, to find favor in your eyes.’”}%
\verse{And the messengers returned to Jacob and said, “We came to your brother, to Esau, and he is coming to meet you, and four hundred men are with him.”}%
\verse{Then Jacob was very frightened and distressed. So he divided the people, flocks, cattle, and camels that were with him into two companies.}%
\verse{And he thought, “If Esau comes to one company and destroys it, the remaining company will be able to escape.”}%
\verse{Then Jacob said, “O God of my father Abraham, and God of my father Isaac, O Adonai, who said to me, ‘Return to your land and to your family, and I will deal well with you.’}%
\verse{I am not worthy\lebnote{“I am small”} of all the loyal love and all the faithfulness that you have shown\lebnote{Or “done”} your servant, for with only my staff I crossed this Jordan, and now I have become two camps.}%
\verse{Please rescue me from the hand of my brother, from the hand of Esau, for I fear him, lest he come and attack mother and children alike.}%
\verse{Now you yourself said, ‘I will surely deal well with you and make your offspring as the sand of the sea that cannot be counted for abundance.’”}%
\verse{And he lodged there that night. Then he took from what he had with him\lebnote{“from that which was going from his hand”} a gift for Esau his brother:}%
\verse{two hundred female goats, twenty male goats, two hundred ewes, twenty rams,}%
\verse{thirty milk camels with their young, forty cows, ten bulls, twenty female donkeys, and ten male donkeys.}%
\verse{And he put them under the hand of his servants, herd by herd,\lebnote{“every herd by its own”} and said to his servants, “Cross on ahead before me, and put some distance between herds.\lebnote{“between a herd and between a herd”}}%
\verse{And he instructed the foremost, saying, “When Esau my brother comes upon you and asks you, saying, ‘Whose are you and where are you going? To whom do these animals belong ahead of you?’}%
\verse{Then you must say, ‘To your servant, to Jacob. It is a gift sent to my lord, to Esau. Now behold, he is also coming after us.’”}%
\verse{And he also instructed the second servant and the third, and everyone else who was behind the herds, saying, “You must speak to Esau according to this word when you find him.}%
\verse{And moreover, you shall say, ‘Look, your servant Jacob is behind us.’” For he thought, “Let me appease him\lebnote{“let me cover his face”} with the gift going before me, and afterward I will see his face. Perhaps he will show me favor.”\lebnote{“lift up my face”}}%
\verse{So the gift passed on before him, but he himself spent that night in the camp.}%
\verseWithHeading{Jacob Wrestles with God}{That night he arose and took his two wives, his two female servants, and his eleven children and crossed the ford of the Jabbok.}%
\verse{And he took them and sent them across the stream. Then he sent across all his possessions.}%
\verse{And Jacob remained alone, and a man wrestled with him until the breaking of the dawn.}%
\verse{And when he\lnCV{} saw that he could not prevail against him, he struck his hip socket, so that Jacob’s hip socket was sprained as he wrestled with him.}%
\verse{Then he\lnCV{} said, “Let me go, for dawn is breaking.” But he answered, “I will not let you go unless you bless me.”}%
\verse{Then he said to him, “What is your name?” And he said, “Jacob.”}%
\verse{And he said, “Your name shall no longer be called Jacob, but Israel,\lebnote{“Israel” means “he struggles with God,” or “God struggles”} for you have struggled with God and with men and have prevailed.”}%
\verse{Then Jacob asked and said, “Please tell me your name.” And he said, “Why do you ask this — for my name?” And he blessed him there.}%
\verse{Then Jacob called the name of the place Peniel which means “I have seen God face to face and my life was spared.”}%
\verse{Then the sun rose upon him as he passed Penuel, and he was limping because of his hip.}%
\verse{Therefore the Israelites\lebnote{“sons/children of Israel”} do not eat the sinew of the sciatic nerve that is upon the socket of the hip unto this day, because he struck the socket of the thigh of Jacob at the sinew of the sciatic nerve.}%
\end{biblechapter}%
\begin{biblechapter}% Genesis 33
\verseWithHeading{Jacob Meets Esau and Settles at Shechem}{And Jacob lifted up his eyes and looked. And behold, Esau was coming and four hundred men were with him. And he\lebnote{That is, Jacob} divided the children among Leah and among Rachel, and among the two of his female servants.}%
\verse{And he put the female slaves and their children first, then Leah and her children next, then Rachel with Joseph last.}%
\verse{And he himself passed on before them and bowed down to the ground seven times until he came to his brother.}%
\verse{But Esau ran to meet him, and embraced him, and fell upon his neck and kissed him, and they wept.}%
\verse{Then Esau lifted up his eyes and saw the women and the children and said, “Who are these with you?” And he said, “The children whom God has graciously given your servant.”}%
\verse{Then the female servants drew near, they and their children, and they bowed down.}%
\verse{Then Leah and her children drew near and bowed down, and afterward Joseph and Rachel drew near and they bowed down.}%
\verse{And he\lnCW{} said, “What do you mean by\lebnote{“What to you?”} all this company that I have met?” Then he said, “To find favor in the eyes of my lord.”}%
\verse{Then Esau said, “I have enough\lnCX{} my brother; keep what you have.”\lebnote{“let what is to you be to you”}}%
\verse{And Jacob said, “No, please, if I have found favor in your eyes, you must take my gift from my hand, for then I have seen your face which is like seeing the face of God, and you have received me.}%
\verse{Please take my gift which has been brought to you, for God has dealt graciously with me, and because I have enough.”\lnCX{} And he urged him, so he took it.}%
\verse{Then he\lnCW{} said, “Let us journey and go on, and I will go ahead of you.”}%
\verse{But he said to him, “My lord knows that the children are frail, and the flocks and the cattle which are nursing are a concern to me. Now if they drove them hard for a day all the flocks would die.}%
\verse{Let my lord pass on before his servant and I will move along slowly at the pace\lnCY{} of the livestock that are ahead of me, and at the pace\lnCY{} of the children until I come to my lord in Seir.”}%
\verse{And Esau said, “Let me leave some of my people with you.” But he said, “What need is there?\lebnote{“What is this?”} Let me find favor in the eyes of my lord.”}%
\verse{So Esau turned that day on his way to Seir.}%
\verse{But Jacob traveled on to Succoth, and he built for himself a house, and he made shelters for his livestock. Therefore he called the name of the place Succoth.}%
\verse{And Jacob came safely to the city of Shechem which is in the land of Canaan, on his way\lebnote{“at his coming”} from Paddan-Aram. And he camped before the city.}%
\verse{And he bought a piece of land where he pitched his tent for one hundred pieces of money\lebnote{Hebrew \textit{kesitah}} from the hand of the sons of Hamor, father of Shechem.}%
\verse{And there he erected an altar and called it “El Elohe Israel.”\lebnote{That is, “\textit{El} the God of Israel”}}%
\end{biblechapter}%
\begin{biblechapter}% Genesis 34
\verseWithHeading{The Rape of Dinah and the Massacre at Shechem}{Now Dinah the daughter of Leah, whom she had borne to Jacob, went out to see the daughters of the land.}%
\verse{And Shechem, the son of Hamor the Hivite, the prince of the land, saw her. And he took her and lay with her and raped her.}%
\verse{And his soul clung to Dinah, the daughter of Jacob, and he loved the girl and spoke tenderly\lebnote{“to the heart”} to the girl.}%
\verse{So Shechem said to Hamor his father, saying, “Get this girl for me as a wife.”}%
\verse{And Jacob heard that Dinah his daughter had been defiled, but his sons were with his flocks in the field. And Jacob kept silent until they came.}%
\verse{And Hamor, father of Shechem, went out to Jacob to speak with him.}%
\verse{And the sons of Jacob came in from the field when they heard it. And the men were distressed and very angry because he had done a disgraceful thing in Israel by having sexual relations with the daughter of Jacob — something that\lebnote{“and thus”} should not be done.}%
\verse{And Hamor spoke with them saying, “Shechem my son is in love with\lebnote{“his soul longs for”} your daughter. Please give her to him for a wife.}%
\verse{Make marriages with us. Give us your daughters and take our daughters for yourselves.}%
\verse{You shall dwell with us and the land shall be before you; settle and trade in it, and acquire property in it.”}%
\verse{Then Shechem said to her father and to her brothers, “Let me find favor in your eyes, and whatever you say to me I will do.}%
\verse{Make the bride price and gift as high as you like;\lebnote{“Increase upon me very bride price and gift”} I will give what you say to me. But give me the girl as a wife.”}%
\verse{Then the sons of Jacob answered Shechem and his father Hamor speaking deceitfully, because he had defiled Dinah their sister.}%
\verse{And they said to them, “We cannot do this thing, to give our sister to a man who is uncircumcised, for that is a disgrace for us.}%
\verse{Only on this condition will we give consent to you; if you will become like us — every male among you to be circumcised.}%
\verse{Then we will give our daughters to you, and we will take for ourselves your daughters, and we will live with you and become one family.}%
\verse{But if you will not listen to us, to be circumcised, then we will take our daughters and we will go.”}%
\verse{And their words were good in the eyes of Hamor and in the eyes of Shechem, the son of Hamor.}%
\verse{And the young man did not delay to do the thing, for he wanted the daughter of Jacob. Now he was the most honored of his father’s house.}%
\verse{Then Hamor and his son Shechem came to the gate of their city, and they spoke to the men of their city, saying,}%
\verse{“These men are at peace with us. Let them dwell in the land and let them trade in it. Now, behold, the land is broad enough for them.\lebnote{“broad on sides before them”} Let us take their daughters as wives, and let us give our daughters to them.}%
\verse{Only on this condition will they give consent to us, to live with us and to become one family — when every male among us is circumcised as they are circumcised.}%
\verse{Will not their livestock and their property and all their animals be ours? Only let us give consent to them so they will live among us.”}%
\verse{And all those who went out of the gate of his city listened to Hamor and Shechem. Every male was circumcised, all those who went out of the gate of his city.}%
\verse{And it happened that on the third day, while they were in pain, two of the sons of Jacob, Simeon and Levi, the brothers of Dinah, each took his sword and came against the unsuspecting city and killed all the males.}%
\verse{They killed Hamor and his son Shechem with the edge of the sword, and they took Dinah from the house of Shechem and went out.}%
\verse{The other sons of Jacob came upon the slain and plundered the city, because they had defiled their sister.}%
\verse{They took their flocks and their cattle and their donkeys, and whatever was in the field.}%
\verse{They captured and plundered all that was in the houses — all their wealth, their little ones, and their women.}%
\verse{Then Jacob said to Simeon and Levi, “You have brought trouble on me, making me stink among the inhabitants of the land, among the Canaanites and the Perizzites! I am few in number! If they gather against me and attack me, I will be destroyed — I and my household!”}%
\verse{But they said, “Shall he treat our sister like a prostitute?”}%
\end{biblechapter}%
\begin{biblechapter}% Genesis 35
\verseWithHeading{Jacob Goes Back to Bethel}{And God said to Jacob, “Arise, go up to Bethel and dwell there, and make an altar to the God who appeared to you when you fled from before Esau your brother.”}%
\verse{Then Jacob said to his household and to all who were with him, “Get rid of the foreign gods that are in your midst and purify yourselves and change your garments.}%
\verse{Then let us make ready and let us go up to Bethel, so that I can make an altar there to the God who answered me in the day of my trouble, and who has been with me on the way that I have gone.”}%
\verse{So they gave to Jacob all the foreign gods that were in their hands, and the ornamental rings that were in their ears. And Jacob buried them under the oak which was near Shechem.}%
\verse{Then they set out on their journey, and the terror of God was upon the cities that were all around them, so that they did not pursue after the sons of Jacob.}%
\verse{And Jacob came to Luz which was in the land of Canaan (that is Bethel), he and all the people that were with him.}%
\verse{And he built an altar there and called the place El-Bethel, for there God had appeared to him when he fled before his brother.}%
\verse{And Deborah, the nurse of Rebekah, died. And she was buried below Bethel, under the oak. And its name was called Allon-Bacuth.\lebnote{“The Oak of Weeping”}}%
\verse{And God appeared to Jacob again when he came from Paddan-Aram, and he blessed him.}%
\verse{And God said to him, “Your name is Jacob. Your name shall no longer be called Jacob, but Israel shall be your name.” Then his name was called Israel.}%
\verse{And God said to him, “I am El-Shaddai.\lebnote{Possibly “God Almighty,” but more likely “God of the Wilderness”} Be fruitful and multiply. A nation and an assemblage of nations shall be from you, and kings shall go out from your loins.}%
\verse{And as for the land that I gave to Abraham and to Isaac, I will give it to you. And to your descendants after you I will give the land.}%
\verse{And God went up from him at the place where he spoke with him.}%
\verse{And Jacob set up a pillar at the place where God had spoken with him, a pillar of stone. And he poured out a drink offering upon it, and poured oil on it.}%
\verse{And Jacob called the name of the place where God had spoken with him Bethel.}%
\verseWithHeading{The Death of Rachel}{Then they journeyed from Bethel. And when they were still some distance\lebnote{“there was still the stretch of land”} from Ephrath, Rachel went into labor. And she had hard labor.}%
\verse{And when her labor was the most difficult\lebnote{“at her distress at giving birth”} the midwife said to her, “Do not be afraid for you have another son.”\lebnote{“for this one also to you is a son”}}%
\verse{And it happened that when her life was departing (for she was dying), she called his name Ben-Oni.\lebnote{“Son of my sorrow”} But his father called him Benjamin.\lebnote{“Son of the right hand”}}%
\verse{And Rachel died and she was buried on the way to Ephrath (that is, Bethlehem).}%
\verse{And Jacob erected a pillar at her burial site. That is the pillar of the burial site of Rachel unto this day.}%
\verse{And Israel journeyed on and pitched his tent beyond the tower of Eder.\lebnote{Or “Migdal-Eder”}}%
\verse{And while Israel was living in that land Reuben went and had sexual relations with Bilhah, his father’s concubine. And Israel heard about it.\innerVerseHeading{The Twelve Sons of Jacob}Now the sons of Jacob were twelve.}%
\verse{The sons of Leah: The firstborn of Jacob was Reuben. Then Simeon, Levi, Judah, Issachar, and Zebulun.}%
\verse{The sons of Rachel: Joseph and Benjamin.}%
\verse{The sons of Bilhah, the female servant of Rachel: Dan and Naphtali.}%
\verse{The sons of Zilpah, the female servant of Leah: Gad and Asher. These were the sons of Jacob who were born to him in Paddan-Aram.}%
\verseWithHeading{The Death of Isaac}{And Jacob came to Isaac his father at Mamre, or Kiriath-Arba (that is, Hebron), where Abraham and Isaac dwelled as aliens.}%
\verse{Now the days of Isaac were one hundred and eighty years.}%
\verse{And Isaac passed away and died, and was gathered to his people, old and full of days. And his sons Esau and Jacob buried him.}%
\end{biblechapter}%
\begin{biblechapter}% Genesis 36
\verseWithHeading{The Descendants of Esau}{Now these are the descendants of Esau (that is, Edom).}%
\verse{Esau took his wives from the daughters of Canaan: Adah, daughter of Elon, the Hittite, and Oholibamah, daughter of Anah, the daughter of Zibeon, the Hivite,}%
\verse{and Basemath, the daughter of Ishmael, the sister of Nebaioth.}%
\verse{And Adah bore to Esau Eliphaz; and Basemath bore Reuel;}%
\verse{and Oholibamah bore Jeush and Jalam, and Korah. These are the sons of Esau who were born to him in the land of Canaan.}%
\verse{And Esau took his wives and his sons and his daughters, and all the persons of his household, and his sheep and goats, and all his cattle, and all the goods that he had acquired in the land of Canaan, and went to a land away from his brother Jacob.}%
\verse{For their possessions were too many to live together,\lebnote{“much from living together”} so that the land of their sojourning was not able to support them on account of their livestock.}%
\verse{So Esau dwelled in the hill country of Seir (Esau, that is Edom).}%
\verse{Now these are the descendants of Esau, the father of Edom, in the hill country of Seir.}%
\verse{These are the names of the sons of Esau: Eliphaz, the son of Adah, the wife of Esau; Reuel, the son of Basemath, the wife of Esau.}%
\verse{The sons of Eliphaz were Teman, Omar, Zepho, Gatam, and Kenaz.}%
\verse{(Now Timnah was the concubine of Eliphaz, the son of Esau. And she bore Amalek to Eliphaz.) These are the sons of Adah, the wife of Esau.}%
\verse{Now these are the sons of Reuel: Nahath, Zerah, Shammah, and Mizzah. These are the sons of Basemath, the wife of Esau.}%
\verse{Now these are the sons of Oholibamah, the daughter of Anah, daughter of Zibeon, the wife of Esau: She bore to Esau Jeush, Jalam, and Korah.}%
\verse{These are the chiefs of the sons of Esau. The sons of Eliphaz, the firstborn of Esau: the chiefs of Teman, Omar, Zepho, Kenaz,}%
\verse{Korah, Gatam, and Amalek. These are the chiefs of Eliphaz in the land of Edom. These are the sons of Adah.}%
\verse{Now these are the sons Reuel, the son of Esau: the chiefs Nahath, Zerah, Shammah, and Mizzah. These are the chiefs of Reuel in the land of Edom. These are the sons of Basemath, the wife of Esau.}%
\verse{Now these are the sons of Oholibamah, the wife of Esau: the chiefs Jeush, Jalam, and Korah. These are the chiefs born of Oholibamah, the daughter of Anah, the wife of Esau.}%
\verse{These are the sons of Esau, and these are their chiefs (that is, Edom).}%
\verse{These are the sons of Seir, the Horite, the inhabitants of the land: Lotan, Shobal, Zibeon, Anah,}%
\verse{Dishon, Ezer, and Dishan. These are the chiefs of the Horites, the sons of Seir in the land of Edom.}%
\verse{And the sons of Lotan were Hori and Hemam. And Lotan’s sister was Timna.}%
\verse{Now these are the sons of Shobal: Alvan, Manahath, Ebal, Shepho, and Onam.}%
\verse{Now these are the sons of Zibeon: Aiah and Anah — he is Anah who found the hot springs in the desert while he pastured the donkeys of Zibeon his father.}%
\verse{Now these are the sons of Anah: Dishon and Oholibamah, the daughter of Anah.}%
\verse{Now these are the sons of Dishon: Hemdan, Eshban, Ithran, and Keran.}%
\verse{These are the sons of Ezer: Bilhan, Zaavan, and Akan.}%
\verse{These are the sons of Dishan: Uz and Aran.}%
\verse{These are the chiefs of the Horites: the chiefs Lotan, Shobal, Zibeon, Anah,}%
\verse{Dishon, Ezer, and Dishan. These are the chiefs of the Horites, according to their chiefs in the land of Seir.}%
\verseWithHeading{The Kings of Edom}{Now these are the kings who reigned in the land of Edom before any king ruled over the Israelites.\lebnote{“sons/children of Israel”}}%
\verse{Bela the son of Beor reigned in Edom. And the name of his city was Dinhabah.}%
\verse{And Bela died, and Jobab, the son of Zerah from Bozrah, reigned in his place.}%
\verse{And Jobab died, and Husham from the land of the Temanites reigned in his place.}%
\verse{And Husham died, and Hadad, son of Bedad, who defeated Midian in the field\lebnote{Or “country”} of Moab reigned in his place. And the name of his city was Avith.}%
\verse{And Hadad died, and Samlah from Masrekah reigned in his place.}%
\verse{And Samlah died, and Shaul from Rehoboth on the Euphrates\lebnote{Or “the River”} reigned in his place.}%
\verse{And Shaul died, and Baal-Hanan, the son of Acbor, reigned in his place.}%
\verse{And Baal-Hanan the son of Acbor died, and Hadar reigned in his place. And the name of his city was Pau, and the name of his wife was Mehetabel, the daughter of Matred, daughter of Mezahab.}%
\verse{Now these are the names of the chiefs of Esau according to their families, according to their dwelling places, by their names: the chiefs Timna, Alvah, Jetheth,}%
\verse{Oholibamah, Elah, Pinon,}%
\verse{Kenaz, Teman, Mibzar,}%
\verse{Magdiel, and Iram. These are the chiefs of Edom (that is, Esau, the father of Edom) according to their settlements in the land of their possession.}%
\end{biblechapter}%
\begin{biblechapter}% Genesis 37
\verseWithHeading{The Dreams of Joseph}{And Jacob settled in the land of the sojourning of his father, in the land of Canaan.}%
\verse{These are the generations\lebnote{Or “family records”} of Jacob. Joseph, being seventeen years old, was shepherding the flock with his brothers. Now he was a helper with the sons of Bilhah and the sons of Zilpah, the wives of his father. And Joseph brought a bad report of them to his father.}%
\verse{Now Israel loved Joseph more than all his sons, for he was a son of his old age. And he made a robe with long sleeves\lnCZ{} for him.}%
\verse{When his brothers saw that their father loved him more than all his brothers, they hated him and were not able to speak peaceably to him.}%
\verse{And Joseph dreamed a dream, and he told it to his brothers. And they hated him even more.\lnDA{}}%
\verse{And he said to them, “Listen now to this dream that I dreamed.}%
\verse{Now behold, we were binding sheaves in the midst of the field and, behold, my sheaf stood up and it remained standing. Then behold, your sheaves gathered around and bowed down to my sheaf.”}%
\verse{Then his brothers said to him, “Will you really rule over us?” And they hated him even more\lnDA{} on account of his dream and because of his words.}%
\verse{Then he dreamed yet another dream and told it to his brothers. And he said, “Behold, I dreamed a dream again, and behold, the sun and the moon and eleven stars were bowing down to me.”}%
\verse{And he told it to his father and to his brothers. And his father rebuked him and said to him, “What is this dream that you have dreamed? Will I and your mother and your brothers indeed come to bow down to the ground to you?”}%
\verse{And his brothers were jealous of him, but his father kept the matter in mind.}%
\verseWithHeading{Joseph Sold Into Slavery by his Brothers}{Now his brothers went to pasture the flock of their father in Shechem.}%
\verse{And Israel said to Joseph, “Are not your brothers pasturing in Shechem? Come, let me send you to them.” And he said, “Here I am.”}%
\verse{Then he said to him, “Go now, see if it goes well for your brothers and for the flock,\lebnote{“see the peace of your brothers and the peace of the flock”} then return word to me.” And he sent him from the valley of Hebron, and he arrived at Shechem.}%
\verse{And a man found him, and behold, he\lebnote{That is, Joseph} was wandering about in a field. And the man asked him, “What do you seek?”}%
\verse{And he said, “I am seeking my brothers. Tell me, please, where they are pasturing.”}%
\verse{And the man said, “They have moved on from here, for I heard them saying, ‘Let us go to Dothan.’” Then Joseph went after his brothers and found them in Dothan.}%
\verse{And they saw him from a distance. And before he drew near to them, they conspired against him to kill him.}%
\verse{And each said to his brothers, “Look, this master of dreams is coming.}%
\verse{Now then, come, let us kill him and throw him in one of the pits. Then we will say a wild animal devoured him. Then we will see what his dreams become.”}%
\verse{And Reuben heard it and delivered him from their hand and said, “We must not take his life.”}%
\verse{And Reuben said to them, “You must not shed blood. Throw him into this pit that is in the desert, but do not lay a hand on him” — so that he might rescue him from their hand to return him to his father.}%
\verse{And it happened that as Joseph came to his brothers they stripped Joseph of his robe, the robe with long sleeves,\lnCZ{} that was upon him.}%
\verse{And they took him and threw him into the pit (the pit was empty; there was no water in it).}%
\verse{Then they sat down to eat some food. And they lifted up their eyes and looked, and behold, a caravan of Ishmaelites was coming from Gilead. And their camels were carrying aromatic gum and balm and spices on the way\lebnote{“going to bring down”} to Egypt.}%
\verse{Then Judah said to his brothers, “What profit is there if we kill our brother and conceal his blood?}%
\verse{Come, let us sell him to the Ishmaelites, but our hand shall not be against him, for he is our brother, our own flesh.” And his brothers agreed.}%
\verse{Then Midianite traders passed by. And they\lebnote{That is, the brothers} drew Joseph up and brought him up from the pit, and they sold Joseph to the Ishmaelites for twenty pieces of silver. And they brought Joseph to Egypt.}%
\verse{Then Reuben returned to the pit and, behold, Joseph was not in the pit. And he tore his clothes.}%
\verse{And he returned to his brothers and said, “The boy is gone!\lebnote{“is not”} Now I, what can I do?”\lebnote{“where am I going?”}}%
\verse{Then they took the robe of Joseph and slaughtered a goat, and dipped the robe in the blood.}%
\verse{Then they sent the robe with long sleeves\lnCZ{} and they brought it to their father and said, “We found this; please examine it. Is it the robe of your son or not?”}%
\verse{And he recognized it and said, “The robe of my son! A wild animal has devoured him! Joseph is surely torn to pieces!”}%
\verse{And Jacob tore his clothes and put sackcloth on his loins and mourned for his son many days.}%
\verse{And all his sons and daughters tried to console him, but he refused to be consoled. And he said, “No, I shall go down to my son, to Sheol, mourning.” And his father wept for him.}%
\verse{And the Midianites sold him in Egypt to Potiphar, a court official of Pharaoh, a commander of the imperial guard.}%
\end{biblechapter}%
\begin{biblechapter}% Genesis 38
\verseWithHeading{Judah and Tamar}{And it happened that at that time Judah went down from his brothers and pitched his tent near a certain Adullamite, whose name was Hirah.}%
\verse{And Judah saw the daughter of a certain Canaanite there whose name was Shua. And he took her and went in to her.}%
\verse{And she conceived and bore a son, and he called his name Er.}%
\verse{And she conceived again and bore a son, and he called his name Onan.}%
\verse{And once again she bore a son, and she called his name Shelah. And he\lebnote{That is, Judah} was in Chezib when she bore him.}%
\verse{And Judah took a wife for Er his firstborn, and her name was Tamar.}%
\verse{And Er, the firstborn of Judah, was evil in the eyes of Adonai, and Adonai killed him.}%
\verse{Then Judah said to Onan, “Go in to the wife of your brother and perform the duty of a brother-in-law to her, and raise up offspring for your brother.”}%
\verse{But Onan knew that the offspring would not be for him, so whenever he went in to the wife of his brother he would waste it on the ground\lebnote{Meaning he would spill his semen on the ground} so as not to give offspring to his brother.}%
\verse{And what he did was evil in the sight of Adonai, so he killed him also.}%
\verse{Then Judah said to Tamar, his daughter-in-law, “Stay a widow in your father’s house until Shelah my son grows up,” for he feared he would also die\lebnote{“he thought lest he also will die”} like his brother. So Tamar went and stayed in the house of her father.}%
\verse{And in the course of time\lebnote{“And the days increased”} the daughter of Shua, the wife of Judah, died. When Judah was consoled he went up to his sheepshearers, he and his friend Hirah the Adullamite, to Timnah.}%
\verse{And it was told to Tamar, saying, “Look, your father-in-law is going up to Timnah to shear his sheep.”}%
\verse{So she removed the clothes of her widowhood and covered herself with the veil and disguised herself. And she sat at the entrance to Eynayim, which is on the way to Timnah, for she saw that Shelah was grown but she had not been given to him as a wife.}%
\verse{And Judah saw her and reckoned her to be a prostitute, for she had covered her face.}%
\verse{And he turned aside to her at the roadside and said, “Please come, let me come in to you,” for he did not know that she was his daughter-in-law. And she said, “What will you give to me that you may come in to me?”}%
\verse{And he said, “I will send a kid from the goats of the flock.” And she said, “Only if you give a pledge until you send it.”}%
\verse{And he said, “What is the pledge that I must give to you?” And she said, “your seal, your cord, and your staff that is in your hand.” And he gave them to her and went in to her. And she conceived by him.}%
\verse{And she arose and left, and she removed her veil from herself and put on the garments of her widowhood.}%
\verse{And Judah sent the kid from the goats by the hand of his friend the Adullamite to take back the pledge from the hand of the woman, but he could not find her.}%
\verse{So he asked the men of her place, saying, “Where is that cult prostitute that was at Eynayim by the roadside?” And they said, “There is no cult prostitute here.”}%
\verse{Then he returned to Judah and said, “I could not find her. Morever, the men of the place said, ‘There is no cult prostitute here.’”}%
\verse{And Judah said, “Let her take them for herself, lest we be laughed at.\lebnote{“as contempt”} Behold, I sent this kid, but you could not find her.”}%
\verse{And about three months later\lebnote{“it happened about three months”} it was told to Judah, “Tamar your daughter-in-law has played the whore, and now, behold, she has conceived by prostitution.” And Judah said, “Bring her out and let her be burned.”}%
\verse{She was brought out, but she sent to her father-in-law saying, “By the man to whom these belong I have conceived.” And she said, “Now discern\lebnote{Or “examine”} to whom these belong: the seal and cord and the staff.”}%
\verse{Then Judah recognized them and said, “She is more righteous than I, since I did not give her to my son Shelah.” And he did not know\lebnote{“Know” is a euphemism for “have sexual relations with”} her again.}%
\verse{And it happened that at the time she gave birth that, behold, twins were in her womb.}%
\verse{And it happened that at her labor one child put out a hand. And the midwife took it and tied a crimson thread on his hand saying, “This one came out first.”}%
\verse{Then his hand drew back and, behold, his brother came out, and she said, “What a breach you have made for yourself!” And she called his name Perez.}%
\verse{And afterward his brother who had the crimson thread on his hand came out. And his name was called Zerah.}%
\end{biblechapter}%
\begin{biblechapter}% Genesis 39
\verseWithHeading{Joseph in Potiphar’s House}{Now Joseph had been brought down to Egypt, and Potiphar, a court official of Pharaoh, commander of the guard, an Egyptian, bought him from the hand of the Ishmaelites who had brought him down there.}%
\verse{And Adonai was with Joseph, and he became a successful man. And he was in the house of his master, the Egyptian.}%
\verse{And his master observed that Adonai was with him, and everything that was in his hand to do Adonai made successful.}%
\verse{And Joseph found favor in his eyes and he served him. Then he appointed him\lebnote{That is, Joseph} over his house and all that he owned he put into his hand.}%
\verse{And it happened that from the time he appointed him over his house and over all that he had, Adonai blessed the house of the Egyptian on account of Joseph. And the blessing of Adonai was upon all that he had in the house and in the field.}%
\verse{And he left all that he had in the hand of Joseph, and he did not worry about anything\lebnote{“he did not know with him anything”} except the food that he ate. Now Joseph was well built and handsome.\lebnote{“beautiful of form and beautiful of appearance”}}%
\verse{And it happened that after these things his master’s wife cast her eyes on Joseph, and she said, “Lie with me.”}%
\verse{But he refused and said to his master’s wife, “Look, my master does not worry about\lebnote{“does not know with me”} what is in the house, and everything he owns he has put in my hand.}%
\verse{He has no greater authority in this house than me, and he has not withheld anything from me except you, since you are his wife. Now how could I do this great wickedness and sin against God?”}%
\verse{And it happened that as she spoke to Joseph day after day,\lebnote{“day, day”} he did not heed her to lie beside her or to be with her.}%
\verse{But one particular day\lebnote{“and it happened as this day”} he came into the house to do his work and none of the men of the house were there in the house,}%
\verse{she seized him by his garment and said, “Lie with me!” And he left his garment in her hand and fled, and he went outside.}%
\verse{And it happened that when she saw that he left his garment in her hand and fled outside,}%
\verse{she called to the men of her house and said to them, “Look! He\lebnote{That is, Potiphar} brought a Hebrew man to us to mock us! He came to me to lie with me, and I cried out with a loud voice.}%
\verse{And when he heard me, that I raised my voice and called out, he left his garment beside me and fled, and he went outside.”}%
\verse{Then she put his garment beside her until his master came to his house.}%
\verse{Then she spoke to him according to these words, saying, “The Hebrew slave that you brought to us came to me to make fun of me.}%
\verse{And it happened that as I raised my voice and called out, he left his garment beside me and fled outside.”}%
\verse{And when his master heard the words of his wife that she spoke to him, “This is what your servant did to me,”\lebnote{“according to these words your servant did to me”} he became very angry.\lebnote{“his nostrils became hot”}}%
\verse{And Joseph’s master took him and put him into prison, the place that the king’s prisoners were confined. And he was there in prison.}%
\verse{And Adonai was with Joseph, and showed loyal love to him, and gave him favor in the eyes of the chief of the prison.}%
\verse{And the chief of the prison put all the prisoners that were in the prison into the hand of Joseph. And everything that was done there, he was the one who did it.}%
\verse{The chief of the prison did not worry about\lebnote{“did not pay heed to”} anything in his\lebnote{That is, Joseph’s} hand, since Adonai was with him. And whatever he did Adonai made it successful.}%
\end{biblechapter}%
\begin{biblechapter}% Genesis 40
\verseWithHeading{Joseph Interprets Dreams in Prison}{And it happened that after these things the cupbearer of the king of Egypt and his baker did wrong against their lord, against the king of Egypt.}%
\verse{And Pharaoh was angry with his two officials, with the chief cupbearer and chief baker.}%
\verse{And he put them in custody in the house of the chief of the guard, into the prison where Joseph was confined.}%
\verse{And the chief of the guard appointed Joseph to be with them, and he attended them. And they were in custody many days.\lebnote{“days”}}%
\verse{And the two of them, the cupbearer and the baker of the king of Egypt, who were confined in the prison, dreamed a dream, each his own dream, with its own interpretation.}%
\verse{When Joseph came to them in the morning he looked at them, and behold, they were troubled.}%
\verse{And he asked the court officials of Pharaoh that were with him in the custody of his master’s house, “Why are your faces sad today?”}%
\verse{And they said to him, “We each dreamed a dream, but there is no one to interpret it.” And Joseph said to them, “Do not interpretations belong to God? Please tell them to me.”}%
\verse{Then the chief cupbearer told his dream to Joseph, and he said to him, “In my dream, now behold, there was a vine before me,}%
\verse{and on the vine were three branches. And as it budded, its blossoms came up, and its clusters of grapes grew ripe.}%
\verse{And the cup of Pharaoh was in my hand, and I took the grapes and squeezed them into the cup of Pharaoh. Then I placed the cup into the hand of Pharaoh.”}%
\verse{Then Joseph said to him, “This is its interpretation: The three branches, they are three days.}%
\verse{In three days Pharaoh will lift up your head and will restore you to your office. And you shall put the cup of Pharaoh into his hand as was formerly the custom, when you were his cupbearer.}%
\verse{But remember me when it goes well with you, and please may you show kindness with respect to me, and mention me to Pharaoh, and bring me out of this house.}%
\verse{For I was surely kidnapped from the land of the Hebrews, and here also I have done nothing that they should put me in this pit.”}%
\verse{And when the chief baker saw that the interpretation was good he said to Joseph, “I also dreamed. In my dream, now behold, there were three baskets of bread upon my head.}%
\verse{And in the upper basket were all sorts of baked foods for Pharaoh, but the birds were eating them out of the basket upon my head.”}%
\verse{Then Joseph answered and said, “This is its interpretation: The three baskets, they are three days.}%
\verse{In three days Pharaoh will lift your head from you and hang you on a pole,\lebnote{Or “tree”} and the birds will eat your flesh from you.”}%
\verse{And it happened that on the third day, which was Pharaoh’s birthday, he made a feast for all his servants. And he lifted up the head of the chief cupbearer and the head of the chief baker in the midst of his servants.}%
\verse{And he restored the chief cupbearer to his cupbearing position. And he placed the cup in the hand of Pharaoh.}%
\verse{But the chief baker he hanged as Joseph had interpreted to them.}%
\verse{But the chief cupbearer did not remember Joseph, but forgot him.}%
\end{biblechapter}%
\begin{biblechapter}% Genesis 41
\verseWithHeading{Joseph Interprets Pharaoh’s Dreams}{And it happened that after two full years\lebnote{“two years of days”} Pharaoh dreamed, and behold, he was standing by the Nile.}%
\verse{And behold, seven cows, well built and fat,\lnDB{} were coming up from the Nile, and they grazed among the reeds.}%
\verse{And behold, seven other cows came up after them from the Nile, ugly and gaunt,\lnDC{} and they stood beside those\lebnote{Or “the”} cows on the bank of the Nile.}%
\verse{And the ugly and gaunt\lnDC{} cows ate the seven well built and fat\lnDB{} cows. Then Pharaoh awoke.}%
\verse{And he fell asleep and dreamed a second time, and behold, seven ears of grain, plump and good, were coming out of one stalk.}%
\verse{And behold, seven thin ears of grain, scorched by the east wind, sprouted up after them.}%
\verse{And the thin ears of grain swallowed up the seven plump and full ears of grain. Then Pharaoh awoke, and behold, it was a dream.}%
\verse{And it happened that in the morning his spirit was troubled, and he sent and called all of the magicians\lnDD{} of Egypt, and all its wise men, and Pharaoh told his dream to them. But they had no interpretation\lebnote{“there was no interpretation with them”} for Pharaoh.}%
\verse{Then the chief of the cupbearers spoke with Pharaoh, saying, “I remember my sins today.}%
\verse{Pharaoh was angry with his servants, and he put me and the chief baker in the custody of the house of the chief of the guard.}%
\verse{And we dreamed a dream one night, I and he, each with a dream that had a meaning.\lebnote{“each according to his dream we dreamed”}}%
\verse{And there with us was a young man, a Hebrew servant of the chief of the guard, and we told him the dream, and he interpreted our dreams for us, each according to his dream he interpreted.}%
\verse{And it happened just as he interpreted to us, so it was. He\lnDE{} restored me to my office, and him\lebnote{That is, the chief baker} he\lnDE{} hanged.”}%
\verse{Then Pharaoh sent and called for Joseph, and they brought him quickly from the prison. And he shaved and changed his clothing, and came to Pharaoh.}%
\verse{Then Pharaoh said to Joseph, “I dreamed a dream, but there is none to interpret it. Now, I have heard concerning you that when you hear a dream you can interpret it.”}%
\verse{Then Joseph answered Pharaoh saying, “It is not in my power;\lebnote{“besides me”} God will answer concerning the well-being of Pharaoh.”}%
\verse{And Pharaoh said to Joseph, “Now in my dream, behold, I was standing on the bank of the Nile,}%
\verse{and behold, seven cows, well built and fat,\lnDB{} were coming up from the Nile, and they grazed among the reeds.}%
\verse{And behold, seven other cows came up after them from the Nile, very ugly and gaunt\lnDC{} — never have I seen any as them in all the land of Egypt for ugliness.}%
\verse{And the thin and ugly cows ate the former seven healthy cows.}%
\verse{But when they went into their bellies\lnDF{} it could not be known that they went into their bellies,\lnDF{} for their appearance was as ugly as at the beginning. Then I awoke.}%
\verse{Then I saw in my dream and behold, seven ears of grain were coming out of one stalk, full and good.}%
\verse{And behold, seven withered ears of grain, thin and scorched by the east wind, sprouted up after them.}%
\verse{And the thin ears of grain swallowed up the seven good ears of grain. And I told the magicians,\lnDD{} but there was none to explain it to me.”}%
\verse{Then Joseph said to Pharaoh, “The dreams of Pharaoh are one. God has revealed to Pharaoh what he is about to do.}%
\verse{The seven good cows, they are seven years, and the seven good ears of grain, they are seven years. The dreams are one.}%
\verse{And the seven thin and ugly cows coming up after them, they are seven years, and the seven empty ears of grain, scorched by the east wind, they are also seven years of famine.}%
\verse{This is the word that I have spoken to Pharaoh; God has shown Pharaoh what he is about to do.}%
\verse{Behold, seven years of great abundance are coming throughout the whole land of Egypt.}%
\verse{Then seven years of famine will arise after them, and all the abundance in the land of Egypt will be forgotten. The famine will consume the land.}%
\verse{Abundance in the land will not be known because of the famine that follows,\lebnote{“that thus afterwards”} for it will be very heavy.}%
\verse{Now concerning the repetition of the dream twice to Pharaoh, it is because the matter is established by God, and God will do it quickly.}%
\verse{Now then, let Pharaoh select a man who is discerning and wise, and let him set him over the land of Egypt.}%
\verse{Let Pharaoh do this, and let him appoint supervisors over the land, and let him take one-fifth from the land of Egypt in the seven years of abundance.}%
\verse{Then let them gather all the food of these coming good years and let them pile up grain under the hand of Pharaoh for food in the cities, and let them keep it.}%
\verse{Then the food shall be as a deposit for the land for the seven years of the famine that will be in the land of Egypt, that the land will not perish on account of the famine.”}%
\verseWithHeading{Joseph Rises to Power}{And the plan\lebnote{Or “word”} was good in the eyes of Pharaoh and in the eyes of all his servants.}%
\verse{Then Pharaoh said to his servants, “Can we find a man like this in whom is the spirit of God?”}%
\verse{Then Pharaoh said to Joseph, “Since God has made all of this known to you there is no one as discerning and wise as you.}%
\verse{You shall be over my house, and to your word\lebnote{Or “mouth”} all my people shall submit. Only with respect to the throne will I be greater than you.”}%
\verse{Then Pharaoh said to Joseph, “See, I have set you over all the land of Egypt.”}%
\verse{Then Pharaoh removed his signet ring from his finger and put it on the finger of Joseph. And he clothed him with garments of fine linen, and he put a chain of gold around his neck.}%
\verse{And he had him ride in his second chariot. And they cried out before him, “Kneel!” And Pharaoh set him over all the land of Egypt.}%
\verse{Then Pharaoh said to Joseph, “I am Pharaoh, but without your consent no one will lift his hand or his foot in all the land of Egypt.”}%
\verse{And Pharaoh called the name of Joseph Zaphenath-paneah and gave him Asenath, the daughter of Potiphera, priest of On, as a wife. And Joseph went out over the land of Egypt.}%
\verse{Now Joseph was thirty years old\lebnote{“a son of thirty years”} when he stood before Pharaoh, the king of Egypt. And Joseph went out from the presence of Pharaoh and traveled through the whole land of Egypt.}%
\verse{And the land produced a plenty in the seven years of abundance.}%
\verse{And he gathered all the food of the seven years which occurred in the land of Egypt. And he stored the food in the cities. The food of the field that surrounded each city he stored in its midst.}%
\verse{And Joseph piled up grain like the sand of the sea in great abundance until he stopped counting it, for it could not be counted.\lebnote{“there was no number”}}%
\verse{Before the years of famine came, Asenath, daughter of Potiphera priest of On, bore two sons to him.}%
\verse{And Joseph called the name of the firstborn Manasseh, for he said, “God has caused me to forget all my hardship and all my father’s house.”}%
\verse{And the name of the second he called Ephraim, for he said, “God has made me fruitful in the land of my misfortune.”}%
\verse{And the seven years of abundance which were in the land of Egypt came to an end.}%
\verse{And the seven years of famine began to come as Joseph had said. And there was famine in all of the countries, but in the land of Egypt there was food.}%
\verse{And when all the land of Egypt was hungry the people cried out to Pharaoh for food. And Pharaoh said to all the land of Egypt, “Go to Joseph; what he says to you, you must do.”}%
\verse{And the famine was over the whole land, and Joseph opened all the storehouses\lebnote{Hebrew “that which was in them”} and sold food to the Egyptians. And the famine was severe in the land of Egypt.}%
\verse{And every land came to Egypt to Joseph to buy grain, for the famine was severe in every land.}%
\end{biblechapter}%
\begin{biblechapter}% Genesis 42
\verseWithHeading{Joseph’s Brothers Go to Egypt for Food}{When Jacob realized that there was grain in Egypt, Jacob said to his sons, “Why do you look at one another?”}%
\verse{Then he said, “Look, I have heard that there is grain in Egypt. Go down there and buy grain for us there that we may live and not die.”}%
\verse{And the ten brothers of Joseph went down to buy grain from Egypt.}%
\verse{But Jacob did not send Benjamin, the brother of Joseph, for he feared harm would come to him.\lebnote{“he thought, lest harm encounter him”}}%
\verse{Then the sons of Israel went to buy grain amid those other people who went as well, for there was famine in the land of Canaan.}%
\verse{Now Joseph was the governor over the land. He was the one who sold food to all the people of the land. And the brothers of Joseph came and bowed down to him with their faces to the ground.}%
\verse{And Joseph saw his brothers and recognized them, but he pretended to be a stranger to them. And he spoke with them harshly and said to them, “From where have you come?” And they said, “From the land of Canaan to buy food.”}%
\verse{And Joseph recognized his brothers, but they did not recognize him.}%
\verse{And Joseph remembered the dreams which he had dreamed concerning them, and he said to them, “You are spies! You have come to see the nakedness of the land!”}%
\verse{And they said to him, “No, my lord, but your servants have come to buy food.}%
\verse{We all are sons of one man. We are honest men. We, your servants, are not spies.”}%
\verse{Then he said to them, “No, but you have come to see the nakedness of the land.”}%
\verse{Then they said, “We, your servants, are twelve brothers, the sons of one man in the land of Canaan, but behold, the youngest is with our father today, and one is no more.”}%
\verse{But Joseph said to them, “It is what I said to you — you are spies.}%
\verse{By this you shall be tested. By the life of Pharaoh you will not go out from here unless your youngest brother comes here.}%
\verse{Send one of you, and let him bring your brother, but you will be kept in prison so that your words might be tested to see if there is truth with you. And if not, by the life of Pharaoh surely you are spies.”}%
\verse{Then he gathered them into the prison for three days.}%
\verse{On the third day Joseph said to them, “Do this and you will live; I fear God.}%
\verse{If you are honest, let one of your brothers be kept in prison where you are now being kept,\lebnote{“in the house of your custody”} but the rest of you go, carry grain for the famine for your households.}%
\verse{You must bring your youngest brother to me, and then your words will be confirmed and you will not die.” And they did so.}%
\verse{Then each said to his brother, “Surely we are guilty on account of our brother when we saw the anguish of his soul when he pleaded for mercy to us and we would not listen. Therefore this trouble has come to us.”}%
\verse{Then Reuben answered them, saying, “Did I not say to you, do not sin against the boy? But you did not listen, and now, behold, his blood has been sought.”}%
\verse{Now they did not know that Joseph understood, for the interpreter was between them.}%
\verse{And he turned away from them and wept. Then he returned to them and spoke to them, and took Simeon from them and tied him up in front of them.}%
\verse{Then Joseph gave orders to fill their bags with grain and to return their money to each sack, and to give them provisions for the journey. Thus he did for them.}%
\verse{Then they loaded their grain upon their donkeys and went away from there.}%
\verse{And one of them later opened his sack to give fodder to his donkey at the lodging place and saw his money — behold, it was in the mouth of his sack.}%
\verse{And he said to his brothers, “My money was returned and moreover, behold, it is in my sack!” Then their hearts failed them\lebnote{“their heart went out”} and each of them trembled and said, “What is this God has done to us?”}%
\verse{And when they came to Jacob their father in the land of Canaan they told him everything that had happened to them, saying,}%
\verse{“The man, the lord of the land, spoke harshly to us and treated us as if we were spying out the land.}%
\verse{But we said to him, ‘We are honest; we are not spies.}%
\verse{We are twelve brothers, the sons of our father. One is no more and the youngest is with our father now in the land of Canaan.’}%
\verse{Then the man, the lord of the land, said to us, ‘By this I will know that you are honest. Leave one brother with me, and take food for the famine in your households and go.}%
\verse{And bring your youngest brother to me. Then I will know that you are not spies but you are honest. And I will give your brother back to you, and you will trade in the land.’”}%
\verse{And it happened that when they emptied their sacks, behold, each one’s pouch of money was in his sack. And when they and their father saw the pouches of their money, they were greatly distressed.}%
\verse{And Jacob their father said to them, “You have bereaved me — Joseph is no more and Simeon is no more, and Benjamin you would take! All of this is against me!}%
\verse{Then Reuben said to his father, “You may kill my two sons if I do not bring him back to you. Put him in my hand and I myself will return him to you.”}%
\verse{But he said, “My son shall not go down with you, for his brother is dead and he alone remains. If harm meets him on the journey that you would take, you would bring down my gray head in sorrow to Sheol.”}%
\end{biblechapter}%
\begin{biblechapter}% Genesis 43
\verseWithHeading{Joseph’s Brothers Return to Egypt}{Now the famine in the land was severe.}%
\verse{And it happened that as they finished eating the grain which they had brought from Egypt their father said to them, “Return and buy a little food for us.”}%
\verse{Then Judah said to him, “The man solemnly admonished us, saying, ‘You shall not see my face unless your brother is with you.’}%
\verse{If you will send\lebnote{“if there is a sending”} our brother with us, we will go down and buy food for you,}%
\verse{but if you will not send\lebnote{“if there is no sending”} him, we will not go down, for the man said to us, ‘You shall not see my face unless your brother is with you.’”}%
\verse{Then Israel said, “Why did you bring trouble to me by telling the man you still had a brother?”}%
\verse{And they said, “The man asked explicitly about us and about our family, saying, ‘Is your father still alive? Do you have a brother?’ And we answered him according to these words. How could we know that he would say, ‘Bring down your brother’?”}%
\verse{Then Judah said to his father Israel, “Send the boy with me, and let us arise and go, so that we will live and not die — you, we, and our children.}%
\verse{I myself will be surety for him. You may seek him from my hand. If I do not bring him back to you and present him before you, then I will stand guilty before you forever.}%
\verse{Surely if we had not hesitated by this time we would have returned twice.”}%
\verse{Then their father Israel said to them, “If it must be so then do this. Take some of the best products of the land in your bags and take them down to the man as a gift — a little balm and honey, aromatic gum and myrrh, and pistachios and almonds.}%
\verse{And take double the money in your hands. Take back the money that was returned in the mouth of your sacks. Perhaps it was a mistake.}%
\verse{And take your brother. Now arise and return to the man.}%
\verse{And may El-Shaddai\lebnote{Often translated “God Almighty”} grant you compassion before the man that he may release your other brother to you and Benjamin. As for me, if I am bereaved, I am bereaved.”}%
\verse{So the men took this gift, and they took double money in their hands, and Benjamin, and they rose up and went down to Egypt and stood before Joseph.}%
\verse{When Joseph saw Benjamin with them he said to the one who was over his household, “Bring the men into the house and slaughter and prepare an animal, for the men shall eat with me at noon.”}%
\verse{And the man did as Joseph had said, and the man brought the men into the house of Joseph.}%
\verse{And the men were afraid when they were brought into the house of Joseph. And they said “We were brought here on account of the money that was returned to our sacks the first time, that he might attack us and fall upon us to take us as slaves with our donkeys.”}%
\verse{So they approached the man who was over Joseph’s house and spoke to him at the doorway of the house.}%
\verse{And they said, “Please, my lord, we surely came down once before to buy food,}%
\verse{but when we came to the place of lodging and we opened our sacks, then behold, each one’s money was in the mouth of his sack — our money in its full weight — so we have returned with it in our hands.}%
\verse{Now, other money we have brought down in our hand to buy food. We do not know who put our money in our sacks.”}%
\verse{And he said, “Peace to you; do not be afraid. Your God and the God of your father must have given you a treasure in your sacks; your money came to me.” And he brought Simeon out to them.}%
\verse{Then the man brought the men into Joseph’s house and he gave them water and washed their feet, and gave fodder to their donkeys.}%
\verse{Then they laid out\lebnote{Or “prepared”} the gift until Joseph came at noon, for they had heard that they were to eat food there.}%
\verse{And when Joseph came into the house they brought the gift that was in their hand into the house to him, and they bowed down before him to the ground.}%
\verse{And he greeted them\lebnote{“he asked peace for them”} and said, “Is your father well, the old man of whom you spoke? Is he still alive?”}%
\verse{And they said, “Your servant our father is well; he is still alive.” And they knelt and bowed down.}%
\verse{Then he lifted up his eyes and saw Benjamin his brother, the son of his mother, and said, “Is this your youngest brother of whom you told me?” And he continued,\lebnote{Or “said”} “God be gracious to you, my son.”}%
\verse{Then Joseph hurried away,\lebnote{“made haste”} being overcome with emotion\lebnote{“his compassion boiled up”} toward his brother, and sought for a place to cry. Then he went into a room and wept there.}%
\verse{Then he washed his face and went out, now controlling himself, and said, “Serve the food.”}%
\verse{And they served him\lebnote{That is, Joseph} by himself, and them by themselves, and the Egyptians who were eating with him by themselves, for Egyptians could not dine\lebnote{“were not able to eat food”} with Hebrews, because that was a detestable thing to Egyptians.}%
\verse{And they were seated before him from the firstborn according to his birthright to the youngest according to his youth. And the men looked at one another\lebnote{“each to his companion”} amazed.}%
\verse{And portions were served to them from his table,\lebnote{“his presence”} and the portion of Benjamin was five times greater than the portion of any of them. And they drank and became drunk with him.}%
\end{biblechapter}%
\begin{biblechapter}% Genesis 44
\verseWithHeading{Joseph Tests His Brothers}{Then he commanded the one who was over his household, saying, “Fill the sacks of the men with food as much as they are able to carry, and put each one’s money in the mouth of his sack.}%
\verse{And my cup — the cup of silver — you shall put into the mouth of the sack of the youngest, and the money for his grain. And he did according to the word of Joseph that he had commanded.}%
\verse{When the morning light came the men were sent away, they and their donkeys.}%
\verse{They went out of the city, and had not gone far when Joseph said to the one who was over his house, “Arise! Pursue after the men and overtake them. Then you shall say to them, ‘Why have you repaid evil for good?}%
\verse{Is this not that from which my master drinks? Now he himself certainly practices divination with it. You have done evil in what you have done.’”}%
\verse{When he overtook them he spoke these words to them.}%
\verse{And they said to him, “Why has my lord spoken according to these words? Far be it from your servants to do such a thing!}%
\verse{Behold, the money that we found in the mouth of our sacks we returned to you from the land of Canaan. Now why would we steal silver or gold from the house of my lord?}%
\verse{Whoever is found with it from among your servants shall die. And moreover, we will become slaves to my lord.”}%
\verse{Then he said, “Now also according to your words, thus will it be. He who is found with it shall be my slave, but you shall be innocent.”}%
\verse{Then each man quickly brought down his sack to the ground, and each one opened his sack.}%
\verse{And he searched, beginning with the oldest and finishing with the youngest. And the cup was found in the sack of Benjamin.}%
\verse{Then they tore their clothes, and each one loaded his donkey and they returned to the city.}%
\verse{And Judah and his brothers came to the house of Joseph — now he was still there — they fell before him to the ground.}%
\verse{Then Joseph said to them, “What is this deed that you have done? Did you not know that a man who is like me surely practices divination?”}%
\verse{And Judah said, “What can we say to my lord? What can we speak? Now how can we show ourselves innocent? God has found the guilt of your servants! Behold, we are slaves to my lord, both we and also he in whose hand the cup was found.”}%
\verse{But he said, “Far be it from me to do this! The man in whose hand the cup was found, he will become my slave. But as for you, go up in peace to your father.”}%
\verse{But Judah drew near to him and said, “Please my lord, let your servant speak a word in the ears of my lord, and let not your anger burn\lebnote{“let not your nose become hot”} against your servant, for you are like Pharaoh himself.\lebnote{“like you like Pharaoh”}}%
\verse{My lord had asked his servants, saying, ‘Do you have a father or a brother?’}%
\verse{And we said to my lord, ‘We have an aged father, and a younger brother, the child of his old age, and his brother died, and he alone remains from his mother, and his father loves him.’}%
\verse{Then you said to your servants, ‘Bring him down to me that I may set my eyes upon him.’}%
\verse{Then we said to my lord, ‘The boy cannot leave his father; if he should leave his father, then he\lebnote{That is, the father (Jacob/Israel)} would die.’}%
\verse{Then you said to your servants, ‘Unless your youngest brother comes down with you, you shall not again see my face.’}%
\verse{And it happened that we went up to your servant, my father, and told him the words of my lord.}%
\verse{And when our father said, ‘Buy a little food for us,’}%
\verse{then we said, ‘We cannot go down. If our youngest brother is with us, then we shall go down. For we will not be able to see the face of the man unless our youngest brother is with us.’}%
\verse{Then your servant, my father, said to us, ‘You yourselves know that my wife bore two sons to me.}%
\verse{One went out from me, and I said, “Surely he must have been torn to pieces,” and I have never seen him since.}%
\verse{And if you take this one also from me, and he encounters harm, you will bring down my gray head in sorrow to Sheol.’}%
\verse{So now, when I come to your servant, my father, and the boy is not with us — now his life is bound up with his life —}%
\verse{it shall happen that when he sees that the boy is gone, he will die. And your servants will bring down the gray head of your servant, our father, to Sheol with sorrow.}%
\verse{For your servant is pledged as surety for the boy by my father, saying, If I do not bring him to you, then I shall be culpable to my father forever.}%
\verse{So then, please let your servant remain in place of the boy as a slave to my lord, and let the boy go up with his brothers.}%
\verse{For how can I go up to my father if the boy is not with me? I do not want to see\lebnote{“lest I see”} the misery which will find my father.”}%
\end{biblechapter}%
\begin{biblechapter}% Genesis 45
\verseWithHeading{Joseph Reveals His Identity}{Then Joseph was not able to control himself before all who were standing by him. And he cried out, “Make every man go out from me!” So no one stood with him when Joseph made himself known to his brothers.}%
\verse{And he wept loudly,\lebnote{“he gave his voice with weeping”} so that the Egyptians heard it and the household of Pharaoh heard it.}%
\verse{Then Joseph said to his brothers, “I am Joseph! Is my father still alive?” And his brothers were unable to answer him, for they were dismayed at his presence.}%
\verse{So Joseph said to his brothers, “Come near to me, please.” And they drew near. And he said, “I am Joseph, your brother, whom you sold into Egypt.}%
\verse{So now, do not be distressed and do not be angry with yourselves\lebnote{“in your eyes”} that you sold me here, for God sent me as deliverance before you.}%
\verse{For these two years the famine has been in the midst of the land, but there will be five more years where there is no plowing or harvest.}%
\verse{And God sent me before you all to preserve for you a remnant in the land and to keep alive among you many survivors.}%
\verse{So now, you yourselves did not send me here, but God put me here as father to Pharaoh and as master of all his household, and a ruler over all the land of Egypt.}%
\verse{Hurry, and go up to my father and say to him, ‘Thus says your son Joseph, God has made me lord of all Egypt. Come down to me and do not delay.}%
\verse{You shall settle in the land of Goshen so that you will be near me, you and your children and your grandchildren, and your flocks and your herds and all that you have.}%
\verse{And I will provide for you there, because there are still five years of famine — lest you and your household and all that you have become destitute.’}%
\verse{Now behold, your eyes see, and the eyes of my brother Benjamin see, that it is I\lebnote{“it is my mouth”} who am speaking to you.}%
\verse{And you must tell my father of all my honor in Egypt and all that you have seen. Now hurry and bring my father here.”}%
\verse{Then he fell upon the neck of his brother Benjamin and wept, and Benjamin wept upon his neck.}%
\verse{And he kissed all his brothers and wept upon them. And afterward his brothers spoke with him.}%
\verse{Then the report was heard in the house of Pharaoh, saying, “Joseph’s brothers have come.” And it pleased Pharaoh and his servants.}%
\verse{Then Pharaoh said to Joseph, “Say to your brothers: ‘Do this — load your donkeys and go back to the land of Canaan,}%
\verse{and take your father and your households and come to me, and I will give you the best of the land of Egypt, and you shall eat the fat of the land.’}%
\verse{And you Joseph, are commanded to say this: ‘Do this! Take wagons from the land of Egypt for your little ones and your wives, and bring your father and come!}%
\verse{Do not worry\lebnote{“let not your eyes be troubled”} about your possessions, for the best of all the land of Egypt is yours.’”}%
\verse{And the sons of Israel did so. And Joseph gave them wagons at the word of Pharaoh, and gave them provisions for the journey.}%
\verse{To each and to all of them he gave sets of clothing, but to Benjamin he gave three hundred pieces of silver and five sets of clothing.}%
\verse{And to his father he sent as follows:\lebnote{“according to this”} ten donkeys carrying the best of Egypt, and ten donkeys carrying grain and food and provisions for his father for the journey.}%
\verse{Then he sent his brothers away, and when they departed he said to them, “Do not be agitated on the journey.”}%
\verse{So they went up from Egypt and came to the land of Canaan to Jacob their father.}%
\verse{And they spoke to him, saying, “Joseph is still alive, and he is ruler over all the land of Egypt.” And his heart went numb,\lebnote{Or “became cold”} because he did not believe him.}%
\verse{Then they told him all the words of Joseph that he had spoken to them. And when he saw the wagons that Joseph had sent to carry him, then the spirit of Jacob their father revived.}%
\verse{And Israel said, “It is enough. Joseph my son is still alive. I will go and see him before I die.”}%
\end{biblechapter}%
\begin{biblechapter}% Genesis 46
\verseWithHeading{Jacob and His Offspring Go to Egypt}{So Israel journeyed with all that he had, and he came to Beersheba and offered sacrifices to the God of his father, Isaac.}%
\verse{And God spoke to Israel in visions of the night and said, “Jacob, Jacob.” And he said, “Here I am.”}%
\verse{Then he said, “I am the God of your father. Do not be afraid to go down to Egypt, for I will make you a great nation there.}%
\verse{I myself will go down with you to Egypt, and I myself will also bring you up. And Joseph will place his hand over your eyes.”}%
\verse{So Jacob arose from Beersheba. And the sons of Israel carried their father Jacob, and their little ones and their wives in the wagons Pharaoh had sent to transport him.}%
\verse{And they took their livestock and their possessions that they had acquired in the land of Canaan. And they came to Egypt, Jacob and all his offspring with him,}%
\verse{his sons and his sons’ sons with him, his daughters and his daughters’ daughters with him, into Egypt.}%
\verse{Now these are the names of the sons of Israel, who came into Egypt, Jacob and his sons. Reuben, the firstborn of Jacob}%
\verse{and the sons of Reuben: Enoch, Pallu, Hezron, and Carmi.}%
\verse{The sons of Simeon: Jemuel, Jamin, Ohad, Jakin, Zohar, and Shaul, the son of a Canaanite woman.}%
\verse{The sons of Levi: Gershon, Kohath, and Merari.}%
\verse{The sons of Judah: Er, Onan, Shelah, Perez, and Zerah (but Er and Onan died in the land of Canaan). And the sons of Perez were Hezron and Hamul.}%
\verse{The sons of Issachar: Tolah, Puvah, Iob, and Shimron.}%
\verse{The sons of Zebulun: Sered, Elon, and Jahleel.}%
\verse{These are the sons of Leah that she bore to Jacob in Paddan-Aram, and Dinah his daughter. His sons and daughters were thirty-three persons in all.}%
\verse{The sons of Gad: Ziphion, Haggi, Shuni, Ezbon, Eri, Arodi, and Areli.}%
\verse{The sons of Asher: Imnah, Ishvah, Ishvi, and Beriah, and their sister Serah. And the sons of Beriah: Heber and Malkiel.}%
\verse{There are the sons of Zilpah, whom Laban gave to Leah his daughter, and she bore these to Jacob — sixteen persons.}%
\verse{The sons of Rachel, Jacob’s wife: Joseph and Benjamin.}%
\verse{And Ephraim and Manasseh, whom Asenath, daughter of Potiphera, priest of On bore to him, were born to Joseph in the land of Egypt.}%
\verse{The sons of Benjamin: Bela, Beker, Ashbel, Gera, Naaman, Ehi, Rosh, Muppim, Huppim, and Ard.}%
\verse{These are the sons of Rachel who were born to Jacob — fourteen persons in all.}%
\verse{The sons of Dan: Hushim.}%
\verse{The sons of Naphtali: Jahzeel, Guni, Jezer, and Shillem.}%
\verse{These are the sons of Bilhah whom Laban gave to Rachel his daughter, and she bore these to Jacob — seven persons in all.}%
\verse{All the persons belonging to Jacob who came to Egypt who were his descendants,\lebnote{“those going out of his loins”} not including the wives of the sons of Jacob were sixty-six persons in all.}%
\verse{And the sons of Joseph who were born to him in Egypt were two persons. All the persons of the house of Jacob who came to Egypt were seventy.}%
\verse{He\lnDG{} had sent Judah ahead of him to Joseph to appear before him\lnDG{} in Goshen. And they came to the land of Goshen.}%
\verse{Then Joseph harnessed his chariot and went up to meet Israel his father in Goshen. He presented himself to him and fell upon his neck and wept upon his neck a long time.}%
\verse{Then Israel said to Joseph, “Now let me die since I have seen your face, for you are still alive.”}%
\verse{Then Joseph said to his brothers and to his father’s household, “I will go up and report to Pharaoh, and I will say to him, ‘My brothers and my father’s household who were in the land of Canaan have come to me.}%
\verse{And the men are shepherds, for they are men of livestock, and they have brought their flocks and their cattle and all that they have.’}%
\verse{And it shall be that when Pharaoh calls you he will say, ‘What is your occupation?’}%
\verse{Then you must say, ‘You servants are men of livestock from our childhood until now, both we and also our ancestors,’\lebnote{Or “fathers”} so that you may dwell in the land of Goshen, for every shepherd is a detestable thing to Egyptians.”}%
\end{biblechapter}%
\begin{biblechapter}% Genesis 47
\verseWithHeading{Jacob Settles in Goshen}{So Joseph went and reported to Pharaoh. And he said, “My father and my brothers, with their flocks and their herds, and all that they have, have come from the land of Canaan. Now they are here in the land of Goshen.”}%
\verse{And from among his brothers he took five men and presented them before Pharaoh.}%
\verse{And Pharaoh said to his brothers, “What is your occupation?” And they said to Pharaoh, “Your servants are keepers of sheep, both we and also our ancestors.”\lnDH{}}%
\verse{And they said to Pharaoh, “We have come to sojourn in the land, for there is no pasture for your servant’s flocks, for the famine is severe in the land of Canaan. So now, please let your servants dwell in the land of Goshen.”}%
\verse{Then Pharaoh said to Joseph, “Your father and your brothers have come to you.}%
\verse{The land of Egypt is before you. Settle your father and your brothers in the best of the land. Let them live in the land of Goshen, and if you know there is among them men of ability, then appoint them overseers of my own livestock.”}%
\verse{Then Joseph brought his father Jacob and presented him before Pharaoh. And Jacob blessed Pharaoh.}%
\verse{Then Pharaoh said to Jacob, “How old are you?”\lebnote{“How many are the days of the years of your life?”}}%
\verse{And Jacob said to Pharaoh, “The days of the years of my sojourning are one hundred and thirty years. Few and hard have been the days of the years of my life, and they have not reached the days of the years of the lives of my ancestors\lnDH{} in the days of their sojourning.”}%
\verse{And Jacob blessed Pharaoh, and he went out from the presence of Pharaoh.}%
\verse{And Joseph settled his father and his brothers, and he gave them property in the land of Egypt in the best part of the land, in the land of Rameses, as Pharaoh had instructed.}%
\verse{And Joseph provided his father and his brothers and all the household of his father with food, according to the number of their children.}%
\verseWithHeading{The Famine in Egypt Continues}{Now there was no food in all the land, for the famine was very severe. And the land of Egypt languished, with the land of Canaan, on account of the famine.}%
\verse{And Joseph collected all the money found in the land of Egypt and in the land of Canaan in exchange for the grain that they were buying. And Joseph brought the money into the house of Pharaoh.}%
\verse{And when the money was spent in the land of Egypt and from the land of Canaan, all of Egypt came to Joseph, saying, “Give us food! Why should we die before you? For the money is used up.”}%
\verse{And Joseph said, “Give your livestock and I will give you food in exchange for your livestock if your money is used up.”}%
\verse{So they brought their herds to Joseph, and Joseph gave food to them in exchange for horses, their flocks, and their cattle and donkeys. And he provided them with food in exchange for all their livestock that year.}%
\verse{When that year ended, they came to him in the following year and said to him, “We cannot hide from my lord that our money and livestock belong to my lord. Nothing remains before my lord except our bodies and our land.}%
\verse{Why should we die in front of you, both we and our land? Buy us and our land in exchange for food, then we and our land will be servants to Pharaoh. Then give us seed and we shall live and not die, and the land will not become desolate.”}%
\verse{So Joseph bought all the land of Egypt for Pharaoh, for each Egyptian sold his field, for the famine was severe upon them. And the land became Pharaoh’s.}%
\verse{As for the people, he transferred them to the cities, from one end of the territory of Egypt to the other.}%
\verse{Only the land of the priests he did not buy, for there was an allotment for the priests from Pharaoh, and they lived on\lebnote{“ate”} the allotment that Pharaoh gave to them. Therefore they did not sell their land.}%
\verse{And Joseph said to the people, “Look, I have bought you and your land this day for Pharaoh. Here is seed for you so you can sow the land.}%
\verse{And it shall happen that at the harvest, you must give a fifth to Pharaoh and four-fifths shall be yours, as seed for the field and for your food and for those who are in your households, and as food for your little ones.”}%
\verse{And they said, “You have saved our lives. If we have found favor in the eyes of my lord, we will be servants to Pharaoh.”}%
\verse{So Joseph made it a statute unto this day concerning the land of Egypt: one fifth to Pharaoh. Only the land of the priests alone did not belong to Pharaoh.}%
\verse{So Israel settled in the land of Egypt, in the land of Goshen. And they acquired possessions in it and were fruitful and multiplied greatly.}%
\verse{And Jacob lived in the land of Egypt seventeen years. And the days of Jacob, the years of his life, were one hundred and forty-seven years.}%
\verse{When the time of Israel’s death drew near,\lebnote{“the days of Israel drew near to die”} he called to his son, to Joseph. And he said to him, “If I have found favor in your eyes, please put your hand under my thigh, that you might vow to deal kindly\lebnote{Or “loyal love”} and faithfully with me. Please do not bury me in Egypt,}%
\verse{but let me lie with my ancestors.\lnDH{} Carry me out of Egypt and bury me in their burial site.” And he said, “I will do according to your word.”}%
\verse{Then he said, “Swear to me.” And he swore to him. Then Israel bowed himself on the head of the bed.}%
\end{biblechapter}%
\begin{biblechapter}% Genesis 48
\verseWithHeading{Jacob Blesses Ephraim and Manasseh}{And it happened that after these things, it was said to Joseph, “Behold, your father is ill.” And he took his two sons with him, Ephraim and Manasseh.}%
\verse{And it was told to Jacob, “Behold, your son Joseph has come to you.” Then Israel strengthened himself and he sat up in the bed.}%
\verse{Then Jacob said to Joseph, “El-Shaddai\lebnote{Often translated “God Almighty”} appeared to me in Luz, in the land of Canaan, and blessed me,}%
\verse{and said to me, ‘Behold, I will make you fruitful and make you numerous, and will make you a company of nations. And I will give this land to your offspring after you as an everlasting possession.’}%
\verse{And now, your two sons who were born to you in the land of Egypt before my coming to you in Egypt, are mine. Ephraim and Manasseh shall be mine as Reuben and Simeon are.}%
\verse{And your children whom you father after them shall be yours. By the name of their brothers they shall be called, with respect to their inheritance.}%
\verse{As for me, when I came to Paddan-Aram Rachel died to my sorrow\lebnote{“to me”} in the land of Canaan on the way when there was still some distance to go to Ephrath. And I buried her there on the way to Ephrath (that is, Bethlehem).”}%
\verse{When Israel saw the sons of Joseph he said, “Who are these?”}%
\verse{Then Joseph said to his father, “They are my sons whom God has given me here.” And he said, “Please bring them to me that I may bless them.”}%
\verse{Now the eyes of Israel were dim\lebnote{Or “heavy” (i.e., his eyelids were heavy; his eyes were closed)} on account of old age; he was not able to see. So he brought them near to him, and he kissed them and embraced them.}%
\verse{And Israel said to Joseph, “I did not expect to see your face and behold, God has also shown me your offspring.”}%
\verse{Then Joseph removed them from his knees and bowed down with his face to the ground.}%
\verse{And Joseph took the two of them, Ephraim at his right to the left of Israel, and Manasseh at his left to the right of Israel. And he brought them near to him.}%
\verse{And Israel stretched out his right hand and put it on the head of Ephraim (now he was the younger), and his left hand on the head of Manasseh, crossing his hands, for Manasseh was the firstborn.}%
\verse{And he blessed Joseph and said, “The God before whom my fathers, Abraham and Isaac, walked, The God who shepherded me all my life\lebnote{“since my duration”} unto this day,}%
\verse{The angel who redeemed me from all evil, may he bless the boys. And through them let my name be perpetuated,\lebnote{Or “called”} and the name of my fathers, Abraham and Isaac. And let them multiply into many in the midst of the earth.}%
\verse{When Joseph saw that his father put his right hand on the head of Ephraim, he was displeased. And he took hold of his father’s hand to remove it from the head of Ephraim over to the head of Manasseh.}%
\verse{And Joseph said to his father, “Not so, my father; because this one is the firstborn. Put your right hand upon his head.”}%
\verse{But his father refused and said, “I know, my son; I know. He also shall become a people, and he also shall be great, but his younger brother shall be greater than him, and his offspring shall become a multitude of nations.”}%
\verse{So he blessed them that day, saying, Through you Israel shall pronounce blessing, saying, ‘May God make you like Ephraim and like Manasseh.’” So he put Ephraim before Manasseh.}%
\verse{And Israel said to Joseph, “Behold, I am about to die, but God will be with you and will bring you back to the land of your ancestors.\lebnote{Or “fathers”}}%
\verse{And I have given to you one slope of land rather than your brothers, which I took from the hand of the Amorites by my sword and with my bow.”}%
\end{biblechapter}%
\begin{biblechapter}% Genesis 49
\verseWithHeading{Jacob Blesses His Twelve Sons}{Then Jacob called his sons and said, “Gather together so that I can tell you what will happen with you in days to come.\lebnote{“the latter days”}}%
\verse{Assemble and hear, O sons of Jacob! Listen to Israel your father!}%
\verse{Reuben, you are my firstborn, my strength, and the firstfruit of my vigor, excelling in rank and excelling in power.}%
\verse{Unstable\lebnote{Or “reckless”} as water, you shall not excel any longer, for you went up upon the bed of your father, then defiled it. You went up upon my couch!}%
\verse{Simeon and Levi are brothers; weapons of violence are their swords. Let me\lebnote{Or “my soul”} not come into their council.}%
\verse{Let not my person\lebnote{Or “my glory”} be joined to their company. For in their anger they killed men, and at their pleasure they hamstrung cattle.}%
\verse{Cursed be their anger, for it is fierce, and their wrath, for it is cruel. I will divide them in Jacob, and I will scatter them in Israel.}%
\verse{Judah, as for you, your brothers shall praise you. Your hand shall be on the neck of your enemies. The sons of your father shall bow down to you.}%
\verse{Judah is a lion’s cub. From the prey, my son, you have gone up. He bowed down; he crouched like a lion and as a lioness. Who shall rouse him?}%
\verse{The scepter shall not depart from Judah, nor the ruler’s staff between his feet, until Shiloh comes. And to him shall be the obedience of nations.}%
\verse{Binding his donkey to the vine and his donkey’s colt to the choice vine, he washes his clothing in the wine and his garment in the blood of grapes.}%
\verse{The eyes are darker than wine, and the teeth whiter than milk.}%
\verse{Zebulun shall settle by the shore of the sea. He shall become a haven for ships, and his border shall be at Sidon.}%
\verse{Issachar is a strong donkey, crouching between the sheepfolds.}%
\verse{He saw a resting place that was good, and land that was pleasant. So he bowed his shoulder to the burden and became a servant of forced labor.}%
\verse{Dan shall judge his people as one of the tribes of Israel.}%
\verse{Dan shall be a serpent on the way, a viper on the road that bites the heels of a horse, so that its rider falls backward.}%
\verse{I wait for your salvation, O Adonai.}%
\verse{Bandits shall attack Gad, but he shall attack their heels.}%
\verse{Asher’s food is delicious, and he shall provide from the king’s delicacies.}%
\verse{Naphtali is a doe running free that puts forth beautiful words.}%
\verse{Joseph is the bough\lnDI{} of a fruitful vine, a fruitful bough\lnDI{} by a spring. His branches climb over the wall.}%
\verse{The archers\lebnote{“the masters of the bow”} fiercely attacked him. They shot arrows at him and were hostile to him.}%
\verse{But his bow remained in a steady position; his arms\lebnote{“the arms of his hands”} were made agile by the hands of the Mighty One of Jacob. From there is the Shepherd, the Rock of Israel.}%
\verse{Because of the God of your father he will help you and by Shaddai\lebnote{Or “the Almighty”} he will bless you with the blessings of heaven above, blessings of the deep that crouches beneath, blessings of the breasts and the womb.}%
\verse{The blessings of your father are superior to the blessings of my ancestors, to the bounty of the everlasting hills. May they be on the head of Joseph, and on the forehead of the prince of his brothers.}%
\verse{Benjamin is a devouring wolf, devouring the prey in the morning, and dividing the plunder in the evening.}%
\verseWithHeading{The Death and Burial of Jacob}{All these are the twelve tribes of Israel, and this is what their father said to them when he blessed them, each according to their blessing.}%
\verse{Then he instructed them and said to them, “I am about to be gathered to my people. Bury me among my ancestors\lebnote{Or “fathers”} in the cave that is in the field of Ephron the Hittite,}%
\verse{in the cave that is in the field of Machpelah that is before\lebnote{Or “east of”} Mamre in the land of Canaan, which Abraham bought with the field from Ephron the Hittite as a burial site.}%
\verse{There they buried Abraham and Sarah his wife. There they buried Isaac and Rebekah his wife. And there I buried Leah —}%
\verse{the purchase of the field and the cave which was in it from the Hittites.”}%
\verse{When Jacob finished instructing his sons he drew his feet up to the bed. Then he took his last breath and was gathered to his people.}%
\end{biblechapter}%
\begin{biblechapter}% Genesis 50
\verseWithHeading{Jacob’s Funeral and Joseph’s Remaining Time in Egypt}{Then Joseph fell on the face of his father and wept upon him and kissed him.}%
\verse{And Joseph instructed his servants the physicians to embalm his father. So the physicians embalmed Israel.}%
\verse{Forty days were required for it,\lebnote{“were fulfilled for it”} for thus are the days required for\lebnote{“fulfilled”} embalming. And the Egyptians wept for him seventy days.}%
\verse{When the days of his weeping had passed, Joseph spoke to the household of Pharaoh, saying, “If I have found favor in your eyes, please speak in the hearing of Pharaoh, saying,}%
\verse{‘My father made me swear, saying, “Behold, I am about to die. In the tomb that I have hewed out for myself in the land of Canaan — there you must bury me.” So then, please let me go up and let me bury my father; then I will return.’”}%
\verse{Then Pharaoh said, “Go up and bury your father as he made you swear.”}%
\verse{So Joseph went up to bury his father. And all the servants of Pharaoh, the elders of his household, and all the elders of the land of Egypt, went up with him,}%
\verse{with all the household of Joseph, his brothers, and the household of his father. They left only their little children and their flocks and their herds in the land of Goshen.}%
\verse{And there also went up with him chariots and horsemen. The company was very great.}%
\verse{When they came to the threshing floor of Atad, which was beyond the Jordan, they lamented there with a very great and sorrowful wailing. And he made a mourning ceremony for his father seven days.}%
\verse{And when the Canaanites, the inhabitants of the land, saw the mourning ceremony at the threshing floor of Atad they said, “This is a severe mourning for the Egyptians.” Therefore its name was called Abel-Mizraim, which is beyond the Jordan.}%
\verse{Thus his sons did to him just as he had instructed them.}%
\verse{And his sons carried him to the land of Canaan and buried him in the cave of the field of Machpelah, which field Abraham had bought as a burial site from Ephron the Hittite before\lebnote{Or “east of”} Mamre.}%
\verse{And after burying his father, Joseph returned to Egypt, he and his brothers and all who had gone up with him to bury his father.}%
\verse{And when the brothers of Joseph saw that their father was dead, they said, “It may be that Joseph will hold a grudge against us and pay us back dearly for all the evil that we did to him.”}%
\verse{So they sent word to Joseph saying, “Your father commanded us before his death, saying,}%
\verse{“Thus you must say to Joseph, ‘O, please now forgive the transgression of your brothers and their sin, for they did evil to you.’ So now, please forgive the transgression of the servants of the God of your father.” And Joseph wept when they spoke to him.}%
\verse{Then his brothers went also and fell before him and said, “Behold, we are your servants.”}%
\verse{Then Joseph said to them, “Do not be afraid, for am I in the place of God?}%
\verse{As for you, you planned evil against me, but God planned it for good, in order to do this — to keep many people alive — as it is today.}%
\verse{So then, do not be afraid. I myself will provide for you and your little ones. And he consoled them and spoke kindly\lebnote{“spoke to their heart”} to them.}%
\verseWithHeading{The Death of Joseph}{So Joseph remained in Egypt, he and the house of his father. And Joseph lived one hundred and ten years.}%
\verse{And Joseph saw Ephraim’s children to the third generation. Moreover, the children of Makir, son of Manasseh, were born on the knees of Joseph.}%
\verse{And Joseph said to his brothers, “I am about to die, but God will certainly visit you and bring you up from this land to the land that he swore to Abraham, to Isaac, and to Jacob.”}%
\verse{Then Joseph made the sons of Israel swear an oath, saying, “God will surely visit you, and you shall bring up my bones from here.”}%
\verse{So Joseph died, being one hundred and ten years old. They embalmed him and he was placed in a coffin in Egypt.}%
\end{biblechapter}%
\flushcolsend
\input{leb/content/old-testament/Exo.tex}\flushcolsend
\input{leb/content/old-testament/Lev.tex}\flushcolsend
\biblebook{Numbers}
\begin{biblechapter}% Numbers 1
\verseWithHeading{God Commands Moses to Take a Census}{Adonai spoke to Moses in the desert of Sinai, in the tent of assembly, on the first of the month, in the second year after they came out\lebnote{“of their coming out”} of the land of Egypt, saying,}%
\verse{“Take a census of\lnQA{} the entire community of the Israelites\lnQB{} according to their clans and their families,\lnQC{} according to the number of names, every male individually}%
\verse{from twenty years old\lnQD{} and above, everyone in Israel who is able to go to war. You and Aaron must muster them\lebnote{Or “count them,” or “summon them,” or “enroll them”} for their wars.}%
\verse{A man from each tribe will be with you, each man the head of his family.\lebnote{“the house of his father”}}%
\verse{And these are the names of the men who will assist you:\lebnote{“stand with you”} from Reuben, Elizur son of Shedeur;}%
\verse{from Simeon, Shelumiel son of Zurishaddai;}%
\verse{from Judah, Nahshon son of Amminadab;}%
\verse{from Issachar, Nethanel son of Zuar;}%
\verse{from Zebulun, Eliab son of Helon.}%
\verse{From the descendants of Joseph: from Ephraim, Elishama son of Ammihud; from Manasseh, Gamaliel son of Pedahzur.}%
\verse{From Benjamin, Abidan son of Gideoni;}%
\verse{from Dan, Ahiezer son of Ammishaddai;}%
\verse{from Asher, Pagiel son of Ocran;}%
\verse{from Gad, Eliasaph son of Deuel;}%
\verse{and from Naphtali, Ahira son of Enan.”}%
\verse{These are the ones summoned from the community, the leaders of their ancestors’\lnQE{} tribes; they are the heads of Israel’s clans.}%
\verse{So Moses and Aaron took these men who had been designated by name,}%
\verse{and they summoned the entire community on the first day of the second month. And they registered themselves among their clans according to their families,\lnQC{} according to the number of names from those twenty years old\lnQD{} and above individually,}%
\verse{just as Adonai commanded Moses. And he counted them in the desert of Sinai.}%
\verse{The descendants of Reuben, the firstborn of Israel, their genealogies according to their clans, according to their families,\lnQC{} according to the number of names, every male individually from twenty years old\lnQD{} and above, everyone who is able to go to war:}%
\verse{those who were counted from the tribe of Reuben were forty-six thousand five hundred.}%
\verse{From the descendants of Simeon, their genealogies according to their clans, according to their families,\lnQC{} those who were counted according to the number of their names, every individual male from twenty years old\lnQD{} and above, everyone who is able to go to war:}%
\verse{those who were counted from the tribe of Simeon were fifty-nine thousand three hundred.}%
\verse{From the descendants of Gad, their genealogies according to their clans, according to their families,\lnQC{} according to the number of names, from those twenty years old\lnQD{} and above, everyone who is able to go to war:}%
\verse{those who were counted from the tribe of Gad were forty-five thousand six hundred and fifty.}%
\verse{From the descendants of Judah, their genealogies according to their clans, according to their families,\lnQC{} according to the number of names, from those twenty years old\lnQD{} and above, everyone who is able to go to war:}%
\verse{those who were counted from the tribe of Judah were seventy-four thousand six hundred.}%
\verse{From the descendants of Issachar, their genealogies according to their clans, according to their families,\lnQC{} according to the number of names, from those twenty years old\lnQD{} and above, everyone who is able to go to war:}%
\verse{those who were counted from the tribe of Issachar were fifty-four thousand four hundred.}%
\verse{From the descendants of Zebulun, their genealogies according to their clans, according to their families,\lnQC{} according to the number of names, from those twenty years old\lnQD{} and above, everyone who is able to go to war:}%
\verse{those who were counted from the tribe of Zebulun were fifty-seven thousand four hundred.}%
\verse{From the descendants of Joseph: from the descendants of Ephraim, their genealogies according to their clans, according to their families,\lnQC{} according to the number of names, from those twenty years old\lnQD{} and above, everyone who is able to go to war:}%
\verse{those who were counted from the tribe of Ephraim were forty thousand five hundred.}%
\verse{From the descendants of Manasseh, their genealogies according to their clans, according to their families,\lnQC{} according to the number of names, from those twenty years old\lnQD{} and above, everyone who is able to go to war:}%
\verse{those who were counted from the tribe of Manasseh were thirty-two thousand two hundred.}%
\verse{From the descendants of Benjamin, their genealogies according to their clans, according to their families,\lnQC{} according to the number of names, from those twenty years old\lnQD{} and above, everyone who is able to go to war:}%
\verse{those who were counted from the tribe of Benjamin were thirty-five thousand four hundred.}%
\verse{From the descendants of Dan, their genealogies according to their clans, according to their families,\lnQC{} according to the number of names, from those twenty years old\lnQD{} and above, everyone who is able to go to war:}%
\verse{those who were counted from the tribe of Dan were sixty-two thousand seven hundred.}%
\verse{From the descendants of Asher, their genealogies according to their clans, according to their families,\lnQC{} according to the number of names, from those twenty years old\lnQD{} and above, everyone who is able to go to war:}%
\verse{those who were counted from the tribe of Asher were forty-one thousand five hundred.}%
\verse{From the descendants of Naphtali, their genealogies according to their clans, according to their families,\lebnote{“their fathers”} according to the number of names, from those twenty years old\lnQD{} and above, everyone who is able to go to war:}%
\verse{those who were counted from the tribe of Naphtali were fifty-three thousand four hundred.}%
\verse{These are the ones counted whom Moses and Aaron mustered,\lnQF{} with the twelve leaders of Israel, each one from his family.\lebnote{“the house of his fathers”}}%
\verse{So all those who were counted from the Israelites\lnQB{} according to their families,\lnQC{} from those twenty years old\lnQD{} and above, everyone in Israel who is able to go to war.}%
\verse{All of the ones counted were six hundred and three thousand, five hundred and fifty.}%
\verse{The Levites from their ancestors’\lnQE{} tribe were not mustered\lnQF{} in their midst.}%
\verse{And Adonai spoke to Moses, saying,}%
\verse{“You will not muster\lnQG{} the tribe of Levi, and you will not take a census of\lnQA{} them in the midst of the Israelites.\lnQB{}}%
\verse{You will appoint\lnQG{} them over the tabernacle of the testimony,\lnQH{} over all its vessels, and over all that belongs to it. They will carry the tabernacle and all its vessels, and they will care for it; and they will camp around the tabernacle.}%
\verse{And when the tabernacle is set out, the Levites will take it down,\lebnote{“lower it”} and when encamping the tabernacle the Levites will set it up; the stranger\lebnote{Or “outsider”} that approaches it will be put to death.}%
\verse{The Israelites\lnQB{} will encamp, each in their own camp, and each by their\lebnote{Hebrew “his” or “its”} own banner according to their divisions.}%
\verse{But the Levites will encamp around the tabernacle of the testimony,\lnQH{} and there will not be wrath on the community of the Israelites;\lnQB{} and the Levites will keep the requirements of the tabernacle of the testimony.”\lnQH{}}%
\verse{And the Israelites\lnQB{} did thus; they did everything that Adonai commanded Moses.}%
\end{biblechapter}%
\begin{biblechapter}% Numbers 2
\verseWithHeading{The Arrangement of the Camps}{Adonai spoke to Moses and Aaron, saying,}%
\verse{“The Israelites\lnQI{} will encamp each with his standard, with a banner according to their families;\lnQJ{} they will encamp around the tent of assembly.}%
\verse{The ones who encamp on the eastern side, toward the sunrise, will be of the standard of the camp of Judah according to their divisions; and the leader of the descendants of Judah will be Nahshon son of Amminadab,}%
\verse{and his division and the ones counted\lnQK{} are seventy-four thousand six hundred.}%
\verse{And the ones who encamp next to him will be the tribe of Issachar. And the leader of the descendants of Issachar will be Nethanel son of Zuar,}%
\verse{and his division are fifty-four thousand four hundred.}%
\verse{For the tribe of Zebulun: the leader of the descendants of Zebulun will be Eliab son of Helon,}%
\verse{and his division and the ones counted\lnQL{} are fifty-seven thousand four hundred.}%
\verse{All those counted from the camp of Judah are one hundred and eighty-six thousand four hundred. They will set out first according to their divisions.}%
\verse{“The standard of the camp of Reuben will be to the south according to their divisions. The leader of the descendants will be Elizur son of Shedeur.}%
\verse{And his division and the ones counted\lnQL{} are forty-six thousand five hundred.}%
\verse{Those encamped next to him will be the tribe of Simeon. The leader of the descendants of Simeon will be Shelumiel son of Zurishaddai.}%
\verse{And his division and the ones counted\lnQK{} are fifty-nine thousand three hundred.}%
\verse{For the tribe of Gad: the leader of the descendants of Gad will be Eliasaph son of Reuel.}%
\verse{And his division and the ones counted\lnQK{} are forty-five thousand six hundred and fifty.}%
\verse{All those counted\lnQM{} from the camp of Reuben are one hundred and fifty-one thousand four hundred and fifty. They will set out second according to their divisions.}%
\verse{“The tent of assembly the camp of the Levites will set out in the midst of the camps; they will set out just as they encamped, each according to their standards.\lebnote{“each man on his hand according to their standards”}}%
\verse{“The standard of the camp of Ephraim according to their divisions will be to the west. The leader of the descendants of Ephraim will be Elishama son of Ammihud.}%
\verse{And his division and the ones counted\lnQK{} are forty thousand five hundred.}%
\verse{The tribe of Manasseh will be next to him. The leader of the descendants of the tribe of Manasseh will be Camaliel son of Pedahzur.}%
\verse{And his division and the ones counted\lnQK{} are thirty-two thousand two hundred.}%
\verse{For the tribe of Benjamin: the leader of the descendants of Benjamin will be Abidan son of Gideoni.}%
\verse{And his division and the ones counted\lnQK{} are thirty-five thousand four hundred.}%
\verse{All those counted\lnQM{} from the camp of Ephraim are one hundred and eighty thousand one hundred. They will set out third according to their divisions.}%
\verse{“The standard of the camp of Dan according to their divisions will be to the west. The leader of the descendants of Dan will be Ahiezer son of Ammishaddai.}%
\verse{And his division and the ones counted\lnQK{} are sixty-two thousand seven hundred.}%
\verse{Those encamped next to him will be the tribe of Asher. The leader of the descendants of Asher will be Pagiel son of Ocran.}%
\verse{And his division and the ones counted\lnQK{} are forty-one thousand five hundred.}%
\verse{For the tribe of Naphtali: the leader of the descendants of Naphtali will be Ahira son of Enan.}%
\verse{And his division and the ones counted\lnQK{} are fifty-three thousand four hundred.}%
\verse{All the ones counted\lnQK{} from the camp of Dan are one hundred and fifty-seven thousand six hundred. They will set out last\lebnote{“from behind”} according to their divisions.”}%
\verse{These were the ones counted of the Israelites\lnQI{} according to their families;\lnQJ{} all those counted from the camps according to their divisions were six hundred and three thousand five hundred.}%
\verse{The Levites were not counted in the midst of the Israelites,\lnQI{} just as Adonai commanded Moses.}%
\verse{And the Israelites\lnQI{} did everything that Adonai commanded Moses. They encamped according to their standards, and they\lebnote{Hebrew “each one,” or “each man”} set out each one according to their clans\lebnote{Hebrew “his clans”} among their families.\lebnote{“the house of his fathers”}}%
\end{biblechapter}%
\begin{biblechapter}% Numbers 3
\verseWithHeading{Aaron’s Sons}{These are the genealogies of Aaron and Moses at the time\lebnote{“on a day”} when Adonai spoke to Moses on Mount Sinai.}%
\verse{These are the names of the descendants of Aaron: Nadab the firstborn, Abihu, Eleazar, and Ithamar.}%
\verse{These are the names of the descendants of Aaron, the priests, the anointed ones whom he consecrated as priests.\lebnote{“he filled their hands to serve as a priest”}}%
\verse{Nadab and Abihu died before Adonai\lnQN{} when they presented a strange fire before Adonai\lnQN{} in the desert of Sinai, and they had no children.\lebnote{“sons were not for them”} Eleazar and Ithamar served as priest during the presence of Aaron their father.}%
\verse{Adonai spoke to Moses, saying,}%
\verse{“Bring near the tribe of Levi, and set the tribe\lebnote{“cause it to stand”} before Aaron\lebnote{“in the presence of Aaron”} the priest, and they will minister to him.}%
\verse{They shall observe his duties and the duties of the entire community before the tent of assembly, to do the work of the tabernacle.}%
\verse{And they will keep all the vessels of the tent of assembly and the responsibilities of the Israelites,\lnQO{} to do the work\lebnote{Or “service”} of the tabernacle.}%
\verse{You will give the Levites to Aaron and to his descendants; they are surely assigned to him from among the Israelites.\lnQO{}}%
\verse{But you will count Aaron and his descendants; they will keep their priesthood, and the stranger\lnQP{} who approaches will be put to death.”}%
\verse{Adonai spoke to Moses saying,}%
\verse{“I myself receive the Levites from the midst of the Israelites\lnQO{} in the place of all the firstborn of the offspring of the womb from the Israelites.\lnQO{} The Levites will be mine}%
\verse{because all the firstborn are mine; on the day of my killing all the firstborn in the land of Egypt, I consecrated for myself all the firstborn in Israel, both humankind and animal;\lebnote{“from humankind to animal”} they will be mine. I am Adonai.”}%
\verse{Adonai spoke to Moses in the desert of Sinai, saying,}%
\verse{“Muster\lebnote{Or “count,” or “summon,” or “enroll”} the descendants of Levi according to their families,\lnQQ{} according to their clans. You will count every male from one month\lnQR{} and above.”}%
\verse{So Moses mustered\lnQS{} them according to the command of Adonai,\lebnote{“mouth of Adonai”} just as he commanded.}%
\verse{These were the sons of Levi according to their names: Gershon, Kohath, and Merari.}%
\verse{And these are the names of the sons of Gershon according to their clans: Libni and Shimei.}%
\verse{And the sons of Kohath according to their clans: Amram, Izhar, Hebron, and Uzziel.}%
\verse{The sons of Merari according to their tribes: Mahli and Mushi. These are the clans of the Levites according to their families.\lnQQ{}}%
\verse{To Gershon belonged\lebnote{“To Gershon was”} the clan of the Libnites and the clan of the Shimeites; these are the clans of the Gershonites.}%
\verse{The ones counted\lnQT{} according to the number of every male from one month\lebnote{“the son of a month and above”} and above were seven thousand five hundred.}%
\verse{The clans of the Gershonites will camp behind the tabernacle to the west,}%
\verse{and the leader of the family\lnQU{} of the Gershonites is Eliasaph son of Lael.}%
\verse{And the responsibility of the descendants of Gershon in the tent of assembly is the tabernacle, and the tent covering it and the curtain of the doorway of the tent of the assembly,}%
\verse{and the hangings of the courtyard and the curtain of the doorway of the courtyard that is around the tabernacle and the altar, and its ten cords, all of its use.}%
\verse{To Kohath belonged\lebnote{“For Kohath \textit{was}”} the clan of Amramites,\lebnote{Hebrew “Amramite”} the clan of the Izharites,\lebnote{Hebrew “Izharite”} the clan of the Hebronites,\lebnote{Hebrew “Hebronite”} and the clan of the Uzzielites;\lebnote{Hebrew “Uzzielite”} these were the clans of the Kohathites.\lnQV{}}%
\verse{According to the number of every male from one month\lnQR{} and above there were eight thousand six hundred keeping the responsibility of the sanctuary.}%
\verse{The clan of the descendants of Kohath will encamp on the side of the tabernacle to the south.}%
\verse{The leader of his family\lebnote{“the house of \textit{his} father”} according to the clans of the Kohathites\lnQV{} is Elizaphan the son of Uzziel.}%
\verse{Their responsibility was the ark, the table, the lampstand, the altar, and the vessels of the sanctuary, with which they ministered, and the curtain, and all of its use.}%
\verse{The chief of the leaders\lebnote{“the leader of leaders”} of the Levites\lebnote{Hebrew “Levite”} was Eleazar son of Aaron the priest who had oversight of those keeping the responsibility of the sanctuary.}%
\verse{To Merari belonged\lebnote{“for Merari \textit{was}”} the clan of Mahlites\lebnote{Hebrew “Mahlite”} and the clan of the Mushites:\lebnote{Hebrew “Mushite”} these are the clans of Merari.}%
\verse{The ones counted\lnQT{} according to the number of every male from one month\lnQR{} and above were six thousand two hundred.}%
\verse{The leader of the family\lnQU{} according to the clans of Merari is Zuriel son of Abihail; they will encamp of the side of the tabernacle to the north.}%
\verse{The responsibility of the sons of Merari was the supervision of the frames of the tabernacle, its bars, pillars, bases, and all its vessels and all its service,}%
\verse{and the pillars around the courtyard, and their bases, pegs, and cords.}%
\verse{Those encamped before the tabernacle to the east — before the tent of assembly to the east — were Moses and Aaron and his sons; they will keep the responsibility of the sanctuary for the Israelites;\lebnote{“for the responsibility of the children of Israel”} and the stranger\lnQP{} who approaches will be put to death.}%
\verse{All those counted from the Levites whom Moses and Aaron mustered\lnQS{} according to the word of Adonai,\lebnote{“the mouth of Adonai”} according to their clans, every male from one month\lnQR{} and above were twenty-two thousand.}%
\verse{And Adonai said to Moses, “Muster every firstborn male from the Israelites\lnQO{} from one month\lnQR{} and above and count\lebnote{“and lift up the number of”} their names.}%
\verse{And you will receive the Levites for me — I am Adonai — in the place of all the firstborn among the Israelites,\lnQO{} and the animals\lnQW{} of the Levites in the place of all the firstborn among the animals\lnQW{} among the Israelites.”\lnQO{}}%
\verse{So Moses mustered\lebnote{Or “counted” or “summoned” or “enrolled”} all the firstborn among the Israelites\lnQO{} just as Adonai commanded him.}%
\verse{And all the firstborn males\lebnote{Hebrew “male”} among the number of names from one month\lnQR{} and above, the ones counted,\lnQT{} were twenty-two thousand two hundred and seventy-three.}%
\verse{Adonai spoke to Moses, saying,}%
\verse{“Receive the Levites in the place of all the firstborn among the Israelites,\lnQO{} and the animals of the Levites in the place of their animals; the Levites will be mine. I am Adonai.}%
\verse{And the ransom of the two hundred and seventy-three of the firstborn of the Israelites\lnQO{} who are excessive over the Levites,}%
\verse{you will receive five shekels a person, in the sanctuary shekel; you will collect twenty gerahs\lebnote{Hebrew “gerah”} per shekel.}%
\verse{You will give the money to Aaron, and to his sons the ransom of the ones who are excessive among them.”}%
\verse{And Moses received the money of the redemption from the ones who were excessive from those redeemed of the Levites.}%
\verse{From the firstborn of the Israelites\lnQO{} he took the money, one thousand three hundred and sixty-five shekels, in the sanctuary shekel.}%
\verse{And Moses gave the money of the ransom to Aaron and to his sons according to the word\lebnote{“mouth”} of Adonai just as Adonai commanded Moses.}%
\end{biblechapter}%
\begin{biblechapter}% Numbers 4
\verseWithHeading{The Census of the Kohathites}{Adonai spoke to Moses and Aaron, saying,}%
\verse{“Take a census\lebnote{“and lift up the number of”} of the descendants\lnQX{} of Kohath from the midst of the descendants\lnQX{} of Levi, according to their clans and their families,\lnQY{}}%
\verse{from thirty years old\lnQZ{} and above, up to fifty years old,\lnRA{} everyone who comes to the service to do the work in the tent of assembly.}%
\verse{This is the work of the descendants\lnQX{} of Kohath in the tent of assembly, concerning the holiness of the sanctuary:}%
\verse{When setting out the camp, Aaron and his sons will go and lower the curtain of the covering and cover with it the ark of the testimony.\lebnote{Or “the ark of the covenant”}}%
\verse{They will put on it a covering of fine leather,\lnRB{} and they will spread a cloth of perfect blue over it, and they will place its poles.}%
\verse{And over the table of the presence they will spread out a blue cloth and put on it the plates, dishes, and libation bowls, and the pitchers of the libation; and the bread of continuity will be on it.}%
\verse{They will spread over it a scarlet cloth, and they will cover it with a covering of fine leather,\lnRB{} and they will place its poles.}%
\verse{They will take a blue cloth and cover the lampstand for the light source, its lamps, a pair of its tongs, its small pans, and all the vessels of its oil with which they attend to it.}%
\verse{They will put it and all its vessels inside a covering of fine leather\lnRB{} and put it on the carrying frame.}%
\verse{Over the altar of gold they will spread a blue cloth, and they will cover it with a covering of fine leather\lnRB{} and place its poles.}%
\verse{They will take all the vessels of the cultic service with which they serve in the sanctuary and put them on a blue cloth, and they will cover them with a covering of fine leather;\lnRB{} and they will put them on the carrying frame.}%
\verse{They will remove the fat-soaked ashes from the altar and spread a purple cloth over it;}%
\verse{they will put on it all the vessels with which they serve, the fire pans, forks, shovels, and bowls — all the vessels of the altar. They will spread on it a covering of fine leather;\lnRB{} and they will place its poles.}%
\verse{And when Aaron and his sons have finished covering the sanctuary and all the vessels of the sanctuary when the camp sets out, the descendants\lnQX{} of Kohath will come after to carry these, but they must not touch the sanctuary, or they will die. These are the load of the descendants\lnQX{} of Kohath in the tent of assembly.}%
\verse{“Eleazar son of Aaron the priest is to supervise the oil of the light source, the incense, the regular grain offering,\lebnote{“the grain offering of continuity”} the oil of anointment, the supervision of all the tabernacle and all that is in it, in the sanctuary and in its vessels.”}%
\verse{Adonai spoke to Moses and Aaron, saying,}%
\verse{“You must not cut off the tribe of the clan of the Kohathites\lnRC{} from the midst of the Levites.}%
\verse{Do this to them and they will live and not die when they come near the most holy things. Aaron and his sons will go and appoint them, each one to his task and burden.}%
\verse{But they must not go and look for a moment\lebnote{“as devouring”} at the holy objects.”}%
\verse{Adonai spoke to Moses and Aaron, saying,}%
\verse{“Take a census\lebnote{“And lift up the number of”} of the descendants\lnQX{} of Gershon also, according to their families\lnQY{} and clans,}%
\verse{from those twenty years old\lebnote{“a son of twenty years”} and above until fifty years old;\lnRA{} you will muster\lnRD{} them, all who come to help to do the work of the tent of assembly.}%
\verse{This is the work of the clans of the Gershonites:\lnRE{} to serve and to carry.}%
\verse{They will carry the curtains of the tabernacle and the tent of assembly and its covering and the covering of fine leather,\lnRB{} which is on top of it,\lebnote{“is above upon it”} and the curtain of the doorway of the tent of assembly,}%
\verse{and the curtains of the courtyard, and the covering for the doorway of the gate of the courtyard, which is all around on the tabernacle and altar, and their cords and all the vessels of their work; and all that is done to them they will do.}%
\verse{And all the work of the descendants\lnQX{} of the Gershonites\lnRE{} will be at the command\lnRF{} of Aaron and his sons, for all they are to carry and for all their work, and you will appoint to them responsibility for all they are to carry.}%
\verse{This is the work of the clan of the descendants\lnQX{} of the Gershonites\lnRE{} in the tent of assembly, and their responsibility lies under the direction\lnRG{} of Ithamar son of Aaron the priest.}%
\verse{“For the descendants\lnQX{} of Merari according to their clans, according to their families,\lnQY{} you will muster\lnRD{} them;}%
\verse{from those thirty years old\lnQZ{} and above until fifty years old;\lnRA{} you will muster\lnRD{} them, all who come to do the work of the tent of assembly.}%
\verse{And this is the responsibility of those who are to carry,\lebnote{“them carrying”} all their work in the tent of assembly: the frames of the tabernacle and its bars, pillars, and bases,}%
\verse{and the pillars of the courtyard all around, and their bases, pegs, and cords, with all their vessels and for all their work. You will appoint by name the vessels that they are responsible to carry.}%
\verse{This is the work of the clan of the descendants\lnQX{} of Merari, for all their work in the tent of assembly under the direction\lnRG{} of Ithamar son of Aaron the priest.”}%
\verse{And Moses and Aaron mustered the leaders of the community according to the house of their families,\lnQY{}}%
\verse{from those thirty years old\lnQZ{} and above until fifty years old;\lnRA{} everyone who comes to the service to work in the tent of assembly,}%
\verse{the ones counted\lnRH{} were two thousand seven hundred and fifty.}%
\verse{These were those counted of the clans of the Kohathites,\lnRC{} everyone who served in the tent of assembly whom Moses and Aaron mustered\lnRI{} according to the command\lnRF{} of Adonai by the hand of Moses.\lnRJ{}}%
\verse{And the descendants\lnQX{} of Gershon counted according to their clans and according to their families,\lnQY{}}%
\verse{from those thirty years old\lnQZ{} and above until fifty years old,\lnRA{} everyone who comes to the service to work in the tent of assembly;}%
\verse{the ones counted,\lnRH{} according to their clans, according to their families,\lnQY{} were two thousand six hundred and thirty.}%
\verse{These were those counted of the clans of the descendants\lnQX{} of Gershon, everyone who serves in the tent of assembly whom Moses and Aaron mustered\lnRI{} according to the command\lnRF{} of Adonai.}%
\verse{Those counted of the clans of the descendants\lnQX{} of Merari according to their clans, according to their families,\lnQY{}}%
\verse{from those thirty years old\lebnote{“thirty years and above”} and above until fifty years old,\lnRA{} everyone who comes to the service to work in the tent of assembly,}%
\verse{the ones counted,\lnRH{} according to their clans, were three thousand two hundred.}%
\verse{These were those counted of the clans of the descendants\lnQX{} of Merari, whom Moses and Aaron mustered\lnRI{} according to the command\lnRF{} of Adonai by the hand of Moses.\lnRJ{}}%
\verse{All those counted of the Levites whom Moses and Aaron and all the leaders of Israel mustered\lnRI{} according to their clans, according to their families,\lnQY{}}%
\verse{from those thirty years old\lnQZ{} and above until fifty years old,\lnRA{} everyone who comes to the service to do the work of the service carrying in the tent of assembly,}%
\verse{the ones counted\lnRH{} were eighty thousand five hundred and eighty.}%
\verse{According to the command\lnRF{} of Adonai by the hand of Moses\lnRJ{} they were mustered,\lebnote{Hebrew “he mustered”} each man according to his service and according to their\lnRK{} service and according to their\lnRK{} burden; and so they were counted by him just as Adonai commanded Moses.}%
\end{biblechapter}%
\begin{biblechapter}% Numbers 5
\verseWithHeading{Rules Concerning Those Unclean}{Adonai spoke to Moses, saying,}%
\verse{“Command the Israelites:\lnRL{} they must send everyone from the camp who is afflicted with a rash,\lebnote{The precise meaning is uncertain; many modern translations suggest “leprosy”} everyone with a fluid discharge, and everyone unclean through contact with a corpse.}%
\verse{You will send away both male and female;\lebnote{“from male until female”} you will send them outside the camp.\lnRM{} They must not make unclean their camps where I am dwelling in their midst.”}%
\verse{So the Israelites\lnRL{} did so. They sent them away outside the camp;\lnRM{} just as Adonai spoke to Moses, so did the Israelites.\lnRL{}}%
\verseWithHeading{Rules of Restitution}{Adonai spoke to Moses, saying,}%
\verse{“Speak to the Israelites:\lnRL{} ‘When a man or woman commits\lebnote{“does”} any of the sins of humankind by acting unfaithfully, it is a sin against Adonai, and that person will be guilty;}%
\verse{they will confess their sin that they did and will make restitution for their\lebnote{Hebrew “his”} guilt by adding a fifth to it\lebnote{“on top of it”} and giving it to whomever was wronged.\lebnote{“to whomever he was guilty”}}%
\verse{But if the man does not have a redeemer to make restitution to him for the reparation, the reparation is to be given to Adonai for the priest, in addition to the ram of atonement by which atonement is made for him.}%
\verse{And every contribution of all the holy objects of the Israelites\lnRL{} that they bring to the priest for him will be his.}%
\verse{The holy objects of a man will be for him;\lebnote{That is, the priest} whatever he gives to the priest will be for him.’”}%
\verseWithHeading{Rules Concerning an Unfaithful Wife}{Adonai spoke to Moses, saying,}%
\verse{“Speak to the Israelites\lnRL{} and say to them, ‘If any man’s wife goes astray and acts unfaithfully to him,}%
\verse{and a man sleeps with her and ejaculates and it is hidden from the eyes of her husband and she is concealed, although she is defiled, and there is no witness against her and she was not caught,}%
\verse{if a spirit of jealousy comes over him, and he is jealous of his wife and she is defiled; or if a spirit of jealousy comes over him and he is jealous of his wife but she is not defiled,}%
\verse{he will bring his wife to the priest. And he will bring her offering for her, one-tenth of an ephah of flour. He will not pour oil on it, and he will not put frankincense on it because it is a grain offering of jealousy, a grain offering of remembering,\lebnote{Or “memorial”} a reminding of guilt.}%
\verse{“‘Then the priest will bring her near and present her before\lnRN{} Adonai;}%
\verse{the priest will take holy waters in a clay vessel, and from the dust that is on the floor of the tabernacle, and the priest will put it into the waters.}%
\verse{And the priest will present the woman before\lnRN{} Adonai, and he will uncover the head of the woman; he will then put in her hands the grain offering of the remembering — which is the grain offering of jealousy — and in the hand of the priest will be the waters of bitterness that brings a curse.}%
\verse{Then the priest will make her swear an oath, and he will say to the woman, “If a man has not slept with you, and if you have not had an impurity affair under your husband, go unpunished from the waters of bitterness that brings this curse.}%
\verse{But if you have had an affair under your husband, and if you are defiled and a man other than your husband had intercourse with you,”}%
\verse{the priest will make the woman swear an oath of the sworn oath of the curse, the priest will say to the woman, “May Adonai give you a curse and a sworn oath in the midst of your people with Adonai making\lebnote{Hebrew “giving”} your hip fall away\lnRO{} and your stomach swollen;}%
\verse{and these waters that bring a curse will go into your intestines to cause your womb to swell and to make your hip fall away.”\lnRO{} And the women will say, “Amen. Amen.”}%
\verse{“‘And the priests will write these curses on the scroll, and he will wipe them off into the waters of the bitterness.}%
\verse{He will make the woman drink the waters of the bitterness that brings\lebnote{Hebrew “bring”} a curse, and the waters of bitterness that bring a curse will go into her.}%
\verse{The priest will take the grain offering of jealousy from the hand of the woman, and he will wave the grain offering before Adonai,\lebnote{“before the face of Adonai”} and he will present it to the altar;}%
\verse{the priest will grasp her memorial offering from the grain offering, and he will turn it into smoke on the altar, and afterward he will make the woman drink the waters.}%
\verse{When he has made her drink the waters, it will come about, if she has defiled herself and acted unfaithfully to her husband and the waters of bitterness that bring a curse go into her and her stomach swells and her hip falls away,\lebnote{Or “wastes away”} the woman will be as a curse in the midst of her people.}%
\verse{And if the woman is not defiled, and she is pure, she will go unpunished and be able to conceive children.}%
\verse{“‘This is the regulation of jealousy, when a woman has an affair under her husband and she is defiled,}%
\verse{or when a spirit of jealousy comes over a man and he is jealous of his wife, he will present the woman before Adonai,\lnRN{}and the priest will do to her all of this law.}%
\verse{The man will go unpunished from guilt, and the woman, she will bear her guilt.’”}%
\end{biblechapter}%
\begin{biblechapter}% Numbers 6
\verseWithHeading{Rules Concerning Nazirites}{Adonai spoke to Moses, saying,}%
\verse{“Speak to the Israelites\lnRP{} and say to them, ‘When a man or a woman takes a special vow, a vow of a Nazirite,\lebnote{“one separated”} to keep separate for Adonai,}%
\verse{he will abstain from wine and fermented drink; he will not drink wine vinegar or vinegar of fermented drink; he will not drink the fruit juice of grapes or eat fresh or dry grapes.}%
\verse{All the days of his separation\lnRQ{} you will not eat from anything that is made from the grapevine, from sour grapes to the skin of grapes.}%
\verse{“‘All the days of the vow of his separation a razor will not pass over his head. Until fulfilling the days that he separated himself to Adonai he will be holy and grow long the locks of the hair of his head.}%
\verse{“‘All the days of keeping himself separated for Adonai he will not go to a person who is dead;}%
\verse{for even his father, mother, brother, or sister he will not make himself unclean by their death,\lebnote{“with their dying”} because the separation to his God is on his head.}%
\verse{He will be holy for Adonai all the days of his separation.\lnRQ{}}%
\verse{“‘If someone dies suddenly and makes the head of his separation\lnRQ{} unclean, he will shave off the hair of his head on the day of his cleansing; on the seventh day he will shave it off.}%
\verse{On the eighth day he will bring two turtledoves or two young pigeons\lebnote{“the sons of doves”} to the priest by the doorway of the tent of assembly,}%
\verse{and the priest will offer one for a sin offering and one for a burnt offering, and he will make atonement for him because he sinned concerning the corpse. He will consecrate his head on that day.}%
\verse{He will rededicate to Adonai the days of his separation\lnRQ{} and bring a ram-lamb in its first year\lnRR{} as a guilt offering. The former days of his vow will fall away because his separation was defiled.}%
\verse{“‘This is the regulation of the Nazirite for the day of the fulfilling of the days of his separation: one will bring him to the doorway of the tent of the assembly.}%
\verse{He will present his offering to Adonai, one ram-lamb in its first year\lnRR{} without defect as a burnt offering, and one ewe-lamb in its first year\lebnote{“a daughter of his year”} without defect as a sin offering, and one ram without defect as a fellowship offering;}%
\verse{and a basket of unleavened bread, finely milled flour of ring-shaped bread cakes mixed with oil, and wafers of unleavened bread smeared with oil, and their grain offering and their libations.}%
\verse{The priest will present before Adonai\lebnote{Or “before the face of Adonai”} and offer his sin offering, his burnt offering;}%
\verse{he will offer a ram as a sacrifice of a fellowship offering to Adonai, in addition to the basket of the unleavened bread; the priest will offer his grain offering and his libation.}%
\verse{The Nazirite will shave off the hair of his consecrated head\lnRS{} at the doorway of the tent of assembly, and he will take the hair of his consecrated head,\lnRS{} and he will put it on the fire that is beneath the sacrifice of the fellowship offering.}%
\verse{The priest will take the shoulder from the ram that is boiled, and one ring-shaped bread cake of unleavened bread from the basket, and one wafer of unleavened bread, and he will put them on the hands of the Nazirite after he has shaved his consecrated head.\lebnote{“his separation”}}%
\verse{The priest will wave them as a wave offering before the presence of\lebnote{“before the face of”} Adonai; they\lebnote{Hebrew “it”} are a holy object to the priest, in addition to the breast section of the wave offering, and in addition to the upper thigh of the contribution; and afterward the Nazirite may drink wine.}%
\verse{“‘This is the regulation of the Nazirite who has made a vow of his offering to Adonai according to his separation, in addition to what he can afford.\lebnote{“except from that which overtakes his hand”} In accordance to the word of his vow that he vowed, he will do, concerning the instruction of his separation.’”}%
\verseWithHeading{The Priestly Blessing}{Adonai spoke to Moses, saying,}%
\verse{“Speak to Aaron and his sons, saying, ‘You will bless the Israelites:\lnRP{} You will say to them:}%
\verse{Adonai will bless you and keep you;}%
\verse{Adonai will make shine his face on you and be gracious to you;}%
\verse{Adonai will lift up his face upon you, and he will give you peace.’}%
\verse{And they will put my name on the Israelites,\lnRP{} and I will bless them.”}%
\end{biblechapter}%
\begin{biblechapter}% Numbers 7
\verseWithHeading{The Leaders Make Offerings}{On the day Moses finished setting up the tabernacle and appointed and consecrated it and all its vessels, and the altar and its vessels, and he appointed them and consecrated them,}%
\verse{the leaders of Israel, the heads of the families,\lebnote{“the house of their fathers”} presented an offering; they were the leaders of the tribes and were the ones in charge of the counting.}%
\verse{They brought their offering before the presence of\lebnote{Or “before the face of”} Adonai, six covered utility carts and twelve cattle, a utility cart for two of the leaders, and a bull for each; and they presented them before\lebnote{“before the face of”} the tabernacle.}%
\verse{Adonai said to Moses, saying,}%
\verse{“Take them,\lebnote{“Take from them”} and they will be used to do the work of the tent of the assembly; and you will give them to the Levites, each according to his required service.”\lebnote{“according to the mouth of his work”}}%
\verse{So Moses took the utility carts and the cattle, and he gave them to the Levites.}%
\verse{Two utility carts and four cattle he gave to the descendants\lnRT{} of Gershon according to their required service;\lnRU{}}%
\verse{and four utility carts and eight cattle he gave to the descendants\lnRT{} of Merari according to their required service,\lnRU{} under the authority of\lebnote{“in the hand of”} Ithamar son of Aaron the priest.}%
\verse{But to the descendants\lnRT{} of Kohath he did not give anything because the work of the sanctuary they carried upon them on their shoulders.\lebnote{Hebrew “the shoulder”}}%
\verse{The leaders presented offerings for the dedication of the altar on the day of its anointing, and the leaders presented their offerings before\lebnote{“in the presence of”} the altar.}%
\verse{Adonai said to Moses, “One leader for each day\lebnote{“One leader for the day one leader for the day”} will present their offering for the dedication of the altar.”}%
\verse{And it happened, the one who presented his offering on the first day was Nahshon son of Amminadab from the tribe of Judah.}%
\verse{His offering was one plate of silver — its weight was one hundred and thirty shekels — and one silver bowl weighing seventy shekels according to the sanctuary shekel, both of them\lnRV{} filled with finely milled flour mixed with oil as a grain offering;}%
\verse{one golden dish weighing ten shekels filled with incense;}%
\verse{one young bull,\lebnote{“one bull, a son of a cattle”} one ram, one male lamb in its first year\lnRW{} as a burnt offering;}%
\verse{one he-goat as a sin offering;}%
\verse{and as a sacrifice of the fellowship offering, two cattle, five rams, five he-goats, and five male lambs a year old.\lnRX{} This was the offering of Nahshon son of Amminadab.}%
\verse{On the second day Nethanel son of Zuar, leader of Issachar, presented an offering.}%
\verse{He presented as his offering one silver plate — its weight one hundred and thirty shekels — and one silver bowl for drinking wine weighing seventy shekels\lnRY{} according to the sanctuary shekel, both of them\lnRV{} filled with finely milled flour mixed with oil as a grain offering.}%
\verse{One dish weighing ten shekels filled with incense;}%
\verse{one young \lnRZ{} bull, one ram, a male lamb in its first year\lnRW{} as a burnt offering;}%
\verse{one he-goat as a sin offering;}%
\verse{and for the sacrifice of the fellowship offering, two cattle, five rams, five he-goats, and five male lambs in their first year.\lnRX{} This was the offering of Nethanel son of Zuar.}%
\verse{On the third day Eliab son of Helon, leader of the descendants\lnRT{} of Zebulun:}%
\verse{his offering was one silver plate — its weight one hundred and thirty shekels — and one silver bowl for drinking wine weighing seventy shekels\lnRY{} according to the sanctuary shekel, both of them\lnRV{} filled with finely milled flour mixed with oil as a grain offering;}%
\verse{one golden dish weighing ten shekels\lnSA{} filled with incense;}%
\verse{one young\lnRZ{} bull, one ram, a male lamb in its first year\lnRW{} as a burnt offering;}%
\verse{one he-goat as a sin offering;}%
\verse{and for the sacrifice of the fellowship offering, two cattle, five rams, five he-goats, and five male lambs in their first year.\lnRX{} This was the offering of Eliab son of Helon.}%
\verse{On the fourth day Elizur son of Shedeur, leader of the descendants\lnRT{} of Reuben:}%
\verse{his offering was one silver plate — its weight one hundred and thirty shekels — and one silver bowl for drinking weighing seventy shekels\lnRY{} according to the sanctuary shekel, both of them\lnRV{} filled with finely milled flour mixed with oil as a grain offering;}%
\verse{one golden dish weighing ten shekels\lnSA{} filled with incense;}%
\verse{one young\lnRZ{} bull, one ram, a male lamb in its first year\lnRW{} as a burnt offering;}%
\verse{one he-goat as a sin offering;}%
\verse{and for the sacrifice of the fellowship offering, two cattle, five rams, five he-goats, and five male lambs in their first year.\lnRX{} This was the offering of Elizur son of Shedeur.}%
\verse{On the fifth day Shelumiel son of Zurishaddai, leader of the descendants\lnRT{} of Simeon:}%
\verse{his offering was one silver plate — its weight one hundred and thirty shekels — and one silver bowl for drinking weighing seventy shekels\lnRY{} according to the sanctuary shekel, both of them\lnRV{} filled with finely milled flour mixed with oil as a grain offering;}%
\verse{one golden dish weighing ten shekels\lnSA{} filled with incense;}%
\verse{one young\lnRZ{} bull, one ram, a male lamb in its first year\lnRW{} as a burnt offering;}%
\verse{one he-goat as a sin offering;}%
\verse{and for the sacrifice of the fellowship offering, two cattle, five rams, five he-goats, and five male lambs in their first year.\lnRX{} This was the offering of Shelumiel son of Zurishaddai.}%
\verse{On the sixth day Eliasaph son of Deuel, leader of the descendants\lnRT{} of Gad:}%
\verse{his offering was one silver plate — its weight one hundred and thirty shekels — and one silver bowl for drinking weighing seventy shekels\lnRY{} according to the sanctuary shekel, both of them\lnRV{} filled with finely milled flour mixed with oil as a grain offering;}%
\verse{one golden dish weighing ten shekels\lnSA{} filled with incense;}%
\verse{one young\lnRZ{} bull, one ram, a male lamb in its first year\lnRW{} as a burnt offering;}%
\verse{one he-goat as a sin offering;}%
\verse{and for the sacrifice of the fellowship offering, two cattle, five rams, five he-goats, and five male lambs in their first year.\lnRX{} This was the offering of Eliasaph son of Deuel.}%
\verse{On the seventh day Elishama son of Ammihud, leader of the descendants\lnRT{} of Ephraim:}%
\verse{his offering was one silver plate — its weight one hundred and thirty shekels — and one silver bowl for drinking weighing seventy shekels\lnRY{} according to the sanctuary shekel, both of them\lnRV{} filled with finely milled flour mixed with oil as a grain offering;}%
\verse{one golden dish weighing ten shekels\lnSA{} filled with incense;}%
\verse{one young\lnRZ{} bull, one ram, a male lamb in its first year\lnRW{} as a burnt offering;}%
\verse{one he-goat as a sin offering;}%
\verse{and for the sacrifice of the fellowship offering, two cattle, five rams, five he-goats, and five male lambs in their first year.\lnRX{} This was the offering of Elishama son of Ammihud.}%
\verse{On the eighth day Gamaliel son of Pedahzur, leader of the descendants\lnRT{} of Manasseh:}%
\verse{his offering was one silver plate — its weight one hundred and thirty shekels — and one silver bowl for drinking weighing seventy shekels according to the sanctuary shekel, both of them\lnRV{} filled with finely milled flour mixed with oil as a grain offering;}%
\verse{one golden dish weighing ten shekels\lnSA{} filled with incense;}%
\verse{one young\lnRZ{} bull, one ram, a male lamb in its first year\lnRW{} as a burnt offering;}%
\verse{one he-goat as a sin offering;}%
\verse{and for the sacrifice of the fellowship offering, two cattle, five rams, five he-goats, and five male lambs in their first year.\lnRX{} This was the offering of Gamaliel son of Pedahzur.}%
\verse{On the ninth day Abidan son of Gideoni, leader of the descendants\lnRT{} of Benjamin:}%
\verse{his offering was one silver plate — its weight one hundred and thirty shekels — and one silver bowl for drinking weighing seventy shekels\lnRY{} according to the sanctuary shekel, both of them\lnRV{} filled with finely milled flour mixed with oil as a grain offering;}%
\verse{one golden dish weighing ten shekels\lnSA{} filled with incense;}%
\verse{one young\lnRZ{} bull, one ram, a male lamb in its first year\lnRW{} as a burnt offering;}%
\verse{one he-goat as a sin offering;}%
\verse{and for the sacrifice of the fellowship offering, two cattle, five rams, five he-goats, and five male lambs in their first year.\lnRX{} This was the offering of Abidan son of Gideoni.}%
\verse{On the tenth day Ahiezer son of Ammishaddai, leader of the descendants\lnRT{} of Dan:}%
\verse{his offering was one silver plate — its weight one hundred and thirty shekels — and one silver bowl for drinking weighing seventy shekels\lnRY{} according to the sanctuary shekel, both of them\lnRV{} filled with finely milled flour mixed with oil as a grain offering;}%
\verse{one golden dish weighing ten shekels\lnSA{} filled with incense;}%
\verse{one young\lnRZ{} bull, one ram, a male lamb in its first year\lnRW{} as a burnt offering;}%
\verse{one he-goat as a sin offering;}%
\verse{and for the sacrifice of the fellowship offering, two cattle, five rams, five he-goats, and five male lambs in their first year.\lnRX{} This was the offering of Ahiezer son of Ammishaddai.}%
\verse{On the eleventh day Pagiel son of Ocran, leader of the descendants\lnRT{} of Asher:}%
\verse{his offering was one silver plate — its weight one hundred and thirty shekels — and one silver bowl for drinking weighing seventy shekels\lnRY{} according to the sanctuary shekel, both of them\lnRV{} filled with finely milled flour mixed with oil as a grain offering;}%
\verse{one golden dish weighing ten shekels\lnSA{} filled with incense;}%
\verse{one young\lnRZ{} bull, one ram, a male lamb in its first year\lnRW{} as a burnt offering;}%
\verse{one he-goat as a sin offering;}%
\verse{and for the sacrifice of the fellowship offering, two cattle, five rams, five he-goats, and five male lambs in their first year.\lnRX{} This was the offering of Pagiel son of Ocran.}%
\verse{On the twelfth day Ahira son of Enan, leader of the descendants\lnRT{} of Naphtali:}%
\verse{his offering was one silver plate — its weight one hundred and thirty shekels — and one silver bowl for drinking weighing seventy shekels\lnRY{} according to the sanctuary shekel, both of them\lnRV{} filled with finely milled flour mixed with oil as a grain offering;}%
\verse{one golden dish weighing ten shekels\lnSA{} filled with incense;}%
\verse{one young\lnRZ{} bull, one ram, a male lamb in its first year\lnRW{} as a burnt offering;}%
\verse{one he-goat as a sin offering;}%
\verse{and for the sacrifice of the fellowship offering, two cattle, five rams, five he-goats, and five male lambs in their first year.\lnRX{} This was the offering of Ahira son of Enan.}%
\verse{This was the dedication of the altar on the day of anointing it, from the leaders of Israel: twelve silver plates, twelve silver bowls for drinking wine, twelve golden dishes;}%
\verse{each plate of silver weighed one hundred and thirty shekels, and each bowl for drinking seventy, all the silver of the vessels two thousand four hundred shekels, according to the sanctuary shekel;}%
\verse{the twelve golden dishes filled with incense, each dish weighing ten shekels according to the sanctuary shekel, all the gold of the dishes one hundred and twenty;}%
\verse{all the cattle for the burnt offering twelve bulls, twelve rams, twelve male lambs in their first year,\lnRX{} and their grain offering; and twelve he-goats as a sin offering;}%
\verse{and all the cattle of the sacrifice of the fellowship offering twenty-four bulls, sixty rams, sixty he-goats, sixty male lambs in their first year.\lnRX{} These were the dedication of the altar after its anointing.}%
\verse{And when Moses came to the tent of assembly to speak with him,\lebnote{That is, Adonai} he would hear the voice speaking to him from the atonement cover,\lebnote{Some modern translations have “mercy seat” (see, for example, the NRSV, NASB)} which is on the ark of the testimony, from between the two cherubim, and he would speak to him.}%
\end{biblechapter}%
\begin{biblechapter}% Numbers 8
\verseWithHeading{The Seven Lamps}{Adonai spoke to Moses, saying,}%
\verse{“Speak to Aaron, and say to him: ‘When you are setting up the lamps, the seven lamps will give light in front of the face of the lampstand.’”}%
\verse{And Aaron did so; he set up the lampstand in front of the face of its lamps, just as Adonai commanded Moses.}%
\verse{And this is how the lampstand was made,\lebnote{“the work of the lampstand”} a hammered-work of gold; from its base up to its blossom,\lebnote{Or “flower”} it was hammered-work according to the pattern that Adonai showed Moses; so he made the lampstand.}%
\verseWithHeading{Moses Consecrates the Levites}{Adonai spoke to Moses, saying,}%
\verse{“Take the Levites from the midst of the Israelites\lnSB{} and purify them.}%
\verse{So you will do to them, to purify them: sprinkle on them waters of purification, and they will shave their whole body\lebnote{“they will send a razor on all their body”} and wash their garments.}%
\verse{And they will take a young bull\lnSC{} and its grain offering of finely milled flour mixed with oil, and you will take a second young bull\lnSC{} as a sin offering.}%
\verse{You will bring the Levites before\lnSD{} the tent of assembly, and you will summon the entire community of the Israelites.\lnSB{}}%
\verse{And you will bring the Levites before Adonai,\lnSE{} and the Israelites\lnSB{} will lay their hands on the Levites,}%
\verse{and Aaron will offer\lnSF{} the Levites as a wave offering before Adonai\lnSE{} from the Israelites,\lnSB{} and they will do the work of Adonai.}%
\verse{And the Levites will lay their hands on the head of the one bull and offer it as a sin offering and the other one as a burnt offering to Adonai, to make atonement for the Levites.}%
\verse{And you will present the Levites before\lnSD{} Aaron and before\lnSD{} his sons, and he will offer\lnSF{} them as a wave offering to Adonai.}%
\verse{“And you will separate the Levites from the midst of the Israelites,\lnSB{} and the Levites will be for me.}%
\verse{And after this the Levites will come to serve at the tent of assembly, and you will purify them, and you will offer\lnSF{} them as a wave offering.}%
\verse{For they are given to me exclusively from the midst of the Israelites.\lnSB{} I have taken them for myself in place of the firstborn of every womb, every firstborn from the Israelites.\lnSB{}}%
\verse{For every firstborn among the Israelites\lnSB{} is mine, both humankind and animal. On the day I destroyed every firstborn in the land of Egypt I consecrated them to me,}%
\verse{and I have taken the Levites in the place of every firstborn among the Israelites.\lnSB{}}%
\verse{And I have given the Levites; they are given to Aaron and his sons from the midst of the Israelites\lnSB{} to do the work of the Israelites\lnSB{} in the tent of the assembly and to make atonement for the Israelites,\lnSB{} so a plague will not be among the Israelites\lnSB{} when the Israelites\lnSB{} come near the sanctuary.”}%
\verse{And Moses and Aaron and the entire community of the Israelites\lnSB{} did to the Levites; everything that Adonai commanded Moses concerning the Levites, the Israelites\lnSB{} did to them.}%
\verse{And the Levites purified themselves, and they washed their garments, and Aaron offered them\lebnote{“Aaron waved them”} as a wave offering before Adonai;\lnSE{} and Aaron made atonement for them to purify them.}%
\verse{After this the Levites came to do their work in the tent of assembly before Aaron and his sons. Just as Adonai commanded Moses concerning the Levities, so they did to them.}%
\verse{Adonai spoke to Moses, saying,}%
\verse{“This is what is for the Levites: those twenty-five years old\lebnote{“from a son of twenty-five years”} and above will\lebnote{Hebrew “and he will”} come to help with the service in the work of the tent of assembly;}%
\verse{and those fifty years old\lebnote{“from a son of fifty years”} will\lebnote{Hebrew “he will”} return from the service of the work and will serve no longer.}%
\verse{They\lnSG{} can attend\lebnote{Or “assist”} their brothers in the tent of assembly to keep their responsibilities, but they\lnSG{} will not do work. This is what you will do concerning the Levities and their responsibilities.”}%
\end{biblechapter}%
\begin{biblechapter}% Numbers 9
\verseWithHeading{The Passover}{Adonai spoke to Moses in the desert of Sinai, in the second year after they came out from the land of Egypt, in the first month, saying,}%
\verse{“Let the Israelites\lnSH{} observe the Passover at its appointed time.}%
\verse{On the fourteenth day of this month at twilight\lnSI{} you will perform it at its appointed time according to all its decrees; and according to all its stipulations you will observe it.”}%
\verse{So Moses spoke to the Israelites\lnSH{} to observe the Passover.}%
\verse{And they observed the Passover on the fourteenth day of the month at twilight\lnSI{} in the desert of Sinai. According to all that Adonai commanded Moses, thus the Israelites\lnSH{} did.}%
\verse{And it happened, men who were unclean by a dead person\lnSJ{} were not able to perform the Passover on that day. And they came before\lebnote{“in the presence of”} Moses and Aaron on that day.}%
\verse{And those men said to him, “Although we are unclean by a dead person,\lnSJ{} why are we hindered from presenting the offering of Adonai at its appointed time in the midst of the Israelites?”\lnSH{}}%
\verse{Moses said to them, “Stay. I will hear what Adonai commands to you.”}%
\verse{And Adonai spoke to Moses, saying,}%
\verse{“Speak to the Israelites,\lnSH{} saying, ‘Each man that is unclean by a dead person\lnSJ{} or is on a far journey, you or your descendants,\lebnote{“generations”} he will observe the Passover of Adonai.}%
\verse{On the second month on the fourteenth day at twilight\lnSI{} they will observe it; they will eat it with unleavened bread and bitter plants.}%
\verse{They will leave none of it until morning, and they will not break a bone in it; they will observe it according to every decree of the Passover.}%
\verse{But the man who is clean and not on a journey, and he fails to observe the Passover, that person will be cut off from the people because he did not present the offering of Adonai on its appointed time. That man will bear his guilt.}%
\verse{If an alien dwells with you he will observe the Passover of Adonai according to the decree of the Passover and according to its stipulation; thus you will have one decree for you, for the alien and for the native of the land.’”}%
\verseWithHeading{The Cloud and the Fire}{And on a day setting up the tabernacle, the cloud covered the tent of the tabernacle, the tent of the testimony; in the evening it was on the tabernacle as an appearance of fire until morning.}%
\verse{So it was\lebnote{Hebrew “it will be”} continually; the cloud would cover it and the appearance of fire by night.}%
\verse{Whenever the cloud lifted up from on the tent, after that the Israelites\lnSH{} would set out, and in the place where the cloud dwelled, there the Israelites\lnSH{} camped.}%
\verse{On the command of Adonai\lnSK{} the Israelites\lnSH{} would set out, and on the command of Adonai\lnSK{} they encamped; all the days that the cloud dwelled on the tabernacle they encamped.}%
\verse{And when the cloud prolonged on the tabernacle many days the Israelites\lnSH{} kept the kept requirement of Adonai and did not set out.}%
\verse{When the cloud remained\lnSL{} a number of days on the tabernacle, on the command of Adonai\lnSK{} they encamped; and on the command of Adonai\lnSK{} they set out.}%
\verse{When the cloud remained\lnSL{} from evening until morning, and the cloud lifted up in the morning, they would set out, or if it remained in the daytime and at night, when the cloud lifted up they set out.}%
\verse{When it was two days, a month, or a year\lebnote{“or days”} that the cloud prolonged to dwell on the tabernacle, the Israelites\lnSH{} encamped, and they did not set out; when it lifted up they set out.}%
\verse{On the command of Adonai\lnSK{} they encamped, and on the command of Adonai\lnSK{} they set out. They kept the requirement of Adonai, on the command of Adonai\lnSK{} in the hand of Moses.\lebnote{Or “through Moses”}}%
\end{biblechapter}%
\begin{biblechapter}% Numbers 10
\verseWithHeading{The Silver Trumpets}{Adonai spoke to Moses, saying,}%
\verse{“Make yourself two silver trumpets; make them of hammered-work. You will use them\lebnote{“They will be for you”} for calling the community and for breaking the camp.}%
\verse{You will blow them, and all the community will assemble to the doorway of the tent of assembly.}%
\verse{But if they blow only one, the leaders, the heads of the thousands of Israel, will assemble to you.}%
\verse{When you will blow a blast, the camps that are camping on the east will set out;}%
\verse{when you blow a second blast, the camps that are camping on the south will set out; they will blow a blast for their journeys.}%
\verse{But when summoning the assembly, you will blow, but you will not signal with a loud noise.}%
\verse{The sons of Aaron, the priests, will blow on the trumpets; this will be an eternal decree for your generations.}%
\verse{If you go to war in your land against the enemy who attacks you, you will signal with a loud noise on the trumpets. You will be remembered before\lnSM{} Adonai your God, and you will be rescued from your enemies.}%
\verse{“And on the day of your joy and in your appointed times, at the beginning of your months, you will blow on the trumpets in addition to your burnt offerings and in addition to the sacrifices of your fellowship offerings. And they will be as a memorial for you before\lnSM{} your God; I am Adonai your God.”}%
\verseWithHeading{The Israelites Depart from Sinai}{And it happened, in the second year, in the second month, on the twentieth of the month the cloud was lifted from upon the tabernacle of the testimony.\lebnote{Some modern translations (e.g., the NRSV) have “tabernacle of the covenant”}}%
\verse{And the Israelites\lnSN{} set out for their journey\lebnote{Hebrew “journeys”} from the desert of Sinai, and the cloud dwelled in the desert of Paran.}%
\verse{They set out for the first time\lebnote{“in the beginning”} on the command of Adonai in the hand of Moses.\lebnote{Or “through Moses”}}%
\verse{The standard of the camp of the descendants\lnSO{} of Judah set out for the first time according to their divisions, with Nahshon son of Amminadab over its division.}%
\verse{And Nathanel son of Zuar was over the division of the descendants\lnSO{} of Issachar;}%
\verse{Eliab son of Helon was over the division of the tribe of the descendants\lnSO{} of Zebulun.}%
\verse{The tabernacle was taken down, and the sons of Gershon and the sons of Merari, the bearers of the tabernacle, set out.}%
\verse{And the standard of the camp of Reuben according to their divisions; Elizur son of Shedeur was over their division.}%
\verse{Shelumiel son of Zurishaddai was over the division of the sons of the tribe of Simeon.}%
\verse{Eliasaph son of Deuel was over the division of the tribe of the descendants\lnSO{} of Gad.}%
\verse{The Kohathites, the bearers of the sanctuary, set out, and they set up the tabernacle before they arrived.}%
\verse{And the stand of the camp of the descendants\lnSO{} of Ephraim set out according to their divisions; Elishama son of Ammihud was over its division.}%
\verse{Gamaliel son of Pedahzur was over the division of the tribe of the descendants\lnSO{} of Manasseh.}%
\verse{Abidan son of Gideoni was over the division of the tribe of the descendants\lnSO{} of Benjamin.}%
\verse{Then the standard of the camp of the descendants\lnSO{} of Dan, who formed a rear guard for all the camps, set out according to their divisions; Ahiezer son of Ammishaddai was over its division.}%
\verse{Pagiel son of Ocran was over the division of the tribe of the descendants\lnSO{} of Asher.}%
\verse{Ahira son of Enan was over the division of the tribe of the descendants\lnSO{} of Naphtali.}%
\verse{These were the departures of the Israelites\lnSN{} according to their divisions; and so they set out.}%
\verse{Moses said to Hobab son of Reuel the Midianite, the father-in-law of Moses, “We are setting out to the place that Adonai said, ‘I will give it to you’; go with us, and we will treat you well\lebnote{“do good to you”} because Adonai promised\lebnote{“Adonai spoke”} good concerning Israel.”}%
\verse{But he said to him, “I will not go. I will only go to my land and to my family.”}%
\verse{He\lebnote{That is, Moses} said, “Please, do not abandon us because you know our encampment in the desert, and you should be our guide.\lebnote{“you should be our eyes”}}%
\verse{Moreover, if you go with us, the good that Adonai will do to us we will do to you.”}%
\verse{And so they set out from the mountain of Adonai a journey of three days, with the ark of the covenant of Adonai setting out ahead of them\lebnote{Or “before them”} three days’ journey to search out a resting place for them;}%
\verse{and the cloud of Adonai was over them by day when they set out from the camp.}%
\verse{And whenever the ark was setting out Moses would say, “Rise up, Adonai! May your enemies be scattered; may the ones that hate you flee from your presence.”}%
\verse{And when it rested he would say, “Return, Adonai, to the countless thousands of Israel.”}%
\end{biblechapter}%
\begin{biblechapter}% Numbers 11
\verseWithHeading{The Israelites Complain}{And it happened, the people were like those who complain of hardship\lebnote{“complain of bad”} in the hearing\lebnote{“in the ears”} of Adonai, and Adonai became angry,\lebnote{“his nose became hot”} and the fire of Adonai burned among them, and it consumed the edge of the camp.}%
\verse{Then the people cried out to Moses, and Moses prayed to Adonai, and the fire died down.}%
\verse{And he called the name of that place Taberah\lebnote{This word is difficult, but some modern translations suggest the word in Hebrew means “burning” (see NRSV, NASB)} because the fire of Adonai burned among them.}%
\verse{The riff-raff that were in their midst had a strong desire;\lebnote{“desired a desire”} and the Israelites\lebnote{“sons/children of Israel”} turned back and also wept, and they said, “Who will feed us meat?}%
\verse{We remember the fish that we ate in Egypt for nothing, the cucumber, melon, leek, the onions, and the garlic.}%
\verse{But now our strength is dried up;\lebnote{“our life \textit{is} dry”} there is nothing whatsoever except for the manna before us.”\lebnote{“for the manna of our eyes”}}%
\verse{Now the manna was like coriander seed, and its outward appearance was like that of bdellium-gum.}%
\verse{The people went about and gathered it, and they ground it with mills or crushed it with mortar. Then they boiled it in a pot and made it into bread-cakes; and it tasted like olive oil cakes.}%
\verse{When the dew came down on the camp at night, the manna came down with it.}%
\verse{Moses heard the people weeping according to their\lebnote{Hebrew “its”} clans, each at the doorway of their tents. Then Adonai became very angry,\lebnote{“the nose of Adonai became very hot”} and in the eyes of Moses it was bad.}%
\verse{And Moses said to Adonai, “Why have you brought trouble to your servant? Why have I not found favor in your eyes, that the burdens of all these people have been placed on me?}%
\verse{Did I conceive all these people? If I have fathered them,\lnSP{} that you could say to me, ‘Carry them\lnSP{} in your lap, just as a foster-father carries the suckling on the land that you swore an oath to their ancestors?’\lebnote{Or “fathers”}}%
\verse{From where do I have meat to give all these people? They weep before me, saying, ‘Give us meat and let us eat!’}%
\verse{I am not able to carry all these people along alone; they are too heavy for me.}%
\verse{If this is how you are going to treat me, please kill me immediately if I find favor in your eyes, and do not let me see my misery.”}%
\verse{And Adonai said to Moses, “Gather for me seventy men from the elders of Israel whom you know are elders of the people and their\lebnote{Hebrew “his/its”} officials; take them to the tent of assembly, and they will stand there with you.}%
\verse{I will come down and speak with you there; I will take away from the spirit that is on you, and I will place it on them; and they will bear the burdens of the people with you; you will not bear it alone.}%
\verse{And you will say to the people, ‘Sanctify yourselves tomorrow, for you will eat meat because you have wept in the ears of Adonai, saying, “Who will feed us good meat? It was good for us in Egypt.” Adonai will give to you meat, and you will eat.}%
\verse{You will eat, not one day, or two days, or five days, or ten days, or twenty days,}%
\verse{but for a whole month,\lebnote{“until a period of one month”} until it comes out from your nose and becomes as nausea to you; because you have rejected Adonai, who is in your midst, and you wept before his presence,\lebnote{“before his face”} saying, “Why did we ever leave Egypt?”’”}%
\verse{But Moses said, “There are six hundred thousand on foot, among whom I am in the midst, and you yourself said, ‘I will give meat to them, and they will eat for a whole month.’}%
\verse{Should flocks and cattle be slaughtered for them? Should all the fish of the sea be gathered together for them, to be enough for them?”}%
\verse{And Adonai said to Moses, “Is Adonai’s power limited?\lebnote{“Is Adonai’s hand short?”} Now you will see if my word will happen or not.”}%
\verse{So Moses went out, and he spoke the words of Adonai to the people, and he gathered together seventy men from the elders of the people, and he made them stand\lebnote{“caused them to stand”} all around the tent.}%
\verse{Then Adonai went down in the cloud and spoke to him, and he took away the spirit that was on him, and he put it\lebnote{Or “gave it”} on the seventy elders. And as soon as the spirit was resting on them they prophesied, but they did not do it again.}%
\verse{But two men were left in the camp; the name of one was Eldad, and the name of the second was Medad, and the spirit rested on them; they were among those who were written down, but they did not go out to the tent, so they prophesied in the camp.}%
\verse{So a boy\lebnote{Hebrew “the boy”} ran and told Moses and said, “Eldad and Medad are prophesying in the camp.”}%
\verse{And Joshua son of Nun, the assistant of Moses from time of his youth, answered, “Moses, my lord, stop them.”}%
\verse{But Moses said to him, “Are you jealous for my sake? Would that he\lebnote{That is, Adonai} give all Adonai’s people prophets, that Adonai put his spirit on them!”}%
\verse{Then Moses and the elders of Israel were gathered to the camp.}%
\verseWithHeading{The Quail}{Then a wind set out from Adonai, and it drove quails from the west, and he spread them out on the camp about a day’s journey on one side and about a day’s journey on the other, all around the camp, about two cubits on the surface of the land.}%
\verse{And so the people worked\lebnote{“arose”} all day and all night and all the next day, and they gathered the quail (the least of the ones collecting gathered ten homers).\lebnote{HALOT 330, “a dry measure”}}%
\verse{While the meat was still between their teeth, before it was consumed, Adonai was angry with the people, and Adonai struck a very great plague among the people.}%
\verse{And he called the name of that place Kibroth Hattaavah\lnSQ{} because they buried the people that were greedy.\lebnote{“craved”}}%
\verse{From Kibroth Hattaavah\lnSQ{} the people set out to Hazeroth; and they stayed\lebnote{Hebrew “they were”} in Hazeroth.}%
\end{biblechapter}%
\begin{biblechapter}% Numbers 12
\verseWithHeading{Aaron and Miriam Murmur Against Moses}{And Miriam and Aaron spoke against Moses because of the Cushite woman whom he took (because he took a Cushite wife);}%
\verse{and they said, “Has Adonai spoken only through Moses? Has not Adonai also spoken through us?” And Adonai heard it.}%
\verse{Now the man, Moses, was more humble than any other person on the face of the earth,}%
\verse{and Adonai said suddenly to Moses, Aaron, and Miriam, “Go out, you three, to the tent of assembly.” So the three of them when out.}%
\verse{And Adonai went down in a column of cloud and stood at the doorway of the tent, and he called Aaron and Miriam, and the two of them went,}%
\verse{and he said, “Please hear my words: If there is a prophet among you, I, Adonai, will make myself known to him in a vision. I will speak to him in a dream.\lebnote{Hebrew “the dream”}}%
\verse{Not so with my servant Moses; in all my house he is faithful.}%
\verse{I will speak to him mouth to mouth, in clearness, not in riddles; and he will look at the form of Adonai. Why were you not afraid to speak against my servant, against Moses?”}%
\verse{And Adonai became very angry\lebnote{“And the nose of Adonai became hot”} with them, and he went away.}%
\verse{And the cloud departed from on the tent, and behold, Miriam was infected with a skin disease\lebnote{The precise meaning is uncertain; many modern translations suggest “leprosy”} white like snow; when Aaron turned toward Miriam, behold, she was afflicted with a skin disease.}%
\verse{So Aaron said to Moses, “Please, my lord, please do not put on us this sin in which we were foolish and in which we have sinned.}%
\verse{Please do not let her be like the dead, whose flesh is half consumed when coming out from the womb of its mother.”}%
\verse{And Moses cried to Adonai, saying, “God, please heal her!”\lebnote{“Please heal please her”}}%
\verse{But Adonai said to Moses, “If her father had surely spit in her face, would she not bear her shame for seven days? Let her be confined for seven days to an outside place of the camp, and afterward she may be gathered.”}%
\verse{So Miriam was confined to the outside place of the camp seven days, and the people did not set out until Miriam was gathered.}%
\verse{And afterward the people set out from Hazeroth, and they encamped in the desert of Paran.}%
\end{biblechapter}%
\begin{biblechapter}% Numbers 13
\verseWithHeading{Spies Sent to Spy Out the Land of Canaan}{And Adonai spoke to Moses, saying,}%
\verse{“Send for yourself men, and let them explore the land of Canaan, which I am about to give to the Israelites;\lnSR{} from each tribe of his father send one man,\lebnote{“one man one man from the tribe of his father”} everyone a leader among them.”}%
\verse{So Moses sent them from the desert of Paran on the command of Adonai; all of the men were leaders\lebnote{“heads”} of the Israelites.\lnSR{}}%
\verse{And these are their names: from the tribe of Reuben, Shammua son of Zaccur;}%
\verse{from the tribe of Simeon, Shaphat son of Hori;}%
\verse{from the tribe of Judah, Caleb son of Jephunneh;}%
\verse{from the tribe of Issachar, Igal son of Joseph;}%
\verse{from the tribe of Ephraim, Hoshea son of Nun;}%
\verse{from the tribe of Benjamin, Palti son of Raphu;}%
\verse{from the tribe of Zebulun, Gaddiel son of Sodi;}%
\verse{from the tribe of Joseph, from the tribe of Manasseh, Gaddi son of Susi;}%
\verse{from the tribe of Dan, Ammiel son of Gemalli;}%
\verse{from the tribe of Asher, Sethur son of Michael;}%
\verse{from the tribe of Naphtali, Nahbi son of Vophsi;}%
\verse{from the tribe of Gad, Geuel son of Maki.}%
\verse{These are the names of the men whom Moses sent to explore the land. And Moses called Hoshea son of Nun Joshua.}%
\verse{Moses sent them to explore the land of Canaan, and he said to them, “Go up like this to the Negev,\lnSS{} and go up into the hill country,}%
\verse{and you will see what the land is like and if the people who inhabit it are strong or weak, or whether they are few or many,}%
\verse{and whether the land that they are inhabiting is good or bad, and whether the cities they are inhabiting are camps or fortifications,}%
\verse{and whether the land is fertile or lean, and whether there are trees on it or not. You will show yourself courageous, and you will take some of the fruit of the land.” It was the time of first ripe grapes.}%
\verse{So they went up and explored the land from the desert of Zin until Rehob, at Lebo Hamath.\lebnote{Or “near Hamath”}}%
\verse{They went up through the Negev\lnSS{} and came to Hebron, where\lebnote{Hebrew “and there”} Ahiman, Sheshai, and Talmai the descendants of the Anakites were. (Hebron was built seven years before Zoan in Egypt.)}%
\verse{And they came up to the valley\lnST{} of Eshcol, and they cut off a vine branch and one cluster of grapes from there; they carried it on a pole between two men, with pomegranates and figs.}%
\verse{That place he called the valley\lnST{} of Eshcol on account of the cluster of grapes that the Israelites\lnSR{} cut off from there.}%
\verseWithHeading{The Spies Return}{They returned from exploring the land at the end of forty days.\lebnote{Hebrew “day”}}%
\verse{And they came\lebnote{Hebrew “they went and came”} to Moses and Aaron and to the entire community of the Israelites\lnSR{} in the desert of Paran at Kadesh; they brought back word to them and to all the community, and they showed them the fruit of the land.}%
\verse{And they told him,\lebnote{Hebrew “they told him and said”} “We came to the land that you sent us, and it is flowing of milk and honey; this is its fruit.}%
\verse{Yet the people who are inhabiting it are strong and the cities are fortified and very large; moreover, we saw the descendants of the Anakites there.}%
\verse{The Amalekites are living in the land of the Negev;\lnSS{} the Hittites, Jebusites, and the Amorites are living in the hill country; and the Canaanites are living at the sea and on the banks of the Jordan.”}%
\verse{And Caleb silenced the people before Moses and said, “Surely, let us go up and let us take possession of it because surely we will be able to prevail over it.”}%
\verse{And the men who went up with him said, “We are not able to go up to the people because they are stronger than us.”}%
\verse{And they presented the report of the land that they explored to the Israelites,\lnSR{} saying, “The land that we went through to explore is a land that eats its inhabitants, and all the people whom we saw in its midst are men of great size.\lebnote{“men of measurements”}}%
\verse{There we saw the Nephilim (the descendants\lebnote{Or “sons”} of Anak came from the Nephilim), and we were like grasshoppers in our own sight, and so we were in their eyes.”}%
\end{biblechapter}%
\begin{biblechapter}% Numbers 14
\verseWithHeading{The People Complain}{Then all the community lifted up their voices,\lebnote{Hebrew “they lifted up and gave their voice”} and the people wept during that night.}%
\verse{And all the Israelites\lnSU{} grumbled against Moses and Aaron, and all the community said to them, “If only we had died in the land of Egypt or in this desert!}%
\verse{Why did Adonai bring us into this land to fall by the sword? Our wives and our little children will become plunder; would it not be better for us to return to Egypt?”}%
\verse{They said to each other,\lebnote{“A man said to his brother”} “Let us appoint a leader, and we will return to Egypt.”}%
\verse{Then Moses and Aaron fell on their faces before\lebnote{“in the presence of”} the assembly of the community of the Israelites.\lnSU{}}%
\verse{Joshua son of Nun and Caleb son of Jephunneh, from the explorers of the land, tore their garments.}%
\verse{And they said to all the community of the Israelites,\lnSU{} “The land that we went through to explore is an exceptionally good land.\lebnote{“the land is very very good”}}%
\verse{If Adonai delights in us, then he will bring us into this land, and he will give it to us, a land that is flowing with milk and honey.}%
\verse{Only do not rebel against Adonai, and you will not fear the people of the land, because they will be our food. Their protection\lebnote{“Their shadow”} has been turned from them; Adonai is with us. You should not fear them.”}%
\verse{And all the community said to stone them with stones, but the glory of Adonai appeared in the tent of assembly among the Israelites.\lnSU{}}%
\verse{And Adonai said to Moses, “How long until this people will despise me, and how long until they will not believe in me, and in all the signs that I have done in their\lnSV{} midst?}%
\verse{I will strike them\lnSW{} with disease, and I will dispossess them;\lnSW{} I will make you into a greater and stronger nation than them.”\lnSW{}}%
\verse{And Moses said to Adonai, “Then the Egyptians will hear that you brought up this people from their\lnSV{} midst in your power,}%
\verse{and they will tell it\lebnote{“say”} to the inhabitants of this land. They heard that you, Adonai, are in the midst of this people, that you are seen eye to eye, and your cloud is standing over them, and in a column of cloud you go before them by day and in a column of fire at night.}%
\verse{But if you destroy this people all at once,\lebnote{“as one man”} the nations that will have heard your message will say,}%
\verse{‘Adonai was unable to bring this people in the land that he swore by an oath, and he slaughtered them in the desert.’}%
\verse{But now, please, let the power of my Lord be great, just has you spoke,}%
\verse{‘Adonai is slow to anger\lebnote{“slow of noses”} and great of loyal love, forgiving\lebnote{“lifting up”} sin and rebellion; but surely he leaves nothing unpunished, visiting the sin of the fathers on the sons to the third and fourth generations.’}%
\verse{Please forgive the sin of this people according to the greatness of your loyal love, just as you forgave\lebnote{“lifted up”} this people, from Egypt until now.”}%
\verse{Adonai said, “I have forgiven them according to your word;}%
\verse{but as I am alive, the glory of Adonai will fill all the earth.}%
\verse{But because all the men who have seen my glory and my signs that I did in Egypt and in the desert yet tested me these ten times and did not listen to my voice,}%
\verse{they will not see the land that I swore by oath to their ancestors,\lebnote{Or “fathers”} and all those who despised me will not see it.}%
\verse{But my servant Caleb, because another spirit was with him, he remained true after me, and I will bring him into the land that he entered,\lebnote{Or “he went to”} and his offspring will take possession of it.}%
\verse{And the Amalekites and the Canaanites live in the valleys; tomorrow turn and set out for the desert by way of the Red Sea.”\lebnote{“sea of reed”}}%
\verse{And Adonai spoke to Moses and Aaron, saying,}%
\verse{“How long will I bear this evil community who are grumbling against me? I have heard the grumbling of the Israelites\lnSU{} which they are making\lebnote{“they are grumbling”} against me.}%
\verse{Say to them, ‘Surely as I live,’ declares\lebnote{“declaration of”} Adonai, ‘just as you spoke in my hearing,\lebnote{“in my ears”} so I will do to you;}%
\verse{in this desert your corpses will fall, and all your counted ones, according to all your number, from twenty years old\lebnote{“a son of twenty years”} and above who grumbled against me.}%
\verse{You yourselves will not come into the land that I swore by oath\lebnote{“I lifted up my hand”} to make you to dwell in it, but Caleb son of Jephunneh and Joshua son of Nun.}%
\verse{But your little children, whom you said would be plunder, I will bring them, and they will know the land that you rejected.}%
\verse{But for you, all your corpses will fall in this desert.}%
\verse{And your children will be shepherds in the desert forty years,\lnSX{} and you will bear your unfaithfulness until all your corpses have fallen\lebnote{“until to complete your corpses”} in the desert.}%
\verse{According to the number of the days\lnSY{} that you explored the land, forty days,\lnSY{} a day for each year,\lebnote{“a day for a year a day for a year”} you will bear your sins forty years,\lnSX{} and you will know my opposition.’}%
\verse{I, Adonai, have spoken; I will surely do this to all this evil community who has banded together against me. In this desert they will come to an end, and there they will die.”}%
\verse{As for the men whom Moses sent to explore the land, who returned and made the community grumble against him by spreading a report over the land,}%
\verse{the men who spread the evil report of the land died by the plague before Adonai.\lebnote{“in the presence of Adonai”}}%
\verse{But Joshua son of Nun and Caleb son of Jephunneh lived from among the men who went to explore the land.}%
\verse{And Moses spoke words to all the Israelites,\lnSU{} and the people mourned greatly.}%
\verse{They rose early in the morning and went to the top of the mount, saying, “Here we are. We will go up to the place that Adonai said, because we have sinned.”}%
\verse{But Moses said, “Why are you going against the command of Adonai? It will not succeed.}%
\verse{You should not go up because Adonai is not in your midst; do not let yourselves be defeated in the presence of your enemies,}%
\verse{because the Amalekites\lnSZ{} and the Canaanites\lnTA{} are there before you,\lebnote{“in your presence”} and you will fall by the sword; because you have turned back from Adonai, and Adonai will not be with you.”}%
\verse{But they dared to go to the top of the mountain, and the ark of the covenant of Adonai and Moses did not depart from the midst of the camp.}%
\verse{So the Amalekites\lnSZ{} and the Canaanites\lnTA{} who were living on the mountain descended, and they beat them down, up to Hormah.}%
\end{biblechapter}%
\begin{biblechapter}% Numbers 15
\verseWithHeading{Various Sacrifices and Offerings}{Adonai spoke to Moses, saying,}%
\verse{“Speak to the Israelites\lnTB{} and say to them, ‘When you come into the land of your dwellings that I am about to give to you,}%
\verse{you will make an offering by fire for Adonai from the cattle or from the flock, a burnt offering or a sacrifice to fulfill a vow, or as a freewill offering or at your feasts, to make a fragrance of appeasement for Adonai.}%
\verse{And the one who presents an offering\lebnote{Hebrew “his offering”} for Adonai, he will present a grain offering of finely milled flour; a tenth will be mixed with a fourth of the liquid measure of oil;}%
\verse{and you will add a fourth of wine for the libation upon the burnt offering, or to the sacrifice for each ram-lamb.}%
\verse{Or for the ram you will make a grain offering of two-tenths of finely milled flour mixed into a third of a liquid measure of oil.}%
\verse{You will present a third of the liquid measure of wine for the libation, a fragrance of appeasement for Adonai.}%
\verse{When you prepare a bull\lebnote{“a son of cattle”} as a burnt offering or a sacrifice to fulfill a vow or a fellowship offering for Adonai,}%
\verse{you will present with the bull\lebnote{“the son of the cattle”} a grain offering of three-tenths of finely milled flour mixed with half a liquid measure of oil,}%
\verse{and you will present half a liquid measure of wine as a libation, as an offering made by fire, a fragrance of appeasement for Adonai.}%
\verse{“‘This is how it should be done for each bull, or for the each ram, or for the small four-footed mammal, or ram-lambs, or goats.}%
\verse{According to the number that you prepare, so should you do to each according to their number.}%
\verse{Every native must do these things to present an offering made by fire, a fragrance of appeasement for Adonai.}%
\verse{If an alien dwells among you, or whoever is in your midst throughout your generations,\lnTC{} and prepares an offering made by fire, a fragrance of appeasement for Adonai, he should do as you do.}%
\verse{For the assembly, there will be one decree for you and for the alien who dwells among you; it is an eternal decree for all your generations. You as well as the alien\lebnote{“like you like the alien”} will be before Adonai.\lnTD{}}%
\verse{There will be one law and one stipulation for you and for the alien dwelling among you.’”}%
\verse{Adonai spoke to Moses, saying,}%
\verse{“Speak to the Israelites\lnTB{} and say to them, ‘When you come into the land to which I am about to bring you,}%
\verse{whenever you eat from the food of the land, you will lift up a contribution to Adonai.}%
\verse{You must lift up a contribution of the first batch of your ring-shaped dough bread; you must lift it up as a contribution of the threshing floor.}%
\verse{You will give to Adonai a contribution from the first of your dough throughout your generations.\lnTC{}}%
\verse{“‘But if you go astray and you do not follow\lebnote{Or “do”} all these commandments that Adonai commanded to Moses,}%
\verse{all that Adonai commanded you by the hand of Moses\lebnote{Or “through Moses”} from the day that Adonai commanded and beyond, throughout your generations,\lnTC{}}%
\verse{and if it was done unintentionally without the knowledge\lebnote{“from the eyes”} of the community, then the entire community must prepare one young bull\lebnote{“a bull a son of cattle”} as a burnt offering, as a fragrance of appeasement for Adonai, and its grain offering and its libation, according to the stipulation, and one male goat as a sin offering.}%
\verse{The priest will make atonement for all of the community of the Israelites,\lnTB{} and they will be forgiven\lebnote{“it will be forgiven to them”} because it was unintentional; they will bring their offering, an offering made by fire for Adonai, their sin offering before Adonai\lnTD{} for their unintentional sin.}%
\verse{All of the community of the Israelites\lnTB{} will be forgiven, as well as the alien that dwells in their midst, because the whole community was involved in the unintentional wrong.}%
\verse{“‘If one person sins unintentionally, that person will present a female goat in its first year\lebnote{“a daughter of a year”} as a sin offering.}%
\verse{And the priest will make atonement for the person who sinned unintentionally\lebnote{“sinned unintentionally when sinning an unintentional wrong”} before Adonai,\lnTD{} to make atonement for him, and he will be forgiven.}%
\verse{For the native among the Israelites\lnTB{} and the alien that dwells in their midst, there will be one law for anyone who commits an unintentional wrong.}%
\verse{But the one who acts presumptuously\lebnote{“who acts with a high hand”} from among the native or alien blasphemes against Adonai, and that person must be cut off from the midst of the people.}%
\verse{Because he despised the word of Adonai and broke his command, that person will be surely cut off and bear the guilt.’”}%
\verseWithHeading{Violation of the Sabbath}{When the Israelites\lnTB{} were in the desert, they found a man who was gathering wood on the day of the Sabbath.}%
\verse{The ones who found him gathering wood brought him to Moses, Aaron, and to all the community.}%
\verse{And they put him under watch because it was not made clear what should be done to him.}%
\verse{And Adonai said to Moses, “Surely the man must be put to death by stoning him; all the community must stone him with stones from outside the camp.”}%
\verse{So the entire community brought him out to a place outside the camp, and they stoned him to death\lebnote{“they stoned him with stones and he died”} just as Adonai commanded Moses.}%
\verseWithHeading{Garment Fringes}{Adonai spoke to Moses, saying,}%
\verse{“Speak to the Israelites,\lnTB{} and tell them to make for themselves tassels\lebnote{Hebrew “tassel”} on the hems of their garments throughout their generations\lebnote{Hebrew “for their generations”} and to put a blue cord on the tassel of the hem.}%
\verse{You will have a tassel for you to look at\lebnote{“and you will look at it”} and remember all the commands of Adonai and do them, and not follow after the unfaithfulness of your own heart and eyes,\lebnote{“after your heart and after your eyes, which you \textit{are} unfaithful after them”}}%
\verse{so that you will remember and do all my commandments, and you will be holy for your God.}%
\verse{I am Adonai your God, who brought you out of the land of Egypt, to be your God; I am Adonai your God.”}%
\end{biblechapter}%
\begin{biblechapter}% Numbers 16
\verseWithHeading{Korah, Dathan, and Abiram Rebel}{Now Korah son of Izhar, son of Kohath, son of Levi, and Dathan and Abiram sons of Eliab, and On son of Peleth, the descendants\lnTE{} of Reuben,}%
\verse{took two hundred and fifty men from the Israelites,\lnTF{} leaders of the community summoned from the assembly, renowned men,\lebnote{“men of name”} and they confronted\lebnote{“they rose up before”} Moses.}%
\verse{They were assembled in front of Moses and Aaron, and they said to them, “You take too much upon yourselves!\lebnote{“It is much for you”} All of the community is holy, every one of them, and Adonai is in their midst, so why do you raise yourselves over the assembly of Adonai?”}%
\verse{When Moses heard this, he fell on his face.}%
\verse{And he said to Korah and to his entire company, saying, “Tomorrow morning Adonai will make known who is his and who is holy, and he will bring him near to him, whomever he chooses he will bring near to him.}%
\verse{Do this: take for yourselves censers, Korah and all of your company;\lebnote{Hebrew “his company”}}%
\verse{tomorrow put fire in them and place incense on them before\lnTG{} Adonai; the man whom Adonai chooses will be the holy one. You take too much upon yourselves, sons of Levi!”}%
\verse{And Moses said to Korah, “Please listen, sons of Levi!}%
\verse{Is it too little for you that the God of Israel set you apart from the community of Israel to allow you to approach him\lebnote{“to bring you near to him”} to do the work of the tabernacle of Adonai, to stand before\lnTG{} the community to serve them?}%
\verse{He has allowed you to approach him,\lebnote{“He has brought you near”} you with all your brothers, the descendants\lnTE{} of Levi, but yet you also seek the priesthood.}%
\verse{Therefore you and your company that has banded together against Adonai. What is Aaron that you grumble against him?”}%
\verse{Moses sent to call for Dathan and Abiram son of Eliab, but they said, “We will not come!\lnTH{}}%
\verse{Is it too little that you have brought us from a land that flows with milk and honey to kill us in the desert, and that you also appoint yourself as a ruler over us?}%
\verse{Surely, you have not brought us to a land that flows with milk and honey, and you have not given us the inheritance of fields and a vineyard. Will you gouge out the eyes of these men? We will not come!”\lnTH{}}%
\verse{Then Moses became angry, and he said to Adonai, “Do not notice their grain offering. I have not offered one donkey from them, and I have not mistreated one of them.”}%
\verse{And Moses said to Korah, “You and your entire company will be before\lnTG{} Adonai tomorrow, you and they and Aaron.}%
\verse{Each one take his censer, and put incense on it\lebnote{Hebrew “on them”} and you will present it before\lnTG{} Adonai, and each of you bring his censer, two hundred and fifty censers, you and Aaron, each his censer.”}%
\verse{So each of them took his censer, and they put fire on them, and they placed incense on them; they stood at the doorway of the tent of the assembly of Moses and Aaron.}%
\verse{And Korah summoned them, the entire community, by the doorway of the tent of assembly, and the glory of Adonai appeared to all the community.}%
\verse{And Adonai spoke to Moses and Aaron, saying,}%
\verse{“Separate yourselves from the midst of this community, that I can destroy them in a moment.”}%
\verse{And they fell on their faces, and they said, “God, God of the spirits of all flesh, will one man sin and you become angry toward the entire community?”}%
\verse{Adonai spoke to Moses, saying,}%
\verse{“Speak to the community, saying, ‘Move away from the dwelling of Korah, Dathan, and Abiram.’”}%
\verse{So Moses stood up and went to Dathan and Abiram; the elders of Israel followed after him.}%
\verse{He said to the community, saying, “Please turn away from the tents of these wicked men, and do not touch anything that belongs to them,\lebnote{“that \textit{is} to them”} or you will be destroyed with all their sins.”}%
\verse{And so they moved away from around the dwellings\lebnote{Hebrew “dwelling”} of Korah, Dathan, and Abiram; and Dathan and Abiram came out standing at the doorway of their tents, with their wives, sons, and little children.}%
\verse{And Moses said, “In this you will know that Adonai has sent me to do all these works; it is not from my heart.}%
\verse{If they die a natural death\lebnote{“If they die like the death of every human”} or if a natural fate is visited upon them,\lebnote{“If the fate of every human is visited upon them”} Adonai has not sent me.}%
\verse{But if Adonai creates something new, and the ground opens its mouth and swallows them up and all that belongs to them,\lebnote{“All that \textit{is} for them”} and they go down alive to Sheol, and you will know that these men have despised Adonai.”}%
\verse{And it happened, as soon as he finished speaking\lebnote{“to speak”} all these words, the ground that was under them split open.}%
\verse{The land opened its mouth and swallowed them up with their houses and every person that belonged to Korah\lebnote{“that was to Korah”} and all the property.}%
\verse{They went down alive to Sheol, they and all that belonged to them, and the land covered over them, and they perished from the midst of the assembly.}%
\verse{All Israel who were around them fled at their cry, because they said, “Lest the land swallow us up!”}%
\verse{And fire went out from Adonai, and it consumed the two hundred and fifty men presenting the incense.}%
\verse{\lebnote{Numbers 16:36–17:13 in the English Bible is 17:1–28 in the Hebrew Bible} And Adonai spoke to Moses, saying,}%
\verse{“Say to Eleazar son of Aaron the priest, ‘Take out\lebnote{“raise up”} the censers from among the place of burning because they are sacred, and scatter the fire outside.}%
\verse{The censers of these who have sinned\lebnote{“sinners”} at the cost of their lives, let them be made into gilded leafing plating for the altar; because they presented them before Adonai,\lebnote{“in the presence of Adonai”} they are holy; and they will be a sign for the Israelites.’”\lnTF{}}%
\verse{Eleazar the priest took the bronze censers that the ones who were burned presented, and they hammered them out thinly as plating for the altar;}%
\verse{it was a memorial for the Israelites,\lnTF{} so that no strange man\lebnote{NASB translates “no layman”} who is not from the offspring\lebnote{“the seed”} of Aaron should approach the presence of Adonai to burn a smoke offering;\lebnote{That is, an incense offering} he will not be like Korah and his company, just as Adonai had spoken to him by the hand of Moses.\lebnote{Or “through Moses”}}%
\verseWithHeading{The Israelites Grumble}{The next day all the community of the Israelites\lnTF{} grumbled against Moses and Aaron, saying, “You have killed the people of Adonai!”}%
\verse{Then, when the community had gathered against Moses and Aaron, they turned to the tent of assembly, and behold, the cloud covered it, and the glory of Adonai appeared.}%
\verse{And Moses and Aaron came to the front of the tent of assembly,}%
\verse{and Adonai spoke to Moses, saying,}%
\verse{“Get away from the midst of this community, and I will finish them in an instant,”\lebnote{Or “in a moment”} but they fell on their faces.}%
\verse{And Moses and Aaron said, “Take the censer, and put fire on it from the altar. Place incense on it, and bring it quickly to the community, and make atonement for them, because wrath went out from the presence of Adonai, and a plague has begun.”}%
\verse{And so Aaron took it just as Moses had spoken, and he ran into the midst of the assembly, for behold, the plague had begun among the people; so he gave the incense and made atonement for the people.}%
\verse{He stood between the dead and between the living, and the plague was stopped.}%
\verse{Those who died by the plague were fourteen thousand seven hundred, besides those who died on account of Korah.}%
\verse{Then Aaron returned to Moses at the doorway of the tent of assembly, and the plague was stopped.}%
\end{biblechapter}%
\begin{biblechapter}% Numbers 17
\verseWithHeading{Aaron’s Staff Is Chosen}{Adonai spoke to Moses, saying,}%
\verse{“Speak to the Israelites,\lnTI{} and take from among them twelve staffs, a staff from each family\lebnote{“a staff from a house of a father”} from among all their leaders according to their families’ households. Write the name of each man on his staff,}%
\verse{and the name of Aaron on the staff of Levi, because one staff is for the head of each of their families.\lebnote{“the house of their fathers”}}%
\verse{You must then put them in the tent of assembly before\lebnote{“in the presence of”} the testimony\lnTJ{} where I meet with you.}%
\verse{And it will happen, the man whom I will choose, his staff will blossom, and so I will rid from upon myself the grumblings of the Israelites,\lnTI{} who are grumbling against you.”}%
\verse{Moses spoke to the Israelites,\lnTI{} and all their leaders gave him a staff for each leader, one from each of their families,\lebnote{“each from the house of their fathers”} twelve staffs, and the staff of Aaron was in the midst of their tribes.}%
\verse{And Moses put the staffs before Adonai in the tent of testimony.}%
\verse{Then the next day, Moses went into the tent of the testimony, and behold the staff of Aaron for the house of Levi blossomed and put forth a flower and produced blossoms, and it produced almonds.}%
\verse{Then Moses brought out to all the Israelites\lnTI{} all the staffs before the presence of Adonai, and they saw, and each man took his staff.}%
\verse{And Adonai said to Moses, “Bring back the staff of Aaron before the testimony\lnTJ{} as a guard and sign for the children of rebellion, and let them finish their grumblings before me and not die.”}%
\verse{So Moses did; just as Adonai commanded him, so he did.}%
\verse{And the Israelites\lnTI{} said to Moses, saying, “Look! We will die! We will be destroyed! All of us will perish!}%
\verse{Anyone who approaches the tabernacle of Adonai will die. Will we all die?”\lebnote{“Will we all die to perish?”}}%
\end{biblechapter}%
\begin{biblechapter}% Numbers 18
\verseWithHeading{The Duties of the Priests and Levities}{Adonai said to Aaron, “You, your sons, and your family with you will bear the guilt of the sanctuary, and you and your sons with you will bear the guilt of your priesthood.}%
\verse{Moreover, bring your brothers with you, the tribe of Levi the tribe of your father, that they may be joined to you and minister to you, you and your sons with you before the tent of testimony.}%
\verse{They will keep your responsibility and the responsibility of all the tent, only they may not come near the vessels of the sanctuary and the altar, so both you and they will not die.}%
\verse{They will be joined to you, and they will keep the responsibility of the tent of assembly for the entire service of the tent; a stranger may not come near you.}%
\verse{You will keep the responsibility of the sanctuary and the responsibility of the altar, and there will no longer be wrath on the Israelites.\lnTK{}}%
\verse{Look, I myself have chosen your brothers the Levites from the midst of the children. They are a gift to you given from Adonai to perform the work of the tent of assembly.}%
\verse{But you with your sons will keep your priesthood to perform your priestly duties for everything at the altar\lebnote{Or “for all the things of the altar”} and for the area behind the curtain.\lebnote{“the house of the curtain”} I give you the priesthood as a gift, but the stranger who approaches will be put to death.”}%
\verseWithHeading{Portions for the Priests}{Adonai spoke to Aaron, “Behold, I myself have given to you the responsibility of my contributions for all the holy objects of the Israelites;\lnTK{} I have given them as a portion to you and your sons as an eternal decree.}%
\verse{This will be for you from the sanctuary of the holy things from the fire; all of their offerings, from every grain offering, from every sin offering, and from every guilt offering which they will bring to me is a most holy thing\lebnote{“a holy object of holiness”} for you and your sons.}%
\verse{You will eat it in the most holy place;\lebnote{Alternatively “as a holy object of holiness”} every male will eat it. It will be a holy object to you.}%
\verse{This is also for you: the contribution of their gift of the wave offerings of the children Israel. I have given them to you and your sons and your daughters with you as an eternal decree; whoever is clean in your house may eat it.}%
\verse{All the finest olive oil and all the finest new wine and their best grain that they have given to Adonai, I have given them to you.}%
\verse{The firstfruits of all that is in their land that they present to Adonai will be for you; whoever is clean in your house may eat it.}%
\verse{All consecrated possessions\lebnote{Hebrew “possession”} in Israel will be for you.}%
\verse{All the first offspring of a womb of any creature that they offer to Adonai, whether human or animal, will be yours; you will surely redeem the firstborn of the human and the unclean firstborn of the animal.}%
\verse{As to their price of redemption, from a one-month-old\lebnote{“a son of a month”} you will redeem them according to your proper value, five shekels of silver according to the shekel of the sanctuary, which is twenty gerah.}%
\verse{Only the firstborn of an ox or the firstborn of a sheep or the firstborn of a goat you will not redeem; they are holy. Their blood you will sprinkle over the altar, and their fat you will turn into smoke as an offering made by fire, a fragrance of appeasement for Adonai.}%
\verse{But their flesh will be for you like the breast section of the wave offering, and it will be for you like the right upper thigh.}%
\verse{All the contributions of holiness that the Israelites\lnTK{} offer to Adonai I have given to you and your sons and your daughters with you as an eternal decree; it is an eternal covenant of salt before\lebnote{“in the presence of”} Adonai to you and your offspring with you.”}%
\verse{Then Adonai said to Aaron, “You will not receive an inheritance in their land, and there will not be a plot of ground for you in the midst of the Israelites.\lnTK{}}%
\verse{“Behold, I have given to the descendants\lebnote{Or “sons”} of Levi every tithe in Israel as an inheritance in return for their service, which they are doing, the work of the tent of assembly.}%
\verse{The Israelites\lnTK{} will not come near again to the tent of assembly, or they will bear sin\lebnote{The NRSV translates “they will incur guilt”} and die.}%
\verse{The Levites\lebnote{Hebrew “Levite”} will perform the service of the tent of assembly, and they will bear their guilt, an eternal decree for all your generations. But they will not receive an inheritance in the midst of the Israelites\lnTK{}}%
\verse{because the tithes\lebnote{Hebrew “tithe”} of the Israelites\lnTK{} that are offered\lebnote{Hebrew “raised up”} to Adonai as a contribution, I have given to the Levites as an inheritance; therefore I said to them, ‘They will not receive an inheritance in the midst of the Israelites.’”\lnTK{}}%
\verse{Adonai spoke to Moses, saying,}%
\verse{“You will speak to the Levites and say to them, ‘When you receive the tithe from the Israelites\lnTK{} that I have given to you from them for your inheritance, you will present\lnTL{} a contribution from it to Adonai, a tithe from a tithe.}%
\verse{Your contribution will be credited to you like the grain from the threshing floor and like the produce from the press.}%
\verse{So you will present\lnTL{} your own contribution to Adonai from all your tithes that you receive from the Israelites;\lnTK{} from it you will give the contribution of Adonai to Aaron the priest.}%
\verse{From all your gifts you will present\lnTL{} every contribution of Adonai, from all its fat, the part that is sacred.’}%
\verse{You will say to them, ‘When you are presenting\lebnote{“you are raising up”} its fat, the rest will be credited to the Levites like a yield of the threshing floor and like a yield from the press.}%
\verse{You may eat it anywhere, you and your household, because it is a wage in return for your service in the tent of assembly.}%
\verse{You will not bear any sin because you have presented\lebnote{“you have raised up”} its fat; you will not defile the holy objects of the Israelites,\lnTK{} or you will die.’”}%
\end{biblechapter}%
\begin{biblechapter}% Numbers 19
\verseWithHeading{Ashes of the Red Heifer}{And Adonai spoke to Moses and Aaron, saying,}%
\verse{“This is the decree of the law that Adonai has commanded, saying, ‘Speak to the Israelites\lnTM{} and let them take to you a red heifer without a physical defect, on which a yoke has not been placed.\lebnote{“has not gone up”}}%
\verse{And you will give it to Eleazar the priest, and it will be brought\lebnote{Or “he will bring it out”} out to a place outside the camp, and it will be slaughtered\lebnote{Or “he will slaughter it”} in his presence.}%
\verse{Then Eleazar the priest will take some of its blood on his finger and spatter it toward the mouth of the tent of assembly seven times.}%
\verse{The heifer will be burned\lebnote{Hebrew “The heifer will burn”} in his sight; its skin, its meat, and its blood, in addition to its offal, will burn.}%
\verse{The priest will take cedar wood, hyssop, and crimson thread, and he will throw them in the midst of the burning heifer.\lebnote{“the burning of the heifer”}}%
\verse{The priest will wash his garments and his body in the water, and afterward he will come to the camp; the priest will be unclean until the evening.}%
\verse{The one who burns it will wash his garments and his body in water; he will be unclean until the evening.}%
\verse{A clean man will gather the ashes of the heifer, and he will put them in a clean place outside the camp;\lebnote{“an outside place of the camp”} it will be for the community of the Israelites\lnTM{} as a requirement for waters of impurity; it is a purification offering.}%
\verse{The one who gathers the ashes of the heifer will wash his garments; he will be unclean until evening. It will be an eternal decree for the Israelites\lnTM{} and for one who dwells as an alien in their midst.}%
\verse{“‘The one who touches a corpse of any person\lebnote{“any human person”} will be unclean for seven days.}%
\verse{He will purify himself on the third day, and on the seventh day he will be clean. If he does not purify himself on the third day, he will not be clean on the seventh day.}%
\verse{Anyone who touches a corpse, the person of a human being who died, and does not purify himself, defiles the tabernacle of Adonai, and that person will be cut off from Israel because the waters of impurity were not sprinkled on him. He will still be unclean, and uncleanness is on him.}%
\verse{“‘This is the law of a person who dies in a tent: everyone who comes into the tent and all who are in the tent will be unclean seven days.}%
\verse{Every container that is opened that does not have a lid cord\lebnote{That is, “that does not have a lid tied shut”} on it is unclean.}%
\verse{Anyone in the open field\lebnote{“upon the face of a field”} who touches one who has been slain,\lebnote{“the dead of sword”} or a corpse, or a bone of a person, or a burial site, he will be unclean for seven days.}%
\verse{For the unclean person they will take\lebnote{Hebrew “he will take”} from the powder of the burnt purification offering,\lebnote{Or “burning of the sin offering”} and they will put\lebnote{Hebrew “he will put”} running water into a container.}%
\verse{A clean person will take hyssop and dip it into the water and sprinkle it on the tent and on all the objects and persons who were there, and on one who touched the bone, or the one slain, or the dead, or the burial site.}%
\verse{The clean person will spatter the unclean on the third day and on the seventh day; and on the seventh day he will purify him, and he will wash his garments; he will bathe in the waters, and in the evening he will be clean.}%
\verse{But the man who is unclean and does not purify himself, that person will be cut off from the midst of the assembly because he defiled the sanctuary of Adonai; the water of impurity was not sprinkled on him; he is unclean.}%
\verse{“‘It will be an eternal decree for them. The one who spatters the waters of impurity will wash his garments, and the one who touches the waters of impurity will be unclean until the evening.}%
\verse{Anything that the unclean person touches will be unclean, and the person who touches it will be unclean until the evening.’”}%
\end{biblechapter}%
\begin{biblechapter}% Numbers 20
\verseWithHeading{Miriam Dies}{Then the entire community of the Israelites\lnTN{} came to the desert of Zin on the first month, and the people stayed in Kadesh; Miriam died and was buried there.}%
\verse{There was no water for the community, and they were gathered before Moses and Aaron.}%
\verse{And the people quarreled with Moses and spoke, saying, “If only we died when our brothers were dying before\lnTO{} Adonai!}%
\verse{Why have you brought the assembly of Adonai, us and our livestock, into this desert to die here?}%
\verse{Why have you brought us from Egypt to bring us to this bad place? It is not a place of seed or figs\lebnote{Hebrew “fig”} or vines\lebnote{Hebrew “vine”} or pomegranate trees,\lebnote{Hebrew “tree”} and there is not water to drink.”}%
\verse{And Moses and Aaron went from the presence of the assembly to the doorway of the tent of assembly. They fell on their faces, and the glory of Adonai appeared to them.}%
\verse{Adonai spoke to Moses, saying,}%
\verse{“Take the staff and summon the community, you and Aaron your brother, and speak to the rock before their eyes, and it will give water. Bring out for them water from the rock, and let the community and their livestock drink.”}%
\verse{So Moses took the staff from before\lnTO{} Adonai just as he command him,}%
\verse{and Moses and Aaron summoned the assembly to the presence of the rock, and he said to them, “Please listen, you rebels; can we bring out water for you from this rock?”}%
\verse{Then Moses lifted up his hand and struck the rock with his staff twice. And abundant water went out, and the community and their livestock drank.}%
\verse{But Adonai said to Moses and Aaron, “Because you have not trusted in me, to regard me as holy in the sight of\lebnote{Or “before the eyes of”} the Israelites,\lnTN{} you will not bring this assembly into the land that I have given to them.”}%
\verse{Those were the waters of Meribah, where the Israelites\lnTN{} quarreled with Adonai, and he showed himself holy among them.}%
\verse{From Kadesh Moses sent messengers to the king of Edom: “Thus your brother Israel has said, ‘You know all the hardship that has found us;}%
\verse{our ancestors\lnTP{} went down to Egypt, and we lived in Egypt a long time,\lebnote{“many days”} and the Egyptians mistreated us and our ancestors.\lnTP{}}%
\verse{Then we cried to Adonai, and he heard our voice; he sent an angel and brought us out from Egypt. And look, we are in Kadesh, a city on the edge of your territory.}%
\verse{Please let us go through your land. We will not go through a field or vineyard, and we will not drink water from a well. We will go along the road of the king; we will not turn aside right or left until we have gone through your territory.’”}%
\verse{Then Edom said to him, “You will not pass through us\lebnote{Hebrew “me”} lest we will go out\lebnote{Hebrew “I will go out”} to meet you with the sword.”}%
\verse{The Israelites\lnTN{} said to him, “We will go up on the main road, and if we\lebnote{Hebrew “I”} and our livestock\lebnote{Hebrew “my livestock”} drink your water, we will pay for it.\lebnote{Hebrew “I will give their worth”} It is only a small matter; let us pass through on our feet.”\lebnote{Hebrew “I will go through on my feet”}}%
\verse{But he said, “You will not go through.” And Edom went out to meet them\lebnote{Hebrew “him”} with a large army and a strong hand.}%
\verse{So Edom refused to give Israel passage through his territory, and Israel turned aside from him.}%
\verseWithHeading{Aaron Dies}{And they set out from Kadesh. The Israelites,\lnTN{} the whole community, came to Mount Hor.}%
\verse{Adonai said to Moses and to Aaron on Mount Hor, on the boundary of the land of Edom, saying,}%
\verse{“Let Aaron be gathered to his people; he will not come into the land that I have given to the Israelites\lnTN{} because you rebelled against my word\lebnote{“my mouth”} at the waters of Meribah.}%
\verse{Take Aaron and Eleazar his son, and take them up Mount Hor.}%
\verse{Strip off Aaron’s garments, and put them on Eleazar his son; Aaron will be gathered to his people, and he will die there.”}%
\verse{So Moses did just as Adonai commanded, and they went up to Mount Hor before the eyes of all the community.}%
\verse{And Moses stripped off Aaron’s garments and put them on Eleazar his son. Aaron died there on the top of the mountain; and Moses and Eleazar went down from the mountain.}%
\verse{All the community saw that Aaron died; so all the house of Israel wept for Aaron thirty days.\lebnote{Hebrew “day”}}%
\end{biblechapter}%
\begin{biblechapter}% Numbers 21
\verseWithHeading{Arad Captured}{The Canaanite king of Arad, who was dwelling in the Negev,\lebnote{An arid region south of the Judean hills} heard that Israel came along the way of Atharim; he fought against Israel and took some of them captive.}%
\verse{Israel made a vow to Adonai, and they said, “If you will surely give this people into our\lebnote{Hebrew “my”} hand, then we\lebnote{Hebrew “I”} will destroy\lebnote{“devote to God”} their cities.”}%
\verse{Adonai heard the voice of Israel; he gave to them the Canaanites, and they destroyed them\lebnote{“they devoted to God”} and their cities. They called the name of the place Hormah.}%
\verse{They set out from Mount Hor by the way of the Red Sea\lebnote{“sea of reed”} to go around the land of Edom; but the people became impatient\lebnote{“the life of the people became short”} along the way.}%
\verse{The people spoke against God and against Moses, “Why have you brought us from Egypt to die in the desert? There is no food and no water, and our hearts detest this miserable food.”}%
\verseWithHeading{The Bronze Serpent}{And Adonai sent among the people poisonous snakes; they bit the people, and many people from Israel died.}%
\verse{The people came to Moses and said, “We have sinned because we have spoken against Adonai and against you. Pray to Adonai and let him remove the snakes\lebnote{Hebrew “snake”} from among us.” So Moses prayed for the people.}%
\verse{And Adonai said to Moses, “Make for yourself a snake and place it on a pole. When\lnTQ{} anyone is bitten and looks at it, that person will live.”}%
\verse{So Moses made a snake of bronze, and he placed it on the pole; whenever\lnTQ{} a snake bit someone, and that person looked at the snake of bronze, he lived.}%
\verse{The Israelites\lebnote{“sons/children of Israel”} set out and encamped at Oboth.}%
\verse{They set out from Oboth and encamped at Iye Abarim in the desert, which was in front of Moab toward the sunrise.\lebnote{“from the east of the sun”}}%
\verse{From there they set out and encamped at the valley of Zered.}%
\verse{From there they set out and encamped beyond Arnon, which is in the desert that goes out from the boundary of the Amorites,\lnTR{} because Arnon is the boundary of Moab, between Moab and the Amorites.\lnTR{}}%
\verse{Therefore thus it is said in the scroll of the Wars of Adonai, “Waheb in Suphah, and the wadis of Arnon,}%
\verse{and the slope of the wadis that spreads out to the dwelling of Ar and lies at the boundary of Moab.”}%
\verse{From there they went to Beer, which is the water well where Adonai spoke to Moses, “Gather the people, that I may give them water.”}%
\verse{Then Israel sang this song, “Arise, well water! Sing to it!}%
\verse{Well water that the princes dug, that the leaders of the people dug, with a staff and with their rods.” And from the desert they continued to Mattanah,}%
\verse{and from Mattanah to Nahaliel, and from Nahaliel to Bamoth;}%
\verse{and from Bamoth to the valley that is in the territory of Moab, by the top of Pisgah, which overlooks the surface of the wasteland.}%
\verseWithHeading{Sihon and Og Defeated}{Israel sent messengers to Sihon, the king of the Amorites,\lnTR{} saying,}%
\verse{“Let us go through your land; we will not turn aside into a field or vineyard; we will not drink well water along the way of the king until we have gone through your territory.”}%
\verse{But Sihon did not allow Israel to go through his territory. Sihon gathered all his people and went out to meet Israel; he came to the desert, to Jahaz, and he fought against Israel.}%
\verse{But Israel struck him with the edge of the sword, and they took possession of his land from Arnon to Jabbok, until the Ammonites,\lnTS{} because the boundary of the Ammonites\lnTS{} was strong.}%
\verse{Israel took all these cities, and Israel inhabited all the cities of the Amorites,\lnTR{} in Heshbon, and in all its environs.\lnTT{}}%
\verse{Because Heshbon was the city of Sihon king of the Amorites,\lnTR{} who had fought against the former king of Moab and taken all his land from his hand until Arnon.}%
\verse{Thus the ones who quote proverbs say, “Come to Heshbon! Let it be built! And let the city of Sihon be established.}%
\verse{Because fire went out from Heshbon, a flame from the city of Sihon; it consumed Ar of Moab, the lords of the\lebnote{Or “the dominant”} high places of Arnon.}%
\verse{Woe to you, Moab! You have perished, people of Chemosh. He has given his sons as fugitives, and his daughters into captivity, to the king of the Amorites,\lnTR{} Sihon.}%
\verse{We destroyed them; Heshbon has perished up to Dibon; we laid waste up to Nophah, which reaches\lebnote{“\textit{is} up to”} Medeba.”}%
\verse{Thus Israel lived in the land of the Amorites.\lnTR{}}%
\verse{Moses sent to explore Jaazer; they captured its environs\lnTT{} and dispossessed the Amorites\lnTR{} who were there.}%
\verse{Then they turned and went up by the way of the Bashan, and Og king of the Bashan and all his people went out to meet them for battle at Edrei.}%
\verse{And Adonai said to Moses, “Do not fear him because I will give him and all his people and all his land into your hand. You will do to him just as you did to Sihon king of the Amorites,\lnTR{} who was living in Heshbon.”}%
\verse{And so they destroyed him and his sons, and all his people until they had not spared a survivor; and they took possession of his land.}%
\end{biblechapter}%
\begin{biblechapter}% Numbers 22
\verseWithHeading{Balak and Balaam}{The Israelites\lnTU{} set out, and they encamped on the desert-plateau of Moab, across from Jericho beyond the Jordan.}%
\verse{Balak son of Zippor saw all that Israel did to the Amorites,\lebnote{Hebrew “Amorite”}}%
\verse{and Moab was very terrified in the presence of the people because they\lebnote{Hebrew “he” or “it”} were numerous; and Moab dreaded the presence of the Israelites.\lnTU{}}%
\verse{And Moab said to the elders of Midian, “Now the crowd will lick up all around us, like a bull devours the grass of the field.” And Balak son of Zippor was king of Moab at that time.}%
\verse{He sent messengers to Balaam son of Beor at Pethor, which is by the river,\lebnote{That is, the Euphrates} in the land of the children of his people, to summon him, saying, “Look! A people went out from Egypt. Look! They cover the surface of the land;\lnTV{} they are about to dwell opposite me.}%
\verse{Now, please go, curse this people for me because they\lnTW{} are stronger than me; perhaps I will be able to strike them\lnTW{} and drive them\lnTW{} out from the land because I know whoever you bless is blessed, and whoever you cursed is cursed.”}%
\verse{So the elders of Moab and the elders of Midian went with a fee for divination in their hand; they came to Balaam and spoke the words of Balak to him.}%
\verse{He said to them, “Spend the night here, and I will return, and I will return word to you, just as Adonai speaks to me.” So the princes of Moab stayed with Balaam.}%
\verse{And God came to Balaam and said, “Who are these men with you?”}%
\verse{And Balaam said to God, “Balak son of Zippor, king of Moab, sent word to me,}%
\verse{‘Look! A people went out from Egypt. Look! They cover the surface of the land.\lnTV{} Now, go, curse them\lnTX{} for me. Perhaps I will be able to attack them\lnTX{} and drive them\lnTX{} out.”}%
\verse{God said to Balaam, “You will not go with them; you will not curse the people, because they\lnTW{} are blessed.”}%
\verse{Balaam got up in the morning, and he said to the princes of Balak, “Go to your land, because Adonai refused to allow me to go with you.”}%
\verse{The princes of Moab got up and went to Balak, and they said, “Balaam refused to come with us.”}%
\verse{Balak again sent many princes, who were more honored than the former.\lebnote{“than these”}}%
\verse{They came to Balaam and said to him, “Thus says Balak son of Zippor, ‘Please, let nothing keep you from coming to me}%
\verse{because I will surely honor you greatly, and all that you say to me I will do. Please, come; curse this people for me.’”}%
\verse{Balaam answered and said to the servants of Balak, “Even though Balak gives to me his house full of silver and gold, I am not able to go beyond the command of Adonai\lebnote{“the mouth of Adonai”} my God to do a little or a lot.}%
\verse{And now please, you also stay here\lebnote{“please stay in this”} the night, and let me find out\lebnote{“let me know”} again what Adonai will say with me.”}%
\verse{And God came to Balaam at night, and he said to him, “If the men have come to call you, get up and go with them; but only the word that I will speak to you, you will do.”}%
\verse{So Balaam got up in the morning and saddled his donkey, and he went with the princes of Moab.}%
\verseWithHeading{Balaam and the Angel}{But God became angry\lebnote{“God’s nose became hot”} because he was going, and the angel of Adonai stood in the road as an adversary to him; he was riding on his donkey, and two servants were with him.}%
\verse{The donkey saw the angel of Adonai standing in the road with his sword drawn in his hand, and the donkey turned aside from the road and went into the field. And Balaam struck the donkey to turn her back to the road.}%
\verse{The angel of Adonai stood in the narrow path of the vineyards, with a wall on either side.\lebnote{“a wall from this and a wall from this”}}%
\verse{When the donkey saw the angel of Adonai, she pressed herself into the wall, and she pressed the foot of Balaam into the wall, so he struck her again.}%
\verse{Then the angel of Adonai went further ahead and stood in a narrow place where there was not a way to turn aside to the right or left.}%
\verse{When the donkey saw the angel of Adonai, she lay down under Balaam, so Balaam became angry,\lebnote{“Balaam’s nose became hot”} and he struck the donkey with his staff.}%
\verse{Adonai opened the mouth of the donkey, and she said to Balaam, “What did I do to you that you struck me these three times?”}%
\verse{Balaam said to the donkey, “Because you made a mockery of me! If only I had a sword in my hand, I would kill you right now!”}%
\verse{The donkey said to Balaam, “Am I not your donkey on which you have ridden all your life until this day? Have I been in the habit of doing this to you?” He said, “No.”}%
\verse{Then Adonai exposed the eyes of Balaam, and he saw the angel of Adonai standing in the road with his sword drawn in his hand, and he bowed down and worshiped to his face.}%
\verse{The angel of Adonai said to him, “Why have you struck this donkey three times? Look, I have come out as an adversary because your conduct is perverse before me.}%
\verse{The donkey saw me and turned aside from me these three times. If she had not turned aside from my face, then I would have killed you and kept her alive.”}%
\verse{Balaam said to the angel of Adonai, “I have sinned because I did not know that you were standing to meet me in the road. Now, if it is displeasing to you,\lebnote{“if it is evil in your eyes”} I will turn back.”}%
\verse{The angel of Adonai said to Balaam, “Go with the men, but speak only the word that I will speak to you.” So Balaam went with the princes of Balak.}%
\verse{When Balak heard that Balaam was coming, he went out to meet him by the city of Moab, which was on the boundary of Aaron at the end of the territory.}%
\verse{And Balak said to Balaam, “Did I not urgently send to meet with you? Why did you not come to me? Am I really not able to honor you?”}%
\verse{Balaam said to Balak, “Look, I came to you now. Am I really able to speak anything at all? I speak the word that God puts in my mouth.”}%
\verse{Balaam went with Balak, and they came to Kiriath-Huzoth.}%
\verse{And Balak sacrificed cattle and sheep, and he sent them to Balaam and to the princes who were with him.}%
\verse{And it happened, in the morning Balak took Balaam and took him up to Bamoth-Baal, and he saw from there the end of the nation.}%
\end{biblechapter}%
\begin{biblechapter}% Numbers 23
\verseWithHeading{Balaam’s Oracles}{Balaam said to Balak, “Build for me this: seven altars. And prepare for me this: seven bulls and seven rams.”}%
\verse{And Balak did just as Balaam spoke, and Balak offered Balaam a bull and a ram on the altar.}%
\verse{And Balaam said to Balak, “Station yourself at your burnt offering, and I will go; perhaps Adonai will come to meet me, and whatever he shows me I will tell to you.” So he went to a barren height.}%
\verse{And God met with Balaam, and he said to him, “I have arranged seven altars, and I have offered a bull and a ram on the altar.”}%
\verse{Adonai put a word in the mouth of Balaam and said, “Return to Balak, and you must speak thus.”}%
\verse{So he returned to him, and behold, he was standing beside his burnt offering, he and all the leaders of Moab.}%
\verse{And he lifted up his oracle and said, “From Aram Balak lead me, from the mountains of the east the king of Moab, ‘Go for me, curse Jacob, and go, denounce Israel.’}%
\verse{How can I curse whom God has not cursed, and how can I denounce whom Adonai has not denounced?}%
\verse{Because from the top of the rocks I see him, from hilltops I watch him. Behold, a people who dwell alone, they do not consider themselves among the nations.}%
\verse{Who can count the dust of Jacob, or as a number the fourth part of Israel? Let my life die the death of an upright person, and let my end be like his!”}%
\verse{And Balak said to Balaam, “What have you done to me? I took you to curse my enemies, and look, you have surely blessed them!”}%
\verse{He answered and said, “Should I not speak\lebnote{“Should I not observe to speak”} what Adonai puts in my mouth?”}%
\verse{Then Balak said, “Please walk with me to another place where you will see them, but you will only see part of them and will not see all of them; and curse them for me from there.”}%
\verse{So he took him to the field of Zophim to the top of Pisgah, and he built seven altars, and he offered a bull and a ram on each altar.}%
\verse{Balaam\lebnote{Hebrew “He”} said to Balak, “Station yourself here at the burnt offering while I myself meet with Adonai there.”}%
\verse{Then Adonai met with Balaam, and he put a word in his mouth, and he said, “Return to Balak, and you must speak thus.”}%
\verse{He came to him, and behold, he was standing at his burnt offering, and the princes of Moab with him. And Balak said to him, “What has Adonai spoken?”}%
\verse{Then he uttered\lebnote{“he lifted up”} his oracle, and said, “Stand up, Balak, and hear; listen to me, son of Zippor!}%
\verse{God is not a man, that he should lie, nor a son of humankind, that he should change his mind. Has he said, and will he not do it? And has he spoken, and will he not fulfill it?}%
\verse{Behold, I have received a command to bless; when he has blessed, I cannot cause it to return.}%
\verse{He has no regard for evil in Jacob, and he does not see trouble in Israel; Adonai his God is with him, and a shout\lebnote{Or “a blast”} of a king is among them.\lnTY{}}%
\verse{God, who brings them out from Egypt, is like the strength\lebnote{Or “like the horns”} of a wild ox for them.\lnTY{}}%
\verse{Because there is no sorcery against Jacob, and there is no divination against Israel. Now\lebnote{Or “At the right time”} it will be said to Jacob and Israel, what God has done!}%
\verse{Look! the people will rise like the lion; he raises himself and will not lie down until he eats the prey and drinks the blood of the slain.”}%
\verse{Then Balak said to Balaam, “Do not curse them\lnTY{} at all, nor bless them\lnTY{} at all!”}%
\verse{But Balaam answered and said to Balak, “Did I not speak to you, saying, ‘Whatever Adonai speaks I will do’?”}%
\verse{Then Balak said to Balaam, “Please, come, I will take you to another place; perhaps it will be acceptable to\lebnote{“it will be right in the eyes of”} God, and you will curse for me from there.”}%
\verse{So Balak took Balaam to the top of Peor, which looks down on the face of the Jeshimon.\lebnote{Or “the wasteland”}}%
\verse{And Balaam said to Balak, “Build for me these seven altars, and prepare for me these seven bulls and seven rams.”}%
\verse{Balak did just as Balaam said, and he offered a bull and a ram on each altar.}%
\end{biblechapter}%
\begin{biblechapter}% Numbers 24
\verseWithHeading{Balaam Continues to Utter Oracles}{And Balaam saw that it pleased\lebnote{“it was good in the eyes of Adonai”} Adonai to bless Israel, and he did not go as other times\lebnote{“as time on time”} to seek out\lebnote{“to meet”} sorcery; instead, he set his face toward the desert.}%
\verse{Balaam lift up his eyes, and he saw Israel dwelling according to its tribes, and the spirit of God was upon it.\lebnote{That is, Israel}}%
\verse{He uttered\lnTZ{} his oracle and said, “The declaration of Balaam son Beor, the declaration of the man whose eyes are closed,}%
\verse{the declaration of the hearer of God’s words,\lnUA{} who sees the revelation of Shaddai,\lnUB{} falling down but whose eyes are uncovered.}%
\verse{How good are your tents, O Jacob, your dwellings, O Israel!}%
\verse{They are spread out like valleys, like gardens on a river, like aloes planted by Adonai, like cedars at the waters.}%
\verse{He will pour water from his buckets, and his offspring will be like many waters; his king will be higher than Agag, and his kingdom will be exalted.}%
\verse{God, who brings him out from Egypt, is like the strength\lebnote{Or “the horns”} of a wild ox for him. He will devour the nations who are his enemies; he will break their bones; he will pierce them with his arrows.}%
\verse{He crouches, he lies down like a lion, and like a lioness, who will rouse him? They who bless you will be blessed, and they who curse you will be cursed.”}%
\verse{Then Balak became angry with\lebnote{“the nose of Balak became hot against”} Balaam, and he clapped his hands and said to Balaam, “I called you to curse my enemies, but look, you have surely blessed them these three times.}%
\verse{Flee\lebnote{“Flee for yourself”} to your place now. I said I would richly honor you, but look, Adonai has withheld honor from you.”}%
\verse{Balaam said to Balak, “Did I not speak to your messengers whom you sent to me, saying,}%
\verse{‘If Balak gave to me the fullness of his house full of silver and gold, I am not able to go beyond the command of Adonai\lebnote{“the mouth of Adonai”} to do good or evil, from my heart; what Adonai speaks, I will speak’?\lebnote{Hebrew “I will speak it”}}%
\verse{And now, look, I am about to go to my people; I will advise you what this people will do to your people in the following days.”\lebnote{“in the last of the days”}}%
\verse{And he uttered\lnTZ{} his oracle and said, “The declaration of Balaam son of Beor, and the declaration of the man whose eye is closed,}%
\verse{the declaration of the hearer of God’s words,\lnUA{} and the knower of the knowledge of the Most High, who sees the vision of Shaddai,\lnUB{} who is falling, and his eyes are revealed.}%
\verse{I see him, but not now; I behold him, but not near; a star will go out from Jacob, and a scepter will rise from Israel; it will crush the foreheads of Moab and destroy all the children of Seth.}%
\verse{Edom will be a captive; Seir, its enemies, will be a captive, and Israel will be acting courageously.\lebnote{“\textit{with} physical strength”}}%
\verse{Someone\lebnote{Hebrew “He”} from Jacob will rule and will destroy a remnant\lebnote{Or “survivor”} from the city.”}%
\verse{And he looked at Amalek, uttered\lnTZ{} his oracle, and said, “Amalek is first\lebnote{Or “among”} of the nations, but his future will be forever ruin.”}%
\verse{And he looked at the Kenites,\lebnote{Hebrew “Kenite”} uttered\lnTZ{} his oracle, and said, “Steady is your dwelling place; in the rock is your nest.}%
\verse{Nevertheless, the Kenite will be burned; how long will Asshur keep\lebnote{Hebrew “take”} you captive?”}%
\verse{Again he uttered\lnTZ{} his oracle and said, “Woe, who will live when God establishes this?\lebnote{Hebrew “it”}}%
\verse{The ships will come from the hand of the Kittim, and they will afflict Asshur and will afflict Eber; also he will be forever ruin.”}%
\verse{Then Balaam got up and went and returned to his place, and Balak also went on his way.}%
\end{biblechapter}%
\begin{biblechapter}% Numbers 25
\verseWithHeading{The Plague of Israel}{When Israel dwelled in Shittim, the people began to prostitute\lebnote{Or “have sexual relations”} themselves with the daughters of Moab.}%
\verse{And they invited the people to the sacrifices of their gods, and the people ate and worshiped their gods.}%
\verse{So Israel was joined together to Baal Peor, and Adonai became angry\lebnote{“the nose of Adonai became hot”} with Israel.}%
\verse{Adonai said to Moses, “Take all the leaders\lebnote{“the heads”} of the people and kill them before the sun, so the fierce anger of Adonai will turn from Israel.”}%
\verse{So Moses said to the judges of Israel, “Each of you kill his men who are joined together with Baal Peor.”}%
\verse{And behold, a man from the Israelites\lnUC{} came and brought to his brothers a Midianite woman before the eyes of Moses and before the eyes of all of the community of the Israelites,\lnUC{} and they were weeping at the doorway of the tent of assembly.}%
\verse{When Phinehas son of Eleazar son of Aaron the priest saw, he got up from the midst of the community and took a spear in his hand.}%
\verse{He went after the man of Israel into the woman’s section of the tent, and he drove the two of them, the man of Israel and the woman, into her belly. And the plague among the Israelites\lnUC{} stopped.}%
\verse{The ones who died in the plague were twenty-four thousand.}%
\verse{Adonai spoke to Moses, saying,}%
\verse{“Phinehas son of Eleazar, son of Aaron the priest, turned away my anger from among the Israelites\lnUC{} when he was jealous with my jealousy in their midst, and I did not destroy the Israelites\lnUC{} with my jealousy.}%
\verse{Therefore say, ‘Behold, I am giving to him my covenant of peace,}%
\verse{and it will be for him and his offspring\lebnote{“seed”} after him a covenant of an eternal priesthood because he was jealous for his God and made atonement for the Israelites.’”\lnUC{}}%
\verse{The name of the man of Israel who was struck with the Midianite woman was Zimri son of Salu, a leader of the family\lnUD{} of the Simeonites.\lebnote{Hebrew “Simeonite”}}%
\verse{The name of the Midianite woman who was struck was Cozbi daughter of Zur, a leader\lebnote{“a head”} of a tribe of the family\lnUD{} in Midian.}%
\verse{Adonai spoke to Moses, saying,}%
\verse{“Attack the Midianites and strike them}%
\verse{because they were attacking you with their deception, with which they have deceived you on the matter of Peor and on the matter of Cozbi the daughter of the leader of Midian, their sister who was struck on the day of the plague because of the matter of Peor.”}%
\end{biblechapter}%
\begin{biblechapter}% Numbers 26
\verseWithHeading{A New Census}{\lebnote{Numbers 26:1a in the English Bible is 25:19b in the Hebrew Bible} And it happened after the plague,\lebnote{In the Hebrew Bible, Numbers 26:1 begins here}Adonai said to Moses and to Eleazar son of Aaron the priest, saying,}%
\verse{“Take a census\lebnote{“Lift up the head\textit{s}”} of the community of the Israelites\lnUE{} from those twenty years old\lnUF{} and above, according to their families,\lebnote{“the house of their fathers”} all who are able to go out to war in Israel.”}%
\verse{So Moses and Eleazar the priest spoke with them on the desert-plateau of Moab by the Jordan across from Jericho, saying,}%
\verse{“Take a census of the community from those twenty years old\lnUF{} and above, just as Adonai commanded Moses.” The Israelites who went out from the land of Egypt were:}%
\verse{Reuben, the firstborn of Israel, the descendants\lnUG{} of Reuben: of Hanoch, the clan of the Hanochites; of Pallu, the clan of the Palluites;\lebnote{Hebrew “Palluite”}}%
\verse{of Hezron, the clan of the Hezronites;\lnUH{} of Carmi, the clan of the Carmites.\lebnote{Hebrew “Carmite”}}%
\verse{These are the clans of the Reubenites,\lebnote{Hebrew “Reubenite”} and the ones counted of them were forty-three thousand seven hundred and thirty.}%
\verse{The children of Pallu: Eliab.}%
\verse{The children of Eliab: Nemuel, Dathan, and Abiram. These are the same Dathan and Abiram who were appointed of the community, who rebelled against Moses and Aaron in the company of Korah, when they rebelled against Adonai,}%
\verse{and the land opened its mouth and swallowed them with Korah, when that company died, when the fire consumed two hundred and fifty men, and they were a sign.\lebnote{That is, a warning sign}}%
\verse{The children\lnUG{} of Korah, however, did not die.}%
\verse{The descendants\lnUG{} of Simeon, according to their clans: of Nemuel, the clan of the Nemuelites;\lebnote{Hebrew “Nemuelite”} of Jamin, the clan of the Jaminites;\lebnote{Hebrew “Jaminite”} of Jakin, the clan of the Jakinites;\lebnote{Hebrew “Jakinite”}}%
\verse{of Zerah, the clan of the Zerahites;\lnUI{} of Shaul, the clan of the Shaulites.\lebnote{Hebrew “Shaulite”}}%
\verse{These were the clans of the Simeonites,\lebnote{Hebrew “Simeonite”} twenty-two thousand two hundred.}%
\verse{The descendants\lnUG{} of Gad according to their clans: of Zephon, the clan of the Zephonites;\lebnote{Hebrew “Zephonite”} of Haggi, the clan of the Haggites;\lebnote{Hebrew “Haggite”} of Shuni, the clan of the Shunites;\lebnote{Hebrew “Shunite”}}%
\verse{of Ozni, the clan of the Oznites;\lebnote{Hebrew “Oznite”} of Eri, the clan of the Erites;\lebnote{Hebrew “Erite”}}%
\verse{of Arod, the clan of the Arodites;\lebnote{Hebrew “Arodite”} of Areli, the clan of the Arelites;\lebnote{Hebrew “Arelite”}}%
\verse{These were the clans of the descendants\lnUG{} of Gad according to the ones counted of them, forty thousand five hundred.}%
\verse{The sons of Judah: Er and Onan; but Er and Onan died in the land of Canaan.}%
\verse{The descendants\lnUG{} of Judah according to their clans were: of Shelah, the clan of the Shelanites;\lebnote{Hebrew “Shelanite”} of Perez, the clan of the Perezites;\lebnote{Hebrew “Perezite”} of Zerah, the clan of the Zerahites.\lnUI{}}%
\verse{The children of Perez were: of Hezron, the clan of the Hezronites;\lnUH{} of Hamul, the clan of the Hamulites.\lebnote{Hebrew “Hamulite”}}%
\verse{These were the clans of Judah according to the ones counted of them, seventy-six thousand five hundred.}%
\verse{The descendants\lnUG{} of Issachar according to their clans: of Tola, the clan of the Tolaites;\lebnote{Hebrew “Tolaite”} of Puvah, the clan of the Punites;\lebnote{Hebrew “Punite”}}%
\verse{of Jashub, the clans of the Jashubites;\lebnote{Hebrew “Jashubite”} of Shimron, the clan of the Shimronites.\lebnote{Hebrew “Shimronite”}}%
\verse{These were the clans of Issachar according to the ones counted of them, sixty-four thousand three hundred.}%
\verse{The descendants\lnUG{} of Zebulun according to their clans: of Sered, the clan of the Seredites;\lebnote{Hebrew “Seredite”} of Elon, the clan of the Elonites;\lebnote{Hebrew “Elonite”} of Jahleel, the clan of the Jahleelites.\lebnote{Hebrew “Jahleelite”}}%
\verse{These were the clans of the Zebulunites\lebnote{Hebrew “Zebulunite”} according to the ones counted of them, sixty thousand five hundred.}%
\verse{The descendants\lnUG{} of Joseph according to their clans: Manasseh and Ephraim.}%
\verse{The descendants\lnUG{} of Manasseh: of Makir, the clan of the Makirites.\lebnote{Hebrew “Makirite”} And Makir fathered Gilead; of Gilead, the clan of the Gileadites.\lebnote{Hebrew “Gileadite”}}%
\verse{These were the descendants\lnUG{} of Gilead: of Iezer, the clan of the Iezerites;\lebnote{Hebrew “Iezerite”} of Helek, the clan of the Helekites;\lebnote{Hebrew “Helekite”}}%
\verse{and of Asriel, the clan of the Asrielites;\lebnote{Hebrew “Asrielite”} and of Shechem, the clan of the Shechemites;\lebnote{Hebrew “Shechemite”}}%
\verse{and of Shemida, the clan of the Shemidaites;\lebnote{Hebrew “Shemidaite”} and of Hepher, the clan of the Hepherites.\lebnote{Hebrew “Hepherite”}}%
\verse{Zelophehad son of Hepher did not have sons, but only daughters; and the names\lebnote{Hebrew “name”} of the daughters of Zelophehad were Mahlah, Noah, Hoglah, Milcah, and Tirzah.}%
\verse{These were the clans of Manasseh, and the ones counted of them were fifty-two thousand seven hundred.}%
\verse{These were the descendants\lnUG{} of Ephraim according to their clans: of Shuthelah, the clan of the Shuthelahites;\lebnote{Hebrew “Shuthelahite”} of Beker, the clan of the Bekerites;\lebnote{Hebrew “Bekerite”} of Tahan, the clan of the Tahanites.\lebnote{Hebrew “Tahanite”}}%
\verse{And these were the descendants\lnUG{} of Shuthelah: of Eran, the family of the Eranites.\lebnote{Hebrew “Eranite”}}%
\verse{These were the clans of the descendants\lnUG{} of Ephraim according to the ones counted of them, thirty-two thousand five hundred. These were the descendants\lnUG{} of Joseph according to their clans.}%
\verse{The descendants\lnUG{} of Benjamin according to their clans: of Bela, the clan of the Belaites;\lebnote{Hebrew “Belaite”} of Ashbel, the clan of the Ashbelites;\lebnote{Hebrew “Ashbelite”} of Ahiram, the clan of the Ahiramites;\lebnote{Hebrew “Ahiramite”}}%
\verse{of Shephupham, the clan of the Shuphamites;\lebnote{Hebrew “Shuphamite”} of Hupham, the clan of the Huphamites.\lebnote{Hebrew “Huphamite”}}%
\verse{The sons of Bela were Ard and Naaman: of Ard, the clan of the Ardites;\lebnote{Hebrew “Ardite”} of Naaman, the clan of the Naamites.\lebnote{Hebrew “Naamite”}}%
\verse{These were the descendants\lnUG{} of Benjamin according to their clans. And the ones counted of them were forty-five thousand six hundred.}%
\verse{These were the descendants\lnUG{} of Dan according to their clans: of Shuham, the clan of the Shuhamites.\lebnote{Hebrew “Shuhamite”} These were the clans of Dan according to their clans.}%
\verse{All the clans of the Shuhamites,\lebnote{Hebrew “Shumhamite”} according to the ones counted of them, were sixty-four thousand four hundred.}%
\verse{The descendants\lnUG{} of Asher according to their clans: of Imnah, the clan of the Imnahites;\lebnote{Hebrew “Imnahite”} of Ishvi, the clan of the Ishvites;\lebnote{Hebrew “Ishvite”} of Beriah, the clan of the Beriahites.\lebnote{Hebrew “Beriahite”}}%
\verse{The descendants\lnUG{} of Beriah: of Heber, the clan of the Heberites;\lebnote{Hebrew “Heberite”} of Malkiel, the clan of the Malkielites.\lebnote{Hebrew “Malkielite”}}%
\verse{The name of the daughter of Asher was Serah.}%
\verse{These were the clans of the descendants\lnUG{} of Asher according to the ones counted of them, fifty-three thousand four hundred.}%
\verse{The descendants\lnUG{} of Naphtali according to their clans: of Jahzeel, the clan of the Jahzeelites;\lebnote{Hebrew “Jahzeelite”} of Guni, the clan of the Gunites;\lebnote{Hebrew “Gunite”}}%
\verse{of Jezer, the clan of the Jezerites;\lebnote{Hebrew “Jezerite”} of Shillem, the clan of the Shillemites.\lebnote{Hebrew “Shillemite”}}%
\verse{These were the clans of Naphtali according to their clans, the ones counted of them, forty-five thousand four hundred.}%
\verse{These were the ones counted of the Israelites,\lnUE{} six hundred and one thousand seven hundred and thirty.}%
\verse{Then Adonai spoke to Moses, saying,}%
\verse{“For these the land must be divided as an inheritance according to the number of names.}%
\verse{For the larger group you must increase their inheritance, and for the smaller group you must make smaller their inheritance; each must be given their\lebnote{Hebrew “his”} inheritance according to the number of the ones counted of them.}%
\verse{Surely the land will be divided by lot. They will inherit according to the names of the tribes of their ancestors.\lebnote{Or “fathers”}}%
\verse{Their\lebnote{Hebrew “His”} inheritance must be divided according to the lot between the larger and smaller groups.”}%
\verse{These are the ones counted of the Levites according to their clans: of Gershon, the clan of the Gershonites;\lebnote{Hebrew “Gershonite”} of Kohath, the clan of the Kohathites;\lebnote{Hebrew “Kohathite”} of Merari, the clan of the Merarites.\lebnote{Hebrew “Merarite”}}%
\verse{These are the clans of Levi: the clan of the Libnites,\lebnote{Hebrew “Libnite”} the clan of the Hebronites,\lebnote{Hebrew “Hebronite”} the clan of the Mahlites,\lebnote{Hebrew “Mahlite”} the clan of the Mushites,\lebnote{Hebrew “Mushite”} the clan of the Korahites.\lebnote{Hebrew “Korahite”} Kohath fathered Amram.}%
\verse{The name of the wife of Amram was Jochebed, the daughter of Levi, whose mother bore her for Levi in Egypt; she bore to Amram: Aaron and Moses and their sister Miriam.}%
\verse{To Aaron were born Nadab and Abihu, Eleazar and Ithamar.}%
\verse{But Nadab and Abihu died when they presented strange fire before\lebnote{“in the presence of”} Adonai.}%
\verse{The ones counted were twenty-three thousand, every male from a month old\lebnote{“a son of one month”} and above, because they were not counted in the midst of the Israelites\lnUE{} since no inheritance was given to them in the midst of the Israelites.\lnUE{}}%
\verse{These were the ones counted by Moses and Eleazar the priest, who counted the Israelites\lnUE{} on the desert-plateau of Moab on the Jordan across Jericho.}%
\verse{And among these there was not a man of those counted by Moses and Aaron the priest, who counted the Israelites\lnUE{} in the desert of Sinai.}%
\verse{For Adonai said to them, “They will surely die in the desert.” And not a man was left over from them, except Caleb son of Jephunneh and Joshua son of Nun.}%
\end{biblechapter}%
\begin{biblechapter}% Numbers 27
\verseWithHeading{Laws of Inheritance}{Then the daughters of Zelophehad, the son of Hepher, the son of Gilead, the son of Makir, the son of Manasseh, of the clan of Manasseh the son of Joseph, came near; and these were the names of his daughters: Mahlah, Noah, Hoglah, and Tirzah.}%
\verse{They stood before Moses and before Eleazar the priest and before the leaders of the entire community at the doorway of the tent of assembly,\lebnote{Or “meeting”} saying,}%
\verse{“Our father died in the desert; he was not in the midst of the company of those who banded together against Adonai in the company of Korah, but he died in his own sin, and he had no sons.}%
\verse{Why should the name of our father disappear from the midst of his clan because he does not have a son? Give us property in the midst of the brothers of our father.”}%
\verse{So Moses brought their case before Adonai.}%
\verse{And Adonai said to Moses, saying,}%
\verse{“The statements of the daughters of Zelophehad are right.\lebnote{“Such as the daughters of Zelophehad \textit{are} speaking”} You must surely give them the property of an inheritance in the midst of their father’s brothers, and you must transfer the inheritance of their father to them.}%
\verse{And you must speak to the Israelites,\lnUJ{} saying, ‘If a man dies and has no son, you must transfer his inheritance to his daughter.}%
\verse{And if he has no daughter, you must give his inheritance to his brothers.}%
\verse{If he has no brothers, then you must give his inheritance to his father’s brothers.}%
\verse{If his father has no brothers, then you must give his inheritance to his nearest relative from his own clan, and he will take possession of it. It will be as a decree of stipulation for the Israelites,\lnUJ{} just as Adonai commanded Moses.’”}%
\verse{Adonai said to Moses, “Go up to this mountain of Abarim, and see the land that I have given to the Israelites.\lnUJ{}}%
\verse{When you see it, you will be gathered to your people, just as Aaron your brother was gathered,}%
\verse{because you rebelled against my word in the desert of Zin when the community quarreled regarding my holiness at the waters.” (These are the waters of Meribah-Kadesh in the desert of Zin.)}%
\verseWithHeading{Joshua Succeeds Moses}{Adonai spoke to Moses, saying,}%
\verse{“Let Adonai, the God of the spirits of all flesh, appoint a man over the community}%
\verse{who will go out before them and will come in before them, and who will lead them out and bring them in, so the community of Adonai will not be like a flock that does not have a shepherd.”}%
\verse{Then Adonai said to Moses, “Take Joshua son of Nun, a man in whom is the spirit, and place your hand on him.}%
\verse{Have him stand before Eleazar the priest and before the entire community, and commission him\lebnote{“command him”} in their sight.\lebnote{“before their eyes”}}%
\verse{You will give to him from your authority so that the entire community of Israel will obey him.\lebnote{“hear \textit{him}”}}%
\verse{He will stand before Eleazar the priest, who will ask for him by the decision\lebnote{Or “judgment”} of the Urim before Adonai. On his command\lnUK{} they will go out, and at his command\lnUK{} they will come in, both he and all of the Israelites\lnUJ{} with him, the entire community.”}%
\verse{Moses did just as Adonai commanded him, and he took Joshua and set him before Eleazar the priest and before the entire community.}%
\verse{And he placed his hands on him and commissioned him\lebnote{“commanded him”} just as Adonai spoke by the hand of Moses.\lebnote{Or “through Moses”}}%
\end{biblechapter}%
\begin{biblechapter}% Numbers 28
\verseWithHeading{Daily Sacrifice}{Adonai spoke to Moses, saying,}%
\verse{“Command the Israelites\lebnote{“sons/children of Israel”} and say to them, ‘You will be careful to present\lebnote{“you will observe to present”} my offering, my food of my offerings made by fire, of a fragrance of appeasement to me, at its appointed time.’}%
\verse{You will say to them, ‘This is the offering made by fire that you will offer to Adonai: two male lambs without defect in their first year\lnUL{} as a continual burnt offering each day.}%
\verse{You will offer one male lamb in the morning, and the second male lamb you will offer at twilight,\lnUM{}}%
\verse{and a tenth of an ephah of finely milled flour as a grain offering, mixed with a fourth of a measure of beaten oil.}%
\verse{It is a continual burnt offering that was ordained\lebnote{“done”} on Mount Sinai as a fragrance of appeasement, an offering made by fire for Adonai.}%
\verse{The libation with it will be a fourth of a liquid measure for each male lamb; in the sanctuary you will pour out the libation of fermented drink for Adonai.}%
\verse{And the second male lamb you will offer at twilight;\lnUM{} as the grain offering of the morning and as its libation you will offer it, an offering made by fire, a fragrance of appeasement for Adonai.}%
\verse{“‘On the day of the Sabbath, two male lambs without defect in their first year,\lnUL{} and two-tenths of finely milled flour mixed with oil for a grain offering and its libation.}%
\verse{This is the burnt offering every Sabbath in addition to the continual burnt offering and its libation.}%
\verse{“‘And at the beginning of each of your months, you will present a burnt offering for Adonai: two bulls and one ram, seven male lambs without defect in their first year;\lnUL{}}%
\verse{and three-tenths of finely milled flour mixed with oil for a grain offering, for each bull; and two-tenths of finely milled flour mixed with oil for a grain offering for the one ram;}%
\verse{and a tenth of finely milled flour mixed with oil as a grain offering for each male lamb, for a burnt offering of a fragrance of appeasement, an offering of fire for Adonai.}%
\verse{Their libations will be half a liquid measure of wine for the bull and a third of a liquid measure of wine for the ram and a fourth of a liquid measure of wine for the male lamb; this is the burnt offering for every month for the months of the year.}%
\verse{And one male goat as a sin offering for Adonai; it will be offered in addition to the continual burnt offering and its libation.}%
\verse{“‘On the fourteenth day of the first month is the Passover for Adonai.}%
\verse{On the fifteenth day of this month is a religious feast, unleavened bread must be eaten for seven days.}%
\verse{On the first day there will be a holy convocation; you will not do any regular work.\lnUN{}}%
\verse{You will present an offering by fire, a burnt offering for Adonai: two bulls and one ram and seven male lambs in their first year;\lnUL{} they will be for you without defect.}%
\verse{For their grain offering, you will offer finely milled flour mixed with oil: three-tenths for the bull and two-tenths for the ram.}%
\verse{You will offer a tenth for each of the seven male lambs;}%
\verse{and a goat for one sin offering to make atonement for you.}%
\verse{You will offer these besides the burnt offering of the morning, which is for the continual burnt offering.}%
\verse{Like this you will offer daily, for seven days, the food of the offering made by fire, a fragrance of appeasement for Adonai; it will be offered in addition to the continual burnt offering and its libation.}%
\verse{On the seventh day you will have a holy convocation; you will not do any regular work.\lnUN{}}%
\verseWithHeading{Offerings for the Festival of Weeks}{“‘And on the day of firstfruits, when you are presenting a new offering for Adonai during your Festival of Weeks, you will have a holy convocation; you will not do any regular work.\lnUN{}}%
\verse{You will present a burnt offering for a fragrance of appeasement for Adonai: two bulls, one ram, seven male lambs in their first year;\lnUL{}}%
\verse{and their grain offering will be finely milled flour mixed with oil: three-tenths for each bull, two-tenths for one ram,}%
\verse{a tenth for each of the male lambs;}%
\verse{and one male goat to make atonement for you.}%
\verse{In addition to the continual burnt offering and its grain offering, you will offer them without defect with their libation.}%
\end{biblechapter}%
\begin{biblechapter}% Numbers 29
\verseWithHeading{Offers for the Seventh Month}{“‘On the seventh month, on the first day of the month, you will have a holy convocation; you will not do any regular work.\lnUO{} It will be a day for you of blowing trumpets.}%
\verse{You will offer a burnt offering as a fragrance of appeasement for Adonai: one bull, one ram, and seven male lambs in their first year;\lnUP{} they will be without defect.}%
\verse{Their grain offering will be finely milled flour mixed with oil: three-tenths for the bull, two-tenths for the ram;}%
\verse{and one-tenth for each of the seven male lambs;}%
\verse{with one male goat for a sin offering, to make atonement for you,}%
\verse{in addition to the burnt offering of the new moon and its grain offering, the continual burnt offering and its grain offering, and their libations, according to their stipulations, as a fragrance of appeasement by fire for Adonai.}%
\verseWithHeading{Offerings for the Day of Atonement}{“‘And on the tenth of this seventh month you will have a holy convocation, and you will afflict yourselves;\lebnote{“you will afflict your lives” or “you will afflict your souls”} you will not do any work.}%
\verse{You will present a burnt offering for Adonai, a fragrance of appeasement: one bull, one ram, seven male lambs in their first year;\lnUP{} they will be without defect.}%
\verse{And their grain offering will be of finely milled flour mixed with oil: three-tenths for the bull, two-tenths for the one ram,}%
\verse{one-tenth for each of the seven male lambs;}%
\verse{one male goat for a sin offering, in addition to the sin offering of atonement and the continual burnt offering and its grain offering, and their libations.}%
\verse{“‘Then on the fifteenth day of the seventh month you will have a holy convocation; you will not do any regular work,\lnUO{} and you will hold a religious feast for Adonai for seven days.}%
\verse{You will present a burnt offering, an offering made by fire as a fragrance of appeasement for Adonai: thirteen bulls, two rams, fourteen male lambs in their first year;\lnUP{} they will be without defect.}%
\verse{And their grain offering will be of finely milled flour mixed with oil: three-tenths for the bull, two-tenths for the one ram,}%
\verse{one-tenth for each of the seven male lambs;}%
\verse{and one male goat for a sin offering, in addition to the continual burnt offering, its grain offering, and its libation.}%
\verse{“‘On the second day: twelve bulls, two rams, fourteen male lambs in their first year;\lnUP{} they will be without defect;}%
\verse{and their grain offering and their libations for the bulls, for the rams, and for the male lambs, by their number according to the stipulation;}%
\verse{and one male goat for a sin offering, in addition to the continual burnt offering and its grain offering, and their libations.}%
\verse{“‘On the third day: eleven bulls, two rams, fourteen male lambs without defect in their first year;\lnUP{}}%
\verse{and their grain offering and their libations for the bulls, for the rams, and for the male lambs, by their number according to the stipulation;}%
\verse{and one male goat for a sin offering, in addition to the continual burnt offering, its grain offering, and its libation.}%
\verse{“‘On the fourth day: ten bulls, two rams, fourteen male lambs without defect in their first year;\lnUP{}}%
\verse{and their grain offering and their libations for the bulls, for the rams, and for the male lambs by their number according to the stipulation;}%
\verse{and one male goat for a sin offering, in addition to the continual burnt offering, its grain offering, and its libation.}%
\verse{“‘On the fifth day: nine bulls, two rams, fourteen male lambs without defect in their first year;\lnUP{}}%
\verse{and their grain offering and their libations for the bulls, for the rams, and for the male lambs by their number according to the stipulation;}%
\verse{and one male goat for a sin offering, in addition to the continual burnt offering, its grain offering, and its libation.}%
\verse{“‘On the sixth day: eight bulls, two rams, fourteen male lambs without defect in their first year;\lnUP{} and their grain offering and their libations for the bulls, for the rams, and for the male lambs by their number according to the stipulation;}%
\verse{and their grain offering and their libations for the bulls, for the rams, and for the male lambs by their number according to the stipulation;}%
\verse{and one male goat for a sin offering, in addition to the continual burnt offering, its grain offering, and its libation.}%
\verse{“‘On the seventh day: seven bulls, two rams, fourteen male lambs without defect in their first year;\lnUP{}}%
\verse{and their grain offering and their libations for the bulls, for the rams, and for the male lambs by their number according to the stipulation;}%
\verse{and one male goat for a sin offering, in addition to the continual burnt offering, its grain offering, and its libation.}%
\verse{“‘On the eighth day you will have an assembly; you will not do any regular work.\lnUO{}}%
\verse{You will present a burnt offering, an offering made by fire as a fragrance of appeasement for Adonai: one bull, one ram, seven male lambs without defect in their first year;\lnUP{}}%
\verse{and their grain offering and their libations for the bulls, for the rams, and for the male lambs by their number according to the stipulation;}%
\verse{and one male goat for a sin offering, in addition to the continual burnt offering, its grain offering, and its libation.}%
\verse{“‘You will present these to Adonai at your appointed time, in addition to your vows and your freewill offerings, for your burnt offerings and for you grain offerings and for your libations and for your fellowship offerings.’”}%
\verse{\lebnote{Numbers 29:40–30:16 in the English Bible is 30:1–17 in the English Bible} So Moses said to the Israelites\lebnote{“sons/children of Israel”} in accordance with all that Adonai commanded Moses.}%
\end{biblechapter}%
\begin{biblechapter}% Numbers 30
\verseWithHeading{Laws for Vows}{Then Moses spoke to the leaders\lebnote{“the heads”} of the tribes concerning the Israelites,\lebnote{“sons/children of Israel”} saying, “This is the word that Adonai commanded:}%
\verse{if a man makes a vow for Adonai or swears an oath with a binding pledge on himself, he must not render his word invalid; he must do all that went out from his mouth.}%
\verse{“If a woman makes a vow to Adonai, and she binds a pledge on herself in her father’s house in your childhood,}%
\verse{but if her father hears her vow or her pledge that she bound on herself and says nothing to her, then all her vows will stand, and every pledge that she binds on her life will stand.}%
\verse{If her father forbids her on the day he hears of it, all her vows or her pledges that she bound on herself will not stand, and Adonai will forgive her because her father has forgiven her.}%
\verse{“If she has a husband\lebnote{“she is to a man”} while bound by her vows or a rash promise of her lips,}%
\verse{and her husband hears of it and is silent on the day he hears it, her vows will stand, and her pledge that she bound upon herself will stand.}%
\verse{But if on the day her husband hears of it, he forbids her, then he will nullify her vow that she is under, and the rash promise of her lips that she bound on herself; and Adonai will forgive her.}%
\verse{“But the vow of a widow or a woman who is divorced, all that she binds on herself will stand on her.}%
\verse{But if she made a vow in her husband’s house, or bound herself on a pledge with a sworn oath,}%
\verse{and her husband heard it but was silent to her, and he did not forbid her, all her vows will stand and every pledge that she bound on herself will stand.}%
\verse{But if her husband nullified them on the day he hears them, all her vows going out of her lips concerning her vows or the pledge on herself will not stand; her husband has nullified them, and Adonai will forgive her.}%
\verse{“Any vow and any sworn oath of a pledge to inflict on herself, her husband can confirm it or her husband can nullify it.}%
\verse{But if her husband is completely silent from day to day, then he confirms all her vows or all her pledges that are on her; he confirms them because he was silent to her on the day he heard them.}%
\verse{But if he indeed nullifies them after he hears them, then he will bear her guilt.”}%
\verse{These are the decrees that Adonai commanded Moses, as between a husband and his wife, and between a father and his daughter, while her childhood is in her father’s house.}%
\end{biblechapter}%
\begin{biblechapter}% Numbers 31
\verseWithHeading{War Against the Midianites}{Adonai spoke to Moses, saying,}%
\verse{“Seek vengeance for the Israelites\lnUQ{} on the Midianites; afterward you will be gathered to your people.”}%
\verse{Moses spoke to the people, saying, “Arm yourself from among your men for the battle, so that they will go\lebnote{“be”} against Midian to mete out the vengeance of Adonai on Midian.}%
\verse{A thousand from each tribe of every tribe of Israel you will send to battle.”}%
\verse{So theywere assigned from the thousands of Israel, a thousand from each tribe, twelve thousand equipped for battle.}%
\verse{Moses sent them, a thousand from each tribe, to the battle, and Phinehas son of Eleazar the priest to the battle with them, and the vessels of the sanctuary and the trumpets of the blast were in his hand.}%
\verse{And they fought against Midian just as Adonai commanded Moses, and they killed every male.}%
\verse{They killed the kings of Midian in addition to the ones they had slain: Evi and Rekem and Zur and Hur and Reba, the five kings of Midian; they also killed Balaam son of Beor by the sword.}%
\verse{The Israelites\lnUQ{} took captive the women of Midian and their children, and they plundered all their domestic animals and all their livestock and all their wealth.}%
\verse{They burned all their cities where they dwelled and all their camps with fire.}%
\verse{They took all the plunder and all the war-booty with the humans\lnUR{} and domestic animals.\lnUS{}}%
\verse{They brought the captives, the war-booty, and the plunder to Moses, and to Eleazar the priest, and to the community of the Israelites,\lnUQ{} to the camp to the desert-plateau of Moab, which was on the Jordan across Jericho.}%
\verse{And Moses and Eleazar the priest and all the leaders of the community went out to meet them outside the camp.}%
\verse{But Moses was angry toward the leaders of the troops, the commanders of the thousands and the commanders of the hundreds, who came from the battle of the war.}%
\verse{And Moses said to them, “You have kept alive every female?}%
\verse{Behold, these women caused\lebnote{“were to”} the Israelites,\lnUQ{} by the word of Balaam, to be in apostasy against Adonai in the matter of Peor, so that the plague was among the community of Adonai.}%
\verse{Now kill every male among the little children, and kill every woman who has had sexual intercourse with a man.\lnUT{}}%
\verse{But all the females who have not had sexual intercourse with a man,\lnUT{} keep alive for yourselves.}%
\verse{And you, camp outside the camp seven days; all who killed a person and all who touched the slain purify yourselves on the third day and on the seventh day, you and your captives.}%
\verse{You will purify yourselves and every garment and every object of hide and all the work of goats’ hair, and every object of wood.”}%
\verse{Then Eleazar the priest said to the men of the battle who came from the war, “This is the decree of the law that Adonai commanded Moses.}%
\verse{Only the gold and the silver, the bronze, the iron, the tin, and the lead —}%
\verse{everything that will go through the fire — you will pass through the fire, and it will be clean, and only in waters of impurity will it be purified. Whatever does not go into the fire you will pass through the waters.}%
\verse{And you will wash your garments on the seventh day and be clean, and afterward you will come into the camp.”}%
\verseWithHeading{Division of the War-Booty}{Adonai said to Moses, saying,}%
\verse{“You and Eleazar the priest and the leaders of the families\lebnote{“fathers”} of the community, take count\lebnote{“lift up the head”} of the war-booty that was captured, both humans\lnUR{} and the domestic animals;\lnUS{}}%
\verse{divide the war-booty between those who engaged in the war, who went out to the battle, and all the community.}%
\verse{Exact a tribute for Adonai from the men of the war, those who went out to the battle, one from five hundred persons, and from the cattle and from the male donkeys and from the flock;}%
\verse{take from their half and give it to Eleazar the priest as a contribution to Adonai.}%
\verse{From half of the Israelites,\lnUQ{} take one share drawn by lot from the fifty from the humans,\lnUR{} from the cattle, from the male donkeys, from the flock, from all the domestic animals,\lnUS{} and give them to the Levities who keep the responsibilities of the tabernacle of Adonai.”}%
\verse{Moses and Eleazar the priest did just as Adonai commanded Moses.}%
\verse{Thus the war-booty that remained of the plunder that the people of the battle plundered was six hundred and seventy-five thousand flocks of sheep,}%
\verse{seventy-two thousand cattle,}%
\verse{sixty-one thousand male donkeys,}%
\verse{and the life of humankind, from the women who did not have sexual intercourse with a man,\lnUT{} all the persons\lebnote{Hebrew “person”} were thirty-two thousand.}%
\verse{The half of the share that was going out to the battle: the number of the flock of sheep was thee hundred and thirty-seven thousand five hundred;}%
\verse{the tribute to Adonai from the flock was six hundred and seventy-five;}%
\verse{and the cattle were thirty-six thousand; and the tribute to Adonai was seventy-two.}%
\verse{Of the male donkeys there were thirty thousand five hundred, and the tribute to Adonai was sixty-one;}%
\verse{the humans\lnUR{} were sixteen thousand, and the tribute to Adonai was thirty-two persons.}%
\verse{And Moses gave away the tribute of the contribution of Adonai to Eleazar the priest, just as Adonai commanded Moses.}%
\verse{From the half of the Israelites,\lnUQ{} which Moses divided from the men who were fighting,}%
\verse{the half that belonged to the community was three hundred and thirty-seven thousand five hundred from the flock,}%
\verse{and thirty-six thousand cattle,}%
\verse{and thirty thousand five hundred male donkeys,}%
\verse{and sixteen thousand humans.\lnUR{}}%
\verse{From the half that belonged to the Israelites,\lnUQ{} Moses took one share drawn by lot out of every fifty humans\lnUR{} and domestic animals, and he gave them to the Levites, who keep the responsibility of the tabernacle of Adonai, just as Adonai commanded Moses.}%
\verse{Then the leaders of the thousands of the army, the commanders of the thousands and the commanders of the hundreds, approached Moses,}%
\verse{and they said to Moses, “Your servants have taken count\lebnote{“lifted up the head”} of the men of war who were in our charge,\lebnote{“in our hand”} and no man is missing from us.}%
\verse{So we brought the offering of Adonai, what each man found, objects of gold, bangles,\lebnote{Hebrew “bangle”} bracelets,\lebnote{Hebrew “bracelet”} rings,\lebnote{Hebrew “ring”} earrings,\lebnote{Hebrew “earring”} and female ornaments,\lebnote{Hebrew “ornament”} to make atonement for ourselves before\lebnote{“in the presence of”} Adonai.”}%
\verse{Moses and Eleazar the priest took the gold from them, all objects of work.}%
\verse{All the gold of the contribution that they raised up to Adonai, from the commanders of the thousands and the commanders of the hundreds, was sixteen thousand seven hundred and fifty shekels.}%
\verse{The men of battle plundered each for himself.}%
\verse{So Moses and Eleazar the priest took the gold from the commanders of the thousands and hundreds, and they brought it to the tent of the assembly as a memorial for the Israelites\lnUQ{} before Adonai.}%
\end{biblechapter}%
\begin{biblechapter}% Numbers 32
\verseWithHeading{Gad and Reuben Inherit Gilead}{The descendants\lnUU{} of Reuben and the descendants\lnUU{} of Gad had a very large number of livestock. And they saw the land of Jazer and the land of Gilead, and behold it was a place for livestock.}%
\verse{The descendants\lnUU{} of Gad and the descendants\lnUU{} of Reuben came, and they said to Moses and to Eleazar the priest and to the leaders of the community, saying,}%
\verse{“Ataroth, Dibon, Jazer, Nimrah, Heshbon, Elealeh, Sebam, Nebo, and Beon,}%
\verse{the land that Adonai struck before the community of Israel, is a land of livestock, and your servants have livestock.”}%
\verse{They said, “If we have found favor in your sight,\lebnote{“your eyes”} let this land be given to your servants as property; do not lead us across the Jordan.”}%
\verse{But Moses said to the descendants\lnUU{} of Gad and to the descendants\lnUU{} of Reuben, “Will your brothers go to war while you yourselves live here?}%
\verse{Why are you discouraging the hearts of the Israelites\lnUV{} from crossing into the land that Adonai gave to them?}%
\verse{This is what your fathers did when I sent them from Kadesh Barnea to see the land.}%
\verse{When they went up to the valley\lebnote{Or “the wadi”} of Eshcol and saw the land, they discouraged the heart of the Israelites\lnUV{} so that they did not come to the land that Adonai gave to them.}%
\verse{So Adonai’s anger burned\lnUW{} on that day, and he swore an oath, saying,}%
\verse{‘The men who went up from Egypt, from those twenty years old\lebnote{“a son of twenty years”} and above, will not see the land that I swore with an oath to Abraham, Isaac, and Jacob because they have not wholly followed me,}%
\verse{except Caleb son of Jephunneh the Kenizzite and Joshua son of Nun, because they followed Adonai wholly.’}%
\verse{And Adonai became angry,\lnUW{} and he made them wander in the desert forty years until the entire generation who did evil in the sight of Adonai\lebnote{“eyes of Adonai”} had died.\lebnote{“had completed”}}%
\verse{Behold, you stand in the place of your fathers, a brood of sinful men, to increase still more Adonai’s fierce anger\lebnote{“the fierce anger of Adonai’s nose”} against Israel.}%
\verse{If you turn from following him, he will again abandon them\lebnote{Hebrew “him”} in the wilderness, and you would have destroyed all these people.”}%
\verse{They came near to him and said, “We will build sheep pens here for the flock of our livestock and cities for our little children;}%
\verse{but we ourselves will become armed and ready before the Israelites\lnUV{} until we have brought them to their place, and our little children will live in the fortified cities because of the inhabitants of the land.}%
\verse{We will not return to our houses until the Israelites\lnUV{} each obtain their\lebnote{Hebrew “his”} inheritance for themselves.}%
\verse{For we will not take possession with them from across the Jordan and beyond because our inheritance has come to us from across the Jordan to the east.”}%
\verse{So Moses said to them, “If you do this thing, if you arm yourselves before\lnUX{} Adonai for the war,}%
\verse{and everyone of you armed cross the Jordan before\lnUX{} Adonai until he has driven out his enemies from before him,}%
\verse{and the land is subdued before\lnUX{} Adonai, then afterward you will return and be free of obligation from Adonai and from Israel, and this land will be your property before\lnUX{} Adonai.}%
\verse{But if you do not do so, behold, you have sinned against Adonai, and know that your sin will find you.}%
\verse{Build for yourselves cities for your little children and sheep pens for your flocks; what has gone out from your mouth you will do.”}%
\verse{So the descendants\lnUU{} of Gad and the descendants\lnUU{} of Reuben said to Moses, saying, “Your servants will do just as my lord commands.}%
\verse{Our little children, our wives, our livestock, and all of our animals\lebnote{Hebrew “animal”} will remain\lebnote{“will be there”} in the cities of Gilead,}%
\verse{but your servants, everyone who is armed for battle, will cross over before\lnUX{} Adonai to the war, just as my lord says.”}%
\verse{So Moses commanded them, Eleazar the priest, Joshua son of Nun, and the heads of the families\lebnote{“fathers”} of the tribes of the Israelites.\lnUV{}}%
\verse{Moses said to them, “If the descendants\lnUU{} of Gad and the descendants\lnUU{} of Reuben, everyone who is armed for the war, cross over the Jordan before\lnUX{} Adonai, and the land is subdued before you, you will give them the land of Gilead as property.}%
\verse{But if they will not cross over with you armed, they will acquire land in your midst in Canaan.”}%
\verse{The descendants\lnUU{} of Gad and the descendants of Reuben answered and said, “What Adonai has commanded your servants, we will do.}%
\verse{We ourselves will cross over armed before\lnUX{} Adonai to the land of Canaan, and the property of our inheritance will remain with us beyond the Jordan.”}%
\verse{So Moses gave to them, to the descendants\lnUU{} of Gad and the descendants\lnUU{} of Reuben, and to half of the tribe of Joseph’s son Manasseh, the kingdom of Sihon the king of Amorites and the kingdom of Og the king of the Bashan, the land with its cities and their territories, the cities of the surrounding land.}%
\verse{The descendants\lnUU{} of Gad rebuilt Dibon, Ataroth, and Aroer,}%
\verse{and Atroth Shophan, Jazer, and Jogbehah,}%
\verse{and Beth Nimrah and Beth Haran, the cities of Mibzar, and the sheep pens for flocks.}%
\verse{The descendants\lnUU{} of Reuben rebuilt Heshbon, Elealeh, and Kiriathaim,}%
\verse{and Nebo, Baal Meon (their names\lebnote{Hebrew “name”} were changed), and Sibmah, and they renamed\lebnote{“they called to names the names”} the cities that they rebuilt.}%
\verse{The descendants\lnUU{} of Makir son of Manasseh went to Gilead, and they captured it and drove out the Amorites\lebnote{Hebrew “Amorite”} who were in it.}%
\verse{So Moses gave Gilead to Makir son of Manasseh, and he lived in it.}%
\verse{And Jair son of Manasseh went and captured their unwalled villages, and he called them Havvoth Jair.}%
\verse{Nobah went and captured Kenath and its villages, and he called it Nobah after his own name.}%
\end{biblechapter}%
\begin{biblechapter}% Numbers 33
\verseWithHeading{The Travels Are Recounted}{These were the journeys of the Israelites,\lnUY{} who went out from the land of Egypt according to their divisions, by the hand of Moses and Aaron.}%
\verse{Moses wrote down their movements according to their journeys on the command of Adonai, and these are their journeys according to their movements.}%
\verse{They set out from Rameses on the first month, on the fifteenth day of the first month; on the next day after the Passover the Israelites\lnUY{} went out boldly\lebnote{“with a hand that was raised”} in the sight\lebnote{“for the eyes”} of all the Egyptians}%
\verse{while the Egyptians were burying all the firstborn among them whom Adonai struck. Adonai also executed punishments among their gods.}%
\verse{Then the Israelites\lnUY{} set out from Rameses, and they camped in Succoth.}%
\verse{They journeyed from Succoth and camped in Etham, which is on the edge of the desert.}%
\verse{Then they set out from Etham and returned to Pi-Hahiroth, which faces Baal Zephon, and they camped before Migdol.}%
\verse{They set out from Pi-Hahiroth and went through the midst of the sea into the desert; and they went a journey of three days into the desert of Etham and camped at Marah.}%
\verse{They set out from Marah and came to Elim, and in Elim there were twelve springs of water and seventy palm trees, and they camped there.}%
\verse{They set out from Elim, and they camped at the Red Sea.\lnUZ{}}%
\verse{They set out from the Red Sea\lnUZ{} and camped at the desert of Sin.}%
\verse{They set out from the desert of Sin and camped at Dophkah.}%
\verse{They set out from Dophkah and camped at Alush.}%
\verse{They set out from Alush and encamped at Rephidim; and it was there that the people had no water to drink.}%
\verse{They set out from Rephidim and camped in the desert of Sinai.}%
\verse{The set out from the desert of Sinai and camped at Kibroth Hattaavah.}%
\verse{They set out from Kibroth Hattaavah and camped at Hazeroth.}%
\verse{They set out from Hazeroth and camped at Rithmah.}%
\verse{They set out from Rithmah and camped at Rimmon Perez.}%
\verse{They set out from Rimmon Perez and camped at Libnah.}%
\verse{They set out from Libnah and camped at Rissah.}%
\verse{They set out from Rissah and camped at Kehelathah.}%
\verse{They set out from Kehelathah and camped at Mount Shapher.}%
\verse{They set out from Mount Shapher and camped at Haradah.}%
\verse{They set out from Haradah and camped at Makheloth.}%
\verse{They set out from Makheloth and camped at Tahath.}%
\verse{They set out from Tahath and camped at Terah.}%
\verse{They set out from Terah and camped at Mithcah.}%
\verse{They set out from Mithcah and camped at Hashmonah.}%
\verse{They set out from Hashmonah and camped at Moserah.}%
\verse{They set out from Moserah and camped at Bene-Jaakan.}%
\verse{They set out from Bene-Jaakan and camped at Hor Haggidgad.}%
\verse{They set out from Hor Haggidgad and camped at Jotbathah.}%
\verse{They set out from Jotbathah and camped at Abronah.}%
\verse{They set out from Abronah and camped at Ezion Geber.}%
\verse{They set out from Ezion Geber and camped in the desert of Zin, that is, Kadesh.}%
\verse{They set out from Kadesh and camped at Mount Hor, at the edge of the land of Edom.}%
\verse{Aaron the priest went up to Mount Hor at the command\lebnote{“mouth”} of Adonai, and he died there in the fortieth year after the Israelites\lnUY{} had gone out from the land of Egypt, in the fifth month on the first day of the month.}%
\verse{Aaron was one hundred and twenty-three years old when he died on Mount Hor.}%
\verse{Now the Canaanite, the king of Arad, who was living in the Negev\lebnote{An arid region south of the Judean hills} in the land of Canaan, heard of the coming of the Israelites.\lnUY{}}%
\verse{Then they set out from Mount Hor and camped at Zalmonah.}%
\verse{They set out from Zalmonah and camped at Punon.}%
\verse{They set out from Punon and camped at Oboth.}%
\verse{They set out from Oboth and camped at Iye Abarim, the boundary of Moab.}%
\verse{They set out from Iyim and camped at Dibon Gad.}%
\verse{They set out from Dibon Gad and camped at Almon-Diblatayim.}%
\verse{They set out from Almon-Diblatayim and camped in the mountains of Abarim, before Nebo.}%
\verse{They set out from the mountains of Abarim and camped on the desert-plateau of Moab by the Jordan across Jericho.}%
\verse{They camped by the Jordan, from Beth-Jeshimoth up to Abel Shittim, on the desert-plateau of Moab.}%
\verse{Then Adonai spoke to Moses on the desert-plateau of Moab by the Jordan across Jericho, saying,}%
\verse{“Speak to the Israelites\lnUY{} and say to them, ‘When you cross the Jordan into the land of Canaan,}%
\verse{you will drive out the inhabitants of the land from your presence, and you will destroy all their idols and all the images of their molten idols, and you will demolish all their high places;}%
\verse{you will dispossess the land and live in it because I have given the land to you to possess it.}%
\verse{You will distribute the land by lot according to your clans; to the larger group you will give a larger inheritance, and to the smaller group you will give less inheritance. However the lot falls for him, there the lot will be. You will distribute it according to the tribes of your ancestors.\lebnote{Or “fathers”}}%
\verse{But if you do not drive out the inhabitants of the land from your presence, then it will happen that whomever you let remain of them will be like irritants in your eyes and like thorns in your sides; they will be your enemies in the land in which you live.}%
\verse{And just as I planned to do to them, I will do to you.’”}%
\end{biblechapter}%
\begin{biblechapter}% Numbers 34
\verseWithHeading{The Land of Canaan Is Divided}{Then Adonai spoke to Moses, saying,}%
\verse{“Command the Israelites\lnVA{} and say to them, ‘When you come into the land of Canaan, this is the land that was allotted to you as an inheritance, the land of Canaan according to its boundaries.}%
\verse{Your southern edge will be from the desert of Zin toward the side of Edom, and your southern border will be from the end of the Salt Sea\lnVB{} to the east;}%
\verse{your boundary will turn from the south to the ascent of Akrabbim\lebnote{Or “Scorpions”} and will pass over to Zin, and its limits will be from the south of Kadesh Barnea; it will continue to Hazar Addar and pass over to Azmon.}%
\verse{The boundary will turn from Azmon to the valley of Egypt, and its limits will be to the sea.}%
\verse{“‘Your western boundary will be the Great Sea;\lnVC{} this will be your western boundary.}%
\verse{Your northern border will be from the Great Sea;\lnVC{} you will make a boundary from the Great Sea\lnVC{} to Mount Hor.}%
\verse{From Mount Hor you will make a boundary to reach Hamath; the limits of the territory will be at Zedad.}%
\verse{The boundary will go out to Ziphron, and its limits will be at Hazar Enan. This will be your boundary to the north.}%
\verse{“‘You will mark out your eastern boundary from Hazar Enan to Shepham;}%
\verse{the boundary will go down from Shepham to Riblah from the east side of Ain, and the boundary will go down and border on the eastern side of the Sea of Kinnereth.}%
\verse{The boundary will go down to the Jordan, and its limits will be at the Salt Sea.\lnVB{} This will be your land according to its boundaries all around.’”}%
\verse{So Moses commanded the Israelites,\lnVA{} saying, “This is the land that you will obtain as an inheritance for yourself by lot, which Adonai commanded to give to the nine and a half tribes.}%
\verse{For the tribe of the children of the Reubenites,\lebnote{Hebrew “Reubenite”} the children of the Gadites,\lebnote{Hebrew “Gadite”} and the half-tribe of Manasseh took their inheritance according to the house of their families.\lebnote{Or “their fathers”}}%
\verse{The two and a half tribes have taken their inheritance from beyond the Jordan across Jericho, east toward the sunrise.”}%
\verse{Adonai spoke to Moses, saying,}%
\verse{“These are the names of the men who divide up the land for your inheritance: Eleazar the priest and Joshua son of Nun.}%
\verse{You will take one leader from every tribe to divide up the land for inheritance.}%
\verse{These are the names of the men: of the tribe of Judah, Caleb son of Jephunneh;}%
\verse{of the tribe of the descendants of Simeon, Samuel son of Ammihud;}%
\verse{of the tribe of Benjamin, Elidad son of Chislon;}%
\verse{of the tribe of the descendants\lnVD{} of Dan, the leader Bukki son of Jogli;}%
\verse{of the descendants\lnVD{} of Joseph, the tribe of the descendants\lnVD{} of Manasseh, the leader Hanniel son of Ephod.}%
\verse{Of the tribe of the descendants\lnVD{} of Ephraim, the leader Kemuel son of Shiphtan;}%
\verse{of the tribe of the descendants\lnVD{} of Zebulum, the leader Elizaphan son of Parnach;}%
\verse{of the tribe of the descendants\lnVD{} of Issachar, the leader Paltiel son of Azzan;}%
\verse{of the tribe of the descendants\lnVD{} of Asher, the leader Ahihud son of Shelomi;}%
\verse{of the tribe of the descendants\lnVD{} of Naphtali, the leader Pedahel son of Ammihud.”}%
\verse{These are those whom Adonai commanded to allot to the Israelites\lnVA{} the land of Canaan.}%
\end{biblechapter}%
\begin{biblechapter}% Numbers 35
\verseWithHeading{Cities for the Levites}{Adonai spoke to Moses on the desert plains of Moab beyond the Jordan across Jericho, saying,}%
\verse{“Command the Israelites\lnVE{} that they give to the Levites from the inheritance of their property cities to live in; and you will give to the Levites pastureland all around the cities.\lebnote{Hebrew “them”}}%
\verse{The cities will be theirs to live in, and their pasturelands will be for their domestic animals, for their possessions, and their animals.\lebnote{Hebrew “animal”}}%
\verse{“The pasturelands of the cities that you will give to the Levites will extend from the wall of the city to a distance of a thousand cubits all around.}%
\verse{You will measure outside the city the eastern edge two thousand cubits, for the southern edge two thousand cubits, for the western edge two thousand cubits, and for the northern edge two thousand cubits, with the city in the middle; this will be for them the pasturelands of the cities.}%
\verse{“All the cities that you will give the Levites will be six cities of refuge, to which the killer can flee; in addition to them you will give forty-two cities.}%
\verse{All the cities that you will give to the Levites will be forty-eight cities, them with their pasturelands.}%
\verse{And the cities that you will give from the property of the Israelites,\lnVE{} you will take more from the larger group and less from the smaller group; each of them will give according to the portion of their inheritance according to the portion that he inherits.”}%
\verseWithHeading{Cities of Refuge}{Adonai spoke to Moses, saying,}%
\verse{“Speak to the Israelites\lnVE{} and say to them, ‘When you cross the Jordan into the land of Canaan,}%
\verse{you will select for yourselves cities for your cities of refuge, that a killer who has killed a person unintentionally can flee there.}%
\verse{The cities will be to you a refuge from a redeemer, so that the killer will not die until he stands before the community for judgment.}%
\verse{The cities that you are to give will be your six cities of refuge.}%
\verse{You will give three cities across the Jordan and three cities in the land of Canaan; they will be cities of refuge.}%
\verse{To the Israelites,\lnVE{} to the alien, and to the temporary resident in their midst there will be these six cities as a refuge to which anyone who unintentionally kills a person may flee.}%
\verse{“‘But if he hit him with an object of iron, so that he dies, the killer must surely be put to death.}%
\verse{And if he hit him with a stone in the hand, by which he will die, and he does die, he is a killer; the killer must surely be put to death.}%
\verse{Or if he hit him with a wooden object, by which he will die, and he does die, he is a killer; the killer must surely be put to death.}%
\verse{The blood avenger himself will put the killer to death; he must put him to death when meeting him.}%
\verse{If he shoves him in hatred, or he throws something at him with intention, and he dies,}%
\verse{or if he hits him in hostility with his hand, and he dies, the one that struck him will put to death the killer when meeting him.}%
\verse{“‘Or if in an instant he shoved him, not in hostility, or threw something at him without intention,}%
\verse{or with any stone, without seeing it dropped on him so that he dies, while he was not seeking his injury,}%
\verse{then the community will judge between the striker and between the blood avenger according to these ordinances.}%
\verse{The community will deliver the killer from the hand of the blood avenger, and the community will restore him to the city of his refuge to which he fled; and he will live there in it until the death of the high priest who was anointed with holy oil.}%
\verse{But if the killer surely goes out of the territory of the city of his refuge to which he fled,}%
\verse{and the blood avenger finds him outside the territory of the city of his refuge, and the blood avenger kills the killer, he will not be guilty of blood\lebnote{“there will not be blood for him”}}%
\verse{because he must live in the city of his refuge until the death of the high priest. But after the death of the high priest the killer will return to the land of his property.}%
\verse{These things will be as a decree of justice for you for your generations in all your dwellings.}%
\verse{“‘If anyone kills a person, the killer will be put to death according to the testimony\lebnote{“according to the mouth”} of witnesses, but someone cannot die on testimony of one person.}%
\verse{Also, you will not take a ransom payment for the life of a killer who is guilty of death; indeed, he must surely be put to death.}%
\verse{You will not take a ransom payment for the one that flees to the city of his refuge, so that he may return to live in the land before the death of the priest.}%
\verse{So you will not pollute the land in which you are; because blood pollutes the land, and no atonement can be made for the land for the blood that is poured out on it except with the blood of the one who poured it out.}%
\verse{You will not defile the land on which you are living because I am living in the midst of it; I am Adonai; I am living in the midst of the Israelites.’”\lnVE{}}%
\end{biblechapter}%
\begin{biblechapter}% Numbers 36
\verseWithHeading{Inheritance Through Marriage}{The leaders\lebnote{“The heads”} of the families\lnVF{} of the clans of descendants\lnVG{} of Gilead the son of Makir, the son of Manasseh, of the clans of the descendants\lnVG{} of Joseph came near and spoke before\lnVH{} Moses and before\lnVH{} the leaders\lebnote{“the heads”} of the families\lebnote{Hebrew “of the fathers”} of the Israelites.\lnVI{}}%
\verse{And they said, “Adonai commanded my lord to give the land by lot as an inheritance to the Israelites,\lnVI{} and my lord was commanded by Adonai to give the inheritance of Zelophehad our brother to his daughters.}%
\verse{But if they become wives to one of the sons from another tribe of the Israelites,\lnVI{} their inheritance will disappear from the inheritance of our ancestors,\lnVF{} and it will be added to the inheritance of the tribe to which they belong; the lot of our inheritance would disappear.}%
\verse{When the Jubilee of the Israelites\lnVI{} comes,\lebnote{“will be”} it will be added to the inheritance of the tribe to which they belong; and their inheritance will disappear from the tribe of our father.”}%
\verse{Then Moses commanded the Israelites\lnVI{} by the command of Adonai, saying, “The tribe of the descendants\lnVG{} of Joseph is right regarding what they are speaking.}%
\verse{This is the word that Adonai commanded the daughters of Zelophehad, saying, ‘Let them marry\lebnote{“Let them be as wives”} whomever they like;\lebnote{“they will be as the good in their eyes”} only they must marry\lebnote{“they must be as wives”} from within the clan of the tribe of their father.}%
\verse{Thus an inheritance of the Israelites\lnVI{} will not go around from tribe to tribe. Rather, the inheritance of each tribe of his father will remain with the Israelites.\lnVI{}}%
\verse{Every daughter who possesses an inheritance from the tribes of the Israelites\lnVI{} will marry\lebnote{“be a wife to”} one of the clan of the tribe of her father, so that the Israelites\lnVI{} will possess the inheritance of his ancestors.\lnVF{}}%
\verse{Therefore an inheritance will not go around from one tribe to another tribe because the tribes of the Israelites\lnVI{} will each hold to their own inheritance.’”}%
\verse{Just as Adonai commanded to Moses, so the daughters of Zelophehad did:}%
\verse{Mahlah, Tirzah, Hoglah, Milcah, and Noah, the daughters of Zelophehad, married\lebnote{“were as wives to”} the sons of their uncles.}%
\verse{They married\lebnote{“They were as wives to”} those from the sons of Manasseh son of Joseph, and their inheritance remained\lebnote{“was”} among the tribe of the clan of their ancestors.\lnVF{}}%
\verse{These were the commands and the stipulations that Adonai commanded by the hand of Moses\lebnote{Or “through Moses”} to the Israelites\lnVI{} on the desert-plateaus of Moab by the Jordan across Jericho.}%
\end{biblechapter}%
\flushcolsend
\input{leb/content/old-testament/Deu.tex}\flushcolsend
\biblebook{Joshua}
\begin{biblechapter}% Joshua 1
\verseWithHeading{Joshua Addresses the Israelites}{After the death of Moses the servant of Adonai, Adonai said to Joshua son of Nun, the assistant\lebnote{Or “servant”} of Moses, saying,}%
\verse{“My servant Moses is dead. Get up and cross the Jordan, you and all this people, into the land that I am giving to them, to the Israelites.\lebnote{“sons/children of Israel”}}%
\verse{Every place that the soles of your feet will tread, I have given it to you, as I promised\lebnote{Or “spoke”} to Moses.}%
\verse{From the wilderness and the Lebanon,\lebnote{“this Lebanon”; “Lebanon” in Hebrew means “white mountain”} up to the great river, the river Euphrates, all of the land of the Hittites, and up to the great sea in the west,\lebnote{“the great sea of the setting sun”} will be your territory.}%
\verse{No one will stand before you\lebnote{Or “in your presence”} all the days of your life. Just as I was with Moses, so will I be with you; I will not fail you, and I will not forsake you.}%
\verse{Be strong and courageous, for you will give the people this land as an inheritance that I swore to their ancestors\lebnote{Or “fathers”} to give them.}%
\verse{Only be strong and very courageous to observe diligently the whole law\lebnote{“to keep \textit{and} to act according to the whole law”} that Moses my servant commanded you. Do not turn aside from it, to the right or left, so that you may succeed wherever you go.\lnBAA{}}%
\verse{The scroll of this law will not depart from your mouth; you will meditate on it day and night so that you may observe diligently all that is written\lebnote{“to keep \textit{and} to act according to all that is written”} in it. For then you will succeed in your ways and prosper.}%
\verse{Did I not command you? Be strong and courageous! Do not fear or be dismayed, for Adonai your God is with you wherever you go.”\lnBAA{}}%
\verse{Then Joshua commanded the officers of the people, saying,}%
\verse{“Pass through the midst of the camp and command the people: ‘Prepare your provisions;\lebnote{Hebrew “provision”} in\lebnote{“for in still”} three days you are to cross the Jordan to go possess the land that Adonai your God is giving you to possess.’”}%
\verse{To the Reubenites, the Gadites, and the half-tribe of Manasseh Joshua said,}%
\verse{“Remember the word that Moses Adonai’s servant commanded you, saying, ‘Adonai your God is giving rest to you, and he is giving\lebnote{Hebrew “he gave”} you this land.’}%
\verse{Your wives, your little children, and your livestock, they will remain in the land that Moses gave to you beyond the Jordan. All of the best fighting men\lebnote{“All the mighty warriors of the troop”} will cross armed in front of your families; they will help you}%
\verse{until Adonai gives rest to your families as well as to you. They will take possession of the land that Adonai your God is giving to them. Then you will return to your own land and take possession of it, the land that Moses the servant of Adonai gave you beyond the Jordan to the east.”\lebnote{“to the sunrise”}}%
\verse{And they answered Joshua, saying, “All that you have commanded us we will do, and wherever you will send us we will go.}%
\verse{Just as we obeyed Moses, so will we obey you. Only may Adonai your God be with you, as he was with Moses.}%
\verse{Whoever rebels against your orders\lebnote{“defiles against your mouth”} and does not obey your words according to what you commanded us will be put to death. Only be strong and courageous.”}%
\end{biblechapter}%
\begin{biblechapter}% Joshua 2
\verseWithHeading{Spies View the Land}{Then Joshua son of Nun secretly sent two men from Acacia Grove\lebnote{Hebrew “Shittim”} as spies, saying, “Go, view the land, especially Jericho.” So they went, and entered the house of a prostitute whose name was Rahab, and spent the night\lebnote{“lay down”} there.}%
\verse{The king of Jericho was told, “Look, some men from the Israelites\lebnote{“sons/children of Israel”} have come here tonight to search out the land.”}%
\verse{And the king of Jericho sent for Rahab, saying, “Bring out the men who came to you, the ones who have entered your house, for they have come to search out the whole land.”}%
\verse{But the woman took the two men and hid them. And she said, “Yes, the men came to me, but I did not know where they were from.}%
\verse{And when it was time to shut the gate for the night,\lebnote{“in the darkness” or “at dark”} the men left, and I do not know where they went. Chase after them quickly, for you may catch up to them.”}%
\verse{(But she had taken them to the roof and had hidden them in the stalks of flax\lebnote{“in the flaxes of the plant stalk”} that she had spread out\lebnote{“had been arranged”} on the roof.)}%
\verse{So the men chased after them on the way to the Jordan\lebnote{“the road of the Jordan”} at the fords; and they shut the gate behind the pursuers that had gone out after them.}%
\verse{Before they went to sleep,\lebnote{“they lay down”} she came up to them on the roof}%
\verse{and said to the men, “I know that Adonai has given you the land, and that dread of you has fallen on us, and that all the inhabitants of the land melt away in fear because of your presence.}%
\verse{For we have heard how Adonai dried up the waters of the Red Sea\lebnote{“sea of reeds”} before you when you went out from Egypt, and what you did to the two kings of the Amorites that were beyond the Jordan, Sihon and Og, whom you utterly destroyed.}%
\verse{We heard this, and our hearts\lebnote{Hebrew “heart”} melted, and no courage was left in anyone\lebnote{“a spirit no longer stood in anyone”} because of your presence. For Adonai your God is God in the heavens above and on the earth below.}%
\verse{So then please swear to me by Adonai, because I have shown loyalty to you,\lebnote{“I have done with you a loyal love”} and you will also show loyalty\lebnote{“you will do … loyal love”} to my family.\lebnote{“with the house of my father”} You must give me a sign of good faith,\lebnote{Or “faithfulness”}}%
\verse{and you will spare my father and mother, my brothers and sisters, and all that belongs to them; you will deliver our lives from death.”}%
\verse{And the men said to her, “Our lives for yours.\lebnote{“Our lives in place of yours to die”} If you do not report this business of ours, we will show you loyalty and faithfulness\lebnote{“we will do with you loyal love and faithfulness”} when Adonai gives us the land.”}%
\verse{Then she lowered them with a rope through the window, as her house was on the outer side of the wall, and she was residing in the wall.}%
\verse{And she said to them, “Go to the mountain, so that the pursuers will not find you, and hide yourselves there three days until the pursuers return, and afterward you may go on your way.”\lebnote{“to your way”}}%
\verse{The men said to her, “We will be released from this oath of yours that you made us swear.}%
\verse{When we come to the land, you must tie this scarlet cord in the window through which you let us down, and you must gather your father and mother, and your brothers, and your whole family to your house.}%
\verse{If anyone goes outside the doors of your house, they will be responsible for their own death,\lebnote{“his blood \textit{will be} on his head”} and we will be innocent. Anyone who will be with you in the house, we will be responsible for their death\lebnote{“his blood \textit{will be} on our head”} if a hand is laid on them.\lebnote{“if a hand will be against him”}}%
\verse{But if you report this business of ours, we will be released from your oath that you made us swear.”}%
\verse{And she said, “According to your word it will be.” Then she sent them away, and they went, and she tied the scarlet cord in the window.}%
\verse{They departed and came to the mountain, and they stayed there three days until the pursuers returned. The pursuers searched all along the way\lebnote{“on all of the road”} but did not find them.}%
\verse{The two men returned and went down from the mountain, and they crossed over and came to Joshua son of Nun, and they told him everything that happened to them.}%
\verse{They said to Joshua, “Surely Adonai has given all the land into our hand; also, all the inhabitants of the land melt away in fear because of our presence.”}%
\end{biblechapter}%
\begin{biblechapter}% Joshua 3
\verseWithHeading{The Israelites Cross the Jordan}{Joshua rose early in the morning, and they set out from Acacia Grove.\lebnote{Hebrew “Shittim”} And they came up to the Jordan, he and all the Israelites,\lnBAB{} and they spent the night there before they crossed over.}%
\verse{At the end of the three days the officers passed through the midst of the camp,}%
\verse{and they commanded the people: “When you see the Levitical priests carrying the ark of the covenant of Adonai your God you must set out from your place and go after it.}%
\verse{But there will be a distance between you and it of about two thousand cubits in measurement.\lebnote{That is, about 1 km} Do not come near it, so that you may know the way that you must go, for you have not passed on this way before.”\lebnote{“from yesterday three days before”}}%
\verse{And Joshua said to the people, “Sanctify yourselves, because tomorrow Adonai will do wonders in your midst.”}%
\verse{And Joshua said to the priests, “Take up the ark of the covenant and cross over ahead of the people.”\lnBAC{} And they took up the ark of the covenant and went ahead of the people.\lnBAC{}}%
\verse{Then Adonai said to Joshua, “This day I will begin exalting you in the sight\lebnote{“in the eyes”} of all Israel, that they may know that I was with Moses, and I will be with you.}%
\verse{You will command the priests carrying the ark of the covenant, saying, ‘At the moment that you come to the edge of the waters of the Jordan, you will stand still in the Jordan.’”}%
\verse{And Joshua said to the Israelites,\lnBAB{} “Come here, and hear the words of Adonai your God.”}%
\verse{Joshua said, “By this you will know that the living God is in your midst, and he will certainly drive out the Canaanites from before you,\lebnote{Or “from your presence”} and the Hittites, Hivites, the Perizzites, the Girgashites, the Amorites, and the Jebusites.}%
\verse{Look! The ark of the covenant of the Lord of all the earth\lnBAD{} is about to cross over ahead of you into the Jordan.}%
\verse{So then, take twelve men from the tribes of Israel, one from each tribe.\lebnote{“one man for the tribe”}}%
\verse{When the soles of the feet of the priests carrying the ark of Adonai, Lord of all the earth,\lnBAD{} rest in the waters of the Jordan, the waters of the Jordan will be cut off upstream,\lebnote{“going down from above”} and they will stand still in one heap.}%
\verse{And it happened, when the people set out from their tents to cross over the Jordan, the priests carrying the ark of the covenant were ahead of the people.\lnBAC{}}%
\verse{When those carrying the ark came up to the Jordan, and the priests carrying the ark dipped their feet in the edge of the water (the Jordan was flowing over its banks during all the days of harvest),}%
\verse{the waters flowing down from above stood still; they stood up in one heap very far from Adam, the city that is beside Zarethan, while the waters flowing down to the sea of the Arabah, the Salt Sea,\lebnote{That is, the Dead Sea} were completely cut off;\lebnote{“they were completed they were cut off”} and the people crossed over opposite Jericho.}%
\verse{And the priests carrying the ark of the covenant of Adonai stood firmly on the dry land in the middle of the Jordan while all Israel crossed on dry ground, until all the nation finished crossing the Jordan.}%
\end{biblechapter}%
\begin{biblechapter}% Joshua 4
\verseWithHeading{The Israelites Make a Memorial}{After all the nation finished crossing the Jordan, Adonai said to Joshua,}%
\verse{“Take twelve men from the people, one man from each tribe,\lebnote{“man one man one from tribe”}}%
\verse{and command them, saying, ‘Take for yourselves twelve stones from the middle of the Jordan where the priests’ feet stood firmly, and bring them over with you, and set them up in the place where you will camp tonight.’”\lebnote{“the lodging place where you will lodge tonight”}}%
\verse{So Joshua summoned the twelve men whom he had appointed from the Israelites,\lnBAE{} one from each tribe.}%
\verse{And Joshua said to them, “Cross over before\lebnote{Or “the presence”} the ark of Adonai your God to the middle of the Jordan, and each one of you lift up a stone on your\lebnote{Hebrew “his”} shoulder, according to the number of the tribes of the Israelites,\lnBAE{}}%
\verse{so that this may be a reminder\lebnote{Or “sign”} among you. When your children ask in the future, saying, ‘What do these stones mean to you?’\lebnote{“What are these stones for you?”}}%
\verse{you will say to them that the waters of the Jordan were cut off from before\lebnote{Or “the presence of”} the ark of the covenant of Adonai. When it crossed the Jordan, the waters of the Jordan were cut off. These stones will be as a memorial for the Israelites\lnBAE{} for eternity.”}%
\verse{Thus the Israelites\lnBAE{} did as Joshua commanded. They took twelve stones from the middle of the Jordan as Adonai told Joshua, according to the number of the tribes of the Israelites,\lnBAE{} and they carried them over with them to the camp,\lebnote{“the place of overnight lodging”} and they put them there.}%
\verse{Then Joshua set up twelve stones in the middle of the Jordan, in the place where the feet of the priests carrying the ark of the covenant stood, and they are there to this day.}%
\verse{The priests carrying the ark remained standing in the middle of the Jordan until everything that Adonai commanded Joshua to tell the people was finished, according to all that Moses commanded Joshua. And the people hastily crossed over.}%
\verse{And it happened, when all the people had finished crossing, the ark of Adonai and the priests crossed over in front of\lebnote{Or “before the presence”} the people.}%
\verse{The children of Reuben, Gad, and the half-tribe of Manasseh crossed over armed before the Israelites,\lnBAE{} as Moses told them.}%
\verse{About forty thousand armed for fighting crossed over before the presence of Adonai to the plains of Jericho for battle.}%
\verse{On that day Adonai exalted Joshua in the sight\lebnote{“in the eyes”} of all Israel, and they respected him\lebnote{Or “feared him”} as they respected\lebnote{Or “feared”} Moses all the days of his life.}%
\verse{Then Adonai said to Joshua, saying,}%
\verse{“Command the priests carrying the ark of the testimony\lebnote{That is, the ark of the covenant} to come up\lebnote{“and let them come up”} from the Jordan.”}%
\verse{So Joshua commanded the priests, saying, “Come up from the Jordan.”}%
\verse{And it happened that when the priests carrying the ark came up from the middle of the Jordan, and the soles of the priests’ feet touched dry land,\lebnote{“were raised from the ground to dry land”} the waters of the Jordan returned to their place and flowed over all its banks as before.}%
\verse{And the people came up from the Jordan on the tenth day of the first month, and they camped in Gilgal on the eastern edge\lebnote{That is, border} of Jericho.}%
\verse{And those twelve stones that they took from the Jordan, Joshua set up in Gilgal.}%
\verse{And he said to the Israelites,\lnBAE{} “When your children ask in the future their parents,\lebnote{“their fathers”} ‘What is the meaning of these stones?’\lebnote{“What \textit{are} these stones?”}}%
\verse{you will let your children know by saying, ‘Israel crossed this Jordan on dry ground.’}%
\verse{For Adonai your God dried up the waters of the Jordan before you, until you had crossed, just as Adonai your God did to the Red Sea,\lebnote{“sea of reeds”} which he dried up before us until we had crossed over,}%
\verse{so that all the peoples of the earth may know that the hand of Adonai is strong, so that you may fear Adonai your God forever.”\lebnote{“all the days”}}%
\end{biblechapter}%
\begin{biblechapter}% Joshua 5
\verseWithHeading{The Israelites Are Circumcised}{And it happened, when all the kings of the Amorites who were beyond the Jordan to the west, and all the kings of the Canaanites\lebnote{Hebrew “Canaanite”} who were by the sea heard that Adonai dried up the waters of the Jordan in front of the Israelites\lnBAF{} until they crossed over, their hearts melted, and there was no courage left in them\lebnote{“a spirit was no longer in them”} because of the presence of the Israelites.\lnBAF{}}%
\verse{At that time Adonai said to Joshua, “Make for yourself knives of flint, and circumcise the Israelites\lnBAF{} a second time.”}%
\verse{So Joshua made knives of flint, and he circumcised the Israelites\lnBAF{} at the hill of the foreskins.\lebnote{Hebrew “Gibeath-haaraloth”}}%
\verse{This is the reason why Joshua circumcised all the people: all the males who went out from Egypt, all the warriors, died in the wilderness as they went out from Egypt on the journey.\lnBAG{}}%
\verse{For all the people who left were circumcised, but all the people born in the wilderness on the journey\lnBAG{} after they left from Egypt were not circumcised.}%
\verse{For forty years the Israelites\lnBAF{} traveled in the wilderness until all the nation, the warriors that left Egypt, perished, because they did not listen to the voice of Adonai. To them Adonai swore that they would not see the land that he\lebnote{Hebrew “Adonai”} swore to their ancestors\lebnote{Or “fathers”} to give to us, a land flowing with milk and honey.}%
\verse{And it was their children whom he raised in their place that Joshua circumcised, for they were uncircumcised, because they had not been circumcised on the journey.\lnBAG{}}%
\verseWithHeading{The Israelites Celebrate Passover in Canaan}{And it happened, when all the nation had finished circumcising, they remained where they were in the camp until they recovered.}%
\verse{And Adonai said to Joshua, “Today I have rolled away the disgrace of Egypt from you.” Therefore, the name of that place is called Gilgal\lebnote{Hebrew “rolling”} to this day.}%
\verse{And the Israelites\lnBAF{} camped at Gilgal, and they kept the Passover on the fourteenth day of the month, in the evening, on the plains of Jericho.}%
\verse{On the next day after the Passover, on that\lebnote{“this”} very day, they ate from the produce of the land, unleavened cakes and roasted corn.}%
\verse{And the manna ceased the day after, when they started eating the produce of the land, and there was no longer manna for the Israelites.\lnBAF{} They ate from the crop of the land of Canaan in that year.}%
\verseWithHeading{The Commander of Adonai’s Army Appears Before Joshua}{And it happened, when Joshua was by Jericho, he looked up,\lebnote{“he lifted up his eyes”} and he saw a man standing opposite him\lebnote{“against him”} with his sword drawn in his hand. And Joshua went to him and said, “Are you with us, or with our adversaries?”}%
\verse{And he said, “Neither. I have come now as the commander of Adonai’s army.” And Joshua fell on his face to the earth, and he bowed down\lebnote{Or “he worshiped”} and said to him, “What is my lord commanding his servant?”}%
\verse{The commander of Adonai’s army said to Joshua, “Take off your sandals\lebnote{Hebrew “sandal”} from your feet,\lebnote{Hebrew “foot”} for the place where you are standing is holy.” And Joshua did so.}%
\end{biblechapter}%
\begin{biblechapter}% Joshua 6
\verseWithHeading{The Battle of Jericho}{Now Jericho was shut up inside and out because of the presence of the Israelites;\lebnote{“sons/children of Israel”} no one was going out or coming in.}%
\verse{And Adonai said to Joshua, “Look, I am giving Jericho into your hand, its king and the soldiers of the army.}%
\verse{You will march around the city, all the warriors circling the city once; you will do so for six days.}%
\verse{And seven priests will bear seven trumpets of rams’ horns before the ark. On the seventh day you will march around the city seven times, and the priests will blow on the trumpets.}%
\verse{And when they blow long on the horn of the ram, when you hear the sound of the trumpet, all the people will shout with a great war cry, and the wall of the city will fall flat,\lebnote{Or “in its place”} and the people will charge, each one straight ahead.”\lebnote{“the people will go up, each before him”}}%
\verse{So Joshua son of Nun summoned the priests and said to them, “Take up the ark of the covenant, and let seven priests carry the trumpets of the rams’ horns before the ark of Adonai.”}%
\verse{And he said\lebnote{Hebrew “they said”} to the people, “Go forward and march around the city, and let the armed men pass before the ark of Adonai.}%
\verse{And when Joshua spoke\lebnote{“And it happened the moment of the saying of Joshua”} to the people, the seven priests carrying the seven trumpets of rams’ horns before the presence of Adonai went forward and they blew the trumpets; and the ark of the covenant of Adonai followed behind them.}%
\verse{And the armed men went before the priests who blew the trumpets, and the rear guard came after the ark, while they were blowing the trumpets.}%
\verse{But Joshua commanded the people, saying, “You will not shout, and you will not let your voice be heard; a word will not go out from your mouth until the day I say to you ‘Shout!’ Then you will shout.”}%
\verse{And the ark of Adonai went around the city, circling once,\lebnote{“circling one occurrence”} and they came into the camp and spent the night in the camp.}%
\verse{Then Joshua got up early in the morning, and the priests took up the ark of Adonai.}%
\verse{The seven priests carrying the seven trumpets of the rams’ horns before the ark of Adonai went on continually, and they blew on the trumpets. And the armed men went before them, and the rear guard came after the ark of Adonai, while the trumpets blew continually.}%
\verse{And they marched around the city once on the second day, and they returned to the camp. They did this for six days.}%
\verse{Then on the seventh day they rose early at dawn, and they marched around the city in this manner seven times. It was only on that day that they marched around the city seven times.}%
\verse{And at the seventh time the priests blew on the trumpets, and Joshua said to the people, “Shout! For Adonai has given you the city.}%
\verse{The city and all that is in it will be devoted to Adonai; only Rahab the prostitute and all who are with her in the house will live, because she hid the messengers whom we sent.}%
\verse{As for you, keep away from the things\lebnote{Hebrew “thing”} devoted to destruction so that you do not take them and bring about your own destruction, making the camp of Israel an object for destruction, bringing trouble upon it.}%
\verse{But all of the silver and gold, and the items of bronze and iron, are holy to Adonai, and they must go to Adonai’s treasury.”}%
\verse{So the people shouted, and they\lebnote{That is, the priests} blew on the trumpets. And when the people heard the sound of the trumpet, they raised a great shout, and the wall fell flat. The people charged, each one straight ahead into the city,\lebnote{“The people went up, each before him”} and they captured it.}%
\verse{And they utterly destroyed by the edge of the sword\lebnote{“by the mouth of \textit{the} sword”} all who were in the city, both men and women, young and old, ox, sheep, and donkey.}%
\verse{Then Joshua said to the two men who spied on the land, “Go to the prostitute’s house and bring out from there the woman and all who belong to her, just as you swore to her.”}%
\verse{So the young men who were spies went and brought Rahab and her father and mother, her brothers, and all who were with her. And they brought all her family out and set them outside the camp of Israel.}%
\verse{And they burned the city and all that was in it with fire; they put only the silver and gold, and the items of copper and iron, into the treasury of the house of Adonai.}%
\verse{But Joshua spared Rahab the prostitute and her family\lebnote{“and the house of her father”} and all who were with her, and she has lived in the midst of Israel until this day, because she hid the messengers whom Joshua sent to spy out Jericho.}%
\verse{And Joshua swore at that time, saying, “Cursed is anyone before Adonai who gets up and builds Jericho, this city. At the cost of his firstborn he will lay its foundation, and at the cost of his youngest he will set up its gates.”}%
\verse{So Adonai was with Joshua, and his fame was in all the land.}%
\end{biblechapter}%
\begin{biblechapter}% Joshua 7
\verseWithHeading{The Sin of Achan}{But the Israelites\lnBAH{} broke faith concerning the devoted things.\lebnote{Or “consecrated possession\textit{s}”} Achan son of Carmi son of Zabdi son of Zerah, of the tribe of Judah, took from the devoted things;\lnBAI{} and Adonai’s anger was kindled\lebnote{“the nose of Adonai became hot”} against the Israelites.\lnBAH{}}%
\verse{Now Joshua sent men from Jericho to Ai, which is near Beth Aven, east of Bethel, and he said to them, “Go up and spy out Ai.” And the men went up and spied out Ai.}%
\verse{And they returned to Joshua and said to him, “Do not let all the people go up and attack Ai; only two or three thousand men should go up because they\lebnote{That is, the people of Ai} are few. Do not make all the people weary up there.”}%
\verse{So about three thousand from the people went up there, and they fled before the men of Ai.}%
\verse{The men of Ai killed about thirty-six of them, and they chased them from the gate up to Shebarim and killed them on the slope. And the hearts\lebnote{Hebrew “heart”} of the people melted and became like water.}%
\verse{And Joshua tore his clothes and fell to the ground on his face before the ark of Adonai until the evening, he and the elders of Israel; and they put dust on their heads.\lebnote{Hebrew “head”}}%
\verse{And Joshua said, “Ah, my Lord! Why did you bring this people across the Jordan to give us into the hand of the Amorites to destroy us? If only we had been content and stayed beyond the Jordan.}%
\verse{Please, my Lord! What can I say after Israel has fled from its enemies?\lebnote{“Israel has turned its neck before its enemies”}}%
\verse{The Canaanites and all the inhabitants of the land will hear of this, and they will surround us and cut off our name from the land. What will you do, for your great name?”}%
\verse{And Adonai said to Joshua, “Stand up! Why\lebnote{“For what this”} have you fallen on your face?}%
\verse{Israel has sinned and transgressed my covenant\lebnote{“crossed my covenant”} that I commanded them. They have taken from the devoted things;\lebnote{Hebrew “thing” or “from the consecrated possession”} they have stolen and acted deceitfully, and they have put them among their belongings.}%
\verse{The Israelites\lnBAH{} were unable to stand before their enemies; they fled from their enemies\lebnote{“they turned \textit{their} neck before their enemies”} because they have become a thing devoted\lebnote{Or “consecrated”} for destruction. I will be with you no more\lebnote{“I will not do again to be with you”} unless you destroy the devoted things\lnBAI{} from among you.}%
\verse{Get up, sanctify the people, and say, ‘Sanctify yourselves for tomorrow. Thus says Adonai the God of Israel: “There are devoted things\lnBAI{} in your midst, O Israel. You will be unable to stand before you enemies until you remove the devoted things\lnBAI{} from your midst.”}%
\verse{In the morning you will come forward, tribe by tribe,\lebnote{“according to your tribes”} and the tribe that Adonai will select by lot will come forward by clans, and the clan that Adonai selects by lot will come forward by families, and the family that Adonai selects by lot will come forward one by one.}%
\verse{The one caught with the devoted things\lnBAI{} will be burned with fire, he and all that belongs to him, because he transgressed the covenant\lebnote{“he crossed the covenant”} of Adonai, and because he did a disgraceful thing in Israel.”’”}%
\verse{So\lebnote{Or “And”} Joshua rose early in the morning and brought forward Israel, tribe by tribe,\lebnote{“according to its tribes”} and the tribe of Judah was selected by lot.}%
\verse{And he brought forward the clans of Judah and selected the clan of the Zerahites\lnBAJ{} by lot. Then he brought forward the clan of the Zerahites,\lnBAJ{} one by one, and Zabdi was selected by lot.}%
\verse{He brought forward his family, one by one, and Achan son of Carmi son of Zabdi son of Zerah, of the tribe of Judah, was selected by lot.}%
\verse{And Joshua said to Achan, “My son, please, give glory to Adonai the God of Israel, and give him a doxology in court.\lebnote{Some interpret “make a confession”} Tell me, please, what you have done; do not hide it from me.”}%
\verse{And Achan answered Joshua and said, “It is true. I have sinned against Adonai the God of Israel, and this is what I did:}%
\verse{I saw among the spoil a beautiful robe from Shinar,\lebnote{“Shinar” refers to Babylonia} two hundred shekels of silver, and one bar of gold that weighed fifty shekels; I coveted them and took them. They are hidden in the ground inside my tent, and the silver is under it.”}%
\verse{Joshua sent messengers, and they ran to the tent; and there they were, hidden in his tent, and the silver was under it.}%
\verse{And they took them from the tent and brought them to Joshua and all the Israelites.\lnBAH{} And they spread them out before the presence of Adonai.}%
\verse{Then Joshua, and all Israel with him, took Achan son of Zerah, the silver, the robe, the bar of gold, his sons and daughters, his cattle and donkeys, his sheep, his tent, and everything that was his, and they brought them to the valley of Achor.\lnBAK{}}%
\verse{And Joshua said, “Why did you bring us trouble? Adonai will bring you trouble on this day.” And all Israel stoned them\lebnote{Hebrew “him”} with stones;\lebnote{Hebrew “stone”} and they burned them with fire after they stoned them with stones.}%
\verse{Then they placed\lebnote{“they raised up”} on top of him a great pile of stones that remains to this day. And Adonai turned from his burning anger,\lebnote{“turned from his burning nose”} and thus the name of that place to this day is called the valley of Achor.\lnBAK{}}%
\end{biblechapter}%
\begin{biblechapter}% Joshua 8
\verseWithHeading{Ai Is Destroyed}{Then Adonai said to Joshua, “Do not fear or be dismayed. Take all the fighting men\lnBAL{} with you and go up immediately to Ai.\lebnote{“get up and go up to Ai”} Look, I am giving into your hand the king of Ai, his city, and his land.}%
\verse{You will do to Ai and its king that which you did to Jericho and its king; you may take only its spoils\lebnote{Hebrew “spoil”} and livestock as booty for yourself. Set for yourself an ambush against the city from behind it.”}%
\verse{So Joshua and all the fighting men\lnBAL{} went up immediately to Ai. Joshua chose thirty thousand of the best fighting men and sent them by night.}%
\verse{And he commanded them, saying, “Look, you are to lay an ambush against the city from behind. Do not go very far from the city and be ready.}%
\verse{Then I and all of the people who are with me will approach the city. And when they go out to meet us as before,\lnBAM{} we will flee from them.\lnBAN{}}%
\verse{They will come out after us until we draw them away from the city, because they will think, ‘They are fleeing from us\lebnote{Or “before our presence”} as before.’\lnBAM{} So we will flee from them.\lnBAN{}}%
\verse{Then you will rise up from the ambush and take possession of the city, for Adonai your God will give it into your hand.}%
\verse{And when you capture the city you will set it on fire as Adonai commanded. Look, I have commanded you.”}%
\verse{So Joshua sent them out, and they went to the place of the ambush, and they sat between Bethel and Ai, to the west of Ai; but Joshua spent the night with the people.\lebnote{“in the middle of the people”}}%
\verse{Joshua rose early in the morning and mustered the people, and he went up with the elders of Israel before the people of Ai.\lebnote{Or “before the presence of the people of Ai”}}%
\verse{All the fighting men\lebnote{“All the people of war”} who were with him went up and drew near before the city\lebnote{“went up, drew near, and came before the city”} and camped north of Ai; there was a valley between him and Ai.}%
\verse{And he took about five thousand men and set them in ambush between Bethel and Ai, to the west of the city.}%
\verse{So they stationed the forces; all the army was north of the city while the rear guard was west.\lebnote{“while its rear guard \textit{was} west of the city”} But Joshua went that night to the middle of the valley.}%
\verse{When the king of Ai saw this, the men of the city hurried and rose early and went out to meet Israel for battle — he\lebnote{That is, the king} and all his army — to the meeting place before the Arabah.\lebnote{A dry region that runs south of the Sea of Galilee along the Jordan Valley} He did not know that there was an ambush for him behind the city.}%
\verse{Then Joshua and all Israel acted like they were beaten before them, and they fled in the direction of the wilderness.\lebnote{“the way of the wilderness”}}%
\verse{All of the people who were in the city were called to pursue after them. As they pursued after Joshua, they were drawn away from the city.}%
\verse{Not a man remained in Ai or Bethel who had not gone out after Israel; they left the city open and pursued after Israel.}%
\verse{And Adonai said to Joshua, “Stretch out the sword\lebnote{Or “spear”} that is in your hand to Ai, because I will give it into your hand.” And Joshua stretched out the sword that was in his hand to the city.}%
\verse{The moment he stretched out his hand, those in the ambush stood up quickly from their place and ran. And they went into the city and captured it, quickly setting the city ablaze with fire.}%
\verse{Then the men of Ai looked behind them, and they saw smoke from the city rising to the sky; they had no power to flee this way or that,\lebnote{“it was not in their hands to flee here and here”} and the people fleeing the wilderness turned around to the pursuers.\lebnote{Hebrew “pursuer”}}%
\verse{And Joshua and all Israel saw that the ambush had captured the city and that the smoke of the city was rising; they returned and struck down the men of Ai.}%
\verse{Then the others from the city came out to meet them, and they found themselves surrounded by Israel,\lebnote{“they were in the middle of Israel”} some on one side, and others on the other side.\lebnote{“these from these, and these from this”} And they\lebnote{That is, Israel} struck them down until no survivor or fugitive was left.}%
\verse{But they captured the king of Ai alive, and they brought him to Joshua.}%
\verse{When Israel finished slaughtering all the inhabitants of Ai in the open field, in the wilderness where they pursued them, and when all of them had fallen by the edge of the sword\lnBAO{} until they all had perished, all Israel returned to Ai and attacked it with the edge of the sword.\lnBAO{}}%
\verse{All the people that fell on that day, both men and women, were twelve thousand — all the inhabitants of Ai.}%
\verse{For Joshua did not draw back his hand that was stretched out with the sword until he had utterly destroyed all the inhabitants of Ai.}%
\verse{Only the livestock and the spoil of that city Israel took as booty for themselves, according to the word of Adonai that Joshua commanded.}%
\verse{So Joshua burned Ai and made it an everlasting heap of rubbish, a desolate place until this day.}%
\verse{The king of Ai he hanged on a tree until the time of evening, and as the sun went down Joshua commanded them, and they brought down his dead body from the tree. Then they threw it at the entrance of the gate of the city, and they raised over it a great heap of stones that remains to this day.}%
\verseWithHeading{Israel Renews the Covenant}{Then Joshua built an altar on Mount Ebal for Adonai the God of Israel,}%
\verse{as Moses Adonai’s servant commanded the Israelites,\lnBAP{} as it is written in the scroll of the law of Moses: “an altar of unhewn\lebnote{Or “whole”} stones on which no one has wielded\lebnote{“waved over them”} an iron implement.”\lebnote{See Exod 20:25} And they offered burnt offerings on it and sacrificed fellowship offerings.}%
\verse{And there Joshua wrote on the stones a copy of the law of Moses, which he\lebnote{That is, Moses} had written, in the presence of the Israelites.\lnBAP{}}%
\verse{Then all Israel, foreigner as well as native,\lebnote{“as the alien as the native”} with the elders, officials, and judges stood on either side\lebnote{“from this and from this”} of the ark before the priests and the Levites, who carried the ark of the covenant of Adonai. Half of them stood in front of Mount Gerizim, and the other half in front of Mount Ebal, as Moses Adonai’s servant had commanded before to bless the people of Israel.}%
\verse{And afterward he read all the words of the law, the blessings\lebnote{Hebrew “blessing”} and the curses,\lebnote{Hebrew “curse”} according to all that was written in the scroll of the law.}%
\verse{There was not a word from all that Moses commanded that Joshua did not read before the assembly of all Israel, and the women, the little children, and the traveling foreigners\lebnote{Hebrew “foreigner”} among them.}%
\end{biblechapter}%
\begin{biblechapter}% Joshua 9
\verseWithHeading{The Gibeonites Act with Cunning}{Now\lebnote{Or “And it happened”} when all the kings who were beyond the Jordan in the hill country and in the Shephelah,\lebnote{Or “lowlands”; a geographical region on the western edge of the hills of Judea} and on all the coast of the great sea toward Lebanon\lebnote{“white mountain”} — the Hittites,\lebnote{Hebrew “Hittite”} the Amorites,\lebnote{Hebrew “Amorite”} the Canaanites,\lebnote{Hebrew “Canaanite”} the Perizzites,\lebnote{Hebrew “Perizzite”} the Hivites,\lnBAQ{} and the Jebusites\lebnote{Hebrew “Jebusite”} — heard of this,}%
\verse{they gathered themselves together to fight with one accord against\lebnote{Hebrew “with”} Joshua and Israel.}%
\verse{But the inhabitants of Gibeon heard what Joshua did to Jericho and Ai,}%
\verse{and they acted on their part with cunning: they went and prepared provisions,\lebnote{The Hebrew is difficult here. Some ancient manuscripts read, “they sent out a delegation/an envoy”} and took worn-out sacks\lebnote{Or “sackcloths”} for their donkeys and old wineskins that were torn and mended.}%
\verse{The sandals on their feet were patched and old, their clothes were old, and their food was dry and crumbled.}%
\verse{And they went to Joshua at the camp at Gilgal and said to him and to the men of Israel, “We have come from a far land; so then make a covenant with us.”\lnBAR{}}%
\verse{And the men of Israel said to the Hivites,\lnBAQ{} “Perhaps you are living among us; how can we make a covenant\lebnote{“cut a covenant”} with you?”}%
\verse{They said to Joshua, “We are your servants.” And Joshua said to them, “Who are you, and from where do you come?”}%
\verse{And they said to him, “Your servants have come from a very far land because of the name of Adonai your God; we have heard of his reputation, of all that he did in Egypt,}%
\verse{and of all that he did to the two kings of the Amorites who were beyond the Jordan — to Sihon king of Heshbon and to Og king of Bashan, who was in Ashtaroth.}%
\verse{So our elders said to us and all the inhabitants of our land, ‘Take in your hand provisions for the journey, and go to meet them, and say to them, “We are your servants; so then make a covenant with us.”’\lnBAR{}}%
\verse{This is our bread; it was hot when we took it from our houses as provisions on the day we set out to come to you. But now, look, it is dry and crumbled.}%
\verse{These are the wineskins that we filled new, but look, they have burst; and these are our clothes and sandals that have worn out from the very long journey.”}%
\verse{So the leaders\lnBAS{} took from their provisions, but they did not ask direction from Adonai.\lebnote{“the mouth of Adonai they did not ask”}}%
\verse{And Joshua made peace with them, and he made a covenant with them\lebnote{“he cut a covenant with them”} to allow them to live happily, and the leaders of the congregation swore an oath to them.}%
\verse{And it happened that at the end of three days, after they made a covenant with them, they heard that they were their neighbors\lebnote{“they \textit{were} near them”} and living among them.}%
\verse{And the Israelites\lnBAT{} set out and went to their cities on the third day (their cities were Gibeon, Kephirah, Beeroth, and Kiriath Jearim).}%
\verse{But the Israelites\lnBAT{} did not attack them, because the leaders of the congregation had sworn to them by Adonai the God of Israel. And all the congregation murmured against their leaders.\lebnote{“against the leaders of the congregation”}}%
\verse{But all the leaders of the congregation said, “We have sworn to them by Adonai the God of Israel, and so we cannot touch them.}%
\verse{This we will do to them: we will let them live so that wrath will not be on us because of the oath we swore to them.”}%
\verse{And the leaders\lnBAS{} said to them, “Let them live.” So they became woodcutters and water carriers for all the congregation, just as the leaders had said to them.}%
\verse{And Joshua summoned them and said, “Why have you deceived us saying, ‘We are very far from you’ when you are living among us?}%
\verse{Therefore you are cursed; some of you will always be slaves as woodcutters and water carriers for the house of my God.”}%
\verse{And they answered Joshua and said, “Because it was told with certainty to your servants that Adonai your God commanded Moses his servant to give to you all the land and to destroy all the inhabitants of the land before you, so we were very afraid for our lives because of you, and so we did this thing.}%
\verse{So then, look, we are in your hand; do with us whatever seems good and right in your eyes.”}%
\verse{So he did this to them: he saved them from the hand of the Israelites,\lnBAT{} and they did not kill them.}%
\verse{And that day Joshua made them woodcutters and water carriers for the congregation and for the altar of Adonai, to this day, in the place that he should choose.}%
\end{biblechapter}%
\begin{biblechapter}% Joshua 10
\verseWithHeading{The Sun Stands Still at Gibeon}{And it happened that when Adoni-Zedek king of Jerusalem heard that Joshua captured Ai and had utterly destroyed it (just as he had done to Jericho and its king, so he did to Ai and its king) and that the inhabitants of Gibeon had made peace with Israel and were among them,}%
\verse{he\lebnote{Hebrew “they”} became very afraid because Gibeon was a very large city, like one of the royal cities,\lebnote{“like one of the cities of the kingship”} and because it was larger than Ai, and all its men were mighty warriors.}%
\verse{So Adoni-Zedek king of Jerusalem sent word to Hohman king of Hebron, to Piram king of Jarmuth, to Japhia king of Lachish, and to Debir king of Eglon, saying,}%
\verse{“Come up and help me, and let us attack Gibeon, because it has made peace with Joshua and the Israelites.”\lnBAU{}}%
\verse{And the five kings of the Amorites,\lnBAV{} the king of Jerusalem, the king of Hebron, the king of Jarmuth, the king of Lachish, and the king of Eglon, gathered together and went up, they and all their forces, and they laid siege to Gibeon\lebnote{“they camped against Gibeon”; see Josh 10:31} and made war against it.}%
\verse{And the men of Gibeon sent word to Joshua at the camp at Gilgal, saying, “Do not abandon\lebnote{“do not release your hands from”} your servant. Come up to us quickly and save us! Help us, for all the kings of the Amorites\lnBAV{} who dwell in the hill country have gathered against us.”}%
\verse{So Joshua went up from Gilgal, he and all the fighting men\lebnote{“all the people of the war”} with him, all the best warriors.\lebnote{“all the mighty warriors of the army”}}%
\verse{And Adonai said to Joshua, “Do not be afraid of them, for I have given them into your hand; no one will withstand you.\lebnote{“not a man of them will stand in your presence”}}%
\verse{Joshua came upon them suddenly by marching up\lebnote{“he went up”} all night from Gilgal.}%
\verse{And Adonai threw them into panic before Israel, who\lebnote{Hebrew “he”} struck them with a great blow at Gibeon and pursued them by the way of the ascent of Beth-horon and struck them as far as Azekah and Makkedah.}%
\verse{And as they were fleeing from Israel, they were on the slope of Beth-horon, and Adonai threw huge stones from the heavens on them as far as Azekah; and more died by the hail stones than those whom the Israelites\lnBAU{} killed by the sword.}%
\verse{Then Joshua spoke to Adonai, on the day Adonai gave the Amorites\lnBAV{} over to the Israelites,\lnBAU{} and he said in the sight of Israel, “Sun in Gibeon, stand still, and moon, in the valley of Aijalon.”}%
\verse{And the sun stood still, and the moon stopped, until the nation took vengeance on its enemies. Is it not written in the scroll of Jashar? The sun stood still in the middle of the heaven and was not in haste to set for about a full day.}%
\verse{There has not been a day like this before it or after, that Adonai listened to the voice of a man; for Adonai fought for Israel.}%
\verse{And Joshua returned and all Israel with him to the camp at Gilgal.}%
\verseWithHeading{The Kings of the Amorites Are Killed}{But these five kings fled and hid themselves in the cave at Makkedah.}%
\verse{And it was told to Joshua, saying, “The five kings were found hidden in the cave at Makkedah.”}%
\verse{And Joshua said, “Roll large stones against the mouth of the cave, and set men in front of it to guard them.}%
\verse{But do not stay there; pursue after your enemies and attack them from the rear. Do not allow them to go into their cities, for Adonai your God has given them into your hand.”}%
\verse{When Joshua and the Israelites\lnBAU{} had finished striking them with a very great blow, until they perished, those of them who survived\lebnote{“the survivors \textit{who} survived”} went into the fortified cities,}%
\verse{and all the people returned to the camp safely\lebnote{Or “in peace”} to Joshua at Makkedah. No one spoke\lebnote{“No one moved his tongue”} against the Israelites.\lnBAU{}}%
\verse{And Joshua said, “Open the mouth of the cave, and bring to me those five kings from the cave.”}%
\verse{And they did so, and brought him these five kings from the cave, the king of Jerusalem, the king of Hebron, the king of Jarmuth, the king of Lachish, and the king of Eglon.}%
\verse{And when they brought these kings to Joshua, Joshua called all the men of Israel and said to the commanders of the fighting men\lebnote{“the men of war”} who had gone with him, “Come near, put your feet on the necks of these kings.” So they came near and put their feet on their necks.}%
\verse{And Joshua said to them, “Do not be afraid or dismayed! Be strong and bold, for thus Adonai will do to all your enemies whom you are about to fight.}%
\verse{And after this Joshua struck them down and killed them, and he hanged them on five trees. And they were hanging on the trees until the evening.}%
\verse{And it happened at the time of sunset,\lebnote{“at the time of the going of the sun”} Joshua commanded, and they took them down from the trees and threw them into the cave where they had hidden themselves, and they put large stones against the mouth of the cave, which are there to this very day.}%
\verse{Joshua captured Makkedah on that day, and he struck it and its king with the edge of the sword;\lnBAW{} he utterly destroyed it and everyone that was in it. He did not leave behind a survivor. So he did to the king of Makkedah just as he did to the king of Jericho.}%
\verseWithHeading{Joshua’s Conquest of the South}{And Joshua passed on, and all of Israel with him, from Makkedah to Libnah, and he fought against Libnah.}%
\verse{And Adonai also gave it into the hand of Israel, and its king and all the people in it he struck with the edge of the sword.\lnBAW{} He left in it no survivor. He did to its king just as he did to the king of Jericho.}%
\verse{And Joshua passed on, and all of Israel with him, from Libnah to Lachish, and he laid siege to it\lebnote{“he camped opposite it”} and fought against it.}%
\verse{And Adonai gave Lachish into the hand of Israel, and he captured it on the second day. He struck it with the edge of the sword,\lnBAW{} and everyone in it, just as he did to Libnah.}%
\verse{Then Horam king of Gezer came up to help Lachish, and Joshua struck him and his people until he left no survivor behind.}%
\verse{And Joshua passed on, and all of Israel with him, from Lachish to Eglon, and they laid siege to it\lebnote{“they camped opposite it”} and fought against it.}%
\verse{And they captured it on that day, and he struck it with the edge of the sword,\lnBAW{} and all the people that were in it on that day he utterly destroyed as he had done to Lachish.}%
\verse{And Joshua went up, and all Israel with him, from Eglon to Hebron, and they fought against it}%
\verse{and captured it, and they struck it with the edge of the sword,\lnBAW{} its king and all its cities, and all the people that were in it; he left behind no survivor, as he had done to Eglon, and he utterly destroyed it and all the people that were in it.}%
\verse{Then Joshua returned to Debir, and all of Israel with him, and they fought against it,}%
\verse{and he captured it and its king and all its cities, and they struck them with the the edge of the sword,\lnBAW{} and they utterly destroyed all the people that were in it; he left behind no survivor, just as he had done to Hebron. Thus he did to Debir and its king what he had done to Libnah and its king.}%
\verse{So Joshua struck all the land — the hill country, the Negev,\lebnote{An arid region south of the Judaean hills} the Shephelah,\lebnote{Or “lowlands”; a geographical region on the western edge of the hills of Judea} and the slopes\lebnote{The slopes of the hills of the western Jordan to the Dead Sea region} — and all their kings; he left behind no survivor, and all that breathed\lebnote{“all of the breath”} he utterly destroyed as Adonai the God of Israel commanded.}%
\verse{Joshua struck them from Kadesh Barnea to Gaza, and all the land of Goshen up to Gibeon;}%
\verse{all of these kings and their land Joshua captured at one time, because Adonai the God of Israel fought for Israel.}%
\verse{And Joshua returned, and all Israel with him, to the camp at Gilgal.}%
\end{biblechapter}%
\begin{biblechapter}% Joshua 11
\verseWithHeading{Joshua’s Conquest of the North}{And it happened, when Jabin king of Hazor heard this, he sent to Jobab king of Madon, to the king of Shimron, to the king of Acshaph,}%
\verse{and to the kings who were in the north in the hill country, in the Arabah\lnBAX{} south of Kinnereth,\lebnote{That is, the Sea of Galilee} in the Shephelah,\lnBAY{} and in Naphoth Dor\lebnote{Or “the heights of Dor”} in the west,}%
\verse{to the Canaanites\lebnote{Hebrew “Canaanite”} in the east and west, the Amorites,\lebnote{Hebrew “Amorite”} the Hittites,\lebnote{Hebrew “Hittite”} the Perizzites,\lebnote{Hebrew “Perizzite”} and the Jebusites\lebnote{Hebrew “Jebusite”} in the hill country, and the Hivites\lnBAZ{} at the foot of\lnBBA{} Hermon in the land of Mizpah.}%
\verse{They came out, they and all their armies with them, as a great army like the sand on the seashore, with very many horses and chariots.}%
\verse{And all these kings joined forces, and they came and camped together by the waters of Merom to fight with Israel.}%
\verse{And Adonai said to Joshua, “Do not be afraid because of their presence, for tomorrow at this time I will hand them over slain to Israel; you will hamstring their horses and burn their chariots with fire.”}%
\verse{So Joshua, and all the fighting men\lebnote{“all the people of war”} with him, came against them suddenly at the waters of Merom, and they attacked them.\lebnote{“they fell upon them”}}%
\verse{And Adonai gave them into the hand of Israel, and they struck them and pursued them up to Great Sidon and Misrephoth Maim, and eastward up to the valley of Mizpeh. And they struck them until they left behind no survivor.}%
\verse{And Joshua did to them as Adonai commanded him; he hamstrung their horses and burned their chariots with fire.}%
\verse{Then Joshua turned back at that time, and he captured Hazor and struck its king with the sword, because Hazor formerly was the head of all these kingdoms.}%
\verse{He struck all the people that were in it with the edge of the sword,\lnBBB{} utterly destroying them. There was no one left who breathed,\lebnote{“No one was left over of any breath”} and he burned Hazor with fire.}%
\verse{And Joshua captured all the cities of these kings, and all their kings, and he utterly destroyed them with the edge of the sword,\lnBBB{} as Moses the servant of Adonai commanded.}%
\verse{Israel did not burn the cities standing on their mounds,\lebnote{Hebrew “mound”} except Hazor alone, which Joshua burned.}%
\verse{And all the spoil and livestock of these cities the Israelites took as booty; they struck the people with the edge of the sword,\lnBBB{} until they had destroyed them — they left behind no one who breathed.}%
\verse{Just as Adonai commanded Moses his servant, so Moses commanded Joshua, and Joshua did; he left nothing undone that Adonai had commanded Moses.}%
\verseWithHeading{A Review of Joshua’s Conquests}{So Joshua took all this land: the hill country, all the Negev,\lebnote{An arid region south of the Judaean hills} all the land of Goshen,\lebnote{A southern region; the name means “on the mountains”} the Shephelah,\lnBAY{} the Arabah,\lnBAX{} and the hill country of Israel and its Shephelah,\lnBAY{}}%
\verse{from Mount Halak that rises to Seir and to Baal Gad in the valley of Lebanon\lebnote{Or “white mountain”} at the foot of\lnBBA{} Mount Hermon; he captured all their kings, struck them, and killed them.}%
\verse{For many days Joshua made war with all these kings.}%
\verse{There was not a city that made peace with the Israelites\lnBBC{} besides the Hivites\lnBAZ{} and the inhabitants of Gibeon — all were taken in battle.\lebnote{“the all they took in the battle”}}%
\verse{For it was Adonai that hardened their hearts,\lebnote{“made their heart\textit{s} strong”} to meet Israel in war in order to utterly destroy them without mercy, that they would destroy them just as Adonai commanded Moses.}%
\verse{At that time Joshua came and exterminated the Anakites from the hill country, from Hebron, Debir, Anab, and from all the hill country of Judah, and from all the hill country of Israel; Joshua utterly destroyed them with their cities.}%
\verse{None of the Anakites were left in the land of the Israelites;\lnBBC{} some remained only in Gaza, Gath, and Ashdod.}%
\verse{Joshua took all the land according to all that Adonai had spoken to Moses; and Joshua gave it as an inheritance to Israel, according to their tribal divisions, and the land rested from war.}%
\end{biblechapter}%
\begin{biblechapter}% Joshua 12
\verseWithHeading{The Kings Conquered by Joshua}{These are the kings of the land whom the Israelites\lnBBD{} defeated, and of whose land they took possession beyond the Jordon to the east,\lebnote{“to the rising of the sun”} from the wadi\lnBBE{} of Arnon up to Mount Hermon, and all the Arabah\lnBBF{} to the east:}%
\verse{Sihon king of the Amorites, who lived in Heshbon, and ruled from Aroer, which is on the edge of the wadi\lnBBE{} of Arnon, from the middle of the valley and half of Gilead, up to the Jabbok River,\lebnote{“Jabbok the wadi”} which marks the border of the Ammonites;\lebnote{“sons of Ammon” or “children of Ammon”}}%
\verse{and the Arabah\lnBBF{} up to the Kinnereth Sea\lebnote{That is, the Sea of Galilee} to the east, and as far as the sea of Arabah, the Salt Sea\lebnote{That is, the Dead Sea} to the east, in the direction of\lebnote{“the way of”} Beth Jeshimoth, and to the area southward, at the foot of\lebnote{“under”} the slopes of Pisgah;\lebnote{Or Ashdoth Pisgah}}%
\verse{the territory of Og king of Bashan, one of the last of the Rephaites, who lived at Ashtaroth and Edrei}%
\verse{and ruled over Mount Hermon and Salecah and over all Bashan up to the border of the Geshurites\lebnote{Hebrew “Geshurite”} and the Maacathites,\lebnote{Hebrew “Maacathite”} and half of Gilead, as far as the border of Sihon king of Heshbon.}%
\verse{Moses Adonai’s servant and the Israelites\lnBBD{} defeated them; and Moses Adonai’s servant gave it as a possession to the Reubenites,\lebnote{Hebrew “Reubenite”} the Gadites,\lebnote{Hebrew “Gadite”} and the half-tribe of Manasseh.}%
\verseWithHeading{The Kings Conquered by Moses}{These are the kings of the land whom Joshua and the Israelites\lnBBD{} defeated beyond to the Jordan to the west, from Baal Gad in the valley of Lebanon,\lebnote{Or “white mountain”} and up to Mount Halak, which rises to Seir. And Joshua gave it as a possession to the tribes of Israel according to their allotments,}%
\verse{in the hill country, the Shephelah,\lebnote{Or “lowlands”; a geographical region on the western edge of the hills of Judea} the Arabah,\lnBBF{} on the slopes, in the wilderness, and in the Negev;\lebnote{An arid region south of the Judaean hills} the Hittites,\lebnote{Hebrew “Hittite”} the Amorites,\lebnote{Hebrew “Amorite”} the Canaanites,\lebnote{Hebrew “Canaanite”} the Perizzites,\lebnote{Hebrew “Perizzite”} the Hivites,\lebnote{Hebrew “Hivite”} and the Jebusites:\lebnote{Hebrew “Jebusite”}}%
\verse{the king of Jericho, one; the king of Ai, which is beside Bethel, one;}%
\verse{the king of Jerusalem, one; the king of Hebron, one;}%
\verse{the king of Jarmuth, one; the king of Lachish, one;}%
\verse{the king of Eglon, one; the king of Gezer, one;}%
\verse{the king of Debir, one; the king of Geder, one;}%
\verse{the king of Hormah, one; the king of Arad, one;}%
\verse{the king of Libnah, one; the king of Adullam, one;}%
\verse{the king of Makkedah, one; the king of Bethel, one;}%
\verse{the king of Tappuah, one; the king of Hepher, one;}%
\verse{the king of Aphek, one; the king of Lasharon, one;}%
\verse{the king of Madon, one; the king of Hazor, one;}%
\verse{the king of Shimron-meron, one; the king of Acshaph, one;}%
\verse{the king of Taanach, one; the king of Megiddo, one;}%
\verse{the king of Kedesh, one; the king of Jokneam in Carmel, one;}%
\verse{the king of Dor in Naphath Dor, one; the king of Goiim for Gilgal, one;}%
\verse{the king of Tirzah, one; all the kings, thirty-one.}%
\end{biblechapter}%
\begin{biblechapter}% Joshua 13
\verseWithHeading{Land Still Remains to be Conquered}{Now Joshua was old and advanced in years,\lebnote{“he went in the days”} and Adonai said to him, “You are old and advanced in years,\lebnote{“you went in the days”} and very much of the land remains to be possessed.}%
\verse{This is the remaining land: all the regions of the Philistines, and all of the Geshurites,\lnBBG{}}%
\verse{from the Shihor, which is east of Egypt,\lebnote{“on the face of Egypt”} up to the border of Ekron to the north, which is reckoned as Canaanite; there are five Philistine rulers: the Gazites,\lebnote{Hebrew “Gazite”} Ashdodites,\lebnote{Hebrew “Ashdodite”} Ashkelonites,\lebnote{Hebrew “Ashkelonite”} Gittites,\lebnote{Hebrew “Gittite”} Ekronites,\lebnote{Hebrew “Ekronite”} and the Avvim.}%
\verse{In the south; all the land of the Canaanites,\lebnote{Hebrew “Canaanite”} and Mearah, which belongs to the Sidonians up to Aphek, to the border of the Amorites,\lnBBH{}}%
\verse{and the land of the Gebalites, and all the Lebanon,\lnBBI{} toward the east,\lebnote{“the rise of the sun”} from Baal Gad at the foot of\lebnote{Or “below” or “under”} Mount Hermon up to Lebo-Hamath;}%
\verse{all the inhabitants of the hill country, from the Lebanon\lnBBI{} up to Misrephoth Maim, and all the Sidonians. I will drive them out from before the Israelites;\lnBBJ{} only allocate it to Israel as an inheritance just as I have commanded you.}%
\verse{Therefore, divide this land as an inheritance to the nine tribes and the half-tribe of Manasseh.”}%
\verse{With it\lebnote{That is, the other half-tribe of Manasseh} the Reubenites,\lebnote{Hebrew “Reubenite”} and the Gadites\lebnote{Hebrew “Gadite”} received their inheritance, which Moses gave them beyond the Jordan to the east, just as Moses Adonai’s servant gave to them:}%
\verse{from Aroer, which is on the edge of the wadi\lnBBK{} of Arnon, and the city which is in the middle of the wadi, and all the plateau from Medeba up to Dibon;}%
\verse{and all the cities of Sihon king of the Amorites,\lnBBH{} who reigned in Heshbon up to the border of the Ammonites;\lnBBL{}}%
\verse{and Gilead, and the border of the Geshurite\lnBBG{} and the Maacathites,\lnBBM{} all of Mount Hermon,\lebnote{Or “the hill country of Hermon”} and Bashan up to Salecah;}%
\verse{all the kingdom of Og in Bashan, who reigned in Ashtaroth and Edrei — he was left over from the survivors\lebnote{Hebrew “survivor”} of the Rephaim; these Moses had defeated and driven out.}%
\verse{But the Israelites\lnBBJ{} did not drive out the Geshurites\lnBBG{} or the Maacathites;\lnBBM{} Geshur and Maacah live among Israel to this day.}%
\verse{Only the tribe of Levites\lebnote{Hebrew “Levite”} Moses did not give an inheritance; the offerings made by fire to Adonai the God of Israel are their\lebnote{Hebrew “his”} inheritance, just as he promised to them.\lebnote{Hebrew “he said to him”}}%
\verseWithHeading{Reuben’s Inheritance}{Moses gave an inheritance to the tribe of the descendants\lnBBN{} of Reuben according to their families.}%
\verse{Their territory was from Aroer, which was on the edge of the wadi\lnBBK{} of Arnon, and the city that is in the middle of the valley, and all the plateau by Medeba;}%
\verse{Heshbon and its cities that are on the plateau; Dibon, Bamoth Baal, Beth Baal Meon,}%
\verse{Jahaz, Kedemoth, Mephaath,}%
\verse{Kiriathaim, Sibmah, and Zereth Shahar on the hill of the valley;}%
\verse{Beth Peor, the slopes of Pisgah, and Beth Jeshimoth;}%
\verse{all of the cities of the plateau, and all the kingdom of Sihon king of the Amorites,\lnBBH{} who reigned in Heshbon and whom Moses defeated with the leaders of Midian, Evi, Rekem, Zur, Hur, and Reba, the princes of Sihon who dwelled in the land.}%
\verse{In addition to their slain, the Israelites\lnBBJ{} killed with the sword Balaam son of Beor, who practiced divination.}%
\verse{And the border of the descendants\lnBBN{} of Reuben was the Jordan and its banks.\lnBBO{} This was the inheritance of the descendants\lnBBN{} of Reuben according to their families, the cities, and their villages.}%
\verseWithHeading{Gad’s Inheritance}{Moses gave an inheritance to the tribe of Gad, to the descendants\lnBBN{} of Gad, according to their families.}%
\verse{Their territory was Jazer and all the cities of Gilead, and half the land of the Ammonites\lnBBL{} up to Aroer, which is east of Rabbah;\lebnote{“which \textit{is} before Rabbah”}}%
\verse{and from Heshbon up to Ramah-Mizpeh and Betonim, and from Mahanaim up to the territory to Debir;}%
\verse{in the valley of Beth Haram, Beth Nimrah, Succoth, Zaphon, and the rest of the kingdom of Sihon king of Heshbon, the Jordan and its banks,\lnBBO{} up to the lower end of the Kinnereth Sea\lebnote{That is, the Sea of Galilee} beyond the Jordan to the east.}%
\verse{This is the inheritance of the Gadites\lebnote{“sons/children of Gad”} according to their families, the cities, and their villages.}%
\verseWithHeading{The Half-Tribe of Manasseh’s Inheritance}{Moses gave an inheritance to the half-tribe of Manasseh; it was for the half-tribe of the descendants\lnBBN{} of Manasseh according to their families.}%
\verse{Their territory was from Mahanaim, all Bashan, all the kingdom of Og king of Bashan, and all the settlements\lebnote{Or “tent villages”} of Jair, which are in Bashan, sixty cities,}%
\verse{and half of Gilead, with Ashtaroth, Edrei, and the cities of the kingdom of Og in Bashan; these were allotted to the children of Makir son of Manasseh, for half of the children of Makir according to their families.}%
\verse{These are the territories that Moses gave as an inheritance on the desert-plateau of Moab, beyond the Jordan, east of Jericho.}%
\verse{But to the tribe of Levi Moses did not give an inheritance; Adonai the God of Israel, he is their inheritance, just as he promised them.\lebnote{“he said to them”}}%
\end{biblechapter}%
\begin{biblechapter}% Joshua 14
\verseWithHeading{The Land Allotted West of the Jordan}{These are the territories that the Israelites\lnBBP{} inherited in the land of Canaan, which Eleazar the priest, Joshua son of Nun, and the heads of the families of the tribes of the Israelites\lnBBP{} gave as an inheritance to them.}%
\verse{Their inheritance was by lot, just as Adonai commanded through the hand of Moses, for the nine tribes and the half-tribe.}%
\verse{For Moses had given an inheritance of the two tribes and the half-tribe beyond the Jordan, but to the Levites he gave no inheritance among them.}%
\verse{For the descendants\lnBBQ{} of Joseph were two tribes, Manasseh and Ephraim, and they did not give a plot of ground to the Levites in the land, only cities to live in, with their pastureland for their flocks and for their goods.}%
\verse{Just as Adonai commanded Moses, so the Israelites\lnBBP{} did; and they allotted the land.}%
\verseWithHeading{Caleb Receives Hebron}{Then the descendants\lnBBQ{} of Judah came to Joshua at Gilgal; and Caleb son of Jephunneh the Kenizzite said to him, “You know the word that Adonai said to Moses the man of God at Kadesh Barnea concerning you and me.}%
\verse{I was forty years old\lebnote{“I \textit{was} a son of forty year\textit{s}”} when Moses Adonai’s servant sent me from Kadesh Barnea to spy out the land, and I returned with an honest report.\lebnote{“with a word just as \textit{was} with my heart”}}%
\verse{My companions who went up with me made the hearts\lebnote{Hebrew “heart”} of the people melt, but I remained true to Adonai my God.}%
\verse{And Moses swore on that day, saying, ‘Surely the land that your foot has trodden on will be an inheritance to you and your sons forever, because you remained true to Adonai my God.’}%
\verse{So then, look, Adonai has kept me alive just as he promised these forty-five years,\lebnote{Hebrew “year”} from the time that Adonai spoke this word to Moses while Israel wandered\lebnote{“went”} in the wilderness. Now look, today I am eighty-five years old.\lebnote{“I \textit{am} a son of eighty-five year\textit{s}”}}%
\verse{Today I am still strong, just as on the day that Moses sent me; as my strength was then, so now also is my strength for war and for daily activities.\lebnote{“for going out and for coming \textit{in}”}}%
\verse{So now give me this hill country that Adonai spoke of on that day, for you heard on that day that the Anakites were there, with great and fortified cities. Perhaps Adonai is with me, and I will drive them out just as Adonai promised.”\lebnote{Or “said”}}%
\verse{And Joshua blessed him and gave Hebron to Caleb son of Jephunneh as an inheritance.}%
\verse{Thus Hebron became the inheritance of Caleb son of Jephunneh the Kenizzite to this day, because he remained true to Adonai the God of Israel.}%
\verse{And the name of Hebron formerly was Kiriath Arba;\lebnote{Or “the city of Arba”} Arba was the greatest person among the Anakites. And the land rested from war.}%
\end{biblechapter}%
\begin{biblechapter}% Joshua 15
\verseWithHeading{The Allotment of Judah}{The allotment for the tribe of the descendants\lnBBR{} of Judah according to their families reached to the border of Edom, to the wilderness of Zin, to the far south.\lebnote{“to the south at the end of south”}}%
\verse{Their southern border was from the end of the Salt Sea,\lnBBS{} from the bay facing southward;}%
\verse{it continues\lnBBT{} to the south to the ascent of Akrabbim, passes along to Zin, it goes up south of Kadesh Barnea, passes along Hezron, goes up to Addar, and makes a turn to Karka;}%
\verse{it passes on\lebnote{“it was to”} to Azmon, continues\lnBBT{} by the wadi of Egypt, and it ends\lnBBU{} at the sea. This will be your southern border.}%
\verse{The eastern border is the Salt Sea\lnBBS{} up to the mouth\lnBBV{} of the Jordan. The border on the northern side runs from the bay of the sea at the mouth\lnBBV{} of the Jordan;}%
\verse{the border goes up to Beth-hoglah and passes along north of Beth Arabah; and the border goes up the stone of Bohan son of Reuben;}%
\verse{and the border goes up to Debir from the valley of Achor, and to the north, turning to Gilgal, which is opposite the ascent of Adummim, which is south of the wadi;\lnBBW{} and the border passes on to the waters of En Shemesh, and it ends at En Rogel.}%
\verse{Then the border goes up by the Valley of Ben Hinnom\lebnote{Or “valley of the son of Hinnom”} to the slope of the Jebusites\lebnote{Hebrew “Jebusite”} from the south (that is, Jerusalem); and the border goes up to the top of the mountain that lies opposite the valley of Hinnom to the west, which is at the end of the valley of Rephaim to the north;}%
\verse{then the border turns from the top of the mountain to the spring of the waters of Nephtoah, and continues\lnBBT{} from there to the cities of Mount Ephron; the border then turns to Baalah (that is, Kiriath Jearim);}%
\verse{and the border goes around from Baalah to the west, to Mount Seir, and passes on to the slope of Mount Jearim from the north (that is, Kesalon), and goes down to Beth Shemesh, and passes along by Timnah.}%
\verse{The border continues\lebnote{“Hebrew “goes out”} to the slope of Ekron to the north, then bends around to Shikkeron, it passes on to Mount Baalah and continues to Jabneel; and the border ends\lnBBU{} at the sea.}%
\verse{And the western border is to the Great Sea\lnBBX{} and its coast. This is the border surrounding the descendants\lnBBR{} of Judah according to their families.}%
\verse{According to the commandment of Adonai to Joshua,\lebnote{“according to the mouth of Adonai to Joshua”} he gave to Caleb son of Jephunneh a plot of ground among the descendants\lnBBR{} of Judah, Kiriath Arba,\lnBBY{} which is Hebron (Arba was Anak’s father).}%
\verse{Caleb drove out from there three of Anak’s sons, Sheshai, Ahiman, and Talmai, the descendants\lebnote{Or “children”} of Anak.}%
\verse{And from there he went up against the inhabitants of Debir (the former name of Debir was Kiriath Sepher).}%
\verse{And Caleb said, “Whoever attacks Kiriath Sepher and captures it, I will give to him my daughter Acsah as a wife.”}%
\verse{Othniel son of Kenaz, the brother of Caleb, captured it, and he gave to him Acsah his daughter as a wife.}%
\verse{When she came to him she urged him to ask her father for a field. So she dismounted from the donkey, and Caleb said to her, “What do you want?”\lebnote{“What \textit{is} for you?”}}%
\verse{And she said to him, “Give to me a gift;\lebnote{Or “blessing”} you have given me the land of the Negev,\lebnote{An arid region south of the Judaean hills} and you must give to me a spring of water.” And he gave to her the upper and lower spring.\lebnote{Joshua 15:13–19 is almost identical to Judges 1:11–15}}%
\verseWithHeading{The Cities of Judah}{This is the inheritance of the tribe of the descendants\lnBBR{} of Judah according to their families:}%
\verse{the cities belonging to the tribe of the descendants\lnBBR{} of Judah to the far south, to the border of Edom to the south, were Kabzeel, Eder, Jagur,}%
\verse{Kinah, Dimonah, Adadah,}%
\verse{Kedesh, Hazor, Ithnan,}%
\verse{Ziph, Telem, Bealoth,}%
\verse{Hazor Hadattah, Kerioth Hezron (that is, Hazor),}%
\verse{Amam, Shema, Moladah,}%
\verse{Hazar Gaddah, Heshmon, Beth Pelet,}%
\verse{Hazar Shual, Beersheba, Biziothiah,}%
\verse{Baalah, Iim, Ezem,}%
\verse{Eltolad, Kesil, Hormah,}%
\verse{Ziklag, Madmannah, Sansannah,}%
\verse{Lebaoth, Shilhim, Ain, and Rimmon; in all, twenty-nine cities and their villages.}%
\verse{In the Shephelah:\lebnote{Or “lowlands”; a geographical region on the western edge of the hills of Judea} Eshtaol, Zorah, Ashnah,}%
\verse{Zanoah, En Gannim, Tappuah, Enam,}%
\verse{Jarmuth, Adullam, Socoh, Azekah,}%
\verse{Shaaraim, Adithaim, Gederah, and Gederothaim; fourteen cities and their villages.}%
\verse{Zenan, Hadashah, Migdal Gad,}%
\verse{Dilean, Mizpah, Joktheel,}%
\verse{Lachish, Bozkath, Eglon,}%
\verse{Cabbon, Lahma, Kitlish,}%
\verse{Gederoth, Beth Dagon, Naamah, and Makkedah; sixteen cities and their villages.}%
\verse{Libnah, Ether, Ashan,}%
\verse{Jephthah, Ashnah, Nezib,}%
\verse{Keilah, Aczib, and Mareshah; nine cities and their villages.}%
\verse{Ekron, its towns and villages;}%
\verse{from Ekron to the sea, and all that were near\lebnote{“\textit{were} on the hand of”} Ashdod and their villages.}%
\verse{Ashdod, its towns and villages; Gaza, its towns and villages, up to the wadi\lnBBW{} of Egypt and the Great Sea\lnBBX{} and its coast.\lebnote{“border”}}%
\verse{And in the hill country: Shamir, Jattir, Socoh,}%
\verse{Dannah, Kiriath Sanna (that is, Debir),}%
\verse{Anab, Eshtemoh, Anim,}%
\verse{Goshen, Holon, and Giloh; eleven cities and their villages.}%
\verse{Arab, Dumah, Eshan,}%
\verse{Janim, Beth-tappuah, Aphekah,}%
\verse{Humtah, Kiriath Arba\lnBBY{} (that is, Hebron), and Zior; nine cities and their villages.}%
\verse{Moan, Carmel, Ziph, Juttah,}%
\verse{Jezreel, Jokdeam, Zanoah,}%
\verse{Kain, Gibeah, and Timnah; ten cities and their villages.}%
\verse{Halhul, Beth Zur, Gedor,}%
\verse{Maarath, Beth Anoth, and Eltekon; six cities and their villages.}%
\verse{Kiriath Baal (that is, Kiriath Jearim) and Rabbah; two cities and their villages.}%
\verse{In the wilderness: Beth Arabah, Middin, Secacah,}%
\verse{Nibshan, the city of Salt, and En Gedi; six cities and their villages.}%
\verse{But the descendants\lnBBR{} of Judah were unable to drive out the Jebusites, the inhabitants of Jerusalem, so the Jebusites live with the descendants\lnBBR{} of Judah in Jerusalem to this day.}%
\end{biblechapter}%
\begin{biblechapter}% Joshua 16
\verseWithHeading{The Allotment of Ephraim and Manasseh}{The allotment of the descendants\lnBBZ{} of Joseph went from the Jordan by Jericho, at the waters of Jericho to the east, into the wilderness, going up from Jericho into the hill country to Bethel;}%
\verse{it continues from Bethel to Luz, and it passes along to the territory of the Arkites\lebnote{Hebrew “Arkite”} at Ataroth.}%
\verse{Then it goes down, to the west, to the territory of the Japhletites,\lebnote{Hebrew “Japhletite”} up to the territory of Lower Beth-horon, then to Gezer, and it ends\lnBCA{} at the sea.}%
\verse{And the descendants\lnBBZ{} of Joseph, Manasseh and Ephraim, received their inheritance.}%
\verse{This was the border of the descendants\lnBBZ{} of Ephraim according to their families: the border of their inheritance to the east was Ataroth Addar, up to Upper Beth-horon.}%
\verse{The border continues to the sea; from Micmethath to the north, the border turns to the east to Taanath Shiloh, and it passes along it from the east to Janoah.}%
\verse{Then it goes down from Janoah to Ataroth and to Naarah; it touches Jericho and ends at the Jordan;}%
\verse{from Tappuah the border goes to the west, to the wadi\lebnote{A valley that is dry most of the year, but contains a stream during the rainy season} of Kanah, and it ends\lnBCA{} at the sea. This is the inheritance of the tribe of the descendants\lnBBZ{} of Ephraim according to their families,}%
\verse{with the cities that were set apart for the descendants\lnBBZ{} of Ephraim in the midst of the inheritance of the descendants\lnBBZ{} of Manasseh, all the cities and their villages.}%
\verse{But they did not drive out the Canananites\lnBCB{} who were dwelling in Gezer, and so the Canaanites\lnBCB{} live in the midst of Ephraim to this day, but they became forced laborers.}%
\end{biblechapter}%
\begin{biblechapter}% Joshua 17
\verseWithHeading{The Allotment of the Other Half-Tribe of Manasseh}{Then the allotment was made for the tribe of Manasseh, because he was the firstborn of Joseph. To Makir, the firstborn of Manasseh, the father of Gilead, were allotted\lebnote{“and there was to him”} Gilead and Bashan, because he was a warrior.\lebnote{“a man of war”}}%
\verse{An allotment was made for the remaining descendants\lnBCC{} of Manasseh, according to their families: For the children of Abiezer, Helek, Asriel, Shechem, Hepher, and Shemida — these were the male descendants\lnBCC{} of Manasseh son of Joseph according to their families.}%
\verse{But Zelophehad son of Hepher, son of Gilead, son of Makir, son of Manasseh, had no sons, only daughters. These are the names of his daughters: Mahlah, Noah, Hoglah, Milcah, and Tirzah.}%
\verse{They came before Eleazar the priest, Joshua son of Nun, and the leaders, saying, “Adonai commanded Moses to give an inheritance to us among our kinsmen.”\lebnote{Or “among our brothers”} Therefore, according to the command of Adonai\lebnote{“mouth of Adonai”} he gave them an inheritance among the kinsmen\lebnote{Or “brothers”} of their father.}%
\verse{Thus ten shares fell to Manasseh, besides the land of Gilead and Bashan, which is beyond the Jordan,}%
\verse{because the daughters of Manasseh received an inheritance among his sons. And the land of Gilead was allotted to the remaining descendants\lnBCC{} of Manasseh.}%
\verse{The border of Manasseh was from Asher to Micmethath, which is opposite Shechem;\lebnote{Or “which faces Shechem”} then the border goes to the south, to the inhabitants of En Tappuah.}%
\verse{The land of Tappuah belonged to Manasseh,\lebnote{“was to Manasseh”} but Tuppuah on the border of Manasseh belonged to the descendants of Ephraim.\lebnote{“was to the children of Ephraim”}}%
\verse{Then the border goes down to the wadi\lebnote{A valley that is dry most of the year, but contains a stream during the rainy season} of Kanah to the south of the wadi. These cities belong to Ephraim among the cities of Manasseh. The border of Manasseh is north of the wadi, and it ends\lebnote{“the goings out of it were”} at the sea.}%
\verse{The south is Ephraim’s, and the north is Manasseh’s; the sea is their\lebnote{Hebrew “its”} border; Asher touches the north and on the east Issachar.}%
\verse{In Issachar and Asher, Manasseh had Beth-shean and its villages, Ibleam and its villages, the inhabitants of Dor and its villages, the inhabitants of En-dor and its villages, the inhabitants of Taanach and its villages, the inhabitants of Megiddo and its villages; the third is Napheth.}%
\verse{But the descendants\lnBCC{} of Manasseh were not able to take possession of these towns; the Canaanites\lnBCD{} were determined to live in this land.}%
\verse{And it happened, when the Israelites\lebnote{“sons/children of Israel”} grew strong, they put the Canaanites\lnBCD{} to forced labor but never drove them out completely.}%
\verseWithHeading{The Tribes of Joseph Object}{The descendants\lnBCC{} of Joseph spoke with Joshua, saying, “Why have you given us\lebnote{Hebrew “to me”} one allotment and one share as an inheritance? We are many people, which Adonai has blessed.”}%
\verse{And Joshua said to them, “If you are many people, go up to the forest and clear a place there for yourselves in the land of the Perizzites\lebnote{Hebrew “Perizzite”} and Rephaim, since the hill country of Ephraim is too narrow for you.”}%
\verse{And the descendants\lnBCC{} of Joseph said, “The hill country is not enough for us, and all of the Canaanites\lnBCD{} living in the land of the valley have chariots\lebnote{Hebrew “chariot”} of iron, those in Beth-shean and its villages, and those in the Jezreel Valley.”}%
\verse{And Joshua said to the house of Joseph, to Ephraim and Manasseh, “You are many people and have great power; you will not have one allotment only;}%
\verse{the hill country will be yours. Even though it is a forest, you will clear it, and it will be yours to its farthest borders. You will drive out the Canaanites,\lnBCD{} even though they have iron chariots and are strong.”}%
\end{biblechapter}%
\begin{biblechapter}% Joshua 18
\verseWithHeading{The Last of the Land Is Divided}{The entire congregation of the Israelites\lnBCE{} assembled at Shiloh, and they set up there the tent of meeting, and the land was subdued before them.\lebnote{Or “in their presence”}}%
\verse{And seven tribes remained among the Israelites\lnBCE{} who had not been apportioned their inheritance.}%
\verse{And Joshua said to the Israelites,\lnBCE{} “How long\lebnote{“Until when”} will you be slack about going to take possession of the land that Adonai, the God of your ancestors,\lebnote{Or “fathers”} has given you?}%
\verse{Provide three men from each tribe,\lebnote{“according to each tribe”} and I will send them so that they may begin to go through the land and write a description of it according to their inheritance,\lebnote{“according to the mouth of their inheritance”} and let them come to me.}%
\verse{They will divide it among themselves into seven portions; Judah will maintain its border\lebnote{“Judah will stand on its border”} from the south, and the house of Joseph will maintain its border\lebnote{“the house of Joseph will stand on its border”} from the north.}%
\verse{Describe the land in seven divisions, and bring it to me here; I will cast lots for you here before Adonai our God.}%
\verse{The Levites among you have no portion, for their inheritance is the priesthood of Adonai; Gad, Reuben, and the half-tribe of Manasseh received their inheritance beyond the Jordan to the east, which Moses Adonai’s servant gave to them.”}%
\verse{And the men went immediately,\lebnote{“the men got up and went”} and Joshua commanded the ones going to describe the land, saying, “Go and walk about through the land, write a description, and return to me, and here I will cast a lot for you before\lnBCF{} Adonai at Shiloh.”}%
\verse{And the men went and passed through the land, and they described\lebnote{“wrote it”} the cities in seven divisions in a book; and they came to Joshua to the camp at Shiloh,}%
\verse{and Joshua cast a lot for them at Shiloh before\lnBCF{} Adonai, and there he divided the land for the Israelites,\lnBCE{} to each a portion.\lebnote{“according to their allotment”}}%
\verseWithHeading{The Allotment of Benjamin}{And the allotment of the tribe of Benjamin came up according to their families, and the border of their allotment fell\lebnote{“went out”} between the descendants\lnBCG{} of Judah and the descendants\lnBCG{} of Joseph.}%
\verse{Their northern border began at the Jordan and went up to the slope of Jericho on the north and continued into the hill country to the west; it ends\lnBCH{} at the wilderness of Beth Aven.}%
\verse{The border passes on from there to Luz, to the slope of Luz to the south (that is, Bethel); then the border goes down to Ataroth Addar to the mountain that is south of Lower Beth-Horon.}%
\verse{Then the border changes direction and turns to the western side southward, from the mountain that is opposite\lebnote{“against the face of”} Beth-Horon to the south. It ends\lnBCH{} at Kiriath Baal (that is, Kiriath Jearim), a town belonging to the descendants\lnBCG{} of Judah. This is the western side.}%
\verse{The southern side begins on the outskirts of Kiriath Jearim, and the border continues to the west to the spring of the waters of Nephtoah;}%
\verse{the border goes down to the foot\lebnote{Or “to the edge”} of the mountain, which is opposite the Valley of Ben Hinnom,\lebnote{Or “valley of the son of-Hinnom”} which is in the valley of Rephaim to the north; then it does down the valley of Hinnom to the slope of the Jebusites\lebnote{Hebrew “Jebusite”} to the south, and then it goes down to En Rogel.}%
\verse{It changes direction from the north, and it continues to En Shemesh; it goes out to Geliloth, which is opposite the ascent of Adummim, and it goes down to the stone of Bohan, son of Reuben.}%
\verse{It passes on to the slope opposite the Arabah\lnBCI{} to the north, and it goes down to the Arabah.\lnBCI{}}%
\verse{The border passes on to the slope of Beth-hoglah to the north and ends\lnBCH{} at the north bay of the Salt Sea\lebnote{That is, the Dead Sea} at the south end of the Jordan. This is the southern border.}%
\verse{The Jordan forms its border on the eastern side. This is the inheritance of the tribe of Benjamin, its borders that surrounds them, according to their families.}%
\verse{Now the towns of the tribes of the descendants\lnBCG{} of Benjamin, according to their families, were Jericho, Beth-hoglah, Emek Keziz,}%
\verse{Beth Arabah, Zemaraim, Bethel,}%
\verse{Avvim, Parah, Ophrah,}%
\verse{Kephar Ammoni, Ophni, and Geba; twelve cities and their villages.}%
\verse{Gibeon, Ramah, Beeroth,}%
\verse{Mizpeh, Kephirah, Mozah,}%
\verse{Rekem, Irpeel, Taralah,}%
\verse{Zela, Haeleph, Jebus (that is, Jerusalem), Gibeah, and Kiriath; fourteen cities and their villages. This is the inheritance of the descendants\lnBCG{} of Benjamin according to their families.}%
\end{biblechapter}%
\begin{biblechapter}% Joshua 19
\verseWithHeading{The Allotment of Simeon}{The second allotment fell\lebnote{“went out”} for Simeon, for the tribe of the descendants\lnBCJ{} of Simeon, according to their families. And their inheritance was in the midst of the inheritance of the descendants\lnBCJ{} of Judah.}%
\verse{And they had as their inheritance Beersheba, Sheba, Moladah,}%
\verse{Hazar Shual, Balah, Ezem,}%
\verse{Eltolad, Bethul, Hormah,}%
\verse{Ziklag, Beth Marcaboth, Hazar Susah,}%
\verse{Beth Lebaoth, and Sharuhen; thirteen cities and their villages.}%
\verse{Ain, Rimmon, Ether, and Ashan; four cities and their villages,}%
\verse{and all the villages that were around these towns up to Baalat-Beor, Ramath of the Negev.\lebnote{An arid region south of the Judaean hills} This was the inheritance of the tribe of the descendants\lnBCJ{} of Simeon according to their families.}%
\verse{Part of the portion allotted to the descendants\lnBCJ{} of Judah became the inheritance of the descendants\lnBCJ{} of Simeon because the portion for the descendants\lnBCJ{} of Judah was too large for them,\lebnote{“large from them”} so the descendants\lnBCJ{} of Simeon inherited from their inheritance.}%
\verseWithHeading{The Allotment of Zebulun}{The third allotment came up for the descendants\lnBCJ{} of Zebulun according to their families. The border of their inheritance went up to Sarid.}%
\verse{Their border goes up to the west, to Maralah; it touches\lnBCK{} Dabbesheth, then the wadi\lebnote{A valley that is dry most of the year, but contains a stream during the rainy season} that is opposite Jokneam.}%
\verse{It turns from Sarid to the east to the sunrise, to the border of Kislot-Tabor; it continues to Daberath and goes up to Japhia.}%
\verse{From there it passes along to the east toward the sunrise, to Gath Hepher and to Eth Kazin, and continuing to Rimmon, it turns to Neah;}%
\verse{it changes direction from the north of Hannathon, and it ends\lnBCL{} at the valley of Yiptah-El;}%
\verse{Kattath, Nahalal, Shimron, Idalah, and Bethlehem; twelve cities and their villages.}%
\verse{This is the inheritance of the descendants\lnBCJ{} of Zebulun according to their families, these cities and their villages.}%
\verseWithHeading{The Allotment of Issachar}{The fourth allotment fell\lnBCM{} for Issachar, for the descendants\lnBCJ{} of Issachar, according to their families.}%
\verse{Their border went to Jezreel, Chesulloth, Shunem,}%
\verse{Hapharaim, Shion, Anaharath,}%
\verse{Rabbith, Kishion, Ebez,}%
\verse{Remeth, En Gannim, En Haddah, and Beth Pazzez;}%
\verse{and the border touches Tabor, Shahazumah, and Beth Shemesh. Its border ends\lebnote{“the goings out of their border were”} at the Jordan; sixteen cities and their villages.}%
\verse{This is the inheritance of the tribe of the descendants\lnBCJ{} of Issachar according to their families, the cities and their villages.}%
\verseWithHeading{The Allotment of Asher}{The fifth allotment fell\lnBCM{} for the tribe of the descendants\lnBCJ{} of Asher according to their families.}%
\verse{Their border was Helkath, Hali, Beten, Acshaph,}%
\verse{Allamelech, Amad, and Mishal; it touches Carmel to the west, and Shihor-Libnat.}%
\verse{Then it turns eastward\lebnote{“to the rising of the sun”} to Beth-dagon and touches Zebulun and the valley of Yiptah-El to the north\lebnote{“left hand/side”} to Beth Emeck and Neiel; it continues to Cabul from the north,}%
\verse{and Ebron, Rehob, Hammon, and Kanah up to Great Sidon;}%
\verse{then the border turns to Ramah, and up to the fortified city of Tyre, where the border turns to Hosah; it ends\lnBCL{} at the sea, from Hebel to Aczib.}%
\verse{Included were Ummah, Aphek, and Rehob; twenty-two cities and their villages.}%
\verse{This is the inheritance of the tribe of the descendants\lnBCJ{} of Asher according to their families, these cities and their villages.}%
\verseWithHeading{The Allotment of Naphtali}{The sixth allotment fell\lnBCM{} for the children of Naphtali, for the children of Naphtali according to their families.}%
\verse{Their border was from Heleph, from the oak in Zaanannim, Adami Nekeb, Jabneel, up to Lakkum; it ends\lnBCL{} at the Jordan;}%
\verse{then the border turns to the west, to Aznoth Tabor, and continues from there to Hukok, and it touches\lnBCK{} Zebulun on the south, Asher on the west, and Judah on the east\lebnote{“the rising of the sun”} at the Jordan.}%
\verse{The fortified cities\lebnote{“the cities of fortification”} are Ziddim, Zer, Hammath, Rakkath, Kinnereth,}%
\verse{Adamah, Ramah, Hazor,}%
\verse{Kedesh, Edrei, En Hazor,}%
\verse{Yiron, Migdal El, Horem, Beth-anath, Beth Shemesh; nineteen cities and their villages.}%
\verse{This is the inheritance of the tribe of the children of Naphtali according to their families, the cities and their villages.}%
\verseWithHeading{The Allotment of Dan}{The seventh lot fell\lnBCM{} for the tribe of the descendants\lnBCJ{} of Dan according to their families.}%
\verse{The border of their inheritance was Zorah, Eshtaol, Ir Shemesh,}%
\verse{Shaalabbin, Aijalon, Ithlah,}%
\verse{Elon, Timnah, Ekron,}%
\verse{Eltekeh, Gibbethon, Baalath,}%
\verse{Jehud, Bene Berak, Gath Rimmon,}%
\verse{Me Jarkon, Rakkon, with the border opposite Joppa.}%
\verse{The border of the descendants\lnBCJ{} of Dan continued beyond them,\lebnote{“from them”} because the descendants\lnBCJ{} of Dan went up and fought with Lesham, and they captured and struck it with the edge of the sword,\lebnote{“\textit{the} mouth of \textit{the} sword”} and they took possession of it and settled in it; and they called Leshem Dan, after the name of Dan their ancestor.\lebnote{Or “father”}}%
\verse{This is the inheritance of the tribe of the descendants\lnBCJ{} of Dan according to their families, these cities and their villages.}%
\verseWithHeading{The Allotment Is Completed}{They finished assigning the land according to its borders, and the Israelites\lnBCN{} gave an inheritance from among them to Joshua son of Nun.}%
\verse{According to the commandment of Adonai,\lebnote{“On the mouth of Adonai”} they gave him the city that he requested, Timnath Serah, in the hill country of Ephraim, and he rebuilt the city and settled\lebnote{Or “dwelt”} in it.}%
\verse{These are the inheritances that Eleazar the priest, Joshua son of Nun, and the heads of the families of the tribes, distributed by allotment to the Israelites,\lnBCN{} at Shiloh before Adonai\lebnote{Or “in the presence of Adonai”} at the entrance of the tent of meeting. And they finished dividing the land.}%
\end{biblechapter}%
\begin{biblechapter}% Joshua 20
\verseWithHeading{Cities of Refuge Are Established}{And Adonai spoke to Joshua, saying,}%
\verse{“Speak to the Israelites,\lnBCO{} saying, ‘Appoint for yourselves cities of refuge, of which I spoke to you through the hand of Moses.}%
\verse{Anyone who kills a person by accident or unintentionally\lebnote{Or “by not knowing”} may flee there; they will be for yourselves a refuge from the avenger of blood.}%
\verse{The killer will flee to one of these cities, stand at the entrance of the gate of the city, and state his case to the elders of that city;\lebnote{“he will speak his words in the ears of the elders of that city”} and they will take him\lebnote{Or “they will gather him”} into the city and give him a place, and he will dwell among them.}%
\verse{And if the avenger of blood pursues after him, they will not hand over the killer into his hand, because he killed his neighbor unintentionally, and he did not hate him previously.\lebnote{“he did not hate him since yesterday and the day before that”}}%
\verse{The killer will stay in that city until he stands before the congregation for the trial, until the death of the one who is the high priest in those days. Then the killer will return\lebnote{“the killer will return and go”} to his city and to his house, to the city from which he fled.’”}%
\verse{So they set apart\lebnote{Or “consecrated”} Kedesh in Galilee in the hill country of Naphtali, Shechem in the hill country of Ephraim, and Kiriath Arba\lebnote{Or “the city of Arba”} (that is, Hebron) in the hill country of Judah.}%
\verse{Beyond the Jordan east of Jericho, they appointed Bezer in the wilderness on the plateau, from the tribe of Reuben, Ramoth in Gilead, from the tribe of Gad, and Golan in the Bashan, from the tribe of Manasseh.}%
\verse{These were the cities designated for all the Israelites,\lnBCO{} and for the foreigners\lebnote{Hebrew “foreigner”} dwelling among them, for anyone that kills a person unintentionally to flee there, and not die by the hand of the avenger of blood, until there is a trial\lebnote{“until he stands”} before the congregation.}%
\end{biblechapter}%
\begin{biblechapter}% Joshua 21
\verseWithHeading{The Allotment of the Levites}{Then the heads of the families of the Levites came to Eleazar the priest, to Joshua son of Nun, and to the heads of the families of the tribes of the Israelites.\lnBCP{}}%
\verse{And they spoke to them at Shiloh in the land of Canaan, saying, “Adonai commanded through the hand of Moses to give us cities to live in, with their pasturelands for our livestock.”}%
\verse{So, by command of Adonai,\lebnote{“by the mouth of Adonai”} the Israelites\lnBCP{} gave the Levites these cities and their pasturelands from their inheritance.}%
\verse{The allotment fell\lebnote{“came out”} for the families of the Kohathites.\lnBCQ{} The descendants\lnBCR{} of Aaron the priest, who were of the Levites, received\lebnote{“they had”} by lot thirteen towns from the tribes of Judah, Simeon, and Benjamin.}%
\verse{The remaining descendants\lnBCR{} of Kohath received by lot ten cities from the families of the tribes of Ephraim, Dan, and the half-tribe of Manasseh.}%
\verse{The descendants\lnBCR{} of Gershon received by lot thirteen cities from the families of the tribes of Issachar, Asher, and Naphtali and from the half-tribe of Manasseh in Bashan.}%
\verse{The descendants\lnBCR{} of the Merarites\lebnote{Hebrew “Merarite”} according to their families received twelve cities from the tribes of Reuben, Gad, and Zebulun.}%
\verse{The Israelites\lnBCP{} gave to the Levites these cities and their pastureland by lot, just as Adonai commanded through the hand of Moses.}%
\verse{They gave these cities, which are here mentioned by name, from the tribe of the families of Judah and from the tribe of the families of Simeon;}%
\verse{and they were for the descendants\lnBCR{} of Aaron, from the families of the Kohathites,\lnBCQ{} from the descendants\lnBCR{} of Levi, because the first lot was theirs.}%
\verse{And they gave to them Kiriath Arba,\lebnote{Or “the city of Arba”} Arba being the father of Anak (that is, Hebron), in the hill country of Judah and the pasturelands surrounding it.}%
\verse{But the field of the city and its villages they gave to Caleb son of Jephunneh as his property.}%
\verse{To the descendants\lnBCR{} of Aaron the priest they gave Hebron, the city of refuge for the killer, and its pasturelands, Libnah and its pasturelands,}%
\verse{Jattir and its pasturelands, Eshtemoa and its pasturelands,}%
\verse{Holon and its pasturelands, Debir and its pasturelands,}%
\verse{Ain and its pasturelands, Juttah and its pasturelands, and Beth Shemesh and its pasturelands; nine cities from these two tribes.}%
\verse{From the tribe of Benjamin, Gibeon and its pasturelands, Geba and its pasturelands,}%
\verse{Anathoth and its pasturelands, Almon and its pasturelands; four cities.}%
\verse{All the cities of the descendants\lnBCR{} of Aaron the priests, thirteen cities and their pasturelands.}%
\verse{For the families of the descendants\lnBCR{} of Kohath, the remaining Levites of the descendants\lnBCR{} of Kohath, they received the cities of their lot from the tribe of Ephraim.}%
\verse{They gave them Shechem, the city of refuge for the killer, and its pasturelands in the hill country of Ephraim, Gezer and its pasturelands,}%
\verse{Kibzaim and its pasturelands, and Beth-horon and its pasturelands; four cities.}%
\verse{From the tribe of Dan, Eltekeh and its pasturelands, Gibbethon and its pasturelands,}%
\verse{Aijalon and its pasturelands, and Gath Rimmon and its pasturelands; four cities.}%
\verse{From the half-tribe of Manasseh, Taanach and its pasturelands and Gath Rimmon with its pasturelands; two cities.}%
\verse{All the cities and their pasturelands for the remaining families of the descendants\lnBCR{} of Kohath were ten.}%
\verse{To the descendants\lnBCR{} of Gershon, one of the families of the Levites, from the half-tribe of Manasseh, Golan in Bashan, a city of refuge for the killer, and its pasturelands, and Eshtarah and its pasturelands; two cities.}%
\verse{From the tribe of Issachar, Kishion and its pasturelands, Daberath and its pasturelands,}%
\verse{Jarmuth and its pasturelands, En Gannim and its pasturelands; four cities.}%
\verse{From the tribe of Asher, Mishal and its pasturelands, Abdon and its pasturelands,}%
\verse{Helkath and its pasturelands, Rehob and its pasturelands; four cities.}%
\verse{From the tribe of Naphtali, Kedesh in Galilee, the city of refuge for the killer, and its pasturelands, Hammoth Dor and its pasturelands, and Kartan and its pasturelands; three cities.}%
\verse{All the cities of the Gershonites\lebnote{Hebrew “Gershonite”} according to their families were thirteen cities and their pasturelands.}%
\verse{To the families of the descendants\lnBCR{} of Merarite, the remaining Levites, from the tribe of Zebulun, Jokneam and its pasturelands, Kartah and its pasturelands,}%
\verse{Dimnah and its pasturelands, and Nahalal and its pasturelands; four cities.}%
\verse{From the tribe of Reuben, Bezer and its pasturelands, Jahaz and its pasturelands,}%
\verse{Kedemoth and its pasturelands, and Mephaath and its pasturelands; four cities.}%
\verse{From the tribe of Gad, Ramoth in Gilead, the city of refuge for the killer, and its pasturelands, Mahanaim and its pasturelands,}%
\verse{Heshbon and its pasturelands, and Jazer and its pasturelands; four cities in all.}%
\verse{All these were the cities of the descendants\lnBCR{} of Merarite according to their families, the remaining families of the Levites; their allotment was twelve cities.}%
\verse{All the cities of the Levites among the property of the Israelites\lnBCP{} were forty-eight cities and their pasturelands.}%
\verse{Each of these cities had pasturelands surrounding them; so it was for all of these cities.}%
\verse{And Adonai gave to Israel all the land that he swore to give to their ancestors,\lnBCS{} and they took possession of it and settled in it.\lebnote{Or “dwelled in it”}}%
\verse{Adonai gave them rest on every side, according to all that he had sworn to their ancestors,\lnBCS{} and nobody from all their enemies withstood them, for Adonai had given all their enemies into their hand.}%
\verse{And nothing failed from\lebnote{“Not a thing fell”} all the good things\lebnote{Hebrew “thing”} that Adonai promised to the house of Israel; everything came to pass.\lebnote{“everything it came”}}%
\end{biblechapter}%
\begin{biblechapter}% Joshua 22
\verseWithHeading{The Eastern Tribes Return}{Then Joshua summoned the Reubenites, the Gadites, and the half-tribe of Manasseh,}%
\verse{and he said to them, “You have observed all that Moses Adonai’s servant commanded you, and you have listened\lebnote{Or “you have obeyed”} to my voice in all that I have commanded you;}%
\verse{you have not forsaken your kinsmen\lnBCT{} these many days, up to this day, and you have observed the obligation of the command of Adonai your God.}%
\verse{So then, Adonai your God has given rest to your kinsmen,\lnBCT{} just as he promised them; so then, turn and go to your tents to the land of your possession, which Moses Adonai’s servant gave to you beyond the Jordan.}%
\verse{Only be very careful to observe the commandment and law that Moses Adonai’s servant commanded you, to love Adonai your God, to walk in all his ways, to keep his commandments, to hold fast to him, and to serve him with all your heart and with all your soul.”\lebnote{Or “inner self”}}%
\verse{And Joshua blessed them and sent them away, and they went to their tents.}%
\verse{And to the half-tribe of Manasseh Moses had given a possession in Bashan, but to the other half Joshua had given a possession with their kinsmen\lnBCT{} beyond the Jordan to the west; and when Joshua sent them to their tents and blessed them,}%
\verse{he said to them, “Return to your tents with much wealth, and with very much livestock, with silver, gold, copper, iron, and with very much clothing; divide the war-booty of your enemies with your kinsmen.”\lnBCT{}}%
\verse{So the descendants\lnBCU{} of Reuben, Gad, and the half-tribe of Manasseh returned home and departed with the Israelites\lnBCV{} at Shiloh, which is in the land of Canaan, to go to the land of Gilead to the land of their possession, which they had acquired according to the command of Adonai\lebnote{“on the mouth of Adonai”} through the hand of Moses.}%
\verse{And they came to the region of the Jordan that is in the land of Canaan, and the descendants\lnBCU{} of Reuben, Gad, and the half-tribe of Manasseh built there an altar on the Jordan, a large and imposing altar.\lebnote{“a large altar to appearance”}}%
\verse{And the Israelites\lnBCV{} heard it said that the descendants\lnBCU{} of Reuben, Gad, and the half-tribe of Manasseh had built an altar next to the land of Canaan, in the region of the Jordan, on the side belonging to the Israelites.\lnBCV{}}%
\verse{When the Israelites\lnBCV{} heard of it, the whole congregation of the Israelites\lnBCV{} gathered at Shiloh to go up against them for battle.}%
\verse{And the Israelites\lnBCV{} sent to the descendants\lnBCU{} of Reuben, Gad, and the half-tribe of Manasseh, to the land of Gilead, Phinehas the priest son of Eleazar,}%
\verse{and ten leaders with him, one leader for each\lebnote{“one leader, one leader”} family\lebnote{“of the house of father”} from each of the tribes of Israel; and each one was the head of his family\lebnote{“house of their fathers”} among the clans of Israel.\lebnote{Or “thousands of Israel”}}%
\verse{They came to the descendants\lnBCU{} of Reuben, Gad, and the half-tribe of Manasseh, to the land of Gilead, and they spoke with them, saying,}%
\verse{“Thus says all the congregation of Adonai: ‘What is this treachery that you have committed against the God of Israel by turning away today from following Adonai, by building for yourselves an altar to rebel today against Adonai?}%
\verse{Is not the sin of Peor enough for us,\lebnote{“too little for us”} from which we have not cleansed ourselves today, and for which a plague came to the congregation of Adonai,}%
\verse{that you must turn today from following Adonai? If you rebel today against Adonai, tomorrow he will be angry with all of the congregation of Israel;}%
\verse{if, however, the land of your property is unclean, cross over to the land of Adonai’s property, where Adonai’s tabernacle resides,\lebnote{Or “stands”} and take possession among us. But you must not rebel against Adonai or against us by building for yourselves an altar other than the altar of Adonai our God.}%
\verse{Did not Achan son of Zerah commit treachery with devoted things,\lebnote{Or “consecrated possession”} and wrath fell on all the congregation of Israel? And he alone\lebnote{“And he \textit{is} one man that”} did not perish because of his iniquity.’”}%
\verse{And the descendants\lnBCU{} of Reuben, Gad, and the half-tribe of Manasseh spoke with the heads of the clans\lnBCW{} of Israel,}%
\verse{“Adonai, God of gods! Adonai, God of gods knows. And let Israel itself know, if it was in rebellion or treachery against Adonai, do not spare us this day}%
\verse{for building for ourselves an altar to turn away from Adonai, or if it was to offer burnt offerings,\lnBCX{} grain offerings,\lnBCX{} or fellowship offerings on it, may Adonai himself take vengeance.}%
\verse{But in fact, we have done this because of anxiety, because of a reason, saying, ‘In the future your children may say to our children, ‘What is the relationship between you and Adonai the God of Israel?\lebnote{“What is to you and to Adonai the God of Israel”}}%
\verse{Adonai has made the Jordan a border between us and you, the descendants\lnBCU{} of Reuben and Gad; you have no portion in Adonai.’ So your children may put an end to our children worshiping\lebnote{Or “seeing”} Adonai.}%
\verse{So we said, ‘Let us build immediately for ourselves an altar, not for burnt offerings\lnBCX{} or for sacrifices;\lnBCY{}}%
\verse{instead, it is a witness between us and you, and between our generations after us for performing the serving of Adonai in his presence with our burnt offerings, sacrifices, and fellowship offerings; so that your children may not say in the future to our children, “You have no portion in Adonai.”’}%
\verse{And we thought, if they say to us and to our children in the future, we can say, ‘Look at this replica of the altar of Adonai, which our ancestors\lebnote{Or “fathers”} made, not for burnt offerings\lnBCX{} or sacrifices;\lnBCY{} rather, it is a witness between us and you.’}%
\verse{Far be it from us to rebel against Adonai, to turn today from following Adonai, to build an altar for burnt offerings,\lnBCX{} grain offerings,\lnBCX{} or sacrifices,\lnBCY{} instead of the altar of Adonai our God that is before his tabernacle.”}%
\verse{Phinehas the priest, the leaders of the congregation, and the heads of the clans\lnBCW{} of Israel who were with him heard the words that the descendants\lnBCU{} of Reuben, Gad, and Manasseh spoke, and they were satisfied.\lebnote{“it was good in their eyes”}}%
\verse{Phinehas the priest, son of Eleazar, said to the descendants\lnBCU{} of Reuben, Gad, and Manasseh, “Today we know that Adonai is among us, because you have not committed this treachery against Adonai. Therefore you have rescued the Israelites\lnBCV{} from the hand of Adonai.”}%
\verse{And Phinehas the priest, son of Eleazar, and the leaders returned from the descendants\lnBCU{} of Reuben and Gad, from the land of Gilead, to the land of Canaan to the Israelites,\lnBCV{} and they gave them their report.\lebnote{“they brought back a word to them”}}%
\verse{The report satisfied the Israelites;\lebnote{“The report was good in the eyes of the children of Israel”} they blessed God, and they did not speak of going up for battle against them to destroy the land in which the descendants\lnBCU{} of Reuben and Gad were living.\lebnote{Or “dwelling”}}%
\verse{The descendants\lnBCU{} of Reuben and Gad called the altar Witness, “Because,” they said, “it is a witness between us that Adonai is God.”}%
\end{biblechapter}%
\begin{biblechapter}% Joshua 23
\verseWithHeading{Joshua’s Farewell Address}{And it happened, after a long time,\lebnote{“many days”} after Adonai had given rest to Israel from all their surrounding enemies, and after Joshua was old and well-advanced in years,\lebnote{“he went into the days”}}%
\verse{Joshua summoned all Israel, their elders, heads, judges, and officials, and he said to them, “I am old and well-advanced in years,\lebnote{“I went into the days”}}%
\verse{and you have seen all that Adonai your God has done to all these nations for your sake,\lebnote{“because of your presence”} for Adonai your God is fighting for you.}%
\verse{Look! I have allotted to you these remaining nations as an inheritance for your tribes, from the Jordan, with all the nations that I have cut off, to the Great Sea\lebnote{That is, the Mediterranean} in the west.\lebnote{“to the setting of the sun”}}%
\verse{And Adonai your God will push them back before you\lebnote{“from your presence”} and drive them out of your sight,\lebnote{“from your face”} and you will possess their land, just as Adonai your God promised to you.}%
\verse{Be very strong to observe carefully all that is written in the scroll of the law of Moses so as not to turn aside from it, to the right or left,}%
\verse{so as not to go among these remaining nations with you; do not profess\lebnote{“do not mention”} the name of their gods, and do not swear by them, serve them, or bow down to them.\lebnote{“bow yourselves down to them”}}%
\verse{But hold fast to Adonai your God, just as you have done up to this day.}%
\verse{Adonai has driven out before you great and strong nations; and as for you, nobody has withstood\lebnote{“a man has not stood in your presence”} you to this day.}%
\verse{One of your men put to flight a thousand, for Adonai your God is fighting for you, just as he promised you.}%
\verse{Take utmost care for the sake of your life to love Adonai your God,}%
\verse{for if indeed you turn back and join these remaining nations among you,\lebnote{“with you”} and you intermarry with them, marrying their women and they yours,\lebnote{“you go into them and they into you”}}%
\verse{know for certain that Adonai your God will not continue to drive out\lebnote{“will not drive out again”} these nations from before you; they will be for you a snare and a trap, a whip on your sides and thorns in your eyes, until you perish from this good land that Adonai your God has given to you.}%
\verse{Look! I am about to die,\lebnote{“I am going today on the way of all the earth”} and you know in all your hearts\lebnote{Hebrew “heart”} and souls\lebnote{Hebrew “soul”; or “inner self”} that not one thing failed\lebnote{“fell”} from all the good things that Adonai your God promised concerning you; everything has been fulfilled;\lebnote{“has come \textit{out}”} not one thing failed.\lebnote{“not one thing fell from it”}}%
\verse{But just as all the good things\lebnote{Hebrew “thing”} came to you that Adonai your God promised, so will Adonai bring to you all the bad things\lebnote{Hebrew “things”} until he has destroyed you from this good land that Adonai your God has given to you.}%
\verse{If you transgress the covenant of Adonai your God, which he commanded to you, and you go and serve other gods and bow down to them, Adonai’s anger will be kindled\lebnote{“Adonai’s nose will become hot”} against you, and you will perish quickly from the good land that he has given to you.”}%
\end{biblechapter}%
\begin{biblechapter}% Joshua 24
\verseWithHeading{Joshua Recounts Their History}{And Joshua gathered all the tribes of Israel to Shechem; he summoned the elders of Israel, their heads, their judges, and their officials, and they presented themselves before God.}%
\verse{And Joshua said to all the people, “Thus says Adonai the God of Israel: ‘Long ago\lebnote{“from ancient”} your ancestors\lnBCZ{} — Terah the father of Abraham and the father of Nahor — lived beyond the river,\lnBDA{} and they served other gods.}%
\verse{I took your ancestor\lebnote{Or “father”} Abraham from beyond the river\lnBDA{} and led him through all the land of Canaan, and I increased his offspring; I gave him Isaac,}%
\verse{and to Isaac I gave Jacob and Esau. To Esau I gave the hill country of Seir to possess, but Jacob and his children went down to Egypt.}%
\verse{And I sent Moses and Aaron, and I plagued Egypt with what I did in its midst; and afterward I brought you out.}%
\verse{When I brought out your ancestors\lnBCZ{} from Egypt, you came to the sea, and the Egyptians pursued after your ancestors\lnBCZ{} with chariots\lebnote{Hebrew “chariot”} and horsemen to the Red Sea.\lebnote{“sea of reed”}}%
\verse{They cried out to Adonai, and he put darkness between you and the Egyptians, and he brought the sea over them\lebnote{That is, the Egyptians} and covered them; your own eyes saw what I did in Egypt. Then you lived in the wilderness for many days.}%
\verse{And I brought you to the land of the Amorites\lnBDB{} who lived beyond the Jordan; they fought you, and I gave them into your hand; you took possession of their land, and I destroyed them before you.\lebnote{“in your presence”}}%
\verse{Then Balak son of Zippor, king of Moab, set out and fought against Israel, and he sent and summoned Balaam son of Beor to curse you,}%
\verse{but I was not willing to listen to Balaam, and he richly blessed you. So I rescued you from his hand,}%
\verse{and you crossed the Jordan and came to Jericho. And the citizens of Jericho, the Amorites,\lnBDB{} the Perizzites,\lebnote{Hebrew “Perizzite”} the Canaanites,\lebnote{Hebrew “Canaanite”} the Hittites,\lebnote{Hebrew “Hittite”} the Girgashites,\lebnote{Hebrew “Girgashite”} the Hivites,\lebnote{Hebrew “Hivite”} and the Jebusites,\lebnote{Hebrew “Jebusite”} fought against you, and I gave them into your hand.}%
\verse{I sent before you the hornet and they drove out before you two kings of the Amorites;\lnBDB{} but not by your sword or bow.}%
\verse{I gave to you a land that you have not labored on, and cities that you have not built, and you live\lnBDC{} in them; you eat from vineyards and olive groves that you have not planted.’}%
\verseWithHeading{The Israelites Promise to Serve Adonai}{“So now, revere Adonai and serve him in sincerity and faithfulness; remove the gods that your ancestors\lnBCZ{} served beyond the river\lnBDA{} and in Egypt, and serve Adonai.}%
\verse{But if it is bad in your eyes to serve Adonai, choose for yourselves today whom you want to serve, whether it is the gods that your ancestors\lnBCZ{} served beyond the river,\lnBDA{} or the gods of the Amorites\lnBDB{} in whose land you are living; but as for me and my household, we will serve Adonai.”}%
\verse{And the people answered and said, “Far be it from us that we would forsake Adonai to serve other gods,}%
\verse{for Adonai our God brought us and our ancestors\lnBCZ{} from the land of Egypt, from the house of slavery, and did these great signs before our eyes. He protected us along the entire way that we went, and among all the peoples through whose midst we passed.}%
\verse{And Adonai drove out all the people before us, the Amorites\lebnote{“Hebrew “Amorite”} who live\lnBDC{} in the land. We will serve Adonai, for he is our God.”}%
\verse{But Joshua said to the people, “You cannot serve Adonai, for he is a holy and jealous God; he will not forgive your transgressions or your sins.}%
\verse{If you forsake Adonai and serve foreign gods, he will turn and bring disaster to you; he will destroy you after he has done good to you.”}%
\verse{And the people said to Joshua, “No, we will serve Adonai.”}%
\verse{And Joshua said to the people, “You are witnesses against yourselves that you have chosen for yourselves to serve Adonai.” And they said, “We are witnesses.”}%
\verse{He said, “Remove the foreign gods that are in your midst, and incline your hearts to Adonai the God of Israel.”}%
\verse{And the people said to Joshua, “We will serve Adonai our God, and we will listen to his voice.”}%
\verse{So Joshua made a covenant\lebnote{“cut a covenant”} with the people on that day, and he established for them a statute and a judgment at Shechem.}%
\verse{Then Joshua wrote these words in a scroll of the law of God, and he took a large stone and set it up there under a large tree, which is at the shrine of Adonai.}%
\verse{And Joshua said to all the people, “Look, this stone will be a witness against us, for it has heard all the words of Adonai that he spoke with us. It will be as a witness against you, so that you do not deny your God.”}%
\verse{Then Joshua sent the people away to their inheritance.}%
\verse{After these things Joshua son of Nun servant of Adonai died; he was one hundred and ten years old.\lebnote{“a son of one hundred and ten years”}}%
\verse{They buried him in the territory of his inheritance, at Timnath-Serah, which is in the hill country of Ephraim, north of Mount Gaash.}%
\verse{Israel served Adonai all the days of Joshua, and all the days of the elders who lived long after Joshua, and who had known all the work that Adonai did for Israel.\lebnote{Joshua 24:28–31 is repeated in Judges 2:6–10}}%
\verse{The bones of Jacob, which the Israelites\lebnote{“sons/children of Israel”} had brought out from Egypt, they buried at Shechem, in a piece of land that Jacob had bought from the children of Hamor, the father of Shechem, for one hundred pieces of money;\lebnote{Hebrew \textit{kesitah}} it became an inheritance for the descendants\lebnote{Or “sons”} of Joseph.}%
\verse{And Eleazar son of Aaron died; and they buried him in Gibeah in the hill country of Ephraim, which had been given to his son Phinehas.}%
\end{biblechapter}%
\flushcolsend
\biblebook{Judges}
\begin{biblechapter}% Judges 1
\verseWithHeading{Israel Continues Its Conquest}{ויהי it came to pass, אחרי Now after מות the death יהושׁע of Joshua וישׁאלו asked בני that the children ישׂראל of Israel ביהוה the LORD, לאמר saying, מי Who יעלה shall go up לנו אל for us against הכנעני the Canaanites בתחלה first, להלחם׃ to fight}%
\verse{ויאמר said, יהוה And the LORD יהודה Judah יעלה shall go up: הנה behold, נתתי I have delivered את הארץ the land בידו׃ into his hand.}%
\verse{ויאמר said יהודה And Judah לשׁמעון unto Simeon אחיו his brother, עלה Come up אתי with בגורלי me into my lot, ונלחמה that we may fight בכנעני against the Canaanites; והלכתי will go גם likewise אני and I אתך with בגורלך thee into thy lot. וילך went אתו with שׁמעון׃ So Simeon}%
\verse{ויעל went up; יהודה And Judah ויתן delivered יהוה and the LORD את הכנעני the Canaanites והפרזי and the Perizzites בידם into their hand: ויכום and they slew בבזק of them in Bezek עשׂרת ten אלפים thousand אישׁ׃ men.}%
\verse{וימצאו And they found את אדני בזק Adoni-bezek בבזק in Bezek: וילחמו and they fought בו ויכו against him, and they slew את הכנעני the Canaanites ואת הפרזי׃ and the Perizzites.}%
\verse{וינס fled; אדני בזק But Adoni-bezek וירדפו and they pursued אחריו after ויאחזו him, and caught אתו ויקצצו him, and cut off את בהנות his thumbs ידיו his thumbs ורגליו׃ and his great toes.}%
\verse{ויאמר said, אדני בזק And Adoni-bezek שׁבעים Threescore and ten מלכים kings, בהנות having their thumbs ידיהם having their thumbs ורגליהם and their great toes מקצצים cut off, היו מלקטים gathered תחת under שׁלחני my table: כאשׁר as עשׂיתי I have done, כן so שׁלם hath requited לי אלהים God ויביאהו me. And they brought ירושׁלם him to Jerusalem, וימת he died. שׁם׃ and there}%
\verse{וילחמו had fought בני Now the children יהודה of Judah בירושׁלם against Jerusalem, וילכדו and had taken אותה ויכוה it, and smitten לפי it with the edge חרב of the sword, ואת העיר the city שׁלחו and set באשׁ׃ on fire.}%
\verse{ואחר And afterward ירדו went down בני the children יהודה of Judah להלחם to fight בכנעני against the Canaanites, יושׁב that dwelt ההר in the mountain, והנגב and in the south, והשׁפלה׃ and in the valley.}%
\verse{וילך went יהודה And Judah אל הכנעני the Canaanites היושׁב that dwelt בחברון in Hebron: ושׁם (now the name חברון of Hebron לפנים before קרית ארבע Kirjath-arba:) ויכו and they slew את שׁשׁי Sheshai, ואת אחימן and Ahiman, ואת תלמי׃ and Talmai.}%
\verse{וילך he went משׁם אל against יושׁבי the inhabitants דביר of Debir: ושׁם and the name דביר of Debir לפנים before קרית ספר׃ Kirjath-sepher:}%
\verse{ויאמר said, כלב And Caleb אשׁר He that יכה smiteth את קרית ספר Kirjath-sepher, ולכדה and taketh ונתתי it, to him will I give לו את עכסהתי my daughter לאשׁה׃ to wife.}%
\verse{וילכדה took עתניאל And Othniel בן the son קנז of Kenaz, אחי brother, כלב Caleb's הקטן younger ממנו ויתן it: and he gave לו את עכסהתו his daughter לאשׁה׃ to wife.}%
\verse{ויהי And it came to pass, בבואה when she came ותסיתהו that she moved לשׁאול him to ask מאת אביה her father השׂדה a field: ותצנח and she lighted מעל from off החמור ass; ויאמר said לה כלב and Caleb מה׃ unto her, What}%
\verse{ותאמר And she said לו הבה unto him, Give לי ברכה me a blessing: כי for ארץ land; הנגב me a south נתתני thou hast given ונתתה give לי גלת me also springs מים of water. ויתן gave לה כלב And Caleb את גלת springs עלית her the upper ואת גלת springs. תחתית׃ and the nether}%
\verse{ובני And the children קיני of the Kenite, חתן father-in-law, משׁה Moses' עלו went up מעיר out of the city התמרים of palm trees את with בני the children יהודה of Judah מדבר into the wilderness יהודה of Judah, אשׁר which בנגב in the south ערד of Arad; וילך and they went וישׁב and dwelt את among העם׃ the people.}%
\verse{וילך went יהודה And Judah את with שׁמעון Simeon אחיו his brother, ויכו and they slew את הכנעני the Canaanites יושׁב that inhabited צפת Zephath, ויחרימו and utterly destroyed אותה ויקרא was called את שׁם it. And the name העיר of the city חרמה׃ Hormah.}%
\verse{וילכד took יהודה Also Judah את עזה Gaza ואת גבולה withthe coast ואת אשׁקלון thereof, and Askelon ואת גבולה with the coast ואת עקרון thereof, and Ekron ואת גבולה׃ with the coast}%
\verse{ויהי was יהוה And the LORD את with יהודה Judah; וירשׁ and he drove out את ההר the mountain; כי but לא could not להורישׁ drive out את ישׁבי the inhabitants העמק of the valley, כי because רכב they had chariots ברזל׃ of iron.}%
\verse{ויתנו And they gave לכלב unto Caleb, את חברון Hebron כאשׁר as דבר said: משׁה Moses ויורשׁ and he expelled משׁם thence את שׁלשׁה the three בני sons הענק׃ of Anak.}%
\verse{ואת היבוסי the Jebusites ישׁב that inhabited ירושׁלם Jerusalem; לא did not הורישׁו drive out בני And the children בנימן of Benjamin וישׁב dwell היבוסי but the Jebusites את with בני the children בנימן of Benjamin בירושׁלם in Jerusalem עד unto היום day. הזה׃ this}%
\verse{ויעלו went up בית And the house יוסף of Joseph, גם also הם they בית אל against Bethel: ויהוה and the LORD עמם׃ with}%
\verse{ויתירו sent to descry בית And the house יוסף of Joseph בבית אל Bethel. ושׁם (Now the name העיר of the city לפנים before לוז׃ Luz.)}%
\verse{ויראו saw השׁמרים And the spies אישׁ a man יוצא come forth מן out of העיר the city, ויאמרו and they said לו הראנו unto him, Show נא us, we pray thee, את מבוא the entrance העיר into the city, ועשׂינו and we will show עמך and we will show חסד׃ thee mercy.}%
\verse{ויראם And when he showed את מבוא them the entrance העיר into the city, ויכו they smote את העיר the city לפי with the edge חרב of the sword; ואת האישׁ the man ואת כל and all משׁפחתו his family. שׁלחו׃ but they let go}%
\verse{וילך went האישׁ And the man ארץ into the land החתים of the Hittites, ויבן and built עיר a city, ויקרא and called שׁמה the name לוז thereof Luz: הוא which שׁמה the name עד thereof unto היום day. הזה׃ this}%
\verse{ולא Neither הורישׁ drive out מנשׁה did Manasseh את בית שׁאן Beth-shean ואת בנותיה and her towns, ואת תענך nor Taanach ואת בנתיה and her towns, ואת ישׁב nor the inhabitants דור of Dor ואת בנותיה and her towns, ואת יושׁבי nor the inhabitants יבלעם of Ibleam ואת בנתיה and her towns, ואת יושׁבי nor the inhabitants מגדו of Megiddo ואת בנותיה and her towns: ויואל would הכנעני but the Canaanites לשׁבת dwell בארץ land. הזאת׃ in that}%
\verse{ויהי And it came to pass, כי when חזק was strong, ישׂראל Israel וישׂם that they put את הכנעני the Canaanites למס to tribute, והורישׁ utterly drive them out. לא and did not הורישׁו׃ utterly drive them out.}%
\verse{ואפרים did Ephraim לא Neither הורישׁ drive out את הכנעני the Canaanites היושׁב that dwelt בגזר in Gezer; וישׁב dwelt הכנעני but the Canaanites בקרבו among בגזר׃ in Gezer}%
\verse{זבולן did Zebulun לא Neither הורישׁ drive out את יושׁבי the inhabitants קטרון of Kitron, ואת יושׁבי nor the inhabitants נהלל of Nahalol; וישׁב dwelt הכנעני but the Canaanites בקרבו among ויהיו them, and became למס׃ tributaries.}%
\verse{אשׁר did Asher לא Neither הורישׁ drive out את ישׁבי the inhabitants עכו of Accho, ואת יושׁבי nor the inhabitants צידון of Zidon, ואת אחלב nor of Ahlab, ואת אכזיב nor of Achzib, ואת חלבה nor of Helbah, ואת אפיק nor of Aphik, ואת רחב׃ nor of Rehob:}%
\verse{וישׁב dwelt האשׁרי But the Asherites בקרב among הכנעני the Canaanites, ישׁבי the inhabitants הארץ of the land: כי for לא they did not הורישׁו׃ drive them out.}%
\verse{נפתלי did Naphtali לא Neither הורישׁ drive out את ישׁבי the inhabitants בית שׁמשׁ of Beth-shemesh, ואת ישׁבי nor the inhabitants בית ענת of Beth-anath; וישׁב but he dwelt בקרב among הכנעני the Canaanites, ישׁבי the inhabitants הארץ of the land: וישׁבי nevertheless the inhabitants בית שׁמשׁ of Beth-shemesh ובית ענת and of Beth-anath היו became להם למס׃ tributaries}%
\verse{וילחצו forced האמרי And the Amorites את בני the children דן of Dan ההרה into the mountain: כי for לא they would not נתנו suffer לרדת them to come down לעמק׃ to the valley:}%
\verse{ויואל would האמרי But the Amorites לשׁבת dwell בהר in mount חרס Heres באילון in Aijalon, ובשׁעלבים and in Shaalbim: ותכבד prevailed, יד yet the hand בית of the house יוסף of Joseph ויהיו so that they became למס׃ tributaries.}%
\verse{וגבול And the coast האמרי of the Amorites ממעלה עקרבים מהסלע from the rock, ומעלה׃ and upward.}%
\end{biblechapter}%
\begin{biblechapter}% Judges 2
\verseWithHeading{Israel Disobeys Adonai}{ויעל came up מלאך And an angel יהוה of the LORD מן from הגלגל Gilgal אל to הבכים Bochim, ויאמר and said, אעלה אתכםמצרים ואביא and have brought אתכם אל you unto הארץ the land אשׁר which נשׁבעתי I swore לאבתיכם unto your fathers; ואמר and I said, לא אפר break בריתי my covenant אתכם with לעולם׃}%
\verse{ואתם And ye לא no תכרתו shall make ברית league ליושׁבי with the inhabitants הארץ land; הזאת of this מזבחותיהם their altars: תתצון ye shall throw down ולא but ye have not שׁמעתם obeyed בקלי my voice: מה why זאת this? עשׂיתם׃ have ye done}%
\verse{וגם Wherefore I also אמרתי said, לא I will not אגרשׁ אותםפניכם from before והיו you; but they shall be לכם לצדים in your sides, ואלהיהם and their gods יהיו shall be לכם למוקשׁ׃ a snare}%
\verse{ויהי And it came to pass, כדבר spoke מלאך when the angel יהוה of the LORD את הדברים words האלה these אל unto כל all בני the children ישׂראל of Israel, וישׂאו lifted up העם that the people את קולם their voice, ויבכו׃ and wept.}%
\verse{ויקראו And they called שׁם the name המקום place ההוא of that בכים Bochim: ויזבחו and they sacrificed שׁם there ליהוה׃ unto the LORD.}%
\verseWithHeading{Joshua Dies}{וישׁלח go, יהושׁע And when Joshua את העם had let the people וילכו went בני the children ישׂראל of Israel אישׁ every man לנחלתו unto his inheritance לרשׁת to possess את הארץ׃ the land.}%
\verse{ויעבדו served העם And the people את יהוה the LORD כל all ימי the days יהושׁע of Joshua, וכל and all ימי the days הזקנים of the elders אשׁר that האריכו outlived ימים אחריהושׁוע Joshua, אשׁר who ראו had seen את כל all מעשׂה works יהוה of the LORD, הגדול the great אשׁר that עשׂה he did לישׂראל׃ for Israel.}%
\verse{וימת died, יהושׁע And Joshua בן the son נון of Nun, עבד the servant יהוה of the LORD, בן old. מאה a hundred ועשׂר and ten שׁנים׃ years}%
\verse{ויקברו And they buried אותו בגבול him in the border נחלתו of his inheritance בתמנת חרס in Timnath-heres, בהר in the mount אפרים of Ephraim, מצפון on the north side להר of the hill געשׁ׃ Gaash.}%
\verse{וגם And also כל all הדור generation ההוא that נאספו were gathered אל unto אבותיו their fathers: ויקם and there arose דור generation אחר another אחריהם after אשׁר them, which לא not ידעו knew את יהוה the LORD, וגם nor yet את המעשׂה the works אשׁר which עשׂה he had done לישׂראל׃ for Israel.}%
\verseWithHeading{Israel Worships the Baals}{ויעשׂו did בני And the children ישׂראל of Israel את הרע evil בעיני in the sight יהוה of the LORD, ויעבדו and served את הבעלים׃ Baalim:}%
\verse{ויעזבו And they forsook את יהוה the LORD אלהי God אבותם of their fathers, המוציא אותםארץ of the land מצרים of Egypt, וילכו and followed אחרי and followed אלהים gods, אחרים other מאלהי of the gods העמים of the people אשׁר that סביבותיהם round about וישׁתחוו them, and bowed themselves להם ויכעסו unto them, and provoked את יהוה׃ the LORD}%
\verse{ויעזבו And they forsook את יהוה the LORD, ויעבדו and served לבעל Baal ולעשׁתרות׃ and Ashtaroth.}%
\verse{ויחר was hot אף And the anger יהוה of the LORD בישׂראל against Israel, ויתנם and he delivered ביד them into the hands שׁסים of spoilers וישׁסו that spoiled אותם וימכרם them, and he sold ביד them into the hands אויביהם of their enemies מסביב round about, ולא not יכלו so that they could עוד any longer לעמד stand לפני before אויביהם׃ their enemies.}%
\verse{בכל אשׁר as יצאו they went out, יד the hand יהוה of the LORD היתה was בם לרעה against them for evil, כאשׁר and as דבר had said, יהוה the LORD וכאשׁר נשׁבע had sworn יהוה the LORD להם ויצר distressed. להם מאד׃ unto them: and they were greatly}%
\verse{ויקם raised up יהוה Nevertheless the LORD שׁפטים judges, ויושׁיעום which delivered מיד them out of the hand שׁסיהם׃ of those that spoiled}%
\verse{וגם And yet אל unto שׁפטיהם their judges, לא they would not שׁמעו hearken כי but זנו they went a whoring אחרי after אלהים gods, אחרים other וישׁתחוו and bowed themselves להם סרו unto them: they turned מהר quickly מן out of הדרך the way אשׁר which הלכו walked אבותם their fathers לשׁמע in, obeying מצות the commandments יהוה of the LORD; לא not עשׂו they did כן׃ so.}%
\verse{וכי And when הקים raised them up יהוה the LORD להם שׁפטים judges, והיה was יהוה then the LORD עם with השׁפט the judge, והושׁיעם and delivered מיד them out of the hand איביהם of their enemies כל all ימי the days השׁופט of the judge: כי for ינחם it repented יהוה the LORD מנאקתם because of their groanings מפני by reason of them לחציהם that oppressed ודחקיהם׃ them and vexed}%
\verse{והיה And it came to pass, במות was dead, השׁופט when the judge ישׁבו they returned, והשׁחיתו and corrupted מאבותם more than their fathers, ללכת in following אחרי in following אלהים gods אחרים other לעבדם to serve ולהשׁתחות them, and to bow down להם לא not הפילו unto them; they ceased ממעלליהם from their own doings, ומדרכם הקשׁה׃}%
\verse{ויחר was hot אף And the anger יהוה of the LORD בישׂראל against Israel; ויאמר and he said, יען Because אשׁר that עברו hath transgressed הגוי people הזה this את בריתי my covenant אשׁר which צויתי I commanded את אבותם their fathers, ולא and have not שׁמעו hearkened לקולי׃ unto my voice;}%
\verse{גם also אני I לא will not אוסיף henceforth להורישׁ drive out אישׁ any מפניהם from before מן from before הגוים the nations אשׁר which עזב left יהושׁע Joshua וימת׃ when he died:}%
\verse{למען That נסות through them I may prove בם את ישׂראל Israel, השׁמרים will keep הם whether they את דרך the way יהוה of the LORD ללכת to walk בם כאשׁר therein, as שׁמרו did keep אבותם their fathers אם or לא׃ not.}%
\verse{וינח יהוה Therefore the LORD את הגוים nations, האלה those לבלתי without הורישׁם driving them out מהר hastily; ולא neither נתנם delivered ביד he them into the hand יהושׁע׃ of Joshua.}%
\end{biblechapter}%
\begin{biblechapter}% Judges 3
\verseWithHeading{Some Nations Remain in the Land}{ואלה Now these הגוים the nations אשׁר which הניח יהוה the LORD לנסות to prove בם את ישׂראל Israel את כל by them, as many אשׁר by them, as many לא as had not ידעו known את כל all מלחמות the wars כנען׃ of Canaan;}%
\verse{רק Only למען that דעת דרות the generations בני of the children ישׂראל of Israel ללמדם to teach מלחמה them war, רק at the least אשׁר such as לפנים before לא nothing ידעום׃ might know,}%
\verse{חמשׁת five סרני lords פלשׁתים of the Philistines, וכל and all הכנעני the Canaanites, והצידני and the Sidonians, והחוי and the Hivites ישׁב that dwelt הר in mount הלבנון Lebanon, מהר from mount בעל חרמון Baal-hermon עד unto לבוא the entering in חמת׃ of Hamath.}%
\verse{ויהיו And they were לנסות to prove בם את ישׂראל Israel לדעת by them, to know הישׁמעו whether they would hearken את מצות unto the commandments יהוה of the LORD, אשׁר which צוה he commanded את אבותם their fathers ביד by the hand משׁה׃ of Moses.}%
\verse{ובני And the children ישׂראל of Israel ישׁבו dwelt בקרב among הכנעני the Canaanites, החתי Hittites, והאמרי and Amorites, והפרזי and Perizzites, והחוי and Hivites, והיבוסי׃ and Jebusites:}%
\verse{ויקחו And they took את בנותיהם their daughters להם לנשׁים to be their wives, ואת בנותיהם their daughters נתנו and gave לבניהם to their sons, ויעבדו and served את אלהיהם׃ their gods.}%
\verseWithHeading{Othniel}{ויעשׂו did בני And the children ישׂראל of Israel את הרע evil בעיני in the sight יהוה of the LORD, וישׁכחו and forgot את יהוה the LORD אלהיהם their God, ויעבדו and served את הבעלים Baalim ואת האשׁרות׃ and the groves.}%
\verse{ויחר was hot אף Therefore the anger יהוה of the LORD בישׂראל against Israel, וימכרם and he sold ביד them into the hand כושׁן רשׁעתים of Cushan-rishathaim מלך king ארם נהרים of Mesopotamia: ויעבדו served בני and the children ישׂראל of Israel את כושׁן רשׁעתים Cushan-rishathaim שׁמנה eight שׁנים׃ years.}%
\verse{ויזעקו cried בני And when the children ישׂראל of Israel אל unto יהוה the LORD, ויקם raised up יהוה the LORD מושׁיע a deliverer לבני to the children ישׂראל of Israel, ויושׁיעם who delivered את עתניאל them, Othniel בן the son קנז of Kenaz, אחי brother. כלב Caleb's הקטן younger ממנו׃}%
\verse{ותהי came עליו upon רוח And the Spirit יהוה of the LORD וישׁפט him, and he judged את ישׂראל Israel, ויצא and went out למלחמה to war: ויתן delivered יהוה and the LORD בידו into his hand; את כושׁן רשׁעתים Cushan-rishathaim מלך king ארם ותעז prevailed ידו and his hand על against כושׁן רשׁעתים׃ Cushan-rishathaim.}%
\verse{ותשׁקט had rest הארץ And the land ארבעים forty שׁנה years. וימת died. עתניאל And Othniel בן the son קנז׃ of Kenaz}%
\verseWithHeading{Ehud}{ויספו again בני And the children ישׂראל of Israel לעשׂות did הרע evil בעיני in the sight יהוה of the LORD: ויחזק strengthened יהוה and the LORD את עגלון Eglon מלך the king מואב of Moab על against ישׂראל Israel, על because כי because עשׂו they had done את הרע evil בעיני in the sight יהוה׃ of the LORD.}%
\verse{ויאסף And he gathered אליו unto את בני him the children עמון of Ammon ועמלק and Amalek, וילך and went ויך and smote את ישׂראל Israel, ויירשׁו and possessed את עיר the city התמרים׃ of palm trees.}%
\verse{ויעבדו served בני So the children ישׂראל of Israel את עגלון Eglon מלך the king מואב of Moab שׁמונה eighteen עשׂרה eighteen שׁנה׃ years.}%
\verse{ויזעקו cried בני But when the children ישׂראל of Israel אל unto יהוה the LORD, ויקם raised them up יהוה the LORD להם מושׁיע a deliverer, את אהודן the son גרא of Gera, בן הימיני אישׁ a man אטר יד and by him ימינו וישׁלחו sent בני the children ישׂראל of Israel בידו מנחה a present לעגלון unto Eglon מלך the king מואב׃ of Moab.}%
\verse{ויעשׂ made לו אהוד חרב him a dagger ולה שׁני which had two פיות edges, גמד of a cubit ארכה length; ויחגר and he did gird אותה מתחת it under למדיו his raiment על upon ירך thigh. ימינו׃ his right}%
\verse{ויקרב And he brought את המנחה the present לעגלון unto Eglon מלך king מואב of Moab: ועגלון and Eglon אישׁ man. בריא fat מאד׃ a very}%
\verse{ויהי כאשׁר And when כלה he had made an end להקריב to offer את המנחה the present, וישׁלח he sent away את העם the people נשׂאי that bore המנחה׃ the present.}%
\verse{והוא But he שׁב himself turned again מן from הפסילים the quarries אשׁר that את by הגלגל Gilgal, ויאמר and said, דבר errand סתר I have a secret לי אליך המלך unto thee, O king: ויאמר who said, הס Keep silence. ויצאו by him went out מעליו from כל And all העמדים that stood עליו׃}%
\verse{ואהוד בא came אליו unto והוא him; and he ישׁב was sitting בעלית parlor, המקרה in a summer אשׁר which לו לבדו he had for himself alone. ויאמר said, אהוד דבר I have a message אלהים from God לי אליך unto ויקם thee. And he arose מעל out of הכסא׃ seat.}%
\verse{וישׁלח put forth אהוד אתד hand, שׂמאלו his left ויקח and took את החרב the dagger מעל from ירך thigh, ימינו his right ויתקעה and thrust בבטנו׃ it into his belly:}%
\verse{ויבא went in גם also הנצב And the haft אחר after הלהב the blade; ויסגר closed החלב and the fat בעד upon הלהב the blade, כי so that לא he could not שׁלף draw החרב the dagger מבטנו out of his belly; ויצא came out. הפרשׁדנה׃ and the dirt}%
\verse{ויצא went forth אהוד המסדרונה through the porch, ויסגר and shut דלתות the doors העליה of the parlor בעדו ונעל׃ and locked}%
\verse{והוא When he יצא was gone out, ועבדיו his servants באו came; ויראו and when they saw והנה that, behold, דלתות the doors העליה of the parlor נעלות locked, ויאמרו they said, אך Surely מסיך covereth הוא he את רגליו his feet בחדר chamber. המקרה׃ in his summer}%
\verse{ויחילו And they tarried עד till בושׁ they were ashamed: והנה and, behold, איננו not פתח he opened דלתות the doors העליה of the parlor; ויקחו therefore they took את המפתח a key, ויפתחו and opened והנה and, behold, אדניהם their lord נפל fallen down ארצה on the earth. מת׃ dead}%
\verse{ואהוד נמלט escaped עד while התמהמהם they tarried, והוא עבר and passed beyond את הפסילים the quarries, וימלט and escaped השׂעירתה׃ unto Seirath.}%
\verse{ויהי And it came to pass, בבואו when he was come, ויתקע that he blew בשׁופר a trumpet בהר in the mountain אפרים of Ephraim, וירדו went down עמו with בני and the children ישׂראל of Israel מן him from ההר the mount, והוא and he לפניהם׃ before}%
\verse{ויאמר And he said אלהם unto רדפו them, Follow אחרי after כי me: for נתן hath delivered יהוה the LORD את איביכם your enemies את מואב the Moabites בידכם into your hand. וירדו And they went down אחריו after וילכדו him, and took את מעברות the fords הירדן of Jordan למואב toward Moab, ולא not נתנו and suffered אישׁ a man לעבר׃ to pass over.}%
\verse{ויכו And they slew את מואב of Moab בעת time ההיא at that כעשׂרת about ten אלפים thousand אישׁ men, כל all שׁמן lusty, וכל and all אישׁ men חיל of valor; ולא not נמלט and there escaped אישׁ׃ a man.}%
\verse{ותכנע was subdued מואב So Moab ביום day ההוא that תחת under יד the hand ישׂראל of Israel. ותשׁקט had rest הארץ And the land שׁמונים fourscore שׁנה׃ years.}%
\verseWithHeading{Shamgar}{ואחריו And after היה him was שׁמגר Shamgar בן the son ענת of Anath, ויך which slew את פלשׁתים of the Philistines שׁשׁ six מאות hundred אישׁ men במלמד goad: הבקר with an ox וישׁע delivered גם also הוא and he את ישׂראל׃ Israel.}%
\end{biblechapter}%
\begin{biblechapter}% Judges 4
\verseWithHeading{Deborah and Barak}{ויספו again בני And the children ישׂראל of Israel לעשׂות did הרע evil בעיני in the sight יהוה of the LORD, ואהוד מת׃ was dead.}%
\verse{וימכרם sold יהוה And the LORD ביד them into the hand יבין of Jabin מלך king כנען of Canaan, אשׁר that מלך reigned בחצור in Hazor; ושׂר the captain צבאו of whose host סיסרא Sisera, והוא which יושׁב dwelt בחרשׁת in Harosheth of the Gentiles. הגוים׃}%
\verse{ויצעקו cried בני And the children ישׂראל of Israel אל unto יהוה the LORD: כי for תשׁע he had nine מאות hundred רכב chariots ברזל of iron; לו והוא he לחץ oppressed את בני the children ישׂראל of Israel. בחזקה עשׂרים and twenty שׁנה׃ years}%
\verse{ודבורה And Deborah, אשׁה a prophetess, נביאה a prophetess, אשׁת the wife לפידות of Lapidoth, היא she שׁפטה judged את ישׂראל Israel בעת time. ההיא׃ at that}%
\verse{והיא And she יושׁבת dwelt תחת under תמר the palm tree דבורה of Deborah בין between הרמה Ramah ובין בית אל and Bethel בהר in mount אפרים Ephraim: ויעלו came up אליה to בני and the children ישׂראל of Israel למשׁפט׃ her for judgment.}%
\verse{ותשׁלח And she sent ותקרא and called לברק Barak בן the son אבינעם of Abinoam מקדשׁ נפתלי of Naphtali ותאמר and said אליו unto הלא him, Hath not צוה commanded, יהוה the LORD אלהי God ישׂראל of Israel לך Go ומשׁכת and draw בהר toward mount תבור Tabor, ולקחת and take עמך with עשׂרת thee ten אלפים thousand אישׁ men מבני of the children נפתלי ומבני and of the children זבלון׃ of Zebulun?}%
\verse{ומשׁכתי And I will draw אליך unto אל thee to נחל the river קישׁון Kishon את סיסרא Sisera, שׂר the captain צבא army, יבין of Jabin's ואת רכבו with his chariots ואת המונו and his multitude; ונתתיהו and I will deliver בידך׃ him into thine hand.}%
\verse{ויאמר said אליה unto ברק And Barak אם her, If תלכי thou wilt go עמי with והלכתי me, then I will go: ואם but if לא thou wilt not תלכי go עמי with לא me, I will not אלך׃ go.}%
\verse{ותאמר And she said, הלך takest אלך and went עמך with אפס כי for לא thee: notwithstanding תהיה be תפארתך for thine honor; על הדרך the journey אשׁר that אתה thou הולך כייד into the hand אשׁה of a woman. ימכר shall sell יהוה the LORD את סיסרא Sisera ותקם arose, דבורה And Deborah ותלך עם with ברק Barak קדשׁה׃ to Kedesh.}%
\verse{ויזעק called ברק And Barak את זבולן Zebulun ואת נפתלי and Naphtali קדשׁה to Kedesh; ויעל and he went up ברגליו at his feet: עשׂרת with ten אלפי thousand אישׁ men ותעל went up עמו with דבורה׃ and Deborah}%
\verse{וחבר Now Heber הקיני נפרד had severed himself מקין מבני of the children חבב of Hobab חתן the father-in-law משׁה of Moses, ויט and pitched אהלו his tent עד unto אלון the plain בצענים of Zaanaim, אשׁר which את by קדשׁ׃ Kedesh.}%
\verse{ויגדו And they showed לסיסרא Sisera כי that עלה was gone up ברק Barak בן the son אבינעם of Abinoam הר to mount תבור׃ Tabor.}%
\verse{ויזעק gathered together סיסרא And Sisera את כל all רכבו his chariots, תשׁע nine מאות hundred רכב chariots ברזל of iron, ואת כל and all העם the people אשׁר that אתו with מחרשׁת הגויםל unto נחל the river קישׁון׃ of Kishon.}%
\verse{ותאמר said דברה And Deborah אל unto ברק Barak, קום Up; כי for זה this היום the day אשׁר in which נתן hath delivered יהוה the LORD את סיסרא Sisera בידך into thine hand: הלא is not יהוה the LORD יצא gone out לפניך before וירד went down ברק thee? So Barak מהר from mount תבור Tabor, ועשׂרת and ten אלפים thousand אישׁ men אחריו׃ after}%
\verse{ויהם discomfited יהוה And the LORD את סיסרא Sisera, ואת כל and all הרכב chariots, ואת כל and all המחנה host, לפי with the edge חרב of the sword לפני before ברק Barak; וירד lighted down סיסרא so that Sisera מעל off המרכבה chariot, וינס and fled away ברגליו׃ on his feet.}%
\verse{וברק But Barak רדף pursued אחרי after הרכב the chariots, ואחרי and after המחנה the host, עד unto חרשׁת Harosheth of the Gentiles: הגוים ויפל fell כל and all מחנה the host סיסרא of Sisera לפי upon the edge חרב of the sword; לא there was not נשׁאר left. עד אחד׃ a man}%
\verse{וסיסרא Howbeit Sisera נס fled away ברגליו on his feet אל to אהל the tent יעל of Jael אשׁת the wife חבר of Heber הקיני the Kenite: כי for שׁלום peace בין between יבין Jabin מלך the king חצור of Hazor ובין בית and the house חבר of Heber הקיני׃ the Kenite.}%
\verse{ותצא went out יעל And Jael לקראת to meet סיסרא Sisera, ותאמר and said אליו unto סורה him, Turn in, אדני my lord, סורה turn in אלי to אל not. תירא me; fear ויסר And when he had turned in אליה unto האהלה her into the tent, ותכסהו she covered בשׂמיכה׃ him with a mantle.}%
\verse{ויאמר And he said אליה unto השׁקיני to drink; נא her, Give me, I pray thee, מעט a little מים water כי for צמאתי I am thirsty. ותפתח And she opened את נאוד a bottle החלב of milk, ותשׁקהו and gave him drink, ותכסהו׃ and covered}%
\verse{ויאמר Again he said אליה unto עמד her, Stand פתח in the door האהל of the tent, והיה and it shall be, אם when אישׁ any man יבוא doth come ושׁאלך and inquire ואמר of thee, and say, הישׁ Is פה here? אישׁ there any man ואמרת that thou shalt say, אין׃ No.}%
\verse{ותקח took יעל Then Jael אשׁת wife חבר Heber's את יתד a nail האהל of the tent, ותשׂם and took את המקבת a hammer בידה in her hand, ותבוא and went אליו unto בלאט ותתקע him, and smote את היתד the nail ברקתו ותצנח and fastened בארץ it into the ground: והוא for he נרדם was fast asleep ויעף and weary. וימת׃ So he died.}%
\verse{והנה And, behold, ברק as Barak רדף pursued את סיסרא Sisera, ותצא came out יעל Jael לקראתו to meet ותאמר him, and said לו לך unto him, Come, ואראך and I will show את האישׁ thee the man אשׁר whom אתה thou מבקשׁ seekest. ויבא And when he came אליה into והנה her behold, סיסרא Sisera נפל lay מת dead, והיתד and the nail ברקתו׃}%
\verse{ויכנע subdued אלהים So God ביום day ההוא on that את יבין Jabin מלך the king כנען of Canaan לפני before בני the children ישׂראל׃ of Israel.}%
\verse{ותלך prospered, יד And the hand בני of the children ישׂראל of Israel הלוך and prevailed וקשׁה and prevailed על against יבין Jabin מלך the king כנען of Canaan, עד until אשׁר until הכריתו they had destroyed את יבין Jabin מלך king כנען׃ of Canaan.}%
\end{biblechapter}%
\begin{biblechapter}% Judges 5
\verseWithHeading{The Song of Deborah and Barak}{ותשׁר Then sang דבורה Deborah וברק and Barak בן the son אבינעם of Abinoam ביום day, ההוא on that לאמר׃ saying,}%
\verse{בפרע for the avenging פרעות for the avenging בישׂראל of Israel, בהתנדב willingly offered themselves. עם when the people ברכו Praise יהוה׃ ye the LORD}%
\verse{שׁמעו Hear, מלכים O ye kings; האזינו give ear, רזנים O ye princes; אנכי I, ליהוה unto the LORD; אנכי I, אשׁירה will sing אזמר I will sing ליהוה to the LORD אלהי God ישׂראל׃ of Israel.}%
\verse{יהוה LORD, בצאתך when thou wentest out משׂעיר בצעדך when thou marchedst משׂדה out of the field אדום of Edom, ארץ the earth רעשׁה trembled, גם and שׁמים the heavens נטפו dropped, גם also עבים the clouds נטפו dropped מים׃ water.}%
\verse{הרים The mountains נזלו melted מפני from before יהוה the LORD, זה that סיני Sinai מפני from before יהוה the LORD אלהי God ישׂראל׃ of Israel.}%
\verse{בימי In the days שׁמגר of Shamgar בן the son ענת of Anath, בימי in the days יעל of Jael, חדלו were unoccupied, ארחות the highways והלכי and the travelers נתיבות and the travelers ילכו walked ארחות through byways. עקלקלות׃ through byways.}%
\verse{חדלו ceased, פרזון the villages בישׂראל in Israel, חדלו they ceased עד until that שׁקמתי arose, דבורה that I Deborah שׁקמתי that I arose אם a mother בישׂראל׃ in Israel.}%
\verse{יבחר They chose אלהים gods; חדשׁים new אז then לחם war שׁערים in the gates: מגן was there a shield אם יראה seen ורמח or spear בארבעים among forty אלף thousand בישׂראל׃ in Israel?}%
\verse{לבי My heart לחוקקי toward the governors ישׂראל of Israel, המתנדבים that offered themselves willingly בעם among the people. ברכו Bless יהוה׃ ye the LORD.}%
\verse{רכבי ye that ride אתנות asses, צחרות on white ישׁבי ye that sit על in מדין judgment, והלכי and walk על by דרך the way. שׂיחו׃ Speak,}%
\verse{מקול מחצצים of archers בין in משׁאבים the places of drawing water, שׁם there יתנו shall they rehearse צדקות the righteous acts יהוה of the LORD, צדקת the righteous acts פרזנו of his villages בישׂראל in Israel: אז then ירדו go down לשׁערים to the gates. עם shall the people יהוה׃ of the LORD}%
\verse{עורי Awake, עורי awake, דבורה Deborah: עורי awake, עורי awake, דברי utter שׁיר a song: קום arise, ברק Barak, ושׁבה and lead thy captivity captive, שׁביך and lead thy captivity captive, בן thou son אבינעם׃ of Abinoam.}%
\verse{אז Then ירד have dominion שׂריד he made him that remaineth לאדירים over the nobles עם among the people: יהוה the LORD ירד made me have dominion לי בגבורים׃ over the mighty.}%
\verse{מני Out of אפרים Ephraim שׁרשׁם a root בעמלק of them against Amalek; אחריך after בנימין thee, Benjamin, בעממיך among thy people; מני out of מכיר Machir ירדו came down מחקקים governors, ומזבולן משׁכים they that handle בשׁבט the pen ספר׃ of the writer.}%
\verse{ושׂרי And the princes בישׂשכר of Issachar עם with דברה Deborah; וישׂשכר even Issachar, כן and also ברק Barak: בעמק into the valley. שׁלח he was sent ברגליו on foot בפלגות For the divisions ראובן of Reuben גדלים great חקקי thoughts לב׃ of heart.}%
\verse{למה Why ישׁבת abodest בין thou among המשׁפתים the sheepfolds, לשׁמע to hear שׁרקות the bleatings עדרים of the flocks? לפלגות For the divisions ראובן of Reuben גדולים great חקרי searchings לב׃ of heart.}%
\verse{גלעד Gilead בעבר beyond הירדן Jordan: שׁכן abode ודן did Dan למה and why יגור remain אניות in ships? אשׁר Asher ישׁב continued לחוף shore, ימים on the sea ועל in מפרציו his breaches. ישׁכון׃ and abode}%
\verse{זבלון Zebulun עם a people חרף jeopardized נפשׁו their lives למות unto the death ונפתלי and Naphtali על in מרומי the high places שׂדה׃ of the field.}%
\verse{באו came מלכים The kings נלחמו fought, אז then נלחמו fought מלכי the kings כנען of Canaan בתענך in Taanach על by מי the waters מגדו of Megiddo; בצע כסף of money. לא no לקחו׃ they took}%
\verse{מן from שׁמים heaven; נלחמו They fought הכוכבים the stars ממסלותם in their courses נלחמו fought עם against סיסרא׃ Sisera.}%
\verse{נחל The river קישׁון of Kishon גרפם swept them away, נחל river, קדומים that ancient נחל the river קישׁון Kishon. תדרכי thou hast trodden down נפשׁי O my soul, עז׃ strength.}%
\verse{אז Then הלמו broken עקבי hooves סוס were the horses' מדהרות by the means of the prancings, דהרות the prancings אביריו׃ of their mighty ones.}%
\verse{אורו Curse מרוז ye Meroz, אמר said מלאך the angel יהוה of the LORD, ארו curse ye bitterly ארור curse ye bitterly ישׁביה the inhabitants כי thereof; because לא not באו they came לעזרת to the help יהוה of the LORD, לעזרת to the help יהוה of the LORD בגבורים׃ against the mighty.}%
\verse{תברך Blessed מנשׁים above women יעל shall Jael אשׁת the wife חבר of Heber הקיני the Kenite מנשׁים shall she be above women באהל in the tent. תברך׃ be, blessed}%
\verse{מים water, שׁאל He asked חלב milk; נתנה she gave בספל dish. אדירים in a lordly הקריבה she brought forth חמאה׃ butter}%
\verse{ידה her hand ליתד to the nail, תשׁלחנה She put וימינה and her right hand להלמות hammer; עמלים to the workmen's והלמה and with the hammer she smote סיסרא Sisera, מחקה she smote off ראשׁו his head, ומחצה when she had pierced וחלפה and stricken through רקתו׃}%
\verse{בין At רגליה her feet כרע he bowed, נפל he fell, שׁכב he lay down: בין at רגליה her feet כרע he bowed, נפל he fell: באשׁר where כרע he bowed, שׁם there נפל he fell down שׁדוד׃ dead.}%
\verse{בעד at החלון a window, נשׁקפה looked out ותיבב and cried אם The mother סיסרא of Sisera בעד through האשׁנב the lattice, מדוע Why בשׁשׁ is רכבו his chariot לבוא in coming? מדוע why אחרו tarry פעמי the wheels מרכבותיו׃ of his chariots?}%
\verse{חכמות Her wise שׂרותיה ladies תענינה answered אף her, yea, היא she תשׁיב returned אמריה׃}%
\verse{הלא Have they not ימצאו sped? יחלקו have they divided שׁלל the prey; רחם a damsel two; רחמתים לראשׁ to every גבר man שׁלל a prey צבעים of divers colors, לסיסרא to Sisera שׁלל a prey צבעים of divers colors רקמה of needlework, צבע of divers colors רקמתים of needlework לצוארי on both sides, for the necks שׁלל׃ of the spoil?}%
\verse{כן So יאבדו perish, כל let all אויביך thine enemies יהוה O LORD: ואהביו but them that love כצאת when he goeth forth השׁמשׁ him as the sun בגברתו in his might. ותשׁקט had rest הארץ And the land ארבעים forty שׁנה׃ years.}%
\end{biblechapter}%
\begin{biblechapter}% Judges 6
\verseWithHeading{The Midianites Oppresses Israel}{ויעשׂו did בני And the children ישׂראל of Israel הרע evil בעיני in the sight יהוה of the LORD: ויתנם delivered יהוה and the LORD ביד them into the hand מדין of Midian שׁבע seven שׁנים׃ years.}%
\verse{ותעז prevailed יד And the hand מדין of Midian על against ישׂראל Israel: מפני because מדין of the Midianites עשׂו made להם בני the children ישׂראל of Israel את המנהרות them the dens אשׁר which בהרים in the mountains, ואת המערות and caves, ואת המצדות׃ and strongholds.}%
\verse{והיה And it was, אם when זרע had sown, ישׂראל Israel ועלה came up, מדין that the Midianites ועמלק ובני and the children קדם of the east, ועלו even they came up עליו׃ against}%
\verse{ויחנו And they encamped עליהם against וישׁחיתו them, and destroyed את יבול the increase הארץ of the earth, עד till בואך thou come עזה unto Gaza, ולא no ישׁאירו and left מחיה sustenance בישׂראל for Israel, ושׂה neither sheep, ושׁור nor ox, וחמור׃ nor ass.}%
\verse{כי For הם they ומקניהם with their cattle יעלו came up ואהליהם and their tents, יבאו and they came כדי ארבה grasshoppers לרב for multitude; ולהם ולגמליהם and their camels אין were without מספר number: ויבאו and they entered בארץ into the land לשׁחתה׃ to destroy}%
\verse{וידל impoverished ישׂראל And Israel מאד was greatly מפני because מדין of the Midianites; ויזעקו cried בני and the children ישׂראל of Israel אל unto יהוה׃ the LORD.}%
\verse{ויהי And it came to pass, כי when זעקו cried בני the children ישׂראל of Israel אל unto יהוה the LORD על because of אדות because of מדין׃ the Midianites,}%
\verse{וישׁלח sent יהוה That the LORD אישׁ נביא a prophet אל unto בני the children ישׂראל of Israel, ויאמר which said להם כה unto them, Thus אמר saith יהוה the LORD אלהי God ישׂראל of Israel, אנכי I העליתי אתכםמצרים ואציאתכם מבית out of the house עבדים׃ of bondage;}%
\verse{ואצל And I delivered אתכם מיד you out of the hand מצרים ומיד and out of the hand כל of all לחציכם that oppressed ואגרשׁ אותםפניכם from before ואתנה you, and gave לכם את ארצם׃ you their land;}%
\verse{ואמרה And I said לכם אני unto you, I יהוה the LORD אלהיכם your God; לא not תיראו fear את אלהי the gods האמרי of the Amorites, אשׁר in whose אתם ye יושׁבים dwell: בארצם land ולא but ye have not שׁמעתם obeyed בקולי׃ my voice.}%
\verseWithHeading{The Angel of Adonai Calls Gideon}{ויבא And there came מלאך an angel יהוה of the LORD, וישׁב and sat תחת under האלה an oak אשׁר which בעפרה in Ophrah, אשׁר that ליואשׁ unto Joash אבי העזרי the Abi-ezrite: וגדעון Gideon בנו and his son חבט threshed חטים wheat בגת by the winepress, להניס to hide מפני from מדין׃ the Midianites.}%
\verse{וירא appeared אליו unto מלאך And the angel יהוה of the LORD ויאמר him, and said אליו unto יהוה him, The LORD עמך with גבור thee, thou mighty man החיל׃ of valor.}%
\verse{ויאמר said אליו unto גדעון And Gideon בי him, Oh אדני my Lord, וישׁ be יהוה if the LORD עמנו with ולמה us, why מצאתנו befallen כל then is all זאת this ואיה us? and where כל all נפלאתיו his miracles אשׁר which ספרו told לנו אבותינו our fathers לאמר us of, saying, הלא Did not ממצרים העלנו bring us up יהוה the LORD ועתה but now נטשׁנו hath forsaken יהוה the LORD ויתננו us, and delivered בכף us into the hands מדין׃ of the Midianites.}%
\verse{ויפן looked אליו upon יהוה And the LORD ויאמר him, and said, לך Go in בכחך thy might, זה this והושׁעת and thou shalt save את ישׂראל Israel מכף from the hand מדין of the Midianites: הלא have not שׁלחתיך׃ I sent}%
\verse{ויאמר And he said אליו unto בי him, Oh אדני my Lord, במה wherewith אושׁיע shall I save את ישׂראל Israel? הנה behold, אלפי my family הדל poor במנשׁה in Manasseh, ואנכי and I הצעיר the least בבית house. אבי׃ in my father's}%
\verse{ויאמר said אליו unto יהוה And the LORD כי him, Surely אהיה I will be עמך with והכית thee, and thou shalt smite את מדין the Midianites כאישׁ man. אחד׃ as one}%
\verse{ויאמר And he said אליו unto אם him, If נא now מצאתי I have found חן grace בעיניך in thy sight, ועשׂית then show לי אות me a sign שׁאתה that thou מדבר talkest עמי׃ with}%
\verse{אל not נא I pray thee, תמשׁ Depart מזה hence, עד until באי I come אליך unto והצאתי thee, and bring forth את מנחתי my present, והנחתי לפניך before ויאמר thee. And he said, אנכי I אשׁב will tarry עד until שׁובך׃ thou come again.}%
\verse{וגדעון And Gideon בא went in, ויעשׂ and made ready גדי a kid, עזים a kid, ואיפת of an ephah קמח of flour: מצות and unleavened cakes הבשׂר the flesh שׂם he put בסל in a basket, והמרק the broth שׂם and he put בפרור in a pot, ויוצא and brought out אליו unto אל him under תחת him under האלה the oak, ויגשׁ׃ and presented}%
\verse{ויאמר said אליו unto מלאך And the angel האלהים of God קח him, Take את הבשׂר the flesh ואת המצות and the unleavened cakes, והנח אל upon הסלע rock, הלז this ואת המרק the broth. שׁפוך and pour out ויעשׂ And he did כן׃ so.}%
\verse{וישׁלח put forth מלאך Then the angel יהוה of the LORD את קצה the end המשׁענת of the staff אשׁר that בידו in his hand, ויגע and touched בבשׂר the flesh ובמצות and the unleavened cakes; ותעל and there rose up האשׁ fire מן out of הצור the rock, ותאכל and consumed את הבשׂר the flesh ואת המצות and the unleavened cakes. ומלאך Then the angel יהוה of the LORD הלך departed מעיניו׃ out of his sight.}%
\verse{וירא perceived גדעון And when Gideon כי that מלאך an angel יהוה of the LORD, הוא he ויאמר said, גדעון Gideon אהה Alas, אדני O Lord יהוה GOD! כי for על because כן because ראיתי I have seen מלאך an angel יהוה of the LORD פנים face אל to פנים׃ face.}%
\verse{ויאמר said לו יהוה And the LORD שׁלום unto him, Peace לך אל not: תירא unto thee; fear לא thou shalt not תמות׃ die.}%
\verse{ויבן built שׁם there גדעון Then Gideon מזבח an altar ליהוה unto the LORD, ויקרא and called לו יהוה שׁלוםד unto היום day הזה this עודנו it yet בעפרת in Ophrah אבי העזרי׃ of the Abi-ezrites.}%
\verse{ויהי And it came to pass בלילה night, ההוא the same ויאמר said לו יהוה that the LORD קח unto him, Take את פר young bullock, השׁור young bullock, אשׁר that לאביך thy father's ופר bullock השׁני even the second שׁבע of seven שׁנים years והרסת old, and throw down את מזבח the altar הבעל of Baal אשׁר that לאביך thy father ואת האשׁרה the grove אשׁר עליו by תכרת׃ hath, and cut down}%
\verse{ובנית And build מזבח an altar ליהוה unto the LORD אלהיך thy God על upon ראשׁ the top המעוז rock, הזה of this במערכה in the ordered place, ולקחת and take את הפר bullock, השׁני the second והעלית and offer עולה a burnt sacrifice בעצי with the wood האשׁרה of the grove אשׁר which תכרת׃ thou shalt cut down.}%
\verse{ויקח took גדעון Then Gideon עשׂרה ten אנשׁים men מעבדיו of his servants, ויעשׂ and did כאשׁר as דבר had said אליו unto יהוה the LORD ויהי him: and it was, כאשׁר because ירא he feared את בית household, אביו his father's ואת אנשׁי and the men העיר of the city, מעשׂות that he could not do יומם by day, ויעשׂ that he did לילה׃ by night.}%
\verseWithHeading{Gideon Destroys the Altar of Baal}{וישׁכימו arose early אנשׁי And when the men העיר of the city בבקר in the morning, והנה behold, נתץ was cast down, מזבח the altar הבעל of Baal והאשׁרה and the grove אשׁר that עליו by כרתה was cut down ואת הפר bullock השׁני it, and the second העלה was offered על upon המזבח the altar הבנוי׃ built.}%
\verse{ויאמרו And they said אישׁ one אל to רעהו another, מי Who עשׂה hath done הדבר thing? הזה this וידרשׁו And when they inquired ויבקשׁו and asked, ויאמרו they said, גדעון Gideon בן the son יואשׁ of Joash עשׂה hath done הדבר thing. הזה׃ this}%
\verse{ויאמרו said אנשׁי Then the men העיר of the city אל unto יואשׁ Joash, הוצא Bring out את בנך thy son, וימת that he may die: כי because נתץ he hath cast down את מזבח the altar הבעל of Baal, וכי and because כרת he hath cut down האשׁרה the grove אשׁר that עליו׃ by}%
\verse{ויאמר said יואשׁ And Joash לכל unto all אשׁר that עמדו stood עליו against האתם him, Will ye תריבון plead לבעל for Baal? אם if אתם will ye תושׁיעון save אותו אשׁר him? he that יריב will plead לו יומת for him, let him be put to death עד whilst הבקר morning: אם אלהים a god, הוא he ירב let him plead לו כי for himself, because נתץ hath cast down את מזבחו׃ his altar.}%
\verse{ויקרא he called לו ביום day ההוא Therefore on that ירבעל him Jerubbaal, לאמר saying, ירב plead בו הבעל Let Baal כי against him, because נתץ he hath thrown down את מזבחו׃ his altar.}%
\verse{וכל Then all מדין the Midianites ועמלק ובני and the children קדם of the east נאספו were gathered יחדו together, ויעברו and went over, ויחנו and pitched בעמק in the valley יזרעאל׃ of Jezreel.}%
\verse{ורוח But the Spirit יהוה of the LORD לבשׁה came upon את גדעון Gideon, ויתקע and he blew בשׁופר a trumpet; ויזעק was gathered אביעזר and Abi-ezer אחריו׃ after}%
\verse{ומלאכים messengers שׁלח And he sent בכל throughout all מנשׁה Manasseh; ויזעק was gathered גם also הוא who אחריו after ומלאכים messengers שׁלח him: and he sent באשׁר unto Asher, ובזבלון and unto Zebulun, ובנפתלי and unto Naphtali; ויעלו and they came up לקראתם׃ to meet}%
\verseWithHeading{Gideon Tests Adonai With the Fleece}{ויאמר said גדעון And Gideon אל unto האלהים God, אם If ישׁך מושׁיע thouwilt save בידי by mine hand, את ישׂראל Israel כאשׁר as דברת׃ thou hast said,}%
\verse{הנה Behold, אנכי I מציג will put את גזת a fleece הצמר of wool בגרן in the floor; אם if טל the dew יהיה be על on הגזה the fleece לבדה only, ועל upon כל all הארץ the earth חרב and dry וידעתי then shall I know כי that תושׁיע thou wilt save בידי by mine hand, את ישׂראל Israel כאשׁר as דברת׃ thou hast said.}%
\verse{ויהי And it was כן so: וישׁכם for he rose up early ממחרת on the morrow, ויזר and thrust את הגזה the fleece וימץ and wrung טל the dew מן on the morrow, הגזה the fleece, מלוא הספל a bowl מים׃ of water.}%
\verse{ויאמר said גדעון And Gideon אל unto האלהים God, אל Let not יחר be hot אפך thine anger בי ואדברה against me and I will speak אך but הפעם this once: אנסה let me prove, נא I pray thee, רק but הפעם this once בגזה with the fleece; יהי be נא let it now חרב dry אל upon הגזה the fleece, לבדה only ועל and upon כל all הארץ the ground יהיה let there be טל׃ dew.}%
\verse{ויעשׂ did אלהים And God כן so בלילה night: ההוא that ויהי for it was חרב dry אל upon הגזה the fleece לבדה only, ועל on כל all הארץ the ground. היה and there was טל׃ dew}%
\end{biblechapter}%
\begin{biblechapter}% Judges 7
\verseWithHeading{Gideon’s Three Hundred Men}{וישׁכם him, rose up early, ירבעל Then Jerubbaal, הוא who גדעון Gideon, וכל and all העם the people אשׁר that אתו with ויחנו and pitched על beside עין חרד the well ומחנה so that the host מדין of the Midianites היה were לו מצפון on the north side מגבעת of them, by the hill המורה of Moreh, בעמק׃ in the valley.}%
\verse{ויאמר said יהוה And the LORD אל unto גדעון Gideon, רב thee too many העם The people אשׁר that אתך with מתתי for me to give את מדין the Midianites בידם into their hands, פן lest יתפאר vaunt themselves עלי against ישׂראל Israel לאמר me, saying, ידי Mine own hand הושׁיעה׃ hath saved}%
\verse{ועתה קרא proclaim נא therefore go to, באזני in the ears העם of the people, לאמר saying, מי Whosoever ירא fearful וחרד and afraid, ישׁב let him return ויצפר and depart early מהר from mount הגלעד Gilead. וישׁב And there returned מן from mount העם the people עשׂרים twenty ושׁנים and two אלף thousand; ועשׂרת ten אלפים thousand. נשׁארו׃ and there remained}%
\verse{ויאמר said יהוה And the LORD אל unto גדעון Gideon, עוד yet העם The people רב many; הורד אותםל unto המים the water, ואצרפנו and I will try לך שׁם them for thee there: והיה and it shall be, אשׁר of whom אמר I say אליך unto זה thee, This ילך shall go אתך with הוא thee, the same ילך shall go אתך with וכל thee; and of whomsoever אשׁר thee; and of whomsoever אמר I say אליך unto זה thee, This לא shall not ילך go עמך with הוא thee, the same לא shall not ילך׃ go.}%
\verse{ויורד So he brought down את העם the people אל unto המים the water: ויאמר said יהוה and the LORD אל unto גדעון Gideon, כל Every one אשׁר that ילק lappeth בלשׁונו with his tongue, מן of המים the water כאשׁר as ילק lappeth, הכלב a dog תציג him shalt thou set אותו לבד by himself; וכל likewise every one אשׁר that יכרע boweth down על upon ברכיו his knees לשׁתות׃ to drink.}%
\verse{ויהי were מספר And the number המלקקים of them that lapped, בידם their hand אל to פיהם their mouth, שׁלשׁ three מאות hundred אישׁ men: וכל but all יתר the rest העם of the people כרעו bowed down על upon ברכיהם their knees לשׁתות to drink מים׃ water.}%
\verse{ויאמר said יהוה And the LORD אל unto גדעון Gideon, בשׁלשׁ By the three מאות hundred האישׁ men המלקקים that lapped אושׁיע will I save אתכם ונתתי you, and deliver את מדין the Midianites בידך into thine hand: וכל and let all העם the people ילכו go אישׁ every man למקמו׃ unto his place.}%
\verse{ויקחו took את צדה victuals העם So the people בידם in their hand, ואת שׁופרתיהם and their trumpets: ואת כל all אישׁ every man ישׂראל Israel שׁלח and he sent אישׁ men: לאהליו unto his tent, ובשׁלשׁ those three מאות hundred האישׁ החזיק and retained ומחנה and the host מדין of Midian היה was לו מתחת beneath בעמק׃ him in the valley.}%
\verse{ויהי And it came to pass בלילה night, ההוא the same ויאמר said אליו unto יהוה that the LORD קום him, Arise, רד get thee down במחנה unto the host; כי for נתתיו I have delivered בידך׃ it into thine hand.}%
\verse{ואם But if ירא אתה thou לרדת to go down, רד go אתה thou ופרה with Phurah נערך thy servant אל to המחנה׃ the host:}%
\verse{ושׁמעת And thou shalt hear מה what ידברו they say; ואחר and afterward תחזקנה be strengthened ידיך shall thine hands וירדת to go down במחנה unto the host. וירד Then went he down הוא ופרה with Phurah נערו his servant אל unto קצה the outside החמשׁים of the armed men אשׁר that במחנה׃ in the host.}%
\verse{ומדין And the Midianites ועמלק וכל and all בני the children קדם of the east נפלים lay בעמק along in the valley כארבה like grasshoppers לרב for multitude; ולגמליהם and their camels אין without מספר number, כחול as the sand שׁעל by שׂפת side הים the sea לרב׃ for multitude.}%
\verse{ויבא was come, גדעון And when Gideon והנה behold, אישׁ a man מספר that told לרעהו unto his fellow, חלום a dream ויאמר and said, הנה Behold, חלום a dream, חלמתי I dreamed והנה and, lo, צלול a cake לחם bread שׂערים of barley מתהפך tumbled במחנה into the host מדין of Midian, ויבא and came עד unto האהל a tent, ויכהו and smote ויפל it that it fell, ויהפכהו and overturned למעלה and overturned ונפל lay along. האהל׃ it, that the tent}%
\verse{ויען answered רעהו And his fellow ויאמר and said, אין nothing else זאת This בלתי save אם save חרב the sword גדעון of Gideon בן the son יואשׁ of Joash, אישׁ a man ישׂראל of Israel: נתן delivered האלהים hath God בידו into his hand את מדין Midian, ואת כל and all המחנה׃ the host.}%
\verse{ויהי And it was כשׁמע heard גדעון when Gideon את מספר the telling החלום of the dream, ואת שׁברו and the interpretation וישׁתחו thereof, that he worshiped, וישׁב and returned אל into מחנה the host ישׂראל of Israel, ויאמר and said, קומו Arise; כי for נתן hath delivered יהוה the LORD בידכם into your hand את מחנה the host מדין׃ of Midian.}%
\verse{ויחץ And he divided את שׁלשׁ the three מאות hundred האישׁ men שׁלשׁה three ראשׁים companies, ויתן and he put שׁופרות a trumpet ביד hand, כלם in every man's וכדים pitchers, רקים with empty ולפדים and lamps בתוך within הכדים׃ the pitchers.}%
\verse{ויאמר And he said אליהם unto ממני on תראו them, Look וכן likewise: תעשׂו me, and do והנה and, behold, אנכי when I בא come בקצה to the outside המחנה of the camp, והיה it shall be כאשׁר as אעשׂה I do, כן so תעשׂון׃ shall ye do.}%
\verse{ותקעתי When I blow בשׁופר with a trumpet, אנכי I וכל and all אשׁר that אתי with ותקעתם me, then blow בשׁופרות the trumpets גם also אתם ye סביבות on every side כל of all המחנה the camp, ואמרתם and say, ליהוה of the LORD, ולגדעון׃ and of Gideon.}%
\verse{ויבא him, came גדעון So Gideon, ומאה and the hundred אישׁ men אשׁר that אתו with בקצה unto the outside המחנה of the camp ראשׁ in the beginning האשׁמרת watch; התיכונה of the middle אך and they had but הקם newly set הקימו newly set את השׁמרים the watch: ויתקעו and they blew בשׁופרות the trumpets, ונפוץ and broke הכדים the pitchers אשׁר that בידם׃ in their hands.}%
\verse{ויתקעו blew שׁלשׁת And the three הראשׁים companies בשׁופרות the trumpets, וישׁברו and broke הכדים the pitchers, ויחזיקו and held ביד hands, שׂמאולם in their left בלפדים the lamps וביד hands ימינם in their right השׁופרות and the trumpets לתקוע to blow ויקראו and they cried, חרב The sword ליהוה of the LORD, ולגדעון׃ and of Gideon.}%
\verse{ויעמדו And they stood אישׁ every man תחתיו in his place סביב round about למחנה the camp: וירץ ran, כל and all המחנה the host ויריעו and cried, ויניסו׃ and fled.}%
\verse{ויתקעו blew שׁלשׁ And the three מאות hundred השׁופרות the trumpets, וישׂם set יהוה and the LORD את חרב sword אישׁ every man's ברעהו against his fellow, ובכל even throughout all המחנה the host: וינס fled המחנה and the host עד to בית השׁטה Beth-shittah צררתה in Zererath, עד to שׂפת the border אבל מחולה of Abel-meholah, על unto טבת׃ Tabbath.}%
\verse{ויצעק gathered themselves together אישׁ And the men ישׂראל of Israel מנפתלי ומן and out of אשׁר Asher, ומן and out of כל all מנשׁה Manasseh, וירדפו and pursued אחרי after מדין׃ the Midianites.}%
\verse{ומלאכים messengers שׁלח sent גדעון And Gideon בכל throughout all הר mount אפרים Ephraim, לאמר saying, רדו Come down לקראת against מדין the Midianites, ולכדו and take להם את המים before them the waters עד unto בית ברה Beth-barah ואת הירדן and Jordan. ויצעק gathered themselves together, כל Then all אישׁ the men אפרים of Ephraim וילכדו and took את המים the waters עד unto בית ברה Beth-barah ואת הירדן׃ and Jordan.}%
\verse{וילכדו And they took שׁני two שׂרי princes מדין of the Midianites, את ערב Oreb ואת זאב and Zeeb; ויהרגו and they slew את עורב Oreb בצור upon the rock עורב Oreb, ואת זאב and Zeeb הרגו they slew ביקב at the winepress זאב of Zeeb, וירדפו and pursued אל and pursued מדין Midian, וראשׁ the heads ערב of Oreb וזאב and Zeeb הביאו and brought אל to גדעון Gideon מעבר on the other side לירדן׃ Jordan.}%
\end{biblechapter}%
\begin{biblechapter}% Judges 8
\verseWithHeading{Gideon Pursues Zebah and Zalmunna}{ויאמרו said אליו unto אישׁ And the men אפרים of Ephraim מה him, Why הדבר הזה us thus, עשׂית hast thou served לנו לבלתי us not, קראות that thou calledst לנו כי when הלכת thou wentest להלחם to fight במדין with the Midianites? ויריבון And they did chide אתו with בחזקה׃ him sharply.}%
\verse{ויאמר And he said אליהם unto מה them, What עשׂיתי have I done עתה now ככם הלוא in comparison of you? not טוב better עללות the gleaning of the grapes אפרים of Ephraim מבציר than the vintage אביעזר׃ of Abi-ezer?}%
\verse{בידכם into your hands נתן hath delivered אלהים God את שׂרי the princes מדין of Midian, את ערב Oreb ואת זאב and Zeeb: ומה and what יכלתי was I able עשׂות to do ככם אז in comparison of you? Then רפתה was abated רוחם their anger מעליו toward בדברו him, when he had said הדבר הזה׃ that.}%
\verse{ויבא came גדעון And Gideon הירדנה to Jordan, עבר passed over, הוא he, ושׁלשׁ and the three מאות hundred האישׁ men אשׁר that אתו with עיפים him, faint, ורדפים׃ yet pursuing}%
\verse{ויאמר And he said לאנשׁי סכות of Succoth, תנו Give, נא I pray you, ככרות loaves לחם of bread לעם unto the people אשׁר that ברגלי follow כי me; for עיפים faint, הם they ואנכי and I רדף am pursuing אחרי after זבח Zebah וצלמנע and Zalmunna, מלכי kings מדין׃ of Midian.}%
\verse{ויאמר said, שׂרי And the princes סכות of Succoth הכף the hands זבח of Zebah וצלמנע and Zalmunna עתה now בידך in thine hand, כי that נתן we should give לצבאך unto thine army? לחם׃ bread}%
\verse{ויאמר said, גדעון And Gideon לכן Therefore בתת hath delivered יהוה when the LORD את זבח Zebah ואת צלמנע and Zalmunna בידי into mine hand, ודשׁתי then I will tear את בשׂרכם your flesh את with קוצי the thorns המדבר of the wilderness ואת and with הברקנים׃ briers.}%
\verse{ויעל And he went up משׁם thence פנואל to Penuel, וידבר and spoke אליהם unto כזאת them likewise: ויענו answered אותו אנשׁי and the men פנואל of Penuel כאשׁר him as ענו had answered אנשׁי the men סכות׃ of Succoth}%
\verse{ויאמר And he spoke גם also לאנשׁי unto the men פנואל of Penuel, לאמר saying, בשׁובי When I come again בשׁלום in peace, אתץ I will break down את המגדל tower. הזה׃ this}%
\verse{וזבח Now Zebah וצלמנע and Zalmunna בקרקר in Karkor, ומחניהם and their hosts עמם with כחמשׁת them, about fifteen עשׂר them, about fifteen אלף thousand כל all הנותרים that were left מכל of all מחנה the hosts בני of the children קדם of the east: והנפלים for there fell מאה a hundred ועשׂרים and twenty אלף thousand אישׁ men שׁלף that drew חרב׃ sword.}%
\verse{ויעל went up גדעון And Gideon דרך by the way השׁכוני of them that dwelt באהלים in tents מקדם on the east לנבח of Nobah ויגבהה and Jogbehah, ויך and smote את המחנה the host: והמחנה for the host היה was בטח׃ secure.}%
\verse{וינוסו fled, זבח And when Zebah וצלמנע and Zalmunna וירדף he pursued אחריהם after וילכד them, and took את שׁני the two מלכי kings מדין of Midian, את זבח Zebah ואת צלמנע and Zalmunna, וכל all המחנה the host. החריד׃ and discomfited}%
\verse{וישׁב returned גדעון And Gideon בן the son יואשׁ of Joash מן from המלחמה battle מלמעלה before החרס׃ the sun}%
\verse{וילכד And caught נער a young man מאנשׁי סכות of Succoth, וישׁאלהו and inquired ויכתב of him: and he described אליו unto את שׂרי him the princes סכות of Succoth, ואת זקניה and the elders שׁבעים thereof, threescore and seventeen ושׁבעה thereof, threescore and seventeen אישׁ׃ of the men}%
\verse{ויבא And he came אל unto אנשׁי the men סכות of Succoth, ויאמר and said, הנה Behold זבח Zebah וצלמנע and Zalmunna, אשׁר with whom חרפתם ye did upbraid אותי לאמר me, saying, הכף the hands זבח of Zebah וצלמנע and Zalmunna עתה now בידך in thine hand, כי that נתן we should give לאנשׁיך unto thy men היעפים לחם׃ bread}%
\verse{ויקח And he took את זקני the elders העיר of the city, ואת קוצי and thorns המדבר of the wilderness ואת הברקנים and briers, וידע and with them he taught בהם את אנשׁי the men סכות׃ of Succoth.}%
\verse{ואת מגדל the tower פנואל of Penuel, נתץ And he beat down ויהרג and slew את אנשׁי the men העיר׃ of the city.}%
\verse{ויאמר Then said אל he unto זבח Zebah ואל צלמנע and Zalmunna, איפה What manner האנשׁים of men אשׁר whom הרגתם ye slew בתבור at Tabor? ויאמרו And they answered, כמוך As thou כמוהם אחד so they; each one כתאר resembled בני the children המלך׃ of a king.}%
\verse{ויאמר And he said, אחי my brethren, בני the sons אמי of my mother: הם They חי liveth, יהוה the LORD לו if החיתם אותםא I would not הרגתי slay אתכם׃}%
\verse{ויאמר And he said ליתר unto Jether בכורו his firstborn, קום Up, הרג slay אותם ולא not שׁלף drew הנער them. But the youth חרבו his sword: כי for ירא he feared, כי because עודנו he yet נער׃ a youth.}%
\verse{ויאמר said, זבח Then Zebah וצלמנע and Zalmunna קום Rise אתה thou, ופגע and fall בנו כי upon us: for כאישׁ as the man גבורתו his strength. ויקם arose, גדעון And Gideon ויהרג and slew את זבח Zebah ואת צלמנע and Zalmunna, ויקח and took away את השׂהרנים the ornaments אשׁר that בצוארי necks. גמליהם׃ on their camels'}%
\verse{ויאמרו said אישׁ Then the men ישׂראל of Israel אל unto גדעון Gideon, משׁל Rule בנו גם thou over us, both אתה thou, גם and בנך thy son, גם also: בן and thy son's בנך son כי for הושׁעתנו thou hast delivered מיד us from the hand מדין׃ of Midian.}%
\verse{ויאמר said אלהם unto גדעון And Gideon לא will not אמשׁל rule אני them, I בכם ולא over you, neither ימשׁל rule בני shall my son בכם יהוה over you: the LORD ימשׁל׃ shall rule}%
\verse{ויאמר said אלהם unto גדעון And Gideon אשׁאלה them, I would desire מכם of שׁאלה a request ותנו you, that ye would give לי אישׁ me every man נזם the earrings שׁללו of his prey. כי (For נזמי earrings, זהב they had golden להם כי because ישׁמעאלים Ishmaelites.) הם׃ they}%
\verse{ויאמרו And they answered, נתון נתןיפרשׂו And they spread את השׂמלה a garment, וישׁליכו and did cast שׁמה therein אישׁ every man נזם the earrings שׁללו׃ of his prey.}%
\verse{ויהי was משׁקל And the weight נזמי earrings הזהב of the golden אשׁר that שׁאל he requested אלף a thousand ושׁבע and seven מאות hundred זהב of gold; לבד beside מן beside השׂהרנים ornaments, והנטפות and collars, ובגדי raiment הארגמן and purple שׁעל that on מלכי the kings מדין of Midian, ולבד and beside מן and beside הענקות the chains אשׁר that בצוארי necks. גמליהם׃ about their camels'}%
\verse{ויעשׂ made אותו גדעון And Gideon לאפוד an ephod ויצג thereof, and put אותו בעירו it in his city, בעפרה in Ophrah: ויזנו went thither a whoring כל and all ישׂראל Israel אחריו after שׁם went thither a whoring ויהי it: which thing became לגדעון unto Gideon, ולביתו and to his house. למוקשׁ׃ a snare}%
\verse{ויכנע subdued מדין Thus was Midian לפני before בני the children ישׂראל of Israel, ולא no יספו more. לשׂאת so that they lifted up ראשׁם their heads ותשׁקט was in quietness הארץ And the country ארבעים forty שׁנה years בימי in the days גדעון׃ of Gideon.}%
\verseWithHeading{The Death of Gideon}{וילך went ירבעל And Jerubbaal בן the son יואשׁ of Joash וישׁב and dwelt בביתו׃ in his own house.}%
\verse{ולגדעון And Gideon היו had שׁבעים threescore and ten בנים sons יצאי begotten: ירכו of his body כי for נשׁים wives. רבות many היו׃ he had}%
\verse{ופילגשׁו And his concubine אשׁר that בשׁכם in Shechem, ילדה bore לו גם also היא she בן him a son, וישׂם he called את שׁמו whose name אבימלך׃ Abimelech.}%
\verse{וימת died גדעון And Gideon בן the son יואשׁ of Joash בשׂיבה old age, טובה in a good ויקבר and was buried בקבר in the sepulcher יואשׁ of Joash אביו his father, בעפרה in Ophrah אבי העזרי׃ of the Abi-ezrites.}%
\verse{ויהי And it came to pass, כאשׁר as soon as מת was dead, גדעון Gideon וישׁובו turned again, בני that the children ישׂראל of Israel ויזנו and went a whoring אחרי after הבעלים Baalim, וישׂימו and made להם בעל ברית Baal-berith לאלהים׃ their god.}%
\verse{ולא not זכרו remembered בני And the children ישׂראל of Israel את יהוה the LORD אלהיהם their God, המציל who had delivered אותם מיד them out of the hands כל of all איביהם their enemies מסביב׃ on every side:}%
\verse{ולא Neither עשׂו showed חסד they kindness עם to בית the house ירבעל of Jerubbaal, גדעון Gideon, ככל according to all הטובה the goodness אשׁר which עשׂה he had showed עם unto ישׂראל׃ Israel.}%
\end{biblechapter}%
\begin{biblechapter}% Judges 9
\verseWithHeading{Abimelech Attempts to Become King}{וילך went אבימלך And Abimelech בן the son ירבעל of Jerubbaal שׁכמה to Shechem אל unto אחי brethren, אמו his mother's וידבר and communed אליהם with ואל them, and with כל all משׁפחת the family בית of the house אבי father, אמו of his mother's לאמר׃ saying,}%
\verse{דברו Speak, נא I pray you, באזני in the ears כל of all בעלי the men שׁכם of Shechem, מה Whether טוב better לכם המשׁל reign בכם שׁבעים threescore and ten אישׁ persons, כל for you, either that all בני the sons ירבעל of Jerubbaal, אם over you, or משׁל reign בכם אישׁ אחד that one וזכרתם over you? remember כי also that עצמכם your bone ובשׂרכם and your flesh. אני׃ I}%
\verse{וידברו spoke אחי brethren אמו And his mother's עליו of באזני him in the ears כל of all בעלי the men שׁכם of Shechem את כל all הדברים words: האלה these ויט inclined to follow לבם and their hearts אחרי inclined to follow אבימלך Abimelech; כי for אמרו they said, אחינו our brother. הוא׃ He}%
\verse{ויתנו And they gave לו שׁבעים him threescore and ten כסף of silver מבית out of the house בעל ברית of Baal-berith, וישׂכר hired בהם אבימלך wherewith Abimelech אנשׁים ריקים vain ופחזים and light וילכו which followed אחריו׃ which followed}%
\verse{ויבא And he went בית house אביו unto his father's עפרתה at Ophrah, ויהרג and slew את אחיו his brethren בני the sons ירבעל of Jerubbaal, שׁבעים threescore and ten אישׁ persons, על upon אבן stone: אחת one ויותר was left; יותם notwithstanding yet Jotham בן son ירבעל of Jerubbaal הקטן the youngest כי for נחבא׃ he hid himself.}%
\verse{ויאספו gathered together, כל And all בעלי the men שׁכם of Shechem וכל and all בית מלואילכו and went, וימליכו and made את אבימלך Abimelech למלך king, עם by אלון the plain מצב of the pillar אשׁר that בשׁכם׃ in Shechem.}%
\verse{ויגדו And when they told ליותם to Jotham, וילך he went ויעמד and stood בראשׁ in the top הר of mount גרזים Gerizim, וישׂא and lifted up קולו his voice, ויקרא and cried, ויאמר and said להם שׁמעו unto them, Hearken אלי unto בעלי me, ye men שׁכם of Shechem, וישׁמע may hearken אליכם unto אלהים׃ that God}%
\verse{הלוך went forth הלכו went forth העצים The trees למשׁח to anoint עליהם over מלך a king ויאמרו them; and they said לזית unto the olive tree, מלוכה Reign עלינו׃ thou over}%
\verse{ויאמר said להם הזית But the olive tree החדלתי unto them, Should I leave את דשׁני my fatness, אשׁר wherewith בי יכבדו by me they honor אלהים God ואנשׁים and man, והלכתי and go לנוע to be promoted על over העצים׃ the trees?}%
\verse{ויאמרו said העצים And the trees לתאנה to the fig tree, לכי Come את thou, מלכי reign עלינו׃ over}%
\verse{ותאמר said להם התאנה But the fig tree החדלתי unto them, Should I forsake את מתקי my sweetness, ואת תנובתי fruit, הטובה and my good והלכתי and go לנוע to be promoted על over העצים׃ the trees?}%
\verse{ויאמרו Then said העצים the trees לגפן unto the vine, לכי Come את thou, מלוכי reign עלינו׃ over}%
\verse{ותאמר said להם הגפן And the vine החדלתי unto them, Should I leave את תירושׁי my wine, המשׂמח which cheereth אלהים God ואנשׁים and man, והלכתי and go לנוע to be promoted על over העצים׃ the trees?}%
\verse{ויאמרו Then said כל all העצים the trees אל unto האטד the bramble, לך Come אתה thou, מלך reign עלינו׃ over}%
\verse{ויאמר said האטד And the bramble אל unto העצים the trees, אם If באמת in truth אתם ye משׁחים anoint אתי למלך me king עליכם over באו you, come חסו put your trust בצלי in my shadow: ואם and if אין not, תצא come out אשׁ let fire מן of האטד the bramble, ותאכל and devour את ארזי the cedars הלבנון׃ of Lebanon.}%
\verse{ועתה Now אם therefore, if באמת truly ובתמים and sincerely, עשׂיתם ye have done ותמליכו אתבימלך ואם and if טובה עשׂיתם ye have dealt עם with ירבעל Jerubbaal ועם ביתו and his house, ואם כגמול unto him according to the deserving ידיו of his hands; עשׂיתם׃ and have done}%
\verse{אשׁר (For נלחם fought אבי my father עליכם for וישׁלך you, and adventured את נפשׁו his life מנגד far, ויצל and delivered אתכם מיד you out of the hand מדין׃ of Midian:}%
\verse{ואתם And ye קמתם are risen up על against בית house אבי my father's היום this day, ותהרגו and have slain את בניו his sons, שׁבעים threescore and ten אישׁ persons, על upon אבן stone, אחת one ותמליכו king את אבימלך and have made Abimelech, בן the son אמתו of his maidservant, על over בעלי the men שׁכם of Shechem, כי because אחיכם your brother;) הוא׃ he}%
\verse{ואם If באמת truly ובתמים and sincerely עשׂיתם ye then have dealt עם with ירבעל Jerubbaal ועם and with ביתו his house היום day, הזה this שׂמחו rejoice באבימלך ye in Abimelech, וישׂמח rejoice גם also הוא׃ and let him}%
\verse{ואם But if אין not, תצא come out אשׁ let fire מאבימלך Abimelech. ותאכל and devour את בעלי the men שׁכם of Shechem, ואת בית and the house מלוא of Millo; ותצא come out אשׁ and let fire מבעלי from the men שׁכם of Shechem, ומבית and from the house מלוא of Millo, ותאכל and devour את אבימלך׃}%
\verse{וינס ran away, יותם And Jotham ויברח and fled, וילך and went בארה to Beer, וישׁב and dwelt שׁם there, מפני for fear אבימלך of Abimelech אחיו׃ his brother.}%
\verseWithHeading{The Downfall of Shechem and Abimelech}{וישׂר had reigned אבימלך When Abimelech על over ישׂראל Israel, שׁלשׁ three שׁנים׃ years}%
\verse{וישׁלח sent אלהים Then God רוח spirit רעה an evil בין between אבימלך Abimelech ובין בעלי and the men שׁכם of Shechem; ויבגדו dealt treacherously בעלי and the men שׁכם of Shechem באבימלך׃ with Abimelech:}%
\verse{לבוא might come, חמס That the cruelty שׁבעים to the threescore and ten בני sons ירבעל of Jerubbaal ודמם and their blood לשׂום be laid על upon אבימלך Abimelech אחיהם their brother, אשׁר which הרג slew אותם ועל them; and upon בעלי the men שׁכם of Shechem, אשׁר which חזקו אתדיו להרג him in the killing את אחיו׃ of his brethren.}%
\verse{וישׂימו set לו בעלי And the men שׁכם of Shechem מארבים liers in wait על for him in ראשׁי the top ההרים of the mountains, ויגזלו and they robbed את כל all אשׁר that יעבר came along עליהם by בדרך that way ויגד them: and it was told לאבימלך׃ Abimelech.}%
\verse{ויבא came געל And Gaal בן the son עבד of Ebed ואחיו with his brethren, ויעברו and went over בשׁכם to Shechem: ויבטחו put their confidence בו בעלי and the men שׁכם׃ of Shechem}%
\verse{ויצאו And they went out השׂדה into the fields, ויבצרו and gathered את כרמיהם their vineyards, וידרכו and trod ויעשׂו and made הלולים merry, ויבאו and went into בית the house אלהיהם of their god, ויאכלו and did eat וישׁתו and drink, ויקללו and cursed את אבימלך׃ Abimelech.}%
\verse{ויאמר said, געל And Gaal בן the son עבד of Ebed מי Who אבימלך Abimelech, ומי and who שׁכם Shechem, כי that נעבדנו we should serve הלא him? not בן the son ירבעל of Jerubbaal? וזבל and Zebul פקידו his officer? עבדו serve את אנשׁי the men חמור of Hamor אבי the father שׁכם ומדוע for why נעבדנו serve אנחנו׃ should we}%
\verse{ומי יתןת העם people הזה this בידי were under my hand! ואסירה then would I remove את אבימלך Abimelech. ויאמר And he said לאבימלך to Abimelech, רבה Increase צבאך thine army, וצאה׃ and come out.}%
\verse{וישׁמע heard זבל And when Zebul שׂר the ruler העיר of the city את דברי the words געל of Gaal בן the son עבד of Ebed, ויחר was kindled. אפו׃ his anger}%
\verse{וישׁלח And he sent מלאכים messengers אל unto אבימלך Abimelech בתרמה privily, לאמר saying, הנה Behold, געל Gaal בן the son עבד of Ebed ואחיו and his brethren באים be come שׁכמה to Shechem; והנם and, behold, צרים they fortify את העיר the city עליך׃ against}%
\verse{ועתה Now קום therefore up לילה by night, אתה thou והעם and the people אשׁר that אתך with וארב thee, and lie in wait בשׂדה׃ in the field:}%
\verse{והיה And it shall be, בבקר in the morning, כזרח is up, השׁמשׁ as soon as the sun תשׁכים thou shalt rise early, ופשׁטת and set על upon העיר the city: והנה and, behold, הוא he והעם and the people אשׁר that אתו with יצאים him come out אליך against ועשׂית thee, then mayest thou do לו כאשׁר to them as תמצא thou shalt find ידך׃ occasion.}%
\verse{ויקם rose up, אבימלך And Abimelech וכל and all העם the people אשׁר that עמו with לילה him, by night, ויארבו and they laid wait על against שׁכם Shechem ארבעה in four ראשׁים׃ companies.}%
\verse{ויצא went out, געל And Gaal בן the son עבד of Ebed ויעמד and stood פתח in the entering שׁער of the gate העיר of the city: ויקם rose up, אבימלך and Abimelech והעם and the people אשׁר that אתו with מן him, from המארב׃ lying in wait.}%
\verse{וירא saw געל And when Gaal את העם the people, ויאמר he said אל to זבל Zebul, הנה Behold, עם there come people down יורד there come people down מראשׁי from the top ההרים of the mountains. ויאמר said אליו unto זבל And Zebul את צל the shadow ההרים of the mountains אתה him, Thou ראה seest כאנשׁים׃ as men.}%
\verse{ויסף again עוד געל And Gaal לדבר spoke ויאמר and said, הנה See עם there come people down יורדים there come people down מעם by טבור the middle הארץ of the land, וראשׁ company אחד and another בא come מדרך אלון the plain מעוננים׃ of Meonenim.}%
\verse{ויאמר Then said אליו unto זבל Zebul איה him, Where אפוא now פיך thy mouth, אשׁר wherewith תאמר thou saidst, מי Who אבימלך Abimelech, כי that נעבדנו we should serve הלא him? not זה this העם the people אשׁר that מאסתה thou hast despised? בו צא go out, נא I pray עתה now, והלחם׃ and fight}%
\verse{ויצא went out געל And Gaal לפני before בעלי the men שׁכם of Shechem, וילחם and fought באבימלך׃ with Abimelech.}%
\verse{וירדפהו chased אבימלך And Abimelech וינס him, and he fled מפניו before ויפלו were overthrown חללים wounded, רבים him, and many עד unto פתח the entering השׁער׃ of the gate.}%
\verse{וישׁב dwelt אבימלך And Abimelech בארומה at Arumah: ויגרשׁ thrust out זבל and Zebul את געל Gaal ואת אחיו and his brethren, משׁבת that they should not dwell בשׁכם׃ in Shechem.}%
\verse{ויהי And it came to pass ממחרת on the morrow, ויצא went out העם that the people השׂדה into the field; ויגדו and they told לאבימלך׃ Abimelech.}%
\verse{ויקח And he took את העם the people, ויחצם and divided לשׁלשׁה them into three ראשׁים companies, ויארב and laid wait בשׂדה in the field, וירא and looked, והנה and, behold, העם the people יצא come forth מן out of העיר the city; ויקם and he rose up עליהם against ויכם׃ them, and smote}%
\verse{ואבימלך And Abimelech, והראשׁים and the company אשׁר that עמו with פשׁטו him, rushed forward, ויעמדו and stood פתח in the entering שׁער of the gate העיר of the city: ושׁני and the two הראשׁים companies פשׁטו ran על upon כל all אשׁר that בשׂדה in the fields, ויכום׃ and slew}%
\verse{ואבימלך And Abimelech נלחם fought בעיר against the city כל all היום day; ההוא that וילכד and he took את העיר the city, ואת העם the people אשׁר that בה הרג and slew ויתץ therein, and beat down את העיר the city, ויזרעה and sowed מלח׃ it with salt.}%
\verse{וישׁמעו heard כל And when all בעלי the men מגדל of the tower שׁכם of Shechem ויבאו they entered אל into צריח a hold בית of the house אל of the god ברית׃ Berith.}%
\verse{ויגד And it was told לאבימלך Abimelech, כי that התקבצו were gathered together. כל all בעלי the men מגדל of the tower שׁכם׃ of Shechem}%
\verse{ויעל got him up אבימלך And Abimelech הר to mount צלמון Zalmon, הוא he וכל and all העם the people אשׁר that אתו with ויקח took אבימלך him; and Abimelech את הקרדמות an axe בידו in his hand, ויכרת and cut down שׂוכת a bough עצים from the trees, וישׂאה and took וישׂם it, and laid על on שׁכמו his shoulder, ויאמר and said אל unto העם the people אשׁר that עמו with מה him, What ראיתם ye have seen עשׂיתי me do, מהרו make haste, עשׂו do כמוני׃ as I}%
\verse{ויכרתו cut down גם likewise כל And all העם the people אישׁ every man שׂוכה his bough, וילכו and followed אחרי and followed אבימלך Abimelech, וישׂימו and put על to הצריח the hold, ויציתו and set עליהם upon את הצריח the hold באשׁ on fire וימתו died גם also, כל them; so that all אנשׁי the men מגדל of the tower שׁכם of Shechem כאלף about a thousand אישׁ men ואשׁה׃ and women.}%
\verse{וילך Then went אבימלך Abimelech אל to תבץ Thebez, ויחן and encamped בתבץ against Thebez, וילכדה׃ and took}%
\verse{ומגדל tower עז a strong היה But there was בתוך within העיר the city, וינסו fled שׁמה and thither כל all האנשׁים והנשׁים and women, וכל and all בעלי they העיר of the city, ויסגרו and shut בעדם to ויעלו them, and got them up על to גג the top המגדל׃ of the tower.}%
\verse{ויבא came אבימלך And Abimelech עד unto המגדל the tower, וילחם and fought בו ויגשׁ against it, and went hard עד unto פתח the door המגדל of the tower לשׂרפו to burn באשׁ׃ it with fire.}%
\verse{ותשׁלך cast אשׁה woman אחת And a certain פלח a piece רכב of a millstone על upon ראשׁ head, אבימלך Abimelech's ותרץ and all to broke את גלגלתו׃ his skull.}%
\verse{ויקרא Then he called מהרה hastily אל unto הנער the young man נשׂא his armorbearer, כליו his armorbearer, ויאמר and said לו שׁלף unto him, Draw חרבך thy sword, ומותתני and slay פן me, that יאמרו men say לי אשׁה not of me, A woman הרגתהו slew וידקרהו thrust him through, נערו him. And his young man וימת׃ and he died.}%
\verse{ויראו saw אישׁ And when the men ישׂראל of Israel כי that מת was dead, אבימלך Abimelech וילכו they departed אישׁ every man למקמו׃ unto his place.}%
\verse{וישׁב rendered אלהים Thus God את רעת the wickedness אבימלך of Abimelech, אשׁר which עשׂה he did לאביו unto his father, להרג in slaying את שׁבעים his seventy אחיו׃ brethren:}%
\verse{ואת כל And all רעת the evil אנשׁי of the men שׁכם of Shechem השׁיב render אלהים did God בראשׁם upon their heads: ותבא them came אליהם and upon קללת the curse יותם of Jotham בן the son ירבעל׃ of Jerubbaal.}%
\end{biblechapter}%
\begin{biblechapter}% Judges 10
\verseWithHeading{The Philistines and Ammonites Afflict the Israelites}{ויקם there arose אחרי And after אבימלך Abimelech להושׁיע to defend את ישׂראל Israel תולע Tola בן the son פואה of Puah, בן the son דודו of Dodo, אישׁ a man ישׂשכר of Issachar; והוא and he ישׁב dwelt בשׁמיר in Shamir בהר in mount אפרים׃ Ephraim.}%
\verse{וישׁפט And he judged את ישׂראל Israel עשׂרים twenty ושׁלשׁ and three שׁנה years, וימת and died, ויקבר and was buried בשׁמיר׃ in Shamir.}%
\verse{ויקם him arose אחריו And after יאיר Jair, הגלעדי a Gileadite, וישׁפט and judged את ישׂראל Israel עשׂרים twenty ושׁתים and two שׁנה׃ years.}%
\verse{ויהי And he had לו שׁלשׁים thirty בנים sons רכבים that rode על on שׁלשׁים thirty עירים ass colts, ושׁלשׁים and they had thirty עירים להםהם יקראו which are called חות יאיר Havoth-jair עד unto היום day, הזה this אשׁר which בארץ in the land הגלעד׃ of Gilead.}%
\verse{וימת died, יאיר And Jair ויקבר and was buried בקמון׃ in Camon.}%
\verse{ויספו again בני And the children ישׂראל of Israel לעשׂות did הרע evil בעיני in the sight יהוה of the LORD, ויעבדו and served את הבעלים Baalim, ואת העשׁתרות and Ashtaroth, ואת אלהי and the gods ארם of Syria, ואת אלהי and the gods צידון of Zidon, ואת אלהי and the gods מואב of Moab, ואת אלהי and the gods בני of the children עמון of Ammon, ואת אלהי and the gods פלשׁתים of the Philistines, ויעזבו and forsook את יהוה the LORD, ולא not עבדוהו׃ and served}%
\verse{ויחר was hot אף And the anger יהוה of the LORD בישׂראל against Israel, וימכרם and he sold ביד them into the hands פלשׁתים of the Philistines, וביד and into the hands בני of the children עמון׃ of Ammon.}%
\verse{וירעצו they vexed וירצצו and oppressed את בני the children ישׂראל of Israel: בשׁנה year ההיא And that שׁמנה eighteen עשׂרה eighteen שׁנה years, את כל all בני the children ישׂראל of Israel אשׁר that בעבר on the other side הירדן Jordan בארץ in the land האמרי of the Amorites, אשׁר which בגלעד׃ in Gilead.}%
\verse{ויעברו passed over בני Moreover the children עמון of Ammon את הירדן Jordan להלחם to fight גם also ביהודה against Judah, ובבנימין and against Benjamin, ובבית and against the house אפרים of Ephraim; ותצר distressed. לישׂראל so that Israel מאד׃ was sore}%
\verse{ויזעקו cried בני And the children ישׂראל of Israel אל unto יהוה the LORD, לאמר saying, חטאנו We have sinned לך וכי against thee, both because עזבנו we have forsaken את אלהינו our God, ונעבד and also served את הבעלים׃ Baalim.}%
\verse{ויאמר said יהוה And the LORD אל בני the children ישׂראל of Israel, הלא not ממצרים ומן and from האמרי the Amorites, ומן from בני the children עמון of Ammon, ומן and from פלשׁתים׃ the Philistines?}%
\verse{וצידונים The Zidonians ועמלק also, and the Amalekites, ומעון and the Maonites, לחצו did oppress אתכם ותצעקו you; and ye cried אלי to ואושׁיעה me, and I delivered אתכם מידם׃ you out of their hand.}%
\verse{ואתם Yet ye עזבתם have forsaken אותי ותעבדו me, and served אלהים gods: אחרים other לכן wherefore לא you no אוסיף more. להושׁיע I will deliver אתכם׃}%
\verse{לכו Go וזעקו and cry אל unto האלהים the gods אשׁר which בחרתם ye have chosen; בם המה let them יושׁיעו deliver לכם בעת you in the time צרתכם׃ of your tribulation.}%
\verse{ויאמרו said בני And the children ישׂראל of Israel אל unto יהוה the LORD, חטאנו We have sinned: עשׂה do אתה thou לנו ככל unto us whatsoever הטוב seemeth good בעיניך unto אך us only, הצילנו thee; deliver נא we pray thee, היום day. הזה׃ this}%
\verse{ויסירו And they put away את אלהי gods הנכר the strange מקרבם from among ויעבדו them, and served את יהוה the LORD: ותקצר was grieved נפשׁו and his soul בעמל for the misery ישׂראל׃ of Israel.}%
\verse{ויצעקו were gathered together, בני Then the children עמון of Ammon ויחנו and encamped בגלעד in Gilead. ויאספו assembled themselves together, בני And the children ישׂראל of Israel ויחנו and encamped במצפה׃ in Mizpeh.}%
\verse{ויאמרו said העם And the people שׂרי princes גלעד of Gilead אישׁ one אל to רעהו another, מי What האישׁ man אשׁר that יחל will begin להלחם to fight בבני against the children עמון of Ammon? יהיה he shall be לראשׁ head לכל over all ישׁבי the inhabitants גלעד׃ of Gilead.}%
\end{biblechapter}%
\begin{biblechapter}% Judges 11
\verseWithHeading{Jephthah}{ויפתח Now Jephthah הגלעדי the Gileadite היה was גבור a mighty man חיל of valor, והוא and he בן the son אשׁה of a harlot: זונה of a harlot: ויולד begot גלעד and Gilead את יפתח׃ Jephthah.}%
\verse{ותלד bore אשׁת wife גלעד And Gilead's לו בנים him sons; ויגדלו grew up, בני sons האשׁה and his wife's ויגרשׁו and they thrust out את יפתח Jephthah, ויאמרו and said לו לא unto him, Thou shalt not תנחל inherit בבית house; אבינו in our father's כי for בן the son אשׁה woman. אחרת of a strange אתה׃ thou}%
\verse{ויברח fled יפתח Then Jephthah מפני from אחיו his brethren, וישׁב and dwelt בארץ in the land טוב of Tob: ויתלקטו and there were gathered אל to יפתח Jephthah, אנשׁים ריקים vain ויצאו and went out עמו׃ with}%
\verse{ויהי And it came to pass מימים in process of time, וילחמו made war בני that the children עמון of Ammon עם against ישׂראל׃ Israel.}%
\verse{ויהי And it was כאשׁר so, that when נלחמו made war בני the children עמון of Ammon עם against ישׂראל Israel, וילכו went זקני the elders גלעד of Gilead לקחת to fetch את יפתח Jephthah מארץ out of the land טוב׃ of Tob:}%
\verse{ויאמרו And they said ליפתח unto Jephthah, לכה Come, והייתה and be לנו לקצין our captain, ונלחמה that we may fight בבני with the children עמון׃ of Ammon.}%
\verse{ויאמר said יפתח And Jephthah לזקני unto the elders גלעד of Gilead, הלא Did not אתם ye שׂנאתם hate אותי ותגרשׁוני me, and expel מבית house? אבי me out of my father's ומדוע and why באתם are ye come אלי unto עתה me now כאשׁר when צר׃ ye are in distress?}%
\verse{ויאמרו said זקני And the elders גלעד of Gilead אל unto יפתח Jephthah, לכן Therefore עתה thee now, שׁבנו we turn again אליך to והלכת that thou mayest go עמנו with ונלחמת us, and fight בבני against the children עמון of Ammon, והיית and be לנו לראשׁ our head לכל over all ישׁבי the inhabitants גלעד׃ of Gilead.}%
\verse{ויאמר said יפתח And Jephthah אל unto זקני the elders גלעד of Gilead, אם If משׁיבים אתם ye אותי להלחם to fight בבני against the children עמון of Ammon, ונתן deliver יהוה and the LORD אותם לפני them before אנכי me, shall I אהיה be לכם לראשׁ׃ your head?}%
\verse{ויאמרו said זקני And the elders גלעד of Gilead אל unto יפתח Jephthah, יהוה The LORD יהיה be שׁמע witness בינותינו between אם us, if לא not כדברך according to thy words. כן so נעשׂה׃ we do}%
\verse{וילך went יפתח Then Jephthah עם with זקני the elders גלעד of Gilead, וישׂימו made העם and the people אותו עליהם over לראשׁ him head ולקצין and captain וידבר uttered יפתח them: and Jephthah את כל all דבריו his words לפני before יהוה the LORD במצפה׃ in Mizpeh.}%
\verse{וישׁלח sent יפתח And Jephthah מלאכים messengers אל unto מלך the king בני of the children עמון of Ammon, לאמר saying, מה What לי ולך כי hast thou to do with me, that באת thou art come אלי against להלחם me to fight בארצי׃ in my land?}%
\verse{ויאמר answered מלך And the king בני of the children עמון of Ammon אל unto מלאכי the messengers יפתח of Jephthah, כי Because לקח took away ישׂראל Israel את ארצי my land, בעלותו when they came up ממצרים מארנוןעד even unto היבק Jabbok, ועד and unto הירדן Jordan: ועתה now השׁיבה אתהןשׁלום׃ peaceably.}%
\verse{ויוסף again עוד again יפתח And Jephthah וישׁלח sent מלאכים messengers אל unto מלך the king בני of the children עמון׃ of Ammon:}%
\verse{ויאמר And said לו כה unto him, Thus אמר saith יפתח Jephthah, לא took not away לקח took not away ישׂראל Israel את ארץ the land מואב of Moab, ואת ארץ nor the land בני of the children עמון׃ of Ammon:}%
\verse{כי But בעלותם came up ממצרים וילך and walked ישׂראל when Israel במדבר through the wilderness עד unto ים sea, סוף the Red ויבא and came קדשׁה׃ to Kadesh;}%
\verse{וישׁלח sent ישׂראל Then Israel מלאכים messengers אל unto מלך the king אדום of Edom, לאמר saying, אעברה pass נא Let me, I pray thee, בארצך through thy land: ולא would not שׁמע hearken מלך but the king אדום of Edom וגם And in like manner אל unto מלך the king מואב of Moab: שׁלח they sent ולא not אבה but he would וישׁב abode ישׂראל and Israel בקדשׁ׃ in Kadesh.}%
\verse{וילך Then they went along במדבר through the wilderness, ויסב and compassed את ארץ the land אדום of Edom, ואת ארץ and the land מואב of Moab, ויבא and came ממזרח שׁמשׁארץ of the land מואב of Moab, ויחנון and pitched בעבר on the other side ארנון of Arnon, ולא not באו but came בגבול within the border מואב of Moab: כי for ארנון Arnon גבול the border מואב׃ of Moab.}%
\verse{וישׁלח sent ישׂראל And Israel מלאכים messengers אל unto סיחון Sihon מלך king האמרי of the Amorites, מלך the king חשׁבון of Heshbon; ויאמר said לו ישׂראל and Israel נעברה unto him, Let us pass, נא we pray thee, בארצך through thy land עד into מקומי׃ my place.}%
\verse{ולא not האמין trusted סיחון But Sihon את ישׂראל Israel עבר to pass בגבלו through his coast: ויאסף סיחון but Sihon את כלמו ויחנו and pitched ביהצה in Jahaz, וילחם and fought עם against ישׂראל׃ Israel.}%
\verse{ויתן delivered יהוה And the LORD אלהי God ישׂראל of Israel את סיחון Sihon ואת כל and all עמו his people ביד into the hand ישׂראל of Israel, ויכום and they smote ויירשׁ possessed ישׂראל them: so Israel את כל all ארץ the land האמרי of the Amorites, יושׁב the inhabitants הארץ country. ההיא׃ of that}%
\verse{ויירשׁו And they possessed את כל all גבול the coasts האמרי of the Amorites, מארנון ועד even unto היבק Jabbok, ומן and from המדבר the wilderness ועד even unto הירדן׃ Jordan.}%
\verse{ועתה So now יהוה the LORD אלהי God ישׂראל of Israel הורישׁ hath dispossessed את האמרי the Amorites מפני from before עמו his people ישׂראל Israel, ואתה and shouldest thou תירשׁנו׃ possess}%
\verse{הלא Wilt not את אשׁר that which יורישׁך thou possess כמושׁ Chemosh אלהיך thy god אותו תירשׁ giveth thee to possess? ואת כלשׁר הורישׁ shall drive out יהוה the LORD אלהינו our God מפנינו from before אותו נירשׁ׃ us, them will we possess.}%
\verse{ועתה And now הטוב any thing better טוב any thing better אתה thou מבלק בן the son צפור of Zippor, מלך king מואב of Moab? הרוב did he ever strive רב did he ever strive עם against ישׂראל Israel, אם or נלחם did he ever fight נלחם׃ did he ever fight}%
\verse{בשׁבת dwelt ישׂראל While Israel בחשׁבון in Heshbon ובבנותיה and her towns, ובערעור and in Aroer ובבנותיה and her towns, ובכל and in all הערים the cities אשׁר that על along by ידי the coasts ארנון of Arnon, שׁלשׁ three מאות hundred שׁנה years? ומדוע why לא therefore did ye not הצלתם recover בעת time? ההיא׃ within that}%
\verse{ואנכי Wherefore I לא have not חטאתי sinned לך ואתה against thee, but thou עשׂה doest אתי רעה me wrong להלחם to war בי ישׁפט the Judge יהוה against me: the LORD השׁפט be judge היום this day בין between בני the children ישׂראל of Israel ובין בני and the children עמון׃ of Ammon.}%
\verse{ולא not שׁמע hearkened מלך Howbeit the king בני of the children עמון of Ammon אל unto דברי the words יפתח of Jephthah אשׁר which שׁלח he sent אליו׃ he sent}%
\verseWithHeading{Jephthah Makes a Vow}{ותהי came על upon יפתח Jephthah, רוח Then the Spirit יהוה of the LORD ויעבר and he passed over את הגלעד Gilead, ואת מנשׁה and Manasseh, ויעבר and passed over את מצפה Mizpeh גלעד of Gilead, וממצפה גלעד of Gilead עבר he passed over בני the children עמון׃ of Ammon.}%
\verse{וידר vowed יפתח And Jephthah נדר a vow ליהוה unto the LORD, ויאמר and said, אם If נתון thou shalt without fail deliver תתן thou shalt without fail deliver את בני the children עמון of Ammon בידי׃ into mine hands,}%
\verse{והיה Then it shall be, היוצא cometh forth אשׁר that whatsoever יצא מדלתי of the doors ביתי of my house לקראתי to meet בשׁובי me, when I return בשׁלום in peace מבני from the children עמון of Ammon, והיה shall surely be ליהוה the LORD's, והעליתהו and I will offer it up עולה׃ for a burnt offering.}%
\verse{ויעבר passed over יפתח So Jephthah אל unto בני the children עמון of Ammon להלחם to fight בם ויתנם delivered יהוה against them; and the LORD בידו׃ them into his hands.}%
\verse{ויכם And he smote מערוער ועד even till בואך thou come מנית to Minnith, עשׂרים twenty עיר cities, ועד and unto אבל the plain כרמים of the vineyards, מכה slaughter. גדולה great מאד with a very ויכנעו were subdued בני Thus the children עמון of Ammon מפני before בני the children ישׂראל׃ of Israel.}%
\verse{ויבא came יפתח And Jephthah המצפה to Mizpeh אל unto ביתו his house, והנה and, behold, בתו his daughter יצאת came out לקראתו to meet בתפים him with timbrels ובמחלות and with dances: ורק היא and she יחידה only child; אין her he had neither לו ממנו beside בן son או nor בת׃ daughter.}%
\verse{ויהי And it came to pass, כראותו when he saw אותה ויקרע her, that he rent את בגדיו his clothes, ויאמר and said, אהה Alas, בתי my daughter! הכרע thou hast brought me very low, הכרעתני thou hast brought me very low, ואת and thou היית art בעכרי one of them that trouble ואנכי me: for I פציתי have opened פי my mouth אל unto יהוה the LORD, ולא אוכלשׁוב׃ go back.}%
\verse{ותאמר And she said אליו unto אבי him, My father, פציתה thou hast opened את פיך thy mouth אל unto יהוה the LORD, עשׂה do לי כאשׁר to me according to that which יצא hath proceeded out מפיך of thy mouth; אחרי forasmuch אשׁר forasmuch עשׂה hath taken לך יהוה as the LORD נקמות vengeance מאיביך for thee of thine enemies, מבני of the children עמון׃ of Ammon.}%
\verse{ותאמר And she said אל unto אביה her father, יעשׂה be done לי הדבר thing הזה Let this הרפה for me: let me alone ממני for me: let me alone שׁנים two חדשׁים months, ואלכה that I may go up וירדתי and down על upon ההרים the mountains, ואבכה and bewail על and bewail בתולי my virginity, אנכי I ורעיתי׃}%
\verse{ויאמר And he said, לכי Go. וישׁלח אותהׁני two חדשׁים months: ותלך went היא and she ורעותיה with her companions, ותבך and bewailed על and bewailed בתוליה her virginity על upon ההרים׃ the mountains.}%
\verse{ויהי And it came to pass מקץ at the end שׁנים of two חדשׁים months, ותשׁב that she returned אל unto אביה her father, ויעשׂ who did לה את נדרו with her to his vow אשׁר which נדר he had vowed: והיא and she לא no ידעה knew אישׁ man. ותהי And it was חק a custom בישׂראל׃ in Israel,}%
\verse{מימים days ימימה תלכנה went בנות the daughters ישׂראל of Israel לתנות to lament לבת the daughter יפתח of Jephthah הגלעדי the Gileadite ארבעת four ימים בשׁנה׃ in a year.}%
\end{biblechapter}%
\begin{biblechapter}% Judges 12
\verseWithHeading{Tribal Conflict Between Gilead and Ephraim}{ויצעק gathered themselves together, אישׁ And the men אפרים of Ephraim ויעבר and went צפונה northward, ויאמרו and said ליפתח unto Jephthah, מדוע Wherefore עברת passedst thou over להלחם to fight בבני against the children עמון of Ammon, ולנו לא and didst not קראת call ללכת us to go עמך with ביתך thine house נשׂרף thee? we will burn עליך upon באשׁ׃ thee with fire.}%
\verse{ויאמר said יפתח And Jephthah אליהם unto אישׁ at great strife ריב הייתי were אני them, I ועמי and my people ובני with the children עמון of Ammon; מאד ואזעק and when I called אתכם ולא me not הושׁעתם you, ye delivered אותי מידם׃ out of their hands.}%
\verse{ואראה And when I saw כי that אינך not, מושׁיע ye delivered ואשׂימה I put נפשׁי my life בכפי in my hands, ואעברה and passed over אל against בני the children עמון >> r="#000000">of Ammon, ויתנם delivered יהוה and the LORD בידי them into my hand: ולמה wherefore עליתם then are ye come up אלי unto היום day, הזה me this להלחם׃ to fight}%
\verse{ויקבץ gathered together יפתח Then Jephthah את כל all אנשׁי the men גלעד of Gilead, וילחם and fought את with אפרים Ephraim: ויכו smote אנשׁי and the men גלעד of Gilead את אפרים Ephraim, כי because אמרו they said, פליטי fugitives אפרים of Ephraim אתם Ye גלעד Gileadites בתוך among אפרים the Ephraimites, בתוך among מנשׁה׃ the Manassites.}%
\verse{וילכד took גלעד And the Gileadites את מעברות the passages הירדן of Jordan לאפרים before the Ephraimites: והיה and it was כי that when יאמרו said, פליטי which were escaped אפרים those Ephraimites אעברה Let me go over; ויאמרו said לו אנשׁי גלעד of Gilead האפרתי an Ephraimite? אתה unto him, thou ויאמר If he said, לא׃ Nay;}%
\verse{ויאמרו Then said לו אמר they unto him, Say נא now שׁבלת Shibboleth: ויאמר and he said סבלת Sibboleth: ולא for he could not יכין frame לדבר to pronounce כן right. ויאחזו Then they took אותו וישׁחטוהו him, and slew אל him at מעברות the passages הירדן of Jordan: ויפל and there fell בעת time ההיא at that מאפרים ארבעים forty ושׁנים and two אלף׃ thousand.}%
\verse{וישׁפט judged יפתח And Jephthah את ישׂראל Israel שׁשׁ six שׁנים years. וימת Then died יפתח Jephthah הגלעדי the Gileadite, ויקבר and was buried בערי in the cities גלעד׃ of Gilead.}%
\verseWithHeading{Ibzan, Elon, and Abdon}{וישׁפט judged אחריו And after את ישׂראל Israel. אבצן him Ibzan מבית לחם׃}%
\verse{ויהי And he had לו שׁלשׁים thirty בנים sons, ושׁלשׁים and thirty בנות daughters, שׁלח he sent החוצה abroad, ושׁלשׁים thirty בנות daughters הביא and took in לבניו for his sons. מן from החוץ abroad וישׁפט And he judged את ישׂראל Israel שׁבע seven שׁנים׃ years.}%
\verse{וימת Then died אבצן Ibzan, ויקבר and was buried בבית לחם׃ at Bethlehem.}%
\verse{וישׁפט judged אחריו And after את ישׂראל Israel; אילון him Elon, הזבולני a Zebulonite, וישׁפט and he judged את ישׂראל Israel עשׂר ten שׁנים׃ years.}%
\verse{וימת died, אלון הזבולני the Zebulonite ויקבר and was buried באילון in Aijalon בארץ in the country זבולן׃}%
\verse{וישׁפט judged אחריו And after את ישׂראל Israel. עבדון him Abdon בן the son הלל of Hillel, הפרעתוני׃ a Pirathonite,}%
\verse{ויהי And he had לו ארבעים forty בנים sons ושׁלשׁים and thirty בני nephews, בנים nephews, רכבים that rode על on שׁבעים threescore and ten עירם ass colts: וישׁפט and he judged את ישׂראל Israel שׁמנה eight שׁנים׃ years.}%
\verse{וימת died, עבדון And Abdon בן the son הלל of Hillel הפרעתוני the Pirathonite ויקבר and was buried בפרעתון in Pirathon בארץ in the land אפרים of Ephraim, בהר in the mount העמלקי׃ of the Amalekites.}%
\end{biblechapter}%
\begin{biblechapter}% Judges 13
\verseWithHeading{Samson’s Parents}{ויספו again בני And the children ישׂראל of Israel לעשׂות did הרע evil בעיני in the sight יהוה of the LORD; ויתנם delivered יהוה and the LORD ביד them into the hand פלשׁתים of the Philistines ארבעים forty שׁנה׃ years.}%
\verse{ויהי And there was אישׁ man אחד a certain מצרעה ממשׁפחת of the family הדני of the Danites, ושׁמו whose name מנוח Manoah; ואשׁתו and his wife עקרה barren, ולא not. ילדה׃ and bore}%
\verse{וירא appeared מלאך And the angel יהוה of the LORD אל unto האשׁה the woman, ויאמר and said אליה unto הנה her, Behold נא now, את thou עקרה barren, ולא not: ילדת and bearest והרית but thou shalt conceive, וילדת and bear בן׃ a son.}%
\verse{ועתה Now השׁמרי therefore beware, נא I pray thee, ואל not תשׁתי and drink יין wine ושׁכר nor strong drink, ואל not תאכלי and eat כל any טמא׃ unclean}%
\verse{כי For, הנך lo, הרה thou shalt conceive, וילדת and bear בן a son; ומורה razor לא and no יעלה shall come על on ראשׁו his head: כי for נזיר a Nazarite אלהים unto God יהיה shall be הנער the child מן from הבטן the womb: והוא and he יחל shall begin להושׁיע to deliver את ישׂראל Israel מיד out of the hand פלשׁתים׃ of the Philistines.}%
\verse{ותבא came האשׁה Then the woman ותאמר and told לאישׁה her husband, לאמר saying, אישׁ A man האלהים of God בא came אלי unto ומראהו me, and his countenance כמראה like the countenance מלאך of an angel האלהים of God, נורא terrible: מאד very ולא him not שׁאלתיהו but I asked אי whence מזה הוא he ואת שׁמו he me his name: לא neither הגיד׃ told}%
\verse{ויאמר But he said לי הנך unto me, Behold, הרה thou shalt conceive, וילדת and bear בן a son; ועתה and now אל no תשׁתי drink יין wine ושׁכר nor strong drink, ואל neither תאכלי eat כל any טמאה unclean כי for נזיר a Nazarite אלהים to God יהיה shall be הנער the child מן from הבטן the womb עד to יום the day מותו׃ of his death.}%
\verse{ויעתר entreated מנוח Then Manoah אל entreated יהוה the LORD, ויאמר and said, בי O אדוני my Lord, אישׁ let the man האלהים of God אשׁר which שׁלחת thou didst send יבוא come נא עוד again אלינו unto ויורנו us, and teach מה us what נעשׂה we shall do לנער unto the child היולד׃ that shall be born.}%
\verse{וישׁמע hearkened האלהים And God בקול to the voice מנוח of Manoah; ויבא came מלאך and the angel האלהים of God עוד again אל unto האשׁה the woman והיא as she יושׁבת sat בשׂדה in the field: ומנוח but Manoah אישׁה her husband אין not עמה׃ with}%
\verse{ותמהר made haste, האשׁה her husband, ותרץ and ran, ותגד and showed לאישׁה And the woman ותאמר and said אליו unto הנה him, Behold, נראה hath appeared אלי unto האישׁ the man אשׁר me, that בא came ביום me the day. אלי׃ unto}%
\verse{ויקם arose, וילך and went מנוח And Manoah אחרי after אשׁתו his wife, ויבא and came אל to האישׁ the man, ויאמר and said לו האתה unto him, thou האישׁ the man אשׁר that דברת spakest אל unto האשׁה the woman? ויאמר And he said, אני׃ I}%
\verse{ויאמר said, מנוח And Manoah עתה Now יבא come to pass. דבריך let thy words מה How יהיה shall we order משׁפט shall we order הנער the child, ומעשׂהו׃ and shall we do}%
\verse{ויאמר said מלאך And the angel יהוה אל unto מנוח Manoah, מכל אשׁר that אמרתי I said אל unto האשׁה the woman תשׁמר׃ let her beware.}%
\verse{מכל of any אשׁר that יצא cometh מגפן of the vine, היין wine לא She may not תאכל eat ויין ושׁכר or strong drink, אל neither תשׁת let her drink וכל any טמאה unclean אל nor תאכל eat כל all אשׁר that צויתיה I commanded תשׁמר׃ her let her observe.}%
\verse{ויאמר said מנוח And Manoah אל unto מלאך the angel יהוה of the LORD, נעצרה let us detain נא I pray thee, אותך ונעשׂה thee, until we shall have made ready לפניך for גדי a kid עזים׃ a kid}%
\verse{ויאמר said מלאך And the angel יהוה of the LORD אל unto מנוח Manoah, אם Though תעצרני thou detain לא me, I will not אכל eat בלחמך of thy bread: ואם and if תעשׂה thou wilt offer עלה a burnt offering, ליהוה it unto the LORD. תעלנה thou must offer כי For לא not ידע knew מנוח Manoah כי that מלאך an angel יהוה of the LORD. הוא׃ he}%
\verse{ויאמר said מנוח And Manoah אל unto מלאך the angel יהוה of the LORD, מי What שׁמך thy name, כי that when יבא come to pass דבריך thy sayings וכבדנוך׃ we may do thee honor?}%
\verse{ויאמר said לו מלאך And the angel יהוה of the LORD למה unto him, Why זה thou thus תשׁאל askest לשׁמי after my name, והוא seeing it פלאי׃ secret?}%
\verse{ויקח took מנוח So Manoah את גדי a kid העזים a kid ואת המנחה a meat offering, ויעל and offered על upon הצור a rock ליהוה unto the LORD: ומפלא wondrously; לעשׂות and did ומנוח and Manoah ואשׁתו and his wife ראים׃ looked on.}%
\verse{ויהי For it came to pass, בעלות went up הלהב when the flame מעל from off המזבח the altar, השׁמימה toward heaven ויעל ascended מלאך that the angel יהוה of the LORD בלהב in the flame המזבח of the altar. ומנוח And Manoah ואשׁתו and his wife ראים looked on ויפלו and fell על on פניהם their faces ארצה׃ to the ground.}%
\verse{ולא did no יסף more עוד מלאך But the angel יהוה of the LORD להראה appear אל to מנוח Manoah ואל and to אשׁתו his wife. אז Then ידע knew מנוח Manoah כי that מלאך an angel יהוה of the LORD. הוא׃ he}%
\verse{ויאמר said מנוח And Manoah אל unto אשׁתו his wife, מות נמותי because אלהים God. ראינו׃ we have seen}%
\verse{ותאמר said לו אשׁתו But his wife לו unto him, If חפץ were pleased יהוה the LORD להמיתנו to kill לא us, he would not לקח have received מידנו at our hands, עלה a burnt offering ומנחה and a meat offering ולא neither הראנו would he have showed את כל us all אלה these וכעת would as at this time לא nor השׁמיענו have told כזאת׃ us as these.}%
\verse{ותלד bore האשׁה And the woman בן a son, ותקרא and called את שׁמו his name שׁמשׁון Samson: ויגדל grew, הנער and the child ויברכהו blessed יהוה׃ and the LORD}%
\verse{ותחל began רוח And the Spirit יהוה of the LORD לפעמו to move במחנה him at times in the camp דן of Dan בין between צרעה Zorah ובין אשׁתאל׃ and Eshtaol.}%
\end{biblechapter}%
\begin{biblechapter}% Judges 14
\verseWithHeading{Samson Marries}{וירד went down שׁמשׁון And Samson תמנתה to Timnath, וירא and saw אשׁה a woman בתמנתה in Timnath מבנות of the daughters פלשׁתים׃ of the Philistines.}%
\verse{ויעל And he came up, ויגד and told לאביו his father ולאמו and his mother, ויאמר and said, אשׁה a woman ראיתי I have seen בתמנתה in Timnath מבנות of the daughters פלשׁתים of the Philistines: ועתה now קחו therefore get אותה לי לאשׁה׃ her for me to wife.}%
\verse{ויאמר said לו אביו Then his father ואמו and his mother האין unto him, never בבנות among the daughters אחיך of thy brethren, ובכל or among all עמי my people, אשׁה a woman כי that אתה thou הולך goest לקחת to take אשׁה a wife מפלשׁתים הערלים of the uncircumcised ויאמר said שׁמשׁון And Samson אל unto אביו his father, אותה קח Get לי כי her for me; for היא she ישׁרה pleaseth me well. בעיני׃ pleaseth me well.}%
\verse{ואביו But his father ואמו and his mother לא not ידעו knew כי that מיהוה היא it כי that תאנה an occasion הוא he מבקשׁ sought מפלשׁתים the Philistines ובעת time ההיא for at that פלשׁתים משׁלים had dominion בישׂראל׃ over Israel.}%
\verse{וירד שׁמשׁוןאביו and his father ואמו and his mother, תמנתה to Timnath, ויבאו and came עד to כרמי the vineyards תמנתה of Timnath: והנה and, behold, כפיר a young אריות lion שׁאג roared לקראתו׃ against}%
\verse{ותצלח came mightily עליו upon רוח And the Spirit יהוה of the LORD וישׁסעהו him, and he rent כשׁסע him as he would have rent הגדי a kid, ומאומה and nothing אין and nothing בידו in his hand: ולא not הגיד but he told לאביו his father ולאמו or his mother את אשׁר what עשׂה׃ he had done.}%
\verse{וירד And he went down, וידבר and talked לאשׁה with the woman; ותישׁר בעיניׁמשׁון׃}%
\verse{וישׁב he returned מימים לקחתה to take ויסר her, and he turned aside לראות to see את מפלת the carcass האריה of the lion: והנה and, behold, עדת a swarm דבורים of bees בגוית in the carcass האריה of the lion. ודבשׁ׃ and honey}%
\verse{וירדהו And he took אל thereof in כפיו his hands, וילך and went on הלוך and went on ואכל eating, וילך and came אל to אביו his father ואל אמו and mother, ויתן and he gave להם ויאכלו them, and they did eat: ולא not הגיד but he told להם כי them that מגוית out of the carcass האריה of the lion. רדה he had taken הדבשׁ׃ the honey}%
\verse{וירד went down אביהו So his father אל unto האשׁה the woman: ויעשׂ made שׁם there שׁמשׁון and Samson משׁתה a feast; כי for כן so יעשׂו to do. הבחורים׃ used the young men}%
\verse{ויהי And it came to pass, כראותם when they saw אותו ויקחו him, that they brought שׁלשׁים thirty מרעים companions ויהיו to be אתו׃ with}%
\verse{ויאמר said להם שׁמשׁון And Samson אחודה put forth נא unto them, I will now לכם חידה a riddle אם unto you: if הגד ye can certainly declare תגידו ye can certainly declare אותה לי שׁבעת it me within the seven ימי days המשׁתה of the feast, ומצאתם and find out, ונתתי then I will give לכם שׁלשׁים you thirty סדינים sheets ושׁלשׁים and thirty חלפת change בגדים׃ of garments:}%
\verse{ואם But if לא ye cannot תוכלו ye cannot להגיד declare לי ונתתם give אתם me, then shall ye לי שׁלשׁים me thirty סדינים sheets ושׁלשׁים and thirty חליפות change בגדים of garments. ויאמרו And they said לו חודה unto him, Put forth חידתך thy riddle, ונשׁמענה׃ that we may hear}%
\verse{ויאמר And he said להם מהאכל יצא came forth מאכל meat, ומעז and out of the strong יצא came forth מתוק sweetness. ולא not יכלו And they could להגיד expound החידה the riddle. שׁלשׁת in three ימים׃ days}%
\verse{ויהי And it came to pass ביום day, השׁביעי on the seventh ויאמרו that they said לאשׁת wife, שׁמשׁון unto Samson's פתי Entice את אישׁך thy husband, ויגד that he may declare לנו את החידה unto us the riddle, פן lest נשׂרף we burn אותך ואתית house אביך thee and thy father's באשׁ with fire: הלירשׁנו us to take that we have? קראתם have ye called לנו הלא׃ not}%
\verse{ותבך wept אשׁת wife שׁמשׁון And Samson's עליו before ותאמר him, and said, רק Thou dost but שׂנאתני hate ולא me not: אהבתני me, and lovest החידה a riddle חדת thou hast put forth לבני unto the children עמי of my people, ולי לא and hast not הגדתה told ויאמר me. And he said לה הנה unto her, Behold, לאבי my father ולאמי nor my mother, לא I have not הגדתי told ולך אגיד׃ and shall I tell}%
\verse{ותבך And she wept עליו before שׁבעת him the seven הימים days, אשׁר while היה lasted: להם המשׁתה their feast ויהי and it came to pass ביום day, השׁביעי on the seventh ויגד that he told לה כי her, because הציקתהו she lay sore ותגד upon him: and she told החידה the riddle לבני to the children עמה׃ of her people.}%
\verse{ויאמרו said לו אנשׁי And the men העיר of the city ביום day השׁביעי unto him on the seventh בטרם before יבא went down, החרסה the sun מה What מתוק sweeter מדבשׁ than honey? ומה and what עז stronger מארי than a lion? ויאמר And he said להם לולא unto them, If חרשׁתם ye had not plowed בעגלתי with my heifer, לא ye had not מצאתם found out חידתי׃ my riddle.}%
\verse{ותצלח came עליו upon רוח And the Spirit יהוה of the LORD וירד him, and he went down אשׁקלון to Ashkelon, ויך and slew מהם שׁלשׁים thirty אישׁ men ויקח them, and took את חליצותם their spoil, ויתן and gave החליפות change of garments למגידי unto them which expounded החידה the riddle. ויחר was kindled, אפו And his anger ויעל and he went up בית house. אביהו׃ to his father's}%
\verse{ותהי was אשׁת wife שׁמשׁון But Samson's למרעהו to his companion, אשׁר whom רעה׃ he had used as his friend.}%
\end{biblechapter}%
\begin{biblechapter}% Judges 15
\verseWithHeading{Samson Defeats the Philistines}{ויהי But it came to pass מימים within a while בימי after, in the time קציר harvest, חטים of wheat ויפקד visited שׁמשׁון that Samson את אשׁתו his wife בגדי with a kid; עזים with a kid; ויאמר and he said, אבאה I will go in אל to אשׁתי my wife החדרה into the chamber. ולא would not נתנו suffer אביה But her father לבוא׃ him to go in.}%
\verse{ויאמר said, אביה And her father אמר אמרתיי that שׂנא thou hadst utterly hated שׂנאתה thou hadst utterly hated ואתננה her; therefore I gave למרעך her to thy companion: הלא not אחתה sister הקטנה her younger טובה fairer ממנה than תהי she? take נא her, I pray thee, לך תחתיה׃ instead}%
\verse{ויאמר said להם שׁמשׁון And Samson נקיתי shall I be more blameless הפעם concerning them, Now מפלשׁתים כי though עשׂה do אני I עמם do רעה׃ them a displeasure.}%
\verse{וילך went שׁמשׁון And Samson וילכד and caught שׁלשׁ three מאות hundred שׁועלים foxes, ויקח and took לפדים firebrands, ויפן and turned זנב tail אל to זנב tail, וישׂם and put לפיד firebrand אחד a בין between שׁני two הזנבות tails. בתוך׃ in the midst}%
\verse{ויבער And when he had set אשׁ on fire, בלפידים the brands וישׁלח he let go בקמות into the standing corn פלשׁתים of the Philistines, ויבער and burnt up מגדישׁ both the shocks, ועד and also קמה the standing corn, ועד with כרם the vineyards זית׃ olives.}%
\verse{ויאמרו said, פלשׁתים Then the Philistines מי Who עשׂה hath done זאת this? ויאמרו And they answered, שׁמשׁון Samson, חתן the son-in-law התמני of the Timnite, כי because לקח he had taken את אשׁתו his wife, ויתנה and given למרעהו her to his companion. ויעלו came up, פלשׁתים And the Philistines וישׂרפו and burnt אותה ואתביה her and her father באשׁ׃ with fire.}%
\verse{ויאמר said להם שׁמשׁון And Samson אם unto them, Though תעשׂון ye have done כזאת this, כי yet אם yet נקמתי will I be avenged בכם ואחר of you, and after that אחדל׃ I will cease.}%
\verse{ויך And he smote אותם שׁוק them hip על and ירך thigh מכה slaughter: גדולה with a great וירד and he went down וישׁב and dwelt בסעיף in the top סלע of the rock עיטם׃ Etam.}%
\verse{ויעלו went up, פלשׁתים Then the Philistines ויחנו and pitched ביהודה in Judah, וינטשׁו and spread themselves בלחי׃ in Lehi.}%
\verse{ויאמרו said, אישׁ And the men יהודה of Judah למה Why עליתם are ye come up עלינו against ויאמרו us? And they answered, לאסור To bind את שׁמשׁון Samson עלינו are we come up, לעשׂות to do לו כאשׁר to him as עשׂה׃ he hath done}%
\verse{וירדו went שׁלשׁת Then three אלפים thousand אישׁ men מיהודה אל to סעיף the top סלע of the rock עיטם Etam, ויאמרו and said לשׁמשׁון to Samson, הלא thou not ידעת Knowest כי that משׁלים rulers בנו פלשׁתים the Philistines ומה over us? what זאת this עשׂית thou hast done לנו ויאמר unto us? And he said להם כאשׁר unto them, As עשׂו they did לי כן unto me, so עשׂיתי׃ have I done}%
\verse{ויאמרו And they said לו לאסרך to bind ירדנו unto him, We are come down לתתך thee, that we may deliver ביד thee into the hand פלשׁתים of the Philistines. ויאמר said להם שׁמשׁון And Samson השׁבעו unto them, Swear לי פן unto me, that תפגעון ye will not fall בי אתם׃ upon me yourselves.}%
\verse{ויאמרו And they spoke לו לאמר unto him, saying, לא No; כי but אסר we will bind thee fast, נאסרך we will bind thee fast, ונתנוך and deliver בידם thee into their hand: והמת but surely we will not kill לא נמיתךיאסרהו thee. And they bound בשׁנים him with two עבתים cords, חדשׁים new ויעלוהו and brought him up מן from הסלע׃ the rock.}%
\verse{הוא when he בא came עד unto לחי Lehi, ופלשׁתים the Philistines הריעו shouted לקראתו against ותצלח came mightily עליו upon רוח him: and the Spirit יהוה of the LORD ותהיינה became העבתים him, and the cords אשׁר that על upon זרועותיו his arms כפשׁתים as flax אשׁר that בערו was burnt באשׁ with fire, וימסו loosed אסוריו and his bands מעל from off ידיו׃ his hands.}%
\verse{וימצא And he found לחי jawbone חמור of an ass, טריה a new וישׁלח and put forth ידו his hand, ויקחה and took ויך it, and slew בה אלף a thousand אישׁ׃ men}%
\verse{ויאמר said, שׁמשׁון And Samson בלחי With the jawbone החמור of an ass, חמור חמרתיםלחי with the jaw החמור of an ass הכיתי have I slain אלף a thousand אישׁ׃ men.}%
\verse{ויהי And it came to pass, ככלתו when he had made an end לדבר of speaking, וישׁלך that he cast away הלחי the jawbone מידו out of his hand, ויקרא and called למקום place ההוא that רמת לחי׃ Ramath-lehi.}%
\verse{ויצמא מאדיקרא and called אל on יהוה the LORD, ויאמר and said, אתה Thou נתת hast given ביד into the hand עבדך of thy servant: את התשׁועה deliverance הגדלה great הזאת this ועתה and now אמות shall I die בצמא for thirst, ונפלתי and fall ביד into the hand הערלים׃ of the uncircumcised?}%
\verse{ויבקע cleaved אלהים But God את המכתשׁ a hollow place אשׁר that בלחי in the jaw, ויצאו and there came ממנו therefrom; מים water וישׁת and when he had drunk, ותשׁב came again, רוחו his spirit ויחי and he revived: על wherefore כן wherefore קרא he called שׁמה the name עין הקורא thereof En-hakkore, אשׁר which בלחי in Lehi עד unto היום day. הזה׃ this}%
\verse{וישׁפט And he judged את ישׂראל Israel בימי in the days פלשׁתים of the Philistines עשׂרים twenty שׁנה׃ years.}%
\end{biblechapter}%
\begin{biblechapter}% Judges 16
\verseWithHeading{Samson and Delilah}{וילך Then went שׁמשׁון Samson עזתה to Gaza, וירא and saw שׁם there אשׁה a harlot, זונה a harlot, ויבא and went in אליה׃ unto}%
\verse{לעזתים the Gazites, לאמר saying, בא is come שׁמשׁון Samson הנה hither. ויסבו And they compassed ויארבו in, and laid wait לו כל for him all הלילה night בשׁער in the gate העיר of the city, ויתחרשׁו and were quiet כל all הלילה the night, לאמר saying, עד when אור it is day, הבקר In the morning, והרגנהו׃ we shall kill}%
\verse{וישׁכב lay שׁמשׁון And Samson עד till חצי midnight, הלילה midnight, ויקם and arose בחצי at midnight, הלילה at midnight, ויאחז and took בדלתות the doors שׁער of the gate העיר of the city, ובשׁתי and the two המזוזות posts, ויסעם and went away עם with them, bar and all, הבריח with them, bar and all, וישׂם and put על upon כתפיו his shoulders, ויעלם and carried them up אל to ראשׁ the top ההר of a hill אשׁר that על before פני before חברון׃ Hebron.}%
\verse{ויהי And it came to pass אחרי afterward, כן afterward, ויאהב that he loved אשׁה a woman בנחל in the valley שׂרק of Sorek, ושׁמה whose name דלילה׃ Delilah.}%
\verse{ויעלו came up אליה unto סרני And the lords פלשׁתים of the Philistines ויאמרו her, and said לה פתי unto her, Entice אותו וראי him, and see במה wherein כחו strength גדול his great ובמה and by what נוכל we may prevail לו ואסרנהו against him, that we may bind לענתו him to afflict ואנחנו him: and we נתן will give לך אישׁ thee every one אלף of us eleven hundred ומאה of us eleven hundred כסף׃ of silver.}%
\verse{ותאמר said דלילה And Delilah אל to שׁמשׁון Samson, הגידה Tell נא me, I pray thee, לי במה wherein כחך strength גדול thy great ובמה and wherewith תאסר thou mightest be bound לענותך׃ to afflict}%
\verse{ויאמר said אליה unto שׁמשׁון And Samson אם her, If יאסרני they bind בשׁבעה me with seven יתרים withes לחים green אשׁר that לא were never חרבו dried, וחליתי then shall I be weak, והייתי and be כאחד as another האדם׃ man.}%
\verse{ויעלו brought up לה סרני Then the lords פלשׁתים of the Philistines שׁבעה to her seven יתרים withes לחים green אשׁר which לא had not חרבו been dried, ותאסרהו׃ and she bound}%
\verse{והארב Now men lying in wait, ישׁב abiding לה בחדר with her in the chamber. ותאמר And she said אליו unto פלשׁתים him, The Philistines עליך upon שׁמשׁון thee, Samson. וינתק And he broke את היתרים the withes, כאשׁר as ינתק is broken פתיל a thread הנערת of tow בהריחו when it toucheth אשׁ the fire. ולא was not נודע known. כחו׃ So his strength}%
\verse{ותאמר said דלילה And Delilah אל unto שׁמשׁון Samson, הנה Behold, התלת thou hast mocked בי ותדבר me, and told אלי me, and told כזבים עתה now הגידה tell נא me, I pray thee, לי במה wherewith תאסר׃ thou mightest be bound.}%
\verse{ויאמר And he said אליה unto אם her, If אסור they bind me fast יאסרוני they bind me fast בעבתים ropes חדשׁים with new אשׁר that לא never נעשׂה were occupied, בהם מלאכה were occupied, וחליתי then shall I be weak, והייתי and be כאחד as another האדם׃ man.}%
\verse{ותקח therefore took דלילה Delilah עבתים ropes, חדשׁים new ותאסרהו and bound בהם ותאמר him therewith, and said אליו unto פלשׁתים him, The Philistines עליך upon שׁמשׁון thee, Samson. והארב And liers in wait ישׁב abiding בחדר in the chamber. וינתקם And he broke מעל them from off זרעתיו his arms כחוט׃ like a thread.}%
\verse{ותאמר said דלילה And Delilah אל unto שׁמשׁון Samson, עד הנהתלת thou hast mocked בי ותדבר me, and told אלי me, and told כזבים me lies: הגידה tell לי במה me wherewith תאסר thou mightest be bound. ויאמר And he said אליה unto אם her, If תארגי thou weavest את שׁבע the seven מחלפות locks ראשׁי of my head עם with המסכת׃ the web.}%
\verse{ותתקע And she fastened ביתד with the pin, ותאמר and said אליו unto פלשׁתים him, The Philistines עליך upon שׁמשׁון thee, Samson. וייקץ And he awaked משׁנתו out of his sleep, ויסע and went away את היתד with the pin הארג of the beam, ואת המסכת׃ and with the web.}%
\verse{ותאמר And she said אליו unto איך him, How תאמר canst thou say, אהבתיך I love ולבך thee, when thine heart אין not אתי with זה שׁלשׁ me these three פעמים times, התלת me? thou hast mocked בי ולא and hast not הגדת told לי במה me wherein כחך strength גדול׃ thy great}%
\verse{ויהי And it came to pass, כי when הציקה she pressed לו בדבריה with her words, כל him daily הימים him daily ותאלצהו and urged ותקצר was vexed נפשׁו him, that his soul למות׃ unto death;}%
\verse{ויגד That he told לה את כל her all לבו his heart, ויאמר and said לה מורה a razor לא unto her, There hath not עלה come על upon ראשׁי mine head; כי for נזיר a Nazarite אלהים unto God אני I מבטן womb: אמי from my mother's אם if גלחתי I be shaven, וסר will go ממני womb: כחי then my strength וחליתי me, and I shall become weak, והייתי and be ככל like any האדם׃ man.}%
\verse{ותרא saw דלילה And when Delilah כי that הגיד he had told לה את כל her all לבו his heart, ותשׁלח she sent ותקרא and called לסרני for the lords פלשׁתים of the Philistines, לאמר saying, עלו Come up הפעם this once, כי for הגיד he hath showed לה את כל me all לבו his heart. ועלו came up אליה unto סרני Then the lords פלשׁתים of the Philistines ויעלו her, and brought הכסף money בידם׃ in their hand.}%
\verse{ותישׁנהו And she made him sleep על upon ברכיה her knees; ותקרא and she called לאישׁ for a man, ותגלח and she caused him to shave off את שׁבע the seven מחלפות locks ראשׁו of his head; ותחל and she began לענותו to afflict ויסר went כחו him, and his strength מעליו׃ from}%
\verse{ותאמר And she said, פלשׁתים The Philistines עליך upon שׁמשׁון thee, Samson. ויקץ And he awoke משׁנתו out of his sleep, ויאמר and said, אצא I will go out כפעם as at other times before, בפעם as at other times before, ואנער and shake myself. והוא And he לא not ידע knew כי that יהוה the LORD סר was departed מעליו׃ from}%
\verse{ויאחזוהו took פלשׁתים But the Philistines וינקרו him, and put out את עיניו his eyes, ויורידו אותוזתה to Gaza, ויאסרוהו and bound בנחשׁתים him with fetters of brass; ויהי and he did טוחן grind בבית house. האסירים׃ in the prison}%
\verse{ויחל began שׂער Howbeit the hair ראשׁו of his head לצמח to grow again כאשׁר after גלח׃ he was shaven.}%
\verse{וסרני Then the lords פלשׁתים of the Philistines נאספו gathered them together לזבח for to offer זבח sacrifice גדול a great לדגון unto Dagon אלהיהם their god, ולשׂמחה and to rejoice: ויאמרו for they said, נתן hath delivered אלהינו Our god בידנו into our hand. את שׁמשׁון Samson אויבינו׃ our enemy}%
\verse{ויראו saw אתו העם And when the people ויהללו him, they praised את אלהיהם their god: כי for אמרו they said, נתן hath delivered אלהינו Our god בידנו into our hands את אויבנו our enemy, ואת מחריב and the destroyer ארצנו of our country, ואשׁר which הרבה אתללינו׃}%
\verse{ויהי And it came to pass, כי when טוב were merry, לבם their hearts ויאמרו that they said, קראו Call לשׁמשׁון for Samson, וישׂחק that he may make us sport. לנו ויקראו And they called לשׁמשׁון for Samson מבית האסיריםיצחק and he made them sport: לפניהם and he made them sport: ויעמידו and they set אותו בין him between העמודים׃ the pillars.}%
\verse{ויאמר said שׁמשׁון And Samson אל unto הנער the lad המחזיק that held בידו him by the hand, הניחה אותיהימשׁני אתעמדים the pillars אשׁר whereupon הבית the house נכון standeth, עליהם whereupon ואשׁען that I may lean עליהם׃ upon}%
\verse{והבית Now the house מלא was full האנשׁים והנשׁים and women; ושׁמה there; כל and all סרני the lords פלשׁתים of the Philistines ועל and upon הגג the roof כשׁלשׁת about three אלפים thousand אישׁ of men ואשׁה and women, הראים that beheld בשׂחוק made sport. שׁמשׁון׃ while Samson}%
\verseWithHeading{Samson’s Revenge}{ויקרא called שׁמשׁון And Samson אל unto יהוה GOD, ויאמר and said, אדני O Lord יהוה the LORD, זכרני remember נא me, I pray thee, וחזקני and strengthen נא me, I pray thee, אך only הפעם once, הזה this האלהים O God, ואנקמה נקםחת משׁתי for my two עיני eyes. מפלשׁתים׃}%
\verse{וילפת took hold שׁמשׁון And Samson את שׁני of the two עמודי pillars התוך middle אשׁר which הבית the house נכון stood, עליהם upon ויסמך which it was borne up, עליהם and on אחד of the one בימינו with his right hand, ואחד and of the other בשׂמאלו׃ with his left.}%
\verse{ויאמר said, שׁמשׁון And Samson תמות die נפשׁי Let me עם with פלשׁתים the Philistines. ויט And he bowed בכח himself with might; ויפל fell הבית and the house על upon הסרנים the lords, ועל and upon כל all העם the people אשׁר that בו ויהיו were המתים therein. So the dead אשׁר which המית he slew במותו at his death רבים more מאשׁר than which המית he slew בחייו׃ in his life.}%
\verse{וירדו came down, אחיו Then his brethren וכל and all בית the house אביהו of his father וישׂאו and took אתו ויעלו him, and brought up, ויקברו and buried אותו בין him between צרעה Zorah ובין אשׁתאל and Eshtaol בקבר in the burial place מנוח of Manoah אביו his father. והוא And he שׁפט judged את ישׂראל Israel עשׂרים twenty שׁנה׃ years.}%
\end{biblechapter}%
\begin{biblechapter}% Judges 17
\verseWithHeading{Micah’s Idolatry}{ויהי And there was אישׁ a man מהר of mount אפרים Ephraim, ושׁמו whose name מיכיהו׃}%
\verse{ויאמר And he said לאמו unto his mother, אלף ומאהכסף of silver אשׁר that לקח were taken לך ואתי from thee, about which thou אלית cursedst, וגם of also אמרת and spakest באזני in mine ears, הנה behold, הכסף the silver אתי with אני me; I לקחתיו took ותאמר said, אמו it. And his mother ברוך Blessed בני my son. ליהוה׃ of the LORD,}%
\verse{וישׁב And when he had restored את אלף the eleven hundred ומאה the eleven hundred הכסף of silver לאמו to his mother, ותאמר said, אמו his mother הקדשׁ הקדשׁתית הכסף the silver ליהוה unto the LORD מידי from my hand לבני for my son, לעשׂות to make פסל a graven image ומסכה and a molten image: ועתה now אשׁיבנו׃ therefore I will restore}%
\verse{וישׁב Yet he restored את הכסף the money לאמו unto his mother; ותקח took אמו and his mother מאתים two hundred כסף of silver, ותתנהו and gave לצורף them to the founder, ויעשׂהו who made פסל thereof a graven image ומסכה and a molten image: ויהי and they were בבית in the house מיכיהו׃}%
\verse{והאישׁ And the man מיכה Micah לו בית had a house אלהים of gods, ויעשׂ and made אפוד an ephod, ותרפים and teraphim, וימלא אתד אחד one מבניו of his sons, ויהי who became לו לכהן׃ his priest.}%
\verse{בימים days ההם In those אין no מלך king בישׂראל in Israel, אישׁ every man הישׁר right בעיניו in his own eyes. יעשׂה׃ did}%
\verse{ויהי And there was נער a young man מבית לחם יהודה of Judah, ממשׁפחת of the family יהודה והוא who לוי a Levite, והוא and he גר sojourned שׁם׃ there.}%
\verse{וילך departed האישׁ And the man מהעיר out of the city מבית לחם יהודהגור to sojourn באשׁר where ימצא he could find ויבא and he came הר to mount אפרים Ephraim עד to בית the house מיכה of Micah, לעשׂות as he journeyed. דרכו׃ as he journeyed.}%
\verse{ויאמר said לו מיכה And Micah מאין תבוא comest ויאמר thou? And he said אליו unto לוי a Levite אנכי him, I מבית לחם יהודהאנכי and I הלך go לגור to sojourn באשׁר where אמצא׃ I may find}%
\verse{ויאמר said לו מיכה And Micah שׁבה unto him, Dwell עמדי with והיה me, and be לי לאב unto me a father ולכהן and a priest, ואנכי and I אתן will give לך עשׂרת thee ten כסף of silver לימים by the year, וערך and a suit בגדים of apparel, ומחיתך and thy victuals. וילך went in. הלוי׃ So the Levite}%
\verse{ויואל was content הלוי And the Levite לשׁבת to dwell את with האישׁ the man; ויהי was הנער and the young man לו כאחד unto him as one מבניו׃ of his sons.}%
\verse{וימלא מיכה And Micah את ידלוי the Levite; ויהי became לו הנער and the young man לכהן his priest, ויהי and was בבית in the house מיכה׃ of Micah.}%
\verse{ויאמר Then said מיכה Micah, עתה Now ידעתי know כי I that ייטיב will do me good, יהוה the LORD לי כי seeing היה I have לי הלוי a Levite לכהן׃ to priest.}%
\end{biblechapter}%
\begin{biblechapter}% Judges 18
\verseWithHeading{The Tribe of Dan Seeks Territory}{בימים days ההם In those אין no מלך king בישׂראל in Israel: ובימים days ההם and in those שׁבט the tribe הדני of the Danites מבקשׁ sought לו נחלה them an inheritance לשׁבת to dwell כי in; for לא had not נפלה fallen לו עד unto היום day ההוא that בתוך unto them among שׁבטי the tribes ישׂראל of Israel. בנחלה׃ inheritance}%
\verse{וישׁלחו sent בני And the children דן of Dan ממשׁפחתם of their family חמשׁה five אנשׁים men מקצותם from their coasts, אנשׁים men בני חיל of valor, מצרעה ומאשׁתאלרגל to spy out את הארץ the land, ולחקרה and to search ויאמרו it; and they said אלהם unto לכו them, Go, חקרו search את הארץ the land: ויבאו who when they came הר to mount אפרים Ephraim, עד to בית the house מיכה of Micah, וילינו they lodged שׁם׃ there.}%
\verse{המה When they עם by בית the house מיכה of Micah, והמה they הכירו knew את קול the voice הנער of the young man הלוי the Levite: ויסורו and they turned in שׁם thither, ויאמרו and said לו מי unto him, Who הביאך brought הלם thee hither? ומה and what אתה thou עשׂה makest בזה in this ומה and what לך פה׃ hast thou here?}%
\verse{ויאמר And he said אלהם unto כזה them, Thus וכזה עשׂה dealeth לי מיכה Micah וישׂכרני with me, and hath hired ואהי me, and I am לו לכהן׃ his priest.}%
\verse{ויאמרו And they said לו שׁאל unto him, Ask counsel, נא we pray thee, באלהים of God, ונדעה that we may know התצליח shall be prosperous. דרכנו whether our way אשׁר which אנחנו we הלכים go עליה׃}%
\verse{ויאמר said להם הכהן And the priest לכו unto them, Go לשׁלום in peace: נכח before יהוה the LORD דרככם your way אשׁר wherein תלכו׃ ye go.}%
\verse{וילכו departed, חמשׁת Then the five האנשׁים men ויבאו and came לישׁה to Laish, ויראו and saw את העם the people אשׁר that בקרבה therein, יושׁבת how they dwelt לבטח careless, כמשׁפט after the manner צדנים of the Zidonians, שׁקט quiet ובטח and secure; ואין and no מכלים that might put to shame דבר in thing; בארץ in the land, יורשׁ magistrate עצר magistrate ורחקים far המה and they מצדנים ודבר business אין and had no להם עם with אדם׃ man.}%
\verse{ויבאו And they came אל unto אחיהם their brethren צרעה to Zorah ואשׁתאל and Eshtaol: ויאמרו said להם אחיהם and their brethren מה unto them, What אתם׃ ye?}%
\verse{ויאמרו And they said, קומה Arise, ונעלה that we may go up עליהם against כי them: for ראינו we have seen את הארץ the land, והנה and, behold, טובה good: מאד it very ואתם and ye מחשׁים still? אל be not slothful תעצלו be not slothful ללכת to go, לבא to enter לרשׁת to possess את הארץ׃ the land.}%
\verse{כבאכם When ye go, תבאו ye shall come אל unto עם a people בטח secure, והארץ land: רחבת and to a large ידים it into your hands; כי for נתנה hath given אלהים God בידכם מקום a place אשׁר where אין no שׁם מחסור want כל of any דבר thing אשׁר that בארץ׃ in the earth.}%
\verse{ויסעו And there went משׁם from thence ממשׁפחת of the family הדני of the Danites, מצרעה ומאשׁתאלׁשׁ six מאות hundred אישׁ men חגור appointed כלי with weapons מלחמה׃ of war.}%
\verse{ויעלו And they went up, ויחנו and pitched בקרית יערים in Kirjath-jearim, ביהודה in Judah: על wherefore כן wherefore קראו they called למקום place ההוא that מחנה דן Mahaneh-dan עד unto היום day: הזה this הנה behold, אחרי behind קרית יערים׃ Kirjath-jearim.}%
\verse{ויעברו And they passed משׁם thence הר unto mount אפרים Ephraim, ויבאו and came עד unto בית the house מיכה׃ of Micah.}%
\verse{ויענו Then answered חמשׁת the five האנשׁים ההלכים that went לרגל to spy out את הארץ the country לישׁ of Laish, ויאמרו and said אל unto אחיהם their brethren, הידעתם Do ye know כי that ישׁ there is בבתים houses האלה in these אפוד an ephod, ותרפים and teraphim, ופסל and a graven image, ומסכה and a molten image? ועתה now דעו therefore consider מה what תעשׂו׃ ye have to do.}%
\verse{ויסורו And they turned שׁמה thitherward, ויבאו and came אל to בית the house הנער of the young man הלוי the Levite, בית unto the house מיכה of Micah, וישׁאלו and saluted לו לשׁלום׃ and saluted}%
\verse{ושׁשׁ And the six מאות hundred אישׁ men חגורים appointed כלי with their weapons מלחמתם of war, נצבים stood פתח by the entering השׁער of the gate. אשׁר which מבני of the children דן׃ of Dan,}%
\verse{ויעלו that went חמשׁת And the five האנשׁים men ההלכים לרגל to spy out את הארץ the land באו came in שׁמה thither, לקחו took את הפסלאת האפוד and the ephod, ואת התרפים and the teraphim, ואת המסכה and the molten image: והכהן and the priest נצב stood פתח in the entering השׁער of the gate ושׁשׁ with the six מאות hundred האישׁ men החגור appointed כלי with weapons המלחמה׃ of war.}%
\verse{ואלה And these באו went בית house, מיכה into Micah's ויקחו and fetched את פסל the carved image, האפוד the ephod, ואת התרפים and the teraphim, ואת המסכה and the molten image. ויאמר Then said אליהם unto הכהן the priest מה them, What אתם ye? עשׂים׃ do}%
\verse{ויאמרו And they said לו החרשׁ unto him, Hold thy peace, שׂים lay ידך thine hand על upon פיך thy mouth, ולך and go עמנו with והיה us, and be לנו לאב to us a father ולכהן and a priest: הטוב better היותך for thee to be כהן a priest לבית unto the house אישׁ man, אחד of one או or היותך that thou be כהן a priest לשׁבט unto a tribe ולמשׁפחה and a family בישׂראל׃ in Israel?}%
\verse{וייטב was glad, לב heart הכהן And the priest's ויקח and he took את האפוד the ephod, ואת התרפים and the teraphim, ואת הפסל and the graven image, ויבא and went בקרב in the midst העם׃ of the people.}%
\verse{ויפנו So they turned וילכו and departed, וישׂימו and put את הטף the little ones ואת המקנה and the cattle ואת הכבודה and the carriage לפניהם׃ before}%
\verse{המה when they הרחיקו were a good way מבית from the house מיכה of Micah, והאנשׁים אשׁר that בבתים in the houses אשׁר עם near בית house מיכה to Micah's נזעקו were gathered together, וידביקו and overtook את בני the children דן׃ of Dan.}%
\verse{ויקראו And they cried אל unto בני the children דן of Dan. ויסבו And they turned פניהם their faces, ויאמרו and said למיכה unto Micah, מה What לך כי aileth thee, that נזעקת׃ thou comest with such a company?}%
\verse{ויאמר And he said, את אלהי my gods אשׁר which עשׂיתי I made, לקחתם Ye have taken away ואת הכהן and the priest, ותלכו and ye are gone away: ומה and what לי עוד have I more? ומה and what זה this תאמרו ye say אלי unto מה׃ me, What}%
\verse{ויאמרו said אליו unto בני And the children דן of Dan אל him, Let not תשׁמע be heard קולך thy voice עמנו among פן us, lest יפגעו run בכם אנשׁים fellows מרי angry נפשׁ angry ואספתה upon thee, and thou lose נפשׁך thy life, ונפשׁ with the lives ביתך׃ of thy household.}%
\verse{וילכו went בני And the children דן of Dan לדרכם their way: וירא saw מיכה and when Micah כי that חזקים too strong המה they ממנו for ויפן him, he turned וישׁב and went back אל unto ביתו׃ his house.}%
\verse{והמה And they לקחו took את אשׁר which עשׂה had made, מיכה Micah ואת הכהן and the priest אשׁר which היה he had, לו ויבאו and came על unto לישׁ Laish, על unto עם a people שׁקט at quiet ובטח and secure: ויכו and they smote אותם לפי them with the edge חרב of the sword, ואת העיר the city שׂרפו and burnt באשׁ׃ with fire.}%
\verse{ואין And no מציל deliverer, כי because רחוקה far היא it מצידון ודבר business אין and they had no להם עם with אדם man; והיא and it בעמק was in the valley אשׁר that לבית רחוב by Beth-rehob. ויבנו And they built את העיר a city, וישׁבו׃ and dwelt}%
\verse{ויקראו And they called שׁם the name העיר of the city דן Dan, בשׁם after the name דן of Dan אביהם their father, אשׁר who יולד was born לישׂראל unto Israel: ואולם howbeit לישׁ Laish שׁם the name העיר of the city לראשׁנה׃ at the first.}%
\verse{ויקימו set up להם בני And the children דן of Dan את הפסל the graven image: ויהונתן בן the son גרשׁם בן the son מנשׁה of Manasseh, הוא he ובניו and his sons היו were כהנים priests לשׁבט to the tribe הדני of Dan עד until יום the day גלות of the captivity הארץ׃ of the land.}%
\verse{וישׂימו And they set them up להם את פסל graven image, מיכה Micah's אשׁר which עשׂה he made, כל all ימי the time היות was בית that the house האלהים of God בשׁלה׃ in Shiloh.}%
\end{biblechapter}%
\begin{biblechapter}% Judges 19
\verseWithHeading{The Concubine and the Levite}{ויהי And it came to pass בימים days, ההם in those ומלך king אין when no בישׂראל in Israel, ויהי that there was אישׁ a certain לוי Levite גר sojourning בירכתי on the side הר of mount אפרים Ephraim, ויקח who took לו אשׁה to him a concubine פילגשׁ to him a concubine מבית לחם יהודה׃}%
\verse{ותזנה played the whore עליו against פילגשׁו And his concubine ותלך him, and went away מאתו אל him unto בית house אביה her father's אל to בית לחם יהודהתהי and was שׁם there ימים whole ארבעה four חדשׁים׃ months.}%
\verse{ויקם arose, אישׁה And her husband וילך and went אחריה after לדבר her, to speak על friendly לבה friendly להשׁיבו unto her, to bring her again, ונערו having his servant עמו with וצמד him, and a couple חמרים of asses: ותביאהו and she brought בית house: אביה him into her father's ויראהו saw אבי and when the father הנערה of the damsel וישׂמח him, he rejoiced לקראתו׃ to meet}%
\verse{ויחזק retained בו חתנו And his father-in-law, אבי father, הנערה the damsel's וישׁב him; and he abode אתו with שׁלשׁת him three ימים days: ויאכלו so they did eat וישׁתו and drink, וילינו and lodged שׁם׃ there.}%
\verse{ויהי And it came to pass ביום day, הרביעי on the fourth וישׁכימו when they arose early בבקר in the morning, ויקם that he rose up ללכת to depart: ויאמר said אבי father הנערה and the damsel's אל unto חתנו his son-in-law, סעד Comfort לבך thine heart פת with a morsel לחם of bread, ואחר and afterward תלכו׃ go your way.}%
\verse{וישׁבו And they sat down, ויאכלו and did eat שׁניהם both יחדו of them together: וישׁתו and drink ויאמר had said אבי father הנערה for the damsel's אל unto האישׁ the man, הואל Be content, נא I pray thee, ולין and tarry all night, ויטב be merry. לבך׃ and let thine heart}%
\verse{ויקם rose up האישׁ And when the man ללכת to depart, ויפצר urged בו חתנו his father-in-law וישׁב again. וילן him: therefore he lodged שׁם׃ there}%
\verse{וישׁכם And he arose early בבקר in the morning ביום day החמישׁי on the fifth ללכת to depart: ויאמר said, אבי father הנערה and the damsel's סעד Comfort נא I pray thee. לבבך thine heart, והתמהמהו And they tarried עד until נטות afternoon, היום afternoon, ויאכלו and they did eat שׁניהם׃ both}%
\verse{ויקם rose up האישׁ And when the man ללכת to depart, הוא he, ופילגשׁו and his concubine, ונערו and his servant, ויאמר said לו חתנו his father-in-law, אבי father, הנערה the damsel's הנה unto him, Behold, נא now רפה draweth היום the day לערב toward evening, לינו tarry all night: נא I pray you הנה behold, חנות groweth to an end, היום the day לין lodge פה here, וייטב may be merry; לבבך that thine heart והשׁכמתם get you early מחר and tomorrow לדרככם on your way, והלכת that thou mayest go לאהלך׃ home.}%
\verse{ולא not אבה would האישׁ But the man ללון tarry that night, ויקם but he rose up וילך and departed, ויבא and came עד over against נכח over against יבוס Jebus, היא which ירושׁלם Jerusalem; ועמו and with צמד him two חמורים asses חבושׁים saddled, ופילגשׁו his concubine עמו׃ also with}%
\verse{הם when they עם by יבוס Jebus, והיום the day רד spent; מאד was far ויאמר said הנער and the servant אל unto אדניו his master, לכה Come, נא I pray thee, ונסורה and let us turn in אל into עיר city היבוסי of the Jebusites, הזאת this ונלין׃ and lodge}%
\verse{ויאמר said אליו unto אדניו And his master לא him, We will not נסור turn aside אל hither into עיר the city נכרי of a stranger, אשׁר that לא not מבני of the children ישׂראל of Israel; הנה ועברנו we will pass over עד to גבעה׃ Gibeah.}%
\verse{ויאמר And he said לנערו unto his servant, לך Come, ונקרבה and let us draw near באחד to one המקמות of these places ולנו to lodge all night, בגבעה in Gibeah, או or ברמה׃ in Ramah.}%
\verse{ויעברו And they passed on וילכו and went their way; ותבא went down להם השׁמשׁ and the sun אצל upon them by הגבעה Gibeah, אשׁר which לבנימן׃ to Benjamin.}%
\verse{ויסרו And they turned aside שׁם thither, לבוא to go in ללון to lodge בגבעה in Gibeah: ויבא and when he went in, וישׁב he sat him down ברחוב in a street העיר of the city: ואין for no אישׁ man מאסף that took אותם הביתה them into his house ללון׃ to lodging.}%
\verse{והנה And, behold, אישׁ man זקן an old בא there came מן from מעשׂהו his work מן out of השׂדה the field בערב at even, והאישׁ but the men מהר also of mount אפרים Ephraim; והוא and he גר sojourned בגבעה in Gibeah: ואנשׁי המקום of the place בני ימיני׃ Benjamites.}%
\verse{וישׂא And when he had lifted up עיניו his eyes, וירא he saw את האישׁ man הארח a wayfaring ברחב in the street העיר of the city: ויאמר said, האישׁ man הזקן and the old אנה תלך goest ומאין thou? and whence תבוא׃ comest}%
\verse{ויאמר And he said אליו unto עברים passing אנחנו him, We מבית לחם יהודהד toward ירכתי the side הר of mount אפרים Ephraim; משׁם from thence אנכי I: ואלך and I went עד to בית לחם יהודהאת בית to the house יהוה of the LORD; אני but I הלך going ואין and there no אישׁ man מאסף that receiveth אותי הביתה׃ me to house.}%
\verse{וגם both תבן straw גם and מספוא provender ישׁ Yet there is לחמורינו for our asses; וגם also לחם bread ויין and wine ישׁ and there is לי ולאמתך for me, and for thy handmaid, ולנער and for the young man עם with עבדיך thy servants: אין no מחסור want כל of any דבר׃ thing.}%
\verse{ויאמר said, האישׁ man הזקן And the old שׁלום Peace לך רק with thee; howsoever כל all מחסורך thy wants עלי upon רק me; only ברחוב in the street. אל not תלן׃ lodge}%
\verse{ויביאהו So he brought לביתו him into his house, ויבול and gave provender לחמורים unto the asses: וירחצו and they washed רגליהם their feet, ויאכלו and did eat וישׁתו׃ and drink.}%
\verse{המה as they מיטיבים אתבם והנה behold, אנשׁי the men העיר of the city, אנשׁי certain בני sons בליעל of Belial, נסבו אתבית of the house, מתדפקים beat על at הדלת the door, ויאמרו and spoke אל to האישׁ the man בעל the master הבית thine house, הזקן the old man, לאמר saying, הוצא Bring forth את האישׁשׁר that בא came אל into ביתך ונדענו׃ that we may know}%
\verse{ויצא went out אליהם unto האישׁ And the man, בעל the master הבית of the house, ויאמר them, and said אלהם unto אל them, Nay, אחי my brethren, אל do not wickedly; תרעו do not wickedly; נא I pray you, אחרי seeing that אשׁר seeing that בא is come האישׁ man הזה this אל into ביתי mine house, אל not תעשׂו do את הנבלה folly. הזאת׃ this}%
\verse{הנה Behold, בתי my daughter הבתולה a maiden, ופילגשׁהו and his concubine; אוציאה them I will bring out נא now, אותם וענו and humble אותם ועשׂו ye them, and do להם הטוב good בעיניכם with them what seemeth ולאישׁ man הזה unto you: but unto this לא not תעשׂו do דבר a thing. הנבלה vile הזאת׃ so}%
\verse{ולא not אבו would האנשׁים לשׁמע hearken לו ויחזק took האישׁ But the men בפילגשׁו his concubine, ויצא and brought her forth אליהם unto החוץ and brought her forth וידעו them; and they knew אותה ויתעללו her, and abused בה כל her all הלילה the night עד until הבקר the morning: וישׁלחוה they let her go. בעלות began to spring, השׁחר׃ and when the day}%
\verse{ותבא Then came האשׁה the woman לפנות in the dawning הבקר of the day, ותפל and fell down פתח at the door בית house האישׁ of the man's אשׁר where אדוניה her lord שׁם where עד till האור׃ it was light.}%
\verse{ויקם rose up אדניה And her lord בבקר in the morning, ויפתח and opened דלתות the doors הבית of the house, ויצא and went out ללכת to go לדרכו his way: והנה and, behold, האשׁה the woman פילגשׁו his concubine נפלת was fallen down פתח the door הבית of the house, וידיה and her hands על upon הסף׃ the threshold.}%
\verse{ויאמר And he said אליה unto קומי her, Up, ונלכה and let us be going. ואין But none ענה answered. ויקחה Then the man took על her upon החמור an ass, ויקם rose up, האישׁ and the man וילך and got למקמו׃ him unto his place.}%
\verse{ויבא And when he was come אל into ביתו his house, ויקח he took את המאכלת a knife, ויחזק and laid hold בפילגשׁו on his concubine, וינתחה and divided לעצמיה her, with her bones, לשׁנים into twelve עשׂר into twelve נתחים pieces, וישׁלחה and sent בכל her into all גבול the coasts ישׂראל׃ of Israel.}%
\verse{והיה And it was כל so, that all הראה that saw ואמר it said, לא There was no נהיתה deed done ולא nor נראתה seen כזאת such למיום from the day עלות came up בני that the children ישׂראל of Israel מארץ out of the land מצרים of Egypt עד unto היום day: הזה this שׂימו consider לכם עליה of עצו it, take advice, ודברו׃ and speak}%
\end{biblechapter}%
\begin{biblechapter}% Judges 20
\verseWithHeading{The Punishment of Benjamin}{ויצאו went out, כל Then all בני the children ישׂראל of Israel ותקהל was gathered together העדה and the congregation כאישׁ man, אחד as one למדן ועד even to באר שׁבע Beer-sheba, וארץ with the land הגלעד of Gilead, אל unto יהוה the LORD המצפה׃ in Mizpeh.}%
\verse{ויתיצבו presented themselves פנות And the chief כל of all העם the people, כל of all שׁבטי the tribes ישׂראל of Israel, בקהל in the assembly עם of the people האלהים of God, ארבע four מאות hundred אלף thousand אישׁ footmen רגלי footmen שׁלף that drew חרב׃ sword.}%
\verse{וישׁמעו heard בני (Now the children בנימן of Benjamin כי that עלו were gone up בני the children ישׂראל of Israel המצפה to Mizpeh.) ויאמרו Then said בני the children ישׂראל of Israel, דברו Tell איכה how נהיתה was הרעה wickedness? הזאת׃ this}%
\verse{ויען answered האישׁ the husband הלוי אישׁאשׁה of the woman הנרצחה that was slain, ויאמר and said, הגבעתה into Gibeah אשׁר that לבנימן to Benjamin, באתי I came אני I ופילגשׁי and my concubine, ללון׃ to lodge.}%
\verse{ויקמו rose עלי against בעלי And the men הגבעה of Gibeah ויסבו עלי upon את הביתילה me by night, אותי דמו thought להרג to have slain ואת פילגשׁי me: and my concubine ענו have they forced, ותמת׃ that she is dead.}%
\verse{ואחז And I took בפילגשׁי my concubine, ואנתחה and cut her in pieces, ואשׁלחה and sent בכל her throughout all שׂדה the country נחלת of the inheritance ישׂראל of Israel: כי for עשׂו they have committed זמה lewdness ונבלה and folly בישׂראל׃ in Israel.}%
\verse{הנה Behold, כלכם ye all בני children ישׂראל of Israel; הבו give לכם דבר your advice ועצה and counsel. הלם׃ here}%
\verse{ויקם arose כל And all העם the people כאישׁ man, אחד as one לאמר saying, לא We will not נלך go אישׁ any לאהלו to his tent, ולא neither נסור turn אישׁ will we any לביתו׃ into his house.}%
\verse{ועתה But now זה this הדבר the thing אשׁר which נעשׂה we will do לגבעה to Gibeah; עליה against בגורל׃ by lot}%
\verse{ולקחנו And we will take עשׂרה ten אנשׁים men למאה of a hundred לכל throughout all שׁבטי the tribes ישׂראל of Israel, ומאה and a hundred לאלף of a thousand, ואלף and a thousand לרבבה out of ten thousand, לקחת to fetch צדה victual לעם for the people, לעשׂות that they may do, לבואם when they come לגבע to Gibeah בנימן of Benjamin, ככל according to all הנבלה the folly אשׁר that עשׂה they have wrought בישׂראל׃ in Israel.}%
\verse{ויאסף were gathered כל So all אישׁ the men ישׂראל of Israel אל against העיר the city, כאישׁ man. אחד as one חברים׃ knit together}%
\verse{וישׁלחו sent שׁבטי And the tribes ישׂראל of Israel אנשׁים בכל through all שׁבטי the tribe בנימן of Benjamin, לאמר saying, מה What הרעה wickedness הזאת this אשׁר that נהיתה׃ is done}%
\verse{ועתה Now תנו therefore deliver את האנשׁים the men, בני the children בליעל of Belial, אשׁר which בגבעה in Gibeah, ונמיתם that we may put them to death, ונבערה and put away רעה evil מישׂראל of Israel: ולא not אבו would בנימן of Benjamin לשׁמע hearken בקול to the voice אחיהם of their brethren בני But the children ישׂראל׃}%
\verse{ויאספו gathered themselves together בני But the children בנימן of Benjamin מן out of הערים the cities הגבעתה unto Gibeah, לצאת to go out למלחמה to battle עם against בני the children ישׂראל׃ of Israel.}%
\verse{ויתפקדו were numbered בני And the children בנימן of Benjamin ביום time ההוא at that מהערים out of the cities עשׂרים twenty ושׁשׁה and six אלף thousand אישׁ men שׁלף that drew חרב sword, לבד beside מישׁבי the inhabitants הגבעה of Gibeah, התפקדו which were numbered שׁבע seven מאות hundred אישׁ men. בחור׃ chosen}%
\verse{מכל every one העם people הזה this שׁבע seven מאות hundred אישׁ men בחור אטרד ימינול זה every one קלע could sling באבן stones אל at השׂערה a hair ולא and not יחטא׃ miss.}%
\verse{ואישׁ And the men ישׂראל of Israel, התפקדו were numbered לבד beside מבנימן ארבע four מאות hundred אלף thousand אישׁ men שׁלף that drew חרב sword: כל all זה these אישׁ men מלחמה׃ of war.}%
\verse{ויקמו arose, ויעלו and went up בית אל to the house of God, וישׁאלו and asked counsel באלהים of God, ויאמרו and said, בני And the children ישׂראל of Israel מי Which יעלה of us shall go up לנו בתחלה first למלחמה to the battle עם against בני the children בנימן of Benjamin? ויאמר said, יהוה And the LORD יהודה Judah בתחלה׃ first.}%
\verse{ויקומו rose up בני And the children ישׂראל of Israel בבקר in the morning, ויחנו and encamped על against הגבעה׃ Gibeah.}%
\verse{ויצא went out אישׁ And the men ישׂראל of Israel למלחמה to battle עם against בנימן Benjamin; ויערכו put themselves in array אתם אישׁ and the men ישׂראל of Israel מלחמה to fight אל them at הגבעה׃ Gibeah.}%
\verse{ויצאו came forth בני And the children בנימן of Benjamin מן out of הגבעה Gibeah, וישׁחיתו and destroyed down בישׂראל of the Israelites ביום day ההוא that שׁנים and two ועשׂרים twenty אלף thousand אישׁ men. ארצה׃ to the ground}%
\verse{ויתחזק encouraged themselves, העם And the people אישׁ the men ישׂראל of Israel ויספו again לערך and set מלחמה their battle במקום in the place אשׁר where ערכו in array שׁם where ביום day. הראשׁון׃ the first}%
\verse{ויעלו went up בני (And the children ישׂראל of Israel ויבכו and wept לפני before יהוה the LORD עד until הערב even, וישׁאלו and asked counsel ביהוה of the LORD, לאמר saying, האוסיף again לגשׁת Shall I go up למלחמה to battle עם against בני the children בנימן of Benjamin אחי my brother? ויאמר said, יהוה And the LORD עלו Go up אליו׃ against}%
\verse{ויקרבו came near בני And the children ישׂראל of Israel אל against בני the children בנימן of Benjamin ביום day. השׁני׃ the second}%
\verse{ויצא went forth בנימן And Benjamin לקראתם against מן them out of הגבעה Gibeah ביום day, השׁני the second וישׁחיתו and destroyed בבני of the children ישׂראל of Israel עוד again שׁמנת eighteen עשׂר eighteen אלף thousand אישׁ men; ארצה down to the ground כל all אלה these שׁלפי drew חרב׃ the sword.}%
\verse{ויעלו went up, כל Then all בני the children ישׂראל of Israel, וכל and all העם the people, ויבאו and came בית אל unto the house of God, ויבכו and wept, וישׁבו and sat שׁם there לפני before יהוה the LORD, ויצומו and fasted ביום day ההוא that עד until הערב even, ויעלו and offered עלות burnt offerings ושׁלמים and peace offerings לפני before יהוה׃ the LORD.}%
\verse{וישׁאלו inquired בני And the children ישׂראל of Israel ביהוה of the LORD, ושׁם there ארון (for the ark ברית of the covenant האלהים of God בימים days, ההם׃ in those}%
\verse{ופינחס And Phinehas, בן the son אלעזר of Eleazar, בן the son אהרן of Aaron, עמד stood לפניו before בימים days,) ההם it in those לאמר saying, האוסף again עוד Shall I yet לצאת go out למלחמה to battle עם against בני the children בנימן of Benjamin אחי my brother, אם or אחדל shall I cease? ויאמר said, יהוה And the LORD עלו Go up; כי for מחר tomorrow אתננו I will deliver בידך׃ them into thine hand.}%
\verse{וישׂם set ישׂראל And Israel ארבים liers in wait אל round about הגבעה Gibeah. סביב׃ round about}%
\verse{ויעלו went up בני And the children ישׂראל of Israel אל against בני the children בנימן of Benjamin ביום day, השׁלישׁי on the third ויערכו and put themselves in array אל against הגבעה Gibeah, כפעם as at other times. בפעם׃ as at other times.}%
\verse{ויצאו went out בני And the children בנימן of Benjamin לקראת against העם the people, הנתקו were drawn away מן from העיר the city; ויחלו and they began להכות to smite מהעם of the people, חללים kill, כפעם as at other times, בפעם as at other times, במסלות in the highways, אשׁר of which אחת one עלה goeth up בית אל to the house of God, ואחת and the other גבעתה to Gibeah בשׂדה in the field, כשׁלשׁים about thirty אישׁ men בישׂראל׃ of Israel.}%
\verse{ויאמרו said, בני And the children בנימן of Benjamin נגפים smitten down הם They לפנינו before כבראשׁנה us, as at the first. ובני But the children ישׂראל of Israel אמרו said, ננוסה Let us flee, ונתקנהו and draw מן them from העיר the city אל unto המסלות׃ the highways.}%
\verse{וכל And all אישׁ the men ישׂראל of Israel קמו rose up ממקומו out of their place, ויערכו and put themselves in array בבעל תמר at Baal-tamar: וארב and the liers in wait ישׂראל of Israel מגיח came forth ממקמו out of their places, ממערה out of the meadows גבע׃}%
\verse{ויבאו And there came מנגד against לגבעה Gibeah עשׂרת ten אלפים thousand אישׁ men בחור מכל out of all ישׂראל Israel, והמלחמה and the battle כבדה was sore: והם but they לא not ידעו knew כי that נגעת near עליהם הרעה׃ evil}%
\verse{ויגף smote יהוה And the LORD את בנימן Benjamin לפני before ישׂראל Israel: וישׁחיתו destroyed בני and the children ישׂראל of Israel בבנימן of the Benjamites ביום day ההוא that עשׂרים twenty וחמשׁה and five אלף thousand ומאה and a hundred אישׁ men: כל all אלה these שׁלף drew חרב׃ the sword.}%
\verse{ויראו saw בני So the children בנימן of Benjamin כי that נגפו they were smitten: ויתנו gave אישׁ for the men ישׂראל of Israel מקום place לבנימן to the Benjamites, כי because בטחו they trusted אל unto הארב the liers in wait אשׁר which שׂמו they had set אל beside הגבעה׃ Gibeah.}%
\verse{והארב And the liers in wait החישׁו hasted, ויפשׁטו and rushed אל upon הגבעה Gibeah; וימשׁך drew along, הארב and the liers in wait ויך and smote את כל all העיר the city לפי with the edge חרב׃ of the sword.}%
\verse{והמועד an appointed sign היה Now there was לאישׁ between the men ישׂראל of Israel עם הארב and the liers in wait, הרב that they should make a great להעלותם rise up משׂאת flame העשׁן with smoke מן out of העיר׃ the city.}%
\verse{ויהפך retired אישׁ And when the men ישׂראל of Israel במלחמה in the battle, ובנימן Benjamin החל began להכות to smite חללים kill באישׁ of the men ישׂראל of Israel כשׁלשׁים about thirty אישׁ persons: כי for אמרו they said, אך Surely נגוף are smitten down נגף are smitten down הוא they לפנינו before כמלחמה battle. הראשׁנה׃ us, as the first}%
\verse{והמשׂאת But when the flame החלה began לעלות to arise up מן out of העיר the city עמוד with a pillar עשׁן of smoke, ויפן looked בנימן the Benjamites אחריו behind והנה them, and, behold, עלה ascended up כליל the flame העיר of the city השׁמימה׃ to heaven.}%
\verse{ואישׁ And when the men ישׂראל of Israel הפך turned again, ויבהל were amazed: אישׁ the men בנימן of Benjamin כי for ראה they saw כי that נגעה was come עליו upon הרעה׃ evil}%
\verse{ויפנו Therefore they turned לפני before אישׁ the men ישׂראל of Israel אל unto דרך the way המדבר of the wilderness; והמלחמה but the battle הדביקתהו overtook ואשׁר them; and them which מהערים out of the cities משׁחיתים they destroyed אותו בתוכו׃ in the midst}%
\verse{כתרו אתנימן הרדיפהו chased מנוחה with ease הדריכהו them, trod them down עד over against נכח over against הגבעה Gibeah ממזרח שׁמשׁ׃}%
\verse{ויפלו And there fell מבנימן שׁמנה eighteen עשׂר eighteen אלף thousand אישׁ men; את כל all אלה these אנשׁי men חיל׃ of valor.}%
\verse{ויפנו And they turned וינסו and fled המדברה toward the wilderness אל unto סלע the rock הרמון of Rimmon: ויעללהו and they gleaned במסלות of them in the highways חמשׁת five אלפים thousand אישׁ men; וידביקו and pursued hard אחריו after עד them unto גדעם Gidom, ויכו and slew ממנו of אלפים two thousand אישׁ׃ men}%
\verse{ויהי were כל So that all הנפלים which fell מבנימן עשׂרים twenty וחמשׁה and five אלף thousand אישׁ men שׁלף that drew חרב the sword; ביום day ההוא that את כל all אלה these אנשׁי men חיל׃ of valor.}%
\verse{ויפנו turned וינסו and fled המדברה to the wilderness אל unto סלע the rock הרמון Rimmon, שׁשׁ But six מאות hundred אישׁ men וישׁבו and abode בסלע in the rock רמון Rimmon ארבעה four חדשׁים׃ months.}%
\verse{ואישׁ And the men ישׂראל of Israel שׁבו turned again אל upon בני the children בנימן of Benjamin, ויכום and smote לפי them with the edge חרב of the sword, מעיר of city, מתם as well the men עד as בהמה the beast, עד כל and all הנמצא that came to hand: גם also כל all הערים the cities הנמצאות that they came to. שׁלחו they set באשׁ׃ on fire}%
\end{biblechapter}%
\begin{biblechapter}% Judges 21
\verseWithHeading{A Decision Is Made About the Tribe of Benjamin}{ואישׁ Now the men ישׂראל of Israel נשׁבע had sworn במצפה in Mizpeh, לאמר saying, אישׁ any ממנו of לא There shall not יתן us give בתו his daughter לבנימן unto Benjamin לאשׁה׃ to wife.}%
\verse{ויבא came העם And the people בית אל to the house of God, וישׁבו and abode שׁם there עד till הערב even לפני before האלהים God, וישׂאו and lifted up קולם their voices, ויבכו and wept בכי sore; גדול׃ sore;}%
\verse{ויאמרו And said, למה why יהוה O LORD אלהי God ישׂראל of Israel, היתה come to pass זאת is this בישׂראל in Israel, להפקד lacking היום that there should be today מישׂראל שׁבט tribe אחד׃ one}%
\verse{ויהי And it came to pass ממחרת on the morrow, וישׁכימו rose early, העם that the people ויבנו and built שׁם there מזבח an altar, ויעלו and offered עלות burnt offerings ושׁלמים׃ and peace offerings.}%
\verse{ויאמרו said, בני And the children ישׂראל of Israel מי Who אשׁר that לא came not up עלה came not up בקהל with the congregation מכל among all שׁבטי the tribes ישׂראל of Israel אל unto יהוה the LORD? כי For השׁבועה oath הגדולה a great היתה they had made לאשׁר concerning לא him that came not up עלה him that came not up אל to יהוה the LORD המצפה to Mizpeh, לאמר saying, מות יומת׃}%
\verse{וינחמו repented בני And the children ישׂראל of Israel אל them for בנימן Benjamin אחיו their brother, ויאמרו and said, נגדע cut off היום this day. שׁבט tribe אחד There is one מישׂראל׃}%
\verse{מה How נעשׂה shall we do להם לנותרים for them that remain, לנשׁים for wives ואנחנו seeing we נשׁבענו have sworn ביהוה by the LORD לבלתי that we will not תת give להם מבנותינו them of our daughters לנשׁים׃ to wives?}%
\verse{ויאמרו And they said, מי What אחד one משׁבטי of the tribes ישׂראל of Israel אשׁר that לא came not up עלה came not up אל to יהוה the LORD? המצפה to Mizpeh והנה And, behold, לא none בא there came אישׁ none אל to המחנה the camp מיבישׁ גלעדל to הקהל׃ the assembly.}%
\verse{ויתפקד were numbered, העם For the people והנה and, behold, אין none שׁם there. אישׁ none מיושׁבי of the inhabitants יבשׁ גלעד׃}%
\verse{וישׁלחו sent שׁם hither העדה And the congregation שׁנים twelve עשׂר twelve אלף thousand אישׁ men מבני החיליצוו and commanded אותם לאמר them, saying, לכו Go והכיתם and smite את יושׁבי the inhabitants יבשׁ גלעדפי with the edge חרב of the sword, והנשׁים with the women והטף׃ and the children.}%
\verse{וזה And this הדבר the thing אשׁר that תעשׂו ye shall do, כל every זכר male, וכל and every אשׁה woman ידעת that hath lain משׁכב that hath lain זכר by man. תחרימו׃ Ye shall utterly destroy}%
\verse{וימצאו And they found מיושׁבי among the inhabitants יבישׁ גלעדרבע four מאות hundred נערה young בתולה virgins, אשׁר that לא no ידעה had known אישׁ man למשׁכב by lying זכר with any male: ויביאו and they brought אותם אל them unto המחנה the camp שׁלה to Shiloh, אשׁר which בארץ in the land כנען׃ of Canaan.}%
\verse{וישׁלחו sent כל And the whole העדה congregation וידברו to speak אל to בני the children בנימן of Benjamin אשׁר that בסלע in the rock רמון Rimmon, ויקראו and to call להם שׁלום׃ peaceably}%
\verse{וישׁב came again בנימן And Benjamin בעת time; ההיא at that ויתנו and they gave להם הנשׁים them wives אשׁר which חיו they had saved alive מנשׁי of the women יבשׁ גלעדלא them not. מצאו they sufficed להם כן׃ and yet so}%
\verse{והעם And the people נחם repented them לבנימן for Benjamin, כי because עשׂה had made יהוה that the LORD פרץ a breach בשׁבטי in the tribes ישׂראל׃ of Israel.}%
\verse{ויאמרו said, זקני Then the elders העדה of the congregation מה How נעשׂה shall we do לנותרים for them that remain, לנשׁים for wives כי seeing נשׁמדה are destroyed מבנימן אשׁה׃ the women}%
\verse{ויאמרו And they said, ירשׁת an inheritance פליטה for them that be escaped לבנימן of Benjamin, ולא be not ימחה destroyed שׁבט that a tribe מישׂראל׃}%
\verse{ואנחנו Howbeit we לא not נוכל may לתת give להם נשׁים them wives מבנותינו of our daughters: כי for נשׁבעו have sworn, בני the children ישׂראל of Israel לאמר saying, ארור Cursed נתן he that giveth אשׁה a wife לבנימן׃ to Benjamin.}%
\verse{ויאמרו Then they said, הנה Behold, חג a feast יהוה of the LORD בשׁלו in Shiloh מימים ימימהשׁר which מצפונה on the north side לבית אל מזרחה on the east side השׁמשׁ on the east side למסלה of the highway העלה that goeth up מבית אל שׁכמה to Shechem, ומנגב and on the south ללבונה׃ of Lebonah.}%
\verse{ויצו Therefore they commanded את בני the children בנימן of Benjamin, לאמר saying, לכו Go וארבתם and lie in wait בכרמים׃ in the vineyards;}%
\verse{וראיתם And see, והנה and, behold, אם if יצאו come out בנות the daughters שׁילו of Shiloh לחול to dance במחלות in dances, ויצאתם then come ye out מן of הכרמים the vineyards, וחטפתם and catch לכם אישׁ you every man אשׁתו his wife מבנות of the daughters שׁילו of Shiloh, והלכתם and go ארץ to the land בנימן׃ of Benjamin.}%
\verse{והיה And it shall be, כי when יבאו come אבותם their fathers או or אחיהם their brethren לרוב us to complain, אלינו unto ואמרנו that we will say אליהם unto חנונו them, Be favorable אותם כי unto them for our sakes: because לא not לקחנו we reserved אישׁ to each man אשׁתו his wife במלחמה in the war: כי for לא did not אתם ye נתתם give להם כעת unto them at this time, תאשׁמו׃ ye should be guilty.}%
\verse{ויעשׂו did כן so, בני And the children בנימן of Benjamin וישׂאו and took נשׁים wives, למספרם according to their number, מן of המחללות them that danced, אשׁר whom גזלו they caught: וילכו and they went וישׁובו and returned אל unto נחלתם their inheritance, ויבנו and repaired את הערים the cities, וישׁבו׃ and dwelt}%
\verse{ויתהלכו departed משׁם thence בני And the children ישׂראל of Israel בעת time, ההיא at that אישׁ every man לשׁבטו to his tribe ולמשׁפחתו and to his family, ויצאו and they went out משׁם from thence אישׁ every man לנחלתו׃ to his inheritance.}%
\verse{בימים days ההם In those אין no מלך king בישׂראל in Israel: אישׁ every man הישׁר right בעיניו in his own eyes. יעשׂה׃ did}%
\end{biblechapter}%
\flushcolsend
\input{leb/content/old-testament/Ruth.tex}\flushcolsend
\input{leb/content/old-testament/1Sam.tex}\flushcolsend
\biblebook{2 Samuel}
\begin{biblechapter}% 2 Samuel 1
\verseWithHeading{The Report of Saul’s Death by the Amalekite}{ויהי Now it came to pass אחרי after מות שׁאול of Saul, ודוד when David שׁב was returned מהכות from the slaughter את העמלק of the Amalekites, וישׁב had abode דוד and David בצקלג in Ziklag; ימים days שׁנים׃ two}%
\verse{ויהי It came even to pass ביום day, השׁלישׁי on the third והנה that, behold, אישׁ a man בא came מן out of המחנה the camp מעם from שׁאול Saul ובגדיו with his clothes קרעים rent, ואדמה and earth על upon ראשׁו his head: ויהי and it was, בבאו when he came אל to דוד David, ויפל that he fell ארצה to the earth, וישׁתחו׃ and did obeisance.}%
\verse{ויאמר said לו דוד And David אי מזהבוא comest ויאמר thou? And he said אליו unto ממחנה ישׂראל of Israel נמלטתי׃ am I escaped.}%
\verse{ויאמר said אליו unto דוד And David מה him, How היה went הדבר the matter? הגד tell נא I pray thee, לי ויאמר me. And he answered, אשׁר That נס are fled העם the people מן from המלחמה the battle, וגם also הרבה and many נפל are fallen מן of העם the people וימתו and dead; וגם also. שׁאול and Saul ויהונתן and Jonathan בנו his son מתו׃ are dead}%
\verse{ויאמר said דוד And David אל unto הנער the young man המגיד that told לו איך him, How ידעת knowest כי thou that מת be dead? שׁאול Saul ויהונתן and Jonathan בנו׃ his son}%
\verse{ויאמר him said, הנער And the young man המגיד that told לו נקרא by chance נקריתי As I happened בהר upon mount הגלבע Gilboa, והנה behold, שׁאול Saul נשׁען leaned על upon חניתו his spear; והנה and, lo, הרכב the chariots ובעלי and horsemen הפרשׁים and horsemen הדבקהו׃ followed hard}%
\verse{ויפן And when he looked אחריו behind ויראני him, he saw ויקרא me, and called אלי unto ואמר me. And I answered, הנני׃}%
\verse{ויאמר And he said לי מי unto me, Who אתה thou? ויאמר And I answered אליו him, עמלקי an Amalekite. אנכי׃ I}%
\verse{ויאמר He said אלי unto עמד me again, Stand, נא I pray thee, עלי upon ומתתני me, and slay כי me: for אחזני is come השׁבץ anguish כי upon me, because כל whole עוד yet נפשׁי׃ my life}%
\verse{ואעמד So I stood עליו upon ואמתתהו him, and slew כי him, because ידעתי I was sure כי that לא he could not יחיה live אחרי after that נפלו he was fallen: ואקח and I took הנזר the crown אשׁר that על upon ראשׁו his head, ואצעדה and the bracelet אשׁר that על on זרעו his arm, ואביאם and have brought אל unto אדני my lord. הנה׃ them hither}%
\verse{ויחזק took hold דוד Then David בבגדו on his clothes, ויקרעם and rent וגם them; and likewise כל all האנשׁים the men אשׁר that אתו׃ with}%
\verse{ויספדו And they mourned, ויבכו and wept, ויצמו and fasted עד until הערב even, על for שׁאול Saul, ועל and for יהונתן Jonathan בנו his son, ועל and for עם the people יהוה ועל and for בית the house ישׂראל of Israel; כי because נפלו they were fallen בחרב׃ by the sword.}%
\verse{ויאמר said דוד And David אל unto הנער the young man המגיד that told לו אי מזהתה thou? ויאמר And he answered, בן the son אישׁ of a stranger, גר of a stranger, עמלקי an Amalekite. אנכי׃ I}%
\verse{ויאמר said אליו unto דוד And David איך him, How לא wast thou not יראת afraid לשׁלח to stretch forth ידך thine hand לשׁחת to destroy את משׁיח anointed? יהוה׃ the LORD's}%
\verse{ויקרא called דוד And David לאחד one מהנערים of the young men, ויאמר and said, גשׁ Go near, פגע fall בו ויכהו upon him. And he smote וימת׃ him that he died.}%
\verse{ויאמר said אליו unto דוד And David דמיך him, Thy blood על upon ראשׁך thy head; כי for פיך thy mouth ענה hath testified בך לאמר against thee, saying, אנכי I מתתי have slain את משׁיח anointed. יהוה׃ the LORD's}%
\verseWithHeading{David Laments Jonathan with the “Song of the Bow”}{ויקנן lamented דוד And David את with הקינה lamentation הזאת this על over שׁאול Saul ועל and over יהונתן Jonathan בנו׃ his son:}%
\verse{ויאמר (Also he bade ללמד them teach בני the children יהודה of Judah קשׁת the bow: הנה behold, כתובה written על in ספר the book הישׁר׃ of Jasher.)}%
\verse{הצבי The beauty ישׂראל of Israel על upon במותיך thy high places: חלל is slain איך how נפלו fallen! גבורים׃ are the mighty}%
\verse{אל not תגידו Tell בגת in Gath, אל not תבשׂרו publish בחוצת in the streets אשׁקלון of Askelon; פן lest תשׂמחנה rejoice, בנות the daughters פלשׁתים of the Philistines פן lest תעלזנה triumph. בנות the daughters הערלים׃ of the uncircumcised}%
\verse{הרי Ye mountains בגלבע of Gilboa, אל no טל dew, ואל neither מטר rain, עליכם upon ושׂדי you, nor fields תרומת of offerings: כי for שׁם there נגעל is vilely cast away, מגן the shield גבורים of the mighty מגן the shield שׁאול of Saul, בלי not משׁיח בשׁמן׃ with oil.}%
\verse{מדם חללים of the slain, מחלב from the fat גבורים of the mighty, קשׁת the bow יהונתן of Jonathan לא not נשׂוג turned אחור back, וחרב and the sword שׁאול of Saul לא not תשׁוב returned ריקם׃ empty.}%
\verse{שׁאול Saul ויהונתן and Jonathan הנאהבים lovely והנעימם and pleasant בחייהם in their lives, ובמותם and in their death לא they were not נפרדו divided: מנשׁרים than eagles, קלו they were swifter מאריות than lions. גברו׃ they were stronger}%
\verse{בנות Ye daughters ישׂראל of Israel, אל over שׁאול Saul, בכינה weep המלבשׁכם who clothed שׁני you in scarlet, עם with עדנים delights, המעלה who put on עדי ornaments זהב of gold על upon לבושׁכן׃ your apparel.}%
\verse{איך How נפלו fallen גברים are the mighty בתוך in the midst המלחמה of the battle! יהונתן O Jonathan, על in במותיך thine high places. חלל׃ slain}%
\verse{צר I am distressed לי עליך for אחי thee, my brother יהונתן Jonathan: נעמת pleasant לי מאד very נפלאתה to me was wonderful, אהבתך hast thou been unto me: thy love לי מאהבת passing the love נשׁים׃ of women.}%
\verse{איך How נפלו fallen, גבורים are the mighty ויאבדו perished! כלי and the weapons מלחמה׃ of war}%
\end{biblechapter}%
\begin{biblechapter}% 2 Samuel 2
\verseWithHeading{David Moves to Hebron}{ויהי And it came to pass אחרי after כן this, וישׁאל inquired דוד that David ביהוה of the LORD, לאמר saying, האעלה Shall I go up באחת into any ערי of the cities יהודה of Judah? ויאמר said יהוה And the LORD אליו unto עלה him, Go up. ויאמר said, דוד And David אנה Whither אעלה shall I go up? ויאמר And he said, חברנה׃ Unto Hebron.}%
\verse{ויעל went up שׁם thither, דוד So David וגם also, שׁתי and his two נשׁיו wives אחינעם Ahinoam היזרעלית the Jezreelitess, ואביגיל and Abigail אשׁת wife נבל Nabal's הכרמלי׃ the Carmelite.}%
\verse{ואנשׁיו And his men אשׁר that עמו with העלה bring up, דוד him did David אישׁ every man וביתו with his household: וישׁבו and they dwelt בערי in the cities חברון׃ of Hebron.}%
\verseWithHeading{David Anointed King over Judah at Hebron}{ויבאו came, אנשׁי And the men יהודה of Judah וימשׁחו they anointed שׁם and there את דוד David למלך king על over בית the house יהודה of Judah. ויגדו And they told לדוד David, לאמר saying, אנשׁי the men יבישׁ גלעדשׁר that קברו buried את שׁאול׃ Saul.}%
\verse{וישׁלח sent דוד And David מלאכים messengers אל unto אנשׁי the men יבישׁ גלעדיאמר and said אליהם unto ברכים them, Blessed אתם ye ליהוה of the LORD, אשׁר that עשׂיתם ye have showed החסד kindness הזה this עם unto אדניכם your lord, עם unto שׁאול Saul, ותקברו and have buried אתו׃}%
\verse{ועתה And now יעשׂ show יהוה the LORD עמכם unto חסד kindness ואמת and truth וגם also אנכי you: and I אעשׂה will requite אתכם הטובה kindness, הזאת you this אשׁר because עשׂיתם ye have done הדבר thing. הזה׃ this}%
\verse{ועתה Therefore now תחזקנה be strengthened, ידיכם let your hands והיו and be לבני ye valiant: חיל ye valiant: כי for מת is dead, אדניכם your master שׁאול Saul וגם and also אתי משׁחו have anointed בית the house יהודה of Judah למלך me king עליהם׃ over}%
\verseWithHeading{Ish-Bosheth over Israel}{ואבנר But Abner בן the son נר of Ner, שׂר captain צבא host, אשׁר לשׁאול of Saul's לקח took את אישׁ בשׁת Ish-bosheth בן the son שׁאול of Saul, ויעברהו and brought him over מחנים׃ to Mahanaim;}%
\verse{וימלכהו And made him king אל over הגלעד Gilead, ואל and over האשׁורי ואל and over יזרעאל Jezreel, ועל and over אפרים Ephraim, ועל and over בנימן Benjamin, ועל and over ישׂראל Israel. כלה׃ all}%
\verse{בן son ארבעים forty שׁנה years אישׁ בשׁת Ish-bosheth בן old שׁאול Saul's במלכו when he began to reign על over ישׂראל Israel, ושׁתים two שׁנים years. מלך and reigned אך But בית the house יהודה of Judah היו followed אחרי followed דוד׃ David.}%
\verse{ויהי was מספר הימיםשׁר that היה was דוד David מלך king בחברון in Hebron על over בית the house יהודה of Judah שׁבע seven שׁנים years ושׁשׁה and six חדשׁים׃ months.}%
\verseWithHeading{War between Judah and Israel}{ויצא went out אבנר And Abner בן the son נר of Ner, ועבדי and the servants אישׁ בשׁת of Ish-bosheth בן the son שׁאול of Saul, ממחנים גבעונה׃ to Gibeon.}%
\verse{ויואב And Joab בן the son צרויה of Zeruiah, ועבדי and the servants דוד of David, יצאו went out, ויפגשׁום and met על by ברכת the pool גבעון of Gibeon: יחדו together וישׁבו and they sat down, אלה the one על of הברכה the pool, מזה on the one side ואלה and the other על of הברכה the pool. מזה׃ on the other side}%
\verse{ויאמר said אבנר And Abner אל to יואב Joab, יקומו arise, נא now הנערים Let the young men וישׂחקו and play לפנינו before ויאמר said, יואב us. And Joab יקמו׃ Let them arise.}%
\verse{ויקמו Then there arose ויעברו and went over במספר by number שׁנים twelve עשׂר twelve לבנימן of Benjamin, ולאישׁ בשׁת which to Ish-bosheth בן the son שׁאול of Saul, ושׁנים and twelve עשׂר and twelve מעבדי of the servants דוד׃ of David.}%
\verse{ויחזקו And they caught אישׁ every one בראשׁ by the head, רעהו his fellow וחרבו and his sword בצד side; רעהו in his fellow's ויפלו so they fell down יחדו together: ויקרא was called למקום place ההוא wherefore that חלקת הצרים Helkath-hazzurim, אשׁר which בגבעון׃ in Gibeon.}%
\verse{ותהי And there was המלחמה battle קשׁה sore עד a very מאד a very ביום day; ההוא that וינגף was beaten, אבנר and Abner ואנשׁי and the men ישׂראל of Israel, לפני before עבדי the servants דוד׃ of David.}%
\verse{ויהיו were שׁם And there שׁלשׁה three בני sons צרויה of Zeruiah יואב there, Joab, ואבישׁי and Abishai, ועשׂהאל and Asahel: ועשׂהאל and Asahel קל light ברגליו of foot כאחד as a הצבים roe. אשׁר wild בשׂדה׃ wild}%
\verse{וירדף pursued עשׂהאל And Asahel אחרי after אבנר Abner; ולא not נטה he turned ללכת and in going על to הימין the right ועל hand nor to השׂמאול the left מאחרי from following אבנר׃ Abner.}%
\verse{ויפן looked אבנר Then Abner אחריו behind ויאמר him, and said, האתה thou זה עשׂהאל Asahel? ויאמר And he answered, אנכי׃ I}%
\verse{ויאמר said לו אבנר And Abner נטה to him, Turn thee aside לך על to ימינך thy right או hand or על to שׂמאלך thy left, ואחז and lay thee hold לך אחד on one מהנערים of the young men, וקח and take לך את חלצתו thee his armor. ולא not אבה would עשׂהאל But Asahel לסור turn aside מאחריו׃ from following}%
\verse{ויסף again עוד אבנר And Abner לאמר said אל to עשׂהאל Asahel, סור Turn thee aside לך מאחרי from following למה me: wherefore אככה should I smite ארצה thee to the ground? ואיך how אשׂא then should I hold up פני my face אל to יואב Joab אחיך׃ thy brother?}%
\verse{וימאן Howbeit he refused לסור to turn aside: ויכהו smote אבנר wherefore Abner באחרי with the hinder end of החנית the spear אל him under החמשׁ the fifth ותצא came out החנית that the spear מאחריו behind ויפל him; and he fell down שׁם there, וימת and died תחתו in the same place: ויהי and it came to pass, כל as many הבא as came אל to המקום the place אשׁר where נפל fell down שׁם עשׂהאל Asahel וימת and died ויעמדו׃ stood still.}%
\verse{וירדפו pursued יואב Joab ואבישׁי also and Abishai אחרי after אבנר Abner: והשׁמשׁ and the sun באה went down והמה when they באו were come עד to גבעת the hill אמה of Ammah, אשׁר that על before פני before גיח Giah דרך by the way מדבר of the wilderness גבעון׃ of Gibeon.}%
\verse{ויתקבצו gathered themselves together בני And the children בנימן of Benjamin אחרי after אבנר Abner, ויהיו and became לאגדה troop, אחת one ויעמדו and stood על on ראשׁ the top גבעה hill. אחת׃ of a}%
\verse{ויקרא called אבנר Then Abner אל to יואב Joab, ויאמר and said, הלנצח forever? תאכל devour חרב Shall the sword הלוא thou not ידעתה knowest כי that מרה bitterness תהיה it will be באחרונה in the latter end? ועד how long מתי לא shall it be then, ere תאמר thou bid לעם the people לשׁוב return מאחרי from following אחיהם׃ their brethren?}%
\verse{ויאמר said, יואב And Joab חי liveth, האלהים God כי unless לולא unless דברת thou hadst spoken, כי surely אז then מהבקר in the morning נעלה had gone up העם the people אישׁ every one מאחרי from following אחיו׃ his brother.}%
\verse{ויתקע blew יואב So Joab בשׁופר a trumpet, ויעמדו stood כל and all העם the people ולא no ירדפו still, and pursued עוד more, אחרי after ישׂראל Israel ולא neither יספו they any more. עוד להלחם׃ fought}%
\verse{ואבנר And Abner ואנשׁיו הלכו walked בערבה through the plain, כל all הלילה night ההוא that ויעברו and passed over את הירדן Jordan, וילכו and went through כל all הבתרון Bithron, ויבאו and they came מחנים׃ to Mahanaim.}%
\verse{ויואב And Joab שׁב returned מאחרי from following אבנר Abner: ויקבץ אתל העםיפקדו there lacked מעבדי דודשׁעה nineteen עשׂר nineteen אישׁ men ועשׂהאל׃ and Asahel.}%
\verse{ועבדי But the servants דוד of David הכו had smitten מבנימן ובאנשׁי men, אבנר and of Abner's שׁלשׁ three מאות hundred ושׁשׁים and threescore אישׁ men מתו׃ died.}%
\verse{וישׂאו And they took up את עשׂהאל Asahel, ויקברהו and buried בקבר him in the sepulcher אביו of his father, אשׁר which בית לחם Bethlehem. וילכו went כל all הלילה night, יואב And Joab ואנשׁיו and his men ויאר at break of day. להם בחברון׃ and they came to Hebron}%
\end{biblechapter}%
\begin{biblechapter}% 2 Samuel 3
\verseWithHeading{The House of David Grows Stronger}{ותהי Now there was המלחמה war ארכה long בין between בית the house שׁאול of Saul ובין בית and the house דוד of David: ודוד but David הלך waxed stronger and stronger, וחזק waxed stronger and stronger, ובית and the house שׁאול of Saul הלכים waxed ודלים׃ weaker and weaker.}%
\verse{וילדו born לדוד And unto David בנים were sons בחברון in Hebron: ויהי was בכורו and his firstborn אמנון Amnon, לאחינעם of Ahinoam היזרעאלת׃ the Jezreelitess;}%
\verse{ומשׁנהו And his second, כלאב Chileab, לאביגל of Abigail אשׁת the wife נבל of Nabal הכרמלי the Carmelite; והשׁלשׁי and the third, אבשׁלום Absalom בן the son מעכה of Maacah בת the daughter תלמי of Talmai מלך king גשׁור׃ of Geshur;}%
\verse{והרביעי And the fourth, אדניה Adonijah בן the son חגית of Haggith; והחמישׁי and the fifth, שׁפטיה Shephatiah בן the son אביטל׃ of Abital;}%
\verse{והשׁשׁי And the sixth, יתרעם Ithream, לעגלה by Eglah אשׁת wife. דוד David's אלה These ילדו were born לדוד to David בחברון׃ in Hebron.}%
\verse{ויהי And it came to pass, בהיות while there was המלחמה war בין between בית the house שׁאול of Saul ובין בית and the house דוד of David, ואבנר that Abner היה מתחזק made himself strong בבית for the house שׁאול׃ of Saul.}%
\verse{ולשׁאול And Saul פלגשׁ had a concubine, ושׁמה whose name רצפה Rizpah, בת the daughter איה of Aiah: ויאמר and said אל to אבנר Abner, מדוע Wherefore באתה hast thou gone אל in unto פילגשׁ concubine? אבי׃ my father's}%
\verse{ויחר wroth לאבנר Then was Abner מאד very על for דברי the words אישׁ בשׁת of Ish-bosheth, ויאמר and said, הראשׁ head, כלב a dog's אנכי I אשׁר which ליהודה against Judah היום this day אעשׂה do show חסד kindness עם unto בית the house שׁאול of Saul אביך thy father, אל to אחיו his brethren, ואל and to מרעהו his friends, ולא and have not המציתך delivered ביד thee into the hand דוד of David, ותפקד that thou chargest עלי that thou chargest עון with a fault האשׁה concerning this woman? היום׃ me today}%
\verse{כה So יעשׂה do אלהים God לאבנר to Abner, וכה also, יסיף and more לו כי except, כאשׁר as נשׁבע hath sworn יהוה the LORD לדוד to David, כי even כן so אעשׂה׃ I do}%
\verse{להעביר To translate הממלכה the kingdom מבית from the house שׁאול of Saul, ולהקים and to set up את כסא the throne דוד of David על over ישׂראל Israel ועל and over יהודה Judah, מדן ועד even to באר שׁבע׃ Beer-sheba.}%
\verse{ולא not יכל And he could עוד again, להשׁיב answer את אבנר Abner דבר a word מיראתו because he feared אתו׃}%
\verseWithHeading{Abner Pledges Support for David}{וישׁלח sent אבנר And Abner מלאכים messengers אל to דוד David תחתו on his behalf, לאמר saying, למי Whose ארץ the land? לאמר saying כרתה Make בריתך thy league אתי with והנה me, and, behold, ידי my hand עמך with להסב thee, to bring about אליך unto את כל all ישׂראל׃ Israel}%
\verse{ויאמר And he said, טוב Well; אני I אכרת will make אתך with ברית a league אך thee: but דבר thing אחד one אנכי I שׁאל require מאתך of לאמר thee, that is, לא Thou shalt not תראה see את פני my face, כי except אם except לפני thou first הביאך bring את מיכל Michal בת daughter, שׁאול Saul's בבאך when thou comest לראות to see את פני׃ my face.}%
\verse{וישׁלח sent דוד And David מלאכים messengers אל to אישׁ בשׁת Ish-bosheth בן son, שׁאול Saul's לאמר saying, תנה Deliver את אשׁתי my wife את מיכל Michal, אשׁר which ארשׂתי I espoused לי במאה to me for a hundred ערלות foreskins פלשׁתים׃ of the Philistines.}%
\verse{וישׁלח sent, אישׁ בשׁת And Ish-bosheth ויקחה and took מעם her from אישׁ husband, מעם from פלטיאל Phaltiel בן the son לושׁ׃ of Laish.}%
\verse{וילך went אתה with אישׁה And her husband הלוך her along ובכה weeping אחריה behind עד her to בחרים Bahurim. ויאמר Then said אליו unto אבנר Abner לך him, Go, שׁוב return. וישׁב׃ And he returned.}%
\verse{ודבר communication אבנר And Abner היה had עם with זקני the elders ישׂראל of Israel, לאמר saying, גם in times תמול in times גם past שׁלשׁם past הייתם מבקשׁיםת דוד for David למלך king עליכם׃ over}%
\verse{ועתה Now עשׂו then do כי for יהוה the LORD אמר hath spoken אל of דוד David, לאמר saying, ביד By the hand דוד David עבדי of my servant הושׁיע I will save את עמי my people ישׂראל Israel מיד out of the hand פלשׁתים of the Philistines, ומיד and out of the hand כל of all איביהם׃ their enemies.}%
\verse{וידבר spoke גם also אבנר And Abner באזני in the ears בנימין of Benjamin: וילך went גם also אבנר and Abner לדבר to speak באזני in the ears דוד of David בחברון in Hebron את כל all אשׁר that טוב seemed good בעיני seemed good ישׂראל to Israel, ובעיני and that seemed good כל to the whole בית house בנימן׃ of Benjamin.}%
\verse{ויבא came אבנר So Abner אל to דוד David חברון to Hebron, ואתו with עשׂרים and twenty אנשׁים men ויעשׂ made דוד him. And David לאבנר Abner ולאנשׁים and the men אשׁר that אתו with משׁתה׃ him a feast.}%
\verse{ויאמר said אבנר And Abner אל unto דוד David, אקומה I will arise ואלכה and go, ואקבצה and will gather אל unto אדני my lord המלך the king, את כל all ישׂראל Israel ויכרתו that they may make אתך with ברית a league ומלכת thee, and that thou mayest reign בכל over all אשׁר that תאוה desireth. נפשׁך thine heart וישׁלח sent דוד And David את אבנר Abner וילך away; and he went בשׁלום׃ in peace.}%
\verse{והנה And, behold, עבדי the servants דוד of David ויואב and Joab בא came מהגדוד from a troop, ושׁלל spoil רב in a great עמם with הביאו and brought ואבנר them: but Abner איננו not עם with דוד David בחברון in Hebron; כי for שׁלחו he had sent him away, וילך and he was gone בשׁלום׃ in peace.}%
\verse{ויואב When Joab וכל and all הצבא the host אשׁר that אתו with באו him were come, ויגדו they told ליואב Joab, לאמר saying, בא came אבנר Abner בן the son נר of Ner אל to המלך the king, וישׁלחהו and he hath sent him away, וילך and he is gone בשׁלום׃ in peace.}%
\verseWithHeading{Joab Assassinates Abner}{ויבא came יואב Then Joab אל to המלך the king, ויאמר and said, מה What עשׂיתה hast thou done? הנה behold, בא came אבנר Abner אליך unto למה thee; why זה שׁלחתו it thou hast sent him away, וילך and he is quite gone? הלוך׃}%
\verse{ידעת Thou knowest את אבנר Abner בן the son נר of Ner, כי that לפתתך to deceive בא he came ולדעת thee, and to know את מוצאך thy going out ואת מבואךלדעת and to know את כל all אשׁר that אתה thou עשׂה׃ doest.}%
\verse{ויצא was come out יואב And when Joab מעם from דוד David, וישׁלח he sent מלאכים messengers אחרי after אבנר Abner, וישׁבו אתובור from the well הסרה of Sirah: ודוד but David לא not. ידע׃ knew}%
\verse{וישׁב was returned אבנר And when Abner חברון to Hebron, ויטהו took him aside יואב Joab אל in תוך in השׁער the gate לדבר to speak אתו with בשׁלי him quietly, ויכהו and smote שׁם him there החמשׁ under the fifth וימת that he died, בדם for the blood עשׂה אל of Asahel אחיו׃ his brother.}%
\verse{וישׁמע heard דוד when David מאחרי כןיאמר he said, נקי guiltless אנכי I וממלכתי and my kingdom מעם before יהוה the LORD עד עולם forever מדמי from the blood אבנר of Abner בן the son נר׃ of Ner:}%
\verse{יחלו Let it rest על on ראשׁ the head יואב of Joab, ואל and on כל all בית house; אביו his father's ואל and let there not יכרת fail מבית from the house יואב of Joab זב one that hath an issue, ומצרע or that is a leper, ומחזיק or that leaneth בפלך on a staff, ונפל or that falleth בחרב on the sword, וחסר or that lacketh לחם׃ bread.}%
\verse{ויואב So Joab ואבישׁי and Abishai אחיו his brother הרגו slew לאבנר Abner, על because אשׁר because המית he had slain את עשׂהאל Asahel אחיהם their brother בגבעון at Gibeon במלחמה׃ in the battle.}%
\verse{ויאמר said דוד And David אל to יואב Joab, ואל and to כל all העם the people אשׁר that אתו with קרעו him, Rend בגדיכם your clothes, וחגרו and gird שׂקים you with sackcloth, וספדו and mourn לפני before אבנר Abner. והמלך And king דוד David הלך אחרי followed המטה׃ the bier.}%
\verse{ויקברו And they buried את אבנר Abner בחברון in Hebron: וישׂא lifted up המלך and the king את קולו his voice, ויבך and wept אל at קבר the grave אבנר of Abner; ויבכו wept. כל and all העם׃ the people}%
\verse{ויקנן lamented המלך And the king אל over אבנר Abner, ויאמר and said, הכמות dieth? נבל as a fool ימות Died אבנר׃ Abner}%
\verse{ידך Thy hands לא not אסרות bound, ורגליך thy feet לא nor לנחשׁתים into fetters: הגשׁו put כנפול falleth לפני before בני as a man עולה wicked נפלת fellest ויספו again כל thou. And all העם the people לבכות wept עליו׃ over}%
\verse{ויבא came כל And when all העם the people להברות to eat את דוד to cause David לחם meat בעוד while it was yet היום day, וישׁבע swore, דוד David לאמר saying, כה So יעשׂה do לי אלהים God וכה also, יסיף to me, and more כי if אם if לפני till בוא be down. השׁמשׁ the sun אטעם I taste לחם bread, או or כל aught else, מאומה׃ aught else,}%
\verse{וכל And all העם the people הכירו took notice וייטב and it pleased them: בעיניהם , and it pleased them: ככל as whatsoever אשׁר as whatsoever עשׂה did המלך the king בעיני pleased כל all העם the people. טוב׃}%
\verse{וידעו understood כל For all העם the people וכל and all ישׂראל Israel ביום day ההוא that כי that לא not היתה it was מהמלך of the king להמית to slay את אבנר Abner בן the son נר׃ of Ner.}%
\verse{ויאמר said המלך And the king אל unto עבדיו his servants, הלוא ye not תדעו Know כי that שׂר there is a prince וגדול and a great man נפל fallen היום day הזה this בישׂראל׃ in Israel?}%
\verse{ואנכי And I היום this day רך weak, ומשׁוח though anointed מלך king; והאנשׁים men האלה and these בני the sons צרויה of Zeruiah קשׁים too hard ממני for ישׁלם shall reward יהוה me: the LORD לעשׂה the doer הרעה of evil כרעתו׃ according to his wickedness.}%
\end{biblechapter}%
\begin{biblechapter}% 2 Samuel 4
\verseWithHeading{Ish-Bosheth is Assassinated}{וישׁמע heard בן son שׁאול And when Saul's כי that מת was dead אבנר Abner בחברון in Hebron, וירפו were feeble, ידיו his hands וכל and all ישׂראל the Israelites נבהלו׃ were troubled.}%
\verse{ושׁני two אנשׁים men שׂרי captains גדודים of bands: היו had בן son שׁאול And Saul's שׁם the name האחד of the one בענה Baanah, ושׁם and the name השׁני of the other רכב Rechab, בני the sons רמון of Rimmon הבארתי a Beerothite, מבני of the children בנימן of Benjamin: כי (for גם also בארות Beeroth תחשׁב was reckoned על to בנימן׃ Benjamin:}%
\verse{ויברחו fled הבארתים And the Beerothites גתימה to Gittaim, ויהיו and were שׁם there גרים sojourners עד until היום day.) הזה׃ this}%
\verse{וליהונתן And Jonathan, בן son, שׁאול Saul's בן had a son נכה lame רגלים of feet. בן old חמשׁ five שׁנים years היה He was בבא came שׁמעת when the tidings שׁאול of Saul ויהונתן and Jonathan מיזרעאל ותשׂאהו took him up, אמנתו and his nurse ותנס and fled: ויהי and it came to pass, בחפזה as she made haste לנוס to flee, ויפל that he fell, ויפסח and became lame. ושׁמו And his name מפיבשׁת׃ Mephibosheth.}%
\verse{וילכו went, בני And the sons רמון of Rimmon הבארתי the Beerothite, רכב Rechab ובענה ויבאו and came כחם about the heat היום of the day אל to בית the house אישׁ בשׁת of Ish-bosheth, והוא who שׁכב lay את משׁכב on a bed הצהרים׃ at noon.}%
\verse{והנה באו And they came עד into תוך the midst הבית of the house, לקחי they would have fetched חטים wheat; ויכהו and they smote אל him under החמשׁ the fifth ורכב and Rechab ובענה and Baanah אחיו his brother נמלטו׃ escaped.}%
\verse{ויבאו For when they came הבית into the house, והוא he שׁכב lay על on מטתו his bed בחדר in his bedchamber, משׁכבו in his bedchamber, ויכהו and they smote וימתהו him, and slew ויסירו אתאשׁו his head, ויקחו him, and took את ראשׁוילכו and got them away דרך through הערבה the plain כל all הלילה׃ night.}%
\verse{ויבאו And they brought את ראשׁ the head אישׁ בשׁת of Ish-bosheth אל unto דוד David חברון to Hebron, ויאמרו and said אל to המלך the king, הנה Behold ראשׁ the head אישׁ בשׁת of Ish-bosheth בן the son שׁאול of Saul איבך thine enemy, אשׁר which בקשׁ sought את נפשׁך thy life; ויתן hath avenged יהוה and the LORD לאדני my lord המלך the king נקמות hath avenged היום day הזה this משׁאול ומזרעו׃ and of his seed.}%
\verse{ויען answered דוד And David את רכב Rechab ואת בענה and Baanah אחיו his brother, בני the sons רמון of Rimmon הבארתי the Beerothite, ויאמר and said להם חי liveth, יהוה unto them, the LORD אשׁר who פדה hath redeemed את נפשׁי my soul מכל out of all צרה׃ adversity,}%
\verse{כי When המגיד one told לי לאמר me, saying, הנה Behold, מת is dead, שׁאול Saul והוא היה to have כמבשׂר brought good tidings, בעיניו thinking ואחזה I took hold בו ואהרגהו of him, and slew בצקלג him in Ziklag, אשׁר who לתתי that I would have given לו בשׂרה׃ him a reward for his tidings:}%
\verse{אף How much more, כי when אנשׁים men רשׁעים wicked הרגו have slain את אישׁ person צדיק a righteous בביתו in his own house על upon משׁכבו his bed? ועתה therefore now הלוא shall I not אבקשׁ require את דמו his blood מידכם of your hand, ובערתי אתכםן of your hand, הארץ׃ the earth?}%
\verse{ויצו commanded דוד And David את הנערים his young men, ויהרגום and they slew ויקצצו them, and cut off את ידיהם their hands ואת רגליהם and their feet, ויתלו and hanged up על over הברכה the pool בחברון in Hebron. ואת ראשׁ the head אישׁ בשׁת of Ish-bosheth, לקחו But they took ויקברו and buried בקבר in the sepulcher אבנר of Abner בחברון׃ in Hebron.}%
\end{biblechapter}%
\begin{biblechapter}% 2 Samuel 5
\verseWithHeading{David Anointed as King over All of Israel}{ויבאו Then came כל all שׁבטי the tribes ישׂראל of Israel אל to דוד David חברונה unto Hebron, ויאמרו and spoke, לאמר saying, הננו עצמך thy bone ובשׂרך and thy flesh. אנחנו׃ we}%
\verse{גם Also אתמול in time past, גם שׁלשׁום in time past, בהיות was שׁאול when Saul מלך king עלינו over אתה us, thou הייתה wast מוציא he that leddest out והמבי and broughtest את ישׂראל in Israel: ויאמר said יהוה and the LORD לך אתה to thee Thou תרעה shalt feed את עמי my people את ישׂראל Israel, ואתה and thou תהיה shalt be לנגיד a captain על over ישׂראל׃ Israel.}%
\verse{ויבאו came כל So all זקני the elders ישׂראל of Israel אל to המלך the king חברונה to Hebron; ויכרת made להם המלך and king דוד David ברית a league בחברון with them in Hebron לפני before יהוה the LORD: וימשׁחו and they anointed את דוד David למלך king על over ישׂראל׃ Israel.}%
\verse{בן old שׁלשׁים thirty שׁנה years דוד David במלכו when he began to reign, ארבעים forty שׁנה years. מלך׃ he reigned}%
\verse{בחברון In Hebron מלך he reigned על over יהודה Judah שׁבע seven שׁנים years ושׁשׁה and six חדשׁים months: ובירושׁלם and in Jerusalem מלך he reigned שׁלשׁים thirty ושׁלשׁ and three שׁנה years על over כל all ישׂראל Israel ויהודה׃ and Judah.}%
\verseWithHeading{The Capture and Growth of Jerusalem}{וילך went המלך And the king ואנשׁיו and his men ירושׁלם to Jerusalem אל unto היבסי the Jebusites, יושׁב the inhabitants הארץ of the land: ויאמר which spoke לדוד unto David, לאמר saying, לא thou shalt not תבוא come in הנה hither: כי אםסירך thou take away העורים the blind והפסחים לאמר thinking, לא cannot יבוא come in דוד David הנה׃ hither.}%
\verse{וילכד took דוד Nevertheless David את מצדת the stronghold ציון of Zion: היא the same עיר the city דוד׃ of David.}%
\verse{ויאמר said דוד And David ביום day, ההוא on that כל Whosoever מכה and smiteth יבסי the Jebusites, ויגע getteth up בצנור to the gutter, ואת הפסחים and the lame ואת העורים and the blind, שׂנאו hated נפשׁ soul, דוד of David's על כןאמרו they said, עור The blind ופסח and the lame לא shall not יבוא come אל into הבית׃ the house.}%
\verse{וישׁב dwelt דוד So David במצדה in the fort, ויקרא and called לה עיר it the city דוד of David. ויבן built דוד And David סביב round about מן from המלוא Millo וביתה׃ and inward.}%
\verse{וילך went on, דוד And David הלוך went on, וגדול and grew great, ויהוה and the LORD אלהי God צבאות of hosts עמו׃ with}%
\verse{וישׁלח sent חירם And Hiram מלך king צר of Tyre מלאכים messengers אל to דוד David, ועצי trees, ארזים and cedar וחרשׁי and carpenters, עץ and carpenters, וחרשׁי and masons: אבן קיריבנו and they built בית a house. לדוד׃ David}%
\verse{וידע perceived דוד And David כי that הכינו had established יהוה the LORD למלך him king על over ישׂראל Israel, וכי and that נשׂא he had exalted ממלכתו his kingdom בעבור sake. עמו for his people ישׂראל׃ Israel's}%
\verse{ויקח took דוד And David עוד more פלגשׁים concubines ונשׁים and wives מירושׁלם אחרי after באו he was come מחברון ויולדו born עוד and there were yet לדוד to David. בנים sons ובנות׃ and daughters}%
\verse{ואלה And these שׁמות the names הילדים of those that were born לו בירושׁלם unto him in Jerusalem; שׁמוע Shammua, ושׁובב and Shobab, ונתן and Nathan, ושׁלמה׃ and Solomon,}%
\verse{ויבחר Ibhar ואלישׁוע also, and Elishua, ונפג and Nepheg, ויפיע׃ and Japhia,}%
\verse{ואלישׁמע And Elishama, ואלידע and Eliada, ואליפלט׃ and Eliphalet.}%
\verseWithHeading{War with the Philistines}{וישׁמעו heard פלשׁתים But when the Philistines כי that משׁחו they had anointed את דוד David למלך king על over ישׂראל Israel, ויעלו came up כל all פלשׁתים the Philistines לבקשׁ to seek את דוד David; וישׁמע heard דוד and David וירד and went down אל to המצודה׃ the hold.}%
\verse{ופלשׁתים The Philistines באו also came וינטשׁו and spread themselves בעמק in the valley רפאים׃ of Rephaim.}%
\verse{וישׁאל inquired דוד And David ביהוה of the LORD, לאמר saying, האעלה Shall I go up אל to פלשׁתים the Philistines? התתנם wilt thou deliver בידי them into mine hand? ויאמר said יהוה And the LORD אל unto דוד David, עלה Go up: כי for נתן אתןת הפלשׁתים the Philistines בידך׃ into thine hand.}%
\verse{ויבא came דוד And David בבעל פרצים to Baal-perazim, ויכם smote שׁם them there, דוד and David ויאמר and said, פרץ hath broken forth upon יהוה The LORD את איבי mine enemies לפני before כפרץ me, as the breach מים of waters. על כןרא he called שׁם the name המקום place ההוא of that בעל פרצים׃ Baal-perazim.}%
\verse{ויעזבו they left שׁם And there את עצביהם their images, וישׂאם burned דוד and David ואנשׁיו׃}%
\verse{ויספו again, עוד yet פלשׁתים And the Philistines לעלות came up וינטשׁו and spread themselves בעמק in the valley רפאים׃ of Rephaim.}%
\verse{וישׁאל inquired דוד And when David ביהוה of the LORD, ויאמר he said, לא Thou shalt not תעלה go up; הסב fetch a compass אל fetch a compass אחריהם behind ובאת them, and come להם ממול upon them over against בכאים׃ the mulberry trees.}%
\verse{ויהי And let it be, בשׁמעך when thou hearest את קול the sound צעדה of a going בראשׁי in the tops הבכאים of the mulberry trees, אז that then תחרץ thou shalt bestir כי thyself: for אז then יצא go out יהוה shall the LORD לפניך before להכות thee, to smite במחנה the host פלשׁתים׃ of the Philistines.}%
\verse{ויעשׂ did דוד And David כן so, כאשׁר as צוהו had commanded יהוה the LORD ויך him; and smote את פלשׁתים the Philistines מגבע עדאך until thou come גזר׃ to Gazer.}%
\end{biblechapter}%
\begin{biblechapter}% 2 Samuel 6
\verseWithHeading{David Brings the Ark of Adonai to Jerusalem}{ויסף עוד Again, דוד David את כל all בחור בישׂראל of Israel, שׁלשׁים thirty אלף׃ thousand.}%
\verse{ויקם arose, וילך and went דוד And David וכל with all העם the people אשׁר that אתו with מבעלי יהודה להעלות to bring up משׁם from thence את ארון the ark האלהים of God, אשׁר whose נקרא is called שׁם name שׁם by the name יהוה of the LORD צבאות of hosts ישׁב that dwelleth הכרבים the cherubims. עליו׃ that dwelleth}%
\verse{וירכבו And they set את ארון the ark האלהים of God אל upon עגלה cart, חדשׁה a new וישׂאהו and brought מבית it out of the house אבינדב of Abinadab אשׁר that בגבעה ועזא and Uzzah ואחיו and Ahio, בני the sons אבינדב of Abinadab, נהגים drove את העגלה cart. חדשׁה׃ the new}%
\verse{וישׂאהו And they brought מבית it out of the house אבינדב of Abinadab אשׁר which בגבעה עם accompanying ארון the ark האלהים of God: ואחיו and Ahio הלך went לפני before הארון׃ the ark.}%
\verse{ודוד And David וכל and all בית the house ישׂראל of Israel משׂחקים played לפני before יהוה the LORD בכל on all עצי wood, ברושׁים manner of fir ובכנרות even on harps, ובנבלים and on psalteries, ובתפים and on timbrels, ובמנענעים and on cornets, ובצלצלים׃ and on cymbals.}%
\verse{ויבאו And when they came עד to גרן threshingfloor, נכון Nachon's וישׁלח put forth עזא Uzzah אל to ארון the ark האלהים of God, ויאחז and took hold בו כי of it; for שׁמטו shook הבקר׃ the oxen}%
\verse{ויחר was kindled אף And the anger יהוה of the LORD בעזה against Uzzah; ויכהו smote שׁם him there האלהים and God על for השׁל error; וימת he died שׁם and there עם by ארון the ark האלהים׃ of God.}%
\verse{ויחר was displeased, לדוד And David על because אשׁר because פרץ had made יהוה the LORD פרץ a breach בעזה upon Uzzah: ויקרא and he called the name למקום place ההוא of the פרץ עזה Perez-uzzah עד to היום day. הזה׃ this}%
\verse{וירא was afraid דוד And David את יהוה of the LORD ביום day, ההוא that ויאמר and said, איך How יבוא come אלי to ארון shall the ark יהוה׃ of the LORD}%
\verse{ולא not אבה would דוד So David להסיר remove אליו unto את ארון the ark יהוה of the LORD על him into עיר the city דוד of David: ויטהו carried it aside דוד but David בית into the house עבד אדום of Obed-edom הגתי׃ the Gittite.}%
\verse{וישׁב continued ארון And the ark יהוה of the LORD בית in the house עבד אדם of Obed-edom הגתי the Gittite שׁלשׁה three חדשׁים months: ויברך blessed יהוה and the LORD את עבד אדם Obed-edom, ואת כל and all ביתו׃ his household.}%
\verse{ויגד And it was told למלך king דוד David, לאמר saying, ברך hath blessed יהוה The LORD את בית the house עבד אדם of Obed-edom, ואת כל and all אשׁר that לו בעבור unto him, because ארון of the ark האלהים of God. וילך went דוד So David ויעל and brought up את ארון the ark האלהים of God מבית from the house עבד אדם of Obed-edom עיר into the city דוד of David בשׂמחה׃ with gladness.}%
\verse{ויהי And it was כי that when צעדו had gone נשׂאי they that bore ארון the ark יהוה of the LORD שׁשׁה six צעדים paces, ויזבח he sacrificed שׁור oxen ומריא׃ and fatlings.}%
\verse{ודוד And David מכרכר danced בכל with all עז might; לפני before יהוה the LORD ודוד and David חגור girded אפוד ephod. בד׃ with a linen}%
\verse{ודוד So David וכל and all בית the house ישׂראל of Israel מעלים brought up את ארון the ark יהוה of the LORD בתרועה with shouting, ובקול and with the sound שׁופר׃ of the trumpet.}%
\verse{והיה And as ארון the ark יהוה of the LORD בא came עיר into the city דוד of David, ומיכל Michal בת daughter שׁאול Saul's נשׁקפה looked בעד through החלון a window, ותרא and saw את המלך king דוד David מפזז leaping ומכרכר and dancing לפני before יהוה the LORD; ותבז and she despised לו בלבה׃ him in her heart.}%
\verse{ויבאו And they brought in את ארון the ark יהוה of the LORD, ויצגו and set אתו במקומו it in his place, בתוך in the midst האהל of the tabernacle אשׁר that נטה had pitched לו דוד David ויעל offered דוד for it: and David עלות burnt offerings לפני before יהוה the LORD. ושׁלמים׃ and peace offerings}%
\verse{ויכל had made an end דוד And as soon as David מהעלות of offering העולה burnt offerings והשׁלמים and peace offerings, ויברך he blessed את העם the people בשׁם in the name יהוה of the LORD צבאות׃ of hosts.}%
\verse{ויחלק And he dealt לכל among all העם the people, לכל among the whole המון multitude ישׂראל of Israel, למאישׁ as men, ועד as well to אשׁה the women לאישׁ to every one חלת cake לחם of bread, אחת a ואשׁפר good piece אחד and a ואשׁישׁה flagon אחת and a וילך departed כל So all העם the people אישׁ every one לביתו׃ to his house.}%
\verse{וישׁב returned דוד Then David לברך to bless את ביתו his household. ותצא came out מיכל And Michal בת the daughter שׁאול of Saul לקראת to meet דוד David, ותאמר and said, מה How נכבד glorious היום today, מלך was the king ישׂראל of Israel אשׁר who נגלה uncovered himself היום today לעיני in the eyes אמהות of the handmaids עבדיו of his servants, כהגלות נגלותחד as one הרקים׃ of the vain fellows}%
\verse{ויאמר said דוד And David אל unto מיכל Michal, לפני before יהוה the LORD, אשׁר which בחר chose בי מאביך me before thy father, ומכל and before all ביתו his house, לצות to appoint אתי נגיד me ruler על over עם the people יהוה of the LORD, על over ישׂראל Israel: ושׂחקתי therefore will I play לפני before יהוה׃ the LORD.}%
\verse{ונקלתי be more vile עוד And I will yet מזאת than thus, והייתי and will be שׁפל base בעיני in mine own sight: ועם and of האמהות the maidservants אשׁר which אמרת thou hast spoken עמם אכבדה׃ of, of them shall I be had in honor.}%
\verse{ולמיכל Therefore Michal בת the daughter שׁאול of Saul לא no היה had לה ילד child עד unto יום the day מותה׃ of her death.}%
\end{biblechapter}%
\begin{biblechapter}% 2 Samuel 7
\verseWithHeading{Adonai Makes a Covenant with David}{ויהי And it came to pass, כי when ישׁב sat המלך the king בביתו in his house, ויהוה and the LORD הניח had given him rest לו מסביב round about מכל from all איביו׃ his enemies;}%
\verse{ויאמר said המלך That the king אל unto נתן Nathan הנביא the prophet, ראה See נא now, אנכי I יושׁב dwell בבית in a house ארזים of cedar, וארון but the ark האלהים of God ישׁב dwelleth בתוך within היריעה׃ curtains.}%
\verse{ויאמר said נתן And Nathan אל to המלך the king, כל all אשׁר that בלבבך in thine heart; לך Go, עשׂה do כי for יהוה the LORD עמך׃ with}%
\verse{ויהי And it came to pass בלילה night, ההוא that ויהי came דבר that the word יהוה of the LORD אל unto נתן Nathan, לאמר׃ saying,}%
\verse{לך Go ואמרת and tell אל and tell עבדי my servant אל דוד David, כה Thus אמר saith יהוה the LORD, האתה Shalt thou תבנה build לי בית me a house לשׁבתי׃ for me to dwell}%
\verse{כי Whereas לא I have not ישׁבתי dwelt בבית in house למיום since the time העלתי that I brought up את בני the children ישׂראל of Israel ממצרים ועד even to היום day, הזה this ואהיה but have מתהלך walked באהל in a tent ובמשׁכן׃ and in a tabernacle.}%
\verse{בכל In all אשׁר wherein התהלכתי I have walked בכל with all בני the children ישׂראל of Israel הדבר I a word דברתי spoke את with אחד any שׁבטי of the tribes ישׂראל of Israel, אשׁר whom צויתי I commanded לרעות to feed את עמי my people את ישׂראל Israel, לאמר saying, למה Why לא ye not בניתם build לי בית me a house ארזים׃ of cedar?}%
\verse{ועתה Now כה therefore תאמר so shalt thou say לעבדי unto my servant לדוד David, כה Thus אמר saith יהוה the LORD צבאות of hosts, אני I לקחתיך took מן thee from הנוה the sheepcote, מאחר from following הצאן the sheep, להיות to be נגיד ruler על over עמי my people, על over ישׂראל׃ Israel:}%
\verse{ואהיה And I was עמך with בכל thee whithersoever אשׁר thee whithersoever הלכת thou wentest, ואכרתה and have cut off את כל all איביך thine enemies מפניך out of thy sight, ועשׂתי and have made לך שׁם name, גדול thee a great כשׁם like unto the name הגדלים of the great אשׁר that בארץ׃ in the earth.}%
\verse{ושׂמתי Moreover I will appoint מקום a place לעמי for my people לישׂראל Israel, ונטעתיו and will plant ושׁכן them, that they may dwell תחתיו in a place ולא no ירגז of their own, and move עוד more; ולא neither יסיפו them any more, בני shall the children עולה of wickedness לענותו afflict כאשׁר as בראשׁונה׃ formerly,}%
\verse{ולמן And as since היום the time אשׁר that צויתי I commanded שׁפטים judges על over עמי my people ישׂראל Israel, והניחתי and have caused thee to rest לך מכל from all איביך thine enemies. והגיד telleth לך יהוה Also the LORD כי thee that בית thee a house. יעשׂה he will make לך יהוה׃}%
\verse{כי And when ימלאו be fulfilled, ימיך thy days ושׁכבת and thou shalt sleep את with אבתיך thy fathers, והקימתי I will set up את זרעך thy seed אחריך after אשׁר thee, which יצא shall proceed ממעיך out of thy bowels, והכינתי and I will establish את ממלכתו׃ his kingdom.}%
\verse{הוא He יבנה shall build בית a house לשׁמי for my name, וכננתי and I will establish את כסא the throne ממלכתו of his kingdom עד forever. עולם׃ forever.}%
\verse{אני I אהיה will be לו לאב his father, והוא and he יהיה shall be לי לבן my son. אשׁר If בהעותו he commit iniquity, והכחתיו I will chasten בשׁבט him with the rod אנשׁים of men, ובנגעי and with the stripes בני of the children אדם׃ of men:}%
\verse{וחסדי But my mercy לא shall not יסור depart away ממנו from כאשׁר him, as הסרתי I took מעם from שׁאול Saul, אשׁר whom הסרתי I put away מלפניך׃ before}%
\verse{ונאמן shall be established ביתך And thine house וממלכתך and thy kingdom עד forever עולם forever לפניך before כסאך thee: thy throne יהיה shall be נכון established עד forever. עולם׃ forever.}%
\verseWithHeading{David Responds to Adonai’s Covenant}{ככל According to all הדברים words, האלה these וככל and according to all החזיון vision, הזה this כן so דבר speak נתן did Nathan אל unto דוד׃ David.}%
\verse{ויבא Then went המלך king דוד David וישׁב in, and sat לפני before יהוה ויאמר and he said, מי Who אנכי I, אדני O Lord יהוה the LORD, ומי and what ביתי my house, כי that הביאתני thou hast brought עד me hitherto? הלם׃ me hitherto?}%
\verse{ותקטן a small עוד was yet זאת And this בעיניך thing in thy sight, אדני O Lord יהוה ותדבר but thou hast spoken גם also אל of בית house עבדך thy servant's למרחוק for a great while to come. וזאת And this תורת the manner האדם of man, אדני O Lord יהוה׃}%
\verse{ומה And what יוסיף say דוד can David עוד more לדבר say אליך unto ואתה thee? for thou, ידעת knowest את עבדך thy servant. אדני Lord יהוה׃ GOD,}%
\verse{בעבור דברךכלבך and according to thine own heart, עשׂית hast thou done את כל all הגדולה great things, הזאת these להודיע know את עבדך׃ to make thy servant}%
\verse{על כןדלת thou art great, אדני God: יהוה כי for אין none כמוך like thee, ואין neither אלהים God זולתך beside בכל thee, according to all אשׁר that שׁמענו we have heard באזנינו׃ with our ears.}%
\verse{ומי And what כעמך like thy people, כישׂראל like Israel, גוי nation אחד one בארץ in the earth אשׁר whom הלכו went אלהים God לפדות to redeem לו לעם for a people ולשׂום to himself, and to make לו שׁם him a name, ולעשׂות and to do לכם הגדולה for you great ונראות things and terrible, לארצך for thy land, מפני before עמך thy people, אשׁר which פדית thou redeemedst לך ממצרים גוים the nations ואלהיו׃ and their gods?}%
\verse{ותכונן For thou hast confirmed לך את עמך to thyself thy people ישׂראל Israel לך לעם a people עד unto thee forever: עולם unto thee forever: ואתה and thou, יהוה LORD, היית art become להם לאלהים׃ their God.}%
\verse{ועתה And now, יהוה O LORD אלהים God, הדבר the word אשׁר that דברת thou hast spoken על concerning עבדך thy servant, ועל and concerning ביתו his house, הקם establish עד forever, עולם , forever, ועשׂה and do כאשׁר as דברת׃ thou hast said.}%
\verse{ויגדל be magnified שׁמך And let thy name עד forever, עולם forever, לאמר saying, יהוה The LORD צבאות of hosts אלהים the God על over ישׂראל Israel: ובית and let the house עבדך of thy servant דוד David יהיה be נכון established לפניך׃ before}%
\verse{כי For אתה thou, יהוה O LORD צבאות of hosts, אלהי God ישׂראל of Israel, גליתה hast revealed את אזן hast revealed עבדך to thy servant, לאמר saying, בית thee a house: אבנה I will build לך על therefore כן therefore מצא found עבדך hath thy servant את לבו in his heart להתפלל to pray אליך unto את התפלה prayer הזאת׃ this}%
\verse{ועתה And now, אדני O Lord יהוה GOD, אתה thou הוא that האלהים God, ודבריך and thy words יהיו be אמת true, ותדבר and thou hast promised אל unto עבדך thy servant: את הטובה goodness הזאת׃ this}%
\verse{ועתה Therefore now הואל let it please וברך thee to bless את בית the house עבדך of thy servant, להיות that it may continue לעולם forever לפניך before כי thee: for אתה thou, אדני O Lord יהוה GOD, דברת hast spoken ומברכתך : and with thy blessing יברך be blessed בית let the house עבדך of thy servant לעולם׃ forever.}%
\end{biblechapter}%
\begin{biblechapter}% 2 Samuel 8
\verseWithHeading{David’s Military Victories Continue}{ויהי it came to pass, אחרי And after כן this ויך smote דוד that David את פלשׁתים the Philistines, ויכניעם and subdued ויקח took דוד them: and David את מתג האמה Metheg-ammah מיד out of the hand פלשׁתים׃ of the Philistines.}%
\verse{ויך And he smote את מואב Moab, וימדדם and measured בחבל them with a line, השׁכב אותםרצה to the ground; וימדד measured שׁני even with two חבלים lines להמית he to put to death, ומלא and with one full החבל line להחיות to keep alive. ותהי became מואב And the Moabites לדוד David's לעבדים servants, נשׂאי brought מנחה׃ gifts.}%
\verse{ויך smote דוד David את הדדעזר also Hadadezer, בן the son רחב of Rehob, מלך king צובה of Zobah, בלכתו as he went להשׁיב to recover ידו his border בנהר־׃ at the river}%
\verse{וילכד took דוד And David ממנו from אלף him a thousand ושׁבע and seven מאות hundred פרשׁים horsemen, ועשׂרים and twenty אלף thousand אישׁ footmen: רגלי footmen: ויעקר hamstrung דוד and David את כל all הרכב the chariot ויותר but reserved ממנו of מאה them a hundred רכב׃ chariots.}%
\verse{ותבא came ארם And when the Syrians דמשׂק of Damascus לעזר to succor להדדעזר Hadadezer מלך king צובה of Zobah, ויך slew דוד David בארם of the Syrians עשׂרים and twenty ושׁנים two אלף thousand אישׁ׃ men.}%
\verse{וישׂם put דוד Then David נצבים garrisons בארם in Syria דמשׂק of Damascus: ותהי became ארם and the Syrians לדוד to David, לעבדים servants נושׂאי brought מנחה gifts. וישׁע preserved יהוה And the LORD את דוד David בכל whithersoever אשׁר whithersoever הלך׃ he went.}%
\verse{ויקח took דוד And David את שׁלטי the shields הזהב of gold אשׁר that היו were אל on עבדי the servants הדדעזר of Hadadezer, ויביאם and brought ירושׁלם׃ them to Jerusalem.}%
\verse{ומבטח ומברתירי cities הדדעזר of Hadadezer, לקח took המלך king דוד David נחשׁת brass. הרבה much מאד׃ exceeding}%
\verse{וישׁמע heard תעי When Toi מלך king חמת of Hamath כי that הכה had smitten דוד David את כל all חיל הדדעזר׃ of Hadadezer,}%
\verse{וישׁלח sent תעי Then Toi את יורם Joram בנו his son אל unto המלך king דוד David, לשׁאל to salute לו לשׁלום to salute ולברכו him, and to bless על him, because אשׁר him, because נלחם he had fought בהדדעזר against Hadadezer, ויכהו and smitten כי him: for אישׁ מלחמות wars תעי with Toi. היה had הדדעזר Hadadezer ובידו with היו And brought כלי him vessels כסף of silver, וכלי and vessels זהב of gold, וכלי and vessels נחשׁת׃ of brass:}%
\verse{גם Which also אתם הקדישׁ did dedicate המלך king דוד David ליהוה unto the LORD, עם with הכסף the silver והזהב and gold אשׁר that הקדישׁ he had dedicated מכל of all הגוים nations אשׁר which כבשׁ׃ he subdued;}%
\verse{מארם וממואבמבני and of the children עמון of Ammon, ומפלשׁתים ומעמלקמשׁלל and of the spoil הדדעזר of Hadadezer, בן son רחב of Rehob, מלך king צובה׃ of Zobah.}%
\verse{ויעשׂ got דוד And David שׁם a name בשׁבו when he returned מהכותו from smiting את ארם of the Syrians בגיא in the valley מלח of salt, שׁמונה eighteen עשׂר eighteen אלף׃ thousand}%
\verse{וישׂם And he put באדום in Edom; נצבים garrisons בכל throughout all אדום Edom שׂם put נצבים he garrisons, ויהי became כל and all אדום they of Edom עבדים servants. לדוד David's ויושׁע preserved יהוה And the LORD את דוד David בכל whithersoever אשׁר whithersoever הלך׃ he went.}%
\verse{וימלך reigned דוד And David על over כל all ישׂראל Israel; ויהי דוד and David עשׂה executed משׁפט judgment וצדקה and justice לכל unto all עמו׃ his people.}%
\verse{ויואב And Joab בן the son צרויה of Zeruiah על over הצבא the host; ויהושׁפט and Jehoshaphat בן the son אחילוד of Ahilud מזכיר׃ recorder;}%
\verse{וצדוק And Zadok בן the son אחיטוב of Ahitub, ואחימלך and Ahimelech בן the son אביתר of Abiathar, כהנים the priests; ושׂריה and Seraiah סופר׃ the scribe;}%
\verse{ובניהו And Benaiah בן the son יהוידע of Jehoiada והכרתי both the Cherethites והפלתי and the Pelethites; ובני sons דוד and David's כהנים chief rulers. היו׃ were}%
\end{biblechapter}%
\begin{biblechapter}% 2 Samuel 9
\verseWithHeading{David Cares for Mephibosheth}{ויאמר said, דוד And David הכי ישׁ Is there עוד yet אשׁר any נותר that is left לבית of the house שׁאול of Saul, ואעשׂה עמוסד him kindness בעבור sake? יהונתן׃}%
\verse{ולבית And of the house שׁאול of Saul עבד a servant ושׁמו whose name ציבא Ziba. ויקראו And when they had called לו אל him unto דוד David, ויאמר said המלך the king אליו unto האתה him, thou ציבא Ziba? ויאמר And he said, עבדך׃ Thy servant}%
\verse{ויאמר said, המלך And the king האפס there not עוד yet אישׁ any לבית of the house שׁאול of Saul, ואעשׂה that I may show עמו unto חסד the kindness אלהים of God ויאמר said ציבא him? And Ziba אל unto המלך the king, עוד hath yet בן a son, ליהונתן Jonathan נכה lame רגלים׃ on feet.}%
\verse{ויאמר said לו המלך And the king איפה unto him, Where הוא he? ויאמר said ציבא And Ziba אל unto המלך the king, הנה Behold, הוא he בית in the house מכיר of Machir, בן the son עמיאל of Ammiel, בלו דבר׃ in Lo-debar.}%
\verse{וישׁלח sent, המלך Then king דוד David ויקחהו and fetched מבית him out of the house מכיר of Machir, בן the son עמיאל of Ammiel, מלו דבר׃}%
\verse{ויבא was come מפיבשׁת Now when Mephibosheth, בן the son יהונתן of Jonathan, בן the son שׁאול of Saul, אל unto דוד David, ויפל he fell על on פניו his face, וישׁתחו and did reverence. ויאמר said, דוד And David מפיבשׁת Mephibosheth. ויאמר And he answered, הנה Behold עבדך׃ thy servant!}%
\verse{ויאמר said לו דוד And David אל not: תירא unto him, Fear כי for עשׂה אעשׂהמך חסד thee kindness בעבור sake, יהונתן for Jonathan אביך thy father's והשׁבתי and will restore לך את כל thee all שׂדה the land שׁאול of Saul אביך thy father; ואתה and thou תאכל shalt eat לחם bread על at שׁלחני my table תמיד׃ continually.}%
\verse{וישׁתחו And he bowed himself, ויאמר and said, מה What עבדך thy servant, כי that פנית thou shouldest look אל upon הכלב dog המת a dead אשׁר such כמוני׃ as I}%
\verse{ויקרא called המלך Then the king אל to ציבא Ziba, נער servant, שׁאול Saul's ויאמר and said אליו unto כל all אשׁר that היה pertained לשׁאול to Saul ולכל and to all ביתו his house. נתתי him, I have given לבן son אדניך׃ unto thy master's}%
\verse{ועבדת shall till לו את האדמה the land אתה Thou ובניך therefore, and thy sons, ועבדיך and thy servants, והבאת for him, and thou shalt bring והיה may have לבן son אדניך in that thy master's לחם food ואכלו to eat: ומפיבשׁת but Mephibosheth בן son אדניך thy master's יאכל shall eat תמיד always לחם bread על at שׁלחני my table. ולציבא Now Ziba חמשׁה had fifteen עשׂר had fifteen בנים sons ועשׂרים and twenty עבדים׃ servants.}%
\verse{ויאמר Then said ציבא Ziba אל unto המלך the king, ככל According to all אשׁר that יצוה hath commanded אדני my lord המלך the king את עבדו his servant, כן so יעשׂה do. עבדך shall thy servant ומפיבשׁת As for Mephibosheth, אכל he shall eat על at שׁלחני my table, כאחד as one מבני המלך׃}%
\verse{ולמפיבשׁת And Mephibosheth בן son, קטן had a young ושׁמו whose name מיכא Micha. וכל And all מושׁב that dwelt בית in the house ציבא of Ziba עבדים servants למפיבשׁת׃ unto Mephibosheth.}%
\verse{ומפיבשׁת So Mephibosheth ישׁב dwelt בירושׁלם in Jerusalem: כי for על at שׁלחן table; המלך the king's תמיד continually הוא he אכל did eat והוא פסח and was lame שׁתי on both רגליו׃ his feet.}%
\end{biblechapter}%
\begin{biblechapter}% 2 Samuel 10
\verseWithHeading{The Ammonites Refuse David’s Loyal Love}{ויהי And it came to pass אחרי after כן this, וימת died, מלך that the king בני of the children עמון of Ammon וימלך reigned חנון and Hanun בנו his son תחתיו׃ in his stead.}%
\verse{ויאמר Then said דוד David, אעשׂה I will show חסד kindness עם unto חנון Hanun בן the son נחשׁ of Nahash, כאשׁר as עשׂה showed אביו his father עמדי חסד kindness וישׁלח sent דוד me. And David לנחמו to comfort ביד him by the hand עבדיו of his servants אל for אביו his father. ויבאו came עבדי servants דוד And David's ארץ into the land בני of the children עמון׃ of Ammon.}%
\verse{ויאמרו said שׂרי And the princes בני of the children עמון of Ammon אל unto חנון Hanun אדניהם their lord, המכבד doth honor דוד thou that David את אביך thy father, בעיניך Thinkest כי that שׁלח he hath sent לך מנחמים comforters הלוא unto thee? hath not בעבור thee, to חקור search את העיר the city, ולרגלה and to spy it out, ולהפכה and to overthrow שׁלח sent דוד David את עבדיו his servants אליך׃ unto}%
\verse{ויקח took חנון Wherefore Hanun את עבדי servants, דוד David's ויגלח and shaved off את חצי the one half זקנם of their beards, ויכרת and cut off את מדויהם their garments בחצי in the middle, עד to שׁתותיהם their buttocks, וישׁלחם׃ and sent them away.}%
\verse{ויגדו When they told לדוד unto David, וישׁלח he sent לקראתם to meet כי them, because היו were האנשׁים נכלמים ashamed: מאד greatly ויאמר said, המלך and the king שׁבו Tarry בירחו at Jericho עד until יצמח be grown, זקנכם your beards ושׁבתם׃ and return.}%
\verseWithHeading{Israel Fights Ammon and Aram}{ויראו saw בני And when the children עמון of Ammon כי that נבאשׁו they stank בדוד before David, וישׁלחו sent בני the children עמון of Ammon וישׂכרו and hired את ארם the Syrians בית רחוב of Beth-rehob, ואת ארם and the Syrians צובא of Zoba, עשׂרים twenty אלף thousand רגלי footmen, ואת מלך and of king מעכה Maacah אלף a thousand אישׁ men, ואישׁ men. טוב שׁנים twelve עשׂר twelve אלף אישׁ׃ thousand}%
\verse{וישׁמע heard דוד And when David וישׁלח of he sent את יואב Joab, ואת כל and all הצבא the host הגברים׃ of the mighty men.}%
\verse{ויצאו came out, בני And the children עמון of Ammon ויערכו in array מלחמה and put the battle פתח at the entering השׁער in of the gate: וארם and the Syrians צובא of Zoba, ורחוב and of Rehob, ואישׁ טובמעכה and Maacah, לבדם by themselves בשׂדה׃ in the field.}%
\verse{וירא saw יואב When Joab כי that היתה was אליו against פני the front המלחמה of the battle מפנים him before ומאחור and behind, ויבחר he chose מכל of all בחורי בישׂראל of Israel, ויערך and put in array לקראת against ארם׃ the Syrians:}%
\verse{ואת יתר And the rest העם of the people נתן he delivered ביד into the hand אבשׁי of Abishai אחיו his brother, ויערך that he might put in array לקראת against בני the children עמון׃ of Ammon.}%
\verse{ויאמר And he said, אם If תחזק be too strong ארם the Syrians ממני for והיתה me, then thou shalt לי לישׁועה and help ואם me: but if בני the children עמון of Ammon יחזקו be too strong ממך for והלכתי thee, then I will come להושׁיע׃}%
\verse{חזק Be of good courage, ונתחזק and let us play the men בעד for עמנו our people, ובעד and for ערי the cities אלהינו of our God: ויהוה and the LORD יעשׂה do הטוב him good. בעיניו׃ that which seemeth}%
\verse{ויגשׁ drew nigh, יואב And Joab והעם and the people אשׁר that עמו with למלחמה him, unto the battle בארם against the Syrians: וינסו and they fled מפניו׃ before}%
\verse{ובני And when the children עמון of Ammon ראו saw כי that נס were fled, ארם the Syrians וינסו then fled מפני they also before אבישׁי Abishai, ויבאו and entered העיר into the city. וישׁב returned יואב So Joab מעל from בני the children עמון of Ammon, ויבא and came ירושׁלם׃ to Jerusalem.}%
\verseWithHeading{The Arameans Regroup for Attack}{וירא saw ארם And when the Syrians כי that נגף they were smitten לפני before ישׂראל Israel, ויאספו they gathered themselves יחד׃ together.}%
\verse{וישׁלח sent, הדדעזר ויצא and brought out את ארם the Syrians אשׁר that מעבר beyond הנהר the river: ויבאו and they came חילם to Helam; ושׁובך and Shobach שׂר the captain צבא of the host הדדעזר לפניהם׃ before}%
\verse{ויגד And when it was told לדוד David, ויאסף אתל ישׂראליעבר and passed over את הירדן Jordan, ויבא and came חלאמה to Helam. ויערכו set themselves in array ארם And the Syrians לקראת against דוד David, וילחמו and fought עמו׃ with}%
\verse{וינס fled ארם And the Syrians מפני before ישׂראל Israel; ויהרג slew דוד and David מארם שׁבע seven מאות hundred רכב chariots וארבעים and forty אלף thousand פרשׁים horsemen, ואת שׁובך Shobach שׂר the captain צבאו of their host, הכה and smote וימת who died שׁם׃ there.}%
\verse{ויראו saw כל And when all המלכים the kings עבדי servants הדדעזר כי that נגפו they were smitten לפני before ישׂראל Israel, וישׁלמו they made peace את with ישׂראל Israel, ויעבדום and served ויראו feared ארם them. So the Syrians להושׁיע to help עוד any more. את בני the children עמון׃ of Ammon}%
\end{biblechapter}%
\begin{biblechapter}% 2 Samuel 11
\verseWithHeading{David Commits Adultery with Bathsheba}{ויהי And it came to pass, לתשׁובת was expired, השׁנה after the year לעת at the time צאת go forth המלאכים when kings וישׁלח sent דוד that David את יואב Joab, ואת עבדיו and his servants עמו with ואת כל him, and all ישׂראל Israel; וישׁחתו and they destroyed את בני the children עמון of Ammon, ויצרו and besieged על and besieged רבה Rabbah. ודוד But David יושׁב tarried בירושׁלם׃ still at Jerusalem.}%
\verse{ויהי And it came to pass לעת in an eveningtide, הערב in an eveningtide, ויקם arose דוד that David מעל from off משׁכבו his bed, ויתהלך and walked על upon גג the roof בית house: המלך of the king's וירא he saw אשׁה a woman רחצת washing מעל and from הגג the roof והאשׁה herself; and the woman טובת beautiful מראה to look מאד׃ very}%
\verse{וישׁלח sent דוד And David וידרשׁ and inquired לאשׁה after the woman. ויאמר And said, הלוא not זאת this בת־שׁבע Bath-sheba, בת the daughter אליעם of Eliam, אשׁת the wife אוריה of Uriah החתי׃ the Hittite?}%
\verse{וישׁלח sent דוד And David מלאכים messengers, ויקחה and took ותבוא her; and she came אליו in unto וישׁכב him, and he lay עמה with her; והיא for she מתקדשׁת was purified מטמאתה from her uncleanness: ותשׁב and she returned אל unto ביתה׃ her house.}%
\verse{ותהר conceived, האשׁה And the woman ותשׁלח and sent ותגד and told לדוד David, ותאמר and said, הרה with child. אנכי׃ I}%
\verse{וישׁלח sent דוד And David אל to יואב Joab, שׁלח sent אלי to את אוריה me Uriah החתי the Hittite. וישׁלח יואב And Joab את אוריה Uriah אל דוד׃ David.}%
\verse{ויבא was come אוריה And when Uriah אליו unto וישׁאל demanded דוד him, David לשׁלום did, יואב how Joab ולשׁלום did, העם and how the people ולשׁלום prospered. המלחמה׃ and how the war}%
\verse{ויאמר said דוד And David לאוריה to Uriah, רד Go down לביתך to thy house, ורחץ and wash רגליך thy feet. ויצא departed אוריה And Uriah מבית המלך from the king. ותצא and there followed אחריו and there followed משׂאת him a mess המלך׃}%
\verse{וישׁכב slept אוריה But Uriah פתח at the door בית house המלך of the king's את with כל all עבדי the servants אדניו of his lord, ולא not ירד and went אל down to ביתו׃ his house.}%
\verse{ויגדו And when they had told לדוד David, לאמר saying, לא went not down ירד went not down אוריה Uriah אל unto ביתו his house, ויאמר said דוד David אל unto אוריה Uriah, הלוא not מדרך from journey? אתה thou בא Camest מדוע why לא didst thou not ירדת go down אל unto ביתך׃ thine house?}%
\verse{ויאמר said אוריה And Uriah אל unto דוד David, הארון The ark, וישׂראל and Israel, ויהודה and Judah, ישׁבים abide בסכות in tents; ואדני and my lord יואב Joab, ועבדי and the servants אדני of my lord, על in פני the open השׂדה fields; חנים are encamped ואני shall I אבוא then go אל into ביתי mine house, לאכל to eat ולשׁתות and to drink, ולשׁכב and to lie עם with אשׁתי my wife? חיך thou livest, וחי liveth, נפשׁך and thy soul אם אעשׂה I will not do את הדבר thing. הזה׃ this}%
\verse{ויאמר said דוד And David אל to אוריה Uriah, שׁב Tarry בזה here גם also, היום today ומחר and tomorrow אשׁלחך I will let thee depart. וישׁב abode אוריה So Uriah בירושׁלם in Jerusalem ביום day, ההוא that וממחרת׃ and the morrow.}%
\verse{ויקרא had called לו דוד And when David ויאכל him, he did eat לפניו before וישׁת and drink וישׁכרהו him; and he made him drunk: ויצא he went out בערב and at even לשׁכב to lie במשׁכבו on his bed עם with עבדי the servants אדניו of his lord, ואל down to ביתו his house. לא not ירד׃ but went}%
\verse{ויהי And it came to pass בבקר in the morning, ויכתב wrote דוד that David ספר a letter אל to יואב Joab, וישׁלח and sent ביד by the hand אוריה׃ of Uriah.}%
\verse{ויכתב And he wrote בספר in the letter, לאמר saying, הבו Set את אוריה ye Uriah אל in מול the forefront פני the forefront המלחמה battle, החזקה of the hottest ושׁבתם and retire מאחריו ye from ונכה him, that he may be smitten, ומת׃ and die.}%
\verse{ויהי And it came to pass, בשׁמור observed יואב when Joab אל observed העיר the city, ויתן that he assigned את אוריה Uriah אל unto המקום a place אשׁר where ידע he knew כי that אנשׁי men חיל valiant שׁם׃ where}%
\verse{ויצאו went out, אנשׁי העיר of the city וילחמו and fought את with יואב Joab: ויפל and there fell מן of העם the people מעבדי of the servants דוד of David; וימת died גם also. אוריה and Uriah החתי׃ the Hittite}%
\verse{וישׁלח sent יואב Then Joab ויגד and told לדוד David את כל all דברי the things המלחמה׃ concerning the war;}%
\verse{ויצו And charged את המלאך the messenger, לאמר saying, ככלותך When thou hast made an end את כלברי the matters המלחמה of the war לדבר of telling אל unto המלך׃ the king,}%
\verse{והיה so be אם And if תעלה arise, חמת wrath המלך that the king's ואמר and he say לך מדוע unto thee, Wherefore נגשׁתם approached אל ye so nigh unto העיר the city להלחם when ye did fight? הלוא ye not ידעתם knew את אשׁר that ירו they would shoot מעל from החומה׃ the wall?}%
\verse{מי Who הכה smote את אבימלך Abimelech בן the son ירבשׁת of Jerubbesheth? הלוא did not אשׁה a woman השׁליכה cast עליו upon פלח a piece רכב of a millstone מעל him from החומה the wall, וימת that he died בתבץ in Thebez? למה why נגשׁתם went ye nigh אל went ye nigh החומה the wall? ואמרת then say גם also. עבדך thou, Thy servant אוריה Uriah החתי the Hittite מת׃ is dead}%
\verse{וילך went, המלאך So the messenger ויבא and came ויגד and showed לדוד David את כל all אשׁר that שׁלחו had sent יואב׃ Joab}%
\verse{ויאמר said המלאך And the messenger אל unto דוד David, כי Surely גברו prevailed עלינו against האנשׁים the men ויצאו us, and came out אלינו unto השׂדה us into the field, ונהיה and we were עליהם upon עד them even unto פתח the entering השׁער׃ of the gate.}%
\verse{ויראו המוראיםל upon עבדך thy servants; מעל from off החומה the wall וימותו be dead, מעבדי and thy servant המלך וגם also. עבדך אוריה Uriah החתי the Hittite מת׃ is dead}%
\verse{ויאמר said דוד Then David אל unto המלאך the messenger, כה Thus תאמר shalt thou say אל unto יואב Joab, אל Let not ירע בעיניך displease את הדבר thing הזה this כי thee, for כזה as well as another: וכזה one תאכל devoureth החרב the sword החזק more strong מלחמתך make thy battle אל against העיר the city, והרסה and overthrow וחזקהו׃ it: and encourage}%
\verse{ותשׁמע heard אשׁת And when the wife אוריה of Uriah כי that מת was dead, אוריה Uriah אישׁה her husband ותספד she mourned על for בעלה׃ her husband.}%
\verse{ויעבר was past, האבל And when the mourning וישׁלח sent דוד David ויאספה and fetched אל her to ביתו his house, ותהי and she became לו לאשׁה his wife, ותלד and bore לו בן him a son. וירע הדבר But the thing אשׁר that עשׂה had done דוד David בעיני displeased יהוה׃ the LORD.}%
\end{biblechapter}%
\begin{biblechapter}% 2 Samuel 12
\verseWithHeading{Nathan Reproves David}{וישׁלח sent יהוה And the LORD את נתן Nathan אל unto דוד David. ויבא And he came אליו unto ויאמר him, and said לו שׁני two אנשׁים היו unto him, There were בעיר city; אחת in one אחד the one עשׁיר rich, ואחד and the other ראשׁ׃ poor.}%
\verse{לעשׁיר The rich היה had צאן flocks ובקר and herds: הרבה many מאד׃ exceeding}%
\verse{ולרשׁ But the poor אין had nothing, כל had nothing, כי save אם save כבשׂה ewe lamb, אחת one קטנה little אשׁר which קנה bought ויחיה and nourished up: ותגדל and it grew up עמו with ועם him, and with בניו his children; יחדו together מפתו of his own meat, תאכל it did eat ומכסו of his own cup, תשׁתה and drank ובחיקו in his bosom, תשׁכב and lay ותהי he had לו כבת׃ unto him as a daughter.}%
\verse{ויבא And there came הלך a traveler לאישׁ man, העשׁיר unto the rich ויחמל and he spared לקחת to take מצאנו of his own flock ומבקרו and of his own herd, לעשׂות to dress לארח for the wayfaring man הבא that was come לו ויקח unto him; but took את כבשׂת lamb, האישׁ man's הראשׁ the poor ויעשׂה and dressed לאישׁ it for the man הבא that was come אליו׃ to}%
\verse{ויחר kindled אף anger דוד And David's באישׁ against the man; מאד was greatly ויאמר and he said אל to נתן Nathan, חי liveth, יהוה the LORD כי shall surely die: בן מותאישׁ the man העשׂה that hath done זאת׃ this}%
\verse{ואת הכבשׂה the lamb ישׁלם And he shall restore ארבעתים fourfold, עקב because אשׁר because עשׂה he did את הדבר thing, הזה this ועל and because אשׁר and because לא he had no חמל׃ pity.}%
\verse{ויאמר said נתן And Nathan אל to דוד David, אתה Thou האישׁ the man. כה Thus אמר saith יהוה the LORD אלהי God ישׂראל of Israel, אנכי I משׁחתיך anointed למלך thee king על over ישׂראל Israel, ואנכי and I הצלתיך delivered מיד thee out of the hand שׁאול׃ of Saul;}%
\verse{ואתנה And I gave לך את בית house, אדניך thee thy master's ואת נשׁי wives אדניך and thy master's בחיקך into thy bosom, ואתנה and gave לך את בית thee the house ישׂראל of Israel ויהודה and of Judah; ואם and if מעט too little, ואספה I would moreover have given לך כהנה unto thee such וכהנה׃ and such}%
\verse{מדוע Wherefore בזית hast thou despised את דבר the commandment יהוה of the LORD, לעשׂות to do הרע evil בעינו in his sight? את אוריה Uriah החתי the Hittite הכית thou hast killed בחרב with the sword, ואת אשׁתו his wife לקחת and hast taken לך לאשׁה thy wife, ואתו הרגת and hast slain בחרב him with the sword בני of the children עמון׃ of Ammon.}%
\verse{ועתה Now לא shall never תסור depart חרב therefore the sword מביתך from thine house; עד עולםקב because כי because בזתני thou hast despised ותקח me, and hast taken את אשׁת the wife אוריה of Uriah החתי the Hittite להיות to be לך לאשׁה׃ thy wife.}%
\verse{כה Thus אמר saith יהוה the LORD, הנני מקים I will raise up עליך against רעה evil מביתך thee out of thine own house, ולקחתי and I will take את נשׁיך thy wives לעיניך before thine eyes, ונתתי and give לרעיך unto thy neighbor, ושׁכב and he shall lie עם with נשׁיך thy wives לעיני in the sight השׁמשׁ sun. הזאת׃ of this}%
\verse{כי For אתה thou עשׂית didst בסתר secretly: ואני but I אעשׂה will do את הדבר thing הזה this נגד before כל all ישׂראל Israel, ונגד and before השׁמשׁ׃ the sun.}%
\verseWithHeading{David Repents, But the Child Dies}{ויאמר said דוד And David אל unto נתן Nathan, חטאתי I have sinned ליהוה against the LORD. ויאמר said נתן And Nathan אל unto דוד David, גם also יהוה The LORD העביר hath put away חטאתך thy sin; לא thou shalt not תמות׃ die.}%
\verse{אפס Howbeit, כי because נאץ thou hast given great occasion נאצת to blaspheme, את איבי to the enemies יהוה of the LORD בדבר deed הזה by this גם also הבן the child הילוד born לך מות unto thee shall surely die. ימות׃ unto thee shall surely die.}%
\verse{וילך departed נתן And Nathan אל unto ביתו his house. ויגף struck יהוה And the LORD את הילד the child אשׁר that ילדה bore אשׁת wife אוריה Uriah's לדוד unto David, ויאנשׁ׃ and it was very sick.}%
\verse{ויבקשׁ therefore besought דוד David את האלהים God בעד for הנער the child; ויצם fasted, דוד and David צום fasted, ובא and went ולן night ושׁכב in, and lay ארצה׃ upon the earth.}%
\verse{ויקמו arose, זקני And the elders ביתו of his house עליו to להקימו him, to raise him up מן from הארץ the earth: ולא not, אבה but he would ולא neither ברא אתם with לחם׃ bread}%
\verse{ויהי And it came to pass ביום day, השׁביעי on the seventh וימת died. הילד that the child ויראו feared עבדי And the servants דוד of David להגיד to tell לו כי him that מת was dead: הילד the child כי for אמרו they said, הנה Behold, בהיות was הילד while the child חי yet alive, דברנו we spoke אליו unto ולא him, and he would not שׁמע hearken בקולנו unto our voice: ואיך how נאמר himself, if we tell אליו himself, if we tell מת is dead? הילד him that the child ועשׂה will he then vex רעה׃ will he then vex}%
\verse{וירא saw דוד But when David כי that עבדיו his servants מתלחשׁים whispered, ויבן perceived דוד David כי that מת was dead: הילד the child ויאמר said דוד therefore David אל unto עבדיו his servants, המת dead? הילד Is the child ויאמרו And they said, מת׃ He is dead.}%
\verse{ויקם arose דוד Then David מהארץ from the earth, וירחץ and washed, ויסך and anointed ויחלף and changed שׂמלתו his apparel, ויבא and came בית into the house יהוה of the LORD, וישׁתחו and worshiped: ויבא then he came אל to ביתו his own house; וישׁאל and when he required, וישׂימו they set לו לחם bread ויאכל׃ before him, and he did eat.}%
\verse{ויאמרו Then said עבדיו his servants אליו unto מה him, What הדבר thing הזה this אשׁר that עשׂיתה thou hast done? בעבור while הילד for the child, חי alive; צמת thou didst fast ותבך and weep וכאשׁר but when מת was dead, הילד the child קמת thou didst rise ותאכל and eat לחם׃ bread.}%
\verse{ויאמר And he said, בעוד was yet הילד While the child חי alive, צמתי I fasted ואבכה and wept: כי for אמרתי I said, מי Who יודע can tell יחנני will be gracious יהוה GOD וחי may live? הילד׃ to me, that the child}%
\verse{ועתה But now מת he is dead, למה wherefore זה wherefore אני should I צם fast? האוכל can להשׁיבו I bring him back עוד again? אני I הלך shall go אליו to והוא him, but he לא shall not ישׁוב return אלי׃ to}%
\verse{וינחם comforted דוד And David את בת־שׁבע Bath-sheba אשׁתו his wife, ויבא and went אליה in unto וישׁכב her, and lay עמה with ותלד her: and she bore בן a son, ויקרא and he called את שׁמו his name שׁלמה Solomon: ויהוה and the LORD אהבו׃ loved}%
\verse{וישׁלח And he sent ביד by the hand נתן of Nathan הנביא the prophet; ויקרא and he called את שׁמו his name ידידיה Jedidiah, בעבור because יהוה׃ of the LORD.}%
\verseWithHeading{Battle with the Ammonites}{וילחם fought יואב And Joab ברבת against Rabbah בני of the children עמון of Ammon, וילכד and took את עיר city. המלוכה׃ the royal}%
\verse{וישׁלח sent יואב And Joab מלאכים messengers אל to דוד David, ויאמר and said, נלחמתי I have fought ברבה against Rabbah, גם and לכדתי have taken את עיר the city המים׃ of waters.}%
\verse{ועתה Now אסף therefore gather את יתר the rest העם of the people וחנה together, and encamp על against העיר the city, ולכדה and take פן it: lest אלכד take אני I את העיר the city, ונקרא and it be called שׁמי my name. עליה׃ after}%
\verse{ויאסף gathered דוד And David את כל all העם the people וילך together, and went רבתה to Rabbah, וילחם and fought בה וילכדה׃ against it, and took}%
\verse{ויקח And he took את עטרת crown מלכם their king's מעל from off ראשׁו his head, ומשׁקלה the weight ככר whereof a talent זהב of gold ואבן stones: יקרה with the precious ותהי and it was על on ראשׁ head. דוד David's ושׁלל the spoil העיר of the city הוציא And he brought forth הרבה abundance. מאד׃ in great}%
\verse{ואת העם the people אשׁר that בה הוציא And he brought forth וישׂם therein, and put במגרה under saws, ובחרצי and under harrows הברזל of iron, ובמגזרת and under axes הברזל of iron, והעביר אותםמלכן the brickkiln: וכן and thus יעשׂה did לכל he unto all ערי the cities בני of the children עמון of Ammon. וישׁב returned דוד So David וכל and all העם the people ירושׁלם׃ unto Jerusalem.}%
\end{biblechapter}%
\begin{biblechapter}% 2 Samuel 13
\verseWithHeading{Amnon Assaults His Sister Tamar}{ויהי And it came to pass אחרי after כן this, ולאבשׁלום that Absalom בן the son דוד of David אחות sister, יפה had a fair ושׁמה whose name תמר Tamar; ויאהבה loved אמנון and Amnon בן the son דוד׃ of David}%
\verse{ויצר לאמנון And Amnon להתחלות that he fell sick בעבור for תמר Tamar; אחתו his sister כי for בתולה a virgin; היא she ויפלא it hard בעיני thought אמנון and Amnon לעשׂות for him to do לה מאומה׃ any thing}%
\verse{ולאמנון But Amnon רע had a friend, ושׁמו whose name יונדב Jonadab, בן the son שׁמעה of Shimeah אחי brother: דוד David's ויונדב and Jonadab אישׁ man. חכם subtle מאד׃ a very}%
\verse{ויאמר And he said לו מדוע unto him, Why אתה thou, ככה lean דל lean בן son, המלך the king's בבקר from day בבקר to day? הלוא wilt thou not תגיד tell לי ויאמר said לו אמנון me? And Amnon את תמר Tamar, אחות sister. אבשׁלם Absalom's אחי my brother אני unto him, I אהב׃ love}%
\verse{ויאמר said לו יהונדב שׁכב unto him, Lay thee down על on משׁכבך thy bed, והתחל and make thyself sick: ובא cometh אביך and when thy father לראותך to see ואמרת thee, say אליו unto תבא come, נא him, I pray thee, תמר Tamar אחותי let my sister ותברני and give לחם me meat, ועשׂתה and dress לעיני in my sight, את הבריה the meat למען that אשׁר that אראה I may see ואכלתי and eat מידה׃ at her hand.}%
\verse{וישׁכב lay down, אמנון So Amnon ויתחל and made himself sick: ויבא was come המלך and when the king לראתו to see ויאמר said אמנון him, Amnon אל unto המלך the king, תבוא come, נא I pray thee, תמר let Tamar אחתי my sister ותלבב and make לעיני in my sight, שׁתי me a couple לבבות of cakes ואברה that I may eat מידה׃ at her hand.}%
\verse{וישׁלח sent דוד Then David אל to תמר Tamar, הביתה home לאמר saying, לכי Go נא now בית house, אמנון Amnon's אחיך to thy brother ועשׂי and dress לו הבריה׃ him meat.}%
\verse{ותלך went תמר So Tamar בית house; אמנון Amnon's אחיה to her brother והוא and he שׁכב was laid down. ותקח And she took את הבצק flour, ותלושׁ and kneaded ותלבב and made cakes לעיניו in his sight, ותבשׁל and did bake את הלבבות׃ the cakes.}%
\verse{ותקח And she took את המשׂרת a pan, ותצק and poured out לפניו before וימאן him; but he refused לאכול to eat. ויאמר said, אמנון And Amnon הוציאו Have out כל all אישׁ men מעלי from ויצאו me. And they went out כל every אישׁ man מעליו׃ from}%
\verse{ויאמר said אמנון And Amnon אל unto תמר Tamar, הביאי Bring הבריה the meat החדר into the chamber, ואברה that I may eat מידך of thine hand. ותקח took תמר And Tamar את הלבבות the cakes אשׁר which עשׂתה she had made, ותבא and brought לאמנון to Amnon אחיה her brother. החדרה׃ into the chamber}%
\verse{ותגשׁ And when she had brought אליו unto לאכל him to eat, ויחזק he took hold בה ויאמר of her, and said לה בואי unto her, Come שׁכבי lie עמי with אחותי׃ me, my sister.}%
\verse{ותאמר And she answered לו אל him, Nay, אחי my brother, אל do not תענני force כי me; for לא no יעשׂה ought to be done כן such thing בישׂראל in Israel: אל not תעשׂה do את הנבלה folly. הזאת׃ thou this}%
\verse{ואני And I, אנה whither אוליך to go? את חרפתי shall I cause my shame ואתה and as for thee, תהיה thou shalt be כאחד as one הנבלים of the fools בישׂראל in Israel. ועתה Now דבר speak נא therefore, I pray thee, אל unto המלך the king; כי for לא he will not ימנעני withhold ממך׃ me from}%
\verse{ולא not אבה Howbeit he would לשׁמע hearken בקולה unto her voice: ויחזק but, being stronger ממנה ויענה than she, forced וישׁכב her, and lay with אתה׃}%
\verse{וישׂנאה hated אמנון Then Amnon שׂנאה her exceedingly; גדולה her greater מאד כי so that גדולה השׂנאה the hatred אשׁר wherewith שׂנאה he hated מאהבה than the love אשׁר wherewith אהבה he had loved ויאמר said לה אמנון her. And Amnon קומי unto her, Arise, לכי׃ be gone.}%
\verse{ותאמר And she said לו אל unto him, no אודת cause: הרעה evil הגדולה greater הזאת this מאחרת than the other אשׁר that עשׂית thou didst עמי unto לשׁלחני in sending me away ולא not אבה me. But he would לשׁמע׃ hearken}%
\verse{ויקרא Then he called את נערו his servant משׁרתו that ministered ויאמר unto him, and said, שׁלחו Put נא now את זאת this מעלי החוצהנעל and bolt הדלת the door אחריה׃ after}%
\verse{ועליה upon כתנת And a garment פסים of divers colors כי her: for כן with such תלבשׁן appareled. בנות daughters המלך were the king's הבתולת virgins מעילים robes ויצא brought אותה משׁרתו Then his servant החוץ her out, ונעל and bolted הדלת the door אחריה׃ after}%
\verse{ותקח put תמר And Tamar אפר ashes על on ראשׁה her head, וכתנת her garment הפסים of divers colors אשׁר that עליה on קרעה and rent ותשׂם her, and laid ידה her hand על on ראשׁה her head, ותלך and went on הלוך and went on וזעקה׃ crying.}%
\verse{ויאמר said אליה unto אבשׁלום And Absalom אחיה her brother האמינון her, Hath Amnon אחיך thy brother היה been עמך with ועתה now אחותי thy peace, my sister: החרישׁי thee? but hold אחיך thy brother; הוא he אל not תשׁיתי אתבך לדבר thing. הזה this ותשׁב remained תמר So Tamar ושׁממה בית house. אבשׁלום Absalom's אחיה׃ in her brother}%
\verseWithHeading{Absalom Kills Amnon}{והמלך But when king דוד David שׁמע heard את כל of all הדברים things, האלה these ויחר wroth. לו מאד׃ he was very}%
\verse{ולא neither דבר spoke אבשׁלום And Absalom עם unto אמנון his brother Amnon למרע bad: ועד nor טוב good כי for שׂנא hated אבשׁלום Absalom את אמנון Amnon, על because דבר אשׁרנה he had forced את תמר Tamar. אחתו׃ his sister}%
\verse{ויהי And it came to pass לשׁנתים after two full years, ימים after two full years, ויהיו had גזזים sheepshearers לאבשׁלום that Absalom בבעל חצור in Baal-hazor, אשׁר which עם beside אפרים Ephraim: ויקרא invited אבשׁלום and Absalom לכל all בני sons. המלך׃ the king's}%
\verse{ויבא came אבשׁלום And Absalom אל to המלך the king, ויאמר and said, הנה Behold נא now, גזזים hath sheepshearers; לעבדך thy servant ילך go נא I beseech thee, המלך let the king, ועבדיו and his servants עם with עבדך׃ thy servant.}%
\verse{ויאמר said המלך And the king אל to אבשׁלום Absalom, אל Nay, בני my son, אל let us not נא now נלך go, כלנו all ולא lest נכבד we be chargeable עליך unto ויפרץ thee. And he pressed בו ולא not אבה him: howbeit he would ללכת go, ויברכהו׃ but blessed}%
\verse{ויאמר Then said אבשׁלום Absalom, ולא If not, ילך go נא I pray thee, אתנו with אמנון Amnon אחי let my brother ויאמר said לו המלך us. And the king למה unto him, Why ילך should he go עמך׃ with}%
\verse{ויפרץ pressed בו אבשׁלום But Absalom וישׁלח go אתו with את אמנון him, that he let Amnon ואת כל and all בני sons המלך׃ the king's}%
\verse{ויצו had commanded אבשׁלום Now Absalom את נעריו his servants, לאמר saying, ראו Mark נא ye now כטוב לב heart אמנון when Amnon's ביין with wine, ואמרתי and when I say אליכם unto הכו you, Smite את אמנון Amnon; והמתם then kill אתו אל not: תיראו him, fear הלוא have not כי אנכי I צויתי commanded אתכם חזקו you? be courageous, והיו and be לבני valiant. חיל׃ valiant.}%
\verse{ויעשׂו did נערי And the servants אבשׁלום of Absalom לאמנון unto Amnon כאשׁר as צוה had commanded. אבשׁלום Absalom ויקמו arose, כל Then all בני sons המלך the king's וירכבו got him up אישׁ and every man על upon פרדו his mule, וינסו׃ and fled.}%
\verse{ויהי And it came to pass, המה while they בדרך were in the way, והשׁמעה that tidings באה came אל to דוד David, לאמר saying, הכה hath slain אבשׁלום Absalom את כל all בני sons, המלך the king's ולא and there is not נותר them left. מהם אחד׃ one}%
\verse{ויקם arose, המלך Then the king ויקרע and tore את בגדיו his garments, וישׁכב and lay ארצה on the earth; וכל and all עבדיו his servants נצבים stood by קרעי rent. בגדים׃ with their clothes}%
\verse{ויען answered יונדב And Jonadab, בן the son שׁמעה of Shimeah אחי brother, דוד David's ויאמר and said, אל Let not יאמר suppose אדני my lord את כל all הנערים the young men בני sons; המלך the king's המיתו they have slain כי for אמנון Amnon לבדו only מת is dead: כי for על by פי the appointment אבשׁלום of Absalom היתה this hath been שׂומה determined מיום from the day ענתו that he forced את תמר Tamar. אחתו׃ his sister}%
\verse{ועתה Now אל therefore let not ישׂם take אדני my lord המלך the king אל to לבו his heart, דבר the thing לאמר to think כל that all בני sons המלך the king's מתו are dead: כי for אם for אמנון Amnon לבדו only מת׃ is dead.}%
\verseWithHeading{Absalom Flees}{ויברח fled. אבשׁלום But Absalom וישׂא lifted up הנער And the young man הצפה that kept the watch את עינו his eyes, וירא and looked, והנה and, behold, עם people רב much הלכים there came מדרך by the way אחריו behind מצד ההר׃}%
\verse{ויאמר said יונדב And Jonadab אל unto המלך the king, הנה Behold, בני sons המלך the king's באו come: כדבר said, עבדך as thy servant כן so היה׃ it is.}%
\verse{ויהי And it came to pass, ככלתו as soon as he had made an end לדבר of speaking, והנה that, behold, בני sons המלך the king's באו came, וישׂאו and lifted up קולם their voice ויבכו and wept: וגם also המלך and the king וכל and all עבדיו his servants בכו wept בכי גדול sore. מאד׃ very}%
\verse{ואבשׁלום But Absalom ברח fled, וילך and went אל to תלמי Talmai, בן the son עמיחור מלך king גשׁור of Geshur. ויתאבל And mourned על for בנו his son כל every הימים׃ day.}%
\verse{ואבשׁלום So Absalom ברח fled, וילך and went גשׁור to Geshur, ויהי and was שׁם there שׁלשׁ three שׁנים׃ years.}%
\verse{ותכל longed דוד David המלך And king לצאת to go forth אל unto אבשׁלום Absalom: כי for נחם he was comforted על concerning אמנון Amnon, כי seeing מת׃ he was dead.}%
\end{biblechapter}%
\begin{biblechapter}% 2 Samuel 14
\verseWithHeading{Joab Plots to Reconcile David with Absalom}{וידע perceived יואב Now Joab בן the son צריה of Zeruiah כי that לב heart המלך the king's על toward אבשׁלום׃ Absalom.}%
\verse{וישׁלח sent יואב And Joab תקועה to Tekoah, ויקח and fetched משׁם thence אשׁה woman, חכמה a wise ויאמר and said אליה unto התאבלי feign thyself to be a mourner, נא her, I pray thee, ולבשׁי and put on נא now בגדי apparel, אבל mourning ואל not תסוכי and anoint שׁמן thyself with oil, והיית but be כאשׁה as a woman זה that ימים time רבים had a long מתאבלת mourned על for מת׃ the dead:}%
\verse{ובאת And come אל to המלך the king, ודברת and speak אליו on כדבר manner הזה this וישׂם put יואב him. So Joab את הדברים the words בפיה׃ in her mouth.}%
\verse{ותאמר spoke האשׁה And when the woman התקעית of Tekoah אל to המלך the king, ותפל she fell על on אפיה her face ארצה to the ground, ותשׁתחו and did obeisance, ותאמר and said, הושׁעה Help, המלך׃ O king.}%
\verse{ויאמר said לה המלך And the king מה unto her, What לך ותאמר aileth thee? And she answered, אבל indeed אשׁה woman, אלמנה a widow אני I וימת is dead. אישׁי׃ and mine husband}%
\verse{ולשׁפחתך And thy handmaid שׁני had two בנים sons, וינצו strove together שׁניהם and they two בשׂדה in the field, ואין and none מציל to part ביניהם to part ויכו smote האחד them, but the one את האחד the other, וימת and slew אתו׃}%
\verse{והנה And, behold, קמה is risen כל the whole המשׁפחה family על against שׁפחתך thine handmaid, ויאמרו and they said, תני Deliver את מכה him that smote אחיו his brother, ונמתהו that we may kill בנפשׁ him, for the life אחיו of his brother אשׁר whom הרג he slew; ונשׁמידה and we will destroy גם also: את היורשׁ the heir וכבו and so they shall quench את גחלתי my coal אשׁר which נשׁארה is left, לבלתי and shall not שׂום לאישׁיׁם name ושׁארית nor remainder על upon פני upon האדמה׃ the earth.}%
\verse{ויאמר said המלך And the king אל unto האשׁה the woman, לכי Go לביתך to thine house, ואני and I אצוה will give charge עליך׃ concerning}%
\verse{ותאמר said האשׁה And the woman התקועית of Tekoah אל unto המלך the king, עלי on אדני My lord, המלך O king, העון the iniquity ועל me, and on בית house: אבי my father's והמלך and the king וכסאו and his throne נקי׃ guiltless.}%
\verse{ויאמר said, המלך And the king המדבר Whosoever saith אליך unto והבאתו thee, bring אלי him to ולא me, and he shall not יסיף thee any עוד more. לגעת׃ touch}%
\verse{ותאמר Then said יזכר remember נא she, I pray thee, המלך let the king את יהוה the LORD אלהיך thy God, מהרבית any more, גאל suffer the revengers הדם of blood לשׁחת to destroy ולא that thou wouldest not ישׁמידו lest they destroy את בני my son. ויאמר And he said, חי liveth, יהוה the LORD אם there shall not יפל fall משׂערת one hair בנך of thy son ארצה׃ to the earth.}%
\verse{ותאמר said, האשׁה Then the woman תדבר speak נא I pray thee, שׁפחתך Let thine handmaid, אל unto אדני my lord המלך the king. דבר word ויאמר And he said, דברי׃ Say on.}%
\verse{ותאמר said, האשׁה And the woman ולמה Wherefore חשׁבתה then hast thou thought כזאת such על a thing against עם the people אלהים of God? ומדבר doth speak המלך for the king הדבר thing הזה this כאשׁם as one which is faulty, לבלתי doth not השׁיב fetch home again המלך in that the king את נדחו׃ his banished.}%
\verse{כי For מות we must needs die, נמות we must needs die, וכמים and as water הנגרים spilt ארצה on the ground, אשׁר which לא cannot יאספו be gathered up ולא again; neither ישׂא respect אלהים doth God נפשׁ person: וחשׁב yet doth he devise מחשׁבות means, לבלתי ידח that his banished ממנו from נדח׃ be not expelled}%
\verse{ועתה Now אשׁר therefore that באתי I am come לדבר to speak אל unto המלך the king, אדני my lord את הדבר thing הזה of this כי because יראני have made me afraid: העם the people ותאמר said, שׁפחתך and thy handmaid אדברה speak נא I will now אל unto המלך the king; אולי it may be יעשׂה will perform המלך that the king את דבר the request אמתו׃ of his handmaid.}%
\verse{כי For ישׁמע will hear, המלך the king להציל to deliver את אמתו his handmaid מכף out of the hand האישׁ of the man להשׁמיד destroy אתי ואתני me and my son יחד together מנחלת out of the inheritance אלהים׃ of God.}%
\verse{ותאמר said, שׁפחתך Then thine handmaid יהיה be נא shall now דבר The word אדני of my lord המלך the king למנוחה comfortable: כי for כמלאך as an angel האלהים of God, כן so אדני my lord המלך the king לשׁמע to discern הטוב good והרע and bad: ויהוה therefore the LORD אלהיך thy God יהי will be עמך׃ with}%
\verse{ויען answered המלך Then the king ויאמר and said אל unto האשׁה the woman, אל not נא me, I pray thee, תכחדי Hide ממני from דבר the thing אשׁר that אנכי I שׁאל shall ask אתך ותאמר said, האשׁה thee. And the woman ידבר speak. נא now אדני Let my lord המלך׃ the king}%
\verse{ויאמר said, המלך And the king היד the hand יואב of Joab אתך with בכל thee in all זאת this? ותען answered האשׁה And the woman ותאמר and said, חי liveth, נפשׁך thy soul אדני my lord המלך the king, אם none can turn אשׁ none can turn להמין to the right ולהשׂמיל hand or to the left מכל from aught אשׁר that דבר hath spoken: אדני my lord המלך the king כי for עבדך thy servant יואב Joab, הוא he צוני bade והוא me, and he שׂם put בפי in the mouth שׁפחתך of thine handmaid: את כל all הדברים words האלה׃ these}%
\verse{לבעבור To סבב fetch about את פני this form הדבר of speech עשׂה done עבדך hath thy servant יואב Joab את הדבר thing: הזה this ואדני and my lord חכם wise, כחכמת according to the wisdom מלאך of an angel האלהים of God, לדעת to know את כל all אשׁר that בארץ׃ in the earth.}%
\verse{ויאמר said המלך And the king אל unto יואב Joab, הנה Behold נא now, עשׂיתי I have done את הדבר thing: הזה this ולך go השׁב therefore, bring את הנער the young man את אבשׁלום׃ Absalom}%
\verse{ויפל fell יואב And Joab אל on פניו his face, ארצה to the ground וישׁתחו and bowed himself, ויברך and thanked את המלך the king: ויאמר said, יואב and Joab היום Today ידע knoweth עבדך thy servant כי that מצאתי I have found חן grace בעיניך in thy sight, אדני my lord, המלך O king, אשׁר in that עשׂה hath fulfilled המלך the king את דבר the request עבדו׃ of his servant.}%
\verseWithHeading{Absalom Returns to Jerusalem}{ויקם arose יואב So Joab וילך and went גשׁורה to Geshur, ויבא and brought את אבשׁלום Absalom ירושׁלם׃ to Jerusalem.}%
\verse{ויאמר said, המלך And the king יסב Let him turn אל to ביתו his own house, ופני my face. לא and let him not יראה see ויסב returned אבשׁלום So Absalom אל to ביתו his own house, ופני face. המלך the king's לא not ראה׃ and saw}%
\verse{וכאבשׁלום as Absalom לא none היה there was אישׁ none יפה for his beauty: בכל But in all ישׂראל Israel להלל praised מאד to be so much מכף from the sole רגלו of his foot ועד even to קדקדו the crown of his head לא no היה there was בו מום׃ blemish}%
\verse{ובגלחו And when he polled את ראשׁו his head, והיה (for it was מקץ ימיםימים אשׁר that יגלח he polled כי because כבד was heavy עליו on וגלחו him, therefore he polled ושׁקל it:) he weighed את שׂער the hair ראשׁו of his head מאתים at two hundred שׁקלים shekels באבן weight. המלך׃ after the king's}%
\verse{ויולדו there were born לאבשׁלום And unto Absalom שׁלושׁה three בנים sons, ובת daughter, אחת and one ושׁמה whose name תמר Tamar: היא she היתה was אשׁה a woman יפת of a fair מראה׃ countenance.}%
\verse{וישׁב dwelt אבשׁלום So Absalom בירושׁלם in Jerusalem, שׁנתים two full years ימים two full years ופני face. המלך the king's לא not ראה׃ and saw}%
\verse{וישׁלח sent אבשׁלום Therefore Absalom אל for יואב Joab, לשׁלח to have sent אתו אל him to המלך the king; ולא not אבה but he would לבוא come אליו to וישׁלח him: and when he sent עוד again שׁנית the second time, ולא not אבה he would לבוא׃ come.}%
\verse{ויאמר Therefore he said אל unto עבדיו his servants, ראו See, חלקת field יואב Joab's אל is near ידי mine, ולו שׁם there; שׂערים and he hath barley לכו go והוצתיה and set באשׁ it on fire. ויצתו set עבדי servants אבשׁלום And Absalom's את החלקה the field באשׁ׃ on fire.}%
\verse{ויקם arose, יואב Then Joab ויבא and came אל to אבשׁלום Absalom הביתה unto house, ויאמר and said אליו unto למה him, Wherefore הציתו set עבדך have thy servants את החלקה my field אשׁר לי באשׁ׃ on fire?}%
\verse{ויאמר answered אבשׁלום And Absalom אל answered יואב Joab, הנה Behold, שׁלחתי I sent אליך unto לאמר thee, saying, בא Come הנה hither, ואשׁלחה that I may send אתך אל thee to המלך the king, לאמר to say, למה Wherefore באתי come מגשׁור טוב good לי עד still: אני am I שׁם for me there ועתה now אראה therefore let me see פני face; המלך the king's ואם and if ישׁ there be בי עון iniquity והמתני׃ in me, let him kill}%
\verse{ויבא came יואב So Joab אל to המלך the king, ויגד and told לו ויקרא him: and when he had called אל for אבשׁלום Absalom, ויבא he came אל to המלך the king, וישׁתחו and bowed לו על himself on אפיו his face ארצה to the ground לפני before המלך the king: וישׁק kissed המלך and the king לאבשׁלום׃ Absalom.}%
\end{biblechapter}%
\begin{biblechapter}% 2 Samuel 15
\verseWithHeading{Absalom Leads a Rebellion}{ויהי And it came to pass מאחרי after כן this, ויעשׂ prepared לו אבשׁלום that Absalom מרכבה him chariots וססים and horses, וחמשׁים and fifty אישׁ men רצים to run לפניו׃ before}%
\verse{והשׁכים rose up early, אבשׁלום And Absalom ועמד and stood על beside יד beside דרך the way השׁער of the gate: ויהי and it was כל that when any האישׁ man אשׁר that יהיה had לו ריב a controversy לבוא came אל to המלך the king למשׁפט for judgment, ויקרא called אבשׁלום then Absalom אליו unto ויאמר him, and said, אי what מזה עיר city אתה thou? ויאמר And he said, מאחד of one שׁבטי of the tribes ישׂראל of Israel. עבדך׃ Thy servant}%
\verse{ויאמר said אליו unto אבשׁלום And Absalom ראה him, See, דברך thy matters טובים good ונכחים and right; ושׁמע to hear אין but no לך מאת המלך׃ man of the king}%
\verse{ויאמר said אבשׁלום Absalom מי moreover, Oh that ישׂמני I were made שׁפט judge בארץ in the land, ועלי unto יבוא might come כל that every אישׁ man אשׁר which יהיה hath לו ריב any suit ומשׁפט or cause והצדקתיו׃ me, and I would do him justice!}%
\verse{והיה And it was בקרב came nigh אישׁ that when any man להשׁתחות to do him obeisance, לו ושׁלח he put forth את ידו his hand, והחזיק and took לו ונשׁק׃ him, and kissed}%
\verse{ויעשׂ did אבשׁלום Absalom כדבר manner הזה And on this לכל to all ישׂראל Israel אשׁר that יבאו came למשׁפט for judgment: אל to המלך the king ויגנב stole אבשׁלום so Absalom את לב the hearts אנשׁי of the men ישׂראל׃ of Israel.}%
\verse{ויהי And it came to pass מקץ after ארבעים forty שׁנה years, ויאמר said אבשׁלום that Absalom אל unto המלך the king, אלכה let me go נא I pray thee, ואשׁלם and pay את נדרי my vow, אשׁר which נדרתי I have vowed ליהוה unto the LORD, בחברון׃ in Hebron.}%
\verse{כי For נדר a vow נדר vowed עבדך thy servant בשׁבתי while I abode בגשׁור at Geshur בארם in Syria, לאמר saying, אם If ישׁיב shall bring me again ישׁיבני יהוה the LORD ירושׁלם indeed to Jerusalem, ועבדתי then I will serve את יהוה׃ the LORD.}%
\verse{ויאמר said לו המלך And the king לך unto him, Go בשׁלום in peace. ויקם So he arose, וילך and went חברונה׃ to Hebron.}%
\verse{וישׁלח sent אבשׁלום But Absalom מרגלים spies בכל throughout all שׁבטי the tribes ישׂראל of Israel, לאמר saying, כשׁמעכם As soon as ye hear את קול the sound השׁפר of the trumpet, ואמרתם then ye shall say, מלך reigneth אבשׁלום Absalom בחברון׃ in Hebron.}%
\verse{ואת And with אבשׁלום Absalom הלכו went מאתים two hundred אישׁ men מירושׁלם קראים called; והלכים and they went לתמם in their simplicity, ולא not ידעו and they knew כל any דבר׃ thing.}%
\verse{וישׁלח sent אבשׁלום And Absalom את אחיתפל for Ahithophel הגילני the Gilonite, יועץ counselor, דוד David's מעירו from his city, מגלה בזבחו while he offered את הזבחים sacrifices. ויהי was הקשׁר And the conspiracy אמץ strong; והעם for the people הולך continually ורב increased את with אבשׁלום׃ Absalom.}%
\verseWithHeading{David Flees from Jerusalem}{ויבא And there came המגיד a messenger אל to דוד David, לאמר saying, היה are לב The hearts אישׁ of the men ישׂראל of Israel אחרי after אבשׁלום׃ Absalom.}%
\verse{ויאמר said דוד And David לכל unto all עבדיו his servants אשׁר that אתו with בירושׁלם him at Jerusalem, קומו Arise, ונברחה כי for לא not תהיה we shall לנו פליטה escape מפני אבשׁלום Absalom: מהרו make speed ללכת to depart, פן lest ימהר us suddenly, והשׂגנו he overtake והדיח and bring עלינו upon את הרעה evil והכה us, and smite העיר the city לפי with the edge חרב׃ of the sword.}%
\verse{ויאמרו said עבדי servants המלך And the king's אל unto המלך the king, ככל whatsoever אשׁר whatsoever יבחר shall appoint. אדני my lord המלך the king הנה Behold, עבדיך׃ thy servants}%
\verse{ויצא went forth, המלך And the king וכל and all ביתו his household ברגליו after ויעזב left המלך him. And the king את עשׂר ten נשׁים women, פלגשׁים concubines, לשׁמר to keep הבית׃ the house.}%
\verse{ויצא went forth, המלך And the king וכל and all העם the people ברגליו after ויעמדו him, and tarried בית המרחק׃}%
\verse{וכל And all עבדיו his servants עברים passed on על beside ידו beside וכל him; and all הכרתי the Cherethites, וכל and all הפלתי the Pelethites, וכל and all הגתים the Gittites, שׁשׁ six מאות hundred אישׁ men אשׁר which באו came ברגלו after מגת עברים passed on על before פני before המלך׃ the king.}%
\verse{ויאמר Then said המלך the king אל to אתי Ittai הגתי the Gittite, למה Wherefore תלך goest גם also אתה thou אתנו with שׁוב us? return ושׁב and abide עם with המלך the king: כי for נכרי a stranger, אתה thou וגם and also גלה an exile. אתה למקומך׃ to thy place,}%
\verse{תמול yesterday, בואך Whereas thou camest והיום should I this day אנועך and down עמנו with ללכת make thee go up ואני us? seeing I הולך go על whither אשׁר whither אני I הולך may, שׁוב return והשׁב thou, and take back את אחיך thy brethren: עמך with חסד mercy ואמת׃ and truth}%
\verse{ויען answered אתי And Ittai את המלך the king, ויאמר and said, חי liveth, יהוה the LORD וחי liveth, אדני and my lord המלך the king כי surely אם surely במקום place אשׁר in what יהיה shall be, שׁם there אדני my lord המלך the king אם whether למות in death אם or לחיים life, כי even שׁם יהיה be. עבדך׃ also will thy servant}%
\verse{ויאמר said דוד And David אל to אתי Ittai, לך Go ועבר and pass over. ויעבר passed over, אתי And Ittai הגתי the Gittite וכל and all אנשׁיו his men, וכל and all הטף the little ones אשׁר that אתו׃ with}%
\verse{וכל And all הארץ the country בוכים wept קול voice, גדול with a loud וכל and all העם the people עברים passed over: והמלך the king עבר also himself passed over בנחל the brook קדרון Kidron, וכל and all העם the people עברים passed over, על toward פני toward דרך the way את המדבר׃ of the wilderness.}%
\verseWithHeading{The Priests Offer Sacrifices for David}{והנה And lo גם also, צדוק Zadok וכל and all הלוים the Levites אתו with נשׂאים him, bearing את ארון the ark ברית of the covenant האלהים of God: ויצקו and they set down את ארון the ark האלהים of God; ויעל went up, אביתר and Abiathar עד until תם had done כל all העם the people לעבור passing מן out of העיר׃ the city.}%
\verse{ויאמר said המלך And the king לצדוק unto Zadok, השׁב Carry back את ארון the ark האלהים of God העיר into the city: אם if אמצא I shall find חן favor בעיני in the eyes יהוה of the LORD, והשׁבני he will bring me again, והראני and show אתו ואתוהו׃ me it, and his habitation:}%
\verse{ואם But if כה he thus יאמר say, לא I have no חפצתי delight בך הנני יעשׂה I, let him do לי כאשׁר to me as טוב good בעיניו׃ seemeth}%
\verse{ויאמר said המלך The king אל also unto צדוק Zadok הכהן the priest, הרואה a seer? אתה thou שׁבה return העיר into the city בשׁלום in peace, ואחימעץ you, Ahimaaz בנך sons ויהונתן and Jonathan בן thy son, אביתר of Abiathar. שׁני and your two בניכם the son אתכם׃ with}%
\verse{ראו See, אנכי I מתמהמה will tarry בעברות המדבר of the wilderness, עד until בוא there come דבר word מעמכם from להגיד׃ you to certify}%
\verse{וישׁב carried צדוק Zadok ואביתר therefore and Abiathar את ארון the ark האלהים of God ירושׁלם again to Jerusalem: וישׁבו and they tarried שׁם׃ there.}%
\verseWithHeading{Hushai Offers to Serve King David}{ודוד And David עלה went up במעלה by the ascent הזיתים of Olivet, עלה as he went up, ובוכה and wept וראשׁ and had his head לו חפוי covered, והוא and he הלך went יחף barefoot: וכל and all העם the people אשׁר that אתו with חפו him covered אישׁ every man ראשׁו his head, ועלו and they went up, עלה as they went up. ובכה׃ weeping}%
\verse{ודוד David, הגיד And told לאמר saying, אחיתפל Ahithophel בקשׁרים among the conspirators עם with אבשׁלום Absalom. ויאמר said, דוד And David סכל into foolishness. נא I pray thee, את עצת turn the counsel אחיתפל of Ahithophel יהוה׃ O LORD,}%
\verse{ויהי And it came to pass, דוד that David בא was come עד to הראשׁ the top אשׁר where ישׁתחוה he worshiped שׁם לאלהים God, והנה behold, לקראתו came to meet חושׁי Hushai הארכי the Archite קרוע rent, כתנתו him with his coat ואדמה and earth על upon ראשׁו׃ his head:}%
\verse{ויאמר said, לו דוד Unto whom David אם If עברת thou passest on אתי with והית me, then thou shalt be עלי unto למשׂא׃ a burden}%
\verse{ואם But if העיר to the city, תשׁוב thou return ואמרת and say לאבשׁלום unto Absalom, עבדך thy servant, אני I המלך O king; אהיה will be עבד servant אביך thy father's ואני I מאז hitherto, ועתה now ואני so I עבדך also thy servant: והפרתה then mayest thou for me defeat לי את עצת the counsel אחיתפל׃ of Ahithophel.}%
\verse{והלוא And not עמך with שׁם there צדוק thee Zadok ואביתר and Abiathar הכהנים the priests? והיה therefore it shall be, כל what הדבר thing אשׁר soever תשׁמע thou shalt hear מבית המלךגיד thou shalt tell לצדוק to Zadok ולאביתר and Abiathar הכהנים׃ the priests.}%
\verse{הנה Behold, שׁם there עמם with שׁני them their two בניהם sons, אחימעץ Ahimaaz לצדוק Zadok's ויהונתן and Jonathan לאביתר Abiathar's ושׁלחתם them ye shall send בידם and by אלי unto כל me every דבר thing אשׁר that תשׁמעו׃ ye can hear.}%
\verse{ויבא came חושׁי So Hushai רעה friend דוד David's העיר into the city, ואבשׁלם and Absalom יבא came ירושׁלם׃ into Jerusalem.}%
\end{biblechapter}%
\begin{biblechapter}% 2 Samuel 16
\verseWithHeading{Ziba Brings Provisions}{ודוד And when David עבר past מעט was a little מהראשׁ the top והנה behold, ציבא Ziba נער the servant מפי בשׁת of Mephibosheth לקראתו וצמד him, with a couple חמרים of asses חבשׁים saddled, ועליהם and upon מאתים them two hundred לחם of bread, ומאה and a hundred צמוקים bunches of raisins, ומאה and a hundred קיץ of summer fruits, ונבל and a bottle יין׃ of wine.}%
\verse{ויאמר said המלך And the king אל unto ציבא Ziba, מה What אלה meanest thou by these? לך ויאמר said, ציבא And Ziba החמורים The asses לבית household המלך for the king's לרכב to ride on; ולהלחם and the bread והקיץ and summer fruit לאכול to eat; הנערים for the young men והיין and the wine, לשׁתות may drink. היעף that such as be faint במדבר׃ in the wilderness}%
\verse{ויאמר said, המלך And the king ואיה And where בן son? אדניך thy master's ויאמר said ציבא And Ziba אל unto המלך the king, הנה Behold, יושׁב he abideth בירושׁלם at Jerusalem: כי for אמר he said, היום Today ישׁיבו restore לי בית shall the house ישׂראל of Israel את ממלכות me the kingdom אבי׃ of my father.}%
\verse{ויאמר Then said המלך the king לצבא to Ziba, הנה Behold, לך כל thine all אשׁר that למפי בשׁת unto Mephibosheth. ויאמר said, ציבא And Ziba השׁתחויתי I humbly beseech אמצא thee I may find חן grace בעיניך in thy sight, אדני my lord, המלך׃ O king.}%
\verseWithHeading{Shimei Curses David}{ובא came המלך And when king דוד David עד to בחורים Bahurim, והנה behold, משׁם thence אישׁ a man יוצא came out ממשׁפחת of the family בית of the house שׁאול of Saul, ושׁמו whose name שׁמעי Shimei, בן the son גרא of Gera: יצא he came forth, יצוא still as he came. ומקלל׃ and cursed}%
\verse{ויסקל And he cast באבנים stones את דוד at David, ואת כל and at all עבדי the servants המלך of king דוד David: וכל and all העם the people וכל and all הגברים the mighty men מימינו on his right hand ומשׂמאלו׃ and on his left.}%
\verse{וכה And thus אמר said שׁמעי Shimei בקללו when he cursed, צא Come out, צא come out, אישׁ man, הדמים thou bloody ואישׁ and thou man הבליעל׃ of Belial:}%
\verse{השׁיב hath returned עליך upon יהוה The LORD כל thee all דמי the blood בית of the house שׁאול of Saul, אשׁר in whose מלכת thou hast reigned; תחתו stead ויתן hath delivered יהוה and the LORD את המלוכה the kingdom ביד into the hand אבשׁלום of Absalom בנך thy son: והנך ברעתך thou in thy mischief, כי because אישׁ man. דמים a bloody אתה׃ thou}%
\verse{ויאמר Then said אבישׁי Abishai בן the son צרויה of Zeruiah אל unto המלך the king, למה Why יקלל curse הכלב dog המת dead הזה should this את אדני my lord המלך the king? אעברה let me go over, נא I pray thee, ואסירה and take off את ראשׁו׃ his head.}%
\verse{ויאמר said, המלך And the king מה What לי ולכם בני have I to do with you, ye sons צריה of Zeruiah? כי because יקלל let him curse, וכי יהוה the LORD אמר hath said לו קלל unto him, Curse את דוד David. ומי Who יאמר shall then say, מדוע Wherefore עשׂיתה hast thou done כן׃ so?}%
\verse{ויאמר said דוד And David אל to אבישׁי Abishai, ואל and to כל all עבדיו his servants, הנה Behold, בני my son, אשׁר which יצא came forth ממעי of my bowels, מבקשׁ seeketh את נפשׁי my life: ואף how much more כי how much more עתה now בן הימיני Benjamite הנחו לו ויקלל and let him curse; כי for אמר hath bidden לו יהוה׃ the LORD}%
\verse{אולי It may be יראה will look יהוה that the LORD בעוני והשׁיב will requite יהוה and that the LORD לי טובה me good תחת for קללתו his cursing היום day. הזה׃ this}%
\verse{וילך went דוד And as David ואנשׁיו and his men בדרך by the way, ושׁמעי Shimei הלך went along בצלע side ההר on the hill's לעמתו over against הלוך as he went, ויקלל him, and cursed ויסקל and threw באבנים stones לעמתו at ועפר him, and cast בעפר׃ dust.}%
\verse{ויבא him, came המלך And the king, וכל and all העם the people אשׁר that אתו with עיפים weary, וינפשׁ and refreshed themselves שׁם׃ there.}%
\verseWithHeading{Hushai Comes to Absalom}{ואבשׁלום And Absalom, וכל and all העם the people אישׁ the men ישׂראל of Israel, באו came ירושׁלם to Jerusalem, ואחיתפל and Ahithophel אתו׃ with}%
\verse{ויהי And it came to pass, כאשׁר when בא was come חושׁי Hushai הארכי the Archite, רעה friend, דוד David's אל unto אבשׁלום Absalom, ויאמר said חושׁי that Hushai אל unto אבשׁלום Absalom, יחי God save המלך the king, יחי God save המלך׃ the king.}%
\verse{ויאמר said אבשׁלום And Absalom אל to חושׁי Hushai, זה this חסדך thy kindness את with רעך to thy friend? למה why לא thou not הלכת wentest את רעך׃ thy friend?}%
\verse{ויאמר said חושׁי And Hushai אל unto אבשׁלם Absalom, לא Nay; כי but אשׁר whom בחר choose, יהוה the LORD, והעם people, הזה and this וכל and all אישׁ the men ישׂראל of Israel, לא אהיה his will I be, ואתו and with אשׁב׃ him will I abide.}%
\verse{והשׁנית And again, למי whom אני should I אעבד serve? הלוא not לפני in the presence בנו of his son? כאשׁר as עבדתי I have served לפני presence, אביך in thy father's כן so אהיה will I be לפניך׃ in thy presence.}%
\verse{ויאמר Then said אבשׁלום Absalom אל to אחיתפל Ahithophel, הבו Give לכם עצה counsel מה among you what נעשׂה׃ we shall do.}%
\verse{ויאמר said אחיתפל And Ahithophel אל unto אבשׁלם Absalom, בוא Go אל in unto פלגשׁי concubines, אביך thy father's אשׁר which הניח לשׁמור to keep הבית the house; ושׁמע shall hear כל and all ישׂראל Israel כי that נבאשׁת thou art abhorred את of אביך thy father: וחזקו thee be strong. ידי then shall the hands כל of all אשׁר that אתך׃ with}%
\verse{ויטו So they spread לאבשׁלום Absalom האהל a tent על upon הגג the top ויבא went אבשׁלום of the house; and Absalom אל in unto פלגשׁי concubines אביו his father's לעיני in the sight כל of all ישׂראל׃ Israel.}%
\verse{ועצת And the counsel אחיתפל of Ahithophel, אשׁר which יעץ he counseled בימים days, ההם in those כאשׁר as ישׁאל had inquired בדבר at the oracle האלהים of God: כן so כל all עצת the counsel אחיתפל of Ahithophel גם both לדוד with David גם and לאבשׁלם׃ with Absalom.}%
\end{biblechapter}%
\begin{biblechapter}% 2 Samuel 17
\verseWithHeading{Hushai Frustrates the Counsel of Ahithophel}{ויאמר said אחיתפל Moreover Ahithophel אל unto אבשׁלם Absalom, אבחרה choose out נא Let me now שׁנים twelve עשׂר twelve אלף thousand אישׁ men, ואקומה and I will arise וארדפה and pursue אחרי after דוד David הלילה׃ this night:}%
\verse{ואבוא And I will come עליו upon והוא him while he יגע weary ורפה and weak ידים handed, והחרדתי אתונס him shall flee; כל and all העם the people אשׁר that אתו with והכיתי and I will smite את המלך the king לבדו׃ only:}%
\verse{ואשׁיבה And I will bring back כל all העם the people אליך unto כשׁוב returned: הכל as if all האישׁ thee: the man אשׁר whom אתה thou מבקשׁ seekest כל all העם the people יהיה shall be שׁלום׃ in peace.}%
\verse{ויישׁר pleased הדבר And the saying בעיני well, אבשׁלם Absalom ובעיני כל and all זקני the elders ישׂראל׃ of Israel.}%
\verse{ויאמר Then said אבשׁלום Absalom, קרא Call נא now גם also, לחושׁי Hushai הארכי the Archite ונשׁמעה and let us hear מה what בפיו saith. גם likewise הוא׃ he}%
\verse{ויבא was come חושׁי And when Hushai אל to אבשׁלום Absalom, ויאמר spoke אבשׁלום Absalom אליו unto לאמר him, saying, כדבר manner: הזה after this דבר hath spoken אחיתפל Ahithophel הנעשׂה shall we do את דברו his saying? אם if אין not; אתה thou. דבר׃ speak}%
\verse{ויאמר said חושׁי And Hushai אל unto אבשׁלום Absalom, לא not טובה good העצה The counsel אשׁר that יעץ hath given אחיתפל Ahithophel בפעם time. הזאת׃ at this}%
\verse{ויאמר For, said חושׁי Hushai, אתה thou ידעת knowest את אביך thy father ואת אנשׁיו and his men, כי that גברים mighty men, המה they ומרי chafed נפשׁ in their minds, המה and they כדב as a bear שׁכול robbed of her whelps בשׂדה in the field: ואביך and thy father אישׁ a man מלחמה of war, ולא and will not ילין lodge את with העם׃ the people.}%
\verse{הנה Behold, עתה now הוא he נחבא is hid באחת in some הפחתים pit, או or באחד in some המקומת place: והיה and it will come to pass, כנפל when some of them be overthrown בהם בתחלה at the first, ושׁמע that whosoever heareth השׁמע that whosoever heareth ואמר it will say, היתה There is מגפה a slaughter בעם among the people אשׁר that אחרי follow אבשׁלם׃ Absalom.}%
\verse{והוא And he גם also בן valiant, חיל valiant, אשׁר whose לבו heart כלב as the heart האריה of a lion, המס shall utterly melt: ימס shall utterly melt: כי for ידע knoweth כל all ישׂראל Israel כי that גבור a mighty man, אביך thy father ובני men. חיל him valiant אשׁר and which אתו׃ with}%
\verse{כי Therefore יעצתי I counsel האסף be generally יאסף gathered עליך unto כל that all ישׂראל Israel מדן ועד even to באר שׁבע Beer-sheba, כחול as the sand אשׁר that על by הים the sea לרב for multitude; ופניך in thine own person. הלכים and that thou go בקרב׃ to battle}%
\verse{ובאנו So shall we come אליו upon באחת him in some המקומת place אשׁר where נמצא he shall be found, שׁם where ונחנו and we will light עליו upon כאשׁר him as יפל falleth הטל the dew על on האדמה the ground: ולא him there shall not נותר be left בו ובכל and of him and of all האנשׁים אשׁר that אתו with גם so much אחד׃ as one.}%
\verse{ואם Moreover, if אל into עיר a city, יאסף he be gotten והשׂיאו bring כל then shall all ישׂראל Israel אל to העיר city, ההיא that חבלים ropes וסחבנו and we will draw אתו עד it into הנחל the river, עד until אשׁר until לא there be not נמצא found שׁם there. גם one small צרור׃ stone}%
\verse{ויאמר said, אבשׁלום And Absalom וכל and all אישׁ the men ישׂראל of Israel טובה better עצת The counsel חושׁי of Hushai הארכי the Archite מעצת than the counsel אחיתפל of Ahithophel. ויהוה For the LORD צוה had appointed להפר to defeat את עצת counsel אחיתפל of Ahithophel, הטובה the good לבעבור to the intent הביא might bring יהוה that the LORD אל upon אבשׁלום Absalom. את הרעה׃ evil}%
\verseWithHeading{Hushai Sends Word to David}{ויאמר Then said חושׁי Hushai אל unto צדוק Zadok ואל and to אביתר Abiathar הכהנים the priests, כזאת Thus וכזאת and thus יעץ counsel אחיתפל did Ahithophel את אבשׁלם Absalom ואת זקני and the elders ישׂראל of Israel; וכזאת and thus וכזאת and thus יעצתי counseled. אני׃ have I}%
\verse{ועתה Now שׁלחו therefore send מהרה quickly, והגידו and tell לדוד David, לאמר saying, אל not תלן Lodge הלילה this night בערבות in the plains המדבר of the wilderness, וגם but עבור speedily תעבור pass over; פן lest יבלע be swallowed up, למלך the king ולכל and all העם the people אשׁר that אתו׃ with}%
\verse{ויהונתן Now Jonathan ואחימעץ and Ahimaaz עמדים stayed בעין רגל by En-rogel; והלכה went השׁפחה and a wench והגידה and told להם והם them; and they ילכו went והגידו and told למלך king דוד David. כי for לא not יוכלו they might להראות be seen לבוא to come העירה׃ into the city:}%
\verse{וירא saw אתם נער Nevertheless a lad ויגד them, and told לאבשׁלם Absalom: וילכו but they went שׁניהם both מהרה of them away quickly, ויבאו and came אל to בית house אישׁ a man's בבחורים in Bahurim, ולו באר which had a well בחצרו in his court; וירדו they went down. שׁם׃ whither}%
\verse{ותקח took האשׁה And the woman ותפרשׂ and spread את המסך a covering על over פני mouth, הבאר the well's ותשׁטח and spread עליו thereon; הרפות ground corn ולא was not נודע known. דבר׃ and the thing}%
\verse{ויבאו came עבדי servants אבשׁלום And when Absalom's אל to האשׁה the woman הביתה to the house, ויאמרו they said, איה Where אחימעץ Ahimaaz ויהונתן and Jonathan? ותאמר said להם האשׁה And the woman עברו unto them, They be gone over מיכל the brook המים of water. ויבקשׁו And when they had sought ולא and could not מצאו find וישׁבו they returned ירושׁלם׃ to Jerusalem.}%
\verse{ויהי And it came to pass, אחרי after לכתם they were departed, ויעלו that they came up מהבאר out of the well, וילכו and went ויגדו and told למלך king דוד David, ויאמרו and said אל unto דוד David, קומו Arise, ועברו and pass quickly over מהרה and pass quickly over את המים the water: כי for ככה thus יעץ counseled עליכם against אחיתפל׃ hath Ahithophel}%
\verse{ויקם arose, דוד Then David וכל and all העם the people אשׁר that אתו with ויעברו him, and they passed over את הירדן Jordan: עד by אור light הבקר the morning עד אחד one לא not נעדר there lacked אשׁר of them that לא was not עבר gone over את הירדן׃ Jordan.}%
\verse{ואחיתפל And when Ahithophel ראה saw כי that לא was not נעשׂתה followed, עצתו his counsel ויחבשׁ he saddled את החמור ass, ויקם and arose, וילך and got אל him home to ביתו his house, אל to עירו his city, ויצו and put אל and put ביתו his household ויחנק in order, and hanged himself, וימת and died, ויקבר and was buried בקבר in the sepulcher אביו׃ of his father.}%
\verse{ודוד Then David בא came מחנימה to Mahanaim. ואבשׁלם And Absalom עבר passed over את הירדן Jordan, הוא he וכל and all אישׁ the men ישׂראל of Israel עמו׃ with}%
\verse{ואת עמשׂא Amasa שׂם made אבשׁלם And Absalom תחת instead יואב of Joab: על captain הצבא of the host ועמשׂא which Amasa בן son, אישׁ a man's ושׁמו whose name יתרא Ithra הישׂראלי an Israelite, אשׁר that בא went אל in to אביגל Abigail בת the daughter נחשׁ of Nahash, אחות sister צרויה to Zeruiah אם mother. יואב׃ Joab's}%
\verse{ויחן pitched ישׂראל So Israel ואבשׁלם and Absalom ארץ in the land הגלעד׃ of Gilead.}%
\verse{ויהי And it came to pass, כבוא was come דוד when David מחנימה to Mahanaim, ושׁבי that Shobi בן the son נחשׁ of Nahash מרבת בני of the children עמון of Ammon, ומכיר and Machir בן the son עמיאל of Ammiel מלא דבר וברזלי and Barzillai הגלעדי the Gileadite מרגלים׃}%
\verse{משׁכב beds, וספות and basins, וכלי vessels, יוצר and earthen וחטים and wheat, ושׂערים and barley, וקמח and flour, וקלי and parched ופול and beans, ועדשׁים and lentils, וקלי׃ and parched}%
\verse{ודבשׁ And honey, וחמאה and butter, וצאן and sheep, ושׁפות and cheese בקר of kine, הגישׁו לדוד for David, ולעם and for the people אשׁר that אתו with לאכול him, to eat: כי for אמרו they said, העם The people רעב hungry, ועיף and weary, וצמא and thirsty, במדבר׃ in the wilderness.}%
\end{biblechapter}%
\begin{biblechapter}% 2 Samuel 18
\verseWithHeading{Absalom Dies in Battle}{ויפקד numbered דוד And David את העם the people אשׁר that אתו with וישׂם him, and set עליהם over שׂרי captains אלפים of thousands ושׂרי and captains מאות׃ of hundreds}%
\verse{וישׁלח sent forth דוד And David את העם of the people השׁלשׁית a third part ביד under the hand יואב of Joab, והשׁלשׁית and a third part ביד under the hand אבישׁי of Abishai בן the son צרויה of Zeruiah, אחי brother, יואב Joab's והשׁלשׁת and a third part ביד under the hand אתי of Ittai הגתי the Gittite. ויאמר said המלך And the king אל unto העם the people, יצא אצאם also. אני you myself עמכם׃ with}%
\verse{ויאמר answered, העם But the people לא Thou shalt not תצא go forth: כי for אם if נס we flee away, ננוס we flee away, לא they will not ישׂימו care אלינו for לב care ואם if ימתו of us die, חצינו half לא us; neither ישׂימו will they care אלינו for לב will they care כי us: but עתה now כמנו worth עשׂרה ten אלפים thousand ועתה of us: therefore now טוב better כי that תהיה לנו מעיר us out of the city. לעזיר׃ thou succor}%
\verse{ויאמר said אליהם unto המלך And the king אשׁר them, What ייטב you best בעיניכם seemeth אעשׂה I will do. ויעמד stood המלך And the king אל by יד side, השׁער the gate וכל and all העם the people יצאו למאות by hundreds ולאלפים׃ and by thousands.}%
\verse{ויצו commanded המלך And the king את יואב Joab ואת אבישׁי and Abishai ואת אתי and Ittai, לאמר saying, לאט gently לי לנער for my sake with the young man, לאבשׁלום with Absalom. וכל And all העם the people שׁמעו heard בצות charge המלך when the king את כל gave all השׂרים the captains על concerning דבר concerning אבשׁלום׃ Absalom.}%
\verse{ויצא went out העם So the people השׂדה into the field לקראת against ישׂראל Israel: ותהי was המלחמה and the battle ביער in the wood אפרים׃ of Ephraim;}%
\verse{וינגפו were slain שׁם Where עם the people ישׂראל of Israel לפני before עבדי the servants דוד of David, ותהי and there was שׁם there המגפה slaughter גדולה a great ביום day ההוא that עשׂרים of twenty אלף׃ thousand}%
\verse{ותהי was שׁם there המלחמה For the battle נפצית scattered על over פני the face כל of all הארץ the country: וירב more היער and the wood לאכל devoured בעם people מאשׁר אכלה devoured. החרב than the sword ביום day ההוא׃ that}%
\verse{ויקרא met אבשׁלום And Absalom לפני עבדי the servants דוד of David. ואבשׁלום And Absalom רכב rode על upon הפרד a mule, ויבא went הפרד and the mule תחת under שׂובך the thick boughs האלה oak, הגדולה of a great ויחזק caught hold ראשׁו and his head באלה of the oak, ויתן and he was taken up בין between השׁמים the heaven ובין הארץ and the earth; והפרד and the mule אשׁר that תחתיו under עבר׃ him went away.}%
\verse{וירא saw אישׁ man אחד And a certain ויגד and told ליואב Joab, ויאמר and said, הנה Behold, ראיתי I saw את אבשׁלם Absalom תלוי hanged באלה׃ in an oak.}%
\verse{ויאמר said יואב And Joab לאישׁ unto the man המגיד that told לו והנה him, And, behold, ראית thou sawest ומדוע and why לא didst thou not הכיתו smite שׁם him there ארצה to the ground? ועלי לתת and I would have given לך עשׂרה thee ten כסף of silver, וחגרה girdle. אחת׃ and a}%
\verse{ויאמר said האישׁ And the man אל unto יואב Joab, ולא Though אנכי I שׁקל should receive על in כפי mine hand, אלף a thousand כסף of silver לא would I not אשׁלח put forth ידי mine hand אל against בן son: המלך the king's כי for באזנינו in our hearing צוה charged המלך the king אתך ואתבישׁי thee and Abishai ואת אתי and Ittai, לאמר saying, שׁמרו מינער the young man באבשׁלום׃ Absalom.}%
\verse{או Otherwise עשׂיתי I should have wrought בנפשׁו against mine own life: שׁקר falsehood וכל דברא for there is no matter יכחד hid מן from המלך the king, ואתה and thou thyself תתיצב wouldest have set thyself מנגד׃ against}%
\verse{ויאמר Then said יואב Joab, לא I may not כן thus אחילה tarry לפניך with ויקח thee. And he took שׁלשׁה three שׁבטים darts בכפו in his hand, ויתקעם and thrust בלב them through the heart אבשׁלום of Absalom, עודנו while חי he yet alive בלב in the midst האלה׃ of the oak.}%
\verse{ויסבו compassed about עשׂרה And ten נערים young men נשׂאי that bore כלי armor יואב Joab's ויכו and smote את אבשׁלום Absalom, וימיתהו׃ and slew}%
\verse{ויתקע blew יואב And Joab בשׁפר the trumpet, וישׁב returned העם and the people מרדף from pursuing אחרי after ישׂראל Israel: כי for חשׂך held back יואב Joab את העם׃ the people.}%
\verse{ויקחו And they took את אבשׁלום Absalom, וישׁליכו and cast אתו ביער in the wood, אל him into הפחת pit הגדול a great ויצבו and laid עליו upon גל heap אבנים of stones גדול great מאד a very וכל him: and all ישׂראל Israel נסו fled אישׁ every one לאהלו׃ to his tent.}%
\verse{ואבשׁלם Now Absalom לקח had taken ויצב and reared up לו בחיו in his lifetime את מצבת for himself a pillar, אשׁר which בעמק dale: המלך in the king's כי for אמר he said, אין I have no לי בן son בעבור to הזכיר keep my name in remembrance: שׁמי keep my name in remembrance: ויקרא and he called למצבת the pillar על after שׁמו his own name: ויקרא and it is called לה יד place. אבשׁלם Absalom's עד unto היום day, הזה׃ this}%
\verse{ואחימעץ Ahimaaz בן the son צדוק of Zadok, אמר Then said ארוצה run, נא Let me now ואבשׂרה and bear את המלך the king כי tidings, how that שׁפטו hath avenged יהוה the LORD מיד איביו׃ him of his enemies.}%
\verse{ויאמר said לו יואב And Joab לא shalt not אישׁ bear tidings בשׂרה bear tidings אתה unto him, Thou היום day, הזה this ובשׂרת but thou shalt bear tidings ביום day: אחר another והיום day הזה but this לא thou shalt bear no tidings, תבשׂר thou shalt bear no tidings, כי because על because בן son המלך the king's מת׃ is dead.}%
\verse{ויאמר Then said יואב Joab לכושׁי to Cushi, לך Go הגד tell למלך the king אשׁר what ראיתה thou hast seen. וישׁתחו bowed himself כושׁי And Cushi ליואב unto Joab, וירץ׃ and ran.}%
\verse{ויסף again עוד yet אחימעץ Ahimaaz בן the son צדוק of Zadok ויאמר Then said אל to יואב Joab, ויהי מהרצה run נא pray גם thee, also אני let me, I אחרי after הכושׁי Cushi. ויאמר said, יואב And Joab למה זהתה wilt thou רץ run, בני my son, ולכה אין seeing that thou hast no בשׂורה tidings מצאת׃ ready?}%
\verse{ויהי מהרוץ let me run. ויאמר And he said לו רוץ unto him, Run. וירץ ran אחימעץ Then Ahimaaz דרך by the way הככר of the plain, ויעבר and overran את הכושׁי׃ Cushi.}%
\verse{ודוד And David יושׁב sat בין between שׁני the two השׁערים gates: וילך went up הצפה and the watchman אל to גג the roof השׁער over the gate אל unto החומה the wall, וישׂא and lifted up את עיניו his eyes, וירא and looked, והנה and behold אישׁ a man רץ running לבדו׃ alone.}%
\verse{ויקרא cried, הצפה And the watchman ויגד and told למלך the king. ויאמר said, המלך And the king אם If לבדו he alone, בשׂורה tidings בפיו in his mouth. וילך And he came הלוך apace, וקרב׃ and drew near.}%
\verse{וירא saw הצפה And the watchman אישׁ man אחר another רץ running: ויקרא called הצפה and the watchman אל unto השׁער the porter, ויאמר and said, הנה Behold אישׁ man רץ running לבדו alone. ויאמר said, המלך And the king גם also זה He מבשׂר׃ bringeth tidings.}%
\verse{ויאמר said, הצפה And the watchman אני Me ראה thinketh את מרוצת the running הראשׁון of the foremost כמרצת is like the running אחימעץ of Ahimaaz בן the son צדוק of Zadok. ויאמר said, המלך And the king אישׁ man, טוב a good זה He ואל בשׂורה tidings. טובה with good יבוא׃ and cometh}%
\verse{ויקרא called, אחימעץ And Ahimaaz ויאמר and said אל unto המלך the king, שׁלום All is well. וישׁתחו למלך before the king, לאפיו upon his face ארצה to the earth ויאמר and said, ברוך Blessed יהוה the LORD אלהיך thy God, אשׁר which סגר hath delivered up את האנשׁים the men אשׁר that נשׂאו lifted up את ידם their hand באדני against my lord המלך׃ the king.}%
\verse{ויאמר said, המלך And the king שׁלום safe? לנער Is the young man לאבשׁלום Absalom ויאמר answered, אחימעץ And Ahimaaz ראיתי I saw ההמון tumult, הגדול a great לשׁלח sent את עבד servant, המלך the king's יואב When Joab ואת עבדך and thy servant, ולא not ידעתי but I knew מה׃ what}%
\verse{ויאמר said המלך And the king סב Turn aside, התיצב stand כה here. ויסב And he turned aside, ויעמד׃ and stood still.}%
\verse{והנה And, behold, הכושׁי Cushi בא came; ויאמר said, הכושׁי and Cushi יתבשׂר Tidings, אדני my lord המלך the king: כי for שׁפטך hath avenged יהוה the LORD היום thee this day מיד of כל all הקמים them that rose up עליך׃ against}%
\verse{ויאמר said המלך And the king אל unto הכושׁי Cushi, השׁלום safe? לנער Is the young man לאבשׁלום Absalom ויאמר answered, הכושׁי And Cushi יהיו be כנער as young man איבי The enemies אדני of my lord המלך the king, וכל and all אשׁר that קמו rise עליך against לרעה׃ thee to do hurt,}%
\verse{וירגז was much moved, המלך And the king ויעל and went up על to עלית the chamber השׁער over the gate, ויבך and wept: וכה thus אמר he said, בלכתו and as he went, בני O my son אבשׁלום Absalom, בני my son, בני my son אבשׁלום Absalom! מי יתןותי had died אני I תחתיך for אבשׁלום thee, O Absalom, בני my son, בני׃ my son!}%
\end{biblechapter}%
\begin{biblechapter}% 2 Samuel 19
\verseWithHeading{King David Weeps for Absalom}{ויגד And it was told ליואב Joab, הנה Behold, המלך the king בכה weepeth ויתאבל and mourneth על for אבשׁלם׃ Absalom.}%
\verse{ותהי that התשׁעה And the victory ביום day ההוא that לאבל was into mourning לכל unto all העם the people: כי for שׁמע heard העם the people ביום day ההוא לאמר say נעצב was grieved המלך how the king על for בנו׃ his son.}%
\verse{ויתגנב them by stealth העם And the people ביום day ההוא that לבוא got העיר into the city, כאשׁר as יתגנב steal away העם people הנכלמים being ashamed בנוסם when they flee במלחמה׃ in battle.}%
\verse{והמלך But the king לאט covered את פניו his face, ויזעק cried המלך and the king קול voice, גדול with a loud בני O my son אבשׁלום Absalom, אבשׁלום O Absalom, בני my son, בני׃ my son!}%
\verse{ויבא came יואב And Joab אל to המלך the king, הבית into the house ויאמר and said, הבשׁת Thou hast shamed היום this day את פני the faces כל of all עבדיך thy servants, הממלטים have saved את נפשׁך thy life, היום which this day ואת נפשׁ and the lives בניך of thy sons ובנתיך and of thy daughters, ונפשׁ and the lives נשׁיך of thy wives, ונפשׁ and the lives פלגשׁיך׃ of thy concubines;}%
\verse{לאהבה In that thou lovest את שׂנאיך thine enemies, ולשׂנא and hatest את אהביך thy friends. כי For הגדת thou hast declared היום this day, כי that אין thou regardest neither לך שׂרים princes ועבדים nor servants: כי for ידעתי I perceive, היום this day כי that לא if אבשׁלום Absalom חי had lived, וכלנו and all היום this day, מתים we had died כי אז then ישׁר it had pleased thee well. בעיניך׃ it had pleased thee well.}%
\verse{ועתה Now קום therefore arise, צא go forth, ודבר and speak על comfortably לב comfortably עבדיך unto thy servants: כי for ביהוה by the LORD, נשׁבעתי I swear כי if אינך thou go not forth, יוצא thou go not forth, אם there will not ילין tarry אישׁ one אתך with הלילה thee this night: ורעה the evil לך זאת and that מכל unto thee than all הרעה אשׁר that באה befell עליך befell מנעריך thee from thy youth עד until עתה׃ now.}%
\verse{ויקם arose, המלך Then the king וישׁב and sat בשׁער in the gate. ולכל unto all העם the people, הגידו And they told לאמר saying, הנה Behold, המלך the king יושׁב doth sit בשׁער in the gate. ויבא came כל And all העם the people לפני before המלך the king: וישׂראל for Israel נס had fled אישׁ every man לאהליו׃ to his tent.}%
\verseWithHeading{David Returns from Exile}{ויהי were כל And all העם the people נדון at strife בכל throughout all שׁבטי the tribes ישׂראל of Israel, לאמר saying, המלך The king הצילנו saved מכף us out of the hand איבינו of our enemies, והוא and he מלטנו delivered מכף us out of the hand פלשׁתים of the Philistines; ועתה and now ברח he is fled מן us out of the hand הארץ the land מעל for אבשׁלום׃ Absalom.}%
\verse{ואבשׁלום And Absalom, אשׁר whom משׁחנו we anointed עלינו over מת us, is dead במלחמה in battle. ועתה Now למה therefore why אתם ye מחרשׁים speak להשׁיב not a word of bringing את המלך׃ the king}%
\verse{והמלך And king דוד David שׁלח sent אל to צדוק Zadok ואל and to אביתר Abiathar הכהנים the priests, לאמר saying, דברו Speak אל unto זקני the elders יהודה of Judah, לאמר saying, למה Why תהיו are אחרנים ye the last להשׁיב to bring את המלך the king אל back to ביתו his house? ודבר seeing the speech כל of all ישׂראל Israel בא is come אל to המלך the king, אל to ביתו׃ his house.}%
\verse{אחי my brethren, אתם Ye עצמי my bones ובשׂרי and my flesh: אתם ye ולמה wherefore תהיו then are אחרנים ye the last להשׁיב to bring back את המלך׃ the king?}%
\verse{ולעמשׂא ye to Amasa, תמרו And say הלוא not עצמי of my bone, ובשׂרי and of my flesh? אתה thou כה so יעשׂה do לי אלהים God וכה also, יוסיף to me, and more אם if לא not שׂר captain צבא of the host תהיה thou be לפני before כל me continually הימים me continually תחת in the room יואב׃ of Joab.}%
\verse{ויט And he bowed את לבב the heart כל of all אישׁ the men יהודה of Judah, כאישׁ man; אחד even as one וישׁלחו so that they sent אל unto המלך the king, שׁוב Return אתה thou, וכל and all עבדיך׃ thy servants.}%
\verse{וישׁב returned, המלך So the king ויבא and came עד to הירדן Jordan. ויהודה And Judah בא came הגלגלה to Gilgal, ללכת to go לקראת to meet המלך the king, להעביר אתמלך אתירדן׃ Jordan.}%
\verse{וימהר hasted שׁמעי And Shimei בן the son גרא of Gera, בן הימיני a Benjamite, אשׁר which מבחורים וירד and came down עם with אישׁ the men יהודה of Judah לקראת to meet המלך king דוד׃ David.}%
\verse{ואלף And a thousand אישׁ men עמו with מבנימן וציבא him, and Ziba נער the servant בית of the house שׁאול of Saul, וחמשׁת and his fifteen עשׂר and his fifteen בניו sons ועשׂרים and his twenty עבדיו servants אתו with וצלחו him; and they went over הירדן Jordan לפני before המלך׃ the king.}%
\verse{ועברה And there went over העברה a ferry boat לעביר to carry over את בית household, המלך the king's ולעשׂות and to do הטוב good. בעינו what he thought ושׁמעי And Shimei בן the son גרא of Gera נפל fell down לפני before המלך the king, בעברו as he was come over בירדן׃ Jordan;}%
\verse{ויאמר And said אל unto המלך the king, אל Let not יחשׁב impute לי אדני my lord עון iniquity ואל unto me, neither תזכר do thou remember את אשׁר that which העוה did perversely עבדך thy servant ביום the day אשׁר that יצא went out אדני my lord המלך the king מירושׁלם לשׂום should take המלך that the king אל it to לבו׃ his heart.}%
\verse{כי For ידע doth know עבדך thy servant כי that אני I חטאתי have sinned: והנה therefore, behold, באתי I am come היום this day ראשׁון the first לכל of all בית the house יוסף of Joseph לרדת to go down לקראת to meet אדני my lord המלך׃ the king.}%
\verse{ויען answered אבישׁי But Abishai בן the son צרויה of Zeruiah ויאמר and said, התחת for זאת this, לא Shall not יומת be put to death שׁמעי Shimei כי because קלל he cursed את משׁיח anointed? יהוה׃ the LORD's}%
\verse{ויאמר said, דוד And David מה What לי ולכם בני have I to do with you, ye sons צרויה of Zeruiah, כי that תהיו be לי היום ye should this day לשׂטן adversaries היום this day יומת be put to death אישׁ unto me? shall there any man בישׂראל in Israel? כי for הלוא do not ידעתי I know כי that היום this day אני I מלך king על over ישׂראל׃ Israel?}%
\verse{ויאמר said המלך Therefore the king אל unto שׁמעי Shimei, לא Thou shalt not תמות die. וישׁבע swore לו המלך׃ And the king}%
\verse{ומפבשׁת And Mephibosheth בן the son שׁאול of Saul ירד came down לקראת to meet המלך the king, ולא and had neither עשׂה dressed רגליו his feet, ולא nor עשׂה trimmed שׂפמו his beard, ואת בגדיו his clothes, לא nor כבס washed למן from היום the day לכת departed המלך the king עד until היום the day אשׁר בא he came בשׁלום׃ in peace.}%
\verse{ויהי And it came to pass, כי when בא he was come ירושׁלם to Jerusalem לקראת to meet המלך the king, ויאמר said לו המלך that the king למה unto him, Wherefore לא not הלכת wentest עמי thou with מפיבשׁת׃ me, Mephibosheth?}%
\verse{ויאמר And he answered, אדני My lord, המלך O king, עבדי my servant רמני deceived כי me: for אמר said, עבדך thy servant אחבשׁה I will saddle לי החמור me an ass, וארכב that I may ride עליה thereon, ואלך and go את to המלך the king; כי because פסח lame. עבדך׃ thy servant}%
\verse{וירגל And he hath slandered בעבדך thy servant אל unto אדני my lord המלך the king; ואדני but my lord המלך the king כמלאך as an angel האלהים of God: ועשׂה do הטוב therefore good בעיניך׃ in thine eyes.}%
\verse{כי For לא היה were כל all בית house אבי my father's כי but אם but אנשׁי men מות dead לאדני my lord המלך the king: ותשׁת yet didst thou set את עבדך thy servant באכלי among them that did eat שׁלחנך at thine own table. ומה What ישׁ therefore have לי עוד any more צדקה right ולזעק to cry עוד אל unto המלך׃ the king?}%
\verse{ויאמר said לו המלך And the king למה unto him, Why תדבר speakest עוד thou any more דבריך of thy matters? אמרתי I have said, אתה Thou וציבא and Ziba תחלקו divide את השׂדה׃ the land.}%
\verse{ויאמר said מפיבשׁת And Mephibosheth אל unto המלך the king, גם Yea, את הכל all, יקח let him take אחרי forasmuch אשׁר forasmuch בא is come again אדני as my lord המלך the king בשׁלום in peace אל unto ביתו׃ his own house.}%
\verse{וברזלי And Barzillai הגלעדי the Gileadite ירד came down מרגלים ויעבר and went over את with המלך the king, הירדן Jordan לשׁלחו to conduct him over את בירדן׃ Jordan.}%
\verse{וברזלי Now Barzillai זקן aged מאד was a very בן old: שׁמנים man, fourscore שׁנה years והוא and he כלכל of sustenance את המלך had provided the king בשׁיבתו while he lay במחנים at Mahanaim; כי for אישׁ man. גדול great הוא he מאד׃ a very}%
\verse{ויאמר said המלך And the king אל unto ברזלי Barzillai, אתה thou עבר Come אתי over with וכלכלתי me, and I will feed אתך עמדיירושׁלם׃ me in Jerusalem.}%
\verse{ויאמר said ברזלי And Barzillai אל unto המלך the king, כמה How ימי long שׁני חיי have I to live, כי that אעלה I should go up את with המלך the king ירושׁלם׃ unto Jerusalem?}%
\verse{בן old: שׁמנים fourscore שׁנה years אנכי I היום this day האדע can I discern בין between טוב good לרע and evil? אם can יטעם taste עבדך thy servant את אשׁר what אכל I eat ואת אשׁר or what אשׁתה I drink? אם can אשׁמע I hear עוד any more בקול the voice שׁרים of singing men ושׁרות and singing women? ולמה wherefore יהיה be עבדך then should thy servant עוד yet למשׂא a burden אל unto אדני my lord המלך׃ the king?}%
\verse{כמעט will go a little way over יעבר will go a little way over עבדך Thy servant את הירדן Jordan את with המלך the king: ולמה and why יגמלני recompense המלך should the king הגמולה a reward? הזאת׃ it me with such}%
\verse{ישׁב turn back again, נא I pray thee, עבדך Let thy servant, ואמת that I may die בעירי in mine own city, עם by קבר the grave אבי of my father ואמי and of my mother. והנה But behold עבדך thy servant כמהם Chimham; יעבר let him go over עם with אדני my lord המלך the king; ועשׂה and do לו את אשׁר to him what טוב good בעיניך׃ shall seem}%
\verse{ויאמר answered, המלך And the king אתי with יעבר shall go over כמהם Chimham ואני me, and I אעשׂה will do לו את הטוב good בעיניך to him that which shall seem וכל unto thee: and whatsoever אשׁר unto thee: and whatsoever תבחר thou shalt require עלי of אעשׂה׃ me, will I do}%
\verse{ויעבר went over כל And all העם the people את הירדן Jordan. והמלך And when the king עבר was come over, וישׁק kissed המלך the king לברזלי Barzillai, ויברכהו and blessed וישׁב him; and he returned למקמו׃ unto his own place.}%
\verse{ויעבר went on המלך Then the king הגלגלה to Gilgal, וכמהן and Chimham עבר went on עמו with וכל him: and all עם the people יהודה of Judah ויעברו conducted את המלך the king, וגם and also חצי half עם the people ישׂראל׃ of Israel.}%
\verse{והנה And, behold, כל all אישׁ the men ישׂראל of Israel באים came אל to המלך the king, ויאמרו and said אל unto המלך the king, מדוע Why גנבוך stolen thee away, אחינו have our brethren אישׁ the men יהודה of Judah ויעברו and have brought את המלך the king, ואת ביתו and his household, את הירדן him, over Jordan? וכל and all אנשׁי men דוד David's עמו׃ with}%
\verse{ויען answered כל And all אישׁ the men יהודה of Judah על answered אישׁ the men ישׂראל of Israel, כי Because קרוב near of kin המלך the king אלי to ולמה us: wherefore זה us: wherefore חרה then be ye angry לך על for הדבר matter? הזה this האכול have we eaten at all אכלנו have we eaten at all מן of המלך the king's אם or נשׂאת hath he given us any gift? נשׂא׃ hath he given us any gift?}%
\verse{ויען answered אישׁ And the men ישׂראל of Israel את אישׁ the men יהודה of Judah, ויאמר and said, עשׂר We have ten ידות parts לי במלך in the king, וגם have also בדוד more in David אני and we ממך than ומדוע ye: why הקלתני then did ye despise ולא should not היה be דברי us, that our advice ראשׁון first לי להשׁיב had in bringing back את מלכי our king? ויקשׁ were fiercer דבר And the words אישׁ of the men יהודה of Judah מדבר than the words אישׁ of the men ישׂראל׃ of Israel.}%
\end{biblechapter}%
\begin{biblechapter}% 2 Samuel 20
\verseWithHeading{Sheba Leads a Revolt}{ושׁם And there נקרא happened אישׁ to be there a man בליעל of Belial, ושׁמו whose name שׁבע Sheba, בן the son בכרי of Bichri, אישׁ every man ימיני ויתקע and he blew בשׁפר a trumpet, ויאמר and said, אין We have no לנו חלק part בדוד in David, ולא neither נחלה have we inheritance לנו בבן in the son ישׁי of Jesse: אישׁ לאהליו to his tents, ישׂראל׃ O Israel.}%
\verse{ויעל went up כל So every אישׁ man ישׂראל of Israel מאחרי from after דוד David, אחרי followed שׁבע Sheba בן the son בכרי of Bichri: ואישׁ but the men יהודה of Judah דבקו cleaved במלכם unto their king, מן from after הירדן Jordan ועד even to ירושׁלם׃ Jerusalem.}%
\verse{ויבא came דוד And David אל to ביתו his house ירושׁלם at Jerusalem; ויקח took המלך and the king את עשׂר the ten נשׁים women פלגשׁים concubines, אשׁר whom הניח לשׁמר to keep הבית the house, ויתנם and put בית them in ward, משׁמרת them in ward, ויכלכלם and fed ואליהם in unto לא not בא them, but went ותהיינה them. So they were צררות shut up עד unto יום the day מתן אלמנות in widowhood. חיות׃ living}%
\verseWithHeading{Joab Assassinates Amasa}{ויאמר Then said המלך the king אל to עמשׂא Amasa, הזעק Assemble לי את אישׁ me the men יהודה of Judah שׁלשׁת within three ימים days, ואתה and be thou פה here עמד׃ present.}%
\verse{וילך went עמשׂא So Amasa להזעיק to assemble את יהודה Judah: וייחר מן than המועד the set time אשׁר which יעדו׃ he had appointed}%
\verse{ויאמר said דוד And David אל to אבישׁי Abishai, עתה Now ירע do us more harm לנו שׁבע shall Sheba בן the son בכרי of Bichri מן than אבשׁלום Absalom: אתה thou קח take את עבדי servants, אדניך thy lord's ורדף and pursue אחריו after פן him, lest מצא he get לו ערים cities, בצרות him fenced והציל and escape עיננו׃ and escape}%
\verse{ויצאו And there went out אחריו after אנשׁי men, יואב him Joab's והכרתי and the Cherethites, והפלתי and the Pelethites, וכל and all הגברים the mighty men: ויצאו and they went out מירושׁלם לרדף to pursue אחרי after שׁבע Sheba בן the son בכרי׃ of Bichri.}%
\verse{הם When they עם at האבן stone הגדולה the great אשׁר which בגבעון in Gibeon, ועמשׂא Amasa בא went לפניהם before ויואב them. And Joab's חגור was girded מדו garment לבשׁו that he had put on ועלו unto חגור חרב a sword מצמדת fastened על him, and upon מתניו his loins בתערה in the sheath והוא thereof; and as he יצא went forth ותפל׃ it fell out.}%
\verse{ויאמר said יואב And Joab לעמשׂא to Amasa, השׁלום in health, אתה thou אחי my brother? ותחז took יד hand ימין with the right יואב And Joab בזקן by the beard עמשׂא Amasa לנשׁק׃ to kiss}%
\verse{ועמשׂא But Amasa לא took no נשׁמר heed בחרב to the sword אשׁר that ביד hand: יואב in Joab's ויכהו so he smote בה אל him therewith in החמשׁ the fifth וישׁפך and shed out מעיו his bowels ארצה to the ground, ולא him not שׁנה and struck לו וימת again; and he died. ויואב So Joab ואבישׁי and Abishai אחיו his brother רדף pursued אחרי after שׁבע Sheba בן the son בכרי׃ of Bichri.}%
\verse{ואישׁ And one עמד עליו by מנערי יואב Joab, ויאמר him, and said, מי He אשׁר that חפץ favoreth ביואב Joab. ומי and he אשׁר that לדוד for David, אחרי after יואב׃}%
\verse{ועמשׂא And Amasa מתגלל wallowed בדם in blood בתוך in the midst המסלה of the highway. וירא saw האישׁ And when the man כי that עמד stood still, כל all העם the people ויסב he removed את עמשׂא Amasa מן out of המסלה the highway השׂדה into the field, וישׁלך and cast עליו upon בגד a cloth כאשׁר him, when ראה he saw כל that every הבא one that came עליו by ועמד׃ him stood still.}%
\verse{כאשׁר When הגה he was removed מן out of המסלה the highway, עבר went on כל all אישׁ the people אחרי after יואב Joab, לרדף to pursue אחרי after שׁבע Sheba בן the son בכרי׃ of Bichri.}%
\verseWithHeading{Wisdom from a Woman under Siege}{ויעבר And he went through בכל all שׁבטי the tribes ישׂראל of Israel אבלה unto Abel, ובית מעכה and to Beth-maachah, וכל and all הברים the Berites: ויקלהו and they were gathered together, ויבאו and went אף also אחריו׃ after}%
\verse{ויבאו And they came ויצרו and besieged עליו and besieged באבלה בית המעכהישׁפכו and they cast up סללה a bank אל against העיר the city, ותעמד and it stood בחל in the trench: וכל and all העם the people אשׁר that את with יואב Joab משׁחיתם battered להפיל to throw it down. החומה׃ the wall,}%
\verse{ותקרא Then cried אשׁה woman חכמה a wise מן out of העיר the city, שׁמעו Hear, שׁמעו hear; אמרו say, נא I pray you, אל unto יואב Joab, קרב Come near עד hither, הנה hither, ואדברה that I may speak אליך׃ with}%
\verse{ויקרב And when he was come near אליה unto ותאמר said, האשׁה her, the woman האתה thou יואב Joab? ויאמר And he answered, אני I ותאמר Then she said לו שׁמע unto him Hear דברי the words אמתך of thine handmaid. ויאמר And he answered, שׁמע do hear. אנכי׃ I}%
\verse{ותאמר Then she spoke, לאמר saying, דבר They were wont ידברו to speak בראשׁנה in old time, לאמר saying, שׁאל ישׁאלואבל at Abel: וכן and so התמו׃ they ended}%
\verse{אנכי I שׁלמי peaceable אמוני faithful ישׂראל in Israel: אתה thou מבקשׁ seekest להמית to destroy עיר a city ואם and a mother בישׂראל in Israel: למה why תבלע wilt thou swallow up נחלת the inheritance יהוה׃ of the LORD?}%
\verse{ויען answered יואב And Joab ויאמר and said, חלילה Far be it, חלילה far be it לי אם from me, that אבלע I should swallow up ואם or אשׁחית׃ destroy.}%
\verse{לא not כן so: הדבר The matter כי but אישׁ a man מהר of mount אפרים Ephraim, שׁבע Sheba בן the son בכרי of Bichri שׁמו by name, נשׂא hath lifted up ידו his hand במלך against the king, בדוד against David: תנו deliver אתו לבדו him only, ואלכה and I will depart מעל from העיר the city. ותאמר said האשׁה And the woman אל unto יואב Joab, הנה Behold, ראשׁו his head משׁלך shall be thrown אליך to בעד thee over החומה׃ the wall.}%
\verse{ותבוא went האשׁה Then the woman אל unto כל all העם the people בחכמתה in her wisdom. ויכרתו And they cut off את ראשׁ the head שׁבע of Sheba בן the son בכרי of Bichri, וישׁלכו and cast out אל to יואב Joab. ויתקע And he blew בשׁופר a trumpet, ויפצו and they retired מעל from העיר the city, אישׁ every man לאהליו to his tent. ויואב And Joab שׁב returned ירושׁלם to Jerusalem אל unto המלך׃ the king.}%
\verse{ויואב Now Joab אל over כל all הצבא the host ישׂראל of Israel: ובניה and Benaiah בן the son יהוידע of Jehoiada על over הכרי ועל and over הפלתי׃ the Pelethites:}%
\verse{ואדרם And Adoram על over המס the tribute: ויהושׁפט and Jehoshaphat בן the son אחילוד of Ahilud המזכיר׃ recorder:}%
\verse{ושׁיא And Sheva ספר scribe: וצדוק and Zadok ואביתר and Abiathar כהנים׃ the priests:}%
\verse{וגם also עירא And Ira היארי the Jairite היה was כהן a chief ruler לדוד׃ about David.}%
\end{biblechapter}%
\begin{biblechapter}% 2 Samuel 21
\verseWithHeading{The Famine Brings Justice}{ויהי Then there was רעב a famine בימי in the days דוד of David שׁלשׁ three שׁנים years, שׁנה year אחרי after שׁנה year; ויבקשׁ inquired דוד and David את פני inquired יהוה of the LORD. ויאמר answered, יהוה And the LORD אל for שׁאול Saul, ואל and for בית house, הדמים bloody על because אשׁר because המית he slew את הגבענים׃ the Gibeonites.}%
\verse{ויקרא called המלך And the king לגבענים the Gibeonites, ויאמר and said אליהם unto והגבענים them; (now the Gibeonites לא not מבני of the children ישׂראל of Israel, המה כי but אם but מיתר of the remnant האמרי of the Amorites; ובני and the children ישׂראל of Israel נשׁבעו had sworn להם ויבקשׁ sought שׁאול unto them: and Saul להכתם to slay בקנאתו them in his zeal לבני to the children ישׂראל of Israel ויהודה׃ and Judah.)}%
\verse{ויאמר said דוד Wherefore David אל unto הגבענים the Gibeonites, מה What אעשׂה shall I do לכם ובמה for you? and wherewith אכפר shall I make the atonement, וברכו that ye may bless את נחלת the inheritance יהוה׃ of the LORD?}%
\verse{ויאמרו said לו הגבענים And the Gibeonites אין unto him, We will have no לי כסף silver וזהב nor gold עם of שׁאול Saul, ועם nor of ביתו his house; ואין neither לנו אישׁ any man להמית for us shalt thou kill בישׂראל in Israel. ויאמר And he said, מה What אתם ye אמרים shall say, אעשׂה׃ will I do}%
\verse{ויאמרו אלמלך the king, האישׁ The man אשׁר that כלנו consumed ואשׁר us, and that דמה devised לנו נשׁמדנו against us we should be destroyed מהתיצב from remaining בכל in any גבל of the coasts ישׂראל׃ of Israel,}%
\verse{ינתן be delivered לנו שׁבעה Let seven אנשׁים men מבניו of his sons והוקענום unto us, and we will hang ליהוה them up unto the LORD בגבעת in Gibeah שׁאול of Saul, בחיר did choose. יהוה the LORD ויאמר said, המלך And the king אני I אתן׃ will give}%
\verse{ויחמל spared המלך But the king על spared מפיבשׁת Mephibosheth, בן the son יהונתן of Jonathan בן the son שׁאול of Saul, על because שׁבעת oath יהוה of the LORD's אשׁר that בינתם between בין them, between דוד David ובין יהונתן and Jonathan בן the son שׁאול׃ of Saul.}%
\verse{ויקח took המלך But the king את שׁני the two בני sons רצפה of Rizpah בת the daughter איה of Aiah, אשׁר whom ילדה she bore לשׁאול unto Saul, את ארמני Armoni ואת מפבשׁת and Mephibosheth; ואת חמשׁת and the five בני sons מיכל of Michal בת the daughter שׁאול of Saul, אשׁר whom ילדה she brought up לעדריאל for Adriel בן the son ברזלי of Barzillai המחלתי׃ the Meholathite:}%
\verse{ויתנם And he delivered ביד them into the hands הגבענים of the Gibeonites, ויקיעם and they hanged בהר them in the hill לפני before יהוה the LORD: ויפלו and they fell שׁבעתים seven יחד together, והם המתו and were put to death בימי in the days קציר of harvest, בראשׁנים in the first תחלת in the beginning קציר harvest. שׂערים׃ of barley}%
\verse{ותקח took רצפה And Rizpah בת the daughter איה of Aiah את השׂק sackcloth, ותטהו and spread לה אל it for her upon הצור the rock, מתחלת from the beginning קציר of harvest עד until נתך dropped מים water עליהם upon מן from the beginning השׁמים heaven, ולא neither נתנה and suffered עוף the birds השׁמים of the air לנוח to rest עליהם on יומם them by day, ואת חית nor the beasts השׂדה of the field לילה׃ by night.}%
\verse{ויגד And it was told לדוד David את אשׁר what עשׂתה had done. רצפה Rizpah בת the daughter איה of Aiah, פלגשׁ the concubine שׁאול׃ of Saul,}%
\verse{וילך went דוד And David ויקח and took את עצמות the bones שׁאול of Saul ואת עצמות and the bones יהונתן of Jonathan בנו his son מאת בעלי the men יבישׁ גלעדשׁר which גנבו had stolen אתם מרחב them from the street בית שׁן of Beth-shan, אשׁר where תלום שׁם where הפלשׁתים the Philistines ביום them, when הכות had slain פלשׁתים the Philistines את שׁאול Saul בגלבע׃ in Gilboa:}%
\verse{ויעל And he brought up משׁם from thence את עצמות the bones שׁאול of Saul ואת עצמות and the bones יהונתן of Jonathan בנו his son; ויאספו and they gathered את עצמות the bones המוקעים׃ of them that were hanged.}%
\verse{ויקברו buried את עצמות And the bones שׁאול of Saul ויהונתן and Jonathan בנו his son בארץ they in the country בנימן of Benjamin בצלע in Zelah, בקבר in the sepulcher קישׁ of Kish אביו his father: ויעשׂו and they performed כל all אשׁר that צוה commanded. המלך the king ויעתר was entreated אלהים God לארץ for the land. אחרי כן׃}%
\verseWithHeading{Battles with the Philistines Recounted}{ותהי had עוד מלחמה yet war לפלשׁתים Moreover the Philistines את again with ישׂראל Israel; וירד went down, דוד and David ועבדיו and his servants עמו with וילחמו him, and fought against את פלשׁתים the Philistines: ויעף waxed faint. דוד׃ and David}%
\verse{וישׁבו בנב And Ishbibenob, אשׁר which בילידי of the sons הרפה ומשׁקל the weight קינו of whose spear שׁלשׁ three מאות hundred משׁקל in weight, נחשׁת of brass והוא he חגור being girded חדשׁה with a new ויאמר thought להכות to have slain את דוד׃ David.}%
\verse{ויעזר succored לו אבישׁי But Abishai בן the son צרויה of Zeruiah ויך him, and smote את הפלשׁתי the Philistine, וימיתהו and killed אז him. Then נשׁבעו swore אנשׁי the men דוד of David לו לאמר unto him, saying, לא no תצא Thou shalt go עוד more אתנו למלחמה out with us to battle, ולא not תכבה that thou quench את נר the light ישׂראל׃ of Israel.}%
\verse{ויהי And it came to pass אחרי after כן this, ותהי that there was עוד again המלחמה a battle בגוב at Gob: עם with פלשׁתים the Philistines אז then הכה slew סבכי Sibbechai החשׁתי the Hushathite את סף Saph, אשׁר which בילדי of the sons הרפה׃}%
\verse{ותהי And there was עוד again המלחמה a battle בגוב in Gob עם with פלשׁתים the Philistines, ויך slew אלחנן where Elhanan בן the son יערי ארגים of Jaare-oregim, בית הלחמי a Bethlehemite, את גלית Goliath הגתי the Gittite, ועץ the staff חניתו of whose spear כמנור beam. ארגים׃ like a weaver's}%
\verse{ותהי And there was עוד yet מלחמה a battle בגת in Gath, ויהי where was אישׁ a man מדין ואצבעת fingers, ידיו that had on every hand ואצבעות toes, רגליו and on every foot שׁשׁ six ושׁשׁ six עשׂרים and twenty וארבע four מספר in number; וגם also הוא and he ילד was born להרפה׃}%
\verse{ויחרף And when he defied את ישׂראל Israel, ויכהו slew יהונתן Jonathan בן the son שׁמעי אחי the brother דוד׃ of David}%
\verse{את ארבעת four אלה These ילדו were born להרפה בגת in Gath, ויפלו and fell ביד by the hand דוד of David, וביד and by the hand עבדיו׃ of his servants.}%
\end{biblechapter}%
\begin{biblechapter}% 2 Samuel 22
\verseWithHeading{The Victory Song of David}{וידבר spoke דוד And David ליהוה unto the LORD את דברי the words השׁירה song הזאת of this ביום in the day הציל had delivered יהוה the LORD אתו מכף him out of the hand כל of all איביו his enemies, ומכף and out of the hand שׁאול׃ of Saul:}%
\verse{ויאמר And he said, יהוה The LORD סלעי my rock, ומצדתי and my fortress, ומפלטי׃ and my deliverer;}%
\verse{אלהי The God צורי of my rock; אחסה in him will I trust: בו מגני my shield, וקרן and the horn ישׁעי of my salvation, משׂגבי my high tower, ומנוסי and my refuge, משׁעי my savior; מחמס me from violence. תשׁעני׃ thou savest}%
\verse{מהלל worthy to be praised: אקרא I will call on יהוה the LORD, ומאיבי from mine enemies. אושׁע׃ so shall I be saved}%
\verse{כי When אפפני compassed משׁברי the waves מות of death נחלי me, the floods בליעל of ungodly יבעתני׃ men made me afraid;}%
\verse{חבלי The sorrows שׁאול of hell סבני compassed me about; קדמני prevented מקשׁי the snares מות׃ of death}%
\verse{בצר In my distress לי אקרא I called upon יהוה the LORD, ואל to אלהי my God: אקרא and cried וישׁמע and he did hear מהיכלו out of his temple, קולי my voice ושׁועתי and my cry באזניו׃ into his ears.}%
\verse{ותגעשׁ shook ותרעשׁ and trembled; הארץ Then the earth מוסדות the foundations השׁמים of heaven ירגזו moved ויתגעשׁו and shook, כי because חרה׃ he was wroth.}%
\verse{עלה There went up עשׁן a smoke באפו out of his nostrils, ואשׁ and fire מפיו out of his mouth תאכל devoured: גחלים coals בערו were kindled ממנו׃ out of his mouth}%
\verse{ויט He bowed שׁמים the heavens וירד also, and came down; וערפל and darkness תחת under רגליו׃ his feet.}%
\verse{וירכב And he rode על upon כרוב a cherub, ויעף and did fly: וירא and he was seen על upon כנפי the wings רוח׃ of the wind.}%
\verse{וישׁת And he made חשׁך darkness סביבתיו round about סכות pavilions חשׁרת him, dark מים waters, עבי thick clouds שׁחקים׃ of the skies.}%
\verse{מנגה נגדו before בערו kindled. גחלי him were coals אשׁ׃ of fire}%
\verse{ירעם thundered מן from שׁמים heaven, יהוה ועליון and the most High יתן uttered קולו׃ his voice.}%
\verse{וישׁלח And he sent out חצים arrows, ויפיצם and scattered ברק them; lightning, ויהמם׃ and discomfited}%
\verse{ויראו appeared, אפקי And the channels ים of the sea יגלו were discovered, מסדות the foundations תבל of the world בגערת at the rebuking יהוה of the LORD, מנשׁמת at the blast רוח of the breath אפו׃ of his nostrils.}%
\verse{ישׁלח He sent ממרום from above, יקחני he took ימשׁני me; he drew ממים רבים׃}%
\verse{יצילני He delivered מאיבי עזשׂנאי from them that hated כי me: for אמצו they were too strong ממני׃ me from my strong enemy,}%
\verse{יקדמני They prevented ביום me in the day אידי of my calamity: ויהי was יהוה but the LORD משׁען׃ my stay.}%
\verse{ויצא He brought me forth למרחב אתיחלצני he delivered כי me, because חפץ׃ he delighted}%
\verse{יגמלני rewarded יהוה The LORD כצדקתי me according to my righteousness: כבר according to the cleanness ידי of my hands ישׁיב׃ hath he recompensed}%
\verse{כי For שׁמרתי I have kept דרכי the ways יהוה of the LORD, ולא and have not רשׁעתי wickedly מאלהי׃}%
\verse{כי For כל all משׁפטו his judgments לנגדי before וחקתיו me: and his statutes, לא I did not אסור depart ממנה׃ from}%
\verse{ואהיה I was תמים also upright לו ואשׁתמרה before him and have kept myself מעוני׃ from mine iniquity.}%
\verse{וישׁב hath recompensed יהוה Therefore the LORD לי כצדקתי me according to my righteousness; כברי according to my cleanness לנגד in עיניו׃ his eye sight.}%
\verse{עם With חסיד the merciful תתחסד thou wilt show thyself merciful, עם with גבור man תמים the upright תתמם׃ thou wilt show thyself upright.}%
\verse{עם With נבר thou wilt show thyself pure; תתבר ועם and with עקשׁ the froward תתפל׃ thou wilt show thyself unsavory.}%
\verse{ואת עם people עני And the afflicted תושׁיע thou wilt save: ועיניך but thine eyes על upon רמים the haughty, תשׁפיל׃ thou mayest bring down.}%
\verse{כי For אתה thou נירי my lamp, יהוה O LORD: ויהוה and the LORD יגיה will lighten חשׁכי׃ my darkness.}%
\verse{כי For בכה ארוץ by thee I have run גדוד through a troop: באלהי by my God אדלג have I leaped שׁור׃ over a wall.}%
\verse{האל God, תמים perfect; דרכו his way אמרת the word יהוה of the LORD צרופה tried: מגן a buckler הוא he לכל to all החסים׃ them that trust}%
\verse{כי For מי who אל God, מבלעדי save יהוה the LORD? ומי and who צור a rock, מבלעדי save אלהינו׃ our God?}%
\verse{האל God מעוזי my strength חיל power: ויתר and he maketh תמים perfect. דרכו׃ my way}%
\verse{משׁוה He maketh רגליו my feet כאילות like hinds' ועל me upon במותי my high places. יעמדני׃ and setteth}%
\verse{מלמד He teacheth ידי my hands למלחמה to war; ונחת is broken קשׁת so that a bow נחושׁה of steel זרעתי׃ by mine arms.}%
\verse{ותתן Thou hast also given לי מגן me the shield ישׁעך of thy salvation: וענתך תרבני׃ hath made me great.}%
\verse{תרחיב Thou hast enlarged צעדי my steps תחתני under ולא did not מעדו slip. קרסלי׃ me; so that my feet}%
\verse{ארדפה I have pursued איבי mine enemies, ואשׁמידם and destroyed ולא them; and turned not again אשׁוב them; and turned not again עד until כלותם׃ I had consumed}%
\verse{ואכלם ואמחצם them, and wounded ולא them, that they could not יקומון arise: ויפלו yea, they are fallen תחת under רגלי׃ my feet.}%
\verse{ותזרני For thou hast girded חיל me with strength למלחמה to battle: תכריע against me hast thou subdued קמי them that rose up תחתני׃ under}%
\verse{ואיבי of mine enemies, תתה Thou hast also given לי ערף me the necks משׂנאי them that hate ואצמיתם׃ that I might destroy}%
\verse{ישׁעו They looked, ואין but none משׁיע to save; אל unto יהוה the LORD, ולא them not. ענם׃ but he answered}%
\verse{ואשׁחקם Then did I beat כעפר them as small as the dust ארץ of the earth, כטיט them as the mire חוצות of the street, אדקם I did stamp ארקעם׃ did spread them abroad.}%
\verse{ותפלטני Thou also hast delivered מריבי me from the strivings עמי of my people, תשׁמרני thou hast kept לראשׁ me head גוים of the heathen: עם a people לא not ידעתי I knew יעבדני׃ shall serve}%
\verse{בני נכרתכחשׁו shall submit themselves לי לשׁמוע unto me: as soon as they hear, אזן unto me: as soon as they hear, ישׁמעו׃ they shall be obedient}%
\verse{בני נכרבלו shall fade away, ויחגרו and they shall be afraid ממסגרותם׃ out of their close places.}%
\verse{חי liveth; יהוה The LORD וברוך and blessed צורי my rock; וירם and exalted אלהי be the God צור of the rock ישׁעי׃ of my salvation.}%
\verse{האל It God הנתן that avengeth נקמת that avengeth לי ומוריד עמים the people תחתני׃ under}%
\verse{ומוציאי And that bringeth me forth מאיבי from mine enemies: ומקמי above them that rose up תרוממני thou also hast lifted me up on high מאישׁ חמסיםצילני׃ against me: thou hast delivered}%
\verse{על כןודך I will give thanks יהוה unto thee, O LORD, בגוים among the heathen, ולשׁמך unto thy name. אזמר׃ and I will sing praises}%
\verse{מגדיל ישׁועות of salvation מלכו for his king: ועשׂה and showeth חסד mercy למשׁיחו to his anointed, לדוד unto David, ולזרעו and to his seed עד forevermore. עולם׃ forevermore.}%
\end{biblechapter}%
\begin{biblechapter}% 2 Samuel 23
\verseWithHeading{David Extols Adonai}{ואלה Now these דברי words דוד of David האחרנים the last נאם said, דוד בן the son ישׁי of Jesse ונאם הגבר and the man הקם raised up על on high, משׁיח the anointed אלהי of the God יעקב of Jacob, ונעים and the sweet זמרות psalmist ישׂראל׃ of Israel,}%
\verse{רוח The Spirit יהוה of the LORD דבר spoke בי ומלתו by me, and his word על in לשׁוני׃ my tongue.}%
\verse{אמר said, אלהי The God ישׂראל of Israel לי דבר spoke צור the Rock ישׂראל of Israel מושׁל to me, He that ruleth over באדם men צדיק just, מושׁל ruling יראת in the fear אלהים׃ of God.}%
\verse{וכאור And as the light בקר of the morning, יזרח riseth, שׁמשׁ the sun בקר a morning לא without עבות clouds; מנגה by clear shining ממטר after rain. דשׁא the tender grass מארץ׃ out of the earth}%
\verse{כי Although לא not כן so ביתי my house עם with אל God; כי yet ברית covenant, עולם with me an everlasting שׂם he hath made לי ערוכה ordered בכל in all ושׁמרה and sure: כי for כל all ישׁעי my salvation, וכל and all חפץ desire, כי although לא he make not יצמיח׃ to grow.}%
\verse{ובליעל But of Belial כקוץ of them as thorns מנד thrust away, כלהם all כי because לא they cannot ביד with hands: יקחו׃ be taken}%
\verse{ואישׁ But the man יגע shall touch בהם ימלא them must be fenced ברזל with iron ועץ and the staff חנית of a spear; ובאשׁ with fire שׂרוף and they shall be utterly burned ישׂרפו and they shall be utterly burned בשׁבת׃ in the place.}%
\verseWithHeading{David’s Faithful Soldiers}{אלה These שׁמות the names הגברים of the mighty men אשׁר whom לדוד David ישׁב that sat בשׁבת in the seat, תחכמני had: The Tachmonite ראשׁ chief השׁלשׁי among the captains; הוא the same עדינו Adino העצנו the Eznite: על against שׁמנה eight מאות hundred, חלל whom he slew בפעם time. אחד׃ at one}%
\verse{ואחרו And after אלעזר him Eleazar בן the son דדי of Dodo בן אחחישׁלשׁה of the three גברים mighty men עם with דוד David, בחרפם when they defied בפלשׁתים the Philistines נאספו gathered together שׁם were there למלחמה to battle, ויעלו were gone away: אישׁ and the men ישׂראל׃ of Israel}%
\verse{הוא He קם arose, ויך and smote בפלשׁתים the Philistines עד until כי until יגעה was weary, ידו his hand ותדבק cleaved ידו and his hand אל unto החרב the sword: ויעשׂ wrought יהוה and the LORD תשׁועה victory גדולה a great ביום day; ההוא that והעם and the people ישׁבו returned אחריו after אך him only לפשׁט׃ to spoil.}%
\verse{ואחריו And after שׁמא בן the son אגא of Agee הררי the Hararite. ויאספו were gathered together פלשׁתים And the Philistines לחיה into a troop, ותהי was שׁם where חלקת a piece השׂדה of ground מלאה עדשׁים of lentils: והעם and the people נס fled מפני from פלשׁתים׃ the Philistines.}%
\verse{ויתיצב But he stood בתוך in the midst החלקה of the ground, ויצילה and defended ויך it, and slew את פלשׁתים the Philistines: ויעשׂ wrought יהוה and the LORD תשׁועה victory. גדולה׃ a great}%
\verse{וירדו went down, שׁלשׁים of the thirty מהשׁלשׁים ראשׁ chief ויבאו and came אל to קציר the harvest time אל in דוד David אל unto מערת the cave עדלם of Adullam: וחית and the troop פלשׁתים of the Philistines חנה pitched בעמק in the valley רפאים׃ of Rephaim.}%
\verse{ודוד And David אז then במצודה in a hold, ומצב and the garrison פלשׁתים of the Philistines אז then בית לחם׃ Bethlehem.}%
\verse{ויתאוה longed, דוד And David ויאמר and said, מי Oh that one ישׁקני would give me drink מים of the water מבאר בית לחם of Bethlehem, אשׁר which בשׁער׃ by the gate!}%
\verse{ויבקעו broke through שׁלשׁת And the three הגברים mighty men במחנה the host פלשׁתים of the Philistines, וישׁאבו and drew מים water מבאר בית לחם of Bethlehem, אשׁר that בשׁער by the gate, וישׂאו and took ויבאו and brought אל to דוד David: ולא not אבה nevertheless he would לשׁתותם drink ויסך אתםיהוה׃ unto the LORD.}%
\verse{ויאמר And he said, חלילה Be it far לי יהוה from me, O LORD, מעשׂתי things did זאת this: הדם the blood האנשׁים ההלכים that went בנפשׁותם in jeopardy of their lives? ולא not אבה therefore he would לשׁתותם drink אלה it. These עשׂו שׁלשׁת these three הגברים׃ mighty men.}%
\verse{ואבישׁי And Abishai, אחי the brother יואב of Joab, בן the son צרויה of Zeruiah, הוא And he ראשׁ was chief השׁלשׁי among three. והוא עורר lifted up את חניתו his spear על against שׁלשׁ three מאות hundred, חלל slew ולו שׁם and had the name בשׁלשׁה׃ among three.}%
\verse{מן most השׁלשׁה of three? הכי Was he not נכבד honorable ויהי therefore he was להם לשׂר their captain: ועד unto השׁלשׁה the three. לא not בא׃ howbeit he attained}%
\verse{ובניהו And Benaiah בן the son יהוידע of Jehoiada, בן the son אישׁ man, חי רב who had done many acts, פעלים who had done many acts, מקבצאל הוא he הכה slew את שׁני two אראל lionlike men מואב of Moab: והוא he ירד went down והכה also and slew את האריה a lion בתוך in the midst הבאר of a pit ביום in time השׁלג׃ of snow:}%
\verse{והוא And he הכה slew את אישׁ man: מצרי and the Egyptian אשׁר מראה a goodly וביד in his hand; המצרי חנית had a spear וירד but he went down אליו to בשׁבט him with a staff, ויגזל and plucked את החנית the spear מיד המצרייהרגהו בחניתו׃ him with his own spear.}%
\verse{אלה These עשׂה did בניהו Benaiah בן the son יהוידע of Jehoiada, ולו שׁם and had the name בשׁלשׁה among three הגברים׃ mighty men.}%
\verse{מן than השׁלשׁים the thirty, נכבד He was more honorable ואל to השׁלשׁה the three. לא not בא but he attained וישׂמהו set דוד And David אל him over משׁמעתו׃ his guard.}%
\verseWithHeading{The Mighty Men of David}{עשׂה אל Asahel אחי the brother יואב of Joab בשׁלשׁים one of the thirty; אלחנן Elhanan בן the son דדו of Dodo בית לחם׃ of Bethlehem,}%
\verse{שׁמה Shammah החרדי the Harodite, אליקא Elika החרדי׃ the Harodite,}%
\verse{חלץ Helez הפלטי the Paltite, עירא Ira בן the son עקשׁ of Ikkesh התקועי׃ the Tekoite,}%
\verse{אביעזר Abiezer הענתתי the Anethothite, מבני Mebunnai החשׁתי׃ the Hushathite,}%
\verse{צלמון Zalmon האחחי the Ahohite, מהרי Maharai הנטפתי׃ the Netophathite,}%
\verse{חלב Heleb בן the son בענה of Baanah, הנטפתי a Netophathite, אתי Ittai בן the son ריבי of Ribai מגבעת בני of the children בנימן׃ of Benjamin,}%
\verse{בניהו Benaiah פרעתני the Pirathonite, הדי Hiddai מנחלי of the brooks געשׁ׃ of Gaash,}%
\verse{אבי עלבון Abi-albon הערבתי the Arbathite, עזמות Azmaveth הברחמי׃ the Barhumite,}%
\verse{אליחבא Eliahba השׁעלבני the Shaalbonite, בני of the sons ישׁן of Jashen, יהונתן׃ Jonathan,}%
\verse{שׁמה Shammah ההררי the Hararite, אחיאם Ahiam בן the son שׁרר of Sharar האררי׃ the Hararite,}%
\verse{אליפלט Eliphelet בן the son אחסבי of Ahasbai, בן the son המעכתי of the Maachathite, אליעם Eliam בן the son אחיתפל of Ahithophel הגלני׃ the Gilonite,}%
\verse{חצרו Hezrai הכרמלי the Carmelite, פערי Paarai הארבי׃ the Arbite,}%
\verse{יגאל Igal בן the son נתן of Nathan מצבה בני Bani הגדי׃ the Gadite,}%
\verse{צלק Zelek העמני the Ammonite, נחרי Naharai הבארתי the Beerothite, נשׂאי armorbearer כלי armorbearer יואב to Joab בן the son צריה׃ of Zeruiah,}%
\verse{עירא Ira היתרי an Ithrite, גרב Gareb היתרי׃ an Ithrite,}%
\verse{אוריה Uriah החתי the Hittite: כל in all. שׁלשׁים thirty ושׁבעה׃ and seven}%
\end{biblechapter}%
\begin{biblechapter}% 2 Samuel 24
\verseWithHeading{David and the Census of the People}{ויסף And again אף the anger יהוה of the LORD לחרות was kindled בישׂראל against Israel, ויסת and he moved את דוד David בהם לאמר against them to say, לך Go, מנה number את ישׂראל Israel ואת יהודה׃ and Judah.}%
\verse{ויאמר said המלך For the king אל to יואב Joab שׂר the captain החיל of the host, אשׁר which אתו with שׁוט him, Go נא now בכל through all שׁבטי the tribes ישׂראל of Israel, מדן ועד even to באר שׁבע Beer-sheba, ופקדו and number את העם ye the people, וידעתי that I may know את מספר the number העם׃ of the people.}%
\verse{ויאמר said יואב And Joab אל unto המלך the king, ויוסף add יהוה Now the LORD אלהיך thy God אל unto העם the people, כהם how many soever they be וכהם how many soever they be מאה a hundredfold, פעמים a hundredfold, ועיני and that the eyes אדני of my lord המלך the king ראות may see ואדני doth my lord המלך the king למה but why חפץ delight בדבר thing? הזה׃ in this}%
\verse{ויחזק prevailed דבר word המלך Notwithstanding the king's אל against יואב Joab, ועל and against שׂרי the captains החיל of the host. ויצא went out יואב And Joab ושׂרי and the captains החיל of the host לפני from the presence המלך of the king, לפקד to number את העם the people את ישׂראל׃ of Israel.}%
\verse{ויעברו And they passed over את הירדן Jordan, ויחנו and pitched בערוער in Aroer, ימין on the right side העיר of the city אשׁר that בתוך in the midst הנחל of the river הגד of Gad, ואל and toward יעזר׃ Jazer:}%
\verse{ויבאו Then they came הגלעדה to Gilead, ואל and to ארץ the land תחתים חדשׁי of Tahtim-hodshi; ויבאו and they came דנה יען to Dan-jaan, וסביב and about אל to צידון׃ Zidon,}%
\verse{ויבאו And came מבצר to the stronghold צר of Tyre, וכל and to all ערי the cities החוי of the Hivites, והכנעני and of the Canaanites: ויצאו and they went out אל נגב the south יהודה Judah, באר שׁבע׃ to Beer-sheba.}%
\verse{וישׁטו So when they had gone בכל through all הארץ the land, ויבאו they came מקצה at the end תשׁעה of nine חדשׁים months ועשׂרים and twenty יום days. ירושׁלם׃ to Jerusalem}%
\verse{ויתן gave up יואב And Joab את מספר the sum מפקד of the number העם of the people אל unto המלך the king: ותהי and there were ישׂראל in Israel שׁמנה eight מאות hundred אלף thousand אישׁ men חיל valiant שׁלף that drew חרב the sword; ואישׁ and the men יהודה of Judah חמשׁ five מאות hundred אלף thousand אישׁ׃ men.}%
\verse{ויך smote לב heart דוד And David's אתו אחרי him after כן that ספר he had numbered את העם the people. ויאמר said דוד And David אל unto יהוה the LORD, חטאתי I have sinned מאד greatly אשׁר in that עשׂיתי I have done: ועתה and now, יהוה O LORD, העבר take away נא I beseech thee, את עון the iniquity עבדך of thy servant; כי for נסכלתי foolishly. מאד׃ I have done very}%
\verse{ויקם was up דוד For when David בבקר in the morning, ודבר the word יהוה of the LORD היה came אל unto גד Gad, הנביא the prophet חזה seer, דוד David's לאמר׃ saying,}%
\verse{הלוך Go ודברת saith אל unto דוד David, כה Thus אמר and say יהוה the LORD, שׁלשׁ thee three אנכי I נוטל offer עליך offer בחר choose לך אחת thee one מהם ואעשׂה׃ them, that I may}%
\verse{ויבא came גד So Gad אל to דוד David, ויגד and told לו ויאמר him, and said לו התבוא come לך שׁבע unto him, Shall seven שׁנים years רעב of famine בארצך unto thee in thy land? אם or שׁלשׁה three חדשׁים months נסך wilt thou flee לפני before צריך thine enemies, והוא while they רדפך pursue ואם thee? or היות that there be שׁלשׁת three ימים days' דבר pestilence בארצך in thy land? עתה now דע advise, וראה and see מה what אשׁיב I shall return שׁלחי to him that sent דבר׃ answer}%
\verse{ויאמר said דוד And David אל unto גד Gad, צר strait: לי מאד I am in a great נפלה let us fall נא now ביד into the hand יהוה of the LORD; כי for רבים great: רחמו his mercies וביד into the hand אדם of man. אל and let me not אפלה׃ fall}%
\verse{ויתן sent יהוה So the LORD דבר a pestilence בישׂראל upon Israel מהבקר from the morning ועד even to עת the time מועד appointed: וימת and there died מן of העם the people מדן ועד even to באר שׁבע Beer-sheba שׁבעים seventy אלף thousand אישׁ׃ men.}%
\verse{וישׁלח stretched out ידו his hand המלאך And when the angel ירושׁלם upon Jerusalem לשׁחתה to destroy וינחם repented him יהוה it, the LORD אל of הרעה the evil, ויאמר and said למלאך to the angel המשׁחית that destroyed בעם the people, רב It is enough: עתה now הרף stay ידך thine hand. ומלאך And the angel יהוה of the LORD היה was עם by גרן the threshingplace האורנה of Araunah היבסי׃ the Jebusite.}%
\verse{ויאמר spoke דוד And David אל unto יהוה the LORD בראתו when he saw את המלאך the angel המכה that smote בעם the people, ויאמר and said, הנה Lo, אנכי I חטאתי have sinned, ואנכי and I העויתי have done wickedly: ואלה but these הצאן sheep, מה what עשׂו have they done? תהי be נא I pray thee, ידך let thine hand, בי ובבית house. אבי׃ against me, and against my father's}%
\verse{ויבא came גד And Gad אל to דוד David, ביום day ההוא that ויאמר and said לו עלה unto him, Go up, הקם rear ליהוה unto the LORD מזבח an altar בגרן in the threshingfloor ארניה of Araunah היבסי׃ the Jebusite.}%
\verse{ויעל went up דוד And David, כדבר according to the saying גד of Gad, כאשׁר as צוה commanded. יהוה׃ the LORD}%
\verse{וישׁקף looked, ארונה And Araunah וירא and saw את המלך the king ואת עבדיו and his servants עברים coming on עליו toward ויצא went out, ארונה him: and Araunah וישׁתחו and bowed himself למלך before the king אפיו on his face ארצה׃ upon the ground.}%
\verse{ויאמר said, ארונה And Araunah מדוע Wherefore בא come אדני is my lord המלך the king אל to עבדו his servant? ויאמר said, דוד And David לקנות To buy מעמך of את הגרן the threshingfloor לבנות thee, to build מזבח an altar ליהוה unto the LORD, ותעצר may be stayed המגפה that the plague מעל from העם׃ the people.}%
\verse{ויאמר said ארונה And Araunah אל unto דוד David, יקח take ויעל and offer up אדני Let my lord המלך the king הטוב what good בעינו unto him: ראה behold, הבקר oxen לעלה for burnt sacrifice, והמרגים and threshing instruments וכלי and instruments הבקר of the oxen לעצים׃ for wood.}%
\verse{הכל All נתן give ארונה these did Araunah, המלך a king, למלך the king. ויאמר said ארונה And Araunah אל unto המלך unto the king, יהוה The LORD אלהיך thy God ירצך׃ accept}%
\verse{ויאמר said המלך And the king אל unto ארונה Araunah, לא Nay; כי but קנו bought אקנה מאותךמחיר thee at a price: ולא neither אעלה will I offer ליהוה unto the LORD אלהי my God עלות burnt offerings חנם doth cost me nothing. ויקן דוד So David את הגרן the threshingfloor ואת הבקר and the oxen בכסף of silver. שׁקלים shekels חמשׁים׃ for fifty}%
\verse{ויבן built שׁם there דוד And David מזבח an altar ליהוה unto the LORD, ויעל and offered עלות burnt offerings ושׁלמים and peace offerings. ויעתר was entreated יהוה So the LORD לארץ for the land, ותעצר was stayed המגפה and the plague מעל from ישׂראל׃ Israel.}%
\end{biblechapter}%
\flushcolsend
\input{leb/content/old-testament/1Kgs.tex}\flushcolsend
\input{leb/content/old-testament/2Kgs.tex}\flushcolsend
\input{leb/content/old-testament/1Chr.tex}\flushcolsend
\input{leb/content/old-testament/2Chr.tex}\flushcolsend
\input{leb/content/old-testament/Ezr.tex}\flushcolsend
\biblebook{Nehemiah}
\begin{biblechapter}% Nehemiah 1
\verseWithHeading{Nehemiah’s Prayer for the People}{דברי The words נחמיה of Nehemiah בן the son חכליה of Hachaliah. ויהי And it came to pass בחדשׁ in the month כסלו Chisleu, שׁנת year, עשׂרים in the twentieth ואני as I הייתי was בשׁושׁן in Shushan הבירה׃ the palace,}%
\verse{ויבא came, חנני That Hanani, אחד one מאחי of my brethren, הוא he ואנשׁים and men מיהודה ואשׁאלם and I asked על them concerning היהודים the Jews הפליטה that had escaped, אשׁר which נשׁארו were left מן of my brethren, השׁבי the captivity, ועל and concerning ירושׁלם׃ Jerusalem.}%
\verse{ויאמרו And they said לי הנשׁארים unto me, The remnant אשׁר that נשׁארו are left מן of השׁבי the captivity שׁם there במדינה in the province ברעה affliction גדלה in great ובחרפה and reproach: וחומת the wall ירושׁלם of Jerusalem מפרצת also broken down, ושׁעריה and the gates נצתו thereof are burned באשׁ׃ with fire.}%
\verse{ויהי And it came to pass, כשׁמעי when I heard את הדברים words, האלה these ישׁבתי that I sat down ואבכה and wept, ואתאבלה and mourned ימים days, ואהי and fasted, צם and fasted, ומתפלל and prayed לפני before אלהי the God השׁמים׃ of heaven,}%
\verse{ואמר And said, אנא I beseech יהוה אלהי God השׁמים of heaven, האל God, הגדול the great והנורא and terrible שׁמר that keepeth הברית covenant וחסד and mercy לאהביו for them that love ולשׁמרי him and observe מצותיו׃ his commandments:}%
\verse{תהי be נא now אזנך Let thine ear קשׁבת attentive, ועיניך and thine eyes פתוחות open, לשׁמע that thou mayest hear אל that thou mayest hear תפלת the prayer עבדך of thy servant, אשׁר which אנכי I מתפלל pray לפניך before היום thee now, יומם day ולילה and night, על for בני the children ישׂראל of Israel עבדיך thy servants, ומתודה and confess על and confess חטאות the sins בני of the children ישׂראל of Israel, אשׁר which חטאנו we have sinned לך ואני against thee: both I ובית house אבי and my father's חטאנו׃ have sinned.}%
\verse{חבל חבלנוך ולא against thee, and have not שׁמרנו kept את המצות the commandments, ואת החקים nor the statutes, ואת המשׁפטים nor the judgments, אשׁר which צוית thou commandedst את משׁה Moses. עבדך׃ thy servant}%
\verse{זכר Remember, נא I beseech את הדבר thee, the word אשׁר that צוית thou commandedst את משׁה Moses, עבדך thy servant לאמר saying, אתם ye תמעלו transgress, אני I אפיץ אתכםעמים׃ among the nations:}%
\verse{ושׁבתם But ye turn אלי unto ושׁמרתם me, and keep מצותי my commandments, ועשׂיתם and do אתם אם them; though יהיה there were נדחכם of you cast out בקצה unto the uttermost part השׁמים of the heaven, משׁם them from thence, אקבצם will I gather והבואתים and will bring אל them unto המקום the place אשׁר that בחרתי I have chosen לשׁכן to set את שׁמי my name שׁם׃ there.}%
\verse{והם Now these עבדיך thy servants ועמך and thy people, אשׁר whom פדית thou hast redeemed בכחך power, הגדול by thy great ובידך hand. החזקה׃ and by thy strong}%
\verse{אנא I beseech אדני O Lord, תהי be נא thee, let now אזנך thine ear קשׁבת attentive אל to תפלת the prayer עבדך of thy servant, ואל and to תפלת the prayer עבדיך of thy servants, החפצים who desire ליראה to fear את שׁמך thy name: והצליחה and prosper, נא I pray לעבדך thee, thy servant היום this day, ותנהו and grant לרחמים him mercy לפני in the sight האישׁ man. הזה of this ואני For I הייתי was משׁקה cupbearer. למלך׃ the king's}%
\end{biblechapter}%
\begin{biblechapter}% Nehemiah 2
\verseWithHeading{Nehemiah Sent}{ויהי And it came to pass בחדשׁ in the month ניסן Nisan, שׁנת year עשׂרים in the twentieth לארתחשׁסתא of Artaxerxes המלך the king, יין wine לפניו before ואשׂא him: and I took up את היין the wine, ואתנה and gave למלך unto the king. ולא Now I had not הייתי been רע sad לפניו׃ in his presence.}%
\verse{ויאמר said לי המלך Wherefore the king מדוע unto me, Why פניך thy countenance רעים sad, ואתה seeing thou אינך not חולה sick? אין nothing זה this כי but אם but רע sorrow לב of heart. ואירא afraid, הרבה sore מאד׃ Then I was very}%
\verse{ואמר And said למלך unto the king, המלך Let the king לעולם forever: יחיה live מדוע why לא should not ירעו be sad, פני my countenance אשׁר when העיר the city, בית the place קברות sepulchers, אבתי of my fathers' חרבה ושׁעריה and the gates אכלו thereof are consumed באשׁ׃ with fire?}%
\verse{ויאמר said לי המלך Then the king על unto me, For מה what זה what אתה dost thou מבקשׁ make request? ואתפלל So I prayed אל to אלהי the God השׁמים׃ of heaven.}%
\verse{ואמר And I said למלך unto the king, אם If על it please המלך the king, טוב it please ואם and if ייטב have found favor עבדך thy servant לפניך in thy sight, אשׁר that תשׁלחני thou wouldest send אל me unto יהודה Judah, אל unto עיר the city קברות sepulchers, אבתי of my fathers' ואבננה׃ that I may build}%
\verse{ויאמר said לי המלך And the king והשׁגל unto me, (the queen יושׁבת also sitting אצלו by עד him,) For מתי how long יהיה be? מהלכך shall thy journey ומתי and when תשׁוב wilt thou return? וייטב לפנימלך the king וישׁלחני to send ואתנה me; and I set לו זמן׃ him a time.}%
\verse{ואומר Moreover I said למלך unto the king, אם If על it please המלך the king, טוב it please אגרות let letters יתנו be given לי על me to פחוות the governors עבר beyond הנהר the river, אשׁר that יעבירוני they may convey me over עד till אשׁר till אבוא I come אל into יהודה׃ Judah;}%
\verse{ואגרת And a letter אל unto אסף Asaph שׁמר the keeper הפרדס forest, אשׁר that למלך of the king's אשׁר which יתן he may give לי עצים me timber לקרות to make beams את שׁערי for the gates הבירה of the palace אשׁר that לבית to the house, ולחומת and for the wall העיר of the city, ולבית and for the house אשׁר אבוא I shall enter אליו into. ויתן granted לי המלך And the king כיד hand אלהי of my God הטובה me, according to the good עלי׃ upon}%
\verse{ואבוא Then I came אל to פחוות the governors עבר beyond הנהר the river, ואתנה and gave להם את אגרות letters. המלך them the king's וישׁלח had sent עמי with המלך Now the king שׂרי captains חיל of the army ופרשׁים׃ and horsemen}%
\verse{וישׁמע heard סנבלט When Sanballat החרני the Horonite, וטוביה and Tobiah העבד the servant, העמני the Ammonite, וירע it grieved להם רעה them exceedingly גדלה them exceedingly אשׁר that בא there was come אדם a man לבקשׁ to seek טובה the welfare לבני of the children ישׂראל׃ of Israel.}%
\verseWithHeading{Nehemiah Inspects the Walls and Decides to Restore Them}{ואבוא So I came אל to ירושׁלם Jerusalem, ואהי and was שׁם there ימים days. שׁלשׁה׃ three}%
\verse{ואקום And I arose לילה in the night, אני I ואנשׁים men מעט and some few עמי with ולא me; neither הגדתי told לאדם I man מה what אלהי my God נתן had put אל in לבי my heart לעשׂות to do לירושׁלם at Jerusalem: ובהמה beast אין neither עמי with כי me, save אם me, save הבהמה the beast אשׁר that אני I רכב׃ rode}%
\verse{ואצאה And I went out בשׁער by the gate הגיא of the valley, לילה by night ואל even before פני even before עין well, התנין the dragon ואל and to שׁער port, האשׁפת the dung ואהי and viewed שׂבר and viewed בחומת the walls ירושׁלם of Jerusalem, אשׁר which המפרוצים were broken down, ושׁעריה and the gates אכלו thereof were consumed באשׁ׃ with fire.}%
\verse{ואעבר Then I went on אל to שׁער the gate העין of the fountain, ואל and to ברכת pool: המלך the king's ואין but no מקום place לבהמה for the beast לעבר me to pass. תחתי׃ under}%
\verse{ואהי and viewed עלה בנחל by the brook, לילה in the night ואהי שׂבר and viewed בחומה the wall, ואשׁוב and turned back, ואבוא and entered בשׁער by the gate הגיא of the valley, ואשׁוב׃ and returned.}%
\verse{והסגנים And the rulers לא not ידעו knew אנה whither הלכתי I went, ומה or what אני I עשׂה did; וליהודים to the Jews, ולכהנים nor to the priests, ולחרים nor to the nobles, ולסגנים nor to the rulers, וליתר nor to the rest עשׂה that did המלאכה the work. עד had I as כן yet לא neither הגדתי׃ told}%
\verse{ואומר Then said אלהם I unto אתם them, Ye ראים see הרעה the distress אשׁר that אנחנו we בה אשׁר in, how ירושׁלם Jerusalem חרבה ושׁעריה and the gates נצתו thereof are burned באשׁ with fire: לכו come, ונבנה and let us build up את חומת the wall ירושׁלם of Jerusalem, ולא no נהיה that we be עוד more חרפה׃ a reproach.}%
\verse{ואגיד Then I told להם את יד them of the hand אלהי of my God אשׁר which היא which טובה was good עלי upon ואף me; as also דברי words המלך the king's אשׁר that אמר he had spoken לי ויאמרו unto me. And they said, נקום Let us rise up ובנינו and build. ויחזקו So they strengthened ידיהם their hands לטובה׃ for good}%
\verse{וישׁמע heard סנבלט But when Sanballat החרני the Horonite, וטביה and Tobiah העבד the servant, העמוני the Ammonite, וגשׁם and Geshem הערבי the Arabian, וילעגו they laughed us to scorn, לנו ויבזו and despised עלינו and despised ויאמרו us, and said, מה What הדבר thing הזה this אשׁר that אתם ye עשׂים do? העל against המלך the king? אתם will ye מרדים׃ rebel}%
\verse{ואשׁיב אותםבר ואומר I them, and said להם אלהי unto them, The God השׁמים of heaven, הוא he יצליח will prosper לנו ואנחנו us; therefore we עבדיו his servants נקום will arise ובנינו and build: ולכם אין but ye have no חלק portion, וצדקה nor right, וזכרון nor memorial, בירושׁלם׃ in Jerusalem.}%
\end{biblechapter}%
\begin{biblechapter}% Nehemiah 3
\verseWithHeading{Organization of the Work}{ויקם rose up אלישׁיב Then Eliashib הכהן priest הגדול the high ואחיו with his brethren הכהנים the priests, ויבנו and they built את שׁער gate; הצאן the sheep המה they קדשׁוהו sanctified ויעמידו it, and set up דלתתיו the doors ועד of it; even unto מגדל the tower המאה of Meah קדשׁוהו they sanctified עד it, unto מגדל the tower חננאל׃ of Hananeel.}%
\verse{ועל unto ידו And next בנו him built אנשׁי ירחו of Jericho. ועל to ידו And next בנה them built זכור Zaccur בן the son אמרי׃ of Imri.}%
\verse{ואת שׁער gate הדגים But the fish בנו build, בני did the sons הסנאה of Hassenaah המה who קרוהו laid the beams ויעמידו thereof, and set up דלתתיו the doors מנעוליו thereof, the locks ובריחיו׃ thereof, and the bars}%
\verse{ועל unto ידם And next החזיק them repaired מרמות Meremoth בן the son אוריה of Urijah, בן the son הקוץ of Koz. ועל unto ידם And next החזיק them repaired משׁלם Meshullam בן the son ברכיה of Berechiah, בן the son משׁיזבאל of Meshezabeel. ועל unto ידם And next החזיק them repaired צדוק Zadok בן the son בענא׃}%
\verse{ועל unto ידם And next החזיקו repaired; התקועים them the Tekoites ואדיריהם but their nobles לא not הביאו put צורם their necks בעבדת to the work אדניהם׃ of their Lord.}%
\verse{ואת שׁער gate הישׁנה Moreover the old החזיקו repaired יוידע Jehoiada בן the son פסח of Paseah, ומשׁלם and Meshullam בן the son בסודיה of Besodeiah; המה they קרוהו laid the beams ויעמידו thereof, and set up דלתתיו the doors ומנעליו thereof, and the locks ובריחיו׃ thereof, and the bars}%
\verse{ועל unto ידם And next החזיק them repaired מלטיה Melatiah הגבעני the Gibeonite, וידון and Jadon המרנתי the Meronothite, אנשׁי the men גבעון of Gibeon, והמצפה and of Mizpah, לכסא unto the throne פחת of the governor עבר on this side הנהר׃ the river.}%
\verse{על unto ידו Next החזיק him repaired עזיאל Uzziel בן the son חרהיה of Harhaiah, צורפים of the goldsmiths. ועל unto ידו Next החזיק him also repaired חנניה Hananiah בן the son הרקחים of the apothecaries, ויעזבו and they fortified ירושׁלם Jerusalem עד unto החומה wall. הרחבה׃ the broad}%
\verse{ועל unto ידם And next החזיק them repaired רפיה Rephaiah בן the son חור of Hur, שׂר the ruler חצי of the half פלך part ירושׁלם׃ of Jerusalem.}%
\verse{ועל unto ידם And next החזיק them repaired ידיה Jedaiah בן the son חרומף of Harumaph, ונגד even over against ביתו his house. ועל unto ידו And next החזיק him repaired חטושׁ Hattush בן the son חשׁבניה׃ of Hashabniah.}%
\verse{מדה piece, שׁנית the other החזיק repaired מלכיה Malchijah בן the son חרם of Harim, וחשׁוב and Hashub בן the son פחת מואב of Pahath-moab, ואת מגדל and the tower התנורים׃ of the furnaces.}%
\verse{ועל unto ידו And next החזיק him repaired שׁלום Shallum בן the son הלוחשׁ of Halohesh, שׂר the ruler חצי of the half פלך part ירושׁלם of Jerusalem, הוא he ובנותיו׃ and his daughters.}%
\verse{את שׁער gate הגיא The valley החזיק repaired חנון Hanun, וישׁבי and the inhabitants זנוח of Zanoah; המה they בנוהו built ויעמידו it, and set up דלתתיו the doors מנעליו thereof, the locks ובריחיו thereof, and the bars ואלף thereof, and a thousand אמה cubits בחומה on the wall עד unto שׁער gate. השׁפות׃ the dung}%
\verse{ואת שׁער gate האשׁפות But the dung החזיק repaired מלכיה Malchiah בן the son רכב of Rechab, שׂר the ruler פלך of part בית הכרם of Beth-haccerem; הוא he יבננו built ויעמיד it, and set up דלתתיו the doors מנעליו thereof, the locks ובריחיו׃ thereof, and the bars}%
\verse{ואת שׁער But the gate העין of the fountain החזיק repaired שׁלון Shallun בן the son כל חזה of Col-hozeh, שׂר the ruler פלך of part המצפה of Mizpah; הוא he יבננו built ויטללנו it, and covered ויעמידו it, and set up דלתתיו the doors מנעליו thereof, the locks ובריחיו thereof, and the bars ואת חומת thereof, and the wall ברכת of the pool השׁלח of Siloah לגן garden, המלך by the king's ועד and unto המעלות the stairs היורדות that go down מעיר from the city דויד׃ of David.}%
\verse{אחריו After החזיק him repaired נחמיה Nehemiah בן the son עזבוק of Azbuk, שׂר the ruler חצי of the half פלך part בית צור of Beth-zur, עד unto נגד over against קברי the sepulchers דויד of David, ועד and to הברכה the pool העשׂויה that was made, ועד and unto בית the house הגברים׃ of the mighty.}%
\verse{אחריו After החזיקו him repaired הלוים the Levites, רחום Rehum בן the son בני of Bani. על unto ידו Next החזיק him repaired חשׁביה Hashabiah, שׂר the ruler חצי of the half פלך part קעילה of Keilah, לפלכו׃ in his part.}%
\verse{אחריו After החזיקו him repaired אחיהם their brethren, בוי Bavai בן the son חנדד of Henadad, שׂר the ruler חצי of the half פלך part קעילה׃ of Keilah.}%
\verse{ויחזק him repaired על to ידו And next עזר Ezer בן the son ישׁוע of Jeshua, שׂר the ruler המצפה of Mizpah, מדה piece שׁנית another מנגד over against עלת the going up הנשׁק to the armory המקצע׃ at the turning}%
\verse{אחריו After החרה earnestly החזיק repaired ברוך him Baruch בן the son זבי of Zabbai מדה piece, שׁנית the other מן from המקצוע the turning עד unto פתח the door בית of the house אלישׁיב of Eliashib הכהן priest. הגדול׃ the high}%
\verse{אחריו After החזיק him repaired מרמות Meremoth בן the son אוריה of Urijah בן the son הקוץ of Koz מדה piece, שׁנית another מפתח from the door בית of the house אלישׁיב of Eliashib ועד even to תכלית the end בית of the house אלישׁיב׃ of Eliashib.}%
\verse{ואחריו And after החזיקו him repaired הכהנים the priests, אנשׁי the men הככר׃ of the plain.}%
\verse{אחריו After החזיק him repaired בנימן Benjamin וחשׁוב and Hashub נגד over against ביתם their house. אחריו After החזיק him repaired עזריה Azariah בן the son מעשׂיה of Maaseiah בן the son ענניה of Ananiah אצל by ביתו׃ his house.}%
\verse{אחריו After החזיק him repaired בנוי Binnui בן the son חנדד of Henadad מדה piece, שׁנית another מבית from the house עזריה of Azariah עד unto המקצוע the turning ועד even unto הפנה׃ the corner.}%
\verse{פלל Palal בן the son אוזי of Uzai, מנגד over against המקצוע the turning והמגדל and the tower היוצא which lieth out מבית house, המלך from the king's העליון high אשׁר that לחצר by the court המטרה of the prison. אחריו After פדיה him Pedaiah בן the son פרעשׁ׃ of Parosh.}%
\verse{והנתינים Moreover the Nethinims היו dwelt ישׁבים dwelt בעפל in Ophel, עד unto נגד over against שׁער gate המים the water למזרח toward the east, והמגדל and the tower היוצא׃ that lieth out.}%
\verse{אחריו After החזיקו repaired התקעים them the Tekoites מדה piece, שׁנית another מנגד over against המגדל tower הגדול the great היוצא that lieth out, ועד even unto חומת the wall העפל׃ of Ophel.}%
\verse{מעל שׁער gate הסוסים the horse החזיקו repaired הכהנים the priests, אישׁ every one לנגד over against ביתו׃ his house.}%
\verse{אחריו After החזיק them repaired צדוק Zadok בן the son אמר of Immer נגד over against ביתו his house. ואחריו After החזיק him repaired שׁמעיה also Shemaiah בן the son שׁכניה of Shechaniah, שׁמר the keeper שׁער gate. המזרח׃ of the east}%
\verse{אחרי After החזיק him repaired חנניה Hananiah בן the son שׁלמיה of Shelemiah, וחנון and Hanun בן son צלף of Zalaph, השׁשׁי the sixth מדה piece. שׁני another אחריו After החזיק him repaired משׁלם Meshullam בן the son ברכיה of Berechiah נגד over against נשׁכתו׃ his chamber.}%
\verse{אחרי After החזיק him repaired מלכיה Malchiah בן son הצרפי the goldsmith's עד unto בית the place הנתינים of the Nethinims, והרכלים and of the merchants, נגד over against שׁער the gate המפקד Miphkad, ועד and to עלית the going up הפנה׃ of the corner.}%
\verse{ובין And between עלית the going up הפנה of the corner לשׁער gate הצאן unto the sheep החזיקו repaired הצרפים the goldsmiths והרכלים׃ and the merchants.}%
\end{biblechapter}%
\begin{biblechapter}% Nehemiah 4
\verseWithHeading{Opposition Begins}{ויהי But it came to pass, כאשׁר that when שׁמע heard סנבלט Sanballat כי that אנחנו we בונים built את החומה the wall, ויחר he was wroth, לו ויכעס and took great indignation, הרבה and took great indignation, וילעג and mocked על and mocked היהודים׃ the Jews.}%
\verse{ויאמר And he spoke לפני before אחיו his brethren וחיל שׁמרון of Samaria, ויאמר and said, מה What היהודים Jews? האמללים these feeble עשׂים do היעזבו will they fortify להם which היזבחו themselves? will they sacrifice? היכלו will they make an end ביום in a day? היחיו will they revive את האבנים the stones מערמות out of the heaps העפר of the rubbish והמה שׂרופות׃ are burned?}%
\verse{וטוביה Now Tobiah העמני the Ammonite אצלו by ויאמר him, and he said, גם Even אשׁר that which הם they בונים build, אם if יעלה go up, שׁועל a fox ופרץ he shall even break down חומת wall. אבניהם׃ their stone}%
\verse{שׁמע Hear, אלהינו O our God; כי for היינו we are בוזה despised: והשׁב and turn חרפתם their reproach אל upon ראשׁם their own head, ותנם and give לבזה them for a prey בארץ in the land שׁביה׃ of captivity:}%
\verse{ואל not תכס עלונם their iniquity, וחטאתם their sin מלפניך from before אל and let not תמחה be blotted out כי thee: for הכעיסו they have provoked to anger לנגד before הבונים׃ the builders.}%
\verse{ונבנה So built את החומה we the wall; ותקשׁר was joined together כל and all החומה the wall עד unto חציה the half ויהי had לב a mind לעם thereof: for the people לעשׂות׃ to work.}%
\verseWithHeading{Opposition Stopped}{ויהי But it came to pass, כאשׁר when שׁמע heard סנבלט Sanballat, וטוביה and Tobiah, והערבים and the Arabians, והעמנים and the Ammonites, והאשׁדודים and the Ashdodites, כי that עלתה were made up, ארוכה were made up, לחמות the walls ירושׁלם of Jerusalem כי that החלו began הפרצים the breaches להסתם to be stopped, ויחר wroth, להם מאד׃ then they were very}%
\verse{ויקשׁרו And conspired כלם all יחדו of them together לבוא to come להלחם to fight בירושׁלם against Jerusalem, ולעשׂות and to hinder לו תועה׃ and to hinder}%
\verse{ונתפלל Nevertheless we made our prayer אל unto אלהינו our God, ונעמיד and set משׁמר a watch עליהם against יומם them day ולילה and night, מפניהם׃ because of}%
\verse{ויאמר said, יהודה And Judah כשׁל is decayed, כח The strength הסבל of the bearers of burdens והעפר rubbish; הרבה and much ואנחנו so that we לא are not נוכל able לבנות to build בחומה׃ the wall.}%
\verse{ויאמרו said, צרינו And our adversaries לא They shall not ידעו know, ולא neither יראו see, עד till אשׁר till נבוא we come אל in תוכם the midst among והרגנום them, and slay והשׁבתנו to cease. את המלאכה׃ them, and cause the work}%
\verse{ויהי And it came to pass, כאשׁר that when באו them came, היהודים the Jews הישׁבים which dwelt אצלם by ויאמרו they said לנו עשׂר unto us ten פעמים times, מכל המקמות places אשׁר whence תשׁובו ye shall return עלינו׃ unto}%
\verse{ואעמיד Therefore set מתחתיות למקום places מאחרי behind לחומה the wall, בצחחיים ואעמיד I even set את העם the people למשׁפחות after their families עם with חרבתיהם their swords, רמחיהם their spears, וקשׁתתיהם׃ and their bows.}%
\verse{וארא And I looked, ואקום and rose up, ואמר and said אל unto החרים the nobles, ואל and to הסגנים the rulers, ואל and to יתר the rest העם of the people, אל Be not תיראו ye afraid מפניהם of את אדני the Lord, הגדול great והנורא and terrible, זכרו them: remember והלחמו and fight על for אחיכם your brethren, בניכם your sons, ובנתיכם and your daughters, נשׁיכם your wives, ובתיכם׃ and your houses.}%
\verse{ויהי And it came to pass, כאשׁר when שׁמעו heard אויבינו our enemies כי that נודע it was known לנו ויפר האלהים unto us, and God את עצתםנשׁוב that we returned כלנו all אל of us to החומה the wall, אישׁ every one אל unto מלאכתו׃ his work.}%
\verse{ויהי And it came to pass מן from היום time ההוא that חצי forth, the half נערי of my servants עשׂים wrought במלאכה in the work, וחצים and the other half מחזיקים of them held והרמחים both the spears, המגנים the shields, והקשׁתות and the bows, והשׁרינים and the habergeons; והשׂרים and the rulers אחרי behind כל all בית the house יהודה׃ of Judah.}%
\verse{הבונים They which built בחומה on the wall, והנשׂאים and they that bore בסבל burdens, עמשׂים with those that laded, באחת with one ידו of his hands עשׂה wrought במלאכה in the work, ואחת and with the other מחזקת held השׁלח׃ a weapon.}%
\verse{והבונים For the builders, אישׁ every one חרבו had his sword אסורים girded על by מתניו his side, ובונים and built. והתוקע And he that sounded בשׁופר the trumpet אצלי׃ by}%
\verse{ואמר And I said אל unto החרים the nobles, ואל and to הסגנים the rulers, ואל and to יתר the rest העם of the people, המלאכה The work הרבה great ורחבה and large, ואנחנו and we נפרדים are separated על upon החומה the wall, רחוקים far אישׁ one מאחיו׃ from another.}%
\verse{במקום place אשׁר In what תשׁמעו ye hear את קול the sound השׁופר of the trumpet, שׁמה ye thither תקבצו resort אלינו unto אלהינו us: our God ילחם׃ shall fight}%
\verse{ואנחנו So we עשׂים labored במלאכה in the work: וחצים and half מחזיקים of them held ברמחים the spears מעלות from the rising השׁחר of the morning עד till צאת appeared. הכוכבים׃ the stars}%
\verse{גם Likewise בעת time ההיא at the same אמרתי said לעם I unto the people, אישׁ Let every one ונערו with his servant ילינו lodge בתוך within ירושׁלם Jerusalem, והיו they may be לנו הלילה that in the night משׁמר a guard והיום on the day. מלאכה׃ to us, and labor}%
\verse{ואין So neither אני I, ואחי nor my brethren, ונערי nor my servants, ואנשׁי המשׁמר of the guard אשׁר which אחרי followed אין me, none אנחנו of us פשׁטים put off בגדינו our clothes, אישׁ nor the men שׁלחו המים׃ for washing.}%
\end{biblechapter}%
\begin{biblechapter}% Nehemiah 5
\verseWithHeading{Nehemiah Deals with Strife}{ותהי And there was צעקת cry העם of the people ונשׁיהם and of their wives גדולה a great אל against אחיהם their brethren היהודים׃ the Jews.}%
\verse{וישׁ For there were אשׁר that אמרים said, בנינו our sons, ובנתינו and our daughters, אנחנו We, רבים many: ונקחה therefore we take up דגן corn ונאכלה that we may eat, ונחיה׃ and live.}%
\verse{וישׁ also there were אשׁר that אמרים said, שׂדתינו our lands, וכרמינו vineyards, ובתינו and houses, אנחנו We ערבים have mortgaged ונקחה that we might buy דגן corn, ברעב׃ because of the dearth.}%
\verse{וישׁ There were אשׁר also that אמרים said, לוינו We have borrowed כסף money למדת tribute, המלך for the king's שׂדתינו our lands וכרמינו׃ and vineyards.}%
\verse{ועתה Yet now כבשׂר our flesh אחינו of our brethren, בשׂרנו as the flesh כבניהם our children בנינו as their children: והנה and, lo, אנחנו we כבשׁים bring into bondage את בנינו our sons ואת בנתינו and our daughters לעבדים to be servants, וישׁ are מבנתינו and of our daughters נכבשׁות brought unto bondage ואין neither לאל in our power ידנו in our power ושׂדתינו men have our lands וכרמינו and vineyards. לאחרים׃ for other}%
\verse{ויחר angry לי מאד And I was very כאשׁר when שׁמעתי I heard את זעקתם their cry ואת הדברים words. האלה׃ and these}%
\verse{וימלך Then I consulted לבי myself, עלי with ואריבה and I rebuked את החרים the nobles, ואת הסגנים and the rulers, ואמרה and said להם משׁא exact אישׁ every one באחיו of his brother. אתם unto them, Ye נשׁאים ואתן And I set עליהם against קהלה assembly גדולה׃ a great}%
\verse{ואמרה And I said להם אנחנו unto them, We קנינו have redeemed את אחינו our brethren היהודים the Jews, הנמכרים which were sold לגוים unto the heathen; כדי after our ability בנו וגם even אתם and will ye תמכרו sell את אחיכם your brethren? ונמכרו or shall they be sold לנו ויחרישׁו unto us? Then held they their peace, ולא nothing מצאו and found דבר׃}%
\verse{ויאמר Also I said, לא It not טוב good הדבר good אשׁר that אתם ye עשׂים do: הלוא ought ye not ביראת in the fear אלהינו of our God תלכו to walk מחרפת because of the reproach הגוים of the heathen אויבינו׃ our enemies?}%
\verse{וגם likewise, אני I אחי my brethren, ונערי and my servants, נשׁים might exact בהם כסף of them money ודגן and corn: נעזבה you, let us leave off נא I pray את המשׁא usury. הזה׃ this}%
\verse{השׁיבו Restore, נא I pray להם כהיום you, to them, even this day, שׂדתיהם their lands, כרמיהם their vineyards, זיתיהם their oliveyards, ובתיהם and their houses, ומאת also the hundredth הכסף of the money, והדגן and of the corn, התירושׁ the wine, והיצהר and the oil, אשׁר that אתם ye נשׁים׃ exact}%
\verse{ויאמרו Then said נשׁיב they, We will restore ומהם לא nothing נבקשׁ and will require כן them; so נעשׂה will we do כאשׁר as אתה thou אומר sayest. ואקרא Then I called את הכהנים the priests, ואשׁביעם and took an oath לעשׂות of them, that they should do כדבר promise. הזה׃ according to this}%
\verse{גם Also חצני my lap, נערתי I shook ואמרה and said, ככה So ינער shake out האלהים God את כל every האישׁ man אשׁר that לא not יקים performeth את הדבר promise, הזה this מביתו from his house, ומיגיעו and from his labor, וככה even thus יהיה be נעור he shaken out, ורק and emptied. ויאמרו said, כל And all הקהל the congregation אמן Amen, ויהללו and praised את יהוהיעשׂ did העם And the people כדבר promise. הזה׃ according to this}%
\verseWithHeading{Nehemiah Denies His Allotment}{גם Moreover מיום from the time אשׁר that צוה I was appointed אתי להיות to be פחם their governor בארץ in the land יהודה of Judah, משׁנת year עשׂרים from the twentieth ועד even unto שׁנת year שׁלשׁים and thirtieth ושׁתים the two לארתחשׁסתא of Artaxerxes המלך the king, שׁנים years, שׁתים twelve עשׂרה , twelve אני I ואחי and my brethren לחם the bread הפחה of the governor. לא have not אכלתי׃ eaten}%
\verse{והפחות governors הראשׁנים But the former אשׁר that לפני before הכבידו me were chargeable על unto העם the people, ויקחו and had taken מהם בלחם them bread ויין and wine, אחר beside כסף of silver; שׁקלים shekels ארבעים forty גם yea, נעריהם even their servants שׁלטו bore rule על over העם the people: ואני I, לא not עשׂיתי did כן but so מפני because יראת of the fear אלהים׃ of God.}%
\verse{וגם Yea, also במלאכת in the work החומה wall, הזאת of this החזקתי I continued ושׂדה we any land: לא neither קנינו bought וכל and all נערי my servants קבוצים gathered שׁם thither על unto המלאכה׃ the work.}%
\verse{והיהודים of the Jews והסגנים and rulers, מאה a hundred וחמשׁים and fifty אישׁ and fifty והבאים beside those that came אלינו unto מן us from among הגוים the heathen אשׁר that סביבתינו about על Moreover at שׁלחני׃ my table}%
\verse{ואשׁר Now which היה was נעשׂה prepared ליום daily אחד daily שׁור ox אחד one צאן sheep; שׁשׁ six בררות choice וצפרים also fowls נעשׂו were prepared לי ובין for me, and once עשׂרת in ten ימים days בכל of all יין sorts of wine: להרבה store ועם yet for all זה this לחם I the bread הפחה of the governor, לא not בקשׁתי required כי because כבדה was heavy העבדה the bondage על upon העם people. הזה׃ this}%
\verse{זכרה Think לי אלהי upon me, my God, לטובה for good, כל to all אשׁר that עשׂיתי I have done על for העם people. הזה׃ this}%
\end{biblechapter}%
\begin{biblechapter}% Nehemiah 6
\verseWithHeading{Enemies Foiled}{ויהי Now it came to pass, כאשׁר when נשׁמע heard לסנבלט Sanballat, וטוביה and Tobiah, ולגשׁם and Geshem הערבי the Arabian, וליתר and the rest איבינו of our enemies, כי that בניתי I had built את החומה the wall, ולא and there was no נותר left בה פרץ breach גם therein; (though עד at העת time ההיא that דלתות the doors לא I had not העמדתי set up בשׁערים׃ upon the gates;)}%
\verse{וישׁלח sent סנבלט That Sanballat וגשׁם and Geshem אלי unto לאמר me, saying, לכה Come, ונועדה let us meet יחדו together בכפירים in the villages בבקעת in the plain אונו of Ono. והמה But they חשׁבים thought לעשׂות to do לי רעה׃ me mischief.}%
\verse{ואשׁלחה And I sent עליהם unto מלאכים messengers לאמר them, saying, מלאכה work, גדולה a great אני I עשׂה doing ולא אוכלרדת come down: למה why תשׁבת cease, המלאכה should the work כאשׁר whilst ארפה I leave וירדתי it, and come down אליכם׃ to}%
\verse{וישׁלחו Yet they sent אלי unto כדבר sort; הזה after this ארבע me four פעמים times ואשׁיב and I answered אותם כדבר manner. הזה׃ them after the same}%
\verse{וישׁלח Then sent אלי unto סנבלט Sanballat כדבר manner הזה me in like פעם time חמישׁית the fifth את נערו his servant ואגרת letter פתוחה with an open בידו׃ in his hand;}%
\verse{כתוב Wherein written, בה בגוים among the heathen, נשׁמע It is reported וגשׁמו and Gashmu אמר saith אתה thou והיהודים and the Jews חשׁבים think למרוד to rebel: על for כן which cause אתה thou בונה buildest החומה the wall, ואתה that thou הוה mayest be להם למלך their king, כדברים words. האלה׃ according to these}%
\verse{וגם And thou hast also נביאים prophets העמדת appointed לקרא to preach עליך of בירושׁלם thee at Jerusalem, לאמר saying, מלך a king ביהודה in Judah: ועתה and now ישׁמע shall it be reported למלך to the king כדברים words. האלה according to these ועתה now לכה Come ונועצה therefore, and let us take counsel יחדו׃ together.}%
\verse{ואשׁלחה Then I sent אליו unto לאמר him, saying, לא no נהיה There are כדברים things האלה such אשׁר done as אתה thou אומר sayest, כי but מלבך them out of thine own heart. אתה thou בודאם׃ feignest}%
\verse{כי For כלם they all מיראים אותנואמר saying, ירפו shall be weakened ידיהם Their hands מן from המלאכה the work, ולא that it be not תעשׂה done. ועתה Now חזק therefore, strengthen את ידי׃ my hands.}%
\verse{ואני Afterward I באתי came בית unto the house שׁמעיה of Shemaiah בן the son דליה of Delaiah בן the son מהיטבאל of Mehetabeel, והוא who עצור shut up; ויאמר and he said, נועד Let us meet together אל in בית the house האלהים of God, אל within תוך within ההיכל the temple, ונסגרה and let us shut דלתות the doors ההיכל of the temple: כי for באים they will come להרגך to slay ולילה thee; yea, in the night באים will they come להרגך׃ to slay}%
\verse{ואמרה And I said, האישׁ כמוני as I יברח flee? ומי and who כמוני אשׁר that, יבוא would go אל into ההיכל the temple וחי to save his life? לא I will not אבוא׃ go in.}%
\verse{ואכירה I perceived והנה And, lo, לא had not אלהים that God שׁלחו sent כי him; but הנבואה this prophecy דבר that he pronounced עלי against וטוביה me: for Tobiah וסנבלט and Sanballat שׂכרו׃ had hired}%
\verse{למען Therefore שׂכור hired, הוא he למען that אירא I should be afraid, ואעשׂה and do כן so, וחטאתי and sin, והיה and they might have להם לשׁם report, רע for an evil למען that יחרפוני׃ they might reproach}%
\verse{זכרה think אלהי My God, לטוביה thou upon Tobiah ולסנבלט and Sanballat כמעשׂיו their works, אלה according to these וגם and לנועדיה Noadiah, הנביאה on the prophetess וליתר and the rest הנביאים of the prophets, אשׁר that היו would have מיראים put me in fear. אותי׃}%
\verseWithHeading{The Wall is Completed}{ותשׁלם was finished החומה So the wall בעשׂרים in the twenty וחמשׁה and fifth לאלול of Elul, לחמשׁים in fifty ושׁנים and two יום׃ days.}%
\verse{ויהי And it came to pass, כאשׁר that when שׁמעו heard כל all אויבינו our enemies ויראו us saw כל and all הגוים the heathen אשׁר that סביבתינו about ויפלו cast down מאד they were much בעיניהם in their own eyes: וידעו for they perceived כי that מאת of אלהינו our God. נעשׂתה was wrought המלאכה work הזאת׃ this}%
\verse{גם Moreover בימים days ההם in those מרבים many חרי the nobles יהודה of Judah אגרתיהם letters הולכות sent על unto טוביה Tobiah, ואשׁר לטוביה and of Tobiah באות came אליהם׃ unto}%
\verse{כי For רבים many ביהודה in Judah בעלי sworn שׁבועה sworn לו כי unto him, because חתן the son in law הוא he לשׁכניה of Shechaniah בן the son ארח of Arah; ויהוחנן Johanan בנו and his son לקח had taken את בת the daughter משׁלם of Meshullam בן the son ברכיה׃ of Berechiah.}%
\verse{גם Also טובתיו his good deeds היו they reported אמרים they reported לפני before ודברי my words היו me, and uttered מוציאים me, and uttered לו אגרות letters שׁלח sent טוביה to him. Tobiah ליראני׃ to put me in fear.}%
\end{biblechapter}%
\begin{biblechapter}% Nehemiah 7
\verse{ויהי Now it came to pass, כאשׁר when נבנתה was built, החומה the wall ואעמיד and I had set up הדלתות the doors, ויפקדו were appointed, השׁוערים and the porters והמשׁררים and the singers והלוים׃ and the Levites}%
\verse{ואצוה charge את חנני Hanani, אחי That I gave my brother ואת חנניה and Hananiah שׂר the ruler הבירה of the palace, על over ירושׁלם Jerusalem: כי for הוא he כאישׁ man, אמת a faithful וירא and feared את האלהים God מרבים׃ above many.}%
\verse{ויאמר And I said להם לא unto them, Let not יפתחו be opened שׁערי the gates ירושׁלם of Jerusalem עד until חם השׁמשׁ the sun ועד and while הם they עמדים stand יגיפו by, let them shut הדלתות the doors, ואחזו and bar והעמיד and appoint משׁמרות watches ישׁבי of the inhabitants ירושׁלם of Jerusalem, אישׁ every one במשׁמרו in his watch, ואישׁ and every one נגד over against ביתו׃ his house.}%
\verse{והעיר Now the city רחבת large ידים large וגדולה and great: והעם but the people מעט few בתוכה therein, ואין not בתים and the houses בנוים׃ built.}%
\verseWithHeading{Lists of the Exiles Who Returned}{ויתן put אלהי And my God אל into לבי mine heart ואקבצה to gather together את החרים the nobles, ואת הסגנים and the rulers, ואת העם and the people, להתיחשׂ that they might be reckoned by genealogy. ואמצא And I found ספר a register היחשׂ of the genealogy העולים of them which came up בראשׁונה at the first, ואמצא and found כתוב׃ written}%
\verse{אלה These בני the children המדינה of the province, העלים that went up משׁבי out of the captivity, הגולה of those that had been carried awa, אשׁר whom הגלה had carried away, נבוכדנצר Nebuchadnezzar מלך the king בבל of Babylon וישׁובו and came again לירושׁלם to Jerusalem וליהודה and to Judah, אישׁ every one לעירו׃ unto his city;}%
\verse{הבאים Who came עם with זרבבל Zerubbabel, ישׁוע Jeshua, נחמיה Nehemiah, עזריה Azariah, רעמיה Raamiah, נחמני Nahamani, מרדכי Mordecai, בלשׁן Bilshan, מספרת Mispereth, בגוי Bigvai, נחום Nehum, בענה Baanah. מספר The number, אנשׁי of the men עם of the people ישׂראל׃ of Israel}%
\verse{בני The children פרעשׁ of Parosh, אלפים two thousand מאה a hundred ושׁבעים seventy ושׁנים׃ and two.}%
\verse{בני The children שׁפטיה of Shephatiah, שׁלשׁ three מאות hundred שׁבעים seventy ושׁנים׃ and two.}%
\verse{בני The children ארח of Arah, שׁשׁ six מאות hundred חמשׁים fifty ושׁנים׃ and two.}%
\verse{בני The children פחת מואב of Pahath-moab, לבני of the children ישׁוע of Jeshua ויואב and Joab, אלפים two thousand ושׁמנה and eight מאות hundred שׁמנה eighteen. עשׂר׃ eighteen.}%
\verse{בני The children עילם of Elam, אלף a thousand מאתים two hundred חמשׁים fifty וארבעה׃ and four.}%
\verse{בני The children זתוא of Zattu, שׁמנה eight מאות hundred ארבעים forty וחמשׁה׃ and five.}%
\verse{בני The children זכי of Zaccai, שׁבע seven מאות hundred ושׁשׁים׃ and threescore.}%
\verse{בני The children בנוי of Binnui, שׁשׁ six מאות hundred ארבעים forty ושׁמנה׃ and eight.}%
\verse{בני The children בבי of Bebai, שׁשׁ six מאות hundred עשׂרים twenty ושׁמנה׃ and eight.}%
\verse{בני The children עזגד of Azgad, אלפים two thousand שׁלשׁ three מאות hundred עשׂרים twenty ושׁנים׃ and two.}%
\verse{בני The children אדניקם of Adonikam, שׁשׁ six מאות hundred שׁשׁים threescore ושׁבעה׃ and seven.}%
\verse{בני The children בגוי of Bigvai, אלפים two thousand שׁשׁים threescore ושׁבעה׃ and seven.}%
\verse{בני The children עדין of Adin, שׁשׁ six מאות hundred חמשׁים fifty וחמשׁה׃ and five.}%
\verse{בני The children אטר of Ater לחזקיה of Hezekiah, תשׁעים ninety ושׁמנה׃ and eight.}%
\verse{בני The children חשׁם of Hashum, שׁלשׁ three מאות hundred עשׂרים twenty ושׁמנה׃ and eight.}%
\verse{בני The children בצי of Bezai, שׁלשׁ three מאות hundred עשׂרים twenty וארבעה׃ and four.}%
\verse{בני The children חריף of Hariph, מאה a hundred שׁנים and twelve. עשׂר׃ and twelve.}%
\verse{בני The children גבעון of Gibeon, תשׁעים ninety וחמשׁה׃ and five.}%
\verse{אנשׁי בית לחם of Bethlehem ונטפה and Netophah, מאה a hundred שׁמנים fourscore ושׁמנה׃ and eight.}%
\verse{אנשׁי The men ענתות of Anathoth, מאה a hundred עשׂרים twenty ושׁמנה׃ and eight.}%
\verse{אנשׁי The men בית עזמות of Beth-azmaveth, ארבעים forty ושׁנים׃ and two.}%
\verse{אנשׁי The men קרית יערים of Kirjath-jearim, כפירה Chephirah, ובארות and Beeroth, שׁבע seven מאות hundred ארבעים forty ושׁלשׁה׃ and three.}%
\verse{אנשׁי The men הרמה of Ramah וגבע and Geba, שׁשׁ six מאות hundred עשׂרים twenty ואחד׃ and one.}%
\verse{אנשׁי The men מכמס of Michmas, מאה a hundred ועשׂרים and twenty ושׁנים׃ and two.}%
\verse{אנשׁי The men בית אל of Bethel והעי and Ai, מאה a hundred עשׂרים twenty ושׁלשׁה׃ and three.}%
\verse{אנשׁי The men נבו Nebo, אחר of the other חמשׁים fifty ושׁנים׃ and two.}%
\verse{בני The children עילם Elam, אחר of the other אלף a thousand מאתים two hundred חמשׁים fifty וארבעה׃ and four.}%
\verse{בני The children חרם of Harim, שׁלשׁ three מאות hundred ועשׂרים׃ and twenty.}%
\verse{בני The children ירחו of Jericho, שׁלשׁ three מאות hundred ארבעים forty וחמשׁה׃ and five.}%
\verse{בני The children לד of Lod, חדיד Hadid, ואונו and Ono, שׁבע seven מאות hundred ועשׂרים twenty ואחד׃ and one.}%
\verse{בני The children סנאה of Senaah, שׁלשׁת three אלפים thousand תשׁע nine מאות hundred ושׁלשׁים׃ and thirty.}%
\verse{הכהנים The priests: בני the children ידעיה of Jedaiah, לבית of the house ישׁוע of Jeshua, תשׁע nine מאות hundred שׁבעים seventy ושׁלשׁה׃ and three.}%
\verse{בני The children אמר of Immer, אלף a thousand חמשׁים fifty ושׁנים׃ and two.}%
\verse{בני The children פשׁחור of Pashur, אלף a thousand מאתים two hundred ארבעים forty ושׁבעה׃ and seven.}%
\verse{בני The children חרם of Harim, אלף a thousand שׁבעה and seventeen. עשׂר׃ and seventeen.}%
\verse{הלוים The Levites: בני the children ישׁוע of Jeshua, לקדמיאל of Kadmiel, לבני of the children להודוה of Hodevah, שׁבעים seventy וארבעה׃ and four.}%
\verse{המשׁררים The singers: בני the children אסף of Asaph, מאה a hundred ארבעים forty ושׁמנה׃ and eight.}%
\verse{השׁערים The porters: בני the children שׁלום of Shallum, בני the children אטר of Ater, בני the children טלמן of Talmon, בני the children עקוב of Akkub, בני the children חטיטא of Hatita, בני the children שׁבי of Shobai, מאה a hundred שׁלשׁים thirty ושׁמנה׃ and eight.}%
\verse{הנתינים The Nethinims: בני the children צחא of Ziha, בני the children חשׂפא of Hashupha, בני the children טבעות׃ of Tabbaoth,}%
\verse{בני The children קירס of Keros, בני the children סיעא of Sia, בני the children פדון׃ of Padon,}%
\verse{בני The children לבנה of Lebana, בני the children חגבה of Hagaba, בני the children שׁלמי׃ of Shalmai,}%
\verse{בני The children חנן of Hanan, בני the children גדל of Giddel, בני the children גחר׃ of Gahar,}%
\verse{בני The children ראיה of Reaiah, בני the children רצין of Rezin, בני the children נקודא׃ of Nekoda,}%
\verse{בני The children גזם of Gazzam, בני the children עזא of Uzza, בני the children פסח׃ of Phaseah,}%
\verse{בני The children בסי of Besai, בני the children מעונים of Meunim, בני the children נפושׁסים׃}%
\verse{בני The children בקבוק of Bakbuk, בני the children חקופא of Hakupha, בני the children חרחור׃ of Harhur,}%
\verse{בני The children בצלית of Bazlith, בני the children מחידא of Mehida, בני the children חרשׁא׃ of Harsha,}%
\verse{בני The children ברקוס of Barkos, בני the children סיסרא of Sisera, בני the children תמח׃ of Tamah,}%
\verse{בני The children נציח of Neziah, בני the children חטיפא׃ of Hatipha.}%
\verse{בני The children עבדי servants: שׁלמה of Solomon's בני the children סוטי of Sotai, בני the children סופרת of Sophereth, בני the children פרידא׃ of Perida,}%
\verse{בני The children יעלא of Jaala, בני the children דרקון of Darkon, בני the children גדל׃ of Giddel,}%
\verse{בני The children שׁפטיה of Shephatiah, בני the children חטיל of Hattil, בני the children פכרת הצביים of Pochereth בני the children אמון׃ of Amon.}%
\verse{כל All הנתינים the Nethinims, ובני and the children עבדי servants, שׁלמה of Solomon's שׁלשׁ three מאות hundred תשׁעים ninety ושׁנים׃ and two.}%
\verse{ואלה And these העולים they which went up מתל מלח תל חרשׁא Tel-haresha, כרוב Cherub, אדון Addon, ואמר and Immer: ולא not יכלו but they could להגיד show בית house, אבותם their father's וזרעם nor their seed, אם whether מישׂראל הם׃ they}%
\verse{בני The children דליה of Delaiah, בני the children טוביה of Tobiah, בני the children נקודא of Nekoda, שׁשׁ six מאות hundred וארבעים forty ושׁנים׃ and two.}%
\verse{ומן And of הכהנים the priests: בני the children חביה of Habaiah, בני the children הקוץ of Koz, בני the children ברזלי of Barzillai, אשׁר which לקח took מבנות of the daughters ברזלי of Barzillai הגלעדי the Gileadite אשׁה to wife, ויקרא and was called על after שׁמם׃ their name.}%
\verse{אלה These בקשׁו sought כתבם their register המתיחשׂים those that were reckoned by genealogy, ולא but it was not נמצא found: ויגאלו therefore were they, as polluted, מן put from הכהנה׃ the priesthood.}%
\verse{ויאמר said התרשׁתא And the Tirshatha להם אשׁר unto them, that לא they should not יאכלו eat מקדשׁ הקדשׁיםד till עמד there stood הכהן a priest לאורים with Urim ותומים׃ and Thummim.}%
\verse{כל The whole הקהל congregation כאחד together ארבע forty רבוא forty אלפים and two thousand שׁלשׁ three מאות hundred ושׁשׁים׃ and threescore,}%
\verse{מלבד עבדיהם their menservants ואמהתיהם and their maidservants, אלה of whom שׁבעת seven אלפים thousand שׁלשׁ three מאות hundred שׁלשׁים thirty ושׁבעה and seven: ולהם משׁררים singing ומשׁררות men and singing מאתים and they had two hundred וארבעים forty וחמשׁה׃ and five}%
\verse{סוסיהם Their horses, שׁבע seven מאות hundred שׁלשׁים thirty ושׁשׁה and six: פרדיהם their mules, מאתים two hundred ארבעים forty וחמשׁה׃ and five:}%
\verse{גמלים camels, ארבע four מאות hundred שׁלשׁים thirty וחמשׁה and five: חמרים asses. שׁשׁת six אלפים thousand שׁבע seven מאות hundred ועשׂרים׃ and twenty}%
\verse{ומקצת ראשׁי of the chief האבות of the fathers נתנו gave למלאכה unto the work. התרשׁתא The Tirshatha נתן gave לאוצר to the treasure זהב of gold, דרכמנים drams אלף a thousand מזרקות basins, חמשׁים fifty כתנות garments. כהנים priests' שׁלשׁים and thirty וחמשׁ five מאות׃ hundred}%
\verse{ומראשׁי האבות of the fathers נתנו gave לאוצר to the treasure המלאכה of the work זהב of gold, דרכמונים drams שׁתי twenty thousand רבות twenty thousand וכסף of silver. מנים pound אלפים and two thousand ומאתים׃ and two hundred}%
\verse{ואשׁר And which נתנו gave שׁארית the rest העם of the people זהב of gold, דרכמונים drams שׁתי twenty רבוא thousand וכסף of silver, מנים pound אלפים and two thousand וכתנת garments. כהנים priests' שׁשׁים and threescore ושׁבעה׃ and seven}%
\verse{וישׁבו dwelt הכהנים So the priests, והלוים and the Levites, והשׁוערים and the porters, והמשׁררים and the singers, ומן and of העם the people, והנתינים and the Nethinims, וכל and all ישׂראל Israel, בעריהם in their cities; ויגע came, החדשׁ month השׁביעי and when the seventh ובני the children ישׂראל of Israel בעריהם׃ in their cities.}%
\end{biblechapter}%
\begin{biblechapter}% Nehemiah 8
\verseWithHeading{Ezra Reads the Law to the People}{ויאספו gathered themselves together כל And all העם the people כאישׁ man אחד as one אל into הרחוב the street אשׁר that לפני before שׁער gate; המים the water ויאמרו and they spoke לעזרא unto Ezra הספר the scribe להביא to bring את ספר the book תורת of the law משׁה of Moses, אשׁר which צוה had commanded יהוה the LORD את ישׂראל׃ to Israel.}%
\verse{ויביא brought עזרא And Ezra הכהן the priest את התורה the law לפני before הקהל the congregation מאישׁ both of men ועד אשׁה and women, וכל and all מבין with understanding, לשׁמע that could hear ביום day אחד upon the first לחדשׁ month. השׁביעי׃ of the seventh}%
\verse{ויקרא And he read בו לפני therein before הרחוב the street אשׁר that לפני before שׁער gate המים the water מן from האור the morning עד until מחצית midday, היום midday, נגד before האנשׁים the men והנשׁים and the women, והמבינים and those that could understand; ואזני and the ears כל of all העם the people אל unto ספר the book התורה׃ of the law.}%
\verse{ויעמד stood עזרא And Ezra הספר the scribe על upon מגדל a pulpit עץ of wood, אשׁר which עשׂו they had made לדבר for the purpose; ויעמד him stood אצלו and beside מתתיה Mattithiah, ושׁמע and Shema, ועניה and Anaiah, ואוריה and Urijah, וחלקיה and Hilkiah, ומעשׂיה and Maaseiah, על on ימינו his right hand; ומשׂמאלו and on his left hand, פדיה Pedaiah, ומישׁאל and Mishael, ומלכיה and Malchiah, וחשׁם and Hashum, וחשׁבדנה and Hashbadana, זכריה Zechariah, משׁלם׃ Meshullam.}%
\verse{ויפתח opened עזרא And Ezra הספר the book לעיני in the sight כל of all העם the people; כי (for מעל above כל all העם the people;) היה he was וכפתחו and when he opened עמדו stood up: כל it, all העם׃ the people}%
\verse{ויברך blessed עזרא And Ezra את יהוה the LORD, האלהים God. הגדול the great ויענו answered, כל And all העם the people אמן Amen, אמן Amen, במעל with lifting up ידיהם their hands: ויקדו and they bowed וישׁתחו their heads, and worshiped ליהוה the LORD אפים with faces ארצה׃ to the ground.}%
\verse{וישׁוע Also Jeshua, ובני and Bani, ושׁרביה and Sherebiah, ימין Jamin, עקוב Akkub, שׁבתי Shabbethai, הודיה Hodijah, מעשׂיה Maaseiah, קליטא Kelita, עזריה Azariah, יוזבד Jozabad, חנן Hanan, פלאיה Pelaiah, והלוים and the Levites, מבינים to understand את העם caused the people לתורה the law: והעם and the people על in עמדם׃}%
\verse{ויקראו So they read בספר in the book בתורת in the law האלהים of God מפרשׁ distinctly, ושׂום and gave שׂכל the sense, ויבינו and caused to understand במקרא׃ the reading.}%
\verse{ויאמר said נחמיה And Nehemiah, הוא which התרשׁתא the Tirshatha, ועזרא and Ezra הכהן the priest הספר the scribe, והלוים and the Levites המבינים that taught את העם the people, לכל unto all העם the people, היום day קדשׁ holy הוא This ליהוה unto the LORD אלהיכם your God; אל not, תתאבלו mourn ואל nor תבכו weep. כי For בוכים wept, כל all העם the people כשׁמעם when they heard את דברי the words התורה׃ of the law.}%
\verse{ויאמר Then he said להם לכו unto them, Go אכלו your way, eat משׁמנים the fat, ושׁתו and drink ממתקים the sweet, ושׁלחו and send מנות portions לאין unto them for whom nothing נכון is prepared: לו כי for קדושׁ holy היום day לאדנינו ואל neither תעצבו be ye sorry; כי for חדות the joy יהוה of the LORD היא מעזכם׃ is your strength.}%
\verse{והלוים So the Levites מחשׁים stilled לכל all העם the people, לאמר saying, הסו Hold your peace, כי for היום the day קדשׁ holy; ואל neither תעצבו׃ be ye grieved.}%
\verse{וילכו went כל And all העם the people לאכל their way to eat, ולשׁתות and to drink, ולשׁלח and to send מנות portions, ולעשׂות and to make שׂמחה mirth, גדולה great כי because הבינו they had understood בדברים the words אשׁר that הודיעו׃ were declared}%
\verseWithHeading{Festival of Booths}{וביום day השׁני And on the second נאספו were gathered together ראשׁי the chief האבות of the fathers לכל of all העם the people, הכהנים the priests, והלוים and the Levites, אל unto עזרא Ezra הספר the scribe, ולהשׂכיל even to understand אל even to understand דברי the words התורה׃ of the law.}%
\verse{וימצאו And they found כתוב written בתורה in the law אשׁר which צוה had commanded יהוה the LORD ביד by משׁה Moses, אשׁר that ישׁבו should dwell בני the children ישׂראל of Israel בסכות in booths בחג in the feast בחדשׁ month: השׁביעי׃ of the seventh}%
\verse{ואשׁר And that ישׁמיעו they should publish ויעבירו and proclaim קול and proclaim בכל in all עריהם their cities, ובירושׁלם and in Jerusalem, לאמר saying, צאו Go forth ההר unto the mount, והביאו and fetch עלי branches, זית olive ועלי branches, עץ and pine שׁמן and pine ועלי branches, הדס and myrtle ועלי branches, תמרים and palm ועלי and branches עץ trees, עבת of thick לעשׂת to make סכת booths, ככתוב׃ as written.}%
\verse{ויצאו went forth, העם So the people ויביאו and brought ויעשׂו and made להם סכות themselves booths, אישׁ every one על upon גגו the roof of his house, ובחצרתיהם and in their courts, ובחצרות and in the courts בית of the house האלהים of God, וברחוב and in the street שׁער gate, המים of the water וברחוב and in the street שׁער of the gate אפרים׃ of Ephraim.}%
\verse{ויעשׂו made כל And all הקהל the congregation השׁבים of them that were come again מן out of השׁבי the captivity סכות booths, וישׁבו and sat בסכות under the booths: כי for לא had not עשׂו done מימי since the days ישׁוע of Jeshua בן the son נון of Nun כן so. בני the children ישׂראל of Israel עד unto היום day ההוא that ותהי And there was שׂמחה gladness. גדולה great מאד׃ very}%
\verse{ויקרא he read בספר in the book תורת of the law האלהים of God. יום Also day ביום by day, מן from היום day הראשׁון the first עד unto היום day, האחרון the last ויעשׂו And they kept חג the feast שׁבעת seven ימים days; וביום day השׁמיני and on the eighth עצרת a solemn assembly, כמשׁפט׃ according unto the manner.}%
\end{biblechapter}%
\begin{biblechapter}% Nehemiah 9
\verseWithHeading{The Nation Confesses}{וביום day עשׂרים Now in the twenty וארבעה and fourth לחדשׁ month הזה of this נאספו were assembled בני the children ישׂראל of Israel בצום with fasting, ובשׂקים and with sackclothes, ואדמה and earth עליהם׃ upon}%
\verse{ויבדלו separated themselves זרע And the seed ישׂראל of Israel מכל from all בני strangers, נכר strangers, ויעמדו and stood ויתודו and confessed על and confessed חטאתיהם their sins, ועונות and the iniquities אבתיהם׃ of their fathers.}%
\verse{ויקומו And they stood up על in עמדם their place, ויקראו and read בספר in the book תורת of the law יהוה of the LORD אלהיהם their God רבעית fourth היום part of the day; ורבעית and fourth מתודים part they confessed, ומשׁתחוים and worshiped ליהוה the LORD אלהיהם׃ their God.}%
\verse{ויקם Then stood up על upon מעלה the stairs, הלוים of the Levites, ישׁוע Jeshua, ובני and Bani, קדמיאל Kadmiel, שׁבניה Shebaniah, בני Bunni, שׁרביה Sherebiah, בני Bani, כנני Chenani, ויזעקו and cried בקול voice גדול with a loud אל unto יהוה the LORD אלהיהם׃ their God.}%
\verse{ויאמרו said, הלוים Then the Levites, ישׁוע Jeshua, וקדמיאל and Kadmiel, בני Bani, חשׁבניה Hashabniah, שׁרביה Sherebiah, הודיה Hodijah, שׁבניה Shebaniah, פתחיה Pethahiah, קומו Stand up ברכו bless את יהוה the LORD אלהיכם your God מן forever העולם forever עד and העולם ever: ויברכו and blessed שׁם name, כבודך be thy glorious ומרומם which is exalted על above כל all ברכה blessing ותהלה׃ and praise.}%
\verse{אתה Thou, הוא thou, יהוה LORD לבדך alone; את עשׂית hast made את השׁמים heaven, שׁמי the heaven השׁמים of heavens, וכל with all צבאם their host, הארץ the earth, וכל and all אשׁר that עליה therein, הימים the seas, וכל and all אשׁר that בהם ואתה thou מחיה preservest את כלם them all; וצבא and the host השׁמים of heaven לך משׁתחוים׃ worshipeth}%
\verse{אתה Thou הוא יהוה the LORD האלהים the God, אשׁר who בחרת didst choose באברם Abram, והוצאתו and broughtest him forth מאור כשׂדים of the Chaldees, ושׂמת and gavest שׁמו him the name אברהם׃ of Abraham;}%
\verse{ומצאת And foundest את לבבו his heart נאמן faithful לפניך before וכרות thee, and madest עמו with הברית a covenant לתת him to give את ארץ the land הכנעני of the Canaanites, החתי the Hittites, האמרי the Amorites, והפרזי and the Perizzites, והיבוסי and the Jebusites, והגרגשׁי and the Girgashites, לתת to give לזרעו to his seed, ותקם and hast performed את דבריך thy words; כי for צדיק righteous: אתה׃ thou}%
\verse{ותרא And didst see את עני the affliction אבתינו of our fathers במצרים in Egypt, ואת זעקתם their cry שׁמעת and heardest על by ים sea; סוף׃ the Red}%
\verse{ותתן And showedst אתת signs ומפתים and wonders בפרעה upon Pharaoh, ובכל and on all עבדיו his servants, ובכל and on all עם the people ארצו of his land: כי for ידעת thou knewest כי that הזידו they dealt proudly עליהם against ותעשׂ them. So didst thou get לך שׁם thee a name, כהיום day. הזה׃ as this}%
\verse{והים the sea בקעת And thou didst divide לפניהם before ויעברו them, so that they went through בתוך the midst הים of the sea ביבשׁה on the dry land; ואת רדפיהם and their persecutors השׁלכת thou threwest במצולת into the deeps, כמו as אבן a stone במים waters. עזים׃ into the mighty}%
\verse{ובעמוד pillar; ענן by a cloudy הנחיתם Moreover thou leddest יומם them in the day ובעמוד by a pillar אשׁ of fire, לילה and in the night להאיר to give them light להם אתדרך in the way אשׁר wherein ילכו׃ they should go.}%
\verse{ועל also upon הר mount סיני Sinai, ירדת Thou camest down ודבר and spakest עמהם with משׁמים them from heaven, ותתן and gavest להם משׁפטים judgments, ישׁרים them right ותורות laws, אמת and true חקים statutes ומצות and commandments: טובים׃ good}%
\verse{ואת שׁבת sabbath, קדשׁך unto them thy holy הודעת And madest known להם ומצוות them precepts, וחקים statutes, ותורה and laws, צוית and commandedst להם ביד by the hand משׁה of Moses עבדך׃ thy servant:}%
\verse{ולחם them bread משׁמים from heaven נתתה And gavest להם לרעבם for their hunger, ומים water מסלע for them out of the rock הוצאת and broughtest forth להם לצמאם for their thirst, ותאמר and promisedst להם לבוא them that they should go in לרשׁת to possess את הארץ the land אשׁר which נשׂאת אתדך לתת׃ to give}%
\verse{והם But they ואבתינו and our fathers הזידו dealt proudly, ויקשׁו and hardened את ערפם their necks, ולא not שׁמעו and hearkened אל to מצותיך׃ thy commandments,}%
\verse{וימאנו And refused לשׁמע to obey, ולא neither זכרו were mindful נפלאתיך of thy wonders אשׁר that עשׂית thou didst עמהם among ויקשׁו them; but hardened את ערפם their necks, ויתנו appointed ראשׁ a captain לשׁוב to return לעבדתם to their bondage: במרים and in their rebellion ואתה but thou אלוה a God סליחות ready to pardon, חנון gracious ורחום and merciful, ארך slow אפים to anger, ורב and of great וחסד kindness, ולא them not. עזבתם׃ and forsookest}%
\verse{אף Yea, כי when עשׂו they had made להם עגל calf, מסכה them a molten ויאמרו and said, זה This אלהיך thy God אשׁר that העלך brought thee up ממצרים ויעשׂו and had wrought נאצות provocations; גדלות׃ great}%
\verse{ואתה Yet thou ברחמיך mercies הרבים in thy manifold לא them not עזבתם forsookest במדבר in the wilderness: את עמוד the pillar הענן of the cloud לא not סר departed מעליהם from ביומם them by day, להנחתם to lead בהדרך them in the way; ואת עמוד neither the pillar האשׁ of fire בלילה by night, להאיר to show them light, להם ואת הדרך and the way אשׁר wherein ילכו׃ they should go.}%
\verse{ורוחך spirit הטובה also thy good נתת Thou gavest להשׂכילם to instruct ומנך thy manna לא not מנעת them, and withheldest מפיהם from their mouth, ומים them water נתתה and gavest להם לצמאם׃ for their thirst.}%
\verse{וארבעים Yea, forty שׁנה years כלכלתם didst thou sustain במדבר them in the wilderness, לא nothing; חסרו they lacked שׂלמתיהם their clothes לא waxed not old, בלו waxed not old, ורגליהם and their feet לא not. בצקו׃ swelled}%
\verse{ותתן Moreover thou gavest להם ממלכות them kingdoms ועממים and nations, ותחלקם and didst divide לפאה them into corners: ויירשׁו so they possessed את ארץ the land סיחון of Sihon, ואת ארץ and the land מלך of the king חשׁבון of Heshbon, ואת ארץ and the land עוג of Og מלך king הבשׁן׃ of Bashan.}%
\verse{ובניהם Their children הרבית also multipliedst כככבי thou as the stars השׁמים of heaven, ותביאם and broughtest אל them into הארץ the land, אשׁר concerning which אמרת thou hadst promised לאבתיהם to their fathers, לבוא that they should go in לרשׁת׃ to possess}%
\verse{ויבאו went in הבנים So the children ויירשׁו and possessed את הארץ the land, ותכנע and thou subduedst לפניהם before את ישׁבי them the inhabitants הארץ of the land, הכנענים the Canaanites, ותתנם and gavest בידם them into their hands, ואת מלכיהם with their kings, ואת עממי and the people הארץ of the land, לעשׂות that they might do בהם כרצונם׃ with them as they would.}%
\verse{וילכדו And they took ערים cities, בצרות strong ואדמה land, שׁמנה and a fat ויירשׁו and possessed בתים houses מלאים full כל of all טוב goods, ברות wells חצובים digged, כרמים vineyards, וזיתים and oliveyards, ועץ trees מאכל and fruit לרב in abundance: ויאכלו so they did eat, וישׂבעו and were filled, וישׁמינו and became fat, ויתעדנו and delighted themselves בטובך goodness. הגדול׃ in thy great}%
\verse{וימרו Nevertheless they were disobedient, וימרדו and rebelled בך וישׁלכו against thee, and cast את תורתך thy law אחרי behind גום their backs, ואת נביאיך thy prophets הרגו and slew אשׁר which העידו testified בם להשׁיבם against them to turn אליך them to ויעשׂו thee, and they wrought נאצות provocations. גדולת׃ great}%
\verse{ותתנם Therefore thou deliveredst ביד them into the hand צריהם of their enemies, ויצרו להם ובעת them: and in the time צרתם of their trouble, יצעקו when they cried אליך unto ואתה thee, thou משׁמים from heaven; תשׁמע heardest וכרחמיך mercies הרבים and according to thy manifold תתן thou gavest להם מושׁיעים them saviors, ויושׁיעום who saved מיד them out of the hand צריהם׃ of their enemies.}%
\verse{וכנוח But after they had rest, להם ישׁובו again לעשׂות they did רע evil לפניך before ותעזבם thee: therefore leftest ביד thou them in the hand איביהם of their enemies, וירדו so that they had the dominion בהם וישׁובו over them: yet when they returned, ויזעקוך and cried ואתה unto thee, thou משׁמים from heaven; תשׁמע heardest ותצילם didst thou deliver כרחמיך them according to thy mercies; רבות and many עתים׃ times}%
\verse{ותעד And testifiedst בהם להשׁיבם against them, that thou mightest bring them again אל unto תורתך thy law: והמה yet they הזידו dealt proudly, ולא not שׁמעו and hearkened למצותיך unto thy commandments, ובמשׁפטיך against thy judgments, חטאו but sinned בם אשׁר (which יעשׂה do, אדם if a man וחיה he shall live בהם ויתנו כתף the shoulder, סוררת וערפם their neck, הקשׁו and hardened ולא and would not שׁמעו׃ hear.}%
\verse{ותמשׁך didst thou forbear עליהם didst thou forbear שׁנים years רבות Yet many ותעד them, and testifiedst בם ברוחך against them by thy spirit ביד in נביאיך thy prophets: ולא yet would they not האזינו give ear: ותתנם therefore gavest ביד thou them into the hand עמי of the people הארצת׃ of the lands.}%
\verse{וברחמיך mercies' הרבים Nevertheless for thy great לא sake thou didst not עשׂיתם utterly consume כלה utterly consume ולא them, nor עזבתם forsake כי them; for אל God. חנון a gracious ורחום and merciful אתה׃ thou}%
\verse{ועתה Now אלהינו therefore, our God, האל God, הגדול the great, הגבור the mighty, והנורא and the terrible שׁומר who keepest הברית covenant והחסד and mercy, אל let not ימעט seem little לפניך before את כל all התלאה the trouble אשׁר thee, that מצאתנו hath come upon למלכינו us, on our kings, לשׂרינו on our princes, ולכהנינו and on our priests, ולנביאנו and on our prophets, ולאבתינו and on our fathers, ולכל and on all עמך thy people, מימי since the time מלכי of the kings אשׁור of Assyria עד unto היום day. הזה׃ this}%
\verse{ואתה Howbeit thou צדיק just על in כל all הבא that is brought עלינו upon כי us; for אמת right, עשׂית thou hast done ואנחנו but we הרשׁענו׃ have done wickedly:}%
\verse{ואת מלכינו Neither have our kings, שׂרינו our princes, כהנינו our priests, ואבתינו our fathers, לא nor עשׂו kept תורתך thy law, ולא nor הקשׁיבו hearkened אל unto מצותיך thy commandments ולעדותיך and thy testimonies, אשׁר wherewith העידת׃ thou didst testify}%
\verse{והם For they במלכותם thee in their kingdom, ובטובך goodness הרב and in thy great אשׁר that נתת thou gavest להם ובארץ land הרחבה them, and in the large והשׁמנה and fat אשׁר which נתת thou gavest לפניהם before לא have not עבדוך served ולא them, neither שׁבו turned ממעלליהם works. הרעים׃ they from their wicked}%
\verse{הנה Behold, אנחנו we היום this day, עבדים servants והארץ and the land אשׁר that נתתה thou gavest לאבתינו unto our fathers לאכל to eat את פריה the fruit ואת טובה thereof and the good הנה thereof, behold, אנחנו we עבדים servants עליה׃ in}%
\verse{ותבואתה increase מרבה And it yieldeth much למלכים unto the kings אשׁר whom נתתה thou hast set עלינו over בחטאותינו us because of our sins: ועל over גויתינו our bodies, משׁלים also they have dominion ובבהמתנו and over our cattle, כרצונם at their pleasure, ובצרה distress. גדולה in great אנחנו׃ and we}%
\verse{ובכל And because of all זאת this אנחנו we כרתים make אמנה a sure וכתבים and write ועל seal החתום seal שׂרינו and our princes, לוינו Levites, כהנינו׃ priests,}%
\end{biblechapter}%
\begin{biblechapter}% Nehemiah 10
\verseWithHeading{The Names of Those Who Signed the Covenant}{ועל החתומיםחמיה Nehemiah, התרשׁתא the Tirshatha, בן the son חכליה of Hachaliah, וצדקיה׃ and Zidkijah,}%
\verse{שׂריה Seraiah, עזריה Azariah, ירמיה׃ Jeremiah,}%
\verse{פשׁחור Pashur, אמריה Amariah, מלכיה׃ Malchijah,}%
\verse{חטושׁ Hattush, שׁבניה Shebaniah, מלוך׃ Malluch,}%
\verse{חרם Harim, מרמות Meremoth, עבדיה׃ Obadiah,}%
\verse{דניאל Daniel, גנתון Ginnethon, ברוך׃ Baruch,}%
\verse{משׁלם Meshullam, אביה Abijah, מימן׃ Mijamin,}%
\verse{מעזיה Maaziah, בלגי Bilgai, שׁמעיה Shemaiah: אלה these הכהנים׃ the priests.}%
\verse{והלוים And the Levites: וישׁוע both Jeshua בן the son אזניה of Azaniah, בנוי Binnui מבני of the sons חנדד of Henadad, קדמיאל׃ Kadmiel;}%
\verse{ואחיהם And their brethren, שׁבניה Shebaniah, הודיה קליטא Kelita, פלאיה Pelaiah, חנן׃ Hanan,}%
\verse{מיכא Micha, רחוב Rehob, חשׁביה׃ Hashabiah,}%
\verse{זכור Zaccur, שׁרביה Sherebiah, שׁבניה׃ Shebaniah,}%
\verse{הודיה בני Bani, בנינו׃ Beninu.}%
\verse{ראשׁי The chief העם of the people; פרעשׁ Parosh, פחת מואב Pahath-moab, עילם Elam, זתוא Zatthu, בני׃ Bani,}%
\verse{בני Bunni, עזגד Azgad, בבי׃ Bebai,}%
\verse{אדניה Adonijah, בגוי Bigvai, עדין׃ Adin,}%
\verse{אטר Ater, חזקיה Hizkijah, עזור׃ Azzur,}%
\verse{הודיה חשׁם Hashum, בצי׃ Bezai,}%
\verse{חריף Hariph, ענתות Anathoth, נובי׃ Nebai,}%
\verse{מגפיעשׁ Magpiash, משׁלם Meshullam, חזיר׃ Hezir,}%
\verse{משׁיזבאל Meshezabeel, צדוק Zadok, ידוע׃ Jaddua,}%
\verse{פלטיה Pelatiah, חנן Hanan, עניה׃ Anaiah,}%
\verse{הושׁע Hoshea, חנניה Hananiah, חשׁוב׃ Hashub,}%
\verse{הלוחשׁ Hallohesh, פלחא Pileha, שׁובק׃ Shobek,}%
\verse{רחום Rehum, חשׁבנה Hashabnah, מעשׂיה׃ Maaseiah,}%
\verse{ואחיה And Ahijah, חנן Hanan, ענן׃ Anan,}%
\verse{מלוך Malluch, חרם Harim, בענה׃ Baanah.}%
\verseWithHeading{A Summary of the Covenant}{ושׁאר And the rest העם of the people, הכהנים the priests, הלוים the Levites, השׁוערים the porters, המשׁררים the singers, הנתינים the Nethinims, וכל and all הנבדל they that had separated themselves מעמי from the people הארצות of the lands אל unto תורת the law האלהים of God, נשׁיהם their wives, בניהם their sons, ובנתיהם and their daughters, כל every יודע one having knowledge, מבין׃ and having understanding;}%
\verse{מחזיקים They cleaved על to אחיהם their brethren, אדיריהם their nobles, ובאים and entered באלה into a curse, ובשׁבועה and into an oath, ללכת to walk בתורת law, האלהים in God's אשׁר which נתנה was given ביד by משׁה Moses עבד the servant האלהים of God, ולשׁמור and to observe ולעשׂות and do את כל all מצות the commandments יהוה of the LORD אדנינו ומשׁפטיו and his judgments וחקיו׃ and his statutes;}%
\verse{ואשׁר And that לא we would not נתן give בנתינו our daughters לעמי unto the people הארץ of the land, ואת בנתיהם their daughters לא nor נקח take לבנינו׃ for our sons:}%
\verse{ועמי And the people הארץ of the land המביאים bring את המקחות ware וכל or any שׁבר victuals ביום day השׁבת on the sabbath למכור to sell, לא we would not נקח buy מהם בשׁבת them on the sabbath, וביום day: קדשׁ or on the holy ונטשׁ and we would leave את השׁנה year, השׁביעית the seventh ומשׁא and the exaction כל of every יד׃ debt.}%
\verse{והעמדנו Also we made עלינו for מצות ordinances לתת us, to charge עלינו ourselves שׁלשׁית with the third part השׁקל of a shekel בשׁנה yearly לעבדת for the service בית of the house אלהינו׃ of our God;}%
\verse{ללחם המערכתמנחת meat offering, התמיד and for the continual ולעולת burnt offering, התמיד and for the continual השׁבתות of the sabbaths, החדשׁים of the new moons, למועדים for the set feasts, ולקדשׁים and for the holy ולחטאות and for the sin offerings לכפר to make an atonement על for ישׂראל Israel, וכל and all מלאכת the work בית of the house אלהינו׃ of our God.}%
\verse{והגורלות הפלנול for קרבן offering, העצים the wood הכהנים the priests, הלוים the Levites, והעם and the people, להביא to bring לבית into the house אלהינו of our God, לבית after the houses אבתינו of our fathers, לעתים at times מזמנים appointed שׁנה year בשׁנה by year, לבער to burn על upon מזבח the altar יהוה of the LORD אלהינו our God, ככתוב as written בתורה׃ in the law:}%
\verse{ולהביא And to bring את בכורי the firstfruits אדמתנו of our ground, ובכורי and the firstfruits כל of all פרי fruit כל of all עץ trees, שׁנה year בשׁנה by year, לבית unto the house יהוה׃ of the LORD:}%
\verse{ואת בכרות Also the firstborn בנינו of our sons, ובהמתינו and of our cattle, ככתוב as written בתורה in the law, ואת בכוריקרינו of our herds וצאנינו and of our flocks, להביא to bring לבית to the house אלהינו of our God, לכהנים unto the priests המשׁרתים that minister בבית in the house אלהינו׃ of our God:}%
\verse{ואת ראשׁית the firstfruits עריסתינו of our dough, ותרומתינו and our offerings, ופרי and the fruit כל of all עץ manner of trees, תירושׁ of wine ויצהר and of oil, נביא And we should bring לכהנים unto the priests, אל to לשׁכות the chambers בית of the house אלהינו of our God; ומעשׂר and the tithes אדמתנו of our ground ללוים unto the Levites, והם that the same הלוים Levites המעשׂרים might have the tithes בכל in all ערי the cities עבדתנו׃ of our tillage.}%
\verse{והיה shall be הכהן And the priest בן the son אהרן of Aaron עם with הלוים the Levites, בעשׂר take tithes: הלוים when the Levites והלוים and the Levites יעלו shall bring up את מעשׂר the tithe המעשׂר of the tithes לבית unto the house אלהינו of our God, אל to הלשׁכות the chambers, לבית house. האוצר׃ into the treasure}%
\verse{כי For אל unto הלשׁכות the chambers, יביאו shall bring בני the children ישׂראל of Israel ובני and the children הלוי of Levi את תרומת the offering הדגן of the corn, התירושׁ of the new wine, והיצהר and the oil, ושׁם where כלי the vessels המקדשׁ of the sanctuary, והכהנים and the priests המשׁרתים that minister, והשׁוערים and the porters, והמשׁררים and the singers: ולא and we will not נעזב forsake את בית the house אלהינו׃ of our God.}%
\end{biblechapter}%
\begin{biblechapter}% Nehemiah 11
\verseWithHeading{The Leaders and Servants in Jerusalem}{וישׁבו dwelt שׂרי And the rulers העם of the people בירושׁלם at Jerusalem: ושׁאר the rest העם of the people הפילו also cast גורלות lots, להביא to bring אחד one מן of העשׂרה ten לשׁבת to dwell בירושׁלם in Jerusalem עיר city, הקדשׁ the holy ותשׁע and nine הידות parts בערים׃ in cities.}%
\verse{ויברכו blessed העם And the people לכל all האנשׁים the men, המתנדבים that willingly offered themselves לשׁבת to dwell בירושׁלם׃ at Jerusalem.}%
\verse{ואלה Now these ראשׁי the chief המדינה of the province אשׁר that ישׁבו dwelt בירושׁלם in Jerusalem: ובערי but in the cities יהודה of Judah ישׁבו dwelt אישׁ every one באחזתו in his possession בעריהם in their cities, ישׂראל Israel, הכהנים the priests, והלוים and the Levites, והנתינים and the Nethinims, ובני and the children עבדי servants. שׁלמה׃ of Solomon's}%
\verse{ובירושׁלם And at Jerusalem ישׁבו dwelt מבני of the children יהודה of Judah, ומבני and of the children בנימן of Benjamin. מבני the son יהודה of Judah; עתיה Athaiah בן the son עזיה of Uzziah, בן the son זכריה of Zechariah, בן the son אמריה of Amariah, בן the son שׁפטיה of Shephatiah, בן of the children מהללאל of Mahalaleel, מבני פרץ׃ of Perez;}%
\verse{ומעשׂיה And Maaseiah בן the son ברוך of Baruch, בן the son כל חזה of Col-hozeh, בן the son חזיה of Hazaiah, בן the son עדיה of Adaiah, בן the son יויריב of Joiarib, בן the son זכריה of Zechariah, בן the son השׁלני׃}%
\verse{כל All בני the sons פרץ of Perez הישׁבים that dwelt בירושׁלם at Jerusalem ארבע four מאות hundred שׁשׁים threescore ושׁמנה and eight אנשׁי חיל׃ valiant}%
\verse{ואלה And these בני the sons בנימן of Benjamin; סלא Sallu בן the son משׁלם of Meshullam, בן the son יועד of Joed, בן the son פדיה of Pedaiah, בן the son קוליה of Kolaiah, בן the son מעשׂיה of Maaseiah, בן the son איתיאל of Ithiel, בן the son ישׁעיה׃ of Jesaiah.}%
\verse{ואחריו And after גבי him Gabbai, סלי Sallai, תשׁע nine מאות hundred עשׂרים twenty ושׁמנה׃ and eight.}%
\verse{ויואל And Joel בן the son זכרי of Zichri פקיד their overseer: עליהם their overseer: ויהודה and Judah בן the son הסנואה of Senuah על over העיר the city. משׁנה׃ second}%
\verse{מן Of הכהנים the priests: ידעיה Jedaiah בן the son יויריב of Joiarib, יכין׃ Jachin.}%
\verse{שׂריה Seraiah בן the son חלקיה of Hilkiah, בן the son משׁלם of Meshullam, בן the son צדוק of Zadok, בן the son מריות of Meraioth, בן the son אחיטוב of Ahitub, נגד the ruler בית of the house האלהים׃ of God.}%
\verse{ואחיהם And their brethren עשׂי that did המלאכה the work לבית of the house שׁמנה eight מאות hundred עשׂרים twenty ושׁנים and two: ועדיה and Adaiah בן the son ירחם of Jeroham, בן the son פלליה of Pelaliah, בן the son אמצי of Amzi, בן the son זכריה of Zechariah, בן the son פשׁחור of Pashur, בן the son מלכיה׃ of Malchiah,}%
\verse{ואחיו And his brethren, ראשׁים chief לאבות of the fathers, מאתים two hundred ארבעים forty ושׁנים and two: ועמשׁסי and Amashai בן the son עזראל of Azareel, בן the son אחזי of Ahasai, בן the son משׁלמות of Meshillemoth, בן the son אמר׃ of Immer,}%
\verse{ואחיהם And their brethren, גבורי mighty men חיל of valor, מאה a hundred עשׂרים twenty ושׁמנה and eight: ופקיד and their overseer עליהם and their overseer זבדיאל Zabdiel, בן the son הגדולים׃ of the great}%
\verse{ומן Also of הלוים the Levites: שׁמעיה Shemaiah בן the son חשׁוב of Hashub, בן the son עזריקם of Azrikam, בן the son חשׁביה of Hashabiah, בן the son בוני׃ of Bunni;}%
\verse{ושׁבתי And Shabbethai ויוזבד and Jozabad, על the oversight המלאכה business החיצנה of the outward לבית of the house האלהים of God. מראשׁי of the chief הלוים׃ of the Levites,}%
\verse{ומתניה And Mattaniah בן the son מיכה of Micha, בן the son זבדי of Zabdi, בן the son אסף of Asaph, ראשׁ the principal התחלה to begin יהודה the thanksgiving לתפלה in prayer: ובקבקיה and Bakbukiah משׁנה the second מאחיו among his brethren, ועבדא and Abda בן the son שׁמוע of Shammua, בן the son גלל of Galal, בן the son ידיתון׃ of Jeduthun.}%
\verse{כל All הלוים the Levites בעיר city הקדשׁ in the holy מאתים two hundred שׁמנים fourscore וארבעה׃ and four.}%
\verse{והשׁוערים Moreover the porters, עקוב Akkub, טלמון Talmon, ואחיהם and their brethren השׁמרים that kept בשׁערים the gates, מאה a hundred שׁבעים seventy ושׁנים׃ and two.}%
\verse{ושׁאר And the residue ישׂראל of Israel, הכהנים of the priests, הלוים the Levites, בכל in all ערי the cities יהודה of Judah, אישׁ every one בנחלתו׃ in his inheritance.}%
\verse{והנתינים But the Nethinims ישׁבים dwelt בעפל in Ophel: וציחא and Ziha וגשׁפא and Gispa על over הנתינים׃ the Nethinims.}%
\verse{ופקיד The overseer הלוים also of the Levites בירושׁלם at Jerusalem עזי Uzzi בן the son בני of Bani, בן the son חשׁביה of Hashabiah, בן the son מתניה of Mattaniah, בן the son מיכא of Micha. מבני אסף of Asaph, המשׁררים the singers לנגד over מלאכת the business בית of the house האלהים׃ of God.}%
\verse{כי For מצות commandment המלך the king's עליהם concerning ואמנה them, that a certain portion על should be for המשׁררים the singers, דבר due יום for every day. ביומו׃ for every day.}%
\verse{ופתחיה And Pethahiah בן the son משׁיזבאל of Meshezabeel, מבני of the children זרח of Zerah בן the son יהודה of Judah, ליד hand המלך at the king's לכל in all דבר matters לעם׃ concerning the people.}%
\verseWithHeading{Villages Outside of Jerusalem}{ואל And for החצרים the villages, בשׂדתם with their fields, מבני of the children יהודה of Judah ישׁבו dwelt בקרית הארבע at Kirjath-arba, ובנתיה and the villages ובדיבן thereof, and at Dibon, ובנתיה and the villages וביקבצאל thereof, and at Jekabzeel, וחצריה׃ and the villages}%
\verse{ובישׁוע And at Jeshua, ובמולדה and at Moladah, ובבית פלט׃ and at Beth-phelet,}%
\verse{ובחצר שׁועל And at Hazar-shual, ובבאר שׁבע and at Beer-sheba, ובנתיה׃ and the villages}%
\verse{ובצקלג And at Ziklag, ובמכנה and at Mekonah, ובבנתיה׃ and in the villages}%
\verse{ובעין רמון And at En-rimmon, ובצרעה and at Zareah, ובירמות׃ and at Jarmuth,}%
\verse{זנח Zanoah, עדלם Adullam, וחצריהם and their villages, לכישׁ at Lachish, ושׂדתיה and the fields עזקה thereof, at Azekah, ובנתיה and the villages ויחנו thereof. And they dwelt מבאר שׁבע עד unto גיא the valley הנם׃ of Hinnom.}%
\verse{ובני The children בנימן also of Benjamin מגבע מכמשׂ at Michmash, ועיה and Aija, ובית אל and Bethel, ובנתיה׃ and their villages,}%
\verse{ענתות at Anathoth, נב Nob, ענניה׃ Ananiah,}%
\verse{חצור Hazor, רמה Ramah, גתים׃ Gittaim,}%
\verse{חדיד Hadid, צבעים Zeboim, נבלט׃ Neballat,}%
\verse{לד Lod, ואונו and Ono, גי the valley החרשׁים׃}%
\verse{ומן And of הלוים the Levites מחלקות divisions יהודה Judah, לבנימין׃ in Benjamin.}%
\end{biblechapter}%
\begin{biblechapter}% Nehemiah 12
\verseWithHeading{The Leaders and Servants in Jerusalem}{ואלה Now these הכהנים the priests והלוים and the Levites אשׁר that עלו went up עם with זרבבל Zerubbabel בן the son שׁאלתיאל of Shealtiel, וישׁוע and Jeshua: שׂריה Seraiah, ירמיה Jeremiah, עזרא׃ Ezra,}%
\verse{אמריה Amariah, מלוך Malluch, חטושׁ׃ Hattush,}%
\verse{שׁכניה Shechaniah, רחם Rehum, מרמת׃ Meremoth,}%
\verse{עדוא Iddo, גנתוי Ginnetho, אביה׃ Abijah,}%
\verse{מימין Miamin, מעדיה Maadiah, בלגה׃ Bilgah,}%
\verse{שׁמעיה Shemaiah, ויויריב and Joiarib, ידעיה׃ Jedaiah,}%
\verse{סלו Sallu, עמוק Amok, חלקיה Hilkiah, ידעיה Jedaiah. אלה These ראשׁי the chief הכהנים of the priests ואחיהם and of their brethren בימי in the days ישׁוע׃ of Jeshua.}%
\verse{והלוים Moreover the Levites: ישׁוע Jeshua, בנוי Binnui, קדמיאל Kadmiel, שׁרביה Sherebiah, יהודה Judah, מתניה Mattaniah, על over הידות the thanksgiving, הוא he ואחיו׃ and his brethren.}%
\verse{ובקבקיה Also Bakbukiah וענו and Unni, אחיהם their brethren, לנגדם over against למשׁמרות׃ them in the watches.}%
\verse{וישׁוע And Jeshua הוליד begot את יויקים Joiakim, ויויקים Joiakim הוליד also begot את אלישׁיב Eliashib, ואלישׁיב and Eliashib את יוידע׃ Joiada,}%
\verse{ויוידע And Joiada הוליד begot את יונתן Jonathan, ויונתן and Jonathan הוליד begot את ידוע׃ Jaddua.}%
\verse{ובימי And in the days יויקים of Joiakim היו were כהנים priests, ראשׁי the chief האבות of the fathers: לשׂריה of Seraiah, מריה Meraiah; לירמיה of Jeremiah, חנניה׃ Hananiah;}%
\verse{לעזרא Of Ezra, משׁלם Meshullam; לאמריה of Amariah, יהוחנן׃ Jehohanan;}%
\verse{למלוכי Of Melicu, יונתן Jonathan; לשׁבניה of Shebaniah, יוסף׃ Joseph;}%
\verse{לחרם Of Harim, עדנא Adna; למריות of Meraioth, חלקי׃ Helkai;}%
\verse{לעדיא Of Iddo, זכריה Zechariah; לגנתון of Ginnethon, משׁלם׃ Meshullam;}%
\verse{לאביה Of Abijah, זכרי Zichri; למנימין of Miniamin, למועדיה of Moadiah, פלטי׃ Piltai;}%
\verse{לבלגה Of Bilgah, שׁמוע Shammua; לשׁמעיה of Shemaiah, יהונתן׃ Jehonathan;}%
\verse{וליויריב And of Joiarib, מתני Mattenai; לידעיה of Jedaiah, עזי׃ Uzzi;}%
\verse{לסלי Of Sallai, קלי Kallai; לעמוק of Amok, עבר׃ Eber;}%
\verse{לחלקיה Of Hilkiah, חשׁביה Hashabiah; לידעיה of Jedaiah, נתנאל׃ Nethaneel.}%
\verse{הלוים The Levites בימי in the days אלישׁיב of Eliashib, יוידע Joiada, ויוחנן and Johanan, וידוע and Jaddua, כתובים recorded ראשׁי chief אבות of the fathers: והכהנים also the priests, על to מלכות the reign דריושׁ of Darius הפרסי׃ the Persian.}%
\verse{בני The sons לוי of Levi, ראשׁי the chief האבות of the fathers, כתובים written על in ספר the book דברי of the chronicles, הימים the days ועד even until ימי יוחנן of Johanan בן the son אלישׁיב׃ of Eliashib.}%
\verse{וראשׁי And the chief הלוים of the Levites: חשׁביה Hashabiah, שׁרביה Sherebiah, וישׁוע and Jeshua בן the son קדמיאל of Kadmiel, ואחיהם with their brethren לנגדם over against להלל them, to praise להודות to give thanks, במצות according to the commandment דויד of David אישׁ the man האלהים of God, משׁמר ward לעמת over against משׁמר׃ ward.}%
\verse{מתניה Mattaniah, ובקבקיה and Bakbukiah, עבדיה Obadiah, משׁלם Meshullam, טלמון Talmon, עקוב Akkub, שׁמרים keeping שׁוערים porters משׁמר the ward באספי at the thresholds השׁערים׃ of the gates.}%
\verse{אלה These בימי in the days יויקים of Joiakim בן the son ישׁוע of Jeshua, בן the son יוצדק of Jozadak, ובימי and in the days נחמיה of Nehemiah הפחה the governor, ועזרא and of Ezra הכהן the priest, הסופר׃ the scribe.}%
\verseWithHeading{Dedication of the Wall of Jerusalem}{ובחנכת And at the dedication חומת of the wall ירושׁלם of Jerusalem בקשׁו they sought את הלוים the Levites מכל out of all מקומתם their places, להביאם to bring לירושׁלם them to Jerusalem, לעשׂת to keep חנכה the dedication ושׂמחה with gladness, ובתודות both with thanksgivings, ובשׁיר and with singing, מצלתים cymbals, נבלים psalteries, ובכנרות׃ and with harps.}%
\verse{ויאספו gathered themselves together, בני And the sons המשׁררים of the singers ומן both out of הככר the plain country סביבות round about ירושׁלם Jerusalem, ומן and from חצרי the villages נטפתי׃ of Netophathi;}%
\verse{ומבית הגלגל of Gilgal, ומשׂדות and out of the fields גבע of Geba ועזמות and Azmaveth: כי for חצרים them villages בנו had built להם המשׁררים the singers סביבות round about ירושׁלם׃ Jerusalem.}%
\verse{ויטהרו purified themselves, הכהנים And the priests והלוים and the Levites ויטהרו and purified את העם the people, ואת השׁערים and the gates, ואת החומה׃ and the wall.}%
\verse{ואעלה Then I brought up את שׂרי the princes יהודה of Judah מעל upon לחומה the wall, ואעמידה and appointed שׁתי two תודת thanks, גדולת great ותהלכת לימין on the right hand מעל upon לחומה the wall לשׁער gate: האשׁפת׃ toward the dung}%
\verse{וילך them went אחריהם And after הושׁעיה Hoshaiah, וחצי and half שׂרי of the princes יהודה׃ of Judah,}%
\verse{ועזריה And Azariah, עזרא Ezra, ומשׁלם׃ and Meshullam,}%
\verse{יהודה Judah, ובנימן and Benjamin, ושׁמעיה and Shemaiah, וירמיה׃ and Jeremiah,}%
\verse{ומבני the son הכהנים בחצצרות with trumpets; זכריה Zechariah בן the son יונתן of Jonathan, בן the son שׁמעיה of Shemaiah, בן the son מתניה of Mattaniah, בן the son מיכיה of Michaiah, בן the son זכור of Zaccur, בן אסף׃ of Asaph:}%
\verse{ואחיו And his brethren, שׁמעיה Shemaiah, ועזראל and Azarael, מללי Milalai, גללי Gilalai, מעי Maai, נתנאל Nethaneel, ויהודה and Judah, חנני Hanani, בכלי instruments שׁיר with the musical דויד of David אישׁ the man האלהים of God, ועזרא and Ezra הסופר the scribe לפניהם׃ before}%
\verse{ועל And at שׁער gate, העין the fountain ונגדם which was over against עלו them, they went up על by מעלות the stairs עיר of the city דויד of David, במעלה at the going up לחומה of the wall, מעל above לבית the house דויד of David, ועד even unto שׁער gate המים the water מזרח׃ eastward.}%
\verse{והתודה thanks השׁנית And the other ההולכת went למואל over against ואני and I אחריה after וחצי them, and the half העם of the people מעל upon להחומה the wall, מעל from beyond למגדל the tower התנורים of the furnaces ועד even unto החומה wall; הרחבה׃ the broad}%
\verse{ומעל and above לשׁער the gate אפרים of Ephraim, ועל and above שׁער gate, הישׁנה ועלׁער gate, הדגים the fish ומגדל and the tower חננאל of Hananeel, ומגדל and the tower המאה of Meah, ועד even unto שׁער gate: הצאן the sheep ועמדו and they stood still בשׁער gate. המטרה׃ in the prison}%
\verse{ותעמדנה So stood שׁתי the two התודת thanks בבית in the house האלהים of God, ואני and I, וחצי and the half הסגנים of the rulers עמי׃ with}%
\verse{והכהנים And the priests; אליקים Eliakim, מעשׂיה Maaseiah, מנימין Miniamin, מיכיה Michaiah, אליועיני Elioenai, זכריה Zechariah, חנניה Hananiah, בחצצרות׃ with trumpets;}%
\verse{ומעשׂיה And Maaseiah, ושׁמעיה and Shemaiah, ואלעזר and Eleazar, ועזי and Uzzi, ויהוחנן and Jehohanan, ומלכיה and Malchijah, ועילם and Elam, ועזר and Ezer. וישׁמיעו sang loud, המשׁררים And the singers ויזרחיה with Jezrahiah הפקיד׃ overseer.}%
\verse{ויזבחו they offered ביום day ההוא Also that זבחים sacrifices, גדולים great וישׂמחו and rejoiced: כי for האלהים God שׂמחם had made them rejoice שׂמחה joy: גדולה with great וגם also הנשׁים the wives והילדים and the children שׂמחו rejoiced: ותשׁמע was heard שׂמחת so that the joy ירושׁלם of Jerusalem מרחוק׃ even afar off.}%
\verseWithHeading{Celebration at the Temple}{ויפקדו appointed ביום time ההוא And at that אנשׁים were some על over הנשׁכות the chambers לאוצרות for the treasures, לתרומות for the offerings, לראשׁית for the firstfruits, ולמעשׂרות and for the tithes, לכנוס to gather בהם לשׂדי into them out of the fields הערים of the cities מנאות the portions התורה of the law לכהנים for the priests וללוים and Levites: כי for שׂמחת rejoiced יהודה Judah על for הכהנים the priests ועל and for הלוים the Levites העמדים׃ that waited.}%
\verse{וישׁמרו kept משׁמרת the ward אלהיהם of their God, ומשׁמרת and the ward הטהרה of the purification, והמשׁררים And both the singers והשׁערים and the porters כמצות according to the commandment דויד of David, שׁלמה of Solomon בנו׃ his son.}%
\verse{כי For בימי in the days דויד of David ואסף and Asaph מקדם of old ראשׁ chief המשׁררים of the singers, ושׁיר and songs תהלה of praise והדות and thanksgiving לאלהים׃ unto God.}%
\verse{וכל And all ישׂראל Israel בימי in the days זרבבל of Zerubbabel, ובימי and in the days נחמיה of Nehemiah, נתנים gave מניות the portions המשׁררים of the singers והשׁערים and the porters, דבר his portion: יום every day ביומו every day ומקדשׁים and they sanctified ללוים unto the Levites; והלוים and the Levites מקדשׁים sanctified לבני unto the children אהרן׃ of Aaron.}%
\end{biblechapter}%
\begin{biblechapter}% Nehemiah 13
\verseWithHeading{Israel Separates Itself}{ביום day ההוא On that נקרא they read בספר in the book משׁה of Moses באזני in the audience העם of the people; ונמצא and therein was found כתוב written, בו אשׁר that לא should not יבוא come עמני the Ammonite ומאבי and the Moabite בקהל into the congregation האלהים of God עד forever; עולם׃ forever;}%
\verse{כי Because לא not קדמו they met את בני the children ישׂראל of Israel בלחם with bread ובמים and with water, וישׂכר but hired עליו against את בלעם Balaam לקללו them, that he should curse ויהפך turned אלהינו them: howbeit our God הקללה the curse לברכה׃ into a blessing.}%
\verse{ויהי Now it came to pass, כשׁמעם when they had heard את התורה the law, ויבדילו that they separated כל all ערב the mixed multitude. מישׂראל׃}%
\verseWithHeading{Nehemiah Brings Reform}{ולפני And before מזה this, אלישׁיב Eliashib הכהן the priest, נתון having the oversight בלשׁכת of the chamber בית of the house אלהינו of our God, קרוב allied לטוביה׃ unto Tobiah:}%
\verse{ויעשׂ And he had prepared לו לשׁכה chamber, גדולה for him a great ושׁם where היו formerly לפנים formerly נתנים they laid את המנחה the meat offerings, הלבונה the frankincense, והכלים and the vessels, ומעשׂר and the tithes הדגן of the corn, התירושׁ the new wine, והיצהר and the oil, מצות which was commanded הלוים to the Levites, והמשׁררים and the singers, והשׁערים and the porters; ותרומת and the offerings הכהנים׃ of the priests.}%
\verse{ובכל But in all זה this לא not הייתי was בירושׁלם I at Jerusalem: כי for בשׁנת year שׁלשׁים and thirtieth ושׁתים in the two לארתחשׁסתא of Artaxerxes מלך king בבל of Babylon באתי came אל I unto המלך the king, ולקץ and after ימים certain days נשׁאלתי obtained I leave מן of המלך׃ the king:}%
\verse{ואבוא And I came לירושׁלם to Jerusalem, ואבינה and understood ברעה of the evil אשׁר that עשׂה did אלישׁיב Eliashib לטוביה for Tobiah, לעשׂות in preparing לו נשׁכה him a chamber בחצרי in the courts בית of the house האלהים׃ of God.}%
\verse{וירע And it grieved לי מאד me sore: ואשׁליכה therefore I cast forth את כל all כלי stuff בית the household טוביה of Tobiah החוץ out מן of הלשׁכה׃ the chamber.}%
\verse{ואמרה Then I commanded, ויטהרו and they cleansed הלשׁכות the chambers: ואשׁיבה brought I again שׁם and thither כלי the vessels בית of the house האלהים of God, את with המנחה the meat offering והלבונה׃ and the frankincense.}%
\verse{ואדעה And I perceived כי that מניות the portions הלוים of the Levites לא had not נתנה been given ויברחו were fled אישׁ every one לשׂדהו to his field. הלוים for the Levites והמשׁררים and the singers, עשׂי that did המלאכה׃ the work,}%
\verse{ואריבה Then contended את I with הסגנים the rulers, ואמרה and said, מדוע Why נעזב forsaken? בית is the house האלהים of God ואקבצם And I gathered them together, ואעמדם and set על them in עמדם׃ their place.}%
\verse{וכל all יהודה Judah הביאו Then brought מעשׂר the tithe הדגן of the corn והתירושׁ and the new wine והיצהר and the oil לאוצרות׃ unto the treasuries.}%
\verse{ואוצרה And I made treasurers על over אוצרות the treasuries, שׁלמיה Shelemiah הכהן the priest, וצדוק and Zadok הסופר the scribe, ופדיה Pedaiah: מן and of הלוים the Levites, ועל and next to ידם and next to חנן them Hanan בן the son זכור of Zaccur, בן the son מתניה of Mattaniah: כי for נאמנים faithful, נחשׁבו they were counted ועליהם and their office לחלק to distribute לאחיהם׃ unto their brethren.}%
\verse{זכרה Remember לי אלהי me, O my God, על concerning זאת this, ואל and wipe not out תמח and wipe not out חסדי my good deeds אשׁר that עשׂיתי I have done בבית for the house אלהי of my God, ובמשׁמריו׃ and for the offices}%
\verseWithHeading{Nehemiah Begins Sabbath Reforms}{בימים days ההמה In those ראיתי saw ביהודה I in Judah דרכים treading גתות wine presses בשׁבת on the sabbath, ומביאים and bringing in הערמות sheaves, ועמסים and lading על and lading החמרים asses; ואף as also יין wine, ענבים grapes, ותאנים and figs, וכל and all משׂא burdens, ומביאים which they brought into ירושׁלם Jerusalem ביום day: השׁבת on the sabbath ואעיד and I testified ביום in the day מכרם wherein they sold ציד׃ victuals.}%
\verse{והצרים men of Tyre ישׁבו There dwelt בה מביאים also therein, which brought דאג fish, וכל and all מכר manner of ware, ומכרים and sold בשׁבת on the sabbath לבני unto the children יהודה of Judah, ובירושׁלם׃ and in Jerusalem.}%
\verse{ואריבה Then I contended את with חרי the nobles יהודה of Judah, ואמרה and said להם מה unto them, What הדבר thing הרע evil הזה this אשׁר that אתם ye עשׂים do, ומחללים and profane את יום day? השׁבת׃ the sabbath}%
\verse{הלוא not כה thus, עשׂו Did אבתיכם your fathers ויבא bring אלהינו and did not our God עלינו upon את כל all הרעה evil הזאת this ועל us, and upon העיר city? הזאת this ואתם yet ye מוסיפים חרון wrath על upon ישׂראל Israel לחלל by profaning את השׁבת׃ the sabbath.}%
\verse{ויהי And it came to pass, כאשׁר that when צללו began to be dark שׁערי the gates ירושׁלם of Jerusalem לפני before השׁבת the sabbath, ואמרה I commanded ויסגרו should be shut, הדלתות that the gates ואמרה and charged אשׁר that לא they should not יפתחום be opened עד till אחר after השׁבת the sabbath: ומנערי and of my servants העמדתי set על I at השׁערים the gates, לא there should no יבוא be brought in משׂא burden ביום day. השׁבת׃ on the sabbath}%
\verse{וילינו lodged הרכלים So the merchants ומכרי and sellers כל of all ממכר kind of ware מחוץ without לירושׁלם Jerusalem פעם once ושׁתים׃ or twice.}%
\verse{ואעידה Then I testified בהם ואמרה against them, and said אליהם unto מדוע them, Why אתם ye לנים lodge נגד about החומה the wall? אם if תשׁנו ye do again, יד hands אשׁלח I will lay בכם מן on you. From העת time ההיא that לא they no באו forth came בשׁבת׃ on the sabbath.}%
\verse{ואמרה And I commanded ללוים the Levites אשׁר that יהיו they should מטהרים cleanse themselves, ובאים and they should come שׁמרים keep השׁערים the gates, לקדשׁ to sanctify את יום day. השׁבת the sabbath גם also, זאת this זכרה Remember לי אלהי me, O my God, וחוסה and spare עלי and spare כרב me according to the greatness חסדך׃ of thy mercy.}%
\verseWithHeading{Mixed Marriages are Condemned}{גם also בימים days ההם In those ראיתי saw את היהודים I Jews השׁיבו had married נשׁים wives אשׁדודיות of Ashdod, עמוניות of Ammon, מואביות׃ of Moab:}%
\verse{ובניהם And their children חצי half מדבר spoke אשׁדודית in the speech of Ashdod, ואינם not מכירים and could לדבר speak יהודית in the Jews' language, וכלשׁון but according to the language עם of each ועם׃ people.}%
\verse{ואריב And I contended עמם with ואקללם them, and cursed ואכה them, and smote מהם אנשׁים certain ואמרטם them, and plucked off their hair, ואשׁביעם and made them swear באלהים by God, אם Ye shall not תתנו give בנתיכם your daughters לבניהם unto their sons, ואם nor תשׂאו take מבנתיהם their daughters לבניכם׃ unto your sons,}%
\verse{הלוא Did not על by אלה these things? חטא sin שׁלמה Solomon מלך king ישׂראל of Israel ובגוים nations הרבים yet among many לא there no היה was מלך king כמהו like him, ואהוב beloved לאלהיו of his God, היה who was ויתנהו made אלהים and God מלך him king על over כל all ישׂראל Israel: גם nevertheless אותו החטיאו cause to sin. הנשׁים women הנכריות׃ even him did outlandish}%
\verse{ולכם הנשׁמע Shall we then hearken לעשׂת unto you to do את כל all הרעה evil, הגדולה great הזאת this למעל to transgress באלהינו against our God להשׁיב in marrying נשׁים wives? נכריות׃ strange}%
\verse{ומבני the son יוידע of Joiada, בן אלישׁיב of Eliashib הכהן priest, הגדול the high חתן son in law לסנבלט to Sanballat החרני the Horonite: ואבריחהו therefore I chased מעלי׃ him from}%
\verse{זכרה Remember להם אלהי them, O my God, על because גאלי they have defiled הכהנה the priesthood, וברית and the covenant הכהנה of the priesthood, והלוים׃ and of the Levites.}%
\verse{וטהרתים Thus cleansed מכל נכר strangers, ואעמידה and appointed משׁמרות the wards לכהנים of the priests וללוים and the Levites, אישׁ every one במלאכתו׃ in his business;}%
\verse{ולקרבן offering, העצים And for the wood בעתים at times מזמנות appointed, ולבכורים and for the firstfruits. זכרה Remember לי אלהי me, O my God, לטובה׃ for good.}%
\end{biblechapter}%
\flushcolsend
\input{leb/content/old-testament/Esth.tex}\flushcolsend
\biblebook{Job}
\begin{biblechapter}% Job 1
\verseWithHeading{Job’s Character and Greatness}{אישׁ a man היה There was בארץ in the land עוץ of Uz, איוב Job; שׁמו whose name והיה was האישׁ man ההוא and that תם perfect וישׁר and upright, וירא and one that feared אלהים God, וסר and eschewed מרע׃ evil.}%
\verse{ויולדו And there were born לו שׁבעה unto him seven בנים sons ושׁלושׁ and three בנות׃ daughters.}%
\verse{ויהי also was מקנהו His substance שׁבעת seven אלפי thousand צאן sheep, ושׁלשׁת and three אלפי thousand גמלים camels, וחמשׁ and five מאות hundred צמד yoke בקר of oxen, וחמשׁ and five מאות hundred אתונות she asses, ועבדה household; רבה great מאד and a very ויהי was האישׁ man ההוא so that this גדול the greatest מכל of all בני the men קדם׃ of the east.}%
\verse{והלכו went בניו And his sons ועשׂו and feasted משׁתה and feasted בית houses, אישׁ every one יומו his day; ושׁלחו and sent וקראו and called לשׁלשׁת for their three אחיתיהם sisters לאכל to eat ולשׁתות and to drink עמהם׃ with}%
\verse{ויהי And it was כי so, when הקיפו were gone about, ימי the days המשׁתה of feasting וישׁלח sent איוב that Job ויקדשׁם and sanctified והשׁכים them, and rose up early בבקר in the morning, והעלה and offered עלות burnt offerings מספר to the number כלם of them all: כי for אמר said, איוב Job אולי It may be חטאו have sinned, בני that my sons וברכו and cursed אלהים God בלבבם in their hearts. ככה Thus יעשׂה did איוב Job כל continually. הימים׃ continually.}%
\verse{ויהי Now there was היום a day ויבאו came בני when the sons האלהים of God להתיצב to present themselves על before יהוה the LORD, ויבוא came גם also השׂטן and Satan בתוכם׃ among}%
\verse{ויאמר said יהוה אל unto השׂטן Satan, מאין תבא comest ויען answered השׂטן thou? Then Satan את יהוהיאמר and said, משׁוט בארץ in the earth, ומהתהלך׃ and from walking up and down}%
\verse{ויאמר said יהוה And the LORD אל unto השׂטן Satan, השׂמת לבךל עבדי my servant איוב Job, כי that אין none כמהו like him בארץ in the earth, אישׁ man, תם a perfect וישׁר and an upright ירא one that feareth אלהים God, וסר and escheweth מרע׃ evil?}%
\verse{ויען answered השׂטן Then Satan את יהוה the LORD, ויאמר and said, החנם for naught? ירא fear איוב Doth Job אלהים׃ God}%
\verse{הלא Hast not את thou שׂכת made a hedge בעדו about ובעד him, and about ביתו his house, ובעד and about כל all אשׁר that לו מסביב he hath on every side? מעשׂה the work ידיו of his hands, ברכת thou hast blessed ומקנהו and his substance פרץ is increased בארץ׃ in the land.}%
\verse{ואולם But שׁלח put forth נא now, ידך thine hand וגע and touch בכל all אשׁר that לו אם he hath, and לא על thee to פניך thy face. יברכך׃ he will curse}%
\verse{ויאמר said יהוה And the LORD אל unto השׂטן Satan, הנה Behold, כל all אשׁר that לו בידך he hath in thy power; רק only אליו upon אל himself put not forth תשׁלח himself put not forth ידך thine hand. ויצא went forth השׂטן So Satan מעם from פני the presence יהוה׃ of the LORD.}%
\verseWithHeading{The Adversary’s Attack on Job’s Possessions}{ויהי And there was היום a day ובניו when his sons ובנתיו and his daughters אכלים eating ושׁתים and drinking יין wine בבית house: אחיהם brother's הבכור׃ in their eldest}%
\verse{ומלאך a messenger בא And there came אל unto איוב Job, ויאמר and said, הבקר The oxen היו were חרשׁות plowing, והאתנות and the asses רעות feeding על beside ידיהם׃ beside}%
\verse{ותפל fell שׁבא And the Sabeans ותקחם and took them away; ואת הנערים the servants הכו yea, they have slain לפי with the edge חרב of the sword; ואמלטה am escaped רק only אני and I לבדי alone להגיד׃ to tell}%
\verse{עוד yet זה While he מדבר speaking, וזה also another, בא there came ויאמר and said, אשׁ The fire אלהים of God נפלה is fallen מן from השׁמים heaven, ותבער and hath burned up בצאן the sheep, ובנערים and the servants, ותאכלם and consumed ואמלטה am escaped רק only אני them; and I לבדי alone להגיד׃ to tell}%
\verse{עוד yet זה While he מדבר speaking, וזה also another, בא there came ויאמר and said, כשׂדים The Chaldeans שׂמו made שׁלשׁה out three ראשׁים bands, ויפשׁטו and fell על upon הגמלים the camels, ויקחום and have carried them away, ואת הנערים the servants הכו yea, and slain לפי with the edge חרב of the sword; ואמלטה am escaped רק only אני and I לבדי alone להגיד׃ to tell}%
\verse{עד זה While he מדבר speaking, וזה also another, בא there came ויאמר and said, בניך Thy sons ובנותיך and thy daughters אכלים eating ושׁתים and drinking יין wine בבית house: אחיהם brother's הבכור׃ in their eldest}%
\verse{והנה And, behold, רוח wind גדולה a great באה there came מעבר from המדבר the wilderness, ויגע and smote בארבע the four פנות corners הבית of the house, ויפל and it fell על upon הנערים the young men, וימותו and they are dead; ואמלטה am escaped רק only אני and I לבדי alone להגיד׃ to tell}%
\verse{ויקם arose, איוב Then Job ויקרע and rent את מעלו his mantle, ויגז and shaved את ראשׁו his head, ויפל and fell down ארצה upon the ground, וישׁתחו׃ and worshiped,}%
\verse{ויאמר And said, ערם Naked יצתי came I out מבטן אמיערם and naked אשׁוב shall I return שׁמה thither: יהוה the LORD נתן gave, ויהוה and the LORD לקח hath taken away; יהי be שׁם the name יהוה of the LORD. מברך׃ blessed}%
\verse{בכל In all זאת this לא not, חטא sinned איוב Job ולא nor נתן charged תפלה foolishly. לאלהים׃ God}%
\end{biblechapter}%
\begin{biblechapter}% Job 2
\verseWithHeading{The Adversary’s Attack on Job’s Person}{ויהי Again there was היום a day ויבאו came בני when the sons האלהים of God להתיצב to present themselves על before יהוה the LORD, ויבוא came גם also השׂטן and Satan בתכם among להתיצב them to present himself על before יהוה׃ the LORD.}%
\verse{ויאמר said יהוה And the LORD אל unto השׂטן Satan, אי מזהבא comest ויען answered השׂטן thou? And Satan את יהוה the LORD, ויאמר and said, משׁט בארץ in the earth, ומהתהלך׃ and from walking up and down}%
\verse{ויאמר said יהוה And the LORD אל unto השׂטן Satan, השׂמת לבךל עבדי my servant איוב Job, כי that אין none כמהו like him בארץ in the earth, אישׁ man, תם a perfect וישׁר and an upright ירא one that feareth אלהים God, וסר and escheweth מרע evil? ועדנו and still מחזיק he holdeth fast בתמתו his integrity, ותסיתני although thou movedst בו לבלעו me against him, to destroy חנם׃ him without cause.}%
\verse{ויען answered השׂטן And Satan את יהוה the LORD, ויאמר and said, עור בעד for עור וכל yea, all אשׁר that לאישׁ a man יתן hath will he give בעד נפשׁו׃ his life.}%
\verse{אולם But שׁלח put forth נא now, ידך thine hand וגע and touch אל and touch עצמו his bone ואל thee to בשׂרו and his flesh, אם and לא אלניך thy face. יברכך׃ he will curse}%
\verse{ויאמר said יהוה אל unto השׂטן Satan, הנו בידך he in thine hand; אך but את נפשׁו his life. שׁמר׃ save}%
\verseWithHeading{Job’s Blameless Behavior}{ויצא השׂטןאת פני the presence יהוה of the LORD, ויך and smote את איוב Job בשׁחין boils רע with sore מכף from the sole רגלו of his foot עד unto קדקדו׃ his crown.}%
\verse{ויקח And he took לו חרשׂ him a potsherd להתגרד to scrape himself בו והוא withal; and he ישׁב sat down בתוך among האפר׃ the ashes.}%
\verse{ותאמר Then said לו אשׁתו his wife עדך unto him, Dost thou still מחזיק retain בתמתך thine integrity? ברך curse אלהים God, ומת׃ and die.}%
\verse{ויאמר But he said אליה unto כדבר her, Thou speakest אחת as one הנבלות of the foolish women תדברי speaketh. גם What? את הטוב good נקבל shall we receive מאת האלהים of God, ואת הרע evil? לא and shall we not נקבל receive בכל In all זאת this לא did not חטא sin איוב Job בשׂפתיו׃ with his lips.}%
\verse{וישׁמעו heard שׁלשׁת three רעי friends איוב Now when Job's את כל of all הרעה evil הזאת this הבאה that was come עליו upon ויבאו him, they came אישׁ every one ממקמו from his own place; אליפז Eliphaz התימני the Temanite, ובלדד and Bildad השׁוחי the Shuhite, וצופר and Zophar הנעמתי the Naamathite: ויועדו for they had made an appointment יחדו together לבוא to come לנוד to mourn לו ולנחמו׃ with him and to comfort}%
\verse{וישׂאו And when they lifted up את עיניהם their eyes מרחוק afar off, ולא him not, הכירהו and knew וישׂאו they lifted up קולם their voice, ויבכו and wept; ויקרעו and they rent אישׁ every one מעלו his mantle, ויזרקו and sprinkled עפר dust על upon ראשׁיהם their heads השׁמימה׃ toward heaven.}%
\verse{וישׁבו So they sat down אתו with לארץ him upon the ground שׁבעת seven ימים days ושׁבעת and seven לילות nights, ואין and none דבר spoke אליו unto דבר a word כי him: for ראו they saw כי that גדל was very great. הכאב grief מאד׃ was very great.}%
\end{biblechapter}%
\begin{biblechapter}% Job 3
\verseWithHeading{Job Regrets His Birth}{אחרי After כן this פתח opened איוב Job את פיהו his mouth, ויקלל and cursed את יומו׃ his day.}%
\verse{ויען spoke, איוב And Job ויאמר׃ and said,}%
\verse{יאבד perish יום Let the day אולד wherein I was born, בו והלילה and the night אמר it was said, הרה גבר׃ There is a man child}%
\verse{היום day ההוא Let that יהי be חשׁך darkness; אל let not ידרשׁהו regard אלוה God ממעל it from above, ואל neither תופע shine עליו upon נהרה׃ let the light}%
\verse{יגאלהו stain חשׁך Let darkness וצלמות and the shadow of death תשׁכן dwell עליו upon עננה it; let a cloud יבעתהו terrify כמרירי it; let the blackness יום׃ of the day}%
\verse{הלילה night, ההוא As that יקחהו seize upon אפל let darkness אל it; let it not יחד be joined בימי unto the days שׁנה of the year, במספר into the number ירחים of the months. אל let it not יבא׃ come}%
\verse{הנה Lo, הלילה night ההוא let that יהי be גלמוד solitary, אל let no תבא come רננה׃ joyful voice}%
\verse{יקבהו Let them curse אררי it that curse יום the day, העתידים who are ready ערר to raise up לויתן׃ their mourning.}%
\verse{יחשׁכו thereof be dark; כוכבי Let the stars נשׁפו of the twilight יקו let it look לאור for light, ואין but none; ואל neither יראה let it see בעפעפי the dawning שׁחר׃ of the day:}%
\verse{כי Because לא it shut not up סגר it shut not up דלתי the doors בטני of my womb, ויסתר nor hid עמל sorrow מעיני׃ from mine eyes.}%
\verseWithHeading{Job Wishes He Had Died}{למה Why לא I not מרחם from the womb? אמות died מבטן of the belly? יצאתי when I came out ואגוע׃ did I give up the ghost}%
\verse{מדוע Why קדמוני prevent ברכים did the knees ומה me? or why שׁדים the breasts כי that אינק׃ I should suck?}%
\verse{כי For עתה now שׁכבתי should I have lain still ואשׁקוט and been quiet, ישׁנתי I should have slept: אז then ינוח׃ had I been at rest,}%
\verse{עם With מלכים kings ויעצי and counselors ארץ of the earth, הבנים which built חרבות׃ desolate places}%
\verse{או Or עם with שׂרים princes זהב that had gold, להם הממלאים who filled בתיהם their houses כסף׃ with silver:}%
\verse{או Or כנפל untimely birth טמון as a hidden לא I had not אהיה been; כעללים as infants לא never ראו saw אור׃ light.}%
\verse{שׁם There רשׁעים the wicked חדלו cease רגז troubling; ושׁם and there ינוחו be at rest. יגיעי the weary כח׃ the weary}%
\verse{יחד together; אסירים the prisoners שׁאננו rest לא not שׁמעו they hear קול the voice נגשׂ׃ of the oppressor.}%
\verse{קטן The small וגדול and great שׁם are there; הוא ועבד and the servant חפשׁי free מאדניו׃ from his master.}%
\verseWithHeading{Job Wishes He Might Die}{למה Wherefore יתן given לעמל to him that is in misery, אור is light וחיים and life למרי unto the bitter נפשׁ׃ soul;}%
\verse{המחכים Which long למות for death, ואיננו but it not; ויחפרהו and dig ממטמונים׃ for it more than for hid treasures;}%
\verse{השׂמחים אלי exceedingly, גיל exceedingly, ישׂישׂו are glad, כי when ימצאו they can find קבר׃ the grave?}%
\verse{לגבר to a man אשׁר whose דרכו way נסתרה is hid, ויסך hath hedged in? אלוה God בעדו׃ and whom}%
\verse{כי For לפני before לחמי I eat, אנחתי my sighing תבא cometh ויתכו are poured out כמים like the waters. שׁאגתי׃ and my roarings}%
\verse{כי For פחד פחדתייאתיני is come upon ואשׁר me, and that which יגרתי I was afraid of יבא׃ is come}%
\verse{לא I was not שׁלותי in safety, ולא neither שׁקטתי had I rest, ולא neither נחתי was I quiet; ויבא came. רגז׃ yet trouble}%
\end{biblechapter}%
\begin{biblechapter}% Job 4
\verseWithHeading{Eliphaz’s First Response to Job}{ויען answered אליפז Then Eliphaz התימני the Temanite ויאמר׃ and said,}%
\verse{הנסה דברליך with תלאה thee, wilt thou be grieved? ועצר withhold במלין himself from speaking? מי but who יוכל׃ can}%
\verse{הנה Behold, יסרת thou hast instructed רבים many, וידים hands. רפות the weak תחזק׃ and thou hast strengthened}%
\verse{כושׁל him that was falling, יקימון have upheld מליך Thy words וברכים knees. כרעות the feeble תאמץ׃ and thou hast strengthened}%
\verse{כי But עתה now תבוא it is come אליך upon ותלא thee, and thou faintest; תגע it toucheth עדיך it toucheth ותבהל׃ thee, and thou art troubled.}%
\verse{הלא not יראתך כסלתך thy confidence, תקותך thy hope, ותם and the uprightness דרכיך׃ of thy ways?}%
\verse{זכר Remember, נא I pray thee, מי who הוא נקי being innocent? אבד perished, ואיפה or where ישׁרים were the righteous נכחדו׃ cut off?}%
\verse{כאשׁר Even as ראיתי I have seen, חרשׁי they that plow און iniquity, וזרעי and sow עמל wickedness, יקצרהו׃ reap}%
\verse{מנשׁמת אלוה of God יאבדו they perish, ומרוח and by the breath אפו of his nostrils יכלו׃ are they consumed.}%
\verse{שׁאגת The roaring אריה of the lion, וקול and the voice שׁחל of the fierce lion, ושׁני and the teeth כפירים of the young lions, נתעו׃ are broken.}%
\verse{לישׁ The old lion אבד perisheth מבלי for lack טרף of prey, ובני whelps לביא and the stout lion's יתפרדו׃ are scattered abroad.}%
\verse{ואלי to דבר Now a thing יגנב was secretly brought ותקח received אזני me, and mine ear שׁמץ a little מנהו׃ thereof.}%
\verse{בשׂעפים In thoughts מחזינות from the visions לילה of the night, בנפל falleth תרדמה when deep sleep על on אנשׁים׃ men,}%
\verse{פחד Fear קראני came upon ורעדה me, and trembling, ורב which made all עצמותי my bones הפחיד׃ to shake.}%
\verse{ורוח Then a spirit על before פני my face; יחלף passed תסמר stood up: שׂערת the hair בשׂרי׃ of my flesh}%
\verse{יעמד It stood still, ולא but I could not אכיר discern מראהו the form תמונה thereof: an image לנגד before עיני mine eyes, דממה silence, וקול a voice, אשׁמע׃ and I heard}%
\verse{האנושׁ מאלוהצדק אםעשׂהו than his maker? יטהר be more pure גבר׃ shall a man}%
\verse{הן Behold, בעבדיו in his servants; לא he put no trust יאמין he put no trust ובמלאכיו and his angels ישׂים he charged תהלה׃ with folly:}%
\verse{אף How much less שׁכני them that dwell in בתי houses חמר of clay, אשׁר whose בעפר in the dust, יסודם foundation ידכאום are crushed לפני before עשׁ׃ the moth?}%
\verse{מבקר from morning לערב to evening: יכתו They are destroyed מבלי without any משׂים regarding לנצח forever יאבדו׃ they perish}%
\verse{הלא Doth not נסע in them go away? יתרם their excellency בם ימותו they die, ולא even without בחכמה׃ wisdom.}%
\end{biblechapter}%
\begin{biblechapter}% Job 5
\verseWithHeading{Eliphaz’s Response Continues}{קרא Call נא now, הישׁ if there be עונך any that will answer ואל thee; and to מי which מקדשׁים of the saints תפנה׃ wilt thou turn?}%
\verse{כי For לאויל the foolish יהרג killeth כעשׂ wrath ופתה the silly תמית slayeth קנאה׃ man, and envy}%
\verse{אני I ראיתי have seen אויל the foolish משׁרישׁ taking root: ואקוב I cursed נוהו his habitation. פתאם׃ but suddenly}%
\verse{ירחקו are far בניו His children מישׁע from safety, וידכאו and they are crushed בשׁער in the gate, ואין neither מציל׃ any to deliver}%
\verse{אשׁר Whose קצירו harvest רעב יאכל eateth up, ואל it even out of מצנים the thorns, יקחהו and taketh ושׁאף swalloweth up צמים and the robber חילם׃ their substance.}%
\verse{כי Although לא cometh not forth יצא cometh not forth מעפר of the dust, און affliction ומאדמה out of the ground; לא neither יצמח spring עמל׃ doth trouble}%
\verse{כי Yet אדם man לעמל unto trouble, יולד is born ובני as the sparks רשׁף as the sparks יגביהו upward. עוף׃ fly}%
\verse{אולם אני I אדרשׁ would seek אל unto אל God, ואל and unto אלהים God אשׂים would I commit דברתי׃ my cause:}%
\verse{עשׂה Which doeth גדלות great things ואין and unsearchable; חקר and unsearchable; נפלאות marvelous things עד without אין without מספר׃ number:}%
\verse{הנתן Who giveth מטר rain על upon פני upon ארץ the earth, ושׁלח and sendeth מים waters על upon פני upon חוצות׃ the fields:}%
\verse{לשׂום To set up שׁפלים those that be low; למרום on high וקדרים that those which mourn שׂגבו may be exalted ישׁע׃ to safety.}%
\verse{מפר He disappointeth מחשׁבות the devices ערומים of the crafty, ולא cannot תעשׂינה perform ידיהם so that their hands תושׁיה׃ enterprise.}%
\verse{לכד He taketh חכמים the wise בערמם in their own craftiness: ועצת and the counsel נפתלים of the froward נמהרה׃ is carried headlong.}%
\verse{יומם in the daytime, יפגשׁו They meet חשׁך with darkness וכלילה as in the night. ימשׁשׁו and grope בצהרים׃ in the noonday}%
\verse{וישׁע But he saveth מחרב from the sword, מפיהם from their mouth, ומיד and from the hand חזק of the mighty. אביון׃ the poor}%
\verse{ותהי hath לדל So the poor תקוה hope, ועלתה and iniquity קפצה stoppeth פיה׃ her mouth.}%
\verse{הנה Behold, אשׁרי happy אנושׁ the man יוכחנו correcteth: אלוה whom God ומוסר thou the chastening שׁדי of the Almighty: אל not תמאס׃ therefore despise}%
\verse{כי For הוא he יכאיב maketh sore, ויחבשׁ and bindeth up: ימחץ he woundeth, וידו and his hands תרפינה׃ make whole.}%
\verse{בשׁשׁ thee in six צרות troubles: יצילך He shall deliver ובשׁבע yea, in seven לא there shall no יגע touch בך רע׃ evil}%
\verse{ברעב In famine פדך he shall redeem ממות thee from death: ובמלחמה and in war מידי from the power חרב׃ of the sword.}%
\verse{בשׁוט from the scourge לשׁון of the tongue: תחבא Thou shalt be hid ולא neither תירא shalt thou be afraid משׁד of destruction כי when יבוא׃ it cometh.}%
\verse{לשׁד At destruction ולכפן and famine תשׂחק thou shalt laugh: ומחית of the beasts הארץ of the earth. אל neither תירא׃ shalt thou be afraid}%
\verse{כי For עם with אבני the stones השׂדה of the field: בריתך thou shalt be in league וחית and the beasts השׂדה of the field השׁלמה׃ shall be at peace}%
\verse{וידעת And thou shalt know כי שׁלום in peace; אהלך that thy tabernacle ופקדת and thou shalt visit נוך thy habitation, ולא and shalt not sin. תחטא׃ and shalt not sin.}%
\verse{וידעת Thou shalt know כי also that רב great, זרעך thy seed וצאצאיך and thine offspring כעשׂב as the grass הארץ׃ of the earth.}%
\verse{תבוא Thou shalt come בכלח in a full age, אלי to קבר grave כעלות cometh in גדישׁ like as a shock of corn בעתו׃ in his season.}%
\verse{הנה Lo זאת this, חקרנוה we have searched כן it, so היא it שׁמענה hear ואתה thou דע׃ it, and know}%
\end{biblechapter}%
\begin{biblechapter}% Job 6
\verseWithHeading{Job’s Second Speech: A Response to Eliphaz}{ויען answered איוב But Job ויאמר׃ and said,}%
\verse{לו Oh that שׁקול were throughly weighed, ישׁקל were throughly weighed, כעשׂי my grief והיתי במאזנים in the balances ישׂאו laid יחד׃ together!}%
\verse{כי For עתה now מחול than the sand ימים of the sea: יכבד it would be heavier על therefore כן therefore דברי my words לעו׃ are swallowed up.}%
\verse{כי For חצי the arrows שׁדי of the Almighty עמדי within אשׁר whereof חמתם me, the poison שׁתה drinketh up רוחי my spirit: בעותי the terrors אלוה of God יערכוני׃ do set themselves in array against}%
\verse{הינהק bray פרא Doth the wild ass עלי over דשׁא when he hath grass? אם or יגעה loweth שׁור the ox על בלילו׃ his fodder?}%
\verse{היאכל be eaten תפל Can that which is unsavory מבלי without מלח salt? אם or ישׁ is there טעם taste בריר in the white חלמות׃ of an egg?}%
\verse{מאנה refused לנגוע to touch נפשׁי my soul המה The things כדוי as my sorrowful לחמי׃ meat.}%
\verse{מי Oh that יתן I might תבוא have שׁאלתי my request; ותקותי the thing that I long for! יתן would grant אלוה׃ and that God}%
\verse{ויאל Even that it would please אלוה God וידכאני to destroy יתר me; that he would let loose ידו his hand, ויבצעני׃ and cut me off!}%
\verse{ותהי have עוד Then should I yet נחמתי comfort; ואסלדה yea, I would harden בחילה myself in sorrow: לא let him not יחמול spare; כי for לא I have not כחדתי concealed אמרי the words קדושׁ׃ of the Holy One.}%
\verse{מה What כחי my strength, כי that איחל I should hope? ומה and what קצי mine end, כי that אאריך I should prolong נפשׁי׃ my life?}%
\verse{אם כח my strength אבנים of stones? כחי the strength אם בשׂרי or my flesh נחושׁ׃ of brass?}%
\verse{האם אין not עזרתי my help בי ותשׁיה in me? and is wisdom נדחה driven ממני׃ quite from}%
\verse{למס To him that is afflicted מרעהו from his friend; חסד pity ויראת the fear שׁדי of the Almighty. יעזוב׃ but he forsaketh}%
\verse{אחי My brethren בגדו have dealt deceitfully כמו as נחל a brook, כאפיק as the stream נחלים of brooks יעברו׃ they pass away;}%
\verse{הקדרים Which are blackish מני by reason of קרח the ice, עלימו wherein יתעלם is hid: שׁלג׃ the snow}%
\verse{בעת What time יזרבו they wax warm, נצמתו they vanish: בחמו נדעכו they are consumed ממקומם׃ out of their place.}%
\verse{ילפתו are turned aside; ארחות The paths דרכם of their way יעלו they go בתהו to nothing, ויאבדו׃ and perish.}%
\verse{הביטו looked, ארחות The troops תמא of Tema הליכת the companies שׁבא of Sheba קוו׃ waited}%
\verse{בשׁו They were confounded כי because בטח they had hoped; באו they came עדיה thither, ויחפרו׃ and were ashamed.}%
\verse{כי For עתה now הייתם ye are לא nothing; תראו ye see חתת casting down, ותיראו׃ and are afraid.}%
\verse{הכי Did אמרתי I say, הבו Bring לי ומכחכם me of your substance? שׁחדו unto me? or, Give a reward בעדי׃ for}%
\verse{ומלטוני Or, Deliver מיד me from the hand צר ומידריצים of the mighty? תפדוני׃ or, Redeem}%
\verse{הורוני Teach ואני me, and I אחרישׁ will hold my tongue: ומה wherein שׁגיתי I have erred. הבינו׃ and cause me to understand}%
\verse{מה How נמרצו forcible אמרי words! ישׁר are right ומה but what יוכיח doth your arguing הוכח reprove? מכם׃ reprove?}%
\verse{הלהוכח to reprove מלים words, תחשׁבו Do ye imagine ולרוח as wind? אמרי and the speeches נאשׁ׃ of one that is desperate,}%
\verse{אף Yea, על ye overwhelm יתום the fatherless, תפילו ye overwhelm ותכרו and ye dig על for ריעכם׃ your friend.}%
\verse{ועתה Now הואילו therefore be content, פנו look בי ועל upon me; for evident unto פניכם upon me; for evident unto אם you if אכזב׃ I lie.}%
\verse{שׁבו Return, נא I pray you, אל let it not תהי be עולה iniquity; ושׁבי yea, return עוד again, צדקי׃ my righteousness}%
\verse{הישׁ Is there בלשׁוני in my tongue? עולה iniquity אם חכי my taste לא cannot יבין discern הוות׃ perverse things?}%
\end{biblechapter}%
\begin{biblechapter}% Job 7
\verseWithHeading{Job’s Second Speech: A Response to Eliphaz}{הלא not צבא an appointed time לאנושׁ to man על upon ארץ וכימי his days שׂכיר of a hireling? ימיו׃ also like the days}%
\verse{כעבד As a servant ישׁאף earnestly desireth צל the shadow, וכשׂכיר and as a hireling יקוה looketh for פעלו׃ his work:}%
\verse{כן So הנחלתי am I made to possess לי ירחי months שׁוא of vanity, ולילות nights עמל and wearisome מנו׃ are appointed}%
\verse{אם When שׁכבתי I lie down, ואמרתי I say, מתי When אקום shall I arise, ומדד be gone? ערב and the night ושׂבעתי and I am full נדדים of tossings to and fro עדי unto נשׁף׃ the dawning of the day.}%
\verse{לבשׁ is clothed בשׂרי My flesh רמה with worms וגישׁ and clods עפר of dust; עורי my skin רגע is broken, וימאס׃ and become loathsome.}%
\verse{ימי My days קלו are swifter מני than ארג a weaver's shuttle, ויכלו and are spent באפס without תקוה׃ hope.}%
\verse{זכר O remember כי that רוח wind: חיי my life לא shall no תשׁוב more עיני mine eye לראות see טוב׃ good.}%
\verse{לא me no תשׁורני me shall see עין The eye ראי of him that hath seen עיניך thine eyes בי ואינני׃ upon me, and I not.}%
\verse{כלה is consumed ענן the cloud וילך and vanisheth away: כן so יורד he that goeth down שׁאול to the grave לא no יעלה׃ shall come up}%
\verse{לא no ישׁוב He shall return עוד more לביתו to his house, ולא neither יכירנו know עוד him any more. מקמו׃ shall his place}%
\verse{גם Therefore אני I לא will not אחשׂך refrain פי my mouth; אדברה I will speak בצר in the anguish רוחי of my spirit; אשׂיחה I will complain במר in the bitterness נפשׁי׃ of my soul.}%
\verse{הים a sea, אני I אם or תנין a whale, כי that תשׂים thou settest עלי over משׁמר׃ a watch}%
\verse{כי When אמרתי I say, תנחמני shall comfort ערשׂי My bed ישׂא shall ease בשׂיחי משׁכבי׃ me, my couch}%
\verse{וחתתני Then thou scarest בחלמות me with dreams, ומחזינות me through visions: תבעתני׃ and terrifiest}%
\verse{ותבחר chooseth מחנק strangling, נפשׁי So that my soul מות death מעצמותי׃ rather than my life.}%
\verse{מאסתי I loathe לא I would not לעלם always: אחיה live חדל let me alone; ממני let me alone; כי for הבל vanity. ימי׃ my days}%
\verse{מה What אנושׁ man, כי that תגדלנו thou shouldest magnify וכי him? and that תשׁית thou shouldest set אליו upon לבך׃ thine heart}%
\verse{ותפקדנו And thou shouldest visit לבקרים him every morning, לרגעים him every moment? תבחננו׃ try}%
\verse{כמה How long לא wilt thou not תשׁעה depart ממני from לא me, nor תרפני let me alone עד till בלעי I swallow down רקי׃ my spittle?}%
\verse{חטאתי I have sinned; מה what אפעל shall I do לך נצר unto thee, O thou preserver האדם of men? למה why שׂמתני hast thou set למפגע me as a mark לך ואהיה עלי to למשׂא׃ a burden}%
\verse{ומה And why לא dost thou not תשׂא pardon פשׁעי my transgression, ותעביר and take away את עוני mine iniquity? כי for עתה now לעפר in the dust; אשׁכב shall I sleep ושׁחרתני and thou shalt seek me in the morning, ואינני׃ but I not}%
\end{biblechapter}%
\begin{biblechapter}% Job 8
\verseWithHeading{Bildad’s First Response to Job}{ויען Then answered בלדד Bildad השׁוחי the Shuhite, ויאמר׃ and said,}%
\verse{עד אןמלל wilt thou speak אלה these ורוח wind? כביר a strong אמרי and the words פיך׃ of thy mouth}%
\verse{האל Doth God יעות pervert משׁפט judgment? ואם or שׁדי doth the Almighty יעות pervert צדק׃ justice?}%
\verse{אם If בניך thy children חטאו have sinned לו וישׁלחם against him, and he have cast them away ביד for פשׁעם׃ their transgression;}%
\verse{אם If אתה thou תשׁחר אל to אל ואלׁדי the Almighty; תתחנן׃ and make thy supplication}%
\verse{אם If זך pure וישׁר and upright; אתה thou כי surely עתה now יעיר he would awake עליך for ושׁלם prosperous. נות thee, and make the habitation צדקך׃ of thy righteousness}%
\verse{והיה was ראשׁיתך Though thy beginning מצער small, ואחריתך yet thy latter end ישׂגה increase. מאד׃ should greatly}%
\verse{כי For שׁאל inquire, נא I pray thee, לדר age, רישׁון of the former וכונן and prepare לחקר thyself to the search אבותם׃ of their fathers:}%
\verse{כי (For תמול yesterday, אנחנו we ולא nothing, נדע and know כי because צל a shadow:) ימינו our days עלי upon ארץ׃ earth}%
\verse{הלא Shall not הם they יורוך teach יאמרו thee, tell לך ומלבם out of their heart? יוצאו thee, and utter מלים׃ words}%
\verse{היגאה grow up גמא Can the rush בלא without בצה mire? ישׂגה grow אחו can the flag בלי without מים׃ water?}%
\verse{עדנו Whilst it yet באבו in his greenness, לא not יקטף cut down, ולפני before כל any חציר herb. ייבשׁ׃ it withereth}%
\verse{כן So ארחות the paths כל of all שׁכחי that forget אל God; ותקות hope חנף and the hypocrite's תאבד׃ shall perish:}%
\verse{אשׁר Whose יקוט shall be cut off, כסלו hope ובית web. עכבישׁ a spider's מבטחו׃ and whose trust}%
\verse{ישׁען He shall lean על upon ביתו his house, ולא but it shall not יעמד stand: יחזיק he shall hold it fast, בו ולא but it shall not יקום׃ endure.}%
\verse{רטב green הוא He לפני before שׁמשׁ the sun, ועל in גנתו his garden. ינקתו and his branch תצא׃ shooteth forth}%
\verse{על about גל the heap, שׁרשׁיו His roots יסבכו are wrapped בית the place אבנים of stones. יחזה׃ seeth}%
\verse{אם If יבלענו he destroy ממקומו him from his place, וכחשׁ then shall deny בו לא him, I have not ראיתיך׃ seen}%
\verse{הן Behold, הוא this משׂושׂ the joy דרכו of his way, ומעפר and out of the earth אחר shall others יצמחו׃ grow.}%
\verse{הן Behold, אל God לא will not ימאס cast away תם a perfect ולא neither יחזיק will he help ביד will he help מרעים׃ the evildoers:}%
\verse{עד Till ימלה he fill שׂחוק with laughing, פיך thy mouth ושׂפתיך and thy lips תרועה׃ with rejoicing.}%
\verse{שׂנאיך They that hate ילבשׁו thee shall be clothed בשׁת with shame; ואהל and the dwelling place רשׁעים of the wicked איננו׃ shall come to naught.}%
\end{biblechapter}%
\begin{biblechapter}% Job 9
\verseWithHeading{Job’s Third Speech: A Response to Bildad}{ויען answered איוב Then Job ויאמר׃ and said,}%
\verse{אמנם of a truth: ידעתי I know כי so כן so ומה but how יצדק be just אנושׁ should man עם with אל׃ God?}%
\verse{אם If יחפץ he will לריב contend עמו with לא him, he cannot יעננו answer אחת him one מני of אלף׃ a thousand.}%
\verse{חכם wise לבב in heart, ואמיץ and mighty כח in strength: מי who הקשׁה hath hardened אליו against וישׁלם׃ him, and hath prospered?}%
\verse{המעתיק Which removeth הרים the mountains, ולא not: ידעו and they know אשׁר which הפכם overturneth באפו׃ them in his anger.}%
\verse{המרגיז Which shaketh ארץ the earth ממקומה out of her place, ועמודיה and the pillars יתפלצון׃ thereof tremble.}%
\verse{האמר Which commandeth לחרס the sun, ולא not; יזרח and it riseth ובעד כוכבים the stars. יחתם׃ and sealeth up}%
\verse{נטה spreadeth out שׁמים the heavens, לבדו Which alone ודורך and treadeth על upon במתי the waves ים׃ of the sea.}%
\verse{עשׂה Which maketh עשׁ Arcturus, כסיל Orion, וכימה and Pleiades, וחדרי and the chambers תמן׃ of the south.}%
\verse{עשׂה Which doeth גדלות great things עד past אין past חקר finding out; ונפלאות yea, and wonders עד without אין without מספר׃ number.}%
\verse{הן Lo, יעבר he goeth עלי by ולא not: אראה me, and I see ויחלף he passeth on ולא him not. אבין׃ also, but I perceive}%
\verse{הן Behold, יחתף he taketh away, מי who ישׁיבנו can hinder מי him? who יאמר will say אליו unto מה him, What תעשׂה׃ doest}%
\verse{אלוה God לא will not ישׁיב withdraw אפו his anger, תחתו under שׁחחו do stoop עזרי helpers רהב׃}%
\verse{אף כינכי shall I אעננו answer אבחרה him, choose out דברי my words עמו׃ with}%
\verse{אשׁר Whom, אם though צדקתי I were righteous, לא would I not אענה answer, למשׁפטי to my judge. אתחנן׃ I would make supplication}%
\verse{אם If קראתי I had called, ויענני and he had answered לא me; would I not אאמין believe כי that יאזין he had hearkened קולי׃ unto my voice.}%
\verse{אשׁר For בשׂערה me with a tempest, ישׁופני he breaketh והרבה and multiplieth פצעי my wounds חנם׃ without cause.}%
\verse{לא He will not יתנני suffer השׁב me to take רוחי my breath, כי but ישׂבעני filleth ממררים׃ me with bitterness.}%
\verse{אם If לכח of strength, אמיץ strong: הנה lo, ואם and if למשׁפט of judgment, מי who יועידני׃ shall set me a time}%
\verse{אם If אצדק I justify פי myself, mine own mouth ירשׁיעני shall condemn תם perfect, אני me: I ויעקשׁני׃ it shall also prove me perverse.}%
\verse{תם perfect, אני I לא would I not אדע know נפשׁי my soul: אמאס I would despise חיי׃ my life.}%
\verse{אחת one היא This על therefore כן , therefore אמרתי I said תם the perfect ורשׁע and the wicked. הוא He מכלה׃ destroyeth}%
\verse{אם If שׁוט the scourge ימית slay פתאם suddenly, למסת at the trial נקים of the innocent. ילעג׃ he will laugh}%
\verse{ארץ The earth נתנה is given ביד into the hand רשׁע of the wicked: פני the faces שׁפטיה of the judges יכסה he covereth אם thereof; if לא not, אפוא where, מי who הוא׃ he?}%
\verse{וימי Now my days קלו are swifter מני than רץ a post: ברחו they flee away, לא no ראו they see טובה׃ good.}%
\verse{חלפו They are passed away עם as אניות ships: אבה the swift כנשׁר as the eagle יטושׂ hasteth עלי to אכל׃ the prey.}%
\verse{אם If אמרי I say, אשׁכחה I will forget שׂיחי my complaint, אעזבה I will leave off פני my heaviness, ואבליגה׃ and comfort}%
\verse{יגרתי I am afraid כל of all עצבתי my sorrows, ידעתי I know כי that לא thou wilt not תנקני׃ hold me innocent.}%
\verse{אנכי I ארשׁע be wicked, למה why זה then הבל I in vain? איגע׃ labor}%
\verse{אם If התרחצתי I wash myself במו שׁלג water, והזכותי so clean; בבר never כפי׃ and make my hands}%
\verse{אז Yet בשׁחת me in the ditch, תטבלני shalt thou plunge ותעבוני shall abhor שׂלמותי׃ and mine own clothes}%
\verse{כי For לא not אישׁ a man, כמני as I אעננו I should answer נבוא him, we should come יחדו together במשׁפט׃ in judgment.}%
\verse{לא ישׁ is there בינינו between מוכיח any daysman ישׁת us, might lay ידו his hand על upon שׁנינו׃ us both.}%
\verse{יסר מעלי from שׁבטו ואמתו his fear אל me, and let not תבעתני׃ terrify}%
\verse{אדברה would I speak, ולא and not איראנו fear כי him; but לא not כן so אנכי עמדי׃ with}%
\end{biblechapter}%
\begin{biblechapter}% Job 10
\verseWithHeading{Job Continues His Response to Bildad}{נקטה is weary נפשׁי My soul בחיי of my life; אעזבה I will leave עלי upon שׂיחי my complaint אדברה myself; I will speak במר in the bitterness נפשׁי׃ of my soul.}%
\verse{אמר I will say אל unto אלוה God, אל Do not תרשׁיעני condemn הודיעני me; show על me wherefore מה me wherefore תריבני׃ thou contendest}%
\verse{הטוב good לך כי unto thee that תעשׁק thou shouldest oppress, כי that תמאס thou shouldest despise יגיע the work כפיך of thine hands, ועל upon עצת the counsel רשׁעים of the wicked? הופעת׃ and shine}%
\verse{העיני Hast thou eyes בשׂר of flesh? לך אם or כראות seest אנושׁ thou as man תראה׃ seeth?}%
\verse{הכימי thy days אנושׁ of man? ימיך as the days אם שׁנותיך thy years כימי days, גבר׃ as man's}%
\verse{כי That תבקשׁ thou inquirest לעוני after mine iniquity, ולחטאתי after my sin? תדרושׁ׃ and searchest}%
\verse{על דעתך Thou knowest כי that לא I am not ארשׁע wicked; ואין and none מידך out of thine hand. מציל׃ that can deliver}%
\verse{ידיך Thine hands עצבוני have made ויעשׂוני me and fashioned יחד me together סביב round about; ותבלעני׃ yet thou dost destroy}%
\verse{זכר Remember, נא I beseech thee, כי that כחמר me as the clay; עשׂיתני thou hast made ואל עפרשׁיבני׃ and wilt thou bring me into dust again?}%
\verse{הלא Hast thou not כחלב as milk, תתיכני poured me out וכגבנה me like cheese? תקפיאני׃ and curdled}%
\verse{עור me with skin ובשׂר and flesh, תלבישׁני Thou hast clothed ובעצמות me with bones וגידים and sinews. תסככני׃}%
\verse{חיים me life וחסד and favor, עשׂית עמדיפקדתך and thy visitation שׁמרה hath preserved רוחי׃ my spirit.}%
\verse{ואלה And these צפנת hast thou hid בלבבך in thine heart: ידעתי I know כי that זאת this עמך׃ with}%
\verse{אם If חטאתי I sin, ושׁמרתני then thou markest ומעוני me from mine iniquity. לא me, and thou wilt not תנקני׃ acquit}%
\verse{אם If רשׁעתי I be wicked, אללי woe לי וצדקתי unto me; and I be righteous, לא will I not אשׂא lift up ראשׁי my head. שׂבע קלון of confusion; וראה therefore see עניי׃ thou mine affliction;}%
\verse{ויגאה For it increaseth. כשׁחל me as a fierce lion: תצודני Thou huntest ותשׁב and again תתפלא׃ thou showest thyself marvelous}%
\verse{תחדשׁ Thou renewest עדיך thy witnesses נגדי against ותרב me, and increasest כעשׂך thine indignation עמדי חליפות me; changes וצבא and war עמי׃ upon}%
\verse{ולמה Wherefore מרחם out of the womb? הצאתני then hast thou brought me forth אגוע Oh that I had given up the ghost, ועין eye לא and no תראני׃ had seen}%
\verse{כאשׁר as though לא I had not הייתי I should have been אהיה been; מבטן from the womb לקבר to the grave. אובל׃ I should have been carried}%
\verse{הלא not מעט few? ימי my days יחדל cease ישׁית let me alone, ממני let me alone, ואבליגה that I may take comfort מעט׃ a little,}%
\verse{בטרם Before אלך I go ולא I shall not אשׁוב return, אל to ארץ the land חשׁך of darkness וצלמות׃ and the shadow of death;}%
\verse{ארץ A land עיפתה of darkness, כמו as אפל darkness צלמות of the shadow of death, ולא without סדרים any order, ותפע and the light כמו as אפל׃ darkness.}%
\end{biblechapter}%
\begin{biblechapter}% Job 11
\verseWithHeading{Zophar’s First Response to Job}{ויען Then answered צפר Zophar הנעמתי the Naamathite, ויאמר׃ and said,}%
\verse{הרב the multitude דברים of words לא Should not יענה be answered? ואם and should אישׁ a man שׂפתים full of talk יצדק׃ be justified?}%
\verse{בדיך Should thy lies מתים make men יחרישׁו hold their peace? ותלעג and when thou mockest, ואין shall no מכלם׃ man make thee ashamed?}%
\verse{ותאמר For thou hast said, זך pure, לקחי My doctrine ובר clean הייתי and I am בעיניך׃ in thine eyes.}%
\verse{ואולם But מי oh that יתן would אלוה God דבר speak, ויפתח and open שׂפתיו his lips עמך׃ against}%
\verse{ויגד And that he would show לך תעלמות thee the secrets חכמה of wisdom, כי that כפלים double לתושׁיה to that which is! ודע Know כי therefore that ישׁה exacteth לך אלוה God מעונך׃ of thee than thine iniquity}%
\verse{החקר Canst thou by searching אלוה God? תמצא find out אם עד unto תכלית perfection? שׁדי the Almighty תמצא׃ canst thou find out}%
\verse{גבהי as high שׁמים as heaven; מה what תפעל canst thou do? עמקה deeper משׁאול than hell; מה what תדע׃ canst thou know?}%
\verse{ארכה thereof longer מארץ than the earth, מדה The measure ורחבה and broader מני than the earth, ים׃ the sea.}%
\verse{אם If יחלף he cut off, ויסגיר and shut up, ויקהיל or gather together, ומי then who ישׁיבנו׃ can hinder}%
\verse{כי For הוא he ידע knoweth מתי men: שׁוא vain וירא he seeth און wickedness ולא also; will he not יתבונן׃ then consider}%
\verse{ואישׁ man נבוב For vain ילבב would be wise, ועיר colt. פרא a wild ass's אדם though man יולד׃ be born}%
\verse{אם If אתה thou הכינות prepare לבך thine heart, ופרשׂת and stretch out אליו toward כפך׃ thine hands}%
\verse{אם If און iniquity בידך in thine hand, הרחיקהו put it far away, ואל and let not תשׁכן dwell באהליך in thy tabernacles. עולה׃ wickedness}%
\verse{כי For אז then תשׂא shalt thou lift up פניך thy face ממום without spot; והיית yea, thou shalt be מצק steadfast, ולא and shalt not תירא׃ fear:}%
\verse{כי Because אתה thou עמל misery, תשׁכח shalt forget כמים as waters עברו pass away: תזכר׃ remember}%
\verse{ומצהרים than the noonday; יקום shall be clearer חלד And age תעפה thou shalt shine forth, כבקר as the morning. תהיה׃ thou shalt be}%
\verse{ובטחת And thou shalt be secure, כי because ישׁ there תקוה is hope; וחפרת yea, thou shalt dig לבטח in safety. תשׁכב׃ thou shalt take thy rest}%
\verse{ורבצת Also thou shalt lie down, ואין and none מחריד shall make afraid; וחלו shall make suit פניך unto רבים׃ yea, many}%
\verse{ועיני But the eyes רשׁעים of the wicked תכלינה shall fail, ומנוס and they shall not escape, אבד מנהםתקותם and their hope מפח the giving up נפשׁ׃ of the ghost.}%
\end{biblechapter}%
\begin{biblechapter}% Job 12
\verseWithHeading{Job’s Fourth Speech}{ויען answered איוב And Job ויאמר׃ and said,}%
\verse{אמנם No doubt כי but אתם ye עם the people, ועמכם with תמות shall die חכמה׃ and wisdom}%
\verse{גם But לי לבב I have understanding כמוכם as well as you; לא not נפל inferior אנכי I מכם to ואת you: yea, מי who אין knoweth not כמו such things as these? אלה׃ such things as these?}%
\verse{שׂחק one mocked לרעהו of his neighbor, אהיה I am קרא who calleth לאלוה upon God, ויענהו and he answereth שׂחוק laughed to scorn. צדיק him: the just תמים׃ upright}%
\verse{לפיד a lamp בוז despised לעשׁתות in the thought שׁאנן of him that is at ease. נכון He that is ready למועדי to slip רגל׃ with feet}%
\verse{ישׁליו prosper, אהלים The tabernacles לשׁדדים of robbers ובטחות are secure; למרגיזי and they that provoke אל God לאשׁר into whose הביא bringeth אלוה God בידו׃ hand}%
\verse{ואולם But שׁאל ask נא now בהמות the beasts, ותרך and they shall teach ועוף thee; and the fowls השׁמים of the air, ויגד׃ and they shall tell}%
\verse{או Or שׂיח speak לארץ to the earth, ותרך and it shall teach ויספרו shall declare לך דגי thee: and the fishes הים׃ of the sea}%
\verse{מי Who לא not ידע knoweth בכל in all אלה these כי that יד the hand יהוה עשׂתה hath wrought זאת׃ this?}%
\verse{אשׁר In whose בידו hand נפשׁ the soul כל of every חי living thing, ורוח and the breath כל of all בשׂר mankind. אישׁ׃ mankind.}%
\verse{הלא Doth not אזן the ear מלין words? תבחן try וחך and the mouth אכל his meat? יטעם׃ taste}%
\verse{בישׁישׁים With the ancient חכמה wisdom; וארך and in length ימים of days תבונה׃ understanding.}%
\verse{עמו With חכמה him wisdom וגבורה and strength, לו עצה he hath counsel ותבונה׃ and understanding.}%
\verse{הן Behold, יהרוס he breaketh down, ולא and it cannot יבנה be built again: יסגר he shutteth up על he shutteth up אישׁ a man, ולא and there can be no opening. יפתח׃ and there can be no opening.}%
\verse{הן Behold, יעצר he withholdeth במים the waters, ויבשׁו and they dry up: וישׁלחם also he sendeth them out, ויהפכו and they overturn ארץ׃ the earth.}%
\verse{עמו With עז him strength ותושׁיה and wisdom: לו שׁגג the deceived ומשׁגה׃ and the deceiver}%
\verse{מוליך He leadeth יועצים counselors שׁולל away spoiled, ושׁפטים and maketh the judges יהולל׃ fools.}%
\verse{מוסר the bond מלכים of kings, פתח He looseth ויאסר and girdeth אזור with a girdle. במתניהם׃ their loins}%
\verse{מוליך He leadeth כהנים princes שׁולל away spoiled, ואתנים the mighty. יסלף׃ and overthroweth}%
\verse{מסיר He removeth away שׂפה the speech לנאמנים of the trusty, וטעם the understanding זקנים of the aged. יקח׃ and taketh away}%
\verse{שׁופך He poureth בוז contempt על upon נדיבים princes, ומזיח the strength אפיקים of the mighty. רפה׃ and weakeneth}%
\verse{מגלה He discovereth עמקות deep things מני out of חשׁך darkness, ויצא and bringeth out לאור to light צלמות׃ the shadow of death.}%
\verse{משׂגיא He increaseth לגוים the nations, ויאבדם and destroyeth שׁטח them: he enlargeth לגוים the nations, וינחם׃ and straiteneth}%
\verse{מסיר He taketh away לב the heart ראשׁי of the chief עם of the people הארץ of the earth, ויתעם and causeth them to wander בתהו in a wilderness לא no דרך׃ way.}%
\verse{ימשׁשׁו They grope חשׁך in the dark ולא without אור light, ויתעם and he maketh them to stagger כשׁכור׃ like drunken}%
\end{biblechapter}%
\begin{biblechapter}% Job 13
\verseWithHeading{Job’s Fourth Speech Continues}{הן Lo, כל all ראתה hath seen עיני mine eye שׁמעה hath heard אזני mine ear ותבן׃ and understood}%
\verse{כדעתכם What ye know, ידעתי know גם also: אני do I לא not נפל inferior אנכי I מכם׃ unto}%
\verse{אולם Surely אני I אל to שׁדי the Almighty, אדבר would speak והוכח to reason אל with אל God. אחפץ׃ and I desire}%
\verse{ואולם But אתם ye טפלי forgers שׁקר of lies, רפאי physicians אלל of no value. כלכם׃ ye all}%
\verse{מי יתן ye would החרשׁ תחרישׁוןתהי and it should be לכם לחכמה׃ your wisdom.}%
\verse{שׁמעו Hear נא now תוכחתי my reasoning, ורבות to the pleadings שׂפתי of my lips. הקשׁיבו׃ and hearken}%
\verse{הלאל for God? תדברו Will ye speak עולה wickedly ולו תדברו and talk רמיה׃ deceitfully}%
\verse{הפניו his person? תשׂאון Will ye accept אם לאל for God? תריבון׃ will ye contend}%
\verse{הטוב Is it good כי that יחקר אתכםם or כהתל mocketh באנושׁ תהתלו׃ another, do ye mock}%
\verse{הוכח יוכיחתכם אם you, if בסתר ye do secretly פנים persons. תשׂאון׃ accept}%
\verse{הלא Shall not שׂאתו his excellency תבעת make you afraid? אתכם ופחדו and his dread יפל fall עליכם׃ upon}%
\verse{זכרניכם Your remembrances משׁלי אפר unto ashes, לגבי your bodies חמר of clay. גביכם׃ to bodies}%
\verse{החרישׁו Hold your peace, ממני let me alone, ואדברה may speak, אני that I ויעבר and let come עלי on מה׃ me what}%
\verse{על מהשׂא do I take בשׂרי my flesh בשׁני in my teeth, ונפשׁי my life אשׂים and put בכפי׃ in mine hand?}%
\verse{הן Though יקטלני he slay לא איחל me, yet will I trust אך in him: but דרכי mine own ways אל before פניו before אוכיח׃ I will maintain}%
\verse{גם also הוא He לי לישׁועה my salvation: כי for לא shall not לפניו before חנף a hypocrite יבוא׃ come}%
\verse{שׁמעו שׁמועלתי my speech, ואחותי and my declaration באזניכם׃ with your ears.}%
\verse{הנה Behold נא now, ערכתי I have ordered משׁפט cause; ידעתי I know כי that אני I אצדק׃ shall be justified.}%
\verse{מי Who הוא he יריב will plead עמדי with כי me? for עתה now, אחרישׁ if I hold my tongue, ואגוע׃ I shall give up the ghost.}%
\verseWithHeading{Job Argues His Case with God}{אך Only שׁתים two אל תעשׂ do עמדי unto אז me: then מפניך from לא will I not אסתר׃ hide myself}%
\verse{כפך thine hand מעלי far from הרחק Withdraw ואמתך thy dread אל me: and let not תבעתני׃ make me afraid.}%
\verse{וקרא Then call ואנכי thou, and I אענה will answer: או or אדבר let me speak, והשׁיבני׃ and answer}%
\verse{כמה How לי עונות many mine iniquities וחטאות and sins? פשׁעי my transgression וחטאתי and my sin. הדיעני׃ make me to know}%
\verse{למה Wherefore פניך thou thy face, תסתיר hidest ותחשׁבני and holdest לאויב׃ me for thine enemy?}%
\verse{העלה a leaf נדף driven to and fro? תערוץ Wilt thou break ואת קשׁ stubble? יבשׁ the dry תרדף׃ and wilt thou pursue}%
\verse{כי For תכתב thou writest עלי against מררות bitter things ותורישׁני me, and makest me to possess עונות the iniquities נעורי׃ of my youth.}%
\verse{ותשׂם Thou puttest בסד also in the stocks, רגלי my feet ותשׁמור and lookest narrowly כל unto all ארחותי my paths; על upon שׁרשׁי the heels רגלי of my feet. תתחקה׃ thou settest a print}%
\verse{והוא And he, כרקב as a rotten thing, יבלה consumeth, כבגד as a garment אכלו eaten. עשׁ׃ that is moth}%
\end{biblechapter}%
\begin{biblechapter}% Job 14
\verseWithHeading{Job Continues to Argue His Case with God}{אדם Man ילוד born אשׁה of a woman קצר of few ימים days, ושׂבע רגז׃ of trouble.}%
\verse{כציץ like a flower, יצא He cometh forth וימל and is cut down: ויברח he fleeth כצל also as a shadow, ולא not. יעמוד׃ and continueth}%
\verse{אף And על upon זה such a one, פקחת dost thou open עינך thine eyes ואתי תביא and bringest במשׁפט me into judgment עמך׃ with}%
\verse{מי Who יתן can bring טהור a clean מטמא out of an unclean? לא not אחד׃ one.}%
\verse{אם Seeing חרוצים determined, ימיו his days מספר the number חדשׁיו of his months אתך with חקו his bounds עשׂית thee, thou hast appointed ולא that he cannot יעבור׃ pass;}%
\verse{שׁעה Turn מעליו from ויחדל him, that he may rest, עד till ירצה he shall accomplish, כשׂכיר as a hireling, יומו׃ his day.}%
\verse{כי For ישׁ there is לעץ of a tree, תקוה hope אם if יכרת it be cut down, ועוד again, יחליף that it will sprout וינקתו and that the tender לא branch thereof will not תחדל׃ cease.}%
\verse{אם Though יזקין thereof wax old בארץ in the earth, שׁרשׁו the root ובעפר in the ground; ימות thereof die גזעו׃ and the stock}%
\verse{מריח מים of water יפרח it will bud, ועשׂה and bring forth קציר boughs כמו like נטע׃ a plant.}%
\verse{וגבר But man ימות dieth, ויחלשׁ and wasteth away: ויגוע giveth up the ghost, אדם yea, man ואיו׃ and where}%
\verse{אזלו fail מים the waters מני from ים the sea, ונהר and the flood יחרב decayeth ויבשׁ׃ and drieth up:}%
\verse{ואישׁ So man שׁכב lieth down, ולא not: יקום and riseth עד till בלתי no more, שׁמים the heavens לא they shall not יקיצו awake, ולא nor יערו be raised משׁנתם׃ out of their sleep.}%
\verse{מי O that יתן thou wouldest בשׁאול me in the grave, תצפנני hide תסתירני that thou wouldest keep me secret, עד until שׁוב be past, אפך thy wrath תשׁית that thou wouldest appoint לי חק me a set time, ותזכרני׃ and remember}%
\verse{אם If ימות die, גבר a man היחיה shall he live כל all ימי the days צבאי of my appointed time איחל will I wait, עד till בוא come. חליפתי׃ my change}%
\verse{תקרא Thou shalt call, ואנכי and I אענך will answer למעשׂה to the work ידיך of thine hands. תכסף׃ thee: thou wilt have a desire}%
\verse{כי For עתה now צעדי my steps: תספור thou numberest לא dost thou not תשׁמור watch על over חטאתי׃ my sin?}%
\verse{חתם sealed up בצרור in a bag, פשׁעי My transgression ותטפל and thou sewest up על and thou sewest up עוני׃ mine iniquity.}%
\verse{ואולם And surely הר the mountain נופל falling יבול cometh to naught, וצור and the rock יעתק is removed ממקמו׃ out of his place.}%
\verse{אבנים the stones: שׁחקו wear מים The waters תשׁטף thou washest away ספיחיה the things which grow עפר of the dust ארץ of the earth; ותקות the hope אנושׁ of man. האבדת׃ and thou destroyest}%
\verse{תתקפהו Thou prevailest לנצח forever ויהלך against him, and he passeth: משׁנה thou changest פניו his countenance, ותשׁלחהו׃ and sendest him away.}%
\verse{יכבדו come to honor, בניו His sons ולא not; ידע and he knoweth ויצערו and they are brought low, ולא not יבין׃ but he perceiveth}%
\verse{אך But בשׂרו his flesh עליו upon יכאב him shall have pain, ונפשׁו and his soul עליו within תאבל׃ him shall mourn.}%
\end{biblechapter}%
\begin{biblechapter}% Job 15
\verseWithHeading{Eliphaz’s Second Response to Job}{ויען Then answered אליפז Eliphaz התימני the Temanite, ויאמר׃ and said,}%
\verse{החכם Should a wise יענה man utter דעת knowledge, רוח vain וימלא and fill קדים with the east wind? בטנו׃ his belly}%
\verse{הוכח Should he reason בדבר talk? לא with unprofitable יסכון with unprofitable ומלים or with speeches לא wherewith he can do no יועיל׃ good?}%
\verse{אף Yea, אתה thou תפר castest off יראה fear, ותגרע and restrainest שׂיחה prayer לפני before אל׃ God.}%
\verse{כי For יאלף uttereth עונך thine iniquity, פיך thy mouth ותבחר and thou choosest לשׁון the tongue ערומים׃ of the crafty.}%
\verse{ירשׁיעך condemneth פיך Thine own mouth ולא thee, and not אני I: ושׂפתיך yea, thine own lips יענו׃ testify}%
\verse{הראישׁון thou the first אדם man תולד was born? ולפני before גבעות the hills? חוללת׃ or wast thou made}%
\verse{הבסוד the secret אלוה of God? תשׁמע Hast thou heard ותגרע and dost thou restrain אליך to חכמה׃ wisdom}%
\verse{מה What ידעת knowest ולא not? נדע thou, that we know תבין understandest ולא not עמנו in הוא׃ thou, which}%
\verse{גם With us both שׂב the grayheaded גם and ישׁישׁ very aged men, בנו כביר much elder מאביך than thy father. ימים׃ much elder}%
\verse{המעט small ממך with תנחמות the consolations אל of God ודבר thing לאט thee? is there any secret עמך׃ with}%
\verse{מה Why יקחך carry thee away? לבך doth thine heart ומה and what ירזמון wink at, עיניך׃ do thy eyes}%
\verse{כי That תשׁיב thou turnest אל against אל God, רוחך thy spirit והצאת go out מפיך of thy mouth? מלין׃ and lettest words}%
\verse{מה What אנושׁ man, כי that יזכה he should be clean? וכי that יצדק he should be righteous? ילוד and born אשׁה׃ of a woman,}%
\verse{הן Behold, בקדשׁו in his saints; לא he putteth no trust יאמין he putteth no trust ושׁמים yea, the heavens לא are not זכו clean בעיניו׃ in his sight.}%
\verse{אף כיתעב abominable ונאלח and filthy אישׁ man, שׁתה which drinketh כמים like water? עולה׃ iniquity}%
\verse{אחוך I will show שׁמע thee, hear לי וזה me; and that חזיתי I have seen ואספרה׃ I will declare;}%
\verse{אשׁר Which חכמים wise יגידו men have told ולא and have not כחדו hid מאבותם׃ from their fathers,}%
\verse{להם לבדם Unto whom alone נתנה was given, הארץ the earth ולא and no עבר passed זר stranger בתוכם׃ among}%
\verse{כל all ימי days, רשׁע The wicked man הוא מתחולל travaileth with pain ומספר and the number שׁנים of years נצפנו is hidden לעריץ׃ to the oppressor.}%
\verse{קול sound פחדים A dreadful באזניו in his ears: בשׁלום in prosperity שׁודד the destroyer יבואנו׃ shall come upon}%
\verse{לא not יאמין He believeth שׁוב that he shall return מני out of חשׁך darkness, וצפו is waited for הוא and he אלי of חרב׃ the sword.}%
\verse{נדד wandereth abroad הוא He ללחם for bread, איה ידע he knoweth כי that נכון is ready בידו at his hand. יום the day חשׁך׃ of darkness}%
\verse{יבעתהו shall make him afraid; צר Trouble ומצוקה and anguish תתקפהו they shall prevail against כמלך him, as a king עתיד ready לכידור׃ to the battle.}%
\verse{כי For נטה he stretcheth out אל against אל God, ידו his hand ואל against שׁדי the Almighty. יתגבר׃ and strengtheneth himself}%
\verse{ירוץ He runneth אליו upon בצואר him, on neck, בעבי upon the thick גבי bosses מגניו׃ of his bucklers:}%
\verse{כי Because כסה he covereth פניו his face בחלבו with his fatness, ויעשׂ and maketh פימה collops of fat עלי on כסל׃ flanks.}%
\verse{וישׁכון And he dwelleth in ערים cities, נכחדות desolate בתים in houses which לא no ישׁבו man inhabiteth, למו אשׁר which התעתדו are ready לגלים׃ to become heaps.}%
\verse{לא He shall not יעשׁר be rich, ולא neither יקום continue, חילו shall his substance ולא neither יטה shall he prolong לארץ thereof upon the earth. מנלם׃ the perfection}%
\verse{לא He shall not יסור depart מני out of חשׁך darkness; ינקתו his branches, תיבשׁ shall dry up שׁלהבת the flame ויסור shall he go away. ברוח and by the breath פיו׃ of his mouth}%
\verse{אל Let not יאמן trust בשׁו in vanity: נתעה him that is deceived כי for שׁוא vanity תהיה shall be תמורתו׃ his recompense.}%
\verse{בלא before יומו his time, תמלא It shall be accomplished וכפתו and his branch לא shall not רעננה׃ be green.}%
\verse{יחמס He shall shake off כגפן as the vine, בסרו his unripe grape וישׁלך and shall cast off כזית as the olive. נצתו׃ his flower}%
\verse{כי For עדת the congregation חנף of hypocrites גלמוד desolate, ואשׁ and fire אכלה shall consume אהלי the tabernacles שׁחד׃ of bribery.}%
\verse{הרה They conceive עמל mischief, וילד and bring forth און vanity, ובטנם and their belly תכין prepareth מרמה׃ deceit.}%
\end{biblechapter}%
\begin{biblechapter}% Job 16
\verseWithHeading{Job’s Fifth Speech}{ויען answered איוב Then Job ויאמר׃ and said,}%
\verse{שׁמעתי I have heard כאלה such things: רבות many מנחמי comforters עמל miserable כלכם׃ ye all.}%
\verse{הקץ have an end? לדברי words רוח Shall vain או or מה what ימריצך emboldeneth כי thee that תענה׃ thou answerest?}%
\verse{גם also אנכי I ככם אדברה could speak לו as ye if ישׁ were נפשׁכם your soul תחת נפשׁיחבירה I could heap up עליכם against במלים words ואניעה you, and shake עליכם at במו mine ראשׁי׃ head}%
\verse{אאמצכם I would strengthen במו you with פי my mouth, וניד and the moving שׂפתי of my lips יחשׂך׃ should assuage}%
\verse{אם Though אדברה I speak, לא is not יחשׂך assuaged: כאבי my grief ואחדלה and I forbear, מה what מני what יהלך׃ am I eased?}%
\verse{אך But עתה now הלאני he hath made me weary: השׁמות thou hast made desolate כל all עדתי׃ my company.}%
\verse{ותקמטני And thou hast filled me with wrath, לעד a witness היה is ויקם rising up בי כחשׁי and my leanness בפני to my face. יענה׃ in me beareth witness}%
\verse{אפו in his wrath, טרף He teareth וישׂטמני who hateth חרק me: he gnasheth עלי upon בשׁניו me with his teeth; צרי mine enemy ילטושׁ sharpeneth עיניו׃ his eyes}%
\verse{פערו They have gaped עלי upon בפיהם me with their mouth; בחרפה reproachfully; הכו they have smitten לחיי me upon the cheek יחד together עלי against יתמלאון׃ they have gathered themselves}%
\verse{יסגירני hath delivered אל God אל me to עויל the ungodly, ועל into ידי the hands רשׁעים of the wicked. ירטני׃ and turned me over}%
\verse{שׁלו at ease, הייתי I was ויפרפרני but he hath broken me asunder: ואחז he hath also taken בערפי by my neck, ויפצפצני and shaken me to pieces, ויקימני and set me up לו למטרה׃ for his mark.}%
\verse{יסבו compass me round about, עלי compass me round about, רביו His archers יפלח he cleaveth כליותי my reins ולא asunder, and doth not יחמול spare; ישׁפך he poureth out לארץ upon the ground. מררתי׃ my gall}%
\verse{יפרצני He breaketh פרץ me with breach על upon פני upon פרץ breach, ירץ he runneth עלי upon כגבור׃ me like a giant.}%
\verse{שׂק sackcloth תפרתי I have sewed עלי upon גלדי my skin, ועללתי and defiled בעפר in the dust. קרני׃ my horn}%
\verse{פני My face חמרמרה is foul מני with בכי weeping, ועל and on עפעפי my eyelids צלמות׃ the shadow of death;}%
\verse{על for לא Not חמס injustice בכפי in mine hands: ותפלתי also my prayer זכה׃ pure.}%
\verse{ארץ O earth, אל not תכסי cover דמי thou my blood, ואל no יהי have מקום place. לזעקתי׃ and let my cry}%
\verse{גם Also עתה now, הנה behold, בשׁמים in heaven, עדי my witness ושׂהדי and my record במרומים׃ on high.}%
\verse{מליצי scorn רעי My friends אל unto אלוה God. דלפה poureth out עיני׃ me: mine eye}%
\verse{ויוכח O that one might plead לגבר for a man עם with אלוה God, ובן as a man אדם as a man לרעהו׃ for his neighbor!}%
\verse{כי When שׁנות years מספר a few יאתיו are come, וארח the way לא I shall not אשׁוב return. אהלך׃ then I shall go}%
\end{biblechapter}%
\begin{biblechapter}% Job 17
\verseWithHeading{Job’s Fifth Speech, Continued}{רוחי My breath חבלה is corrupt, ימי my days נזעכו are extinct, קברים׃ the graves}%
\verse{אם לא not התלים mockers עמדי with ובהמרותם in their provocation? תלן continue עיני׃ me? and doth not mine eye}%
\verse{שׂימה Lay down נא now, ערבני put me in a surety עמך with מי thee; who הוא he לידי hands יתקע׃ will strike}%
\verse{כי For לבם their heart צפנת thou hast hid משׂכל from understanding: על therefore כן therefore לא shalt thou not תרמם׃ exalt}%
\verse{לחלק flattery יגיד He that speaketh רעים to friends, ועיני even the eyes בניו of his children תכלנה׃ shall fail.}%
\verse{והצגני He hath made למשׁל me also a byword עמים of the people; ותפת as a tabret. לפנים and formerly אהיה׃ I was}%
\verse{ותכה also is dim מכעשׂ by reason of sorrow, עיני Mine eye ויצרי my members כצל as a shadow. כלם׃ and all}%
\verse{ישׁמו shall be astonished ישׁרים Upright על at זאת this, ונקי and the innocent על against חנף the hypocrite. יתערר׃ shall stir up himself}%
\verse{ויאחז also shall hold צדיק The righteous דרכו on his way, וטהר ידים hands יסיף shall be stronger and stronger. אמץ׃ shall be stronger and stronger.}%
\verse{ואולם But כלם as for you all, תשׁבו do ye return, ובאו and come נא now: ולא for I cannot אמצא find בכם חכם׃ wise}%
\verse{ימי My days עברו are past, זמתי my purposes נתקו are broken off, מורשׁי the thoughts לבבי׃ of my heart.}%
\verse{לילה the night ליום into day: ישׂימו They change אור the light קרוב short מפני because חשׁך׃ of darkness.}%
\verse{אם If אקוה I wait, שׁאול the grave ביתי mine house: בחשׁך in the darkness. רפדתי I have made יצועי׃ my bed}%
\verse{לשׁחת to corruption, קראתי I have said אבי my father: אתה Thou אמי my mother, ואחתי and my sister. לרמה׃ to the worm,}%
\verse{ואיה And where אפו now תקותי my hope? ותקותי as for my hope, מי who ישׁורנה׃ shall see}%
\verse{בדי to the bars שׁאל of the pit, תרדנה They shall go down אם when יחד together על in עפר the dust. נחת׃ rest}%
\end{biblechapter}%
\begin{biblechapter}% Job 18
\verseWithHeading{Bildad’s Second Speech}{ויען Then answered בלדד Bildad השׁחי the Shuhite, ויאמר׃ and said,}%
\verse{עד אנהשׂימון ye make קנצי an end למלין of words? תבינו mark, ואחר and afterwards נדבר׃ we will speak.}%
\verse{מדוע Wherefore נחשׁבנו are we counted כבהמה as beasts, נטמינו reputed vile בעיניכם׃ in your sight?}%
\verse{טרף He teareth נפשׁו himself באפו in his anger: הלמענך for תעזב be forsaken ארץ shall the earth ויעתק be removed צור thee? and shall the rock ממקמו׃ out of his place?}%
\verse{גם Yea, אור the light רשׁעים of the wicked ידעך shall be put out, ולא shall not יגה shine. שׁביב and the spark אשׁו׃ of his fire}%
\verse{אור The light חשׁך shall be dark באהלו in his tabernacle, ונרו and his candle עליו with ידעך׃ shall be put out}%
\verse{יצרו shall be straitened, צעדי The steps אונו of his strength ותשׁליכהו shall cast him down. עצתו׃ and his own counsel}%
\verse{כי For שׁלח he is cast ברשׁת into a net ברגליו by his own feet, ועל upon שׂבכה a snare. יתהלך׃ and he walketh}%
\verse{יאחז shall take בעקב by the heel, פח The gin יחזק shall prevail עליו against צמים׃ the robber}%
\verse{טמון laid בארץ for him in the ground, חבלו The snare ומלכדתו and a trap עלי for him in נתיב׃ the way.}%
\verse{סביב on every side, בעתהו shall make him afraid בלהות Terrors והפיצהו and shall drive לרגליו׃ him to his feet.}%
\verse{יהי shall be רעב hunger-bitten, אנו His strength ואיד and destruction נכון ready לצלעו׃ at his side.}%
\verse{יאכל It shall devour בדי the strength עורו of his skin: יאכל shall devour בדיו his strength. בכור the firstborn מות׃ of death}%
\verse{ינתק shall be rooted out מאהלו of his tabernacle, מבטחו His confidence ותצעדהו and it shall bring למלך him to the king בלהות׃ of terrors.}%
\verse{תשׁכון It shall dwell באהלו in his tabernacle, מבלי because none לו יזרה shall be scattered על upon נוהו his habitation. גפרית׃ of his: brimstone}%
\verse{מתחת beneath, שׁרשׁיו His roots יבשׁו shall be dried up וממעל and above ימל be cut off. קצירו׃ shall his branch}%
\verse{זכרו His remembrance אבד shall perish מני from ארץ the earth, ולא and he shall have no שׁם name לו על in פני the street. חוץ׃ the street.}%
\verse{יהדפהו He shall be driven מאור from light אל into חשׁך darkness, ומתבל out of the world. ינדהו׃ and chased}%
\verse{לא He shall neither נין have son לו ולא nor נכד nephew בעמו among his people, ואין nor שׂריד any remaining במגוריו׃ in his dwellings.}%
\verse{על at יומו his day, נשׁמו shall be astonished אחרנים They that come after וקדמנים as they that went before אחזו שׂער׃ were frightened.}%
\verse{אך Surely אלה such משׁכנות the dwellings עול of the wicked, וזה and this מקום the place לא not ידע knoweth אל׃ God.}%
\end{biblechapter}%
\begin{biblechapter}% Job 19
\verseWithHeading{Job’s Sixth Speech: A Response to Bildad}{ויען answered איוב Then Job ויאמר׃ and said,}%
\verse{עד אנהוגיון will ye vex נפשׁי my soul, ותדכאונני and break me in pieces במלים׃ with words?}%
\verse{זה These עשׂר ten פעמים times תכלימוני have ye reproached לא me: ye are not תבשׁו ashamed תהכרו׃ ye make yourselves strange}%
\verse{ואף And אמנם be it indeed שׁגיתי I have erred, אתי with תלין remaineth משׁוגתי׃ mine error}%
\verse{אם If אמנם indeed עלי against תגדילו ye will magnify ותוכיחו me, and plead עלי against חרפתי׃ me my reproach:}%
\verse{דעו Know אפו now כי that אלוה God עותני hath overthrown ומצודו עלי me, and hath compassed הקיף׃ me, and hath compassed}%
\verse{הן Behold, אצעק I cry out חמס of wrong, ולא but I am not אענה heard: אשׁוע I cry aloud, ואין but no משׁפט׃ judgment.}%
\verse{ארחי my way גדר He hath fenced up ולא that I cannot אעבור pass, ועל in נתיבותי my paths. חשׁך darkness ישׂים׃ and he hath set}%
\verse{כבודי my glory, מעלי הפשׁיטיסר and taken עטרת the crown ראשׁי׃ my head.}%
\verse{יתצני He hath destroyed סביב me on every side, ואלך and I am gone: ויסע hath he removed כעץ like a tree. תקותי׃ and mine hope}%
\verse{ויחר He hath also kindled עלי against אפו his wrath ויחשׁבני me, and he counteth לו כצריו׃ me unto him as his enemies.}%
\verse{יחד together, יבאו come גדודיו His troops ויסלו and raise up עלי against דרכם their way ויחנו me, and encamp סביב round about לאהלי׃ my tabernacle.}%
\verse{אחי מעלי from הרחיק וידעי me, and mine acquaintance אך are verily זרו estranged ממני׃ from}%
\verse{חדלו have failed, קרובי My kinsfolk ומידעי and my familiar friends שׁכחוני׃ have forgotten}%
\verse{גרי They that dwell ביתי in mine house, ואמהתי and my maids, לזר me for a stranger: תחשׁבני count נכרי an alien הייתי I am בעיניהם׃ in their sight.}%
\verse{לעבדי my servant, קראתי I called ולא and he gave no יענה answer; במו him with פי my mouth. אתחנן׃ I entreated}%
\verse{רוחי My breath זרה is strange לאשׁתי to my wife, וחנתי though I entreated לבני for the children's בטני׃ of mine own body.}%
\verse{גם Yea, עוילים young children מאסו despised בי אקומה me; I arose, וידברו׃ and they spoke}%
\verse{תעבוני abhorred כל All מתי friends סודי my inward וזה me: and they whom אהבתי I loved נהפכו׃ are turned}%
\verse{בעורי to my skin ובבשׂרי and to my flesh, דבקה cleaveth עצמי My bone ואתמלטה and I am escaped בעור with the skin שׁני׃ of my teeth.}%
\verse{חנני Have pity upon חנני me, have pity upon אתם me, O ye רעי my friends; כי for יד the hand אלוה of God נגעה׃ hath touched}%
\verse{למה Why תרדפני do ye persecute כמו me as אל God, ומבשׂרי with my flesh? לא and are not תשׂבעו׃ satisfied}%
\verse{מי Oh that יתן were אפו now ויכתבון written! מלי my words מי oh that יתן they were בספר in a book! ויחקו׃ printed}%
\verse{בעט pen ברזל with an iron ועפרת and lead לעד forever! בצור in the rock יחצבון׃ That they were graven}%
\verse{ואני For I ידעתי know גאלי my redeemer חי liveth, ואחרון at the latter על upon עפר the earth: יקום׃ and he shall stand}%
\verse{ואחר And after עורי my skin נקפו destroy זאת this ומבשׂרי , yet in my flesh אחזה shall I see אלוה׃ God:}%
\verse{אשׁר Whom אני I אחזה shall see לי ועיני for myself, and mine eyes ראו shall behold, ולא and not זר another; כלו be consumed כליתי my reins בחקי׃ within}%
\verse{כי But תאמרו ye should say, מה Why נרדף persecute לו ושׁרשׁ we him, seeing the root דבר of the matter נמצא׃ is found}%
\verse{גורו Be ye afraid לכם מפני of חרב the sword: כי for חמה wrath עונות the punishments חרב of the sword, למען that תדעון ye may know שׁדין׃ a judgment.}%
\end{biblechapter}%
\begin{biblechapter}% Job 20
\verseWithHeading{Zophar’s Second Speech}{ויען Then answered צפר Zophar הנעמתי the Naamathite, ויאמר׃ and said,}%
\verse{לכן Therefore שׂעפי do my thoughts ישׁיבוני cause me to answer, ובעבור and for חושׁי׃ I make haste.}%
\verse{מוסר the check כלמתי of my reproach, אשׁמע I have heard ורוח and the spirit מבינתי of my understanding יענני׃ causeth me to answer.}%
\verse{הזאת thou this ידעת Knowest מני of עד old, מני since שׂים was placed אדם man עלי upon ארץ׃ earth,}%
\verse{כי That רננת the triumphing רשׁעים of the wicked מקרוב short, ושׂמחת and the joy חנף of the hypocrite עדי for רגע׃ a moment?}%
\verse{אם Though יעלה mount up לשׁמים to the heavens, שׂיאו his excellency וראשׁו and his head לעב unto the clouds; יגיע׃ reach}%
\verse{כגללו like his own dung: לנצח forever יאבד he shall perish ראיו they which have seen יאמרו him shall say, איו׃ Where}%
\verse{כחלום as a dream, יעוף He shall fly away ולא and shall not ימצאוהו be found: וידד yea, he shall be chased away כחזיון as a vision לילה׃ of the night.}%
\verse{עין The eye שׁזפתו also saw ולא him shall no תוסיף more; ולא neither עוד any more תשׁורנו behold מקומו׃ shall his place}%
\verse{בניו His children ירצו shall seek to please דלים the poor, וידיו and his hands תשׁבנה shall restore אונו׃ their goods.}%
\verse{עצמותיו His bones מלאו are full עלומו of his youth, ועמו with על him in עפר the dust. תשׁכב׃ which shall lie down}%
\verse{אם Though תמתיק be sweet בפיו in his mouth, רעה wickedness יכחידנה he hide תחת it under לשׁונו׃ his tongue;}%
\verse{יחמל עליהלא it not; יעזבנה it, and forsake וימנענה but keep it still בתוך within חכו׃ his mouth:}%
\verse{לחמו his meat במעיו in his bowels נהפך is turned, מרורת the gall פתנים of asps בקרבו׃ within}%
\verse{חיל riches, בלע He hath swallowed down ויקאנו and he shall vomit them up again: מבטנו his belly. יורשׁנו shall cast them out אל׃ God}%
\verse{ראשׁ the poison פתנים of asps: יינק He shall suck תהרגהו shall slay לשׁון tongue אפעה׃ the viper's}%
\verse{אל He shall not ירא see בפלגות the rivers, נהרי the floods, נחלי the brooks דבשׁ of honey וחמאה׃ and butter.}%
\verse{משׁיב shall he restore, יגע That which he labored for ולא and shall not יבלע swallow down: כחיל according to substance תמורתו the restitution ולא and he shall not יעלס׃ rejoice}%
\verse{כי Because רצץ he hath oppressed עזב hath forsaken דלים the poor; בית a house גזל he hath violently taken away ולא not; יבנהו׃ which he built}%
\verse{כי Surely לא he shall not ידע feel שׁלו quietness בבטנו in his belly, בחמודו of that which he desired. לא he shall not ימלט׃ save}%
\verse{אין There shall none שׂריד be left; לאכלו of his meat על therefore כן therefore לא shall no יחיל man look טובו׃ for his goods.}%
\verse{במלאות In the fullness שׂפקו of his sufficiency יצר he shall be in straits: לו כל every יד hand עמל of the wicked תבואנו׃ shall come}%
\verse{יהי he למלא is about to fill בטנו his belly, ישׁלח shall cast בו חרון the fury אפו of his wrath וימטר upon him, and shall rain עלימו upon בלחומו׃ him while he is eating.}%
\verse{יברח He shall flee מנשׁק weapon, ברזל from the iron תחלפהו shall strike him through. קשׁת the bow נחושׁה׃ of steel}%
\verse{שׁלף It is drawn, ויצא and cometh out מגוה of the body; וברק yea, the glittering sword ממררתו out of his gall: יהלך cometh עליו upon אמים׃ terrors}%
\verse{כל All חשׁך darkness טמון hid לצפוניו in his secret places: תאכלהו shall consume אשׁ a fire לא not נפח blown ירע שׂריד with him that is left באהלו׃ in his tabernacle.}%
\verse{יגלו shall reveal שׁמים The heaven עונו his iniquity; וארץ and the earth מתקוממה׃ shall rise up}%
\verse{יגל shall depart, יבול The increase ביתו of his house נגרות shall flow away ביום in the day אפו׃ of his wrath.}%
\verse{זה This חלק the portion אדם man רשׁע of a wicked מאלהים ונחלת and the heritage אמרו appointed מאל׃}%
\end{biblechapter}%
\begin{biblechapter}% Job 21
\verseWithHeading{Job’s Seventh Speech: A Response to Zophar}{ויען answered איוב But Job ויאמר׃ and said,}%
\verse{שׁמעו שׁמועלתי my speech, ותהי be זאת and let this תנחומתיכם׃ your consolations.}%
\verse{שׂאוני Suffer ואנכי me that I אדבר may speak; ואחר and after that דברי I have spoken, תלעיג׃ mock on.}%
\verse{האנכי As for me, לאדם to man? שׂיחי my complaint ואם and if מדוע why לא should not תקצר be troubled? רוחי׃ my spirit}%
\verse{פנו אליהשׁמו me, and be astonished, ושׂימו and lay יד hand על upon פה׃ mouth.}%
\verse{ואם Even when זכרתי I remember ונבהלתי I am afraid, ואחז taketh hold בשׂרי on my flesh. פלצות׃ and trembling}%
\verse{מדוע Wherefore רשׁעים do the wicked יחיו live, עתקו become old, גם yea, גברו are mighty חיל׃ in power?}%
\verse{זרעם Their seed נכון is established לפניהם in their sight עמם with וצאצאיהם them, and their offspring לעיניהם׃ before their eyes.}%
\verse{בתיהם Their houses שׁלום safe מפחד from fear, ולא neither שׁבט the rod אלוה of God עליהם׃ upon}%
\verse{שׁורו Their bull עבר engendereth, ולא not; יגעל and faileth תפלט calveth, פרתו their cow ולא and casteth not her calf. תשׁכל׃ and casteth not her calf.}%
\verse{ישׁלחו They send forth כצאן like a flock, עויליהם their little ones וילדיהם and their children ירקדון׃ dance.}%
\verse{ישׂאו They take כתף the timbrel וכנור and harp, וישׂמחו and rejoice לקול at the sound עוגב׃ of the organ.}%
\verse{יבלו בטוב in wealth, ימיהם their days וברגע and in a moment שׁאול to the grave. יחתו׃ go down}%
\verse{ויאמרו Therefore they say לאל unto God, סור Depart ממנו from ודעת the knowledge דרכיך of thy ways. לא not חפצנו׃ us; for we desire}%
\verse{מה What שׁדי the Almighty, כי that נעבדנו we should serve ומה him? and what נועיל profit כי should we have, if נפגע׃ we pray}%
\verse{הן Lo, לא not בידם in their hand: טובם their good עצת the counsel רשׁעים of the wicked רחקה is far מני׃ from}%
\verse{כמה How oft נר is the candle רשׁעים of the wicked ידעך put out! ויבא and cometh עלימו upon אידם their destruction חבלים sorrows יחלק them! distributeth באפו׃ in his anger.}%
\verse{יהיו They are כתבן as stubble לפני before רוח the wind, וכמץ and as chaff גנבתו carrieth away. סופה׃ that the storm}%
\verse{אלוה God יצפן layeth up לבניו for his children: אונו his iniquity ישׁלם he rewardeth אליו he rewardeth וידע׃ him, and he shall know}%
\verse{יראו shall see עינו His eyes כידו his destruction, ומחמת of the wrath שׁדי of the Almighty. ישׁתה׃ and he shall drink}%
\verse{כי For מה what חפצו pleasure בביתו he in his house אחריו after ומספר him, when the number חדשׁיו of his months חצצו׃ is cut off in the midst?}%
\verse{הלאל God ילמד Shall teach דעת knowledge? והוא seeing he רמים those that are high. ישׁפוט׃ judgeth}%
\verse{זה One ימות dieth בעצם strength, תמו in his full כלו being wholly שׁלאנן at ease ושׁליו׃ and quiet.}%
\verse{עטיניו His breasts מלאו are full חלב of milk, ומח with marrow. עצמותיו and his bones ישׁקה׃ are moistened}%
\verse{וזה And another ימות dieth בנפשׁ of his soul, מרה in the bitterness ולא and never אכל eateth בטובה׃ with pleasure.}%
\verse{יחד alike על in עפר the dust, ישׁכבו They shall lie down ורמה and the worms תכסה shall cover עליהם׃ shall cover}%
\verse{הן Behold, ידעתי I know מחשׁבותיכם your thoughts, ומזמות and the devices עלי against תחמסו׃ ye wrongfully imagine}%
\verse{כי For תאמרו ye say, איה Where בית the house נדיב of the prince? ואיה and where אהל places משׁכנות the dwelling רשׁעים׃ of the wicked?}%
\verse{הלא Have ye not שׁאלתם asked עוברי them that go דרך by the way? ואתתם their tokens, לא and do ye not תנכרו׃ know}%
\verse{כי That ליום to the day איד of destruction? יחשׂך is reserved רע the wicked ליום to the day עברות of wrath. יובלו׃ they shall be brought forth}%
\verse{מי Who יגיד shall declare על to פניו his face? דרכו his way והוא him he עשׂה hath done? מי and who ישׁלם׃ shall repay}%
\verse{והוא Yet shall he לקברות to the grave, יובל be brought ועל in גדישׁ the tomb. ישׁקוד׃ and shall remain}%
\verse{מתקו shall be sweet לו רגבי The clods נחל of the valley ואחריו after כל unto him, and every אדם man ימשׁוך shall draw ולפניו before אין him, as innumerable מספר׃ him, as innumerable}%
\verse{ואיך How תנחמוני then comfort הבל ye me in vain, ותשׁובתיכם seeing in your answers נשׁאר there remaineth מעל׃ falsehood?}%
\end{biblechapter}%
\begin{biblechapter}% Job 22
\verseWithHeading{Eliphaz’s Third Speech}{ויען answered אליפז Then Eliphaz התמני the Temanite ויאמר׃ and said,}%
\verse{הלאל unto God, יסכן be profitable גבר Can a man כי as he that יסכן may be profitable עלימו unto משׂכיל׃ is wise}%
\verse{החפץ any pleasure לשׁדי to the Almighty, כי that תצדק thou art righteous? ואם or בצע gain כי that תתם thou makest thy ways perfect? דרכיך׃ thou makest thy ways perfect?}%
\verse{המיראתך thee for fear יכיחך Will he reprove יבוא of thee? will he enter עמך with במשׁפט׃ thee into judgment?}%
\verse{הלא not רעתך thy wickedness רבה great? ואין infinite? קץ infinite? לעונתיך׃ and thine iniquities}%
\verse{כי For תחבל thou hast taken a pledge אחיך from thy brother חנם for naught, ובגדי of their clothing. ערומים the naked תפשׁיט׃ and stripped}%
\verse{לא Thou hast not מים given water עיף to the weary תשׁקה to drink, ומרעב תמנע and thou hast withheld לחם׃ bread}%
\verse{ואישׁ man, זרוע But the mighty לו הארץ he had the earth; ונשׂוא and the honorable man פנים and the honorable man ישׁב׃ dwelt}%
\verse{אלמנות שׁלחתיקם empty, וזרעות and the arms יתמים of the fatherless ידכא׃ have been broken.}%
\verse{על כןביבותיך round about פחים snares ויבהלך troubleth פחד fear פתאם׃ thee, and sudden}%
\verse{או Or חשׁך darkness, לא thou canst not תראה see; ושׁפעת and abundance מים of waters תכסך׃ cover}%
\verse{הלא not אלוה God גבה in the height שׁמים of heaven? וראה and behold ראשׁ the height כוכבים of the stars, כי how רמו׃ high they are!}%
\verse{ואמרת And thou sayest, מה How ידע know? אל doth God הבעד through ערפל the dark cloud? ישׁפוט׃ can he judge}%
\verse{עבים Thick clouds סתר a covering לו ולא not; יראה to him, that he seeth וחוג in the circuit שׁמים of heaven. יתהלך׃ and he walketh}%
\verse{הארח way עולם the old תשׁמר Hast thou marked אשׁר which דרכו have trodden? מתי men און׃ wicked}%
\verse{אשׁר Which קמטו were cut down ולא out of עת time, נהר with a flood: יוצק was overflown יסודם׃ whose foundation}%
\verse{האמרים Which said לאל unto God, סור Depart ממנו from ומה us: and what יפעל do שׁדי׃ can the Almighty}%
\verse{והוא Yet he מלא filled בתיהם their houses טוב with good ועצת but the counsel רשׁעים of the wicked רחקה is far מני׃ from}%
\verse{יראו see צדיקים The righteous וישׂמחו and are glad: ונקי and the innocent ילעג׃ laugh them to scorn.}%
\verse{אם Whereas לא is not נכחד cut down, קימנו our substance ויתרם but the remnant אכלה consumeth. אשׁ׃ of them the fire}%
\verse{הסכן Acquaint נא now עמו thyself with ושׁלם him, and be at peace: בהם תבואתך shall come טובה׃ thereby good}%
\verse{קח Receive, נא I pray thee, מפיו from his mouth, תורה the law ושׂים and lay up אמריו his words בלבבך׃ in thine heart.}%
\verse{אם If תשׁוב thou return עד to שׁדי the Almighty, תבנה thou shalt be built up, תרחיק far עולה thou shalt put away iniquity מאהלך׃ from thy tabernacles.}%
\verse{ושׁית Then shalt thou lay up על as עפר dust, בצר gold ובצור as the stones נחלים of the brooks. אופיר׃ and the of Ophir}%
\verse{והיה shall be שׁדי Yea, the Almighty בצריך thy defense, וכסף of silver. תועפות׃ and thou shalt have plenty}%
\verse{כי For אז then על in שׁדי the Almighty, תתענג shalt thou have thy delight ותשׂא and shalt lift up אל unto אלוה God. פניך׃ thy face}%
\verse{תעתיר Thou shalt make thy prayer אליו unto וישׁמעך him, and he shall hear ונדריך thy vows. תשׁלם׃ thee, and thou shalt pay}%
\verse{ותגזר Thou shalt also decree אומר a thing, ויקם and it shall be established לך ועל upon דרכיך thy ways. נגה shall shine אור׃ unto thee: and the light}%
\verse{כי When השׁפילו are cast down, ותאמר then thou shalt say, גוה lifting up; ושׁח the humble person. עינים the humble person. יושׁע׃ and he shall save}%
\verse{ימלט He shall deliver אי the island נקי of the innocent: ונמלט and it is delivered בבר by the pureness כפיך׃ of thine hands.}%
\end{biblechapter}%
\begin{biblechapter}% Job 23
\verseWithHeading{Job’s Eighth Speech: A Response to Eliphaz}{ויען answered איוב Then Job ויאמר׃ and said,}%
\verse{גם Even היום today מרי bitter: שׂחי my complaint ידי my stroke כבדה is heavier על than אנחתי׃ my groaning.}%
\verse{מי יתןדעתי I knew ואמצאהו where I might find אבוא him! I might come עד to תכונתו׃ his seat!}%
\verse{אערכה I would order לפניו before משׁפט cause ופי my mouth אמלא him, and fill תוכחות׃ with arguments.}%
\verse{אדעה I would know מלים the words יענני he would answer ואבינה me, and understand מה what יאמר׃ he would say}%
\verse{הברב כח power? יריב Will he plead עמדי against לא No; אך but הוא he ישׂם׃ would put}%
\verse{שׁם There ישׁר the righteous נוכח might dispute עמו with ואפלטה him; so should I be delivered לנצח forever משׁפטי׃ from my judge.}%
\verse{הן Behold, קדם forward, אהלך I go ואיננו but he not ואחור and backward, ולא but I cannot אבין׃ perceive}%
\verse{שׂמאול On the left hand, בעשׂתו where he doth work, ולא but I cannot אחז behold יעטף he hideth ימין himself on the right hand, ולא that I cannot אראה׃ see}%
\verse{כי But ידע he knoweth דרך the way עמדי that I take: בחנני he hath tried כזהב as gold. אצא׃ me, I shall come forth}%
\verse{באשׁרו his steps, אחזה hath held רגלי My foot דרכו his way שׁמרתי have I kept, ולא and not אט׃ declined.}%
\verse{מצות from the commandment שׂפתיו of his lips; ולא Neither אמישׁ have I gone back מחקי more than my necessary צפנתי I have esteemed אמרי the words פיו׃ of his mouth}%
\verse{והוא But he באחד in one ומי and who ישׁיבנו can turn ונפשׁו him? and his soul אותה desireth, ויעשׂ׃ even he doeth.}%
\verse{כי For ישׁלים he performeth חקי appointed וכהנה such רבות for me: and many עמו׃ with}%
\verse{על כןפניו at his presence: אבהל am I troubled אתבונן when I consider, ואפחד I am afraid ממנו׃ at his presence:}%
\verse{ואל For God הרך maketh my heart soft לבי maketh my heart soft ושׁדי and the Almighty הבהילני׃ troubleth}%
\verse{כי Because לא I was not נצמתי cut off מפני before חשׁך the darkness, ומפני from my face. כסה hath he covered אפל׃ the darkness}%
\end{biblechapter}%
\begin{biblechapter}% Job 24
\verseWithHeading{Job’s Eighth Speech, Continued}{מדוע Why, משׁדי לא are not נצפנו hidden עתים seeing times וידעו do they that know לא him not חזו see ימיו׃ his days?}%
\verse{גבלות the landmarks; ישׂיגו remove עדר flocks, גזלו they violently take away וירעו׃ and feed}%
\verse{חמור the ass יתומים of the fatherless, ינהגו They drive away יחבלו שׁורלמנה׃}%
\verse{יטו They turn אביונים the needy מדרך out of the way: יחד themselves together. חבאו hide עניי ארץ׃ of the earth}%
\verse{הן Behold, פראים wild asses במדבר in the desert, יצאו go they forth בפעלם to their work; משׁחרי rising quickly לטרף for a prey: ערבה the wilderness לו לחם food לנערים׃ for them for children.}%
\verse{בשׂדה in the field: בלילו his corn יקצירו They reap וכרם the vintage רשׁע of the wicked. ילקשׁו׃ and they gather}%
\verse{ערום They cause the naked ילינו to lodge מבלי without לבושׁ clothing, ואין that no כסות covering בקרה׃ in the cold.}%
\verse{מזרם with the showers הרים of the mountains, ירטבו They are wet ומבלי for want מחסה of a shelter. חבקו and embrace צור׃ the rock}%
\verse{יגזלו They pluck משׁד from the breast, יתום the fatherless ועל עני the poor. יחבלו׃ and take a pledge}%
\verse{ערום naked הלכו They cause to go בלי without לבושׁ clothing, ורעבים נשׂאו and they take away עמר׃ the sheaf}%
\verse{בין within שׁורתם their walls, יצהירו make oil יקבים winepresses, דרכו tread ויצמאו׃ and suffer thirst.}%
\verse{מעיר from out of the city, מתים Men ינאקו groan ונפשׁ and the soul חללים of the wounded תשׁוע crieth out: ואלוה yet God לא not ישׂים layeth תפלה׃ folly}%
\verse{המה They היו are במרדי of those that rebel אור against the light; לא not הכירו they know דרכיו the ways ולא thereof, nor ישׁבו abide בנתיבתיו׃ in the paths}%
\verse{לאור with the light יקום rising רוצח The murderer יקטל killeth עני the poor ואביון and needy, ובלילה and in the night יהי is כגנב׃ as a thief.}%
\verse{ועין The eye נאף also of the adulterer שׁמרה waiteth נשׁף for the twilight, לאמר saying, לא No תשׁורני shall see עין eye וסתר me: and disguiseth פנים face. ישׂים׃ me: and disguiseth}%
\verse{חתר they dig through בחשׁך In the dark בתים houses, יומם for themselves in the daytime: חתמו they had marked למו לא not ידעו they know אור׃ the light.}%
\verse{כי For יחדו to them even as בקר the morning למו צלמות the shadow of death: כי if יכיר know בלהות the terrors צלמות׃ of the shadow of death.}%
\verse{קל swift הוא He על as פני as מים the waters; תקלל is cursed חלקתם their portion בארץ in the earth: לא not יפנה he beholdeth דרך the way כרמים׃ of the vineyards.}%
\verse{ציה Drought גם and חם heat יגזלו consume מימי waters: שׁלג the snow שׁאול the grave חטאו׃ have sinned.}%
\verse{ישׁכחהו shall forget רחם The womb מתקו shall feed sweetly רמה him; the worm עוד more לא on him; he shall be no יזכר remembered; ותשׁבר shall be broken כעץ as a tree. עולה׃ and wickedness}%
\verse{רעה He evil entreateth עקרה the barren לא not: תלד beareth ואלמנה to the widow. לא and doeth not ייטיב׃ good}%
\verse{ומשׁך He draweth אבירים also the mighty בכחו with his power: יקום he riseth up, ולא and no יאמין is sure בחיין׃ of life.}%
\verse{יתן it be given לו לבטח him in safety, וישׁען whereon he resteth; ועיניהו yet his eyes על upon דרכיהם׃ their ways.}%
\verse{רומו They are exalted מעט for a little while, ואיננו but are gone והמכו and brought low; ככל as all יקפצון they are taken out of the way וכראשׁ as the tops שׁבלת of the ears of corn. ימלו׃ and cut off}%
\verse{ואם And if לא not אפו now, מי who יכזיבני will make me a liar, וישׂם and make לאל nothing מלתי׃ my speech}%
\end{biblechapter}%
\begin{biblechapter}% Job 25
\verseWithHeading{Bildad’s Third Speech}{ויען Then answered בלדד Bildad השׁחי the Shuhite, ויאמר׃ and said,}%
\verse{המשׁל Dominion ופחד and fear עמו with עשׂה him, he maketh שׁלום peace במרומיו׃ in his high places.}%
\verse{הישׁ Is there מספר any number לגדודיו of his armies? ועל and upon מי whom לא doth not יקום arise? אורהו׃ his light}%
\verse{ומה How יצדק be justified אנושׁ then can man עם with אל God? ומה or how יזכה can he be clean ילוד born אשׁה׃ of a woman?}%
\verse{הן Behold עד even ירח to the moon, ולא not; יאהיל and it shineth וכוכבים yea, the stars לא are not pure זכו are not pure בעיניו׃ in his sight.}%
\verse{אף כינושׁ man, רמה a worm? ובן and the son אדם of man, תולעה׃ a worm?}%
\end{biblechapter}%
\begin{biblechapter}% Job 26
\verseWithHeading{Job’s Ninth Speech: A Response to Bildad}{ויען answered איוב But Job ויאמר׃ and said,}%
\verse{מה How עזרת hast thou helped ללא without כח power? הושׁעת savest זרוע thou the arm לא no עז׃ strength?}%
\verse{מה How יעצת hast thou counseled ללא no חכמה wisdom? ותושׁיה the thing לרב and hast thou plentifully הודעת׃ declared}%
\verse{את To מי whom הגדת hast thou uttered מלין words? ונשׁמת spirit מי and whose יצאה came ממך׃ from}%
\verse{הרפאים Dead יחוללו are formed מתחת from under מים the waters, ושׁכניהם׃ and the inhabitants}%
\verse{ערום naked שׁאול Hell נגדו before ואין hath no כסות covering. לאבדון׃ him, and destruction}%
\verse{נטה He stretcheth out צפון the north על over תהו the empty place, תלה hangeth ארץ the earth על upon בלי nothing. מה׃}%
\verse{צרר He bindeth up מים the waters בעביו in his thick clouds; ולא is not rent נבקע is not rent ענן and the cloud תחתם׃ under}%
\verse{מאחז He holdeth back פני the face כסה of his throne, פרשׁז spreadeth עליו upon עננו׃ his cloud}%
\verse{חק with bounds, חג עלני the waters מים the waters עד until תכלית come to an end. אור the day עם חשׁך׃ and night}%
\verse{עמודי The pillars שׁמים of heaven ירופפו tremble ויתמהו and are astonished מגערתו׃ at his reproof.}%
\verse{בכחו with his power, רגע He divideth הים the sea ובתובנתו and by his understanding מחץ he smiteth through רהב׃ the proud.}%
\verse{ברוחו By his spirit שׁמים the heavens; שׁפרה he hath garnished חללה hath formed ידו his hand נחשׁ serpent. בריח׃ the crooked}%
\verse{הן Lo, אלה these קצות parts דרכו of his ways: ומה but how שׁמץ little דבר a portion נשׁמע is heard בו ורעם of him? but the thunder גבורתו of his power מי who יתבונן׃ can understand?}%
\end{biblechapter}%
\begin{biblechapter}% Job 27
\verseWithHeading{Job Continues His Final Speech}{ויסף Moreover איוב Job שׂאת continued משׁלו his parable, ויאמר׃ and said,}%
\verse{חי liveth, אל God הסיר hath taken away משׁפטי my judgment; ושׁדי and the Almighty, המר hath vexed נפשׁי׃ my soul;}%
\verse{כי כל All עוד the while נשׁמתי my breath בי ורוח in me, and the spirit אלוה of God באפי׃ in my nostrils;}%
\verse{אם shall not תדברנה speak שׂפתי My lips עולה wickedness, ולשׁוני my tongue אם nor יהגה utter רמיה׃ deceit.}%
\verse{חלילה God forbid לי אם that אצדיק I should justify אתכם עד you: till אגוע I die לא I will not אסיר remove תמתי mine integrity ממני׃ from}%
\verse{בצדקתי My righteousness החזקתי I hold fast, ולא and will not ארפה let it go: לא shall not יחרף reproach לבבי my heart מימי׃}%
\verse{יהי be כרשׁע as the wicked, איבי Let mine enemy ומתקוממי and he that riseth up against כעול׃ me as the unrighteous.}%
\verse{כי For מה what תקות the hope חנף of the hypocrite, כי though יבצע he hath gained, כי when ישׁל taketh away אלוה God נפשׁו׃ his soul?}%
\verse{הצעקתו his cry ישׁמע hear אל Will God כי when תבוא cometh עליו upon צרה׃ trouble}%
\verse{אם על in שׁדי the Almighty? יתענג Will he delight himself יקרא call upon אלוה God? בכל will he always עת׃ will he always}%
\verse{אורה I will teach אתכם ביד you by the hand אל of God: אשׁר which עם with שׁדי the Almighty לא will I not אכחד׃ conceal.}%
\verse{הן Behold, אתם ye yourselves כלכם all חזיתם have seen ולמה why זה then are ye thus הבל altogether תהבלו׃ vain?}%
\verse{זה This חלק the portion אדם man רשׁע of a wicked עם with אל God, ונחלת and the heritage עריצים of oppressors, משׁדי יקחו׃ they shall receive}%
\verse{אם If ירבו be multiplied, בניו his children למו for חרב the sword: וצאצאיו and his offspring לא shall not ישׂבעו be satisfied לחם׃ with bread.}%
\verse{שׂרידו Those that remain במות in death: יקברו of him shall be buried ואלמנתיו and his widows לא shall not תבכינה׃ weep.}%
\verse{אם Though יצבר he heap up כעפר as the dust, כסף silver וכחמר as the clay; יכין and prepare מלבושׁ׃ raiment}%
\verse{יכין He may prepare וצדיק but the just ילבשׁ shall put on, וכסף the silver. נקי and the innocent יחלק׃ shall divide}%
\verse{בנה He buildeth כעשׁ as a moth, ביתו his house וכסכה and as a booth עשׂה maketh. נצר׃ the keeper}%
\verse{עשׁיר The rich man ישׁכב shall lie down, ולא but he shall not יאסף be gathered: עיניו his eyes, פקח he openeth ואיננו׃ and he not.}%
\verse{תשׂיגהו take hold כמים on him as waters, בלהות Terrors לילה in the night. גנבתו stealeth him away סופה׃ a tempest}%
\verse{ישׂאהו carrieth him away, קדים The east wind וילך and he departeth: וישׂערהו and as a storm hurleth ממקמו׃ him out of his place.}%
\verse{וישׁלך For shall cast עליו upon ולא him, and not יחמל spare: מידו out of his hand. ברוח he would fain flee יברח׃ he would fain flee}%
\verse{ישׂפק shall clap עלימו at כפימו their hands וישׁרק him, and shall hiss עליו him, and shall hiss ממקמו׃ him out of his place.}%
\end{biblechapter}%
\begin{biblechapter}% Job 28
\verseWithHeading{Job’s Discourse on Wisdom}{כי Surely ישׁ there is לכסף for the silver, מוצא a vein ומקום and a place לזהב for gold יזקו׃ they fine}%
\verse{ברזל Iron מעפר out of the earth, יקח is taken ואבן the stone. יצוק molten נחושׁה׃ and brass}%
\verse{קץ an end שׂם He setteth לחשׁך to darkness, ולכל all תכלית perfection: הוא and searcheth out חוקר and searcheth out אבן the stones אפל of darkness, וצלמות׃ and the shadow of death.}%
\verse{פרץ breaketh out נחל The flood מעם from גר the inhabitant; הנשׁכחים forgotten מני from רגל the foot: דלו they are dried up, מאנושׁ from men. נעו׃ they are gone away}%
\verse{ארץ the earth, ממנה out of יצא it cometh לחם bread: ותחתיה and under נהפך it is turned up כמו as it were אשׁ׃ fire.}%
\verse{מקום of it the place ספיר of sapphires: אבניה The stones ועפרת and it hath dust זהב׃ of gold.}%
\verse{נתיב a path לא which no ידעו knoweth, עיט fowl ולא hath not שׁזפתו seen: עין eye איה׃ and which the vulture's}%
\verse{לא have not הדריכהו trodden בני whelps שׁחץ The lion's לא it, nor עדה passed עליו by שׁחל׃ the fierce lion}%
\verse{בחלמישׁ upon the rock; שׁלח He putteth forth ידו his hand הפך he overturneth משׁרשׁ by the roots. הרים׃ the mountains}%
\verse{בצורות among the rocks; יארים rivers בקע He cutteth out וכל every יקר precious thing. ראתה seeth עינו׃ and his eye}%
\verse{מבכי from overflowing; נהרות the floods חבשׁ He bindeth ותעלמה and hid יצא bringeth he forth אור׃ to light.}%
\verse{והחכמה shall wisdom מאין תמצא be found? ואי and where זה and where מקום the place בינה׃ of understanding?}%
\verse{לא not ידע knoweth אנושׁ Man ערכה the price ולא thereof; neither תמצא is it found בארץ in the land החיים׃ of the living.}%
\verse{תהום The depth אמר saith, לא not בי היא It וים in me: and the sea אמר saith, אין not עמדי׃ with}%
\verse{לא It cannot יתן be gotten סגור תחתיהלא neither ישׁקל be weighed כסף shall silver מחירה׃ the price}%
\verse{לא It cannot תסלה be valued בכתם with the gold אופיר of Ophir, בשׁהם onyx, יקר with the precious וספיר׃ or the sapphire.}%
\verse{לא cannot יערכנה equal זהב The gold וזכוכית and the crystal ותמורתה it: and the exchange כלי of it jewels פז׃ of fine gold.}%
\verse{ראמות of coral, וגבישׁ or of pearls: לא No יזכר mention shall be made ומשׁך for the price חכמה of wisdom מפנינים׃ above rubies.}%
\verse{לא shall not יערכנה equal פטדת The topaz כושׁ of Ethiopia בכתם gold. טהור with pure לא it, neither תסלה׃ shall it be valued}%
\verse{והחכמה wisdom? מאין תבוא then cometh ואי and where זה and where מקום the place בינה׃ of understanding?}%
\verse{ונעלמה Seeing it is hid מעיני from the eyes כל of all חי living, ומעוף from the fowls השׁמים of the air. נסתרה׃ and kept close}%
\verse{אבדון Destruction ומות and death אמרו say, באזנינו thereof with our ears. שׁמענו We have heard שׁמעה׃ the fame}%
\verse{אלהים God הבין understandeth דרכה the way והוא thereof, and he ידע knoweth את מקומה׃ the place}%
\verse{כי For הוא he לקצות to the ends הארץ of the earth, יביט looketh תחת under כל the whole השׁמים heaven; יראה׃ seeth}%
\verse{לעשׂות To make לרוח for the winds; משׁקל the weight ומים the waters תכן and he weigheth במדה׃ by measure.}%
\verse{בעשׂתו When he made למטר for the rain, חק a decree ודרך and a way לחזיז for the lightning קלות׃ of the thunder:}%
\verse{אז Then ראה did he see ויספרה it, and declare הכינה it; he prepared וגם it, yea, חקרה׃ and searched it out.}%
\verse{ויאמר he said, לאדם And unto man הן Behold, יראת the fear אדני of the Lord, היא that חכמה wisdom; וסור and to depart מרע from evil בינה׃ understanding.}%
\end{biblechapter}%
\begin{biblechapter}% Job 29
\verseWithHeading{Job’s Final Defense}{ויסף Moreover איוב Job שׂאת continued משׁלו his parable, ויאמר׃ and said,}%
\verse{מי Oh that יתנני I were כירחי as months קדם past, כימי as the days אלוה God ישׁמרני׃ preserved}%
\verse{בהלו shined נרו When his candle עלי upon ראשׁי my head, לאורו by his light אלך I walked חשׁך׃ darkness;}%
\verse{כאשׁר As הייתי I was בימי in the days חרפי of my youth, בסוד when the secret אלוה of God עלי upon אהלי׃ my tabernacle;}%
\verse{בעוד yet שׁדי When the Almighty עמדי with סביבותי about נערי׃ me, my children}%
\verse{ברחץ When I washed הליכי my steps בחמה with butter, וצור and the rock יצוק poured me out עמדי poured me out פלגי rivers שׁמן׃ of oil;}%
\verse{בצאתי When I went out שׁער to the gate עלי through קרת the city, ברחוב in the street! אכין I prepared מושׁבי׃ my seat}%
\verse{ראוני saw נערים The young men ונחבאו me, and hid themselves: וישׁישׁים and the aged קמו arose, עמדו׃ stood up.}%
\verse{שׂרים The princes עצרו refrained במלים talking, וכף hand ישׂימו and laid לפיהם׃ on their mouth.}%
\verse{קול their peace, נגידים The nobles נחבאו held ולשׁונם and their tongue לחכם to the roof of their mouth. דבקה׃ cleaved}%
\verse{כי When אזן the ear שׁמעה heard ותאשׁרני then it blessed ועין me; and when the eye ראתה saw ותעידני׃ it gave witness}%
\verse{כי Because אמלט I delivered עני the poor משׁוע that cried, ויתום and the fatherless, ולא and none עזר׃ to help}%
\verse{ברכת The blessing אבד of him that was ready to perish עלי upon תבא came ולב heart אלמנה me: and I caused the widow's ארנן׃ to sing for joy.}%
\verse{צדק righteousness, לבשׁתי I put on וילבשׁני and it clothed כמעיל as a robe וצניף and a diadem. משׁפטי׃ me: my judgment}%
\verse{עינים eyes הייתי I was לעור to the blind, ורגלים and feet לפסח to the lame. אני׃ I}%
\verse{אב a father אנכי I לאביונים to the poor: ורב and the cause לא not ידעתי I knew אחקרהו׃ I searched out.}%
\verse{ואשׁברה And I broke מתלעות the jaws עול of the wicked, ומשׁניו out of his teeth. אשׁליך and plucked טרף׃ the spoil}%
\verse{ואמר Then I said, עם in קני my nest, אגוע I shall die וכחול as the sand. ארבה and I shall multiply ימים׃ days}%
\verse{שׁרשׁי My root פתוח spread out אלי by מים the waters, וטל and the dew ילין lay all night בקצירי׃ upon my branch.}%
\verse{כבודי My glory חדשׁ fresh עמדי in וקשׁתי me, and my bow בידי in my hand. תחליף׃ was renewed}%
\verse{לי שׁמעו Unto me gave ear, ויחלו and waited, וידמו and kept silence למו at עצתי׃ my counsel.}%
\verse{אחרי After דברי my words לא they spoke not again ישׁנו they spoke not again ועלימו upon תטף dropped מלתי׃ and my speech}%
\verse{ויחלו And they waited כמטר for me as for the rain; לי ופיהם their mouth פערו and they opened למלקושׁ׃ wide for the latter rain.}%
\verse{אשׂחק I laughed אלהם on לא not; יאמינו them, they believed ואור and the light פני of my countenance לא they cast not down. יפילון׃ they cast not down.}%
\verse{אבחר I chose out דרכם their way, ואשׁב and sat ראשׁ chief, ואשׁכון and dwelt כמלך as a king בגדוד in the army, כאשׁר as אבלים the mourners. ינחם׃ one comforteth}%
\end{biblechapter}%
\begin{biblechapter}% Job 30
\verseWithHeading{Job’s Final Defense Continued}{ועתה But now שׂחקו עליעירים younger ממני than לימים younger אשׁר whose מאסתי I would have disdained אבותם fathers לשׁית to have set עם with כלבי the dogs צאני׃ of my flock.}%
\verse{גם Yea, כח the strength ידיהם of their hands למה whereto לי עלימו me, in whom אבד was perished? כלח׃ old age}%
\verse{בחסר For want ובכפן and famine גלמוד solitary; הערקים fleeing ציה into the wilderness אמשׁ in former time שׁואה desolate ומשׁאה׃ and waste.}%
\verse{הקטפים Who cut up מלוח mallows עלי by שׂיח the bushes, ושׁרשׁ roots רתמים and juniper לחמם׃ their meat.}%
\verse{מן from גו among יגרשׁו They were driven forth יריעו (they cried עלימו after כגנב׃ them as a thief;)}%
\verse{בערוץ in the clefts נחלים of the valleys, לשׁכן To dwell חרי caves עפר of the earth, וכפים׃ and the rocks.}%
\verse{בין Among שׂיחים the bushes ינהקו they brayed; תחת under חרול the nettles יספחו׃ they were gathered together.}%
\verse{בני children נבל of fools, גם yea, בני children בלי of base men: שׁם of base men: נכאו they were viler מן than הארץ׃ the earth.}%
\verse{ועתה And now נגינתם I their song, הייתי am ואהי yea, I am להם למלה׃ their byword.}%
\verse{תעבוני They abhor רחקו me, they flee far מני from ומפני in my face. לא not חשׂכו me, and spare רק׃ to spit}%
\verse{כי Because יתרו my cord, פתח he hath loosed ויענני and afflicted ורסן the bridle מפני before שׁלחו׃ me, they have also let loose}%
\verse{על Upon ימין right פרחח the youth; יקומו rise רגלי my feet, שׁלחו they push away ויסלו and they raise up עלי against ארחות me the ways אידם׃ of their destruction.}%
\verse{נתסו They mar נתיבתי my path, להותי my calamity, יעילו they set forward לא they have no עזר׃ helper.}%
\verse{כפרץ breaking in רחב as a wide יאתיו They came תחת in שׁאה the desolation התגלגלו׃ they rolled themselves}%
\verse{ההפך are turned עלי upon בלהות Terrors תרדף me: they pursue כרוח as the wind: נדבתי my soul וכעב as a cloud. עברה passeth away ישׁעתי׃ and my welfare}%
\verse{ועתה And now עלי upon תשׁתפך is poured out נפשׁי my soul יאחזוני have taken hold upon ימי me; the days עני׃ of affliction}%
\verse{לילה me in the night season: עצמי My bones נקר are pierced מעלי in וערקי and my sinews לא take no rest. ישׁכבון׃ take no rest.}%
\verse{ברב כח force יתחפשׂ changed: לבושׁי is my garment כפי as the collar כתנתי of my coat. יאזרני׃ it bindeth me about}%
\verse{הרני He hath cast לחמר me into the mire, ואתמשׁל and I am become כעפר like dust ואפר׃ and ashes.}%
\verse{אשׁוע I cry אליך unto ולא thee, and thou dost not תענני hear עמדתי me: I stand up, ותתבנן׃ and thou regardest}%
\verse{תהפך Thou art become לאכזר cruel לי בעצם to me: with thy strong ידך hand תשׂטמני׃ thou opposest thyself against}%
\verse{תשׂאני Thou liftest me up אל to רוח the wind; תרכיבני thou causest me to ride ותמגגני and dissolvest תשׁוה׃ my substance.}%
\verse{כי For ידעתי I know מות me death, תשׁיבני thou wilt bring ובית and the house מועד appointed לכל for all חי׃ living.}%
\verse{אך Howbeit לא he will not בעי to the grave, ישׁלח stretch out יד hand אם though בפידו in his destruction. להן שׁוע׃ they cry}%
\verse{אם לא Did not בכיתי I weep לקשׁה for him that was in trouble? יום for him that was in trouble? עגמה grieved נפשׁי was my soul לאביון׃ for the poor?}%
\verse{כי When טוב good, קויתי I looked for ויבא came רע then evil ואיחלה and when I waited לאור for light, ויבא there came אפל׃ darkness.}%
\verse{מעי My bowels רתחו boiled, ולא not: דמו and rested קדמני prevented ימי the days עני׃ of affliction}%
\verse{קדר mourning הלכתי I went בלא without חמה the sun: קמתי I stood up, בקהל in the congregation. אשׁוע׃ I cried}%
\verse{אח a brother הייתי I am לתנים to dragons, ורע and a companion לבנות to owls. יענה׃ to owls.}%
\verse{עורי My skin שׁחר is black מעלי upon ועצמי me, and my bones חרה are burned מני upon חרב׃ heat.}%
\verse{ויהי also is לאבל to mourning, כנרי My harp ועגבי and my organ לקול into the voice בכים׃ of them that weep.}%
\end{biblechapter}%
\begin{biblechapter}% Job 31
\verseWithHeading{Job’s Final Defense Continued}{ברית a covenant כרתי I made לעיני with mine eyes; ומה why אתבונן then should I think על upon בתולה׃ a maid?}%
\verse{ומה For what חלק portion אלוה of God ממעל from above? ונחלת and inheritance שׁדי of the Almighty ממרמים׃ from on high?}%
\verse{הלא not איד destruction לעול to the wicked? ונכר and a strange לפעלי to the workers און׃ of iniquity?}%
\verse{הלא Doth not הוא he יראה see דרכי my ways, וכל all צעדי my steps? יספור׃ and count}%
\verse{אם If הלכתי I have walked עם with שׁוא vanity, ותחשׁ hath hasted על to מרמה deceit; רגלי׃ or if my foot}%
\verse{ישׁקלני Let me be weighed במאזני balance, צדק in an even וידע may know אלוה that God תמתי׃ mine integrity.}%
\verse{אם If תטה hath turned אשׁרי my step מני out of הדרך the way, ואחר after עיני mine eyes, הלך walked לבי and mine heart ובכפי to mine hands; דבק hath cleaved מאום׃ and if any blot}%
\verse{אזרעה let me sow, ואחר and let another יאכל eat; וצאצאי yea, let my offspring ישׁרשׁו׃ be rooted out.}%
\verse{אם If נפתה have been deceived לבי mine heart על by אשׁה a woman, ועל at פתח door; רעי my neighbor's ארבתי׃ or I have laid wait}%
\verse{תטחן grind לאחר unto another, אשׁתי let my wife ועליה upon יכרעון bow down אחרין׃ and let others}%
\verse{כי For הוא this זמה a heinous crime; והיא yea, it עון an iniquity פלילים׃ the judges.}%
\verse{כי For אשׁ a fire היא it עד to אבדון destruction, תאכל consumeth ובכל all תבואתי mine increase. תשׁרשׁ׃ and would root out}%
\verse{אם If אמאס I did despise משׁפט the cause עבדי of my manservant ואמתי or of my maidservant, ברבם when they contended עמדי׃ with}%
\verse{ומה What אעשׂה then shall I do כי when יקום riseth up? אל God וכי and when יפקד he visiteth, מה what אשׁיבנו׃ shall I answer}%
\verse{הלא Did not בבטן me in the womb עשׂני he that made עשׂהו make ויכננו fashion ברחם us in the womb? אחד׃ him? and did not one}%
\verse{אם If אמנע I have withheld מחפץ from desire, דלים the poor ועיני or have caused the eyes אלמנה of the widow אכלה׃ to fail;}%
\verse{ואכל Or have eaten פתי my morsel לבדי myself alone, ולא hath not אכל eaten יתום and the fatherless ממנה׃ thereof;}%
\verse{כי (For מנעורי from my youth גדלני he was brought up כאב with me, as a father, ומבטן אמי her from my mother's אנחנה׃ and I have guided}%
\verse{אם If אראה I have seen אובד any perish מבלי for want לבושׁ of clothing, ואין without כסות covering; לאביון׃ or any poor}%
\verse{אם If לא have not ברכוני blessed חלצו his loins ומגז with the fleece כבשׂי of my sheep; יתחמם׃ me, and he were warmed}%
\verse{אם If הניפותי I have lifted up על against יתום the fatherless, ידי my hand כי when אראה I saw בשׁער in the gate: עזרתי׃ my help}%
\verse{כתפי let mine arm משׁכמה from my shoulder blade, תפול fall ואזרעי and mine arm מקנה from the bone. תשׁבר׃ be broken}%
\verse{כי For פחד a terror אלי to איד destruction אל God ומשׂאתו me, and by reason of his highness לא not אוכל׃ I could}%
\verse{אם If שׂמתי I have made זהב gold כסלי my hope, ולכתם to the fine gold, אמרתי or have said מבטחי׃ my confidence;}%
\verse{אם If אשׂמח I rejoiced כי because רב great, חילי my wealth וכי and because כביר much; מצאה had gotten ידי׃ mine hand}%
\verse{אם If אראה I beheld אור the sun כי when יהל it shined, וירח or the moon יקר brightness; הלך׃ walking}%
\verse{ויפת enticed, בסתר hath been secretly לבי And my heart ותשׁק hath kissed ידי my hand: לפי׃ or my mouth}%
\verse{גם also הוא This עון an iniquity פלילי the judge: כי for כחשׁתי I should have denied לאל the God ממעל׃ above.}%
\verse{אם If אשׂמח I rejoiced בפיד at the destruction משׂנאי of him that hated והתעררתי me, or lifted up myself כי when מצאו found רע׃ evil}%
\verse{ולא Neither נתתי have I suffered לחטא to sin חכי my mouth לשׁאל by wishing באלה a curse נפשׁו׃ to his soul.}%
\verse{אם If לא not, אמרו said מתי the men אהלי of my tabernacle מי Oh that יתן we had מבשׂרו לא we cannot נשׂבע׃ be satisfied.}%
\verse{בחוץ in the street: לא did not ילין lodge גר The stranger דלתי my doors לארח to the traveler. אפתח׃ I opened}%
\verse{אם If כסיתי I covered כאדם as Adam, פשׁעי my transgressions לטמון by hiding בחבי in my bosom: עוני׃ mine iniquity}%
\verse{כי אערוץ Did I fear המון multitude, רבה a great ובוז or did the contempt משׁפחות of families יחתני terrify ואדם me, that I kept silence, לא went not out אצא went not out פתח׃ of the door?}%
\verse{מי Oh that יתן one would לי שׁמע hear לי הן me! behold, תוי my desire שׁדי the Almighty יענני would answer וספר a book. כתב had written אישׁ me, and mine adversary ריבי׃ me, and mine adversary}%
\verse{אם Surely לא על it upon שׁכמי my shoulder, אשׂאנו I would take אענדנו bind עטרות׃ it a crown}%
\verse{מספר unto him the number צעדי of my steps; אגידנו I would declare כמו as נגיד a prince אקרבנו׃ would I go near unto}%
\verse{אם If עלי against אדמתי my land תזעק cry ויחד likewise תלמיה me, or that the furrows יבכיון׃ thereof complain;}%
\verse{אם If כחה the fruits אכלתי I have eaten בלי thereof without כסף money, ונפשׁ their life: בעליה or have caused the owners הפחתי׃ thereof to lose}%
\verse{תחת instead חטה of wheat, יצא grow חוח Let thistles ותחת instead שׂערה of barley. באשׁה and cockle תמו are ended. דברי The words איוב׃ of Job}%
\end{biblechapter}%
\begin{biblechapter}% Job 32
\verseWithHeading{Elihu Rebukes Job and His Three Friends}{וישׁבתו ceased שׁלשׁת three האנשׁים האלה So these מענות to answer את איוב Job, כי because הוא he צדיק righteous בעיניו׃ in his own eyes.}%
\verse{ויחר Then was kindled אף the wrath אליהוא of Elihu בן the son ברכאל of Barachel הבוזי the Buzite, ממשׁפחת of the kindred רם of Ram: באיוב against Job חרה kindled, אפו was his wrath על because צדקו he justified נפשׁו himself מאלהים׃}%
\verse{ובשׁלשׁת Also against his three רעיו friends חרה kindled, אפו was his wrath על because אשׁר because לא no מצאו they had found מענה answer, וירשׁיעו and had condemned את איוב׃ Job.}%
\verse{ואליהו Now Elihu חכה had waited את איוב till Job בדברים had spoken, כי because זקנים elder המה they ממנו than לימים׃ elder}%
\verse{וירא saw אליהוא When Elihu כי that אין no מענה answer בפי in the mouth שׁלשׁת of three האנשׁים men, ויחר was kindled. אפו׃ then his wrath}%
\verse{ויען answered אליהוא And Elihu בן the son ברכאל of Barachel הבוזי the Buzite ויאמר and said, צעיר young, אני I לימים very old; ואתם and ye ישׁישׁים very old; על wherefore כן wherefore זחלתי I was afraid, ואירא and durst מחות not show דעי you mine opinion. אתכם׃}%
\verse{אמרתי I said, ימים Days ידברו should speak, ורב and multitude שׁנים of years ידיעו should teach חכמה׃ wisdom.}%
\verse{אכן But רוח a spirit היא באנושׁ in man: ונשׁמת and the inspiration שׁדי of the Almighty תבינם׃ giveth them understanding.}%
\verse{לא men are not רבים Great יחכמו wise: וזקנים neither do the aged יבינו understand משׁפט׃ judgment.}%
\verse{לכן Therefore אמרתי I said, שׁמעה Hearken לי אחוה will show דעי mine opinion. אף also אני׃ to me; I}%
\verse{הן Behold, הוחלתי I waited לדבריכם for your words; אזין I gave ear עד to תבונתיכם your reasons, עד whilst תחקרון ye searched out מלין׃ what to say.}%
\verse{ועדיכם unto אתבונן Yea, I attended והנה you, and, behold, אין none לאיוב Job, מוכיח you that convinced עונה that answered אמריו his words: מכם׃ of}%
\verse{פן Lest תאמרו ye should say, מצאנו We have found out חכמה wisdom: אל God ידפנו thrusteth him down, לא not אישׁ׃ man.}%
\verse{ולא Now he hath not ערך directed אלי against מלין words ובאמריכם him with your speeches. לא me: neither אשׁיבנו׃ will I answer}%
\verse{חתו They were amazed, לא no ענו they answered עוד more: העתיקו they left off מהם מלים׃ speaking.}%
\verse{והוחלתי When I had waited, כי (for לא not, ידברו they spoke כי but עמדו stood still, לא no ענו answered עוד׃ more;)}%
\verse{אענה will answer אף also אני I חלקי my part, אחוה will show דעי mine opinion. אף also אני׃ I}%
\verse{כי For מלתי I am full מלים of matter, הציקתני me constraineth רוח the spirit בטני׃ within}%
\verse{הנה Behold, בטני my belly כיין as wine לא hath no יפתח vent; כאבות bottles. חדשׁים like new יבקע׃ it is ready to burst}%
\verse{אדברה I will speak, וירוח that I may be refreshed: לי אפתח I will open שׂפתי my lips ואענה׃ and answer.}%
\verse{אל Let me not, נא I pray you, אשׂא accept פני person, אישׁ any man's ואל unto אדם man. לא neither אכנה׃ let me give flattering titles}%
\verse{כי For לא not ידעתי I know אכנה to give flattering titles; כמעט would soon ישׂאני take me away. עשׂני׃ my maker}%
\end{biblechapter}%
\begin{biblechapter}% Job 33
\verseWithHeading{Elihu Rebukes Job}{ואולם Wherefore, שׁמע hear נא I pray thee, איוב Job, מלי my speeches, וכל to all דברי my words. האזינה׃ and hearken}%
\verse{הנה Behold, נא now פתחתי I have opened פי my mouth, דברה hath spoken לשׁוני my tongue בחכי׃ in my mouth.}%
\verse{ישׁר the uprightness לבי of my heart: אמרי My words ודעת knowledge שׂפתי and my lips ברור clearly. מללו׃ shall utter}%
\verse{רוח The Spirit אל of God עשׂתני hath made ונשׁמת me, and the breath שׁדי of the Almighty תחיני׃ hath given me life.}%
\verse{אם If תוכל thou canst השׁיבני answer ערכה me, set n order לפני before התיצבה׃ me, stand up.}%
\verse{הן Behold, אני I כפיך according to thy wish לאל in God's מחמר out of the clay. קרצתי am formed גם also אני׃ stead: I}%
\verse{הנה Behold, אמתי my terror לא shall not תבעתך make thee afraid, ואכפי shall my hand עליך upon לא neither יכבד׃ be heavy}%
\verse{אך Surely אמרת thou hast spoken באזני in mine hearing, וקול the voice מלין of words, אשׁמע׃ and I have heard}%
\verse{זך am clean אני I בלי without פשׁע transgression, חף innocent; אנכי I ולא neither עון׃ iniquity}%
\verse{הן Behold, תנואות occasions עלי against ימצא he findeth יחשׁבני me, he counteth לאויב׃ me for his enemy,}%
\verse{ישׂם He putteth בסד in the stocks, רגלי my feet ישׁמר he marketh כל all ארחתי׃ my paths.}%
\verse{הן Behold, זאת this לא thou art not צדקת just: אענך I will answer כי thee, that ירבה is greater אלוה God מאנושׁ׃ than man.}%
\verse{מדוע Why אליו against ריבות dost thou strive כי him? for כל of any דבריו of his matters. לא he giveth not יענה׃ account}%
\verse{כי For באחת once, ידבר speaketh אל God ובשׁתים yea twice, לא it not. ישׁורנה׃ perceiveth}%
\verse{בחלום In a dream, חזיון in a vision לילה of the night, בנפל falleth תרדמה when deep sleep על upon אנשׁים men, בתנומות in slumberings עלי upon משׁכב׃ the bed;}%
\verse{אז Then יגלה he openeth אזן the ears אנשׁים ובמסרם their instruction, יחתם׃ and sealeth}%
\verse{להסיר That he may withdraw אדם man מעשׂה purpose, וגוה pride מגבר from man. יכסה׃ and hide}%
\verse{יחשׂך He keepeth back נפשׁו his soul מני from שׁחת the pit, וחיתו and his life מעבר from perishing בשׁלח׃ by the sword.}%
\verse{והוכח He is chastened במכאוב also with pain על upon משׁכבו his bed, וריב עצמיו of his bones אתן׃ with strong}%
\verse{וזהמתו abhorreth חיתו So that his life לחם bread, ונפשׁו and his soul מאכל meat. תאוה׃ dainty}%
\verse{יכל is consumed away, בשׂרו His flesh מראי that it cannot be seen; ושׁפי עצמותיו and his bones לא were not ראו׃ seen}%
\verse{ותקרב draweth near לשׁחת unto the grave, נפשׁו Yea, his soul וחיתו and his life לממתים׃ to the destroyers.}%
\verse{אם If ישׁ there be עליו with מלאך a messenger מליץ him, an interpreter, אחד one מני among אלף a thousand, להגיד to show לאדם unto man ישׁרו׃ his uprightness:}%
\verse{ויחננו Then he is gracious ויאמר unto him, and saith, פדעהו Deliver מרדת him from going down שׁחת to the pit: מצאתי I have found כפר׃ a ransom.}%
\verse{רטפשׁ shall be fresher בשׂרו His flesh מנער ישׁוב he shall return לימי to the days עלומיו׃ of his youth:}%
\verse{יעתר He shall pray אל unto אלוה God, וירצהו and he will be favorable וירא unto him: and he shall see פניו his face בתרועה with joy: וישׁב for he will render לאנושׁ unto man צדקתו׃ his righteousness.}%
\verse{ישׁר He looketh על upon אנשׁים ויאמר and say, חטאתי I have sinned, וישׁר right, העויתי and perverted ולא me not; שׁוה׃ and it profited}%
\verse{פדה He will deliver נפשׁי his soul מעבר from going בשׁחת into the pit, וחיתי and his life באור the light. תראה׃ shall see}%
\verse{הן Lo, כל all אלה these יפעל worketh אל God פעמים oftentimes שׁלושׁ oftentimes עם with גבר׃ man,}%
\verse{להשׁיב To bring back נפשׁו his soul מני from שׁחת the pit, לאור to be enlightened באור with the light החיים׃ of the living.}%
\verse{הקשׁב Mark well, איוב O Job, שׁמע hearken לי החרשׁ unto me: hold thy peace, ואנכי and I אדבר׃ will speak.}%
\verse{אם If ישׁ thou hast מלין any thing to say, השׁיבני answer דבר me: speak, כי for חפצתי I desire צדקך׃ to justify}%
\verse{אם If אין not, אתה שׁמע hearken לי החרשׁ unto me: hold thy peace, ואאלפך and I shall teach חכמה׃ thee wisdom.}%
\end{biblechapter}%
\begin{biblechapter}% Job 34
\verseWithHeading{Elihu Asserts God’s Justice}{ויען answered אליהוא Furthermore Elihu ויאמר׃ and said,}%
\verse{שׁמעו Hear חכמים O ye wise מלי my words, וידעים unto me, ye that have knowledge. האזינו׃ and give ear}%
\verse{כי For אזן the ear מלין words, תבחן trieth וחך as the mouth יטעם tasteth לאכל׃ meat.}%
\verse{משׁפט to us judgment: נבחרה Let us choose לנו נדעה let us know בינינו among מה ourselves what טוב׃ good.}%
\verse{כי For אמר hath said, איוב Job צדקתי I am righteous: ואל and God הסיר hath taken away משׁפטי׃ my judgment.}%
\verse{על against משׁפטי my right? אכזב Should I lie אנושׁ incurable חצי my wound בלי without פשׁע׃ transgression.}%
\verse{מי What גבר man כאיוב like Job, ישׁתה drinketh up לעג scorning כמים׃ like water?}%
\verse{וארח Which goeth לחברה in company עם with פעלי the workers און of iniquity, וללכת and walketh עם with אנשׁי men. רשׁע׃ wicked}%
\verse{כי For אמר he hath said, לא nothing יסכן It profiteth גבר a man ברצתו that he should delight עם himself with אלהים׃ God.}%
\verse{לכן Therefore אנשׁי unto me, ye men לבב of understanding: שׁמעו hearken לי חללה far be it לאל from God, מרשׁע wickedness; ושׁדי and the Almighty, מעול׃ iniquity.}%
\verse{כי For פעל the work אדם of a man ישׁלם shall he render לו וכארח according to ways. אישׁ unto him, and cause every man ימצאנו׃ to find}%
\verse{אף Yea, אמנם surely אל God לא will not ירשׁיע do wickedly, ושׁדי will the Almighty לא neither יעות pervert משׁפט׃ judgment.}%
\verse{מי Who פקד hath given him a charge עליו hath given him a charge ארצה over the earth? ומי or who שׂם hath disposed תבל world? כלה׃ the whole}%
\verse{אם If ישׂים he set אליו upon לבו his heart רוחו himself his spirit ונשׁמתו and his breath; אליו unto יאסף׃ man, he gather}%
\verse{יגוע shall perish כל All בשׂר flesh יחד together, ואדם and man על unto עפר dust. ישׁוב׃ shall turn again}%
\verse{ואם If בינה now understanding, שׁמעה hear זאת this: האזינה hearken לקול to the voice מלי׃ of my words.}%
\verse{האף Shall even שׂונא he that hateth משׁפט right יחבושׁ govern? ואם צדיק just? כביר him that is most תרשׁיע׃ and wilt thou condemn}%
\verse{האמר to say למלך to a king, בליעל wicked? רשׁע ungodly? אל to נדיבים׃ princes,}%
\verse{אשׁר that לא not נשׂא accepteth פני the persons שׂרים of princes, ולא nor נכר regardeth שׁוע the rich לפני more than דל the poor? כי for מעשׂה the work ידיו of his hands. כלם׃ they all}%
\verse{רגע In a moment ימתו shall they die, וחצות at midnight, לילה at midnight, יגעשׁו shall be troubled עם and the people ויעברו and pass away: ויסירו shall be taken away אביר and the mighty לא without ביד׃ hand.}%
\verse{כי For עיניו his eyes על upon דרכי the ways אישׁ of man, וכל all צעדיו his goings. יראה׃ and he seeth}%
\verse{אין no חשׁך darkness, ואין nor צלמות shadow of death, להסתר may hide themselves. שׁם where פעלי the workers און׃ of iniquity}%
\verse{כי For לא he will not על upon אישׁ man ישׂים lay עוד more להלך that he should enter אל with אל God. במשׁפט׃ into judgment}%
\verse{ירע He shall break in pieces כבירים mighty men לא without חקר number, ויעמד and set אחרים others תחתם׃ in their stead.}%
\verse{לכן Therefore יכיר he knoweth מעבדיהם their works, והפך and he overturneth לילה in the night, וידכאו׃ so that they are destroyed.}%
\verse{תחת them as רשׁעים wicked men ספקם He striketh במקום in the open ראים׃ sight}%
\verse{אשׁר עלן סרו they turned back מאחריו from וכל any דרכיו of his ways: לא him, and would not השׂכילו׃ consider}%
\verse{להביא to come עליו unto צעקת So that they cause the cry דל of the poor וצעקת the cry עניים of the afflicted. ישׁמע׃ him, and he heareth}%
\verse{והוא When he ישׁקט giveth quietness, ומי who ירשׁע then can make trouble? ויסתר and when he hideth פנים face, ומי who ישׁורנו then can behold ועל him? whether against גוי a nation, ועל or against אדם a man יחד׃ only:}%
\verse{ממלך reign not, אדם חנףמקשׁי be ensnared. עם׃ lest the people}%
\verse{כי Surely it is meet אל unto אל God, האמר to be said נשׂאתי I have borne לא I will not אחבל׃ offend}%
\verse{בלעדי not אחזה I see אתה thou הרני teach אם me: if עול iniquity, פעלתי I have done לא no אסיף׃ I will do}%
\verse{המעמך ישׁלמנה he will recompense כי it, whether מאסת thou refuse, כי or whether אתה thou תבחר choose; ולא and not אני I: ומה what ידעת thou knowest. דבר׃ therefore speak}%
\verse{אנשׁי לבב of understanding יאמרו tell לי וגבר man חכם me, and let a wise שׁמע׃ hearken}%
\verse{איוב Job לא without בדעת knowledge, ידבר hath spoken ודבריו and his words לא without בהשׂכיל׃ wisdom.}%
\verse{אבי My desire יבחן may be tried איוב Job עד unto נצח the end על because תשׁבת of answers באנשׁי men. און׃ for wicked}%
\verse{כי For יסיף he addeth על unto חטאתו his sin, פשׁע rebellion בינינו among יספוק he clappeth וירב us, and multiplieth אמריו his words לאל׃ against God.}%
\end{biblechapter}%
\begin{biblechapter}% Job 35
\verseWithHeading{Elihu Condemns Job}{ויען spoke אליהו Elihu ויאמר׃ moreover, and said,}%
\verse{הזאת thou this חשׁבת Thinkest למשׁפט to be right, אמרת thou saidst, צדקי My righteousness מאל׃}%
\verse{כי For תאמר thou saidst, מה What יסכן advantage לך מה will it be unto thee? What אעיל profit מחטאתי׃}%
\verse{אני I אשׁיבך will answer מלין will answer ואת רעיך thee, and thy companions עמך׃ with}%
\verse{הבט Look שׁמים unto the heavens, וראה and see; ושׁור and behold שׁחקים the clouds גבהו are higher ממך׃ than}%
\verse{אם If חטאת thou sinnest, מה what תפעל doest בו ורבו be multiplied, פשׁעיך thou against him? or thy transgressions מה what תעשׂה׃ doest}%
\verse{אם If צדקת thou be righteous, מה what תתן givest לו או thou him? or מה what מידך he of thine hand? יקח׃ receiveth}%
\verse{לאישׁ a man כמוך as thou רשׁעך Thy wickedness ולבן the son אדם of man. צדקתך׃ and thy righteousness}%
\verse{מרב עשׁוקים of oppressions יזעיקו they make to cry: ישׁועו they cry out מזרוע by reason of the arm רבים׃ of the mighty.}%
\verse{ולא But none אמר saith, איה אלוה God עשׂי my maker, נתן who giveth זמרות songs בלילה׃ in the night;}%
\verse{מלפנו Who teacheth מבהמות us more than the beasts ארץ of the earth, ומעוף than the fowls השׁמים of heaven? יחכמנו׃ and maketh us wiser}%
\verse{שׁם There יצעקו they cry, ולא but none יענה giveth answer, מפני because גאון of the pride רעים׃ of evil men.}%
\verse{אך Surely שׁוא vanity, לא will not ישׁמע hear אל God ושׁדי will the Almighty לא neither ישׁורנה׃ regard}%
\verse{אף כיאמר thou sayest לא thou shalt not תשׁורנו see דין him, judgment לפניו before ותחולל׃ him; therefore trust}%
\verse{ועתה But now, כי because אין not פקד he hath visited אפו in his anger; ולא not ידע yet he knoweth בפשׁ extremity: מאד׃ in great}%
\verse{ואיוב Therefore doth Job הבל in vain; יפצה open פיהו his mouth בבלי without דעת knowledge. מלין words יכבר׃ he multiplieth}%
\end{biblechapter}%
\begin{biblechapter}% Job 36
\verseWithHeading{Elihu Extols God’s Greatness}{ויסף also proceeded, אליהוא Elihu ויאמר׃ and said,}%
\verse{כתר Suffer לי זעיר me a little, ואחוך and I will show כי thee that עוד yet לאלוה on God's מלים׃ to speak}%
\verse{אשׂא I will fetch דעי my knowledge למרחוק from afar, ולפעלי to my Maker. אתן and will ascribe צדק׃ righteousness}%
\verse{כי For אמנם truly לא not שׁקר false: מלי my words תמים he that is perfect דעות in knowledge עמך׃ with}%
\verse{הן Behold, אל God כביר mighty, ולא not ימאס and despiseth כביר mighty כח in strength לב׃ wisdom.}%
\verse{לא יחיהשׁע of the wicked: ומשׁפט right עניים to the poor. יתן׃ but giveth}%
\verse{לא not יגרע He withdraweth מצדיק from the righteous: עיניו his eyes ואת but with מלכים kings לכסא on the throne; וישׁיבם yea, he doth establish לנצח them forever, ויגבהו׃ and they are exalted.}%
\verse{ואם And if אסורים bound בזקים in fetters, ילכדון be holden בחבלי in cords עני׃ of affliction;}%
\verse{ויגד Then he showeth להם פעלם them their work, ופשׁעיהם and their transgressions כי that יתגברו׃ they have exceeded.}%
\verse{ויגל He openeth אזנם also their ear למוסר to discipline, ויאמר and commandeth כי that ישׁבון they return מאון׃ from iniquity.}%
\verse{אם If ישׁמעו they obey ויעבדו and serve יכלו they shall spend ימיהם their days בטוב in prosperity, ושׁניהם and their years בנעימים׃ in pleasures.}%
\verse{ואם But if לא not, ישׁמעו they obey בשׁלח by the sword, יעברו they shall perish ויגועו and they shall die בבלי without דעת׃ knowledge.}%
\verse{וחנפי But the hypocrites לב in heart ישׂימו heap up אף wrath: לא not ישׁועו they cry כי when אסרם׃ he bindeth}%
\verse{תמת die בנער in youth, נפשׁם They וחיתם and their life בקדשׁים׃ among the unclean.}%
\verse{יחלץ He delivereth עני the poor בעניו in his affliction, ויגל and openeth בלחץ in oppression. אזנם׃ their ears}%
\verse{ואף Even so הסיתך would he have removed מפי צרחב a broad place, לא no מוצק straitness; תחתיה where ונחת and that which should be set שׁלחנך on thy table מלא full דשׁן׃ of fatness.}%
\verse{ודין the judgment רשׁע of the wicked: מלאת But thou hast fulfilled דין judgment ומשׁפט and justice יתמכו׃ take hold}%
\verse{כי Because חמה wrath, פן lest יסיתך he take thee away בספק with stroke: ורב then a great כפר ransom אל cannot יטך׃ deliver}%
\verse{היערך Will he esteem שׁועך thy riches? לא not בצר gold, וכל nor all מאמצי the forces כח׃ of strength.}%
\verse{אל not תשׁאף Desire הלילה the night, לעלות are cut off עמים when people תחתם׃ in their place.}%
\verse{השׁמר Take heed, אל תפן regard not אל און iniquity: כי for על for זה this בחרת hast thou chosen מעני׃ rather than affliction.}%
\verse{הן Behold, אל God ישׂגיב exalteth בכחו by his power: מי who כמהו like him? מורה׃ teacheth}%
\verse{מי Who פקד hath enjoined עליו hath enjoined דרכו him his way? ומי or who אמר can say, פעלת Thou hast wrought עולה׃ iniquity?}%
\verse{זכר Remember כי that תשׂגיא thou magnify פעלו his work, אשׁר which שׁררו אנשׁים׃ men}%
\verse{כל Every אדם man חזו may see בו אנושׁ it; man יביט may behold מרחוק׃ afar off.}%
\verse{הן Behold, אל God שׂגיא great, ולא not נדע and we know מספר can the number שׁניו of his years ולא neither חקר׃ be searched out.}%
\verse{כי For יגרע he maketh small נטפי the drops מים of water: יזקו they pour down מטר rain לאדו׃ according to the vapor}%
\verse{אשׁר Which יזלו do drop שׁחקים the clouds ירעפו distill עלי upon אדם man רב׃ abundantly.}%
\verse{אף אםבין can understand מפרשׂי the spreadings עב of the clouds, תשׁאות the noise סכתו׃ of his tabernacle?}%
\verse{הן Behold, פרשׂ he spreadeth עליו upon אורו his light ושׁרשׁי the bottom הים of the sea. כסה׃ it, and covereth}%
\verse{כי For בם ידין by them judgeth עמים he the people; יתן he giveth אכל meat למכביר׃ in abundance.}%
\verse{על With כפים clouds כסה he covereth אור the light; ויצו and commandeth עליה it by במפגיע׃ that cometh between.}%
\verse{יגיד thereof showeth עליו concerning רעו The noise מקנה it, the cattle אף also על concerning עולה׃ the vapor.}%
\end{biblechapter}%
\begin{biblechapter}% Job 37
\verseWithHeading{Elihu Extols God’s Majesty}{אף also לזאת At this יחרד trembleth, לבי my heart ויתר and is moved ממקומו׃ out of his place.}%
\verse{שׁמעו שׁמוערגז the noise קלו of his voice, והגה and the sound מפיו of his mouth. יצא׃ goeth out}%
\verse{תחת it under כל the whole השׁמים heaven, ישׁרהו He directeth ואורו and his lightning על unto כנפות the ends הארץ׃ of the earth.}%
\verse{אחריו After ישׁאג roareth: קול it a voice ירעם he thundereth בקול with the voice גאונו of his excellency; ולא and he will not יעקבם stay כי them when ישׁמע is heard. קולו׃ his voice}%
\verse{ירעם thundereth אל God בקולו with his voice; נפלאות marvelously עשׂה doeth גדלות great things ולא he, which we cannot נדע׃ comprehend.}%
\verse{כי For לשׁלג to the snow, יאמר he saith הוא Be ארץ thou the earth; וגשׁם rain, מטר likewise to the small וגשׁם rain מטרות and to the great עזו׃ of his strength.}%
\verse{ביד the hand כל of every אדם man; יחתום He sealeth up לדעת may know כל that all אנשׁי men מעשׂהו׃ his work.}%
\verse{ותבא go חיה Then the beasts במו into ארב dens, ובמעונתיה in their places. תשׁכן׃ and remain}%
\verse{מן Out of החדר the south תבוא cometh סופה the whirlwind: וממזרים out of the north. קרה׃ and cold}%
\verse{מנשׁמת אל of God יתן is given: קרח frost ורחב and the breadth מים of the waters במוצק׃ is straitened.}%
\verse{אף Also ברי by watering יטריח he wearieth עב the thick cloud: יפיץ he scattereth ענן cloud: אורו׃ his bright}%
\verse{והוא And it מסבות round about מתהפך is turned בתחבולתו by his counsels: לפעלם כל whatsoever אשׁר whatsoever יצום he commandeth על them upon פני the face תבל of the world ארצה׃ in the earth.}%
\verse{אם whether לשׁבט for correction, אם or לארצו for his land, אם or לחסד for mercy. ימצאהו׃ He causeth it to come,}%
\verse{האזינה Hearken זאת unto this, איוב O Job: עמד stand still, והתבונן and consider נפלאות the wondrous אל׃ works of God.}%
\verse{התדע Dost thou know בשׂום disposed אלוה when God עליהם disposed והופיע to shine? אור them, and caused the light עננו׃ of his cloud}%
\verse{התדע עלפלשׂי the balancings עב of the clouds, מפלאות the wondrous works תמים of him which is perfect דעים׃ in knowledge?}%
\verse{אשׁר How בגדיך thy garments חמים warm, בהשׁקט when he quieteth ארץ the earth מדרום׃ by the south}%
\verse{תרקיע him spread out עמו Hast thou with לשׁחקים the sky, חזקים strong, כראי looking glass? מוצק׃ as a molten}%
\verse{הודיענו Teach מה us what נאמר we shall say לו לא unto him; we cannot נערך order מפני by reason חשׁך׃ of darkness.}%
\verse{היספר Shall it be told לו כי him that אדבר I speak? אם if אמר speak, אישׁ a man כי surely יבלע׃ he shall be swallowed up.}%
\verse{ועתה And now לא not ראו see אור light בהיר the bright הוא which בשׁחקים in the clouds: ורוח but the wind עברה passeth, ותטהרם׃ and cleanseth}%
\verse{מצפון out of the north: זהב Fair weather יאתה cometh על with אלוה God נורא terrible הוד׃ majesty.}%
\verse{שׁדי the Almighty, לא we cannot מצאנהו find him out: שׂגיא excellent כח in power, ומשׁפט and in judgment, ורב and in plenty צדקה of justice: לא he will not יענה׃ afflict.}%
\verse{לכן do therefore יראוהו fear אנשׁים לא not יראה him: he respecteth כל any חכמי wise לב׃ of heart.}%
\end{biblechapter}%
\begin{biblechapter}% Job 38
\verseWithHeading{Adonai Challenges Job}{ויען answered יהוה Then the LORD את איוב Job מנהסערה the whirlwind, ויאמר׃ and said,}%
\verse{מי Who זה this מחשׁיך that darkeneth עצה counsel במלין by words בלי without דעת׃ knowledge?}%
\verse{אזר Gird up נא now כגבר like a man; חלציך thy loins ואשׁאלך for I will demand והודיעני׃ of thee, and answer}%
\verseWithHeading{Adonai Interrogates Job}{איפה Where היית wast ביסדי thou when I laid the foundations ארץ of the earth? הגד declare, אם if ידעת thou hast understanding. בינה׃ thou hast understanding.}%
\verse{מי Who שׂם hath laid ממדיה the measures כי thereof, if תדע thou knowest? או or מי who נטה hath stretched עליה upon קו׃ the line}%
\verse{על מהדניה are the foundations הטבעו thereof fastened? או or מי who ירה laid אבן stone פנתה׃ the corner}%
\verse{ברן sang יחד together, כוכבי stars בקר When the morning ויריעו shouted for joy? כל and all בני the sons אלהים׃ of God}%
\verse{ויסך Or shut up בדלתים with doors, ים the sea בגיחו when it broke forth, מרחם of the womb? יצא׃ it had issued out}%
\verse{בשׂומי When I made ענן the cloud לבשׁו the garment וערפל thereof, and thick darkness חתלתו׃ a swaddling band}%
\verse{ואשׁבר And broke up עליו for חקי it my decreed ואשׂים and set בריח bars ודלתים׃ and doors,}%
\verse{ואמר And said, עד פה and here תבוא shalt thou come, ולא but no תסיף further: ופא ישׁית be stayed? בגאון shall thy proud גליך׃ waves}%
\verse{המימיך since thy days; צוית Hast thou commanded בקר the morning ידעתה to know שׁחר מקמו׃ his place;}%
\verse{לאחז That it might take hold בכנפות of the ends הארץ of the earth, וינערו might be shaken רשׁעים that the wicked ממנה׃ out of}%
\verse{תתהפך It is turned כחמר as clay חותם the seal; ויתיצבו and they stand כמו as לבושׁ׃ a garment.}%
\verse{וימנע is withheld, מרשׁעים אורם their light וזרוע arm רמה and the high תשׁבר׃ shall be broken.}%
\verse{הבאת Hast thou entered עד into נבכי the springs ים of the sea? ובחקר in the search תהום of the depth? התהלכת׃ or hast thou walked}%
\verse{הנגלו been opened לך שׁערי Have the gates מות of death ושׁערי the doors צלמות of the shadow of death? תראה׃ unto thee? or hast thou seen}%
\verse{התבננת עדחבי the breadth ארץ of the earth? הגד declare אם if ידעת thou knowest כלה׃ it all.}%
\verse{אי where זה where הדרך the way ישׁכן dwelleth? אור light וחשׁך and darkness, אי זהקמו׃ the place}%
\verse{כי That תקחנו thou shouldest take אל it to גבולו the bound וכי thereof, and that תבין thou shouldest know נתיבות the paths ביתו׃ the house}%
\verse{ידעת Knowest כי thou because אז thou wast then תולד born? ומספר or the number ימיך of thy days רבים׃ great?}%
\verse{הבאת Hast thou entered אל into אצרות the treasures שׁלג of the snow? ואצרות the treasures ברד of the hail, תראה׃ or hast thou seen}%
\verse{אשׁר Which חשׂכתי I have reserved לעת against the time צר of trouble, ליום against the day קרב of battle ומלחמה׃ and war?}%
\verse{אי זהדרך way יחלק parted, אור is the light יפץ scattereth קדים the east wind עלי upon ארץ׃ the earth?}%
\verse{מי Who פלג hath divided לשׁטף for the overflowing of waters, תעלה a watercourse ודרך or a way לחזיז for the lightning קלות׃ of thunder;}%
\verse{להמטיר To cause it to rain על on ארץ the earth, לא no אישׁ man מדבר the wilderness, לא wherein no אדם׃ man;}%
\verse{להשׂביע To satisfy שׁאה the desolate ומשׁאה and waste ולהצמיח to spring forth? מצא and to cause the bud דשׁא׃ of the tender herb}%
\verse{הישׁ Hath למטר the rain אב a father? או or מי who הוליד hath begotten אגלי the drops טל׃ of dew?}%
\verse{מבטן מי who יצא came הקרח the ice? וכפר and the hoary frost שׁמים of heaven, מי ילדו׃ hath engendered}%
\verse{כאבן as a stone, מים The waters יתחבאו are hid ופני and the face תהום of the deep יתלכדו׃ is frozen.}%
\verse{התקשׁר Canst thou bind מעדנות the sweet influences כימה of Pleiades, או or משׁכות the bands כסיל of Orion? תפתח׃ loose}%
\verse{התציא Canst thou bring forth מזרות Mazzaroth בעתו in his season? ועישׁ Arcturus על with בניה his sons? תנחם׃ or canst thou guide}%
\verse{הידעת Knowest חקות thou the ordinances שׁמים of heaven? אם תשׂים canst thou set משׁטרו the dominion בארץ׃ thereof in the earth?}%
\verse{התרים Canst thou lift up לעב to the clouds, קולך thy voice ושׁפעת that abundance מים of waters תכסך׃ may cover}%
\verse{התשׁלח Canst thou send ברקים lightnings, וילכו that they may go, ויאמרו and say לך הננו׃}%
\verse{מי Who שׁת hath put בטחות in the inward parts? חכמה wisdom או or מי who נתן hath given לשׂכוי to the heart? בינה׃ understanding}%
\verse{מי Who יספר can number שׁחקים the clouds בחכמה in wisdom? ונבלי the bottles שׁמים of heaven, מי or who ישׁכיב׃ can stay}%
\verse{בצקת groweth עפר When the dust למוצק into hardness, ורגבים and the clods ידבקו׃ cleave fast together?}%
\verse{התצוד Wilt thou hunt ללביא for the lion? טרף the prey וחית the appetite כפירים of the young lions, תמלא׃ or fill}%
\verse{כי When ישׁחו they couch במעונות in dens, ישׁבו abide בסכה in the covert למו to lie in wait? ארב׃ to lie in wait?}%
\verse{מי Who יכין provideth לערב for the raven צידו his food? כי when ילדו his young ones אל God, אל unto ישׁועו cry יתעו they wander לבלי for lack אכל׃ of meat.}%
\end{biblechapter}%
\begin{biblechapter}% Job 39
\verse{הידעת Knowest עת thou the time לדת bring forth? יעלי when the wild goats סלע of the rock חלל do calve? אילות when the hinds תשׁמר׃ canst thou mark}%
\verse{תספר Canst thou number ירחים the months תמלאנה they fulfill? וידעת or knowest עת thou the time לדתנה׃ when they bring forth?}%
\verse{תכרענה They bow ילדיהן their young ones, תפלחנה themselves, they bring forth חבליהם their sorrows. תשׁלחנה׃ they cast out}%
\verse{יחלמו are in good liking, בניהם Their young ones ירבו they grow up בבר with corn; יצאו they go forth, ולא not שׁבו׃ and return}%
\verse{מי Who שׁלח hath sent out פרא the wild ass חפשׁי free? ומסרות the bands ערוד of the wild ass? מי or who פתח׃ hath loosed}%
\verse{אשׁר Whose שׂמתי I have made ערבה the wilderness, ביתו house ומשׁכנותיו his dwellings. מלחה׃ and the barren land}%
\verse{ישׂחק He scorneth להמון the multitude קריה of the city, תשׁאות he the crying נוגשׂ of the driver. לא neither ישׁמע׃ regardeth}%
\verse{יתור The range הרים of the mountains מרעהו his pasture, ואחר after כל every ירוק green thing. ידרושׁ׃ and he searcheth}%
\verse{היאבה be willing רים Will the unicorn עבדך to serve אם thee, or ילין abide על by אבוסך׃ thy crib?}%
\verse{התקשׁר Canst thou bind רים the unicorn בתלם in the furrow? עבתו with his band אם or ישׂדד will he harrow עמקים the valleys אחריך׃ after}%
\verse{התבטח Wilt thou trust בו כי him, because רב great? כחו his strength ותעזב or wilt thou leave אליו to יגיעך׃ thy labor}%
\verse{התאמין Wilt thou believe בו כי him, that ישׁוב he will bring home זרעך thy seed, וגרנך thy barn? יאסף׃ and gather}%
\verse{כנף wings רננים unto the peacocks? נעלסה the goodly אם or אברה wings חסידה and feathers ונצה׃ unto the ostrich?}%
\verse{כי Which תעזב leaveth לארץ in the earth, בציה her eggs ועל them in עפר dust, תחמם׃ and warmeth}%
\verse{ותשׁכח And forgetteth כי that רגל the foot תזורה may crush וחית beast השׂדה them, or that the wild תדושׁה׃ may break}%
\verse{הקשׁיח She is hardened בניה against her young ones, ללא as though not לה לריק is in vain יגיעה hers: her labor בלי without פחד׃ fear;}%
\verse{כי Because השׁה hath deprived אלוה God חכמה her of wisdom, ולא neither חלק hath he imparted לה בבינה׃ to her understanding.}%
\verse{כעת What time במרום on high, תמריא she lifteth up herself תשׂחק she scorneth לסוס the horse ולרכבו׃ and his rider.}%
\verse{התתן Hast thou given לסוס the horse גבורה strength? התלבישׁ hast thou clothed צוארו his neck רעמה׃ with thunder?}%
\verse{התרעישׁנו Canst thou make him afraid כארבה as a grasshopper? הוד the glory נחרו of his nostrils אימה׃ terrible.}%
\verse{יחפרו He paweth בעמק in the valley, וישׂישׂ and rejoiceth בכח in strength: יצא he goeth on לקראת to meet נשׁק׃ the armed men.}%
\verse{ישׂחק He mocketh לפחד at fear, ולא and is not יחת frightened; ולא neither ישׁוב turneth he back מפני from חרב׃ the sword.}%
\verse{עליו against תרנה rattleth אשׁפה The quiver להב him, the glittering חנית spear וכידון׃ and the shield.}%
\verse{ברעשׁ with fierceness ורגז and rage: יגמא He swalloweth ארץ the ground ולא neither יאמין believeth כי he that קול the sound שׁופר׃ of the trumpet.}%
\verse{בדי among שׁפר the trumpets, יאמר He saith האח Ha, ha; ומרחוק afar off, יריח and he smelleth מלחמה the battle רעם the thunder שׂרים of the captains, ותרועה׃ and the shouting.}%
\verse{המבינתך by thy wisdom, יאבר fly נץ Doth the hawk יפרשׂ stretch כנפו her wings לתימן׃ toward the south?}%
\verse{אם על at פיך thy command, יגביה mount up נשׁר Doth the eagle וכי ירים and make her nest on high? קנו׃ and make her nest on high?}%
\verse{סלע on the rock, ישׁכן She dwelleth ויתלנן and abideth על upon שׁן the crag סלע of the rock, ומצודה׃ and the strong place.}%
\verse{משׁם חפר she seeketh אכל the prey, למרחוק afar off. עיניו her eyes יביטו׃ behold}%
\verse{ואפרחו Her young ones יעלעו also suck up דם blood: ובאשׁר and where חללים the slain שׁם there הוא׃ she.}%
\end{biblechapter}%
\begin{biblechapter}% Job 40
\verse{ויען answered יהוה Moreover the LORD את איוב Job, ויאמר׃ and said,}%
\verse{הרב Shall he that contendeth עם with שׁדי the Almighty יסור instruct מוכיח he that reproveth אלוה God, יעננה׃ let him answer}%
\verseWithHeading{Job Responds to Adonai}{ויען answered איוב Then Job את יהוה the LORD, ויאמר׃ and said,}%
\verse{הן Behold, קלתי I am vile; מה what אשׁיבך shall I answer ידי mine hand שׂמתי thee? I will lay למו upon פי׃ my mouth.}%
\verse{אחת Once דברתי have I spoken; ולא but I will not אענה answer: ושׁתים yea, twice; ולא אוסיף׃}%
\verseWithHeading{Adonai Challenges Job Again}{ויען Then answered יהוה the LORD את איוב unto Job מנסערה the whirlwind, ויאמר׃ and said,}%
\verse{אזר Gird up נא now כגבר like a man: חלציך thy loins אשׁאלך I will demand והודיעני׃ of thee, and declare}%
\verseWithHeading{Adonai Interrogates Job Again}{האף Wilt thou also תפר disannul משׁפטי my judgment? תרשׁיעני wilt thou condemn למען me, that תצדק׃ thou mayest be righteous?}%
\verse{ואם זרוע Hast thou an arm כאל like God? לך ובקול with a voice כמהו like him? תרעם׃ or canst thou thunder}%
\verse{עדה Deck נא thyself now גאון majesty וגבה and excellency; והוד thyself with glory והדר and beauty. תלבשׁ׃ and array}%
\verse{הפץ Cast abroad עברות the rage אפך of thy wrath: וראה and behold כל every one גאה proud, והשׁפילהו׃ and abase}%
\verse{ראה Look on כל every one גאה proud, הכניעהו bring him low; והדך and tread down רשׁעים the wicked תחתם׃ in their place.}%
\verse{טמנם Hide בעפר them in the dust יחד together; פניהם their faces חבשׁ bind בטמון׃ in secret.}%
\verse{וגם also אני Then will I אודך confess כי unto thee that תושׁע can save לך ימינך׃ thine own right hand}%
\verse{הנה Behold נא now בהמות behemoth, אשׁר which עשׂיתי I made עמך with חציר grass כבקר as an ox. יאכל׃ thee; he eateth}%
\verse{הנה Lo נא now, כחו his strength במתניו in his loins, ואנו and his force בשׁרירי in the navel בטנו׃ of his belly.}%
\verse{יחפץ He moveth זנבו his tail כמו like ארז a cedar: גידי the sinews פחדו of his stones ישׂרגו׃ are wrapped together.}%
\verse{עצמיו His bones אפיקי strong pieces נחושׁה of brass; גרמיו his bones כמטיל like bars ברזל׃ of iron.}%
\verse{הוא He ראשׁית the chief דרכי of the ways אל of God: העשׂו he that made יגשׁ to approach חרבו׃ him can make his sword}%
\verse{כי Surely בול food, הרים the mountains ישׂאו bring him forth לו וכל all חית the beasts השׂדה of the field ישׂחקו play. שׁם׃ where}%
\verse{תחת under צאלים the shady trees, ישׁכב He lieth בסתר in the covert קנה of the reed, ובצה׃ and fens.}%
\verse{יסכהו cover צאלים The shady trees צללו him their shadow; יסבוהו compass him about. ערבי the willows נחל׃ of the brook}%
\verse{הן Behold, יעשׁק he drinketh up נהר a river, לא not: יחפוז hasteth יבטח he trusteth כי that יגיח he can draw up ירדן Jordan אל into פיהו׃ his mouth.}%
\verse{בעיניו it with his eyes: יקחנו He taketh במוקשׁים through snares. ינקב pierceth אף׃ nose}%
\end{biblechapter}%
\begin{biblechapter}% Job 41
\verse{תמשׁך Canst thou draw out לויתן leviathan בחכה with a hook? ובחבל with a cord תשׁקיע thou lettest down? לשׁנו׃ or his tongue}%
\verse{התשׂים Canst thou put אגמון a hook באפו into his nose? ובחוח through with a thorn? תקוב or bore לחיו׃ his jaw}%
\verse{הירבה Will he make many אליך unto תחנונים supplications אם ידבר thee? will he speak אליך unto רכות׃ soft}%
\verse{היכרת Will he make ברית a covenant עמך with תקחנו thee? wilt thou take לעבד him for a servant עולם׃ forever?}%
\verse{התשׂחק Wilt thou play בו כצפור with him as a bird? ותקשׁרנו or wilt thou bind לנערותיך׃ him for thy maidens?}%
\verse{יכרו make a banquet עליו of חברים Shall the companions יחצוהו him? shall they part בין him among כנענים׃ the merchants?}%
\verse{התמלא Canst thou fill בשׂכות with barbed irons? עורו his skin ובצלצל spears? דגים with fish ראשׁו׃ or his head}%
\verse{שׂים Lay עליו upon כפך thine hand זכר him, remember מלחמה the battle, אל do no תוסף׃ more.}%
\verse{הן Behold, תחלתו the hope נכזבה of him is in vain: הגם even אל מראיו the sight יטל׃ shall not be cast down}%
\verse{לא None אכזר fierce כי that יעורנו dare stir him up: ומי who הוא לפני before יתיצב׃ then is able to stand}%
\verse{מי Who הקדימני hath prevented ואשׁלם me, that I should repay תחת under כל the whole השׁמים heaven לי הוא׃}%
\verse{לא I will not אחרישׁ conceal בדיו his parts, ודבר nor גבורות his power, וחין nor his comely ערכו׃ proportion.}%
\verse{מי Who גלה can discover פני the face לבושׁו of his garment? בכפל with his double רסנו bridle? מי who יבוא׃ can come}%
\verse{דלתי the doors פניו of his face? מי Who פתח can open סביבות round about. שׁניו his teeth אימה׃ terrible}%
\verse{גאוה pride, אפיקי מגניםגור shut up together חותם seal. צר׃ a close}%
\verse{אחד One באחד to another, יגשׁו is so near ורוח air לא that no יבוא can come ביניהם׃ between}%
\verse{אישׁ one באחיהו to another, ידבקו They are joined יתלכדו they stick together, ולא that they cannot יתפרדו׃ be sundered.}%
\verse{עטישׁתיו By his sneezes תהל doth shine, אור a light ועיניו and his eyes כעפעפי like the eyelids שׁחר׃ of the morning.}%
\verse{מפיו לפידים burning lamps, יהלכו go כידודי sparks אשׁ of fire יתמלטו׃ leap out.}%
\verse{מנחיריו יצא goeth עשׁן smoke, כדוד pot נפוח as of a seething ואגמן׃ or caldron.}%
\verse{נפשׁו His breath גחלים coals, תלהט kindleth ולהב and a flame מפיו of his mouth. יצא׃ goeth out}%
\verse{בצוארו In his neck ילין remaineth עז strength, ולפניו before תדוץ is turned into joy דאבה׃ and sorrow}%
\verse{מפלי The flakes בשׂרו of his flesh דבקו are joined together: יצוק they are firm עליו in בל themselves; they cannot ימוט׃ be moved.}%
\verse{לבו His heart יצוק is as firm כמו as אבן a stone; ויצוק yea, as hard כפלח as a piece תחתית׃ of the nether}%
\verse{משׂתו יגורו are afraid: אלים himself, the mighty משׁברים by reason of breakings יתחטאו׃ they purify themselves.}%
\verse{משׂיגהו of him that layeth חרב The sword בלי at him cannot תקום hold: חנית the spear, מסע the dart, ושׁריה׃ nor the habergeon.}%
\verse{יחשׁב He esteemeth לתבן as straw, ברזל iron לעץ wood. רקבון as rotten נחושׁה׃ brass}%
\verse{לא cannot יבריחנו make him flee: בן קשׁתקשׁ with him into stubble. נהפכו are turned לו אבני slingstones קלע׃ slingstones}%
\verse{כקשׁ as stubble: נחשׁבו are counted תותח Darts וישׂחק he laugheth לרעשׁ at the shaking כידון׃ of a spear.}%
\verse{תחתיו under חדודי Sharp חרשׂ stones ירפד him: he spreadeth חרוץ sharp pointed things עלי upon טיט׃ the mire.}%
\verse{ירתיח to boil כסיר like a pot: מצולה He maketh the deep ים the sea ישׂים he maketh כמרקחה׃ like a pot of ointment.}%
\verse{אחריו after יאיר to shine נתיב He maketh a path יחשׁב him; would think תהום the deep לשׂיבה׃ hoary.}%
\verse{אין there is not על Upon עפר earth משׁלו his like, העשׂו who is made לבלי without חת׃ fear.}%
\verse{את כל all גבה high יראה He beholdeth הוא he מלך a king על over כל all בני the children שׁחץ׃ of pride.}%
\end{biblechapter}%
\begin{biblechapter}% Job 42
\verseWithHeading{Job’s Repentance and Restoration}{ויען answered איוב Then Job את יהוה the LORD, ויאמר׃ and said,}%
\verse{ידעת I know כי that כל every תוכל thou canst do ולא and no יבצר can be withheld ממך from מזמה׃ thought}%
\verse{מי Who זה he מעלים that hideth עצה counsel בלי without דעת knowledge? לכן therefore הגדתי have I uttered ולא not; אבין that I understood נפלאות things too wonderful ממני for ולא not. אדע׃ me, which I knew}%
\verse{שׁמע Hear, נא I beseech thee, ואנכי and I אדבר will speak: אשׁאלך I will demand והודיעני׃ of thee, and declare}%
\verse{לשׁמע I have heard אזן of the ear: שׁמעתיך of thee by the hearing ועתה but now עיני mine eye ראתך׃ seeth}%
\verse{על in כן אמאס I abhor ונחמתי and repent על עפר dust ואפר׃ and ashes.}%
\verse{ויהי And it was אחר that after דבר had spoken יהוה the LORD את הדברים words האלה these אל unto איוב Job, ויאמר said יהוה the LORD אל to אליפז Eliphaz התימני the Temanite, חרה is kindled אפי My wrath בך ובשׁני against thee, and against thy two רעיך friends: כי for לא ye have not דברתם spoken אלי of נכונה me right, כעבדי as my servant איוב׃ Job}%
\verse{ועתה unto you now קחו Therefore take לכם שׁבעה seven פרים bullocks ושׁבעה and seven אילים rams, ולכו and go אל to עבדי my servant איוב Job, והעליתם and offer up עולה yourselves a burnt offering; בעדכם for ואיוב Job עבדי and my servant יתפלל shall pray עליכם for כי you: for אם you: for פניו him אשׂא will I accept: לבלתי lest עשׂות I deal עמכם with נבלה you folly, כי in that לא ye have not דברתם spoken אלי of נכונה me right, כעבדי like my servant איוב׃ Job.}%
\verse{וילכו went, אליפז So Eliphaz התימני the Temanite ובלדד and Bildad השׁוחי the Shuhite צפר Zophar הנעמתי the Naamathite ויעשׂו and did כאשׁר according as דבר commanded אליהם commanded יהוה the LORD וישׂא also accepted יהוה them: the LORD את פני also accepted איוב׃ Job.}%
\verse{ויהוה And the LORD שׁב turned את שׁבית the captivity איוב of Job, בהתפללו when he prayed בעד for רעהו his friends: ויסף gave יהוה also the LORD את כל as much as he had before. אשׁר as much as he had before. לאיוב Job למשׁנה׃ twice}%
\verse{ויבאו Then came אליו there unto כל him all אחיו his brethren, וכל and all אחיתיו his sisters, וכל and all ידעיו they that had been of his acquaintance לפנים before, ויאכלו and did eat עמו with לחם bread בביתו him in his house: וינדו and they bemoaned לו וינחמו him, and comforted אתו על him over כל all הרעה the evil אשׁר that הביא had brought יהוה the LORD עליו upon ויתנו also gave לו אישׁ him: every man קשׂיטה him a piece of money, אחת him a piece of money, ואישׁ and every one נזם earring זהב of gold. אחד׃ an}%
\verse{ויהוה So the LORD ברך blessed את אחרית the latter end איוב of Job מראשׁתו more than his beginning: ויהי for he had לו ארבעה fourteen עשׂר fourteen אלף thousand צאן sheep, ושׁשׁת and six אלפים thousand גמלים camels, ואלף and a thousand צמד yoke בקר of oxen, ואלף and a thousand אתונות׃ she asses.}%
\verse{ויהי He had לו שׁבענה also seven בנים sons ושׁלושׁ and three בנות׃ daughters.}%
\verse{ויקרא And he called שׁם the name האחת of the first, ימימה Jemima; ושׁם and the name השׁנית of the second, קציעה Kezia; ושׁם and the name השׁלישׁית of the third, קרן הפוך׃ Keren-happuch.}%
\verse{ולא were no נמצא found נשׁים women יפות fair כבנות as the daughters איוב of Job: בכל And in all הארץ the land ויתן gave להם אביהם and their father נחלה them inheritance בתוך among אחיהם׃ their brethren.}%
\verse{ויחי lived איוב Job אחרי After זאת this מאה a hundred וארבעים and forty שׁנה years, וירא and saw את בניו his sons, ואת בני andhis sons' בניו sons, ארבעה four דרות׃ generations.}%
\verse{וימת died, איוב So Job זקן old ושׂבע and full ימים׃ of days.}%
\end{biblechapter}%
\flushcolsend
\biblebook{Psalm}
\begin{biblechapter}% Psalm 1
\verseWithHeading{The Ways of the Righteous and the Wicked}{אשׁרי Blessed האישׁ the man אשׁר that לא not הלך walketh בעצת in the counsel רשׁעים of the ungodly, ובדרך in the way חטאים of sinners, לא nor עמד standeth ובמושׁב in the seat לצים of the scornful. לא nor ישׁב׃ sitteth}%
\verse{כי אםתורת in the law יהוה חפצו his delight ובתורתו and in his law יהגה doth he meditate יומם day ולילה׃ and night.}%
\verse{והיה And he shall be כעץ like a tree שׁתול planted על by פלגי the rivers מים of water, אשׁר that פריו his fruit יתן bringeth forth בעתו in his season; ועלהו his leaf לא also shall not יבול wither; וכל and whatsoever אשׁר and whatsoever יעשׂה he doeth יצליח׃ shall prosper.}%
\verse{לא not כן so: הרשׁעים The ungodly כי but אם but כמץ like the chaff אשׁר which תדפנו driveth away. רוח׃ the wind}%
\verse{על כןא shall not יקמו stand רשׁעים the ungodly במשׁפט in the judgment, וחטאים nor sinners בעדת in the congregation צדיקים׃ of the righteous.}%
\verse{כי For יודע knoweth יהוה the LORD דרך the way צדיקים of the righteous: ודרך but the way רשׁעים of the ungodly תאבד׃ shall perish.}%
\end{biblechapter}%
\begin{biblechapter}% Psalm 2
\verseWithHeading{The Messiah’s Reign}{למה Why רגשׁו rage, גוים do the heathen ולאמים and the people יהגו imagine ריק׃ a vain thing?}%
\verse{יתיצבו set themselves, מלכי The kings ארץ of the earth ורוזנים and the rulers נוסדו take counsel יחד together, על against יהוה the LORD, ועל and against משׁיחו׃ his anointed,}%
\verse{ננתקה אתוסרותימו ונשׁליכה and cast away ממנו from עבתימו׃ their cords}%
\verse{יושׁב He that sitteth בשׁמים in the heavens ישׂחק shall laugh: אדני the Lord ילעג׃ shall have them in derision.}%
\verse{אז Then ידבר shall he speak אלימו unto באפו them in his wrath, ובחרונו them in his sore displeasure. יבהלמו׃ and vex}%
\verse{ואני Yet have I נסכתי set מלכי my king על upon ציון of Zion. הר hill קדשׁי׃ my holy}%
\verse{אספרה אל unto חק the decree: יהוה אמר hath said אלי בני my Son; אתה me, Thou אני have I היום this day ילדתיך׃ begotten}%
\verse{שׁאל Ask ממני of ואתנה me, and I shall give גוים the heathen נחלתך thine inheritance, ואחזתך thy possession. אפסי and the uttermost parts ארץ׃ of the earth}%
\verse{תרעם Thou shalt break בשׁבט them with a rod ברזל of iron; ככלי vessel. יוצר like a potter's תנפצם׃ thou shalt dash them in pieces}%
\verse{ועתה now מלכים therefore, O ye kings: השׂכילו Be wise הוסרו be instructed, שׁפטי ye judges ארץ׃ of the earth.}%
\verse{עבדו Serve את יהוה the LORD ביראה with fear, וגילו and rejoice ברעדה׃ with trembling.}%
\verse{נשׁקו Kiss בר the Son, פן lest יאנף he be angry, ותאבדו and ye perish דרך the way, כי when יבער is kindled כמעט but a little. אפו his wrath אשׁרי Blessed כל all חוסי׃ they that put their trust}%
\end{biblechapter}%
\begin{biblechapter}% Psalm 3
\verseWithHeading{A Call to Adonai in Distress}{מזמור A Psalm לדוד of David, בברחו when he fled מפני from אבשׁלום Absalom בנו׃ his son. יהוה LORD, מה how רבו are they increased צרי that trouble רבים me! many קמים they that rise up עלי׃ against}%
\verse{רבים Many אמרים which say לנפשׁי of my soul, אין no ישׁועתה help לו באלהים for him in God. סלה׃ Selah.}%
\verse{ואתה But thou, יהוה O LORD, מגן a shield בעדי for כבודי me; my glory, ומרים and the lifter up ראשׁי׃ of mine head.}%
\verse{קולי with my voice, אל unto יהוה אקרא I cried ויענני and he heard מהר hill. קדשׁו me out of his holy סלה׃ Selah.}%
\verse{אני I שׁכבתי laid me down ואישׁנה and slept; הקיצותי I awaked; כי for יהוה the LORD יסמכני׃ sustained}%
\verse{לא I will not אירא be afraid מרבבות of ten thousands עם of people, אשׁר that סביב me round about. שׁתו have set עלי׃ against}%
\verse{קומה Arise, יהוה O LORD; הושׁיעני save אלהי me, O my God: כי for הכית thou hast smitten את כל all איבי mine enemies לחי the cheek bone; שׁני the teeth רשׁעים of the ungodly. שׁברת׃ thou hast broken}%
\verse{ליהוה unto the LORD: הישׁועה Salvation על upon עמך thy people. ברכתך thy blessing סלה׃ Selah.}%
\end{biblechapter}%
\begin{biblechapter}% Psalm 4
\verseWithHeading{Safety in Adonai}{למנצח To the chief Musician בנגינות on Neginoth, מזמור A Psalm לדוד׃ of David. בקראי me when I call, ענני Hear אלהי O God צדקי of my righteousness: בצר me in distress; הרחבת thou hast enlarged לי חנני have mercy ושׁמע upon me, and hear תפלתי׃ my prayer.}%
\verse{בני O ye sons אישׁ of men, עד how long מה how long כבודי my glory לכלמה into shame? תאהבון will ye love ריק vanity, תבקשׁו seek after כזב leasing? סלה׃ Selah.}%
\verse{ודעו But know כי that הפלה hath set apart יהוה the LORD חסיד him that is godly לו יהוה for himself: the LORD ישׁמע will hear בקראי when I call אליו׃ unto}%
\verse{רגזו Stand in awe, ואל not: תחטאו and sin אמרו commune בלבבכם with your own heart על upon משׁכבכם your bed, ודמו and be still. סלה׃ Selah.}%
\verse{זבחו Offer זבחי the sacrifices צדק of righteousness, ובטחו and put your trust אל in יהוה׃ the LORD.}%
\verse{רבים many אמרים that say, מי Who יראנו will show טוב us good? נסה lift thou up עלינו upon אור the light פניך of thy countenance יהוה׃ LORD,}%
\verse{נתתה Thou hast put שׂמחה gladness בלבי in my heart, מעת more than in the time דגנם their corn ותירושׁם and their wine רבו׃ increased.}%
\verse{בשׁלום in peace, יחדו I will both אשׁכבה lay me down ואישׁן and sleep: כי for אתה thou, יהוה LORD, לבדד only לבטח in safety. תושׁיבני׃ makest me dwell}%
\end{biblechapter}%
\begin{biblechapter}% Psalm 5
\verseWithHeading{A Prayer for Guidance and Protection}{למנצח To the chief Musician אל upon הנחילות Nehiloth, מזמור A Psalm לדוד׃ of David. אמרי to my words, האזינה Give ear יהוה O LORD, בינה consider הגיגי׃ my meditation.}%
\verse{הקשׁיבה Hearken לקול unto the voice שׁועי of my cry, מלכי my King, ואלהי and my God: כי for אליך unto אתפלל׃ thee will I pray.}%
\verse{יהוה O LORD; בקר in the morning, תשׁמע shalt thou hear קולי My voice בקר in the morning אערך will I direct לך ואצפה׃ unto thee, and will look up.}%
\verse{כי For לא not אל a God חפץ that hath pleasure רשׁע in wickedness: אתה thou לא neither יגרך dwell רע׃ shall evil}%
\verse{לא shall not יתיצבו stand הוללים The foolish לנגד עיניך thy sight: שׂנאת thou hatest כל all פעלי workers און׃ of iniquity.}%
\verse{תאבד Thou shalt destroy דברי them that speak כזב leasing: אישׁ man. דמים the bloody ומרמה and deceitful יתעב will abhor יהוה׃ the LORD}%
\verse{ואני But as for me, ברב in the multitude חסדך of thy mercy: אבוא I will come ביתך thy house אשׁתחוה will I worship אל toward היכל temple. קדשׁך thy holy ביראתך׃ in thy fear}%
\verse{יהוה me, O LORD, נחני Lead בצדקתך in thy righteousness למען because of שׁוררי mine enemies; הושׁר straight לפני before my face. דרכך׃ make thy way}%
\verse{כי For אין no בפיהו in their mouth; נכונה faithfulness קרבם their inward part הוות very wickedness; קבר sepulcher; פתוח an open גרונם their throat לשׁונם with their tongue. יחליקון׃ they flatter}%
\verse{האשׁימם Destroy אלהים thou them, O God; יפלו let them fall ממעצותיהם by their own counsels; ברב in the multitude פשׁעיהם of their transgressions; הדיחמו cast them out כי for מרו׃ they have rebelled}%
\verse{וישׂמחו in thee rejoice: כל But let all חוסי those that put their trust בך לעולם let them ever ירננו shout for joy, ותסך because thou defendest עלימו them: ויעלצו be joyful בך אהבי let them also that love שׁמך׃ thy name}%
\verse{כי For אתה thou, תברך wilt bless צדיק the righteous; יהוה LORD, כצנה him as a shield. רצון with favor תעטרנו׃ wilt thou compass}%
\end{biblechapter}%
\begin{biblechapter}% Psalm 6
\verseWithHeading{An Appeal for Forgiveness and Deliverance}{למנצח To the chief Musician בנגינות on Neginoth על upon השׁמינית Sheminith, מזמור A Psalm לדוד׃ of David. יהוה O LORD, אל me not באפך in thine anger, תוכיחני rebuke ואל neither בחמתך me in thy hot displeasure. תיסרני׃ chasten}%
\verse{חנני Have mercy יהוה כי for אמלל weak: אני I רפאני heal יהוה כי me; for נבהלו are vexed. עצמי׃ my bones}%
\verse{ונפשׁי My soul נבהלה vexed: מאד is also sore ואת but thou, יהוה O LORD, עד how long? מתי׃ how long?}%
\verse{שׁובה Return, יהוה O LORD, חלצה deliver נפשׁי my soul: הושׁיעני oh save למען חסדך׃}%
\verse{כי For אין no במות in death זכרך remembrance בשׁאול of thee: in the grave מי who יודה׃ shall give thee thanks?}%
\verse{יגעתי I am weary באנחתי with my groaning; אשׂחה to swim; בכל all לילה the night מטתי make I my bed בדמעתי with my tears. ערשׂי my couch אמסה׃ I water}%
\verse{עשׁשׁה is consumed מכעס because of grief; עיני Mine eye עתקה it waxeth old בכל because of all צוררי׃ mine enemies.}%
\verse{סורו Depart ממני from כל me, all פעלי ye workers און of iniquity; כי for שׁמע hath heard יהוה the LORD קול the voice בכיי׃ of my weeping.}%
\verse{שׁמע hath heard יהוה The LORD תחנתי my supplication; יהוה the LORD תפלתי my prayer. יקח׃ will receive}%
\verse{יבשׁו be ashamed ויבהלו vexed: מאד and sore כל Let all איבי mine enemies ישׁבו let them return יבשׁו be ashamed רגע׃ suddenly.}%
\end{biblechapter}%
\begin{biblechapter}% Psalm 7
\verseWithHeading{Prayer for Deliverance from Enemies}{שׁגיון Shiggaion לדוד of David, אשׁר which שׁר he sang ליהוה unto the LORD, על concerning דברי the words כושׁ of Cush בן־ימיני׃ יהוה O LORD אלהי my God, בך חסיתי in thee do I put my trust: הושׁיעני save מכל me from all רדפי them that persecute והצילני׃ me, and deliver}%
\verse{פן Lest יטרף he tear כאריה like a lion, נפשׁי my soul פרק rending in pieces, ואין while none מציל׃ to deliver.}%
\verse{יהוה O LORD אלהי my God, אם if עשׂיתי I have done זאת this; אם if ישׁ there be עול iniquity בכפי׃ in my hands;}%
\verse{אם If גמלתי I have rewarded שׁולמי unto him that was at peace רע evil ואחלצה with me; (yea, I have delivered צוררי is mine enemy:) ריקם׃ him that without cause}%
\verse{ירדף persecute אויב Let the enemy נפשׁי my soul, וישׂג and take וירמס yea, let him tread down לארץ upon the earth, חיי my life וכבודי mine honor לעפר in the dust. ישׁכן and lay סלה׃ Selah.}%
\verse{קומה Arise, יהוה O LORD, באפך in thine anger, הנשׂא lift up thyself בעברות because of the rage צוררי of mine enemies: ועורה and awake אלי משׁפט me the judgment צוית׃ thou hast commanded.}%
\verse{ועדת So shall the congregation לאמים of the people תסובבך compass thee about: ועליה for their sakes למרום thou on high. שׁובה׃ therefore return}%
\verse{יהוה The LORD ידין shall judge עמים the people: שׁפטני judge יהוה me, O LORD, כצדקי according to my righteousness, וכתמי and according to mine integrity עלי׃ in}%
\verse{יגמר come to an end; נא Oh רע let the wickedness רשׁעים of the wicked ותכונן but establish צדיק the just: ובחן trieth לבות וכליות and reins. אלהים God צדיק׃ for the righteous}%
\verse{מגני My defense על of אלהים God, מושׁיע which saveth ישׁרי the upright לב׃ in heart.}%
\verse{אלהים God שׁופט judgeth צדיק the righteous, ואל and God זעם is angry בכל every יום׃ day.}%
\verse{אם If לא not, ישׁוב he turn חרבו his sword; ילטושׁ he will whet קשׁתו his bow, דרך he hath bent ויכוננה׃ and made it ready.}%
\verse{ולו הכין He hath also prepared כלי for him the instruments מות of death; חציו his arrows לדלקים against the persecutors. יפעל׃ he ordaineth}%
\verse{הנה Behold, יחבל he travaileth און with iniquity, והרה and hath conceived עמל mischief, וילד and brought forth שׁקר׃ falsehood.}%
\verse{בור a pit, כרה He made ויחפרהו and digged ויפל it, and is fallen בשׁחת into the ditch יפעל׃ he made.}%
\verse{ישׁוב shall return עמלו His mischief בראשׁו upon his own head, ועל upon קדקדו his own pate. חמסו and his violent dealing ירד׃ shall come down}%
\verse{אודה I will praise יהוה the LORD כצדקו according to his righteousness: ואזמרה and will sing praise שׁם to the name יהוה of the LORD עליון׃ most high.}%
\end{biblechapter}%
\begin{biblechapter}% Psalm 8
\verseWithHeading{Adonai’s Glory in Creation}{למנצח To the chief Musician על upon הגתית Gittith, מזמור A Psalm לדוד׃ of David. יהוה O LORD אדנינו our Lord, מה how אדיר excellent שׁמך thy name בכל in all הארץ the earth! אשׁר who תנה hast set הודך thy glory על above השׁמים׃ the heavens.}%
\verse{מפי עוללים of babes וינקים and sucklings יסדת hast thou ordained עז strength למען because of צורריך thine enemies, להשׁבית that thou mightest still אויב the enemy ומתנקם׃ and the avenger.}%
\verse{כי When אראה I consider שׁמיך thy heavens, מעשׂי the work אצבעתיך of thy fingers, ירח the moon וכוכבים and the stars, אשׁר which כוננתה׃ thou hast ordained;}%
\verse{מה What אנושׁ is man, כי that תזכרנו thou art mindful ובן of him? and the son אדם of man, כי that תפקדנו׃ thou visitest}%
\verse{ותחסרהו מעטאלהים than the angels, וכבוד him with glory והדר and honor. תעטרהו׃ and hast crowned}%
\verse{תמשׁילהו Thou madest him to have dominion במעשׂי over the works ידיך of thy hands; כל all שׁתה thou hast put תחת under רגליו׃ his feet:}%
\verse{צנה sheep ואלפים and oxen, כלם All וגם yea, בהמות and the beasts שׂדי׃ of the field;}%
\verse{צפור The fowl שׁמים of the air, ודגי and the fish הים of the sea, עבר passeth through ארחות the paths ימים׃ of the seas.}%
\verse{יהוה O LORD אדנינו our Lord, מה how אדיר excellent שׁמך thy name בכל in all הארץ׃ the earth!}%
\end{biblechapter}%
\begin{biblechapter}% Psalm 9
\verseWithHeading{Praise for Adonai’s Justice}{למנצח To the chief Musician עלמות upon לבן Muth-labben, מזמור A Psalm לדוד׃ of David. אודה I will praise יהוה O LORD, בכל with my whole לבי heart; אספרה I will show forth כל all נפלאותיך׃ thy marvelous works.}%
\verse{אשׂמחה I will be glad ואעלצה and rejoice בך אזמרה in thee: I will sing praise שׁמך to thy name, עליון׃ O thou most High.}%
\verse{בשׁוב are turned אויבי When mine enemies אחור back, יכשׁלו they shall fall ויאבדו and perish מפניך׃ at thy presence.}%
\verse{כי For עשׂית thou hast maintained משׁפטי my right ודיני and my cause; ישׁבת thou satest לכסא in the throne שׁופט judging צדק׃ right.}%
\verse{גערת Thou hast rebuked גוים the heathen, אבדת thou hast destroyed רשׁע the wicked, שׁמם their name מחית thou hast put out לעולם forever ועד׃ and ever.}%
\verse{האויב O thou enemy, תמו are come to a perpetual end: חרבות destructions לנצח are come to a perpetual end: וערים cities; נתשׁת and thou hast destroyed אבד is perished זכרם their memorial המה׃ with them.}%
\verse{ויהוה But the LORD לעולם forever: ישׁב shall endure כונן he hath prepared למשׁפט for judgment. כסאו׃ his throne}%
\verse{והוא And he ישׁפט shall judge תבל the world בצדק in righteousness, ידין he shall minister judgment לאמים to the people במישׁרים׃ in uprightness.}%
\verse{ויהי also will be יהוה The LORD משׂגב a refuge לדך for the oppressed, משׂגב a refuge לעתות in times בצרה׃ of trouble.}%
\verse{ויבטחו will put their trust בך יודעי And they that know שׁמך thy name כי in thee: for לא hast not עזבת forsaken דרשׁיך them that seek יהוה׃ thou, LORD,}%
\verse{זמרו Sing praises ליהוה to the LORD, ישׁב which dwelleth ציון in Zion: הגידו declare בעמים among the people עלילותיו׃ his doings.}%
\verse{כי When דרשׁ he maketh inquisition דמים for blood, אותם זכר he remembereth לא not שׁכח them: he forgetteth צעקת the cry עניים׃}%
\verse{חננני Have mercy יהוה upon me, O LORD; ראה consider עניי my trouble משׂנאי of them that hate מרוממי me, thou that liftest me up משׁערי from the gates מות׃ of death:}%
\verse{למען That אספרה I may show forth כל all תהלתיך thy praise בשׁערי in the gates בת of the daughter ציון of Zion: אגילה I will rejoice בישׁועתך׃ in thy salvation.}%
\verse{טבעו are sunk down גוים The heathen בשׁחת in the pit עשׂו they made: ברשׁת in the net זו which טמנו they hid נלכדה taken. רגלם׃ is their own foot}%
\verse{נודע is known יהוה The LORD משׁפט the judgment עשׂה he executeth: בפעל in the work כפיו of his own hands. נוקשׁ רשׁע the wicked הגיון Higgaion. סלה׃ Selah.}%
\verse{ישׁובו shall be turned רשׁעים The wicked לשׁאולה into hell, כל all גוים the nations שׁכחי that forget אלהים׃ God.}%
\verse{כי For לא shall not לנצח always ישׁכח be forgotten: אביון the needy תקות the expectation ענוים תאבד shall perish לעד׃ forever.}%
\verse{קומה Arise, יהוה אל let not יעז prevail: אנושׁ ישׁפטו be judged גוים let the heathen על in פניך׃ thy sight.}%
\verse{שׁיתה Put יהוה מורה them in fear, להם themselves ידעו may know גוים the nations אנושׁ men. המה סלה׃ Selah.}%
\end{biblechapter}%
\begin{biblechapter}% Psalm 10
\verseWithHeading{A Prayer for God to Throw down the Wicked}{למה Why יהוה O LORD? תעמד standest ברחוק thou afar off, תעלים hidest לעתות thou in times בצרה׃ of trouble?}%
\verse{בגאות in pride רשׁע The wicked ידלק doth persecute עני the poor: יתפשׂו let them be taken במזמות in the devices זו that חשׁבו׃ they have imagined.}%
\verse{כי For הלל boasteth רשׁע the wicked על of תאות desire, נפשׁו his heart's ובצע the covetous, ברך and blesseth נאץ abhorreth. יהוה׃ the LORD}%
\verse{רשׁע The wicked, כגבה through the pride אפו of his countenance, בל will not ידרשׁ seek אין not אלהים God כל in all מזמותיו׃ his thoughts.}%
\verse{יחילו grievous; דרכו His ways בכל are always עת are always מרום far above משׁפטיך thy judgments מנגדו out of his sight: כל all צורריו his enemies, יפיח׃ he puffeth}%
\verse{אמר He hath said בלבו in his heart, בל I shall not אמוט be moved: לדר for never ודר אשׁרא ברע׃ in adversity.}%
\verse{אלה of cursing פיהו His mouth מלא is full ומרמות and deceit ותך and fraud: תחת under לשׁונו his tongue עמל mischief ואון׃ and vanity.}%
\verse{ישׁב He sitteth במארב in the lurking places חצרים of the villages: במסתרים in the secret places יהרג doth he murder נקי the innocent: עיניו his eyes לחלכה against the poor. יצפנו׃ are privily set}%
\verse{יארב He lieth in wait במסתר secretly כאריה as a lion בסכה in his den: יארב he lieth in wait לחטוף to catch עני the poor: יחטף he doth catch עני the poor, במשׁכו when he draweth ברשׁתו׃ him into his net.}%
\verse{ודכה He croucheth, ישׁח humbleth ונפל may fall בעצומיו by his strong ones. חלכאים׃ himself, that the poor}%
\verse{אמר He hath said בלבו in his heart, שׁכח hath forgotten: אל God הסתיר he hideth פניו his face; בל he will never ראה see לנצח׃ he will never}%
\verse{קומה Arise, יהוה O LORD; אל O God, נשׂא lift up ידך thine hand: אל not תשׁכח forget עניים׃}%
\verse{על מהאץ contemn רשׁע doth the wicked אלהים God? אמר he hath said בלבו in his heart, לא Thou wilt not תדרשׁ׃ require}%
\verse{ראתה Thou hast seen כי for אתה thou עמל mischief וכעס and spite, תביט beholdest לתת to requite בידך with thy hand: עליך himself unto יעזב committeth חלכה the poor יתום of the fatherless. אתה thee; thou היית art עוזר׃ the helper}%
\verse{שׁבר Break זרוע thou the arm רשׁע of the wicked ורע and the evil תדרושׁ seek out רשׁעו his wickedness בל none. תמצא׃ thou find}%
\verse{יהוה The LORD מלך King עולם forever ועד and ever: אבדו are perished גוים the heathen מארצו׃ out of his land.}%
\verse{תאות the desire ענוים of the humble: שׁמעת thou hast heard יהוה LORD, תכין thou wilt prepare לבם their heart, תקשׁיב to hear: אזנך׃ thou wilt cause thine ear}%
\verse{לשׁפט To judge יתום the fatherless ודך and the oppressed, בל may no יוסיף more עוד more לערץ oppress. אנושׁ מן of הארץ׃ the earth}%
\end{biblechapter}%
\begin{biblechapter}% Psalm 11
\verseWithHeading{Confidence in Adonai’s Righteousness}{למנצח To the chief Musician, לדוד of David. ביהוה In the LORD חסיתי put I my trust: איך how תאמרו say לנפשׁי ye to my soul, נודו Flee הרכם to your mountain? צפור׃ a bird}%
\verse{כי For, הנה lo, הרשׁעים the wicked ידרכון bend קשׁת bow, כוננו they make ready חצם their arrow על upon יתר the string, לירות shoot במו that they may privily אפל that they may privily לישׁרי at the upright לב׃ in heart.}%
\verse{כי If השׁתות the foundations יהרסון be destroyed, צדיק can the righteous מה what פעל׃ do?}%
\verse{יהוה The LORD בהיכל temple, קדשׁו in his holy יהוה the LORD's בשׁמים in heaven: כסאו throne עיניו his eyes יחזו behold, עפעפיו his eyelids יבחנו try, בני the children אדם׃ of men.}%
\verse{יהוה The LORD צדיק the righteous: יבחן trieth ורשׁע but the wicked ואהב and him that loveth חמס violence שׂנאה hateth. נפשׁו׃ his soul}%
\verse{ימטר he shall rain על Upon רשׁעים the wicked פחים snares, אשׁ fire וגפרית and brimstone, ורוח tempest: זלעפות and a horrible מנת the portion כוסם׃ of their cup.}%
\verse{כי For צדיק the righteous יהוה LORD צדקות righteousness; אהב loveth ישׁר the upright. יחזו doth behold פנימו׃ his countenance}%
\end{biblechapter}%
\begin{biblechapter}% Psalm 12
\verseWithHeading{Human Faithlessness and God’s Faithfulness}{למנצח To the chief Musician על upon השׁמינית Sheminith, מזמור A Psalm לדוד׃ of David. הושׁיעה Help, יהוה LORD; כי for גמר ceaseth; חסיד the godly man כי for פסו fail אמונים the faithful מבני from among the children אדם׃ of men.}%
\verse{שׁוא vanity ידברו They speak אישׁ every one את with רעהו his neighbor: שׂפת lips חלקות flattering בלב with a double heart ולב with a double heart ידברו׃ do they speak.}%
\verse{יכרת shall cut off יהוה The LORD כל all שׂפתי lips, חלקות flattering לשׁון the tongue מדברת that speaketh גדלות׃ proud things:}%
\verse{אשׁר Who אמרו have said, ללשׁננו With our tongue נגביר will we prevail; שׂפתינו our lips אתנו מי our own: who אדון׃ lord}%
\verse{משׁד עניים of the poor, מאנקת for the sighing אביונים of the needy, עתה now אקום will I arise, יאמר saith יהוה the LORD; אשׁית I will set בישׁע in safety יפיח׃ puffeth}%
\verse{אמרות The words יהוה of the LORD אמרות words: טהרות pure כסף silver צרוף tried בעליל in a furnace לארץ of earth, מזקק purified שׁבעתים׃ seven}%
\verse{אתה Thou יהוה them, O LORD, תשׁמרם shalt keep תצרנו thou shalt preserve מן them from הדור generation זו this לעולם׃ forever.}%
\verse{סביב on every side, רשׁעים The wicked יתהלכון walk כרם are exalted. זלות when the vilest לבני אדם׃ men}%
\end{biblechapter}%
\begin{biblechapter}% Psalm 13
\verseWithHeading{Trust in the Salvation of Adonai}{למנצח To the chief Musician, מזמור A Psalm לדוד׃ of David. עד how long אנה how long יהוה me, O LORD? תשׁכחני wilt thou forget נצח forever? עד אנהסתיר wilt thou hide את פניך thy face ממני׃ from}%
\verse{עד how long אנה how long אשׁית shall I take עצות counsel בנפשׁי in my soul, יגון sorrow בלבבי in my heart יומם daily? עד אנהרום be exalted איבי shall mine enemy עלי׃ over}%
\verse{הביטה Consider ענני hear יהוה me, O LORD אלהי my God: האירה lighten עיני mine eyes, פן lest אישׁן I sleep המות׃ the death;}%
\verse{פן Lest יאמר say, איבי mine enemy יכלתיו I have prevailed against צרי him; those that trouble יגילו me rejoice כי when אמוט׃ I am moved.}%
\verse{ואני But I בחסדך in thy mercy; בטחתי have trusted יגל shall rejoice לבי my heart בישׁועת׃ in thy salvation.}%
\verse{אשׁירה I will sing ליהוה unto the LORD, כי because גמל he hath dealt bountifully עלי׃ with}%
\end{biblechapter}%
\begin{biblechapter}% Psalm 14
\verseWithHeading{The Folly of the Godless and God’s Final Triumph}{למנצח To the chief Musician, לדוד of David. אמר hath said נבל The fool בלבו in his heart, אין no אלהים God. השׁחיתו They are corrupt, התעיבו they have done abominable עלילה works, אין none עשׂה that doeth טוב׃ good.}%
\verse{יהוה The LORD משׁמים from heaven השׁקיף looked down על upon בני the children אדם of men, לראות to see הישׁ if there were משׂכיל any that did understand, דרשׁ seek את אלהים׃ God.}%
\verse{הכל They are all סר gone aside, יחדו they are together נאלחו become filthy: אין none עשׂה that doeth טוב good, אין no, גם not אחד׃ one.}%
\verse{הלא no ידעו knowledge? כל Have all פעלי the workers און of iniquity אכלי who eat up עמי my people אכלו they eat לחם bread, יהוה upon the LORD. לא not קראו׃ and call}%
\verse{שׁם There פחדו were they in great fear: פחד were they in great fear: כי for אלהים God בדור in the generation צדיק׃ of the righteous.}%
\verse{עצת the counsel עני of the poor, תבישׁו Ye have shamed כי because יהוה the LORD מחסהו׃ his refuge.}%
\verse{מי יתןציון ישׁועת the salvation ישׂראל of Israel בשׁוב bringeth back יהוה when the LORD שׁבות the captivity עמו of his people, יגל shall rejoice, יעקב Jacob ישׂמח shall be glad. ישׂראל׃ Israel}%
\end{biblechapter}%
\begin{biblechapter}% Psalm 15
\verseWithHeading{A Description of Those Who May Dwell with Adonai}{מזמור A Psalm לדוד of David. יהוה LORD, מי who יגור shall abide באהלך in thy tabernacle? מי who ישׁכן shall dwell בהר hill? קדשׁך׃ in thy holy}%
\verse{הולך He that walketh תמים uprightly, ופעל and worketh צדק righteousness, ודבר and speaketh אמת the truth בלבבו׃ in his heart.}%
\verse{לא not רגל backbiteth על with לשׁנו his tongue, לא nor עשׂה doeth לרעהו to his neighbor, רעה evil וחרפה a reproach לא nor נשׂא taketh up על against קרבו׃ his neighbor.}%
\verse{נבזה is contemned; בעיניו In whose eyes נמאס a vile person ואת יראיהוה the LORD. יכבד but he honoreth נשׁבע sweareth להרע to hurt, ולא not. ימר׃ and changeth}%
\verse{כספו his money לא nor נתן בנשׁך to usury, ושׁחד reward על against נקי the innocent. לא shall never לקח taketh עשׂה He that doeth אלה these לא ימוט be moved. לעולם׃ shall never}%
\end{biblechapter}%
\begin{biblechapter}% Psalm 16
\verseWithHeading{Confidence in Adonai}{מכתם Michtam לדוד of David. שׁמרני Preserve אל me, O God: כי for חסיתי׃ in thee do I put my trust.}%
\verse{אמרת thou hast said ליהוה unto the LORD, אדני my Lord: אתה Thou טובתי my goodness בל not עליך׃ to}%
\verse{לקדושׁים to the saints אשׁר that בארץ in the earth, המה ואדירי and the excellent, כל in whom all חפצי׃ my delight.}%
\verse{ירבו shall be multiplied עצבותם Their sorrows אחר another מהרו hasten בל will I not אסיך offer, נסכיהם their drink offerings מדם of blood ובל nor אשׂא take up את שׁמותם their names על into שׂפתי׃ my lips.}%
\verse{יהוה The LORD מנת חלקי of mine inheritance וכוסי and of my cup: אתה thou תומיך maintainest גורלי׃ my lot.}%
\verse{חבלים The lines נפלו are fallen לי בנעמים unto me in pleasant אף yea, נחלת heritage. שׁפרה a goodly עלי׃ I have}%
\verse{אברך I will bless את יהוה the LORD, אשׁר who יעצני hath given me counsel: אף also לילות me in the night seasons. יסרוני instruct כליותי׃ my reins}%
\verse{שׁויתי I have set יהוה the LORD לנגדי before תמיד always כי me: because מימיני at my right hand, בל I shall not אמוט׃ be moved.}%
\verse{לכן Therefore שׂמח is glad, לבי my heart ויגל rejoiceth: כבודי and my glory אף also בשׂרי my flesh ישׁכן shall rest לבטח׃ in hope.}%
\verse{כי For לא thou wilt not תעזב leave נפשׁי my soul לשׁאול in hell; לא neither תתן wilt thou suffer חסידך thine Holy One לראות to see שׁחת׃ corruption.}%
\verse{תודיעני Thou wilt show ארח me the path חיים of life: שׂבע fullness שׂמחות of joy; את פניך thy presence נעמות pleasures בימינך at thy right hand נצח׃ forevermore.}%
\end{biblechapter}%
\begin{biblechapter}% Psalm 17
\verseWithHeading{Prayer for Vindication and Protection}{תפלה A Prayer לדוד of David. שׁמעה Hear יהוה O LORD צדק the right, הקשׁיבה attend רנתי unto my cry, האזינה give ear תפלתי unto my prayer, בלא not שׂפתי lips. מרמה׃ out of feigned}%
\verse{מלפניך from thy presence; משׁפטי Let my sentence יצא come forth עיניך let thine eyes תחזינה behold מישׁרים׃ the things that are equal.}%
\verse{בחנת Thou hast proved לבי mine heart; פקדת thou hast visited לילה in the night; צרפתני thou hast tried בל nothing; תמצא me, shalt find זמתי I am purposed בל shall not יעבר transgress. פי׃ my mouth}%
\verse{לפעלות Concerning the works אדם of men, בדבר by the word שׂפתיך of thy lips אני I שׁמרתי have kept ארחות the paths פריץ׃ of the destroyer.}%
\verse{תמך Hold up אשׁרי my goings במעגלותיך in thy paths, בל not. נמוטו slip פעמי׃ my footsteps}%
\verse{אני I קראתיך have called upon כי thee, for תענני thou wilt hear אל me, O God: הט incline אזנך thine ear לי שׁמע unto me, and hear אמרתי׃ my speech.}%
\verse{הפלה חסדיךושׁיע O thou that savest חוסים them which put their trust ממתקוממים from those that rise up בימינך׃ by thy right hand}%
\verse{שׁמרני Keep כאישׁון me as the apple בת me as the apple עין of the eye, בצל me under the shadow כנפיך of thy wings, תסתירני׃ hide}%
\verse{מפני רשׁעים the wicked זו that שׁדוני oppress איבי enemies, בנפשׁ me, my deadly יקיפו compass me about. עלי׃ compass me about.}%
\verse{חלבמו their own fat: סגרו They are enclosed in פימו with their mouth דברו they speak בגאות׃ proudly.}%
\verse{אשׁרינו us in our steps: עתה They have now סבבוני compassed עיניהם their eyes ישׁיתו they have set לנטות bowing down בארץ׃ to the earth;}%
\verse{דמינו Like כאריה as a lion יכסוף is greedy לטרוף of his prey, וככפיר and as it were a young lion ישׁב lurking במסתרים׃ in secret places.}%
\verse{קומה Arise, יהוה O LORD, קדמה disappoint פניו disappoint הכריעהו him, cast him down: פלטה deliver נפשׁי my soul מרשׁע from the wicked, חרבך׃ thy sword:}%
\verse{ממתים from men ידך thy hand, יהוה O LORD, ממתים מחלד of the world, חלקם their portion בחיים in life, וצפינך תמלא thou fillest בטנם and whose belly ישׂבעו they are full בנים of children, והניחו יתרם the rest לעולליהם׃ of their to their babes.}%
\verse{אני As for me, בצדק in righteousness: אחזה I will behold פניך thy face אשׂבעה I shall be satisfied, בהקיץ when I awake, תמונתך׃ with thy likeness.}%
\end{biblechapter}%
\begin{biblechapter}% Psalm 18
\verseWithHeading{Praise to God for His Deliverance}{למנצח To the chief Musician, לעבד the servant יהוה of the LORD, לדוד of David, אשׁר who דבר spoke ליהוה unto the LORD את דברי the words השׁירה song הזאת of this ביום in the day הציל delivered יהוה the LORD אותו מכף him from the hand כל of all איביו his enemies, ומיד and from the hand שׁאול׃ of Saul: ויאמר And he said, ארחמך I will love יהוה thee, O LORD, חזקי׃ my strength.}%
\verse{יהוה The LORD סלעי my rock, ומצודתי and my fortress, ומפלטי and my deliverer; אלי my God, צורי my strength, אחסה in whom I will trust; בו מגני my buckler, וקרן and the horn ישׁעי of my salvation, משׂגבי׃ my high tower.}%
\verse{מהלל to be praised: אקרא I will call יהוה upon the LORD, ומן from איבי mine enemies. אושׁע׃ so shall I be saved}%
\verse{אפפוני compassed חבלי The sorrows מות of death ונחלי me, and the floods בליעל of ungodly men יבעתוני׃ made me afraid.}%
\verse{חבלי The sorrows שׁאול of hell סבבוני compassed me about: קדמוני prevented מוקשׁי the snares מות׃ of death}%
\verse{בצר In my distress לי אקרא I called upon יהוה the LORD, ואל unto אלהי my God: אשׁוע and cried ישׁמע he heard מהיכלו out of his temple, קולי my voice ושׁועתי and my cry לפניו before תבוא came באזניו׃ him, into his ears.}%
\verse{ותגעשׁ shook ותרעשׁ and trembled; הארץ Then the earth ומוסדי the foundations הרים also of the hills ירגזו moved ויתגעשׁו and were shaken, כי because חרה׃ he was wroth.}%
\verse{עלה There went up עשׁן a smoke באפו out of his nostrils, ואשׁ and fire מפיו out of his mouth תאכל devoured: גחלים coals בערו were kindled ממנו׃ out of his mouth}%
\verse{ויט He bowed שׁמים the heavens וירד also, and came down: וערפל and darkness תחת under רגליו׃ his feet.}%
\verse{וירכב And he rode על upon כרוב a cherub, ויעף and did fly: וידא yea, he did fly על upon כנפי the wings רוח׃ of the wind.}%
\verse{ישׁת He made חשׁך סתרו his secret place; סביבותיו round about סכתו his pavilion חשׁכת darkness מים waters עבי thick clouds שׁחקים׃ of the skies.}%
\verse{מנגה נגדו before עביו him his thick clouds עברו passed, ברד hail וגחלי and coals אשׁ׃ of fire.}%
\verse{וירעם also thundered בשׁמים in the heavens, יהוה The LORD ועליון and the Highest יתן gave קלו his voice; ברד hail וגחלי and coals אשׁ׃ of fire.}%
\verse{וישׁלח Yea, he sent out חציו his arrows, ויפיצם and scattered וברקים lightnings, רב them; and he shot out ויהמם׃ and discomfited}%
\verse{ויראו were seen, אפיקי Then the channels מים of waters ויגלו were discovered מוסדות and the foundations תבל of the world מגערתך at thy rebuke, יהוה O LORD, מנשׁמת at the blast רוח of the breath אפך׃ of thy nostrils.}%
\verse{ישׁלח He sent ממרום from above, יקחני he took ימשׁני me, he drew ממים רבים׃}%
\verse{יצילני He delivered מאיבי enemy, עז me from my strong ומשׂנאי and from them which hated כי me: for אמצו they were too strong ממני׃ enemy,}%
\verse{יקדמוני They prevented ביום me in the day אידי of my calamity: ויהי was יהוה but the LORD למשׁען׃ my stay.}%
\verse{ויוציאני He brought me forth למרחב also into a large place; יחלצני he delivered כי me, because חפץ׃ he delighted}%
\verse{יגמלני rewarded יהוה The LORD כצדקי me according to my righteousness; כבר according to the cleanness ידי of my hands ישׁיב׃ hath he recompensed}%
\verse{כי For שׁמרתי I have kept דרכי the ways יהוה of the LORD, ולא and have not רשׁעתי wickedly departed מאלהי׃}%
\verse{כי For כל all משׁפטיו his judgments לנגדי before וחקתיו his statutes לא me, and I did not אסיר put away מני׃ from}%
\verse{ואהי I was תמים also upright עמו before ואשׁתמר him, and I kept myself מעוני׃ from mine iniquity.}%
\verse{וישׁב recompensed יהוה Therefore hath the LORD לי כצדקי me according to my righteousness, כבר according to the cleanness ידי of my hands לנגד in עיניו׃ his eyesight.}%
\verse{עם With חסיד the merciful תתחסד thou wilt show thyself merciful; עם with גבר man תמים an upright תתמם׃ thou wilt show thyself upright;}%
\verse{עם With נבר the pure תתברר thou wilt show thyself pure; ועם and with עקשׁ the froward תתפתל׃ thou wilt show thyself froward.}%
\verse{כי For אתה thou wilt עם people; עני the afflicted תושׁיע save ועינים looks. רמות high תשׁפיל׃ but wilt bring down}%
\verse{כי For אתה thou תאיר wilt light נרי my candle: יהוה the LORD אלהי my God יגיה will enlighten חשׁכי׃ my darkness.}%
\verse{כי For בך ארץ by thee I have run through גדוד a troop; ובאלהי and by my God אדלג have I leaped over שׁור׃ a wall.}%
\verse{האל God, תמים perfect: דרכו his way אמרת the word יהוה of the LORD צרופה is tried: מגן a buckler הוא he לכל to all החסים׃ those that trust}%
\verse{כי For מי who אלוה God מבלעדי save יהוה the LORD? ומי or who צור a rock זולתי save אלהינו׃ our God?}%
\verse{האל God המאזרני that girdeth חיל me with strength, ויתן and maketh תמים perfect. דרכי׃ my way}%
\verse{משׁוה He maketh רגלי my feet כאילות like hinds' ועל me upon במתי my high places. יעמידני׃ and setteth}%
\verse{מלמד He teacheth ידי my hands למלחמה to war, ונחתה is broken קשׁת so that a bow נחושׁה of steel זרועתי׃ by mine arms.}%
\verse{ותתן Thou hast also given לי מגן me the shield ישׁעך of thy salvation: וימינך and thy right hand תסעדני hath holden me up, וענותך תרבני׃ hath made me great.}%
\verse{תרחיב Thou hast enlarged צעדי my steps תחתי under ולא did not מעדו slip. קרסלי׃ me, that my feet}%
\verse{ארדוף I have pursued אויבי mine enemies, ואשׂיגם and overtaken ולא them: neither אשׁוב did I turn again עד till כלותם׃ they were consumed.}%
\verse{אמחצם I have wounded ולא them that they were not יכלו able קום to rise: יפלו they are fallen תחת under רגלי׃ my feet.}%
\verse{ותאזרני For thou hast girded חיל me with strength למלחמה unto the battle: תכריע thou hast subdued קמי me those that rose up תחתי׃ under}%
\verse{ואיבי of mine enemies; נתתה Thou hast also given לי ערף me the necks ומשׂנאי them that hate אצמיתם׃ that I might destroy}%
\verse{ישׁועו They cried, ואין but none מושׁיע to save על unto יהוה the LORD, ולא them not. ענם׃ but he answered}%
\verse{ואשׁחקם Then did I beat them small כעפר as the dust על before פני before רוח the wind: כטיט as the dirt חוצות in the streets. אריקם׃ I did cast them out}%
\verse{תפלטני Thou hast delivered מריבי me from the strivings עם of the people; תשׂימני thou hast made לראשׁ me the head גוים of the heathen: עם a people לא I have not ידעתי known יעבדוני׃ shall serve}%
\verse{לשׁמע אזןשׁמעו of me, they shall obey לי בני me: the strangers נכר me: the strangers יכחשׁו׃ shall submit}%
\verse{בני נכרבלו shall fade away, ויחרגו and be afraid ממסגרותיהם׃ their close places.}%
\verse{חי liveth; יהוה The LORD וברוך and blessed צורי my rock; וירום be exalted. אלוהי ישׁעי׃ of my salvation}%
\verse{האל God הנותן that avengeth נקמות that avengeth לי וידבר me, and subdueth עמים the people תחתי׃ under}%
\verse{מפלטי He delivereth מאיבי me from mine enemies: אף yea, מן me from mine enemies: קמי those that rise up תרוממני thou liftest me up מאישׁ man. חמס me from the violent תצילני׃ against me: thou hast delivered}%
\verse{על כןודך will I give thanks בגוים among the heathen, יהוה unto thee, O LORD, ולשׁמך unto thy name. אזמרה׃ and sing praises}%
\verse{מגדל ישׁועות deliverance מלכו giveth he to his king; ועשׂה and showeth חסד mercy למשׁיחו to his anointed, לדוד to David, ולזרעו and to his seed עד forevermore. עולם׃ forevermore.}%
\end{biblechapter}%
\begin{biblechapter}% Psalm 19
\verseWithHeading{Adonai’s Creation and Law}{למנצח To the chief Musician, מזמור A Psalm לדוד׃ of David. השׁמים The heavens מספרים declare כבוד the glory אל of God; ומעשׂה his handiwork. ידיו his handiwork. מגיד showeth הרקיע׃ and the firmament}%
\verse{יום Day ליום unto day יביע uttereth אמר speech, ולילה and night ללילה unto night יחוה showeth דעת׃ knowledge.}%
\verse{אין no אמר speech ואין nor דברים language, בלי is not נשׁמע heard. קולם׃ their voice}%
\verse{בכל through all הארץ the earth, יצא is gone out קום Their line ובקצה to the end תבל of the world. מליהם and their words לשׁמשׁ for the sun, שׂם In them hath he set אהל׃ a tabernacle}%
\verse{והוא Which כחתן as a bridegroom יצא coming out מחפתו of his chamber, ישׂישׂ rejoiceth כגבור as a strong man לרוץ to run ארח׃ a race.}%
\verse{מקצה from the end השׁמים of the heaven, מוצאו His going forth ותקופתו and his circuit על unto קצותם the ends ואין of it: and there is nothing נסתר hid מחמתו׃ from the heat}%
\verse{תורת The law יהוה of the LORD תמימה perfect, משׁיבת converting נפשׁ the soul: עדות the testimony יהוה of the LORD נאמנה sure, מחכימת making wise פתי׃ the simple.}%
\verse{פקודי The statutes יהוה of the LORD ישׁרים right, משׂמחי rejoicing לב the heart: מצות the commandment יהוה of the LORD ברה pure, מאירת enlightening עינים׃ the eyes.}%
\verse{יראת The fear יהוה of the LORD טהורה clean, עומדת enduring לעד forever: משׁפטי the judgments יהוה of the LORD אמת true צדקו righteous יחדו׃ altogether.}%
\verse{הנחמדים More to be desired מזהב than gold, ומפז fine gold: רב yea, than much ומתוקים sweeter מדבשׁ also than honey ונפת and the honeycomb. צופים׃ and the honeycomb.}%
\verse{גם Moreover עבדך by them is thy servant נזהר warned: בהם בשׁמרם in keeping עקב reward. רב׃ of them great}%
\verse{שׁגיאות errors? מי Who יבין can understand מנסתרות thou me from secret נקני׃ cleanse}%
\verse{גם also מזדים from presumptuous חשׂך Keep back עבדך thy servant אל let them not ימשׁלו have dominion בי אז over me: then איתם shall I be upright, ונקיתי and I shall be innocent מפשׁע transgression. רב׃ from the great}%
\verse{יהיו be לרצון acceptable אמרי Let the words פי of my mouth, והגיון and the meditation לבי of my heart, לפניך in thy sight, יהוה O LORD, צורי my strength, וגאלי׃ and my redeemer.}%
\end{biblechapter}%
\begin{biblechapter}% Psalm 20
\verseWithHeading{God’s Blessing on the King}{למנצח To the chief Musician, מזמור A Psalm לדוד׃ of David. יענך hear יהוה The LORD ביום thee in the day צרה of trouble; ישׂגבך defend שׁם the name אלהי of the God יעקב׃ of Jacob}%
\verse{ישׁלח Send עזרך thee help מקדשׁ from the sanctuary, ומציון יסעדך׃ and strengthen}%
\verse{יזכר Remember כל all מנחתך thy offerings, ועולתך thy burnt sacrifice; ידשׁנה and accept סלה׃ Selah.}%
\verse{יתן Grant לך כלבבך thee according to thine own heart, וכל all עצתך thy counsel. ימלא׃ and fulfill}%
\verse{נרננה We will rejoice בישׁועתך in thy salvation, ובשׁם and in the name אלהינו of our God נדגל we will set up banners: ימלא fulfill יהוה the LORD כל all משׁאלותיך׃ thy petitions.}%
\verse{עתה Now ידעתי know כי I that הושׁיע saveth יהוה the LORD משׁיחו his anointed; יענהו he will hear משׁמי heaven קדשׁו him from his holy בגברות strength ישׁע with the saving ימינו׃ of his right hand.}%
\verse{אלה Some ברכב in chariots, ואלה and some בסוסים in horses: ואנחנו but we בשׁם the name יהוה of the LORD אלהינו our God. נזכיר׃ will remember}%
\verse{המה They כרעו are brought down ונפלו and fallen: ואנחנו but we קמנו are risen, ונתעודד׃ and stand upright.}%
\verse{יהוה LORD: הושׁיעה Save, המלך let the king יעננו hear ביום us when קראנו׃ we call.}%
\end{biblechapter}%
\begin{biblechapter}% Psalm 21
\verseWithHeading{Joy in the Salvation of Adonai}{למנצח To the chief Musician, מזמור A Psalm לדוד׃ of David. יהוה O LORD; בעזך in thy strength, ישׂמח shall joy מלך The king ובישׁועתך and in thy salvation מה how יגיל shall he rejoice! מאד׃ greatly}%
\verse{תאות desire, לבו him his heart's נתתה Thou hast given לו וארשׁת the request שׂפתיו of his lips. בל and hast not מנעת withheld סלה׃ Selah.}%
\verse{כי For תקדמנו thou preventest ברכות him with the blessings טוב of goodness: תשׁית thou settest לראשׁו on his head. עטרת a crown פז׃ of pure gold}%
\verse{חיים life שׁאל He asked ממך of נתתה thee, thou gavest לו ארך him, length ימים of days עולם forever ועד׃ and ever.}%
\verse{גדול great כבודו His glory בישׁועתך in thy salvation: הוד honor והדר and majesty תשׁוה hast thou laid עליו׃ upon}%
\verse{כי For תשׁיתהו thou hast made ברכות him most blessed לעד forever: תחדהו thou hast made him exceeding glad בשׂמחה thou hast made him exceeding glad את with פניך׃ thy countenance.}%
\verse{כי For המלך the king בטח trusteth ביהוה in the LORD, ובחסד and through the mercy עליון of the most High בל he shall not ימוט׃ be moved.}%
\verse{תמצא shall find out ידך Thine hand לכל all איביך thine enemies: ימינך thy right hand תמצא shall find out שׂנאיך׃ those that hate}%
\verse{תשׁיתמו Thou shalt make כתנור oven אשׁ them as a fiery לעת in the time פניך of thine anger: יהוה the LORD באפו in his wrath, יבלעם shall swallow them up ותאכלם shall devour אשׁ׃ and the fire}%
\verse{פרימו Their fruit מארץ from the earth, תאבד shalt thou destroy וזרעם and their seed מבני from among the children אדם׃ of men.}%
\verse{כי For נטו they intended עליך against רעה evil חשׁבו thee: they imagined מזמה a mischievous device, בל they are not יוכלו׃ able}%
\verse{כי Therefore תשׁיתמו shalt thou make שׁכם them turn their back, במיתריך upon thy strings תכונן thou shalt make ready על against פניהם׃ the face}%
\verse{רומה Be thou exalted, יהוה LORD, בעזך in thine own strength: נשׁירה will we sing ונזמרה and praise גבורתך׃ thy power.}%
\end{biblechapter}%
\begin{biblechapter}% Psalm 22
\verseWithHeading{Suffering and Waiting for Deliverance}{למנצח To the chief Musician על upon אילת Aijeleth השׁחר Shahar, מזמור A Psalm לדוד׃ of David. אלי My God, אלי my God, למה why עזבתני hast thou forsaken רחוק me? far מישׁועתי from helping דברי me, the words שׁאגתי׃ of my roaring?}%
\verse{אלהי O my God, אקרא I cry יומם in the daytime, ולא not; תענה but thou hearest ולילה and in the night season, ולא and am not דומיה׃ silent.}%
\verse{ואתה But thou קדושׁ holy, יושׁב that inhabitest תהלות the praises ישׂראל׃ of Israel.}%
\verse{בך בטחו trusted אבתינו Our fathers בטחו in thee: they trusted, ותפלטמו׃ and thou didst deliver}%
\verse{אליך unto זעקו They cried ונמלטו thee, and were delivered: בך בטחו they trusted ולא in thee, and were not בושׁו׃ confounded.}%
\verse{ואנכי But I תולעת a worm, ולא and no אישׁ man; חרפת a reproach אדם of men, ובזוי and despised עם׃ of the people.}%
\verse{כל All ראי they that see ילעגו me laugh לי יפטירו me to scorn: they shoot out בשׂפה the lip, יניעו they shake ראשׁ׃ the head,}%
\verse{גל He trusted אל on יהוה the LORD יפלטהו he would deliver יצילהו him: let him deliver כי him, seeing חפץ׃ he delighted}%
\verse{כי But אתה thou גחי he that took מבטן me out of the womb: מבטיחי thou didst make me hope על upon שׁדי breasts. אמי׃ my mother's}%
\verse{עליך upon השׁלכתי I was cast מרחם thee from the womb: מבטן belly. אמי from my mother's אלי my God אתה׃ thou}%
\verse{אל Be not תרחק far ממני from כי me; for צרה trouble קרובה near; כי for אין none עוזר׃ to help.}%
\verse{סבבוני have compassed פרים bulls רבים Many אבירי me: strong בשׁן of Bashan כתרוני׃ have beset me round.}%
\verse{פצו They gaped עלי upon פיהם me their mouths, אריה lion. טרף a ravening ושׁאג׃ and a roaring}%
\verse{כמים like water, נשׁפכתי I am poured out והתפרדו are out of joint: כל and all עצמותי my bones היה is לבי my heart כדונג like wax; נמס it is melted בתוך in the midst מעי׃ of my bowels.}%
\verse{יבשׁ is dried up כחרשׂ like a potsherd; כחי My strength ולשׁוני and my tongue מדבק cleaveth מלקוחי to my jaws; ולעפר me into the dust מות of death. תשׁפתני׃ and thou hast brought}%
\verse{כי For סבבוני have compassed כלבים dogs עדת me: the assembly מרעים of the wicked הקיפוני have enclosed כארי ידי my hands ורגלי׃ and my feet.}%
\verse{אספר I may tell כל all עצמותי my bones: המה they יביטו look יראו׃ stare}%
\verse{יחלקו They part בגדי my garments להם ועל upon לבושׁי my vesture. יפילו among them, and cast גורל׃ lots}%
\verse{ואתה thou יהוה from me, O LORD: אל But be not תרחק far אילותי O my strength, לעזרתי thee to help חושׁה׃ haste}%
\verse{הצילה Deliver מחרב from the sword; נפשׁי my soul מיד from the power כלב of the dog. יחידתי׃ my darling}%
\verse{הושׁיעני Save מפי mouth: אריה me from the lion's ומקרני me from the horns רמים of the unicorns. עניתני׃ for thou hast heard}%
\verse{אספרה I will declare שׁמך thy name לאחי unto my brethren: בתוך in the midst קהל of the congregation אהללך׃ will I praise}%
\verse{יראי Ye that fear יהוה the LORD, הללוהו praise כל him; all זרע ye the seed יעקב of Jacob, כבדוהו glorify וגורו him; and fear ממנו him; and fear כל him, all זרע ye the seed ישׂראל׃ of Israel.}%
\verse{כי For לא he hath not בזה despised ולא nor שׁקץ abhorred ענות the affliction עני of the afflicted; ולא neither הסתיר hath he hid פניו his face ממנו from ובשׁועו him; but when he cried אליו unto שׁמע׃ him, he heard.}%
\verse{מאתך תהלתי My praise בקהל congregation: רב thee in the great נדרי my vows אשׁלם I will pay נגד before יראיו׃ them that fear}%
\verse{יאכלו shall eat ענוים The meek וישׂבעו and be satisfied: יהללו they shall praise יהוה the LORD דרשׁיו that seek יחי shall live לבבכם him: your heart לעד׃ forever.}%
\verse{יזכרו shall remember וישׁבו and turn אל unto יהוה the LORD: כל All אפסי the ends ארץ of the world וישׁתחוו shall worship לפניך before כל and all משׁפחות the kindreds גוים׃ of the nations}%
\verse{כי For ליהוה the LORD's: המלוכה the kingdom ומשׁל and he the governor בגוים׃ among the nations.}%
\verse{אכלו shall eat וישׁתחוו and worship: כל All דשׁני fat ארץ upon earth לפניו before יכרעו shall bow כל all יורדי they that go down עפר to the dust ונפשׁו his own soul. לא him: and none חיה׃ can keep alive}%
\verse{זרע A seed יעבדנו shall serve יספר him; it shall be accounted לאדני to the Lord לדור׃ for a generation.}%
\verse{יבאו They shall come, ויגידו and shall declare צדקתו his righteousness לעם unto a people נולד that shall be born, כי that עשׂה׃ he hath done}%
\end{biblechapter}%
\begin{biblechapter}% Psalm 23
\verseWithHeading{Adonai the Shepherd}{מזמור A Psalm לדוד of David. יהוה The LORD רעי my shepherd; לא I shall not אחסר׃ want.}%
\verse{בנאות pastures: דשׁא in green ירביצני He maketh me to lie down על me beside מי waters. מנחות the still ינהלני׃ he leadeth}%
\verse{נפשׁי my soul: ישׁובב He restoreth ינחני he leadeth במעגלי me in the paths צדק of righteousness למען שׁמו׃}%
\verse{גם Yea, כי though אלך I walk בגיא through the valley צלמות of the shadow of death, לא no אירא I will fear רע evil: כי for אתה thou עמדי with שׁבטך me; thy rod ומשׁענתך and thy staff המה they ינחמני׃ comfort}%
\verse{תערך Thou preparest לפני before שׁלחן a table נגד me in the presence צררי of mine enemies: דשׁנת thou anointest בשׁמן with oil; ראשׁי my head כוסי my cup רויה׃ runneth over.}%
\verse{אך Surely טוב goodness וחסד and mercy ירדפוני shall follow כל me all ימי the days חיי of my life: ושׁבתי and I will dwell בבית in the house יהוה of the LORD לארך forever. ימים׃ forever.}%
\end{biblechapter}%
\begin{biblechapter}% Psalm 24
\verseWithHeading{The King of Glory}{לדוד of David. מזמור A Psalm ליהוה the LORD's, הארץ The earth ומלואה and the fullness תבל thereof; the world, וישׁבי׃ and they that dwell}%
\verse{כי For הוא he על it upon ימים the seas, יסדה hath founded ועל it upon נהרות the floods. יכוננה׃ and established}%
\verse{מי Who יעלה shall ascend בהר into the hill יהוה of the LORD? ומי or who יקום shall stand במקום place? קדשׁו׃ in his holy}%
\verse{נקי He that hath clean כפים hands, ובר and a pure לבב heart; אשׁר who לא hath not נשׂא lifted up לשׁוא unto vanity, נפשׁי his soul ולא nor נשׁבע sworn למרמה׃ deceitfully.}%
\verse{ישׂא He shall receive ברכה the blessing מאת יהוה the LORD, וצדקה and righteousness מאלהי ישׁעו׃ of his salvation.}%
\verse{זה This דור the generation דרשׁו of them that seek מבקשׁי him, that seek פניך thy face, יעקב O Jacob. סלה׃ Selah.}%
\verse{שׂאו Lift up שׁערים O ye gates; ראשׁיכם your heads, והנשׂאו and be ye lift up, פתחי doors; עולם ye everlasting ויבוא shall come in. מלך and the King הכבוד׃ of glory}%
\verse{מי Who זה this מלך King הכבוד of glory? יהוה The LORD עזוז strong וגבור and mighty, יהוה the LORD גבור mighty מלחמה׃ in battle.}%
\verse{שׂאו Lift up שׁערים O ye gates; ראשׁיכם your heads, ושׂאו even lift up, פתחי doors; עולם ye everlasting ויבא shall come in. מלך and the King הכבוד׃ of glory}%
\verse{מי Who הוא is זה this מלך King הכבוד of glory? יהוה The LORD צבאות of hosts, הוא he מלך the King הכבוד of glory. סלה׃ Selah.}%
\end{biblechapter}%
\begin{biblechapter}% Psalm 25
\verseWithHeading{A Prayer for Guidance, Deliverance, and Forgiveness}{לדוד of David. אליך Unto יהוה thee, O LORD, נפשׁי my soul. אשׂא׃ do I lift up}%
\verse{אלהי O my God, בך בטחתי I trust אל in thee: let me not אבושׁה be ashamed, אל let not יעלצו triumph איבי׃ mine enemies}%
\verse{גם Yea, כל let none קויך that wait לא let none יבשׁו on thee be ashamed: יבשׁו let them be ashamed הבוגדים which transgress ריקם׃ without cause.}%
\verse{דרכיך me thy ways, יהוה O LORD; הודיעני Show ארחותיך me thy paths. למדני׃ teach}%
\verse{הדריכני Lead באמתך me in thy truth, ולמדני and teach כי me: for אתה thou אלהי the God ישׁעי of my salvation; אותך קויתי on thee do I wait כל all היום׃ the day.}%
\verse{זכר Remember, רחמיך thy tender mercies יהוה O LORD, וחסדיך and thy lovingkindnesses; כי for מעולם ever of old. המה׃ they}%
\verse{חטאות the sins נעורי of my youth, ופשׁעי nor my transgressions: אל not תזכר Remember כחסדך according to thy mercy זכר remember לי אתה thou למען טובךהוה׃ O LORD.}%
\verse{טוב Good וישׁר and upright יהוה the LORD: על therefore כן therefore יורה will he teach חטאים sinners בדרך׃ in the way.}%
\verse{ידרך will he guide ענוים The meek במשׁפט in judgment: וילמד will he teach ענוים and the meek דרכו׃ his way.}%
\verse{כל All ארחות the paths יהוה of the LORD חסד mercy ואמת and truth לנצרי unto such as keep בריתו his covenant ועדתיו׃ and his testimonies.}%
\verse{למען שׁמךהוה O LORD, וסלחת pardon לעוני mine iniquity; כי for רב great. הוא׃ it}%
\verse{מי What זה he that האישׁ man ירא feareth יהוה the LORD? יורנו him shall he teach בדרך in the way יבחר׃ he shall choose.}%
\verse{נפשׁו His soul בטוב at ease; תלין shall dwell וזרעו and his seed יירשׁ shall inherit ארץ׃ the earth.}%
\verse{סוד The secret יהוה of the LORD ליראיו with them that fear ובריתו them his covenant. להודיעם׃ him; and he will show}%
\verse{עיני Mine eyes תמיד ever אל toward יהוה the LORD; כי for הוא he יוציא shall pluck my feet out מרשׁת of the net. רגלי׃ shall pluck my feet out}%
\verse{פנה Turn אלי thee unto וחנני me, and have mercy כי upon me; for יחיד desolate ועני and afflicted. אני׃ I}%
\verse{צרות The troubles לבבי of my heart הרחיבו are enlarged: ממצוקותי of my distresses. הוציאני׃ bring thou me out}%
\verse{ראה Look upon עניי mine affliction ועמלי and my pain; ושׂא and forgive לכל all חטאותי׃ my sins.}%
\verse{ראה Consider אויבי mine enemies; כי for רבו they are many; ושׂנאת and they hate חמס me with cruel שׂנאוני׃ hatred.}%
\verse{שׁמרה O keep נפשׁי my soul, והצילני and deliver אל me: let me not אבושׁ be ashamed; כי for חסיתי׃ I put my trust}%
\verse{תם Let integrity וישׁר and uprightness יצרוני preserve כי me; for קויתיך׃ I wait on}%
\verse{פדה Redeem אלהים O God, את ישׂראל Israel, מכל out of all צרותיו׃ his troubles.}%
\end{biblechapter}%
\begin{biblechapter}% Psalm 26
\verseWithHeading{A Prayer for Vindication}{לדוד of David. שׁפטני Judge יהוה me, O LORD; כי for אני I בתמי in mine integrity: הלכתי have walked וביהוה also in the LORD; בטחתי I have trusted לא I shall not אמעד׃ slide.}%
\verse{בחנני Examine יהוה me, O LORD, ונסני and prove צרופה me; try כליותי my reins ולבי׃ and my heart.}%
\verse{כי For חסדך thy lovingkindness לנגד before עיני mine eyes: והתהלכתי and I have walked באמתך׃ in thy truth.}%
\verse{לא I have not ישׁבתי sat עם with מתי persons, שׁוא vain ועם with נעלמים dissemblers. לא neither אבוא׃ will I go in}%
\verse{שׂנאתי I have hated קהל the congregation מרעים of evildoers; ועם with רשׁעים the wicked. לא and will not אשׁב׃ sit}%
\verse{ארחץ I will wash בנקיון in innocency: כפי mine hands ואסבבה so will I compass את מזבחך thine altar, יהוה׃ O LORD:}%
\verse{לשׁמע That I may publish בקול with the voice תודה of thanksgiving, ולספר and tell כל of all נפלאותיך׃ thy wondrous works.}%
\verse{יהוה LORD, אהבתי I have loved מעון the habitation ביתך of thy house, ומקום and the place משׁכן dwelleth. כבודך׃ where thine honor}%
\verse{אל not תאסף Gather עם with חטאים sinners, נפשׁי my soul ועם with אנשׁי men: דמים bloody חיי׃ nor my life}%
\verse{אשׁר In whose בידיהם hands זמה mischief, וימינם and their right hand מלאה is full שׁחד׃ of bribes.}%
\verse{ואני But as for me, בתמי in mine integrity: אלך I will walk פדני redeem וחנני׃ me, and be merciful}%
\verse{רגלי My foot עמדה standeth במישׁור in an even place: במקהלים in the congregations אברך will I bless יהוה׃ the LORD.}%
\end{biblechapter}%
\begin{biblechapter}% Psalm 27
\verseWithHeading{A Declaration of Trust}{לדוד of David. יהוה The LORD אורי my light וישׁעי and my salvation; ממי whom אירא shall I fear? יהוה the LORD מעוז the strength חיי of my life; ממי of whom אפחד׃ shall I be afraid?}%
\verse{בקרב came עלי upon מרעים When the wicked, לאכל me to eat up את בשׂרי my flesh, צרי mine enemies ואיבי and my foes, לי המה they כשׁלו stumbled ונפלו׃ and fell.}%
\verse{אם Though תחנה should encamp עלי against מחנה a host לא shall not יירא fear: לבי me, my heart אם though תקום should rise עלי against מלחמה war בזאת me, in this אני I בוטח׃ confident.}%
\verse{אחת One שׁאלתי have I desired מאת יהוה the LORD, אותה אבקשׁ that will I seek after; שׁבתי that I may dwell בבית in the house יהוה of the LORD כל all ימי the days חיי of my life, לחזות to behold בנעם the beauty יהוה of the LORD, ולבקר and to inquire בהיכלו׃ in his temple.}%
\verse{כי For יצפנני he shall hide בסכה me in his pavilion: ביום in the time רעה of trouble יסתרני shall he hide בסתר in the secret אהלו of his tabernacle בצור upon a rock. ירוממני׃ me; he shall set me up}%
\verse{ועתה And now ירום be lifted up ראשׁי shall mine head על above איבי mine enemies סביבותי round about ואזבחה me: therefore will I offer באהלו in his tabernacle זבחי sacrifices תרועה of joy; אשׁירה I will sing, ואזמרה yea, I will sing praises ליהוה׃ unto the LORD.}%
\verse{שׁמע Hear, יהוה O LORD, קולי with my voice: אקרא I cry וחנני have mercy וענני׃ also upon me, and answer}%
\verse{לך אמר said לבי my heart בקשׁו Seek פני ye my face; את פניך unto thee, Thy face, יהוה LORD, אבקשׁ׃ will I seek.}%
\verse{אל not תסתר Hide פניך thy face ממני from אל me not, תט me; put not thy servant away באף in anger: עבדך עזרתי my help; היית thou hast been אל neither תטשׁני leave ואל תעזבני forsake אלהי me, O God ישׁעי׃ of my salvation.}%
\verse{כי When אבי my father ואמי and my mother עזבוני forsake ויהוה me, then the LORD יאספני׃ will take me up.}%
\verse{הורני Teach יהוה O LORD, דרכך me thy way, ונחני and lead בארח path, מישׁור me in a plain למען because of שׁוררי׃ mine enemies.}%
\verse{אל me not תתנני Deliver בנפשׁ over unto the will צרי of mine enemies: כי for קמו are risen up בי עדי witnesses שׁקר false ויפח against me, and such as breathe out חמס׃ cruelty.}%
\verse{לולא unless האמנתי I had believed לראות to see בטוב the goodness יהוה of the LORD בארץ in the land חיים׃ of the living.}%
\verse{קוה Wait אל on יהוה the LORD: חזק be of good courage, ויאמץ and he shall strengthen לבך thine heart: וקוה wait, אל I say, on יהוה׃ the LORD.}%
\end{biblechapter}%
\begin{biblechapter}% Psalm 28
\verseWithHeading{A Prayer for Help, and Joy in Its Answer}{לדוד of David. אליך Unto יהוה O LORD אקרא thee will I cry, צורי my rock; אל be not תחרשׁ silent ממני to פן me: lest, תחשׁה thou be silent ממני to ונמשׁלתי עםורדי them that go down בור׃ into the pit.}%
\verse{שׁמע Hear קול the voice תחנוני of my supplications, בשׁועי when I cry אליך unto בנשׂאי thee, when I lift up ידי my hands אל toward דביר oracle. קדשׁך׃ thy holy}%
\verse{אל me not תמשׁכני Draw עם away with רשׁעים the wicked, ועם and with פעלי the workers און of iniquity, דברי which speak שׁלום peace עם to רעיהם their neighbors, ורעה but mischief בלבבם׃ in their hearts.}%
\verse{תן Give להם כפעלם them according to their deeds, וכרע and according to the wickedness מעלליהם of their endeavors: כמעשׂה them after the work ידיהם of their hands; תן give להם השׁב render גמולם to them their desert. להם׃}%
\verse{כי Because לא not יבינו they regard אל they regard פעלת the works יהוה of the LORD, ואל מעשׂה nor the operation ידיו of his hands, יהרסם he shall destroy ולא them, and not יבנם׃ build them up.}%
\verse{ברוך Blessed יהוה the LORD, כי because שׁמע he hath heard קול the voice תחנוני׃ of my supplications.}%
\verse{יהוה The LORD עזי my strength ומגני and my shield; בו בטח trusted לבי my heart ונעזרתי in him, and I am helped: ויעלז greatly rejoiceth; לבי therefore my heart ומשׁירי and with my song אהודנו׃ will I praise}%
\verse{יהוה The LORD עז their strength, למו ומעוז strength ישׁועות the saving משׁיחו of his anointed. הוא׃ and he}%
\verse{הושׁיעה Save את עמך thy people, וברך and bless את נחלתך thine inheritance: ורעם feed ונשׂאם them also, and lift them up עד forever. העולם׃ forever.}%
\end{biblechapter}%
\begin{biblechapter}% Psalm 29
\verseWithHeading{Praise to God for His Glory and Strength}{מזמור A Psalm לדוד of David. הבו Give ליהוה unto the LORD, בני אליםבו give ליהוה unto the LORD כבוד glory ועז׃ and strength.}%
\verse{הבו Give ליהוה unto the LORD כבוד the glory שׁמו due unto his name; השׁתחוו worship ליהוה the LORD בהדרת in the beauty קדשׁ׃ of holiness.}%
\verse{קול The voice יהוה of the LORD על upon המים the waters: אל the God הכבוד of glory הרעים thundereth: יהוה the LORD על upon מים waters. רבים׃ many}%
\verse{קול The voice יהוה of the LORD בכח powerful; קול the voice יהוה of the LORD בהדר׃ full of majesty.}%
\verse{קול The voice יהוה of the LORD שׁבר breaketh ארזים the cedars; וישׁבר breaketh יהוה yea, the LORD את ארזי the cedars הלבנון׃ of Lebanon.}%
\verse{וירקידם He maketh them also to skip כמו like עגל a calf; לבנון Lebanon ושׂרין and Sirion כמו like בן a young ראמים׃ unicorn.}%
\verse{קול The voice יהוה of the LORD חצב divideth להבות the flames אשׁ׃ of fire.}%
\verse{קול The voice יהוה of the LORD יחיל shaketh מדבר the wilderness; יחיל shaketh יהוה the LORD מדבר the wilderness קדשׁ׃ of Kadesh.}%
\verse{קול The voice יהוה of the LORD יחולל to calve, אילות maketh the hinds ויחשׂף and discovereth יערות the forests: ובהיכלו and in his temple כלו doth every one אמר speak of כבוד׃ glory.}%
\verse{יהוה The LORD למבול upon the flood; ישׁב sitteth וישׁב sitteth יהוה yea, the LORD מלך King לעולם׃ forever.}%
\verse{יהוה The LORD עז strength לעמו unto his people; יתן will give יהוה the LORD יברך will bless את עמו his people בשׁלום׃ with peace.}%
\end{biblechapter}%
\begin{biblechapter}% Psalm 30
\verseWithHeading{Thanksgiving for Answered Prayer}{מזמור A Psalm שׁיר Song חנכת the dedication הבית of the house לדוד׃ of David. ארוממך I will extol יהוה thee, O LORD; כי for דליתני thou hast lifted me up, ולא and hast not שׂמחת to rejoice איבי made my foes}%
\verse{יהוה O LORD אלהי my God, שׁועתי I cried אליך unto ותרפאני׃ thee, and thou hast healed}%
\verse{יהוה O LORD, העלית thou hast brought up מן from שׁאול the grave: נפשׁי my soul חייתני thou hast kept me alive, מיורדי בור׃}%
\verse{זמרו Sing ליהוה unto the LORD, חסידיו O ye saints והודו of his, and give thanks לזכר at the remembrance קדשׁו׃ of his holiness.}%
\verse{כי For רגע a moment; באפו his anger חיים life: ברצונו in his favor בערב for a night, ילין may endure בכי weeping ולבקר in the morning. רנה׃ but joy}%
\verse{ואני I אמרתי said, בשׁלוי And in my prosperity בל אמוט be moved. לעולם׃}%
\verse{יהוה LORD, ברצונך by thy favor העמדתה to stand להררי thou hast made my mountain עז strong: הסתרת thou didst hide פניך thy face, הייתי I was נבהל׃ troubled.}%
\verse{אליך to יהוה thee, O LORD; אקרא I cried ואל and unto אדני the LORD אתחנן׃ I made supplication.}%
\verse{מה What בצע profit בדמי in my blood, ברדתי when I go down אל to שׁחת the pit? היודך praise עפר Shall the dust היגיד thee? shall it declare אמתך׃ thy truth?}%
\verse{שׁמע Hear, יהוה O LORD, וחנני and have mercy יהוה upon me: LORD, היה be עזר׃ thou my helper.}%
\verse{הפכת Thou hast turned מספדי for me my mourning למחול into dancing: לי פתחת thou hast put off שׂקי my sackcloth, ותאזרני and girded שׂמחה׃ me with gladness;}%
\verse{למען To the end that יזמרך may sing praise כבוד glory ולא to thee, and not ידם be silent. יהוה O LORD אלהי my God, לעולם unto thee forever. אודך׃ I will give thanks}%
\end{biblechapter}%
\begin{biblechapter}% Psalm 31
\verseWithHeading{Adonai Is a Fortress}{למנצח To the chief Musician, מזמור A Psalm לדוד׃ of David. בך יהוה In thee, O LORD, חסיתי do I put my trust; אל let me never אבושׁה be ashamed: לעולם let me never בצדקתך me in thy righteousness. פלטני׃ deliver}%
\verse{הטה Bow down אלי to אזנך thine ear מהרה me speedily: הצילני me; deliver היה be לי לצור rock, מעוז thou my strong לבית for a house מצודות of defense להושׁיעני׃ to save}%
\verse{כי For סלעי my rock ומצודתי and my fortress; אתה thou ולמען שׁמךנחני lead ותנהלני׃ me, and guide}%
\verse{תוציאני Pull me out מרשׁת of the net זו that טמנו they have laid privily לי כי for me: for אתה thou מעוזי׃ my strength.}%
\verse{בידך Into thine hand אפקיד I commit רוחי my spirit: פדיתה thou hast redeemed אותי יהוה me, O LORD אל God אמת׃ of truth.}%
\verse{שׂנאתי I have hated השׁמרים them that regard הבלי vanities: שׁוא lying ואני but I אל in יהוה the LORD. בטחתי׃ trust}%
\verse{אגילה I will be glad ואשׂמחה and rejoice בחסדך in thy mercy: אשׁר for ראית thou hast considered את עניי my trouble; ידעת thou hast known בצרות in adversities; נפשׁי׃ my soul}%
\verse{ולא And hast not הסגרתני shut me up ביד into the hand אויב of the enemy: העמדת thou hast set במרחב in a large room. רגלי׃ my feet}%
\verse{חנני Have mercy יהוה upon me, O LORD, כי for צר לי עשׁשׁה is consumed בכעס with grief, עיני mine eye נפשׁי my soul ובטני׃ and my belly.}%
\verse{כי For כלו is spent ביגון with grief, חיי my life ושׁנותי and my years באנחה with sighing: כשׁל faileth בעוני because of mine iniquity, כחי my strength ועצמי and my bones עשׁשׁו׃ are consumed.}%
\verse{מכל among all צררי mine enemies, הייתי I was חרפה a reproach ולשׁכני among my neighbors, מאד but especially ופחד and a fear למידעי to mine acquaintance: ראי they that did see בחוץ me without נדדו fled ממני׃ among all}%
\verse{נשׁכחתי I am forgotten כמת as a dead man מלב out of mind: הייתי I am ככלי vessel. אבד׃ like a broken}%
\verse{כי For שׁמעתי I have heard דבת the slander רבים of many: מגור fear מסביב on every side: בהוסדם while they took counsel יחד together עלי against לקחת to take away נפשׁי my life. זממו׃ me, they devised}%
\verse{ואני But I עליך in בטחתי trusted יהוה thee, O LORD: אמרתי I said, אלהי my God. אתה׃ Thou}%
\verse{בידך in thy hand: עתתי My times הצילני deliver מיד me from the hand אויבי of mine enemies, ומרדפי׃ and from them that persecute}%
\verse{האירה to shine פניך Make thy face על upon עבדך thy servant: הושׁיעני save בחסדך׃ me for thy mercies'}%
\verse{יהוה O LORD; אל Let me not אבושׁה be ashamed, כי for קראתיך I have called upon יבשׁו be ashamed, רשׁעים thee: let the wicked ידמו let them be silent לשׁאול׃ in the grave.}%
\verse{תאלמנה be put to silence; שׂפתי lips שׁקר Let the lying הדברות which speak על against צדיק the righteous. עתק grievous things בגאוה proudly ובוז׃ and contemptuously}%
\verse{מה how רב great טובך thy goodness, אשׁר which צפנת thou hast laid up ליראיך for them that fear פעלת thee; thou hast wrought לחסים for them that trust בך נגד in thee before בני the sons אדם׃ of men!}%
\verse{תסתירם Thou shalt hide בסתר them in the secret פניך of thy presence מרכסי from the pride אישׁ of man: תצפנם thou shalt keep them secretly בסכה in a pavilion מריב from the strife לשׁנות׃ of tongues.}%
\verse{ברוך Blessed יהוה כי for הפליא he hath showed me his marvelous kindness חסדו he hath showed me his marvelous kindness לי בעיר city. מצור׃ in a strong}%
\verse{ואני For I אמרתי said בחפזי in my haste, נגרזתי I am cut off מנגד from before עיניך thine eyes: אכן nevertheless שׁמעת thou heardest קול the voice תחנוני of my supplications בשׁועי when I cried אליך׃ unto}%
\verse{אהבו O love את יהוה the LORD, כל all חסידיו ye his saints: אמונים the faithful, נצר preserveth יהוה the LORD ומשׁלם rewardeth על and plentifully יתר and plentifully עשׂה doer. גאוה׃ the proud}%
\verse{חזקו Be of good courage, ויאמץ and he shall strengthen לבבכם your heart, כל all המיחלים ye that hope ליהוה׃ in the LORD.}%
\end{biblechapter}%
\begin{biblechapter}% Psalm 32
\verseWithHeading{Thanksgiving for Forgiveness of Sins}{לדוד of David, משׂכיל Maschil. אשׁרי Blessed נשׂוי forgiven, פשׁע transgression כסוי covered. חטאה׃ sin}%
\verse{אשׁרי Blessed אדם the man לא not יחשׁב imputeth יהוה unto whom the LORD לו עון iniquity, ואין no ברוחו and in whose spirit רמיה׃ guile.}%
\verse{כי When החרשׁתי I kept silence, בלו waxed old עצמי my bones בשׁאגתי through my roaring כל all היום׃ the day}%
\verse{כי For יומם day ולילה and night תכבד was heavy עלי upon ידך thy hand נהפך is turned לשׁדי me: my moisture בחרבני into the drought קיץ of summer. סלה׃ Selah.}%
\verse{חטאתי my sin אודיעך I acknowledged ועוני unto thee, and mine iniquity לא have I not כסיתי hid. אמרתי I said, אודה I will confess עלי my transgressions פשׁעי my transgressions ליהוה unto the LORD; ואתה and thou נשׂאת forgavest עון the iniquity חטאתי of my sin. סלה׃ Selah.}%
\verse{על For זאת this יתפלל pray כל shall every חסיד one that is godly אליך unto לעת thee in a time מצא when thou mayest be found: רק surely לשׁטף in the floods מים waters רבים of great אליו unto לא they shall not יגיעו׃ come nigh}%
\verse{אתה Thou סתר my hiding place; לי מצר me from trouble; תצרני thou shalt preserve רני with songs פלט of deliverance. תסובבני thou shalt compass me about סלה׃ Selah.}%
\verse{אשׂכילך I will instruct ואורך thee and teach בדרך thee in the way זו which תלך thou shalt go: איעצה עליךיני׃ thee with mine eye.}%
\verse{אל ye not תהיו Be כסוס as the horse, כפרד as the mule, אין have no הבין understanding: במתג with bit ורסן and bridle, עדיו whose mouth לבלום must be held in בל lest קרב they come near אליך׃ unto thee.}%
\verse{רבים Many מכאובים sorrows לרשׁע to the wicked: והבוטח but he that trusteth ביהוה in the LORD, חסד mercy יסובבנו׃ shall compass him about.}%
\verse{שׂמחו Be glad ביהוה in the LORD, וגילו and rejoice, צדיקים ye righteous: והרנינו and shout for joy, כל all ישׁרי upright לב׃ in heart.}%
\end{biblechapter}%
\begin{biblechapter}% Psalm 33
\verseWithHeading{Praise to Adonai for His Character and Creation}{רננו Rejoice צדיקים O ye righteous: ביהוה in the LORD, לישׁרים for the upright. נאוה is comely תהלה׃ praise}%
\verse{הודו Praise ליהוה the LORD בכנור with harp: בנבל unto him with the psaltery עשׂור an instrument of ten strings. זמרו׃ sing}%
\verse{שׁירו Sing לו שׁיר song; חדשׁ unto him a new היטיבו skilfully נגן play בתרועה׃ with a loud noise.}%
\verse{כי For ישׁר right; דבר the word יהוה of the LORD וכל and all מעשׂהו his works באמונה׃ in truth.}%
\verse{אהב He loveth צדקה righteousness ומשׁפט and judgment: חסד of the goodness יהוה of the LORD. מלאה is full הארץ׃ the earth}%
\verse{בדבר By the word יהוה of the LORD שׁמים were the heavens נעשׂו made; וברוח of them by the breath פיו of his mouth. כל and all צבאם׃ the host}%
\verse{כנס He gathereth כנד together as a heap: מי the waters הים of the sea נתן he layeth up באצרות in storehouses. תהומות׃ the depth}%
\verse{ייראו fear מיהוה כל Let all הארץ the earth ממנו of יגורו stand in awe כל let all ישׁבי the inhabitants תבל׃ of the world}%
\verse{כי For הוא he אמר spoke, ויהי and it was הוא he צוה commanded, ויעמד׃ and it stood fast.}%
\verse{יהוה The LORD הפיר to naught: עצת bringeth the counsel גוים of the heathen הניא of none effect. מחשׁבות he maketh the devices עמים׃ of the people}%
\verse{עצת The counsel יהוה of the LORD לעולם forever, תעמד standeth מחשׁבות the thoughts לבו of his heart לדר to all ודר׃ generations.}%
\verse{אשׁרי Blessed הגוי the nation אשׁר whose יהוה the LORD; אלהיו God העם the people בחר he hath chosen לנחלה׃ for his own inheritance.}%
\verse{משׁמים from heaven; הביט looketh יהוה The LORD ראה he beholdeth את כל all בני the sons האדם׃ of men.}%
\verse{ממכון שׁבתו of his habitation השׁגיח he looketh אל upon כל all ישׁבי the inhabitants הארץ׃ of the earth.}%
\verse{היצר He fashioneth יחד alike; לבם their hearts המבין he considereth אל he considereth כל all מעשׂיהם׃ their works.}%
\verse{אין There is no המלך king נושׁע saved ברב by the multitude חיל of a host: גבור a mighty man לא is not ינצל delivered ברב by much כח׃ strength.}%
\verse{שׁקר a vain thing הסוס A horse לתשׁועה for safety: וברב by his great חילו strength. לא neither ימלט׃ shall he deliver}%
\verse{הנה Behold, עין the eye יהוה of the LORD אל upon יראיו them that fear למיחלים him, upon them that hope לחסדו׃ in his mercy;}%
\verse{להציל To deliver ממות from death, נפשׁם their soul ולחיותם and to keep them alive ברעב׃ in famine.}%
\verse{נפשׁנו Our soul חכתה waiteth ליהוה for the LORD: עזרנו our help ומגננו and our shield. הוא׃ he}%
\verse{כי For בו ישׂמח shall rejoice לבנו our heart כי in him, because בשׁם name. קדשׁו in his holy בטחנו׃ we have trusted}%
\verse{יהי be חסדך Let thy mercy, יהוה O LORD, עלינו upon כאשׁר us, according as יחלנו׃ we hope}%
\end{biblechapter}%
\begin{biblechapter}% Psalm 34
\verseWithHeading{Thanksgiving for Adonai’s Deliverance}{לדוד of David, בשׁנותו when he changed את טעמו his behavior לפני before אבימלך Abimelech; ויגרשׁהו who drove him away, וילך׃ and he departed. אברכה I will bless את יהוה the LORD בכל at all עת times: תמיד continually תהלתו his praise בפי׃ in my mouth.}%
\verse{ביהוה in the LORD: תתהלל shall make her boast נפשׁי My soul ישׁמעו shall hear ענוים the humble וישׂמחו׃ and be glad.}%
\verse{גדלו O magnify ליהוה the LORD אתי with ונרוממה me, and let us exalt שׁמו his name יחדו׃ together.}%
\verse{דרשׁתי I sought את יהוה the LORD, וענני and he heard ומכל me from all מגורותי my fears. הצילני׃ me, and delivered}%
\verse{הביטו They looked אליו unto ונהרו him, and were lightened: ופניהם and their faces אל were not יחפרו׃ ashamed.}%
\verse{זה This עני poor קרא man cried, ויהוה and the LORD שׁמע heard ומכל him out of all צרותיו his troubles. הושׁיעו׃ and saved}%
\verse{חנה encampeth מלאך The angel יהוה of the LORD סביב round about ליראיו them that fear ויחלצם׃ him, and delivereth}%
\verse{טעמו O taste וראו and see כי that טוב good: יהוה the LORD אשׁרי blessed הגבר the man יחסה׃ trusteth}%
\verse{יראו O fear את יהוה the LORD, קדשׁיו ye his saints: כי for אין no מחסור want ליראיו׃ to them that fear}%
\verse{כפירים The young lions רשׁו do lack, ורעבו and suffer hunger: ודרשׁי but they that seek יהוה the LORD לא shall not יחסרו want כל any טוב׃ good}%
\verse{לכו Come, בנים ye children, שׁמעו hearken לי יראת you the fear יהוה of the LORD. אלמדכם׃ unto me: I will teach}%
\verse{מי What האישׁ man החפץ desireth חיים life, אהב loveth ימים days, לראות that he may see טוב׃ good?}%
\verse{נצר Keep לשׁונך thy tongue מרע from evil, ושׂפתיך and thy lips מדבר from speaking מרמה׃ guile.}%
\verse{סור Depart מרע from evil, ועשׂה and do טוב good; בקשׁ seek שׁלום peace, ורדפהו׃ and pursue}%
\verse{עיני The eyes יהוה of the LORD אל upon צדיקים the righteous, ואזניו and his ears אל unto שׁועתם׃ their cry.}%
\verse{פני The face יהוה of the LORD בעשׂי against them that do רע evil, להכרית to cut off מארץ of them from the earth. זכרם׃ the remembrance}%
\verse{צעקו cry, ויהוה and the LORD שׁמע heareth, ומכל them out of all צרותם their troubles. הצילם׃ and delivereth}%
\verse{קרוב nigh יהוה The LORD לנשׁברי unto them that are of a broken לב heart; ואת דכאי such as be of a contrite רוח spirit. יושׁיע׃ and saveth}%
\verse{רבות Many רעות the afflictions צדיק of the righteous: ומכלם him out of them all. יצילנו delivereth יהוה׃ but the LORD}%
\verse{שׁמר He keepeth כל all עצמותיו his bones: אחת one מהנה of them לא not נשׁברה׃ is broken.}%
\verse{תמותת shall slay רשׁע the wicked: רעה Evil ושׂנאי and they that hate צדיק the righteous יאשׁמו׃ shall be desolate.}%
\verse{פודה redeemeth יהוה The LORD נפשׁ the soul עבדיו of his servants: ולא and none יאשׁמו in him shall be desolate. כל and none החסים׃ of them that trust}%
\end{biblechapter}%
\begin{biblechapter}% Psalm 35
\verseWithHeading{A Prayer for Rescue from Enemies}{לדוד of David. ריבה Plead יהוה O LORD, את יריבי them that strive לחם with me: fight against את לחמי׃ them that fight against}%
\verse{החזק Take hold מגן of shield וצנה and buckler, וקומה and stand up בעזרתי׃ for mine help.}%
\verse{והרק Draw out חנית also the spear, וסגר and stop לקראת against רדפי them that persecute אמר me: say לנפשׁי unto my soul, ישׁעתך thy salvation. אני׃ I}%
\verse{יבשׁו Let them be confounded ויכלמו and put to shame מבקשׁי that seek after נפשׁי my soul: יסגו let them be turned אחור back ויחפרו and brought to confusion חשׁבי that devise רעתי׃ my hurt.}%
\verse{יהיו Let them be כמץ as chaff לפני before רוח the wind: ומלאך and let the angel יהוה of the LORD דוחה׃ chase}%
\verse{יהי be דרכם Let their way חשׁך dark וחלקלקות and slippery: ומלאך and let the angel יהוה of the LORD רדפם׃ persecute}%
\verse{כי For חנם without cause טמנו have they hid לי שׁחת a pit, רשׁתם for me their net חנם without cause חפרו they have digged לנפשׁי׃ for my soul.}%
\verse{תבואהו come upon שׁואה Let destruction לא him at unawares; ידע him at unawares; ורשׁתו and let his net אשׁר that טמן he hath hid תלכדו catch בשׁואה himself: into that very destruction יפל׃ let him fall.}%
\verse{ונפשׁי And my soul תגיל shall be joyful ביהוה in the LORD: תשׂישׂ it shall rejoice בישׁועתו׃ in his salvation.}%
\verse{כל All עצמותי my bones תאמרנה shall say, יהוה LORD, מי who כמוך like unto thee, מציל which deliverest עני the poor מחזק from him that is too strong ממנו from him that is too strong ועני him, yea, the poor ואביון and the needy מגזלו׃ from him that spoileth}%
\verse{יקומון did rise up; עדי witnesses חמס False אשׁר that לא not. ידעתי I knew ישׁאלוני׃ they laid to my charge}%
\verse{ישׁלמוני They rewarded רעה me evil תחת for טובה good שׁכול the spoiling לנפשׁי׃ of my soul.}%
\verse{ואני But as for me, בחלותם when they were sick, לבושׁי my clothing שׂק sackcloth: עניתי I humbled בצום with fasting; נפשׁי my soul ותפלתי and my prayer על into חיקי mine own bosom. תשׁוב׃ returned}%
\verse{כרע as though my friend כאח brother: לי התהלכתי I behaved myself כאבל as one that mourneth אם mother. קדר heavily, שׁחותי׃ I bowed down}%
\verse{ובצלעי But in mine adversity שׂמחו they rejoiced, ונאספו and gathered themselves together: נאספו gathered themselves together עלי against נכים the abjects ולא not; ידעתי me, and I knew קרעו they did tear ולא not: דמו׃ and ceased}%
\verse{בחנפי With hypocritical לעגי mockers מעוג in feasts, חרק they gnashed עלי upon שׁנימו׃ me with their teeth.}%
\verse{אדני Lord, כמה how long תראה wilt thou look on? השׁיבה rescue נפשׁי my soul משׁאיהם from their destructions, מכפירים from the lions. יחידתי׃ my darling}%
\verse{אודך I will give thee thanks בקהל congregation: רב in the great בעם people. עצום thee among much אהללך׃ I will praise}%
\verse{אל Let not ישׂמחו rejoice לי איבי them that are mine enemies שׁקר wrongfully שׂנאי that hate חנם me without a cause. יקרצו over me: let them wink עין׃ with the eye}%
\verse{כי For לא not שׁלום peace: ידברו they speak ועל against רגעי quiet ארץ in the land. דברי matters מרמות deceitful יחשׁבון׃ but they devise}%
\verse{וירחיבו Yea, they opened עלי wide against פיהם their mouth אמרו me, said, האח Aha, האח aha, ראתה hath seen עינינו׃ our eye}%
\verse{ראיתה thou hast seen, יהוה O LORD: אל keep not תחרשׁ silence: אדני O Lord, אל be not תרחק far ממני׃ from}%
\verse{העירה Stir up thyself, והקיצה and awake למשׁפטי to my judgment, אלהי my God ואדני and my Lord. לריבי׃ unto my cause,}%
\verse{שׁפטני Judge כצדקך according to thy righteousness; יהוה me, O LORD אלהי my God, ואל and let them not ישׂמחו׃ rejoice}%
\verse{אל Let them not יאמרו say בלבם in their hearts, האח Ah, נפשׁנו so would we have it: אל let them not יאמרו say, בלענוהו׃ We have swallowed him up.}%
\verse{יבשׁו Let them be ashamed ויחפרו and brought to confusion יחדו together שׂמחי that rejoice רעתי at mine hurt: ילבשׁו let them be clothed בשׁת with shame וכלמה and dishonor המגדילים that magnify עלי׃ against}%
\verse{ירנו Let them shout for joy, וישׂמחו and be glad, חפצי that favor צדקי my righteous cause: ויאמרו yea, let them say תמיד continually, יגדל be magnified, יהוה Let the LORD החפץ which hath pleasure שׁלום in the prosperity עבדו׃ of his servant.}%
\verse{ולשׁוני And my tongue תהגה shall speak צדקך of thy righteousness כל all היום the day תהלתך׃ of thy praise}%
\end{biblechapter}%
\begin{biblechapter}% Psalm 36
\verseWithHeading{Human Wickedness and God’s Love}{למנצח To the chief Musician, לעבד the servant יהוה of the LORD. לדוד׃ of David נאם saith פשׁע The transgression לרשׁע of the wicked בקרב within לבי my heart, אין no פחד fear אלהים of God לנגד before עיניו׃ his eyes.}%
\verse{כי For החליק he flattereth אליו he flattereth בעיניו himself in his own eyes, למצא be found עונו until his iniquity לשׂנא׃ to be hateful.}%
\verse{דברי The words פיו of his mouth און iniquity ומרמה and deceit: חדל he hath left off להשׂכיל to be wise, להיטיב׃ to do good.}%
\verse{און mischief יחשׁב He deviseth על upon משׁכבו his bed; יתיצב he setteth himself על in דרך a way לא not טוב good; רע evil. לא not ימאס׃ he abhorreth}%
\verse{יהוה O LORD, בהשׁמים in the heavens; חסדך Thy mercy, אמונתך thy faithfulness עד unto שׁחקים׃ the clouds.}%
\verse{צדקתך Thy righteousness כהררי mountains; אל like the great משׁפטך thy judgments תהום deep: רבה a great אדם man ובהמה and beast. תושׁיע thou preservest יהוה׃ O LORD,}%
\verse{מה How יקר excellent חסדך thy lovingkindness, אלהים O God! ובני therefore the children אדם of men בצל under the shadow כנפיך of thy wings. יחסיון׃ put their trust}%
\verse{ירוין They shall be abundantly satisfied מדשׁן with the fatness ביתך of thy house; ונחל of the river עדניך of thy pleasures. תשׁקם׃ and thou shalt make them drink}%
\verse{כי For עמך with מקור thee the fountain חיים of life: באורך in thy light נראה shall we see אור׃ light.}%
\verse{משׁך O continue חסדך thy lovingkindness לידעיך unto them that know וצדקתך thee; and thy righteousness לישׁרי to the upright לב׃ in heart.}%
\verse{אל Let not תבואני come against רגל the foot גאוה of pride ויד the hand רשׁעים of the wicked אל me, and let not תנדני׃ remove}%
\verse{שׁם There נפלו fallen: פעלי are the workers און of iniquity דחו they are cast down, ולא and shall not יכלו be able קום׃ to rise.}%
\end{biblechapter}%
\begin{biblechapter}% Psalm 37
\verseWithHeading{The Protection of the Righteous and the Destruction of the Wicked}{לדוד of David. אל neither תתחר במרעים because of evildoers, אל תקנא be thou envious בעשׂי against the workers עולה׃ of iniquity.}%
\verse{כי For כחציר like the grass, מהרה they shall soon ימלו be cut down וכירק as the green דשׁא herb. יבולון׃ and wither}%
\verse{בטח Trust ביהוה in the LORD, ועשׂה and do טוב good; שׁכן shalt thou dwell ארץ in the land, ורעה thou shalt be fed. אמונה׃ and verily}%
\verse{והתענג Delight thyself על also in יהוה the LORD; ויתן and he shall give לך משׁאלת thee the desires לבך׃ of thine heart.}%
\verse{גול Commit על unto יהוה the LORD; דרכך thy way ובטח trust עליו also in והוא him; and he יעשׂה׃ shall bring to pass.}%
\verse{והוציא And he shall bring forth כאור as the light, צדקך thy righteousness ומשׁפטך and thy judgment כצהרים׃ as the noonday.}%
\verse{דום Rest ליהוה in the LORD, והתחולל and wait patiently לו אל for him: fret not thyself תתחר for him: fret not thyself במצליח because of him who prospereth דרכו in his way, באישׁ because of the man עשׂה who bringeth wicked devices to pass. מזמות׃ who bringeth wicked devices to pass.}%
\verse{הרף Cease מאף from anger, ועזב and forsake חמה wrath: אל fret not thyself תתחר fret not thyself אך in any wise להרע׃ to do evil.}%
\verse{כי For מרעים evildoers יכרתון shall be cut off: וקוי but those that wait upon יהוה the LORD, המה they יירשׁו shall inherit ארץ׃ the earth.}%
\verse{ועוד For yet מעט a little ואין not רשׁע while, and the wicked והתבוננת yea, thou shalt diligently consider על : yea, thou shalt diligently consider מקומו his place, ואיננו׃ and it not}%
\verse{וענוים But the meek יירשׁו shall inherit ארץ the earth; והתענגו and shall delight themselves על in רב the abundance שׁלום׃ of peace.}%
\verse{זמם plotteth רשׁע The wicked לצדיק against the just, וחרק and gnasheth עליו upon שׁניו׃ him with his teeth.}%
\verse{אדני The Lord ישׂחק shall laugh לו כי at him: for ראה he seeth כי that יבא is coming. יומו׃ his day}%
\verse{חרב the sword, פתחו have drawn out רשׁעים The wicked ודרכו and have bent קשׁתם their bow, להפיל to cast down עני the poor ואביון and needy, לטבוח to slay ישׁרי such as be of upright דרך׃ conversation.}%
\verse{חרבם Their sword תבוא shall enter בלבם into their own heart, וקשׁתותם and their bows תשׁברנה׃ shall be broken.}%
\verse{טוב man hath better מעט A little לצדיק that a righteous מהמון than the riches רשׁעים wicked. רבים׃ of many}%
\verse{כי For זרועות the arms רשׁעים of the wicked תשׁברנה shall be broken: וסומך upholdeth צדיקים the righteous. יהוה׃ but the LORD}%
\verse{יודע knoweth יהוה The LORD ימי the days תמימם of the upright: ונחלתם and their inheritance לעולם forever. תהיה׃ shall be}%
\verse{לא They shall not יבשׁו be ashamed בעת time: רעה in the evil ובימי and in the days רעבון of famine ישׂבעו׃ they shall be satisfied.}%
\verse{כי But רשׁעים the wicked יאבדו shall perish, ואיבי and the enemies יהוה of the LORD כיקר as the fat כרים of lambs: כלו they shall consume; בעשׁן into smoke כלו׃ shall they consume away.}%
\verse{לוה borroweth, רשׁע The wicked ולא not ישׁלם and payeth וצדיק again: but the righteous חונן showeth mercy, ונותן׃ and giveth.}%
\verse{כי For מברכיו blessed יירשׁו of him shall inherit ארץ the earth; ומקלליו and cursed יכרתו׃ of him shall be cut off.}%
\verse{מיהוה מצעדי The steps גבר of a man כוננו are ordered ודרכו in his way. יחפץ׃ and he delighteth}%
\verse{כי Though יפל he fall, לא he shall not יוטל be utterly cast down: כי for יהוה the LORD סומך upholdeth ידו׃ his hand.}%
\verse{נער young, הייתי I have been גם and זקנתי am old; ולא yet have I not ראיתי seen צדיק the righteous נעזב forsaken, וזרעו nor his seed מבקשׁ begging לחם׃ bread.}%
\verse{כל היוםונן merciful, ומלוה and lendeth; וזרעו and his seed לברכה׃ blessed.}%
\verse{סור Depart מרע evil, ועשׂה and do טוב good; ושׁכן and dwell לעולם׃ forevermore.}%
\verse{כי For יהוה the LORD אהב loveth משׁפט judgment, ולא not יעזב and forsaketh את חסידיו his saints; לעולם forever: נשׁמרו they are preserved וזרע but the seed רשׁעים of the wicked נכרת׃ shall be cut off.}%
\verse{צדיקים The righteous יירשׁו shall inherit ארץ the land, וישׁכנו and dwell לעד forever. עליה׃ therein}%
\verse{פי The mouth צדיק of the righteous יהגה speaketh חכמה wisdom, ולשׁונו and his tongue תדבר talketh משׁפט׃ of judgment.}%
\verse{תורת The law אלהיו of his God בלבו in his heart; לא none תמעד shall slide. אשׁריו׃ of his steps}%
\verse{צופה watcheth רשׁע The wicked לצדיק the righteous, ומבקשׁ and seeketh להמיתו׃ to slay}%
\verse{יהוה The LORD לא will not יעזבנו leave בידו him in his hand, ולא nor ירשׁיענו condemn בהשׁפטו׃ him when he is judged.}%
\verse{קוה Wait אל on יהוה the LORD, ושׁמר and keep דרכו his way, וירוממך and he shall exalt לרשׁת thee to inherit ארץ the land: בהכרת are cut off, רשׁעים when the wicked תראה׃ thou shalt see}%
\verse{ראיתי I have seen רשׁע the wicked עריץ in great power, ומתערה and spreading himself כאזרח bay tree. רענן׃ like a green}%
\verse{ויעבר Yet he passed away, והנה and, lo, איננו he not: ואבקשׁהו yea, I sought ולא him, but he could not נמצא׃ be found.}%
\verse{שׁמר Mark תם the perfect וראה and behold ישׁר the upright: כי for אחרית the end לאישׁ of man שׁלום׃ peace.}%
\verse{ופשׁעים But the transgressors נשׁמדו shall be destroyed יחדו together: אחרית the end רשׁעים of the wicked נכרתה׃ shall be cut off.}%
\verse{ותשׁועת But the salvation צדיקים of the righteous מיהוה מעוזם their strength בעת in the time צרה׃ of trouble.}%
\verse{ויעזרם shall help יהוה And the LORD ויפלטם them, and deliver יפלטם them: he shall deliver מרשׁעים them from the wicked, ויושׁיעם and save כי them, because חסו׃ they trust}%
\end{biblechapter}%
\begin{biblechapter}% Psalm 38
\verseWithHeading{A Prayer of Repentance}{מזמור A Psalm לדוד of David, להזכיר׃ to bring to remembrance. יהוה O LORD, אל me not בקצפך in thy wrath: תוכיחני rebuke ובחמתך me in thy hot displeasure. תיסרני׃ neither chasten}%
\verse{כי For חציך thine arrows נחתו stick fast בי ותנחת presseth me sore. עלי presseth me sore. ידך׃ in me, and thy hand}%
\verse{אין no מתם soundness בבשׂרי in my flesh מפני because זעמך of thine anger; אין neither שׁלום rest בעצמי in my bones מפני because חטאתי׃ of my sin.}%
\verse{כי For עונתי mine iniquities עברו are gone over ראשׁי mine head: כמשׂא burden כבד as a heavy יכבדו they are too heavy ממני׃ for}%
\verse{הבאישׁו stink נמקו are corrupt חבורתי My wounds מפני because אולתי׃ of my foolishness.}%
\verse{נעויתי I am troubled; שׁחתי I am bowed down עד greatly; מאד greatly; כל all היום the day קדר mourning הלכתי׃ I go}%
\verse{כי For כסלי my loins מלאו are filled נקלה with a loathsome ואין and no מתם soundness בבשׂרי׃ in my flesh.}%
\verse{נפוגותי I am feeble ונדכיתי broken: עד and sore מאד and sore שׁאגתי I have roared מנהמת by reason of the disquietness לבי׃ of my heart.}%
\verse{אדני Lord, נגדך before כל all תאותי my desire ואנחתי thee; and my groaning ממך from לא is not נסתרה׃ hid}%
\verse{לבי My heart סחרחר panteth, עזבני faileth כחי my strength ואור me: as for the light עיני of mine eyes, גם also הם it אין is gone אתי׃ from}%
\verse{אהבי My lovers ורעי and my friends מנגד נגעיעמדו stand aloof וקרובי and my kinsmen מרחק afar off. עמדו׃ stand}%
\verse{וינקשׁו lay snares מבקשׁי They also that seek after נפשׁי my life ודרשׁי and they that seek רעתי my hurt דברו speak הוות mischievous things, ומרמות deceits כל all היום the day יהגו׃ and imagine}%
\verse{ואני But I, כחרשׁ as a deaf לא not; אשׁמע heard וכאלם and as a dumb man לא not יפתח openeth פיו׃ his mouth.}%
\verse{ואהי Thus I was כאישׁ as a man אשׁר that לא not, שׁמע heareth ואין no בפיו and in whose mouth תוכחות׃ reproofs.}%
\verse{כי For לך יהוה הוחלתי do I hope: אתה thou תענה wilt hear, אדני O Lord אלהי׃ my God.}%
\verse{כי For אמרתי I said, פן lest ישׂמחו they should rejoice לי במוט slippeth, רגלי over me: when my foot עלי against הגדילו׃ they magnify}%
\verse{כי For אני I לצלע to halt, נכון ready ומכאובי and my sorrow נגדי before תמיד׃ continually}%
\verse{כי For עוני mine iniquity; אגיד I will declare אדאג I will be sorry מחטאתי׃ for my sin.}%
\verse{ואיבי But mine enemies חיים lively, עצמו they are strong: ורבו are multiplied. שׂנאי and they that hate שׁקר׃ me wrongfully}%
\verse{ומשׁלמי They also that render רעה evil תחת for טובה good ישׂטנוני are mine adversaries; תחת because רדופי I follow טוב׃}%
\verse{אל me not, תעזבני Forsake יהוה O LORD: אלהי O my God, אל be not תרחק far ממני׃ from}%
\verse{חושׁה Make haste לעזרתי to help אדני me, O Lord תשׁועתי׃ my salvation.}%
\end{biblechapter}%
\begin{biblechapter}% Psalm 39
\verseWithHeading{The Brevity of Human Life}{למנצח To the chief Musician, לידיתון to Jeduthun, מזמור A Psalm לדוד׃ of David. אמרתי I said, אשׁמרה I will take heed דרכי to my ways, מחטוא בלשׁוני with my tongue: אשׁמרה I will keep לפי my mouth מחסום with a bridle, בעד רשׁע the wicked לנגדי׃ is before}%
\verse{נאלמתי I was dumb דומיה with silence, החשׁיתי I held my peace, מטוב from good; וכאבי and my sorrow נעכר׃ was stirred.}%
\verse{חם was hot לבי My heart בקרבי within בהגיגי me, while I was musing תבער burned: אשׁ the fire דברתי spoke בלשׁוני׃ I with my tongue,}%
\verse{הודיעני make me to know יהוה LORD, קצי mine end, ומדת and the measure ימי of my days, מה what היא it אדעה I may know מה how חדל frail אני׃ I}%
\verse{הנה Behold, טפחות a handbreadth; נתתה thou hast made ימי my days וחלדי and mine age כאין as nothing נגדך before אך thee: verily כל every הבל vanity. כל altogether אדם man נצב at his best state סלה׃ Selah.}%
\verse{אך Surely בצלם in a vain show: יתהלך walketh אישׁ every man אך surely הבל in vain: יהמיון they are disquieted יצבר he heapeth up ולא not ידע and knoweth מי who אספם׃ shall gather}%
\verse{ועתה And now, מה what קויתי wait אדני Lord, תוחלתי I for? my hope לך היא׃}%
\verse{מכל me from all פשׁעי my transgressions: הצילני Deliver חרפת the reproach נבל of the foolish. אל me not תשׂימני׃ make}%
\verse{נאלמתי I was dumb, לא not אפתח I opened פי my mouth; כי because אתה thou עשׂית׃ didst}%
\verse{הסר Remove מעלי away from נגעך thy stroke מתגרת by the blow ידך of thine hand. אני me: I כליתי׃ am consumed}%
\verse{בתוכחות When thou with rebukes על for עון iniquity, יסרת dost correct אישׁ man ותמס to consume away כעשׁ like a moth: חמודו thou makest his beauty אך surely הבל vanity. כל every אדם man סלה׃ Selah.}%
\verse{שׁמעה Hear תפלתי my prayer, יהוה O LORD, ושׁועתי unto my cry; האזינה and give ear אל at דמעתי my tears: אל hold not thy peace תחרשׁ hold not thy peace כי for גר a stranger אנכי I עמך with תושׁב thee, a sojourner, ככל as all אבותי׃ my fathers}%
\verse{השׁע ממניאבליגה me, that I may recover strength, בטרם before אלך I go ואינני׃ hence, and be no}%
\end{biblechapter}%
\begin{biblechapter}% Psalm 40
\verseWithHeading{God’s Faithfulness and Deliverance}{למנצח To the chief Musician, לדוד of David. מזמור׃ A Psalm קוה I waited קויתי patiently יהוה for the LORD; ויט and he inclined אלי unto וישׁמע me, and heard שׁועתי׃ my cry.}%
\verse{ויעלני He brought me up מבור pit, שׁאון also out of a horrible מטיט clay, היון out of the miry ויקם and set על upon סלע a rock, רגלי my feet כונן established אשׁרי׃ my goings.}%
\verse{ויתן And he hath put בפי in my mouth, שׁיר song חדשׁ a new תהלה praise לאלהינו unto our God: יראו shall see רבים many וייראו and fear, ויבטחו and shall trust ביהוה׃ in the LORD.}%
\verse{אשׁרי Blessed הגבר that man אשׁר that שׂם maketh יהוה the LORD מבטחו his trust, ולא not פנה and respecteth אל and respecteth רהבים the proud, ושׂטי nor such as turn aside כזב׃ to lies.}%
\verse{רבות Many, עשׂית hast done, אתה thou יהוה O LORD אלהי my God, נפלאתיך thy wonderful works ומחשׁבתיך and thy thoughts אלינו to אין usward: they cannot ערך be reckoned up in order אליך unto thee: אגידה I would declare ואדברה and speak עצמו they are more מספר׃ than can be numbered.}%
\verse{זבח Sacrifice ומנחה and offering לא thou didst not חפצת desire; אזנים mine ears כרית hast thou opened: לי עולה burnt offering וחטאה and sin offering לא hast thou not שׁאלת׃ required.}%
\verse{אז Then אמרתי said הנה I, Lo, באתי I come: במגלת in the volume ספר of the book כתוב written עלי׃ of}%
\verse{לעשׂות to do רצונך thy will, אלהי O my God: חפצתי I delight ותורתך yea, thy law בתוך within מעי׃ my heart.}%
\verse{בשׂרתי I have preached צדק righteousness בקהל congregation: רב in the great הנה lo, שׂפתי my lips, לא I have not אכלא refrained יהוה O LORD, אתה thou ידעת׃ knowest.}%
\verse{צדקתך thy righteousness לא I have not כסיתי hid בתוך within לבי my heart; אמונתך thy faithfulness ותשׁועתך and thy salvation: אמרתי I have declared לא I have not כחדתי concealed חסדך thy lovingkindness ואמתך and thy truth לקהל congregation. רב׃ from the great}%
\verse{אתה thou יהוה me, O LORD: לא not תכלא Withhold רחמיך thy tender mercies ממני from חסדך let thy lovingkindness ואמתך and thy truth תמיד continually יצרוני׃ preserve}%
\verse{כי For אפפו have compassed me about: עלי have compassed me about: רעות evils עד innumerable אין מספרשׂיגוני have taken hold עונתי mine iniquities ולא upon me, so that I am not יכלתי able לראות to look up; עצמו they are more משׂערות than the hairs ראשׁי of mine head: ולבי therefore my heart עזבני׃ faileth}%
\verse{רצה Be pleased, יהוה O LORD, להצילני to deliver יהוה me: O LORD, לעזרתי to help חושׁה׃ make haste}%
\verse{יבשׁו Let them be ashamed ויחפרו and confounded יחד together מבקשׁי that seek after נפשׁי my soul לספותה to destroy יסגו it; let them be driven אחור backward ויכלמו and put to shame חפצי that wish רעתי׃ me evil.}%
\verse{ישׁמו Let them be desolate על for עקב a reward בשׁתם of their shame האמרים that say לי האח unto me, Aha, האח׃ aha.}%
\verse{ישׂישׂו thee rejoice וישׂמחו and be glad בך כל Let all מבקשׁיך those that seek יאמרו say תמיד continually, יגדל be magnified. יהוה The LORD אהבי in thee: let such as love תשׁועתך׃ thy salvation}%
\verse{ואני But I עני poor ואביון and needy; אדני the Lord יחשׁב thinketh לי עזרתי my help ומפלטי and my deliverer; אתה upon me: thou אלהי O my God. אל make no תאחר׃ tarrying,}%
\end{biblechapter}%
\begin{biblechapter}% Psalm 41
\verseWithHeading{Thanksgiving for God’s Provision in Time of Sickness}{למנצח To the chief Musician, מזמור A Psalm לדוד׃ of David. אשׁרי Blessed משׂכיל he that considereth אל he that considereth דל the poor: ביום him in time רעה of trouble. ימלטהו will deliver יהוה׃ the LORD}%
\verse{יהוה The LORD ישׁמרהו will preserve ויחיהו him, and keep him alive; יאשׁר he shall be blessed בארץ upon the earth: ואל and thou wilt not תתנהו deliver בנפשׁ him unto the will איביו׃ of his enemies.}%
\verse{יהוה The LORD יסעדנו will strengthen על him upon ערשׂ the bed דוי of languishing: כל all משׁכבו his bed הפכת thou wilt make בחליו׃ in his sickness.}%
\verse{אני I אמרתי said, יהוה LORD, חנני be merciful רפאה unto me: heal נפשׁי my soul; כי for חטאתי׃ I have sinned}%
\verse{אויבי Mine enemies יאמרו speak רע evil לי מתי of me, When ימות shall he die, ואבד perish? שׁמו׃ and his name}%
\verse{ואם And if בא he come לראות to see שׁוא vanity: ידבר he speaketh לבו his heart יקבץ gathereth און iniquity לו יצא to itself; he goeth לחוץ abroad, ידבר׃ he telleth}%
\verse{יחד together עלי against יתלחשׁו me whisper כל All שׂנאי that hate עלי me: against יחשׁבו me do they devise רעה׃ my hurt.}%
\verse{דבר disease, בליעל An evil יצוק cleaveth fast בו ואשׁר unto him: and that שׁכב he lieth לא no יוסיף more. לקום׃ he shall rise up}%
\verse{גם Yea, אישׁ mine own familiar friend, שׁלומי mine own familiar friend, אשׁר in whom בטחתי I trusted, בו אוכל which did eat לחמי of my bread, הגדיל hath lifted up עלי against עקב׃ heel}%
\verse{ואתה But thou, יהוה חנני be merciful והקימני unto me, and raise me up, ואשׁלמה that I may requite להם׃}%
\verse{בזאת By this ידעתי I know כי that חפצת thou favorest בי כי me, because לא doth not יריע triumph איבי mine enemy עלי׃ over}%
\verse{ואני And as for me, בתמי me in mine integrity, תמכת thou upholdest בי ותציבני and settest לפניך me before thy face לעולם׃ forever.}%
\verse{ברוך Blessed יהוה the LORD אלהי God ישׂראל of Israel מהעולם from everlasting, ועד and to העולם everlasting. אמן Amen, ואמן׃ and Amen.}%
\end{biblechapter}%
\begin{biblechapter}% Psalm 42
\verseWithHeading{Hope in God in the Midst of Despair}{למנצח To the chief Musician, משׂכיל Maschil, לבני for the sons קרח׃ of Korah. כאיל As the hart תערג panteth על after אפיקי brooks, מים the water כן so נפשׁי my soul תערג panteth אליך after אלהים׃ thee, O God.}%
\verse{צמאה thirsteth נפשׁי My soul לאלהים for God, לאל God: חי for the living מתי when אבוא shall I come ואראה and appear פני before אלהים׃ God?}%
\verse{היתה have been לי דמעתי My tears לחם my meat יומם day ולילה and night, באמר say אלי unto כל while they continually היום while they continually איה me, Where אלהיך׃ thy God?}%
\verse{אלה these אזכרה When I remember ואשׁפכה I pour out עלי in נפשׁי my soul כי me: for אעבר I had gone בסך with the multitude, אדדם I went עד with them to בית the house אלהים of God, בקול with the voice רנה of joy ותודה and praise, המון with a multitude חוגג׃ that kept holyday.}%
\verse{מה Why תשׁתוחחי art thou cast down, נפשׁי O my soul? ותהמי and art thou disquieted עלי in הוחילי me? hope לאלהים thou in God: כי for עוד I shall yet אודנו praise ישׁועות him the help פניו׃ of his countenance.}%
\verse{אלהי O my God, עלי within נפשׁי my soul תשׁתוחח is cast down על me: therefore כן me: therefore אזכרך will I remember מארץ thee from the land ירדן of Jordan, וחרמונים and of the Hermonites, מהר from the hill מצער׃ Mizar.}%
\verse{תהום Deep אל unto תהום deep קורא calleth לקול at the noise צנוריך of thy waterspouts: כל all משׁבריך thy waves וגליך and thy billows עלי over עברו׃ are gone}%
\verse{יומם in the daytime, יצוה will command יהוה the LORD חסדו his lovingkindness ובלילה and in the night שׁירה his song עמי with תפלה me, my prayer לאל unto the God חיי׃ of my life.}%
\verse{אומרה I will say לאל unto God סלעי my rock, למה Why שׁכחתני hast thou forgotten למה me? why קדר I mourning אלך go בלחץ because of the oppression אויב׃ of the enemy?}%
\verse{ברצח with a sword בעצמותי in my bones, חרפוני reproach צוררי mine enemies באמרם me; while they say אלי unto כל daily היום daily איה me, Where אלהיך׃ thy God?}%
\verse{מה Why תשׁתוחחי art thou cast down, נפשׁי O my soul? ומה and why תהמי art thou disquieted עלי within הוחילי me? hope לאלהים thou in God: כי for עוד I shall yet אודנו praise ישׁועת him, the health פני of my countenance, ואלהי׃ and my God.}%
\end{biblechapter}%
\begin{biblechapter}% Psalm 43
\verseWithHeading{A Prayer for Rescue}{שׁפטני Judge אלהים me, O God, וריבה and plead ריבי my cause מגוי nation: לא against an ungodly חסיד against an ungodly מאישׁ man. מרמה me from the deceitful ועולה and unjust תפלטני׃ O deliver}%
\verse{כי For אתה thou אלהי the God מעוזי of my strength: למה why זנחתני dost thou cast me off? למה why קדר I mourning אתהלך go בלחץ because of the oppression אויב׃ of the enemy?}%
\verse{שׁלח O send out אורך thy light ואמתך and thy truth: המה let them ינחוני lead יביאוני me; let them bring אל me unto הר hill, קדשׁך thy holy ואל and to משׁכנותיך׃ thy tabernacles.}%
\verse{ואבואה Then will I go אל unto מזבח the altar אלהים of God, אל unto אל God שׂמחת my exceeding גילי joy: ואודך will I praise בכנור yea, upon the harp אלהים thee, O God אלהי׃ my God.}%
\verse{מה Why תשׁתוחחי art thou cast down, נפשׁי O my soul? ומה and why תהמי art thou disquieted עלי within הוחילי me? hope לאלהים in God: כי for עוד I shall yet אודנו praise ישׁועת him, the health פני of my countenance, ואלהי׃ and my God.}%
\end{biblechapter}%
\begin{biblechapter}% Psalm 44
\verseWithHeading{Present Defeat and Past Deliverance}{למנצח To the chief Musician לבני for the sons קרח of Korah, משׂכיל׃ Maschil. אלהים O God, באזנינו with our ears, שׁמענו We have heard אבותינו our fathers ספרו have told לנו פעל us, work פעלת thou didst בימיהם in their days, בימי in the times קדם׃ of old.}%
\verse{אתה thou ידך with thy hand, גוים the heathen הורשׁת didst drive out ותטעם and plantedst תרע them; thou didst afflict לאמים the people, ותשׁלחם׃ and cast them out.}%
\verse{כי For לא they got not בחרבם by their own sword, ירשׁו in possession ארץ the land וזרועם did their own arm לא neither הושׁיעה save למו כי them: but ימינך thy right hand, וזרועך and thine arm, ואור and the light פניך of thy countenance, כי because רציתם׃ thou hadst a favor}%
\verse{אתה Thou הוא מלכי art my King, אלהים O God: צוה command ישׁועות deliverances יעקב׃ for Jacob.}%
\verse{בך צרינו our enemies: ננגח Through thee will we push down בשׁמך through thy name נבוס will we tread them under קמינו׃ that rise up against}%
\verse{כי For לא I will not בקשׁתי in my bow, אבטח trust וחרבי shall my sword לא neither תושׁיעני׃ save}%
\verse{כי But הושׁעתנו thou hast saved מצרינו us from our enemies, ומשׂנאינו that hated הבישׁות׃ and hast put them to shame}%
\verse{באלהים In God הללנו we boast כל all היום the day ושׁמך thy name לעולם forever. נודה long, and praise סלה׃ Selah.}%
\verse{אף But זנחת thou hast cast off, ותכלימנו and put us to shame; ולא and goest not forth תצא and goest not forth בצבאותינו׃ with our armies.}%
\verse{תשׁיבנו Thou makest us to turn אחור back מני from צר the enemy: ומשׂנאינו and they which hate שׁסו׃ us spoil}%
\verse{תתננו Thou hast given כצאן us like sheep מאכל for meat; ובגוים us among the heathen. זריתנו׃ and hast scattered}%
\verse{תמכר Thou sellest עמך thy people בלא for naught, הון for naught, ולא and dost not רבית increase במחיריהם׃ by their price.}%
\verse{תשׂימנו Thou makest חרפה us a reproach לשׁכנינו to our neighbors, לעג a scorn וקלס and a derision לסביבותינו׃ to them that are round about}%
\verse{תשׂימנו Thou makest משׁל us a byword בגוים among the heathen, מנוד a shaking ראשׁ of the head בל among the people. אמים׃}%
\verse{כל continually היום continually כלמתי My confusion נגדי before ובשׁת me, and the shame פני of my face כסתני׃ hath covered}%
\verse{מקול מחרף of him that reproacheth ומגדף and blasphemeth; מפני by reason אויב of the enemy ומתנקם׃ and avenger.}%
\verse{כל All זאת this באתנו is come upon ולא us; yet have we not שׁכחנוך forgotten ולא thee, neither שׁקרנו have we dealt falsely בבריתך׃ in thy covenant.}%
\verse{לא is not נסוג turned אחור back, לבנו Our heart ותט declined אשׁרינו neither have our steps מני from ארחך׃ thy way;}%
\verse{כי Though דכיתנו thou hast sore broken במקום us in the place תנים of dragons, ותכס and covered עלינו and covered בצלמות׃ us with the shadow of death.}%
\verse{אם If שׁכחנו we have forgotten שׁם the name אלהינו of our God, ונפרשׂ or stretched out כפינו our hands לאל god; זר׃ to a strange}%
\verse{הלא Shall not אלהים God יחקר search זאת this כי out? for הוא he ידע knoweth תעלמות the secrets לב׃ of the heart.}%
\verse{כי Yea, עליך for הרגנו thy sake are we killed כל all היום the day נחשׁבנו long; we are counted כצאן as sheep טבחה׃ for the slaughter.}%
\verse{עורה Awake, למה why תישׁן sleepest אדני thou, O Lord? הקיצה arise, אל not תזנח cast לנצח׃ off forever.}%
\verse{למה Wherefore פניך thou thy face, תסתיר hidest תשׁכח forgettest ענינו our affliction ולחצנו׃ and our oppression?}%
\verse{כי For שׁחה is bowed down לעפר to the dust: נפשׁנו our soul דבקה cleaveth לארץ unto the earth. בטננו׃ our belly}%
\verse{קומה Arise עזרתה for our help, לנו ופדנו and redeem למען חסדך׃}%
\end{biblechapter}%
\begin{biblechapter}% Psalm 45
\verseWithHeading{Celebration of a Royal Wedding}{למנצח To the chief Musician על upon שׁשׁנים Shoshannim, לבני for the sons קרח of Korah, משׂכיל Maschil, שׁיר A Song ידידת׃ of loves. רחשׁ is inditing לבי My heart דבר matter: טוב a good אני I מעשׂי of the things which I have made אמר speak למלך touching the king: לשׁוני my tongue עט the pen סופר writer. מהיר׃ of a ready}%
\verse{יפיפית Thou art fairer מבני than the children אדם of men: הוצק is poured חן grace בשׂפתותיך into thy lips: על therefore כן therefore ברכך hath blessed אלהים God לעולם׃ thee forever.}%
\verse{חגור Gird חרבך thy sword על upon ירך thigh, גבור O mighty, הודך with thy glory והדרך׃ and thy majesty.}%
\verse{והדרך And in thy majesty צלח prosperously רכב ride על because דבר because אמת of truth וענוה and meekness צדק righteousness; ותורך shall teach נוראות thee terrible ימינך׃ and thy right hand}%
\verse{חציך Thine arrows שׁנונים sharp עמים the people תחתיך under יפלו fall בלב in the heart אויבי enemies; המלך׃ of the king's}%
\verse{כסאך Thy throne, אלהים O God, עולם forever ועד and ever: שׁבט the scepter מישׁר a right שׁבט scepter. מלכותך׃ of thy kingdom}%
\verse{אהבת Thou lovest צדק righteousness, ותשׂנא and hatest רשׁע wickedness: על therefore כן therefore משׁחך hath anointed אלהים God, אלהיך thy God, שׁמן thee with the oil שׂשׂון of gladness מחבריך׃ above thy fellows.}%
\verse{מר of myrrh, ואהלות and aloes, קציעות cassia, כל All בגדתיך thy garments מן out of היכלי palaces, שׁן the ivory מני שׂמחוך׃ they have made thee glad.}%
\verse{בנות daughters מלכים Kings' ביקרותיך among thy honorable נצבה did stand שׁגל the queen לימינך women: upon thy right hand בכתם in gold אופיר׃ of Ophir.}%
\verse{שׁמעי Hearken, בת O daughter, וראי and consider, והטי and incline אזנך thine ear; ושׁכחי forget עמך also thine own people, ובית house; אביך׃ and thy father's}%
\verse{ויתאו greatly desire המלך So shall the king יפיך thy beauty: כי for הוא he אדניך thy Lord; והשׁתחוי׃ and worship}%
\verse{ובת And the daughter צר of Tyre במנחה with a gift; פניך thy favor. יחלו shall entreat עשׁירי the rich עם׃ among the people}%
\verse{כל all כבודה glorious בת daughter מלך The king's פנימה within: ממשׁבצות of wrought זהב gold. לבושׁה׃ her clothing}%
\verse{לרקמות in raiment of needlework: תובל She shall be brought למלך unto the king בתולות the virgins אחריה that follow רעותיה her companions מובאות׃ her shall be brought}%
\verse{תובלנה shall they be brought: בשׂמחת With gladness וגיל and rejoicing תבאינה they shall enter בהיכל palace. מלך׃ into the king's}%
\verse{תחת Instead אבתיך of thy fathers יהיו shall be בניך thy children, תשׁיתמו whom thou mayest make לשׂרים princes בכל in all הארץ׃ the earth.}%
\verse{אזכירה to be remembered שׁמך I will make thy name בכל in all דר generations: ודר generations: על therefore כן therefore עמים shall the people יהודך praise לעלם thee forever ועד׃ and ever.}%
\end{biblechapter}%
\begin{biblechapter}% Psalm 46
\verseWithHeading{God Provides for and Protects His People}{למנצח To the chief Musician לבני for the sons קרח of Korah, על upon עלמות Alamoth. שׁיר׃ A Song אלהים God לנו מחסה our refuge ועז and strength, עזרה help בצרות in trouble. נמצא present מאד׃ a very}%
\verse{על כןא will not נירא we fear, בהמיר be removed, ארץ though the earth ובמוט be carried הרים and though the mountains בלב into the midst ימים׃ of the sea;}%
\verse{יהמו thereof roar יחמרו be troubled, מימיו the waters ירעשׁו shake הרים the mountains בגאותו with the swelling סלה׃ thereof. Selah.}%
\verse{נהר a river, פלגיו the streams ישׂמחו whereof shall make glad עיר the city אלהים of God, קדשׁ the holy משׁכני of the tabernacles עליון׃ of the most High.}%
\verse{אלהים God בקרבה in the midst בל of her; she shall not תמוט be moved: יעזרה shall help אלהים God לפנות her, right בקר׃ early.}%
\verse{המו raged, גוים The heathen מטו were moved: ממלכות the kingdoms נתן he uttered בקולו his voice, תמוג melted. ארץ׃ the earth}%
\verse{יהוה The LORD צבאות of hosts עמנו with משׂגב our refuge. לנו אלהי us; the God יעקב of Jacob סלה׃ Selah.}%
\verse{לכו Come, חזו behold מפעלות the works יהוה of the LORD, אשׁר what שׂם he hath made שׁמות desolations בארץ׃ in the earth.}%
\verse{משׁבית to cease מלחמות He maketh wars עד unto קצה the end הארץ of the earth; קשׁת the bow, ישׁבר he breaketh וקצץ and cutteth חנית the spear עגלות the chariot ישׂרף in sunder; he burneth באשׁ׃ in the fire.}%
\verse{הרפו Be still, ודעו and know כי that אנכי I אלהים God: ארום I will be exalted בגוים among the heathen, ארום I will be exalted בארץ׃ in the earth.}%
\verse{יהוה The LORD צבאות of hosts עמנו with משׂגב our refuge. לנו אלהי us; the God יעקב of Jacob סלה׃ Selah.}%
\end{biblechapter}%
\begin{biblechapter}% Psalm 47
\verseWithHeading{God Is King over All the Earth}{למנצח To the chief Musician, לבני for the sons קרח of Korah. מזמור׃ A Psalm כל all העמים ye people; תקעו O clap כף your hands, הריעו shout לאלהים unto God בקול with the voice רנה׃ of triumph.}%
\verse{כי For יהוה the LORD עליון most high נורא terrible; מלך King גדול a great על over כל all הארץ׃ the earth.}%
\verse{ידבר He shall subdue עמים the people תחתינו under ולאמים us, and the nations תחת under רגלינו׃ our feet.}%
\verse{יבחר He shall choose לנו את נחלתנו our inheritance את גאון for us, the excellency יעקב of Jacob אשׁר whom אהב he loved. סלה׃ Selah.}%
\verse{עלה is gone up אלהים God בתרועה with a shout, יהוה the LORD בקול with the sound שׁופר׃ of a trumpet.}%
\verse{זמרו Sing praises אלהים to God, זמרו sing praises: זמרו sing praises למלכנו unto our King, זמרו׃ sing praises.}%
\verse{כי For מלך the King כל of all הארץ the earth: אלהים God זמרו sing ye praises משׂכיל׃ with understanding.}%
\verse{מלך reigneth אלהים God על over גוים the heathen: אלהים God ישׁב sitteth על upon כסא the throne קדשׁו׃ of his holiness.}%
\verse{נדיבי The princes עמים of the people נאספו are gathered together, עם the people אלהי of the God אברהם of Abraham: כי for לאלהים unto God: מגני the shields ארץ of the earth מאד he is greatly נעלה׃ exalted.}%
\end{biblechapter}%
\begin{biblechapter}% Psalm 48
\verseWithHeading{The Greatness of God in Zion}{שׁיר A Song מזמור Psalm לבני for the sons קרח׃ of Korah. גדול Great יהוה the LORD, ומהלל to be praised מאד and greatly בעיר in the city אלהינו of our God, הר the mountain קדשׁו׃ of his holiness.}%
\verse{יפה Beautiful נוף for situation, משׂושׂ the joy כל of the whole הארץ earth, הר mount ציון Zion, ירכתי the sides צפון of the north, קרית the city מלך King. רב׃ of the great}%
\verse{אלהים God בארמנותיה in her palaces נודע is known למשׂגב׃ for a refuge.}%
\verse{כי For, הנה lo, המלכים the kings נועדו were assembled, עברו they passed by יחדו׃ together.}%
\verse{המה They ראו saw כן so תמהו they marveled; נבהלו they were troubled, נחפזו׃ hasted away.}%
\verse{רעדה Fear אחזתם took hold upon שׁם them there, חיל pain, כיולדה׃ as of a woman in travail.}%
\verse{ברוח wind. קדים with an east תשׁבר Thou breakest אניות the ships תרשׁישׁ׃ of Tarshish}%
\verse{כאשׁר As שׁמענו we have heard, כן so ראינו have we seen בעיר in the city יהוה צבאות of hosts, בעיר in the city אלהינו of our God: אלהים God יכוננה will establish עד it forever. עולם it forever. סלה׃ Selah.}%
\verse{דמינו We have thought אלהים O God, חסדך of thy lovingkindness, בקרב in the midst היכלך׃ of thy temple.}%
\verse{כשׁמך According to thy name, אלהים O God, כן so תהלתך thy praise על unto קצוי the ends ארץ of the earth: צדק of righteousness. מלאה is full ימינך׃ thy right hand}%
\verse{ישׂמח rejoice, הר Let mount ציון Zion תגלנה be glad, בנות let the daughters יהודה of Judah למען because of משׁפטיך׃ thy judgments.}%
\verse{סבו Walk about ציון Zion, והקיפוה and go round about ספרו her: tell מגדליה׃ the towers}%
\verse{שׁיתו לבכםחילה her bulwarks, פסגו consider ארמנותיה her palaces; למען that תספרו ye may tell לדור to the generation אחרון׃ following.}%
\verse{כי For זה this אלהים God אלהינו our God עולם forever ועד and ever: הוא he ינהגנו will be our guide על unto מות׃}%
\end{biblechapter}%
\begin{biblechapter}% Psalm 49
\verseWithHeading{Wealth and the Fate of the Wicked}{למנצח To the chief Musician, לבני for the sons קרח of Korah. מזמור׃ A Psalm שׁמעו Hear זאת this, כל all העמים people; האזינו give ear, כל all ישׁבי inhabitants חלד׃ of the world:}%
\verse{גם Both בני אדם low גם and בני אישׁ high, יחד together. עשׁיר rich ואביון׃ and poor,}%
\verse{פי My mouth ידבר shall speak חכמות of wisdom; והגות and the meditation לבי of my heart תבונות׃ of understanding.}%
\verse{אטה I will incline למשׁל to a parable: אזני mine ear אפתח I will open בכנור upon the harp. חידתי׃ my dark saying}%
\verse{למה Wherefore אירא should I fear בימי in the days רע of evil, עון the iniquity עקבי יסובני׃ shall compass me about?}%
\verse{הבטחים They that trust על in חילם their wealth, וברב in the multitude עשׁרם of their riches; יתהללו׃ and boast themselves}%
\verse{אח his brother, לא None פדה can by any means יפדה redeem אישׁ לא nor יתן give לאלהים to God כפרו׃ a ransom}%
\verse{ויקר precious, פדיון (For the redemption נפשׁם of their soul וחדל and it ceaseth לעולם׃ forever:)}%
\verse{ויחי live עוד That he should still לנצח forever, לא not יראה see השׁחת׃ corruption.}%
\verse{כי For יראה he seeth חכמים wise men ימותו die, יחד likewise כסיל the fool ובער and the brutish person יאבדו perish, ועזבו and leave לאחרים to others. חילם׃ their wealth}%
\verse{קרבם Their inward thought בתימו their houses לעולם forever, משׁכנתם their dwelling places לדר to all generations; ודר to all generations; קראו בשׁמותם after their own names. עלי they call אדמות׃ lands}%
\verse{ואדם Nevertheless man ביקר in honor בל not: ילין abideth נמשׁל he is like כבהמות the beasts נדמו׃ perish.}%
\verse{זה This דרכם their way כסל their folly: למו ואחריהם yet their posterity בפיהם their sayings. ירצו approve סלה׃ Selah.}%
\verse{כצאן Like sheep לשׁאול in the grave; שׁתו they are laid מות death ירעם shall feed וירדו shall have dominion בם ישׁרים on them; and the upright לבקר over them in the morning; וצירם לבלות shall consume שׁאול in the grave מזבל׃ from their dwelling.}%
\verse{אך But אלהים God יפדה will redeem נפשׁי my soul מיד from the power שׁאול of the grave: כי for יקחני he shall receive סלה׃ me. Selah.}%
\verse{אל Be not תירא thou afraid כי when יעשׁר is made rich, אישׁ one כי when ירבה is increased; כבוד the glory ביתו׃ of his house}%
\verse{כי For when לא nothing במותו he dieth יקח he shall carry הכל nothing לא shall not ירד descend אחריו after כבודו׃ away: his glory}%
\verse{כי Though נפשׁו his soul: בחייו while he lived יברך he blessed ויודך and will praise כי thee, when תיטיב׃ thou doest well}%
\verse{תבוא He shall go עד to דור the generation אבותיו of his fathers; עד they shall never נצח לאראו see אור׃ light.}%
\verse{אדם Man ביקר in honor, ולא not, יבין and understandeth נמשׁל is like כבהמות the beasts נדמו׃ perish.}%
\end{biblechapter}%
\begin{biblechapter}% Psalm 50
\verseWithHeading{An Oracle Concerning Sacrifices}{מזמור A Psalm לאסף of Asaph. אל The mighty אלהים God, יהוה the LORD, דבר hath spoken, ויקרא and called ארץ the earth ממזרח from the rising שׁמשׁ of the sun עד unto מבאו׃ the going down}%
\verse{מציון מכלל the perfection יפי of beauty, אלהים God הופיע׃ hath shined.}%
\verse{יבא shall come, אלהינו Our God ואל and shall not יחרשׁ keep silence: אשׁ a fire לפניו before תאכל shall devour וסביביו round about נשׂערה tempestuous מאד׃ him, and it shall be very}%
\verse{יקרא He shall call אל to השׁמים the heavens מעל from above, ואל and to הארץ the earth, לדין that he may judge עמו׃ his people.}%
\verse{אספו לי חסידירתי unto me; those that have made בריתי a covenant עלי with me by זבח׃ sacrifice.}%
\verse{ויגידו shall declare שׁמים And the heavens צדקו his righteousness: כי for אלהים God שׁפט judge הוא himself. סלה׃ Selah.}%
\verse{שׁמעה Hear, עמי O my people, ואדברה and I will speak; ישׂראל O Israel, ואעידה and I will testify בך אלהים God, אלהיך thy God. אנכי׃ against thee: I}%
\verse{לא I will not על thee for זבחיך thy sacrifices אוכיחך reprove ועולתיך or thy burnt offerings, לנגדי before תמיד׃ continually}%
\verse{לא no אקח I will take מביתך out of thy house, פר bullock ממכלאתיך out of thy folds. עתודים׃ he goats}%
\verse{כי For לי כל every חיתו beast יער of the forest בהמות mine, the cattle בהררי hills. אלף׃ upon a thousand}%
\verse{ידעתי I know כל all עוף the fowls הרים of the mountains: וזיז and the wild beasts שׂדי of the field עמדי׃ mine.}%
\verse{אם If ארעב I were hungry, לא I would not אמר tell לך כי thee: for לי תבל the world ומלאה׃ mine, and the fullness}%
\verse{האוכל Will I eat בשׂר the flesh אבירים of bulls, ודם the blood עתודים of goats? אשׁתה׃ or drink}%
\verse{זבח Offer לאלהים unto God תודה thanksgiving; ושׁלם and pay לעליון unto the most High: נדריך׃ thy vows}%
\verse{וקראני And call upon ביום me in the day צרה of trouble: אחלצך I will deliver ותכבדני׃ thee, and thou shalt glorify}%
\verse{ולרשׁע But unto the wicked אמר saith, אלהים God מה What לך לספר hast thou to do to declare חקי my statutes, ותשׂא or thou shouldest take בריתי my covenant עלי in פיך׃ thy mouth?}%
\verse{ואתה Seeing thou שׂנאת hatest מוסר instruction, ותשׁלך and castest דברי my words אחריך׃ behind}%
\verse{אם When ראית thou sawest גנב a thief, ותרץ then thou consentedst עמו with ועם with מנאפים adulterers. חלקך׃ him, and hast been partaker}%
\verse{פיך thy mouth שׁלחת Thou givest ברעה to evil, ולשׁונך and thy tongue תצמיד frameth מרמה׃ deceit.}%
\verse{תשׁב Thou sittest באחיך against thy brother; תדבר speakest בבן son. אמך thine own mother's תתן thou slanderest דפי׃ thou slanderest}%
\verse{אלה These עשׂית hast thou done, והחרשׁתי and I kept silence; דמית thou thoughtest היות that I was אהיה altogether כמוך as thyself: אוכיחך I will reprove ואערכה thee, and set in order לעיניך׃ before thine eyes.}%
\verse{בינו consider נא Now זאת this, שׁכחי ye that forget אלוה God, פן lest אטרף I tear in pieces, ואין and none מציל׃ to deliver.}%
\verse{זבח Whoso offereth תודה praise יכבדנני glorifieth ושׂם me: and to him that ordereth דרך conversation אראנו will I show בישׁע the salvation אלהים׃ of God.}%
\end{biblechapter}%
\begin{biblechapter}% Psalm 51
\verseWithHeading{A Prayer of Repentance and Plea for Mercy}{למנצח To the chief Musician, מזמור A Psalm לדוד׃ of David, בבוא came אליו unto נתן when Nathan הנביא the prophet כאשׁר him, after בא he had gone in אל to בת־שׁבע׃ Bathsheba. חנני Have mercy אלהים upon me, O God, כחסדך according to thy lovingkindness: כרב according unto the multitude רחמיך of thy tender mercies מחה blot out פשׁעי׃ my transgressions.}%
\verse{הרבה me throughly כבסני Wash מעוני from mine iniquity, ומחטאתי me from my sin. טהרני׃ and cleanse}%
\verse{כי For פשׁעי my transgressions: אני I אדע acknowledge וחטאתי and my sin נגדי before תמיד׃ ever}%
\verse{לך לבדך Against thee, thee only, חטאתי have I sinned, והרע evil בעיניך in thy sight: עשׂיתי and done למען that תצדק thou mightest be justified בדברך when thou speakest, תזכה be clear בשׁפטך׃ when thou judgest.}%
\verse{הן Behold, בעוון in iniquity; חוללתי I was shapen ובחטא and in sin יחמתני conceive אמי׃ did my mother}%
\verse{הן Behold, אמת truth חפצת thou desirest בטחות in the inward parts: ובסתם and in the hidden חכמה wisdom. תודיעני׃ thou shalt make me to know}%
\verse{תחטאני Purge באזוב me with hyssop, ואטהר and I shall be clean: תכבסני wash ומשׁלג than snow. אלבין׃ me, and I shall be whiter}%
\verse{תשׁמיעני Make me to hear שׂשׂון joy ושׂמחה and gladness; תגלנה may rejoice. עצמות the bones דכית׃ thou hast broken}%
\verse{הסתר Hide פניך thy face מחטאי from my sins, וכל all עונתי mine iniquities. מחה׃ and blot out}%
\verse{לב heart, טהור in me a clean ברא Create לי אלהים O God; ורוח spirit נכון a right חדשׁ and renew בקרבי׃ within}%
\verse{אל me not תשׁליכני Cast מלפניך away from thy presence; ורוח spirit קדשׁך thy holy אל not תקח and take ממני׃ away from thy presence;}%
\verse{השׁיבה Restore לי שׂשׂון unto me the joy ישׁעך of thy salvation; ורוח spirit. נדיבה תסמכני׃ and uphold}%
\verse{אלמדה will I teach פשׁעים transgressors דרכיך thy ways; וחטאים and sinners אליך unto ישׁובו׃ shall be converted}%
\verse{הצילני Deliver מדמים me from bloodguiltiness, אלהים O God, אלהי thou God תשׁועתי of my salvation: תרנן shall sing aloud לשׁוני my tongue צדקתך׃ of thy righteousness.}%
\verse{אדני O Lord, שׂפתי thou my lips; תפתח open ופי and my mouth יגיד shall show forth תהלתך׃ thy praise.}%
\verse{כי For לא not תחפץ thou desirest זבח sacrifice; ואתנה else would I give עולה in burnt offering. לא not תרצה׃ thou delightest}%
\verse{זבחי The sacrifices אלהים of God רוח spirit: נשׁברה a broken לב heart, נשׁבר a broken ונדכה and a contrite אלהים O God, לא thou wilt not תבזה׃ despise.}%
\verse{היטיבה Do good ברצונך in thy good pleasure את ציון unto Zion: תבנה build חומות thou the walls ירושׁלם׃ of Jerusalem.}%
\verse{אז Then תחפץ shalt thou be pleased זבחי with the sacrifices צדק of righteousness, עולה with burnt offering וכליל and whole burnt offering: אז then יעלו shall they offer על upon מזבחך thine altar. פרים׃ bullocks}%
\end{biblechapter}%
\begin{biblechapter}% Psalm 52
\verseWithHeading{God’s Judgment on the Wicked and Love for the Faithful}{למנצח To the chief Musician, משׂכיל Maschil, לדוד׃ of David, בבוא came דואג when Doeg האדמי the Edomite ויגד and told לשׁאול Saul, ויאמר and said לו בא is come דוד unto him, David אל to בית the house אחימלך׃ of Ahimelech. מה Why תתהלל boastest thou thyself ברעה in mischief, הגבור O mighty man? חסד the goodness אל of God כל continually. היום׃ continually.}%
\verse{הוות mischiefs; תחשׁב deviseth לשׁונך Thy tongue כתער razor, מלטשׁ like a sharp עשׂה working רמיה׃ deceitfully.}%
\verse{אהבת Thou lovest רע evil מטוב more than good; שׁקר lying מדבר rather than to speak צדק righteousness. סלה׃ Selah.}%
\verse{אהבת Thou lovest כל all דברי words, בלע devouring לשׁון tongue. מרמה׃ O deceitful}%
\verse{גם shall likewise אל God יתצך destroy לנצח thee forever, יחתך he shall take thee away, ויסחך and pluck thee out מאהל of dwelling place, ושׁרשׁך and root מארץ thee out of the land חיים of the living. סלה׃ Selah.}%
\verse{ויראו also shall see, צדיקים The righteous וייראו and fear, ועליו at ישׂחקו׃ and shall laugh}%
\verse{הנה Lo, הגבר the man לא not ישׂים made אלהים God מעוזו his strength; ויבטח but trusted ברב in the abundance עשׁרו of his riches, יעז strengthened בהותו׃ himself in his wickedness.}%
\verse{ואני But I כזית olive tree רענן like a green בבית in the house אלהים of God: בטחתי I trust בחסד in the mercy אלהים of God עולם forever ועד׃ and ever.}%
\verse{אודך I will praise לעולם thee forever, כי because עשׂית thou hast done ואקוה and I will wait on שׁמך thy name; כי for טוב good נגד before חסידיך׃ thy saints.}%
\end{biblechapter}%
\begin{biblechapter}% Psalm 53
\verseWithHeading{The Folly of the Godless and Salvation for Israel}{למנצח To the chief Musician על upon מחלת Mahalath, משׂכיל Maschil, לדוד׃ of David. אמר hath said נבל The fool בלבו in his heart, אין no אלהים God. השׁחיתו Corrupt והתעיבו are they, and have done abominable עול iniquity: אין none עשׂה that doeth טוב׃ good.}%
\verse{אלהים God משׁמים from heaven השׁקיף looked down על upon בני the children אדם of men, לראות to see הישׁ if there were משׂכיל that did understand, דרשׁ that did seek את אלהים׃ God.}%
\verse{כלו Every one סג of them is gone back: יחדו they are altogether נאלחו become filthy; אין none עשׂה that doeth טוב good, אין no, גם not אחד׃ one.}%
\verse{הלא no ידעו knowledge? פעלי Have the workers און of iniquity אכלי who eat up עמי my people אכלו they eat לחם bread: אלהים God. לא they have not קראו׃ called upon}%
\verse{שׁם There פחדו were they in great fear, פחד were they in great fear, לא no היה was: פחד fear כי for אלהים God פזר hath scattered עצמות the bones חנך of him that encampeth הבשׁתה thee: thou hast put to shame, כי because אלהים God מאסם׃ hath despised}%
\verse{מי יתןציון ישׁעות the salvation ישׂראל of Israel בשׁוב bringeth back אלהים When God שׁבות the captivity עמו of his people, יגל shall rejoice, יעקב Jacob ישׂמח shall be glad. ישׂראל׃ Israel}%
\end{biblechapter}%
\begin{biblechapter}% Psalm 54
\verseWithHeading{Answered Prayer for Deliverance from Adversaries}{למנצח To the chief Musician בנגינת on Neginoth, משׂכיל Maschil, לדוד׃ of David, בבוא came הזיפים when the Ziphims ויאמרו and said לשׁאול to Saul, הלא Doth not דוד David מסתתר hide himself עמנו׃ with אלהים me, O God, בשׁמך by thy name, הושׁיעני us? Save ובגבורתך me by thy strength. תדינני׃ and judge}%
\verse{אלהים O God; שׁמע Hear תפלתי my prayer, האזינה give ear לאמרי to the words פי׃ of my mouth.}%
\verse{כי For זרים strangers קמו are risen up עלי against ועריצים me, and oppressors בקשׁו seek after נפשׁי my soul: לא they have not שׂמו set אלהים God לנגדם before סלה׃ them. Selah.}%
\verse{הנה Behold, אלהים God עזר mine helper: לי אדני the Lord בסמכי with them that uphold נפשׁי׃ my soul.}%
\verse{ישׁוב He shall reward הרע evil לשׁררי unto mine enemies: באמתך in thy truth. הצמיתם׃ cut them off}%
\verse{בנדבה I will freely אזבחה sacrifice לך אודה unto thee: I will praise שׁמך thy name, יהוה O LORD; כי for טוב׃ good.}%
\verse{כי For מכל me out of all צרה trouble: הצילני he hath delivered ובאיבי upon mine enemies. ראתה hath seen עיני׃ and mine eye}%
\end{biblechapter}%
\begin{biblechapter}% Psalm 55
\verseWithHeading{Betrayal of a Friend and Trust in God}{למנצח To the chief Musician בנגינת on Neginoth, משׂכיל Maschil, לדוד׃ of David. האזינה Give ear אלהים O God; תפלתי to my prayer, ואל and hide not thyself תתעלם and hide not thyself מתחנתי׃ from my supplication.}%
\verse{הקשׁיבה Attend לי וענני unto me, and hear אריד me: I mourn בשׂיחי in my complaint, ואהימה׃ and make a noise;}%
\verse{מקול אויב of the enemy, מפני because עקת of the oppression רשׁע of the wicked: כי for ימיטו they cast עלי upon און iniquity ובאף me, and in wrath ישׂטמוני׃ they hate}%
\verse{לבי My heart יחיל is sore pained בקרבי within ואימות me: and the terrors מות of death נפלו are fallen עלי׃ upon}%
\verse{יראה Fearfulness ורעד and trembling יבא are come בי ותכסני hath overwhelmed פלצות׃ upon me, and horror}%
\verse{ואמר And I said, מי Oh that יתן I had לי אבר wings כיונה like a dove! אעופה would I fly away, ואשׁכנה׃ and be at rest.}%
\verse{הנה Lo, ארחיק far off, נדד would I wander אלין remain במדבר in the wilderness. סלה׃ Selah.}%
\verse{אחישׁה I would hasten מפלט my escape לי מרוח from the windy סעה storm מסער׃ tempest.}%
\verse{בלע Destroy, אדני O Lord, פלג divide לשׁונם their tongues: כי for ראיתי I have seen חמס violence וריב and strife בעיר׃ in the city.}%
\verse{יומם Day ולילה and night יסובבה they go about על it upon חומתיה the walls ואון thereof: mischief ועמל also and sorrow בקרבה׃ in the midst}%
\verse{הוות Wickedness בקרבה in the midst ולא not ימישׁ depart מרחבה her streets. תך thereof: deceit ומרמה׃ and guile}%
\verse{כי For לא not אויב an enemy יחרפני reproached ואשׂא me; then I could have borne לא neither משׂנאי he that hated עלי against הגדיל me did magnify ואסתר me; then I would have hid myself ממנו׃ from}%
\verse{ואתה But thou, אנושׁ a man כערכי mine equal, אלופי my guide, ומידעי׃ and mine acquaintance.}%
\verse{אשׁר יחדו together, נמתיק We took sweet סוד counsel בבית unto the house אלהים of God נהלך walked ברגשׁ׃ in company.}%
\verse{ישׁימות עלימו upon ירדו them, let them go down שׁאול into hell: חיים quick כי for רעות wickedness במגורם in their dwellings, בקרבם׃ among}%
\verse{אני As for me, I אל upon אלהים God; אקרא will call ויהוה and the LORD יושׁיעני׃ shall save}%
\verse{ערב Evening, ובקר and morning, וצהרים and at noon, אשׂיחה will I pray, ואהמה and cry aloud: וישׁמע and he shall hear קולי׃ my voice.}%
\verse{פדה He hath delivered בשׁלום in peace נפשׁי my soul מקרב from the battle לי כי against me: for ברבים many היו there were עמדי׃ with}%
\verse{ישׁמע shall hear, אל God ויענם and afflict וישׁב them, even he that abideth קדם of old. סלה Selah. אשׁר Because אין they have no חליפות changes, למו ולא not יראו therefore they fear אלהים׃ God.}%
\verse{שׁלח He hath put forth ידיו his hands בשׁלמיו against such as be at peace חלל with him: he hath broken בריתו׃ his covenant.}%
\verse{חלקו were smoother מחמאת than butter, פיו of his mouth וקרב but war לבו in his heart: רכו were softer דבריו his words משׁמן than oil, והמה yet they פתחות׃ drawn swords.}%
\verse{השׁלך Cast על upon יהוה the LORD, יהבך thy burden והוא and he יכלכלך shall sustain לא thee: he shall never יתן suffer לעולם thee: he shall never מוט to be moved. לצדיק׃ the righteous}%
\verse{ואתה But thou, אלהים O God, תורדם shalt bring them down לבאר into the pit שׁחת of destruction: אנשׁי דמים bloody ומרמה and deceitful לא shall not יחצו live out half ימיהם their days; ואני but I אבטח׃ will trust}%
\end{biblechapter}%
\begin{biblechapter}% Psalm 56
\verseWithHeading{Prayer for Deliverance and Confidence in God}{למנצח To the chief Musician על upon יונת אלםחקים לדוד of David, מכתם Michtam באחז took אתו פלשׁתים when the Philistines בגת׃ him in Gath. חנני Be merciful אלהים unto me, O God: כי for שׁאפני would swallow me up; אנושׁ man כל daily היום daily לחם he fighting ילחצני׃ oppresseth}%
\verse{שׁאפו swallow up: שׁוררי Mine enemies כל would daily היום would daily כי for רבים many לחמים that fight לי מרום׃ against me, O thou most High.}%
\verse{יום What time אירא I am afraid, אני I אליך in אבטח׃ will trust}%
\verse{באלהים In God אהלל I will praise דברו his word, באלהים in God בטחתי I have put my trust; לא I will not אירא fear מה what יעשׂה can do בשׂר׃ flesh}%
\verse{כל Every היום day דברי my words: יעצבו they wrest עלי against כל all מחשׁבתם their thoughts לרע׃ me for evil.}%
\verse{יגורו They gather themselves together, יצפינו they hide המה themselves, they עקבי my steps, ישׁמרו mark כאשׁר when קוו they wait for נפשׁי׃ my soul.}%
\verse{על by און iniquity? פלט למו באף in anger עמים the people, הורד cast down אלהים׃ O God.}%
\verse{נדי my wanderings: ספרתה Thou tellest אתה thou שׂימה put דמעתי my tears בנאדך into thy bottle: הלא not בספרתך׃ in thy book?}%
\verse{אז then ישׁובו turn אויבי shall mine enemies אחור back: ביום When אקרא I cry זה this ידעתי I know; כי for אלהים׃ God}%
\verse{באלהים In God אהלל will I praise דבר word: ביהוה in the LORD אהלל will I praise דבר׃ word.}%
\verse{באלהים In God בטחתי have I put my trust: לא I will not אירא be afraid מה what יעשׂה can do אדם׃ man}%
\verse{עלי upon אלהים me, O God: נדריך Thy vows אשׁלם I will render תודת׃ praises}%
\verse{כי For הצלת thou hast delivered נפשׁי my soul ממות from death: הלא not רגלי my feet מדחי from falling, להתהלך that I may walk לפני before אלהים God באור in the light החיים׃ of the living.}%
\end{biblechapter}%
\begin{biblechapter}% Psalm 57
\verseWithHeading{Prayer for Rescue from Enemies}{למנצח To the chief Musician, אל תשׁחת Al-taschith, לדוד of David, מכתם Michtam בברחו when he fled מפני from שׁאול Saul במערה׃ in the cave. חנני Be merciful אלהים unto me, O God, חנני be merciful כי unto me: for בך חסיה trusteth נפשׁי my soul ובצל in thee: yea, in the shadow כנפיך of thy wings אחסה will I make my refuge, עד until יעבר be overpast. הוות׃ calamities}%
\verse{אקרא I will cry לאלהים unto God עליון most high; לאל unto God גמר that performeth עלי׃ for}%
\verse{ישׁלח He shall send משׁמים from heaven, ויושׁיעני and save חרף me the reproach שׁאפי of him that would swallow me up. סלה Selah. ישׁלח shall send forth אלהים God חסדו his mercy ואמתו׃ and his truth.}%
\verse{נפשׁי My soul בתוך among לבאם lions: אשׁכבה I lie להטים them that are set on fire, בני the sons אדם of men, שׁניהם whose teeth חנית spears וחצים and arrows, ולשׁונם and their tongue חרב sword. חדה׃ a sharp}%
\verse{רומה Be thou exalted, על above השׁמים the heavens; אלהים O God, על above כל all הארץ the earth. כבודך׃ thy glory}%
\verse{רשׁת a net הכינו They have prepared לפעמי for my steps; כפף is bowed down: נפשׁי my soul כרו they have digged לפני before שׁיחה a pit נפלו whereof they are fallen בתוכה me, into the midst סלה׃ Selah.}%
\verse{נכון is fixed, לבי My heart אלהים O God, נכון is fixed: לבי my heart אשׁירה I will sing ואזמרה׃ and give praise.}%
\verse{עורה Awake up, כבודי my glory; עורה awake, הנבל psaltery וכנור and harp: אעירה I will awake שׁחר׃ early.}%
\verse{אודך I will praise בעמים among the people: אדני thee, O Lord, אזמרך I will sing בל אמים׃ unto thee among the nations.}%
\verse{כי For גדל great עד unto שׁמים the heavens, חסדך thy mercy ועד unto שׁחקים the clouds. אמתך׃ and thy truth}%
\verse{רומה Be thou exalted, על above שׁמים the heavens: אלהים O God, על above כל all הארץ the earth. כבודך׃ thy glory}%
\end{biblechapter}%
\begin{biblechapter}% Psalm 58
\verseWithHeading{Judgment on the Wicked}{למנצח To the chief Musician, אל תשׁחת Al-taschith, לדוד of David. מכתם׃ Michtam האמנם Do ye indeed אלם O congregation? צדק righteousness, תדברון speak מישׁרים uprightly, תשׁפטו do ye judge בני O ye sons אדם׃ of men?}%
\verse{אף Yea, בלב in heart עולת wickedness; תפעלון ye work בארץ in the earth. חמס the violence ידיכם of your hands תפלסון׃ ye weigh}%
\verse{זרו are estranged רשׁעים The wicked מרחם from the womb: תעו they go astray מבטן as soon as they be born, דברי speaking כזב׃ lies.}%
\verse{חמת Their poison למו כדמות like חמת the poison נחשׁ of a serpent: כמו like פתן adder חרשׁ the deaf יאטם stoppeth אזנו׃ her ear;}%
\verse{אשׁר Which לא will not ישׁמע hearken לקול to the voice מלחשׁים of charmers, חובר charming חברים charming מחכם׃ never so wisely.}%
\verse{אלהים O God, הרס Break שׁנימו their teeth, בפימו in their mouth: מלתעות the great teeth כפירים of the young lions, נתץ break out יהוה׃ O LORD.}%
\verse{ימאסו Let them melt away כמו as מים waters יתהלכו run למו ידרך continually: he bendeth חצו his arrows, כמו let them be as יתמללו׃ cut in pieces.}%
\verse{כמו As שׁבלול a snail תמס melteth, יהלך let pass away: נפל the untimely birth אשׁת of a woman, בל they may not חזו see שׁמשׁ׃ the sun.}%
\verse{בטרם Before יבינו can feel סירתיכם your pots אטד the thorns, כמו both חי living, כמו and in חרון wrath. ישׂערנו׃ he shall take them away as with a whirlwind,}%
\verse{ישׂמח shall rejoice צדיק The righteous כי when חזה he seeth נקם the vengeance: פעמיו his feet ירחץ he shall wash בדם in the blood הרשׁע׃ of the wicked.}%
\verse{ויאמר shall say, אדם So that a man אך Verily פרי a reward לצדיק for the righteous: אך verily ישׁ he is אלהים a God שׁפטים that judgeth בארץ׃ in the earth.}%
\end{biblechapter}%
\begin{biblechapter}% Psalm 59
\verseWithHeading{A Prayer for Protection}{למנצח To the chief Musician, אל תשׁחת Al-taschith, לדוד of David; מכתם Michtam בשׁלח sent, שׁאול when Saul וישׁמרו and they watched את הבית the house להמיתו׃ to kill הצילני him. Deliver מאיבי me from mine enemies, אלהי O my God: ממתקוממי me from them that rise up against תשׂגבני׃ defend}%
\verse{הצילני Deliver מפעלי me from the workers און of iniquity, ומאנשׁי דמים me from bloody הושׁיעני׃ and save}%
\verse{כי For, הנה lo, ארבו they lie in wait לנפשׁי for my soul: יגורו are gathered עלי against עזים the mighty לא me; not פשׁעי my transgression, ולא nor חטאתי my sin, יהוה׃ O LORD.}%
\verse{בלי without עון fault: ירוצון They run ויכוננו and prepare themselves עורה awake לקראתי to help וראה׃ me, and behold.}%
\verse{ואתה Thou יהוה therefore, O LORD אלהים God צבאות of hosts, אלהי the God ישׂראל of Israel, הקיצה awake לפקד to visit כל all הגוים the heathen: אל be not תחן merciful כל to any בגדי transgressors. און wicked סלה׃ Selah.}%
\verse{ישׁובו They return לערב at evening: יהמו they make a noise ככלב like a dog, ויסובבו and go round about עיר׃ the city.}%
\verse{הנה Behold, יביעון they belch out בפיהם with their mouth: חרבות swords בשׂפתותיהם in their lips: כי for מי who, שׁמע׃ doth hear?}%
\verse{ואתה But thou, יהוה O LORD, תשׂחק shalt laugh למו תלעג in derision. לכל at them; thou shalt have all גוים׃ the heathen}%
\verse{עזו his strength אליך upon אשׁמרה will I wait כי thee: for אלהים God משׂגבי׃ my defense.}%
\verse{אלהי The God חסדו of my mercy יקדמני shall prevent אלהים me: God יראני shall let me see בשׁררי׃ upon mine enemies.}%
\verse{אל them not, תהרגם Slay פן lest ישׁכחו forget: עמי my people הניעמו scatter בחילך them by thy power; והורידמו and bring them down, מגננו our shield. אדני׃ O Lord}%
\verse{חטאת the sin פימו of their mouth דבר the words שׂפתימו of their lips וילכדו let them even be taken בגאונם in their pride: ומאלה and for cursing ומכחשׁ and lying יספרו׃ they speak.}%
\verse{כלה Consume בחמה in wrath, כלה consume ואינמו that they not וידעו and let them know כי that אלהים God משׁל ruleth ביעקב in Jacob לאפסי unto the ends הארץ of the earth. סלה׃ Selah.}%
\verse{וישׁובו let them return; לערב And at evening יהמו let them make a noise ככלב like a dog, ויסובבו and go round about עיר׃ the city.}%
\verse{המה Let them ינועון wander up and down לאכל for meat, אם if לא they be not ישׂבעו satisfied. וילינו׃ and grudge}%
\verse{ואני But I אשׁיר will sing עזך of thy power; וארנן yea, I will sing aloud 'ecבקר in the morning: חסדך of thy mercy כי for היית thou hast been משׂגב my defense לי ומנוס and refuge ביום in the day צר׃ of my trouble.}%
\verse{עזי thee, O my strength, אליך Unto אזמרה will I sing: כי for אלהים God משׂגבי my defense, אלהי the God חסדי׃ of my mercy.}%
\end{biblechapter}%
\begin{biblechapter}% Psalm 60
\verseWithHeading{A Lament After a Defeat and a Prayer for Restoration}{למנצח To the chief Musician על upon שׁושׁן עדות Shushan-eduth, מכתם Michtam לדוד of David, ללמד׃ to teach; בהצותו when he strove with את ארם נהרים Aram-naharaim ואת ארם צובה and with Aram-zobah, וישׁב returned, יואב when Joab ויך and smote את אדום of Edom בגיא in the valley מלח of salt שׁנים twelve עשׂר twelve אלף׃ thousand. אלהים O God, זנחתנו thou hast cast us off, פרצתנו thou hast scattered אנפת us, thou hast been displeased; תשׁובב O turn}%
\verse{הרעשׁתה to tremble; ארץ Thou hast made the earth פצמתה thou hast broken רפה it: heal שׁבריה the breaches כי thereof; for מטה׃ it shaketh.}%
\verse{הראיתה Thou hast showed עמך thy people קשׁה hard השׁקיתנו things: thou hast made us to drink יין the wine תרעלה׃ of astonishment.}%
\verse{נתתה Thou hast given ליראיך to them that fear נס a banner להתנוסס thee, that it may be displayed מפני because קשׁט of the truth. סלה׃ Selah.}%
\verse{למען That יחלצון may be delivered; ידידיך thy beloved הושׁיעה save ימינך thy right hand, ועננו׃}%
\verse{אלהים God דבר hath spoken בקדשׁו in his holiness; אעלזה I will rejoice, אחלקה I will divide שׁכם Shechem, ועמק the valley סכות of Succoth. אמדד׃ and mete out}%
\verse{לי גלעד Gilead ולי מנשׁה mine, and Manasseh ואפרים mine; Ephraim מעוז also the strength ראשׁי of mine head; יהודה Judah מחקקי׃ my lawgiver;}%
\verse{מואב Moab סיר my washpot; רחצי my washpot; על over אדום Edom אשׁליך will I cast out נעלי my shoe: עלי thou because פלשׁת Philistia, התרעעי׃ triumph}%
\verse{מי Who יבלני will bring עיר city? מצור me the strong מי who נחני will lead עד me into אדום׃ Edom?}%
\verse{הלא not אתה thou, אלהים O God, זנחתנו hadst cast us off? ולא didst not תצא go out אלהים and O God, בצבאותינו׃ with our armies?}%
\verse{הבה Give לנו עזרת us help מצר from trouble: ושׁוא for vain תשׁועת the help אדם׃ of man.}%
\verse{באלהים Through God נעשׂה we shall do חיל valiantly: והוא for he יבוס shall tread down צרינו׃ our enemies.}%
\end{biblechapter}%
\begin{biblechapter}% Psalm 61
\verseWithHeading{Confidence in God’s Protection}{למנצח To the chief Musician על upon נגינת Neginah, לדוד׃ of David. שׁמעה Hear אלהים O God; רנתי my cry, הקשׁיבה attend unto תפלתי׃ my prayer.}%
\verse{מקצה הארץ of the earth אליך unto אקרא will I cry בעטף is overwhelmed: לבי thee, when my heart בצור me to the rock ירום is higher ממני than תנחני׃ lead}%
\verse{כי For היית thou hast been מחסה a shelter לי מגדל tower עז for me, a strong מפני from אויב׃ the enemy.}%
\verse{אגורה I will abide באהלך in thy tabernacle עולמים forever: אחסה I will trust בסתר in the covert כנפיך of thy wings. סלה׃ Selah.}%
\verse{כי For אתה thou, אלהים O God, שׁמעת hast heard לנדרי my vows: נתת thou hast given ירשׁת the heritage יראי of those that fear שׁמך׃ thy name.}%
\verse{ימים life: על ימילך the king's תוסיף Thou wilt prolong שׁנותיו his years כמו as דר many generations. ודר׃ many generations.}%
\verse{ישׁב He shall abide עולם forever: לפני before אלהים God חסד mercy ואמת and truth, מן O prepare ינצרהו׃ may preserve}%
\verse{כן So אזמרה will I sing praise שׁמך unto thy name לעד forever, לשׁלמי perform נדרי my vows. יום יום׃}%
\end{biblechapter}%
\begin{biblechapter}% Psalm 62
\verseWithHeading{Confidence in God’s Salvation}{למנצח To the chief Musician, על to ידותון Jeduthun, מזמור A Psalm לדוד׃ of David. אך Truly אל upon אלהים God: דומיה waiteth נפשׁי my soul ממנו from ישׁועתי׃ him my salvation.}%
\verse{אך only הוא He צורי my rock וישׁועתי and my salvation; משׂגבי my defense; לא I shall not אמוט moved. רבה׃ be greatly}%
\verse{עד אנההותתו will ye imagine mischief על against אישׁ a man? תרצחו ye shall be slain כלכם all כקיר wall נטוי of you: as a bowing גדר fence. הדחויה׃ a tottering}%
\verse{אך They only משׂאתו from his excellency: יעצו consult להדיח to cast down ירצו they delight כזב in lies: בפיו with their mouth, יברכו they bless ובקרבם inwardly. יקללו but they curse סלה׃ Selah.}%
\verse{אך thou only לאלהים upon God; דומי wait נפשׁי My soul, כי for ממנו from תקותי׃ my expectation}%
\verse{אך only הוא He צורי my rock וישׁועתי and my salvation: משׂגבי my defense; לא I shall not אמוט׃ be moved.}%
\verse{על In אלהים God ישׁעי my salvation וכבודי and my glory: צור the rock עזי of my strength, מחסי my refuge, באלהים׃ in God.}%
\verse{בטחו Trust בו בכל in him at all עת times; עם people, שׁפכו pour out לפניו before לבבכם your heart אלהים him: God מחסה a refuge לנו סלה׃ for us. Selah.}%
\verse{אך Surely הבל vanity, בני men of low degree אדם men of low degree כזב a lie: בני men of high degree אישׁ men of high degree במאזנים in the balance, לעלות to be laid המה they מהבל than vanity. יחד׃ altogether}%
\verse{אל not תבטחו Trust בעשׁק in oppression, ובגזל in robbery: אל and become not vain תהבלו and become not vain חיל riches כי if ינוב increase, אל not תשׁיתו set לב׃ your heart}%
\verse{אחת once; דבר hath spoken אלהים God שׁתים twice זו this; שׁמעתי have I heard כי that עז power לאלהים׃ unto God.}%
\verse{ולך אדני Also unto thee, O Lord, חסד mercy: כי for אתה thou תשׁלם renderest לאישׁ to every man כמעשׂהו׃ according to his work.}%
\end{biblechapter}%
\begin{biblechapter}% Psalm 63
\verseWithHeading{Longing for God}{מזמור A Psalm לדוד of David, בהיותו when he was במדבר in the wilderness יהודה׃ of Judah. אלהים O God, אלי my God; אתה thou אשׁחרך early will I seek צמאה thirsteth לך נפשׁי thee: my soul כמה longeth לך בשׂרי for thee, my flesh בארץ land, ציה for thee in a dry ועיף and thirsty בלי where no מים׃ water}%
\verse{כן so בקדשׁ thee in the sanctuary. חזיתיך I have seen לראות To see עזך thy power וכבודך׃ and thy glory,}%
\verse{כי Because טוב better חסדך thy lovingkindness מחיים than life, שׂפתי my lips ישׁבחונך׃ shall praise}%
\verse{כן Thus אברכך will I bless בחיי thee while I live: בשׁמך in thy name. אשׂא I will lift up כפי׃ my hands}%
\verse{כמו as חלב marrow ודשׁן and fatness; תשׂבע shall be satisfied נפשׁי My soul ושׂפתי lips: רננות with joyful יהלל shall praise פי׃ and my mouth}%
\verse{אם When זכרתיך I remember על thee upon יצועי my bed, באשׁמרות on thee in the watches. אהגה׃ meditate}%
\verse{כי Because היית thou hast been עזרתה my help, לי ובצל therefore in the shadow כנפיך of thy wings ארנן׃ will I rejoice.}%
\verse{דבקה followeth hard נפשׁי My soul אחריך after בי תמכה upholdeth ימינך׃ thee: thy right hand}%
\verse{והמה But those לשׁואה to destroy יבקשׁו seek נפשׁי my soul, יבאו shall go בתחתיות into the lower parts הארץ׃ of the earth.}%
\verse{יגירהו They shall fall על by ידי by חרב the sword: מנת a portion שׁעלים for foxes. יהיו׃ they shall be}%
\verse{והמלך But the king ישׂמח shall rejoice באלהים in God; יתהלל by him shall glory: כל every one הנשׁבע that sweareth בו כי but יסכר shall be stopped. פי the mouth דוברי of them that speak שׁקר׃ lies}%
\end{biblechapter}%
\begin{biblechapter}% Psalm 64
\verseWithHeading{A Plea for Divine Retribution}{למנצח To the chief Musician, מזמור A Psalm לדוד׃ of David. שׁמע Hear אלהים O God, קולי my voice, בשׂיחי in my prayer: מפחד from fear אויב of the enemy. תצר preserve חיי׃ my life}%
\verse{תסתירני Hide מסוד me from the secret counsel מרעים of the wicked; מרגשׁת from the insurrection פעלי of the workers און׃ of iniquity:}%
\verse{אשׁר Who שׁננו whet כחרב like a sword, לשׁונם their tongue דרכו bend חצם their arrows, דבר words: מר׃ bitter}%
\verse{לירות That they may shoot במסתרים in secret תם at the perfect: פתאם suddenly ירהו do they shoot ולא not. ייראו׃ at him, and fear}%
\verse{יחזקו They encourage למו דבר matter: רע themselves an evil יספרו they commune לטמון of laying snares privily; מוקשׁים of laying snares privily; אמרו they say, מי Who יראה׃ shall see}%
\verse{יחפשׂו They search out עולת iniquities; תמנו they accomplish חפשׂ a diligent search: מחפשׂ a diligent search: וקרב both the inward אישׁ of every one ולב and the heart, עמק׃ deep.}%
\verse{וירם shall shoot אלהים But God חץ at them an arrow; פתאום suddenly היו shall they be מכותם׃ wounded.}%
\verse{ויכשׁילוהו to fall עלימו upon לשׁונם So they shall make their own tongue יתנדדו them shall flee away. כל themselves: all ראה׃ that see}%
\verse{וייראו shall fear, כל And all אדם men ויגידו and shall declare פעל the work אלהים of God; ומעשׂהו of his doing. השׂכילו׃ for they shall wisely consider}%
\verse{ישׂמח shall be glad צדיק The righteous ביהוה וחסה and shall trust בו ויתהללו shall glory. כל in him; and all ישׁרי the upright לב׃ in heart}%
\end{biblechapter}%
\begin{biblechapter}% Psalm 65
\verseWithHeading{Thanksgiving for God’s Provision}{למנצח To the chief Musician, מזמור A Psalm לדוד of David. שׁיר׃ Song לך דמיה waiteth תהלה Praise אלהים for thee, O God, בציון in Zion: ולך ישׁלם be performed. נדר׃ and unto thee shall the vow}%
\verse{שׁמע O thou that hearest תפלה prayer, עדיך unto כל thee shall all בשׂר flesh יבאו׃ come.}%
\verse{דברי עונתברו prevail מני against פשׁעינו me: our transgressions, אתה thou תכפרם׃ shalt purge them away.}%
\verse{אשׁרי Blessed תבחר thou choosest, ותקרב and causest to approach ישׁכן he may dwell חצריך in thy courts: נשׂבעה we shall be satisfied בטוב with the goodness ביתך of thy house, קדשׁ of thy holy היכלך׃ temple.}%
\verse{נוראות terrible things בצדק in righteousness תעננו wilt thou answer אלהי us, O God ישׁענו of our salvation; מבטח the confidence כל of all קצוי the ends ארץ of the earth, וים the sea: רחקים׃ and of them that are afar off}%
\verse{מכין setteth fast הרים the mountains; בכחו Which by his strength נאזר girded בגבורה׃ with power:}%
\verse{משׁביח Which stilleth שׁאון the noise ימים of the seas, שׁאון the noise גליהם of their waves, והמון and the tumult לאמים׃ of the people.}%
\verse{וייראו are afraid ישׁבי They also that dwell קצות in the uttermost parts מאותתיך at thy tokens: מוצאי thou makest the outgoings בקר of the morning וערב and evening תרנין׃ to rejoice.}%
\verse{פקדת Thou visitest הארץ the earth, ותשׁקקה and waterest רבת it: thou greatly תעשׁרנה enrichest פלג it with the river אלהים of God, מלא is full מים of water: תכין thou preparest דגנם them corn, כי when כן thou hast so תכינה׃ provided}%
\verse{תלמיה Thou waterest the ridges רוה thereof abundantly: נחת thou settlest גדודיה the furrows ברביבים with showers: תמגגנה thereof: thou makest it soft צמחה the springing תברך׃ thou blessest}%
\verse{עטרת Thou crownest שׁנת the year טובתך with thy goodness; ומעגליך and thy paths ירעפון drop דשׁן׃ fatness.}%
\verse{ירעפו They drop נאות the pastures מדבר of the wilderness: וגיל rejoice גבעות and the little hills תחגרנה׃ on every side.}%
\verse{לבשׁו are clothed כרים The pastures הצאן with flocks; ועמקים the valleys יעטפו also are covered over בר with corn; יתרועעו they shout for joy, אף they also ישׁירו׃ sing.}%
\end{biblechapter}%
\begin{biblechapter}% Psalm 66
\verseWithHeading{Thanksgiving to God for His Works}{למנצח To the chief Musician, שׁיר A Song מזמור Psalm. הריעו Make a joyful noise לאלהים unto God, כל all הארץ׃ ye lands:}%
\verse{זמרו Sing forth כבוד the honor שׁמו of his name: שׂימו make כבוד glorious. תהלתו׃ his praise}%
\verse{אמרו Say לאלהים unto God, מה How נורא terrible מעשׂיך thy works! ברב through the greatness עזך of thy power יכחשׁו submit לך איביך׃ shall thine enemies}%
\verse{כל All הארץ the earth ישׁתחוו shall worship לך ויזמרו thee, and shall sing לך יזמרו unto thee; they shall sing שׁמך thy name. סלה׃ Selah.}%
\verse{לכו Come וראו and see מפעלות the works אלהים of God: נורא terrible עלילה doing על toward בני the children אדם׃ of men.}%
\verse{הפך He turned ים the sea ליבשׁה into dry בנהר through the flood יעברו they went ברגל on foot: שׁם there נשׂמחה׃ did we rejoice}%
\verse{משׁל He ruleth בגבורתו by his power עולם forever; עיניו his eyes בגוים the nations: תצפינה behold הסוררים the rebellious אל let not ירימו exalt למו סלה׃ themselves. Selah.}%
\verse{ברכו O bless עמים ye people, אלהינו our God, והשׁמיעו to be heard: קול and make the voice תהלתו׃ of his praise}%
\verse{השׂם Which holdeth נפשׁנו our soul בחיים in life, ולא not נתן and suffereth למוט to be moved. רגלנו׃ our feet}%
\verse{כי For בחנתנו hast proved אלהים thou, O God, צרפתנו us: thou hast tried כצרף is tried. כסף׃ us, as silver}%
\verse{הבאתנו Thou broughtest במצודה us into the net; שׂמת thou laidst מועקה affliction במתנינו׃ upon our loins.}%
\verse{הרכבת to ride אנושׁ Thou hast caused men לראשׁנו over our heads; באנו we went באשׁ through fire ובמים and through water: ותוציאנו but thou broughtest us out לרויה׃ into a wealthy}%
\verse{אבוא I will go ביתך into thy house בעולות with burnt offerings: אשׁלם I will pay לך נדרי׃ thee my vows,}%
\verse{אשׁר Which פצו have uttered, שׂפתי my lips ודבר hath spoken, פי and my mouth בצר׃ when I was in trouble.}%
\verse{עלות unto thee burnt sacrifices מחים of fatlings, אעלה I will offer לך עם with קטרת the incense אילים of rams; אעשׂה I will offer בקר bullocks עם with עתודים goats. סלה׃ Selah.}%
\verse{לכו Come שׁמעו hear, ואספרה and I will declare כל all יראי ye that fear אלהים God, אשׁר what עשׂה he hath done לנפשׁי׃ for my soul.}%
\verse{אליו unto פי him with my mouth, קראתי I cried ורומם and he was extolled תחת with לשׁוני׃ my tongue.}%
\verse{און iniquity אם If ראיתי I regard בלבי in my heart, לא will not ישׁמע hear אדני׃ the Lord}%
\verse{אכן verily שׁמע hath heard אלהים God הקשׁיב he hath attended בקול to the voice תפלתי׃ of my prayer.}%
\verse{ברוך Blessed אלהים God, אשׁר which לא hath not הסיר turned away תפלתי my prayer, וחסדו nor his mercy מאתי׃ from}%
\end{biblechapter}%
\begin{biblechapter}% Psalm 67
\verseWithHeading{A Prayer of Blessing}{למנצח To the chief Musician בנגינת מזמור A Psalm שׁיר׃ Song. אלהים God יחננו be merciful ויברכנו unto us, and bless יאר to shine פניו us; cause his face אתנו upon סלה׃ us; Selah.}%
\verse{לדעת may be known בארץ upon earth, דרכך That thy way בכל among all גוים nations. ישׁועתך׃ thy saving health}%
\verse{יודוך praise עמים Let the people אלהים thee, O God; יודוך praise עמים the people כלם׃ let all}%
\verse{ישׂמחו be glad וירננו and sing for joy: לאמים O let the nations כי for תשׁפט thou shalt judge עמים the people מישׁור righteously, ולאמים the nations בארץ upon earth. תנחם and govern סלה׃ Selah.}%
\verse{יודוך praise עמים Let the people אלהים thee, O God; יודוך praise עמים the people כלם׃ let all}%
\verse{ארץ shall the earth נתנה yield יבולה her increase; יברכנו shall bless אלהים God, אלהינו׃ our own God,}%
\verse{יברכנו shall bless אלהים God וייראו shall fear אתו כל us; and all אפסי the ends ארץ׃ of the earth}%
\end{biblechapter}%
\begin{biblechapter}% Psalm 68
\verseWithHeading{Praise to God for Providing Victory}{למנצח To the chief Musician, לדוד of David. מזמור A Psalm שׁיר׃ Song יקום arise, אלהים Let God יפוצו be scattered: אויביו let his enemies וינוסו him flee משׂנאיו let them also that hate מפניו׃ before}%
\verse{כהנדף is driven away, עשׁן As smoke תנדף drive away: כהמס melteth דונג as wax מפני before אשׁ the fire, יאבדו perish רשׁעים let the wicked מפני at the presence אלהים׃ of God.}%
\verse{וצדיקים But let the righteous ישׂמחו be glad; יעלצו let them rejoice לפני before אלהים God: וישׂישׂו rejoice. בשׂמחה׃ yea, let them exceedingly}%
\verse{שׁירו Sing לאלהים unto God, זמרו sing praises שׁמו to his name: סלו extol לרכב him that rideth בערבות upon the heavens ביה JAH, שׁמו by his name ועלזו and rejoice לפניו׃ before}%
\verse{אבי A father יתומים of the fatherless, ודין and a judge אלמנות of the widows, אלהים God במעון habitation. קדשׁו׃ in his holy}%
\verse{אלהים God מושׁיב setteth יחידים the solitary ביתה in families: מוציא he bringeth out אסירים those which are bound בכושׁרות with chains: אך but סוררים the rebellious שׁכנו dwell צחיחה׃ in a dry}%
\verse{אלהים O God, בצאתך when thou wentest forth לפני before עמך thy people, בצעדך when thou didst march בישׁימון through the wilderness; סלה׃ Selah:}%
\verse{ארץ The earth רעשׁה shook, אף also שׁמים the heavens נטפו dropped מפני at the presence אלהים of God: זה itself סיני Sinai מפני at the presence אלהים of God, אלהי the God ישׂראל׃ of Israel.}%
\verse{גשׁם rain, נדבות a plentiful תניף didst send אלהים Thou, O God, נחלתך thine inheritance, ונלאה when it was weary. אתה whereby thou כוננתה׃ didst confirm}%
\verse{חיתך Thy congregation ישׁבו hath dwelt בה תכין hast prepared בטובתך of thy goodness לעני for the poor. אלהים׃ therein: thou, O God,}%
\verse{אדני The Lord יתן gave אמר the word: המבשׂרות of those that published צבא the company רב׃ great}%
\verse{מלכי Kings צבאות of armies ידדון did flee apace: ידדון did flee apace: ונות and she that tarried בית at home תחלק divided שׁלל׃ the spoil.}%
\verse{אם Though תשׁכבון ye have lain בין among שׁפתים the pots, כנפי the wings יונה of a dove נחפה covered בכסף with silver, ואברותיה with yellow בירקרק and her feathers חרוץ׃ gold.}%
\verse{בפרשׂ scattered שׁדי When the Almighty מלכים kings בה תשׁלג in it, it was as snow בצלמון׃ in Salmon.}%
\verse{הר The hill אלהים of God הר the hill בשׁן of Bashan; הר hill גבננים a high הר the hill בשׁן׃ of Bashan.}%
\verse{למה Why תרצדון leap הרים hills? גבננים ye, ye high ההר the hill חמד desireth אלהים God לשׁבתו to dwell אף in; yea, יהוה the LORD ישׁכן will dwell לנצח׃ forever.}%
\verse{רכב The chariots אלהים of God רבתים twenty thousand, אלפי thousands שׁנאן of angels: אדני the Lord בם סיני among them, Sinai, בקדשׁ׃ in the holy}%
\verse{עלית Thou hast ascended למרום on high, שׁבית thou hast led captivity captive: שׁבי thou hast led captivity captive: לקחת thou hast received מתנות gifts באדם for men; ואף also, סוררים yea, the rebellious לשׁכן might dwell יה that the LORD אלהים׃ God}%
\verse{ברוך Blessed אדני the Lord, יום daily יום daily יעמס loadeth לנו האל us the God ישׁועתנו of our salvation. סלה׃ Selah.}%
\verse{האל our God לנו אל the God למושׁעות of salvation; וליהוה and unto GOD אדני the Lord למות from death. תוצאות׃ the issues}%
\verse{אך But אלהים God ימחץ shall wound ראשׁ the head איביו of his enemies, קדקד scalp שׂער the hairy מתהלך of such a one as goeth on still באשׁמיו׃ in his trespasses.}%
\verse{אמר said, אדני The Lord מבשׁן אשׁיב I will bring again אשׁיב I will bring again ממצלות from the depths ים׃ of the sea:}%
\verse{למען That תמחץ may be dipped רגלך thy foot בדם in the blood לשׁון the tongue כלביך of thy dogs מאיבים of enemies, מנהו׃ of enemies,}%
\verse{ראו They have seen הליכותיך thy goings, אלהים O God; הליכות the goings אלי of my God, מלכי my King, בקדשׁ׃ in the sanctuary.}%
\verse{קדמו went before, שׁרים The singers אחר after; נגנים the players on instruments בתוך among עלמות the damsels תופפות׃ playing with timbrels.}%
\verse{במקהלות in the congregations, ברכו Bless אלהים ye God יהוה the Lord, ממקור from the fountain ישׂראל׃ of Israel.}%
\verse{שׁם There בנימן Benjamin צעיר little רדם their ruler, שׂרי the princes יהודה of Judah רגמתם their council, שׂרי the princes זבלון of Zebulun, שׂרי the princes נפתלי׃ of Naphtali.}%
\verse{צוה hath commanded אלהיך Thy God עזך thy strength: עוזה strengthen, אלהים O God, זו that which פעלת׃ thou hast wrought}%
\verse{מהיכלך על at ירושׁלם Jerusalem לך יובילו bring מלכים shall kings שׁי׃ presents}%
\verse{גער Rebuke חית the company קנה of spearmen, עדת the multitude אבירים of the bulls, בעגלי with the calves עמים of the people, מתרפס submit himself ברצי with pieces כסף of silver: בזר scatter עמים thou the people קרבות in war. יחפצו׃ delight}%
\verse{יאתיו shall come out חשׁמנים Princes מני of מצרים Egypt; כושׁ Ethiopia תריץ shall soon stretch out ידיו her hands לאלהים׃ unto God.}%
\verse{ממלכות ye kingdoms הארץ of the earth; שׁירו Sing לאלהים unto God, זמרו O sing praises אדני unto the Lord; סלה׃ Selah:}%
\verse{לרכב To him that rideth בשׁמי upon the heavens שׁמי of heavens, קדם of old; הן lo, יתן he doth send out בקולו his voice, קול voice. עז׃ a mighty}%
\verse{תנו Ascribe עז ye strength לאלהים unto God: על over ישׂראל Israel, גאותו his excellency ועזו and his strength בשׁחקים׃ in the clouds.}%
\verse{נורא terrible אלהים O God, ממקדשׁיך out of thy holy places: אל the God ישׂראל of Israel הוא he נתן that giveth עז strength ותעצמות and power לעם unto people. ברוך Blessed אלהים׃ God.}%
\end{biblechapter}%
\begin{biblechapter}% Psalm 69
\verseWithHeading{A Plea for Deliverance from Persecution}{למנצח To the chief Musician על upon שׁושׁנים Shoshannim, לדוד׃ of David. הושׁיעני Save אלהים me, O God; כי for באו are come in מים the waters עד unto נפשׁ׃ soul.}%
\verse{טבעתי I sink ביון mire, מצולה in deep ואין where no מעמד standing: באתי I am come במעמקי into deep מים waters, ושׁבלת where the floods שׁטפתני׃ overflow}%
\verse{יגעתי I am weary בקראי of my crying: נחר is dried: גרוני my throat כלו fail עיני mine eyes מיחל while I wait לאלהי׃ for my God.}%
\verse{רבו are more משׂערות than the hairs ראשׁי of mine head: שׂנאי They that hate חנם me without a cause עצמו are mighty: מצמיתי they that would destroy איבי me, mine enemies שׁקר wrongfully, אשׁר which לא גזלתיז then אשׁיב׃ I restored}%
\verse{אלהים O God, אתה thou ידעת knowest לאולתי my foolishness; ואשׁמותי and my sins ממך from לא are not נכחדו׃ hid}%
\verse{אל Let not יבשׁו be ashamed בי קויך them that wait on אדני thee, O Lord יהוה GOD צבאות of hosts, אל for my sake: let not יכלמו thee be confounded בי מבקשׁיך those that seek אלהי for my sake, O God ישׂראל׃ of Israel.}%
\verse{כי Because עליך for thy sake נשׂאתי I have borne חרפה reproach; כסתה hath covered כלמה shame פני׃ my face.}%
\verse{מוזר a stranger הייתי I am become לאחי unto my brethren, ונכרי and an alien לבני children. אמי׃ unto my mother's}%
\verse{כי For קנאת the zeal ביתך of thine house אכלתני hath eaten me up; וחרפות and the reproaches חורפיך of them that reproached נפלו thee are fallen עלי׃ upon}%
\verse{ואבכה When I wept, בצום with fasting, נפשׁי my soul ותהי that was לחרפות׃ to my reproach.}%
\verse{ואתנה I made לבושׁי also my garment; שׂק sackcloth ואהי and I became להם למשׁל׃ a proverb}%
\verse{ישׂיחו speak בי ישׁבי They that sit שׁער in the gate ונגינות against me; and I the song שׁותי of the drunkards. שׁכר׃ of the drunkards.}%
\verse{ואני But as for me, תפלתי my prayer לך יהוה unto thee, O LORD, עת time: רצון an acceptable אלהים O God, ברב in the multitude חסדך of thy mercy ענני hear באמת me, in the truth ישׁעך׃ of thy salvation.}%
\verse{הצילני Deliver מטיט me out of the mire, ואל and let me not אטבעה sink: אנצלה let me be delivered משׂנאי from them that hate וממעמקי me, and out of the deep מים׃ waters.}%
\verse{אל Let not תשׁטפני overflow שׁבלת the waterflood מים the waterflood ואל me, neither תבלעני swallow me up, מצולה let the deep ואל and let not תאטר shut עלי upon באר the pit פיה׃ her mouth}%
\verse{ענני Hear יהוה me, O LORD; כי for טוב good: חסדך thy lovingkindness כרב me according to the multitude רחמיך of thy tender mercies. פנה turn אלי׃ unto}%
\verse{ואל not תסתר And hide פניך thy face מעבדך from thy servant; כי for צר לי מהרנני׃ hear}%
\verse{קרבה Draw nigh אל unto נפשׁי my soul, גאלה redeem למען me because of איבי mine enemies. פדני׃ it: deliver}%
\verse{אתה Thou ידעת hast known חרפתי my reproach, ובשׁתי and my shame, וכלמתי and my dishonor: נגדך before כל all צוררי׃ mine adversaries}%
\verse{חרפה Reproach שׁברה hath broken לבי my heart; ואנושׁה and I am full of heaviness: ואקוה and I looked לנוד to take pity, ואין but none; ולמנחמים and for comforters, ולא none. מצאתי׃ but I found}%
\verse{ויתנו They gave בברותי for my meat; ראשׁ me also gall ולצמאי and in my thirst ישׁקוני to drink. חמץ׃ they gave me vinegar}%
\verse{יהי become שׁלחנם Let their table לפניהם before לפח a snare ולשׁלומים them: and for welfare, למוקשׁ׃ a trap.}%
\verse{תחשׁכנה be darkened, עיניהם Let their eyes מראות that they see not; ומתניהם and make their loins תמיד continually המעד׃ to shake.}%
\verse{שׁפך Pour out עליהם upon זעמך thine indignation וחרון them, and let thy wrathful אפך anger ישׂיגם׃ take hold}%
\verse{תהי be טירתם Let their habitation נשׁמה desolate; באהליהם in their tents. אל none יהי let ישׁב׃ dwell}%
\verse{כי For אתה thou אשׁר whom הכית hast smitten; רדפו they persecute ואל to מכאוב the grief חלליך of those whom thou hast wounded. יספרו׃ and they talk}%
\verse{תנה Add עון iniquity על unto עונם their iniquity: ואל and let them not יבאו come בצדקתך׃ into thy righteousness.}%
\verse{ימחו Let them be blotted מספר out of the book חיים of the living, ועם with צדיקים the righteous. אל and not יכתבו׃ be written}%
\verse{ואני But I עני poor וכואב and sorrowful: ישׁועתך let thy salvation, אלהים O God, תשׂגבני׃ set me up on high.}%
\verse{אהללה I will praise שׁם the name אלהים of God בשׁיר with a song, ואגדלנו and will magnify בתודה׃ him with thanksgiving.}%
\verse{ותיטב also shall please ליהוה the LORD משׁור better than an ox פר bullock מקרן that hath horns מפריס׃ and hooves.}%
\verse{ראו shall see ענוים The humble ישׂמחו be glad: דרשׁי that seek אלהים God. ויחי shall live לבבכם׃ and your heart}%
\verse{כי For שׁמע heareth אל heareth אביונים the poor, יהוה the LORD ואת אסיריו his prisoners. לא not בזה׃ and despiseth}%
\verse{יהללוהו praise שׁמים Let the heaven וארץ and earth ימים him, the seas, וכל and every thing רמשׂ׃ that moveth}%
\verse{כי For אלהים God יושׁיע will save ציון Zion, ויבנה and will build ערי the cities יהודה of Judah: וישׁבו that they may dwell שׁם there, וירשׁוה׃ and have it in possession.}%
\verse{וזרע The seed עבדיו also of his servants ינחלוה shall inherit ואהבי it: and they that love שׁמו his name ישׁכנו׃ shall dwell}%
\end{biblechapter}%
\begin{biblechapter}% Psalm 70
\verseWithHeading{A Prayer for Deliverance from Enemies}{למנצח To the chief Musician, לדוד of David, להזכיר׃ to bring to remembrance. אלהים O God, להצילני to deliver יהוה me, O LORD. לעזרתי to help חושׁה׃ me; make haste}%
\verse{יבשׁו Let them be ashamed ויחפרו and confounded מבקשׁי that seek after נפשׁי my soul: יסגו let them be turned אחור backward, ויכלמו and put to confusion, חפצי that desire רעתי׃ my hurt.}%
\verse{ישׁובו Let them be turned back על for עקב a reward בשׁתם of their shame האמרים that say, האח Aha, האח׃ aha.}%
\verse{ישׂישׂו thee rejoice וישׂמחו and be glad בך כל Let all מבקשׁיך those that seek ויאמרו say תמיד continually, יגדל be magnified. אלהים Let God אהבי in thee: and let such as love ישׁועתך׃ thy salvation}%
\verse{ואני But I עני poor ואביון and needy: אלהים unto me, O God: חושׁה make haste לי עזרי my help ומפלטי and my deliverer; אתה thou יהוה O LORD, אל make no תאחר׃ tarrying.}%
\end{biblechapter}%
\begin{biblechapter}% Psalm 71
\verseWithHeading{A Prayer to God the Rock of Refuge}{בך יהוה In thee, O LORD, חסיתי do I put my trust: אל let me never אבושׁה be put to confusion. לעולם׃ let me never}%
\verse{בצדקתך me in thy righteousness, תצילני Deliver ותפלטני and cause me to escape: הטה incline אלי unto אזנך thine ear והושׁיעני׃ me, and save}%
\verse{היה Be לי לצור thou my strong מעון habitation, לבוא resort: תמיד whereunto I may continually צוית thou hast given commandment להושׁיעני to save כי me; for סלעי my rock ומצודתי and my fortress. אתה׃ thou}%
\verse{אלהי me, O my God, פלטני Deliver מיד out of the hand רשׁע of the wicked, מכף out of the hand מעול of the unrighteous וחומץ׃ and cruel}%
\verse{כי For אתה thou תקותי my hope, אדני O Lord יהוה מבטחי my trust מנעורי׃ from my youth.}%
\verse{עליך By נסמכתי thee have I been holden up מבטן from the womb: ממעי bowels: אמי me out of my mother's אתה thou גוזי art he that took בך תהלתי my praise תמיד׃ continually}%
\verse{כמופת as a wonder הייתי I am לרבים unto many; ואתה but thou מחסי refuge. עז׃ my strong}%
\verse{ימלא be filled פי Let my mouth תהלתך thy praise כל all היום the day. תפארתך׃ thy honor}%
\verse{אל me not תשׁליכני לעת in the time זקנה of old age; ככלות faileth. כחי when my strength אל תעזבני׃ forsake}%
\verse{כי For אמרו speak אויבי mine enemies לי ושׁמרי against me; and they that lay wait for נפשׁי my soul נועצו take counsel יחדו׃ together,}%
\verse{לאמר Saying, אלהים God עזבו hath forsaken רדפו him: persecute ותפשׂוהו and take כי him; for אין none מציל׃ to deliver}%
\verse{אלהים O God, אל be not תרחק far ממני from אלהי me: O my God, לעזרתי for my help. חישׁה׃}%
\verse{יבשׁו Let them be confounded יכלו consumed שׂטני that are adversaries נפשׁי to my soul; יעטו let them be covered חרפה reproach וכלמה and dishonor מבקשׁי that seek רעתי׃ my hurt.}%
\verse{ואני But I תמיד continually, איחל will hope והוספתי thee more and more. על כלהלתך׃ and will yet praise}%
\verse{פי My mouth יספר shall show forth צדקתך thy righteousness כל all היום the day; תשׁועתך thy salvation כי for לא not ידעתי I know ספרות׃ the numbers}%
\verse{אבוא I will go בגברות in the strength אדני of the Lord יהוה GOD: אזכיר I will make mention צדקתך of thy righteousness, לבדך׃ of thine only.}%
\verse{אלהים O God, למדתני thou hast taught מנעורי me from my youth: ועד and hitherto הנה and hitherto אגיד have I declared נפלאותיך׃ thy wondrous works.}%
\verse{וגם Now also עד when זקנה I am old ושׂיבה and grayheaded, אלהים O God, אל me not; תעזבני forsake עד until אגיד I have showed זרועך thy strength לדור unto generation, לכל to every one יבוא is to come. גבורתך׃ thy power}%
\verse{וצדקתך Thy righteousness אלהים also, O God, עד very מרום high, אשׁר who עשׂית hast done גדלות great things: אלהים O God, מי who כמוך׃ like unto thee!}%
\verse{אשׁר which הראיתנו hast showed צרות troubles, רבות me great ורעות and sore תשׁוב me again, תחיינו shalt quicken ומתהמות from the depths הארץ of the earth. תשׁוב again תעלני׃ and shalt bring me up}%
\verse{תרב Thou shalt increase גדלתי my greatness, ותסב me on every side. תנחמני׃ and comfort}%
\verse{גם will also אני I אודך praise בכלי thee with the psaltery, נבל thee with the psaltery, אמתך thy truth, אלהי O my God: אזמרה unto thee will I sing לך בכנור with the harp, קדושׁ O thou Holy One ישׂראל׃ of Israel.}%
\verse{תרננה shall greatly rejoice שׂפתי My lips כי when אזמרה I sing לך ונפשׁי unto thee; and my soul, אשׁר which פדית׃ thou hast redeemed.}%
\verse{גם also לשׁוני My tongue כל all היום the day תהגה shall talk צדקתך of thy righteousness כי long: for בשׁו they are confounded, כי for חפרו they are brought unto shame, מבקשׁי that seek רעתי׃ my hurt.}%
\end{biblechapter}%
\begin{biblechapter}% Psalm 72
\verseWithHeading{A Prayer for the Prosperity of God’s Anointed King}{לשׁלמה for Solomon. אלהים O God, משׁפטיך thy judgments, למלך the king תן Give וצדקתך and thy righteousness לבן son. מלך׃ unto the king's}%
\verse{ידין He shall judge עמך thy people בצדק with righteousness, וענייך and thy poor במשׁפט׃ with judgment.}%
\verse{ישׂאו shall bring הרים The mountains שׁלום peace לעם to the people, וגבעות and the little hills, בצדקה׃ by righteousness.}%
\verse{ישׁפט He shall judge עניי the poor עם of the people, יושׁיע he shall save לבני the children אביון of the needy, וידכא and shall break in pieces עושׁק׃ the oppressor.}%
\verse{ייראוך They shall fear עם thee as long שׁמשׁ as the sun ולפני endure, ירח and moon דור throughout all generations. דורים׃ throughout all generations.}%
\verse{ירד He shall come down כמטר like rain על upon גז the mown grass: כרביבים as showers זרזיף water ארץ׃ the earth.}%
\verse{יפרח flourish; בימיו In his days צדיק shall the righteous ורב and abundance שׁלום of peace עד so long בלי endureth. ירח׃ as the moon}%
\verse{וירד He shall have dominion מים also from sea עד to ים sea, ומנהר and from the river עד unto אפסי the ends ארץ׃ of the earth.}%
\verse{לפניו before יכרעו shall bow ציים They that dwell in the wilderness ואיביו him; and his enemies עפר the dust. ילחכו׃ shall lick}%
\verse{מלכי The kings תרשׁישׁ of Tarshish ואיים and of the isles מנחה presents: ישׁיבו shall bring מלכי the kings שׁבא of Sheba וסבא and Seba אשׁכר gifts. יקריבו׃ shall offer}%
\verse{וישׁתחוו shall fall down לו כל Yea, all מלכים kings כל before him: all גוים nations יעבדוהו׃ shall serve}%
\verse{כי For יציל he shall deliver אביון the needy משׁוע when he crieth; ועני the poor ואין also, and that hath no עזר׃ helper.}%
\verse{יחס עלל the poor ואביון and needy, ונפשׁות the souls אביונים of the needy. יושׁיע׃ and shall save}%
\verse{מתוך from deceit ומחמס and violence: יגאל He shall redeem נפשׁם their soul וייקר and precious דמם shall their blood בעיניו׃ be in his sight.}%
\verse{ויחי And he shall live, ויתן and to him shall be given לו מזהב of the gold שׁבא of Sheba: ויתפלל prayer בעדו also shall be made for תמיד him continually; כל daily היום daily יברכנהו׃ shall he be praised.}%
\verse{יהי There shall be פסת a handful בר of corn בארץ in the earth בראשׁ upon the top הרים of the mountains; ירעשׁ thereof shall shake כלבנון like Lebanon: פריו the fruit ויציצו shall flourish מעיר and of the city כעשׂב like grass הארץ׃ of the earth.}%
\verse{יהי shall endure שׁמו His name לעולם forever: לפני as long שׁמשׁ as the sun: ינין shall be continued שׁמו his name ויתברכו and shall be blessed בו כל in him: all גוים nations יאשׁרוהו׃ shall call him blessed.}%
\verse{ברוך Blessed יהוה the LORD אלהים God, אלהי the God ישׂראל of Israel, עשׂה doeth נפלאות wondrous things. לבדו׃ who only}%
\verse{וברוך And blessed שׁם name כבודו his glorious לעולם forever: וימלא be filled כבודו his glory; את כל and let the whole הארץ earth אמן Amen, ואמן׃ and Amen.}%
\verse{כלו are ended. תפלות The prayers דוד of David בן the son ישׁי׃ of Jesse}%
\end{biblechapter}%
\begin{biblechapter}% Psalm 73
\verseWithHeading{The Wicked and the Righteous Contrasted}{מזמור A Psalm לאסף of Asaph. אך Truly טוב good לישׂראל to Israel, אלהים God לברי to such as are of a clean לבב׃ heart.}%
\verse{ואני But as for me, כמעט were almost נטוי gone; רגלי my feet כאין had well nigh שׁפכה slipped. אשׁרי׃ my steps}%
\verse{כי For קנאתי I was envious בהוללים at the foolish, שׁלום the prosperity רשׁעים of the wicked. אראה׃ I saw}%
\verse{כי For אין no חרצבות bands למותם in their death: ובריא firm. אולם׃ but their strength}%
\verse{בעמל in trouble אנושׁ men; אינמו They not ועם like אדם men. לא neither ינגעו׃ are they plagued}%
\verse{לכן Therefore ענקתמו compasseth them about as a chain; גאוה pride יעטף covereth שׁית them a garment. חמס׃ violence}%
\verse{יצא stand out מחלב with fatness: עינמו Their eyes עברו they have more משׂכיות could wish. לבב׃ than heart}%
\verse{ימיקו They are corrupt, וידברו and speak ברע wickedly עשׁק oppression: ממרום loftily. ידברו׃ they speak}%
\verse{שׁתו They set בשׁמים against the heavens, פיהם their mouth ולשׁונם and their tongue תהלך walketh בארץ׃ through the earth.}%
\verse{לכן Therefore ישׁיב return עמו his people הלם hither: ומי and waters מלא of a full ימצו׃ are wrung out}%
\verse{ואמרו And they say, איכה How ידע know? אל doth God וישׁ and is there דעה knowledge בעליון׃ in the most High?}%
\verse{הנה Behold, אלה these רשׁעים the ungodly, ושׁלוי who prosper עולם in the world; השׂגו they increase חיל׃ riches.}%
\verse{אך Verily ריק vain, זכיתי I have cleansed לבבי my heart וארחץ and washed בנקיון in innocency. כפי׃ my hands}%
\verse{ואהי long have I been נגוע plagued, כל For all היום the day ותוכחתי and chastened לבקרים׃ every morning.}%
\verse{אם If אמרתי I say, אספרה I will speak כמו thus; הנה behold, דור the generation בניך of thy children. בגדתי׃ I should offend}%
\verse{ואחשׁבה When I thought לדעת to know זאת this, עמל too painful היא it בעיני׃ for}%
\verse{עד Until אבוא I went אל into מקדשׁי the sanctuary אל of God; אבינה understood לאחריתם׃ I their end.}%
\verse{אך Surely בחלקות them in slippery places: תשׁית thou didst set למו הפלתם thou castedst them down למשׁואות׃ into destruction.}%
\verse{איך How היו are לשׁמה they into desolation, כרגע as in a moment! ספו they are utterly consumed תמו they are utterly consumed מן with בלהות׃ terrors.}%
\verse{כחלום As a dream מהקיץ when awaketh; אדני O Lord, בעיר when thou awakest, צלמם their image. תבזה׃ thou shalt despise}%
\verse{כי Thus יתחמץ was grieved, לבבי my heart וכליותי in my reins. אשׁתונן׃ and I was pricked}%
\verse{ואני I, בער So foolish ולא and ignorant: אדע and ignorant: בהמות a beast הייתי I was עמך׃ before}%
\verse{ואני Nevertheless I תמיד continually עמך with אחזת thee: thou hast holden ביד hand. ימיני׃ by my right}%
\verse{בעצתך me with thy counsel, תנחני Thou shalt guide ואחר and afterward כבוד me glory. תקחני׃ receive}%
\verse{מי Whom לי בשׁמים have I in heaven ועמך beside לא and none חפצתי I desire בארץ׃ upon earth}%
\verse{כלה faileth: שׁארי My flesh ולבבי and my heart צור the strength לבבי of my heart, וחלקי and my portion אלהים God לעולם׃ forever.}%
\verse{כי For, הנה lo, רחקיך they that are far יאבדו from thee shall perish: הצמתה thou hast destroyed כל all זונה them that go a whoring ממך׃ from}%
\verse{ואני But it קרבת for me to draw near אלהים to God: לי טוב good שׁתי I have put באדני in the Lord יהוה GOD, מחסי my trust לספר that I may declare כל all מלאכותיך׃ thy works.}%
\end{biblechapter}%
\begin{biblechapter}% Psalm 74
\verseWithHeading{A Lament in Time of National Defeat}{משׂכיל Maschil לאסף of Asaph. למה why אלהים O God, זנחת hast thou cast off לנצח forever? יעשׁן smoke אפך doth thine anger בצאן against the sheep מרעיתך׃ of thy pasture?}%
\verse{זכר Remember עדתך thy congregation, קנית thou hast purchased קדם of old; גאלת thou hast redeemed; שׁבט the rod נחלתך of thine inheritance, הר mount ציון Zion, זה this שׁכנת׃ wherein thou hast dwelt.}%
\verse{הרימה Lift up פעמיך thy feet למשׁאות desolations; נצח unto the perpetual כל all הרע hath done wickedly אויב the enemy בקדשׁ׃ in the sanctuary.}%
\verse{שׁאגו roar צרריך Thine enemies בקרב in the midst מועדך of thy congregations; שׂמו they set up אותתם their ensigns אתות׃ signs.}%
\verse{יודע was famous כמביא according as he had lifted up למעלה according as he had lifted up בסבך upon the thick עץ trees. קרדמות׃ axes}%
\verse{ועת פתוחיה the carved work יחד thereof at once בכשׁיל with axes וכילפת and hammers. יהלמון׃ they break down}%
\verse{שׁלחו They have cast באשׁ fire מקדשׁך into thy sanctuary, לארץ to the ground. חללו they have defiled משׁכן the dwelling place שׁמך׃ of thy name}%
\verse{אמרו They said בלבם in their hearts, נינם Let us destroy יחד them together: שׂרפו they have burned up כל all מועדי the synagogues אל of God בארץ׃ in the land.}%
\verse{אותתינו our signs: לא not ראינו We see אין no עוד more נביא any prophet: ולא neither אתנו among ידע us any that knoweth עד how long. מה׃ how long.}%
\verse{עד how long מתי how long אלהים O God, יחרף reproach? צר shall the adversary ינאץ blaspheme אויב shall the enemy שׁמך thy name לנצח׃ forever?}%
\verse{למה Why תשׁיב withdrawest ידך thou thy hand, וימינך even thy right hand? מקרב out of חוקך thy bosom. כלה׃ pluck}%
\verse{ואלהים For God מלכי my King מקדם of old, פעל working ישׁועות salvation בקרב in the midst הארץ׃ of the earth.}%
\verse{אתה Thou פוררת didst divide בעזך by thy strength: ים the sea שׁברת thou didst break ראשׁי the heads תנינים of the dragons על in המים׃ the waters.}%
\verse{אתה Thou רצצת didst break ראשׁי the heads לויתן of leviathan תתננו in pieces, gavest מאכל him meat לעם to the people לציים׃ inhabiting the wilderness.}%
\verse{אתה Thou בקעת didst cleave מעין the fountain ונחל and the flood: אתה thou הובשׁת driedst up נהרות rivers. איתן׃ mighty}%
\verse{לך יום The day אף also לך לילה thine, the night אתה thine: thou הכינות hast prepared מאור the light ושׁמשׁ׃ and the sun.}%
\verse{אתה Thou הצבת hast set כל all גבולות the borders ארץ of the earth: קיץ summer וחרף and winter. אתה thou יצרתם׃ hast made}%
\verse{זכר Remember זאת this, אויב the enemy חרף hath reproached, יהוה O LORD, ועם people נבל and the foolish נאצו have blasphemed שׁמך׃ thy name.}%
\verse{אל not תתן O deliver לחית unto the multitude נפשׁ the soul תורך of thy turtledove חית the congregation ענייך of thy poor אל not תשׁכח forget לנצח׃ forever.}%
\verse{הבט Have respect לברית unto the covenant: כי for מלאו are full מחשׁכי the dark places ארץ of the earth נאות of the habitations חמס׃ of cruelty.}%
\verse{אל O let not ישׁב return דך the oppressed נכלם ashamed: עני let the poor ואביון and needy יהללו praise שׁמך׃ thy name.}%
\verse{קומה Arise, אלהים O God, ריבה plead thine own cause: ריבך plead thine own cause: זכר remember חרפתך reproacheth מני how נבל the foolish man כל thee daily. היום׃ thee daily.}%
\verse{אל not תשׁכח Forget קול the voice צרריך of thine enemies: שׁאון the tumult קמיך of those that rise up against עלה thee increaseth תמיד׃ continually.}%
\end{biblechapter}%
\begin{biblechapter}% Psalm 75
\verseWithHeading{Thanksgiving for God’s Future Help}{למנצח To the chief Musician, אל תשׁחת Al-taschith, מזמור A Psalm לאסף of Asaph. שׁיר׃ Song הודינו do we give thanks, לך אלהים Unto thee, O God, הודינו do we give thanks: וקרוב is near שׁמך for thy name ספרו declare. נפלאותיך׃ thy wondrous works}%
\verse{כי When אקח I shall receive מועד the congregation אני I מישׁרים uprightly. אשׁפט׃ will judge}%
\verse{נמגים thereof are dissolved: ארץ The earth וכל and all ישׁביה the inhabitants אנכי I תכנתי bear up עמודיה the pillars סלה׃ of it. Selah.}%
\verse{אמרתי I said להוללים unto the fools, אל תהלולרשׁעים and to the wicked, אל תרימורן׃ the horn:}%
\verse{אל תרימומרום on high: קרנכם your horn תדברו speak בצואר neck. עתק׃ a stiff}%
\verse{כי For לא neither ממוצא from the east, וממערב nor from the west, ולא nor ממדבר from the south. הרים׃ promotion}%
\verse{כי But אלהים God שׁפט the judge: זה one, ישׁפיל he putteth down וזה up another. ירים׃ and setteth}%
\verse{כי For כוס a cup, ביד in the hand יהוה of the LORD ויין and the wine חמר is red; מלא it is full מסך of mixture; ויגר and he poureth out מזה of the same: אך but שׁמריה the dregs ימצו shall wring out, ישׁתו drink כל thereof, all רשׁעי the wicked ארץ׃ of the earth}%
\verse{ואני But I אגיד will declare לעלם forever; אזמרה I will sing praises לאלהי to the God יעקב׃ of Jacob.}%
\verse{וכל All קרני the horns רשׁעים of the wicked אגדע also will I cut off; תרוממנה shall be exalted. קרנות the horns צדיק׃ of the righteous}%
\end{biblechapter}%
\begin{biblechapter}% Psalm 76
\verseWithHeading{Praise to God for His Rescue of Israel}{למנצח To the chief Musician בנגינת on Neginoth, מזמור A Psalm לאסף of Asaph. שׁיר׃ Song נודע known: ביהודה In Judah אלהים God בישׂראל in Israel. גדול great שׁמו׃ his name}%
\verse{ויהי also is בשׁלם In Salem סכו his tabernacle, ומעונתו and his dwelling place בציון׃ in Zion.}%
\verse{שׁמה There שׁבר broke רשׁפי he the arrows קשׁת of the bow, מגן the shield, וחרב and the sword, ומלחמה and the battle. סלה׃ Selah.}%
\verse{נאור more glorious אתה Thou אדיר excellent מהררי than the mountains טרף׃ of prey.}%
\verse{אשׁתוללו are spoiled, אבירי לבמו they have slept שׁנתם their sleep: ולא and none מצאו have found כל and none אנשׁי חיל of might ידיהם׃ their hands.}%
\verse{מגערתך אלהי O God יעקב of Jacob, נרדם are cast into a dead sleep. ורכב both the chariot וסוס׃ and horse}%
\verse{אתה Thou, נורא to be feared: אתה thou, ומי and who יעמד may stand לפניך in thy sight מאז when once אפך׃ thou art angry?}%
\verse{משׁמים heaven; השׁמעת to be heard דין Thou didst cause judgment ארץ the earth יראה feared, ושׁקטה׃ and was still,}%
\verse{בקום arose למשׁפט to judgment, אלהים When God להושׁיע to save כל all ענוי the meek ארץ of the earth. סלה׃ Selah.}%
\verse{כי Surely חמת the wrath אדם of man תודך shall praise שׁארית thee: the remainder חמת of wrath תחגר׃ shalt thou restrain.}%
\verse{נדרו Vow, ושׁלמו and pay ליהוה unto the LORD אלהיכם your God: כל let all סביביו that be round about יובילו him bring שׁי presents למורא׃ unto him that ought to be feared.}%
\verse{יבצר He shall cut off רוח the spirit נגידים of princes: נורא terrible למלכי to the kings ארץ׃ of the earth.}%
\end{biblechapter}%
\begin{biblechapter}% Psalm 77
\verseWithHeading{Remembering God’s Help for Israel}{למנצח To the chief Musician, על to ידיתון Jeduthun, לאסף of Asaph. מזמור׃ A Psalm קולי with my voice, אל unto אלהים God ואצעקה I cried קולי with my voice; אל unto אלהים God והאזין and he gave ear אלי׃ unto}%
\verse{ביום In the day צרתי of my trouble אדני the Lord: דרשׁתי I sought ידי my sore לילה in the night, נגרה ran ולא not: תפוג and ceased מאנה refused הנחם to be comforted. נפשׁי׃ my soul}%
\verse{אזכרה I remembered אלהים God, ואהמיה and was troubled: אשׂיחה I complained, ותתעטף was overwhelmed. רוחי and my spirit סלה׃ Selah.}%
\verse{אחזת Thou holdest שׁמרות waking: עיני mine eyes נפעמתי I am so troubled ולא that I cannot אדבר׃ speak.}%
\verse{חשׁבתי I have considered ימים the days מקדם of old, שׁנות the years עולמים׃ of ancient times.}%
\verse{אזכרה I call to remembrance נגינתי my song בלילה in the night: עם with לבבי mine own heart: אשׂיחה I commune ויחפשׂ made diligent search. רוחי׃ and my spirit}%
\verse{הלעולמים forever? יזנח cast off אדני Will the Lord ולא no יסיף more? לרצות and will he be favorable עוד׃ more?}%
\verse{האפס clean gone לנצח forever? חסדו Is his mercy גמר fail אמר doth promise לדר forevermore? ודר׃ forevermore?}%
\verse{השׁכח forgotten חנות to be gracious? אל Hath God אם קפץ shut up באף hath he in anger רחמיו his tender mercies? סלה׃ Selah.}%
\verse{ואמר And I said, חלותי my infirmity: היא This שׁנות the years ימין of the right hand עליון׃ of the most High.}%
\verse{אזכיר I will remember מעללי the works יה of the LORD: כי surely אזכרה I will remember מקדם of old. פלאך׃ thy wonders}%
\verse{והגיתי I will meditate בכל also of all פעלך thy work, ובעלילותיך of thy doings. אשׂיחה׃ and talk}%
\verse{אלהים O God, בקדשׁ in the sanctuary: דרכך Thy way, מי who אל a God גדול great כאלהים׃ as God?}%
\verse{אתה Thou האל the God עשׂה that doest פלא wonders: הודעת thou hast declared בעמים among the people. עזך׃ thy strength}%
\verse{גאלת redeemed בזרוע Thou hast with arm עמך thy people, בני the sons יעקב of Jacob ויוסף and Joseph. סלה׃ Selah.}%
\verse{ראוך saw מים The waters אלהים thee, O God, ראוך saw מים the waters יחילו thee; they were afraid: אף also ירגזו were troubled. תהמות׃ the depths}%
\verse{זרמו poured out מים water: עבות The clouds קול a sound: נתנו sent out שׁחקים the skies אף also חצציך יתהלכו׃ went abroad.}%
\verse{קול The voice רעמך of thy thunder בגלגל in the heaven: האירו lightened ברקים the lightnings תבל the world: רגזה trembled ותרעשׁ and shook. הארץ׃ the earth}%
\verse{בים in the sea, דרכך Thy way ושׁביליך and thy path במים waters, רבים in the great ועקבותיך and thy footsteps לא are not נדעו׃ known.}%
\verse{נחית Thou leddest כצאן like a flock עמך thy people ביד by the hand משׁה of Moses ואהרן׃ and Aaron.}%
\end{biblechapter}%
\begin{biblechapter}% Psalm 78
\verseWithHeading{God’s Faithfulness in Israel’s History}{משׂכיל Maschil לאסף of Asaph. האזינה Give ear, עמי O my people, תורתי my law: הטו incline אזנכם your ears לאמרי to the words פי׃ of my mouth.}%
\verse{אפתחה I will open במשׁל in a parable: פי my mouth אביעה I will utter חידות dark sayings מני of קדם׃ old:}%
\verse{אשׁר Which שׁמענו we have heard ונדעם and known, ואבותינו and our fathers ספרו׃ have told}%
\verse{לא We will not נכחד hide מבניהם from their children, לדור to the generation אחרון to come מספרים showing תהלות the praises יהוה of the LORD, ועזוזו and his strength, ונפלאותיו and his wonderful works אשׁר that עשׂה׃ he hath done.}%
\verse{ויקם For he established עדות a testimony ביעקב in Jacob, ותורה a law שׂם and appointed בישׂראל in Israel, אשׁר which צוה he commanded את אבותינו our fathers, להודיעם that they should make them known לבניהם׃ to their children:}%
\verse{למען That ידעו might know דור the generation אחרון to come בנים the children יולדו should be born; יקמו should arise ויספרו and declare לבניהם׃ to their children:}%
\verse{וישׂימו That they might set באלהים in God, כסלם their hope ולא and not ישׁכחו forget מעללי the works אל of God, ומצותיו his commandments: ינצרו׃ but keep}%
\verse{ולא And might not יהיו be כאבותם as their fathers, דור generation; סורר a stubborn ומרה and rebellious דור a generation לא was not הכין set not their heart aright, לבו ולאאמנה steadfast את with אל God. רוחו׃ and whose spirit}%
\verse{בני The children אפרים of Ephraim, נושׁקי armed, רומי carrying קשׁת bows, הפכו turned back ביום in the day קרב׃ of battle.}%
\verse{לא not שׁמרו They kept ברית the covenant אלהים of God, ובתורתו in his law; מאנו and refused ללכת׃ to walk}%
\verse{וישׁכחו And forgot עלילותיו his works, ונפלאותיו and his wonders אשׁר that הראם׃ he had showed}%
\verse{נגד he in the sight אבותם of their fathers, עשׂה did פלא Marvelous things בארץ in the land מצרים of Egypt, שׂדה the field צען׃ of Zoan.}%
\verse{בקע He divided ים the sea, ויעבירם and caused them to pass through; ויצב to stand מים and he made the waters כמו as נד׃ a heap.}%
\verse{וינחם also he led בענן them with a cloud, יומם In the daytime וכל and all הלילה the night באור with a light אשׁ׃ of fire.}%
\verse{יבקע He cleaved צרים the rocks במדבר in the wilderness, וישׁק and gave drink כתהמות depths. רבה׃ as the great}%
\verse{ויוצא נוזליםסלע of the rock, ויורד to run down כנהרות like rivers. מים׃ and caused waters}%
\verse{ויוסיפו more עוד yet לחטא And they sinned לו למרות against him by provoking עליון the most High בציה׃ in the wilderness.}%
\verse{וינסו And they tempted אל God בלבבם in their heart לשׁאל by asking אכל meat לנפשׁם׃ for their lust.}%
\verse{וידברו Yea, they spoke באלהים against God; אמרו they said, היוכל Can אל God לערך furnish שׁלחן a table במדבר׃ in the wilderness?}%
\verse{הן Behold, הכה he smote צור the rock, ויזובו gushed out, מים that the waters ונחלים and the streams ישׁטפו overflowed; הגם also? לחם bread יוכל can תת he give אם יכין can he provide שׁאר flesh לעמו׃ for his people?}%
\verse{לכן Therefore שׁמע heard יהוה the LORD ויתעבר and was wroth: ואשׁ so a fire נשׂקה was kindled ביעקב against Jacob, וגם also אף and anger עלה came up בישׂראל׃ against Israel;}%
\verse{כי Because לא not האמינו they believed באלהים in God, ולא not בטחו and trusted בישׁועתו׃ in his salvation:}%
\verse{ויצו Though he had commanded שׁחקים the clouds ממעל from above, ודלתי the doors שׁמים of heaven, פתח׃ and opened}%
\verse{וימטר And had rained down עליהם upon מן manna לאכל them to eat, ודגן them of the corn שׁמים of heaven. נתן׃ and had given}%
\verse{לחם food: אבירים angels' אכל did eat אישׁ Man צידה them meat שׁלח he sent להם לשׂבע׃ to the full.}%
\verse{יסע to blow קדים He caused an east wind בשׁמים in the heaven: וינהג he brought בעזו and by his power תימן׃ in the south wind.}%
\verse{וימטר He rained עליהם also upon כעפר them as dust, שׁאר flesh וכחול like as the sand ימים of the sea: עוף fowls כנף׃ and feathered}%
\verse{ויפל And he let fall בקרב in the midst מחנהו of their camp, סביב round about למשׁכנתיו׃ their habitations.}%
\verse{ויאכלו So they did eat, וישׂבעו filled: מאד and were well ותאותם them their own desire; יבא for he gave להם׃}%
\verse{לא They were not זרו estranged מתאותם from their lust. עוד yet אכלם But while their meat בפיהם׃ in their mouths,}%
\verse{ואף The wrath אלהים of God עלה came בהם ויהרג upon them, and slew במשׁמניהם the fattest ובחורי the chosen ישׂראל of Israel. הכריע׃ of them, and smote down}%
\verse{בכל For all זאת this חטאו they sinned עוד still, ולא not האמינו and believed בנפלאותיו׃ for his wondrous works.}%
\verse{ויכל did he consume בהבל in vanity, ימיהם Therefore their days ושׁנותם and their years בבהלה׃ in trouble.}%
\verse{אם When הרגם he slew ודרשׁוהו them, then they sought ושׁבו him: and they returned ושׁחרו and inquired early אל׃ after God.}%
\verse{ויזכרו And they remembered כי that אלהים God צורם their rock, ואל God עליון and the high גאלם׃ their redeemer.}%
\verse{ויפתוהו Nevertheless they did flatter בפיהם him with their mouth, ובלשׁונם unto him with their tongues. יכזבו׃ and they lied}%
\verse{ולבם For their heart לא was not right נכון was not right עמו with ולא him, neither נאמנו were they steadfast בבריתו׃ in his covenant.}%
\verse{והוא But he, רחום full of compassion, יכפר forgave עון iniquity, ולא not: ישׁחית and destroyed והרבה yea, many להשׁיב a time turned he his anger away, אפו a time turned he his anger away, ולא and did not יעיר stir up כל all חמתו׃ his wrath.}%
\verse{ויזכר For he remembered כי that בשׂר flesh; המה they רוח a wind הולך that passeth away, ולא and cometh not again. ישׁוב׃ and cometh not again.}%
\verse{כמה How oft ימרוהו did they provoke במדבר him in the wilderness, יעציבוהו grieve בישׁימון׃ him in the desert!}%
\verse{וישׁובו Yea, they turned back וינסו and tempted אל God, וקדושׁ the Holy One ישׂראל of Israel. התוו׃ and limited}%
\verse{לא not זכרו They remembered את ידו his hand, יום the day אשׁר when פדם he delivered מני them from צר׃ the enemy.}%
\verse{אשׁר How שׂם he had wrought במצרים in Egypt, אתותיו his signs ומופתיו and his wonders בשׂדה in the field צען׃ of Zoan:}%
\verse{ויהפך And had turned לדם into blood; יאריהם their rivers ונזליהם and their floods, בל that they could not ישׁתיון׃ drink.}%
\verse{ישׁלח He sent בהם ערב divers sorts of flies ויאכלם among them, which devoured וצפרדע them; and frogs, ותשׁחיתם׃ which destroyed}%
\verse{ויתן He gave לחסיל unto the caterpillar, יבולם also their increase ויגיעם and their labor לארבה׃ unto the locust.}%
\verse{יהרג He destroyed בברד with hail, גפנם their vines ושׁקמותם and their sycamore trees בחנמל׃ with frost.}%
\verse{ויסגר He gave up לברד also to the hail, בעירם their cattle ומקניהם and their flocks לרשׁפים׃ to hot thunderbolts.}%
\verse{ישׁלח He cast בם חרון upon them the fierceness אפו of his anger, עברה wrath, וזעם and indignation, וצרה and trouble, משׁלחת by sending מלאכי angels רעים׃ evil}%
\verse{יפלס He made נתיב a way לאפו to his anger; לא not חשׂך he spared ממות from death, נפשׁם their soul וחיתם their life לדבר over to the pestilence; הסגיר׃ but gave}%
\verse{ויך And smote כל all בכור the firstborn במצרים in Egypt; ראשׁית the chief אונים of strength באהלי in the tabernacles חם׃ of Ham:}%
\verse{ויסע to go forth כצאן like sheep, עמו But made his own people וינהגם and guided כעדר like a flock. במדבר׃ them in the wilderness}%
\verse{וינחם And he led לבטח them on safely, ולא not: פחדו so that they feared ואת אויביהם their enemies. כסה overwhelmed הים׃ but the sea}%
\verse{ויביאם And he brought אל them to גבול the border קדשׁו of his sanctuary, הר mountain, זה this קנתה had purchased. ימינו׃ his right hand}%
\verse{ויגרשׁ He cast out מפניהם also before גוים the heathen ויפילם them, and divided בחבל by line, נחלה them an inheritance וישׁכן to dwell באהליהם in their tents. שׁבטי and made the tribes ישׂראל׃ of Israel}%
\verse{וינסו Yet they tempted וימרו and provoked את אלהים God, עליון the most high ועדותיו his testimonies: לא not שׁמרו׃ and kept}%
\verse{ויסגו But turned back, ויבגדו and dealt unfaithfully כאבותם like their fathers: נהפכו they were turned aside כקשׁת bow. רמיה׃ like a deceitful}%
\verse{ויכעיסוהו For they provoked him to anger בבמותם with their high places, ובפסיליהם with their graven images. יקניאוהו׃ and moved him to jealousy}%
\verse{שׁמע heard אלהים When God ויתעבר he was wroth, וימאס abhorred מאד and greatly בישׂראל׃ Israel:}%
\verse{ויטשׁ So that he forsook משׁכן the tabernacle שׁלו of Shiloh, אהל the tent שׁכן he placed באדם׃ among men;}%
\verse{ויתן And delivered לשׁבי into captivity, עזו his strength ותפארתו and his glory ביד hand. צר׃ into the enemy's}%
\verse{ויסגר He gave לחרב also unto the sword; עמו his people ובנחלתו with his inheritance. התעבר׃ and was wroth}%
\verse{בחוריו their young men; אכלה consumed אשׁ The fire ובתולתיו and their maidens לא were not הוללו׃ given to marriage.}%
\verse{כהניו Their priests בחרב by the sword; נפלו fell ואלמנתיו and their widows לא made no תבכינה׃ lamentation.}%
\verse{ויקץ awaked כישׁן as one out of sleep, אדני Then the Lord כגבור like a mighty man מתרונן that shouteth מיין׃ by reason of wine.}%
\verse{ויך And he smote צריו his enemies אחור in the hinder parts: חרפת reproach. עולם them to a perpetual נתן׃ he put}%
\verse{וימאס Moreover he refused באהל the tabernacle יוסף of Joseph, ובשׁבט the tribe אפרים of Ephraim: לא not בחר׃ and chose}%
\verse{ויבחר But chose את שׁבט the tribe יהודה of Judah, את הר the mount ציון Zion אשׁר which אהב׃ he loved.}%
\verse{ויבן And he built כמו like רמים high מקדשׁו his sanctuary כארץ like the earth יסדה which he hath established לעולם׃ forever.}%
\verse{ויבחר He chose בדוד David עבדו also his servant, ויקחהו and took ממכלאת צאן׃}%
\verse{מאחר עלות the ewes great with young הביאו he brought לרעות him to feed ביעקב Jacob עמו his people, ובישׂראל and Israel נחלתו׃ his inheritance.}%
\verse{וירעם So he fed כתם them according to the integrity לבבו of his heart; ובתבונות them by the skilfulness כפיו of his hands. ינחם׃ and guided}%
\end{biblechapter}%
\begin{biblechapter}% Psalm 79
\verseWithHeading{A Lament for Jerusalem after Its Destruction}{מזמור A Psalm לאסף of Asaph. אלהים O God, באו are come גוים the heathen בנחלתך into thine inheritance; טמאו have they defiled; את היכל temple קדשׁך thy holy שׂמו they have laid את ירושׁלם Jerusalem לעיים׃ on heaps.}%
\verse{נתנו have they given את נבלת The dead bodies עבדיך of thy servants מאכל meat לעוף unto the fowls השׁמים of the heaven, בשׂר the flesh חסידיך of thy saints לחיתו unto the beasts ארץ׃ of the earth.}%
\verse{שׁפכו have they shed דמם Their blood כמים like water סביבות round about ירושׁלם Jerusalem; ואין and none קובר׃ to bury}%
\verse{היינו We are become חרפה a reproach לשׁכנינו to our neighbors, לעג a scorn וקלס and derision לסביבותינו׃ to them that are round about}%
\verse{עד מההוה LORD? תאנף wilt thou be angry לנצח forever? תבער burn כמו like אשׁ fire? קנאתך׃ shall thy jealousy}%
\verse{שׁפך Pour out חמתך thy wrath אל upon הגוים the heathen אשׁר that לא have not ידעוך known ועל thee, and upon ממלכות the kingdoms אשׁר that בשׁמך upon thy name. לא קראו׃ called}%
\verse{כי For אכל they have devoured את יעקב Jacob, ואת נוהו his dwelling place. השׁמו׃ and laid waste}%
\verse{אל not תזכר O remember לנו עונת iniquities: ראשׁנים against us former מהר יקדמונו prevent רחמיך let thy tender mercies כי us: for דלונו we are brought very low. מאד׃ we are brought very low.}%
\verse{עזרנו Help אלהי us, O God ישׁענו of our salvation, על for דבר for כבוד the glory שׁמך of thy name: והצילנו and deliver וכפר us, and purge away על us, and purge away חטאתינו our sins, למען שׁמך׃}%
\verse{למה Wherefore יאמרו say, הגוים should the heathen איה Where אלהיהם their God? יודע let him be known בגיים among the heathen לעינינו in our sight נקמת the revenging דם of the blood עבדיך of thy servants השׁפוך׃ shed.}%
\verse{תבוא come לפניך before אנקת Let the sighing אסיר כגדל thee; according to the greatness זרועך of thy power הותר preserve בני thou those that are appointed תמותה׃ to die;}%
\verse{והשׁב And render לשׁכנינו unto our neighbors שׁבעתים sevenfold אל into חיקם their bosom חרפתם their reproach, אשׁר wherewith חרפוך they have reproached אדני׃ thee, O Lord.}%
\verse{ואנחנו So we עמך thy people וצאן and sheep מרעיתך of thy pasture נודה will give thee thanks לך לעולם forever: לדר to all generations. ודר to all generations. נספר we will show forth תהלתך׃ thy praise}%
\end{biblechapter}%
\begin{biblechapter}% Psalm 80
\verseWithHeading{A Prayer to Restore Israel}{למנצח To the chief Musician אל upon שׁשׁנים עדות Shoshannim-eduth, לאסף of Asaph. מזמור׃ A Psalm רעה O Shepherd ישׂראל of Israel, האזינה Give ear, נהג thou that leadest כצאן like a flock; יוסף Joseph ישׁב thou that dwellest הכרובים the cherubims, הופיעה׃ shine forth.}%
\verse{לפני Before אפרים Ephraim ובנימן and Benjamin ומנשׁה and Manasseh עוררה stir up את גבורתך thy strength, ולכה and come לישׁעתה׃ save}%
\verse{אלהים O God, השׁיבנו Turn us again, והאר to shine; פניך and cause thy face ונושׁעה׃ and we shall be saved.}%
\verse{יהוה O LORD אלהים God צבאות of hosts, עד how long מתי how long עשׁנת wilt thou be angry בתפלת against the prayer עמך׃ of thy people?}%
\verse{האכלתם Thou feedest לחם them with the bread דמעה of tears; ותשׁקמו to drink בדמעות and givest them tears שׁלישׁ׃ in great measure.}%
\verse{תשׂימנו Thou makest מדון us a strife לשׁכנינו unto our neighbors: ואיבינו and our enemies ילעגו׃ laugh}%
\verse{אלהים O God צבאות of hosts, השׁיבנו Turn us again, והאר to shine; פניך and cause thy face ונושׁעה׃ and we shall be saved.}%
\verse{גפן a vine ממצרים תסיע Thou hast brought תגרשׁ thou hast cast out גוים the heathen, ותטעה׃ and planted}%
\verse{פנית Thou preparedst לפניה before ותשׁרשׁ it, and didst cause it to take deep root, שׁרשׁיה it, and didst cause it to take deep root, ותמלא and it filled ארץ׃ the land.}%
\verse{כסו were covered הרים The hills צלה with the shadow וענפיה of it, and the boughs ארזי cedars. אל׃ thereof the goodly}%
\verse{תשׁלח She sent out קצירה her boughs עד unto ים the sea, ואל unto נהר the river. יונקותיה׃ and her branches}%
\verse{למה Why פרצת hast thou broken down גדריה her hedges, וארוה do pluck כל so that all עברי they which pass by דרך׃ the way}%
\verse{יכרסמנה doth waste חזיר The boar מיער out of the wood וזיז it, and the wild beast שׂדי of the field ירענה׃ doth devour}%
\verse{אלהים thee, O God צבאות of hosts: שׁוב Return, נא we beseech הבט look down משׁמים from heaven, וראה and behold, ופקד and visit גפן vine; זאת׃ this}%
\verse{וכנה And the vineyard אשׁר which נטעה hath planted, ימינך thy right hand ועל בן and the branch אמצתה׃ thou madest strong}%
\verse{שׂרפה burned באשׁ with fire, כסוחה cut down: מגערת at the rebuke פניך of thy countenance. יאבדו׃ they perish}%
\verse{תהי be ידך Let thy hand על upon אישׁ the man ימינך of thy right hand, על upon בן the son אדם of man אמצת׃ thou madest strong}%
\verse{ולא So will not נסוג we go back ממך from תחינו thee: quicken ובשׁמך upon thy name. נקרא׃ us, and we will call}%
\verse{יהוה O LORD אלהים God צבאות of hosts, השׁיבנו Turn us again, האר to shine; פניך cause thy face ונושׁעה׃ and we shall be saved.}%
\end{biblechapter}%
\begin{biblechapter}% Psalm 81
\verseWithHeading{An Appeal from God to Israel}{למנצח To the chief Musician על upon הגתית Gittith, לאסף׃ of Asaph. הרנינו Sing aloud לאלהים unto God עוזנו our strength: הריעו make a joyful noise לאלהי unto the God יעקב׃ of Jacob.}%
\verse{שׂאו Take זמרה a psalm, ותנו and bring תף hither the timbrel, כנור harp נעים the pleasant עם with נבל׃ the psaltery.}%
\verse{תקעו Blow up בחדשׁ in the new moon, שׁופר the trumpet בכסה in the time appointed, ליום day. חגנו׃ on our solemn feast}%
\verse{כי For חק a statute לישׂראל for Israel, הוא this משׁפט a law לאלהי of the God יעקב׃ of Jacob.}%
\verse{עדות a testimony, ביהוסף in Joseph שׂמו This he ordained בצאתו when he went out על through ארץ the land מצרים of Egypt: שׂפת a language לא not. ידעתי I understood אשׁמע׃ I heard}%
\verse{הסירותי I removed מסבל from the burden: שׁכמו his shoulder כפיו his hands מדוד from the pots. תעברנה׃ were delivered}%
\verse{בצרה in trouble, קראת Thou calledst ואחלצך and I delivered אענך thee; I answered בסתר thee in the secret place רעם of thunder: אבחנך I proved על thee at מי the waters מריבה of Meribah. סלה׃ Selah.}%
\verse{שׁמע Hear, עמי O my people, ואעידה and I will testify בך ישׂראל unto thee: O Israel, אם if תשׁמע׃ thou wilt hearken}%
\verse{לא There shall no יהיה be בך אל god זר strange ולא in thee; neither תשׁתחוה shalt thou worship לאל god. נכר׃ any strange}%
\verse{אנכי I יהוה the LORD אלהיך thy God, המעלך which brought מארץ thee out of the land מצרים of Egypt: הרחב wide, פיך open thy mouth ואמלאהו׃ and I will fill}%
\verse{ולא would not שׁמע hearken עמי But my people לקולי to my voice; וישׂראל and Israel לא none אבה׃ would}%
\verse{ואשׁלחהו So I gave them up בשׁרירות lust: לבם unto their own hearts' ילכו they walked במועצותיהם׃ in their own counsels.}%
\verse{לו Oh that עמי my people שׁמע had hearkened לי ישׂראל unto me, Israel בדרכי in my ways! יהלכו׃ had walked}%
\verse{כמעט I should soon אויביהם their enemies, אכניע have subdued ועל against צריהם their adversaries. אשׁיב and turned ידי׃ my hand}%
\verse{משׂנאי The haters יהוה of the LORD יכחשׁו should have submitted לו ויהי should have endured עתם themselves unto him: but their time לעולם׃ forever.}%
\verse{ויאכילהו He should have fed מחלב them also with the finest חטה of the wheat: ומצור out of the rock דבשׁ and with honey אשׂביעך׃ should I have satisfied}%
\end{biblechapter}%
\begin{biblechapter}% Psalm 82
\verseWithHeading{God Commands Justice}{מזמור A Psalm לאסף of Asaph. אלהים God נצב standeth בעדת in the congregation אל of the mighty; בקרב among אלהים the gods. ישׁפט׃ he judgeth}%
\verse{עד מתישׁפטו will ye judge עול unjustly, ופני the persons רשׁעים of the wicked? תשׂאו and accept סלה׃ Selah.}%
\verse{שׁפטו Defend דל the poor ויתום and fatherless: עני to the afflicted ורשׁ and needy. הצדיקו׃ do justice}%
\verse{פלטו Deliver דל the poor ואביון and needy: מיד out of the hand רשׁעים of the wicked. הצילו׃ rid}%
\verse{לא not, ידעו They know ולא neither יבינו will they understand; בחשׁכה in darkness: יתהלכו they walk on ימוטו are out of course. כל all מוסדי ארץ׃ of the earth}%
\verse{אני I אמרתי have said, אלהים Ye gods; אתם of you ובני children עליון of the most High. כלכם׃ and all}%
\verse{אכן But כאדם like men, תמותון ye shall die וכאחד like one השׂרים of the princes. תפלו׃ and fall}%
\verse{קומה Arise, אלהים O God, שׁפטה judge הארץ the earth: כי for אתה thou תנחל shalt inherit בכל all הגוים׃ nations.}%
\end{biblechapter}%
\begin{biblechapter}% Psalm 83
\verseWithHeading{A Request to Act against Israel’s Neighbors}{שׁיר A Song מזמור Psalm לאסף׃ of Asaph. אלהים O God: אל hold not thy peace, דמי לך אל and be not תחרשׁ hold not thy peace, ואל תשׁקט still, אל׃ O God.}%
\verse{כי For, הנה lo, אויביך thine enemies יהמיון make a tumult: ומשׂנאיך and they that hate נשׂאו thee have lifted up ראשׁ׃ the head.}%
\verse{על against עמך thy people, יערימו They have taken crafty סוד counsel ויתיעצו and consulted על against צפוניך׃ thy hidden ones.}%
\verse{אמרו They have said, לכו Come, ונכחידם and let us cut them off מגוי from a nation; ולא may be no יזכר in remembrance. שׁם that the name ישׂראל of Israel עוד׃ more}%
\verse{כי For נועצו they have consulted together לב with one consent: יחדו with one consent: עליך against ברית they are confederate יכרתו׃ they are confederate}%
\verse{אהלי The tabernacles אדום of Edom, וישׁמעאלים and the Ishmaelites; מואב of Moab, והגרים׃ and the Hagarenes;}%
\verse{גבל Gebal, ועמון and Ammon, ועמלק and Amalek; פלשׁת the Philistines עם with ישׁבי the inhabitants צור׃ of Tyre;}%
\verse{גם also אשׁור Assur נלוה is joined עמם with היו them: they have זרוע helped לבני the children לוט of Lot. סלה׃ Selah.}%
\verse{עשׂה Do להם כמדין unto them as the Midianites; כסיסרא as Sisera, כיבין as Jabin, בנחל at the brook קישׁון׃ of Kison:}%
\verse{נשׁמדו perished בעין דאר at Endor: היו they became דמן dung לאדמה׃ for the earth.}%
\verse{שׁיתמו Make נדיבמו their nobles כערב like Oreb, וכזאב and like Zeeb: וכזבח as Zebah, וכצלמנע and as Zalmunna: כל yea, all נסיכמו׃ their princes}%
\verse{אשׁר Who אמרו said, נירשׁה Let us take לנו את נאות to ourselves the houses אלהים׃ of God}%
\verse{אלהי O my God, שׁיתמו make כגלגל them like a wheel; כקשׁ as the stubble לפני before רוח׃ the wind.}%
\verse{כאשׁ As the fire תבער burneth יער a wood, וכלהבה and as the flame תלהט setteth the mountains on fire; הרים׃ setteth the mountains on fire;}%
\verse{כן So תרדפם persecute בסערך them with thy tempest, ובסופתך with thy storm. תבהלם׃ and make them afraid}%
\verse{מלא Fill פניהם their faces קלון with shame; ויבקשׁו that they may seek שׁמך thy name, יהוה׃ O LORD.}%
\verse{יבשׁו Let them be confounded ויבהלו and troubled עדי forever; עד forever; ויחפרו yea, let them be put to shame, ויאבדו׃ and perish:}%
\verse{וידעו That may know כי that אתה thou, שׁמך whose name יהוה JEHOVAH, לבדך alone עליון the most high על over כל all הארץ׃ the earth.}%
\end{biblechapter}%
\begin{biblechapter}% Psalm 84
\verseWithHeading{The Joy of Worshiping in the Temple}{למנצח To the chief Musician על upon הגתית Gittith, לבני for the sons קרח of Korah. מזמור׃ A Psalm מה How ידידות amiable משׁכנותיך thy tabernacles, יהוה O LORD צבאות׃ of hosts!}%
\verse{נכספה longeth, וגם yea even כלתה fainteth נפשׁי My soul לחצרות for the courts יהוה of the LORD: לבי my heart ובשׂרי and my flesh ירננו crieth out אל for אל God. חי׃ the living}%
\verse{גם Yea, צפור the sparrow מצאה hath found בית a house, ודרור and the swallow קן a nest לה אשׁר for herself, where שׁתה she may lay אפרחיה her young, את מזבחותיך thine altars, יהוה O LORD צבאות of hosts, מלכי my King, ואלהי׃ and my God.}%
\verse{אשׁרי Blessed יושׁבי they that dwell ביתך in thy house: עוד they will be still יהללוך praising סלה׃ thee. Selah.}%
\verse{אשׁרי Blessed אדם the man עוז whose strength לו בך מסלות the ways בלבבם׃ in thee; in whose heart}%
\verse{עברי passing בעמק through the valley הבכא of Baca מעין it a well; ישׁיתוהו make גם also ברכות the pools. יעטה filleth מורה׃ the rain}%
\verse{ילכו They go מחיל from strength אל to חיל strength, יראה appeareth אל before אלהים God. בציון׃ in Zion}%
\verse{יהוה O LORD אלהים God צבאות of hosts, שׁמעה hear תפלתי my prayer: האזינה give ear, אלהי O God יעקב of Jacob. סלה׃ Selah.}%
\verse{מגננו our shield, ראה Behold, אלהים O God והבט and look upon פני the face משׁיחך׃ of thine anointed.}%
\verse{כי For טוב better יום a day בחצריך in thy courts מאלף than a thousand. בחרתי I had rather הסתופף be a doorkeeper בבית in the house אלהי of my God, מדור than to dwell באהלי in the tents רשׁע׃ of wickedness.}%
\verse{כי For שׁמשׁ a sun ומגן and shield: יהוה the LORD אלהים God חן grace וכבוד and glory: יתן will give יהוה the LORD לא no ימנע will he withhold טוב good להלכים from them that walk בתמים׃ uprightly.}%
\verse{יהוה O LORD צבאות of hosts, אשׁרי blessed אדם the man בטח׃ that trusteth}%
\end{biblechapter}%
\begin{biblechapter}% Psalm 85
\verseWithHeading{Hope in God’s Future Help}{למנצח To the chief Musician, לבני for the sons קרח of Korah. מזמור׃ A Psalm רצית thou hast been favorable יהוה LORD, ארצך unto thy land: שׁבת thou hast brought back שׁבות the captivity יעקב׃ of Jacob.}%
\verse{נשׂאת Thou hast forgiven עון the iniquity עמך of thy people, כסית thou hast covered כל all חטאתם their sin. סלה׃ Selah.}%
\verse{אספת Thou hast taken away כל all עברתך thy wrath: השׁיבות thou hast turned מחרון from the fierceness אפך׃ of thine anger.}%
\verse{שׁובנו Turn אלהי us, O God ישׁענו of our salvation, והפר us to cease. כעסך and cause thine anger עמנו׃ toward}%
\verse{הלעולם with us forever? תאנף Wilt thou be angry בנו תמשׁך wilt thou draw out אפך thine anger לדר to all generations? ודר׃ to all generations?}%
\verse{הלא not אתה Wilt thou תשׁוב us again: תחינו revive ועמך that thy people ישׂמחו׃ may rejoice}%
\verse{הראנו Show יהוה O LORD, חסדך us thy mercy, וישׁעך us thy salvation. תתן׃ and grant}%
\verse{אשׁמעה I will hear מה what ידבר will speak: האל God יהוה the LORD כי for ידבר he will speak שׁלום peace אל unto עמו his people, ואל and to חסידיו his saints: ואל but let them not ישׁובו turn again לכסלה׃ to folly.}%
\verse{אך Surely קרוב nigh ליראיו them that fear ישׁעו his salvation לשׁכן may dwell כבוד him; that glory בארצנו׃ in our land.}%
\verse{חסד Mercy ואמת and truth נפגשׁו are met together; צדק righteousness ושׁלום and peace נשׁקו׃ have kissed}%
\verse{אמת Truth מארץ out of the earth; תצמח shall spring וצדק and righteousness משׁמים from heaven. נשׁקף׃ shall look down}%
\verse{גם Yea, יהוה the LORD יתן shall give הטוב good; וארצנו and our land תתן shall yield יבולה׃ her increase.}%
\verse{צדק Righteousness לפניו before יהלך shall go וישׂם him; and shall set לדרך in the way פעמיו׃ of his steps.}%
\end{biblechapter}%
\begin{biblechapter}% Psalm 86
\verseWithHeading{A Prayer for Help against Ruthless Men}{תפלה A Prayer לדוד of David. הטה Bow down יהוה O LORD אזנך thine ear, ענני hear כי me: for עני poor ואביון and needy. אני׃ I}%
\verse{שׁמרה Preserve נפשׁי my soul; כי for חסיד holy: אני I הושׁע save עבדך thy servant אתה O thou אלהי my God, הבוטח that trusteth אליך׃ in thee.}%
\verse{חנני Be merciful אדני unto me, O Lord: כי for אליך unto אקרא I cry כל thee daily. היום׃ thee daily.}%
\verse{שׂמח Rejoice נפשׁ the soul עבדך of thy servant: כי for אליך unto אדני thee, O Lord, נפשׁי my soul. אשׂא׃ do I lift up}%
\verse{כי For אתה thou, אדני Lord, טוב good, וסלח and ready to forgive; ורב and plenteous חסד in mercy לכל unto all קראיך׃ them that call upon}%
\verse{האזינה Give ear, יהוה O LORD, תפלתי unto my prayer; והקשׁיבה and attend בקול to the voice תחנונותי׃ of my supplications.}%
\verse{ביום In the day צרתי of my trouble אקראך I will call upon כי thee: for תענני׃ thou wilt answer}%
\verse{אין none כמוך like unto thee, באלהים Among the gods אדני O Lord; ואין neither כמעשׂיך׃ like unto thy works.}%
\verse{כל All גוים nations אשׁר whom עשׂית thou hast made יבואו shall come וישׁתחוו and worship לפניך before אדני thee, O Lord; ויכבדו and shall glorify לשׁמך׃ thy name.}%
\verse{כי For גדול great, אתה thou ועשׂה and doest נפלאות wondrous things: אתה thou אלהים God לבדך׃ alone.}%
\verse{הורני Teach יהוה O LORD; דרכך me thy way, אהלך I will walk באמתך in thy truth: יחד unite לבבי my heart ליראה to fear שׁמך׃ thy name.}%
\verse{אודך I will praise אדני thee, O Lord אלהי my God, בכל with all לבבי my heart: ואכבדה and I will glorify שׁמך thy name לעולם׃ forevermore.}%
\verse{כי For חסדך thy mercy גדול great עלי toward והצלת me: and thou hast delivered נפשׁי my soul משׁאול hell. תחתיה׃ from the lowest}%
\verse{אלהים O God, זדים the proud קמו are risen עלי against ועדת me, and the assemblies עריצים of violent בקשׁו have sought after נפשׁי my soul; ולא and have not שׂמוך set לנגדם׃ thee before}%
\verse{ואתה But thou, אדני O Lord, אל a God רחום full of compassion, וחנון and gracious, ארך longsuffering, אפים longsuffering, ורב and plenteous חסד in mercy ואמת׃ and truth.}%
\verse{פנה O turn אלי unto וחנני me, and have mercy תנה upon me; give עזך thy strength לעבדך unto thy servant, והושׁיעה and save לבן the son אמתך׃ of thine handmaid.}%
\verse{עשׂה עמיות me a token לטובה for good; ויראו me may see שׂנאי that they which hate ויבשׁו and be ashamed: כי because אתה thou, יהוה LORD, עזרתני hast helped ונחמתני׃ me, and comforted}%
\end{biblechapter}%
\begin{biblechapter}% Psalm 87
\verseWithHeading{Foreign Nations Come to Worship in Jerusalem}{לבני for the sons קרח of Korah. מזמור A Psalm שׁיר Song יסודתו His foundation בהררי mountains. קדשׁ׃ in the holy}%
\verse{אהב loveth יהוה The LORD שׁערי the gates ציון of Zion מכל more than all משׁכנות the dwellings יעקב׃ of Jacob.}%
\verse{נכבדות Glorious things מדבר are spoken בך עיר of thee, O city האלהים of God. סלה׃ Selah.}%
\verse{אזכיר I will make mention רהב of Rahab ובבל and Babylon לידעי to them that know הנה me: behold פלשׁת Philistia, וצור and Tyre, עם with כושׁ Ethiopia; זה this ילד was born שׁם׃ there.}%
\verse{ולציון And of Zion יאמר it shall be said, אישׁ This ואישׁ and that man ילד was born בה והוא himself יכוננה shall establish עליון׃ in her: and the highest}%
\verse{יהוה The LORD יספר shall count, בכתוב when he writeth up עמים the people, זה this ילד was born שׁם there. סלה׃ Selah.}%
\verse{ושׁרים As well the singers כחללים as the players on instruments כל all מעיני׃ my springs}%
\end{biblechapter}%
\begin{biblechapter}% Psalm 88
\verseWithHeading{A Prayer for Help in Despair}{שׁיר A Song מזמור Psalm לבני for the sons קרח of Korah, למנצח to the chief Musician על upon מחלת Mahalath לענות Leannoth, משׂכיל Maschil להימן of Heman האזרחי׃ the Ezrahite. יהוה O LORD אלהי God ישׁועתי of my salvation, יום day צעקתי I have cried בלילה night נגדך׃ before}%
\verse{תבוא come לפניך before תפלתי Let my prayer הטה thee: incline אזנך thine ear לרנתי׃ unto my cry;}%
\verse{כי For שׂבעה is full ברעות of troubles: נפשׁי my soul וחיי and my life לשׁאול unto the grave. הגיעו׃ draweth nigh}%
\verse{נחשׁבתי I am counted עם with יורדי them that go down בור into the pit: הייתי I am כגבר as a man אין no איל׃ strength:}%
\verse{במתים among the dead, חפשׁי Free כמו like חללים the slain שׁכבי that lie קבר in the grave, אשׁר whom לא no זכרתם thou rememberest עוד more: והמה and they מידך from thy hand. נגזרו׃ are cut off}%
\verse{שׁתני Thou hast laid בבור pit, תחתיות me in the lowest במחשׁכים in darkness, במצלות׃ in the deeps.}%
\verse{עלי upon סמכה lieth hard חמתך Thy wrath וכל with all משׁבריך thy waves. ענית me, and thou hast afflicted סלה׃ Selah.}%
\verse{הרחקת Thou hast put away מידעי mine acquaintance ממני far from שׁתני me; thou hast made תועבות me an abomination למו כלא unto them: shut up, ולא and I cannot אצא׃ come forth.}%
\verse{עיני Mine eye דאבה mourneth מני by reason of עני affliction: קראתיך I have called יהוה LORD, בכל daily יום daily שׁטחתי upon thee, I have stretched out אליך unto thee. כפי׃ my hands}%
\verse{הלמתים to the dead? תעשׂה Wilt thou show פלא wonders אם רפאים shall the dead יקומו arise יודוך praise סלה׃ thee? Selah.}%
\verse{היספר be declared בקבר in the grave? חסדך Shall thy lovingkindness אמונתך thy faithfulness באבדון׃ in destruction?}%
\verse{היודע be known בחשׁך in the dark? פלאך Shall thy wonders וצדקתך and thy righteousness בארץ in the land נשׁיה׃ of forgetfulness?}%
\verse{ואני thee have I אליך But unto יהוה O LORD; שׁועתי cried, ובבקר and in the morning תפלתי shall my prayer תקדמך׃ prevent}%
\verse{למה why יהוה LORD, תזנח castest thou off נפשׁי my soul? תסתיר hidest פניך thou thy face ממני׃ from}%
\verse{עני afflicted אני I וגוע and ready to die מנער from youth up: נשׂאתי I suffer אמיך thy terrors אפונה׃ I am distracted.}%
\verse{עלי over עברו goeth חרוניך Thy fierce wrath בעותיך me; thy terrors צמתותני׃ have cut me off.}%
\verse{סבוני They came round about כמים like water; כל me daily היום me daily הקיפו they compassed me about עלי they compassed me about יחד׃ together.}%
\verse{הרחקת hast thou put far ממני from אהב Lover ורע and friend מידעי me, mine acquaintance מחשׁך׃ into darkness.}%
\end{biblechapter}%
\begin{biblechapter}% Psalm 89
\verseWithHeading{Remembering the Covenant with David, and Sorrow for Lost Blessings}{משׂכיל Maschil לאיתן of Ethan האזרחי׃ the Ezrahite. חסדי of the mercies יהוה of the LORD עולם forever: אשׁירה I will sing לדר to all generations. ודר to all generations. אודיע will I make known אמונתך thy faithfulness בפי׃ with my mouth}%
\verse{כי For אמרתי I have said, עולם forever: חסד Mercy יבנה shall be built up שׁמים in the very heavens. תכן shalt thou establish אמונתך׃ thy faithfulness}%
\verse{כרתי I have made ברית a covenant לבחירי with my chosen, נשׁבעתי I have sworn לדוד unto David עבדי׃ my servant,}%
\verse{עד forever, עולם forever, אכין will I establish זרעך Thy seed ובניתי and build up לדר to all generations. ודור >> r="#000000">to all generations. כסאך thy throne סלה׃ Selah.}%
\verse{ויודו shall praise שׁמים And the heavens פלאך thy wonders, יהוה O LORD: אף also אמונתך thy faithfulness בקהל in the congregation קדשׁים׃ of the saints.}%
\verse{כי For מי who בשׁחק in the heaven יערך can be compared ליהוה unto the LORD? ידמה can be likened ליהוה unto the LORD? בבני among the sons אלים׃ of the mighty}%
\verse{אל God נערץ to be feared בסוד in the assembly קדשׁים of the saints, רבה is greatly ונורא and to be had in reverence על of כל all סביביו׃ about}%
\verse{יהוה O LORD אלהי God צבאות of hosts, מי who כמוך like unto thee? חסין a strong יה LORD ואמונתך or to thy faithfulness סביבותיך׃ round about}%
\verse{אתה Thou מושׁל rulest בגאות the raging הים of the sea: בשׂוא thereof arise, גליו when the waves אתה thou תשׁבחם׃ stillest}%
\verse{אתה Thou דכאת hast broken כחלל in pieces, as one that is slain; רהב Rahab בזרוע arm. עזך with thy strong פזרת thou hast scattered אויביך׃ thine enemies}%
\verse{לך שׁמים The heavens אף also לך ארץ thine, the earth תבל thine: the world ומלאה and the fullness אתה thereof, thou יסדתם׃ hast founded}%
\verse{צפון The north וימין and the south אתה thou בראתם hast created תבור them: Tabor וחרמון and Hermon בשׁמך in thy name. ירננו׃ shall rejoice}%
\verse{לך זרוע arm: עם גבורה Thou hast a mighty תעז strong ידך is thy hand, תרום high ימינך׃ is thy right hand.}%
\verse{צדק Justice ומשׁפט and judgment מכון the habitation כסאך of thy throne: חסד mercy ואמת and truth יקדמו shall go פניך׃ before thy face.}%
\verse{אשׁרי Blessed העם the people יודעי that know תרועה the joyful sound: יהוה O LORD, באור in the light פניך of thy countenance. יהלכון׃ they shall walk,}%
\verse{בשׁמך In thy name יגילון shall they rejoice כל all היום the day: ובצדקתך and in thy righteousness ירומו׃ shall they be exalted.}%
\verse{כי For תפארת the glory עזמו of their strength: אתה thou וברצנך and in thy favor תרים shall be exalted. קרננו׃ our horn}%
\verse{כי For ליהוה the LORD מגננו our defense; ולקדושׁ and the Holy One ישׂראל of Israel מלכנו׃ our king.}%
\verse{אז Then דברת thou spakest בחזון in vision לחסידיך to thy holy one, ותאמר and saidst, שׁויתי I have laid עזר help על upon גבור mighty; הרימותי I have exalted בחור chosen מעם׃ out of the people.}%
\verse{מצאתי I have found דוד David עבדי my servant; בשׁמן oil קדשׁי with my holy משׁחתיו׃ have I anointed}%
\verse{אשׁר whom ידי my hand תכון shall be established: עמו With אף also זרועי mine arm תאמצנו׃ shall strengthen}%
\verse{לא shall not ישׁא exact אויב The enemy בו ובן the son עולה of wickedness לא upon him; nor יעננו׃ afflict}%
\verse{וכתותי And I will beat down מפניו before his face, צריו his foes ומשׂנאיו them that hate אגוף׃ and plague}%
\verse{ואמונתי But my faithfulness וחסדי and my mercy עמו with ובשׁמי him: and in my name תרום be exalted. קרנו׃ shall his horn}%
\verse{ושׂמתי I will set בים also in the sea, ידו his hand ובנהרות in the rivers. ימינו׃ and his right hand}%
\verse{הוא He יקראני shall cry אבי my father, אתה unto me, Thou אלי my God, וצור and the rock ישׁועתי׃ of my salvation.}%
\verse{אף Also אני I בכור him firstborn, אתנהו will make עליון higher למלכי than the kings ארץ׃ of the earth.}%
\verse{לעולם for him forevermore, אשׁמור will I keep לו חסדי My mercy ובריתי and my covenant נאמנת׃ shall stand fast}%
\verse{ושׂמתי also will I make לעד forever, זרעו His seed וכסאו and his throne כימי as the days שׁמים׃ of heaven.}%
\verse{אם If יעזבו forsake בניו his children תורתי my law, ובמשׁפטי in my judgments; לא not ילכון׃ and walk}%
\verse{אם If חקתי my statutes, יחללו they break ומצותי my commandments; לא not ישׁמרו׃ and keep}%
\verse{ופקדתי Then will I visit בשׁבט with the rod, פשׁעם their transgression ובנגעים with stripes. עונם׃ and their iniquity}%
\verse{וחסדי Nevertheless my lovingkindness לא will I not אפיר utterly take מעמו from ולא him, nor אשׁקר suffer my faithfulness to fail. באמונתי׃ suffer my faithfulness to fail.}%
\verse{לא will I not אחלל break, בריתי My covenant ומוצא the thing that is gone out שׂפתי of my lips. לא nor אשׁנה׃ alter}%
\verse{אחת Once נשׁבעתי have I sworn בקדשׁי by my holiness אם that I will not לדוד unto David. אכזב׃ lie}%
\verse{זרעו His seed לעולם forever, יהיה shall endure וכסאו and his throne כשׁמשׁ as the sun נגדי׃ before}%
\verse{כירח as the moon, יכון It shall be established עולם forever ועד witness בשׁחק in heaven. נאמן and a faithful סלה׃ Selah.}%
\verse{ואתה But thou זנחת hast cast off ותמאס and abhorred, התעברת thou hast been wroth עם with משׁיחך׃ thine anointed.}%
\verse{נארתה Thou hast made void ברית the covenant עבדך of thy servant: חללת thou hast profaned לארץ to the ground. נזרו׃ his crown}%
\verse{פרצת Thou hast broken down כל all גדרתיו his hedges; שׂמת thou hast brought מבצריו his strongholds מחתה׃ to ruin.}%
\verse{שׁסהו spoil כל All עברי that pass by דרך the way היה him: he is חרפה a reproach לשׁכניו׃ to his neighbors.}%
\verse{הרימות Thou hast set up ימין the right hand צריו of his adversaries; השׂמחת to rejoice. כל thou hast made all אויביו׃ his enemies}%
\verse{אף Thou hast also תשׁיב turned צור the edge חרבו of his sword, ולא and hast not הקימתו made him to stand במלחמה׃ in the battle.}%
\verse{השׁבת to cease, מטהרו וכסאו his throne לארץ down to the ground. מגרתה׃ and cast}%
\verse{הקצרת hast thou shortened: ימי The days עלומיו of his youth העטית thou hast covered עליו thou hast covered בושׁה him with shame. סלה׃ Selah.}%
\verse{עד מההוה LORD? תסתר wilt thou hide thyself לנצח forever? תבער burn כמו like אשׁ fire? חמתך׃ shall thy wrath}%
\verse{זכר Remember אני my מה how חלד short על time is: wherefore מה time is: wherefore שׁוא in vain? בראת hast thou made כל all בני men אדם׃ men}%
\verse{מי What גבר man יחיה liveth, ולא and shall not יראה see מות death? ימלט shall he deliver נפשׁו his soul מיד from the hand שׁאול of the grave? סלה׃ Selah.}%
\verse{איה where חסדיך lovingkindnesses, הראשׁנים thy former אדני Lord, נשׁבעת thou swarest לדוד unto David באמונתך׃ in thy truth?}%
\verse{זכר Remember, אדני Lord, חרפת the reproach עבדיך of thy servants; שׂאתי I do bear בחיקי in my bosom כל all רבים the mighty עמים׃ people;}%
\verse{אשׁר Wherewith חרפו have reproached, אויביך thine enemies יהוה O LORD; אשׁר wherewith חרפו they have reproached עקבות the footsteps משׁיחך׃ of thine anointed.}%
\verse{ברוך Blessed יהוה the LORD לעולם forevermore. אמן Amen, ואמן׃ and Amen.}%
\end{biblechapter}%
\begin{biblechapter}% Psalm 90
\verseWithHeading{God’s Eternity and Human Frailty}{תפלה A Prayer למשׁה of Moses אישׁ the man האלהים of God. אדני Lord, מעון our dwelling place אתה thou היית hast been לנו בדר in all generations. ודר׃ in all generations.}%
\verse{בטרם Before הרים the mountains ילדו were brought forth, ותחולל or ever thou hadst formed ארץ the earth ותבל and the world, ומעולם even from everlasting עד to עולם everlasting, אתה thou אל׃ God.}%
\verse{תשׁב Thou turnest אנושׁ man עד to דכא destruction; ותאמר and sayest, שׁובו Return, בני ye children אדם׃ of men.}%
\verse{כי For אלף a thousand שׁנים years בעיניך in thy sight כיום as yesterday אתמול as yesterday כי when יעבר it is past, ואשׁמורה and a watch בלילה׃ in the night.}%
\verse{זרמתם Thou carriest them away as with a flood; שׁנה a sleep: יהיו they are בבקר in the morning כחציר like grass יחלף׃ groweth up.}%
\verse{בבקר In the morning יציץ it flourisheth, וחלף and groweth up; לערב in the evening ימולל it is cut down, ויבשׁ׃ and withereth.}%
\verse{כי For כלינו we are consumed באפך by thine anger, ובחמתך and by thy wrath נבהלנו׃ are we troubled.}%
\verse{שׁת Thou hast set עונתינו our iniquities לנגדך before עלמנו thee, our secret למאור in the light פניך׃ of thy countenance.}%
\verse{כי For כל all ימינו our days פנו are passed away בעברתך in thy wrath: כלינו we spend שׁנינו our years כמו as הגה׃ a tale}%
\verse{ימי The days שׁנותינו of our years בהם שׁבעים threescore years and ten; שׁנה threescore years and ten; ואם and if בגבורת by reason of strength שׁמונים fourscore שׁנה years, ורהבם yet their strength עמל labor ואון and sorrow; כי for גז cut off, חישׁ it is soon ונעפה׃ and we fly away.}%
\verse{מי Who יודע knoweth עז the power אפך of thine anger? וכיראתך even according to thy fear, עברתך׃ thy wrath.}%
\verse{למנות to number ימינו our days, כן So הודע teach ונבא that we may apply לבב hearts חכמה׃ unto wisdom.}%
\verse{שׁובה Return, יהוה O LORD, עד how long? מתי how long? והנחם and let it repent על thee concerning עבדיך׃ thy servants.}%
\verse{שׂבענו O satisfy בבקר us early חסדך with thy mercy; ונרננה that we may rejoice ונשׂמחה and be glad בכל all ימינו׃ our days.}%
\verse{שׂמחנו Make us glad כימות according to the days עניתנו thou hast afflicted שׁנות us, the years ראינו we have seen רעה׃ evil.}%
\verse{יראה appear אל unto עבדיך thy servants, פעלך Let thy work והדרך and thy glory על unto בניהם׃ their children.}%
\verse{ויהי be נעם And let the beauty אדני of the LORD אלהינו our God עלינו upon ומעשׂה thou the work ידינו of our hands כוננה us: and establish עלינו upon ומעשׂה us; yea, the work ידינו of our hands כוננהו׃ establish}%
\end{biblechapter}%
\begin{biblechapter}% Psalm 91
\verseWithHeading{God’s Protection in Times of Crisis}{ישׁב He that dwelleth בסתר in the secret place עליון of the most High בצל under the shadow שׁדי of the Almighty. יתלונן׃ shall abide}%
\verse{אמר I will say ליהוה of the LORD, מחסי my refuge ומצודתי and my fortress: אלהי my God; אבטח׃ in him will I trust.}%
\verse{כי Surely הוא he יצילך shall deliver מפח thee from the snare יקושׁ of the fowler, מדבר הוות׃}%
\verse{באברתו thee with his feathers, יסך He shall cover לך ותחת and under כנפיו his wings תחסה shalt thou trust: צנה shield וסחרה and buckler. אמתו׃ his truth}%
\verse{לא Thou shalt not תירא be afraid מפחד for the terror לילה by night; מחץ for the arrow יעוף flieth יומם׃ by day;}%
\verse{מדבר באפל in darkness; יהלך walketh מקטב for the destruction ישׁוד wasteth צהרים׃ at noonday.}%
\verse{יפל shall fall מצדך at thy side, אלף A thousand ורבבה and ten thousand מימינך at thy right hand; אליך come nigh לא it shall not יגשׁ׃ come nigh}%
\verse{רק Only בעיניך with thine eyes תביט shalt thou behold ושׁלמת the reward רשׁעים of the wicked. תראה׃ and see}%
\verse{כי Because אתה thou יהוה the LORD, מחסי my refuge, עליון the most High, שׂמת hast made מעונך׃ thy habitation;}%
\verse{לא There shall no תאנה befall אליך befall רעה evil ונגע shall any plague לא thee, neither יקרב come nigh באהלך׃ thy dwelling.}%
\verse{כי For מלאכיו he shall give his angels יצוה charge לך לשׁמרך over thee, to keep בכל thee in all דרכיך׃ thy ways.}%
\verse{על in כפים hands, ישׂאונך They shall bear thee up פן lest תגף thou dash באבן against a stone. רגלך׃ thy foot}%
\verse{על upon שׁחל the lion ופתן and adder: תדרך Thou shalt tread תרמס shalt thou trample under feet כפיר the young lion ותנין׃ and the dragon}%
\verse{כי Because בי חשׁק he hath set his love ואפלטהו upon me, therefore will I deliver אשׂגבהו him: I will set him on high, כי because ידע he hath known שׁמי׃ my name.}%
\verse{יקראני He shall call upon ואענהו me, and I will answer עמו with אנכי him: I בצרה him in trouble; אחלצהו I will deliver ואכבדהו׃ him, and honor}%
\verse{ארך With long ימים life אשׂביעהו will I satisfy ואראהו him, and show בישׁועתי׃ him my salvation.}%
\end{biblechapter}%
\begin{biblechapter}% Psalm 92
\verseWithHeading{Thanksgiving to Adonai for Victory}{מזמור A Psalm שׁיר Song ליום day. השׁבת׃ for the sabbath טוב good להדות to give thanks ליהוה unto the LORD, ולזמר and to sing praises לשׁמך unto thy name, עליון׃ O most High:}%
\verse{להגיד To show forth בבקר in the morning, חסדך thy lovingkindness ואמונתך and thy faithfulness בלילות׃ every night,}%
\verse{עלי Upon עשׂור an instrument of ten strings, ועלי and upon נבל the psaltery; עלי with הגיון a solemn sound. בכנור׃ upon the harp}%
\verse{כי For שׂמחתני hast made me glad יהוה thou, LORD, בפעלך through thy work: במעשׂי in the works ידיך of thy hands. ארנן׃ I will triumph}%
\verse{מה how גדלו great מעשׂיך are thy works! יהוה O LORD, מאד are very עמקו deep. מחשׁבתיך׃ thy thoughts}%
\verse{אישׁ man בער A brutish לא not; ידע knoweth וכסיל doth a fool לא neither יבין understand את זאת׃ this.}%
\verse{בפרח spring רשׁעים When the wicked כמו as עשׂב the grass, ויציצו do flourish; כל and when all פעלי the workers און of iniquity להשׁמדם that they shall be destroyed עדי forever: עד׃ forever:}%
\verse{ואתה But thou, מרום high לעלם forevermore. יהוה׃ LORD,}%
\verse{כי For, הנה lo, איביך thine enemies, יהוה O LORD, כי for, הנה lo, איביך thine enemies יאבדו shall perish; יתפרדו shall be scattered. כל all פעלי the workers און׃ of iniquity}%
\verse{ותרם shalt thou exalt כראים like a unicorn: קרני >> r="#000000">But my horn בלתי I shall be anointed בשׁמן oil. רענן׃ with fresh}%
\verse{ותבט also shall see עיני Mine eye בשׁורי on mine enemies, בקמים that rise up עלי against מרעים of the wicked תשׁמענה shall hear אזני׃ mine ears}%
\verse{צדיק The righteous כתמר like the palm tree: יפרח shall flourish כארז like a cedar בלבנון in Lebanon. ישׂגה׃ he shall grow}%
\verse{שׁתולים Those that be planted בבית in the house יהוה of the LORD בחצרות in the courts אלהינו of our God. יפריחו׃ shall flourish}%
\verse{עוד They shall still ינובון bring forth fruit בשׂיבה in old age; דשׁנים fat ורעננים and flourishing; יהיו׃ they shall be}%
\verse{להגיד To show כי that ישׁר upright: יהוה the LORD צורי my rock, ולא and no עלתה׃ unrighteousness}%
\end{biblechapter}%
\begin{biblechapter}% Psalm 93
\verseWithHeading{Adonai Is King Over All the Earth}{יהוה The LORD מלך reigneth, גאות with majesty; לבשׁ he is clothed לבשׁ is clothed יהוה the LORD עז with strength, התאזר he hath girded himself: אף also תכון is established, תבל the world בל that it cannot תמוט׃ be moved.}%
\verse{נכון established כסאך Thy throne מאז of old: מעולם from everlasting. אתה׃ thou}%
\verse{נשׂאו have lifted up, נהרות The floods יהוה O LORD, נשׂאו have lifted up נהרות the floods קולם their voice; ישׂאו lift up נהרות the floods דכים׃ their waves.}%
\verse{מקלות than the noise מים waters, רבים of many אדירים mightier משׁברי waves ים of the sea. אדיר the mighty במרום on high יהוה׃ The LORD}%
\verse{עדתיך Thy testimonies נאמנו sure: מאד are very לביתך thine house, נאוה becometh קדשׁ holiness יהוה O LORD, לארך forever. ימים׃ forever.}%
\end{biblechapter}%
\begin{biblechapter}% Psalm 94
\verseWithHeading{A Prayer for Retribution against Oppressors}{אל God, נקמות to whom vengeance יהוה O LORD אל belongeth; O God, נקמות to whom vengeance הופיע׃ belongeth, show thyself.}%
\verse{הנשׂא Lift up thyself, שׁפט thou judge הארץ of the earth: השׁב render גמול a reward על to גאים׃ the proud.}%
\verse{עד how long מתי how long רשׁעים shall the wicked, יהוה LORD, עד how long מתי how long רשׁעים shall the wicked יעלזו׃ triumph?}%
\verse{יביעו shall they utter ידברו speak עתק hard things? יתאמרו boast themselves? כל all פעלי the workers און׃ of iniquity}%
\verse{עמך thy people, יהוה O LORD, ידכאו They break in pieces ונחלתך thine heritage. יענו׃ and afflict}%
\verse{אלמנה the widow וגר and the stranger, יהרגו They slay ויתומים the fatherless. ירצחו׃ and murder}%
\verse{ויאמרו Yet they say, לא shall not יראה see, יה The LORD ולא neither יבין regard אלהי shall the God יעקב׃ of Jacob}%
\verse{בינו Understand, בערים ye brutish בעם among the people: וכסילים and fools, מתי when תשׂכילו׃ will ye be wise?}%
\verse{הנטע He that planted אזן the ear, הלא shall he not ישׁמע hear? אם יצר he that formed עין the eye, הלא shall he not יביט׃ see?}%
\verse{היסר He that chastiseth גוים the heathen, הלא shall not יוכיח he correct? המלמד he that teacheth אדם man דעת׃ knowledge,}%
\verse{יהוה The LORD ידע knoweth מחשׁבות the thoughts אדם of man, כי that המה they הבל׃ vanity.}%
\verse{אשׁרי Blessed הגבר the man אשׁר whom תיסרנו thou chastenest, יה O LORD, ומתורתך him out of thy law; תלמדנו׃ and teachest}%
\verse{להשׁקיט That thou mayest give him rest לו מימי from the days רע of adversity, עד until יכרה be digged לרשׁע for the wicked. שׁחת׃ the pit}%
\verse{כי For לא will not יטשׁ cast off יהוה the LORD עמו his people, ונחלתו his inheritance. לא neither יעזב׃ will he forsake}%
\verse{כי But עד unto צדק righteousness: ישׁוב shall return משׁפט judgment ואחריו shall follow כל and all ישׁרי the upright לב׃ in heart}%
\verse{מי Who יקום will rise up לי עם for me against מרעים the evildoers? מי who יתיצב will stand up לי עם for me against פעלי the workers און׃ of iniquity?}%
\verse{לולי Unless יהוה the LORD עזרתה my help, לי כמעט had almost שׁכנה dwelt דומה in silence. נפשׁי׃ my soul}%
\verse{אם When אמרתי I said, מטה slippeth; רגלי My foot חסדך thy mercy, יהוה O LORD, יסעדני׃ held me up.}%
\verse{ברב In the multitude שׂרעפי of my thoughts בקרבי within תנחומיך me thy comforts ישׁעשׁעו delight נפשׁי׃ my soul.}%
\verse{היחברך have fellowship כסא Shall the throne הוות of iniquity יצר with thee, which frameth עמל mischief עלי by חק׃ a law?}%
\verse{יגודו They gather themselves together על against נפשׁ the soul צדיק of the righteous, ודם blood. נקי the innocent ירשׁיעו׃ and condemn}%
\verse{ויהי is יהוה But the LORD לי למשׂגב my defense; ואלהי and my God לצור the rock מחסי׃ of my refuge.}%
\verse{וישׁב And he shall bring עליהם upon את אונם them their own iniquity, וברעתם in their own wickedness; יצמיתם and shall cut them off יצמיתם shall cut them off. יהוה the LORD אלהינו׃ our God}%
\end{biblechapter}%
\begin{biblechapter}% Psalm 95
\verseWithHeading{A Call to Worship and Obey}{לכו O come, נרננה let us sing ליהוה unto the LORD: נריעה let us make a joyful noise לצור to the rock ישׁענו׃ of our salvation.}%
\verse{נקדמה Let us come before פניו his presence בתודה with thanksgiving, בזמרות unto him with psalms. נריע׃ and make a joyful noise}%
\verse{כי For אל God, גדול a great יהוה the LORD ומלך King גדול and a great על above כל all אלהים׃ gods.}%
\verse{אשׁר בידו In his hand מחקרי the deep places ארץ of the earth: ותועפות the strength הרים׃ of the hills}%
\verse{אשׁר לו הים The sea והוא his, and he עשׂהו made ויבשׁת the dry ידיו it: and his hands יצרו׃ formed}%
\verse{באו O come, נשׁתחוה let us worship ונכרעה and bow down: נברכה let us kneel לפני before יהוה the LORD עשׂנו׃ our maker.}%
\verse{כי For הוא he אלהינו our God; ואנחנו and we עם the people מרעיתו of his pasture, וצאן and the sheep ידו of his hand. היום Today אם if בקלו his voice, תשׁמעו׃ ye will hear}%
\verse{אל not תקשׁו Harden לבבכם your heart, כמריבה as in the provocation, כיום as the day מסה of temptation במדבר׃ in the wilderness:}%
\verse{אשׁר When נסוני tempted אבותיכם your fathers בחנוני me, proved גם me, and ראו saw פעלי׃ my work.}%
\verse{ארבעים Forty שׁנה years אקוט long was I grieved בדור with generation, ואמר and said, עם a people תעי that do err לבב in their heart, הם It והם and they לא have not ידעו known דרכי׃ my ways:}%
\verse{אשׁר Unto whom נשׁבעתי I swore באפי in my wrath אם that יבאון they should not enter אל into מנוחתי׃ my rest.}%
\end{biblechapter}%
\begin{biblechapter}% Psalm 96
\verseWithHeading{Adonai the King Comes in Judgment}{שׁירו O sing ליהוה unto the LORD שׁיר song: חדשׁ a new שׁירו sing ליהוה unto the LORD, כל all הארץ׃ the earth.}%
\verse{שׁירו Sing ליהוה unto the LORD, ברכו bless שׁמו his name; בשׂרו show forth מיום from day ליום to day. ישׁועתו׃ his salvation}%
\verse{ספרו Declare בגוים among the heathen, כבודו his glory בכל among all העמים people. נפלאותיו׃ his wonders}%
\verse{כי For גדול great, יהוה the LORD ומהלל to be praised: מאד and greatly נורא to be feared הוא he על above כל all אלהים׃ gods.}%
\verse{כי For כל all אלהי the gods העמים of the nations אלילים idols: ויהוה but the LORD שׁמים the heavens. עשׂה׃ made}%
\verse{הוד Honor והדר and majesty לפניו before עז him: strength ותפארת and beauty במקדשׁו׃ in his sanctuary.}%
\verse{הבו Give ליהוה unto the LORD, משׁפחות O ye kindreds עמים of the people, הבו give ליהוה unto the LORD כבוד glory ועז׃ and strength.}%
\verse{הבו Give ליהוה unto the LORD כבוד the glory שׁמו his name: שׂאו bring מנחה an offering, ובאו and come לחצרותיו׃ into his courts.}%
\verse{השׁתחוו O worship ליהוה the LORD בהדרת in the beauty קדשׁ of holiness: חילו fear מפניו before כל him, all הארץ׃ the earth.}%
\verse{אמרו Say בגוים among the heathen יהוה the LORD מלך reigneth: אף also תכון shall be established תבל the world בל that it shall not תמוט be moved: ידין he shall judge עמים the people במישׁרים׃ righteously.}%
\verse{ישׂמחו rejoice, השׁמים Let the heavens ותגל be glad; הארץ and let the earth ירעם roar, הים let the sea ומלאו׃ and the fullness}%
\verse{יעלז be joyful, שׂדי Let the field וכל and all אשׁר that בו אז therein: then ירננו rejoice כל shall all עצי the trees יער׃ of the wood}%
\verse{לפני Before יהוה the LORD: כי for בא he cometh, כי for בא he cometh לשׁפט to judge הארץ the earth: ישׁפט he shall judge תבל the world בצדק with righteousness, ועמים and the people באמונתו׃ with his truth.}%
\end{biblechapter}%
\begin{biblechapter}% Psalm 97
\verseWithHeading{Adonai’s Glorious Reign}{יהוה The LORD מלך reigneth; תגל rejoice; הארץ let the earth ישׂמחו be glad איים of isles רבים׃ let the multitude}%
\verse{ענן Clouds וערפל and darkness סביביו round about צדק him: righteousness ומשׁפט and judgment מכון the habitation כסאו׃ of his throne.}%
\verse{אשׁ A fire לפניו before תלך goeth ותלהט him, and burneth up סביב round about. צריו׃ his enemies}%
\verse{האירו enlightened ברקיו His lightnings תבל the world: ראתה saw, ותחל and trembled. הארץ׃ the earth}%
\verse{הרים The hills כדונג like wax נמסו melted מלפני at the presence יהוה of the LORD, מלפני at the presence אדון of the Lord כל of the whole הארץ׃ earth.}%
\verse{הגידו declare השׁמים The heavens צדקו his righteousness, וראו see כל and all העמים the people כבודו׃ his glory.}%
\verse{יבשׁו Confounded כל be all עבדי they that serve פסל graven images, המתהללים that boast themselves באלילים of idols: השׁתחוו worship לו כל him, all אלהים׃ gods.}%
\verse{שׁמעה heard, ותשׂמח and was glad; ציון Zion ותגלנה rejoiced בנות and the daughters יהודה of Judah למען because of משׁפטיך thy judgments, יהוה׃ O LORD.}%
\verse{כי For אתה thou, יהוה LORD, עליון high על above כל all הארץ the earth: מאד far נעלית thou art exalted על above כל all אלהים׃ gods.}%
\verse{אהבי Ye that love יהוה the LORD, שׂנאו hate רע evil: שׁמר he preserveth נפשׁות the souls חסידיו of his saints; מיד them out of the hand רשׁעים of the wicked. יצילם׃ he delivereth}%
\verse{אור Light זרע is sown לצדיק for the righteous, ולישׁרי for the upright לב in heart. שׂמחה׃ and gladness}%
\verse{שׂמחו Rejoice צדיקים ye righteous; ביהוה in the LORD, והודו and give thanks לזכר at the remembrance קדשׁו׃ of his holiness.}%
\end{biblechapter}%
\begin{biblechapter}% Psalm 98
\verseWithHeading{Praise to Adonai for His Salvation and Judgment}{מזמור A Psalm. שׁירו O sing ליהוה unto the LORD שׁיר song; חדשׁ a new כי for נפלאות marvelous things: עשׂה he hath done הושׁיעה hath gotten him the victory. לו ימינו his right hand, וזרוע arm, קדשׁו׃ and his holy}%
\verse{הודיע hath made known יהוה The LORD ישׁועתו his salvation: לעיני in the sight הגוים of the heathen. גלה hath he openly showed צדקתו׃ his righteousness}%
\verse{זכר He hath remembered חסדו his mercy ואמונתו and his truth לבית toward the house ישׂראל of Israel: ראו have seen כל all אפסי the ends ארץ of the earth את ישׁועת the salvation אלהינו׃ of our God.}%
\verse{הריעו Make a joyful noise ליהוה unto the LORD, כל all הארץ the earth: פצחו make a loud noise, ורננו and rejoice, וזמרו׃ and sing praise.}%
\verse{זמרו Sing ליהוה unto the LORD בכנור with the harp; בכנור with the harp, וקול and the voice זמרה׃ of a psalm.}%
\verse{בחצצרות With trumpets וקול and sound שׁופר of cornet הריעו make a joyful noise לפני before המלך the King. יהוה׃ the LORD,}%
\verse{ירעם roar, הים Let the sea ומלאו and the fullness תבל thereof; the world, וישׁבי׃ and they that dwell}%
\verse{נהרות Let the floods ימחאו clap כף hands: יחד together הרים let the hills ירננו׃ be joyful}%
\verse{לפני Before יהוה the LORD; כי for בא he cometh לשׁפט to judge הארץ the earth: ישׁפט shall he judge תבל the world, בצדק with righteousness ועמים and the people במישׁרים׃ with equity.}%
\end{biblechapter}%
\begin{biblechapter}% Psalm 99
\verseWithHeading{Adonai Is a Holy King}{יהוה The LORD מלך reigneth; ירגזו tremble: עמים let the people ישׁב he sitteth כרובים the cherubims; תנוט be moved. הארץ׃ let the earth}%
\verse{יהוה The LORD בציון in Zion; גדול great ורם high הוא and he על above כל all העמים׃ the people.}%
\verse{יודו Let them praise שׁמך name; גדול thy great ונורא and terrible קדושׁ holy. הוא׃ it}%
\verse{ועז strength מלך The king's משׁפט judgment; אהב also loveth אתה thou כוננת dost establish מישׁרים equity, משׁפט judgment וצדקה and righteousness ביעקב in Jacob. אתה thou עשׂית׃ executest}%
\verse{רוממו Exalt יהוה ye the LORD אלהינו our God, והשׁתחוו and worship להדם at his footstool; רגליו at his footstool; קדושׁ holy. הוא׃ he}%
\verse{משׁה Moses ואהרן and Aaron בכהניו among his priests, ושׁמואל and Samuel בקראי among them that call upon שׁמו his name; קראים they called אל upon יהוה the LORD, והוא and he יענם׃ answered}%
\verse{בעמוד pillar: ענן them in the cloudy ידבר He spoke אליהם unto שׁמרו they kept עדתיו his testimonies, וחק and the ordinance נתן׃ he gave}%
\verse{יהוה אלהינו our God: אתה thou עניתם Thou answeredst אל a God נשׂא that forgavest היית wast להם ונקם them, though thou tookest vengeance על of עלילותם׃ their inventions.}%
\verse{רוממו Exalt יהוה the LORD אלהינו our God, והשׁתחוו and worship להר hill; קדשׁו at his holy כי for קדושׁ holy. יהוה the LORD אלהינו׃ our God}%
\end{biblechapter}%
\begin{biblechapter}% Psalm 100
\verseWithHeading{Worship God with Joy}{מזמור A Psalm לתודה of praise. הריעו Make a joyful noise ליהוה unto the LORD, כל all הארץ׃ ye lands.}%
\verse{עבדו Serve את יהוה the LORD בשׂמחה with gladness: באו come לפניו before his presence ברננה׃ with singing.}%
\verse{דעו Know כי ye that יהוה the LORD הוא he אלהים God: הוא he עשׂנו hath made ולא us, and not אנחנו we ourselves; עמו his people, וצאן and the sheep מרעיתו׃ of his pasture.}%
\verse{באו Enter שׁעריו into his gates בתודה with thanksgiving, חצרתיו into his courts בתהלה with praise: הודו be thankful לו ברכו unto him, bless שׁמו׃ his name.}%
\verse{כי For טוב good; יהוה the LORD לעולם everlasting; חסדו his mercy ועד to דר all generations. ודר all generations. אמונתו׃ and his truth}%
\end{biblechapter}%
\begin{biblechapter}% Psalm 101
\verseWithHeading{A Promise to Act with Integrity}{לדוד of David. מזמור A Psalm חסד of mercy ומשׁפט and judgment: אשׁירה I will sing לך יהוה unto thee, O LORD, אזמרה׃ will I sing.}%
\verse{אשׂכילה I will behave myself wisely בדרך way. תמים in a perfect מתי O when תבוא wilt thou come אלי unto אתהלך me? I will walk בתם with a perfect לבבי heart. בקרב within ביתי׃ my house}%
\verse{לא no אשׁית I will set לנגד before עיני mine eyes: דבר thing בליעל wicked עשׂה the work סטים of them that turn aside; שׂנאתי I hate לא shall not ידבק׃ cleave}%
\verse{לבב heart עקשׁ A froward יסור shall depart ממני from רע a wicked לא me: I will not אדע׃ know}%
\verse{מלושׁני slandereth בסתר Whoso privily רעהו his neighbor, אותו אצמית him will I cut off: גבה him that hath a high עינים look ורחב and a proud לבב heart אתו לא will not אוכל׃ I suffer.}%
\verse{עיני Mine eyes בנאמני upon the faithful ארץ of the land, לשׁבת that they may dwell עמדי with הלך me: he that walketh בדרך way, תמים in a perfect הוא he ישׁרתני׃ shall serve}%
\verse{לא shall not ישׁב dwell בקרב within ביתי my house: עשׂה He that worketh רמיה deceit דבר he that telleth שׁקרים lies לא shall not יכון tarry לנגד in עיני׃ my sight.}%
\verse{לבקרים I will early אצמית destroy כל all רשׁעי the wicked ארץ of the land; להכרית that I may cut off מעיר from the city יהוה of the LORD. כל all פעלי doers און׃ wicked}%
\end{biblechapter}%
\begin{biblechapter}% Psalm 102
\verseWithHeading{A Plea for Personal and National Help}{תפלה A Prayer לעני of the afflicted, כי when יעטף he is overwhelmed, ולפני before יהוה the LORD. ישׁפך and poureth out שׂיחו׃ his complaint יהוה O LORD, שׁמעה Hear תפלתי my prayer, ושׁועתי and let my cry אליך unto תבוא׃ come}%
\verse{אל not תסתר Hide פניך thy face ממני from ביום me in the day צר I am in trouble; לי הטה incline אלי unto אזנך thine ear ביום me: in the day אקרא I call מהר ענני׃ answer}%
\verse{כי For כלו are consumed בעשׁן like smoke, ימי my days ועצמותי and my bones כמו a hearth. קד נחרו׃ are burned}%
\verse{הוכה is smitten, כעשׂב like grass; ויבשׁ and withered לבי My heart כי so שׁכחתי that I forget מאכל to eat לחמי׃ my bread.}%
\verse{מקול אנחתי of my groaning דבקה cleave עצמי my bones לבשׂרי׃ to my skin.}%
\verse{דמיתי I am like לקאת a pelican מדבר of the wilderness: הייתי I am ככוס like an owl חרבות׃ of the desert.}%
\verse{שׁקדתי I watch, ואהיה and am כצפור as a sparrow בודד alone על upon גג׃ the house top.}%
\verse{כל me all היום the day; חרפוני reproach אויבי Mine enemies מהוללי they that are mad בי נשׁבעו׃ against me are sworn}%
\verse{כי For אפר ashes כלחם like bread, אכלתי I have eaten ושׁקוי my drink בבכי with weeping, מסכתי׃ and mingled}%
\verse{מפני זעמך of thine indignation וקצפך and thy wrath: כי for נשׂאתני thou hast lifted me up, ותשׁליכני׃ and cast me down.}%
\verse{ימי My days כצל like a shadow נטוי that declineth; ואני and I כעשׂב like grass. איבשׁ׃ am withered}%
\verse{ואתה But thou, יהוה O LORD, לעולם forever; תשׁב shalt endure וזכרך and thy remembrance לדר unto all generations. ודר׃ unto all generations.}%
\verse{אתה Thou תקום shalt arise, תרחם have mercy upon ציון Zion: כי for עת the time לחננה to favor כי her, yea, בא is come. מועד׃ the set time,}%
\verse{כי For רצו take pleasure in עבדיך thy servants את אבניה her stones, ואת עפרה the dust יחננו׃ and favor}%
\verse{וייראו shall fear גוים So the heathen את שׁם the name יהוה of the LORD, וכל and all מלכי the kings הארץ of the earth את כבודך׃ thy glory.}%
\verse{כי When בנה shall build up יהוה the LORD ציון Zion, נראה he shall appear בכבודו׃ in his glory.}%
\verse{פנה אלפלת the prayer הערער of the destitute, ולא and not בזה despise את תפלתם׃ their prayer.}%
\verse{תכתב shall be written זאת This לדור for the generation אחרון to come: ועם and the people נברא which shall be created יהלל shall praise יה׃ the LORD.}%
\verse{כי For השׁקיף he hath looked down ממרום from the height קדשׁו of his sanctuary; יהוה did the LORD משׁמים from heaven אל behold ארץ the earth; הביט׃ behold}%
\verse{לשׁמע To hear אנקת the groaning אסיר of the prisoner; לפתח to loose בני those that are appointed תמותה׃ to death;}%
\verse{לספר To declare בציון in Zion, שׁם the name יהוה of the LORD ותהלתו and his praise בירושׁלם׃ in Jerusalem;}%
\verse{בהקבץ are gathered עמים When the people יחדו together, וממלכות and the kingdoms, לעבד to serve את יהוה׃ the LORD.}%
\verse{ענה He weakened בדרך in the way; כחו my strength קצר he shortened ימי׃ my days.}%
\verse{אמר I said, אלי O my God, אל me not תעלני take בחצי away in the midst ימי of my days: בדור throughout all generations. דורים throughout all generations. שׁנותיך׃ thy years}%
\verse{לפנים Of old הארץ of the earth: יסדת hast thou laid the foundation ומעשׂה the work ידיך of thy hands. שׁמים׃ and the heavens}%
\verse{המה They יאבדו shall perish, ואתה but thou תעמד shalt endure: וכלם yea, all כבגד like a garment; יבלו of them shall wax old כלבושׁ as a vesture תחליפם shalt thou change ויחלפו׃ them, and they shall be changed:}%
\verse{ואתה But thou הוא the same, ושׁנותיך and thy years לא shall have no יתמו׃ end.}%
\verse{בני The children עבדיך of thy servants ישׁכונו shall continue, וזרעם and their seed לפניך before יכון׃ shall be established}%
\end{biblechapter}%
\begin{biblechapter}% Psalm 103
\verseWithHeading{Thanksgiving for Adonai’s Compassion}{לדוד of David. ברכי Bless נפשׁי O my soul: את יהוה the LORD, וכל and all קרבי that is within את שׁם name. קדשׁו׃ me, his holy}%
\verse{ברכי Bless נפשׁי O my soul, את יהוה the LORD, ואל not תשׁכחי and forget כל all גמוליו׃ his benefits:}%
\verse{הסלח Who forgiveth לכל all עונכי thine iniquities; הרפא who healeth לכל all תחלאיכי׃ thy diseases;}%
\verse{הגואל Who redeemeth משׁחת from destruction; חייכי thy life המעטרכי who crowneth חסד thee with lovingkindness ורחמים׃ and tender mercies;}%
\verse{המשׂביע Who satisfieth בטוב with good עדיך thy mouth תתחדשׁ is renewed כנשׁר like the eagle's. נעוריכי׃ thy youth}%
\verse{עשׂה executeth צדקות righteousness יהוה The LORD ומשׁפטים and judgment לכל for all עשׁוקים׃ that are oppressed.}%
\verse{יודיע He made known דרכיו his ways למשׁה unto Moses, לבני unto the children ישׂראל of Israel. עלילותיו׃ his acts}%
\verse{רחום merciful וחנון and gracious, יהוה The LORD ארך slow אפים to anger, ורב and plenteous חסד׃ in mercy.}%
\verse{לא He will not לנצח always יריב chide: ולא neither לעולם forever. יטור׃ will he keep}%
\verse{לא He hath not כחטאינו with us after our sins; עשׂה dealt לנו ולא nor כעונתינו us according to our iniquities. גמל rewarded עלינו׃ rewarded}%
\verse{כי For כגבה is high שׁמים as the heaven על above הארץ the earth, גבר great חסדו is his mercy על toward יראיו׃ them that fear}%
\verse{כרחק far hath he removed מזרח as the east ממערב is from the west, הרחיק ממנו is from the west, את פשׁעינו׃ our transgressions}%
\verse{כרחם pitieth אב Like as a father על pitieth בנים children, רחם pitieth יהוה the LORD על pitieth יראיו׃ them that fear}%
\verse{כי For הוא he ידע knoweth יצרנו our frame; זכור he remembereth כי that עפר dust. אנחנו׃ we}%
\verse{אנושׁ כחציר as grass: ימיו his days כציץ as a flower השׂדה of the field, כן so יציץ׃ he flourisheth.}%
\verse{כי For רוח the wind עברה passeth בו ואיננו over it, and it is gone; ולא it no יכירנו thereof shall know עוד more. מקומו׃ and the place}%
\verse{וחסד But the mercy יהוה of the LORD מעולם from everlasting ועד to עולם everlasting על upon יראיו them that fear וצדקתו him, and his righteousness לבני unto children's בנים׃ children;}%
\verse{לשׁמרי To such as keep בריתו his covenant, ולזכרי and to those that remember פקדיו his commandments לעשׂותם׃ to do}%
\verse{יהוה The LORD בשׁמים in the heavens; הכין hath prepared כסאו his throne ומלכותו and his kingdom בכל over all. משׁלה׃ ruleth}%
\verse{ברכו Bless יהוה the LORD, מלאכיו ye his angels, גברי that excel כח in strength, עשׂי that do דברו his commandments, לשׁמע hearkening בקול unto the voice דברו׃ of his word.}%
\verse{ברכו Bless יהוה ye the LORD, כל all צבאיו his hosts; משׁרתיו ministers עשׂי of his, that do רצונו׃ his pleasure.}%
\verse{ברכו Bless יהוה the LORD, כל all מעשׂיו his works בכל in all מקמות places ממשׁלתו of his dominion: ברכי bless נפשׁי O my soul. את יהוה׃ the LORD,}%
\end{biblechapter}%
\begin{biblechapter}% Psalm 104
\verseWithHeading{Praise to Adonai for His Creation and Providence}{ברכי Bless נפשׁי O my soul. את יהוה the LORD, יהוה O LORD אלהי my God, גדלת great; מאד thou art very הוד with honor והדר and majesty. לבשׁת׃ thou art clothed}%
\verse{עטה Who coverest אור with light כשׂלמה as a garment: נוטה who stretchest out שׁמים the heavens כיריעה׃ like a curtain:}%
\verse{המקרה Who layeth the beams במים in the waters: עליותיו of his chambers השׂם who maketh עבים the clouds רכובו his chariot: המהלך who walketh על upon כנפי the wings רוח׃ of the wind:}%
\verse{עשׂה Who maketh מלאכיו his angels רוחות spirits; משׁרתיו his ministers אשׁ fire: להט׃ a flaming}%
\verse{יסד ארץ of the earth, על מכוניה the foundations בל it should not תמוט be removed עולם forever. ועד׃ forever.}%
\verse{תהום it with the deep כלבושׁ as a garment: כסיתו Thou coveredst על above הרים the mountains. יעמדו stood מים׃ the waters}%
\verse{מן At גערתך thy rebuke ינוסון they fled; מן at קול the voice רעמך of thy thunder יחפזון׃ they hasted away.}%
\verse{יעלו They go up הרים by the mountains; ירדו they go down בקעות by the valleys אל unto מקום the place זה which יסדת thou hast founded להם׃}%
\verse{גבול a bound שׂמת Thou hast set בל that they may not יעברון pass over; בל that they turn not again ישׁובון that they turn not again לכסות to cover הארץ׃ the earth.}%
\verse{המשׁלח He sendeth מעינים the springs בנחלים into the valleys, בין among הרים the hills. יהלכון׃ run}%
\verse{ישׁקו They give drink כל to every חיתו beast שׂדי of the field: ישׁברו quench פראים the wild asses צמאם׃ their thirst.}%
\verse{עליהם By עוף them shall the fowls השׁמים of the heaven ישׁכון have their habitation, מבין among עפאים the branches. יתנו sing קול׃ sing}%
\verse{משׁקה He watereth הרים the hills מעליותיו from his chambers: מפרי with the fruit מעשׂיך of thy works. תשׂבע is satisfied הארץ׃ the earth}%
\verse{מצמיח to grow חציר He causeth the grass לבהמה for the cattle, ועשׂב and herb לעבדת for the service האדם of man: להוציא that he may bring forth לחם food מן out of הארץ׃ the earth;}%
\verse{ויין And wine ישׂמח maketh glad לבב the heart אנושׁ of man, להצהיל to shine, פנים to make face משׁמן oil ולחם and bread לבב heart. אנושׁ man's יסעד׃ strengtheneth}%
\verse{ישׂבעו are full עצי The trees יהוה of the LORD ארזי the cedars לבנון of Lebanon, אשׁר which נטע׃ he hath planted;}%
\verse{אשׁר שׁםפרים the birds יקננו make their nests: חסידה the stork, ברושׁים the fir trees ביתה׃ her house.}%
\verse{הרים hills הגבהים The high ליעלים for the wild goats; סלעים the rocks מחסה a refuge לשׁפנים׃ for the conies.}%
\verse{עשׂה He appointed ירח the moon למועדים for seasons: שׁמשׁ the sun ידע knoweth מבואו׃ his going down.}%
\verse{תשׁת Thou makest חשׁך darkness, ויהי and it is לילה night: בו תרמשׂ do creep כל wherein all חיתו the beasts יער׃ of the forest}%
\verse{הכפירים The young lions שׁאגים roar לטרף after their prey, ולבקשׁ and seek מאל אכלם׃ their meat}%
\verse{תזרח ariseth, השׁמשׁ The sun יאספון they gather themselves together, ואל in מעונתם their dens. ירבצון׃ and lay them down}%
\verse{יצא goeth forth אדם Man לפעלו unto his work ולעבדתו and to his labor עדי ערב׃ the evening.}%
\verse{מה how רבו manifold מעשׂיך are thy works! יהוה O LORD, כלם them all: בחכמה in wisdom עשׂית hast thou made מלאה is full הארץ the earth קנינך׃ of thy riches.}%
\verse{זה this הים sea, גדול great ורחב and wide ידים and wide שׁם wherein רמשׂ things creeping ואין innumerable, מספר innumerable, חיות beasts. קטנות both small עם and גדלות׃ great}%
\verse{שׁם There אניות the ships: יהלכון go לויתן that leviathan, זה thou יצרת hast made לשׂחק׃ to play}%
\verse{כלם all אליך upon ישׂברון These wait לתת thee; that thou mayest give אכלם their meat בעתו׃ in due season.}%
\verse{תתן thou givest להם ילקטון them they gather: תפתח thou openest ידך thine hand, ישׂבעון they are filled טוב׃ with good.}%
\verse{תסתיר Thou hidest פניך thy face, יבהלון they are troubled: תסף thou takest away רוחם their breath, יגועון they die, ואל to עפרם their dust. ישׁובון׃ and return}%
\verse{תשׁלח Thou sendest forth רוחך thy spirit, יבראון they are created: ותחדשׁ and thou renewest פני the face אדמה׃ of the earth.}%
\verse{יהי shall endure כבוד The glory יהוה of the LORD לעולם forever: ישׂמח shall rejoice יהוה the LORD במעשׂיו׃ in his works.}%
\verse{המביט He looketh לארץ on the earth, ותרעד and it trembleth: יגע he toucheth בהרים the hills, ויעשׁנו׃ and they smoke.}%
\verse{אשׁירה I will sing ליהוה unto the LORD בחיי as long as I live: אזמרה I will sing praise לאלהי to my God בעודי׃ while I have my being.}%
\verse{יערב him shall be sweet: עליו of שׂיחי My meditation אנכי I אשׂמח will be glad ביהוה׃ in the LORD.}%
\verse{יתמו be consumed חטאים Let the sinners מן out of הארץ the earth, ורשׁעים and let the wicked עוד more. אינם be no ברכי Bless נפשׁי O my soul. את יהוה thou the LORD, הללו Praise יה׃ ye the LORD.}%
\end{biblechapter}%
\begin{biblechapter}% Psalm 105
\verseWithHeading{Praise to Adonai for His Work on Behalf of Israel}{הודו O give thanks ליהוה unto the LORD; קראו call בשׁמו upon his name: הודיעו make known בעמים among the people. עלילותיו׃ his deeds}%
\verse{שׁירו Sing לו זמרו unto him, sing psalms לו שׂיחו unto him: talk בכל ye of all נפלאותיו׃ his wondrous works.}%
\verse{התהללו Glory בשׁם name: קדשׁו ye in his holy ישׂמח of them rejoice לב let the heart מבקשׁי that seek יהוה׃ the LORD.}%
\verse{דרשׁו Seek יהוה the LORD, ועזו and his strength: בקשׁו seek פניו his face תמיד׃ evermore.}%
\verse{זכרו Remember נפלאותיו his marvelous works אשׁר that עשׂה he hath done; מפתיו his wonders, ומשׁפטי and the judgments פיו׃ of his mouth;}%
\verse{זרע O ye seed אברהם of Abraham עבדו his servant, בני ye children יעקב of Jacob בחיריו׃ his chosen.}%
\verse{הוא He יהוה the LORD אלהינו our God: בכל in all הארץ the earth. משׁפטיו׃ his judgments}%
\verse{זכר He hath remembered לעולם forever, בריתו his covenant דבר the word צוה he commanded לאלף to a thousand דור׃ generations.}%
\verse{אשׁר Which כרת he made את with אברהם Abraham, ושׁבועתו and his oath לישׂחק׃ unto Isaac;}%
\verse{ויעמידה And confirmed ליעקב the same unto Jacob לחק for a law, לישׂראל to Israel ברית covenant: עולם׃ an everlasting}%
\verse{לאמר Saying, לך אתן Unto thee will I give את ארץ the land כנען of Canaan, חבל the lot נחלתכם׃ of your inheritance:}%
\verse{בהיותם When they were מתי a few men מספר in number; כמעט yea, very few, וגרים׃ and strangers}%
\verse{ויתהלכו When they went מגוי from one nation אל to גוי another, מממלכה from kingdom אל to עם people; אחר׃ another}%
\verse{לא no הניח אדם man לעשׁקם to do them wrong: ויוכח yea, he reproved עליהם for their sakes; מלכים׃ kings}%
\verse{אל not תגעו Touch במשׁיחי mine anointed, ולנביאי and do my prophets אל no תרעו׃ harm.}%
\verse{ויקרא Moreover he called for רעב a famine על upon הארץ the land: כל the whole מטה staff לחם of bread. שׁבר׃ he broke}%
\verse{שׁלח He sent לפניהם before אישׁ a man לעבד for a servant: נמכר was sold יוסף׃ them, Joseph,}%
\verse{ענו they hurt בכבל with fetters: רגליו Whose feet ברזל in iron: באה was laid נפשׁו׃ he}%
\verse{עד Until עת the time בא came: דברו that his word אמרת the word יהוה of the LORD צרפתהו׃ tried}%
\verse{שׁלח sent מלך The king ויתירהו and loosed משׁל him; the ruler עמים of the people, ויפתחהו׃ and let him go free.}%
\verse{שׂמו He made אדון him lord לביתו of his house, ומשׁל and ruler בכל of all קנינו׃ his substance:}%
\verse{לאסר To bind שׂריו his princes בנפשׁו at his pleasure; וזקניו and teach his senators יחכם׃ wisdom.}%
\verse{ויבא also came into ישׂראל Israel מצרים Egypt; ויעקב and Jacob גר sojourned בארץ in the land חם׃ of Ham.}%
\verse{ויפר And he increased את עמו his people מאד greatly; ויעצמהו and made them stronger מצריו׃ than their enemies.}%
\verse{הפך He turned לבם their heart לשׂנא to hate עמו his people, להתנכל to deal subtlely בעבדיו׃ with his servants.}%
\verse{שׁלח He sent משׁה Moses עבדו his servant; אהרן Aaron אשׁר whom בחר׃ he had chosen.}%
\verse{שׂמו They showed בם דברי his signs אתותיו his signs ומפתים among them, and wonders בארץ in the land חם׃ of Ham.}%
\verse{שׁלח He sent חשׁך darkness, ויחשׁך and made it dark; ולא not מרו and they rebelled את דברוו׃ against his word.}%
\verse{הפך He turned את מימיהם their waters לדם into blood, וימת and slew את דגתם׃ their fish.}%
\verse{שׁרץ brought forth frogs in abundance, ארצם Their land צפרדעים brought forth frogs in abundance, בחדרי in the chambers מלכיהם׃ of their kings.}%
\verse{אמר He spoke, ויבא and there came ערב divers sorts of flies, כנים lice בכל in all גבולם׃ their coasts.}%
\verse{נתן He gave גשׁמיהם for rain, ברד them hail אשׁ fire להבות flaming בארצם׃ in their land.}%
\verse{ויך He smote גפנם their vines ותאנתם also and their fig trees; וישׁבר and broke עץ the trees גבולם׃ of their coasts.}%
\verse{אמר He spoke, ויבא came, ארבה and the locusts וילק and caterpillars, ואין and that without מספר׃ number,}%
\verse{ויאכל And did eat up כל all עשׂב the herbs בארצם in their land, ויאכל and devoured פרי the fruit אדמתם׃ of their ground.}%
\verse{ויך He smote כל also all בכור the firstborn בארצם in their land, ראשׁית the chief לכל of all אונם׃ their strength.}%
\verse{ויוציאם He brought them forth בכסף also with silver וזהב and gold: ואין and not בשׁבטיו among their tribes. כושׁל׃ one feeble}%
\verse{שׂמח was glad מצרים Egypt בצאתם when they departed: כי for נפל of them fell פחדם the fear עליהם׃ upon}%
\verse{פרשׂ He spread ענן a cloud למסך for a covering; ואשׁ and fire להאיר to give light לילה׃ in the night.}%
\verse{שׁאל asked, ויבא and he brought שׂלו quails, ולחם them with the bread שׁמים of heaven. ישׂביעם׃ and satisfied}%
\verse{פתח He opened צור the rock, ויזובו gushed out; מים and the waters הלכו they ran בציות in the dry places נהר׃ a river.}%
\verse{כי For זכר he remembered את דבר promise, קדשׁו his holy את אברהם Abraham עבדו׃ his servant.}%
\verse{ויוצא And he brought forth עמו his people בשׂשׂון with joy, ברנה with gladness: את בחיריו׃ his chosen}%
\verse{ויתן And gave להם ארצות them the lands גוים of the heathen: ועמל the labor לאמים of the people; יירשׁו׃ and they inherited}%
\verse{בעבור That ישׁמרו they might observe חקיו his statutes, ותורתיו his laws. ינצרו and keep הללו Praise יה׃ ye the LORD.}%
\end{biblechapter}%
\begin{biblechapter}% Psalm 106
\verseWithHeading{Praise to Adonai for His Faithfulness in Israel’s History}{הללויה Praise הודו O give thanks ליהוה unto the LORD; כי for טוב good: כי for לעולם forever. חסדו׃ his mercy}%
\verse{מי Who ימלל can utter גבורות the mighty acts יהוה of the LORD? ישׁמיע can show forth כל all תהלתו׃ his praise?}%
\verse{אשׁרי Blessed שׁמרי they that keep משׁפט judgment, עשׂה he that doeth צדקה righteousness בכל at all עת׃ times.}%
\verse{זכרני Remember יהוה me, O LORD, ברצון with the favor עמך thy people: פקדני O visit בישׁועתך׃ me with thy salvation;}%
\verse{לראות That I may see בטובת the good בחיריך of thy chosen, לשׂמח that I may rejoice בשׂמחת in the gladness גויך of thy nation, להתהלל that I may glory עם with נחלתך׃ thine inheritance.}%
\verse{חטאנו We have sinned עם with אבותינו our fathers, העוינו we have committed iniquity, הרשׁענו׃ we have done wickedly.}%
\verse{אבותינו Our fathers במצרים in Egypt; לא not השׂכילו understood נפלאותיך thy wonders לא not זכרו they remembered את רב the multitude חסדיך of thy mercies; וימרו but provoked על at ים the sea, בים sea. סוף׃ at the Red}%
\verse{ויושׁיעם Nevertheless he saved למען שׁמוהודיע to be known. את גבורתו׃ that he might make his mighty power}%
\verse{ויגער He rebuked בים sea סוף the Red ויחרב also, and it was dried up: ויוליכם so he led בתהמות them through the depths, כמדבר׃ as through the wilderness.}%
\verse{ויושׁיעם And he saved מיד them from the hand שׂונא of him that hated ויגאלם and redeemed מיד them from the hand אויב׃ of the enemy.}%
\verse{ויכסו covered מים And the waters צריהם their enemies: אחד one מהם לא there was not נותר׃ them left.}%
\verse{ויאמינו Then believed בדבריו they his words; ישׁירו they sang תהלתו׃ his praise.}%
\verse{מהרו They soon שׁכחו forgot מעשׂיו his works; לא not חכו they waited לעצתו׃ for his counsel:}%
\verse{ויתאוו תאוהמדבר in the wilderness, וינסו and tempted אל God בישׁימון׃ in the desert.}%
\verse{ויתן And he gave להם שׁאלתם them their request; וישׁלח but sent רזון leanness בנפשׁם׃ into their soul.}%
\verse{ויקנאו They envied למשׁה Moses במחנה also in the camp, לאהרן Aaron קדושׁ the saint יהוה׃ of the LORD.}%
\verse{תפתח opened ארץ The earth ותבלע and swallowed up דתן Dathan, ותכס and covered על and covered עדת the company אבירם׃ of Abiram.}%
\verse{ותבער was kindled אשׁ And a fire בעדתם in their company; להבה the flame תלהט burned up רשׁעים׃ the wicked.}%
\verse{יעשׂו They made עגל a calf בחרב in Horeb, וישׁתחוו and worshiped למסכה׃ the molten image.}%
\verse{וימירו Thus they changed את כבודם their glory בתבנית into the similitude שׁור of an ox אכל that eateth עשׂב׃ grass.}%
\verse{שׁכחו They forgot אל God מושׁיעם their savior, עשׂה which had done גדלות great things במצרים׃ in Egypt;}%
\verse{נפלאות Wondrous works בארץ in the land חם of Ham, נוראות terrible things על by ים sea. סוף׃ the Red}%
\verse{ויאמר Therefore he said להשׁמידם that he would destroy לולי them, had not משׁה Moses בחירו his chosen עמד stood בפרץ him in the breach, לפניו before להשׁיב to turn away חמתו his wrath, מהשׁחית׃ lest he should destroy}%
\verse{וימאסו Yea, they despised בארץ land, חמדה the pleasant לא not האמינו they believed לדברו׃ his word:}%
\verse{וירגנו But murmured באהליהם in their tents, לא not שׁמעו hearkened בקול unto the voice יהוה׃ of the LORD.}%
\verse{וישׂא Therefore he lifted up ידו his hand להם להפיל against them, to overthrow אותם במדבר׃ them in the wilderness:}%
\verse{ולהפיל To overthrow זרעם their seed בגוים also among the nations, ולזרותם and to scatter בארצות׃ them in the lands.}%
\verse{ויצמדו They joined themselves לבעל פעור also unto Baalpeor, ויאכלו and ate זבחי the sacrifices מתים׃ of the dead.}%
\verse{ויכעיסו Thus they provoked to anger במעלליהם with their inventions: ותפרץ broke in בם מגפה׃ and the plague}%
\verse{ויעמד Then stood up פינחס Phinehas, ויפלל and executed judgment: ותעצר was stayed. המגפה׃ and the plague}%
\verse{ותחשׁב And that was counted לו לצדקה unto him for righteousness לדר unto all generations ודר unto all generations עד forevermore. עולם׃ forevermore.}%
\verse{ויקציפו They angered על also at מי the waters מריבה of strife, וירע so that it went ill למשׁה with Moses בעבורם׃ for their sakes:}%
\verse{כי Because המרו they provoked את רוחו his spirit, ויבטא so that he spoke unadvisedly בשׂפתיו׃ with his lips.}%
\verse{לא They did not השׁמידו destroy את העמים the nations, אשׁר concerning whom אמר commanded יהוה להם׃}%
\verse{ויתערבו But were mingled בגוים among the heathen, וילמדו and learned מעשׂיהם׃ their works.}%
\verse{ויעבדו And they served את עצביהם their idols: ויהיו which were להם למוקשׁ׃ a snare}%
\verse{ויזבחו Yea, they sacrificed את בניהם their sons ואת בנותיהם and their daughters לשׁדים׃ unto devils,}%
\verse{וישׁפכו And shed דם blood, נקי innocent דם the blood בניהם of their sons ובנותיהם and of their daughters, אשׁר whom זבחו they sacrificed לעצבי unto the idols כנען of Canaan: ותחנף was polluted הארץ and the land בדמים׃ with blood.}%
\verse{ויטמאו Thus were they defiled במעשׂיהם with their own works, ויזנו and went a whoring במעלליהם׃ with their own inventions.}%
\verse{ויחר kindled אף Therefore was the wrath יהוה of the LORD בעמו against his people, ויתעב insomuch that he abhorred את נחלתו׃ his own inheritance.}%
\verse{ויתנם And he gave ביד them into the hand גוים of the heathen; וימשׁלו them ruled בהם שׂנאיהם׃ and they that hated}%
\verse{וילחצום also oppressed אויביהם Their enemies ויכנעו them, and they were brought into subjection תחת under ידם׃ their hand.}%
\verse{פעמים times רבות Many יצילם did he deliver והמה them; but they ימרו provoked בעצתם with their counsel, וימכו and were brought low בעונם׃ for their iniquity.}%
\verse{וירא Nevertheless he regarded בצר their affliction, להם בשׁמעו when he heard את רנתם׃ their cry:}%
\verse{ויזכר And he remembered להם בריתו for them his covenant, וינחם and repented כרב according to the multitude חסדו׃ of his mercies.}%
\verse{ויתן He made אותם לרחמים them also to be pitied לפני of כל all שׁוביהם׃ those that carried them captives.}%
\verse{הושׁיענו Save יהוה us, O LORD אלהינו our God, וקבצנו and gather מן us from among הגוים the heathen, להדות to give thanks לשׁם name, קדשׁך unto thy holy להשׁתבח to triumph בתהלתך׃ in thy praise.}%
\verse{ברוך Blessed יהוה the LORD אלהי God ישׂראל of Israel מן from העולם everlasting ועד to העולם everlasting: ואמר say, כל and let all העם the people אמן Amen. הללו Praise יה׃ ye the LORD.}%
\end{biblechapter}%
\begin{biblechapter}% Psalm 107
\verseWithHeading{Thanksgiving to Adonai for His of Deliverance}{הדו O give thanks ליהוה unto the LORD, כי for טוב good: כי for לעולם forever. חסדו׃ his mercy}%
\verse{יאמרו say גאולי Let the redeemed יהוה of the LORD אשׁר whom גאלם he hath redeemed מיד from the hand צר׃ of the enemy;}%
\verse{ומארצות them out of the lands, קבצם And gathered ממזרח from the east, וממערב and from the west, מצפון from the north, ומים׃ and from the south.}%
\verse{תעו They wandered במדבר in the wilderness בישׁימון in a solitary דרך way; עיר city מושׁב to dwell לא no מצאו׃ they found}%
\verse{רעבים Hungry גם and צמאים thirsty, נפשׁם their soul בהם תתעטף׃ fainted}%
\verse{ויצעקו Then they cried אל unto יהוה בצר in their trouble, להם ממצוקותיהם them out of their distresses. יצילם׃ he delivered}%
\verse{וידריכם And he led them forth בדרך way, ישׁרה by the right ללכת that they might go אל to עיר a city מושׁב׃ of habitation.}%
\verse{יודו Oh that would praise ליהוה the LORD חסדו his goodness, ונפלאותיו and his wonderful works לבני to the children אדם׃ of men!}%
\verse{כי For השׂביע he satisfieth נפשׁ soul, שׁקקה the longing ונפשׁ soul רעבה the hungry מלא and filleth טוב׃ with goodness.}%
\verse{ישׁבי Such as sit חשׁך in darkness וצלמות and in the shadow of death, אסירי bound עני in affliction וברזל׃ and iron;}%
\verse{כי Because המרו they rebelled against אמרי the words אל of God, ועצת the counsel עליון of the most High: נאצו׃ and contemned}%
\verse{ויכנע Therefore he brought down בעמל with labor; לבם their heart כשׁלו they fell down, ואין and none עזר׃ to help.}%
\verse{ויזעקו Then they cried אל unto יהוה בצר in their trouble, להם ממצקותיהם them out of their distresses. יושׁיעם׃ he saved}%
\verse{יוציאם He brought them out מחשׁך of darkness וצלמות and the shadow of death, ומוסרותיהם their bands ינתק׃ and broke}%
\verse{יודו Oh that would praise ליהוה the LORD חסדו his goodness, ונפלאותיו and his wonderful works לבני to the children אדם׃ of men!}%
\verse{כי For שׁבר he hath broken דלתות the gates נחשׁת of brass, ובריחי the bars ברזל of iron גדע׃ and cut}%
\verse{אולים Fools מדרך because פשׁעם of their transgression, ומעונתיהם and because of their iniquities, יתענו׃ are afflicted.}%
\verse{כל all manner אכל of meat; תתעב abhorreth נפשׁם Their soul ויגיעו and they draw near עד unto שׁערי the gates מות׃ of death.}%
\verse{ויזעקו Then they cry אל unto יהוה בצר in their trouble, להם ממצקותיהם them out of their distresses. יושׁיעם׃ he saveth}%
\verse{ישׁלח He sent דברו his word, וירפאם and healed וימלט them, and delivered משׁחיתותם׃ from their destructions.}%
\verse{יודו Oh that would praise ליהוה the LORD חסדו his goodness, ונפלאותיו and his wonderful works לבני to the children אדם׃ of men!}%
\verse{ויזבחו And let them sacrifice זבחי the sacrifices תודה of thanksgiving, ויספרו and declare מעשׂיו his works ברנה׃ with rejoicing.}%
\verse{יורדי They that go down הים to the sea באניות in ships, עשׂי that do מלאכה business במים waters; רבים׃ in great}%
\verse{המה These ראו see מעשׂי the works יהוה of the LORD, ונפלאותיו and his wonders במצולה׃ in the deep.}%
\verse{ויאמר For he commandeth, ויעמד and raiseth רוח wind, סערה the stormy ותרומם which lifteth up גליו׃ the waves}%
\verse{יעלו They mount up שׁמים to the heaven, ירדו they go down תהומות again to the depths: נפשׁם their soul ברעה because of trouble. תתמוגג׃ is melted}%
\verse{יחוגו They reel to and fro, וינועו and stagger כשׁכור like a drunken man, וכל חכמתםתבלע׃}%
\verse{ויצעקו Then they cry אל unto יהוה בצר in their trouble, להם וממצוקתיהם of their distresses. יוציאם׃ and he bringeth them out}%
\verse{יקם He maketh סערה the storm לדממה a calm, ויחשׁו thereof are still. גליהם׃ so that the waves}%
\verse{וישׂמחו Then are they glad כי because ישׁתקו they be quiet; וינחם so he bringeth אל them unto מחוז haven. חפצם׃ their desired}%
\verse{יודו Oh that would praise ליהוה the LORD חסדו his goodness, ונפלאותיו and his wonderful works לבני to the children אדם׃ of men!}%
\verse{וירממוהו Let them exalt בקהל him also in the congregation עם of the people, ובמושׁב him in the assembly זקנים of the elders. יהללוהו׃ and praise}%
\verse{ישׂם He turneth נהרות rivers למדבר into a wilderness, ומצאי and the watersprings מים and the watersprings לצמאון׃ into dry ground;}%
\verse{ארץ land פרי A fruitful למלחה into barrenness, מרעת for the wickedness ישׁבי׃ of them that dwell}%
\verse{ישׂם He turneth מדבר the wilderness לאגם into a standing מים water, וארץ ground ציה and dry למצאי into watersprings. מים׃ into watersprings.}%
\verse{ויושׁב he maketh the hungry to dwell, שׁם And there רעבים he maketh the hungry to dwell, ויכוננו that they may prepare עיר a city מושׁב׃ for habitation;}%
\verse{ויזרעו And sow שׂדות the fields, ויטעו and plant כרמים vineyards, ויעשׂו which may yield פרי fruits תבואה׃ of increase.}%
\verse{ויברכם He blesseth וירבו them also, so that they are multiplied מאד greatly; ובהמתם their cattle לא and suffereth not ימעיט׃ to decrease.}%
\verse{וימעטו Again, they are minished וישׁחו and brought low מעצר through oppression, רעה affliction, ויגון׃ and sorrow.}%
\verse{שׁפך He poureth בוז contempt על upon נדיבים princes, ויתעם and causeth them to wander בתהו in the wilderness, לא no דרך׃ way.}%
\verse{וישׂגב אביוןעוני from affliction, וישׂם and maketh כצאן like a flock. משׁפחות׃ families}%
\verse{יראו shall see ישׁרים The righteous וישׂמחו and rejoice: וכל and all עולה iniquity קפצה shall stop פיה׃ her mouth.}%
\verse{מי Whoso חכם wise, וישׁמר and will observe אלה these ויתבוננו even they shall understand חסדי the lovingkindness יהוה׃ of the LORD.}%
\end{biblechapter}%
\begin{biblechapter}% Psalm 108
\verseWithHeading{Prayer to Adonai for Victory over Enemies}{שׁיר A Song מזמור Psalm לדוד׃ of David. נכון is fixed; לבי my heart אלהים O God, אשׁירה I will sing ואזמרה and give praise, אף even כבודי׃ with my glory.}%
\verse{עורה Awake, הנבל psaltery וכנור and harp: אעירה I will awake שׁחר׃ early.}%
\verse{אודך I will praise בעמים among the people: יהוה thee, O LORD, ואזמרך and I will sing praises בל אמים׃ unto thee among the nations.}%
\verse{כי For גדול great מעל above שׁמים the heavens: חסדך thy mercy ועד unto שׁחקים the clouds. אמתך׃ and thy truth}%
\verse{רומה Be thou exalted, על above שׁמים the heavens: אלהים O God, ועל above כל all הארץ the earth; כבודך׃ and thy glory}%
\verse{למען That יחלצון may be delivered: ידידיך thy beloved הושׁיעה save ימינך thy right hand, וענני׃ and answer}%
\verse{אלהים God דבר hath spoken בקדשׁו in his holiness; אעלזה I will rejoice, אחלקה I will divide שׁכם Shechem, ועמק the valley סכות of Succoth. אמדד׃ and mete out}%
\verse{לי גלעד Gilead לי מנשׁה mine; Manasseh ואפרים mine; Ephraim מעוז also the strength ראשׁי of mine head; יהודה Judah מחקקי׃ my lawgiver;}%
\verse{מואב Moab סיר my washpot; רחצי my washpot; על over אדום Edom אשׁליך will I cast out נעלי my shoe; עלי over פלשׁת Philistia אתרועע׃ will I triumph.}%
\verse{מי Who יבלני will bring עיר city? מבצר me into the strong מי who נחני will lead עד me into אדום׃ Edom?}%
\verse{הלא not אלהים O God, זנחתנו hast cast us off? ולא and wilt not תצא go forth אלהים thou, O God, בצבאתינו׃ with our hosts?}%
\verse{הבה Give לנו עזרת us help מצר from trouble: ושׁוא for vain תשׁועת the help אדם׃ of man.}%
\verse{באלהים Through God נעשׂה we shall do חיל valiantly: והוא for he יבוס shall tread down צרינו׃ our enemies.}%
\end{biblechapter}%
\begin{biblechapter}% Psalm 109
\verseWithHeading{A Prayer for Help against Enemies}{למנצח To the chief Musician, לדוד of David. מזמור A Psalm אלהי thy peace, O God תהלתי of my praise; אל not תחרשׁ׃ Hold}%
\verse{כי For פי the mouth רשׁע of the wicked ופי and the mouth מרמה of the deceitful עלי against פתחו are opened דברו me: they have spoken אתי against לשׁון tongue. שׁקר׃ me with a lying}%
\verse{ודברי also with words שׂנאה of hatred; סבבוני They compassed me about וילחמוני and fought against חנם׃ me without a cause.}%
\verse{תחת For אהבתי my love ישׂטנוני they are my adversaries: ואני but I תפלה׃ prayer.}%
\verse{וישׂימו עליעה me evil תחת for טובה good, ושׂנאה and hatred תחת for אהבתי׃ my love.}%
\verse{הפקד Set עליו over רשׁע thou a wicked man ושׂטן him: and let Satan יעמד stand על at ימינו׃ his right hand.}%
\verse{בהשׁפטו When he shall be judged, יצא let him be condemned: רשׁע let him be condemned: ותפלתו his prayer תהיה become לחטאה׃ sin.}%
\verse{יהיו be ימיו Let his days מעטים few; פקדתו his office. יקח take אחר׃ let another}%
\verse{יהיו be בניו Let his children יתומים fatherless, ואשׁתו and his wife אלמנה׃ a widow.}%
\verse{ונוע be continually vagabonds, ינועו be continually vagabonds, בניו Let his children ושׁאלו and beg: ודרשׁו let them seek מחרבותיהם׃ also out of their desolate places.}%
\verse{ינקשׁ catch נושׁה Let the extortioner לכל all אשׁר that לו ויבזו spoil זרים he hath; and let the strangers יגיעו׃ his labor.}%
\verse{אל none יהי Let there be לו משׁך to extend חסד mercy ואל unto him: neither יהי let there be חונן any to favor ליתומיו׃ his fatherless children.}%
\verse{יהי be אחריתו Let his posterity להכרית cut off; בדור in the generation אחר following ימח be blotted out. שׁמם׃ let their name}%
\verse{יזכר be remembered עון Let the iniquity אבתיו of his fathers אל with יהוה the LORD; וחטאת the sin אמו of his mother אל and let not תמח׃ be blotted out.}%
\verse{יהיו Let them be נגד before יהוה the LORD תמיד continually, ויכרת that he may cut off מארץ of them from the earth. זכרם׃ the memory}%
\verse{יען Because אשׁר that לא not זכר he remembered עשׂות to show חסד mercy, וירדף but persecuted אישׁ man, עני the poor ואביון and needy ונכאה לבב in heart. למותת׃ that he might even slay}%
\verse{ויאהב As he loved קללה cursing, ותבואהו so let it come ולא not חפץ unto him: as he delighted בברכה in blessing, ותרחק so let it be far ממנו׃ from}%
\verse{וילבשׁ As he clothed קללה himself with cursing כמדו like as with his garment, ותבא so let it come כמים like water, בקרבו into his bowels וכשׁמן and like oil בעצמותיו׃ into his bones.}%
\verse{תהי Let it be לו כבגד unto him as the garment יעטה covereth ולמזח him, and for a girdle תמיד continually. יחגרה׃ wherewith he is girded}%
\verse{זאת this פעלת the reward שׂטני of mine adversaries מאת יהוה the LORD, והדברים and of them that speak רע evil על against נפשׁי׃ my soul.}%
\verse{ואתה thou יהוה me, O GOD אדני the Lord, עשׂה But do אתי for למען שׁמךי because טוב good, חסדך thy mercy הצילני׃ deliver}%
\verse{כי For עני poor ואביון and needy, אנכי I ולבי and my heart חלל is wounded בקרבי׃ within}%
\verse{כצל like the shadow כנטותו when it declineth: נהלכתי I am gone ננערתי I am tossed up and down כארבה׃ as the locust.}%
\verse{ברכי My knees כשׁלו are weak מצום through fasting; ובשׂרי and my flesh כחשׁ faileth משׁמן׃ of fatness.}%
\verse{ואני I הייתי became חרפה also a reproach להם יראוני unto them: they looked upon יניעון me they shaked ראשׁם׃ their heads.}%
\verse{עזרני Help יהוה me, O LORD אלהי my God: הושׁיעני O save כחסדך׃ me according to thy mercy:}%
\verse{וידעו they may know כי That ידך thy hand; זאת that this אתה thou, יהוה LORD, עשׂיתה׃ hast done}%
\verse{יקללו curse, המה Let them ואתה thou: תברך but bless קמו when they arise, ויבשׁו let them be ashamed; ועבדך but let thy servant ישׂמח׃ rejoice.}%
\verse{ילבשׁו be clothed שׂוטני Let mine adversaries כלמה with shame, ויעטו and let them cover כמעיל as with a mantle. בשׁתם׃ themselves with their own confusion,}%
\verse{אודה praise יהוה the LORD מאד I will greatly בפי with my mouth; ובתוך him among רבים the multitude. אהללנו׃ yea, I will praise}%
\verse{כי For יעמד he shall stand לימין at the right hand אביון of the poor, להושׁיע to save משׁפטי from those that condemn נפשׁו׃ his soul.}%
\end{biblechapter}%
\begin{biblechapter}% Psalm 110
\verseWithHeading{Adonai Gives Authority to His Messiah}{לדוד of David. מזמור A Psalm נאם said יהוה The LORD לאדני unto my Lord, שׁב Sit לימיני thou at my right hand, עד until אשׁית I make איביך thine enemies הדם thy footstool. לרגליך׃ thy footstool.}%
\verse{מטה the rod עזך of thy strength ישׁלח shall send יהוה The LORD מציון רדה rule בקרב thou in the midst איביך׃ of thine enemies.}%
\verse{עמך Thy people נדבת willing ביום in the day חילך of thy power, בהדרי in the beauties קדשׁ of holiness מרחם from the womb משׁחר of the morning: לך טל thou hast the dew ילדתיך׃ of thy youth.}%
\verse{נשׁבע hath sworn, יהוה The LORD ולא and will not ינחם repent, אתה Thou כהן a priest לעולם forever על after דברתי the order מלכי צדק׃ of Melchizedek.}%
\verse{אדני The Lord על at ימינך thy right hand מחץ shall strike through ביום in the day אפו of his wrath. מלכים׃ kings}%
\verse{ידין He shall judge בגוים among the heathen, מלא he shall fill גויות with the dead bodies; מחץ he shall wound ראשׁ the heads על over ארץ countries. רבה׃ many}%
\verse{מנחל of the brook בדרך in the way: ישׁתה He shall drink על therefore כן therefore ירים shall he lift up ראשׁ׃ the head.}%
\end{biblechapter}%
\begin{biblechapter}% Psalm 111
\verseWithHeading{Praise to God for His Work and Commands}{הללו Praise יה ye the LORD. אודה I will praise יהוה the LORD בכל with whole לבב heart, בסוד in the assembly ישׁרים of the upright, ועדה׃ and the congregation.}%
\verse{גדלים great, מעשׂי The works יהוה of the LORD דרושׁים sought out לכל of all חפציהם׃ them that have pleasure}%
\verse{הוד honorable והדר and glorious: פעלו His work וצדקתו and his righteousness עמדת endureth לעד׃ forever.}%
\verse{זכר to be remembered: עשׂה He hath made לנפלאתיו his wonderful works חנון gracious ורחום and full of compassion. יהוה׃ the LORD}%
\verse{טרף meat נתן He hath given ליראיו unto them that fear יזכר be mindful לעולם him: he will ever בריתו׃ of his covenant.}%
\verse{כח the power מעשׂיו of his works, הגיד He hath showed לעמו his people לתת that he may give להם נחלת them the heritage גוים׃ of the heathen.}%
\verse{מעשׂי The works ידיו of his hands אמת verity ומשׁפט and judgment; נאמנים sure. כל all פקודיו׃ his commandments}%
\verse{סמוכים They stand fast לעד forever לעולם and ever, עשׂוים done באמת in truth וישׁר׃ and uprightness.}%
\verse{פדות redemption שׁלח He sent לעמו unto his people: צוה he hath commanded לעולם forever: בריתו his covenant קדושׁ holy ונורא and reverend שׁמו׃ his name.}%
\verse{ראשׁית the beginning חכמה of wisdom: יראת The fear יהוה of the LORD שׂכל understanding טוב a good לכל have all עשׂיהם they that do תהלתו his praise עמדת endureth לעד׃ forever.}%
\end{biblechapter}%
\begin{biblechapter}% Psalm 112
\verseWithHeading{The Path of the Righteous and the Path of the Wicked}{הללו Praise יה ye the LORD. אשׁרי Blessed אישׁ the man ירא feareth את יהוה the LORD, במצותיו in his commandments. חפץ delighteth מאד׃ greatly}%
\verse{גבור mighty בארץ upon earth: יהיה shall be זרעו His seed דור the generation ישׁרים of the upright יברך׃ shall be blessed.}%
\verse{הון Wealth ועשׁר and riches בביתו in his house: וצדקתו and his righteousness עמדת endureth לעד׃ forever.}%
\verse{זרח there ariseth בחשׁך in the darkness: אור light לישׁרים Unto the upright חנון gracious, ורחום and full of compassion, וצדיק׃ and righteous.}%
\verse{טוב A good אישׁ man חונן showeth favor, ומלוה and lendeth: יכלכל he will guide דבריו his affairs במשׁפט׃ with discretion.}%
\verse{כי Surely לעולם forever: לא he shall not ימוט be moved לזכר remembrance. עולם in everlasting יהיה shall be צדיק׃ the righteous}%
\verse{משׁמועה tidings: רעה of evil לא He shall not יירא be afraid נכון is fixed, לבו his heart בטח trusting ביהוה׃ in the LORD.}%
\verse{סמוך established, לבו His heart לא he shall not יירא be afraid, עד until אשׁר until יראה he see בצריו׃ upon his enemies.}%
\verse{פזר He hath dispersed, נתן he hath given לאביונים to the poor; צדקתו his righteousness עמדת endureth לעד forever; קרנו his horn תרום shall be exalted בכבוד׃ with honor.}%
\verse{רשׁע The wicked יראה shall see וכעס and be grieved; שׁניו with his teeth, יחרק he shall gnash ונמס and melt away: תאות the desire רשׁעים of the wicked תאבד׃ shall perish.}%
\end{biblechapter}%
\begin{biblechapter}% Psalm 113
\verseWithHeading{God’s Majesty and Care for the Needy}{הללו Praise יה ye the LORD. הללו Praise, עבדי O ye servants יהוה of the LORD, הללו praise את שׁם the name יהוה׃ of the LORD.}%
\verse{יהי be שׁם the name יהוה of the LORD מברך Blessed מעתה from this time forth ועד and forevermore. עולם׃ and forevermore.}%
\verse{ממזרח שׁמשׁ of the sun עד unto מבואו the going down מהלל to be praised. שׁם name יהוה׃ of the same the LORD's}%
\verse{רם high על above כל all גוים nations, יהוה The LORD על above השׁמים the heavens. כבודו׃ his glory}%
\verse{מי Who כיהוה like unto the LORD אלהינו our God, המגביהי on high, לשׁבת׃ who dwelleth}%
\verse{המשׁפילי Who humbleth לראות to behold בשׁמים in heaven, ובארץ׃ and in the earth!}%
\verse{מקימי He raiseth up מעפר out of the dust, דל the poor מאשׁפת out of the dunghill; ירים lifteth אביון׃ the needy}%
\verse{להושׁיבי That he may set עם with נדיבים princes, עם with נדיבי the princes עמו׃ of his people.}%
\verse{מושׁיבי woman to keep עקרת He maketh the barren הבית house, אם mother הבנים of children. שׂמחה a joyful הללו Praise יה׃ ye the LORD.}%
\end{biblechapter}%
\begin{biblechapter}% Psalm 114
\verseWithHeading{Praise to God for His Works During the Exodus}{בצאת went out ישׂראל When Israel ממצרים בית the house יעקב of Jacob מעם from a people לעז׃ of strange language;}%
\verse{היתה was יהודה Judah לקדשׁו his sanctuary, ישׂראל Israel ממשׁלותיו׃ his dominion.}%
\verse{הים The sea ראה saw וינס and fled: הירדן Jordan יסב was driven לאחור׃ back.}%
\verse{ההרים The mountains רקדו skipped כאילים like rams, גבעות the little hills כבני like lambs. צאן׃ like lambs.}%
\verse{מה What לך הים thee, O thou sea, כי that תנוס thou fleddest? הירדן thou Jordan, תסב thou wast driven לאחור׃ back?}%
\verse{ההרים Ye mountains, תרקדו ye skipped כאילים like rams; גבעות ye little hills, כבני like lambs? צאן׃ like lambs?}%
\verse{מלפני at the presence אדון of the Lord, חולי Tremble, ארץ thou earth, מלפני at the presence אלוה of the God יעקב׃ of Jacob;}%
\verse{ההפכי Which turned הצור the rock אגם a standing מים water, חלמישׁ the flint למעינו into a fountain מים׃ of waters.}%
\end{biblechapter}%
\begin{biblechapter}% Psalm 115
\verseWithHeading{Dead Idols and the Living God}{לא Not לנו יהוה unto us, O LORD, לא not לנו כי unto us, but לשׁמך unto thy name תן give כבוד glory, על for חסדך thy mercy, על for אמתך׃ thy truth's}%
\verse{למה Wherefore יאמרו say, הגוים should the heathen איה Where נא now אלהיהם׃ their God?}%
\verse{ואלהינו But our God בשׁמים in the heavens: כל whatsoever אשׁר whatsoever חפץ he hath pleased. עשׂה׃ he hath done}%
\verse{עצביהם Their idols כסף silver וזהב and gold, מעשׂה the work ידי hands. אדם׃ of men's}%
\verse{פה They have mouths, להם ולא not: ידברו but they speak עינים eyes להם ולא not: יראו׃ have they, but they see}%
\verse{אזנים They have ears, להם ולא not: ישׁמעו but they hear אף noses להם ולא not: יריחון׃ have they, but they smell}%
\verse{ידיהם They have hands, ולא not: ימישׁון but they handle רגליהם feet ולא not: יהלכו have they, but they walk לא neither יהגו speak בגרונם׃ they through their throat.}%
\verse{כמוהם like unto them; יהיו them are עשׂיהם They that make כל every one אשׁר that בטח׃ trusteth}%
\verse{ישׂראל O Israel, בטח trust ביהוה thou in the LORD: עזרם their help ומגנם and their shield. הוא׃ he}%
\verse{בית O house אהרן of Aaron, בטחו trust ביהוה in the LORD: עזרם their help ומגנם and their shield. הוא׃ he}%
\verse{יראי Ye that fear יהוה the LORD, בטחו trust ביהוה in the LORD: עזרם their help ומגנם and their shield. הוא׃ he}%
\verse{יהוה The LORD זכרנו hath been mindful יברך of us: he will bless יברך he will bless את בית the house ישׂראל of Israel; יברך he will bless את בית the house אהרן׃ of Aaron.}%
\verse{יברך He will bless יראי them that fear יהוה the LORD, הקטנים small עם and הגדלים׃ great.}%
\verse{יסף shall increase you more and more, יהוה The LORD עליכם shall increase you more and more, עליכם ועלניכם׃ you and your children.}%
\verse{ברוכים blessed אתם Ye ליהוה of the LORD עשׂה which made שׁמים heaven וארץ׃ and earth.}%
\verse{השׁמים The heaven, שׁמים the heavens, ליהוה the LORD's: והארץ but the earth נתן hath he given לבני to the children אדם׃ of men.}%
\verse{לא not המתים The dead יהללו praise יה the LORD, ולא neither כל any ירדי that go down דומה׃ into silence.}%
\verse{ואנחנו But we נברך will bless יה the LORD מעתה from this time forth ועד and forevermore. עולם and forevermore. הללו Praise יה׃ the LORD.}%
\end{biblechapter}%
\begin{biblechapter}% Psalm 116
\verseWithHeading{Thanksgiving for God’s Deliverance}{אהבתי I love כי because ישׁמע he hath heard יהוה the LORD, את קולי my voice תחנוני׃ my supplications.}%
\verse{כי Because הטה he hath inclined אזנו his ear לי ובימי as long as I live. אקרא׃ unto me, therefore will I call upon}%
\verse{אפפוני compassed חבלי The sorrows מות of death ומצרי me, and the pains שׁאול of hell מצאוני got hold upon צרה trouble ויגון and sorrow. אמצא׃ me: I found}%
\verse{ובשׁם I upon the name יהוה of the LORD; אקרא Then called אנה I beseech יהוה O LORD, מלטה thee, deliver נפשׁי׃ my soul.}%
\verse{חנון Gracious יהוה the LORD, וצדיק and righteous; ואלהינו yea, our God מרחם׃ merciful.}%
\verse{שׁמר preserveth פתאים the simple: יהוה The LORD דלותי I was brought low, ולי יהושׁיע׃ and he helped}%
\verse{שׁובי Return נפשׁי O my soul; למנוחיכי כי for יהוה the LORD גמל hath dealt bountifully עליכי׃ with}%
\verse{כי For חלצת thou hast delivered נפשׁי my soul ממות from death, את עיני mine eyes מן from death, דמעה tears, את רגלי my feet מדחי׃ from falling.}%
\verse{אתהלך I will walk לפני before יהוה the LORD בארצות in the land החיים׃ of the living.}%
\verse{האמנתי I believed, כי therefore אדבר spoken: אני have I עניתי afflicted: מאד׃ I was greatly}%
\verse{אני I אמרתי said בחפזי in my haste, כל All האדם men כזב׃ liars.}%
\verse{מה What אשׁיב shall I render ליהוה unto the LORD כל all תגמולוהי his benefits עלי׃ toward}%
\verse{כוס the cup ישׁועות of salvation, אשׂא I will take ובשׁם upon the name יהוה of the LORD. אקרא׃ and call}%
\verse{נדרי my vows ליהוה unto the LORD אשׁלם I will pay נגדה in the presence נא now לכל of all עמו׃ his people.}%
\verse{יקר Precious בעיני in the sight יהוה of the LORD המותה the death לחסידיו׃ of his saints.}%
\verse{אנה truly יהוה O LORD, כי אני I עבדך thy servant; אני I עבדך thy servant, בן the son אמתך of thine handmaid: פתחת thou hast loosed למוסרי׃ my bonds.}%
\verse{לך אזבח I will offer זבח to thee the sacrifice תודה of thanksgiving, ובשׁם upon the name יהוה of the LORD. אקרא׃ and will call}%
\verse{נדרי my vows ליהוה unto the LORD אשׁלם I will pay נגדה in the presence נא now לכל of all עמו׃ his people,}%
\verse{בחצרות In the courts בית house, יהוה of the LORD's בתוככי in the midst ירושׁלם of thee, O Jerusalem. הללו Praise יה׃ ye the LORD.}%
\end{biblechapter}%
\begin{biblechapter}% Psalm 117
\verseWithHeading{Let All Peoples Praise Adonai}{הללו O praise את יהוה the LORD, כל all גוים ye nations: שׁבחוהו praise כל him, all האמים׃ ye people.}%
\verse{כי For גבר is great עלינו toward חסדו his merciful kindness ואמת us: and the truth יהוה of the LORD לעולם forever. הללו Praise יה׃ ye the LORD.}%
\end{biblechapter}%
\begin{biblechapter}% Psalm 118
\verseWithHeading{Praise to God for His Loyal Love}{הודו O give thanks ליהוה unto the LORD; כי for טוב good: כי because לעולם forever. חסדו׃ his mercy}%
\verse{יאמר say, נא now ישׂראל Let Israel כי that לעולם forever. חסדו׃ his mercy}%
\verse{יאמרו say, נא now בית Let the house אהרן of Aaron כי that לעולם forever. חסדו׃ his mercy}%
\verse{יאמרו say, נא Let them now יראי that fear יהוה the LORD כי that לעולם forever. חסדו׃ his mercy}%
\verse{מן in המצר distress: קראתי I called upon יה the LORD ענני answered במרחב me, in a large place. יה׃ the LORD}%
\verse{יהוה The LORD לי לא on my side; I will not אירא fear: מה what יעשׂה do לי אדם׃ can man}%
\verse{יהוה The LORD לי בעזרי taketh my part with them that help ואני me: therefore shall I אראה see בשׂנאי׃ upon them that hate}%
\verse{טוב better לחסות to trust ביהוה in the LORD מבטח than to put confidence באדם׃ in man.}%
\verse{טוב better לחסות to trust ביהוה in the LORD מבטח than to put confidence בנדיבים׃ in princes.}%
\verse{כל All גוים nations סבבוני compassed me about: בשׁם but in the name יהוה of the LORD כי אמילם׃ will I destroy}%
\verse{סבוני They compassed me about; גם yea, סבבוני they compassed me about: בשׁם but in the name יהוה of the LORD כי אמילם׃ I will destroy}%
\verse{סבוני They compassed me about כדבורים like bees; דעכו they are quenched כאשׁ as the fire קוצים of thorns: בשׁם for in the name יהוה of the LORD כי אמילם׃ I will destroy}%
\verse{דחה דחיתנינפל at me that I might fall: ויהוה but the LORD עזרני׃ helped}%
\verse{עזי my strength וזמרת and song, יה The LORD ויהי and is become לי לישׁועה׃ my salvation.}%
\verse{קול The voice רנה of rejoicing וישׁועה and salvation באהלי in the tabernacles צדיקים of the righteous: ימין the right hand יהוה of the LORD עשׂה doeth חיל׃ valiantly.}%
\verse{ימין The right hand יהוה of the LORD רוממה ימין the right hand יהוה of the LORD עשׂה doeth חיל׃ valiantly.}%
\verse{לא I shall not אמות die, כי but אחיה live, ואספר and declare מעשׂי the works יה׃ of the LORD.}%
\verse{יסר hath chastened me sore: יסרני hath chastened me sore: יה The LORD ולמות unto death. לא but he hath not נתנני׃ given me over}%
\verse{פתחו Open לי שׁערי to me the gates צדק of righteousness: אבא I will go בם אודה into them, I will praise יה׃ the LORD:}%
\verse{זה This השׁער gate ליהוה of the LORD, צדיקים into which the righteous יבאו׃ shall enter.}%
\verse{אודך I will praise כי thee: for עניתני thou hast heard ותהי me, and art become לי לישׁועה׃ my salvation.}%
\verse{אבן The stone מאסו refused הבונים the builders היתה is become לראשׁ the head פנה׃ of the corner.}%
\verse{מאת יהוה the LORD's היתה is זאת This היא it נפלאת marvelous בעינינו׃ in our eyes.}%
\verse{זה This היום the day עשׂה hath made; יהוה the LORD נגילה we will rejoice ונשׂמחה׃ and be glad}%
\verse{אנא I beseech יהוה thee, O LORD: הושׁיעה Save נא now, אנא I beseech יהוה O LORD, הצליחה prosperity. נא׃ thee, send now}%
\verse{ברוך Blessed הבא he that cometh בשׁם in the name יהוה of the LORD: ברכנוכם we have blessed מבית you out of the house יהוה׃ of the LORD.}%
\verse{אל God יהוה the LORD, ויאר לנו אסרו bind חג the sacrifice בעבתים with cords, עד unto קרנות the horns המזבח׃ of the altar.}%
\verse{אלי my God, אתה Thou ואודך and I will praise אלהי thee: my God, ארוממך׃ I will exalt}%
\verse{הודו O give thanks ליהוה unto the LORD; כי for טוב good: כי for לעולם forever. חסדו׃ his mercy}%
\end{biblechapter}%
\begin{biblechapter}% Psalm 119
\verseWithHeading{Aleph}{אשׁרי Blessed תמימי the undefiled דרך in the way, ההלכים who walk בתורת in the law יהוה׃}%
\verse{אשׁרי Blessed נצרי they that keep עדתיו his testimonies, בכל him with the whole לב heart. ידרשׁוהו׃ seek}%
\verse{אף They also לא no פעלו do עולה iniquity: בדרכיו in his ways. הלכו׃ they walk}%
\verse{אתה Thou צויתה hast commanded פקדיך thy precepts לשׁמר to keep מאד׃ diligently.}%
\verse{אחלי O that יכנו were directed דרכי my ways לשׁמר to keep חקיך׃ thy statutes!}%
\verse{אז Then לא shall I not אבושׁ be ashamed, בהביטי when I have respect אל unto כל all מצותיך׃ thy commandments.}%
\verse{אודך I will praise בישׁר thee with uprightness לבב of heart, בלמדי when I shall have learned משׁפטי judgments. צדקך׃ thy righteous}%
\verse{את חקיך thy statutes: אשׁמר I will keep אל me not תעזבני O forsake עד utterly. מאד׃ utterly.}%
\verseWithHeading{Beth}{במה Wherewithal יזכה cleanse נער shall a young man את ארחו his way? לשׁמר by taking heed כדברך׃ according to thy word.}%
\verse{בכל With my whole לבי heart דרשׁתיך have I sought אל thee: O let me not תשׁגני wander ממצותיך׃ from thy commandments.}%
\verse{בלבי in mine heart, צפנתי have I hid אמרתך Thy word למען that לא I might not אחטא׃ sin}%
\verse{ברוך Blessed אתה thou, יהוה למדני teach חקיך׃ me thy statutes.}%
\verse{בשׂפתי With my lips ספרתי have I declared כל all משׁפטי the judgments פיך׃ of thy mouth.}%
\verse{בדרך in the way עדותיך of thy testimonies, שׂשׂתי I have rejoiced כעל as כל in all הון׃ riches.}%
\verse{בפקדיך in thy precepts, אשׂיחה I will meditate ואביטה and have respect ארחתיך׃ unto thy ways.}%
\verse{בחקתיך in thy statutes: אשׁתעשׁע I will delight myself לא I will not אשׁכח forget דברך׃ thy word.}%
\verseWithHeading{Gimel}{גמל Deal bountifully על with עבדך thy servant, אחיה I may live, ואשׁמרה and keep דברך׃ thy word.}%
\verse{גל Open עיני thou mine eyes, ואביטה that I may behold נפלאות wondrous things מתורתך׃ out of thy law.}%
\verse{גר a stranger אנכי I בארץ in the earth: אל not תסתר hide ממני from מצותיך׃ thy commandments}%
\verse{גרסה breaketh נפשׁי My soul לתאבה for the longing אל unto משׁפטיך thy judgments בכל at all עת׃ times.}%
\verse{גערת Thou hast rebuked זדים the proud ארורים cursed, השׁגים which do err ממצותיך׃ from thy commandments.}%
\verse{גל Remove מעלי from חרפה me reproach ובוז and contempt; כי for עדתיך thy testimonies. נצרתי׃ I have kept}%
\verse{גם also ישׁבו did sit שׂרים Princes בי נדברו speak עבדך against me: thy servant ישׂיח did meditate בחקיך׃ in thy statutes.}%
\verse{גם also עדתיך Thy testimonies שׁעשׁעי my delight אנשׁי עצתי׃ my counselors.}%
\verseWithHeading{Daleth}{דבקה cleaveth לעפר unto the dust: נפשׁי My soul חיני quicken כדברך׃ thou me according to thy word.}%
\verse{דרכי my ways, ספרתי I have declared ותענני and thou heardest למדני me: teach חקיך׃ me thy statutes.}%
\verse{דרך the way פקודיך of thy precepts: הבינני Make me to understand ואשׂיחה so shall I talk בנפלאותיך׃ of thy wondrous works.}%
\verse{דלפה melteth נפשׁי My soul מתוגה for heaviness: קימני strengthen כדברך׃ thou me according unto thy word.}%
\verse{דרך me the way שׁקר of lying: הסר Remove ממני from ותורתך and grant me thy law חנני׃ graciously.}%
\verse{דרך the way אמונה of truth: בחרתי I have chosen משׁפטיך thy judgments שׁויתי׃ have I laid}%
\verse{דבקתי I have stuck בעדותיך unto thy testimonies: יהוה אל put me not תבישׁני׃ to shame.}%
\verse{דרך the way מצותיך of thy commandments, ארוץ I will run כי when תרחיב thou shalt enlarge לבי׃ my heart.}%
\verseWithHeading{Heʼ}{הורני Teach יהוה דרך the way חקיך of thy statutes; ואצרנה and I shall keep עקב׃ it the end.}%
\verse{הבינני Give me understanding, ואצרה and I shall keep תורתך thy law; ואשׁמרנה yea, I shall observe בכל it with whole לב׃ heart.}%
\verse{הדריכני Make me to go בנתיב in the path מצותיך of thy commandments; כי for בו חפצתי׃ therein do I delight.}%
\verse{הט Incline לבי my heart אל unto עדותיך thy testimonies, ואל and not אל to בצע׃ covetousness.}%
\verse{העבר Turn away עיני mine eyes מראות from beholding שׁוא vanity; בדרכך thou me in thy way. חיני׃ quicken}%
\verse{הקם Establish לעבדך unto thy servant, אמרתך thy word אשׁר who ליראתך׃ to thy fear.}%
\verse{העבר Turn away חרפתי my reproach אשׁר which יגרתי I fear: כי for משׁפטיך thy judgments טובים׃ good.}%
\verse{הנה Behold, תאבתי I have longed לפקדיך after thy precepts: בצדקתך me in thy righteousness. חיני׃ quicken}%
\verseWithHeading{Waw}{ויבאני come חסדך Let thy mercies יהוה תשׁועתך thy salvation, כאמרתך׃ according to thy word.}%
\verse{ואענה to answer חרפי him that reproacheth דבר So shall I have wherewith כי me: for בטחתי I trust בדברך׃ in thy word.}%
\verse{ואל not תצל And take מפי out of my mouth; דבר the word אמת of truth עד utterly מאד utterly כי for למשׁפטך in thy judgments. יחלתי׃ I have hoped}%
\verse{ואשׁמרה So shall I keep תורתך thy law תמיד continually לעולם forever ועד׃ and ever.}%
\verse{ואתהלכה And I will walk ברחבה at liberty: כי for פקדיך thy precepts. דרשׁתי׃ I seek}%
\verse{ואדברה I will speak בעדתיך of thy testimonies נגד also before מלכים kings, ולא and will not אבושׁ׃ be ashamed.}%
\verse{ואשׁתעשׁע And I will delight myself במצותיך in thy commandments, אשׁר which אהבתי׃ I have loved.}%
\verse{ואשׂא also will I lift up כפי My hands אל unto מצותיך thy commandments, אשׁר which אהבתי I have loved; ואשׂיחה and I will meditate בחקיך׃ in thy statutes.}%
\verseWithHeading{Zayin}{זכר Remember דבר the word לעבדך unto thy servant, על upon אשׁר which יחלתני׃ thou hast caused me to hope.}%
\verse{זאת This נחמתי my comfort בעניי in my affliction: כי for אמרתך thy word חיתני׃ hath quickened}%
\verse{זדים The proud הליצני in derision: עד have had me greatly מאד have had me greatly מתורתך from thy law. לא have I not נטיתי׃ declined}%
\verse{זכרתי I remembered משׁפטיך thy judgments מעולם of old, יהוה ואתנחם׃ and have comforted myself.}%
\verse{זלעפה Horror אחזתני hath taken hold upon מרשׁעים me because of the wicked עזבי that forsake תורתך׃ thy law.}%
\verse{זמרות my songs היו have been לי חקיך Thy statutes בבית in the house מגורי׃ of my pilgrimage.}%
\verse{זכרתי I have remembered בלילה in the night, שׁמך thy name, יהוה ואשׁמרה and have kept תורתך׃ thy law.}%
\verse{זאת This היתה I had, לי כי because פקדיך thy precepts. נצרתי׃ I kept}%
\verseWithHeading{Heth}{חלקי my portion, יהוה אמרתי I have said לשׁמר that I would keep דבריך׃ thy words.}%
\verse{חליתי I entreated פניך thy favor בכל with whole לב heart: חנני be merciful כאמרתך׃ unto me according to thy word.}%
\verse{חשׁבתי I thought דרכי on my ways, ואשׁיבה and turned רגלי my feet אל unto עדתיך׃ thy testimonies.}%
\verse{חשׁתי I made haste, ולא not התמהמהתי and delayed לשׁמר to keep מצותיך׃ thy commandments.}%
\verse{חבלי The bands רשׁעים of the wicked עודני have robbed תורתך thy law. לא me: I have not שׁכחתי׃ forgotten}%
\verse{חצות לילהקום I will rise להודות to give thanks לך על unto thee because of משׁפטי judgments. צדקך׃ thy righteous}%
\verse{חבר a companion אני I לכל of all אשׁר that יראוך fear ולשׁמרי thee, and of them that keep פקודיך׃ thy precepts.}%
\verse{חסדך of thy mercy: יהוה מלאה is full הארץ The earth, חקיך me thy statutes. למדני׃ teach}%
\verseWithHeading{Teth}{טוב well עשׂית Thou hast dealt עם with עבדך thy servant, יהוה כדברך׃ according unto thy word.}%
\verse{טוב me good טעם judgment ודעת and knowledge: למדני Teach כי for במצותיך thy commandments. האמנתי׃ I have believed}%
\verse{טרם Before אענה was afflicted אני I שׁגג I went astray: ועתה but now אמרתך thy word. שׁמרתי׃ have I kept}%
\verse{טוב good, אתה Thou ומטיב and doest good; למדני teach חקיך׃ me thy statutes.}%
\verse{טפלו have forged עלי against שׁקר a lie זדים The proud אני me: I בכל with whole לב heart. אצר will keep פקודיך׃ thy precepts}%
\verse{טפשׁ is as fat כחלב as grease; לבם Their heart אני I תורתך in thy law. שׁעשׁעתי׃ delight}%
\verse{טוב good לי כי for me that עניתי I have been afflicted; למען that אלמד I might learn חקיך׃ thy statutes.}%
\verse{טוב better לי תורת The law פיך of thy mouth מאלפי unto me than thousands זהב of gold וכסף׃ and silver.}%
\verseWithHeading{Yod}{ידיך Thy hands עשׂוני have made ויכוננוני me and fashioned הבינני me: give me understanding, ואלמדה that I may learn מצותיך׃ thy commandments.}%
\verse{יראיך They that fear יראוני when they see וישׂמחו thee will be glad כי me; because לדברך in thy word. יחלתי׃ I have hoped}%
\verse{ידעתי I know, יהוה כי that צדק right, משׁפטיך thy judgments ואמונה and thou in faithfulness עניתני׃ hast afflicted}%
\verse{יהי be נא Let, I pray thee, חסדך thy merciful kindness לנחמני for my comfort, כאמרתך according to thy word לעבדך׃ unto thy servant.}%
\verse{יבאוני come רחמיך Let thy tender mercies ואחיה unto me, that I may live: כי for תורתך thy law שׁעשׁעי׃ my delight.}%
\verse{יבשׁו be ashamed; זדים Let the proud כי for שׁקר with me without a cause: עותוני they dealt perversely אני I אשׂיח will meditate בפקודיך׃ in thy precepts.}%
\verse{ישׁובו thee turn לי יראיך Let those that fear וידעו unto me, and those that have known עדתיך׃ thy testimonies.}%
\verse{יהי be לבי Let my heart תמים sound בחקיך in thy statutes; למען that לא I be not אבושׁ׃ ashamed.}%
\verseWithHeading{Kaph}{כלתה fainteth לתשׁועתך for thy salvation: נפשׁי My soul לדברך in thy word. יחלתי׃ I hope}%
\verse{כלו fail עיני Mine eyes לאמרתך for thy word, לאמר saying, מתי When תנחמני׃ wilt thou comfort}%
\verse{כי For הייתי I am become כנאד like a bottle בקיטור in the smoke; חקיך thy statutes. לא do I not שׁכחתי׃ forget}%
\verse{כמה How many ימי the days עבדך of thy servant? מתי when תעשׂה wilt thou execute ברדפי on them that persecute משׁפט׃ judgment}%
\verse{כרו have digged לי זדים The proud שׁיחות pits אשׁר for me, which לא not כתורתך׃ after thy law.}%
\verse{כל All מצותיך thy commandments אמונה faithful: שׁקר me wrongfully; רדפוני they persecute עזרני׃ help}%
\verse{כמעט They had almost כלוני consumed בארץ me upon earth; ואני but I לא not עזבתי forsook פקודיך׃ thy precepts.}%
\verse{כחסדך me after thy lovingkindness; חיני Quicken ואשׁמרה so shall I keep עדות the testimony פיך׃ of thy mouth.}%
\verseWithHeading{Lamed}{לעולם Forever, יהוה דברך thy word נצב is settled בשׁמים׃ in heaven.}%
\verse{לדר unto all generations: ודר unto all generations: אמונתך Thy faithfulness כוננת thou hast established ארץ the earth, ותעמד׃ and it abideth.}%
\verse{למשׁפטיך according to thine ordinances: עמדו They continue היום this day כי for הכל all עבדיך׃ thy servants.}%
\verse{לולי Unless תורתך thy law שׁעשׁעי my delights, אז I should then אבדתי have perished בעניי׃ in mine affliction.}%
\verse{לעולם לאשׁכח forget פקודיך thy precepts: כי for בם חייתני׃ with them thou hast quickened}%
\verse{לך אני I הושׁיעני thine, save כי me; for פקודיך thy precepts. דרשׁתי׃ I have sought}%
\verse{לי קוו have waited רשׁעים The wicked לאבדני for me to destroy עדתיך thy testimonies. אתבונן׃ me: I will consider}%
\verse{לכל of all תכלה perfection: ראיתי I have seen קץ an end רחבה broad. מצותך thy commandment מאד׃ exceeding}%
\verseWithHeading{Mem}{מה O how אהבתי love תורתך I thy law! כל all היום the day. היא it שׂיחתי׃ my meditation}%
\verse{מאיבי than mine enemies: תחכמני hast made me wiser מצותך Thou through thy commandments כי for לעולם ever היא׃ they}%
\verse{מכל than all מלמדי my teachers: השׂכלתי I have more understanding כי for עדותיך thy testimonies שׂיחה׃ my meditation.}%
\verse{מזקנים more than the ancients, אתבונן I understand כי because פקודיך thy precepts. נצרתי׃ I keep}%
\verse{מכל from every ארח way, רע evil כלאתי I have refrained רגלי my feet למען that אשׁמר I might keep דברך׃ thy word.}%
\verse{ממשׁפטיך from thy judgments: לא I have not סרתי departed כי for אתה thou הורתני׃ hast taught}%
\verse{מה How נמלצו sweet are לחכי unto my taste! אמרתך thy words מדבשׁ than honey לפי׃ to my mouth!}%
\verse{מפקודיך אתבונן I get understanding: על therefore כן therefore שׂנאתי I hate כל every ארח way. שׁקר׃ false}%
\verseWithHeading{Nun}{נר a lamp לרגלי unto my feet, דברך Thy word ואור and a light לנתיבתי׃ unto my path.}%
\verse{נשׁבעתי I have sworn, ואקימה and I will perform לשׁמר that I will keep משׁפטי judgments. צדקך׃ thy righteous}%
\verse{נעניתי I am afflicted עד very much: מאד very much: יהוה חיני quicken כדברך׃ according unto thy word.}%
\verse{נדבות the freewill offerings פי of my mouth, רצה Accept, נא I beseech thee, יהוה ומשׁפטיך me thy judgments. למדני׃ and teach}%
\verse{נפשׁי My soul בכפי in my hand: תמיד continually ותורתך thy law. לא yet do I not שׁכחתי׃ forget}%
\verse{נתנו have laid רשׁעים The wicked פח a snare לי ומפקודיך from thy precepts. לא not תעיתי׃ for me: yet I erred}%
\verse{נחלתי have I taken as a heritage עדותיך Thy testimonies לעולם forever: כי for שׂשׂון the rejoicing לבי of my heart. המה׃ they}%
\verse{נטיתי I have inclined לבי mine heart לעשׂות to perform חקיך thy statutes לעולם always, עקב׃ the end.}%
\verseWithHeading{Samek}{סעפים thoughts: שׂנאתי I hate ותורתך but thy law אהבתי׃ do I love.}%
\verse{סתרי my hiding place ומגני and my shield: אתה Thou לדברך in thy word. יחלתי׃ I hope}%
\verse{סורו Depart ממני from מרעים me, ye evildoers: ואצרה for I will keep מצות the commandments אלהי׃ of my God.}%
\verse{סמכני Uphold כאמרתך me according unto thy word, ואחיה that I may live: ואל and let me not תבישׁני be ashamed משׂברי׃ of my hope.}%
\verse{סעדני Hold thou me up, ואושׁעה and I shall be safe: ואשׁעה and I will have respect בחקיך unto thy statutes תמיד׃ continually.}%
\verse{סלית Thou hast trodden down כל all שׁוגים them that err מחקיך from thy statutes: כי for שׁקר falsehood. תרמיתם׃ their deceit}%
\verse{סגים dross: השׁבת Thou puttest away כל all רשׁעי the wicked ארץ of the earth לכן therefore אהבתי I love עדתיך׃ thy testimonies.}%
\verse{סמר trembleth מפחדך for fear בשׂרי My flesh וממשׁפטיך of thy judgments. יראתי׃ of thee; and I am afraid}%
\verseWithHeading{ʽAyin}{עשׂיתי I have done משׁפט judgment וצדק and justice: בל me not תניחני לעשׁקי׃ to mine oppressors.}%
\verse{ערב Be surety עבדך for thy servant לטוב for good: אל let not יעשׁקני oppress זדים׃ the proud}%
\verse{עיני Mine eyes כלו fail לישׁועתך for thy salvation, ולאמרת and for the word צדקך׃ of thy righteousness.}%
\verse{עשׂה Deal עם with עבדך thy servant כחסדך according unto thy mercy, וחקיך me thy statutes. למדני׃ and teach}%
\verse{עבדך thy servant; אני I הבינני give me understanding, ואדעה that I may know עדתיך׃ thy testimonies.}%
\verse{עת time לעשׂות to work: ליהוה הפרו they have made void תורתך׃ thy law.}%
\verse{על כןהבתי I love מצותיך thy commandments מזהב above gold; ומפז׃ yea, above fine gold.}%
\verse{על כןל I esteem all פקודי precepts כל all ישׁרתי right; כל every ארח way. שׁקר false שׂנאתי׃ I hate}%
\verseWithHeading{Pe}{פלאות wonderful: עדותיך Thy testimonies על therefore כן therefore נצרתם keep נפשׁי׃ doth my soul}%
\verse{פתח The entrance דבריך of thy words יאיר giveth light; מבין it giveth understanding פתיים׃ unto the simple.}%
\verse{פי my mouth, פערתי I opened ואשׁאפה and panted: כי for למצותיך for thy commandments. יאבתי׃ I longed}%
\verse{פנה Look אלי thou upon וחנני me, and be merciful כמשׁפט unto me, as thou usest to do לאהבי unto those that love שׁמך׃ thy name.}%
\verse{פעמי my steps הכן Order באמרתך in thy word: ואל and let not תשׁלט have dominion בי כל any און׃ iniquity}%
\verse{פדני Deliver מעשׁק me from the oppression אדם of man: ואשׁמרה so will I keep פקודיך׃ thy precepts.}%
\verse{פניך Make thy face האר to shine בעבדך upon thy servant; ולמדני and teach את חקיך׃ me thy statutes.}%
\verse{פלגי Rivers מים of waters ירדו run down עיני mine eyes, על because לא not שׁמרו they keep תורתך׃ thy law.}%
\verseWithHeading{Tsade}{צדיק Righteous אתה thou, יהוה וישׁר and upright משׁפטיך׃ thy judgments.}%
\verse{צוית thou hast commanded צדק righteous עדתיך Thy testimonies ואמונה faithful. מאד׃ and very}%
\verse{צמתתני hath consumed קנאתי My zeal כי me, because שׁכחו have forgotten דבריך thy words. צרי׃ mine enemies}%
\verse{צרופה pure: אמרתך Thy word מאד very ועבדך therefore thy servant אהבה׃ loveth}%
\verse{צעיר small אנכי I ונבזה and despised: פקדיך thy precepts. לא do not שׁכחתי׃ I forget}%
\verse{צדקתך Thy righteousness צדק righteousness, לעולם an everlasting ותורתך and thy law אמת׃ the truth.}%
\verse{צר Trouble ומצוק and anguish מצאוני have taken hold on מצותיך me: thy commandments שׁעשׁעי׃ my delights.}%
\verse{צדק The righteousness עדותיך of thy testimonies לעולם everlasting: הבינני give me understanding, ואחיה׃ and I shall live.}%
\verseWithHeading{Qoph}{קראתי I cried בכל with whole לב heart; ענני hear יהוה חקיך thy statutes. אצרה׃ I will keep}%
\verse{קראתיך I cried הושׁיעני unto thee; save ואשׁמרה me, and I shall keep עדתיך׃ thy testimonies.}%
\verse{קדמתי I prevented בנשׁף the dawning ואשׁועה of the morning, and cried: לדבריך in thy word. יחלתי׃ I hoped}%
\verse{קדמו prevent עיני Mine eyes אשׁמרות the watches, לשׂיח that I might meditate באמרתך׃ in thy word.}%
\verse{קולי my voice שׁמעה Hear כחסדך according unto thy lovingkindness: יהוה כמשׁפטך me according to thy judgment. חיני׃ quicken}%
\verse{קרבו They draw nigh רדפי that follow after זמה mischief: מתורתך from thy law. רחקו׃ they are far}%
\verse{קרוב near, אתה Thou יהוה וכל and all מצותיך thy commandments אמת׃ truth.}%
\verse{קדם of old ידעתי I have known מעדתיך כי that לעולם them forever. יסדתם׃ thou hast founded}%
\verseWithHeading{Resh}{ראה Consider עניי mine affliction, וחלצני and deliver כי me: for תורתך thy law. לא I do not שׁכחתי׃ forget}%
\verse{ריבה Plead ריבי my cause, וגאלני and deliver לאמרתך me according to thy word. חיני׃ me: quicken}%
\verse{רחוק far מרשׁעים from the wicked: ישׁועה Salvation כי for חקיך thy statutes. לא not דרשׁו׃ they seek}%
\verse{רחמיך thy tender mercies, רבים Great יהוה כמשׁפטיך me according to thy judgments. חיני׃ quicken}%
\verse{רבים Many רדפי my persecutors וצרי and mine enemies; מעדותיך from thy testimonies. לא do I not נטיתי׃ decline}%
\verse{ראיתי I beheld בגדים the transgressors, ואתקוטטה and was grieved; אשׁר because אמרתך thy word. לא not שׁמרו׃ they kept}%
\verse{ראה Consider כי how פקודיך thy precepts: אהבתי I love יהוה כחסדך according to thy lovingkindness. חיני׃ quicken}%
\verse{ראשׁ the beginning: דברך Thy word אמת true ולעולם forever. כל and every one משׁפט judgments צדקך׃ of thy righteous}%
\verseWithHeading{Sin/Shin}{שׂרים Princes רדפוני have persecuted חנם me without a cause: ומדבריך of thy word. פחד standeth in awe לבי׃ but my heart}%
\verse{שׂשׂ rejoice אנכי I על at אמרתך thy word, כמוצא as one that findeth שׁלל spoil. רב׃ great}%
\verse{שׁקר lying: שׂנאתי I hate ואתעבה and abhor תורתך thy law אהבתי׃ do I love.}%
\verse{שׁבע Seven ביום times a day הללתיך do I praise על thee because of משׁפטי judgments. צדקך׃ thy righteous}%
\verse{שׁלום peace רב Great לאהבי have they which love תורתך thy law: ואין and nothing למו מכשׁול׃ shall offend}%
\verse{שׂברתי I have hoped לישׁועתך for thy salvation, יהוה ומצותיך thy commandments. עשׂיתי׃ and done}%
\verse{שׁמרה hath kept נפשׁי My soul עדתיך thy testimonies; ואהבם and I love מאד׃ them exceedingly.}%
\verse{שׁמרתי I have kept פקודיך thy precepts ועדתיך and thy testimonies: כי for כל all דרכי my ways נגדך׃ before}%
\verseWithHeading{Taw}{תקרב come near רנתי Let my cry לפניך before יהוה כדברך according to thy word. הבינני׃ give me understanding}%
\verse{תבוא come תחנתי Let my supplication לפניך before כאמרתך me according to thy word. הצילני׃ thee: deliver}%
\verse{תבענה shall utter שׂפתי My lips תהלה praise, כי when תלמדני thou hast taught חקיך׃ me thy statutes.}%
\verse{תען shall speak לשׁוני My tongue אמרתך of thy word: כי for כל all מצותיך thy commandments צדק׃ righteousness.}%
\verse{תהי Let ידך thine hand לעזרני help כי me; for פקודיך thy precepts. בחרתי׃ I have chosen}%
\verse{תאבתי I have longed לישׁועתך for thy salvation, יהוה ותורתך and thy law שׁעשׁעי׃ my delight.}%
\verse{תחי live, נפשׁי Let my soul ותהללך and it shall praise ומשׁפטך thee; and let thy judgments יעזרני׃ help}%
\verse{תעיתי I have gone astray כשׂה sheep; אבד like a lost בקשׁ seek עבדך thy servant; כי for מצותיך thy commandments. לא I do not שׁכחתי׃ forget}%
\end{biblechapter}%
\begin{biblechapter}% Psalm 120
\verseWithHeading{Prayer for Deliverance from Enemies}{שׁיר A Song המעלות of degrees. אל unto יהוה the LORD, בצרתה In my distress לי קראתי I cried ויענני׃ and he heard}%
\verse{יהוה O LORD, הצילה Deliver נפשׁי my soul, משׂפת lips, שׁקר from lying מלשׁון tongue. רמיה׃ from a deceitful}%
\verse{מה What יתן shall be given לך ומה unto thee? or what יסיף shall be done לך לשׁון tongue? רמיה׃ unto thee, thou false}%
\verse{חצי arrows גבור of the mighty, שׁנונים Sharp עם with גחלי coals רתמים׃ of juniper.}%
\verse{אויה Woe לי כי is me, that גרתי I sojourn משׁך in Mesech, שׁכנתי I dwell עם in אהלי the tents קדר׃ of Kedar!}%
\verse{רבת hath long שׁכנה dwelt לה נפשׁי My soul עם with שׂונא him that hateth שׁלום׃ peace.}%
\verse{אני I שׁלום peace: וכי but when אדבר I speak, המה they למלחמה׃ for war.}%
\end{biblechapter}%
\begin{biblechapter}% Psalm 121
\verseWithHeading{Trust in God’s Protection}{שׁיר A Song למעלות of degrees. אשׂא I will lift up עיני mine eyes אל unto ההרים the hills, מאין from whence יבא cometh עזרי׃ my help.}%
\verse{עזרי My help מעם from יהוה the LORD, עשׂה which made שׁמים heaven וארץ׃ and earth.}%
\verse{אל He will not יתן suffer למוט to be moved: רגלך thy foot אל thee will not ינום slumber. שׁמרך׃ he that keepeth}%
\verse{הנה Behold, לא shall neither ינום slumber ולא nor יישׁן sleep. שׁומר he that keepeth ישׂראל׃ Israel}%
\verse{יהוה The LORD שׁמרך thy keeper: יהוה the LORD צלך thy shade על upon יד hand. ימינך׃ thy right}%
\verse{יומם thee by day, השׁמשׁ The sun לא shall not יככה smite וירח nor the moon בלילה׃ by night.}%
\verse{יהוה The LORD ישׁמרך shall preserve מכל thee from all רע evil: ישׁמר he shall preserve את נפשׁך׃ thy soul.}%
\verse{יהוה The LORD ישׁמר shall preserve צאתך thy going out ובואך and thy coming in מעתה from this time forth, ועד and even forevermore. עולם׃ and even forevermore.}%
\end{biblechapter}%
\begin{biblechapter}% Psalm 122
\verseWithHeading{Jerusalem the Site of God’s Presence}{שׁיר A Song המעלות of degrees לדוד of David. שׂמחתי I was glad באמרים when they said לי בית into the house יהוה of the LORD. נלך׃ unto me, Let us go}%
\verse{עמדות stand היו shall רגלינו Our feet בשׁעריך within thy gates, ירושׁלם׃ O Jerusalem.}%
\verse{ירושׁלם Jerusalem הבנויה is built כעיר as a city שׁחברה that is compact לה יחדו׃ together:}%
\verse{שׁשׁם עלו go up, שׁבטים the tribes שׁבטי the tribes יה of the LORD, עדות unto the testimony לישׂראל of Israel, להדות to give thanks לשׁם unto the name יהוה׃ of the LORD.}%
\verse{כי For שׁמה there ישׁבו are set כסאות thrones למשׁפט of judgment, כסאות the thrones לבית of the house דויד׃ of David.}%
\verse{שׁאלו Pray שׁלום for the peace ירושׁלם of Jerusalem: ישׁליו they shall prosper אהביך׃ that love}%
\verse{יהי be שׁלום Peace בחילך within thy walls, שׁלוה prosperity בארמנותיך׃ within thy palaces.}%
\verse{למען For אחי my brethren ורעי and companions' אדברה say, נא sakes, I will now שׁלום׃ Peace}%
\verse{למען Because of בית the house יהוה of the LORD אלהינו our God אבקשׁה I will seek טוב׃ thy good.}%
\end{biblechapter}%
\begin{biblechapter}% Psalm 123
\verseWithHeading{Prayer for Adonai’s Action in the Face of Scorn}{שׁיר A Song המעלות of degrees. אליך Unto נשׂאתי thee lift I up את עיני mine eyes, הישׁבי O thou that dwellest בשׁמים׃ in the heavens.}%
\verse{הנה Behold, כעיני as the eyes עבדים of servants אל unto יד the hand אדוניהם of their masters, כעיני as the eyes שׁפחה of a maiden אל unto יד the hand גברתה of her mistress; כן so עינינו our eyes אל upon יהוה the LORD אלהינו our God, עד until שׁיחננו׃ that he have mercy upon}%
\verse{חננו Have mercy upon יהוה us, O LORD, חננו have mercy upon כי us: for רב we are exceedingly שׂבענו filled בוז׃ with contempt.}%
\verse{רבת is exceedingly שׂבעה filled לה נפשׁנו Our soul הלעג with the scorning השׁאננים of those that are at ease, הבוז with the contempt לגאיונים׃ of the proud.}%
\end{biblechapter}%
\begin{biblechapter}% Psalm 124
\verseWithHeading{Thanksgiving for Adonai’s Help}{שׁיר A Song המעלות of degrees לדוד of David. לולי If יהוה the LORD שׁהיה who was לנו יאמר say; נא on our side, now ישׂראל׃ may Israel}%
\verse{לולי If יהוה the LORD שׁהיה who was לנו בקום rose up עלינו against אדם׃ on our side, when men}%
\verse{אזי Then חיים quick, בלעונו they had swallowed us up בחרות was kindled אפם׃ when their wrath}%
\verse{אזי Then המים the waters שׁטפונו had overwhelmed נחלה us, the stream עבר had gone על over נפשׁנו׃ our soul:}%
\verse{אזי Then עבר had gone על over נפשׁנו our soul. המים waters הזידונים׃ the proud}%
\verse{ברוך Blessed יהוה the LORD, שׁלא who hath not נתננו given טרף us a prey לשׁניהם׃ to their teeth.}%
\verse{נפשׁנו Our soul כצפור as a bird נמלטה is escaped מפח out of the snare יוקשׁים of the fowlers: הפח the snare נשׁבר is broken, ואנחנו and we נמלטנו׃ are escaped.}%
\verse{עזרנו Our help בשׁם in the name יהוה of the LORD, עשׂה who made שׁמים heaven וארץ׃ and earth.}%
\end{biblechapter}%
\begin{biblechapter}% Psalm 125
\verseWithHeading{Confidence in Adonai’s Protection}{שׁיר A Song המעלות of degrees. הבטחים They that trust ביהוה in the LORD כהר as mount ציון Zion, לא cannot ימוט be removed, לעולם forever. ישׁב׃ abideth}%
\verse{ירושׁלם Jerusalem, הרים As the mountains סביב round about לה ויהוה so the LORD סביב round about לעמו his people מעתה from henceforth ועד even forever. עולם׃ even forever.}%
\verse{כי For לא shall not ינוח rest שׁבט the rod הרשׁע of the wicked על upon גורל the lot הצדיקים of the righteous; למען lest לא lest ישׁלחו put forth הצדיקים the righteous בעולתה unto iniquity. ידיהם׃ their hands}%
\verse{היטיבה יהוה O LORD, לטובים unto good, ולישׁרים and upright בלבותם׃ in their hearts.}%
\verse{והמטים As for such as turn aside עקלקלותם unto their crooked ways, יוליכם shall lead them forth יהוה the LORD את with פעלי the workers האון of iniquity: שׁלום peace על upon ישׂראל׃ Israel.}%
\end{biblechapter}%
\begin{biblechapter}% Psalm 126
\verseWithHeading{A Prayer for Restoration}{שׁיר A Song המעלות of degrees. בשׁוב turned again יהוה When the LORD את שׁיבת the captivity ציון of Zion, היינו we were כחלמים׃ like them that dream.}%
\verse{אז Then ימלא filled שׂחוק with laughter, פינו was our mouth ולשׁוננו and our tongue רנה with singing: אז then יאמרו said בגוים they among the heathen, הגדיל great things יהוה The LORD לעשׂות hath done עם for אלה׃ them.}%
\verse{הגדיל great things יהוה The LORD לעשׂות hath done עמנו for היינו us; we are שׂמחים׃ glad.}%
\verse{שׁובה Turn again יהוה O LORD, את שׁבותנו our captivity, כאפיקים as the streams בנגב׃ in the south.}%
\verse{הזרעים They that sow בדמעה in tears ברנה in joy. יקצרו׃ shall reap}%
\verse{הלוך ילךבכה and weepeth, נשׂא bearing משׁך precious הזרע seed, בא shall doubtless come again יבוא shall doubtless come again ברנה with rejoicing, נשׂא bringing אלמתיו׃ his sheaves}%
\end{biblechapter}%
\begin{biblechapter}% Psalm 127
\verseWithHeading{A Prayer for Protection and Prosperity}{שׁיר A Song המעלות of degrees לשׁלמה for Solomon. אם it: except יהוה the LORD לא it: except יבנה build בית the house, שׁוא in vain עמלו they labor בוניו that build בו אם יהוה the LORD לא ישׁמר keep עיר the city, שׁוא in vain. שׁקד waketh שׁומר׃ the watchman}%
\verse{שׁוא vain לכם משׁכימי early, קום for you to rise up מאחרי late, שׁבת to sit up אכלי to eat לחם the bread העצבים of sorrows: כן so יתן he giveth לידידו his beloved שׁנא׃ sleep.}%
\verse{הנה Lo, נחלת a heritage יהוה of the LORD: בנים children שׂכר reward. פרי the fruit הבטן׃ of the womb}%
\verse{כחצים As arrows ביד in the hand גבור of a mighty כן man; so בני children הנעורים׃ of the youth.}%
\verse{אשׁרי Happy הגבר the man אשׁר that מלא full את אשׁפתו hath his quiver מהם לא them: they shall not יבשׁו be ashamed, כי but ידברו they shall speak את with אויבים the enemies בשׁער׃ in the gate.}%
\end{biblechapter}%
\begin{biblechapter}% Psalm 128
\verseWithHeading{Blessed Is Everyone Who Fears Adonai}{שׁיר A Song המעלות of degrees. אשׁרי Blessed כל every one ירא that feareth יהוה ההלך that walketh בדרכיו׃ in his ways.}%
\verse{יגיע the labor כפיך of thine hands: כי For תאכל thou shalt eat אשׁריך happy וטוב׃ thou and well}%
\verse{אשׁתך Thy wife כגפן vine פריה as a fruitful בירכתי by the sides ביתך of thine house: בניך thy children כשׁתלי plants זיתים like olive סביב round about לשׁלחנך׃ thy table.}%
\verse{הנה Behold, כי that כן thus יברך be blessed גבר shall the man ירא that feareth יהוה׃ the LORD.}%
\verse{יברכך shall bless יהוה The LORD מציון וראה and thou shalt see בטוב the good ירושׁלם of Jerusalem כל all ימי the days חייך׃ of thy life.}%
\verse{וראה Yea, thou shalt see בנים thy children's לבניך children, שׁלום peace על upon ישׂראל׃ Israel.}%
\end{biblechapter}%
\begin{biblechapter}% Psalm 129
\verseWithHeading{Victory Over the Enemies of Zion}{שׁיר A Song המעלות of degrees. רבת Many a time צררוני have they afflicted מנעורי me from my youth, יאמר say: נא now ישׂראל׃ may Israel}%
\verse{רבת Many a time צררוני have they afflicted מנעורי me from my youth: גם yet לא they have not יכלו׃ prevailed}%
\verse{על upon גבי my back: חרשׁו The plowers חרשׁים plowed האריכו they made long למענותם׃ their furrows.}%
\verse{יהוה The LORD צדיק righteous: קצץ he hath cut asunder עבות the cords רשׁעים׃ of the wicked.}%
\verse{יבשׁו be confounded ויסגו and turned אחור back כל Let them all שׂנאי that hate ציון׃ Zion.}%
\verse{יהיו Let them be כחציר as the grass גגות the housetops, שׁקדמת before שׁלף it groweth up: יבשׁ׃ which withereth}%
\verse{שׁלא not מלא filleth כפו his hand; קוצר Wherewith the mower וחצנו sheaves his bosom. מעמר׃ nor he that bindeth}%
\verse{ולא Neither אמרו say, העברים do they which go by ברכת The blessing יהוה of the LORD אליכם upon ברכנו you: we bless אתכם בשׁם you in the name יהוה׃ of the LORD.}%
\end{biblechapter}%
\begin{biblechapter}% Psalm 130
\verseWithHeading{Hope for the Redemption of Adonai}{שׁיר A Song המעלות of degrees. ממעמקים קראתיך have I cried יהוה׃ unto thee, O LORD.}%
\verse{אדני Lord, שׁמעה hear בקולי my voice: תהיינה be אזניך let thine ears קשׁבות attentive לקול to the voice תחנוני׃ of my supplications.}%
\verse{אם If עונות iniquities, תשׁמר shouldest mark יה thou, LORD, אדני O Lord, מי who יעמד׃ shall stand?}%
\verse{כי But עמך with הסליחה forgiveness למען thee, that תורא׃ thou mayest be feared.}%
\verse{קויתי I wait for יהוה the LORD, קותה doth wait, נפשׁי my soul ולדברו and in his word הוחלתי׃ do I hope.}%
\verse{נפשׁי My soul לאדני for the Lord משׁמרים more than they that watch לבקר for the morning: שׁמרים they that watch לבקר׃ for the morning.}%
\verse{יחל hope ישׂראל Let Israel אל in יהוה the LORD: כי for עם with יהוה the LORD החסד mercy, והרבה him plenteous עמו and with פדות׃ redemption.}%
\verse{והוא And he יפדה shall redeem את ישׂראל Israel מכל from all עונתיו׃ his iniquities.}%
\end{biblechapter}%
\begin{biblechapter}% Psalm 131
\verseWithHeading{Calm Trust in Adonai}{שׁיר A Song המעלות of degrees לדוד of David. יהוה LORD, לא is not גבה haughty, לבי my heart ולא nor רמו lofty: עיני mine eyes ולא neither הלכתי do I exercise בגדלות myself in great matters, ובנפלאות or in things too high ממני׃ for}%
\verse{אם לאׁויתי I have behaved ודוממתי and quieted נפשׁי myself, כגמל as a child that is weaned עלי of אמו his mother: כגמל even as a weaned child. עלי נפשׁי׃ my soul}%
\verse{יחל hope ישׂראל Let Israel אל in יהוה the LORD מעתה from henceforth ועד and forever. עולם׃ and forever.}%
\end{biblechapter}%
\begin{biblechapter}% Psalm 132
\verseWithHeading{Adonai Dwells in Zion}{שׁיר A Song המעלות of degrees. זכור remember יהוה LORD, לדוד David, את כל all ענותו׃ his afflictions:}%
\verse{אשׁר How נשׁבע he swore ליהוה unto the LORD, נדר vowed לאביר unto the mighty יעקב׃ of Jacob;}%
\verse{אם Surely אבא I will not come באהל into the tabernacle ביתי of my house, אם nor אעלה go up על into ערשׂ my bed; יצועי׃ my bed;}%
\verse{אם I will not אתן give שׁנת sleep לעיני to mine eyes, לעפעפי to mine eyelids, תנומה׃ slumber}%
\verse{עד Until אמצא I find out מקום a place ליהוה for the LORD, משׁכנות a habitation לאביר for the mighty יעקב׃ of Jacob.}%
\verse{הנה Lo, שׁמענוה we heard באפרתה of it at Ephratah: מצאנוה we found בשׂדי it in the fields יער׃ of the wood.}%
\verse{נבואה We will go למשׁכנותיו into his tabernacles: נשׁתחוה we will worship להדם at his footstool. רגליו׃ at his footstool.}%
\verse{קומה Arise, יהוה O LORD, למנוחתך into thy rest; אתה thou, וארון and the ark עזך׃ of thy strength.}%
\verse{כהניך Let thy priests ילבשׁו be clothed צדק with righteousness; וחסידיך and let thy saints ירננו׃ shout for joy.}%
\verse{בעבור sake דוד David's עבדך For thy servant אל turn not away תשׁב turn not away פני the face משׁיחך׃ of thine anointed.}%
\verse{נשׁבע hath sworn יהוה The LORD לדוד unto David; אמת truth לא he will not ישׁוב turn ממנה from מפרי בטנך of thy body אשׁית will I set לכסא׃ upon thy throne.}%
\verse{אם If ישׁמרו will keep בניך thy children בריתי my covenant ועדתי and my testimony זו אלמדם I shall teach גם shall also בניהם them, their children עדי forevermore. עד ישׁבו sit לכסא׃ upon thy throne}%
\verse{כי For בחר hath chosen יהוה the LORD בציון Zion; אוה he hath desired למושׁב׃ for his habitation.}%
\verse{זאת This מנוחתי my rest עדי forever: עד פה here אשׁב will I dwell; כי for אותיה׃ I have desired}%
\verse{צידה her provision: ברך אברךביוניה her poor אשׂביע I will satisfy לחם׃ with bread.}%
\verse{וכהניה her priests אלבישׁ I will also clothe ישׁע with salvation: וחסידיה and her saints רנן shall shout aloud for joy. ירננו׃ shall shout aloud for joy.}%
\verse{שׁם There אצמיח to bud: קרן will I make the horn לדוד of David ערכתי I have ordained נר a lamp למשׁיחי׃ for mine anointed.}%
\verse{אויביו His enemies אלבישׁ will I clothe בשׁת with shame: ועליו but upon יציץ flourish. נזרו׃ himself shall his crown}%
\end{biblechapter}%
\begin{biblechapter}% Psalm 133
\verseWithHeading{The People of God Dwell in Unity}{שׁיר A Song המעלות of degrees לדוד of David. הנה Behold, מה how טוב good ומה and how נעים pleasant שׁבת to dwell אחים for brethren גם יחד׃}%
\verse{כשׁמן ointment הטוב like the precious על upon הראשׁ the head, ירד that ran down על upon הזקן the beard, זקן beard: אהרן Aaron's שׁירד that went down על to פי the skirts מדותיו׃ of his garments;}%
\verse{כטל As the dew חרמון of Hermon, שׁירד that descended על upon הררי the mountains ציון of Zion: כי for שׁם there צוה commanded יהוה the LORD את הברכה the blessing, חיים life עד forevermore. העולם׃ forevermore.}%
\end{biblechapter}%
\begin{biblechapter}% Psalm 134
\verseWithHeading{Praising Adonai in the Temple at Night}{שׁיר A Song המעלות of degrees. הנה Behold, ברכו bless את יהוה ye the LORD, כל all עבדי servants יהוה of the LORD, העמדים stand בבית in the house יהוה of the LORD. בלילות׃ which by night}%
\verse{שׂאו Lift up ידכם your hands קדשׁ the sanctuary, וברכו and bless את יהוה׃ the LORD.}%
\verse{יברכך bless יהוה The LORD מציון עשׂה that made שׁמים heaven וארץ׃ and earth}%
\end{biblechapter}%
\begin{biblechapter}% Psalm 135
\verseWithHeading{Praise to God for His Power and Redemption}{הללו Praise יה ye the LORD. הללו Praise את שׁם ye the name יהוה of the LORD; הללו praise עבדי O ye servants יהוה׃ of the LORD.}%
\verse{שׁעמדים בבית in the house יהוה of the Lord, בחצרות in the courts בית of the house אלהינו׃ of our God,}%
\verse{הללו Praise יה the LORD; כי for טוב good: יהוה the LORD זמרו sing praises לשׁמו unto his name; כי for נעים׃ pleasant.}%
\verse{כי For יעקב Jacob בחר hath chosen לו יה the LORD ישׂראל unto himself, Israel לסגלתו׃ for his peculiar treasure.}%
\verse{כי For אני I ידעתי know כי that גדול great, יהוה the LORD ואדנינו מכל above all אלהים׃ gods.}%
\verse{כל and all אשׁר חפץ pleased, יהוה the LORD עשׂה did בשׁמים he in heaven, ובארץ and in earth, בימים in the seas, וכל תהומות׃ deep places.}%
\verse{מעלה to ascend נשׂאים He causeth the vapors מקצה from the ends הארץ of the earth; ברקים lightnings למטר for the rain; עשׂה he maketh מוצא he bringeth רוח the wind מאוצרותיו׃ out of his treasuries.}%
\verse{שׁהכה בכורי the firstborn מצרים of Egypt, מאדם both of man עד and בהמה׃ beast.}%
\verse{שׁלח sent אתות tokens ומפתים and wonders בתוככי into the midst מצרים of thee, O Egypt, בפרעה upon Pharaoh, ובכל and upon all עבדיו׃ his servants.}%
\verse{שׁהכה גוים nations, רבים great והרג and slew מלכים kings; עצומים׃ mighty}%
\verse{לסיחון Sihon מלך king האמרי of the Amorites, ולעוג and Og מלך king הבשׁן of Bashan, ולכל and all ממלכות the kingdoms כנען׃ of Canaan:}%
\verse{ונתן And gave ארצם their land נחלה a heritage, נחלה a heritage לישׂראל unto Israel עמו׃ his people.}%
\verse{יהוה O LORD, שׁמך Thy name, לעולם forever; יהוה O LORD, זכרך thy memorial, לדר throughout all generations. ודר׃ throughout all generations.}%
\verse{כי For ידין will judge יהוה the LORD עמו his people, ועל concerning עבדיו his servants. יתנחם׃ and he will repent himself}%
\verse{עצבי The idols הגוים of the heathen כסף silver וזהב and gold, מעשׂה the work ידי hands. אדם׃ of men's}%
\verse{פה They have mouths, להם ולא not; ידברו but they speak עינים eyes להם ולא not; יראו׃ have they, but they see}%
\verse{אזנים They have ears, להם ולא not; יאזינו but they hear אף neither אין neither ישׁ is there רוח breath בפיהם׃ in their mouths.}%
\verse{כמוהם like unto them: יהיו them are עשׂיהם They that make כל every one אשׁר that בטח׃ trusteth}%
\verse{בית O house ישׂראל of Israel: ברכו Bless את יהוה the LORD, בית O house אהרן of Aaron: ברכו bless את יהוה׃ the LORD,}%
\verse{בית O house הלוי of Levi: ברכו Bless את יהוה the LORD, יראי ye that fear יהוה the LORD, ברכו bless את יהוה׃ the LORD.}%
\verse{ברוך Blessed יהוה be the LORD מציון שׁכן which dwelleth ירושׁלם at Jerusalem. הללו Praise יה׃ ye the LORD.}%
\end{biblechapter}%
\begin{biblechapter}% Psalm 136
\verseWithHeading{Praise to God for His Creation and Deliverance}{הודו O give thanks ליהוה unto the LORD; כי for טוב good: כי for לעולם forever. חסדו׃ his mercy}%
\verse{הודו O give thanks לאלהי unto the God האלהים of gods: כי for לעולם forever. חסדו׃ his mercy}%
\verse{הודו O give thanks לאדני to the Lord האדנים of lords: כי for לעולם forever. חסדו׃ his mercy}%
\verse{לעשׂה doeth נפלאות wonders: גדלות great לבדו To him who alone כי for לעולם forever. חסדו׃ his mercy}%
\verse{לעשׂה made השׁמים the heavens: בתבונה To him that by wisdom כי for לעולם forever. חסדו׃ his mercy}%
\verse{לרקע To him that stretched out הארץ the earth על above המים the waters: כי for לעולם forever. חסדו׃ his mercy}%
\verse{לעשׂה To him that made אורים lights: גדלים great כי for לעולם forever: חסדו׃ his mercy}%
\verse{את השׁמשׁ The sun לממשׁלת to rule ביום by day: כי for לעולם forever: חסדו׃ his mercy}%
\verse{את הירח The moon וכוכבים and stars לממשׁלות to rule בלילה by night: כי for לעולם forever. חסדו׃ his mercy}%
\verse{למכה To him that smote מצרים Egypt בבכוריהם in their firstborn: כי for לעולם forever: חסדו׃ his mercy}%
\verse{ויוצא And brought out ישׂראל Israel מתוכם from among כי them: for לעולם forever: חסדו׃ his mercy}%
\verse{ביד hand, חזקה With a strong ובזרוע arm: נטויה and with a stretched out כי for לעולם forever. חסדו׃ his mercy}%
\verse{לגזר To him which divided ים sea סוף the Red לגזרים into parts: כי for לעולם forever: חסדו׃ his mercy}%
\verse{והעביר to pass through ישׂראל And made Israel בתוכו the midst כי of it: for לעולם forever: חסדו׃ his mercy}%
\verse{ונער פרעה Pharaoh וחילו and his host בים sea: סוף in the Red כי for לעולם forever. חסדו׃ his mercy}%
\verse{למוליך To him which led עמו his people במדבר through the wilderness: כי for לעולם forever. חסדו׃ his mercy}%
\verse{למכה To him which smote מלכים kings: גדלים great כי for לעולם forever: חסדו׃ his mercy}%
\verse{ויהרג And slew מלכים kings: אדירים famous כי for לעולם forever: חסדו׃ his mercy}%
\verse{לסיחון Sihon מלך king האמרי of the Amorites: כי for לעולם forever: חסדו׃ his mercy}%
\verse{ולעוג And Og מלך the king הבשׁן of Bashan: כי for לעולם forever: חסדו׃ his mercy}%
\verse{ונתן And gave ארצם their land לנחלה for a heritage: כי for לעולם forever: חסדו׃ his mercy}%
\verse{נחלה a heritage לישׂראל unto Israel עבדו his servant: כי for לעולם forever. חסדו׃ his mercy}%
\verse{שׁבשׁפלנו us in our low estate: זכר Who remembered לנו כי for לעולם forever: חסדו׃ his mercy}%
\verse{ויפרקנו And hath redeemed מצרינו us from our enemies: כי for לעולם forever. חסדו׃ his mercy}%
\verse{נתן Who giveth לחם food לכל to all בשׂר flesh: כי for לעולם forever. חסדו׃ his mercy}%
\verse{הודו O give thanks לאל unto the God השׁמים of heaven: כי for לעולם forever. חסדו׃ his mercy}%
\end{biblechapter}%
\begin{biblechapter}% Psalm 137
\verseWithHeading{Lament During the Babylonian Exile}{על By נהרות the rivers בבל of Babylon, שׁם there ישׁבנו we sat down, גם yea, בכינו we wept, בזכרנו when we remembered את ציון׃ Zion.}%
\verse{על upon ערבים the willows בתוכה in the midst תלינו We hanged כנרותינו׃ our harps}%
\verse{כי For שׁם there שׁאלונו required שׁובינו they that carried us away captive דברי of us a song; שׁיר of us a song; ותוללינו and they that wasted שׂמחה us mirth, שׁירו Sing לנו משׁיר us of the songs ציון׃ of Zion.}%
\verse{איך How נשׁיר shall we sing את שׁיר song יהוה the LORD's על in אדמת land? נכר׃ a strange}%
\verse{אם If אשׁכחך I forget ירושׁלם thee, O Jerusalem, תשׁכח forget ימיני׃ let my right hand}%
\verse{תדבק cleave לשׁוני thee, let my tongue לחכי to the roof of my mouth; אם If לא I do not אזכרכי remember אם if לא not אעלה I prefer את ירושׁלם Jerusalem על above ראשׁ my chief שׂמחתי׃ joy.}%
\verse{זכר Remember, יהוה O LORD, לבני the children אדום of Edom את יום in the day ירושׁלם of Jerusalem; האמרים who said, ערו Raze ערו raze עד to היסוד׃ the foundation}%
\verse{בת O daughter בבל of Babylon, השׁדודה who art to be destroyed; אשׁרי happy שׁישׁלם , that rewardeth לך את גמולך thee as שׁגמלת׃ thou hast served}%
\verse{אשׁרי Happy שׁיאחז , that taketh ונפץ and dasheth את עלליך thy little ones אל against הסלע׃ the stones.}%
\end{biblechapter}%
\begin{biblechapter}% Psalm 138
\verseWithHeading{Thanksgiving for Adonai’s Goodness}{לדוד of David. אודך I will praise בכל thee with my whole לבי heart: נגד before אלהים the gods אזמרך׃ will I sing praise}%
\verse{אשׁתחוה I will worship אל toward היכל temple, קדשׁך thy holy ואודה and praise את שׁמך thy name על and for חסדך thy lovingkindness ועל above אמתך thy truth: כי for הגדלת thou hast magnified על כל all שׁמך thy name. אמרתך׃ thy word}%
\verse{ביום In the day קראתי when I cried ותענני thou answeredst תרהבני me, strengthenedst בנפשׁי in my soul. עז׃ me strength}%
\verse{יודוך shall praise יהוה thee, O LORD, כל All מלכי the kings ארץ of the earth כי when שׁמעו they hear אמרי the words פיך׃ of thy mouth.}%
\verse{וישׁירו Yea, they shall sing בדרכי in the ways יהוה of the LORD: כי for גדול great כבוד the glory יהוה׃ of the LORD.}%
\verse{כי Though רם high, יהוה the LORD ושׁפל unto the lowly: יראה yet hath he respect וגבה but the proud ממרחק afar off. יידע׃ he knoweth}%
\verse{אם Though אלך I walk בקרב in the midst צרה of trouble, תחיני thou wilt revive על against אף the wrath איבי of mine enemies, תשׁלח me: thou shalt stretch forth ידך thine hand ותושׁיעני shall save ימינך׃ and thy right hand}%
\verse{יהוה The LORD יגמר will perfect בעדי concerneth יהוה O LORD, חסדך me: thy mercy, לעולם forever: מעשׂי the works ידיך of thine own hands. אל not תרף׃ forsake}%
\end{biblechapter}%
\begin{biblechapter}% Psalm 139
\verseWithHeading{The Knowledge of God}{למנצח To the chief Musician, לדוד of David. מזמור A Psalm יהוה O LORD, חקרתני thou hast searched ותדע׃ me, and known}%
\verse{אתה Thou ידעת knowest שׁבתי my downsitting וקומי and mine uprising, בנתה thou understandest לרעי my thought מרחוק׃ afar off.}%
\verse{ארחי my path ורבעי and my lying down, זרית Thou compassest וכל all דרכי my ways. הסכנתה׃ and art acquainted}%
\verse{כי For אין not מלה a word בלשׁוני in my tongue, הן lo, יהוה O LORD, ידעת thou knowest כלה׃ it altogether.}%
\verse{אחור me behind וקדם and before, צרתני Thou hast beset ותשׁת and laid עלי upon כפכה׃ thine hand}%
\verse{פלאיה too wonderful דעת knowledge ממני for נשׂגבה me; it is high, לא אוכל׃}%
\verse{אנה Whither אלך shall I go מרוחך from thy spirit? ואנה or whither מפניך from thy presence? אברח׃ shall I flee}%
\verse{אם If אסק I ascend up שׁמים into heaven, שׁם there: אתה thou ואציעה if I make my bed שׁאול in hell, הנך׃}%
\verse{אשׂא I take כנפי the wings שׁחר of the morning, אשׁכנה dwell באחרית in the uttermost parts ים׃ of the sea;}%
\verse{גם Even שׁם there ידך shall thy hand תנחני lead ותאחזני shall hold ימינך׃ me, and thy right hand}%
\verse{ואמר If I say, אך Surely חשׁך the darkness ישׁופני shall cover ולילה me; even the night אור shall be light בעדני׃ about}%
\verse{גם Yea, חשׁך the darkness לא not יחשׁיך hideth ממך from ולילה thee; but the night כיום as the day: יאיר shineth כחשׁיכה the darkness כאורה׃ and the light}%
\verse{כי For אתה thou קנית hast possessed כליתי my reins: תסכני thou hast covered בבטן womb. אמי׃ me in my mother's}%
\verse{אודך I will praise על thee; for כי thee; for נוראות I am fearfully נפליתי wonderfully נפלאים made: marvelous מעשׂיך thy works; ונפשׁי and my soul ידעת knoweth מאד׃ right well.}%
\verse{לא was not נכחד hid עצמי My substance ממך from אשׁר thee, when עשׂיתי I was made בסתר in secret, רקמתי curiously wrought בתחתיות in the lowest parts ארץ׃ of the earth.}%
\verse{גלמי my substance, yet being unperfect; ראו did see עיניך Thine eyes ועל and in ספרך thy book כלם all יכתבו were written, ימים in continuance יצרו were fashioned, ולא when none אחד׃ when none}%
\verse{ולי מה How יקרו precious רעיך also are thy thoughts אל unto me, O God! מה how עצמו great ראשׁיהם׃ is the sum}%
\verse{אספרם I should count מחול than the sand: ירבון them, they are more in number הקיצתי when I awake, ועודי I am still עמך׃ with thee.}%
\verse{אם Surely תקטל thou wilt slay אלוה O God: רשׁע the wicked, ואנשׁי דמים me therefore, ye bloody סורו depart מני׃ from}%
\verse{אשׁר For יאמרך they speak למזמה against thee wickedly, נשׂא take לשׁוא in vain. עריך׃ thine enemies}%
\verse{הלוא Do not משׂנאיך I hate יהוה them, O LORD, אשׂנא that hate ובתקוממיך with those that rise up against אתקוטט׃ thee? and am not I grieved}%
\verse{תכלית them with perfect שׂנאה hatred: שׂנאתים I hate לאויבים them mine enemies. היו׃ I count}%
\verse{חקרני Search אל me, O God, ודע and know לבבי my heart: בחנני try ודע me, and know שׂרעפי׃ my thoughts:}%
\verse{וראה And see אם if דרך way עצב wicked בי ונחני in me, and lead בדרך me in the way עולם׃ everlasting.}%
\end{biblechapter}%
\begin{biblechapter}% Psalm 140
\verseWithHeading{Prayer for Help in the Face of Enemies}{למנצח To the chief Musician, מזמור A Psalm לדוד׃ of David. חלצני Deliver יהוה me, O LORD, מאדם man: רע from the evil מאישׁ man; חמסים me from the violent תנצרני׃ preserve}%
\verse{אשׁר Which חשׁבו imagine רעות mischiefs בלב in heart; כל continually יום continually יגורו are they gathered together מלחמות׃ war.}%
\verse{שׁננו They have sharpened לשׁונם their tongues כמו like נחשׁ a serpent; חמת poison עכשׁוב adders' תחת under שׂפתימו their lips. סלה׃ Selah.}%
\verse{שׁמרני Keep יהוה me, O LORD, מידי from the hands רשׁע of the wicked; מאישׁ חמסיםנצרני preserve אשׁר who חשׁבו have purposed לדחות to overthrow פעמי׃ my goings.}%
\verse{טמנו have hid גאים The proud פח a snare לי וחבלים for me, and cords; פרשׂו they have spread רשׁת a net ליד by the wayside; מעגל by the wayside; מקשׁים gins שׁתו they have set לי סלה׃ for me. Selah.}%
\verse{אמרתי I said ליהוה unto the LORD, אלי my God: אתה Thou האזינה hear יהוה O LORD. קול the voice תחנוני׃ of my supplications,}%
\verse{יהוה O GOD אדני the Lord, עז the strength ישׁועתי of my salvation, סכתה thou hast covered לראשׁי my head ביום in the day נשׁק׃ of battle.}%
\verse{אל not, תתן Grant יהוה O LORD, מאויי the desires רשׁע of the wicked: זממו his wicked device; אל not תפק further ירומו they exalt סלה׃ themselves. Selah.}%
\verse{ראשׁ the head מסבי of those that compass me about, עמל let the mischief שׂפתימו of their own lips יכסומו׃ cover}%
\verse{ימיטו fall עליהם upon גחלים Let burning coals באשׁ into the fire; יפלם them: let them be cast במהמרות into deep pits, בל not יקומו׃ that they rise}%
\verse{אישׁ an evil speaker לשׁון an evil speaker בל Let not יכון be established בארץ in the earth: אישׁ man חמס the violent רע evil יצודנו shall hunt למדחפת׃ to overthrow}%
\verse{ידעת I know כי that יעשׂה will maintain יהוה the LORD דין the cause עני of the afflicted, משׁפט the right אבינים׃ of the poor.}%
\verse{אך Surely צדיקים the righteous יודו shall give thanks לשׁמך unto thy name: ישׁבו shall dwell ישׁרים the upright את פניך׃ in thy presence.}%
\end{biblechapter}%
\begin{biblechapter}% Psalm 141
\verseWithHeading{Prayer for God’s Help in Maintaining Integrity}{מזמור A Psalm לדוד of David. יהוה LORD, קראתיך I cry חושׁה unto thee: make haste לי האזינה unto me; give ear קולי unto my voice, בקראי׃ when I cry}%
\verse{תכון be set forth תפלתי Let my prayer קטרת thee incense; לפניך before משׂאת the lifting up כפי of my hands מנחת sacrifice. ערב׃ the evening}%
\verse{שׁיתה Set יהוה O LORD, שׁמרה a watch, לפי before my mouth; נצרה keep על keep דל the door שׂפתי׃ of my lips.}%
\verse{אל not תט Incline לבי my heart לדבר thing, רע to evil להתעולל to practice עללות works ברשׁע wicked את with אישׁים פעלי that work און iniquity: ובל and let me not אלחם eat במנעמיהם׃ of their dainties.}%
\verse{יהלמני smite צדיק Let the righteous חסד me; a kindness: ויוכיחני and let him reprove שׁמן oil, ראשׁ me; an excellent אל shall not יני break ראשׁי my head: כי for עוד yet ותפלתי my prayer ברעותיהם׃ also in their calamities.}%
\verse{נשׁמטו are overthrown בידי places, סלע in stony שׁפטיהם When their judges ושׁמעו they shall hear אמרי my words; כי for נעמו׃ they are sweet.}%
\verse{כמו as when פלח one cutteth ובקע and cleaveth בארץ upon the earth. נפזרו are scattered עצמינו Our bones לפי mouth, שׁאול׃ at the grave's}%
\verse{כי But אליך unto יהוה אדני the Lord: עיני mine eyes בכה חסיתי in thee is my trust; אל תער leave not my soul destitute. נפשׁי׃}%
\verse{שׁמרני Keep מידי me from פח the snares יקשׁו they have laid לי ומקשׁות for me, and the gins פעלי of the workers און׃ of iniquity.}%
\verse{יפלו fall במכמריו into their own nets, רשׁעים Let the wicked יחד withal אנכי that I עד whilst אעבור׃ escape.}%
\end{biblechapter}%
\begin{biblechapter}% Psalm 142
\verseWithHeading{A Prayer for Deliverance from Pursuers}{משׂכיל Maschil לדוד of David; בהיותו when he was במערה in the cave. תפלה׃ A Prayer קולי with my voice; אל unto יהוה the LORD אזעק I cried קולי with my voice אל unto יהוה the LORD אתחנן׃ did I make my supplication.}%
\verse{אשׁפך I poured out לפניו before שׂיחי my complaint צרתי him my trouble. לפניו before אגיד׃ him; I showed}%
\verse{בהתעטף was overwhelmed עלי within רוחי When my spirit ואתה me, then thou ידעת knewest נתיבתי my path. בארח In the way זו wherein אהלך I walked טמנו have they privily laid פח׃ a snare}%
\verse{הביט I looked ימין on right hand, וראה and beheld, ואין but no man לי מכיר that would know אבד failed מנוס me: refuge ממני failed אין me; no man דורשׁ cared לנפשׁי׃ for my soul.}%
\verse{זעקתי I cried אליך unto יהוה thee, O LORD: אמרתי I said, אתה Thou מחסי my refuge חלקי my portion בארץ in the land החיים׃ of the living.}%
\verse{הקשׁיבה Attend אל unto רנתי my cry; כי for דלותי מאדצילני deliver מרדפי me from my persecutors; כי for אמצו they are stronger ממני׃ me from my persecutors;}%
\verse{הוציאה ממסגר of prison, נפשׁי להודות that I may praise את שׁמך thy name: בי יכתרו shall compass me about; צדיקים the righteous כי for תגמל thou shalt deal bountifully עלי׃ with}%
\end{biblechapter}%
\begin{biblechapter}% Psalm 143
\verseWithHeading{A Prayer for Rescue from Enemies}{מזמור A Psalm לדוד of David. יהוה O LORD, שׁמע Hear תפלתי my prayer, האזינה give ear אל to תחנוני my supplications: באמנתך in thy faithfulness ענני answer בצדקתך׃ me, in thy righteousness.}%
\verse{ואל not תבוא And enter במשׁפט into judgment את with עבדך thy servant: כי for לא shall no יצדק be justified. לפניך in thy sight כל man חי׃ living}%
\verse{כי For רדף hath persecuted אויב the enemy נפשׁי my soul; דכא he hath smitten לארץ down to the ground; חיתי my life הושׁיבני he hath made me to dwell במחשׁכים in darkness, כמתי dead. עולם׃ as those that have been long}%
\verse{ותתעטף overwhelmed עלי within רוחי Therefore is my spirit בתוכי within ישׁתומם me is desolate. לבי׃ me; my heart}%
\verse{זכרתי I remember ימים the days מקדם of old; הגיתי I meditate בכל on all פעלך thy works; במעשׂה on the work ידיך of thy hands. אשׂוחח׃ I muse}%
\verse{פרשׂתי I stretch forth ידי my hands אליך unto thee: נפשׁי my soul כארץ land. עיפה after thee, as a thirsty לך סלה׃ Selah.}%
\verse{מהר ענני Hear יהוה O LORD: כלתה faileth: רוחי my spirit אל not תסתר hide פניך thy face ממני from ונמשׁלתי me, lest I be like עם unto ירדי them that go down בור׃ into the pit.}%
\verse{השׁמיעני Cause me to hear בבקר in the morning; חסדך thy lovingkindness כי for בך בטחתי in thee do I trust: הודיעני cause me to know דרך the way זו wherein אלך I should walk; כי for אליך unto נשׂאתי I lift up נפשׁי׃ my soul}%
\verse{הצילני Deliver מאיבי from mine enemies: יהוה me, O LORD, אליך כסתי׃}%
\verse{למדני Teach לעשׂות me to do רצונך thy will; כי for אתה thou אלוהי רוחך thy spirit טובה good; תנחני lead בארץ me into the land מישׁור׃ of uprightness.}%
\verse{למען שׁמךהוה me, O LORD, תחיני Quicken בצדקתך for thy righteousness' תוציא sake bring my soul out מצרה of trouble. נפשׁי׃ sake bring my soul out}%
\verse{ובחסדך And of thy mercy תצמית cut off איבי mine enemies, והאבדת and destroy כל all צררי them that afflict נפשׁי my soul: כי for אני I עבדך׃ thy servant.}%
\end{biblechapter}%
\begin{biblechapter}% Psalm 144
\verseWithHeading{A Prayer for National Safety}{לדוד of David. ברוך Blessed יהוה the LORD צורי my strength, המלמד which teacheth ידי my hands לקרב to war, אצבעותי my fingers למלחמה׃ to fight:}%
\verse{חסדי My goodness, ומצודתי and my fortress; משׂגבי my high tower, ומפלטי and my deliverer; לי מגני my shield, ובו חסיתי and in whom I trust; הרודד who subdueth עמי my people תחתי׃ under}%
\verse{יהוה מה what אדם man, ותדעהו that thou takest knowledge בן of him! the son אנושׁ ותחשׁבהו׃ that thou makest account}%
\verse{אדם Man להבל to vanity: דמה is like ימיו his days כצל as a shadow עובר׃ that passeth away.}%
\verse{יהוה O LORD, הט Bow שׁמיך thy heavens, ותרד and come down: גע touch בהרים the mountains, ויעשׁנו׃ and they shall smoke.}%
\verse{ברוק Cast forth ברק lightning, ותפיצם and scatter שׁלח them: shoot out חציך thine arrows, ותהמם׃ and destroy}%
\verse{שׁלח Send ידיך thine hand ממרום from above; פצני rid והצילני me, and deliver ממים waters, רבים me out of great מיד from the hand בני children; נכר׃ of strange}%
\verse{אשׁר Whose פיהם mouth דבר speaketh שׁוא vanity, וימינם and their right hand ימין a right hand שׁקר׃ of falsehood.}%
\verse{אלהים unto thee, O God: שׁיר I will sing חדשׁ a new אשׁירה song לך בנבל upon a psaltery עשׂור an instrument of ten strings אזמרה׃ will I sing praises}%
\verse{הנותן that giveth תשׁועה salvation למלכים unto kings: הפוצה who delivereth את דוד David עבדו his servant מחרב sword. רעה׃ from the hurtful}%
\verse{פצני Rid והצילני me, and deliver מיד me from the hand בני children, נכר of strange אשׁר whose פיהם mouth דבר speaketh שׁוא vanity, וימינם and their right hand ימין a right hand שׁקר׃ of falsehood:}%
\verse{אשׁר That בנינו our sons כנטעים as plants מגדלים grown up בנעוריהם in their youth; בנותינו our daughters כזוית as corner stones, מחטבות polished תבנית the similitude היכל׃ of a palace:}%
\verse{מזוינו our garners מלאים full, מפיקים affording מזן אלן צאוננו our sheep מאליפות may bring forth thousands מרבבות and ten thousands בחוצותינו׃ in our streets:}%
\verse{אלופינו our oxen מסבלים strong to labor; אין no פרץ breaking in, ואין nor יוצאת going out ואין ; that no צוחה complaining ברחבתינו׃ in our streets.}%
\verse{אשׁרי Happy העם people, שׁככה that is in such a case: לו אשׁרי happy העם people, שׁיהוה אלהיו׃ whose God}%
\end{biblechapter}%
\begin{biblechapter}% Psalm 145
\verseWithHeading{A Song of God’s Majesty and Love}{תהלה of praise. לדוד David's ארוממך I will extol אלוהי המלך O king; ואברכה and I will bless שׁמך thy name לעולם forever ועד׃ and ever.}%
\verse{בכל Every יום day אברכך will I bless ואהללה thee; and I will praise שׁמך thy name לעולם forever ועד׃ and ever.}%
\verse{גדול Great יהוה the LORD, ומהלל to be praised; מאד and greatly ולגדלתו and his greatness אין unsearchable. חקר׃ unsearchable.}%
\verse{דור One generation לדור to another, ישׁבח shall praise מעשׂיך thy works וגבורתיך thy mighty acts. יגידו׃ and shall declare}%
\verse{הדר honor כבוד of the glorious הודך of thy majesty, ודברי works. נפלאותיך and of thy wondrous אשׂיחה׃ I will speak}%
\verse{ועזוז of the might נוראתיך of thy terrible acts: יאמרו And shall speak וגדולתיך thy greatness. אספרנה׃ and I will declare}%
\verse{זכר the memory רב of thy great טובך goodness, יביעו They shall abundantly utter וצדקתך of thy righteousness. ירננו׃ and shall sing}%
\verse{חנון gracious, ורחום and full of compassion; יהוה The LORD ארך slow אפים to anger, וגדל and of great חסד׃ mercy.}%
\verse{טוב good יהוה The LORD לכל to all: ורחמיו and his tender mercies על over כל all מעשׂיו׃ his works.}%
\verse{יודוך shall praise יהוה thee, O LORD; כל All מעשׂיך thy works וחסידיך and thy saints יברכוכה׃ shall bless}%
\verse{כבוד of the glory מלכותך of thy kingdom, יאמרו They shall speak וגבורתך of thy power; ידברו׃ and talk}%
\verse{להודיע To make known לבני to the sons האדם of men גבורתיו his mighty acts, וכבוד and the glorious הדר majesty מלכותו׃ of his kingdom.}%
\verse{מלכותך Thy kingdom מלכות kingdom, כל an everlasting עלמים an everlasting וממשׁלתך and thy dominion בכל throughout all דור generations. ודור׃ generations.}%
\verse{סומך upholdeth יהוה The LORD לכל all הנפלים that fall, וזוקף and raiseth up לכל all הכפופים׃ bowed down.}%
\verse{עיני The eyes כל of all אליך upon ישׂברו wait ואתה thee; and thou נותן givest להם אתכלם them their meat בעתו׃ in due season.}%
\verse{פותח Thou openest את ידך thine hand, ומשׂביע and satisfiest לכל of every חי living thing. רצון׃ the desire}%
\verse{צדיק righteous יהוה The LORD בכל in all דרכיו his ways, וחסיד and holy בכל in all מעשׂיו׃ his works.}%
\verse{קרוב nigh יהוה The LORD לכל unto all קראיו them that call לכל upon him, to all אשׁר that יקראהו call באמת׃ upon him in truth.}%
\verse{רצון the desire יראיו of them that fear יעשׂה He will fulfill ואת שׁועתם their cry, ישׁמע him: he also will hear ויושׁיעם׃ and will save}%
\verse{שׁומר preserveth יהוה The LORD את כל all אהביו them that love ואת כל him: but all הרשׁעים the wicked ישׁמיד׃ will he destroy.}%
\verse{תהלת the praise יהוה of the LORD: ידבר shall speak פי My mouth ויברך bless כל and let all בשׂר flesh שׁם name קדשׁו his holy לעולם forever ועד׃ and ever.}%
\end{biblechapter}%
\begin{biblechapter}% Psalm 146
\verseWithHeading{Praise to Adonai for His Help}{הללו Praise יה ye the LORD. הללי Praise נפשׁי O my soul. את יהוה׃ the LORD,}%
\verse{אהללה will I praise יהוה the LORD: בחיי While I live אזמרה I will sing praises לאלהי unto my God בעודי׃ while I have any being.}%
\verse{אל Put not תבטחו your trust בנדיבים in princes, בבן in the son אדם of man, שׁאין in whom no לו תשׁועה׃ help.}%
\verse{תצא goeth forth, רוחו His breath ישׁב he returneth לאדמתו to his earth; ביום day ההוא in that very אבדו perish. עשׁתנתיו׃ his thoughts}%
\verse{אשׁרי Happy שׁאל יעקב of Jacob בעזרו for his help, שׂברו whose hope על in יהוה the LORD אלהיו׃ his God:}%
\verse{עשׂה Which made שׁמים heaven, וארץ and earth, את הים the sea, ואת כל and all אשׁר that בם השׁמר therein which keepeth אמת truth לעולם׃ forever:}%
\verse{עשׂה Which executeth משׁפט judgment לעשׁוקים for the oppressed: נתן which giveth לחם food לרעבים to the hungry. יהוה The LORD מתיר looseth אסורים׃ the prisoners:}%
\verse{יהוה The LORD פקח openeth עורים the blind: יהוה the LORD זקף raiseth כפופים them that are bowed down: יהוה the LORD אהב loveth צדיקים׃ the righteous:}%
\verse{יהוה The LORD שׁמר preserveth את גרים the strangers; יתום the fatherless ואלמנה and widow: יעודד he relieveth ודרך but the way רשׁעים of the wicked יעות׃ he turneth upside down.}%
\verse{ימלך shall reign יהוה The LORD לעולם forever, אלהיך thy God, ציון O Zion, לדר unto all generations. ודר unto all generations. הללו Praise יה׃ ye the LORD.}%
\end{biblechapter}%
\begin{biblechapter}% Psalm 147
\verseWithHeading{Praise to Adonai for His Providence}{הללו Praise יה ye the LORD: כי for טוב good זמרה to sing praises אלהינו unto our God; כי for נעים pleasant; נאוה is comely. תהלה׃ praise}%
\verse{בונה doth build up ירושׁלם Jerusalem: יהוה The LORD נדחי the outcasts ישׂראל of Israel. יכנס׃ he gathereth together}%
\verse{הרפא He healeth לשׁבורי the broken לב in heart, ומחבשׁ and bindeth up לעצבותם׃ their wounds.}%
\verse{מונה He telleth מספר the number לכוכבים of the stars; לכלם them all שׁמות by names. יקרא׃ he calleth}%
\verse{גדול Great אדונינו our Lord, ורב and of great כח power: לתבונתו his understanding אין infinite. מספר׃ infinite.}%
\verse{מעודד lifteth up ענוים the meek: יהוה The LORD משׁפיל he casteth the wicked down רשׁעים he casteth the wicked down עדי ארץ׃ the ground.}%
\verse{ענו Sing ליהוה unto the LORD בתודה with thanksgiving; זמרו sing praise לאלהינו unto our God: בכנור׃ upon the harp}%
\verse{המכסה Who covereth שׁמים the heaven בעבים with clouds, המכין who prepareth לארץ for the earth, מטר rain המצמיח to grow הרים upon the mountains. חציר׃ who maketh grass}%
\verse{נותן He giveth לבהמה to the beast לחמה his food, לבני to the young ערב ravens אשׁר which יקראו׃ cry.}%
\verse{לא not בגבורת in the strength הסוס of the horse: יחפץ He delighteth לא he taketh not pleasure בשׁוקי in the legs האישׁ of a man. ירצה׃ he taketh not pleasure}%
\verse{רוצה taketh pleasure in יהוה The LORD את יראיו them that fear את המיחלים him, in those that hope לחסדו׃ in his mercy.}%
\verse{שׁבחי Praise ירושׁלם O Jerusalem; את יהוה the LORD, הללי praise אלהיך thy God, ציון׃ O Zion.}%
\verse{כי For חזק he hath strengthened בריחי the bars שׁעריך of thy gates; ברך he hath blessed בניך thy children בקרבך׃ within}%
\verse{השׂם He maketh גבולך thy borders, שׁלום peace חלב thee with the finest חטים of the wheat. ישׂביעך׃ filleth}%
\verse{השׁלח He sendeth forth אמרתו his commandment ארץ earth: עד very מהרה swiftly. ירוץ runneth דברו׃ his word}%
\verse{הנתן He giveth שׁלג snow כצמר like wool: כפור the hoarfrost כאפר like ashes. יפזר׃ he scattereth}%
\verse{משׁליך He casteth forth קרחו his ice כפתים like morsels: לפני before קרתו his cold? מי who יעמד׃ can stand}%
\verse{ישׁלח He sendeth out דברו his word, וימסם and melteth ישׁב to blow, רוחו them: he causeth his wind יזלו flow. מים׃ the waters}%
\verse{מגיד He showeth דברו his word ליעקב unto Jacob, חקיו his statutes ומשׁפטיו and his judgments לישׂראל׃ unto Israel.}%
\verse{לא He hath not עשׂה dealt כן so לכל with any גוי nation: ומשׁפטים and judgments, בל they have not ידעום known הללו them. Praise יה׃ ye the LORD.}%
\end{biblechapter}%
\begin{biblechapter}% Psalm 148
\verseWithHeading{Let All Creation Praise Adonai}{הללו Praise יה ye the LORD. הללו Praise את יהוה ye the LORD מן from השׁמים the heavens: הללוהו praise במרומים׃ him in the heights.}%
\verse{הללוהו Praise כל ye him, all מלאכיו his angels: הללוהו praise כל ye him, all צבאו׃ his hosts.}%
\verse{הללוהו Praise שׁמשׁ ye him, sun וירח and moon: הללוהו praise כל him, all כוכבי ye stars אור׃ of light.}%
\verse{הללוהו Praise שׁמי him, ye heavens השׁמים of heavens, והמים and ye waters אשׁר that מעל above השׁמים׃ the heavens.}%
\verse{יהללו Let them praise את שׁם the name יהוה of the LORD: כי for הוא he צוה commanded, ונבראו׃ and they were created.}%
\verse{ויעמידם He hath also established לעד them forever לעולם and ever: חק a decree נתן he hath made ולא which shall not יעבור׃ pass.}%
\verse{הללו Praise את יהוה the LORD מן from הארץ the earth, תנינים ye dragons, וכל and all תהמות׃ deeps:}%
\verse{אשׁ Fire, וברד and hail; שׁלג snow, וקיטור and vapor; רוח wind סערה stormy עשׂה fulfilling דברו׃ his word:}%
\verse{ההרים Mountains, וכל and all גבעות hills; עץ trees, פרי fruitful וכל and all ארזים׃ cedars:}%
\verse{החיה Beasts, וכל and all בהמה cattle; רמשׂ creeping things, וצפור fowl: כנף׃ and flying}%
\verse{מלכי Kings ארץ of the earth, וכל and all לאמים people; שׂרים princes, וכל and all שׁפטי judges ארץ׃ of the earth:}%
\verse{בחורים Both young men, וגם and בתולות maidens; זקנים old men, עם and נערים׃ children:}%
\verse{יהללו Let them praise את שׁם the name יהוה of the LORD: כי for נשׂגב is excellent; שׁמו his name לבדו alone הודו his glory על above ארץ the earth ושׁמים׃ and heaven.}%
\verse{וירם He also exalteth קרן the horn לעמו of his people, תהלה the praise לכל of all חסידיו his saints; לבני of the children ישׂראל of Israel, עם a people קרבו near הללו unto him. Praise יה׃ ye the LORD.}%
\end{biblechapter}%
\begin{biblechapter}% Psalm 149
\verseWithHeading{Praise to God for His Future Judgment}{הללו Praise יה ye the LORD. שׁירו Sing ליהוה unto the LORD שׁיר song, חדשׁ a new תהלתו his praise בקהל in the congregation חסידים׃ of saints.}%
\verse{ישׂמח rejoice ישׂראל Let Israel בעשׂיו in him that made בני him: let the children ציון of Zion יגילו be joyful במלכם׃ in their King.}%
\verse{יהללו Let them praise שׁמו his name במחול in the dance: בתף unto him with the timbrel וכנור and harp. יזמרו׃ let them sing praises}%
\verse{כי For רוצה taketh pleasure יהוה the LORD בעמו in his people: יפאר he will beautify ענוים the meek בישׁועה׃ with salvation.}%
\verse{יעלזו be joyful חסידים Let the saints בכבוד in glory: ירננו let them sing aloud על upon משׁכבותם׃ their beds.}%
\verse{רוממות אל of God בגרונם in their mouth, וחרב sword פיפיות and a twoedged בידם׃ in their hand;}%
\verse{לעשׂות To execute נקמה vengeance בגוים upon the heathen, תוכחת punishments בל אמים׃ upon the people;}%
\verse{לאסר To bind מלכיהם their kings בזקים with chains, ונכבדיהם and their nobles בכבלי with fetters ברזל׃ of iron;}%
\verse{לעשׂות To execute בהם משׁפט upon them the judgment כתוב written: הדר honor הוא this לכל have all חסידיו his saints. הללו Praise יה׃ ye the LORD.}%
\end{biblechapter}%
\begin{biblechapter}% Psalm 150
\verseWithHeading{Let Everything Praise Adonai}{הללו Praise יה ye the LORD. הללו Praise אל God בקדשׁו in his sanctuary: הללוהו praise ברקיע him in the firmament עזו׃ of his power.}%
\verse{הללוהו Praise בגבורתיו him for his mighty acts: הללוהו praise כרב him according to his excellent גדלו׃ greatness.}%
\verse{הללוהו Praise בתקע him with the sound שׁופר of the trumpet: הללוהו praise בנבל him with the psaltery וכנור׃ and harp.}%
\verse{הללוהו Praise בתף him with the timbrel ומחול and dance: הללוהו praise במנים him with stringed instruments ועוגב׃ and organs.}%
\verse{הללוהו Praise בצלצלי cymbals: שׁמע him upon the loud הללוהו praise בצלצלי cymbals. תרועה׃ him upon the high sounding}%
\verse{כל Let every הנשׁמה thing that hath breath תהלל praise יה the LORD. הללו Praise יה׃ ye the LORD.}%
\end{biblechapter}%
\flushcolsend
\biblebook{Proverbs}
\begin{biblechapter}% Proverbs 1
\verseWithHeading{Prologue}{Proverbs of Solomon, son of David, king of Israel:}%
\verse{To know wisdom and instruction, to understand sayings of understanding,}%
\verse{to gain insightful instruction,\lebnote{“the instruction of insight”} righteousness and justice and equity,}%
\verse{to give shrewdness\lebnote{Or “cleverness,” or “prudence”} to the simple, knowledge and purpose\lebnote{Or “plan”} to the young,}%
\verse{may the wise hear and increase learning, and the one who understands gain direction,\lebnote{Or “guidance”}}%
\verse{to understand a proverb and an expression, words of wisdom and their riddles.}%
\verse{Fear of Adonai is the beginning of knowledge; wisdom and instruction, fools despise.}%
\verseWithHeading{Lecture Against Gang Behavior}{My child, may you keep\lebnote{Or “obey,” or “hear”} your father’s instruction, and do not reject your mother’s teachings,}%
\verse{for they are a garland of favor for your head, and pendants for your neck.}%
\verse{My child, if sinners entice you, do not consent.}%
\verse{If they say, “Come with us! We shall lie in wait for blood; we shall ambush the innocent without cause.\lebnote{Or “wantonly,” or “for nothing”}}%
\verse{Like Sheol,\lebnote{A term for the place where the dead reside, i.e., the Underworld} we will swallow them alive and whole, like those who descend to the pit.\lebnote{“like the descenders of a pit”}}%
\verse{We shall find all precious wealth, we shall fill our houses with booty,}%
\verse{you shall throw your lot in our midst, there will be one purse for all of us.”}%
\verse{My child, do not walk in their way.\lebnote{Or “on their road”} Keep your foot from their paths,}%
\verse{for their feet run to evil, and they hurry to shed blood,}%
\verse{for “in vain\lebnote{Or “without cause,” or “for nothing”} is the net scattered, in the sight of any winged bird.”\lebnote{“bird of wing”}}%
\verse{They lie in wait for their own blood. They ambush their own lives.}%
\verse{Thus are the ways of all who are greedy for gain — it will take the life of its possessors.}%
\verseWithHeading{The Call of Lady Wisdom}{Wisdom calls out in the streets, in the squares she raises her voice.}%
\verse{On a busy corner\lebnote{“head of commotion,” or “head of stirring, bustling”} she cries out, at the entrances of the gates in the city, she speaks her sayings:}%
\verse{“How long,\lebnote{“Until when”} O simple ones, will you love simplicity? And how long will scoffers delight in their scoffing, and fools hate knowledge?}%
\verse{May you turn to my argument!\lebnote{Or “turn at my reproach”} Behold, I shall pour out my spirit upon you; I will make my words known to you.}%
\verse{Because I called out and you refused me, I stretched out my hand, yet there is none who heeds.}%
\verse{You have ignored all my counsel, and my reproof you are not willing to accept.}%
\verse{I will also laugh at your calamity;\lebnote{Or “downfall”} I will mock when panic\lebnote{Or “dread, fear”} comes upon you.}%
\verse{When your panic comes like a storm, and your calamity arrives like a whirlwind, when distress and anguish come upon you,}%
\verse{then they will call\lebnote{Or “cry out \textit{to}”} me, but I will not answer; they will seek me diligently but not find me.}%
\verse{Since they hated knowledge, and did not choose the fear of Adonai,}%
\verse{they were not willing to accept my counsel, they despised all my reproof.}%
\verse{They shall eat from the fruit of their way, and they shall be sated from their own schemes,}%
\verse{for the waywardness of the simple ones will kill them, and the complacency of fools will destroy them.}%
\verse{Whoever listens to me will dwell in security and rest securely\lebnote{Or “be at ease, peace”} from dread and disaster.”}%
\end{biblechapter}%
\begin{biblechapter}% Proverbs 2
\verseWithHeading{The Benefits of Wisdom}{My child, if you will receive\lebnote{Or “take, seize”} my sayings, and hide my commands with you,}%
\verse{in order to incline your ear toward wisdom, then you shall apply your heart to understanding.}%
\verse{For if you cry out for understanding,\lebnote{Or “call to understanding”} if you lift\lebnote{Or “carry, give”} your voice for insight,}%
\verse{if you seek her like silver and search her out like treasure,\lebnote{“like the treasures”}}%
\verse{then you will understand the fear of Adonai, and the knowledge of God you will find.}%
\verse{For Adonai will give wisdom; from his mouth come knowledge and understanding.}%
\verse{For the upright, he stores\lebnote{Or “hides, keeps”} sound judgment, a shield for those who walk uprightly,}%
\verse{in order to guard paths of justice and keep\lebnote{Or “preserve, guard, watch”} the way of his faithful ones.}%
\verse{Then you will understand righteousness and justice and uprightness — every good course —}%
\verse{for wisdom will enter your heart, and knowledge will be pleasing to your soul.\lebnote{Or “inner self”}}%
\verse{Discretion will watch over you; understanding will protect you,}%
\verse{in order to deliver you from the way of evil,\lebnote{Or “evildoer”} from a man who speaks devious things —}%
\verse{those who forsake the paths of uprightness to walk in ways of darkness,}%
\verse{those who are happy to do evil, for they delight in the deviousness of evil,}%
\verse{who are crooked in their ways,\lebnote{“that \textit{in} their ways are crooked”} and devious in their paths;}%
\verse{in order to deliver you from a strange woman, from a foreign woman who flatters\lebnote{Or “makes smooth”} with her sayings,}%
\verse{she who forsakes the partner of her youth and has forgotten the covenant of her God,}%
\verse{for her house sinks to death, and to the dead\lebnote{Or “Rephaim”} are her paths.}%
\verse{Of all who go to her, none shall return, nor do they reach paths of life.}%
\verse{So that you will walk on the road of those who are good, and the paths of those who are righteous you shall keep.}%
\verse{For those who are upright will dwell in the land,\lebnote{Or “earth”} and those who are blameless\lebnote{Or “innocent”} will remain in it.}%
\verse{And those who are wicked will be cut off from the land, and those who are treacherous will be uprooted from it.}%
\end{biblechapter}%
\begin{biblechapter}% Proverbs 3
\verseWithHeading{Wisdom and Piety}{My child, do not forget my instruction, and may your heart guard my commands.}%
\verse{For length of days, years of life, and peace they\lebnote{That is, “my commands” (3:1)} shall add to you.}%
\verse{May loyal love\lebnote{Or “kindness, mercy”} and truth not forsake you; bind them around your neck, write them upon your heart.}%
\verse{And you shall find favor and good sense\lebnote{Or “understanding, prudence”} in the eyes of God and humankind.}%
\verse{Trust Adonai\lebnote{“trust toward Adonai”} with all your heart; do not lean toward your own understanding.}%
\verse{In all your ways acknowledge him, and he will straighten your paths.}%
\verse{Do not be wise in your own eyes; fear Adonai and retreat from evil.}%
\verse{There shall be healing for your flesh,\lebnote{“navel”} and refreshment for your body.}%
\verse{Honor Adonai from your substance,\lebnote{Or “wealth, property”} and from the firstfruits of all that will come to you,}%
\verse{and your barns shall be full of plenty, and your vats shall burst with new wine.}%
\verseWithHeading{Happy Is the One Who Finds Wisdom}{Do not despise the discipline of Adonai, my child. Do not be weary of his reproof}%
\verse{because whomever Adonai will love, he will rebuke, as a father delights in his son.}%
\verse{Happy is the one\lebnote{“a man”} who finds wisdom, and one who obtains understanding.}%
\verse{For her\lebnote{Or “its”; both “wisdom” and “understanding” are feminine nouns and can be read as either “she” or “it”} income is better than the income of silver, and her revenue than that of gold.}%
\verse{She is more precious than rubies, and all your desires shall not compare with her.}%
\verse{Length of days is in her right hand; in her left hand are riches and honor.}%
\verse{Her ways are ways of pleasantness, and all her paths are peace.}%
\verse{She is a tree of life for those who seize her; those who take hold of her are considered happy.}%
\verseWithHeading{Role of Wisdom in Creation and Society}{Adonai in wisdom founded the earth; he established the heavens in understanding.}%
\verse{With his knowledge, depths broke open, and clouds dropped dew.}%
\verse{My child, may they\lebnote{Grammatically, “they” most likely refers to clouds (3:20). In context, “they” may refer to wisdom and understanding (3:13).} not escape from your sight;\lebnote{“eyes”} may you keep sound wisdom and prudence.}%
\verse{They shall be life for your soul\lebnote{Or “inner self”} and adornment\lebnote{Or “favor”} for your neck.}%
\verse{Then you will walk in the confidence of your ways, and your foot will not stumble.}%
\verse{If you sit down, you will not panic,\lnBTH{} and if you lie down, then your sleep shall be sweet.}%
\verse{Do not be afraid of sudden panic,\lnBTH{} or the storm of wickedness that will come.}%
\verse{Adonai will be your confidence\lebnote{“in your confidence”} and guard your foot from capture.}%
\verse{Do not withhold good from its owner\lebnote{Or “lord”} when it is in the power of your hand to do.}%
\verse{Do not say to your neighbor, “Go and return and tomorrow I will give it,” when it is with you.\lebnote{“and there is with you”}}%
\verse{Do not plot harm against your neighbor who dwells in confidence beside you.}%
\verse{Do not quarrel with anyone without cause, when he did not do you harm.}%
\verse{Do not envy a man of violence, and do not choose any of his ways,}%
\verse{for he who is perverse is an abomination of Adonai, but those who are upright are his confidence.}%
\verse{The curse of Adonai is on the house of the wicked, and the abode of the righteous ones he blesses.}%
\verse{With those who scorn, he is scornful, but to those who are humble, he gives favor.}%
\verse{They will inherit the honor of the wise, but stubborn fools, disgrace.}%
\end{biblechapter}%
\begin{biblechapter}% Proverbs 4
\verseWithHeading{The Father’s Wisdom}{Children, listen to the instruction of a father, and be attentive in order to know insight.}%
\verse{For I have given you good instruction; do not forsake my teaching.}%
\verse{When I was a son to my father, tender and alone before my mother,}%
\verse{he taught me and said to me, “May your heart hold fast to my words; guard my commandments and live.}%
\verse{Get wisdom and insight; do not forget and do not turn from the sayings of my mouth.}%
\verse{Do not forsake her — then she will guard you; love her and she will keep you.}%
\verse{The beginning of wisdom: Get wisdom! With all that is in your possession, gain insight.}%
\verse{Cherish her and she will exalt you; she will honor you if you embrace her.}%
\verse{She will give a garland for your head; she shall bestow a crown of glory upon you.”}%
\verseWithHeading{The Right Path}{Listen, my child, take my sayings, and the years of your life shall be great.}%
\verse{In the way of wisdom I have instructed you; I have led you in the path of uprightness.}%
\verse{When you walk,\lebnote{“In your walking”} your step will not be hampered, and if you run, you will not stumble.}%
\verse{Seize the instruction! Do not let go! Guard her, for she is your life.}%
\verse{In the path of the wicked do not go; do not walk in the way of those who do evil.}%
\verse{Avoid it; do not transgress it; turn away from it and pass by.}%
\verse{For they will not sleep if they have not done wrong, and they are robbed of their sleep if they do not cause stumbling.}%
\verse{For they ate the bread of wickedness, and they drank the wine of violence.}%
\verse{But the path of the righteous ones is like the light of dawn, leading and shining until the day is full.\lebnote{“\textit{is} correct”}}%
\verse{The way of the wicked ones is like deep darkness; they do not know what they stumble over.}%
\verseWithHeading{Staying the Course}{My child, be attentive to my words; to my sayings incline your ear.}%
\verse{May they not escape from your sight;\lebnote{“eyes”} keep them in the midst of your heart.}%
\verse{For they are life to those who find them and healing to the entire body.\lebnote{“for all his flesh”}}%
\verse{With all vigilance, keep your heart, for from it comes the source\lebnote{Or “extremity”} of life.}%
\verse{Remove from yourself deceitful speech,\lebnote{“crookedness of mouth”} and abolish devious talk\lebnote{“deviousness of lips”} from yourself.}%
\verse{May your eyes look forward\lebnote{Or “opposite”} and your gaze be straight before you.}%
\verse{May the path of your foot be balanced and all your ways be sure.}%
\verse{Do not swerve right or left; remove your foot from evil.}%
\end{biblechapter}%
\begin{biblechapter}% Proverbs 5
\verse{My child, be attentive to my wisdom, and to my understanding incline your ear;}%
\verse{in order to keep prudence, and knowledge will guard your lips.}%
\verse{For the lips of the strange woman will drip honey, and smoother than oil is her mouth.\lebnote{Or “palate”}}%
\verse{But her end is bitter as the wormwood, sharp as a two-edged sword.}%
\verse{Her feet go down to death; her steps take hold of Sheol.\lebnote{A term for the place where the dead reside, i.e., the Underworld}}%
\verse{She does not observe\lebnote{Or “examine, weigh”} the path of life; her ways wander, and she does not know it.}%
\verseWithHeading{Do Not Commit Adultery Against Wisdom}{Now, O children, listen to me; do not depart from the sayings of my mouth.}%
\verse{Keep your paths far from her, and do not go near to the door of her house,}%
\verse{lest you give your honor to the others, and your years to the merciless,}%
\verse{lest strangers take their fill of your strength, and your labors go to the house of a foreigner,}%
\verse{and you groan at your end, when your flesh and body are consumed,}%
\verse{and say “How I hated discipline, and I despised reproof!”}%
\verse{and “I did not listen to the voice of my teachers, and I did not incline my ear to my instructors!}%
\verse{I was almost at utter\lebnote{Or “all, every, whole”} ruin\lebnote{Or “evil”} in the midst of the assembly and congregation.”}%
\verse{Drink water from your own cistern and flowing waters from inside your own well.}%
\verse{Shall your springs be scattered outward? In the streets, shall there be streams of water?}%
\verse{May they be yours alone, and not for strangers who are with you.}%
\verse{May your fountain be blessed, and rejoice in the wife\lebnote{Or “woman”} of your youth.}%
\verse{She is a deer of love and a doe of grace; may her breasts satisfy you always;\lebnote{“at all time”} by her love may you be intoxicated continually.}%
\verse{Why should you be intoxicated, my child, by a strange woman, and embrace the bosom of a foreigner?}%
\verse{For before the eyes of Adonai are human ways,\lebnote{“the ways of man/humankind”} and all his paths he examines.}%
\verse{His iniquities shall ensnare him, the evildoer, and in the vanity of his sin he shall be caught.}%
\verse{He shall die for lack of\lebnote{“with there is no”} discipline, and in the greatness of his folly he shall be lost.}%
\end{biblechapter}%
\begin{biblechapter}% Proverbs 6
\verseWithHeading{Against Pledges}{My child, if you have pledged to your neighbor, if you have bound yourself\lebnote{“palms of your hands”} to the stranger,}%
\verse{if you are snared by the sayings of your mouth, if you are caught by the sayings of your mouth,}%
\verse{do this, then, my child, and save yourself, for you have come into the palm of your neighbor’s hand:\lebnote{“the palm of the hand of your neighbor”} Go, humble yourself, plead with your neighbor.}%
\verse{Do not give sleep to your eyes, or slumber to your eyelids.}%
\verse{Save yourself like a gazelle from a hand, or like a bird from the hand of a fowler.}%
\verseWithHeading{Against Sloth}{Go to the ant, lazy! Consider its ways and be wise.}%
\verse{It has no chief, officer, or ruler.}%
\verse{In the summer, it prepares its food; in the harvest, it gathers its sustenance.}%
\verse{How long will you lie down, lazy? When will you rise up from your sleep?}%
\verse{A little sleep, a little slumber, a little folding of the hands for rest —}%
\verse{like a robber\lebnote{“one who walks,” that is, a vagabond} shall your poverty come, and what you lack like an armed man.}%
\verseWithHeading{Against Worthlessness}{A worthless man, an evil man, goes around with deceitful speech.\lebnote{“crookedness of mouth”}}%
\verse{Winking in his eye, shuffling in his foot, pointing in his fingers,}%
\verse{perversion in his heart, he devises evil; at all times he will send out discord.\lnBTI{}}%
\verse{Upon such a man,\lebnote{“thus”} suddenly shall his calamity come; in a moment he will be damaged and there is no healing.\lebnote{Or “repair”}}%
\verseWithHeading{What Adonai Hates}{There are six things Adonai hates, and seven things are abominations of his soul:\lebnote{Or “inner self”}}%
\verse{haughty eyes, a lying tongue, and hands that shed innocent blood,}%
\verse{a devising heart, plans of deception, feet that hurry to run to evil,\lebnote{Hebrew “the evil”}}%
\verse{a false witness who breathes lies and sends out discord\lnBTI{} between brothers.}%
\verseWithHeading{Commandment and Instruction as Guardians}{My child, keep the commandment of your father, and do not disregard the instruction of your mother.}%
\verse{Bind them on your heart continually; tie them upon your neck.}%
\verse{When you walk,\lebnote{“In your walking”} she\lebnote{That is, “commandment” and “instruction”} will lead you, When you lie down,\lebnote{“In your lying down”} she will watch over you, and when you awake, she will converse with you.}%
\verse{For like a lamp is a commandment, and instruction is light, and the way of life\lebnote{Hebrew “lives”} is the reproof of discipline,}%
\verse{in order to preserve you from an evil woman,\lebnote{Or “wife”} from the smoothness of the tongue of an adulteress.\lebnote{“a foreign woman”}}%
\verse{Do not desire her beauty in your heart; may she not capture you with her eyelashes.}%
\verse{For the price of a woman, a prostitute,\lebnote{Or “whore”} is the price of a loaf of bread, but the woman belonging to a man\lebnote{“the woman of a man”} hunts precious life.}%
\verseWithHeading{Warning Against Relations with a Married Woman}{Can a man carry fire in his lap, and his clothes not burn?}%
\verse{If a man walks upon the hot coals, will his feet not be burned?}%
\verse{Thus, he who goes to the wife of his neighbor, any who touches her shall not go unpunished.}%
\verse{People do not despise a thief when he steals to fill himself\lebnote{Or “soul,” or “inner self”} when he is hungry.}%
\verse{But if he is found, he will pay sevenfold, every possession of his house he shall give.}%
\verse{He who commits adultery with a woman lacks sense,\lebnote{“heart”} he destroys himself\lebnote{Or “his soul,” or “his life”} who does it.}%
\verse{A wound and dishonor he will find, and his disgrace will not be wiped out.}%
\verse{For jealousy is the fury of a husband, and he will not show restraint on the day of revenge.}%
\verse{He will not accept any compensation,\lebnote{“\textit{the} face of any compensation”} and he will not be willing, though the bribe is large.}%
\end{biblechapter}%
\begin{biblechapter}% Proverbs 7
\verseWithHeading{Warning Against the Strange Woman}{My child, guard my sayings; store my commandments with you.}%
\verse{Keep my commands and live, and my teaching like the apple of your eye.\lebnote{A single word meaning “pupil of the eye”}}%
\verse{Bind them on your fingers; write them on the tablet of your heart.}%
\verse{Say to wisdom,\lebnote{Hebrew “the wisdom”} “you are my sister,” and you shall call insight,\lebnote{Hebrew “the insight”} “intimate friend.”\lebnote{“one who is known.” To “know” is often a euphemism for intercourse. Therefore “intimate friend” may also be read “lover.”}}%
\verse{In order to guard yourself from an adulteress,\lebnote{“a strange woman”} from the foreigner who makes her words smooth.\lebnote{“causes to be smooth her words”}}%
\verse{For at the window of my house, through my lattice, I looked down.}%
\verse{And I saw among the simple, I observed among the youth, a young man lacking sense,\lebnote{“heart”}}%
\verse{passing on the street at\lnBTJ{} her corner, and he takes the road to her house,}%
\verse{at twilight, at the day’s evening, in the midst of night and the darkness.}%
\verse{Then behold! A woman comes to meet him with the garment of a prostitute\lebnote{Or “whore”} and a secret heart.\lebnote{“secret of heart”}}%
\verse{She is loud and stubborn; her feet do not stay at her house.}%
\verse{Now in the street, now in the square, at\lnBTJ{} every corner she lies in wait.}%
\verse{She took hold of\lebnote{Or “strengthened”} him and kissed him. Her face was impudent, and she said to him,}%
\verse{“Sacrifices of peace offerings are upon me; today\lebnote{“the day”} I completed my vows.}%
\verse{So\lebnote{Hebrew “thus”} I have come out to meet you, to seek your face, and I have found you.}%
\verse{With coverings I have adorned my couch, spreads of the linen of Egypt;}%
\verse{I have perfumed my bed with myrrh, aloes, and cinnamon.}%
\verse{Come, let us take our fill of love making, until the morning let us delight in love.}%
\verse{For there is no man\lebnote{Or “husband”} in his home; he has gone on a long journey.\lebnote{“a journey from far”}}%
\verse{The bag of money he took in his hand, for on the day of the full moon he will come home.”}%
\verse{She persuades him with the greatness of her teachings; with her smooth lips she compels him.}%
\verse{He goes after her suddenly; like an ox to the slaughter he goes, and like a stag to the instruction of a fool,}%
\verse{until an arrow pierces his entrails,\lebnote{“heaviness,” often referring to the liver} like a bird rushing into a snare, but he does not know that it will cost him his life.\lebnote{“it is against his life”}}%
\verseWithHeading{Reiteration of the Warning Against a Strange Woman}{And now, my children, listen to me, and be attentive to the sayings of my mouth.}%
\verse{May your heart not turn aside to her ways; do not stray into her path.}%
\verse{For many slain she has laid low, and countless\lebnote{Or “strong men”} are all of her killings.}%
\verse{The ways of Sheol\lebnote{A term for the place where the dead reside, i.e., the Underworld} are her house, descending to chambers of death.}%
\end{biblechapter}%
\begin{biblechapter}% Proverbs 8
\verseWithHeading{Wisdom Calls}{Does not wisdom call, and understanding raise its voice?}%
\verse{Atop the heights\lebnote{“At a head of the heights”} beside the road, at the crossroads she stands.}%
\verse{Beside gates, before towns, at the entrance of doors, she cries out:}%
\verse{“To you, O people,\lebnote{“men”} I call, and my cry is to the children of humankind.}%
\verse{Learn prudence, O simple ones; fools, learn intelligence.\lebnote{“heart”}}%
\verse{Listen! For noble things I will speak, and upright things from the opening of my lips.}%
\verse{My mouth will utter truth, and wickedness is an abomination to my lips.}%
\verse{All sayings of my mouth are in righteousness; none of them are twisted and crooked.}%
\verse{All of them are straight to him who understands, and upright to those who find knowledge.}%
\verse{Take my teaching and not silver; may you choose knowledge rather than choice gold.}%
\verse{For wisdom is better than jewels, and all desires shall not compare with her.}%
\verse{I, wisdom, live with prudence, and I find knowledge and discretion.}%
\verse{The fear of Adonai is hatred of evil, pride, and arrogance and an evil way. And I hate a mouth of perversity.}%
\verse{Advice and sound judgment\lebnote{“To/for me are advice and sound judgment”} are mine; I am understanding, strength is mine.\lebnote{“to/for me is strength”}}%
\verse{By me kings reign, and rulers decree righteousness.}%
\verse{By me rulers rule, and nobles — all judges of righteousness.}%
\verse{I love those who love me,\lebnote{“her lovers”} and those who seek me diligently shall find me.}%
\verse{Fortune and glory are with me, enduring wealth and righteousness.}%
\verse{My fruit is better than gold,\lebnote{“fine gold”} even refined gold, and my yield than choice silver.}%
\verse{In the way of righteousness I walk, in the midst of paths of justice,}%
\verse{in order to endow those who love me with wealth, and I will fill their treasuries.}%
\verseWithHeading{Wisdom at Creation}{“Adonai possessed\lebnote{Or “gained,” or “acquired”} me, the first of his ways, before his acts of old.\lebnote{“from before” or “from then”}}%
\verse{From eternity, I was set up from the first, from the beginning\lebnote{Hebrew “beginnings”} of the earth.}%
\verse{When there were no depths, I was brought forth, when there were no springs of abounding\lebnote{“made heavy”} water.}%
\verse{Before mountains had been shaped, before hills, I was brought forth.}%
\verse{When he had not yet made earth and fields, or the first dust of the world,}%
\verse{when he established\lebnote{“in his establishing”} the heavens, there I was, when he drew\lebnote{“in his drawing”} a circle upon the face of the deep,}%
\verse{when he made\lebnote{“in his making”} skies from above, when he founded fountains of the deep,}%
\verse{when he assigned\lebnote{“in his assigning”} his limits to the sea, that waters shall not transgress his command,\lebnote{“mouth”} when he marked\lebnote{“in his marking”} the foundations of the earth,}%
\verse{I was beside\lebnote{“at the place of”} him, a master workman, and I was delighting day by day, rejoicing before him always,\lebnote{“in all \textit{of} time”}}%
\verse{rejoicing in the world of his earth, and my delight was with the children of humankind.}%
\verseWithHeading{Benefits of Following Wisdom}{“And now, children, listen to me; happy are those who will keep my ways.}%
\verse{Hear teaching and be wise; do not neglect it.}%
\verse{Happy is the person who listens to me, in order to keep watch at my doors day by day, in order to guard the frames of my entrances.}%
\verse{For he who finds me is he who finds\lebnote{Hebrew “those who find”} life, and he obtains favor from Adonai.}%
\verse{But he who misses me injures himself.\lebnote{Or “soul,” or “inner self”} All those who hate me love death.”}%
\end{biblechapter}%
\begin{biblechapter}% Proverbs 9
\verseWithHeading{Wisdom’s Banquet}{Wisdom has built her house; she has hewn her seven pillars.}%
\verse{She has slaughtered her slaughtering, mixed her wine, and also set her table.}%
\verse{She has sent her servant girls,\lebnote{Or “young maidens”} she calls upon the wings of the high places of town,}%
\verse{“Whoever is simple, let him turn here.” As for the one who lacks sense,\lnBTK{} she says to him,}%
\verse{“Come, eat with my bread; drink with the wine I have mixed.}%
\verse{Lay aside simplicity and live; walk in the way of understanding.”}%
\verse{He who corrects a scoffer gains abuse for himself, and he who rebukes the wicked gets hurt.\lebnote{“abuse him”}}%
\verse{Do not rebuke a scoffer, lest he hate you; rebuke the wise and he will love you.}%
\verse{Give to a wise one and he will become more wise; teach\lebnote{“make known to”} a righteous one and he will increase learning.}%
\verseWithHeading{Foolishness’ Banquet}{The start of wisdom is fear of Adonai, and knowledge of the Holy One,\lebnote{Or “holy ones”} insight.}%
\verse{For by me your days shall increase, and years of life shall multiply for you.}%
\verse{If you are wise, you are wise for yourself, and if you scoff, alone you shall bear it.}%
\verse{A woman of foolishness is loud, simple, and does not know it.\lebnote{“what”}}%
\verse{She sits at the door of her house, upon a throne at the high places of town,}%
\verse{in order to call to those who pass by the road, those who go straight on their way:}%
\verse{“Whoever is simple, may he turn here!” As for he who lacks sense,\lnBTK{} she says to him,}%
\verse{“Stolen waters are sweet, and bread of secrecy is pleasant.”}%
\verse{But he does not know that the dead\lebnote{Or “Rephaim”} are there, in the depths of Sheol\lebnote{A term for the place where the dead reside, i.e., the Underworld} are her guests.}%
\end{biblechapter}%
\begin{biblechapter}% Proverbs 10
\verseWithHeading{Proverbs of Solomon}{The proverbs of Solomon: A wise child makes a father glad, but a foolish child grieves his mother.}%
\verse{Treasures of wickedness do not profit, but righteousness delivers from death.}%
\verse{Adonai will not cause a righteous person\lebnote{Or “soul,” or “inner self”} to go hungry, but the craving of the wicked he will thwart.}%
\verse{A slack hand causes poverty, but the hand of the diligent enriches.}%
\verse{He who gathers in the summer is a child who is prudent; he who sleeps at the harvest is a child who brings shame.}%
\verse{Blessings belong to the head of the righteous, but the mouth of the wicked conceals violence.}%
\verse{The memory of righteousness is like a blessing, but the name of the wicked will rot.}%
\verse{A heart of wisdom will heed commandments, but a babbling fool will come to ruin.}%
\verse{Whoever walks in integrity\lebnote{Hebrew “in the integrity”} will walk securely, but whoever follows perversity, his ways will be made known.}%
\verse{The winking of an eye causes\lebnote{Or “gives, sets up”} trouble, and the foolishness of lips comes to ruin.}%
\verse{A fountain of life is a mouth of righteousness, and a mouth of wickedness conceals violence.}%
\verse{Hatred stirs up strife, but love covers over all offenses.}%
\verse{On the lips of one who has understanding, wisdom is found, but a rod is for the back of one who lacks sense.\lebnote{“heart”}}%
\verse{Those who are wise lay up knowledge, but to the mouth of the fool, ruin draws near.}%
\verse{The wealth of the rich is the city of his strength; the ruin of the poor is their poverty.}%
\verse{The wage of the righteous leads to life; the gain of the wicked to sin.}%
\verse{On the path to life is he who guards instruction, but he who rejects rebuke goes astray.}%
\verse{He who conceals hatred has lips of deceit, and he who utters slander — he is a fool.}%
\verse{In many words, transgression is not lacking, but he who restrains his lips is prudent.}%
\verse{Choice silver is a tongue of righteousness, a heart\lebnote{Or “mind”} of wickedness is of little worth.\lebnote{“like a little”}}%
\verse{Lips of righteousness feed many, but fools die for lack of sense.\lebnote{“in the lack of heart, they die”}}%
\verse{The blessing of Adonai makes one rich, and he does not increase sorrow with it.}%
\verse{It is like a sport for a fool to do wrong, wisdom for a person of understanding.}%
\verse{The dread of the wicked will come upon him, but the desire of the righteous will be granted.}%
\verse{With the passing of the tempest, there is no wickedness, but the righteous have a foundation forever.}%
\verse{Like vinegar to the tooth and like smoke to the eyes, thus is the lazy to one who employs\lebnote{Or “sends”} him.}%
\verse{The fear of Adonai adds life, but the years of the wicked are shortened.}%
\verse{The hope of the righteous is gladness, but the expectation of the wicked comes to nothing.}%
\verse{A stronghold for the upright is the way of Adonai, but ruin belongs to evildoers.\lebnote{“them who do evil”}}%
\verse{The righteous one is forever; he will not be removed. But the wicked will not remain in the land.\lebnote{Or “earth”}}%
\verse{The mouth of the righteous brings forth wisdom, but a tongue of perversity will be cut off.}%
\verse{The lips of the righteous know the acceptable, but the mouth of the wicked, the perverse.}%
\end{biblechapter}%
\begin{biblechapter}% Proverbs 11
\verse{Balances of deceit are an abomination of Adonai, but an accurate weight\lebnote{“a full stone” or “a pure stone”} delights him.}%
\verse{Pride comes, then disgrace comes, but wisdom is with the humble.}%
\verse{The integrity of the upright guides them, but the crookedness of the treacherous destroys them.}%
\verse{Wealth does not profit on the day of wrath, but righteousness will deliver from death.}%
\verse{The righteousness of the blameless will keep his ways straight, but the wicked will fall by his wickedness.}%
\verse{The righteousness of the upright will save them, but by a scheme the treacherous will be taken captive.}%
\verse{With the death of a wicked person, hope will die, and the expectation of the godless perishes.}%
\verse{The righteous is delivered from trouble, but the wicked enters into it.}%
\verse{With a mouth, the godless shall destroy his neighbor, but by knowledge the righteous are delivered.}%
\verse{When good is with the righteous, the city rejoices, and with the perishing of the wicked, jubilation.}%
\verse{By the blessing of the upright, a city will be exalted, but by the mouth of the wicked, it will be overthrown.}%
\verse{He who lacks sense\lebnote{“heart”} belittles his neighbor, but a person of intelligence will remain silent.}%
\verse{A gossip walks about telling a secret, but the trustworthy in spirit keeps the matter.}%
\verse{Where there is no guidance, a nation\lebnote{Or “people”} shall fall, but there is safety in an abundance of counsel.}%
\verse{He will suffer trouble when he loans to a stranger, but he who refuses a pledge is safe.}%
\verse{A woman of grace receives honor, but the ruthless gets wealth.}%
\verse{A person of kindness rewards himself,\lebnote{Or “soul,” or “inner self”} but a cruel person harms his own flesh.}%
\verse{The wicked earns\lebnote{Or “does, makes”} deceptive gain,\lebnote{“gain of deception”} but he who sows righteousness, a true reward.\lebnote{“a reward of truth”}}%
\verse{He who is steadfast in righteousness is to life as\lebnote{Hebrew “and”} he who pursues evil is to death.}%
\verse{An abomination of Adonai are the crooked of heart,\lebnote{“crooked ones of heart”; “heart” may also be translated “mind”} but his delight are those with blameless ways.\lebnote{“blameless ones of ways”}}%
\verse{Rest assured,\lebnote{“Hand to hand”} the wicked will not go unpunished, but the offspring\lebnote{Or “seed”} of the righteous will escape.}%
\verse{A ring of gold in the snout of a pig is a woman who is beautiful but without discretion.}%
\verse{The desire of the righteous is only good, but the expectation of the wicked, wrath.}%
\verse{There is one who gives yet grows richer,\lebnote{“adding more”} but he who withholds what is right\lebnote{“from what is right”} only finds need.}%
\verse{A person of blessing will be enriched, and he who gives water also will be refreshed.}%
\verse{He who withholds grain, the people curse him, but a blessing is for the head of him who sells.}%
\verse{He who diligently seeks good seeks favor, but he who inquires of evil, it will come to him.}%
\verse{He who trusts in his wealth is he who will fall, but like a green leaf the righteous will flourish.}%
\verse{He who brings trouble to his household, he will inherit wind,\lebnote{Or “breath, spirit”} and a fool will serve the wise of heart.}%
\verse{The fruit of righteousness is a tree of life, and he who captures souls\lebnote{Or “persons,” or “inner selves”} is wise.}%
\verse{If the righteous on earth will be repaid, how much more\lebnote{“more for”} the wicked and sinner.}%
\end{biblechapter}%
\begin{biblechapter}% Proverbs 12
\verse{He who loves discipline loves knowledge, but he who hates rebuke is stupid.}%
\verse{The good obtains favor from Adonai, but anyone who schemes, he condemns.}%
\verse{A person will not be established by wickedness, but the root of the righteous will not be moved.}%
\verse{A woman of strength\lebnote{Or “honor”} is the crown of her master,\lebnote{Or “lord, owner”} but like rot in his bones is she who brings shame.}%
\verse{The thoughts of the righteous are\lebnote{Hebrew “is”} just; the advice of the wicked is treacherous.}%
\verse{The words of the wicked are an ambush of blood, but the mouth of the upright delivers them.}%
\verse{The wicked are overthrown and are no more,\lebnote{“there is no them”} but the house of the righteous shall stand.}%
\verse{For his mouth of good sense, a man will be recommended, but he who is of perverse mind\lnBTL{} will be despised.}%
\verse{It is better to be lowly and a servant to someone\lebnote{“for him,” “\textit{belonging} to him”} than self-glorifying and lacking food.}%
\verse{The righteous knows the life\lebnote{Or “soul,” or “inner self”} of his animal, but the compassion\lebnote{Hebrew “compassions”} of the wicked is cruel.}%
\verse{He who works his land will have plenty of food, but he who follows worthless things lacks sense.\lnBTL{}}%
\verse{The wicked covets the proceeds of evil, but the root of the righteous bears fruit.\lebnote{“gives, sets up”}}%
\verse{By the transgression of lips, evil is ensnared, but the righteous escapes from trouble.}%
\verse{From the fruit of the mouth of a man, he is filled with good, and the reward of a man’s labor\lebnote{“the hands of a man”} will return to him.}%
\verse{The way of a fool is upright in his own eyes, but he who listens to advice is wise.}%
\verse{As for a fool, on that very day\lebnote{“on the day”} he makes his anger known, but he who ignores an insult is prudent.}%
\verse{He who will speak truth will reveal righteousness, but the witness of falsehood, deceit.}%
\verse{There is one who speaks rashly, like the thrust of a sword, but the tongue of the wise brings healing.}%
\verse{A lip of truth endures forever, but a tongue of deception lasts only a moment.}%
\verse{Deceit is in the heart\lnBTM{} of those who plan\lnBTN{} evil, but to those who plan\lnBTN{} peace, there is joy.}%
\verse{No evil will happen\lebnote{“All evil will not happen”} to the righteous, but the wicked are filled with trouble.}%
\verse{An abomination of Adonai are lips of deceit, but they who act faithfully are his delight.}%
\verse{A clever person conceals knowledge, but the heart\lnBTM{} of a fool\lebnote{Hebrew “fools”} announces folly.}%
\verse{The hand of the diligent ones will rule, but the lazy will belong to forced labor.}%
\verse{Anxiety in the heart\lnBTM{} of a man will weigh him down, but a good word will cheer him.}%
\verse{A righteous person will seek out his neighbor, but the way of the wicked will lead them astray.}%
\verse{The lazy will not roast his game, but diligence is the precious wealth of a man.}%
\verse{On the road of righteousness is life, and on the way of the path, may there be no death.}%
\end{biblechapter}%
\begin{biblechapter}% Proverbs 13
\verse{A wise child hears the discipline of a father, but a scoffer does not listen to a rebuke.}%
\verse{From the fruit of the mouth of a man, he shall eat what is good, but the desire\lebnote{Or “soul”} of the treacherous, wrongdoing.}%
\verse{He who keeps his mouth guards his life;\lnBTO{} he who opens his lips, ruin belongs to him.}%
\verse{The soul\lebnote{Or “life”} of the lazy craves, but there is nothing, but the person of diligence is richly supplied.}%
\verse{The righteous hates a word of falsehood; the wicked will bring shame and disgrace.\lebnote{“he will bring disgrace”}}%
\verse{Righteousness will guard the upright of way, but wickedness will overthrow sin.}%
\verse{There is one who acts rich but has nothing;\lebnote{“there is no”} another who pretends to be poor but has wealth.}%
\verse{The ransom of the life of a man is his wealth, but the poor does not receive a threat.}%
\verse{The light of the righteous will rejoice, but the lamp of the wicked will die out.}%
\verse{Only by insolence is strife set up, and wisdom is with those who take advice.}%
\verse{Wealth gained from haste\lebnote{Or “vanity”} will dwindle, but he who gathers little by little\lebnote{“upon the hand”} will increase it.}%
\verse{Hope that is deferred makes the heart\lebnote{Or “mind”} sick, but a desire fulfilled is a tree of life.}%
\verse{He who despises a word will bring destruction on himself, but he who respects a commandment will be rewarded.}%
\verse{The teaching of the wise is a fountain of life, in order to avoid the snares of death.}%
\verse{Good sense grants favor, but the way of the faithless is coarse.}%
\verse{Anyone who is clever will act with intelligence, but the fool will display folly.}%
\verse{A messenger of wickedness will fall into trouble, but an envoy of the faithful brings healing.}%
\verse{Poverty and disgrace belong to him who ignores instruction, but he who guards reproof will be honored.}%
\verse{A desire fulfilled\lebnote{“made to be,” “brought about”} will be sweet to the soul, but an abomination of fools is turning from evil.}%
\verse{Walk with the wise and be wise, but as for the companion of fools, he will suffer harm.}%
\verse{Misfortune will pursue sinners, but the righteous will be rewarded with prosperity.\lebnote{“it will reward prosperity”}}%
\verse{He who is good will leave an inheritance to his grandchildren,\lebnote{“sons of sons”} and stored up for the righteous is the wealth of a sinner.}%
\verse{There is much food in the field of the poor, but it is swept away\lebnote{“there is sweeping away”} by injustice.\lebnote{“no justice”}}%
\verse{He who withholds his rod hates his child, but he who loves him gives him\lebnote{“visits him”} discipline.}%
\verse{The righteous eats to satisfy his life,\lnBTO{} but the belly of the wicked will lack.}%
\end{biblechapter}%
\begin{biblechapter}% Proverbs 14
\verse{The wisest of women\lebnote{“wise ones of women”} builds her house, but the foolish tears it down with her hands.}%
\verse{He who walks in uprightness fears Adonai, but he who is devious in his ways displeases him.}%
\verse{In the mouth of a fool is the rod of pride, but the lips of the wise preserve them.}%
\verse{When there are no\lebnote{“In there is no”} oxen the manger is empty, but an abundance of crops comes by the strength of an ox.\lebnote{Hebrew “bull”}}%
\verse{A faithful witness does not lie, but he who breathes out falsehood is a witness of deceit.}%
\verse{A scoffer seeks wisdom, but there is none, but knowledge comes easily to him who understands.\lebnote{“to one who understands it comes easy”}}%
\verse{Leave the presence of a foolish man,\lebnote{“from the presence \textit{belonging} to a man of foolishness”} for you will not come to know words of knowledge.\lebnote{“lips of knowledge”}}%
\verse{The wisdom of the clever is understanding his ways, but the folly of fools is deceit.}%
\verse{Fools mock\lebnote{Hebrew “he mocks”} the guilt offering, but among the upright, it is favorable.}%
\verse{The heart\lnBTP{} knows the bitterness of its soul,\lebnote{Or “life,” or “inner self”} but in its joy, it will not share itself with a stranger.}%
\verse{The house of the wicked will be destroyed, but the tent of the upright will flourish.}%
\verse{There is a way that seems upright to\lebnote{“to the face of”} a man, but its end is the way of death.}%
\verse{Even in laughter, a heart may be sad, and the end of joy may be grief.}%
\verse{From his ways, the perverse of heart will be satisfied, and from his own,\lebnote{“from upon him”} so shall a good man.}%
\verse{The simple will believe every word, but the clever will consider his step.}%
\verse{The wise is cautious and turns from evil, but the fool throws off restraint and is confident.}%
\verse{He who is short of temper\lebnote{“nostril”} will act foolishly, and the man who schemes will be hated.}%
\verse{The simple are adorned with folly, but the clever are crowned with knowledge.}%
\verse{The evil bow down before the good, and the wicked at the gates of the righteous.}%
\verse{The poor is disliked even by his neighbor, but the lovers of the rich are many.}%
\verse{He who despises his neighbor is a sinner, but he who has mercy on the poor blesses him.}%
\verse{Have they not erred, those who plan evil? But loyalty and faithfulness belong to those who plan good.}%
\verse{In all toil, there is profit, but the talk\lebnote{“word, matter, thing”} of lips leads only to poverty.}%
\verse{The crown of the wise is their wealth; the folly of fools is folly.}%
\verse{He who saves lives\lebnote{Or “souls,” or “inner selves”} is a witness of truth, but he who utters lies is a betrayer.}%
\verse{In the fear of Adonai, there is confidence of strength, and for his children, there will be refuge.}%
\verse{The fear of Adonai is a fountain of life, in order to turn from the snares of death.}%
\verse{In the multitude of people is the glory of the king, but without a population, a prince is ruined.}%
\verse{He who is slow to anger has great understanding, but the hasty of spirit\lebnote{Or “breath”} exalts folly.}%
\verse{A heart\lnBTP{} of tranquility\lebnote{Or “he who heals”} is life to the flesh, but causes bones of passion to rot.}%
\verse{He who oppresses the poor insults him who made him, but he who has mercy on the poor honors him.}%
\verse{By his evildoing, the wicked will be overthrown, and the righteous will find refuge in his death.}%
\verse{In the heart\lnBTP{} of him who has understanding, wisdom rests, but even in the midst of fools it becomes known.}%
\verse{Righteousness will exalt a nation, but sin is a reproach to a people.}%
\verse{The favor of a king is for the servant who deals wisely, but his wrath will be on him who acts shamefully.}%
\end{biblechapter}%
\begin{biblechapter}% Proverbs 15
\verse{A soft answer will turn away wrath, but a word of trouble will stir anger.\lebnote{“nostril”}}%
\verse{The tongue of the wise will dispense knowledge, but the mouth of fools will pour out folly.}%
\verse{In every place, the eyes of Adonai keep watch over the evil and the good.}%
\verse{Gentleness\lebnote{Or “Healing”} of tongue is a tree of life, but perverseness in it\lebnote{That is, the tongue} causes a break in spirit.}%
\verse{A fool will despise the instruction of his father, but he who guards reproof is prudent.}%
\verse{In the house of the righteous there is much treasure, but the income of the wicked brings trouble.}%
\verse{The lips of the wise will spread knowledge, but the heart\lnBTQ{} of fools, not so.}%
\verse{The sacrifice of the wicked is an abomination of Adonai, but the prayer of the upright is his delight.}%
\verse{An abomination of Adonai is the way of the wicked, but he who pursues righteousness he will love.}%
\verse{Severe discipline belongs to him who forsakes the way; he who hates a rebuke will die.}%
\verse{Sheol\lebnote{A term for the place where the dead reside, i.e., the Underworld} and Abaddon\lebnote{Poetic synonym for “Sheol.” Only mentioned in the OT in relation to Sheol, the grave, or death.} are before Adonai, how much more\lebnote{“more for”} the hearts of the children of men!}%
\verse{A scoffer does not like his rebuke;\lebnote{“rebuke \textit{belonging} to him”} to the wise he will not go.}%
\verse{A heart\lnBTQ{} of gladness will make good countenance,\lebnote{“faces”} but in sorrow of heart\lnBTQ{} a spirit is broken.}%
\verse{The heart of him who understands will seek knowledge, but the faces of fools, they\lebnote{Hebrew “he/it”} will feed on folly.}%
\verse{All the days of the poor are hard, but goodness of heart\lnBTQ{} is a continuous feast.}%
\verse{Better is little with the fear of Adonai than great treasure and trouble with it.\lnBTR{}}%
\verse{Better is a dinner of vegetables when\lebnote{Hebrew “and”} love is there than a fattened ox and hatred with it.\lnBTR{}}%
\verse{A man who is hot-tempered will stir up strife, but he who is slow to anger,\lebnote{“nostrils”} he will calm contention.}%
\verse{The way of the lazy is like a hedge of thorns,\lebnote{Hebrew “thorn”} but the path of the upright is a highway.}%
\verse{A child of wisdom will make a father glad, but a foolish person,\lebnote{“foolishness of humankind”} he despises his mother.}%
\verse{Folly is a joy to him who lacks sense,\lebnote{“heart”} and a person of understanding will walk upright.\lebnote{“he will walk upright walking”}}%
\verse{Plans go wrong when there is no counsel, but with many advisors it will succeed.}%
\verse{Joy belongs to a man with answers in his mouth, and a word in its time, how good it is!}%
\verse{The path of life leads upward for him who has insight, in order to turn away from Sheol below.}%
\verse{The house of the proud, Adonai will tear it down, but he will maintain the property line\lebnote{“boundary”} of the widow.}%
\verse{Plans of evil are an abomination of Adonai, but gracious words are pure.}%
\verse{He who makes trouble for his house is he who is greedy for unjust gain, but he who hates bribes will live.}%
\verse{A heart\lnBTQ{} of righteousness will ponder the answer, but a mouth of wickedness will pour out deceit.}%
\verse{Adonai is far from the wicked, but the prayers of the righteous he will hear.}%
\verse{From the light of the eyes, the heart\lnBTQ{} will rejoice, and good news will enliven the bones.\lebnote{Hebrew “bone”}}%
\verse{The ear of him who listens to admonitions of life, in the midst of the wise it will lodge.}%
\verse{He who ignores instruction despises himself,\lebnote{Or “soul,” or “inner self”} but he who hears admonition gains heart.\lnBTQ{}}%
\verse{Fear of Adonai is the instruction of the wise, and before honor comes humility.}%
\end{biblechapter}%
\begin{biblechapter}% Proverbs 16
\verse{To mortals belong the plans of the heart, but from Adonai comes the answer of the tongue.}%
\verse{All the ways of a man are pure in his own eyes, but Adonai weighs the spirit.\lebnote{Or “breath”}}%
\verse{Commit your work to Adonai, and your plans will be established.}%
\verse{All Adonai has made is for his\lebnote{Or “its”} purpose, and even the wicked for the day of trouble.}%
\verse{An abomination of Adonai are all who are arrogant of heart;\lnBTS{} rest assured,\lebnote{“hand to hand”} he will not go unpunished.}%
\verse{By loyalty and faithfulness, iniquity will be covered over,\lebnote{Or “atoned for”} and by fear of Adonai one turns\lebnote{“turning”} from evil.}%
\verse{When the ways of a man are pleasing to Adonai, even his enemies he will cause to make peace with him.}%
\verse{Better is little with righteousness than great income with no justice.}%
\verse{The mind\lebnote{“heart”} of a person will plan his ways, and Adonai will direct his steps.}%
\verse{A decision is upon the lips of a king; in judgment his mouth will not sin.}%
\verse{A balance and scales of justice belong to Adonai; all the weights of the bag are his work.}%
\verse{An abomination to kings is doing evil, for by righteousness the throne will be established.}%
\verse{The delight of kings are the lips of righteousness, and he who speaks what is upright he will love.}%
\verse{The wrath of a king is a messenger of death, but one who is wise will appease\lebnote{Or “atone, cover over”} it.}%
\verse{In the light of the face of the king there is life, and his favor is like a cloud of spring rain.}%
\verse{Getting wisdom: how much\lebnote{“what”} better than gold! And getting understanding: it is chosen over\lebnote{“from”} silver.}%
\verse{The highway of the upright, it turns from evil, he who guards himself\lnBTT{} keeps his way.}%
\verse{Before destruction comes pride, and before a fall, a haughty spirit.\lebnote{Hebrew “a haughty of spirit”}}%
\verse{Better a lowly spirit with the poor than dividing the spoil with the proud.}%
\verse{He who is attentive to a matter will find goodness, and he who trusts in Adonai, his own happiness.}%
\verse{The wise of heart\lnBTS{} is called perceptive, but he who is pleasant of lips will increase persuasiveness.}%
\verse{A fountain of life is wisdom for its owner,\lebnote{Or “master”} but the instruction\lebnote{Or “punishment”} of fools is folly.}%
\verse{The heart\lnBTS{} of the wise will make his mouth judicious, and upon his lips, it\lebnote{Or “he”} will add persuasiveness.}%
\verse{Pleasant sayings are a honeycomb, sweetness to the soul and healing to the bones.}%
\verse{There is a way that seems upright to\lebnote{“to the face of”} a man, but its end is the way of death.}%
\verse{The life\lnBTT{} of a worker works for him, for his hunger\lebnote{“mouth”} urges him.}%
\verse{A man of wickedness concocts evil, and his lips are like a scorching fire.}%
\verse{A person of perversity will spread dissent, and he who whispers separates a close friend.}%
\verse{A person of violence will entice his neighbor and cause him to walk on a way that is not good.}%
\verse{He who winks his eyes does so in order to plan perverse things; he who purses his lips will bring evil to pass.}%
\verse{A crown of glory is gray hair; by a righteous life it is gained.}%
\verse{He who is slow to anger\lebnote{“nostrils”} is better than him who is mighty, and he who controls his spirit than him who captures a city.}%
\verse{The lot will be cast into the lap, but all of its decisions are from Adonai.}%
\end{biblechapter}%
\begin{biblechapter}% Proverbs 17
\verse{Better a dry morsel and quiet with it than a house filled with feasts of strife.}%
\verse{A slave who deals wisely will rule over a child who acts shamefully, and in the midst of brothers he will share his inheritance.}%
\verse{A crucible is for the silver, and a furnace is for the gold, but Adonai will test hearts.}%
\verse{He who does evil listens to lips of wickedness, and the liar gives heed to the tongue of mischief.}%
\verse{He who mocks the poor insults him who made him; he who rejoices at calamity will not go unpunished.}%
\verse{The crown of the elderly are grandchildren,\lebnote{“sons of sons”} and the glory of children is their fathers.}%
\verse{Fine speech\lebnote{“A lip of fine\textit{ness}”} is not becoming a fool, still less\lebnote{“only for”} is false speech\lebnote{“lip of deceit”} for a ruler.}%
\verse{The bribe is a stone of magic in the eyes of its owner;\lebnote{Or “master”} everywhere\lebnote{“to all which”} he will turn, he will prosper.}%
\verse{He who forgives an affront fosters love, but he who waits on a matter will alienate a friend.}%
\verse{A rebuke strikes him who understands deeper than one hundred blows to a fool.}%
\verse{An evil person will seek only rebellion, and a cruel messenger will be sent against him.}%
\verse{May a man meet a she-bear robbed of offspring and not a fool in his folly.}%
\verse{For he who returns evil for good, evil will not depart from his house.}%
\verse{Like the release of water is the beginning of strife; before it breaks out, stop the quarrel.}%
\verse{He who justifies the wicked and he who condemns the righteous, the two of them are both abominations of Adonai.}%
\verse{Why is this? A price in the hand of a fool, in order to buy wisdom where\lebnote{Hebrew “and”} there is no sense.\lnBTU{}}%
\verse{The friend loves at all times\lebnote{Hebrew “time”}, but a brother is born for adversity.}%
\verse{A person who lacks sense\lnBTU{} pledges;\lebnote{“pledges a hand”} he becomes security before his neighbor.}%
\verse{He who loves transgression loves strife; he who builds his high thresholds seeks destruction.}%
\verse{He who is crooked of heart\lebnote{Or “mind”} will not find goodness, and he who is perverse, by his tongue he will fall into calamity.}%
\verse{He who begets a fool, there is trouble for him; the father of a fool will not rejoice.}%
\verse{A cheerful heart is good medicine, but a downcast spirit will dry out bones.}%
\verse{The wicked will accept a bribe from the lap, in order to pervert the ways of justice.}%
\verse{He who understands sets his face toward wisdom, but the eyes of a fool, to the end of the earth.\lebnote{Or “land”}}%
\verse{A grief to his father is the child of a fool, and bitterness to her who bore him.}%
\verse{Also, imposing a fine on the righteous is not good, nor to flog nobles for uprightness.}%
\verse{He who spares his sayings knows knowledge, and a cool spirit is a man of understanding.}%
\verse{Even a fool who keeps silent shall be considered wise;\lebnote{“wise, he shall be considered”} he who closes his lips is intelligent.}%
\end{biblechapter}%
\begin{biblechapter}% Proverbs 18
\verse{He who is selfish seeks a craving;\lebnote{Or “boundary”} against all sound judgment he shows contempt.}%
\verse{A fool will not take pleasure in understanding, but in expressing his heart.\lnBTV{}}%
\verse{With the coming of wickedness comes contempt also, and with dishonor, disgrace.}%
\verse{Deep waters are words of the mouth of a man; a gushing stream is a fountain of wisdom.}%
\verse{Being partial to faces of evil is not good, nor to subvert the righteous at the judgment.}%
\verse{The lips of a fool will bring strife, and his mouth calls out for a flogging.}%
\verse{The mouth of a fool is ruin to him, and his lips are a snare to his soul.\lebnote{Or “life,” or “inner self”}}%
\verse{The words of a whisper are like delicious morsels, and they themselves go down to inner parts of the body.}%
\verse{Even he who is slack in his work, he is brother to a master\lebnote{Or “owner”} of destruction.}%
\verse{A tower of strength is the name of Adonai; into him\lebnote{Or “it”} the righteous will run and be safe.}%
\verse{The wealth of the rich is his strong city,\lebnote{“a city of his strength”} and like a wall, it is high in his imagination.}%
\verse{In the presence of destruction, the heart\lnBTV{} of a man will be haughty, but in the presence of honor, humble.}%
\verse{He who returns a word before he will hear, folly itself belongs to him as well as\lebnote{Hebrew “and”} shame.}%
\verse{The spirit of a man will endure his sickness, but a broken spirit, who may bear it?}%
\verse{An intelligent mind\lebnote{“heart”} will acquire knowledge, and the ear of the wise will seek knowledge.}%
\verse{The gift of a person will open doors for him, and before the great, it gives him access.}%
\verse{The first in his dispute is deemed righteous, but his neighbor will come and examine him.}%
\verse{The lot will put an end to disputes, and between powerful contenders it will decide.}%
\verse{A brother who is offended is worse than a city of strength, and quarrels are like the bars of a fortification.}%
\verse{From the fruit of a man’s mouth, his stomach will be satisfied, as for the yield of his lips, it will satisfy.}%
\verse{Death and life are in the power\lebnote{“hand”} of the tongue, and those who love her\lebnote{That is, the power of the tongue} will eat of her fruit.}%
\verse{He who finds a wife\lebnote{Or “woman”} finds good, and he will obtain favor from Adonai.}%
\verse{The poor may speak entreaties, but the rich will answer roughly.}%
\verse{A man of many friends will come to ruin, but there is a friend who sticks closer than a brother.}%
\end{biblechapter}%
\begin{biblechapter}% Proverbs 19
\verse{Better a poor person walking in integrity than one who is perverse in his speech\lebnote{“in his lips”} and is a fool.}%
\verse{Also, a life\lnBTW{} without knowledge is not good, and he who moves quickly with his feet misses the mark.\lebnote{Or “sins”}}%
\verse{As for the folly of humankind, its way leads to ruin, and against Adonai his heart\lnBTX{} will rage.}%
\verse{Wealth adds many friends, but the poor will be left by his friends.}%
\verse{A witness of falsehood will not go unpunished, and he who breathes lies will not escape.}%
\verse{Many will seek favor before the generous, and everyone is the friend of a man of gifts.\lebnote{Hebrew “gift”}}%
\verse{All the brothers of the poor, if they hate him, how much more will his friends keep away from him. He pursues them with words, and they are gone.\lebnote{Or “\textit{when} he pursues words \textit{and} not them”}}%
\verse{He who acquires wisdom loves himself;\lnBTW{} he who guards understanding loves to find good.}%
\verse{A false witness will not go unpunished, and he who breathes lies will perish.}%
\verse{For a fool living in luxury is not fitting, any more than it is for a slave to rule over princes.}%
\verse{The understanding of a person makes him slow to his anger,\lebnote{“nostril”} and his glory overlooks offense.}%
\verse{The rage of a king growls like a lion, but his favor is like dew on the grass.}%
\verse{A foolish child is a ruin to his father, and the quarreling of a woman\lnBTY{} is a continuous dripping.}%
\verse{A house and wealth are an inheritance from fathers, but from Adonai comes a woman\lnBTY{} who is prudent.}%
\verse{Laziness will bring on a deep sleep, and a person\lebnote{Or “soul,” or “life”} of idleness will suffer hunger.}%
\verse{He who guards commandments guards his life;\lnBTW{} he who is careless of his ways will be killed.}%
\verse{He who lends to Adonai is he who is kind to the poor, and his benefits he will repay to him.}%
\verse{Discipline your child, for there is hope, but on his destruction do not set your desire.\lnBTW{}}%
\verse{A hot-tempered person pays a penalty; if you rescue him, you will do it yet again.}%
\verse{Listen to advice and accept instruction so that you will gain wisdom for your future.\lebnote{“after \textit{things}”}}%
\verse{Many plans are in the heart\lnBTX{} of a man, but the purpose of Adonai will be established.}%
\verse{The craving of a man is his steadfast loyalty, and it is better to be poor than a liar.\lebnote{“man of lying”}}%
\verse{Fear of Adonai leads to life; he who is filled with it will rest — he will not suffer harm.}%
\verse{A lazy person buries his hand in the dish, and even to his mouth he will not bring it back.}%
\verse{The scoffer you shall strike, and the simple, may they learn prudence, and reprove the intelligent and he will gain knowledge.}%
\verse{He who does violence to a father, he who chases away a mother, is a child who causes shame and brings reproach.}%
\verse{Cease to listen to instruction, my child, and you will stray\lebnote{“\textit{in order} to stray”} from sayings of knowledge.}%
\verse{A worthless witness will mock justice, and the mouth of the wicked will devour iniquity.}%
\verse{Judgments\lebnote{Or “Punishments,” or “Condemnations”} are prepared for the scoffers, and flogging for the back of fools.}%
\end{biblechapter}%
\begin{biblechapter}% Proverbs 20
\verse{Wine is a mocker, strong drink a brawler, and any who go astray by it are not wise.}%
\verse{Roaring like the lion is the dreaded anger of a king; he who provokes him forfeits his life.\lebnote{Or “soul,” or “inner self”}}%
\verse{It is honorable for the man to refrain from strife, but every fool will be quick to quarrel.}%
\verse{The lazy person will not plow in season; he will expect at the harvest, but there will be nothing.}%
\verse{Deep waters are like purpose in the heart\lnBTZ{} of a man, and a man of understanding will draw it\lebnote{Or “her”; referring to “purpose”} out.}%
\verse{Many a person will proclaim his loyalty for himself, but a man who is trustworthy, who can find?}%
\verse{He who walks in his integrity is righteous; happy are his children who follow him.}%
\verse{A king who sits on the throne of judgment winnows all evil with his eyes.}%
\verse{Who will say “I have made my heart\lnBTZ{} clean; I am pure from my sin”?}%
\verse{Stone and stone, measure and measure, both of them\lnBUA{} are an abomination of Adonai.}%
\verse{Even by his acts, a young man\lebnote{Or “young boy,” or “adolescent”} will make himself known, whether his acts are pure and upright.}%
\verse{The ear that hears and the eye that sees, Adonai has made them both.\lnBUA{}}%
\verse{Do not love sleep, lest you become poor; open your eyes and have plenty bread.}%
\verse{“Bad, bad,” the buyer will say, but when one goes to him, then he will boast.}%
\verse{There is gold and many costly stones, but precious jewels are lips of knowledge.}%
\verse{Take his garment, for he has given security to a stranger, and on behalf of a foreigner — take it as pledge.}%
\verse{Bread gained by deceit is sweet for the man, but afterward, his mouth will be filled with gravel.}%
\verse{A plan\lebnote{Hebrew “plans”} will be established by advice, and with guidance make war.}%
\verse{He reveals a secret, he who walks about with gossip, and do not associate with a babbler’s lips.}%
\verse{He who curses his father and his mother, his lamp will be extinguished in the midst of darkness.\lebnote{“in a pupil of darkness”}}%
\verse{An inheritance acquired hastily at the beginning\lebnote{“at the first”} will not be blessed at its end.}%
\verse{Do not say “I will repay evil”; wait for Adonai and he will deliver you.}%
\verse{An abomination of Adonai is a stone and a weight,\lebnote{“a stone and a stone”} and scales of falsehood are not good.}%
\verse{Away from Adonai are the steps of a strong man, and how will humankind understand his ways?}%
\verse{It is a snare to humankind to say rashly “It is holy,” and after vows, to scrutinize.}%
\verse{A wise king winnows the wicked, and he will drive a wheel over them.}%
\verse{The lamp of Adonai is the spirit\lebnote{Or “soul”} of humankind, he who searches every innermost part.\lnBUB{}}%
\verse{Loyalty and faithfulness will preserve a king, and he is upheld with the righteousness of his throne.}%
\verse{The glory of young men is their strength,\lebnote{Or “strengthens them”} but the beauty of the aged is gray hair.}%
\verse{The blows of a wound will cleanse evil, as will\lebnote{Hebrew “and”} beatings of the innermost part.\lnBUB{}}%
\end{biblechapter}%
\begin{biblechapter}% Proverbs 21
\verse{Streams of water are the heart\lebnote{Or “mind”} of a king in the hand of Adonai; wherever\lebnote{“upon all that”} he will desire, he will turn.}%
\verse{Every way of a man is upright in his own eyes, but Adonai weighs hearts.\lebnote{Or “minds”}}%
\verse{Doing righteousness and justice is more acceptable to Adonai than sacrifice.}%
\verse{Haughtiness of the eyes and pride of heart, the lamp of the wicked are sin.}%
\verse{The plans of the diligent only lead to abundance, but all who are hasty, only to want.}%
\verse{He who makes treasure by a lying tongue is a fleeting vapor and seeker\lebnote{Hebrew “seekers”} of death.}%
\verse{The violence of the wicked will sweep them away, for they refuse to do justice.}%
\verse{Crooked is the way of a man and a foreigner, but the pure is upright in his conduct.}%
\verse{Better to dwell on the corner of a roof than to share a house with a woman\lebnote{Or “wife”} of contention.}%
\verse{The soul\lebnote{Or “soul, life, throat”} of the wicked desires evil; his neighbor will not find mercy in his eyes.}%
\verse{With the punishment of a scoffer, the simple will become wise, and with the instruction of the wise, he will obtain knowledge.}%
\verse{The righteous observes the house of the wicked; he throws the wicked to ruin.}%
\verse{He who closes his ear from the cry of the poor, he also will cry out and not be heard.}%
\verse{A gift in secret\lebnote{Hebrew “the secret”} will avert anger,\lebnote{“nostril”} and a concealed bribe,\lebnote{“bribe in the bosom”} strong wrath.}%
\verse{It is a joy to the righteous to do justice, but dismay to those who do evil.}%
\verse{Whoever wanders from the way of understanding, in the assembly of the dead\lebnote{Or “Rephaim”} he will rest.}%
\verse{A man of want is he who loves pleasure; he who loves wine and oil\lebnote{Or “fat”} will not become rich.}%
\verse{A ransom for the righteous is the wicked, and the faithless instead of the upright.}%
\verse{Better to live in a land of wilderness than with a wife of quarrels and provocation.}%
\verse{Precious treasure and oil are in the house of the wise, but the foolish person\lebnote{Or “foolish of mankind”} will devour them.\lebnote{“it”}}%
\verse{He who pursues righteousness and kindness will find life, righteousness, and honor.}%
\verse{To a city of warriors, the wise ascends, and he will bring down the stronghold — its object of trust.}%
\verse{He who guards his mouth and his tongue, he guards his life\lebnote{Or “soul,” or “inner self”} from danger.}%
\verse{The proud, haughty one: “scoffer” is his name; he acts with arrogance of pride.}%
\verse{The craving of a lazy person will kill him, for his hands refuse to work.\lebnote{Or “make”}}%
\verse{All day\lebnote{“All the day”} he craves a craving, but the righteous will give and not hold back.}%
\verse{The sacrifice of the wicked is an abomination; how much more when he brings it in divisiveness!}%
\verse{A false witness will perish, but a man who listens will testify with\lebnote{Hebrew “to the”} success.}%
\verse{A wicked man is strong in his countenance,\lebnote{“his face”} but the upright will appoint his paths.}%
\verse{There is no wisdom, nor understanding, nor counsel to oppose Adonai.}%
\verse{A horse is prepared for the day of battle, but to Adonai belongs the victory.}%
\end{biblechapter}%
\begin{biblechapter}% Proverbs 22
\verse{A reputation\lebnote{“name”} is to be chosen rather than\lebnote{“from”} great riches; favor is better than silver and gold.\lebnote{“from silver and from gold, favor \textit{is} better”}}%
\verse{Rich and poor have much in common; Adonai is the maker of all of them.}%
\verse{The clever sees danger and hides, but the simple go on and suffer.}%
\verse{The reward of humility is the fear of Adonai — wealth and honor and life.}%
\verse{Thorns and snares are in the way of the perverse; he who guards himself\lebnote{Or “soul,” or “life”} will keep away from them.}%
\verse{Train the child concerning his way;\lebnote{“on the mouth of his way”} even when he is old, he will not stray from it.}%
\verse{The rich will rule over the poor, and the borrower is a slave of the lender.\lebnote{“the borrower \textit{belonging} to a man”}}%
\verse{He who sows injustice will reap calamity, and the rod of his anger will fail.}%
\verse{He who is generous\lebnote{“the generous of eye”} will be blessed, for he gives to the poor from his own bread.}%
\verse{Drive out a scoffer and strife will go out; quarrel and abuse will cease.}%
\verse{He who loves purity of heart and hasgracious speech,\lebnote{“grace of his lips”} his friend is the king.}%
\verse{The eyes of Adonai keep watch over knowledge, but he will overthrow the words of the faithless.}%
\verse{A lazy person says “A lion in the street! In the middle of the highway, I shall be killed!”}%
\verse{A deep pit is the mouth of an adulteress,\lebnote{“a strange woman”} he with whom Adonai is angry will fall there.}%
\verse{Folly is bound up in the heart\lnBUC{} of a boy;\lebnote{Or “young man,” or “adolescent”} the rod of discipline will drive it\lebnote{That is, folly} from him.}%
\verse{He who oppresses the poor in order to enrich himself, or gives to the rich, will come to poverty.\lebnote{“only loss”}}%
\verseWithHeading{Words of the Wise}{Incline your ear and hear the words of the wise; you shall apply your heart\lnBUC{} to my teaching.}%
\verse{For it is pleasant if you guard them within you;\lebnote{“in your belly/womb”} together they will be ready upon your lips.}%
\verse{In order for your trust to be in Adonai, I have made them known to you today\lebnote{“the day”} — even\lebnote{Or “only”} you.}%
\verse{Have I not written for you thirty sayings with admonitions and knowledge?}%
\verse{In order to show you what is right — sayings of truth — in order to return a true saying to him who sent you.}%
\verse{Do not rob the poor because he is poor, and do not crush the afflicted at the gate;}%
\verse{For Adonai will plead their case and despoil those who despoil them of life.\lnBUD{}}%
\verse{Do not befriend an owner\lebnote{Or “master”} of anger,\lebnote{“nostril”} and with a man of wrath you shall not associate;}%
\verse{lest you learn his way and become entangled in a snare to yourself.\lnBUD{}}%
\verse{Do not be with those who give a pledge\lebnote{“strike a hand”} by becoming\lebnote{“in the becomings of”} surety.}%
\verse{If there is nothing for you to pay,\lebnote{Or “complete”} why will he take your bed from under you?}%
\verse{Do not remove an ancient boundary marker which your ancestors\lebnote{Or “fathers”} made.}%
\verse{A man who is skillful in his work, you shall see: before kings, he will serve; he will not serve before the commoners.}%
\end{biblechapter}%
\begin{biblechapter}% Proverbs 23
\verse{When you sit to eat with a ruler, you shall surely observe what is before you,}%
\verse{and you shall put a knife to your throat if you have a big appetite.\lebnote{“lord/master of life”}}%
\verse{Do not desire his delicacies, for\lebnote{Hebrew “and”} it is food of deception.}%
\verse{Do not tire in order to become rich; out of your understanding, may you desist.}%
\verse{Your eyes will alight\lebnote{“cause to fly”} on it, but there is nothing to it, for suddenly it will make for itself wings like an eagle and it will be exhausted in the heavens.}%
\verse{Do not eat the bread of the stingy,\lebnote{“stingy of eye”} and do not desire his delicacies.}%
\verse{For, like hair in his throat,\lebnote{“soul,” or “inner self”} so it is.\lebnote{Or “is he”} “Eat and drink!” he will say to you, but his heart will not be with you.}%
\verse{Your morsel you have eaten, you will vomit it up, and you will waste your pleasant words.}%
\verse{In the ears of a fool do not speak, for he will despise the wisdom of your words.}%
\verse{Do not remove an ancient boundary marker, and on the fields of orphans do not encroach;}%
\verse{For their redeemer is strong, he himself will plead their cause against you.}%
\verse{Apply your heart\lnBUE{} to instruction, and your ear to sayings of knowledge.}%
\verse{Do not withhold discipline from a child, if you will beat him with the rod, he will not die.}%
\verse{As for you, with the rod you shall beat him, and his life\lebnote{Or “soul,” or “inner self”} you will save from Sheol.\lebnote{A term for the place where the dead reside, i.e., the Underworld}}%
\verse{My child, if your heart\lnBUE{} is wise, my heart will be glad — even me!}%
\verse{And my insides\lebnote{“kidneys”} will rejoice when your lips speak what is upright.}%
\verse{May your heart\lnBUE{} not envy the sinners, but live in fear of Adonai all day.\lebnote{“all the day”}}%
\verse{Surely\lebnote{“For if”} there is a future, and your hope will not be cut off.}%
\verse{You, my child, hear and be wise, and direct your heart\lnBUE{} on the road.}%
\verse{Do not be among drinkers of wine, among gluttonous eaters of their meat.\lebnote{“meat for them”}}%
\verse{For the drunkard and gluttonous, they will become poor, and with rags, drowsiness will clothe them.}%
\verse{Listen to your father — he who gave you life,\lebnote{“caused you to be born”} and do not despise your mother when\lebnote{Or “for, because”} she is old.}%
\verse{Buy truth and do not sell it, wisdom and instruction and understanding.}%
\verse{The father of the righteous will surely rejoice; he who bears a wise person will be happy with him.}%
\verse{May your father and your mother be glad, and may she who bore you rejoice.}%
\verse{My child, may you give your heart\lnBUE{} to me, and may your eyes delight in my ways.}%
\verse{For a deep pit is a prostitute,\lebnote{Or “whore”} and a narrow well is an adulteress.\lebnote{“a foreign woman”}}%
\verse{She is also like a robber lying in wait, and the faithless among mankind she increases.}%
\verse{To whom is woe? To whom is sorrow? To whom are quarrels? To whom is complaint? To whom are wounds without cause? To whom is redness of the eyes?}%
\verse{To those who linger over wine, to those who come to try mixed wine.}%
\verse{Do not look at wine when it is red, when it sparkles\lebnote{“gives its eye”} on the cup, going down smoothly.}%
\verse{In the end,\lebnote{“end him”} it will bite like a serpent, and it will sting like an adder.}%
\verse{Your eyes will see strange things, and your heart\lnBUE{} will speak perverse things.}%
\verse{And you will be like him who lies down in the heart of the sea, and like him who lies down on top of a mast.}%
\verse{“They struck me; I was not hurt. They beat me; I did not know it. When I will awake, I will continue; I will seek it again.”}%
\end{biblechapter}%
\begin{biblechapter}% Proverbs 24
\verse{Do not envy men of evil, and do not desire to be with them.}%
\verse{For their minds will devise violence, and their lips will speak mischief.}%
\verse{By wisdom a house is built, and by understanding it is established.}%
\verse{And by knowledge, rooms are filled with all riches, precious and pleasant.}%
\verse{The warrior of wisdom is in strength, and a man of knowledge is strong in power.}%
\verse{For with wise guidance you shall make war for yourself, and victory is in an abundance of counsel.}%
\verse{Wisdom is too high for fools; at the gate he will not open his mouth.}%
\verse{He who plans to do evil for\lebnote{Or “to”} himself, they will call him “master\lebnote{Or “owner”} of mischief.”}%
\verse{Devising folly is a sin, and an abomination to humankind is a scoffer.}%
\verse{If you faint on the day of adversity, little is your strength.}%
\verse{Rescue those who are led away to the death and those who stagger to the slaughter. If you hold back,}%
\verse{if you say, “Look, we do not know this,” does not he who weighs hearts perceive it? And he who keeps your soul,\lnBUF{} he knows and will repay humankind according to his deeds.}%
\verse{My child, eat honey, for it is good, and the dripping of the honeycomb is sweet to your taste.}%
\verse{Thus know wisdom for the sake of your soul,\lnBUF{} if you find it, then there is a future, and your hope will not be cut off.}%
\verse{Do not lie in wait like an outlaw against the home of the righteous; do not do violence to his dwelling place.}%
\verse{For seven times the righteous will fall, but he will rise, but the wicked will be overthrown by calamity.}%
\verse{While your enemies are falling, do not rejoice; when he trips himself, may your heart\lnBUG{} not be glad}%
\verse{lest Adonai see and it be evil in his eyes, and turn his anger away from him.}%
\verse{Do not fret because of the evildoers; do not envy the wicked.}%
\verse{For there will not be a future for the evil; the lamp of the wicked will die out.}%
\verse{Fear Adonai, my son, and the king; with those who change, do not associate.}%
\verse{For suddenly their disaster will come, and the ruin of both of them,\lebnote{“\textit{the} two of them”} who knows?}%
\verse{These sayings are also for the wise: Partiality\lebnote{“acknowledge faces”} in judgment is not good.}%
\verse{Whoever says to the guilty, “You are righteous,” the people will curse him; the nations will abhor him.}%
\verse{But they who rebuke will have delight, and upon them blessings of goodness will come.}%
\verse{He will kiss the lips, he who gives an honest answer.}%
\verse{Prepare your work in the street and get it ready for yourself in the field; afterward, then\lebnote{Hebrew “and”} you shall build your house.}%
\verse{Do not be a witness without cause against your neighbor nor deceive with your lips.}%
\verse{Do not say, “Just as he has done to me, so shall I do to him; I will pay back the man according to his deed.”}%
\verse{I passed by the field of a lazy person, and over the vineyard of a person lacking sense;\lebnote{“heart”}}%
\verse{and behold, it was overgrown — all of it was covered with thorns, its surface with nettles, and its stone wall\lebnote{“a wall of his/its stones”} was broken down.}%
\verse{Then I myself saw and my heart\lnBUG{} considered; I looked, and I took hold of instruction:}%
\verse{A little sleep, a little slumber, a little folding of the hands for rest,}%
\verse{and your poverty will come running, and your lack like an armed warrior.}%
\end{biblechapter}%
\begin{biblechapter}% Proverbs 25
\verseWithHeading{More Proverbs of Solomon}{These are also proverbs of Solomon which officials of Hezekiah king of Judah copied:}%
\verse{The glory of God\lebnote{Or “gods”} conceals things, but the glory of kings searches out things.}%
\verse{As heaven is to height and the earth is to depth, so is the heart\lebnote{Or “mind”} of kings — there is no searching.}%
\verse{Remove the dross from silver, and it will become a vessel for the smith.}%
\verse{Remove the wicked before a king, and his throne will be established in righteousness.\lebnote{Hebrew “in the righteousness”}}%
\verse{Do not promote yourself before the king, and in the place of the great ones do not stand.}%
\verse{For it is better that he say to you, “Ascend here,” than he humble you before a noble. What your eyes have seen,}%
\verse{do not hastily bring out to court, for\lebnote{Or “lest”} what will you do at its end, when your neighbor puts you to shame?}%
\verse{Argue your argument with your neighbor himself, the secret of another do not disclose,}%
\verse{lest he who hears shame you and your ill repute will not end.}%
\verse{Apples of gold in a setting of silver is a matter spoken at\lebnote{Hebrew “on”} its proper time.}%
\verse{A ring of gold and an ornament of fine gold is a rebuke of the wise to the ear of a listener.}%
\verse{Like the cold of snow on a day\lebnote{Or “at a season”} of harvest is a faithful messenger to those who send him, and the soul\lnBUH{} of his master is refreshed.\lebnote{Or “returned”}}%
\verse{Like clouds and wind when there is no rain, so too is a man who boasts in a gift of deception.}%
\verse{With patience\lebnote{“With length of face”} a ruler may be persuaded, and a soft tongue will break a bone.\lebnote{Or “strength”}}%
\verse{If you find honey, eat what is sufficient for you, lest you have your fill of it and vomit it out.}%
\verse{Make your foot scarce in the house of your neighbor, lest he become weary of you and hate you.}%
\verse{Like a club and sword and a sharp arrow is a man who bears false witness against his neighbor.}%
\verse{A bad tooth and a lame foot is the trust of a faithless person in a time\lebnote{“on a day”} of trouble.}%
\verse{Like one who removes a garment on a cold day, or like vinegar on natron,\lebnote{A mineral salt found on dry lake beds often used as a preservative.} is he who sings songs to a heavy heart.}%
\verse{If your enemy is hungry, feed him bread, and if thirsty, let him drink water.}%
\verse{For coals of fire you will heap upon his head, and Adonai will reward you.}%
\verse{The wind of the north produces rain, and a backbiting tongue, angry faces.}%
\verse{Better to live upon the corner of a roof than with a woman\lebnote{Or “wife”} of contention and in a shared house.}%
\verse{Like cold water\lebnote{Hebrew “waters”} upon a weary soul,\lnBUH{} so too is good news from a distant place.}%
\verse{Like a muddied spring or a polluted fountain is the righteous who gives way before the wicked.}%
\verse{To eat much honey is not good, nor is seeking one’s honor\lebnote{Hebrew “their honor”} honorable.}%
\verse{A breached city where there is no wall is like a man who has no\lebnote{“there is no”} self-control for his spirit.}%
\end{biblechapter}%
\begin{biblechapter}% Proverbs 26
\verse{Like snow in the summer and like rain at the harvest, so honor is not fitting for a fool.}%
\verse{Like the sparrow is to fluttering and like the swallow is to flying, so an undeserved curse does not go forth.}%
\verse{A whip for the horse, a bridle for the donkey, and a rod for the back of fools.}%
\verse{Do not answer a fool according to his folly lest you become like him — even you.}%
\verse{Answer a fool according to his folly, or else he will be wise in his own eyes.}%
\verse{Like cutting off feet or drinking violence, so is he who sends messages in the hand of a fool.}%
\verse{Like legs that hang limp from a lame person, so is a proverb in the mouth of fools.}%
\verse{Like binding a stone in a sling, so is giving honor to a fool.}%
\verse{Like a thorn that goes up in the hand of a drunkard, so is a proverb in the mouth of fools.}%
\verse{Like an archer who wounds everyone, so is he who hires a fool or he who hires passersby.}%
\verse{Like a dog returning to his vomit is a fool reverting to his folly.}%
\verse{Do you see a man wise in his own eyes? There is more hope for a fool than for him.}%
\verse{A lazy person says “A lion is in the road! A lion among the streets!”}%
\verse{The door turns on its hinge, and a lazy person on his bed.}%
\verse{A lazy person buries his hands in the dish; he is too tired to return it to his mouth.}%
\verse{A lazy person is wiser in his eyes than seven who answer discreetly.}%
\verse{Grabbing onto the ears of a dog passing by is one who meddles in a quarrel that is not his own.}%
\verse{Like a maniac who shoots firebrands, arrows, and death,}%
\verse{so is a man who deceives his neighbor, but says “Am I not joking?”}%
\verse{For lack of wood, a fire goes out, and where there is no whisperer, quarreling will cease.}%
\verse{As charcoal is to hot embers and wood is to fire, so a man of quarrels is to kindling strife.}%
\verse{The words of a whisperer are like delicious morsels, and they go down to the inner parts of the body.}%
\verse{Like impure silver\lebnote{“silver of impurities”} which overlays an earthen vessel, so are smooth lips and an evil heart\lebnote{Or “mind”}.}%
\verse{On his lips, an enemy will pretend, but inside\lebnote{“in his midst”} he will harbor deceit.}%
\verse{When he makes his voice gracious, do not believe him, for seven abominations are in his heart.}%
\verse{Though hatred is covered with guile, its evil will be exposed in the assembly.}%
\verse{He who digs a pit, in it he will fall, and he who rolls a stone, on him it will come back.}%
\verse{A tongue of deceit hates its victim, and a flattering mouth makes ruin.}%
\end{biblechapter}%
\begin{biblechapter}% Proverbs 27
\verse{Do not boast about tomorrow,\lebnote{“on the day of tomorrow”} for you do not know what the day will bring.}%
\verse{May another praise you and not your own mouth, a stranger and not your own lips.}%
\verse{Heavy is a stone and weighty is sand, but the provocation of a fool is heavier than both of them.}%
\verse{Cruel is wrath and overwhelming is anger, but who will stand before jealousy?}%
\verse{Better a rebuke that is open than a love that is hidden.}%
\verse{The wounds of a friend mean well, but the kisses of an enemy are profane.}%
\verse{An appetite\lnBUI{} that is sated spurns honey, but to an appetite\lnBUI{} that is ravenous, all bitterness is sweet.}%
\verse{Like a bird that strays from its nest, so is a man who strays from his place.}%
\verse{Perfume and incense will gladden a heart, and the pleasantness of one’s friend is personal advice.\lebnote{“because of advice of a person”}}%
\verse{As for your friend and a friend of your father, do not forsake them, and the house of your brother, do not enter on the day of your calamity. Better is a close neighbor than a distant brother.}%
\verse{Be wise, my child, and make my heart glad, and I will answer him who reproaches me with a word.}%
\verse{When the clever sees danger, he hides; the simple go on and suffer.}%
\verse{Take his garment, for he gives surety to a stranger, and to an adulteress\lebnote{“a foreign woman”} — so take his pledge.}%
\verse{He who blesses his neighbor with a loud voice early in the morning, a curse will be reckoned to him.}%
\verse{Dripping constantly on a day of heavy rain and a woman\lebnote{Or “wife”} of contention are alike.}%
\verse{In restraining her, he restrains wind,\lebnote{Or “breath, or “spirit”} and his right hand will grasp oil\lebnote{Or “fat”}.}%
\verse{As iron sharpens\lebnote{Or “is united with”} iron, so one man sharpens another.\lebnote{“a man sharpens \textit{the} faces of his friend”}}%
\verse{He who tends a fig tree will eat its fruit, and he who guards his master\lebnote{Or “lord”} will be honored.}%
\verse{As the waters reflect face to face,\lebnote{“the faces to the faces”} so the heart of a person reflects the person.}%
\verse{Sheol\lebnote{A term for the place where the dead reside, i.e., the Underworld} and Abaddon\lebnote{Poetic synonym for “Sheol.” Only mentioned in the OT in relation to Sheol, the grave, or death.} will not be satisfied, and the eyes of a person will not be satisfied either.}%
\verse{A crucible is for the silver, and a furnace for the gold, but a man is tested by the mouth of him who praises him.}%
\verse{If you crush a fool in the mortar with the pestle along with\lebnote{Or “in the midst of”} the crushed grain, it will not drive folly from upon him.}%
\verse{You will surely know the condition\lebnote{“faces”} of your flock; your heart\lebnote{Or “mind”} attends to the herds.}%
\verse{For riches are not forever, nor a crown for generation after generation.}%
\verse{When the grass is gone, then green growth will appear, and the herbs of the mountains will be gathered.}%
\verse{Lambs will be your clothing, and goats the price of the field.}%
\verse{And there will be enough goats’ milk for your food, for the food of your household and the nourishment\lebnote{“life”} of your maidservants.}%
\end{biblechapter}%
\begin{biblechapter}% Proverbs 28
\verse{The wicked flee, but no one pursues, but the righteous is bold like a lion.}%
\verse{By the rebellion of a land, her rulers increase, but by a person of intelligence who knows justice, it will last.}%
\verse{A man who is poor and oppresses the impoverished is a beating rain that leaves\lebnote{“and there is”} no food.}%
\verse{Those who forsake instruction will praise the wicked, but they who guard instruction will struggle against them.}%
\verse{Men of evil do not understand justice, but seekers of Adonai understand completely.\lebnote{Or “everything”}}%
\verse{Better to be poor and walking in one’s integrity than to be crooked of ways when one is rich.}%
\verse{He who keeps instruction is a child of understanding, but the companion of gluttons will shame his father.}%
\verse{He who augments his wealth with interest and with usury gathers it for him who is kind to the poor.}%
\verse{He who turns his ear from listening to instruction, even his prayer is an abomination.}%
\verse{He who misleads the upright onto the way of evil, into his pits he will fall. But as for the blameless, they will inherit good.}%
\verse{A man of wealth is wise in his own eyes, but the intelligent poor sees through him.}%
\verse{When the righteous triumphs, great is the glory, but with the rising of the wicked, a person will be hidden.}%
\verse{He who conceals his transgression will not prosper, but he who confesses and forsakes will obtain mercy.}%
\verse{Happy is the person who fears continuously, but he who is stubborn of heart,\lebnote{Or “mind”} will fall into calamity.}%
\verse{Like a roaring lion and a charging bear is a wicked ruler over a poor people.}%
\verse{A ruler who lacks understanding is\lebnote{Hebrew “and”} a cruel oppressor, but those who hate unjust gain will have long\lebnote{Or “prolong”} days.}%
\verse{A person who is burdened with the blood of another,\lebnote{“life”} until death he will flee; do not take hold of him.}%
\verse{He who walks in integrity will be safe, but he who takes crooked paths will fall in one.}%
\verse{He who tills his ground will have plenty bread, but he who follows fantasies will have plenty of poverty.}%
\verse{A man of faithfulness has abundant blessings, but he who hurries to become rich will not go unpunished.}%
\verse{Showing partiality\lebnote{“Recognizing faces”} is not good, and over a morsel of bread, a strong man will do wrong.}%
\verse{He who hurries for wealth is a man with an evil eye,\lebnote{“a man of evil of eye”} but he does not know that poverty will come upon him.}%
\verse{He who rebukes a person will afterward find more favor than he who flatters with the tongue.}%
\verse{He who robs his father and his mother and says, “There is no crime,” is partner to a man who corrupts.}%
\verse{The greedy person\lebnote{“wide soul”} will stir up strife, but he who trusts in Adonai will be enriched.}%
\verse{He who trusts in his own heart\lebnote{Or “mind,” or “sense”} is a fool, but he who walks in wisdom will be saved.}%
\verse{For he who gives to the poor, there is no lacking, but for he\lebnote{Hebrew “those”} who turns his eyes, there are many curses.}%
\verse{With the rising of the wicked, humankind will be hidden, and with their perishing, the righteous will multiply.}%
\end{biblechapter}%
\begin{biblechapter}% Proverbs 29
\verse{A man who is reproved, being stubborn of neck — suddenly he will be broken, and there will not be healing.}%
\verse{When the righteous are numerous, the people will rejoice, but when the wicked are ruling, people will groan.}%
\verse{A man who loves wisdom will make his parents glad, but the friend of prostitutes\lebnote{Or “whores”} will squander his wealth.}%
\verse{By justice a king gives stability to a land, but a man of bribes will ruin it.}%
\verse{A strong man who flatters his neighbor is spreading a net for his feet.}%
\verse{In transgression, an evil man is a snare, but the righteous will sing and rejoice.\lebnote{Hebrew “he will rejoice”}}%
\verse{The righteous knows the case of the poor, but the wicked does not understand knowledge.}%
\verse{Men of scoffing set a city aflame, but the wise turn away wrath.}%
\verse{If a wise man goes to court with a foolish man, then there is rankling and ridicule,\lebnote{Hebrew “there is ridicule”} but there is no relief.}%
\verse{Men of blood hate the blameless, and they seek the life of the upright.\lnBUJ{}}%
\verse{A fool gives all his breath,\lebnote{Or “spirit”} but the wise holds back in quiet.}%
\verse{A ruler listening to a word of falsehood, all his officials are wicked.}%
\verse{The poor and a man of oppression have this in common: Adonai gives light to the eyes of them both.\lebnote{“light of the eyes of the two of them”}}%
\verse{A king who judges with truthfulness to the poor, his throne will be established forever.}%
\verse{As for a rod and reproof, they\lebnote{Hebrew “it”} will give wisdom, but a neglected child is disgraced by his mother.}%
\verse{With the increase of the wicked, transgression will increase, but the righteous will look on his downfall.}%
\verse{Discipline your children, and they will give you rest, and they will give delight to your soul.\lebnote{Or “life,” or “inner self”}}%
\verse{When there is no prophecy, the people cast off restraint, but as for he who guards instruction, happiness is his.}%
\verse{By words, a servant is not disciplined, for he will understand, but there is no giving heed.}%
\verse{You see a man who is hasty in his words: there is more hope for a fool than him.}%
\verse{He who pampers his servant from childhood, arrogance will be his end.\lebnote{“and its/his end will be arrogance”}}%
\verse{A man of anger will stir strife, and the owner of anger, much transgression.}%
\verse{The pride of a person will bring him humiliation, and the lowly of spirit will obtain honor.}%
\verse{Being a partner with a thief is hating one’s life;\lnBUJ{} a curse he will hear, but not disclose.}%
\verse{The fear of a person will lay a snare, but he who trusts in Adonai will be secure.}%
\verse{Many are those who seek the favor\lebnote{“faces”} of a ruler, but from Adonai one obtains justice.\lebnote{“\textit{comes} justice for a man”}}%
\verse{A man of injustice is an abomination to the righteous, but the upright\lebnote{“upright of way”} is an abomination to the wicked.}%
\end{biblechapter}%
\begin{biblechapter}% Proverbs 30
\verseWithHeading{The Oracle of Agur}{The words of Agur, son of Yakeh, the oracle. Thus says the man to Ithiel, to Ithiel, and Ucal:\lebnote{Or “I am weary, O God; I am weary, O God, and worn out”}}%
\verse{Certainly I am more stupid than a man, and the understanding of humankind is not for me.}%
\verse{And I have not learned wisdom, nor will I know knowledge of the Holy One.\lebnote{Or “holy ones”}}%
\verse{Who has ascended to heaven and come down? Who has gathered the wind in the hollow of his hand? Who has wrapped water in a\lebnote{Hebrew “the”} garment? Who has established all the ends of the earth? What is his name and what is the name of his child? For surely you know.}%
\verse{Every word of God is flawless; he is a shield for him who takes refuge in him.}%
\verse{Do not add to his words lest he rebuke you and you be found a liar.}%
\verse{Two things I ask from you; do not deny me before I die:}%
\verse{Keep falsehood and a lying word\lebnote{“word of a lie”} far from me; do not give me poverty or wealth; provide me with food only for my need.}%
\verse{Or else I will be satisfied and will deny him and say “Who is Adonai?” Or\lebnote{Or “Lest”} I will be poor and will steal and profane the name of my God.}%
\verse{Do not slander a servant to his master, or else he will curse you and you will be guilty.}%
\verse{There is a generation that will curse its father, and its mother it will not bless.}%
\verse{There is a generation that is pure in its own eyes, but from its filthiness it will not be cleansed.}%
\verse{There is a generation — how lofty are their eyes! And their eyelids they will lift.}%
\verse{There is a generation whose teeth are swords, and its jawbones, knives, in order to devour the poor from the earth\lnBUK{} and the needy from humankind.}%
\verse{For the leech, there are two daughters; “Give, give!” they cry. As for three of these, they are not satisfied; as for four, they do not say enough.}%
\verse{Sheol\lebnote{A term for the place where the dead reside, i.e., the Underworld} and barrenness of womb, the land is not satisfied with water, and fire does not say “enough!”}%
\verse{The eye that mocks a father and scorns the obedience of a mother — the ravens of the valley will peck it out, and the offspring of vultures will eat it.}%
\verse{Three of these are too wonderful for me, and four, I do not understand them:}%
\verse{the way of the eagle in the sky, the way of a snake on a rock, the way of a ship in the heart of the sea, and the way of a man with a young woman.}%
\verse{This is the way of a woman committing adultery: she eats and wipes her mouth, and says “I have not done wrong.”}%
\verse{Under three things the earth\lnBUK{} trembles, and under four, it is not able to bear up:}%
\verse{under a slave when he becomes king, and a fool when he is satisfied with food;}%
\verse{under an unloved woman when she gets married, and a maid when she succeeds her mistress.}%
\verse{There are four small things on the earth, and they are exceedingly wise:\lebnote{“wise \textit{ones} from wise \textit{ones}”}}%
\verse{The ants are a people who are not strong, yet they prepare their food in the summer;}%
\verse{the badgers are a people who are not mighty, yet they set their house on the rock;}%
\verse{there is no king for the locust, yet it marches in rank;}%
\verse{a lizard you can seize with hands, yet it is in palaces of kings.}%
\verse{There are three things that are magnificent of stride, and four that are magnificent when moving:}%
\verse{a mighty lion among the beasts,\lebnote{Hebrew “beasts”} but he will not turn back from any face;\lebnote{“faces of all”}}%
\verse{a strutting rooster or he-goat, and a king whose army is with him.}%
\verse{If you have been foolish by exalting yourself, and if you have devised evil, put your hand to your mouth.}%
\verse{For pressing milk produces curd, and pressing the nose produces blood, so pressing anger\lebnote{“nostrils”} produces strife.}%
\end{biblechapter}%
\begin{biblechapter}% Proverbs 31
\verseWithHeading{The Oracle of King Lemuel}{The words of Lemuel, the king — an oracle that his mother taught him:}%
\verse{What, my son? And what, my son in my womb? And what, son of my vows?}%
\verse{Do not give your strength to the women, and your ways to destroy kings.}%
\verse{It is not for the kings, O Lemuel; drinking wine is not for the kings, nor is strong drink for rulers.}%
\verse{Or else he will drink and forget what has been decreed, and he will pervert the rights\lnBUL{} of all the afflicted.\lebnote{“sons of affliction”}}%
\verse{Give strong drink to him who is perishing, and wine to those in bitter distress.\lebnote{“to bitterness of soul”}}%
\verse{He will drink and forget his poverty, and his misery he will not remember any more.}%
\verse{Open your mouth for the mute, for the rights\lnBUL{} of all the needy.\lebnote{“the sons of the needy”}}%
\verse{Open your mouth, judge righteousness, and defend the poor and needy.}%
\verseWithHeading{An Excellent Woman}{A woman\lebnote{Or “wife”} of excellence,\lnBUM{} who will find? For her worth is far more than precious jewels.\lebnote{Prov 31:10–31 is an acrostic poem}}%
\verse{The heart of her husband\lnBUN{} trusts in her, and gain he will not lack.}%
\verse{She does him good, but not harm all the days of her life.}%
\verse{She seeks wool and flax, and she works with the diligence of her hands.\lebnote{“her palms”}}%
\verse{She is like the ships of a merchant; from far off she brings her food;\lebnote{Or “bread”}}%
\verse{And she arises while it is still night, and gives food to her household, and tasks to her servant girls.}%
\verse{She considers a field and buys it, from the fruit of her hand\lebnote{“her palm”} she plants a vineyard.}%
\verse{She girds her waist in strength, and makes her arms strong.}%
\verse{She perceives that her merchandise is good; her lamp does not go out in the night.}%
\verse{Her hands she puts onto the distaff,\lebnote{A stick or spindle onto which wool or flax is wound in preparation for spinning} and her palms hold a spindle.}%
\verse{Her palm she opens to the poor, and her hand she reaches out to the needy.}%
\verse{She does not fear for her house when it snows, for her entire household\lebnote{“all her house, household”} is clothed in crimson.}%
\verse{She makes for herself coverings; her clothing is fine linen and purple.}%
\verse{Her husband\lnBUN{} is known at the gates, in his seat among the elders of the land.}%
\verse{Linen garments she makes and sells, and she supplies sashes for the merchants.}%
\verse{Strength and dignity are her clothing, and she laughs at the future.\lebnote{“to the day/time coming after”}}%
\verse{She opens her mouth with wisdom, and instruction of kindness is upon her tongue.}%
\verse{She looks after the ways of her household, and the bread of idleness she will not eat.}%
\verse{Her children rise and consider her happy, her husband\lnBUN{} also, and he praises her;}%
\verse{“Many daughters have done excellence,\lnBUM{} but you surpass\lebnote{“you ascend over”} all of them.”}%
\verse{Charm\lebnote{Hebrew “The charm”} is deceit and beauty\lebnote{Hebrew “the beauty”} is vain;\lebnote{Or “vapor,” or “emptiness,” or “breath”} but a woman who fears Adonai shall be praised.}%
\verse{Give to her from the fruit of her hand, and may they praise her works in the city gates.}%
\end{biblechapter}%
\flushcolsend
\biblebook{Ecclesiastes}
\begin{biblechapter}% Ecclesiastes 1
\verseWithHeading{Prologue}{דברי The words קהלת of the Preacher, בן the son דוד of David, מלך king בירושׁלם׃ in Jerusalem.}%
\verseWithHeading{Motto Introduced}{הבל Vanity הבלים of vanities, אמר saith קהלת the Preacher, הבל vanity הבלים of vanities; הכל all הבל׃ vanity.}%
\verseWithHeading{All Toil is Profitless and Repetitious}{מה What יתרון profit לאדם hath a man בכל of all עמלו his labor שׁיעמל which he taketh תחת under השׁמשׁ׃ the sun?}%
\verse{דור generation הלך passeth away, ודור and generation בא cometh: והארץ but the earth לעולם forever. עמדת׃ abideth}%
\verse{וזרח also ariseth, השׁמשׁ The sun ובא goeth down, השׁמשׁ and the sun ואל to מקומו his place שׁואף and hasteth זורח arose. הוא he שׁם׃ where}%
\verse{הולך goeth אל toward דרום the south, וסובב and turneth about אל unto צפון the north; סובב continually, סבב continually, הולך it whirleth about הרוח The wind ועל according סביבתיו to his circuits. שׁב returneth again הרוח׃ and the wind}%
\verse{כל All הנחלים the rivers הלכים run אל into הים the sea; והים yet the sea איננו not מלא full; אל unto מקום the place שׁהנחלים from whence the rivers הלכים come, שׁם thither הם they שׁבים return ללכת׃ again.}%
\verse{כל All הדברים things יגעים full of labor; לא cannot יוכל cannot אישׁ man לדבר utter לא is not תשׂבע satisfied עין the eye לראות with seeing, ולא nor תמלא filled אזן the ear משׁמע׃ with hearing.}%
\verse{מה The thing שׁהיה that hath been, הוא it שׁיהיה which shall be; ומה and that שׁנעשׂה which is done הוא that שׁיעשׂה which shall be done: ואין and no כל and no חדשׁ new תחת under השׁמשׁ׃ the sun.}%
\verse{ישׁ Is there דבר thing שׁיאמר whereof it may be said, ראה See, זה this חדשׁ new? הוא it כבר already היה hath been לעלמים of old time, אשׁר which היה was מלפננו׃ before}%
\verse{אין no זכרון remembrance לראשׁנים of former וגם neither לאחרנים after. שׁיהיו shall there be לא ; neither יהיה of that are to come להם זכרון remembrance עם with שׁיהיו that shall come לאחרנה׃}%
\verseWithHeading{Qohelet Introduces His Quest}{אני I קהלת the Preacher הייתי was מלך king על over ישׂראל Israel בירושׁלם׃ in Jerusalem.}%
\verse{ונתתי And I gave את לבי my heart לדרושׁ to seek ולתור and search out בחכמה by wisdom על concerning כל all אשׁר that נעשׂה are done תחת under השׁמים heaven: הוא this ענין travail רע sore נתן given אלהים hath God לבני to the sons האדם of man לענות׃ to be exercised}%
\verse{ראיתי I have seen את כל all המעשׂים the works שׁנעשׂו that are done תחת under השׁמשׁ the sun; והנה and, behold, הכל all הבל vanity ורעות and vexation רוח׃ of spirit.}%
\verse{מעות crooked לא cannot יוכל cannot לתקן be made straight: וחסרון and that which is wanting לא cannot יוכל cannot להמנות׃ be numbered.}%
\verse{דברתי communed אני I עם with לבי mine own heart, לאמר saying, אני I הנה Lo, הגדלתי am come to great estate, והוספתי and have gotten more חכמה wisdom על than כל all אשׁר that היה have been לפני before על me in ירושׁלם Jerusalem: ולבי yea, my heart ראה experience הרבה had great חכמה of wisdom ודעת׃ and knowledge.}%
\verse{ואתנה And I gave לבי my heart לדעת to know חכמה wisdom, ודעת and to know הוללות madness ושׂכלות and folly: ידעתי I perceived שׁגם also זה that this הוא רעיון vexation רוח׃ of spirit.}%
\verse{כי For ברב in much חכמה wisdom רב כעס grief: ויוסיף and he that increaseth דעת knowledge יוסיף increaseth מכאוב׃ sorrow.}%
\end{biblechapter}%
\begin{biblechapter}% Ecclesiastes 2
\verseWithHeading{Qohelet’s Investigation of Self-Indulgence}{אמרתי said אני I בלבי in mine heart, לכה Go to נא now, אנסכה I will prove בשׂמחה thee with mirth, וראה therefore enjoy בטוב pleasure: והנה and, behold, גם also הוא this הבל׃ vanity.}%
\verse{לשׂחוק of laughter, אמרתי I said מהולל mad: ולשׂמחה and of mirth, מה What זה it? עשׂה׃ doeth}%
\verse{תרתי I sought בלבי in mine heart למשׁוך to give ביין unto wine, את בשׂרי myself ולבי mine heart נהג yet acquainting בחכמה with wisdom; ולאחז and to lay hold בסכלות on folly, עד till אשׁר till אראה I might see אי what זה that טוב good לבני for the sons האדם of men, אשׁר which יעשׂו they should do תחת under השׁמים the heaven מספר all ימי the days חייהם׃ of their life.}%
\verseWithHeading{Qohelet’s Investigation of Personal Accomplishment}{הגדלתי I made me great מעשׂי works; בניתי I built לי בתים me houses; נטעתי I planted לי כרמים׃ me vineyards:}%
\verse{עשׂיתי I made לי גנות me gardens ופרדסים and orchards, ונטעתי and I planted בהם עץ trees כל in them of all פרי׃ fruits:}%
\verse{עשׂיתי I made לי ברכות me pools מים of water, להשׁקות to water מהם יער the wood צומח that bringeth forth עצים׃ trees:}%
\verse{קניתי I got עבדים servants ושׁפחות and maidens, ובני servants born בית in my house; היה and had לי גם also מקנה possessions בקר of great וצאן and small cattle הרבה great היה I had לי מכל above all שׁהיו that were לפני before בירושׁלם׃ in Jerusalem}%
\verse{כנסתי I gathered לי גם me also כסף silver וזהב and gold, וסגלת and the peculiar treasure מלכים of kings והמדינות and of the provinces: עשׂיתי I got לי שׁרים me men singers ושׁרות and women singers, ותענוגת and the delights בני of the sons האדם of men, שׁדה musical instruments, ושׁדות׃ musical instruments,}%
\verse{וגדלתי So I was great, והוספתי and increased more מכל than all שׁהיה that were לפני before בירושׁלם me in Jerusalem: אף also חכמתי my wisdom עמדה׃ remained}%
\verse{וכל from any אשׁר שׁאלו desired עיני mine eyes לא not אצלתי I kept מהם לא not מנעתי them, I withheld את לבי my heart מכל in all שׂמחה joy; כי for לבי my heart שׂמח מכל of all עמלי my labor: וזה and this היה was חלקי my portion מכל עמלי׃ my labor.}%
\verse{ופניתי looked אני Then I בכל on all מעשׂי the works שׁעשׂו to do: ידי ובעמל and on the labor שׁעמלתי לעשׂותהנה and, behold, הכל all הבל vanity ורעות and vexation רוח of spirit, ואין and no יתרון profit תחת under השׁמשׁ׃ the sun.}%
\verseWithHeading{The Living Must Abandon the Work of their Hands to Others at Death}{ופניתי turned אני And I לראות myself to behold חכמה wisdom, והוללות and madness, וסכלות and folly: כי for מה what האדם the man שׁיבוא that cometh אחרי after המלך the king? את אשׁר that which כבר hath been already עשׂוהו׃ done.}%
\verse{וראיתי saw אני Then I שׁישׁ that יתרון excelleth לחכמה wisdom מן folly, הסכלות folly, כיתרון excelleth האור as far as light מן darkness. החשׁך׃ darkness.}%
\verse{החכם The wise man's עיניו eyes בראשׁו in his head; והכסיל but the fool בחשׁך in darkness: הולך walketh וידעתי perceived גם also אני and I myself שׁמקרה אחדקרה happeneth את כלם׃ them all.}%
\verse{ואמרתי Then said אני I בלבי in my heart, כמקרה As it happeneth הכסיל to the fool, גם even אני was I יקרני it happeneth ולמה to me; and why חכמתי wise? אני אז then יותר more ודברתי Then I said בלבי in my heart, שׁגם also זה that this הבל׃ vanity.}%
\verse{כי For אין no זכרון remembrance לחכם of the wise עם more than הכסיל of the fool לעולם forever; בשׁכבר seeing that which now הימים in the days הבאים to come הכל shall all נשׁכח be forgotten. ואיך And how ימות dieth החכם the wise עם as הכסיל׃ the fool.}%
\verse{ושׂנאתי Therefore I hated את החיים life; כי because רע grievous עלי unto המעשׂה the work שׁנעשׂה that is wrought תחת under השׁמשׁ the sun כי me: for הכל all הבל vanity ורעות and vexation רוח׃ of spirit.}%
\verse{ושׂנאתי hated אני Yea, I את כל all עמלי my labor שׁאני עמל had taken תחת under השׁמשׁ the sun: שׁאניחנו לאדם it unto the man שׁיהיה that shall be אחרי׃ after}%
\verse{ומי And who יודע knoweth החכם a wise יהיה whether he shall be או or סכל a fool? וישׁלט yet shall he have rule בכל over all עמלי my labor שׁעמלתי ושׁחכמתיחת under השׁמשׁ the sun. גם also זה This הבל׃ vanity.}%
\verse{וסבותי went about אני Therefore I ליאשׁ to despair את לבי to cause my heart על of כל all העמל the labor שׁעמלתי תחת under השׁמשׁ׃ the sun.}%
\verse{כי For ישׁ there is אדם a man שׁעמלו whose labor בחכמה in wisdom, ובדעת and in knowledge, ובכשׁרון and in equity; ולאדם yet to a man שׁלא that hath not עמל labored בו יתננו therein shall he leave חלקו it his portion. גם also זה This הבל vanity ורעה evil. רבה׃ and a great}%
\verse{כי For מה what הוה hath לאדם man בכל of all עמלו his labor, וברעיון and of the vexation לבו of his heart, שׁהוא wherein he עמל hath labored תחת under השׁמשׁ׃ the sun?}%
\verse{כי For כל all ימיו his days מכאבים sorrows, וכעס grief; ענינו and his travail גם yea, בלילה in the night. לא taketh not שׁכב rest לבו his heart גם is also זה This הבל vanity. הוא׃}%
\verseWithHeading{It is Best to Simply Enjoy the Passing Pleasures of Life as Reward for Pleasing God}{אין nothing טוב better באדם for a man, שׁיאכל that he should eat ושׁתה and drink, והראה אתפשׁו טוב good בעמלו in his labor. גם also זה This ראיתי saw, אני I כי that מיד from the hand האלהים of God. היא׃ it}%
\verse{כי For מי who יאכל can eat, ומי or who יחושׁ else can hasten חוץ more ממני׃ than}%
\verse{כי For לאדם to a man שׁטוב that good לפניו in his sight נתן giveth חכמה wisdom, ודעת and knowledge, ושׂמחה and joy: ולחוטא but to the sinner נתן he giveth ענין travail, לאסוף to gather ולכנוס and to heap up, לתת that he may give לטוב to good לפני before האלהים God. גם also זה This הבל vanity ורעות and vexation רוח׃ of spirit.}%
\end{biblechapter}%
\begin{biblechapter}% Ecclesiastes 3
\verseWithHeading{God Has Ordained the Ebb and Flow of Human Activities}{לכל To every זמן a season, ועת and a time לכל to every חפץ purpose תחת under השׁמים׃ the heaven:}%
\verse{עת A time ללדת to be born, ועת and a time למות to die; עת a time לטעת to plant, ועת and a time לעקור to pluck up נטוע׃ planted;}%
\verse{עת A time להרוג to kill, ועת and a time לרפוא to heal; עת a time לפרוץ to break down, ועת and a time לבנות׃ to build up;}%
\verse{עת A time לבכות to weep, ועת and a time לשׂחוק to laugh; עת a time ספוד to mourn, ועת and a time רקוד׃ to dance;}%
\verse{עת A time להשׁליך to cast away אבנים stones, ועת and a time כנוס to gather stones together; אבנים to gather stones together; עת a time לחבוק to embrace, ועת and a time לרחק to refrain מחבק׃ from embracing;}%
\verse{עת A time לבקשׁ to get, ועת and a time לאבד to lose; עת a time לשׁמור to keep, ועת and a time להשׁליך׃ to cast away;}%
\verse{עת A time לקרוע to rend, ועת and a time לתפור to sew; עת a time לחשׁות to keep silence, ועת and a time לדבר׃ to speak;}%
\verse{עת A time לאהב to love, ועת and a time לשׂנא to hate; עת a time מלחמה of war, ועת and a time שׁלום׃ of peace.}%
\verse{מה What יתרון profit העושׂה hath he that worketh באשׁר in that wherein הוא he עמל׃ laboreth?}%
\verseWithHeading{No One Understands God’s Mysterious Plan}{ראיתי I have seen את הענין the travail, אשׁר which נתן hath given אלהים God לבני to the sons האדם of men לענות׃ to be exercised}%
\verse{את הכל every עשׂה He hath made יפה beautiful בעתו in his time: גם also את העלם the world נתן he hath set בלבם in their heart, מבלי no אשׁר so that לא ימצא can find out האדם man את המעשׂה the work אשׁר that עשׂה maketh האלהים God מראשׁ from the beginning ועד to סוף׃ the end.}%
\verse{ידעתי I know כי that אין no טוב good בם כי for אם in them, but לשׂמוח to rejoice, ולעשׂות and to do טוב good בחייו׃ in his life.}%
\verse{וגם And also כל of all האדם שׁיאכלשׁתה and drink, וראה and enjoy טוב the good בכל עמלו his labor, מתת the gift אלהים of God. היא׃ it}%
\verse{ידעתי I know כי that, כל whatsoever אשׁר whatsoever יעשׂה doeth, האלהים God הוא it יהיה shall be לעולם forever: עליו to אין nothing להוסיף can be put וממנו from אין it, nor לגרע any thing taken והאלהים it: and God עשׂה doeth שׁיראו , that should fear מלפניו׃ before}%
\verse{מה שׁהיה is to be כבר is now; הוא and that ואשׁר which להיות been; כבר hath already היה והאלהים and God יבקשׁ requireth את נרדף׃ that which is past.}%
\verseWithHeading{God’s Mysterious Plan Allows Injustice to Exist in the World}{ועוד And moreover ראיתי I saw תחת under השׁמשׁ the sun מקום the place המשׁפט of judgment, שׁמה there; הרשׁע wickedness ומקום and the place הצדק of righteousness, שׁמה there. הרשׁע׃ iniquity}%
\verse{אמרתי said אני I בלבי in mine heart, את הצדיק the righteous ואת הרשׁע and the wicked: ישׁפט shall judge האלהים God כי for עת a time לכל for every חפץ purpose ועל and for כל every המעשׂה work. שׁם׃ there}%
\verse{אמרתי said אני I בלבי in mine heart על concerning דברת the estate בני of the sons האדם of men, לברם might manifest האלהים that God ולראות might see שׁהם them, and that they בהמה themselves are beasts. המה להם׃}%
\verse{כי For מקרה that which befalleth בני the sons האדם of men ומקרה befalleth הבהמה beasts; ומקרה befalleth אחד even one thing להם כמות dieth, זה them: as the one כן so מות dieth זה the other; ורוח breath; אחד one לכל yea, they have all ומותר preeminence האדם so that a man מן above הבהמה a beast: אין hath no כי for הכל all הבל׃ vanity.}%
\verse{הכל All הולך go אל unto מקום place; אחד one הכל all היה are מן of העפר the dust, והכל and all שׁב turn to dust again. אל העפר׃}%
\verse{מי Who יודע knoweth רוח the spirit בני האדם of man העלה goeth היא that למעלה upward, ורוח and the spirit הבהמה of the beast הירדת goeth היא that למטה downward לארץ׃ to the earth?}%
\verse{וראיתי Wherefore I perceive כי that אין nothing טוב better, מאשׁר than that ישׂמח should rejoice האדם a man במעשׂיו in his own works; כי for הוא that חלקו his portion: כי for מי who יביאנו shall bring לראות him to see במה what shall be שׁיהיה אחריו׃ after}%
\end{biblechapter}%
\begin{biblechapter}% Ecclesiastes 4
\verseWithHeading{The Existence of Oppression in the World Makes Human Existence Miserable}{ושׁבתי returned, אני So I ואראה and considered את כל all העשׁקים the oppressions אשׁר that נעשׂים are done תחת under השׁמשׁ the sun: והנה and behold דמעת the tears העשׁקים of oppressed, ואין and they had no להם מנחם comforter; ומיד and on the side עשׁקיהם of their oppressors כח power; ואין but they had no להם מנחם׃ comforter.}%
\verse{ושׁבח praised אני Wherefore I את המתים the dead שׁכבר which are already מתו dead מן more than החיים the living אשׁר which המה which חיים alive. עדנה׃ are yet}%
\verse{וטוב Yea, better משׁניהם than both את אשׁר they, which עדן yet לא hath not היה been, אשׁר who לא hath not ראה seen את המעשׂה work הרע the evil אשׁר that נעשׂה is done תחת under השׁמשׁ׃ the sun.}%
\verseWithHeading{People Need Balance in Their Approach to Labor}{וראיתי considered אני Again, I את כל all עמל travail, ואת כל and every כשׁרון right המעשׂה work, כי that היא for this קנאת is envied אישׁ a man מרעהו of his neighbor. גם also זה This הבל vanity ורעות and vexation רוח׃ of spirit.}%
\verse{הכסיל The fool חבק אתדיו ואכל and eateth את בשׂרו׃ his own flesh.}%
\verse{טוב Better מלא full כף נחת quietness, ממלא חפנים than both the hands עמל travail ורעות and vexation רוח׃ of spirit.}%
\verseWithHeading{Wealth without Someone with Which to Enjoy It is Futile}{ושׁבתי returned, אני Then I ואראה and I saw הבל vanity תחת under השׁמשׁ׃ the sun.}%
\verse{ישׁ There is אחד one ואין and not שׁני a second; גם yea, בן child ואח nor brother: אין he hath neither לו ואין yet no קץ end לכל of all עמלו his labor; גם also עיניו is his eye לא neither תשׂבע satisfied עשׁר with riches; ולמי neither For whom אני do I עמל labor, ומחסר and bereave את נפשׁי my soul מטובה of good? גם זה This הבל vanity, וענין travail. רע a sore הוא׃ yea, it}%
\verseWithHeading{Friends and Family Can Help One Another in Life}{טובים better השׁנים Two מן than האחד one; אשׁר because ישׁ they have להם שׂכר reward טוב a good בעמלם׃ for their labor.}%
\verse{כי For אם if יפלו they fall, האחד the one יקים will lift up את חברו his fellow: ואילו but woe האחד to him alone שׁיפול when he falleth; ואין for not שׁני another להקימו׃ to help him up.}%
\verse{גם Again, אם if ישׁכבו lie together, שׁנים two וחם then they have heat: להם ולאחד can one איך but how יחם׃ be warm}%
\verse{ואם And if יתקפו prevail against האחד one השׁנים him, two יעמדו shall withstand נגדו shall withstand והחוט cord המשׁלשׁ him; and a threefold לא is not במהרה quickly ינתק׃ broken.}%
\verseWithHeading{One Must Be Willing to Listen to Counsel}{טוב Better ילד child מסכן a poor וחכם and a wise ממלך king, זקן than an old וכסיל and foolish אשׁר who לא no ידע will להזהר be admonished. עוד׃ more}%
\verse{כי For מבית הסוריםצא he cometh למלך to reign; כי whereas גם also במלכותו in his kingdom נולד born רשׁ׃ becometh poor.}%
\verse{ראיתי I considered את כל all החיים the living המהלכים which walk תחת under השׁמשׁ the sun, עם with הילד child השׁני the second אשׁר that יעמד shall stand up תחתיו׃ in his stead.}%
\verse{אין no קץ end לכל of all העם the people, לכל of all אשׁר that היה have been לפניהם before גם them: they also האחרונים that come after לא shall not ישׂמחו rejoice בו כי in him. Surely גם also זה this הבל vanity ורעיון and vexation רוח׃ of spirit.}%
\end{biblechapter}%
\begin{biblechapter}% Ecclesiastes 5
\verseWithHeading{Listen to God Rather Than Uttering Rash Vows}{שׁמר Keep רגליך thy foot כאשׁר when תלך thou goest אל to בית the house האלהים of God, וקרוב לשׁמע to hear, מתת than to give הכסילים of fools: זבח the sacrifice כי that אינם not יודעים for they consider לעשׂות they do רע׃ evil.}%
\verse{אל Be not תבהל rash על with פיך thy mouth, ולבך thine heart אל and let not ימהר be hasty להוציא to utter דבר thing לפני before האלהים God: כי for האלהים God בשׁמים in heaven, ואתה and thou על upon הארץ earth: על therefore כן therefore יהיו be דבריך let thy words מעטים׃ few.}%
\verse{כי For בא cometh החלום a dream ברב through the multitude ענין of business; וקול voice כסיל and a fool's ברב by multitude דברים׃ of words.}%
\verse{כאשׁר When תדר thou vowest נדר a vow לאלהים unto God, אל not תאחר defer לשׁלמו to pay כי it; for אין no חפץ pleasure בכסילים in fools: את אשׁר that which תדר thou hast vowed. שׁלם׃ pay}%
\verse{טוב Better אשׁר that לא thou shouldest not תדר vow, משׁתדור ולא and not תשׁלם׃ pay.}%
\verse{אל not תתן Suffer את פיך thy mouth לחטיא to sin; את בשׂרך to cause thy flesh ואל neither תאמר say לפני thou before המלאך the angel, כי that שׁגגה an error: היא it למה wherefore יקצף be angry האלהים should God על at קולך thy voice, וחבל and destroy את מעשׂה the work ידיך׃ of thine hands?}%
\verse{כי For ברב in the multitude חלמות of dreams והבלים also vanities: ודברים words הרבה and many כי but את האלהים thou God. ירא׃ fear}%
\verseWithHeading{Powerful Bureaucrats Exploit the Helpless Poor}{אם If עשׁק the oppression רשׁ of the poor, וגזל and violent perverting משׁפט of judgment וצדק and justice תראה thou seest במדינה in a province, אל not תתמה marvel על at החפץ the matter: כי for גבה higher מעל than גבה the highest שׁמר regardeth; וגבהים and higher עליהם׃ than}%
\verse{ויתרון Moreover the profit ארץ of the earth בכל is for all: היא מלך the king לשׂדה by the field. נעבד׃ is served}%
\verseWithHeading{There is Never Enough Money to Satisfy}{אהב He that loveth כסף silver לא shall not ישׂבע be satisfied כסף with silver; ומי nor he אהב that loveth בהמון abundance לא תבואה with increase: גם also זה this הבל׃ vanity.}%
\verse{ברבות increase, הטובה When goods רבו אוכליה that eat ומה them: and what כשׁרון good לבעליה to the owners כי thereof, saving אם thereof, saving ראית the beholding עיניו׃ with their eyes?}%
\verse{מתוקה sweet, שׁנת The sleep העבד of a laboring man אם whether מעט little ואם or הרבה much: יאכל he eat והשׂבע but the abundance לעשׁיר of the rich איננו will not מניח לו לישׁון׃ him to sleep.}%
\verseWithHeading{Hoarding Wealth Can Backfire}{ישׁ There is רעה evil חולה a sore ראיתי I have seen תחת under השׁמשׁ the sun, עשׁר riches שׁמור kept לבעליו for the owners לרעתו׃ thereof to their hurt.}%
\verse{ואבד perish העשׁר riches ההוא But those בענין travail: רע by evil והוליד and he begetteth בן a son, ואין and nothing בידו in his hand. מאומה׃ and nothing}%
\verse{כאשׁר As יצא he came forth מבטן womb, אמו of his mother's ערום naked ישׁוב shall he return ללכת to go כשׁבא as he came, ומאומה nothing לא nothing ישׂא and shall take בעמלו of his labor, שׁילך which he may carry way בידו׃ in his hand.}%
\verse{וגם also זה And this רעה evil, חולה a sore כל in all עמת points שׁבא as he came, כן so ילך shall he go: ומה and what יתרון profit לו שׁיעמל hath he that hath labored לרוח׃ for the wind?}%
\verse{גם also כל All ימיו his days בחשׁך in darkness, יאכל he eateth וכעס הרבה and much וחליו with his sickness. וקצף׃ and wrath}%
\verseWithHeading{If You Have Wealth, Enjoy It as God Enables}{הנה Behold אשׁר which ראיתי have seen: אני I טוב good אשׁר which יפה and comely לאכול to eat ולשׁתות and to drink, ולראות and to enjoy טובה the good בכל of all עמלו his labor שׁיעמל that he taketh תחת under השׁמשׁ the sun מספר all ימי the days חיו of his life, אשׁר נתן giveth לו האלהים God כי him: for הוא it חלקו׃ his portion.}%
\verse{גם also כל Every האדם man אשׁר to whom נתן hath given לו האלהים God עשׁר riches ונכסים and wealth, והשׁליטו and hath given him power לאכל to eat ממנו thereof, ולשׂאת and to take את חלקו his portion, ולשׂמח and to rejoice בעמלו in his labor; זה this מתת the gift אלהים of God. היא׃}%
\verse{כי For לא he shall not הרבה much יזכר remember את ימי the days חייו of his life; כי because האלהים God מענה בשׂמחת in the joy לבו׃ of his heart.}%
\end{biblechapter}%
\begin{biblechapter}% Ecclesiastes 6
\verseWithHeading{Those Who Have Wealth but Do Not Enjoy It Are Pitiful}{ישׁ There is רעה an evil אשׁר which ראיתי I have seen תחת under השׁמשׁ the sun, ורבה common היא and it על among האדם׃ men:}%
\verse{אישׁ A man אשׁר to whom יתן hath given לו האלהים God עשׁר riches, ונכסים wealth, וכבוד and honor, ואיננו nothing חסר so that he wanteth לנפשׁו for his soul מכל of all אשׁר that יתאוה he desireth, ולא giveth him not power ישׁליטנו giveth him not power האלהים yet God לאכל to eat ממנו of all כי but אישׁ a stranger נכרי a stranger יאכלנו eateth זה it: this הבל vanity, וחלי disease. רע an evil הוא׃ and it}%
\verse{אם If יוליד beget אישׁ a man מאה a hundred ושׁנים years, רבות many יחיה and live ורב many, שׁיהיו be ימי so that the days שׁניו of his years ונפשׁו and his soul לא be not תשׂבע filled מן with הטובה good, וגם and also קבורה burial; לא no היתה he have לו אמרתי I say, טוב better ממנו than הנפל׃ an untimely birth}%
\verse{כי For בהבל in with vanity, בא he cometh ובחשׁך in darkness, ילך and departeth ובחשׁך with darkness. שׁמו and his name יכסה׃ shall be covered}%
\verse{גם Moreover שׁמשׁ the sun, לא he hath not ראה seen ולא nor ידע known נחת hath more rest לזה this מזה׃ than the other.}%
\verse{ואלו Yea, חיה though he live אלף a thousand שׁנים years פעמים twice וטובה good: לא no ראה yet hath he seen הלא do not אל to מקום place? אחד one הכל all הולך׃ go}%
\verseWithHeading{One Must Learn to Be Content with What One Has}{כל All עמל the labor האדם of man לפיהו for his mouth, וגם and yet הנפשׁ the appetite לא is not תמלא׃ filled.}%
\verse{כי For מה what יותר more לחכם hath the wise מן than הכסיל the fool? מה what לעני hath the poor, יודע that knoweth להלך to walk נגד before החיים׃ the living?}%
\verse{טוב Better מראה the sight עינים of the eyes מהלך than the wandering נפשׁ of the desire: גם also זה this הבל vanity ורעות and vexation רוח׃ of spirit.}%
\verseWithHeading{It is Futile for Humans to Complain about God’s Irresistible Will}{מה That שׁהיה which hath been כבר already, נקרא is named שׁמו is named ונודע and it is known אשׁר that הוא it אדם man: ולא neither יוכל may לדין he contend עם with שׁהתקיף ממנו׃ than}%
\verse{כי Seeing ישׁ there be דברים things הרבה many מרבים that increase הבל vanity, מה what יתר the better? לאדם׃ man}%
\verseWithHeading{The Future is Inscrutable to Humans}{כי For מי who יודע knoweth מה what טוב good לאדם for man בחיים in life, מספר all ימי the days חיי life הבלו of his vain ויעשׂם which he spendeth כצל as a shadow? אשׁר for מי who יגיד can tell לאדם a man מה what יהיה shall be אחריו after תחת him under השׁמשׁ׃ the sun?}%
\end{biblechapter}%
\begin{biblechapter}% Ecclesiastes 7
\verseWithHeading{People Generally Do Not Know What is Best for Them}{טוב better שׁם A good name משׁמן ointment; טוב than precious ויום and the day המות of death מיום than the day הולדו׃ of one's birth.}%
\verse{טוב better ללכת to go אל to בית the house אבל of mourning, מלכת than to go אל to בית the house משׁתה of feasting: באשׁר for הוא that סוף the end כל of all האדם men; והחי and the living יתן will lay אל to לבו׃ his heart.}%
\verse{טוב better כעס Sorrow משׂחק than laughter: כי for ברע by the sadness פנים of the countenance ייטב is made better. לב׃ the heart}%
\verse{לב The heart חכמים of the wise בבית in the house אבל of mourning; ולב but the heart כסילים of fools בבית in the house שׂמחה׃ of mirth.}%
\verse{טוב better לשׁמע to hear גערת the rebuke חכם of the wise, מאישׁ than for a man שׁמע to hear שׁיר the song כסילים׃ of fools.}%
\verse{כי For כקול as the crackling הסירים of thorns תחת under הסיר a pot, כן so שׂחק the laughter הכסיל of the fool: וגם also זה this הבל׃ vanity.}%
\verseWithHeading{Wisdom—Although Vulnerable—is Beneficial}{כי Surely העשׁק oppression יהולל mad; חכם maketh a wise man ויאבד destroyeth את לב the heart. מתנה׃ and a gift}%
\verse{טוב Better אחרית the end דבר of a thing מראשׁיתו than the beginning טוב better ארך thereof: the patient רוח in spirit מגבה than the proud רוח׃ in spirit.}%
\verse{אל Be not תבהל hasty ברוחך in thy spirit לכעוס to be angry: כי for כעס anger בחיק in the bosom כסילים of fools. ינוח׃ resteth}%
\verse{אל not תאמר Say מה thou, What היה is שׁהימים הראשׁניםיו were טובים better מאלה than these? כי for לא thou dost not מחכמה wisely שׁאלת inquire על concerning זה׃ this.}%
\verse{טובה good חכמה Wisdom עם with נחלה an inheritance: ויתר and profit לראי to them that see השׁמשׁ׃ the sun.}%
\verse{כי For בצל a defense, החכמה wisdom בצל a defense: הכסף money ויתרון but the excellency דעת of knowledge החכמה wisdom תחיה giveth life בעליה׃ to them that have}%
\verseWithHeading{Humans Must Accept God’s Will and Make the Best of It}{ראה Consider את מעשׂה the work האלהים of God: כי for מי who יוכל can לתקן make straight, את אשׁר which עותו׃ he hath made crooked?}%
\verse{ביום In the day טובה of prosperity היה be בטוב joyful, וביום but in the day רעה of adversity ראה consider: גם also את זה the one לעמת over against זה the other, עשׂה hath set האלהים God על to דברת the end שׁלא ימצא should find האדם that man אחריו after מאומה׃}%
\verseWithHeading{The Law of Retribution Does Not Always Work—but It Does Sometimes}{את הכל All ראיתי have I seen בימי in the days הבלי of my vanity: ישׁ there is צדיק a just אבד that perisheth בצדקו in his righteousness, וישׁ and there is רשׁע a wicked מאריך that prolongeth ברעתו׃ in his wickedness.}%
\verse{אל not תהי Be צדיק righteous הרבה over much; ואל neither תתחכם make thyself over wise: יותר make thyself over wise: למה why תשׁומם׃ shouldest thou destroy thyself?}%
\verse{אל Be not תרשׁע wicked, הרבה over much ואל neither תהי be סכל thou foolish: למה why תמות shouldest thou die בלא before עתך׃ thy time?}%
\verse{טוב good אשׁר that תאחז thou shouldest take hold בזה of this; וגם yea, also מזה from this אל not תנח אתדך thine hand: כי for ירא he that feareth אלהים God יצא shall come forth את כלם׃ of them all.}%
\verseWithHeading{Wisdom is Valuable, but No One is Completely Righteous}{החכמה Wisdom תעז strengtheneth לחכם the wise מעשׂרה more than ten שׁליטים mighty אשׁר which היו are בעיר׃ in the city.}%
\verse{כי For אדם man אין not צדיק a just בארץ upon earth, אשׁר that יעשׂה doeth טוב good, ולא not. יחטא׃ and sinneth}%
\verse{גם Also לכל unto all הדברים words אשׁר that ידברו are spoken; אל no תתן take לבך heed אשׁר lest לא lest תשׁמע thou hear את עבדך thy servant מקללך׃ curse}%
\verse{כי For גם also פעמים oftentimes רבות oftentimes ידע knoweth לבך thine own heart אשׁר that גם likewise את thou thyself קללת hast cursed אחרים׃ others.}%
\verseWithHeading{Absolute Wisdom is Unattainable}{כל All זה this נסיתי have I proved בחכמה by wisdom: אמרתי I said, אחכמה I will be wise; והיא but it רחוקה far ממני׃ from}%
\verse{רחוק far off, מה שׁהיהעמק and exceeding deep, עמק and exceeding deep, מי who ימצאנו׃ can find it out?}%
\verse{סבותי applied אני I ולבי mine heart לדעת to know, ולתור and to search, ובקשׁ and to seek out חכמה wisdom, וחשׁבון and the reason ולדעת רשׁע the wickedness כסל of folly, והסכלות even of foolishness הוללות׃ madness:}%
\verse{ומוצא find אני And I מר more bitter ממות than death את האשׁה the woman, אשׁר whose היא whose מצודים snares וחרמים and nets, לבה heart אסורים bands: ידיה her hands טוב whoso pleaseth לפני whoso pleaseth האלהים God ימלט shall escape ממנה than death וחוטא her; but the sinner ילכד׃ shall be taken}%
\verse{ראה Behold, זה this מצאתי have I found, אמרה saith קהלת the preacher, אחת one לאחת by one, למצא to find out חשׁבון׃ the account:}%
\verse{אשׁר Which עוד yet בקשׁה seeketh, נפשׁי my soul ולא not: מצאתי but I find אדם man אחד one מאלף among a thousand מצאתי have I found; ואשׁה but a woman בכל among all אלה those לא have I not מצאתי׃ found.}%
\verse{לבד only ראה Lo, זה this מצאתי have I found, אשׁר that עשׂה hath made האלהים God את האדם man ישׁר upright; והמה but they בקשׁו have sought out חשׁבנות inventions. רבים׃ many}%
\end{biblechapter}%
\begin{biblechapter}% Ecclesiastes 8
\verseWithHeading{Wisdom is Valuable}{מי Who כהחכם as the wise ומי and who יודע knoweth פשׁר the interpretation דבר of a thing? חכמת wisdom אדם a man's תאיר to shine, פניו maketh his face ועז and the boldness פניו of his face ישׁנא׃ shall be changed.}%
\verse{אני I פי commandment, מלך the king's שׁמור to keep ועל and in regard דברת and in regard שׁבועת of the oath אלהים׃ of God.}%
\verse{אל Be not תבהל hasty מפניו out of his sight: תלך to go אל not תעמד stand בדבר thing; רע in an evil כי for כל whatsoever אשׁר whatsoever יחפץ pleaseth יעשׂה׃ he doeth}%
\verse{באשׁר Where דבר the word מלך of a king שׁלטון power: ומי and who יאמר may say לו מה unto him, What תעשׂה׃ doest}%
\verse{שׁומר Whoso keepeth מצוה the commandment לא no ידע shall feel דבר thing: רע evil ועת both time ומשׁפט and judgment. ידע discerneth לב heart חכם׃ and a wise man's}%
\verse{כי Because לכל to every חפץ purpose ישׁ there is עת time ומשׁפט and judgment, כי therefore רעת the misery האדם of man רבה great עליו׃ upon}%
\verseWithHeading{No One Knows the Future}{כי For איננו not ידע he knoweth מה that שׁיהיה which shall be: כי for כאשׁר him when יהיה it shall be? מי who יגיד׃ can tell}%
\verse{אין no אדם man שׁליט that hath power ברוח over the spirit לכלוא to retain את הרוח the spirit; ואין neither שׁלטון power ביום in the day המות of death: ואין and no משׁלחת discharge במלחמה in war; ולא neither ימלט deliver רשׁע shall wickedness את בעליו׃ those that are given}%
\verseWithHeading{The World Marred by Oppression and Injustice}{את כל All זה this ראיתי have I seen, ונתון and applied את לבי my heart לכל unto every מעשׂה work אשׁר that נעשׂה is done תחת under השׁמשׁ the sun: עת a time אשׁר wherein שׁלט ruleth over האדם one man באדם another לרע׃ to his own hurt.}%
\verse{ובכן And so ראיתי I saw רשׁעים the wicked קברים buried, ובאו who had come וממקום from the place קדושׁ of the holy, יהלכו and gone וישׁתכחו and they were forgotten בעיר in the city אשׁר where כן they had so עשׂו done: גם also זה this הבל׃ vanity.}%
\verseWithHeading{Although Evil is Not Punished Swiftly, God Does Eventually Punish Sinners}{אשׁר Because אין is not נעשׂה executed פתגם sentence against מעשׂה work הרעה an evil מהרה speedily, על therefore כן therefore מלא is fully set לב the heart בני of the sons האדם of men בהם לעשׂות in them to do רע׃ evil.}%
\verse{אשׁר Though חטא a sinner עשׂה do רע evil מאת a hundred times, ומאריך and his be prolonged, לו כי yet גם surely יודע know אני I אשׁר that יהיה it shall be טוב well ליראי fear האלהים God, אשׁר which ייראו מלפניו׃ before}%
\verse{וטוב well לא But it shall not יהיה be לרשׁע with the wicked, ולא neither יאריך shall he prolong ימים days, כצל as a shadow; אשׁר because איננו not ירא מלפני before אלהים׃ God.}%
\verse{ישׁ There is הבל a vanity אשׁר which נעשׂה is done על upon הארץ the earth; אשׁר that ישׁ there be צדיקים just אשׁר unto whom מגיע it happeneth אלהם , unto whom כמעשׂה according to the work הרשׁעים of the wicked; וישׁ again, there be רשׁעים wicked שׁמגיע whom it happeneth אלהם to כמעשׂה according to the work הצדיקים of the righteous: אמרתי I said שׁגם זה that this also הבל׃ vanity.}%
\verseWithHeading{Humans Should Enjoy the Life That God Gives to Them}{ושׁבחתי commended אני Then I את השׂמחה mirth, אשׁר because אין hath no טוב better לאדם a man תחת thing under השׁמשׁ the sun, כי than אם than לאכול to eat, ולשׁתות and to drink, ולשׂמוח and to be merry: והוא for that ילונו shall abide בעמלו with him of his labor ימי the days חייו of his life, אשׁר which נתן giveth לו האלהים God תחת him under השׁמשׁ׃ the sun.}%
\verseWithHeading{No One Can Discover the Rhyme and Reason for Things}{כאשׁר When נתתי I applied את לבי mine heart לדעת to know חכמה wisdom, ולראות and to see את הענין the business אשׁר that נעשׂה is done על upon הארץ the earth: כי (for גם also ביום day ובלילה nor night שׁנה sleep בעיניו with his eyes:) איננו neither ראה׃ seeth}%
\verse{וראיתי Then I beheld את כל all מעשׂה the work האלהים of God, כי that לא cannot יוכל cannot האדם a man למצוא find out את המעשׂה the work אשׁר that נעשׂה is done תחת under השׁמשׁ the sun: בשׁל because אשׁר though יעמל labor האדם a man לבקשׁ to seek out, ולא yet he shall not ימצא find וגם yea אם further; though יאמר think החכם a wise לדעת to know לא yet shall he not יוכל be able למצא׃ to find}%
\end{biblechapter}%
\begin{biblechapter}% Ecclesiastes 9
\verseWithHeading{The Same Fate—Death—Awaits Everyone}{כי For את כל all זה this נתתי I considered אל in לבי my heart ולבור even to declare את כל all זה this, אשׁר that הצדיקים the righteous, והחכמים and the wise, ועבדיהם and their works, ביד in the hand האלהים of God: גם either אהבה love גם or שׂנאה hatred אין no יודע knoweth האדם man הכל all לפניהם׃ before}%
\verse{הכל All כאשׁר alike לכל to all: מקרה event אחד one לצדיק to the righteous, ולרשׁע and to the wicked; לטוב to the good ולטהור and to the clean, ולטמא and to the unclean; ולזבח to him that sacrificeth, ולאשׁר and to him that sacrificeth איננו not: זבח and to him that sacrificeth כטוב as the good, כחטא so the sinner; הנשׁבע he that sweareth, כאשׁר as שׁבועה an oath. ירא׃ that feareth}%
\verse{זה This רע an evil בכל among all אשׁר that נעשׂה are done תחת under השׁמשׁ the sun, כי that מקרה event אחד one לכל unto all: וגם yea, also לב the heart בני of the sons האדם of men מלא is full רע of evil, והוללות and madness בלבבם in their heart בחייהם while they live, ואחריו and after אל that to המתים׃ the dead.}%
\verseWithHeading{Death Deprives Humans of Everything in Life}{כי For מי to him אשׁר that יבחר אל to כל all החיים the living ישׁ there is בטחון hope: כי for לכלב dog חי a living הוא טוב is better מן than האריה lion. המת׃ a dead}%
\verse{כי For החיים the living יודעים know שׁימתו that they shall die: והמתים but the dead אינם not יודעים know מאומה any thing, ואין neither עוד have they any more להם שׂכר a reward; כי for נשׁכח of them is forgotten. זכרם׃ the memory}%
\verse{גם Also אהבתם their love, גם and שׂנאתם their hatred, גם and קנאתם their envy, כבר is now אבדה perished; וחלק a portion אין neither להם עוד have they any more לעולם forever בכל in any אשׁר that נעשׂה is done תחת under השׁמשׁ׃ the sun.}%
\verseWithHeading{Enjoy Life While It Lasts}{לך Go אכל thy way, eat בשׂמחה with joy, לחמך thy bread ושׁתה and drink בלב heart; טוב with a merry יינך thy wine כי for כבר now רצה accepteth האלהים God את מעשׂיך׃ thy works.}%
\verse{בכל always עת always יהיו be בגדיך Let thy garments לבנים white; ושׁמן ointment. על ראשׁך and let thy head אל no יחסר׃ lack}%
\verse{ראה חיים of the life עם with אשׁה the wife אשׁר whom אהבת thou lovest כל all ימי the days חיי in life, הבלך of thy vanity, אשׁר which נתן he hath given לך תחת thee under השׁמשׁ the sun, כל all ימי the days הבלך of thy vanity: כי for הוא that חלקך thy portion בחיים ובעמלך and in thy labor אשׁר which אתה thou עמל takest תחת under השׁמשׁ׃ the sun.}%
\verse{כל אשׁר whither תמצא findeth ידך thy hand לעשׂות to do, בכחך with thy might; עשׂה do כי for אין no מעשׂה work, וחשׁבון nor device, ודעת nor knowledge, וחכמה nor wisdom, בשׁאול in the grave, אשׁר אתה thou הלך goest. שׁמה׃ whither}%
\verseWithHeading{The Injustice of Time and Chance}{שׁבתי I returned, וראה and saw תחת under השׁמשׁ the sun, כי that לא not לקלים to the swift, המרוץ the race ולא nor לגבורים to the strong, המלחמה the battle וגם yet לא neither לחכמים to the wise, לחם bread וגם yet לא nor לנבנים to men of understanding, עשׁר riches וגם yet לא nor לידעים to men of skill; חן favor כי but עת time ופגע and chance יקרה happeneth את כלם׃ to them all.}%
\verse{כי For גם also לא not ידע knoweth האדם man את עתו his time: כדגים as the fishes שׁנאחזים that are taken במצודה net, רעה in an evil וכצפרים and as the birds האחזות that are caught בפח in the snare; כהם יוקשׁים snared בני so the sons האדם of men לעת time, רעה in an evil כשׁתפול when it falleth עליהם upon פתאם׃ suddenly}%
\verseWithHeading{Wisdom—Although Vulnerable—is Superior to Power}{גם also זה This ראיתי have I seen חכמה wisdom תחת under השׁמשׁ the sun, וגדולה great היא and it אלי׃ unto}%
\verse{עיר city, קטנה a little ואנשׁים בה מעט and few ובא within it; and there came אליה against מלך king גדול a great וסבב it, and besieged אתה ובנה it, and built עליה against מצודים bulwarks גדלים׃ great}%
\verse{ומצא Now there was found בה אישׁ man, מסכן in it a poor חכם wise ומלט delivered הוא and he את העיר the city; בחכמתו by his wisdom ואדם man לא yet no זכר remembered את האישׁ man. המסכן poor ההוא׃ that same}%
\verse{ואמרתי Then said אני I, טובה better חכמה Wisdom מגבורה than strength: וחכמת wisdom המסכן nevertheless the poor man's בזויה despised, ודבריו and his words אינם are not נשׁמעים׃ heard.}%
\verseWithHeading{Wisdom—Although Vulnerable—is Superior to Folly}{דברי The words חכמים of wise בנחת in quiet נשׁמעים heard מזעקת more than the cry מושׁל of him that ruleth בכסילים׃ among fools.}%
\verse{טובה better חכמה Wisdom מכלי than weapons קרב of war: וחוטא sinner אחד but one יאבד destroyeth טובה good. הרבה׃ much}%
\end{biblechapter}%
\begin{biblechapter}% Ecclesiastes 10
\verse{זבובי flies מות Dead יבאישׁ a stinking savor: יביע to send forth שׁמן cause the ointment רוקח of the apothecary יקר him that is in reputation מחכמה for wisdom מכבוד honor. סכלות folly מעט׃ a little}%
\verse{לב heart חכם A wise man's לימינו at his right hand; ולב heart כסיל but a fool's לשׂמאלו׃ at his left.}%
\verse{וגם Yea also, בדרך by the way, כשׁהסכל when he that is a fool הלך walketh לבו his wisdom חסר faileth ואמר and he saith לכל to every one סכל a fool. הוא׃ he}%
\verse{אם If רוח the spirit המושׁל of the ruler תעלה rise up עליך against מקומך thy place; אל not תנח כי for מרפא yielding יניח חטאים offenses. גדולים׃ great}%
\verse{ישׁ There is רעה an evil ראיתי I have seen תחת under השׁמשׁ the sun, כשׁגגה as an error שׁיצא which proceedeth מלפני from השׁליט׃ the ruler:}%
\verse{נתן is set הסכל Folly במרומים dignity, רבים in great ועשׁירים and the rich בשׁפל in low place. ישׁבו׃ sit}%
\verse{ראיתי I have seen עבדים servants על upon סוסים horses, ושׂרים and princes הלכים walking כעבדים as servants על upon הארץ׃ the earth.}%
\verseWithHeading{Accidents Happen—Even to Professionals}{חפר He that diggeth גומץ a pit בו יפול shall fall ופרץ into it; and whoso breaketh גדר a hedge, ישׁכנו shall bite נחשׁ׃ a serpent}%
\verse{מסיע Whoso removeth אבנים stones יעצב shall be hurt בהם בוקע therewith; he that cleaveth עצים wood יסכן׃ shall be endangered}%
\verseWithHeading{Hard Work and Skill Alone Cannot Succeed—Wisdom is Necessary}{אם If קהה be blunt, הברזל the iron והוא and he לא do not פנים the edge, קלקל whet וחילים more strength: יגבר then must he put to ויתרון profitable הכשׁיר to direct. חכמה׃ but wisdom}%
\verse{אם Surely ישׁך will bite הנחשׁ the serpent בלוא without לחשׁ enchantment; ואין is no יתרון better. לבעל and a babbler הלשׁון׃ and a babbler}%
\verseWithHeading{The Consequences of Foolishness}{דברי The words פי mouth חכם of a wise man's חן gracious; ושׂפתות but the lips כסיל of a fool תבלענו׃ will swallow up}%
\verse{תחלת The beginning דברי of the words פיהו of his mouth סכלות foolishness: ואחרית and the end פיהו of his talk הוללות madness. רעה׃ mischievous}%
\verse{והסכל A fool ירבה also is full דברים of words: לא cannot ידע tell האדם a man מה what shall be; שׁיהיה shall be ואשׁר and what יהיה מאחריו after מי him, who יגיד׃ can tell}%
\verse{עמל The labor הכסילים of the foolish תיגענו wearieth אשׁר every one of them, because לא not ידע he knoweth ללכת how to go אל to עיר׃ the city.}%
\verse{אי Woe לך ארץ to thee, O land, שׁמלכך when thy king נער a child, ושׂריך and thy princes בבקר in the morning! יאכלו׃ eat}%
\verse{אשׁריך Blessed ארץ thou, O land, שׁמלכך when thy king בן the son חורים of nobles, ושׂריך and thy princes בעת in due season, יאכלו eat בגבורה for strength, ולא and not בשׁתי׃ for drunkenness!}%
\verse{בעצלתים By much slothfulness ימך decayeth; המקרה the building ובשׁפלות and through idleness ידים of the hands ידלף droppeth through. הבית׃ the house}%
\verse{לשׂחוק for laughter, עשׂים is made לחם A feast ויין and wine ישׂמח maketh merry: חיים maketh merry: והכסף but money יענה answereth את הכל׃ all}%
\verse{גם no במדעך not in thy thought; מלך the king, אל not תקלל Curse ובחדרי in thy bedchamber: משׁכבך in thy bedchamber: אל not תקלל and curse עשׁיר the rich כי for עוף a bird השׁמים of the air יוליך shall carry את הקול the voice, ובעל and that which hath הכנפים wings יגיד shall tell דבר׃ the matter.}%
\end{biblechapter}%
\begin{biblechapter}% Ecclesiastes 11
\verseWithHeading{Living in the Light of the Limits of Human Knowledge}{שׁלח Cast לחמך thy bread על upon פני upon המים the waters: כי for ברב it after many הימים days. תמצאנו׃ thou shalt find}%
\verse{תן Give חלק a portion לשׁבעה to seven, וגם and also לשׁמונה to eight; כי for לא not תדע thou knowest מה what יהיה shall be רעה evil על upon הארץ׃ the earth.}%
\verse{אם If ימלאו be full העבים the clouds גשׁם of rain, על upon הארץ the earth: יריקו they empty ואם and if יפול fall עץ the tree בדרום toward the south, ואם or בצפון toward the north, מקום in the place שׁיפול falleth, העץ where the tree שׁם there יהוא׃ it}%
\verse{שׁמר He that observeth רוח the wind לא shall not יזרע sow; וראה and he that regardeth בעבים the clouds לא shall not יקצור׃ reap.}%
\verse{כאשׁר As אינך not יודע thou knowest מה what דרך the way הרוח of the spirit, כעצמים how the bones בבטן in the womb המלאה of her that is with child: ככה even so לא not תדע thou knowest את מעשׂה the works האלהים of God אשׁר who יעשׂה maketh את הכל׃ all.}%
\verse{בבקר In the morning זרע sow את זרעך thy seed, ולערב and in the evening אל not תנח ידך thine hand: כי for אינך not יודע thou knowest אי whether זה whether יכשׁר shall prosper, הזה either this או or זה that, ואם or whether שׁניהם they both כאחד alike טובים׃ good.}%
\verseWithHeading{Enjoy Life to the Fullest within the Auspices of the Fear of God}{ומתוק sweet, האור Truly the light וטוב and a pleasant לעינים for the eyes לראות to behold את השׁמשׁ׃ the sun:}%
\verse{כי But אם if שׁנים years, הרבה many יחיה live האדם a man בכלם in them all; ישׂמח rejoice ויזכר yet let him remember את ימי the days החשׁך of darkness; כי for הרבה many. יהיו they shall be כל All שׁבא that cometh הבל׃ vanity.}%
\verse{שׂמח Rejoice, בחור O young man, בילדותיך in thy youth; ויטיבך cheer לבך and let thy heart בימי thee in the days בחורותך of thy youth, והלך and walk בדרכי in the ways לבך of thine heart, ובמראי and in the sight עיניך of thine eyes: ודע but know כי thou, that על for כל all אלה these יביאך will bring האלהים God במשׁפט׃ thee into judgment.}%
\verse{והסר Therefore remove כעס sorrow מלבך from thy heart, והעבר and put away רעה evil מבשׂרך from thy flesh: כי for הילדות childhood והשׁחרות and youth הבל׃ vanity.}%
\end{biblechapter}%
\begin{biblechapter}% Ecclesiastes 12
\verseWithHeading{Advice to the Young: Life is Short and Then You Die}{וזכר Remember את בוראיך now thy Creator בימי in the days בחורתיך of thy youth, עד while אשׁר while לא not, יבאו come ימי days הרעה the evil והגיעו draw nigh, שׁנים nor the years אשׁר when תאמר thou shalt say, אין I have no לי בהם חפץ׃ pleasure}%
\verse{עד אשׁרא be not תחשׁך darkened, השׁמשׁ the sun, והאור or the light, והירח or the moon, והכוכבים or the stars, ושׁבו return העבים nor the clouds אחר after הגשׁם׃ the rain:}%
\verse{ביום In the day שׁיזעו shall tremble, שׁמרי when the keepers הבית of the house והתעותו shall bow themselves, אנשׁי men החיל and the strong ובטלו cease הטחנות and the grinders כי because מעטו וחשׁכו be darkened, הראות and those that look בארבות׃ out of the windows}%
\verse{וסגרו shall be shut דלתים And the doors בשׁוק in the streets, בשׁפל קול when the sound הטחנה of the grinding ויקום and he shall rise up לקול at the voice הצפור of the bird, וישׁחו shall be brought low; כל and all בנות the daughters השׁיר׃ of music}%
\verse{גם Also מגבה of high, יראו they shall be afraid וחתחתים and fears בדרך in the way, וינאץ shall flourish, השׁקד and the almond tree ויסתבל shall be a burden, החגב and the grasshopper ותפר shall fail: האביונה and desire כי because הלך goeth האדם man אל to בית home, עולמו his long וסבבו go about בשׁוק the streets: הספדים׃ and the mourners}%
\verse{עד Or אשׁר ever לא ever ירחק be loosed, חבל cord הכסף the silver ותרץ be broken, גלת bowl הזהב or the golden ותשׁבר be broken כד or the pitcher על at המבוע the fountain, ונרץ broken הגלגל or the wheel אל at הבור׃ the cistern.}%
\verse{וישׁב return העפר Then shall the dust על to הארץ the earth כשׁהיה as it was: והרוח and the spirit תשׁוב shall return אל unto האלהים God אשׁר who נתנה׃ gave}%
\verseWithHeading{Motto Restated}{הבל Vanity הבלים of vanities, אמר saith הקוהלת the preacher; הכל all הבל׃ vanity.}%
\verseWithHeading{Epilogue}{ויתר And moreover, שׁהיה was קהלת because the preacher חכם wise, עוד he still למד taught דעת knowledge; את העם the people ואזן yea, he gave good heed, וחקר and sought out, תקן set in order משׁלים proverbs. הרבה׃ many}%
\verse{בקשׁ sought קהלת The preacher למצא to find out דברי words: חפץ acceptable וכתוב and written ישׁר upright, דברי words אמת׃ of truth.}%
\verse{דברי The words חכמים of the wise כדרבנות as goads, וכמשׂמרות and as nails נטועים fastened בעלי the masters אספות of assemblies, נתנו are given מרעה אחד׃}%
\verse{ויתר And further, מהמה by these, בני my son, הזהר be admonished: עשׂות of making ספרים books הרבה many אין no קץ end; ולהג study הרבה and much יגעת a weariness בשׂר׃ of the flesh.}%
\verse{סוף the conclusion דבר matter: הכל of the whole נשׁמע Let us hear את האלהים God, ירא Fear ואת מצותיו his commandments: שׁמור and keep כי for זה this כל the whole האדם׃ of man.}%
\verse{כי For את כל every מעשׂה work האלהים God יבא shall bring במשׁפט into judgment, על with כל every נעלם secret thing, אם whether טוב good, ואם or whether רע׃ evil.}%
\end{biblechapter}%
\flushcolsend
\biblebook{Song of Solomon}
\begin{biblechapter}% Song of Solomon 1
\verseWithHeading{Title}{The Song of Songs,\lebnote{This construction conveys a superlative connotation, e.g., “The most exquisite song”} which is for\lebnote{Or “by Solomon” or “about/concerning Solomon”} Solomon.}%
\verseWithHeading{Maiden’s Soliloquy}{May\lebnote{In the maiden’s soliloquy, she thinks about her beloved in her thoughts (“May he kiss me!”), then poetically speaks to him as if he were in her presence (“for your love is better than wine”). To avoid confusion, the translation uses the second-person form throughout vv. 2–4} you kiss me\lebnote{“May he kiss me”} passionately with your lips,\lebnote{“with the kisses of his mouth”} for your love is better than wine.\lebnote{The shift from the third person “he … his” to the second person “you … your” in vv. 2–4 should not be interpreted as suggesting two different referents, that is, one male whom the maiden is addressing as “you,” and another to whom she refers as “he.” Rather, this shift is a poetic device (called “grammatical differentiation”) that is not uncommon in Hebrew poetry (e.g., Gen 49:4; Deut 32:15; Psa 23:2–5; Isa 1:29; 42:20; 54:1; Jer 22:24; Amos 4:1; Mic 7:19; Lam 3:1; Song 4:2; 6:6). This shift is characteristic of a soliloquy, a dramatic or literary form in which a character reveals her thoughts without addressing a listener who is actually present (e.g., 2 Sam 19:4). In this case, the maiden’s private thoughts about her beloved (v. 2a) shift to an imaginary address to her beloved (vv. 2b–4a)}}%
\verse{As fragrance, your perfumes\lebnote{“your oil lotions”} are delightful;\lebnote{“good”} your name is poured out perfume;\lebnote{“oil lotion”} therefore young women love you.}%
\verse{Draw me after you, let us run! May the king bring me into his chambers!\lebnote{Or “The king has brought me into his chambers”} Let us be joyful and let us rejoice in you; let us extol your love more than wine. Rightly do they love you!}%
\verseWithHeading{Maiden’s Self-Description}{I am black but beautiful,\lebnote{Or “black and beautiful”} O maidens of Jerusalem,\lebnote{“O daughters of Jerusalem”} like the tents of Kedar, like the curtains of Solomon.}%
\verse{Do not gaze at me because I am black,\lebnote{This is figurative for the maiden’s physical appearance; her skin was darkly tanned} because the sun has stared at me. The sons of my mother were angry with me; they made me keeper of the vineyards, but my own “vineyard”\lebnote{“my vineyard that for me”} I did not keep.}%
\verseWithHeading{Dialogue between Shepherdess and Shepherd}{Tell me, you whom my heart\lebnote{“soul”} loves, where do you pasture your flock, where do your sheep lie down at the noon? For why should I be like\lebnote{“For to what will I be like”} one who is veiled\lebnote{The reading of the MT (“like one who is veiled”) is supported by the LXX. However, several ancient versions (Syriac Peshitta, Latin Vulgate, Symmachus) reflect an alternate Hebrew textual tradition in which two letters are transposed, resulting in the reading “like one who wanders about.” This makes good sense contextually, since the maiden does not know where her beloved would be at noon} beside the flocks of your companions?}%
\verse{If you do not know, O fairest among women, follow the tracks\lebnote{“in the tracks”} of the flock, and pasture your little lambs\lebnote{Or “your kids”} beside the tents of the shepherds.}%
\verseWithHeading{Man’s Poetic Praise of His Beloved}{To a mare\lebnote{Or “my mare”} among the chariots\lebnote{Or “chariot horses”} of Pharaoh, I compare you, my beloved.}%
\verse{Your cheeks are beautiful with ornaments, your neck with strings of jewels.}%
\verse{We will make ornaments of gold for you with studs\lebnote{Or “droplets”} of silver.}%
\verseWithHeading{Maiden’s Poetic Praise of Her Beloved}{While the king was on his couch, my nard gave its fragrance.}%
\verse{My beloved is to me a pouch\lebnote{“the bag”} of myrrh, he spends the night\lebnote{Or “he lays”} between my breasts.}%
\verse{My beloved is to me a cluster of blossoms of henna in the vineyards of En Gedi.}%
\verseWithHeading{Mutual Admiration}{Look! You are beautiful, my beloved. Look! You are beautiful; your eyes are doves.}%
\verse{Look! You are beautiful, my beloved, truly pleasant. Truly our couch is verdant;\lebnote{“green”}}%
\verse{the beams of our house are cedar; our rafter is cypress.}%
\end{biblechapter}%
\begin{biblechapter}% Song of Solomon 2
\verseWithHeading{Dialogue between Maiden and Her Beloved}{I am a rose\lebnote{More likely “meadow saffron” or “crocus.” Hebrew scholars and botanists suggest the term refers to \textit{Ashodelos} (lily family), \textit{Narcissus tazetta} (narcissus or daffodil), or \textit{Colchicum autumnale} (meadow saffron or crocus) (e.g., Isa 35:1). The location of this flower in Sharon suggests a common wild flower rather than a rose. The maiden compares herself to a simple, common flower of the field} of Sharon, a lily of the valleys.}%
\verse{Like a lily among the thorns,\lebnote{Or “brambles”} so is my love among the maidens.}%
\verse{As an apple tree among the trees of the forest, so is my beloved among the young men. In his shade I sat down with delight,\lebnote{“I sat down and I delighted”} and his fruit was sweet to my palate.}%
\verseWithHeading{Banquet Hall of Love}{He brought me to the house of the wine, and his intention was love toward me.}%
\verse{Sustain me with the raisins, refresh me with the apples, for I am lovesick.\lebnote{“for I myself \textit{am} sick with love”}}%
\verseWithHeading{Double Refrain: Embrace and Adjuration}{His left hand is under my head, and his right hand embraces\lebnote{Or “would embrace me”} me.}%
\verse{I adjure you, O maidens of Jerusalem,\lebnote{“O daughters of Jerusalem”} by the gazelles or by the does of the field, do not arouse or awaken love until it pleases!\lebnote{Or “Do not stir up or awaken the love until it is willing,” or “Do not disturb or interrupt \textit{our} lovemaking until it is satisfied”}}%
\verseWithHeading{Rendezvous in the Countryside}{The voice of my beloved! Look! Here he\lebnote{“this one”} comes leaping upon the mountains, bounding over the hills!}%
\verse{My beloved is like a gazelle or a young stag.\lnBUV{} Look! He is\lebnote{“This \textit{is he}”} standing behind our wall, gazing through\lnBUW{} the window, looking through\lnBUW{} the lattice.}%
\verse{My beloved answered and said to me, “Arise,\lebnote{“Arise, you”} my beloved! Come, my beauty!\lebnote{“And come, you”}}%
\verse{For look! The winter is over; the rainy season\lebnote{“the rain”} has turned and gone away.\lebnote{“is over; it is gone”}}%
\verse{The blossoms appear\lebnote{“is seen”} in the land;\lebnote{“on the earth”} the time of singing\lebnote{Most likely, a subtle word play occurs here since there are two different words in Hebrew spelled the same way: “pruning” and “singing.” The former plays upon the first line and the latter upon the third line} has arrived;\lebnote{“the time of the song arrived”} the voice of the turtledove is heard in our land.}%
\verse{The fig tree puts forth her figs, and the vines are in blossom; they give fragrance. Arise,\lebnote{“Arise, to you!”} my beloved! Come, my beauty!”\lebnote{“My beauty, come, you”}}%
\verse{My dove, in the clefts of the rock, in the secluded place\lebnote{“in the secret place”}\lebnote{Or “in the covering”} in the mountain,\lebnote{“foothold in the rock”}\lebnote{Or “cliff”} Let me see your face, let me hear your voice; for your voice is sweet and your face is lovely.}%
\verse{Catch for us the foxes, the little foxes destroying vineyards, for\lebnote{Or “while”} our vineyards are in blossom!}%
\verseWithHeading{Poetic Refrain(s)}{My beloved belongs to me and I belong to him;\lebnote{“My beloved for me and I for him”} he pastures his flock among the lilies.}%
\verse{Until the day breathes and the shadows flee, turn, my beloved! Be like\lebnote{“Be like for you”} a gazelle\lebnote{Or “a buck gazelle”} or young stag\lnBUV{} on the cleft mountains.\lebnote{Or “the mountains of Bether”}}%
\end{biblechapter}%
\begin{biblechapter}% Song of Solomon 3
\verseWithHeading{Maiden’s Dream (?): Seeking and Finding}{On my bed in the night, I sought\lebnote{Or “I seek”} him whom my heart\lnBUX{} loves. I sought him, but I did not find him.}%
\verse{Now I will arise, and I will go about in the city, in the streets and in the squares; I will seek him whom my heart\lnBUX{} loves. I sought him, but I did not find him.}%
\verse{The sentinels who go about in the city found me. “Have you seen the one whom my heart\lnBUX{} loves?”}%
\verse{Scarcely had I passed\lebnote{“As little that I passed”} by them when I found him whom my heart\lnBUX{} loves. I held him and I would not let him go until I brought him to the house of my mother, into the bedroom chamber of she who conceived me.}%
\verseWithHeading{Adjuration Refrain}{I adjure you, O maidens of Jerusalem,\lebnote{“O daughters of Jerusalem”} by the gazelles or by the does of the field, do not arouse or awaken love until it pleases!\lebnote{Or “Do not stir up or awaken the love until it is willing,” or “Do not disturb or interrupt \textit{our} love-making until it is satisfied”}}%
\verseWithHeading{Royal Wedding Procession}{What is this coming up from the desert like a column of smoke, perfumed with myrrh and frankincense from all the fragrant powders of the merchant?}%
\verse{Look! It is Solomon’s portable couch!\lebnote{“couch” or “portable sedan chair”} Sixty mighty men surround it,\lebnote{“her”} the mighty men of Israel.}%
\verse{All of them wield swords;\lebnote{“holders of sword”} they are trained in warfare,\lebnote{“learnt of war”} each with his sword at his thigh to guard against terror\lebnote{“because of the fear”} in the night.}%
\verse{King Solomon\lebnote{“The king, Solomon”} made for himself a sedan chair from the wood of Lebanon.}%
\verse{He made its column of silver, its back\lebnote{Or “its support,” “its base,” “its headrest,” “its litter,” “its cover”} of gold, its seat of purple; its interior is inlaid with leather\lebnote{Or “love.” The Hebrew term here translated “leather” is spelled the same as the term for “love.” Most likely this is an example of a word play that puns on the intentional ambiguity: “Its interior was inlaid with leather//love by the maidens of Jerusalem”} by the maidens of Jerusalem.\lebnote{“by the daughters of Jerusalem”}}%
\verse{Come out and look, O maidens of Zion,\lebnote{“O daughters of Zion”} at King Solomon,\lebnote{“the king, Solomon”} at the crown with which his mother crowned him on the day of his wedding, on the day of the joy of his heart!}%
\end{biblechapter}%
\begin{biblechapter}% Song of Solomon 4
\verseWithHeading{Groom’s Praise of His Bride}{Oh my!\lnBUY{} You are beautiful, my beloved! Oh my!\lnBUY{} You are beautiful! Your eyes are doves from behind your veil. Your hair is like a flock of goats that move down from the mountains of Gilead.}%
\verse{Your teeth are like a flock of shorn ewes that came up from the washing, all of them bearing twins, and there is none bereaved among them.}%
\verse{Your lips are like a thread of crimson, and your mouth is lovely. Your temple is like pomegranate from behind your veil.}%
\verse{Your neck is like the tower of David, built in courses; a thousand ornaments\lebnote{“shields”} are hung on it, all the shields of the warriors.}%
\verse{Your two breasts are like two fawns, twins of a gazelle that feed among the lilies.}%
\verse{Until the day breathes and the shadows flee, I will go to the mountain of the myrrh, to the hill of the frankincense.}%
\verse{You are completely beautiful, my beloved! You are flawless!\lebnote{“There is no flaw in you!”}}%
\verseWithHeading{The Mountains and Fragrance of Lebanon}{Come\lebnote{Or “You must come”} with me from Lebanon, my bride! Come with me\lebnote{Or “With me”} from Lebanon! Look from the top of Amana, from the top of Senir and Hermon, from the dwelling places of the lions, from the mountains of leopard.}%
\verse{You have stolen (my) heart, my sister bride! You have stolen my heart with one glance from your eyes, with one ornament from your necklaces.}%
\verse{How beautiful is your love, my sister bride! How better is your love than wine, and the fragrance of your oils than any spice!}%
\verse{Your lips drip nectar, my bride; honey and milk are under your lips; the scent of your garments is like the scent of Lebanon.}%
\verseWithHeading{The Locked Garden of Delights Is Unlocked}{A garden locked is my sister bride, a spring enclosed,\lebnote{Or “a source locked”} a fountain sealed.}%
\verse{Your shoots\lebnote{Or “your channel”} are an orchard of pomegranates with choice fruit,\lebnote{“fruit of choice things”} henna with nard;}%
\verse{nard and saffron, calamus and cinnamon spice with all trees of frankincense, myrrh and aloes with all chief spices.}%
\verse{A garden fountain, a well of living water, flowing (streams) from Lebanon.}%
\verse{Awake, O north wind! Come, O south wind! Blow upon my garden! Let its fragrances\lebnote{Or “perfumes”} waft forth!\lebnote{Or “His perfumes can waft down”} Let my beloved come to his garden, let him eat his choice fruit!}%
\end{biblechapter}%
\begin{biblechapter}% Song of Solomon 5
\verse{I have come to my garden, my sister bride, I have gathered my myrrh with my spice, I have eaten my honeycomb with my honey, I have drunk my wine with my milk! Eat, O friends! Drink and become drunk with love!\lebnote{Or “Drink and become drunk, O lovers!”}}%
\verseWithHeading{Maiden’s Dream: Seeking and Not Finding}{I was asleep but\lebnote{Or “and”} my heart was awake. A sound! My beloved knocking!\lebnote{Or “The sound of my beloved knocking!”} “Open to me, my sister, my beloved, my dove, my perfect one! For my head is full of dew, my hair drenched from the moist night air.”\lebnote{“my locks with drops of night”}}%
\verse{I have taken off my tunic, must I put it on?\lebnote{“How will I put it on?”} I have bathed my feet, must I soil them?\lebnote{“How will I soil them?”}}%
\verse{My beloved thrust his hand into the opening, and my inmost yearned for him.}%
\verse{I myself arose to open to my beloved; my hands dripped with myrrh, my fingers with liquid myrrh upon the handles of the bolt.}%
\verse{I opened myself to my beloved, but my beloved had turned and gone;\lebnote{Or “my beloved had left; he was gone”} my heart sank\lebnote{Or “my soul left”} when he turned away.\lebnote{Or “when he was speaking.” Translations equivocate on how to translate this verb, since there are two terms in Hebrew spelled identically: “to speak” and “to turn aside” (HALOT 1:210). The context suggests the latter} I sought him, but I did not find him; I called him, but he did not answer me.}%
\verse{The sentinels making rounds in the city found me; they beat me, they wounded me; they took my cloak\lebnote{Or “mantle”} away from me — those sentinels on the walls!\lebnote{“the sentinels of the walls”}}%
\verseWithHeading{Adjuration Refrain}{I adjure you, O maidens of Jerusalem,\lnBUZ{} if you find my beloved, what will you tell him? Tell him that I am lovesick!\lebnote{“sick \textit{with} love”}}%
\verseWithHeading{Maiden’s Praise of Her Beloved}{How is your beloved better than another lover,\lnBVA{} O most beautiful among women? How is your beloved better than another lover,\lnBVA{} that you adjure us thus?}%
\verse{My beloved is radiant and ruddy,\lebnote{“red”} distinguished among\lebnote{“more than”} ten thousand.}%
\verse{His head is gold, refined gold; his locks are wavy, black as a raven.}%
\verse{His eyes are like doves beside springs\lebnote{Or “streams”} of water, bathed in milk, set like mounted jewels.\lebnote{“dwelling in a setting”}\lebnote{Or “seated at a \textit{suitable} mounting”}}%
\verse{His cheeks are like beds of spice, a tower of fragrances; his lips are lilies dripping liquid myrrh.}%
\verse{His arms are rods\lebnote{“cylinders”}\lebnote{Or “rings”} of gold engraved with\lebnote{“filled with”} jewels; his belly\lebnote{Or “body”} is polished ivory covered with sapphires.\lebnote{Or “works of ivory set with sapphire”}}%
\verse{His legs are columns of alabaster,\lebnote{Or “marble”} set on bases of gold; his appearance is like Lebanon, choice as its cedars.\lebnote{“the cedars”}}%
\verse{His mouth\lebnote{Or “his palate”} is sweet, and he is altogether desirable. This is my beloved; this is my friend, O young women of Jerusalem.\lnBUZ{}}%
\end{biblechapter}%
\begin{biblechapter}% Song of Solomon 6
\verse{Where has your beloved gone, O most beautiful among women? Where has your beloved turned that we may seek him with you?}%
\verse{My beloved has gone down to his garden, to the garden bed of the spice, to pasture his flock and to gather lilies in the garden.}%
\verseWithHeading{Mutual Possession Refrain}{I belong to my beloved and he belongs to me;\lebnote{“I for my beloved and he for me”} he pastures his flock among the lilies.}%
\verseWithHeading{Solomon’s Praise of His Beloved}{You are beautiful, my beloved, as Tirzah, lovely as Jerusalem, overwhelming as an army with banners.\lnBVB{}}%
\verse{Turn away your eyes from before me, for they overwhelm me. Your hair is like a flock of the goats that moves down from Gilead.}%
\verse{Your teeth are like a flock of the ewes that have come up from the washing, all of them bearing twins, and there is none bereaved among them.}%
\verse{Your cheeks behind\lebnote{“from behind”} your veil are like halves of a pomegranate.}%
\verseWithHeading{The Maiden’s Beauty Is without Peer}{Sixty queens there are, eighty concubines, and maidens beyond number.}%
\verse{My dove, she is the one;\lnBVC{}\lebnote{The term “one” functions here as an adjective of quality: “unique, singular, the only one”} my perfect, she is the only one;\lnBVC{}\lebnote{Or “the only daughter of her mother.” Although the latter option is permissible, the term is used elsewhere of the heir as the favored child (e.g., Gen 22:2; Prov 4:3). This nuance is supported by the parallel term “favorite”} she is the favorite of\lebnote{Or “she \textit{is} the pure one.” Since there are two Hebrew terms spelled the same way, some relate this to the adjective that means “pure.” Others relate it to the verb that means “to choose, select.” The parallelism favors the latter}\lebnote{“the favorite for”} her mother who bore her. Maidens see her and consider her fortunate;\lebnote{Or “call her happy” or “call her blessed” or “bless her”} queens and concubines praise her:}%
\verse{“Who is this that looks down like the dawn, beautiful as the moon, bright as the sun,\lebnote{“pure as the glow”}\lebnote{Or “bright as the heat of the sun.” The Hebrew term “glow” poetically refers to the bright rays of the sun (Psa 19:7; Isa 24:23; 30:26)} overwhelming as an army with banners?”\lnBVB{}}%
\verseWithHeading{The Journey to the Valley}{I went down to the orchard of the walnut trees to look at the blossoms of the valley, to see whether the vines have sprouted, whether the pomegranates have blossomed.}%
\verse{I did not know my heart\lebnote{“soul”} set me in a chariot of my princely people.\lebnote{Or “Before I was aware, my soul made me like the chariots of Amminadib” (KJV, ASV) or “Before I knew it, my desire set me mid the chariots of Ammi-nadib” (JPS) or “Before I was aware, my soul set me over the chariots of my noble people” (NASB) or “Before I realized it, my desire set me among the royal chariots of my people” (NIV) or “… among the chariots of Amminadab” (NIV margin) or “… among the chariots of the people of the prince” (NIV margin)}}%
\verse{\lebnote{Song of Songs 6:13–7:13 in the English Bible is 7:1–14 in the Hebrew Bible} Turn, turn,\lnBVD{} O Shulammite!\lebnote{Or “O perfect one,” “O peaceful one,” “O bride.” Many interpreters take this moniker as suggesting the maiden was from the village of Shulem (alternately called Shunem)} Turn, turn\lnBVD{} so that we may look upon you! Why do you look upon the Shulammite as at a dance of the two armies?}%
\end{biblechapter}%
\begin{biblechapter}% Song of Solomon 7
\verseWithHeading{Solomon’s Praise of His Dancing Maiden}{How beautiful are your feet in sandals, O royal princess!\lebnote{Or “O daughter of leader”} The curves of your thighs\lebnote{“thigh”} are like jewels,\lebnote{“ornaments”} the work of the hands of a craftsman.}%
\verse{Your navel is a round wine-mixing bowl\lebnote{“a bowl of the roundness”} that does not lack mixed\lebnote{Or “blended”} wine! Your belly is a heap of wheat encircled with lilies.}%
\verse{Your two breasts are like two fawns, twins of a gazelle.}%
\verse{Your neck is like a tower of ivory; your eyes are pools in Heshbon at the gate of Beth Rabbim. Your nose is like the tower of Lebanon looking out over Damascus.\lebnote{“looking out over the face of Damascus”}}%
\verse{Your head crowns you like Carmel;\lebnote{“Your head \textit{is} on you like the Carmel”}\lebnote{Because of its height and fertility, Mount Carmel is often associated with royalty} the flowing locks of your head are like purple tapestry;\lebnote{“the purple”} a king is held captive in the tresses!}%
\verse{How beautiful you are and how pleasant, O loved one in the delights!}%
\verse{Your stature\lebnote{“this your height”} is like the palm tree, and your breasts are like clusters.}%
\verse{I say, “I will climb up the palm tree; I will lay hold of its fruit clusters.” Let your breasts be pleasing like clusters of the vine and the scent of your breath like the apples.}%
\verse{Your palate is like the best wine that goes down for my beloved, smoothly gliding over my lips and teeth.\lebnote{Or “over lips of sleepers.” One Hebrew textual tradition preserves the reading “lips of those who sleep” (MT). Another Hebrew tradition reads “my lips and my teeth,” as reflected by the ancient versions (LXX, Latin Vulgate, Aramaic Targum, Syriac Peshitta). The latter is adopted here since it makes the most sense poetically}}%
\verseWithHeading{Mutual Possession Refrain}{I belong to my beloved,\lebnote{“I \textit{am} for my beloved”} and he desires me!\lebnote{“and his desire \textit{is} for me.” Or “and his desire belongs to me”}}%
\verseWithHeading{Rendezvous in the Countryside}{Come, my beloved, let us go out to the countryside;\lebnote{“go forth into the field”} let us spend the night\lebnote{Or “lodge”} in the villages.}%
\verse{Let us rise early to go\lebnote{Or “let us go”} to the vineyards; let us see whether the vine has budded,\lebnote{Or “sprouted”} whether the grape blossom has opened, and whether the pomegranates are in bloom;\lebnote{“have bloomed”} there I will give my love to you.}%
\verse{The mandrakes give off their fragrance, and over our doorway is every kind of delicious fruit;\lebnote{Or “over our doorways all choice \textit{things}”} both fresh and dried fruit I have stored up\lebnote{“new also old I have laid up”} for you, O my beloved.}%
\end{biblechapter}%
\begin{biblechapter}% Song of Solomon 8
\verseWithHeading{Maiden’s Fanciful Wish}{How I wish that you were my little brother,\lebnote{“O that he would give you like a brother to me”}\lebnote{The Hebrew construction (which is somewhat misleading if rendered in a woodenly literal sense) is an idiom expressing one’s fanciful wish} who nursed upon my mother’s breasts!\lebnote{“at the breast of my mother”} If I met you outside,\lebnote{“I will find you in the street”} I would kiss you, and no one would despise me!\lebnote{“also they would not despise me”}}%
\verse{I would surely bring you\lebnote{“I would lead you and I would bring you”}\lebnote{The combination of the two verbs creates a hendiadys which may be rendered more cogently as “I would surely bring you …”} to the house of my mother, who would surely teach me;\lebnote{“she will teach me”} I would give you spiced wine to drink,\lebnote{“I would give you to drink from the wine of the spice”} the sweet wine\lebnote{Or “juice”} of my pomegranates.\lebnote{The traditional Hebrew reads the singular “my pomegranate.” However, the plural reading “my pomegranates” is attested in numerous medieval Hebrew manuscripts and is reflected in the ancient versions (Greek Septuagint, Aramaic Targum, Syriac Peshitta). The latter makes the most sense in this context as a euphemistic description of the maiden’s delights}}%
\verseWithHeading{Double Refrain: Embrace and Adjuration}{His left hand is under my head, and his right hand embraces\lebnote{Or “embraced”} me.}%
\verse{I adjure you, O maidens of Jerusalem,\lebnote{“O daughters of Jerusalem”} do not\lebnote{Or “Why must you … before it pleases?”} arouse or awaken love until it pleases!\lebnote{Or “Do not stir up or awaken the love until it is willing,” or “Do not disturb or interrupt \textit{our} lovemaking until it is satisfied”}}%
\verseWithHeading{Up from the Wilderness and under the Apple Tree}{Who is this coming up from the wilderness, leaning upon her beloved? Under the apple tree I awakened you; there your mother conceived you;\lebnote{“was in labor with you”} there she who was in labor gave birth to you.}%
\verseWithHeading{The Nature of Genuine Romantic Love}{Set me as a seal upon your heart, as a seal upon your arm; for love is strong as death; passion is fierce as Sheol; its flashes are flashes of fire; it is a blazing flame.}%
\verse{Many waters cannot quench love; rivers cannot sweep it away.\lebnote{Or “and rivers cannot engulf it”} If a man were to give all the wealth of his house for love,\lebnote{“in the love”} he would be utterly scorned.\lebnote{“they will utterly scorn him”}}%
\verseWithHeading{Maiden’s Virtuous Chastity and Voluptuous Beauty}{We have a little sister,\lebnote{“a little sister for us”} and she does not yet have any breasts.\lebnote{“and there is no breast for her”} What should we do for our sister on the day when she is betrothed?\lebnote{“on the day when it is spoken of her”}\lebnote{Or “on the day when she is spoken for”}}%
\verse{If she is a wall, we will adorn her with a turret of silver;\lebnote{“we will build upon her a camp of silver”}\lebnote{The term translated “turret” refers to the decorative parapet adorning the top of a building. This image is likely figurative for a silver tiara set upon the head} but if she is a door, we will barricade her with boards of cedar.\lebnote{Or “we will enclose her”}}%
\verse{I was a wall, and my breasts were like the towers, so my betrothed viewed me with great delight.\lebnote{“then I was in his eyes as \textit{one who} finds peace”}}%
\verseWithHeading{Solomon’s Vineyard and the Maiden’s Gift}{Solomon had a vineyard\lebnote{“A vineyard was for Solomon”} at Baal-hamon; he entrusted his vineyard to the keepers;\lebnote{“he gave the vineyard to the keepers”} people paid a thousand silver pieces for its fruit.\lebnote{“each one brought a thousand silver \textit{pieces} for his fruit”}}%
\verse{My own “vineyard” belongs to me;\lebnote{“My vineyard that for me before my face”} the “thousand” are for you, O Solomon, and “two hundred” for those who tend its fruit.\lebnote{“and two hundred for \textit{the} keepers \textit{of} his fruit”}}%
\verseWithHeading{Closing Words of Mutual Love}{O you who dwell in the garden, my companions are listening to your voice. Let me hear it!}%
\verse{Flee, my beloved! Be like a gazelle\lebnote{“and be like for you to a gazelle”} or a young stag\lebnote{“to the fawn of the stag”} upon the perfumed mountains!\lebnote{“the mountains of spices”}}%
\end{biblechapter}%
\flushcolsend
\biblebook{Isaiah}
\begin{biblechapter}% Isaiah 1
\verseWithHeading{Superscription}{The vision of Isaiah son of Amoz, which he saw concerning Judah and Jerusalem in the days of Uzziah, Jotham, Ahaz, and Hezekiah, kings of Judah.}%
\verseWithHeading{Rebellious Judah}{Hear, heavens, and listen, earth, for Adonai has spoken: “I reared children and I brought them up, but they rebelled against me.}%
\verse{An ox knows its owner and a donkey the manger of its master. Israel does not know; my people do not understand.}%
\verse{Ah, sinful nation, a people heavy with iniquity, offspring of evildoers, children who deal corruptly. They have forsaken Adonai; they have despised the holy one of Israel. They are estranged and gone backward.}%
\verse{Why do you want to be beaten again? You continue in rebellion. The whole of the head is sick, and the whole of the heart is faint.}%
\verse{From the sole of the foot and up to the head there is no health in it; bruise and sore and bleeding wound have not been cleansed, and they have not been bound up and not softened with the oil.}%
\verse{Your country is desolate, your cities are burned with fire; As for your land, aliens are devouring it in your presence, and it is desolate, like devastation by foreigners.}%
\verse{And the daughter of Zion is left like a booth in a vineyard, like a shelter in a cucumber field, like a city that is besieged.\lebnote{Or “preserved”}}%
\verse{If Adonai of hosts had not left us survivors,\lebnote{Hebrew “survivor”} we would have been as few as Sodom, we would have become like Gomorrah.}%
\verse{Hear the word of Adonai, rulers of Sodom! Listen to the teaching of our God, people of Gomorrah!}%
\verse{What is the abundance of your sacrifices to me? says Adonai. I have had enough of burnt offerings of rams and the fat of fattened animals and I do not delight in the blood of bulls and ram-lambs and goats.}%
\verse{When you come to appear before me, who asked for this from your hand: you trampling my courts?}%
\verse{You must not continue\lebnote{“increase” or “add to”} to bring offerings\lebnote{Hebrew “offering”} of futility, incense — it is an abomination to me; new moon and Sabbath, the calling of a convocation — I cannot endure iniquity with solemn assembly.}%
\verse{Your new moons and your appointed festivals my soul hates; they have become to me like a burden, I am not able to bear them.}%
\verse{And when you stretch out your hands, I will hide my eyes from you; even though you make many prayers,\lebnote{Hebrew “prayer”} I will not be listening. Your hands are full of blood.}%
\verse{Wash! Make yourselves clean! Remove the evil of your doings from before my eyes! Cease to do evil!}%
\verse{Learn to do good! Seek justice! Rescue the oppressed! Defend the orphan! Plead for the widow!}%
\verse{“Come now, and let us argue,” says Adonai. “Even though your sins are like scarlet, they will be white like snow; even though they are red like crimson, they shall become like wool.}%
\verse{If you are willing and you are obedient, you shall eat the good of the land.}%
\verse{But if you refuse and you rebel, you shall be devoured by the sword. For the mouth of Adonai has spoken.”}%
\verseWithHeading{Purifying Jerusalem}{How has a faithful city become like a whore? Full of justice, righteousness lodged in her, but now murderers.}%
\verse{Your silver has become as dross; Your wine is diluted with waters.}%
\verse{Your princes are rebels and companions of thieves. Every one loves a bribe and runs after gifts. They do not defend the orphan and the legal dispute of the widow does not come before them.}%
\verse{Therefore, the declaration of the Lord Adonai of hosts, the Mighty One of Israel: “Ah, I will be relieved of my enemies, and I will avenge myself on my foes.}%
\verse{And I will turn my hand against you; I will purify your dross like lye, and I will remove all of your tin.}%
\verse{And I will restore your judges, as at the first, and your counselors, as at the beginning. After this you will be called\lebnote{“it will be called for you”} the city of righteousness, faithful city.}%
\verse{Zion will be redeemed by justice, and those of her who repent, by righteousness.}%
\verse{But the destruction of rebels and sinners shall be together, and those who forsake Adonai will perish.}%
\verse{For you\lebnote{“they,” but a few manuscripts read “you,” which fits the context better} will be ashamed of the oaks in which you delighted, and you will be disgraced because of the gardens that you have chosen.}%
\verse{For you shall be like an oak withering its leaves, and like a garden where there is no water for her.}%
\verse{And the strong man shall become like tinder, and his work like a spark. And both of them shall burn together, and there is not one to quench them.”}%
\end{biblechapter}%
\begin{biblechapter}% Isaiah 2
\verseWithHeading{The Mountain of Adonai}{The word that Isaiah son of Amoz saw concerning Judah and Jerusalem:}%
\verse{And it shall happen in the future of the days the mountain of the house\lnBVE{} of Adonai shall be established; it will be among the highest\lebnote{“head”} of the mountains, and it shall be raised from the hills. All of the nations shall travel to him;}%
\verse{many peoples shall come. And they shall say, “Come, let us go up to the mountain of Adonai, to the house\lnBVE{} of the God of Jacob, and may he teach us part of his ways, and let us walk in his paths.” For instruction shall go out from Zion, and the word of Adonai from Jerusalem.}%
\verse{He shall judge between the nations and he shall arbitrate for many peoples. They shall beat their swords into ploughshares and their spears into pruning hooks. A nation shall not lift up a sword against a nation, and they shall not learn war again.}%
\verseWithHeading{The Day of Adonai}{House of Jacob, come and let us walk in the light of Adonai.}%
\verse{For you have forsaken your people, house of Jacob, because they are full\lebnote{Possibly “of diviners” was part of the original text here} from the east, and of soothsayers like the Philistines, and they make alliances\lebnote{“they clap \textit{hands}”} with the offspring of foreigners.}%
\verse{And its land is filled with silver and gold, and there is no end to its treasures; and its land is filled with horses, and there is no end to its chariots.}%
\verse{Its land is filled with idols; they bow down to the work of their\lnBVF{} hands, to what they made with their\lnBVF{} fingers.}%
\verse{So humanity is humbled; everyone is humbled, and you must not forgive them.}%
\verse{Enter into the rock and hide yourself in the dust from the presence of the terror of Adonai and from the glory of his majesty.}%
\verse{The haughty eyes\lebnote{“eyes of the haughtiness”} of humanity will\lebnote{The Hebrew is singular} be brought low, and the pride of everyone will be humbled, and Adonai alone will be exalted on that day.}%
\verse{For there is a day for Adonai of hosts against all of the proud and the lofty and against all that is lifted up and humble,\lebnote{Or “it will be humbled”}}%
\verse{and against all the lofty and lifted up cedars of Lebanon, and against all the large trees of Bashan,}%
\verse{and against all the high mountains, and against all the lofty hills,}%
\verse{and against every kind of high tower, and against every kind of fortified wall,}%
\verse{and against all the ships of Tarshish, and against all the ships of desire.}%
\verse{And the haughtiness of the people shall be humbled, and the pride of everyone shall be brought low, and Adonai alone will be exalted on that day.}%
\verse{And the idols shall pass away entirely,}%
\verse{and they will enter into the caves of the rocks and into the holes of the ground from the presence of the terror of Adonai and from the glory of his majesty when he rises\lnBVG{} to terrify the earth.}%
\verse{On that day humanity will throw away its idols of silver and its idols of gold, which they made for it to worship, to the rodents\lebnote{Or “moles”} and to the bats —}%
\verse{to enter into the crevices of the rocks and into the clefts of the crags from the presence of the terror of Adonai and from the glory of his majesty, when he rises\lnBVG{} to terrify the earth.}%
\verse{Turn away from humanity, who has breath in its nostrils,\lebnote{Hebrew “nostril”} for by\lebnote{Or “in”} what is it esteemed?}%
\end{biblechapter}%
\begin{biblechapter}% Isaiah 3
\verseWithHeading{Leaders of Judah and Jerusalem}{For look, the Lord Adonai of hosts is removing every source of support\lebnote{“supplies and supplies”} from Jerusalem and from Judah: all of the supplies of bread and all of the supplies of water,}%
\verse{mighty warrior and man of war, judge and prophet, and diviner and elder,}%
\verse{captain of fifty and the honorable men of rank, and counselor and skillful magicians and skillful enchanter.}%
\verse{And I will make boys their princes, and children shall rule over them.}%
\verse{And the people will be oppressed by each other\lebnote{“man by man”} and a man by his neighbor. The boy will act arrogantly toward the elder, and the dishonorable toward the honorable.}%
\verse{Indeed, a man will seize his brother in the house of his father: “You have a cloak;\lebnote{“a cloak for you”} you shall be a leader for us, and this heap of ruins shall be under your hand!”}%
\verse{He will lift up his voice on that day, saying, “I will not be a healer; in my house there is no bread and there is no cloak. You shall not make me the leader of the people!”}%
\verse{For Jerusalem has stumbled, and Judah has fallen because their speech and their deeds are against Adonai, defying the eyes of his glory.}%
\verse{The look on their faces testifies against them and they declare their sin like Sodom; they do not hide it. Woe to their soul! For they have dealt out evil to themselves.}%
\verse{Tell the innocent\lnBVH{} that it is good for they shall eat the fruit of their deeds.}%
\verse{Woe to the wicked!\lnBVH{} It is bad! For what is done by his hands will be done to him.}%
\verse{My people — children are their oppressors, and women rule over them. My people, your leaders are misleading you, and they confuse the course of your paths.}%
\verse{Adonai takes his stand to conduct a legal case and takes his stand to judge the peoples.}%
\verse{Adonai enters into judgment with the elders of his people and its princes. “And you! You have devoured the vineyard; the spoil of the poor is in your houses!}%
\verse{Why\lebnote{“What to you”} do you crush my people and grind the face of the poor?” declares\lebnote{“declaration of”} the Lord Adonai of hosts.}%
\verseWithHeading{The Pride of Jerusalem’s Women}{And Adonai said: “Because\lebnote{There are two Hebrew words for “because” here} the daughters of Zion are haughty, and they walk with outstretched neck, and they give flirting glances with their eyes, mincing along as they go,\lebnote{“they go walking and mincing along”} and with their feet they rattle their bangles,\lebnote{“they tinkle with their feet”}}%
\verse{the Lord will make the heads\lebnote{Hebrew “head”} of the daughters of Zion scabby, and Adonai will lay their foreheads bare.”}%
\verse{In that day the Lord will take away the finery of the anklets and the headbands and the crescent necklaces,}%
\verse{the pendants and the bracelets and the veils,}%
\verse{the headdresses and the armlets and the sashes, and the perfume boxes\lebnote{“houses of \textit{the} breath”} and the amulets,}%
\verse{the signet rings and the nose rings,}%
\verse{the festal robes and the mantles, and the cloaks and the handbags,}%
\verse{and the mirrors and the linen garments, and the turbans and the wraps.}%
\verse{And this shall happen: There will be a stench instead of perfume, and a rope instead of a sash, and baldness instead of a well-set hairdo, and a clothing wrap of sackcloth instead of a rich robe, branding instead of beauty.}%
\verse{Your men shall fall by the sword, and your warriors in battle.}%
\verse{And her\lebnote{That is, Jerusalem’s} gates shall lament and mourn, and she shall be banished; she shall sit upon the ground.}%
\end{biblechapter}%
\begin{biblechapter}% Isaiah 4
\verse{And seven women shall grasp at one man on that day, saying, “We will eat our own bread, and we will wear our own clothing; only let us be called by your name!\lebnote{“let your name be called upon us”} Take away our disgrace!”}%
\verseWithHeading{The Glory of the Branch of Adonai}{On that day the branch of Adonai shall become beautiful and glorious, and the fruit of the land shall become the pride and glory of the survivors\lebnote{Hebrew “survivor”} of Israel.}%
\verse{And this shall happen: He who is left in Zion and he who remains in Jerusalem will be called holy, everyone written for life in Jerusalem,}%
\verse{when the Lord has washed away the filth of the daughters of Zion and cleansed the blood of Jerusalem from her midst by a spirit of judgment and by a spirit of burning.}%
\verse{Then Adonai will create over all of the site\lebnote{Or “place”} of Mount Zion\lebnote{“the mountain of Zion”} and over her assembly a cloud by day and smoke and the brightness of flaming fire by night. Indeed, over all the glory there will be a canopy,}%
\verse{and it will be a shelter for shade from the heat by day, and a refuge and a hiding place from rainstorm and from rain.}%
\end{biblechapter}%
\begin{biblechapter}% Isaiah 5
\verseWithHeading{The Song of the Vineyard}{Let me sing for my beloved a song of my love concerning his vineyard: My beloved had a vineyard\lebnote{“A vineyard was for my beloved”} on a fertile hill.\lebnote{“a horn of a son of olive oil.” The Hebrew for horn, \textit{qeren}, sounds like the Hebrew for vineyard, \textit{kerem}}}%
\verse{And he dug it and cleared it of stones, and he planted it with choice vines,\lebnote{Hebrew “vine”} and he built a watchtower in the middle of it, and he even hewed out a wine vat in it, and he waited for it to yield grapes — but it yielded wild grapes.}%
\verse{And now, inhabitants of Jerusalem and men\lebnote{Hebrew “man”} of Judah, judge between me and my vineyard.}%
\verse{What more was there to do for my vineyard that I have not done in it? Why did I hope for it to yield grapes, and it yielded wild grapes?}%
\verse{And now let me tell you what I myself am about to do to my vineyard. I will remove its hedge, and it shall become a devastation. I will break down its wall, and it shall become a trampling.}%
\verse{And I will make it a wasteland; it shall not be pruned and hoed, and it shall be overgrown with briers\lebnote{Hebrew “brier”} and thornbushes.\lebnote{Hebrew “thornbush”} And concerning the clouds, I will command them not to send\lebnote{“from sending”} rain down upon it.}%
\verse{For the vineyard of Adonai of hosts is the house of Israel, and the man\lebnote{Or “people”} of Judah is the plantation of his delight. And he waited for justice,\lebnote{The Hebrew word, \textit{mishpat,} sounds like \textit{mishpakh} in the next line} but look! Bloodshed!\lebnote{The Hebrew word, \textit{mishpakh}, sounds like \textit{mishpat} in the previous line} For righteousness,\lebnote{The Hebrew word, \textit{tsedaqah,} sounds like \textit{tsa`aqah} in the next line} but look! A cry of distress!\lebnote{The Hebrew word, \textit{tsa`aqah,} sounds like \textit{tsedaqah} in the previous line}}%
\verseWithHeading{Woes on the Wicked}{Ah! Those who join\lebnote{“touch”} house with house, they join field together with field until there is no place\lebnote{“an end of place”} and you are caused to dwell alone in the midst of the land.}%
\verse{Adonai of hosts said in my ears: Surely\lebnote{“If not”} many houses shall become a desolation, large and beautiful ones without inhabitant.}%
\verse{For ten acres of vineyard shall yield one bath,\lebnote{A bath is a liquid measure} and the seed of a homer will yield an ephah.\lebnote{An ephah is a dry measure equal to one-tenth of a homer}}%
\verse{Ah! Those who rise early in the morning, they pursue strong drink. Those who linger in the evening, wine inflames them.}%
\verse{And there will be lyre and harp, tambourine and flute, and wine at their feasts, but they do not look at the deeds\lebnote{Hebrew “deed”} of Adonai, and they do not see the work of his hands.}%
\verse{Therefore my people will go into exile without knowledge, and their\lnBVI{} nobles\lnBVJ{} will be men of hunger, and their\lnBVI{} multitude is parched with thirst.}%
\verse{Therefore Sheol has enlarged its throat, and it has opened wide its mouth without limit, and her\lebnote{That is, Jerusalem’s} nobles\lnBVJ{} will go down, and her multitude, her tumult and those who revel in her.}%
\verse{And humankind is bowed down, and man is brought low, and the eyes of the haughty are humiliated.}%
\verse{But Adonai of hosts is exalted by justice, and the holy God shows himself holy by righteousness.}%
\verse{And then the lambs will graze as in their pasture, and fatlings, kids\lebnote{Following the Septuagint, which reads the Hebrew \textit{grym} (resident aliens) as \textit{gdym} (young goats/sheep)} will eat among the sites of ruins.\lebnote{“and ruins, fatlings, resident aliens, will eat”}}%
\verse{Ah! Those who drag iniquity along with the cords of falsehood and sin as with rope of the cart,}%
\verse{those who say, “Let him make haste; let him hurry his work so that we may see it and let it draw near and let the plan of the holy one of Israel come so that we may know it!”}%
\verse{Ah! Those who call evil good and good evil, those who put darkness for light and light for darkness, those who put bitter for sweet and sweet for bitter!}%
\verse{Ah! Those who are wise in their own eyes and have understanding in their view!\lebnote{“before their faces”}}%
\verse{Ah! Heroes at drinking wine, and men of capability at mixing strong drink!}%
\verse{Those who acquit the guilty because of a bribe and remove the justice of the innocent from him.}%
\verse{Therefore as the tongue of fire devours the stubble, and dry grass sinks down in the flame, so their root will become like the stench, and their blossom will go up like the dust. For they have rejected the instruction of Adonai of hosts, and they have treated the word of the holy one of Israel with contempt.}%
\verse{Therefore Adonai’s wrath was kindled\lebnote{“the anger of Adonai became hot”} against his people, and he stretched out his hand against them\lnBVK{} and struck them,\lnBVK{} and the mountains quaked, and their corpses\lebnote{Hebrew “corpse”} were like refuse in the middle of the streets.\innerVerseHeading{Adonai’s Outstretched Hand} In all of this his anger has not turned back, and still his hand is stretched out.}%
\verse{And he will raise a signal for a nation\lebnote{The Hebrew is plural, but the following verses refer to the nation as singular} from afar, and he will whistle for it from the end of the earth. And look! It comes quickly, swiftly!}%
\verse{None is weary, and none among him stumbles; none slumbers and none sleeps. And no loincloth on his waist is opened, and no thong of his sandals is drawn away.}%
\verse{Whose arrows are sharp, and all of his bows are bent. The hoofs of his horses are reckoned like flint, and his wheels like the storm wind.}%
\verse{His roaring is like the lion, and he roars like young lions. And he growls and seizes his prey, and he carries it off, and not one can rescue it.}%
\verse{And he will roar over him on that day like the roaring of the sea, and if one looks to the land, look! Darkness! Distress! And the light grows dark with its\lebnote{Presumably the land’s} clouds.}%
\end{biblechapter}%
\begin{biblechapter}% Isaiah 6
\verseWithHeading{Isaiah’s Commission}{In the year of the death of Uzziah the king, I saw the Lord sitting on a high and raised throne, and the hem of his robe was filling the temple.}%
\verse{Seraphs were standing above him. Each had six wings:\lebnote{“six wings six wings for one”} with two he covered his face, and with two he covered his feet, and with two he flew.}%
\verse{And the one called to the other and said, “Holy, holy, holy is Adonai of hosts! The whole earth is full of his glory.”\lebnote{“fullness of all of the earth glory him”}}%
\verse{And the pivots of the thresholds shook from the sound of those who called, and the house\lebnote{Or “temple”} was filled with smoke.}%
\verse{And I said, “Woe to me! For I am destroyed!\lebnote{Or “silenced”} For I am a man of unclean lips,\lnBVL{} and I am living among\lebnote{With an emphatic sense: “in the very heart and midst of”} a people of unclean lips,\lnBVL{} for my eyes have seen the king, Adonai of hosts!”}%
\verse{Then one of the seraphs flew to me, and in his hand was a hot coal he had taken from the altar with tongs.}%
\verse{And he touched my mouth, and he said, “Look! This has touched your lips and has removed your guilt, and your sin is annulled.”}%
\verse{Then I heard the voice of the Lord saying, “Whom shall I send? And who will go for us?” And I said, “I am here! Send me!”}%
\verse{And he said, “Go and say to this people, ‘Keep on listening and do not comprehend! And keep on looking and do not understand!’}%
\verse{Make the heart of this people insensitive, and make its ears unresponsive, and shut its eyes so that it may not look with its eyes and listen with its ears and comprehend with its mind and turn back, and it may be healed for him.”}%
\verse{Then I said, “Until when, Lord?” And he said, “Until the cities lie wasted without inhabitant, and houses without people, and the land is ruined and a waste,}%
\verse{and Adonai sends the people far away, and the abandonment is great in the midst of the land.}%
\verse{And even if only a tenth part remain,\lebnote{“still in her a tenth”} again she will be destroyed\lebnote{“and she will again and she will be to burn”} like a terebinth or like an oak, which although felled, a tree stump remains in them. The seed of holiness will be her tree stump.”}%
\end{biblechapter}%
\begin{biblechapter}% Isaiah 7
\verseWithHeading{The Sign to Ahaz}{This happened in the days of Ahaz, son of Jotham, son of Uzziah, king of Judah. Rezin, king of Aram, and Pekah, son of Remaliah, king of Israel, went\lebnote{The Hebrew is singular} up to Jerusalem for warfare against it,\lnBVM{} but he was not able to fight against it.\lnBVM{}}%
\verse{When it was reported to the house of David, saying “Aram stands by Ephraim,” his heart and the heart of his people shook like the shaking of the trees of the forest because of the wind.}%
\verse{Then Adonai said to Isaiah, “Go out to meet Ahaz, you and Shear-Jashub your son, at the end of the conduit of the upper pool on the highway of the washer’s field.}%
\verse{And you must say to him, ‘Take heed and be quiet! You must not fear, and your heart must not be faint because of these two stumps of smoldering firebrands, because of the fierce anger of\lebnote{“because of the fierceness of the anger of”} Rezin and Aram and the son of Remaliah.}%
\verse{Because Aram has plotted evil against you with Ephraim and the son of Remaliah, saying,}%
\verse{“Let us go up against Judah and let us tear her apart, and let us lay it open and so bring it unto ourselves,\lebnote{“and let us break through her to us”} and let us install the son of Tabeel as king in her midst.”}%
\verse{Thus says the Lord Adonai, “It shall not stand, and it shall not come to pass.}%
\verse{For the head of Aram is Damascus, and the head of Damascus is Rezin, and in sixty-five years from now Ephraim will be too shattered to be a people.}%
\verse{And the head of Ephraim is Samaria, and the head of Samaria is the son of Remaliah. If you do not believe then you will not endure.”’”}%
\verse{And Adonai continued to speak to Ahaz, saying,}%
\verse{“Ask for a sign for yourself from Adonai God; make it deep as Sheol or make it high as above.”}%
\verse{But Ahaz said, “I will not ask, and I will not put Adonai to the test.”}%
\verse{Then he said, “Hear, house of David! Is it too little for you to make men weary, that you should also make my God weary?}%
\verse{Therefore the Lord himself will give you\lebnote{The Hebrew is plural} a sign. Look! the virgin\lebnote{Or “young woman”} is with child and she is about to give birth to a son, and she shall call his name ‘God with us.’}%
\verse{He shall eat curds and honey until he knows to reject the evil and to choose the good.}%
\verse{For before the boy knows to reject the evil and to choose the good, the land whose two kings you dread will be abandoned.\lebnote{“the land which you dread will be abandoned because of the face of her two kings”}}%
\verseWithHeading{That Day}{“Adonai will bring on you and on your people and on the house of your ancestor\lebnote{Or “father”} days that have not come since the day Ephraim departed from Judah: the king of Assyria.”}%
\verse{And this shall happen: On that day, Adonai will whistle for the fly that is at the end of the stream of Egypt and the bee that is in the land of Assyria.}%
\verse{And all of them will come and settle in the rivers of the cliffs and in the clefts of the rocks and on all of the thornbushes and watering places.}%
\verse{On that day, the Lord will shave the head and the hair of the feet with a razor of the one hired from beyond the river — with the king of Assyria — and it will even take off the beard.}%
\verse{And this shall happen: on that day, a young man will keep a young cow of the herd and two sheep alive.}%
\verse{And this shall happen: because of the abundance of milk production, he will eat curds, for every one that is left in the midst of the land will eat curds and honey.}%
\verse{And this shall happen on that day: Every place where there are a thousand vines\lebnote{Hebrew “vine”} for a thousand silver pieces will become briers,\lnBVN{} and it will be thornbushes.\lnBVO{}}%
\verse{One will go there with arrows and bow, for all of the land will be briers\lnBVN{} and thornbushes.\lnBVO{}}%
\verse{And as for all of the hills that they hoed with the hoe, you will not go there, for fear of briers\lnBVN{} and thornbushes.\lnBVO{} And it will become like pastureland for cattle and overtrodden land for sheep.}%
\end{biblechapter}%
\begin{biblechapter}% Isaiah 8
\verseWithHeading{Signs of the Assyrian Invasion}{Then Adonai said to me, “Take yourself a large tablet and write on it with a common stylus pen: Maher-Halal-Hash-Baz.}%
\verse{And I will require reliable witnesses as a witness for me: Uriah the priest and Zechariah son of Jeberekiah.”}%
\verse{And I approached the prophetess, and she conceived, and she gave birth to a son. And Adonai said to me, “Call his name Maher-Halal-Hash-Baz.}%
\verse{For before the boy knows to call ‘my father’ and ‘my mother,’ one will carry away the wealth of Damascus and the spoil of Samaria in the presence of the king of Assyria.”}%
\verseWithHeading{Shiloah Waters and Euphrates Flood}{And Adonai continued to speak to me again, saying,}%
\verse{“Because this people has refused the gently flowing waters of Shiloah and rejoices over Rezin and the son of Remaliah,}%
\verse{therefore look! The Lord is bringing up the waters of the great and mighty river against them, the king of Assyria and all his glory. And he will rise above all his channels, and he will flow over all his banks.}%
\verse{And he will sweep into Judah; he will overflow and he will flood up to the neck. He will reach, and he will spread his wings out over your entire land,\lebnote{“the outspreading of his wings will be the fullness of the breadth of your land”} God with us.”}%
\verse{Be broken, you peoples, and be dismayed. And listen, all distant parts of the earth; gird yourselves and be dismayed; gird yourselves and be dismayed!}%
\verse{Make a plan,\lebnote{“Plan counsel”} but it will be frustrated! Speak a word, but it will not stand, for God is with us!}%
\verseWithHeading{Wait for Adonai}{For Adonai said this to me while his hand weighed heavily on me,\lebnote{“with the strength of the hand”} and he warned me not to walk\lebnote{“instructed me from walking”} in the way of this people, saying,}%
\verse{“You must not call conspiracy everything that this people calls conspiracy, and you must not share its fear,\lebnote{“fear its fear”} and you must not be in dread.}%
\verse{You shall regard Adonai of hosts as holy, and he is your\lebnote{The Hebrew is plural} fear, and he is your dread.}%
\verse{And he will become like a sanctuary and a stumbling-stone, and like a stumbling-rock for the two houses of Israel, like a trap and a snare for the inhabitants of Jerusalem.}%
\verse{And many shall stumble among them, and they shall fall and they shall be broken, and they shall be ensnared and they shall be caught.”}%
\verse{Bind up the testimony; seal the teaching among my disciples.}%
\verse{And I will wait for Adonai, who hides his face from the house of Jacob, and I will await him.}%
\verse{Look! I and the children whom Adonai has given to me are like signs and portents in Israel from Adonai of hosts, the one who dwells on the mountain of Zion.}%
\verse{Now if they tell you, “Consult the ghosts and the spirits, those who chirp and those who mutter. Should not a people consult its gods, the dead on behalf of the living,}%
\verse{for teaching and for testimony?” surely they who speak like this have no dawn.\lebnote{“if not, they say like this word that there is no dawn for him,” which could also mean “if not, let them say a word like this: that there is no dawn for him”}}%
\verse{And it\lebnote{That is, the nation} will pass through it\lebnote{That is, the land} distressed and hungry, and this shall happen: when it is hungry, it will be enraged, and it will curse its king and its gods,\lebnote{Or “God”} and it will face upwar}%
\verse{or look to the earth. But look! Distress and darkness, the gloom of affliction! And it will be thrust into darkness!}%
\end{biblechapter}%
\begin{biblechapter}% Isaiah 9
\verseWithHeading{New Light: The Birth of a King}{\lebnote{Isaiah 9:1–21 in the English Bible is 8:23–9:20 in the Hebrew Bible} But there will be no gloom for those who were in distress.\lebnote{“Indeed there is no gloom for the one to whom there was anxiety for her”} In former times he\lebnote{That is, Adonai} treated the land of Zebulun and Naphtali with contempt, but in the future he will honor the way of the sea beyond the Jordan, Galilee of the nations.}%
\verse{The people who walked in darkness have seen a great light; light has shined on those who lived in a land of darkness.}%
\verse{You have made the nation numerous; you have not\lebnote{The written text (\textit{Kethib}) is “not,” but the reading tradition (\textit{Qere}) is “for it”} made the joy great. They rejoice in your presence as with joy at the harvest, as they rejoice when they divide plunder.}%
\verse{For you have shattered the yoke of its burden and the stick of its shoulder, the rod of its oppressor, on\lebnote{Hebrew “like”; the Hebrew letters for “like” and “on” look alike} the day of Midian.}%
\verse{For every boot that marches and shakes the earth\lebnote{“marching with shaking,” which might mean “marching \textit{is} with shaking”} and garment rolled in blood will\lebnote{“and it will”} be for burning — fire fuel.}%
\verse{For a child has been born for us; a son has been given to us. And the dominion will be on his shoulder, and his name is called Wonderful Counselor, Mighty God, Everlasting Father, Prince of Peace.}%
\verse{His dominion will grow continually, and to peace there will be no end\lebnote{“There is no end to the increase of the dominion and to peace”} on\lnBVP{} the throne of David and over\lnBVP{} his kingdom, to establish it\lebnote{That is, the kingdom} and sustain it with justice and righteousness now and forever. The zeal of Adonai of hosts will do this.}%
\verseWithHeading{Adonai’s Anger against Arrogance}{The Lord has sent out a word against Jacob, and it fell on Israel.}%
\verse{And all of the people knew it, Ephraim and the inhabitants of Samaria in pride and arrogance of heart, saying,}%
\verse{“The bricks have fallen, but we will build with dressed stone. The sycamore-fig trees were felled, but we will replace them with cedars.”}%
\verse{So Adonai strengthened the adversaries of Rezin\lebnote{Or “his adversaries” if a copyist added “of Rezin” in error} against him, and he provoked his enemies —}%
\verse{Aram from the east and Philistines\lebnote{Hebrew “Philistine”} from the west — and they devoured Israel with the whole mouth. He has not turned away his anger in all of this, and his hand is still stretched out.}%
\verse{And the people did not turn to the one who struck it,\lebnote{That is, the people} and they did not seek Adonai of hosts.}%
\verse{So Adonai cut off head and tail from Israel, palm branch and reed in one day.}%
\verse{Elders\lebnote{Hebrew “Elder”} and the respectable\lebnote{“one lifted up of face”} are the head, and prophets\lebnote{Hebrew “prophet”} who teach lies\lebnote{Hebrew “lie”} are the tail.}%
\verse{And the leaders of this people were misleading them, and those who were led were confused.}%
\verse{Therefore the Lord did not rejoice over its young men, and he did not have compassion on its orphans and widows, for everyone was godless and an evildoer, and every mouth was speaking folly. In all of this his anger did not turn away, and still his hand is stretched out.}%
\verse{For wickedness burned like fire; it consumed brier and thorn. And it kindled the thickets of the forest, and they swirled upward in a column of smoke.}%
\verse{The land was burned through the wrath of Adonai of hosts, and the people became like fire fuel. People had no compassion toward each other.\lebnote{“A man does not have compassion toward his brother”}}%
\verse{They devoured on the right but still were hungry and devoured on the left but they were not satisfied. Each one devoured the flesh of his arm,}%
\verse{Manasseh devoured Ephraim, and Ephraim Manasseh; together they were against Judah. In all of this his anger has not turned away, and still his hand is stretched out.}%
\end{biblechapter}%
\begin{biblechapter}% Isaiah 10
\verseWithHeading{Woes on the Wicked}{Ah! Those who decree decrees of evil, and writers who have written harm,}%
\verse{to guide the needy away from legal claims,\lebnote{Hebrew “claim”} and to rob the justice from the poor of my people, to make widows their spoil; and they plunder orphans.}%
\verse{And what will you do at the day of punishment, and at calamity? It comes from afar! To whom will you flee for help, and where will you leave your wealth,}%
\verse{save that they bow down under the prisoners and fall under the slain?\lebnote{“except he will bow down under a prisoner and under slain they will fall”; “under” could also mean “in the place of”} In all of this his anger has not turned away, and still his hand is stretched out.}%
\verseWithHeading{Judgment on Assyria’s Arrogance}{Ah! Assyria, the rod of my anger, and a staff is in their hand: my wrath!}%
\verse{I send him against a godless nation, and I command him against the people of my wrath, to capture spoil and to carry off plunder, and to make them\lebnote{Hebrew “him”} a trampling place, like the clay of the streets.}%
\verse{But he does not think this,\lebnote{Or “so”} and his heart does not plan this. For it is in his heart to destroy and to cut off not a few nations.}%
\verse{For he says, “Are not my commanders altogether kings?}%
\verse{Is not Calno like Carchemish? Is not Hamath like Arpad? Is not Samaria like Damascus?}%
\verse{As my hand has reached to the kingdoms of the idols\lebnote{Hebrew “idol”} — and their images were greater than those of\lebnote{“from”} Jerusalem and Samaria —}%
\verse{shall I not do to Jerusalem and its idols what I have done to Samaria and her idols?”}%
\verse{And this shall happen: when the Lord has finished all his work against Mount Zion\lebnote{“the mountain of Zion”} and Jerusalem, “I will punish the arrogance\lebnote{“fruit of the greatness of the heart”} of the king of Assyria and his haughtiness.”\lebnote{“the pride of the height of his eyes”}}%
\verse{For he says, “I have done it by the strength of my hand and by my wisdom, for I have understanding, and I have removed the boundaries of peoples, and I have plundered their stores, and like a bull I have brought down the inhabitants.\lebnote{Or, “those sitting,” that is, rulers sitting on thrones}}%
\verse{And my hand has found, like a nest, the wealth of the peoples, and like the gathering of forsaken eggs, I myself have gathered all the earth. And there was no fluttering wing or open mouth or chirp.”}%
\verse{Does the ax boast against the one who cuts with it, or the saw magnify itself against the one who moves it to and fro? As if a rod should move the one who lifts it!\lebnote{“As a rod waving even the one who lifts it up”} As if a staff should lift up that which is not wood!\lebnote{“As a staff lifting up not wood”}}%
\verse{Therefore the Lord, Adonai of hosts, will send leanness among his sturdy warriors, and a burning like the burning of fire will burn under his glory.}%
\verse{And the light of Israel will become like a fire, and his holy one like a flame, and it will burn and devour his thorns\lebnote{Hebrew “thorn”} and briers\lebnote{Hebrew “brier”} in one day.}%
\verse{And he will destroy the glory of his forest and orchard completely,\lebnote{“from soul and to body”} and it will be like the wasting away of one who is sick.}%
\verse{And the rest of the trees\lebnote{Hebrew “tree”} of his forest will be a small number, and a boy can write them down.}%
\verseWithHeading{The Return of the Remnant}{And this shall happen: on that day, the remnant of Israel and the survivors\lebnote{Hebrew “survivor”} of the house of Jacob will not continue to lean on the one who struck it but will lean on Adonai, the holy one of Israel, in truth.}%
\verse{A remnant will return — the remnant of Jacob — to the mighty God.}%
\verse{For though your people Israel was like the sand of the sea, only a remnant of it will return. Annihilation is determined, overflowing with righteousness.}%
\verse{For the Lord Adonai of hosts is about to make a complete destruction and a determined end in the midst of all the earth.}%
\verse{Therefore thus says the Lord Adonai of hosts: “My people who live in Zion, you must not be afraid of Assyria. It beats you with the rod, and it lifts up its staff against you as the Egyptians did.\lnBVQ{}}%
\verse{My indignation will come to an end in just a very little while,\lebnote{“for still a little a trifle”} and my anger will be directed to their destruction.”}%
\verse{And Adonai of hosts is going to swing a whip against him, as when Midian was defeated at the rock of Oreb; and his staff will be over the sea, and he will lift him up as he did in Egypt.\lnBVQ{}}%
\verse{And this shall happen: on that day, he will remove his burden from your shoulder and his yoke from your neck, and a yoke will be destroyed because of fat.\lebnote{The meaning of this phrase is uncertain, leading to the conjecture that it belongs with the next verse and by a different word division could mean “he has gone up from Jeshimon”; alternatively, “fat” could be a metaphor for prosperity}}%
\verse{He has come to Aiath, he has passed through Migron; at Micmash he deposited his baggage.}%
\verse{They crossed over the pass; Geba is a place of overnight lodging for us. Ramah trembles; Gibeah of Saul has fled.}%
\verse{Daughter of Gallim, cry out with your voice; Laishah, listen! Anathoth is poor.\lebnote{Or with different vocalization, “Answer her, Anathoth!” which fits the parallelism better}}%
\verse{Madmenah flees! The inhabitants of Gebim bring themselves into safety!}%
\verse{This day\lebnote{“Yet today”} taking a stand\lebnote{“to stand”} at Nob, he will shake his fist at the mountain of the daughter\lebnote{Following the reading tradition (\textit{Qere}); the consonantal text has “house”} of Zion, at the hill of Jerusalem.}%
\verse{Look! The Lord Adonai of hosts is about to lop off the branches\lebnote{Hebrew “branch”} with great power, and the towering trees\lebnote{“the haughty of the height”} will be felled, and the tall trees\lebnote{“height”} will be brought low.}%
\verse{And he will cut down the thickets of the forest with the axe, and Lebanon will fall by the mighty one.}%
\end{biblechapter}%
\begin{biblechapter}% Isaiah 11
\verseWithHeading{The Branch’s Righteous Reign}{And a shoot will come out from the stump of Jesse, and a branch from its roots will bear fruit.}%
\verse{And the spirit of Adonai shall rest on him — a spirit of wisdom and understanding, a spirit of counsel and might, a spirit of knowledge and the fear of Adonai.}%
\verse{And his breath\lebnote{Hebrew “smelling,” perhaps metaphorically as “delight” as in Amos 5:14; perhaps this line was accidentally copied twice from the preceding line} is in the fear of Adonai. And he shall judge not by his eyesight, and he shall rebuke not by what he hears with\lebnote{“the rumors of”} his ears.}%
\verse{But he shall judge the poor with righteousness, and he shall decide for the needy of the earth with rectitude. And he shall strike the earth with the rod of his mouth, and he shall kill the wicked person with the breath of his lips.}%
\verse{And righteousness shall be the belt around his waist, and faithfulness the belt around his loins.}%
\verse{And a wolf shall stay\lebnote{“dwell as an alien”} with a lamb, and a leopard shall lie down with a kid, and a calf and a lion and a fatling together as a small boy leads\lebnote{“and a small boy leading”} them.}%
\verse{And a cow and a bear shall graze; their young shall lie down together. And a lion shall eat straw like the cattle.}%
\verse{And an infant\lebnote{“one nursing”} shall play over a serpent’s hole, and a toddler\lebnote{“one who is weaned”} shall put his hand on a viper’s hole.}%
\verse{They will not injure and they will not destroy on all of my holy mountain,\lebnote{“mountain of holiness”} for the earth will be full of the knowledge of Adonai, as the waters cover the sea.}%
\verse{And this shall happen on that day: the nations shall inquire of the root of Jesse, which shall be standing as a signal to the peoples, and his resting place shall be glorious.}%
\verseWithHeading{The Regathered Remnant}{And this shall happen on that day: The Lord will again extend his hand a second time to acquire the remnant of his people that is left, from Assyria, Egypt, Pathros, Cush, Elam, Shinar, Hamath, and the coastlands of the sea,}%
\verse{and he will raise a signal for the nations. And he will gather the outcasts of Israel, and he will gather the scattered ones of Judah together from the four corners\lebnote{“wings”} of the earth.}%
\verse{And the jealousy of Ephraim shall depart, and the enemies of Judah shall be cut off. Ephraim shall not be jealous of Judah, and Judah shall not be an enemy of Ephraim.}%
\verse{But they shall swoop\lebnote{“fly”} upon the Philistine shoulder,\lebnote{That is, the hill country west of Judah} westward.\lebnote{“sea”} Together they shall plunder the sons of the east. Edom and Moab will be under their command,\lebnote{“the stretching of their hand”} and the sons of Ammon will be their subjugated people.}%
\verse{And Adonai will divide\lebnote{Or “utterly destroy”} the tongue\lebnote{Probably the “gulf”} of the sea of Egypt,\lebnote{That is, the Red Sea} and he will wave his hand over the river\lebnote{That is, the Euphrates River} with his scorching wind; and he will strike it into seven streams, and he will make it passable by foot.\lebnote{“cause to walk with the sandal”}}%
\verse{So there shall be a highway from Assyria for the remnant of his people that remains, as there was for Israel when\lebnote{“in \textit{the} day”} it went up from the land of Egypt.}%
\end{biblechapter}%
\begin{biblechapter}% Isaiah 12
\verseWithHeading{A Song of Thanksgiving}{And you will say on that day, “I will give you thanks, Adonai, for though you were angry with me, your anger turned away, and you comforted me.}%
\verse{Look! God is my salvation; I will trust, and I will not be afraid, for my strength and might is Yah, Adonai; and he has become salvation for me.”}%
\verse{And you will draw water from the wells of salvation in joy.}%
\verse{And you will say on that day, “Give thanks to Adonai; call on his name. Make his deeds known among the peoples; bring to remembrance that his name is exalted.}%
\verse{Sing praises to Adonai, for he has done a glorious thing; this is known in all the earth.}%
\verse{Inhabitant of Zion, shout out and sing for joy, for the holy one of Israel is great in your midst.”}%
\end{biblechapter}%
\begin{biblechapter}% Isaiah 13
\verseWithHeading{An Oracle against Babylon}{The oracle of Babylon that Isaiah son of Amoz saw:}%
\verse{Raise a signal on a bare hill, lift up your voice to them; wave the hand and may they enter the gateways of the noblemen.}%
\verse{I myself I have commanded my consecrated ones, I have also summoned my mighty warriors concerning my anger, the ones who exalt over\lebnote{“exultant of”} my majesty.}%
\verse{A sound, a noise is on the mountains, the likeness of many people! A sound of the roar of the kingdoms, of nations gathering! Adonai of hosts is mustering an army for battle.}%
\verse{They are coming from a distant land, from the end of the heavens, Adonai and the weapons of his indignation, to destroy the whole earth.\lebnote{“all of the land/earth”}}%
\verse{Wail, for the day of Adonai is near; it will come like destruction from Shaddai!\lebnote{Often translated “the Almighty”}}%
\verse{Therefore all hands will grow slack, and every human heart will melt,}%
\verse{and they will be dismayed. Pangs and labor pains will seize them; they will tremble like a woman giving birth. They will stare at one another,\lebnote{“A man will stare at his neighbor”} their faces flushing.\lebnote{“faces of flames”}}%
\verse{Look! The day of Adonai is coming, cruel and wrath and the burning of anger, to make the earth a desolation, and he will destroy its sinners from it.}%
\verse{For the stars of the heavens and their constellations will not flash forth their light; the sun will keep back when it comes out, and the moon will not cause its light to shine.}%
\verse{And I will punish the world for its evil and the wicked for their iniquity. And I will put an end to the pride of the arrogant, and I will bring the haughtiness of tyrants low.}%
\verse{I will make humanity more rare than gold and humankind more than the gold of Ophir.}%
\verse{Therefore I will make the heavens tremble, and the earth will quake from its place because of the wrath of Adonai of hosts, and in the day his anger burns.}%
\verse{And this shall happen: like a hunted gazelle or sheep with none to gather them,\lebnote{“and without one who gathers”} they will each turn to his own people, and they will each flee to his own land.}%
\verse{Everyone who is found will be pierced through, and everyone who is carried away will fall by the sword.}%
\verse{And their children will be dashed into pieces before their eyes; their houses will be plundered, and their wives will be raped.\lebnote{The reading tradition (\textit{Qere}) softens this to “slept with”}}%
\verse{Look! I am stirring the Medes up against them, who do not value silver and do not delight in gold.}%
\verse{And their bows will shatter young men. And they will not show mercy on the fruit of the womb; their eyes will not look compassionately on children.}%
\verse{And Babylon, the glory of kingdoms, the splendor of the Chaldeans’ pride, will be like when God overthrew\lebnote{“the overthrow by God of”} Sodom and Gomorrah.}%
\verse{It will not be inhabited forever, and it will not be dwelled in forever;\lebnote{“until generation and generation”} and no Arab will pitch a tent there, and shepherds will not allow their flocks to lie down there.}%
\verse{But wild animals will lie down there, and their houses will be full of howling creatures, and the daughters of ostriches\lebnote{Hebrew “ostrich”} will live there, and goats will dance there.}%
\verse{And hyenas will answer in its palaces, and jackals in the pleasure palaces; and its time is coming soon,\lebnote{“close to come”} and its days will not be prolonged.}%
\end{biblechapter}%
\begin{biblechapter}% Isaiah 14
\verseWithHeading{The Restoration of Israel}{But Adonai will have compassion on Jacob, and he will again choose Israel and set them on their land, and the immigrant will join himself to them, and they will attach themselves to the house of Jacob.}%
\verse{And the nations will take them and bring them to their place, and the house of Israel will take possession of them\lebnote{That is, the nations} in the land of Adonai as slaves and female slaves. And this will happen: they will take their captors captive and rule over their oppressors.}%
\verseWithHeading{The Downfall of the King of Babylon}{And it shall happen on the day Adonai gives you rest from your pain and turmoil and hard labor which you had to perform,\lebnote{“was worked by you”}}%
\verse{that you will take this taunt against the king of Babylon, and you will say: “How the oppressor has ceased! his insolence\lebnote{The meaning of the word is uncertain; others translate it as “fury,” “hostility,” or even “golden city”} has ceased.}%
\verse{Adonai has broken the staff of the wicked, the scepter of rulers,}%
\verse{that struck the peoples in wrath, a blow without ceasing, that ruled the nations in anger with unrestrained persecution.\lebnote{“persecution without withholding”}}%
\verse{All of the earth rests and is quiet; they break forth into singing.}%
\verse{Even the cypresses rejoice over you, the cedars of Lebanon: ‘Since you were laid down, no wood cutter comes up against us.’}%
\verse{Sheol below is getting excited over you, to meet you when you come;\lebnote{“your entrance”} it arouses the dead spirits for you, all of the leaders of the earth. It raises all of the kings of the nations from their thrones.}%
\verse{All of them will respond and say to you, ‘You yourself also were made weak like us! You have become the same as us!’}%
\verse{Your pride is brought down to Sheol, and the sound of your harps; maggots\lebnote{Hebrew “maggot”} are spread out beneath you like a bed, and your covering is worms.\lebnote{Hebrew “worm”}}%
\verse{How you have fallen from heaven, morning star, son of dawn! You are cut down to the ground, conqueror of nations!}%
\verse{And you yourself said in your heart, ‘I will ascend to heaven; I will raise up my throne above the stars of God; and I will sit on the mountain of assembly on the summit of Zaphon;\lebnote{Or “of the north”}}%
\verse{I will ascend to the high places of the clouds,\lebnote{Hebrew “cloud”} I will make myself like the Most High.’}%
\verse{But you are brought down to Sheol, to the depths of the pit.}%
\verse{Those who see you will stare at you, they will look closely at you: ‘Is this the man who made the earth tremble, who caused kingdoms to shake,}%
\verse{who made the world like the desert and destroyed its cities, who would not let his prisoners go home?’}%
\verse{All the kings of the nations, all of them, lie in glory, each one in his house.}%
\verse{But as for you, you are thrown away from your grave, like an abhorrent shoot, clothed with the slain, those pierced by the sword, those who go down to the stones of the pit, like a corpse that is trodden down.}%
\verse{You will not be united with them in burial because you have destroyed your land, you have killed your people. The descendants\lebnote{Hebrew “descendant”} of evildoers will not be mentioned for eternity!}%
\verse{Prepare a place of slaughter for his sons because of the sin of their ancestors.\lebnote{Or “fathers”} Let them not rise and take possession of the earth or fill up the face of the world with cities.”}%
\verse{“And I will rise up against them,” declares\lnBVR{} Adonai of hosts, “and I will cut off name and a remnant from Babylon, and offspring and posterity,” declares\lnBVR{} Adonai.}%
\verse{“And I will make her a possession of the hedgehog, and pools of water, and I will sweep her away with the broom of destruction,” declares\lnBVR{} Adonai of hosts.}%
\verseWithHeading{Oracle of Judgment on Assyria}{Adonai of hosts has sworn, saying, “Surely\lebnote{The oath formula begins literally “If not”} just as I have intended, so it shall be. And just as I have planned, it shall stand:}%
\verse{to break Assyria in my land, and I will trample him down on my mountains; and he shall remove his yoke from them, and he shall remove his burden from his\lebnote{That is, their} shoulders.”\lebnote{Hebrew “shoulder”}}%
\verse{This is the plan that is planned concerning all of the earth; and this is the hand that is stretched out over all of the nations.}%
\verse{For Adonai of hosts has planned, and who will frustrate it? And his hand is stretched out, and who will turn it back?}%
\verseWithHeading{Oracle of Judgment on Philistia}{In the year of the death of king Ahaz there was this oracle:}%
\verse{You must not rejoice, all you Philistines, that the rod that struck you is broken, for a viper will come forth from the root of the snake, and its fruit will be a flying serpent.}%
\verse{And the firstborn of the poor will graze, and the needy will lie down in security; but I will cause your root to die in famine, and it will kill your remnant.}%
\verse{Wail, gate! Cry, city! Melt,\lebnote{With fear or despair} Philistia, all of you! For smoke is coming from the north, and there is no straggler in his ranks.}%
\verse{And what will one answer the messengers of the nation? That Adonai has founded Zion, and the needy of his people will take refuge in it.}%
\end{biblechapter}%
\begin{biblechapter}% Isaiah 15
\verseWithHeading{Oracle of Judgment on Moab}{An oracle of Moab: Because Ar is devastated in a night, Moab is destroyed; because Kir of Moab is devastated in a night, it is destroyed.}%
\verse{It has gone up to the house,\lebnote{Or “temple”} and Dibon to the high places for weeping over Nebo, and Moab wails over Medeba. Every head is bald,\lebnote{“On all its heads baldness”} every beard is shaved.}%
\verse{They gird themselves with sackcloth in its streets; on its roofs and public squares everyone wails, going down in weeping.}%
\verse{And Heshbon and Elealeh cry\lebnote{The Hebrew is singular} out, their voice is heard as far as Jahaz; therefore the armed ones of Moab cry out; its soul quivers\lebnote{The Hebrew words for “cry out” and “quiver” are similar} for him.}%
\verse{My heart cries out for Moab; its fugitives flee up to Zoar, to Eglath-shelishiyah. For at the ascent of Luhith it goes up it with weeping; for on the road of Horonaim they raise up a cry of destruction.}%
\verse{For the waters of Nimrim are wastelands; for the grass has withered, the vegetation has vanished, there is no greenness.}%
\verse{Therefore they carry the abundance it has made and their store of goods over the river of the poplars.}%
\verse{For a cry for help has encircled the territory of Moab, her wailing is heard as far as Eglaim, and her wailing as far as Beer Elim.}%
\verse{For the waters of Dimon\lebnote{One of the Dead Sea Scrolls reads “Dibon” here} are full of blood; but I will place added things upon Dimon: a lion for the survivors\lebnote{Hebrew “survivor”} of Moab and for the remnant of the land.}%
\end{biblechapter}%
\begin{biblechapter}% Isaiah 16
\verse{Send a ram\lebnote{Possibly “rams” if a scribe accidentally omitted the Hebrew plural ending because the next word begins with that same letter} to the ruler of the land, from Sela across the desert to the mountain of daughter Zion.\lebnote{“the daughter of Zion”}}%
\verse{And this shall be: like a bird fleeing from a thrust away nest shall be the daughters of Moab at the fords of Arnon.}%
\verse{“Bring counsel, make a decision; make your shade like the night in the middle of noonday. Hide the outcasts; you must not betray the fugitive.}%
\verse{Let my outcasts of Moab dwell as aliens among you; be a hiding place for them from the presence of the destroyer.” When the oppressor is no more, destruction has stopped, the one who tramples has\lebnote{“one who tramples have,” with mismatched singular and plural} disappeared from the land,}%
\verse{then a throne shall be established in steadfast love, and one shall sit on it in faithfulness, in the tent of David, judging and seeking justice and zealous for righteousness.}%
\verse{We have heard of the pride of Moab — exceedingly proud — of his arrogance, pride, and insolence; his boasting is not true.\lebnote{“not so his boasting”}}%
\verse{Therefore Moab wails; all of it wails for Moab, for the raisin cakes of Kir-hareseth you moan, utterly devastated.\lebnote{“surely destroyed”}}%
\verse{For Heshbon withers the fields, the vine of Sibmah; rulers of nations have broken down her tendrils, they reached up to Jazer, they wandered to the desert; her shoots spread abroad, they crossed over the sea.}%
\verse{Therefore I weep with the weeping of Jazer for the vine of Sibmah. I drench you with my tears,\lebnote{Hebrew “tear”} Heshbon and Elealeh, for a jubilant shout has fallen over your summer fruit and harvest.}%
\verse{And joy and gladness are\lebnote{The Hebrew is singular} taken away from the fruitful land, and in the vineyards no one exults, no one shouts for joy; no treader treads wine in the presses; I have put to an end to the jubilant shout.}%
\verse{Therefore my heart moans\lebnote{“intestines moan,”; in Hebrew, the “intestines” are the seat of the emotions, which would correspond to the “heart” in English} like a harp for Moab and my inner parts for Kir-heres.}%
\verse{And this shall happen: when Moab appears, when it is weary upon the high place and it comes to its sanctuary to pray, it will not prevail.}%
\verse{This was the word that Adonai spoke to Moab in the past.\lebnote{“from then”}}%
\verse{But now Adonai speaks, saying, “In three years, like the years of a hired worker, the glory of Moab will become contemptible, with all of the great multitude, and the remnant will be a few, small, not strong.}%
\end{biblechapter}%
\begin{biblechapter}% Isaiah 17
\verseWithHeading{Oracle of Judgment on Damascus}{An oracle of Damascus: “Look! Damascus will cease being a city and will become a heap of ruins.}%
\verse{The cities of Aroer will be deserted;\lebnote{These words in Hebrew (and “flocks” in the next line) all begin with the same letter, Ayin} they will be for the flocks, and they will lie down and no one will frighten\lebnote{“there is not one who frightens”} them.}%
\verse{And the fortified city will disappear from Ephraim, and the kingdom from Damascus; and the remnant of Aram will be like the glory of the children of Israel,” declares\lnBVS{} Adonai of hosts.}%
\verse{“And this shall happen: On that day, the glory of Jacob will be brought low, and the fat of his flesh will become lean.}%
\verse{And it shall be as when a reaper gathers\lebnote{“a gathering of a reaper of”} standing grain and he reaps grain with his arm, and it shall be like one who gathers ears of grain in the valley of Rephaim.}%
\verse{And gleanings will be left over in it, as when an olive tree is beaten,\lebnote{“beating of an olive tree”} two or three ripe olive berries in the top of a branch, four or five on its fruitful branches,” declares\lnBVS{} Adonai, the God of Israel.}%
\verse{On that day, mankind will look to its maker, and its eyes will look to the holy one of Israel;}%
\verse{it will not look to the altars, the work of its hands, and it will not see what its fingers made and the poles of Asherah worship and the incense altars.}%
\verse{On that day, its fortified cities\lebnote{“the cities of his fortress”} will be like the abandonment of the wooded place and the summit,\lebnote{Perhaps this difficult phrase originally read “abandonment of the wooded heights of the Amorites”} which they deserted because of the children of Israel; and there will be desolation.}%
\verse{For you have forgotten the God of your salvation, and you have not remembered the rock of your refuge; therefore you plant plants of pleasantness, and you plant\lebnote{“plant it”} a vine of a foreigner.}%
\verse{On your planting day you make them grow, and in the morning of your sowing you bring them into bloom, yet the harvest will flee\lebnote{Reading the same consonants as a verb, \textit{nad}, rather than the noun \textit{ned,} which would mean “a heap \textit{of}\textit{the} harvest”} in a day of sickness and incurable pain.}%
\verseWithHeading{The Roar of the Peoples}{Ah! The noise of many peoples, they make a noise like the noise of the seas! And the roar of nations, they roar like the roar of mighty waters!}%
\verse{The nations roar like the roar of many waters, but he will rebuke him, and he will flee far away. And they are chased like chaff of the mountains before the wind and like tumbleweed before the storm.}%
\verse{At the time of evening, and look, terror! Before morning he is no more. This is the fate of those who plunder us and the lot of those who plunder us.}%
\end{biblechapter}%
\begin{biblechapter}% Isaiah 18
\verseWithHeading{Oracle of Judgment on Cush}{Ah! land of the whirring of wings, which is beyond the rivers of Cush,}%
\verse{that sends messengers by the sea and in vessels of papyrus on the surface of the waters! Go, swift messengers, to a tall\lnBVT{} and smooth\lnBVU{} nation, to a people feared near and far,\lebnote{“from this and beyond”} a mighty, mighty\lebnote{Perhaps this is a gibberish phrase in Hebrew, \textit{qaw-qaw}} and trampling\lebnote{Meaning uncertain} nation, whose land rivers divide.}%
\verse{All you inhabitants of the world and dwellers of the earth, when a signal is raised on the\lebnote{“as a raising of a signal”} mountains, you must look, and when a trumpet is blown,\lebnote{“as a blowing of a trumpet”} you must listen!}%
\verse{For Adonai said this to me: “I will be quiet, and I will look from my dwelling place like clear heat because of light,\lebnote{The meaning of this phrase is uncertain; perhaps the “light” is sunlight} like a cloud of dew in the heat of harvest.”}%
\verse{For before the harvest, when the blossom is complete\lebnote{“as a completion of a blossom”} and a blossom becomes ripening fruit, and one cuts off the shoots with pruning hooks, and one removes, tears away the tendrils.}%
\verse{They shall all be left\lebnote{“be left together”} for birds\lnBVV{} of prey of the mountains and for the animals\lebnote{Hebrew “animal”} of the earth. And the birds\lnBVV{} of prey will pass the summer on it, and every animal of the earth will winter on it.}%
\verse{At that time, a gift\lebnote{Perhaps tribute is meant} will be brought to Adonai of hosts from a tall\lnBVT{} and smooth\lnBVU{} people, and from a people feared near and far, a mighty, mighty and trampling nation, whose land the rivers divide, to the place of the name of Adonai of hosts, the mountain of Zion.}%
\end{biblechapter}%
\begin{biblechapter}% Isaiah 19
\verseWithHeading{Oracle of Judgment on Egypt}{An oracle of Egypt: Look! Adonai is riding on a swift cloud and is coming to Egypt. And the idols of Egypt will tremble in front of him, and the heart of Egypt melts in his inner parts.}%
\verse{“And I will stir up Egyptians\lnBVW{} against Egyptians,\lnBVW{} and each one will fight against his brother and each one against his neighbor, city against city, kingdom against kingdom.}%
\verse{And the spirit of the Egyptians\lnBVW{} will be disturbed in his midst, and I will confuse\lebnote{“engulf”} his plans,\lebnote{Hebrew “plan”} and they will consult the idols and the spirits of the dead, and the ghosts and the spiritists.}%
\verse{And I will hand over the Egyptians\lnBVW{} into the hand of a hard master, and a powerful king will rule over them,” declares\lebnote{“declaration of”} the Lord Adonai of hosts.}%
\verse{And the waters will be dried up from the sea, and the river will be parched and dry.}%
\verse{And the rivers will become foul-smelling; the branches of the Nile of Egypt will become little and dry up; reed and rush will wither.}%
\verse{Bare places by the Nile will be dried up, by the edge of the Nile and all the sown land of the Nile; it will be driven about, and it will be no more.\lebnote{“there is none of it”}}%
\verse{And the fishers will mourn, and all of those who cast fishhooks\lebnote{Hebrew “fishhook”} in the Nile will lament, and those who spread out fishing nets\lebnote{Hebrew “net”} on the surface\lebnote{“face”} of the water will languish.}%
\verse{And the workers of combed flax will be ashamed, and those who weave white linen.}%
\verse{And her weavers will be crushed; all the hired workers\lebnote{“workers of wage”} will be grieved of heart.}%
\verse{Surely the princes of Zoan are foolish; the wise of the counselors of Pharaoh give senseless counsel. How can you say to Pharaoh, “I myself am a son of sages, a descendant\lebnote{“son”} of ancient kings”?}%
\verse{Where are your sages then? Now, let them tell you, and let them know what Adonai of hosts has planned against Egypt.}%
\verse{The princes of Zoan have become foolish; the princes of Memphis are deceived; the leaders of her tribes have led Egypt astray.}%
\verse{Adonai has mixed a spirit of confusion into her midst, and they have caused Egypt to stagger in all of its doings, as when a drunkard staggers\lebnote{“the staggering of a drunkard”} in his vomit.}%
\verse{And there will be nothing for Egypt to do,\lebnote{“not it will be for Egypt a deed which he can do”} head or tail, palm branch or reed.}%
\verseWithHeading{The Future Blessing of Egypt, Assyria, and Egypt}{On that day, Egypt will be like women, and will tremble and be afraid before the waving hand of Adonai of hosts that he is waving against it.}%
\verse{And the land of Judah will become a terror to Egypt, everyone to whom one mentions it will be afraid in himself because of the plan of Adonai of hosts that he is planning against him.}%
\verse{On that day, there will be five cities in the land of Egypt that speak the language\lebnote{“lip”} of Canaan and swear an oath to Adonai of hosts. One will be called “City of the Sun.”}%
\verse{On that day, there will be an altar for Adonai in in the middle of the land of Egypt, and a stone pillar for Adonai at her border.}%
\verse{And it will be a sign and a witness to Adonai of hosts in the land of Egypt; when they cry out to Adonai because of oppressors, he will send them a savior and a defender,\lebnote{“contender”} and he will deliver them.}%
\verse{And Adonai will make himself known to Egypt, and Egypt will know Adonai on that day, and they will worship with sacrifice and offering, and they will make vows\lebnote{Hebrew “vow”} to Adonai, and they will fulfill them.}%
\verse{And Adonai will strike Egypt, striking and healing; and they will return to Adonai, and he will respond to their prayer, and he will heal them.}%
\verse{On that day, there will be a highway from Egypt to Assyria, and Assyria will come into Egypt, and Egypt into Assyria, and Egypt will worship together with Assyria.}%
\verse{On that day, Israel will be the third with Egypt and Assyria, a blessing in the midst of the earth,}%
\verse{whom Adonai of hosts blessed, saying, “May Egypt my people be blessed, and Assyria, the work of my hands, and my inheritance, Israel.”}%
\end{biblechapter}%
\begin{biblechapter}% Isaiah 20
\verseWithHeading{A Sign regarding Egypt and Cush}{In the year the commander-in-chief came\lebnote{“of \textit{the} coming of \textit{the} commander in chief”} to Ashdod, when Sargon the king of Assyria sent him, and he fought against Ashdod and he took it,}%
\verse{at that time, Adonai had spoken by the hand of Isaiah son of Amoz, saying, “Go and loosen the sackcloth from your loins, and take off your sandals\lebnote{Hebrew “sandal”} from your feet,” and he had done so, walking naked and barefoot.}%
\verse{Then\lebnote{Or “And”} Adonai said, “Just as my servant Isaiah has walked naked and barefoot three years as a sign and a portent against Egypt and Cush,}%
\verse{so shall the king of Assyria lead the captives\lebnote{Hebrew “captive”} of Egypt and the exiles of Cush, young and old, naked and barefoot, with bared buttocks,\lebnote{“and bare of buttocks”} the shame of Egypt.}%
\verse{And they shall be dismayed, and they shall be ashamed because of Cush, their hope, and because of Egypt, their pride.}%
\verse{And the inhabitant\lebnote{Hebrew “inhabitant”} of the coastland will say this on that day: ‘Look! This is our hope to whom we fled for help, to be delivered from\lebnote{“from the face of”} the king of Assyria, and how shall we escape?’”}%
\end{biblechapter}%
\begin{biblechapter}% Isaiah 21
\verseWithHeading{The Fall of Babylon}{The oracle of the wilderness of the sea: As storm winds passing over in the Negev, it comes\lebnote{“to come”} from the desert, from a frightful land.}%
\verse{A hard revelation is told to me; the treacherous deals treacherously, and the destroyer destroys. Go up, Elam; lay siege, Media! I put an end to all of her sighing.}%
\verse{Therefore my loins are filled with anguish; pangs have seized me, like the pangs of a woman giving birth. I am too bent to hear,\lebnote{“bent from hearing,” which could also mean “bent because of what I heard”} I am too dismayed to see.\lebnote{“dismayed from seeing,” which could also mean “dismayed because of what I saw”}}%
\verse{My mind\lebnote{“heart”} staggers; fear terrifies me; the twilight I desired\lebnote{“of my desire”} brought\lebnote{“put”} me fear.}%
\verse{Set out the table in order! Spread out the rugs!\lebnote{Hebrew “rug”} Eat! drink! Rise up, commanders; smear\lebnote{That is, prepare with oil} the shield!}%
\verse{For the Lord said this to me: “Go, set a watchman in position. He must announce what he sees.}%
\verse{When\lnBVX{} he sees riders,\lnBVY{} a pair of horsemen, riders\lnBVY{} of donkeys,\lebnote{Hebrew “donkey”} riders\lnBVY{} of camels,\lebnote{Hebrew “camel”} then\lnBVZ{} he must listen attentively, paying attention, paying special\lebnote{“much”} attention.”}%
\verse{Then\lnBVX{} the watchman\lebnote{“lion”} called, “Lord, I am standing on the watchtower continually by day, and I am standing at my post throughout\lebnote{“all of”} the night.}%
\verse{And look at this! A man’s a chariot is coming, a pair of horsemen!” Then\lnBVZ{} he responded and said, “It has fallen! Babylon has fallen! And all the images of her gods are\lebnote{Hebrew “is”} smashed on the ground!”}%
\verse{My downtrodden people and the son of my threshing floor, I will announce to you what I have heard from Adonai of hosts, the God of Israel.}%
\verseWithHeading{Oracle regarding Dumah}{The oracle of Dumah: One is calling to me from Seir, “Watchman, what of\lnBWA{} the night? Watchman, what of\lnBWA{} the night?”}%
\verse{The watchman says, “Morning comes, and also the night. If you will inquire, inquire; come back again.”\lebnote{“return come”}}%
\verseWithHeading{Oracle against Arabia}{An oracle concerning\lebnote{“in”} Arabia: You will spend the night in the thicket in a desert-plateau,\lebnote{The Hebrew for “in a desert-plateau” is the same as for “concerning Arabia” in the preceding line} caravans of Dedanites.}%
\verse{When you happen to meet\lebnote{“Toward encountering”} the thirsty, bring water. Inhabitants of the land of Tema came to meet the fugitive with his bread.}%
\verse{For they have fled from the swords, from the drawn sword and the bent bow, and from the heaviness of the battle.}%
\verse{For the Lord said this to me: “In one more year,\lebnote{“In yet a year”} like the years of a hired worker, all the glory of Kedar will come to an end.}%
\verse{And the remainder of the number of the bows\lebnote{Hebrew “bow”} of the warriors of the sons of Kedar will be few.” For Adonai, the God of Israel, has spoken.}%
\end{biblechapter}%
\begin{biblechapter}% Isaiah 22
\verseWithHeading{Oracle against Jerusalem}{The oracle of the valley of vision: What business do you have going\lebnote{“What to you then that you have gone”} up, all of you, to the roofs,}%
\verse{noisy,\lebnote{“full of noises”} tumultuous city, exultant town? Your slain are not slain by the sword, nor are they dead from battle.}%
\verse{All of your rulers have fled together without a bow; all of you who were found\lebnote{“your found ones”} were captured. They were captured together; they had fled far away.}%
\verse{Therefore I said, “Look away from me, let me weep bitterly;\lebnote{“be bitter in weeping”} you must not insist on comforting me for the destruction of the daughter of my people.”}%
\verse{For the Lord Adonai of hosts has a day of tumult and trampling and confusion\lebnote{These three Hebrew words are alliterative: \textit{m}\textit{e}\textit{hûm}\textit{â}\textit{ûm}\textit{e}\textit{ḇûs}\textit{â}\textit{ûm}\textit{e}\textit{ḇû}\textit{ḵâ}} in the valley of vision, a tearing down of walls\lebnote{Hebrew “wall”} and a cry for help to the mountains.\lebnote{Hebrew “mountain”}}%
\verse{And Elam lifted up the quiver, with chariots\lnBWB{} of men\lebnote{Hebrew “man”} and cavalry. And Kir uncovered the shield.}%
\verse{And this happened: the choicest of your valleys were full of chariots,\lnBWB{} and the cavalry confidently stood at the gate.}%
\verse{And he uncovered the covering of Judah. And you looked, on that day, to the weapons\lebnote{Hebrew “weapon”} of the House of the Forest,}%
\verse{and you saw that the breaches in the walls of the city of David were many, and you gathered the waters of the lower pool.}%
\verse{And you counted the houses of Jerusalem, and you broke down the houses to fortify the wall.}%
\verse{And you made a reservoir between the walls for the waters of the old pool, but you did not look to its maker, and you did not see the one who created it long ago.\lebnote{“its creator from far away”}}%
\verse{And the Lord, Adonai of hosts, called on that day for weeping and mourning, and for baldness and girding with\lebnote{“of”} sackcloth.}%
\verse{But look! Joy and gladness, the killing of oxen\lebnote{Hebrew “ox”} and the slaughtering of sheep, the eating of meat and the drinking of wine! “Let us eat and drink, for tomorrow we die!”}%
\verse{And it was revealed in my ears by Adonai of hosts: “Surely this sin will not be atoned for you until you die!” says the Lord, Adonai of hosts.}%
\verseWithHeading{Oracle regarding Shebna}{The Lord, Adonai of hosts, says this: “Go! Go to this steward, to Shebna, who is over the house:}%
\verse{‘What business do you have\lnBWC{} here, and who do you have\lnBWC{} here, that you have cut a grave cutting here for yourself, carving his grave on the height, a dwelling place for him in the rock?}%
\verse{Look! Adonai is about to really hurl\lebnote{“hurl a hurling”} you, man! And he is about to grasp you firmly;}%
\verse{he will wind a winding tightly around you like a ball, to a wide land.\lebnote{“land broad of sides”} There you shall die, and there the chariots of your splendor will be, disgrace to your master’s house!}%
\verse{And I will push you from your office, and he will throw you down from your position.}%
\verse{And this shall happen: On that day I will call to my servant, Eliakim son of Hilkiah,}%
\verse{and I will clothe him with your tunic, and I will bind your sash firmly about him, and I will put your authority into his hand, and he shall be like a father to the inhabitants\lebnote{Hebrew “inhabitant”} of Jerusalem and to the house of Judah.}%
\verse{And I will put the key of the house of David on his shoulder, and he shall open and no one will be able to shut; and he shall shut and no one will be able to open.}%
\verse{And I will drive him in like a peg into a secure place, and he will become like a throne of glory to the house of his father.}%
\verse{And they will hang all of the heaviness of his father’s house on him, the offspring and the offshoot, all of the small vessels, from the bowls to the jars.\lebnote{“vessels of the bowls to all of the vessels of the jars”}}%
\verse{On that day, declares\lebnote{“declaration of”} Adonai of hosts, the peg that was driven will move away into a secure place, and it will be cut down and fall, and the load that was on her will be cut off. For Adonai has spoken.’”}%
\end{biblechapter}%
\begin{biblechapter}% Isaiah 23
\verseWithHeading{Oracle against Tyre}{The oracle of Tyre: Wail, ships of Tarshish, for the house is destroyed so that no one can enter;\lebnote{“it is destroyed from a house from entering”} it is announced to them from the land of Cyprus.}%
\verse{Be still, inhabitants of the coast, merchant of Sidon, who travels over the sea, they filled you.}%
\verse{And on the great waters is the seed of Shihor, the harvest of the Nile is its produce, and she was the merchandise of the nations.}%
\verse{Be ashamed, Sidon, for the sea, the fortress of the sea said, saying, “I was not in labor, and I have not given birth, and I have not reared young men, brought\lebnote{Or “I have brought”} up young women.”}%
\verse{When the report comes to Egypt, they will be in anguish at the report about\lebnote{“of”} Tyre.}%
\verse{Cross over to Tarshish! Wail, inhabitants of the coast!}%
\verse{Is this your exultant one,\lebnote{“this to you, exultant”} her origin from the days of long ago? Her feet brought her to dwell afar as an alien.}%
\verse{Who has planned this against Tyre, the bestower of crowns,\lebnote{Hebrew “crown”} whose merchants were princes, her traders the honored ones of the earth?}%
\verse{Adonai of hosts has planned it: to defile the pride of all glory, to humble all the honored ones of the earth.}%
\verse{Cross over your own land like the Nile, daughter of Tarshish; there is no longer a harbor.\lebnote{Hebrew “waistband” is emended to “harbor” by transposing two consonants}}%
\verse{He has stretched his hand out over the sea; he has made kingdoms shake. Adonai has commanded concerning Canaan to destroy her fortresses.}%
\verse{And he said, “You will not continue\lebnote{“do again longer”} to exult, crushed one, virgin\lebnote{“virgin of the”} daughter of Sidon. Arise, cross over to Cyprus! There will be no rest for you even there.”}%
\verse{Look at the land of the Chaldeans! This people no longer exists. Assyria destined\lebnote{“this people was not Assyria destined,” which could be translated “It was this people! It was not Assyria. They destined”} it for wild animals. They erected its siege towers; they demolished its citadel fortresses. It made her like a ruin.}%
\verse{Wail, ships of Tarshish! For your fortress is destroyed.}%
\verse{And this will happen on that day: And Tyre will be forgotten seventy years,\lnBWD{} like the days of one king. At the end of seventy years,\lnBWD{} it will be for Tyre like the song of the prostitute:}%
\verse{“Take a harp, go around the city, forgotten prostitute! Do it well, playing a stringed instrument! Make numerous songs,\lebnote{Hebrew “song”} that you may be remembered.”}%
\verse{And this shall happen: at the end of seventy years,\lnBWD{} Adonai will visit Tyre, and she will return to her harlot’s wages, and she will commit fornication with all the kingdoms of the world on the face of the earth.}%
\verse{And this shall happen: her merchandise and her harlot’s wages will be set apart for Adonai; it will not be stored up, and it will not be hoarded, but her merchandise will be for those who live before the presence\lebnote{“face”} of Adonai, for eating to satiation and for fine clothing.}%
\end{biblechapter}%
\begin{biblechapter}% Isaiah 24
\verseWithHeading{The Judgment on the Earth}{Look! Adonai is about to lay the earth waste and is about to devastate it, and he will twist her surface, and he will scatter her inhabitants.}%
\verse{And it shall be as with the people, so with the priest; as with the slave, so with his master; as with the female slave, so with her mistress; as with the buyer, so with the seller; as with the lender, so with the borrower; as with the creditor, so with the one to whom he lends.}%
\verse{The earth shall be utterly laid waste, and it shall be utterly plundered, for Adonai has spoken this word.}%
\verse{The earth dries up, it withers; the world languishes, it withers. The elevated of the people of the earth languish,}%
\verse{and the earth is defiled beneath its inhabitants. For they have transgressed laws; they have passed by statutes;\lebnote{Hebrew “statute”} they have broken the everlasting covenant.}%
\verse{Therefore a curse devours the earth, and the inhabitants in it suffer for their guilt. Therefore the inhabitants of the earth burn, and few men are left.\lebnote{“he is left a man of smallness”}}%
\verse{The new wine dries up; the vine languishes. All the merry of heart sigh;}%
\verse{the joy of the tambourine has stopped. The noise of the jubilant has ceased; the joy of the lyre has stopped.}%
\verse{They do not drink wine with song; strong drink is bitter to those who drink it.}%
\verse{The city of emptiness is broken; every house is shut so that no one can enter;\lebnote{“from entering”}}%
\verse{there is an outcry over the wine in the streets. All joy turns into darkness;\lebnote{“evening”} the joy of the earth disappears.}%
\verse{Desolation is left in the city; the gate is crushed into a state of ruin.}%
\verse{For it shall be like this in the midst of the earth, among the nations, like the beating of an olive tree, like gleanings when a grape harvest is at an end.}%
\verse{They lift up their voices;\lebnote{Hebrew “voice”} they sing for joy; they shout out from the west over the majesty of Adonai.}%
\verse{Therefore glorify Adonai in the east, the name of Adonai the God of Israel in the coastlands of the sea.}%
\verse{We hear songs from the edge\lebnote{“wing”} of the earth: “Glory to the righteous one!” But I say, “Ruin to me! Ruin to me! Woe to me! The treacherous ones deal treacherously, and the treacherous ones deal treacherously with treachery!”}%
\verse{Terror and the pit and the snare are upon you, inhabitants\lebnote{Hebrew “inhabitant”} of the earth!}%
\verse{And this shall happen: The one who flees from the sound of the terror shall fall into the pit, and the one who goes up from inside the pit\lebnote{“the middle of the pit”} shall be caught in the snare, for the windows from heaven are opened, and the foundations of the earth tremble.}%
\verse{The earth is utterly broken; the earth is torn asunder; the earth is shaken violently.}%
\verse{The earth staggers to and fro like the drunkard, and it sways like a hut, and its transgression is heavy upon it, and it falls and does not rise again.}%
\verse{And this shall happen on that day: Adonai will punish the host of heaven in heaven, and the kings of the earth on the earth.}%
\verse{And they will be gathered in a gathering, like a prisoner in\lnBWE{} a pit. And they will be shut in\lnBWE{} a prison and be punished after\lebnote{“from”} many days.}%
\verse{And the full moon will be ashamed and the sun will be ashamed, for Adonai of hosts will rule on Mount Zion\lebnote{“the mountain of Zion”} and in Jerusalem, and before his elders in glory.}%
\end{biblechapter}%
\begin{biblechapter}% Isaiah 25
\verseWithHeading{Praise for Salvation}{Adonai, you are my God; I will exalt you. I will praise your name, for you have done wonderful things,\lebnote{Hebrew “thing”} plans\lebnote{Hebrew “plan”} of old,\lebnote{“from far away “} in faithfulness, trustworthiness.}%
\verse{For you have made\lebnote{“placed”} the city a heap, the fortified city a ruin, the palace of foreigners is no longer\lebnote{“from”} a city; it will never\lebnote{“to eternity not”} be rebuilt.}%
\verse{Therefore strong peoples will glorify you; a city of ruthless nations, they will fear you.}%
\verse{For you have been a refuge to the poor, a refuge to the needy in his distress, a shelter from the rainstorm, a shade from the heat. For the breath of the ruthless was like a rainstorm against a wall,\lebnote{The same consonants with different vowels can be translated “winter rainstorm”}}%
\verse{the noise of foreigners like heat in a dry land. You subdued the heat with the shade of a cloud; the song of the ruthless was silenced.}%
\verse{And on this mountain Adonai of hosts will make for all peoples a rich feast,\lebnote{“feast of fat”} a feast of aged wines, fat filled with marrow, filtered aged wine.}%
\verse{And on this mountain he will destroy\lnBWF{} the face of the shroud, the shroud over all peoples, and the woven covering over all nations.}%
\verse{He will destroy\lnBWF{} death forever, and the Lord Adonai will wipe off the tears from all faces, and he will remove the disgrace of his people from all the earth, for Adonai has spoken.}%
\verse{And one will say, on that day, “Look! This is our God! We have waited for him and he saved us! This is Adonai; we waited for him! Let us be glad, and let us rejoice in his salvation.”}%
\verse{For the hand of Adonai will rest on this mountain, and Moab shall be trampled down under him as a heap of straw is trampled down\lebnote{“the trampling down of a heap of straw”} in waters of\lebnote{These words are in the written Hebrew text, but not in the reading tradition (\textit{Qere})} a dung heap.}%
\verse{And it\lebnote{Moab} will spread out its hands in the midst of it, just as the swimmer spreads out to swim, and its pride will be brought low with the movement\lebnote{Meaning uncertain} of its hands.}%
\verse{And he will throw down the fortification of the high point of your walls; he will bring it low; he will send it\lebnote{“cause \textit{it} to touch”} to the ground, to the dust.}%
\end{biblechapter}%
\begin{biblechapter}% Isaiah 26
\verseWithHeading{Judah’s Song of Praise}{On that day, this song will be sung in the land of Judah: “We have a strong city;\lebnote{“a city of strength for us”} he sets up victory like walls and ramparts!\lebnote{Hebrew “rampart”}}%
\verse{Open the gates so that\lebnote{Or “and”} the righteous nation who keeps faithfulness may enter!}%
\verse{You will protect a firm inclination in peace, in peace because he trusts in you.}%
\verse{Trust in Adonai forever, for in Yah, Adonai you have an everlasting rock.}%
\verse{For he has thrown down the inhabitants of the height, he lays low the lofty city. He lays it low to the ground; he throws her to the dust.}%
\verse{The foot tramples it, the feet of the poor, the steps of the needy.”}%
\verseWithHeading{Adonai’s People Vindicated}{The way of the righteous is a straight path; you clear the level path of the righteous.\lebnote{“\textit{the} level path of the righteous you make level”; the meaning of this phrase is uncertain}}%
\verse{Surely we wait for you in the path of your judgments, Adonai, for your name and renown are the desire of the soul.}%
\verse{I desire you with all my soul in the night; also I seek you with my spirit within me, for when your judgments are upon\lebnote{“to”} the earth, the inhabitants of the world learn righteousness.}%
\verse{Though the wicked person is shown compassion, he does not learn righteousness; he acts unjustly in the land of uprightness, and he does not see the majesty of Adonai.}%
\verse{Adonai, though your hand reaches high, they do not see it. Let them see, and let them be ashamed of the zeal of people. Indeed, let the fire of your enemies consume them.}%
\verse{Adonai, you will establish peace for us, for you have done even all of our works for us.}%
\verse{Adonai, our God, lords besides you ruled over us, but we praise your name alone.\lebnote{“alone in you we praise name you”}}%
\verse{The dead do not live; dead spirits do not rise because you have punished and destroyed them, and you have destroyed all memory of them.}%
\verse{You have added to the nation, Adonai. You have added to the nation; you are honored. You have extended all the ends of the land.}%
\verse{Adonai, in distress they have visited you; they poured out an incantation;\lebnote{“whispering”} your discipline was on them.}%
\verse{Like a pregnant woman on the point of giving birth, she writhes; she cries in her labor pains. So we were because of your presence, Adonai.}%
\verse{We became pregnant, we writhed; we gave birth to wind. We cannot bring about deliverance on the earth, and no inhabitants of the world are born.}%
\verse{Your dead shall live; their corpses\lebnote{“my corpse;” some manuscripts propose an emendation to the suffix to change it to third person masculine plural (“their corpses”), which makes better sense} shall rise. Wake up and sing for joy, dwellers of the dust, for your dew is celestial dew,\lebnote{“dew of light”} and the earth will give birth to dead spirits.\lebnote{Or “you make the land of Rephaim fall”}}%
\verse{Go, my people, enter into your chambers and shut your doors\lebnote{The reading tradition (\textit{Qere}) is singular} behind you; hide for a very little\lebnote{“a little a little”} while, until the wrath has passed over.}%
\verse{For look! Adonai is about to come out from his place to punish the iniquity of the inhabitants\lebnote{Hebrew “inhabitant”} of the earth against him, and the earth will disclose her blood and will no longer cover her slain.}%
\end{biblechapter}%
\begin{biblechapter}% Isaiah 27
\verseWithHeading{Israel Rescued}{On that day, Adonai will punish with his cruel, great and strong sword Leviathan, the fleeing serpent, and Leviathan, the twisting serpent, and he will kill the sea monster that is in the sea.}%
\verse{On that day: “A vineyard of beauty! Sing in praise of it!}%
\verse{I, Adonai, am her keeper; I water it again and again.\lebnote{“by moments”} Lest one afflict harm on it, I guard it night and day;}%
\verse{I have no wrath.\lebnote{“There is no wrath for me”} Whatever gives me thorns and briers, I will step forth against in battle. I will set it on fire altogether.}%
\verse{Or let it grasp at my protection; let it make peace with me; peace let it make with me.”}%
\verse{Days are coming,\lebnote{“The coming ones”} let Jacob take root; Israel will blossom and send out shoots, and they will fill the face of the world with fruit.}%
\verse{Does he strike him as he struck down those who struck him?\lebnote{“like the striking of one striking him”} Or is he killed as those who killed him were killed?\lebnote{“like the killing of his killed ones”}}%
\verse{By expelling\lebnote{Meaning uncertain, derived from the following parallel expression} her, by her sending away, you argue with her. He removed them with his strong wind, in the day of the east wind.}%
\verse{Therefore by this he will make atonement for the guilt of Jacob, and this will be all of the fruit of the removal of his sin: when he makes\lebnote{“in his setting”} all the stones of the altar like crushed stones of chalk, no poles of Asherah worship or incense altars will stand.}%
\verse{For the fortified city is solitary, a settlement deserted and forsaken, like the wilderness; the calf grazes there, lies down there and destroys its branches.}%
\verse{When its branches are dry,\lebnote{“In her branch being dry”} they are broken; women are coming and setting light to it. For it is not a people of understanding; therefore his maker will not have compassion on him, and his creator will not show him favor.}%
\verse{And this shall happen: on that day, Adonai will thresh from the floodwaters of the Euphrates to the wadi\lebnote{A seasonal stream that is often dry} of Egypt, and as for you, you will be gathered one by one, sons of Israel.}%
\verse{And this shall happen: on that day, a great trumpet will be blown, and those who were lost in the land of Assyria will come, and those who were scattered in the land of Egypt, and they will bow down to Adonai on the holy mountain\lebnote{“mountain of holiness”} at Jerusalem.}%
\end{biblechapter}%
\begin{biblechapter}% Isaiah 28
\verseWithHeading{Judgment against the Leaders of Ephraim}{Ah! The garland of the pride of the drunkards of Ephraim and the withering flower of the glory of his beauty, which is at the head of the rich valley,\lnBWG{} ones overcome\lebnote{“ones struck”} with wine!}%
\verse{Look! The Lord has a mighty and strong one, like a rainstorm of hail, a wind storm of destruction, like a rainstorm of mighty overflowing waters, he will put them to the earth with his hand.}%
\verse{The garland of the pride of the drunkards of Ephraim will be trampled\lebnote{The Hebrew is plural} by feet,}%
\verse{and the withering flower of the glory of its beauty which is at the head of the rich valley\lnBWG{} will be like its early fig before summer, which the one who sees it swallows\lebnote{“Which, when the one who sees it sees it, he swallows it”} while it is still in his hand.}%
\verse{In that day, Adonai of hosts will become a garland of glory and a diadem of beauty to the remnant of his people,}%
\verse{and a spirit of justice to the one who sits over judgment, and strength to those who turn back the battle at the gate.}%
\verse{And these also stagger because of wine and stagger because of strong drink; priest and prophet stagger because of strong drink; they are confused\lebnote{Or “swallowed”} because of wine. They stagger because of strong drink; they err in vision. They stagger in the rendering of a decision,}%
\verse{for all the tables are full of disgusting vomit, with no place left.\lebnote{“without a place”}}%
\verse{To whom will he teach knowledge, and to whom will he explain the message? Those who are weaned from milk, those taken from the breast?}%
\verse{For it is blah-blah upon blah-blah, blah-blah upon blah-blah, gah-gah upon gah-gah, gah-gah upon gah-gah,\lnBWH{} a little here, a little there.}%
\verse{For he will speak with stammering\lebnote{“stammer of lip”} and another tongue to this people,}%
\verse{to whom he has said, “This is rest; give rest to the weary; and this is repose”; yet\lebnote{Or “and”} they were not willing to hear.}%
\verse{And to them the word of Adonai will be blah-blah upon blah-blah blah-blah upon blah-blah gah-gah upon gah-gah gah-gah upon gah-gah,\lnBWH{} a little here, a little there, so that they may go and stumble backward and be broken and ensnared and captured.}%
\verseWithHeading{The Cornerstone of Zion}{Therefore hear the word of Adonai, men of mockery, rulers of these people in Jerusalem:}%
\verse{Because you have said, “We have made\lebnote{“cut”} a covenant with death, and we have made an agreement with Sheol. The overwhelming flood, when it passes through, will not come to us, for we have made lies\lnBWI{} our refuge, and we have hidden ourselves in falsehood.”}%
\verse{Therefore the Lord Adonai says this: “Look! I am laying a stone in Zion, a tested\lebnote{“testing”} stone, a precious cornerstone, a founded foundation: ‘The one who trusts will not panic.’\lebnote{“hurry”}}%
\verse{And I will make justice the measuring line, and righteousness the plumb line; and hail will sweep away the refuge of lies,\lnBWI{} and waters will flood over the hiding place.}%
\verse{And your covenant with death will be annulled, and your agreement with Sheol will not stand; you will become a trampling place for the overwhelming flood when it passes through.}%
\verse{As often as it passes through,\lebnote{“From enough his passing through”} it will take you, for morning by morning\lebnote{“in the morning in the morning”} it will pass through, in the day and in the night, and understanding the message will be only terror.”}%
\verse{For the bed is too short to stretch out on,\lebnote{“from stretching oneself”} and the covering is too narrow when wrapping oneself.}%
\verse{For Adonai will rise up as at Mount Perazim; he will rave as in the valley at Gibeon to do his deed — his deed is strange — and to work his work — his work is alien!}%
\verse{And now you must not scoff, or your bonds will be strong, for I have heard from the Lord, Adonai of hosts: complete destruction decreed upon all the land.}%
\verseWithHeading{A Parable of Adonai’s Work}{Listen, and hear my voice! Listen attentively, and hear my word!}%
\verse{Is it all day that the plowman plows, opens to sow seed, harrows his ground?}%
\verse{When he has leveled its surface,\lebnote{“face”} does he not scatter dill, and sow cumin seed, and plant\lebnote{“place”} wheat in planted rows,\lebnote{Hebrew “row”} and barley in an appointed place, and spelt grain as its border?}%
\verse{And his God instructs him about the prescription;\lebnote{That is, “the proper way”} he teaches him.}%
\verse{For dill is not threshed with a threshing sledge, nor is a wheel of a utility cart rolled\lebnote{“turned”} over cumin, but dill is beaten out with a stick, and cumin with a rod.}%
\verse{Grain is crushed fine, but certainly one does not thresh it forever; and one drives the wheel of his cart, but his horses do not crush it.}%
\verse{This also comes forth from Adonai of hosts. He is wonderful in advice; he makes great wisdom.}%
\end{biblechapter}%
\begin{biblechapter}% Isaiah 29
\verseWithHeading{Woe to Jerusalem}{Ah! Ariel, Ariel, the city where David encamped! Add year to year, let festivals recur.}%
\verse{Yet\lnBWJ{} I will inflict Ariel, and there shall be mourning and lamentation, and it shall be to me like an altar hearth.\lebnote{Hebrew “Ariel,” which might mean “altar hearth”}}%
\verse{And I will encamp in a circle\lebnote{The Hebrew text literally reads “like the ball”; the LXX supports an emendation to “David”} against you, and I will lay siege to you with towers\lebnote{Hebrew “tower”} and I will raise up siegeworks against you.}%
\verse{Then\lnBWJ{} you shall be low; you shall speak from the earth, and your words\lebnote{Hebrew “word”} will be low, from dust. And your voice will be from the earth, like a ghost, and your word will whisper from the dust.}%
\verse{But\lnBWJ{} the multitude of your strangers shall be like fine dust, and the multitude of tyrants like chaff that passes by. And it will happen in an instant, suddenly.}%
\verse{You will be punished by Adonai of hosts with thunder and earthquake and great sound, storm wind and tempest and the flame of a devouring fire.}%
\verse{And the multitude of all the nations who fight against Ariel, all those who fight against her and her stronghold, and those who inflict her shall be like a dream, a vision of the night.}%
\verse{And it shall be as when the hungry person dreams — look, he is eating! And he wakes up and his inner self is empty. Or\lnBWJ{} as when the thirsty person dreams — look, he is drinking! And he wakes up and look, he is faint, and his inner self is longing for water. So shall be the multitude of all the nations who fight against Mount Zion.}%
\verse{Be astonished and be amazed! Blind yourselves and be blinded! They are drunk but\lnBWK{} not from wine; they stagger but\lnBWK{} not from strong drink.}%
\verse{For Adonai has poured out upon you a spirit of deep sleep, and he has shut your eyes, the prophets, and he has covered your heads, the seers.}%
\verse{And the vision of all this has become for you like the words of a sealed document. When they give it to one who knows the\lebnote{The reading tradition (\textit{Qere}) has “a” instead of “the”} document, saying, “Read\lnBWL{} this now!” He says, “I am not able, for it is sealed.”}%
\verse{And if the document is given to one who does not know how to read,\lnBWM{} saying, “Read\lnBWL{} this now!” he says, “I do not know how to read.”\lnBWM{}}%
\verse{And the Lord said, “Because this people draw near with its mouth, and with its lips it honors me, and its heart is far from me, and their fear of me is a commandment of men that has been taught,}%
\verse{therefore look, I am again doing something spectacular and a spectacle with this spectacular people. And the wisdom of its wise men shall perish, and the discernment of its discerning ones shall keep itself hidden.”}%
\verse{Ah! Those who make a plan deep, to hide it from Adonai, and their deeds are\lebnote{The Hebrew is singular} in a dark place. And they say, “Who sees us? And who knows us?”}%
\verse{Your perversity! As if a potter\lebnote{“of the one who creates”} shall be regarded as the clay! That the product of its maker says, “He did not make me,” and the thing made into shape says of its potter,\lebnote{“one who creates”} “He has no understanding.”}%
\verseWithHeading{Blessing after Punishment}{In a very little while\lebnote{“Not still a trifling \textit{of time}”} shall not Lebanon change into a fruitful land, and the fruitful land be regarded as a forest?}%
\verse{And on that day, the deaf shall hear the words of a scroll, and the eyes of the blind shall see out of gloom and darkness.}%
\verse{And the meek have joy after joy\lebnote{“shall add joy”} in Adonai, and the needy of the people shall rejoice in the holy one of Israel.}%
\verse{For the tyrant shall be no more, and the scoffer shall come to an end. And all those lying in wait for evil shall be cut off;}%
\verse{those who mislead a person into sin with a word and set a trap for the arbitrator\lebnote{“one who argues”} in the gate and guide away the righteous by emptiness.}%
\verse{Therefore Adonai, who redeemed Abraham, says this to the house of Jacob: “Jacob will no longer be ashamed, and his face will no longer grow pale.}%
\verse{For when he sees his children, the work of my hands, in his midst, they will treat my name as holy, and they will treat the holy one of Jacob as holy, and they will stand in awe of the God of Israel.}%
\verse{And those who err in spirit will acquire\lebnote{“know”} understanding, and those who grumble will learn instruction.}%
\end{biblechapter}%
\begin{biblechapter}% Isaiah 30
\verseWithHeading{Warning against Alliance with Egypt}{“Oh rebellious children!” declares\lebnote{“declaration of”} Adonai, “to make a plan, but\lnBWN{} not from me, and pour out a libation, but\lnBWN{} not from my Spirit, so as to add\lebnote{“for the sake of adding”} sin to sin.}%
\verse{Who go to go down to Egypt, but\lnBWN{} they do not ask of my mouth, to take refuge in the protection of Pharaoh and to take refuge in the shadow of Egypt.}%
\verse{And the protection of Pharaoh shall be shame to you, and the refuge in the shadow of Egypt, humiliation.}%
\verse{For his officials are at Zoan, and his envoys reach to Hanes.}%
\verse{Everyone will start to stink because of a people that cannot profit them, not for help and not for profiting, but for shame and also for disgrace.”}%
\verseWithHeading{Oracle regarding the Negev}{An oracle of the animals of the Negev: Through a land of trouble and distress, of lioness and lion, among\lebnote{“from”} them are snake and flying serpent; they carry their wealth on the backs\lebnote{“shoulder”} of male donkeys and their treasures on the humps\lebnote{Hebrew “hump”} of camels, to a people that cannot profit them.}%
\verse{For\lnBWO{} Egyptians\lebnote{“Egypt”} help with vanity and emptiness, therefore I have called this one “Rahab, they are sitting.”}%
\verse{Now go, write it on a tablet with them, and inscribe it on a scroll, that\lnBWN{} it may be for the time to come,\lebnote{“for last day”} forever, forever.}%
\verse{For it is a people of rebellion, deceitful children, children who are not willing to hear the instruction of Adonai,}%
\verse{who say to those who do see, “You must not see!” and to the seers, “You must not see truth for us; speak smooth things to us, see illusions,}%
\verse{turn aside from the way, turn aside from the path, put an end to the holy one of Israel from our face.”}%
\verse{Therefore the holy one of Israel says this: “Because you are rejecting this word and you trust in oppression and cunning\lebnote{“going wrong”} and you rely on it,}%
\verse{therefore this iniquity shall come for you like a breach about to fall, bulging out on a high wall that breaks\lebnote{“whose breaking comes”} suddenly, in an instant.}%
\verse{And he breaks it like a vessel of a potter\lebnote{“someone who forms”} breaks, that is crushed; he has no compassion, and no potsherd is found among its fragments\lebnote{Hebrew “fragment”} to take fire\lebnote{“for the taking away of fire”} from the hearth, or to skim\lebnote{“for the skimming off of”} water from the cistern.”}%
\verse{For the Lord Adonai, the holy one of Israel, said this: “In returning and rest you shall be saved; your strength shall be in quietness and in trust.” But\lnBWO{} you were not willing,}%
\verse{and you said, “No! For we will flee on horses!”\lnBWP{} Therefore you shall flee! And, “We will ride on swift horses!”\lnBWP{} Therefore your pursuers shall be swift!}%
\verse{One thousand because of\lebnote{“from \textit{the} face of”} a threat of one, because\lebnote{“from \textit{the} face”} of a threat of five you shall flee, until you are left like a flagstaff on top\lebnote{“head”} of a mountain, and like a signal on a hill.}%
\verseWithHeading{Adonai Will Show Mercy}{Therefore Adonai waits to be gracious to you, and therefore he will rise up to show you mercy, for Adonai is a God of justice; blessed are all those who wait for him.}%
\verse{For people will live in Zion; in Jerusalem, you will surely not weep. Surely he will be gracious to you; when he hears the sound of your cry, he will answer you.}%
\verse{And the Lord will give you the bread of distress and the water of oppression, but\lnBWN{} your teachers will not hide themselves\lebnote{Hebrew “himself”} any longer. And your eyes shall see\lebnote{“be seeing”} your teachers.}%
\verse{And your ears shall hear a word from behind you, saying, “this is the way; walk in it,” when you go to your right and when you go to your left.}%
\verse{And you will defile the plating of your silver idols and the covering of your gold image. You will scatter them like contaminated things;\lebnote{Hebrew “contaminated thing”} you will say to it, “Filth!”\lebnote{Or “Get out!”}}%
\verse{And he will give rain for your seed with which you sow the ground, and grain, the produce of the ground, and it will be rich and fertile.\lebnote{“fat”} On that day, your cattle will graze in broad pastures;\lebnote{Hebrew “pasture”}}%
\verse{and the oxen and the donkeys that till\lebnote{Or “tilling”} the ground will eat fodder, sorrel that has been winnowed with shovel and pitchfork.}%
\verse{And there will be streams on every high mountain and elevated hill, watercourses of water, on a day of great slaughter, when towers fall.}%
\verse{And the light of the full moon will be like the light of the sun, and the light of the sun will be sevenfold, like the light of seven days, on the day when Adonai binds up the breakage of his people, and he heals the wound of his blow.}%
\verseWithHeading{Judgment against Assyria}{Look! The name of Adonai comes from afar, burning with his anger and heaviness of cloud. His lips are full of indignation, and his tongue is like a devouring fire.}%
\verse{And his breath is like an overflowing river; it reaches up to the neck to shake the nations with the sieve of worthlessness; and a bridle that leads astray is on the jawbones of the peoples.}%
\verse{You shall have a song\lebnote{“There shall be a song for you”} as in the night when a holy festival is kept, and a gladness of heart like one who goes with the flute, to go to the mountain of Adonai, to the rock of Israel.}%
\verse{And Adonai will cause the majesty of his voice to be heard, and he will cause the descent of his arm to be seen, in furious anger and a flame of devouring fire, with a cloudburst and a rainstorm and stones\lebnote{Hebrew “stone”} of hail.}%
\verse{Indeed, Assyria will be shattered by the voice of Adonai; he strikes with the rod.}%
\verse{And every stroke of the staff of foundation\lebnote{Some translations emend this to “discipline”} that Adonai lays will be on it with timbrels and lyres, and he will fight against it\lebnote{The reading tradition (\textit{Qere}) has “them”} with battles of brandishing.}%
\verse{For Topheth has been prepared from yesterday; indeed, it is made ready for the king. He makes its pile of wood deep and wide; he makes fire and wood abundant.\lebnote{“numerous”} The breath of Adonai burns in it like a stream of sulfur.}%
\end{biblechapter}%
\begin{biblechapter}% Isaiah 31
\verseWithHeading{The Egyptians are No Help}{Ah! Those who go down to Egypt for help! They rely on horses and trust in chariots because they are many, and in horsemen because they are very numerous, and they do not look to the holy one of Israel, and they do not consult Adonai.}%
\verse{And indeed, he is wise, and he brings disaster, and he does not remove his words, and he will rise against the house of evildoers and against the help of workers of iniquity.}%
\verse{And the Egyptians are human and not God, and their horses are flesh and not spirit. And Adonai stretches out his hand, and the helper will stumble, and the one being helped will fall, and together all of them will come to an end.\lebnote{Or “perish”}}%
\verse{For Adonai said this to me: “As which a lion growls and a young lion over its prey when a full group\lebnote{“fullness”} of shepherds is called against him, it is not terrified by their voice, and to their noise it does not respond, so Adonai of hosts will come down to fight upon Mount Zion and upon its hill.}%
\verse{Like birds flying overhead, so Adonai of hosts will protect Jerusalem; he will protect and deliver it; he will pass over and rescue it.}%
\verse{Turn back to the one against whom the sons of Israel have made deep rebellion.}%
\verse{For on that day, each one will reject his idols of silver and his idols of gold which your hands have made in sin for you.}%
\verse{And Assyria shall fall by a sword not of a man, and a sword not of a human shall devour him; and he shall flee from the sword, and his young men shall be put to forced labor.}%
\verse{And his rock will pass over because of terror, and his officers will be terrified because of the flag,” declares\lebnote{“declaration of”} Adonai, who has a fire in Zion and has a furnace in Jerusalem.}%
\end{biblechapter}%
\begin{biblechapter}% Isaiah 32
\verseWithHeading{The Kingdom of Righteousness}{See, a king will rule according to righteousness, and princes will rule according to justice.}%
\verse{And each one will be like a hiding place from the wind and a covering from the rainstorm, like streams of water in a dry region, like the shade of a large rock in a weary land.}%
\verse{And the eyes of those who see will not gaze,\lebnote{The same consonants with different vowels would be translated “be blinded,” which fits the context better} and the ears of those who hear will listen.}%
\verse{And the minds\lnBWQ{} of the rash will understand knowledge,\lebnote{“to know”} and the tongues\lebnote{Hebrew “tongue”} of stammerers will hasten to speak clearly.}%
\verse{A fool will no longer be called noble, and a scoundrel will not be said to be eminent.}%
\verse{For a fool speaks folly, and his mind\lnBWQ{} does iniquity: to behave wickedly,\lebnote{“do ungodliness”} and to speak error concerning Adonai, to leave the throat of the hungry empty, and he deprives the thirsty of drink.}%
\verse{And a scoundrel, his weapons are evil; he plans evil devices to ruin the poor with words of deception even\lebnote{Or “and”} when the speech of the needy is right.}%
\verse{But\lnBWR{} the nobleman plans noble things, and he stands upon noble things.}%
\verseWithHeading{Against the Carefree Women}{Women who are at ease, rise up; hear my voice! Carefree daughters, listen to my word!}%
\verse{In a year\lebnote{“\textit{In} days upon a year”} you will tremble, carefree ones, for the vintage will come to an end; the harvest will not come.}%
\verse{Tremble, you who are at ease; tremble, carefree ones; strip, and strip yourself, and gird yourself on your loins,\lebnote{Although the carefree and at ease addressees in this verse are feminine plural (“you women”), the commands are masculine singular in form}}%
\verse{mourning over breasts, over fields of delight, over the fruitful vine,}%
\verse{over the soil of my people. It goes up in thorns\lebnote{Hebrew “thorn”} and briers,\lebnote{Hebrew “brier”} indeed over all of the houses of joy in the jubilant city.}%
\verse{For the palace will be forsaken, the crowded city\lebnote{“crowd of \textit{the} city”} deserted; the hill and the watchtower will become\lebnote{The Hebrew text has “for the benefit of” following “become”} a cave forever, the joy of wild asses, a pasture for\lebnote{“of”} flocks.}%
\verse{Until a spirit is poured out on us from on high, and the wilderness becomes a fruitful field, and the fruitful field is reckoned as the forest.}%
\verse{Then\lnBWR{} justice will dwell in the wilderness, and righteousness will live in the fruitful field.}%
\verse{And the work of righteousness will be peace, and the work of righteousness, quietness and security forever.}%
\verse{And my people will dwell in a settlement of peace and in a dwelling place of security and in undisturbed resting places.}%
\verse{And it hails when the forest comes down,\lebnote{“the coming down of the forest”} and the city will become low in humiliation.}%
\verse{Happy are you who sow by all waters, who let the foot of the ox and the donkey go free.}%
\end{biblechapter}%
\begin{biblechapter}% Isaiah 33
\verseWithHeading{Adonai’s Judgment and Help}{Ah, destroyer, and yourself not destroyed! And treacherous one, and no one has dealt treacherously with him!\lebnote{Many manuscripts suggest “you” rather than “him”} When you cease\lebnote{“finishing”} destroying, you will be destroyed. When you stop dealing treacherously, one will deal treacherously with you.}%
\verse{Adonai, be gracious to us, we wait for you. Be our\lebnote{Hebrew “their”} arm in the mornings, indeed our salvation in the time of trouble.}%
\verse{At the sound of tumult, peoples fled; because of your exaltation, nations scattered.}%
\verse{And your spoil is gathered, as the gathering of the locust, as a swarm of locusts storming on it.}%
\verse{Adonai is exalted, for he dwells on high; he filled Zion with justice and righteousness,}%
\verse{and he will be the security of your times, an abundance of salvation, wisdom, and knowledge. The fear of Adonai is his treasure.}%
\verse{Look! Their heroes cry out in the street; the messengers of peace weep bitterly.}%
\verse{Highways are deserted; the traveler on the road ceases. One breaks a treaty, he rejects the cities,\lebnote{A Dead Sea Scroll reads “witnesses”} he does not hold man in high regard.}%
\verse{The land mourns; it languishes. Lebanon feels abashed; it withers. Sharon is like the desert, and Bashan and Carmel are losing their leaves.\lebnote{“shaking off”}}%
\verse{“Now I will arise,” says Adonai. “Now I will lift myself up proudly; now I will raise myself.}%
\verse{You conceive dry grass, you bring forth stubble; your breath is a fire; it will consume you.}%
\verse{And the peoples will be burning to lime — they are burned like thorns that have been cut down in the fire.}%
\verse{You who are far away, hear what I have done; and you who are near, know my might!”}%
\verse{Sinners are afraid in Zion; trembling has seized the godless: “Who of us can live\lnBWS{} with devouring fire? Who of us can live\lnBWS{} with everlasting consuming hearths?”}%
\verse{He who walks in righteousness and speaks uprightness, who rejects the gain of extortion, who refuses\lebnote{“shakes his hand from the taking of”} a bribe, who stops up his ears\lebnote{Hebrew “ear”} from hearing bloodshed\lebnote{“blood”} and shuts his eyes from seeing evil.}%
\verse{That one will live on the heights; the fortresses of rocks will be his refuge. His food will be given; his waters\lebnote{Hebrew “water”} will endure.}%
\verse{Your eyes will see the king in his beauty; they will see a distant land.\lebnote{“land of distance”}}%
\verse{Your mind\lebnote{“heart”} will meditate on the terror: “Where is the one who counted? Where is the one who weighed out? Where is the one who counted the towers?”}%
\verse{You will not see the insolent people, the people whose language is too obscure to understand,\lebnote{“obscure of lip than to hear”} whose stammering of tongue cannot be understood.\lebnote{“there is no understanding”}}%
\verse{Look on Zion, the city of our appointed festivals! Your eyes will see Jerusalem, an undisturbed settlement, a tent that is not moved.\lebnote{“not he is loaded”} No one will ever pull out its tent pegs, and none of its ropes will be torn in two.}%
\verse{Rather, there Adonai will be mighty for us, a place of rivers and broad streams,\lebnote{“streams broad of hands”} a galley ship with\lebnote{Or “of”} oars\lebnote{Hebrew “oar”} cannot go in it, and a mighty ship cannot pass through it.}%
\verse{For Adonai is our judge; Adonai is our lawgiver. Adonai is our king; he is the one who will save us.}%
\verse{Your riggings hang slack; they do not hold the base of their mast firm, they do not spread out the sail. Then the prey of spoil in abundance will be divided; the lame will take plunder.}%
\verse{And no inhabitant will say, “I am sick”; the people who live in it, their iniquity will be taken away.}%
\end{biblechapter}%
\begin{biblechapter}% Isaiah 34
\verseWithHeading{Judgment on the Nations}{Come near, nations, to hear; and peoples, listen attentively! Let the earth hear, and that which fills it; the world and all its offspring.}%
\verse{For the anger of Adonai is against all the nations, and his wrath is against all their armies; he has put them under a ban, he has given them up for slaughter.}%
\verse{And their slain shall be cast out; as for\lnBWT{} their corpses, their stench shall go up. And the mountains shall melt with\lnBWU{} their blood,}%
\verse{and all the host of heaven shall rot. And the skies shall roll up like a scroll, and all their host shall wither like the withering of a leaf from a vine, or\lnBWT{} like the withering from a fig tree.}%
\verse{When my sword is drenched in the heavens, look! It will descend upon Edom, and upon the people of my ban, for judgment.}%
\verse{Adonai has a sword;\lebnote{“A sword for Adonai”} it is full of blood. It is covered with\lnBWU{} fat, with\lnBWU{} the blood of lambs and goats, with\lnBWU{} the fat of the kidneys of rams, for Adonai has a sacrifice\lebnote{“a sacrifice for Adonai”} in Bozrah and a great slaughter in the land of Edom.}%
\verse{And wild oxen shall go down with them, and steers with strong bulls. And their land shall be drenched with\lnBWU{} blood, and their soil shall be fattened with\lnBWU{} fat.}%
\verse{For Adonai has a day of vengeance,\lebnote{“a day of vengeance for Adonai”} a year of retribution for the strife of Zion.}%
\verse{And its streams shall be changed to pitch and its soil to sulfur, and its land shall become like burning pitch.}%
\verse{Night and day it shall not be quenched; its smoke shall go up forever. From generation to generation it shall be in ruins; forever and ever there will be no one who passes through her.}%
\verse{But\lebnote{Or “And”} the large bird and the hedgehog shall take possession of it, and the owl and the raven shall live in it. And he shall stretch the measuring line of confusion out over it, and the plumb line of emptiness.}%
\verse{Its nobles — but no kingdom is there — shall call, and all its princes shall be nothing.}%
\verse{And thorns shall go up her citadel fortress, weeds\lebnote{Hebrew “weed”} and thistle plants\lebnote{Hebrew “thistle plant”} in her fortresses; and it shall be the settlement of jackals, green grass for the daughters of an ostrich.}%
\verse{And desert creatures shall meet with hyenas, and a goat-demon shall call to his neighbor; surely there Lilith\lebnote{Hebrew transliteration; possibly a proper name for a Mesopotamian night-demon} shall repose, and she shall find a resting place for herself.}%
\verse{There the owl shall nest and lay and hatch and care for her chicks in her shadow; surely there the birds of prey shall be gathered, each one with her mate.}%
\verse{Seek from the book of Adonai and read;\lebnote{“call”} none of these shall be missing; none shall miss her mate. For my\lebnote{The Dead Sea Scroll has “his”} mouth is the one that\lnBWV{} has commanded, and his spirit is the one that\lnBWV{} has gathered them.}%
\verse{And he is the one that\lnBWV{} has cast the lot for them, and his hand has apportioned it to them with the measuring line; they shall take possession of it forever, they shall live in it from generation to generation.\lebnote{“to generation and generation”}}%
\end{biblechapter}%
\begin{biblechapter}% Isaiah 35
\verseWithHeading{The Ransomed Return to Zion}{Wilderness and dry land shall be glad,\lebnote{The Hebrew has an extra “them,” probably from an accidental of the first letter of the next word} and desert shall rejoice and blossom like the crocus.}%
\verse{It shall blossom abundantly, and it shall rejoice indeed with rejoicing and exulting. The glory of Lebanon shall be given to it, the majesty of Carmel and Sharon. They are the ones who\lebnote{“they”} shall see the glory of Adonai, the majesty of our God.}%
\verse{Strengthen the weak hands and make the staggering knees firm.}%
\verse{Say to those who are hasty of heart, “Be strong; you must not fear! Look! your God will come with vengeance, with divine retribution.\lebnote{“the retribution of gods”} He is the one who\lebnote{“he”} will come and save you.”}%
\verse{Then the eyes of the blind shall be opened, and the ears of the deaf shall be opened.}%
\verse{Then the lame shall leap like the deer, and the tongue of the dumb shall sing for joy, for waters shall break forth in the wilderness and streams in the desert.}%
\verse{And the parched ground shall become a pool, and the thirsty ground springs of water. Her resting place is in a settlement of jackals; the grass shall become like reeds\lebnote{Hebrew “reed”} and rushes.\lebnote{Hebrew “rush”}}%
\verse{And a highway shall be there, and a way, and it shall be called the way of holiness. The unclean shall not travel through it, but\lnBWW{} it is for them, he who walks on the way; and fools shall not wander about.}%
\verse{No lion shall be there, and no ferocious wild beast shall go up it. It shall not be found there, but\lnBWW{} the redeemed shall walk there.}%
\verse{And the ransomed of Adonai shall return, and they shall come to Zion with rejoicing. And everlasting joy shall be on their head; joy and gladness shall overtake them, and sorrow and sighing shall flee.}%
\end{biblechapter}%
\begin{biblechapter}% Isaiah 36
\verseWithHeading{Sennacherib Threatens Jerusalem}{And this happened: In the fourteenth year\lebnote{“four ten year”} of King Hezekiah, Sennacherib king of Assyria went up against all the fortified cities of Judah, and he captured them.}%
\verse{And the king of Assyria sent Rabshakeh\lebnote{Rabshekah is the title of a high Assyrian official} from Lachish to Jerusalem, to King Hezekiah, with a large army, and he stood by the conduit of the upper pool on the highway of the field of the washer.}%
\verse{And Eliakim son of Hilkiah, who was in charge of the palace,\lebnote{“\textit{was} over the house”} came out to him, and Shebna the secretary, and Joah son of Asaph, the reminder.}%
\verse{And Rabshakeh said to them, “Now say to Hezekiah, ‘Thus says the great king, the king of Assyria: “What is this confidence in which you trust?}%
\verse{I said, ‘Only a word of lips! War has power and a plan!’\lebnote{The Hebrew here is awkward; literally “Plan and power for war”} Now, in whom do you trust, that you have rebelled against me?}%
\verse{Look, you trust in the staff of this broken reed, on Egypt, which if a man leans on it, goes into his hand and bores through it! Such is Pharaoh, king of Egypt, to all those who trust in him.}%
\verse{And if you say to me, ‘We trust in Adonai our God,’ was it not he whose high places and altars Hezekiah removed? And he said to Judah and to Jerusalem, ‘You shall bow down in the presence\lebnote{“face”} of this altar.’”}%
\verse{And now please make a wager with my master the king of Assyria, and I will give you two thousand horses, that is, if you are able put\lebnote{“give”} riders for yourself on them!}%
\verse{But how can you drive back one governor among the least of my master’s servants,\lebnote{“the face of the governor of the one of the insignificant servants of my master”} when\lebnote{Or “and”} you trust in Egypt for chariots\lebnote{Hebrew “chariot”} and horsemen?}%
\verse{And now was it without Adonai that I have come up against this land to destroy it? Adonai said to me, “Go up against this land and destroy it!”’”}%
\verse{And Eliakim, Shebna, and Joah said to Rabshakeh, “Please speak to your servants in Aramaic, for we can understand\lebnote{Or “hear”} it, and you must not speak to us in Judean in the hearing\lebnote{“ear”} of the people who are on the wall.”}%
\verse{But\lnBWX{} Rabshakeh said, “Has my master sent me to speak these words to your masters and you? Was it not for the people who sit on the wall, to eat their dung and drink their urine\lebnote{So Masoretic Hebrew text (\textit{Kethib}); the reading tradition (\textit{Qere}) has “feet-water”} with you?”}%
\verse{Then\lnBWX{} Rabshakeh stood and called in a great voice in Judean and said, “Hear the words of the great king, the king of Assyria.}%
\verse{Thus says the king: ‘Do not let Hezekiah deceive you, for he will not be able to deliver you!}%
\verse{And do not let Hezekiah make you rely on Adonai, saying, “Surely Adonai will deliver us; this city will not be given into the hand of the king of Assyria!”}%
\verse{You must not listen to Hezekiah, for thus says the king of Assyria: “Make a blessing\lebnote{That is, a gesture of surrender} with me, and come out to me, and each one will eat from his vine and from his fig tree and drink water from\lebnote{Or “of”} his cistern,}%
\verse{until I come\lebnote{“my coming”} and take you to a land like your land, a land of grain and new wine, a land of bread and vineyards,}%
\verse{lest Hezekiah mislead you, saying, ‘Adonai will save us!’ Did the gods of the nations each save his land from the hand of the king of Assyria?}%
\verse{Where are the gods of Hamath and Arpad? Where are the gods of Sepharvaim? Indeed, have they delivered Samaria from my hand?}%
\verse{Who are there among all the gods of these countries who have saved their land from my hand, that Adonai should save Jerusalem from my hand?”’”}%
\verse{But\lnBWX{} they were silent and did not answer him a word, for the command of the king was, “You must not answer him.”}%
\verse{Then\lnBWX{} Eliakim son of Hilkiah, who was over the palace,\lebnote{“house”} Shebna the secretary, and Joah son of Asaph, the reminder, came to Hezekiah with torn garments and told him the words of Rabshakeh.}%
\end{biblechapter}%
\begin{biblechapter}% Isaiah 37
\verseWithHeading{Hezekiah Consults Isaiah}{And this happened: When King Hezekiah heard, he tore his garments, covered himself with sackcloth, and entered the temple\lnBWY{} of Adonai.}%
\verse{And he sent Eliakim, who was in charge of\lebnote{“over”} the palace,\lnBWY{} and Shebna the secretary, and the elders of the priests covered\lebnote{“covering themselves”} with sackcloth to Isaiah son of Amoz, the prophet.}%
\verse{And they said to him, “Thus says Hezekiah: ‘This day is a day of distress, rebuke, and disgrace, for children have come to the cervical opening, and there is no strength to give birth.}%
\verse{Maybe Adonai your God heard the words of Rabshakeh whom the king of Assyria, his master, has sent to taunt the living God, and he will rebuke the words that Adonai your God hears. And you must lift up a prayer for the benefit of the remnant that is found.’”}%
\verse{When\lnBWZ{} the servants of King Hezekiah came to Isaiah,}%
\verse{Isaiah said to them, “You must say this to your master: ‘Thus says Adonai: “You must not be afraid because of the words that you have heard, with which the servants of the king of Assyria have blasphemed me.}%
\verse{Look! I am about to put\lnBXA{} a spirit in him so that\lnBXB{} he shall hear a rumor and he shall return to his land, and I will cause him to fall by the sword in his land.”’”}%
\verse{And Rabshakeh returned and found the king of Assyria fighting against Libnah, for he had heard that he had left from Lachish.}%
\verse{Now\lnBWZ{} he\lebnote{That is, the king} heard concerning Tirhakah the king of Cush, saying, “He has set out to fight against\lebnote{Or “with”} you.” When\lnBWZ{} he heard it, he sent messengers to Hezekiah, saying,}%
\verse{“You shall say this to Hezekiah, king of Judah: ‘Do not let your God in whom you trust in him deceive you by saying, “Jerusalem will not be given into the hand of the king of Assyria.”}%
\verse{Look! you have heard what the kings of Assyria have done to all lands to destroy them, and you — shall you be delivered?}%
\verse{Did the gods of the nations that my predecessors\lebnote{“fathers”} destroyed deliver them — Gozan, Haran, Rezeph, and the sons of Eden who were in Telassar?}%
\verse{Where is the king of Hamath, the king of Arpad, the king of the city of Sepharvaim, Hena, or\lnBXB{} Ivvah?’”}%
\verseWithHeading{Hezekiah’s Prayer}{And Hezekiah took the letter from the hand of the messengers, and he read\lebnote{Or “called”} it. Then\lnBWZ{} he went up to the temple\lnBWY{} of Adonai, and Hezekiah spread it out before the presence\lebnote{“face”} of Adonai.}%
\verse{And Hezekiah prayed to Adonai, saying,}%
\verse{“Adonai of hosts, God of Israel who is enthroned on\lebnote{“sitting”} the cherubim, you are the one, God by yourself, of all the kingdoms of the earth; you are the one who made the heavens and the earth.}%
\verse{Adonai, extend your ear and hear! Adonai, open your eyes and see, and hear all the words of Sennacherib that he has sent to taunt the living God!}%
\verse{Truly, Adonai, the kings of Assyria have laid waste all the lands\lebnote{The parallel text in 2 Kings has “nations”} and their land,}%
\verse{to set\lnBXA{} their gods in the fire, for they were not gods, but the work of human hands, wood and stone, and they destroyed them.}%
\verse{So\lnBWZ{} now, Adonai, our God, save us from his hand so that\lnBXB{} all the kingdoms of the earth may know that you are Adonai, you alone!”}%
\verseWithHeading{God’s Answer}{And Isaiah son of Amoz sent to Hezekiah, saying, “Thus says Adonai the God of Israel: ‘Because you have prayed to me concerning\lnBXC{} Sennacherib, king of Assyria,}%
\verse{this is the word that Adonai has spoken concerning him: She shows contempt for you; she derides you, virgin daughter of Zion; she shakes her head behind you, daughter of Jerusalem.}%
\verse{Whom have you taunted and blasphemed, and against whom have you raised up your voice and lifted your eyes upward? To the holy one of Israel!}%
\verse{By the hand of your servants you have taunted the Lord, and you have said, “With my many chariots,\lebnote{Hebrew “chariot”} I myself have gone up the height of the mountains, to the remote areas of Lebanon. And I cut off its tall cedars,\lebnote{“the height of its cedars”} the choicest of its junipers. And I came to the height of its limit, the forest of its orchard.\lebnote{Or “Carmel”}}%
\verse{I myself dug and drank waters, and I caused all the streams of Egypt to dry up by the sole of my feet.”}%
\verse{Have you not heard from a long time ago?\lebnote{“distant”} I have made it from days of primeval time, and I formed it. Now I have brought it about, and it is for fortified cities to collapse into heaps of destroyed stones.}%
\verse{And their inhabitants are weak;\lebnote{“short of hand”} they are dismayed, and they are ashamed; they have become like plants\lnBXD{} of the field, and like greens of grass, like grass on\lebnote{Or “of”} the roofs and a cultivated field before\lebnote{“before the face of”} the standing grain.}%
\verse{And I know your sitting down and your going out, and your coming in, and your raging against\lnBXC{} me.}%
\verse{Because you were enraged against\lnBXC{} me, and your noise\lebnote{Or “complacency”} has come up to\lebnote{“in”} my ears, I will put my hook in your nose and my bridle on your lips, and I will turn you back on the way by which you came.}%
\verse{And this shall be the sign for you: the eating of volunteer plants\lnBXD{} this\lebnote{“the”} year, and in the second year self-seeded plants,\lnBXD{} and in the third year sow, reap, plant vineyards, and eat their fruit.}%
\verse{And the remnant of the house of Judah that remain shall grow\lebnote{“add”} roots\lebnote{Hebrew “root”} downwards and make fruit upwards.}%
\verse{For a remnant shall go out from Jerusalem and survivors\lebnote{Hebrew “survivor”} from mountain Zion. The zeal of Adonai of hosts will do this.’}%
\verse{Therefore thus says Adonai concerning\lebnote{Or “to”} the king of Assyria: ‘He shall not come to this city, and he shall not shoot an arrow there, and he shall not meet it with a shield, and he shall not heap a siege ramp up against her.}%
\verse{He shall return by the way that he came, and he shall not come to this city,’ declares\lebnote{“declaration of”} Adonai.}%
\verse{‘And I will defend this city, to save it for my sake and for the sake of David, my servant.’”}%
\verseWithHeading{Sennacherib’s Defeat}{And the angel of Adonai set out and struck one hundred and eighty-five thousand in the camp of the Assyrians. When\lnBWZ{} they rose in the morning, look! All of them were dead corpses.}%
\verse{Then\lnBWZ{} Sennacherib king of Assyria left, went, and returned and lived at Nineveh.}%
\verse{And this happened: he was bowing in worship in the house of Nisroch his god, and Adrammelech and Sharezer, his sons, struck him with the sword. And they themselves escaped to the land of Ararat, and Esar-haddon his son reigned as king in his place.}%
\end{biblechapter}%
\begin{biblechapter}% Isaiah 38
\verseWithHeading{Hezekiah’s Illness}{In those days, Hezekiah became sick to death, and Isaiah son of Amoz, the prophet, came to him and said to him, “Thus says Adonai: ‘Order your house, for you are about to die, and you shall not recover.’”}%
\verse{Then\lnBXE{} Hezekiah turned his face to the wall and prayed to Adonai,}%
\verse{and he said, “O Adonai, please remember how\lebnote{Or “that”} I have walked before your presence\lebnote{“face”} in faithfulness with a whole heart, and I have done the good in your eyes!” And Hezekiah wept with great weeping.}%
\verse{Then\lnBXE{} the word of Adonai came\lebnote{Or “was”} to Isaiah, saying,}%
\verse{“Go and say to Hezekiah, ‘Thus says Adonai, the God of David your ancestor:\lebnote{Or “father”} “I have heard your prayer; I have seen your tears. Look, I am going to\lebnote{Literally, “Behold me; he will add,” but in this context it makes better sense to change the “he” to “I”; most translations follow this emendation} add fifteen years to your days.}%
\verse{And I will deliver you and this city from the hand of the king of Assyria, and I will defend this city.”’}%
\verse{And this is the sign to you from Adonai, that Adonai will do this thing that he has spoken:}%
\verse{Look! I will cause the shadow of the steps, which it had gone down on the steps of Ahaz with the sun, to turn backwards ten steps.” And the sun turned back ten steps on the steps which it had gone down.}%
\verse{A writing of Hezekiah, king of Judah, when he was sick and had recovered from his sickness:}%
\verse{I was the one who said, “I must go in the quiet of my days; I am summoned through the gates of Sheol for the rest of my years.”}%
\verse{I said, “I shall not see Yah! Yah in the land of the living! I shall no more look at humankind among the inhabitants of the world.}%
\verse{My dwelling place is pulled up and removed from me like the tent of my shepherd; I have rolled up my life like a weaver. He cuts me off from the thrum; from day to night you bring me to an end.}%
\verse{I lie down\lebnote{Or “cry out”} until morning; like a lion, so he breaks all my bones; from day to night you bring me to an end.}%
\verse{Like a horse or a crane, so I chirp; I moan like a dove. My eyes are weak toward the height. Lord, I have oppression; lend me support!}%
\verse{What can I say? For\lnBXE{} he has spoken to me, and he himself has done it. I will walk slowly all my years because of the bitterness of my soul.}%
\verse{Lord, they live by them, and the life of my spirit belongs to all among them. And restore me to health and keep me alive!}%
\verse{Look! Bitterness was bitter to me for peace. And you were the one who loved\lebnote{Or possibly “kept back,” which sounds similar in Hebrew} my life from the pit of destruction, for you have cast all my sins behind your back.}%
\verse{For Sheol cannot praise you; death cannot praise you. Those who go down to the pit cannot hope for your faithfulness.}%
\verse{The living, the living one praises you like me today; a father will make your faithfulness known to children.}%
\verse{Adonai, save me, and we will play my music on stringed instruments all the days of our lives at the temple\lnBXF{} of Adonai.”}%
\verse{And Isaiah said, “Let them take\lebnote{“lift up”} a lump of figs, and let them rub it on the boil so that\lebnote{Or “and”} he may recover.”}%
\verse{And Hezekiah said, “What is the sign that I shall go up to the temple\lnBXF{} of Adonai?”}%
\end{biblechapter}%
\begin{biblechapter}% Isaiah 39
\verseWithHeading{The Delegation from Babylon}{At that time, Merodach-Baladan, son of Baladan, king of Babylon, sent letters and a present to Hezekiah, for\lnBXG{} he heard that he had been sick and recovered.}%
\verse{And Hezekiah rejoiced over them and showed them his house of aromatic gum, the silver, gold, spices, good oil, all the house of his weapons, and all that was found in his storehouses. There was nothing that Hezekiah had not shown them in his house or in all his dominion.}%
\verse{And Isaiah the prophet came to King Hezekiah and said to him, “What did these men say, and from where did they come to you?” And Hezekiah answered,\lnBXH{} “They came to me from a distant country, from Babylon.”}%
\verse{And he\lebnote{That is, Isaiah} said, “What have they seen in your house?” And Hezekiah answered,\lnBXH{} “They have seen all that is in my house. There is nothing that I have not shown them in my storehouses.”}%
\verse{And Isaiah said to Hezekiah, “Hear the word of Adonai of hosts:}%
\verse{‘Look! days are coming, and all that is in your house and that which your ancestors\lebnote{Or “fathers”} have stored up to this day shall be carried off to Babylon. Nothing shall be left,’ says Adonai.}%
\verse{‘And some of your sons who go out from you, whom you fathered, shall be taken, and they shall be eunuchs in the palace of the king of Babylon.’”}%
\verse{And Hezekiah said to Isaiah, “The word of Adonai that you have spoken is good,” for\lnBXG{} he thought,\lnBXH{} “Surely there will be peace and security in my days.”}%
\end{biblechapter}%
\begin{biblechapter}% Isaiah 40
\verseWithHeading{Comfort for God’s People}{“Comfort; comfort my people,” says your God.}%
\verse{“Speak to the heart of Jerusalem, and call to her, that her compulsory labor is fulfilled, that her sin is paid for, that she has received\lebnote{“taken”} from the hand of Adonai double for all her sins.”}%
\verse{A voice is calling in the wilderness, “Clear the way of Adonai! Make a highway smooth in the desert for our God!}%
\verse{Every valley shall be lifted up, and every mountain and hill shall become low, And the rough ground shall be like a plain, and the rugged ground like a valley-plain.}%
\verse{And the glory of Adonai shall be revealed, and all humankind\lnBXI{} together shall see it, for the mouth of Adonai has spoken.”}%
\verse{A voice is saying, “Call!” And he said, “What shall I call?” All humankind\lnBXI{} are grass, and all his loyalty is like the flowers of the field.}%
\verse{Grass withers; the flower withers when the breath of Adonai blows on it. Surely the people are grass.}%
\verse{Grass withers; the flower withers, but\lnBXJ{} the word of our God will stand forever.}%
\verse{Get yourself\lebnote{“Go for you”} up to a high mountain, Zion, bringer of good news! Lift up your voice with strength, Jerusalem, bringer of good news! Lift it up; you must not fear! Say to the cities of Judah, “Here is your God!”}%
\verse{Look! The Lord Adonai comes with strength,\lebnote{“strong”} and his arm rules for him. Look! His reward is with him, and his recompense in his presence.\lebnote{“before his face”}}%
\verse{He will feed his flock like a shepherd; he will gather the lambs in his arm, and he will carry them in his bosom; he will lead those who nurse.}%
\verse{Who has measured the waters in the hollow of his hand and marked off the heavens with a span, comprehended the dust of the earth in a third of a measure and weighed out the mountains in the scales,\lebnote{Hebrew “scale”} and the hills in a balance?}%
\verse{Who has measured up the spirit of Adonai or informed him as his counselor\lebnote{“the man of his counsel”}?}%
\verse{With whom has he consulted, that\lnBXJ{} he enlightened him\lebnote{Or “made him understand”} and taught him the path of justice, and taught him knowledge, and made the way of understanding known to him?}%
\verse{Look! The nations are like a drop from a bucket, and they are counted like dust of the balances! Look! He weighs the islands like a thin covering.}%
\verse{And Lebanon is not enough to light a fire, and its animals\lebnote{Hebrew “animal”} not enough for a burnt offering.}%
\verse{All the nations are like nothing before him; they are counted by him as\lebnote{Or “from”} nothing and emptiness.}%
\verse{And to whom will you liken God? And to what likeness will you compare him?}%
\verse{A craftsman pours out the idol, and a goldsmith\lebnote{“refiner”} overlays it with gold, and he smelts chains of silver.}%
\verse{The one who is too impoverished for a gift chooses wood that will not rot; he seeks a skillful artisan for himself to set up an image that will not be knocked over.}%
\verse{Have you not known? Have you not heard? Has it not been told to you from the beginning?\lebnote{“head”} Have you not understood from the foundation of the earth?}%
\verse{He is the one who sits above the circle of the earth, and its inhabitants are like grasshoppers; the one who stretches out the heavens like a veil and spreads them out like a tent to live in,}%
\verse{the one who brings\lebnote{Or “gives”} princes to nothing; he makes rulers of the earth like nothing.}%
\verse{Indeed, hardly are they planted; indeed, hardly are they sown; indeed, hardly has their shoot taken root in the earth when\lebnote{Or “also”} he blows on them and they wither, and the tempest carries them like stubble.}%
\verse{“And to whom you will compare me, and am I equal?” says the holy one.}%
\verse{Lift your eyes up on high, and see! Who created these? The one who brings out their host by number. He calls all them by name. Because he is great of power and mighty of power, no man is missing.}%
\verse{Why\lebnote{“To what”} do you say, Jacob, and you speak, Israel, “My way is hidden from Adonai, and my judgment is passed over by my God?”}%
\verse{Have you not known, or have you not heard? Adonai is the God of eternity, the creator of the ends of the earth! He is not faint, and he does not grow weary! There is no searching his understanding.}%
\verse{He gives power to the weary, and he increases power for the powerless.\lebnote{“not power”}}%
\verse{Even\lnBXK{} young people will be faint and grow weary, and the young will stumble, exhausted.}%
\verse{But\lnBXK{} those who wait for Adonai shall renew their strength. They shall go up with wings\lebnote{Hebrew “wing”} like eagles; they shall run and not grow weary; they shall walk and not be faint.}%
\end{biblechapter}%
\begin{biblechapter}% Isaiah 41
\verseWithHeading{God Helps Israel}{Listen to me in silence, coastlands, and let nations renew their strength. Let them approach, then let them speak; let us draw near together for judgment.}%
\verse{Who has roused salvation from the east, summoned\lebnote{Or “called”} him to his foot, gives nations in his presence,\lebnote{“before his face”} and subjugates kings? He makes\lebnote{Or “gives”} them like the dust with his sword, like scattered stubble with his bow.}%
\verse{He pursues them and passes on in peace; he does not enter the path with his feet.}%
\verse{Who has accomplished and done this, calling the generations from the beginning?\lnBXL{} I, Adonai, am first; and I am the one with the last.}%
\verse{The coastlands have seen and are afraid; the ends of the earth tremble. They have drawn near, and they have come.}%
\verse{Each one helps\lebnote{Hebrew “help”} his neighbor; he says to his brother, “Take courage!”}%
\verse{And the artisan encourages the goldsmith,\lebnote{“refiner”} the one who makes smooth with the hammer encourages the one who strikes the anvil, saying of the soldering, “It is good!” And they strengthen it with nails so it cannot be knocked over.}%
\verse{But\lnBXM{} you, Israel, my servant, Jacob, whom I have chosen, you, the offspring\lebnote{“seed”} of Abraham my friend,\lebnote{“beloved”}}%
\verse{you whom I grasped from the ends of the earth and called from its remotest parts and told, “You are my servant; I have chosen you and I have not rejected you.”}%
\verse{You must not fear, for I am with you; you must not be afraid, for I am your God. I will strengthen you, indeed I will help you, indeed I will take hold of you with the right hand of my salvation.}%
\verse{Look! All those who are angry with you shall be ashamed and humiliated; your opponents\lnBXN{} shall be like nothing and shall become lost.}%
\verse{You shall seek them, but\lnBXO{} you shall not find them; your opponents\lnBXN{} shall be like nothing, and the men of your war like nothing.}%
\verse{For I, Adonai your God, am grasping your right hand; it is I who say\lebnote{“the sayer”} to you, “You must not fear; I myself, I will help you.}%
\verse{You must not fear, O worm of Jacob; people of Israel, I myself, I will help you,” declares\lebnote{“declaration of”} Adonai, “and your redeemer is the holy one of Israel.}%
\verse{Look! I will make you into a new sharp threshing sledge, with\lebnote{“owner of”} sharp edges. You shall thresh and crush the mountains, and you shall make the hills like chaff.}%
\verse{You shall winnow them and the wind shall carry them, and the tempest shall scatter them. And you yourself shall rejoice in Adonai; you shall boast in the holy one of Israel.}%
\verse{The poor and the needy are seeking water and there is none; their tongue is dried up with thirst. I, Adonai, will answer them; I, the God of Israel, will not forsake them.}%
\verse{I will open rivers on the barren heights and fountains in the midst of the valleys. I will make the wilderness like a pool of water and the land of dryness like springs of water.}%
\verse{I will put\lebnote{“give”} the cedar, acacia, myrtle, and olive oil tree in the wilderness; I will set the cypress, elm, and box tree together in the desert}%
\verse{so that they may see and know, and take to heart and understand together that the hand of Adonai has done this, and the holy one of Israel has created it.”}%
\verse{“Present your legal case,” says Adonai. “Bring your evidence,” says the king of Jacob.}%
\verse{Let them bring them, and let them tell us what will happen. Tell us what the former things are so that\lnBXM{} we may take them to our heart and know their outcome.\lebnote{Or “end”} Declare to us the things to come;}%
\verse{tell us the things coming hereafter,\lebnote{“later”} that\lnBXO{} we may know that you are gods. Indeed, do good or\lnBXO{} do evil, that\lnBXO{} we may be afraid and see\lebnote{Following the reading tradition (\textit{Qere}); the written text (\textit{Kethib}) has “fear”} together.}%
\verse{Look! you are nothing, and your work is something worthless; whoever chooses you is an abomination.}%
\verse{I stirred up one from the north, and he has come from the rising of the sun. He shall call on my name, and he shall come on officials as on mortar, and as the potter\lebnote{“creator”} treads clay.}%
\verse{Who declared it from the beginning\lnBXL{} so that\lnBXO{} we might know, and beforehand\lebnote{“from before faces”} so that\lnBXO{} we might say, “He is right!”\lebnote{“righteous”} Indeed, there was no one who declared it; Indeed, there was no one who proclaimed it. Indeed there was no one who heard your words.}%
\verse{First to Zion, look! Look at them! And I give a herald\lebnote{Or “bringer”} of good tidings to Jerusalem.}%
\verse{But\lnBXM{} I look and there is no man, and I look among these and there is no counselor, that\lnBXO{} I might ask them and they might answer a word.}%
\verse{Look! All of them are deception; their works are nothing; their images are wind and emptiness.}%
\end{biblechapter}%
\begin{biblechapter}% Isaiah 42
\verseWithHeading{The Mission of Adonai’s Servant}{Look! here is my servant; I hold him, my chosen one, in whom my soul delights. I have put\lebnote{“given”} my spirit on him; he will bring justice forth to the nations.}%
\verse{He will not cry out and lift up and make his voice heard in the street.}%
\verse{He will not break a broken reed, and he not will extinguish a dim wick. He will bring justice forth in\lebnote{Or “to”} faithfulness.}%
\verse{He will not grow faint, and he will not be broken until he has established justice in the earth. And the coastlands wait for his teaching.}%
\verse{Thus says the God, Adonai, who created the heavens and stretched them out, who spread out the earth and its offspring, who gives breath to the people upon it and spirit to those who walk in it.}%
\verse{“I am Adonai; I have called you in righteousness, and I have grasped your hand and watched over you; and I have given you as a covenant of the people, as a light of the nations,}%
\verse{to open the blind eyes, to bring the prisoner out from the dungeon, those who sit in darkness from the house of imprisonment.}%
\verse{I am Adonai; that is my name, and I do not give my glory to another, nor\lnBXP{} my praise to the idols.}%
\verse{Look! the former things have come, and I declare new things. I announce\lebnote{“cause to hear”} them to you before they sprout up.”}%
\verseWithHeading{A Song of Praise to Adonai}{Sing a new song to Adonai; praise him from the end of the earth, you\lebnote{Or “those”} who go down to the sea and that which fills it, the coastlands and their inhabitants.}%
\verse{Let the desert and its towns lift up their voice, the villages that Kedar inhabits. Let the inhabitants of Sela sing for joy; let them shout loudly from the top\lebnote{“head”} of the mountains.}%
\verse{Let them give\lebnote{Or “establish”} glory to Adonai and declare his praise in the coastlands.}%
\verse{Adonai goes forth like a mighty warrior; he stirs up zeal like a man of war. He raises the war cry, indeed he raises the battle shout; he prevails against his foes.}%
\verse{I have been silent for a long time; I have kept silent; I have restrained myself like one giving birth; I will moan, pant, and gasp together.}%
\verse{I will cause mountains and hills to dry up, and I will cause all their herbage to wither; and I will make rivers like islands, and I will cause pools to dry up.}%
\verse{And I will lead the blind by a road they do not know; I will cause them to tread on paths they have not known. I will make darkness in their presence\lebnote{“before their face”} into light and rough places into level ground. These are the things I will do, and I will not forsake them.}%
\verse{They shall turn back; they shall be greatly ashamed,\lebnote{“ashamed with shame”} those who trust in an image, who say to a cast image, “You are our gods.”}%
\verseWithHeading{Blind and Deaf Israel}{Deaf people, listen! And blind people, look to see!}%
\verse{Who is blind but my servant or\lnBXP{} deaf like my messenger whom I sent? Who is blind like the one who is repaid or\lnBXP{} blind like the servant of Adonai?}%
\verse{You see many things, but\lnBXP{} you do not observe. His ears are open, but\lnBXP{} he does not hear.}%
\verse{Adonai was willing for the sake of his righteousness; he showed his teaching to be great and proved it to be glorious.}%
\verse{But\lebnote{Or “And”} this is a people plundered and looted; all of them are trapped in holes, and they are kept hidden in houses of imprisonment. They have become like plunder, and there is no one who saves; like booty, and there is no one who says, “Restore!”}%
\verse{Who among you will heed this, will listen attentively and listen, for the time to come?\lebnote{“later”}}%
\verse{Who gave Jacob to a plunderer\lebnote{The reading tradition (\textit{Qere}) has “as plunder”} and Israel to those who plunder? Was it not Adonai, against whom\lebnote{Or “this”} we have sinned? And they were not willing to walk in his ways, and they would not obey\lebnote{“hear”} his law.}%
\verse{So\lnBXP{} he poured the wrath of his anger upon him and the strength of war. And it set him afire all around, but\lnBXP{} he did not understand;\lebnote{Or “did not know”} and it burned him, but\lnBXP{} he did not take it to heart.}%
\end{biblechapter}%
\begin{biblechapter}% Isaiah 43
\verseWithHeading{The Restorer of Israel}{But now thus says Adonai, he who created you, Jacob, and he who formed you, Israel: “You must not fear, for I have redeemed you. I have called you by your name; you are mine.\lebnote{“to me you”}}%
\verse{When you pass through the waters, I will be with you, and through the rivers, they shall not flow over you. When you walk through fire, you shall not be burned, and the flame shall not scorch\lebnote{Or “burn”} you.}%
\verse{For I am Adonai, your God, the holy one of Israel, your savior. I give you Egypt as ransom, Cush and Seba in place of you.}%
\verse{Because you are precious in my eyes, you are honored, and I myself love you, and I give people in place of you, and nations in place of your life.}%
\verse{You must not fear, for I am with you. I will bring your offspring\lebnote{“seed”} from the east, and I will gather you from the west.}%
\verse{I will say to the north, ‘Give!’ and to the south, ‘You must not withhold!’ Bring my sons from far away, and my daughters from the end of the earth —}%
\verse{everyone who is called by my name, and whom I created for my glory, whom I formed, indeed whom I made.”}%
\verse{Bring out the people blind yet\lnBXQ{} with eyes, and deaf, though\lnBXQ{} they have ears.}%
\verse{Let all the nations gather together, and let the peoples assemble. Who among them has declared this, and declared\lnBXR{} the former things to us? Let them bring\lebnote{Or “give”} their witnesses, that\lnBXQ{} they may be in the right, and let them hear and say, “It is true!”}%
\verse{“You are my witnesses,” declares\lnBXS{} Adonai, “and my servant whom I have chosen so that you may know and believe in\lebnote{“to”} me and understand that I am he. No god was formed before me\lebnote{“my face”}, and none shall be after me.}%
\verse{I myself am Adonai, and there is no savior besides me!}%
\verse{I myself declared and saved, and I proclaimed.\lnBXR{} And there was no strange god\lebnote{“stranger”} among you. And you are my witnesses,” declares\lnBXS{} Adonai, “and I am God.}%
\verse{Indeed, from this day I am the one, and no one can deliver from my hand. I perform,\lebnote{Or “make”} and who can cancel it?”\lebnote{“bring it back”}}%
\verseWithHeading{Adonai Rescues His People}{Thus says Adonai, your redeemer, the holy one of Israel: “For your sake I will send to Babylon, and I will cause all of them to fall down as fugitives, and the Chaldeans,\lebnote{“Chaldea”} their rejoicing on the ships.}%
\verse{I am Adonai, your holy one, the creator of Israel, your king.”}%
\verse{Thus says Adonai, who makes\lebnote{“gives”} a way in the sea and a path in the mighty waters,}%
\verse{who brings out chariot and horse, army and mighty one. Together they lie down; they cannot rise. They are extinguished, quenched like a wick.}%
\verse{“You must not remember the former things, and you not must consider the former things.}%
\verse{Look! I am about to do a new thing! Now it sprouts! Do you not perceive\lebnote{Or “know”} it? Indeed, I will make\lebnote{“put”} a way in the wilderness, rivers in the desert.}%
\verse{The animals of the field will honor me, jackals and daughters of the ostrich, for I give water in the wilderness, rivers in the desert, to give a drink to my chosen people,}%
\verse{this people whom I formed for myself, so they might make known\lebnote{“count out”} my praise.}%
\verse{But\lebnote{Or “And”} you did not call me, Jacob; for you have become weary of me, Israel.}%
\verse{You have not brought me your sheep for a burnt offering nor\lnBXQ{} honored me with your sacrifice. I have not made you serve with offerings,\lebnote{Hebrew “offering”} nor have I made you weary with frankincense.}%
\verse{You have not bought me spice reed with money or\lnBXQ{} satisfied me with the fat of sacrifices. But you have burdened me with your sins; you have made me weary with your iniquities.}%
\verse{I, I am the one who blots out your transgressions for my sake, and I will not remember your sins.}%
\verse{Take me to court; let us enter into judgment together. You, make an account\lebnote{“count up”} so that you may be in the right.}%
\verse{Your first ancestor\lebnote{Or “father”} sinned, and your representatives\lebnote{“scoffers” or “spokesmen”} transgressed against me.}%
\verse{And I profaned the princes of the sanctuary, and I gave Jacob to destruction, and Israel to reviling.}%
\end{biblechapter}%
\begin{biblechapter}% Isaiah 44
\verseWithHeading{Adonai Blesses Chosen Israel}{“But now hear, Jacob my servant, and Israel, whom I have chosen.}%
\verse{Thus says Adonai, who made you, and who formed you in\lnBXT{} the womb and will help you: you must not fear, my servant Jacob, and Jeshurun whom I have chosen.}%
\verse{For I will pour out water on a thirsty land and streams on dry ground. I will pour my spirit out on your descendants\lebnote{Hebrew “descendant”} and my blessing on your offspring.}%
\verse{And they shall sprout among\lebnote{“in between”} the grass like willows by a watercourse of water.}%
\verse{This one will say, ‘I belong to Adonai!’ And that one will be called by the name of Jacob, and another will write on his hand ‘Adonai’s’ and take the name\lebnote{“he will be titled by the name”} of Israel.”}%
\verseWithHeading{Adonai the Only God}{Thus says Adonai, the king of Israel, and its redeemer, Adonai of hosts: “I am the first, and I am the last, and there is no god besides me.}%
\verse{And who is like me? Let him proclaim\lebnote{Or “call”} it! And let him declare it and set it in order for me since I established an eternal people\lebnote{“from my placing a people of eternity”} and things that are to come, and let them tell them the things that are coming.}%
\verse{You must not tremble,\lnBXU{} and you must not be paralyzed with fear. Have I not made you hear from of old\lebnote{Or “then”} and declared it, and you are my witnesses? Is there a god besides me? And there is no rock! I know none!”}%
\verseWithHeading{Idolatry Is Ridiculous}{All those who form an idol are nothing, and their delightful things do not profit. And their witnesses do not see or know, so they will be ashamed.}%
\verse{Who would form a god and cast an image of which he cannot profit?}%
\verse{Look! all his companions shall be ashamed, and the artisans are human! Let all of them assemble; let them stand up. They shall tremble;\lnBXU{} they shall be ashamed together.}%
\verse{The ironsmith\lebnote{“craftsman of iron”} works in the coals with his tool and forms it with hammers. And he makes it with his strong arm;\lebnote{“the arm of his strength”} indeed, he becomes hungry, and he lacks\lebnote{“there is no”} strength; he does not drink water, and he is faint.}%
\verse{The woodworker\lebnote{“craftsman of wood”} stretches out a line; he makes an outline of it with a marker. He makes it with a knife and makes an outline of it with a compass. He makes it like the image of a man, like the beauty of a human, to dwell in a temple.\lebnote{Or “house”}}%
\verse{Cutting down cedars for himself, he chooses\lebnote{“takes”} a holm tree and an oak, and he lets it grow strong for him among the trees of the forest. He plants a cedar, and the rain makes it grow.}%
\verse{And it becomes fuel for a human,\lebnote{“will become for a man as burning”} and he takes some of it and grows warm; also, he kindles a fire and bakes bread. Also, he makes a god and bows in worship; he makes himself an image and bows down to it!}%
\verse{He burns half of it in the fire; he eats meat over half of it; he roasts a roast and is satisfied. Also he grows warm and says, “Ah! I am warm! I see the fire!”}%
\verse{And he makes the remainder of it into a god! He bows down to his idol, and he bows in worship and prays to him, and he says, “Save me, for you are my god!”}%
\verse{They do not know, and they do not understand, for their eyes are besmeared so that they cannot see,\lebnote{“from seeing”} their minds\lebnote{“hearts”} so that they have no insight.\lebnote{“from having insight”}}%
\verse{And no one takes it to heart,\lebnote{“he does not bring back to his heart”} and there is no knowledge and no understanding to say, “I burned half of it in the fire and also I baked bread on its coals; I roasted meat, and I have eaten. And I shall make the rest of it into an abomination! I shall bow down to a block of wood!”}%
\verse{He feeds on ashes; a deceived mind\lebnote{“heart deceived”} misleads him. And he cannot save himself,\lebnote{“his soul”} and he cannot say, “Is this not an illusion in my right hand?”}%
\verseWithHeading{Adonai Remembers Israel}{“Remember these things, Jacob, and Israel, for you are my servant: I formed you; you are my servant; Israel, you will not be forgotten by me!}%
\verse{I have wiped your transgressions out like a cloud and your sins like mist. Return to me, for I have redeemed you!”}%
\verse{Sing for joy, heavens, for Adonai has done it! Shout, depths of the earth! Break forth, mountains, in rejoicing, forest and every tree in it, for Adonai has redeemed Jacob, and he will show his glory in Israel!}%
\verseWithHeading{Jerusalem Restored through Cyrus}{Thus says Adonai, your redeemer, and he who formed you in\lnBXT{} the womb: “I am Adonai, who made everything, who stretched out the heavens alone, who spread out the earth — who was with me? —}%
\verse{who frustrates\lebnote{Or “breaks”} the signs of oracle priests and makes a fool of diviners, who drives the wise men back and makes their knowledge foolish,}%
\verse{who keeps\lebnote{Or “erects”} the word of his servant and carries out the plan of his messengers, who says of Jerusalem, ‘It shall be inhabited,’ and of the cities of Judah, ‘They shall be rebuilt, and I will restore its ruins’;}%
\verse{who says to the deep, ‘Dry up! And I will cause your rivers to dry up’;}%
\verse{who says of Cyrus, ‘My shepherd,’ and he shall carry out all my wishes\lebnote{Hebrew “wish”}; and saying of Jerusalem, ‘It shall be rebuilt,’ and the temple, ‘It shall be founded.’”}%
\end{biblechapter}%
\begin{biblechapter}% Isaiah 45
\verse{Thus says Adonai to his anointed one, to Cyrus, whose right hand I have grasped to subjugate nations before him, and I uncover\lebnote{Or “let loose”} the loins of kings to open doors before him, and the gates shall not be shut:}%
\verse{“I myself will go before you, and I will level the mountains.\lebnote{“swells”} I will break the doors of bronze and cut throughthe bars of iron.}%
\verse{And I will give you the treasures of darkness and treasures of secret places so that you may know that I am Adonai, the one who calls you by your name, the God of Israel,}%
\verse{for the sake of my servant Jacob, and Israel my chosen one. And I call you by your name; I give you a name of honor, though\lnBXV{} you do not know me.}%
\verse{I am Adonai, and there is none besides me; besides me there is no god. I gird you though\lnBXV{} you do not know me,}%
\verse{so that they may know from the rising of the sun and from the west that there is none besides me; I am Adonai and there is none besides me.}%
\verse{I form light and I create darkness; I make peace and I create evil; I am Adonai; I do all these things.}%
\verse{Trickle, O heavens, from above, and let clouds trickle with righteousness; let the earth open so that\lnBXV{} salvation may be fruitful, and let it cause righteousness to sprout along with it.\lebnote{Or “together”} I myself, Adonai, have created it.}%
\verse{Woe to the one who strives with his maker,\lnBXW{} a potsherd among\lebnote{Or “with”} potsherds of earth! Does the clay say to the one who fashions it, ‘What are you making?’ and ‘Your work has no hands’?}%
\verse{Woe to the one who says to a father, ‘What you are begetting?’ or\lnBXV{} to a woman, ‘With what are you in labor?’”}%
\verse{Thus says Adonai, the holy one of Israel, and its maker:\lnBXW{} “Ask me of the things to come about\lebnote{“over”} my children, and you command me about the work of my hands.}%
\verse{I myself made the earth, and I created humankind upon it. I, my hands, stretched out the heavens, and I commanded all their host.}%
\verse{I myself have stirred him up in righteousness, and I will make all his paths smooth. He himself shall build my city, and he shall set my exiles free, not for price or a gift,” says Adonai of hosts.}%
\verseWithHeading{Adonai the Only Savior}{Thus says Adonai: “The acquisition of Egypt and the merchandise of Cush and the Sabeans, tall men, shall pass over to you; they shall be yours, and they shall walk behind you. They shall pass over in chains, and they shall bow down to you; they will pray to you: ‘Surely God is with you, and there is no other. Besides him there is no God.’”}%
\verse{Surely you are a God who keeps yourself hidden, God of Israel, the savior.}%
\verse{All of them are ashamed and indeed humiliated; the craftsmen of idols go together in insult.}%
\verse{Israel is saved by Adonai with everlasting salvation; you shall not be ashamed, and you shall not be humiliated to all eternity.\lebnote{“for eternity, forever”}}%
\verse{For thus says Adonai, who created\lebnote{“creating”} the heavens, he is God, who formed\lebnote{“forming”} the earth and who made\lebnote{“making”} it. He himself established it; he did not create it as emptiness — he formed it for inhabiting. “I am Adonai and there is none besides me.}%
\verse{I have spoken not in secrecy, in a place, a land, of darkness, I have not said to the descendants of Jacob, ‘Seek me in vain!’\lebnote{“in emptiness,” as in verse 18} I, Adonai, am speaking righteousness, declaring uprightness.}%
\verse{Assemble and come; draw near together, survivors of the nations! They do not know, those who carry their wooden idols\lebnote{“the wood of their idol”} and pray to a god who cannot save.}%
\verse{Declare and present your case, also let them consult together! Who made this known\lebnote{“caused to hear this”} from former times,\lebnote{“before”} declared it from of old?\lebnote{“then”} Was it not I, Adonai? And there is no other god besides me, a righteous God besides me, and no savior besides me.}%
\verse{Turn to me and be saved, all the ends of the earth, for I am God and there is none besides me.}%
\verse{I have sworn by myself; a word that\lnBXV{} shall not return has gone forth from my mouth in righteousness: ‘Every knee shall kneel down to me; every tongue shall swear.’}%
\verse{‘Only in Adonai,’ one shall say to me, ‘are righteousness and strength.’ He shall come to him, and all those who were angry with him shall be ashamed.}%
\verse{In Adonai all the offspring of Israel shall be in the right, and they shall boast.”}%
\end{biblechapter}%
\begin{biblechapter}% Isaiah 46
\verseWithHeading{Babylon’s Idols}{Bel bows down; Nebo is stooping. Their idols are on\lnBXX{} animals\lebnote{Hebrew “animal”} and on\lnBXX{} cattle; your cargo\lebnote{“things that are carried”} is carried as a burden on\lnBXX{} weary animals.}%
\verse{They stoop; they bow down together. They are not able to save the burden, but\lnBXY{} they themselves go\lebnote{“their inner self goes”} in captivity.}%
\verse{“Listen to me, house of Jacob, and all the remnant of the house of Israel who have been carried from the belly, who have been carried from the womb:}%
\verse{Even\lnBXZ{} to your old age I am he; even\lnBXZ{} to your advanced age I myself will support you. I myself have made you, and I myself will carry you, and I myself will support you, and I will save you.}%
\verse{To whom will you liken me, and count as equal, and compare with me, as though\lnBXY{} we were alike?}%
\verse{Those who lavish gold from the purse and weigh out silver in the balance scales; hire a goldsmith\lebnote{“one who smelts”} and he makes him a god; they bow down, indeed they bow in worship.}%
\verse{They carry it on their shoulder; they support it and they set it in its place, and it stands in position. It cannot be removed from its place; even when he cries out to it, it does not answer. It does not save him from his trouble.}%
\verse{Remember this and pluck up courage! Call to mind,\lebnote{“heart”} you transgressors!}%
\verse{Remember the former things from a long time ago,\lebnote{“forever”} for I am God and there is none besides me, God and there is none like me,}%
\verse{who from the beginning declares the end, and from before, things that have not been done, who says, ‘My plan shall stand,’ and, ‘I will accomplish\lebnote{Or “do”} all my wishes,’\lebnote{Hebrew “wish”}}%
\verse{who calls a bird of prey from the east, the man of his\lebnote{The reading tradition (\textit{Qere}) has “my”} plan from a country from afar. Indeed I have spoken; indeed I will bring it to being. I have formed it; indeed I will do it.}%
\verse{Listen to me, strong of heart, far from righteousness!}%
\verse{I bring my righteousness near; it is not far. And my salvation will not delay; and I will put\lebnote{Or “give”} salvation in Zion, for Israel my glory.”}%
\end{biblechapter}%
\begin{biblechapter}% Isaiah 47
\verseWithHeading{Babylon’s Fall}{Come down and sit on the dust, virgin daughter of Babylon! Sit on the ground without a throne, daughter of Chaldea! For they shall no longer call\lnBYA{} you tender and delicate.}%
\verse{Take the pair of mill stones and grind flour! Uncover your veil, strip off your skirt, uncover your thigh, pass through the rivers!}%
\verse{Your nakedness shall be exposed; indeed, your shame shall become visible. I will take vengeance and I will not spare\lebnote{“plead with”} a person.}%
\verse{Our redeemer, Adonai of hosts is his name, the holy one of Israel.}%
\verse{Sit silently and go into the darkness, daughter of Chaldea, for they shall no longer call\lnBYA{} you mistress of kingdoms.}%
\verse{I was angry with my people; I profaned my inheritance, and I gave them into your hand. You did not give\lebnote{“set up”} them mercy; on the aged you made your yoke very heavy.}%
\verse{And you said, “I shall be an eternal mistress forever!” You did not set these things upon your heart; you did not remember its end.}%
\verse{Therefore now hear this, luxuriant one who sits\lebnote{Or “sitting”} in security, who says\lebnote{Or “saying”} in her heart, “I am, and besides me there is no one. I shall not sit as a widow, and I shall not know the loss of children.”}%
\verse{And these two shall come to you in a moment, in one day: the loss of children and widowhood shall come on you completely,\lebnote{“as their fullness”} in spite of your many sorceries, in spite of the power of your great enchantments.}%
\verse{And you felt secure in your wickedness; you said, “No one\lebnote{“none”} sees\lebnote{Or “\textit{is} seeing”} me.” Your wisdom and your knowledge led you astray, and you said in your heart, “I am, and besides me there is no one.”}%
\verse{And evil shall come upon you, you will not know; it will be on the lookout for her. And disaster shall fall upon you; you will not be able to avert\lebnote{“make amends”} it. And ruin shall come on you suddenly; you do not know.}%
\verse{Stand, now, in your enchantments, and in your many sorceries with which you have labored from your youth. Perhaps you may be able to benefit; perhaps you may scare away.\lebnote{“terrify”}}%
\verse{You struggle with your many consultations;\lebnote{Or “advice”} let them stand, now, and save you — those who see the stars, divide the celestial sphere,\lebnote{“heavens,” that is, for astrology} who inform by new moons — from those things that are coming upon you.}%
\verse{Look! They are like stubble; the fire burns them completely. They cannot deliver themselves\lebnote{“their lives”} from the power\lebnote{“hand”} of the flame; there is no coal for warming oneself,\lebnote{“to grow warm”} no fire before which to sit.}%
\verse{So are to you those with whom you have labored, your traders from your youth. They wander, each to his side; there is no one who can save you.}%
\end{biblechapter}%
\begin{biblechapter}% Isaiah 48
\verseWithHeading{Adonai Refines Stubborn Israel}{Hear this, house of Jacob, who are called by the name of Israel and came out from the waters of Judah, who swear by the name of Adonai, and invoke\lebnote{“cause to remember”} the God of Israel, but not in truth and not in righteousness.}%
\verse{For they call themselves after the holy city,\lebnote{“from the city of the holiness”} and they lean on the God of Israel — Adonai of hosts is his name.}%
\verse{“I declared the former things from of old,\lnBYB{} and they went out from my mouth. And I announced\lnBYC{} them suddenly; I acted,\lebnote{Or “did”} and they came to pass,}%
\verse{because I know\lebnote{“knowing me”} that you are obstinate, and your neck an iron sinew, and your forehead bronze.}%
\verse{And I declared them to you from of old; I announced\lnBYC{} them to you before they came to pass so that you would not say, ‘My idol did them, and my image and my cast image commanded them.’}%
\verse{You have heard; see it all. And will you not declare it? I announce\lebnote{“cause to hear”} new things to you from this time and hidden things that\lnBYD{} you have not known.}%
\verse{Now they are created, and not from of old,\lnBYB{} and before today, and you have not heard them so that you could not say, “Look! I knew them.”}%
\verse{Neither have you heard, nor have you known, nor from of old\lnBYB{} has your ear been opened. For I knew you would deal treacherously, very treacherously, and you are called a rebel from the womb.}%
\verse{For the sake of my name I refrain from\lebnote{“make long”} my anger, and for my praise I restrain it for you so as not to cut you off.}%
\verse{Look! I have refined you, but\lnBYD{} not like\lebnote{Or “with”} silver; I have chosen you in the furnace of misery.}%
\verse{For my own sake, for my own sake I do it; for why should it\lebnote{That is, my name} be defiled? And I will not give my glory to another.”}%
\verseWithHeading{Adonai Frees Israel}{“Listen to me, Jacob, and Israel, whom I called:\lebnote{“who was called by me”} I am he. I am the first; also I am the last.}%
\verse{Indeed, my hand founded the earth, and my right hand spread out the heavens; when I summon\lebnote{Or “calling”} them, they stand in position together.}%
\verse{Assemble, all of you, and hear! Who among them declared these things? Adonai loves him; he shall perform\lebnote{Or “do”} his wish against Babylon and his arm against the Chaldeans.\lebnote{“Chaldea”}}%
\verse{I, I myself, I have spoken! Indeed, I have called him. I have brought him, and he will be successful in his way.}%
\verse{Draw near to me; hear this! I have not spoken in secrecy from the beginning;\lebnote{“head”} from the time it came to be,\lebnote{“\textit{of} becoming it”} there I have been; And now the Lord Adonai has sent me and his Spirit.”}%
\verse{Thus says Adonai, your redeemer, the holy one of Israel: “I am Adonai your God, who teaches\lebnote{“teaching”} you to profit, leads you in the way you should go.}%
\verse{O that you had listened attentively to my commandments! Then\lnBYD{} your prosperity would have been like a river, and your righteousness like the waves of the sea.}%
\verse{And your offspring would have been like the sand, and the descendants of your body\lebnote{“intestines”} like its grains.\lebnote{Hebrew “grain”} It would not be cut off, and its name would not be destroyed from my presence.”}%
\verse{Go out from Babylon! Flee from Chaldea! Proclaim\lebnote{Or “announce”} it with a shout\lebnote{Or “sound”} of rejoicing; proclaim\lebnote{Or “cause to hear”} this! Send it forth\lebnote{“Cause it to go out”} to the end of the earth; say, “Adonai has redeemed his servant Jacob!”}%
\verse{And when he led them through the deserts, they were not thirsty; he made water flow from the rock for them, and he split the rock, and the water gushed out.}%
\verse{“There is no peace,” says Adonai, “for the wicked.”}%
\end{biblechapter}%
\begin{biblechapter}% Isaiah 49
\verseWithHeading{Adonai’s Servant Brings Salvation}{Listen to me, coastlands, and listen attentively, peoples from far away! Adonai called me from the womb; from the body\lebnote{“intestines”} of my mother he made my name known.}%
\verse{And he made\lnBYE{} my mouth like a sharp sword; he hid me in the shadow of his hand, and he made\lnBYE{} me like an sharpened arrow; he hid me in his quiver.}%
\verse{And he said to me, “You are my servant, Israel, in whom I will show my glory.”}%
\verse{But\lnBYF{} I myself said, “I have labored in\lebnote{Or “for”} vain; I have used up my strength for nothing and vanity! Nevertheless, my justice is with Adonai, and my reward is with my God.”}%
\verse{And now Adonai says, who formed me from the womb as a servant for him, to bring Jacob back to him, and that Israel might not\lebnote{The reading tradition (\textit{Qere}) has “to him”} be gathered, for\lnBYG{} I am honored in the eyes of Adonai, and my God has become my strength.}%
\verse{And he says, “It is trivial for you to be\lebnote{“from being you”} a servant for me, to raise up the tribes of Jacob and to bring back the preserved of Israel. I will give you as a light to the nations, to be my salvation to the end of the earth.”}%
\verse{Thus says Adonai, the redeemer of Israel, his holy one, to the one who despises\lebnote{A Dead Sea Scroll reads “is despised \textit{with respect to}”} life, to the one who abhors the nation, to the slave of rulers: “Kings shall see and stand up; princes, and they shall bow down, for the sake of Adonai, who is faithful, the holy one of Israel, and he has chosen you.”}%
\verseWithHeading{Adonai Restores the Afflicted}{Thus says Adonai: “I have answered you in a time of favor, and helped you on a day of salvation, and watched over you, and given you as a covenant of the people, to raise up the land, to give the desolate hereditary property as an inheritance,}%
\verse{saying to the prisoners,\lebnote{“captured ones”} “Come out!” to those who are in darkness, “Show yourselves!” they shall feed along\lebnote{“on”} the ways, and their pasturage shall be on all the barren heights.}%
\verse{They shall not be hungry or thirsty, and heat and sun shall not strike them, for he who takes pity on them will lead them, and he will guide them to springs of water.}%
\verse{And I will make\lnBYE{} all my mountains like a road, and my highways shall lead up.\lebnote{Or “be high”}}%
\verse{Look! These shall come from afar, And look! These from the north and from the west and these from the land of Sinim.”}%
\verse{Sing for joy, heavens, and rejoice, earth! Mountains must break forth in rejoicing! For Adonai has comforted his people, and he will take pity on his afflicted ones.}%
\verseWithHeading{Adonai Remembers Zion}{But\lnBYF{} Zion said, “Adonai has forsaken me, and the Lord has forgotten me!”}%
\verse{Can a woman forget her suckling, refrain from having compassion on the child of her womb? Indeed, these may forget, but\lnBYG{} I, I will not forget you!}%
\verse{Look, I have inscribed you on the palms of my hands; your walls are continually before me.}%
\verse{Your children hasten; your destroyers and those who laid you waste depart\lebnote{Or “move away”} from you.}%
\verse{Lift your eyes up all around and see; all of them gather; they come to you. As surely as I live,\lebnote{“Life I”} declares\lebnote{“declaration of”} Adonai, surely you shall put on all of them like an ornament, and you shall bind them on like a bride.}%
\verse{Surely your sites of ruins and desolate places and land of ruins,\lebnote{Hebrew “ruin”} surely now you will be too cramped for your\lebnote{“cramped by”} inhabitants,\lebnote{Hebrew “inhabitant”} and those who engulfed you will be far away.}%
\verse{Yet the children born when you were bereaved\lebnote{“of your bereavement”} will say in your hearing,\lebnote{“ears”} “The place is too cramped for me; make room\lebnote{“approach”} for me so that I can dwell.”}%
\verse{Then\lnBYF{} you will say in your heart, “Who has borne me these?” And, “I was bereaved and barren, exiled and thrust away; so\lnBYG{} who raised these? Look at me! I was left alone; where have these come from?”\lebnote{“these where they”}}%
\verse{Thus says the Lord Adonai: “Look! I will lift my hand up to the nations, and I will raise my signal to the peoples, and they shall bring your sons in their bosom, and your daughters shall be carried on their shoulders.\lebnote{Hebrew “shoulder”}}%
\verse{And kings shall be your guardians,\lebnote{“those who nurture”} and their queens your nurses. They shall bow down, faces\lebnote{“noses”} to the ground, to you, and they will lick up the dust of your feet. Then\lnBYF{} you will know that I am Adonai; those who await me shall not be ashamed.}%
\verse{Can war-booty be taken from the mighty? or can a captive of a righteous\lebnote{The Dead Sea Scroll has “violent”} person be rescued?}%
\verse{But thus says Adonai: “Indeed a captive of the mighty shall be taken, and the war-booty of the tyrant shall be rescued, for\lnBYG{} I myself will dispute with your opponent, and I myself will save your children.}%
\verse{And I will feed your oppressors their own flesh, and they shall be drunk with their blood as with wine. Then\lnBYF{} all flesh shall know that I am Adonai, your savior and redeemer, the strong one of Jacob.”}%
\end{biblechapter}%
\begin{biblechapter}% Isaiah 50
\verse{Thus says Adonai: “Where is this divorce document of your mother’s divorce, with which I dismissed her? or to whom of my creditors did I sell you? Look! you were sold because of your sin, and your mother was dismissed because of your transgressions.}%
\verse{Why was there no man when I came, no one who answered when I called? Do I lack the strength to save?\lebnote{“short is short my hand from redemption”} Or is there no power in me to deliver? Look! by my rebuke I dry up the sea; I make\lnBYH{} the rivers a desert; their fish stink because there is no water, and they\lebnote{Hebrew “she”} die because of thirst.}%
\verse{I clothe the heavens with darkness, and I make\lnBYH{} their covering sackcloth.”}%
\verseWithHeading{The Servant’s Vindication}{The Lord Adonai has given me the tongue of a pupil, to know how to help the weary with a word. He awakens morning by morning,\lebnote{“in the morning in the morning”} awakens an ear for me to listen as do the pupils.}%
\verse{The Lord Adonai has opened an ear for me, and I, I was not rebellious. I did not turn backwards;}%
\verse{I gave my back to those who struck me, and my cheeks to those who pulled out my beard;\lebnote{“face”} I did not hide my face from insults and spittle.}%
\verse{And the Lord Adonai helps me, therefore I have not been put to shame; therefore I have set my face like flint. And I know that I shall not be ashamed;}%
\verse{he who obtains rights for me is near. Who will contend with me? Let us stand together. Who is the master of my judgment? Let him approach me.}%
\verse{Look! The Lord Adonai helps me. Who is the one who will declare me guilty? Look! All of them will be worn out like a garment; the moth will eat them.}%
\verse{Who among you is in fear of Adonai, obeys\lebnote{Or “listens”} the voice of his servant? Who walks in darkness and has no light, trusts in the name of Adonai and depends on his God?}%
\verse{Look! All of you are kindlers of fire, who gird yourselves with flaming arrows. Walk in the light of your fire, and among the flaming arrows you have kindled! You shall have this from my hand: you shall lie down in\lebnote{Or “to”} a place of torment.}%
\end{biblechapter}%
\begin{biblechapter}% Isaiah 51
\verseWithHeading{Adonai Comforts Zion}{“Listen to me, you who pursue righteousness, who seek Adonai. Look to the rock from which you were hewn, and to the excavation of the pit from which you were quarried.}%
\verse{Look to Abraham your father, and to Sarah; she brought you forth. For I called him alone,\lebnote{“one”} but\lnBYI{} I blessed him and made him numerous.”}%
\verse{For Adonai will comfort Zion; he will comfort all its sites of ruins. And he will make\lebnote{“put”} its wilderness like Eden, and its desert like the garden of Adonai. Joy and gladness will be found in it, thanksgiving and the sound\lebnote{“voice”} of song.}%
\verse{“Listen attentively to me, my people, and my nation, listen to me! For a teaching will go out from me, and I will cause my justice to rest for a light to the peoples.}%
\verse{My righteousness is near; my salvation has gone out, and my arms will judge the peoples. The coastlands wait for me, and for my arm they wait.}%
\verse{Lift up your eyes to the heavens and look to the earth beneath, for the heavens will be torn to pieces like smoke, and the earth will be worn out like a garment, and those who inhabit her will die like gnats. But\lnBYJ{} my salvation will be forever, and my righteousness will not be broken to pieces.}%
\verse{Listen to me, you who know righteousness, people who have my teaching in their heart; you must not fear the reproach of men, or be terrified because of their abuse.}%
\verse{For a moth will eat them like garments; a moth will devour\lebnote{Or “eat”} them like wool, but\lnBYI{} my righteousness will be forever, and my salvation for generation after generation.\lebnote{“generation of generations”}”}%
\verse{Awake! Awake; put on strength, O arm of Adonai! Awake as in days of long ago, the generations of a long time back! Are you not the one who cut Rahab in pieces, the one who pierced the sea-dragon?}%
\verse{Are you not the one who dried up the sea, the waters of the great deep, the one who made\lnBYK{} the depths of the sea a way for those who are redeemed to cross over?}%
\verse{So\lnBYJ{} the redeemed ones of Adonai shall return, and they shall come to Zion with singing, and everlasting joy shall be on their heads.\lebnote{Hebrew “head”} Joy and gladness shall appear;\lebnote{“reach”} sorrow and sighing shall flee away!}%
\verse{“I, I am he who comforts you; who are you that\lnBYI{} you are afraid of man? He dies! And of the son of humankind? He is sacrificed\lebnote{“given”} as grass!}%
\verse{And you have forgotten Adonai, your maker, who stretched out the heavens, and founded the earth. And you tremble continually, all day, because of the wrath of the oppressor when he takes aim\lebnote{“sets up”} to destroy. But\lnBYJ{} where is the wrath of the oppressor?}%
\verse{The fettered one shall make haste to be freed. And he shall not die in\lebnote{Or “to”} the pit, and he shall not lack his bread.}%
\verse{For\lnBYJ{} I am Adonai, your God, who stirs up\lebnote{Or “stirring up”} the sea, so that\lnBYI{} its waves roar; Adonai of hosts is his name.}%
\verse{And I have put my words in your mouth, and I have covered you in the shadow of my hand, to plant the heavens and to found the earth, saying to Zion, ‘You are my people.’”}%
\verse{Rouse yourself! Rouse yourself! Stand up, Jerusalem, who have drunk from the hand of Adonai the cup of his wrath; you have drunk the goblet, the cup of staggering; you have drained it out.}%
\verse{There is no one who guides her among\lnBYL{} all the children she has borne, and there is no one who grasps her by the hand among\lnBYL{} all the children she raised.}%
\verse{Two things here have happened to you — who will show sympathy\lebnote{“wander”} for you? — devastation and destruction, famine and sword — who will comfort you?}%
\verse{Your children have fainted; they lie at the head of all the streets, like an antelope in\lebnote{Or “of”} a snare, those who are full of the wrath of Adonai, the rebuke of your God.}%
\verse{Therefore hear now this afflicted one and drunken one but\lnBYI{} not from wine.}%
\verse{Thus says your Lord, Adonai, and your God pleads the cause of his people: “Look! I have taken from your hand the cup of staggering. You shall not continue\lebnote{“do again”} to drink the goblet, the cup of my wrath, any longer.}%
\verse{And I will put it in the hand of your tormenters, who have said to you,\lebnote{“your inner self”} ‘Bow down that\lnBYI{} we may pass\lnBYM{} over you!’ And you have made\lnBYK{} your back like the ground, and like the street for those who pass\lnBYM{} over you.”}%
\end{biblechapter}%
\begin{biblechapter}% Isaiah 52
\verseWithHeading{Adonai Redeems Zion}{Awake! Awake; put on your strength, Zion! Put on the garments of your beauty, Jerusalem, holy city!\lebnote{“city of \textit{the} holiness”} For the uncircumcised and the unclean shall not continue to\lebnote{“do again he shall”} enter you any longer.}%
\verse{Shake yourself free from the dust! Rise up; sit, Jerusalem! Free yourselves from the bonds of your neck, captive daughter of Zion!}%
\verse{For thus says Adonai: “You were sold for nothing, and you shall be redeemed without money.”}%
\verse{For thus says the Lord Adonai: “My people went down to Egypt in the beginning, to dwell as aliens there, and Assyria oppressed him without cause.\lebnote{“in nothing”}}%
\verse{And now what do I have here?”declares\lnBYN{} Adonai, “for my people is taken without cause. Its rulers howl,” declares\lnBYN{} Adonai — “and my name is reviled continually, all day.}%
\verse{Therefore my people shall know my name, therefore in that day, that I am the one who speaks. Here am I.”\lebnote{“Look at me”}}%
\verse{How delightful on the mountains are the feet of him who brings good news, who announces peace, who brings good news, who announces salvation, who says to Zion, “Your God reigns as a king.”}%
\verse{The voices\lnBYO{} of your watchmen! They lift up their voices;\lnBYO{} together they sing for joy; for they clearly\lebnote{“eye in eye “} see Adonai’s return to Zion.}%
\verse{Break forth, sing for joy together, ruins of Jerusalem, for Adonai has comforted his people; he has redeemed Jerusalem.}%
\verse{Adonai has bared his holy arm\lebnote{“the arm of his holiness”} to the eyes of all the nations, and all the ends\lebnote{Hebrew “end”} of the earth shall see the salvation of our God.}%
\verse{Depart, depart, go out from there! You must not touch any unclean thing. Go out from the midst of it, keep clean, you who carry the vessels of Adonai.}%
\verse{For you shall not go out in haste, and you shall not go in flight, for Adonai is going before you, and your rear guard is the God of Israel.}%
\verseWithHeading{The Servant’s Suffering and Exaltation}{Look, my servant shall achieve success; he shall be exalted, and he shall be lifted up, and he shall be very high.}%
\verse{Just as many were appalled at you — such was his appearance beyond human disfigurement, and his form beyond the sons of mankind —}%
\verse{so he shall sprinkle many nations; because of him, kings shall shut their mouths.\lebnote{Hebrew “mouth”} For they shall see what has not been told them, and they shall consider with full attention what they have not heard.}%
\end{biblechapter}%
\begin{biblechapter}% Isaiah 53
\verse{Who has believed our message, and to whom has the arm of Adonai been revealed?}%
\verse{For\lnBYP{} he went\lebnote{Or “grew”} up like a shoot before him, and like a root from dry ground. He had no form and no majesty that\lnBYQ{} we should see him, and no appearance that\lnBYQ{} we should take pleasure in him.}%
\verse{He was despised and rejected by men, a man of suffering, and acquainted with\lebnote{Or “knowledgeable of”} sickness, and like one from whom others hide their faces,\lebnote{“hiding of face from him”} he was despised, and we did not hold him in high regard.}%
\verse{However, he was the one who lifted up our sicknesses, and he carried our pain, yet\lnBYQ{} we ourselves assumed him stricken, struck down by God and afflicted.}%
\verse{But\lnBYP{} he was pierced\lebnote{Or “wounded”} because of our transgressions, crushed because of our iniquities; the chastisement for\lebnote{Or “of”} our peace\lebnote{Or “healing”} was upon him, and by his wounds\lebnote{Hebrew “wound”} we were healed.\lebnote{“it was healed for us”}}%
\verse{All of us have wandered about like sheep; we each have turned to his own way; and Adonai let fall on him the iniquity of us all.}%
\verse{He was oppressed and afflicted, yet\lnBYQ{} he did not open his mouth; he was brought like a lamb to the slaughter, and like a sheep is dumb before its shearers, so\lnBYQ{} he did not open his mouth.}%
\verse{He was taken by restraint of justice, and who concerned himself with his generation? For he was cut off from the land of the living; he received a blow because of the transgression of my people.}%
\verse{He made\lebnote{Or “gave”} his grave with the wicked, and with the rich in his death, although he had done no violence, and there was no deceit in his mouth.}%
\verse{Yet\lnBYP{} Adonai was pleased to crush him; he made him sick.\lebnote{“he made sick”} If she\lebnote{Or “you”} places\lebnote{Or “makes”} his life a guilt offering,\lebnote{Or “compensation”} he will see offspring. He will prolong days, and the will of Adonai will succeed in his hand.}%
\verse{From the trouble of his life\lebnote{Or “soul”} he will see;\lebnote{Dead Sea Scrolls add “light”} he will be satisfied. In his knowledge, the righteous one, my servant, shall declare many righteous,\lebnote{Or “right”} and he is the one who will bear their iniquities.}%
\verse{Therefore, I will divide to him a portion among the many,\lebnote{Or “great”} and with the strong ones he will divide bounty, because\lebnote{“Instead of that”} he poured his life out to death and was counted with the transgressors; and\lebnote{Or “yet”} he was the one who bore the sin of many and will intercede for the transgressors.}%
\end{biblechapter}%
\begin{biblechapter}% Isaiah 54
\verseWithHeading{The Fertile Wife of Adonai}{“Sing for joy, barren woman; who has not borne! Burst forth into rejoicing and rejoice, she who has not been in labor! For the children of the desolate woman are more than the children of the married woman,” says Adonai.}%
\verse{“Enlarge the site of your tent, and let them stretch out the tent curtains of your dwelling place. You must not spare; make your tent cords long and strengthen your pegs,}%
\verse{for you will spread out to the right and to the left. And your descendants\lebnote{Hebrew “descendant”} will be heir to the nations, and they will inhabit desolate towns.}%
\verse{You must not fear, for you will not be ashamed, and you must not be confounded, for you will not feel abashed, for you will forget the shame of your youth, and you will no longer remember the disgrace of your widowhood.}%
\verse{For your husband\lnBYR{} is your maker,\lnBYR{} his name is Adonai of hosts; and your redeemer is the holy one of Israel, he is called the God of all of the earth.}%
\verse{For Adonai has called you like a wife forsaken and hurt of spirit, like\lnBYS{} the wife of childhood when she is rejected, says your God.}%
\verse{I abandoned you for a short\lebnote{“in a small”} moment, but\lnBYS{} I will gather you with great compassion.}%
\verse{I hid my face from you for a moment, in the flowing of anger, but\lnBYS{} I will have compassion on you with everlasting faithfulness” says your redeemer, Adonai.}%
\verse{“For this is like the waters of Noah to me, when I swore that the waters of Noah would never again pass\lebnote{“from the passing over of the waters of Noah again”} over the earth, so I swore that I would not be\lebnote{“from being “} angry at you and rebuke you.}%
\verse{For the mountains may depart, and the hills may sway, but\lnBYS{} my faithfulness shall not depart from you, and my covenant of peace shall not sway,” says Adonai, who has compassion on you,}%
\verse{“O afflicted one, driven away, who is not consoled. Look! I am about to set your stones in hard mortar, and I will lay your foundation with sapphires.}%
\verse{And I will make\lebnote{“place”} your battlements of ruby, and your gates of stones of beryl, and all your wall of precious stones.}%
\verse{And all your children shall be pupils of Adonai, and the peace of your children shall be great.}%
\verse{In righteousness you shall be established. Be far from oppression, for you shall not fear, and from terror, for it shall not come near you.}%
\verse{If indeed one attacks, it is not from me; whoever attacks you shall fall because of you.}%
\verse{Look! I myself have created the craftsman who blows\lebnote{Or “blowing”} the fire of coals, and who produces\lebnote{Or “producing”} a weapon for his work; also\lnBYS{} I myself have created the destroyer to destroy.}%
\verse{Every weapon formed against you shall not succeed, and you shall declare guilty every tongue that rises against you for judgment. This is the inheritance of the servants of Adonai, and their legal right\lebnote{“justice”} from me,” declares\lebnote{“declaration of”} Adonai.}%
\end{biblechapter}%
\begin{biblechapter}% Isaiah 55
\verseWithHeading{Invitation to True Rewards}{“Ho! Everyone thirsty, come to the waters! And whoever has no money, come, buy and eat, and come, buy without money, wine and milk without price!}%
\verse{Why do you weigh out money for what is not food, and your labor for what cannot satisfy?\lebnote{“\textit{what is} not for satiation”} Listen carefully to me, and eat what is good, and let your soul take pleasure in rich\lebnote{“fatty”} food.}%
\verse{Extend your ear, and come to me! Listen so that\lnBYT{} your soul may live, and I will make\lebnote{“cut”} an everlasting covenant with\lebnote{or “for”} you, the enduring proofs of the mercies shown to\lebnote{“mercy of”} David.}%
\verse{Look! I made\lebnote{Or “gave”} him a witness to the peoples, a leader and a commander for the peoples.}%
\verse{Look! You shall call a nation that you do not know, and a nation that does not know you shall run to you, because of Adonai your God, and the holy one of Israel, for he has glorified you.”}%
\verse{Seek Adonai while he lets himself be found;\lebnote{“in his letting himself be found”} call him while he is\lebnote{“in his being”} near.}%
\verse{Let the wicked forsake his way, and the man of sin his thoughts. And let him return to Adonai, that\lnBYT{} he may take pity on him, and to our God, for he will forgive manifold.\lebnote{“make numerous to”}}%
\verse{“For my thoughts are not your thoughts, and your ways are not my ways,” declares\lebnote{“declaration of”} Adonai.}%
\verse{“For as the heavens are higher\lnBYU{} than the earth, so my ways are higher\lnBYU{} than your ways, and my thoughts than your thoughts.}%
\verse{For just as the rain and the snow come down from heaven, and they do not return there except they have watered the earth thoroughly and cause it to bring forth and sprout, and give seed to the sower and bread to the eater,}%
\verse{so shall be my word that goes out from my mouth. It shall not return to me without success, but shall accomplish\lebnote{Or “do”} what I desire and be successful in the thing for which I sent it.}%
\verse{For you shall go out in joy, and you shall be led in peace. The mountains and the hills shall break forth before you, rejoicing and all the trees of the field shall clap hands.\lebnote{Hebrew “hand”}}%
\verse{Instead of the thorn bush, the juniper shall go up; instead of the brier, the myrtle shall go up, and it shall serve as a memorial to Adonai, for an everlasting sign that shall not be cut off.”}%
\end{biblechapter}%
\begin{biblechapter}% Isaiah 56
\verseWithHeading{Salvation for All Who Obey}{Thus says Adonai: “Observe justice and do righteousness, for my salvation is close to coming, and my justice to being revealed.}%
\verse{Happy is the man who does this, and the son of humankind who keeps hold of it, who keeps the Sabbath so as not to profane\lnBYV{} it, and who keeps his hand from doing any\lebnote{Or “all”} evil.”}%
\verse{And do not let the foreigner\lebnote{“son of the foreign country”} who joins himself to Adonai say, “Surely Adonai will separate me from his people.” And do not let the eunuch say, “Look! I am a dry tree!”}%
\verse{For thus says Adonai, “To the eunuchs who keep my Sabbaths, and choose that in which I delight, and who keep hold of my covenant.}%
\verse{And I will give them a monument and a name in my house and within my walls, better than sons and daughters; I will give him an everlasting name that shall not be cut off.}%
\verse{And the foreigners\lebnote{“sons of the foreign country”} who join themselves to Adonai to serve him and to love the name of Adonai, to become his servants, every one who keeps the Sabbath, so as not to profane\lnBYV{} it, and those who keep hold of my covenant,}%
\verse{I will bring them to my holy mountain;\lebnote{“the mountain of my holiness”} I will make them merry in my house of prayer. their burnt offerings and their sacrifices will be accepted\lebnote{“for acceptance”} on my altar, for my house shall be called a house of prayer for all peoples,”}%
\verse{declares\lebnote{“declaration of”} the Lord Adonai, who gathers\lebnote{Or “gathering”} the scattered ones of Israel (still I will gather to\lebnote{Or “upon”} him, to his gathered ones).}%
\verseWithHeading{Wicked Leaders}{All wild animals in the field, come, to devour,\lebnote{Or “eat”} all wild animals in the forest!}%
\verse{His watchmen are blind, none of them know. They are all dumb dogs; they are unable to bark, panting, lying down, loving to slumber.}%
\verse{And the dogs have a greedy appetite;\lebnote{“are strong of soul”} they are never satisfied.\lebnote{“do not know satiation”} And they are the shepherds! They do not have\lebnote{Or “know”} understanding. They all turn to their own way, each one for his own gain, every one of them.\lebnote{“from his end “}}%
\verse{“Come, let me take wine, and let us carouse with intoxicating drink, and tomorrow will be like today, very great indeed.”\lebnote{“excessively”}}%
\end{biblechapter}%
\begin{biblechapter}% Isaiah 57
\verseWithHeading{Idolatry is Adultery}{The righteous one perishes, and there is no one who takes\lebnote{Or “sets”} it to heart. And men of faithfulness are gathered, while there is no one who understands, for the righteous is gathered from the presence of wickedness.}%
\verse{he enters into peace; they will rest on their beds, walking straight ahead of him.}%
\verse{“But\lnBYW{} you, come near here, you children of a soothsayer, offspring of an adulterer and she who commits fornication.}%
\verse{At whom do you make fun? At whom do you open\lebnote{“widen”} your mouth and stick out\lebnote{“make long”} your tongue? Are you not children of transgression, offspring of deception,}%
\verse{who burn with lust among the oaks, under every leafy tree, who slaughter children in the valleys, under the clefts of the rocks?}%
\verse{Your portion is among the smooth stones of the valley; they, they are your lot; indeed, to them you have poured out a drink offering, you have brought a food offering. Shall I relent concerning these things?}%
\verse{You have set your bed upon a high and lofty mountain; indeed, you went up there to slaughter sacrifice.}%
\verse{And you have set your symbol behind the door and the doorpost; for you depart\lebnote{The reading tradition (\textit{Qere}) has “uncover,” which makes no sense in context, but with different vowels it would be “depart”} from me, and you go up; you make your bed wide, and you make a deal with\lebnote{“cut off for yourself from”} them, you have loved their bed; you have seen their genitals.\lebnote{“hand”}}%
\verse{And you climbed down to the king with oil, and you made your perfumes numerous, and you sent your envoys far away,\lebnote{“to from far”} and you sent down deep\lebnote{“humiliated”} to Sheol.}%
\verse{You grow weary by the greatness of your way, but you did not say, ‘Despairing!’ You found the renewal\lebnote{“life”} of your strength,\lebnote{“hand;” compare the “hand” in verse 8} therefore you do not grow weak.}%
\verse{And of whom were you afraid and feared, that you deceived and did not remember me? Did you not place it on your heart? Have I not been silent, even from long ago, and so you do not fear me?}%
\verse{I myself will declare your righteousness and your works, but\lnBYX{} they will not benefit you.}%
\verse{When you cry,\lebnote{“your crying”} let your collection deliver you, and the wind will carry all of them away; a breath will take them away. But\lnBYW{} he who takes refuge in me shall take possession of the land, and he shall inherit my holy mountain.”\lebnote{“mountain of holiness “}}%
\verseWithHeading{Peace for the Contrite}{And one shall say, “Build up, build up! Clear the way! Remove the obstacles\lebnote{Hebrew “obstacle”} from the way of my people!”}%
\verse{For thus says the high and lofty one who resides forever, and whose name is holy: “I reside in a high and holy place, and with the contrite\lnBYY{} and humble of spirit, to revive the spirit of the humble, and to revive the heart of the contrite.\lnBYY{}}%
\verse{For I will not attack forever, and I will not be angry forever, for the spirit will grow faint before me, and the breaths that I myself I have made.}%
\verse{I was angry because of his sin of gain, and I struck him; I hid and I was angry, but\lnBYX{} he walked apostate, in the ways\lebnote{Hebrew “way”} of his heart.}%
\verse{I have seen his ways, but\lnBYX{} I will heal him; and I will lead him and give him and his mourners comfort as a recompense,}%
\verse{creating fruit of lips. Peace, peace to the far and near,” says Adonai, “and I will heal him.}%
\verse{But\lnBYW{} the wicked are like the churning sea, that is not able to keep quiet, and its waters toss up mire and mud.}%
\verse{There is no peace,” says my God, “for the wicked.”}%
\end{biblechapter}%
\begin{biblechapter}% Isaiah 58
\verseWithHeading{True Fasts and Sabbaths}{“Call with the throat; you must not keep back! lift up your voice like a trumpet, and declare\lebnote{Or “announce”} to my people their rebellion, and to the house of Jacob their sins.}%
\verse{Yet\lebnote{Or “And”} they seek me day by day, and they desire the knowledge of my ways like a nation that practiced\lebnote{“did”} righteousness, and had not forsaken the judgment of its God; they ask me for righteous judgments,\lebnote{“judgments of righteousness”} they desire the closeness of God.}%
\verse{‘Why do we fast, and you do not see it? We humiliate our soul, and you do not notice it?’ Look! You find delight on the day of your fast, and you oppress all your workers!}%
\verse{Look! You fast to quarrel and strife, and to strike with a wicked fist.\lebnote{“fist of wickedness”} You shall not fast as you do today,\lebnote{“the day”} to make your voice heard\lebnote{“cause to hear your voice”} on the height.}%
\verse{Is the fast I choose like this, a day for humankind to humiliate himself\lebnote{“his soul”}? To bow his head like a reed, and make\lebnote{“he makes”} his bed on sackcloth and ashes; you call this a fast and a day of pleasure to Adonai?}%
\verse{Is this not the fast I choose: to release the bonds of injustice, to untie the ropes of the yoke, and to let the oppressed go free, and tear\lebnote{“you must tear”} every yoke to pieces?}%
\verse{Is it not to break your bread for the hungry? You must bring home\lebnote{“a house”} the poor, the homeless. When you see the naked, you must cover him, and you must not hide yourself from your relatives.\lebnote{“flesh”}}%
\verse{Then your light shall break forth like the dawn, and your healing shall grow quickly. And your salvation shall go before you; the glory of Adonai will be your rear guard.}%
\verse{Then you shall call, and Adonai himself will answer. You shall cry for help, and he will say, ‘Here I am!’ If you remove from among you the yoke, the finger-pointing\lebnote{“stretching out of \textit{the} finger”} and evil speech,\lebnote{“the speaking of evil”}}%
\verse{if\lnBYZ{} you offer your soul to the hungry, and you satisfy the appetite of the afflicted, then\lnBYZ{} your light shall rise in the darkness, and your darkness will be like noon.}%
\verse{And Adonai will lead you continually, and satisfy your soul in a barren land, and he will strengthen your bones, and you shall be like a well-watered garden, and like a spring of water whose water does not fail.}%
\verse{And they shall rebuild ancient ruins from you; you shall erect the foundations of many generations,\lebnote{“generation and generation”} and you shall be called\lebnote{“shall be called to you”} the bricklayer of the breach, the restorer of paths to live in.}%
\verse{If you hold your foot back from the Sabbath, from doing your affairs on my holy day,\lebnote{“the day of my holiness”} if\lnBYZ{} you call the Sabbath a pleasure, the holy day of Adonai honorable, if\lebnote{“and”} you honor him more than\lebnote{“from”} doing your ways, than finding your affairs\lebnote{Hebrew “affair”} and speaking a word,}%
\verse{then you shall take your pleasure in\lebnote{Or “on”} Adonai, and I will make you ride upon the heights of the earth, and I will feed you the heritage of Jacob your ancestor,\lebnote{Or “father”} for the mouth of Adonai has spoken.”}%
\end{biblechapter}%
\begin{biblechapter}% Isaiah 59
\verseWithHeading{Rampant Transgression}{Look! The hand of Adonai is not too short to save,\lebnote{“from saving”} and his ear is not too dull to hear.\lebnote{“from hearing”}}%
\verse{Rather, your iniquities have been barriers\lebnote{“dividers”} between you and your God, and your sins have hidden his face from you, from hearing.}%
\verse{For your hands are defiled with blood, and your fingers with iniquity. Your lips have spoken lies, your tongue speaks wickedness.}%
\verse{There is nobody who pleads with\lebnote{“calling in”} justice, and there is nobody who judges with honesty. They rely on nothing and speak vanity. They conceive trouble and beget iniquity;}%
\verse{they hatch viper eggs, and they weave a spider web. One who eats their eggs dies, and that which is pressed is hatched as a serpent.}%
\verse{Their webs cannot become clothing, and they cannot cover themselves with their works. Their works are works of iniquity, and deeds\lebnote{Hebrew “deed”} of violence are in their hands.}%
\verse{Their feet run to evil, and they hasten to shed innocent blood. Their thoughts are thoughts of iniquity; devastation and destruction are in their highways.}%
\verse{They do not know the way of peace, and there is no justice in their firm paths. They have made their paths crooked for themselves; everyone who walks\lebnote{“walking”} in it knows no peace.}%
\verse{Therefore justice is far from us, and righteousness does not reach us. We wait for light, but\lnBZA{} look! there is darkness; for brightness, but we walk in darkness.}%
\verse{We grope like the blind along a wall, and we grope as without\lebnote{“like there are no”} eyes. We stumble at noon as in the twilight; among the strong we are like the dead.}%
\verse{We all groan like bears, and we coo mutteringly like doves. We wait for justice, but\lnBZA{} there is none; for salvation, but it is far from us.}%
\verse{For our transgressions are numerous before you, and our sins testify\lebnote{Or “answers”} against us. Indeed, our transgressions are with us, and we know our iniquities:}%
\verse{transgressing and denying Adonai, and turning away from following\lebnote{“back from behind”} our God; speaking oppression and falsehood, conceiving and uttering words of deception from the heart.}%
\verse{And justice is pushed back, and righteousness stands afar; for truth stumbles in the public square, and straightforwardness is unable to enter,}%
\verse{and truth is missing, and he who turns aside from evil is plundered. And Adonai saw, and it was displeasing in his eyes that there was no justice}%
\verse{And he saw that there was no man, and he was appalled that there was no one who intercedes, so\lnBZA{} his arm came to assist him, and his righteousness was what\lebnote{“it”} sustained him.}%
\verse{And he put on righteousness like a breastplate, and a helmet of salvation on his head, and he put on garments of vengeance for clothing, and he wrapped himself in zeal as in a robe.}%
\verse{According to deeds, so he will repay; wrath to his enemies, requital to those who are his enemies.\lebnote{“hostile”} He will repay requital to the coastlands.}%
\verse{So\lebnote{Or “And”} they shall fear the name of Adonai from the west, and his glory from the sunrise, for he will come like a narrow stream; the wind of Adonai drives it on.}%
\verse{“And a redeemer will come to Zion, to those in Jacob who turn away from transgression,” declares\lebnote{“declaration of”} Adonai.}%
\verse{“And as for me, this is my covenant with them, says Adonai: my spirit that is upon you, and my words that I have placed in your mouth shall not depart from your mouth, or\lnBZA{} from the mouths\lnBZB{} of your children,\lebnote{“offspring”} or\lnBZA{} from the mouths\lnBZB{} of your children’s children,”\lebnote{“the offspring of your offspring”} says Adonai, “from now on and forever.”}%
\end{biblechapter}%
\begin{biblechapter}% Isaiah 60
\verseWithHeading{Zion Glorified}{“Arise, shine! For your light has come, and the glory of Adonai has risen on you.}%
\verse{For look! darkness shall cover the earth, and thick darkness the peoples, but\lnBZC{} Adonai will rise on you, and his glory will appear over you.}%
\verse{And nations shall come to your light, and kings to the bright light of your sunrise.}%
\verse{Lift up your eyes all around and see! All of them gather; they come to you. Your sons shall come from afar, and your daughters shall be looked after on the hip.}%
\verse{Then you shall see and you shall be radiant; and your heart shall tremble and open itself wide, because the abundance of the sea shall fall upon you; the wealth of the nations shall come to you.}%
\verse{A multitude of camels shall cover you, the young male camels of Midian and Ephah. All those from Sheba shall come; they shall bring gold and frankincense, and they shall proclaim the praise of Adonai.}%
\verse{All the flocks of Kedar shall be gathered to you; the rams of Nebaioth shall serve you. They shall present a sacrifice for favor on my altar, and I will glorify my honorable house.\lebnote{“house of my honor”}}%
\verse{Who are these? They fly like a cloud, and like doves to their coops.}%
\verse{Because the coastlands wait for me, and the ships of Tarshish are first to bring your children from afar, their silver and gold with them, for the name of Adonai your God, and for the holy one of Israel, because he has glorified you.}%
\verse{And foreigners\lebnote{“sons of a foreign country”} shall build your walls, and their kings shall serve you, for in my anger I struck you, but\lnBZC{} in my favor I have taken pity on you.}%
\verse{And your gates shall continually be open, day and night they shall not be shut, to bring you the wealth of nations, and their kings shall be led.}%
\verse{For the nation and the kingdom that will not serve you shall perish, and the nations shall be utterly devastated.}%
\verse{The glory of Lebanon shall come to you; the cypress, the plane, and the pine together, to glorify the place of my sanctuary, and I will do honor to the place of my feet.}%
\verse{And the children of those who oppressed you shall come to you bending low, and all those who treated you disrespectfully shall bow down at the soles of your feet. And they shall call you the city of Adonai, Zion of the holy one of Israel.}%
\verse{Instead of you being forsaken and hated with no one passing through,\lebnote{Or “and there is not one who passes through”} I will make\lebnote{“place”} you an everlasting object of pride,\lebnote{“pride”} a joy of coming generations.\lebnote{“generation and generation”}}%
\verse{And you shall suck the milk of nations, and suck the breast of kings, and you shall know that I am Adonai your Savior, and your Redeemer, the Strong One of Jacob.}%
\verse{Instead of bronze I will bring gold, and instead of iron I will bring silver, and instead of wood, bronze, and instead of stones, iron. And I will appoint\lebnote{“put”} peace as your overseer, and righteousness as your ruling body.\lebnote{“they who force you to work”}}%
\verse{Violence shall no longer be heard in your land; devastation or\lnBZC{} destruction on your borders. And you shall call your walls Salvation, and your gates, Praise.}%
\verse{The sun shall no longer be your light by day, and for bright light the moon shall not give you light, but\lnBZC{} Adonai will be your everlasting light, and your God your glory.}%
\verse{Your sun shall no longer go down, and your moon shall not wane,\lebnote{“be taken away”} for Adonai himself will be your everlasting light, and your days of mourning shall come to an end.}%
\verse{And all your people shall be righteous; they shall take possession of the land forever, the shoot of his\lebnote{The reading tradition (\textit{Qere}) has “my”} planting, the work of my hands, to show my glory.}%
\verse{The small one shall become a tribe, and the smallest one a mighty nation. I am Adonai; I will hasten it in its time.”}%
\end{biblechapter}%
\begin{biblechapter}% Isaiah 61
\verseWithHeading{The Year of Adonai’s Favor}{The Spirit of the Lord Adonai is upon me, because Adonai has anointed me, he has sent me to bring good news to the oppressed, to bind up the brokenhearted,\lebnote{“those broken of heart”} to proclaim\lnBZD{} release to the captives and liberation to those who are bound,}%
\verse{to proclaim\lnBZD{} the year of Adonai’s favor, and our God’s day of vengeance, to comfort all those in mourning,}%
\verse{to give\lebnote{“place”} for those in mourning in Zion, to give them a head wrap instead of ashes, the oil of joy instead of mourning, a garment of praise instead of a faint spirit. And they will be called\lebnote{“it will be called to them”} oaks of righteousness, the planting of Adonai, to show his glory.}%
\verse{And they shall build the ancient ruins, they shall erect the former deserted places. And they shall restore the devastated cities,\lebnote{“cities of devastation”} the deserted places of many generations.\lebnote{“generation and generation”}}%
\verse{And strangers shall stand and feed your flocks,\lebnote{Hebrew “flock”} and foreigners\lebnote{“sons of a foreign country”} shall be your farmers and vinedressers.}%
\verse{But\lebnote{Or “And”} you shall be called the priests of Adonai, you will be called\lebnote{“it shall be said of you”} servers of our God. You shall eat the wealth of the nations, and you shall boast in their riches.}%
\verse{Instead of your shame, a double portion, and instead of insult, they will rejoice over their portion. Therefore they will take a double portion in their land; they shall possess everlasting joy.\lebnote{“possession of joy everlasting shall be for them”}}%
\verse{For I, Adonai, love justice, hate robbery and\lebnote{Or “in”} injustice, and I will faithfully\lebnote{“in \textit{the} faithfulness”} give their reward, and I will make\lebnote{“cut”} an everlasting covenant with\lebnote{“to”} them.}%
\verse{And their descendants\lnBZE{} will be known among the nations, and their offspring in the midst of the peoples. All those who see them shall recognize them, that they are descendants\lnBZE{} whom Adonai has blessed.}%
\verse{I will rejoice greatly in Adonai; my being shall shout in exultation in my God. For he has clothed me with garments of salvation, he has covered me with the robe of righteousness, as a bridegroom adorns himself with a head wrap like a priest, and as a bride adorns herself with her jewelry.}%
\verse{For as the earth produces\lebnote{“brings out”} its sprout, and as a garden makes its plants sprout, so the Lord Adonai will make righteousness sprout, and praise before all the nations.}%
\end{biblechapter}%
\begin{biblechapter}% Isaiah 62
\verseWithHeading{Zion’s New Identity}{For the sake of Zion I will not be silent, and for the sake of Jerusalem I will not maintain a quiet attitude, until her righteousness goes out like the bright light, and her salvation burns like a torch.}%
\verse{And the nations shall see your righteousness, and all the kings your glory, and you will be called\lebnote{“it shall be called to you”} a new name that the mouth of Adonai will designate.}%
\verse{And you shall be a crown of beauty in the hand of Adonai, and a headband of royalty in the hand of your God.}%
\verse{It shall no longer be said of you, “Forsaken,” and it shall no longer be said of your land, “Desolation!” but you will\lebnote{“to you it shall”} be called “My Delight Is In Her,” and your land, “Married,” for Adonai delights in you, and your land shall be married.}%
\verse{For as a young man marries a virgin, so shall your sons marry you, and as is the joy of the bridegroom over the bride, so shall your God rejoice over you.}%
\verse{I have appointed watchmen upon your walls, Jerusalem; all day and all night they shall never be silent. You who profess Adonai have no rest.}%
\verse{And you must not give him rest until he establishes, and until he makes\lebnote{“places”} Jerusalem an object of praise\lebnote{“a praise”} in the earth.}%
\verse{Adonai has sworn by his right hand, and by his mighty arm,\lebnote{“the arm of his might”} “Surely I will never again give your grain as food to your enemies,\lebnote{“those who are hostile to you”} and surely foreigners\lebnote{“sons of a foreign country”} shall not drink your new wine at which you have labored.”}%
\verse{But those who gather it shall eat it, and they shall praise Adonai, and those who gather it shall drink it in my holy courts.\lebnote{“the courts of my holiness”}}%
\verse{Pass through, pass through the gates! Make the way clear for\lebnote{“of”} the people! Pile up, pile up the highway; clear it of stones!\lebnote{Hebrew “stone”} Lift up an ensign over the peoples!}%
\verse{Look! Adonai has proclaimed to the end of the earth: Say to daughter Zion, “Look, your salvation is coming! Look, his reward is with him, and his reward before him.}%
\verse{And they shall call them “The Holy People,\lebnote{“People of the Holiness”} The Redeemed Of Adonai,” and you\lebnote{“to you it”} shall be called “Sought After, A City Is Not Forsaken.”}%
\end{biblechapter}%
\begin{biblechapter}% Isaiah 63
\verseWithHeading{Adonai’s Day of Vengeance}{Who is this, coming from Edom, from Bozrah in bright red garments? Who is this honored in his garment, lying down in his great strength? “It is I, speaking in justice, mighty\lebnote{“great”} to save!”}%
\verse{Why are your garments\lebnote{Hebrew “garment”} red, and your garments like he who treads in the winepress?}%
\verse{“I have trodden the winepress alone, and there was no man from the peoples with me. And I trod them in my anger, and I trampled them in my wrath, and spattered their juice on my garments, and stained all my clothing,}%
\verse{for the day of vengeance was in my heart, and the year of my blood-vengeance has come!}%
\verse{And I looked, but\lnBZF{} there was no helper, and I was appalled, but\lnBZF{} there was no one who sustains, so\lnBZF{} my arm came to assist me, and my wrath was what\lebnote{“it”} sustained me.}%
\verse{And I trampled peoples in my anger, and I made them drunk in my wrath, and I brought their juice down to the earth.”}%
\verseWithHeading{Remembering Adonai’s Mercy}{I will mention the loyal love of Adonai, the praises of Adonai, according to all that Adonai has done for us, and the greatness of goodness to the house of Israel that he has done to them according to his mercy and the abundance of his loyal love.}%
\verse{And he said, “Surely my people are children; they will not break faith.” And he became a Savior to them.}%
\verse{In all their distress, there was no distress,\lebnote{The reading tradition (\textit{Qere}) has “he had distress”} and the messenger of his presence\lebnote{“face”} saved them, in his love and compassion he himself redeemed them, and he lifted them up, and he supported them all the days of old.}%
\verse{But\lnBZG{} they were the ones who\lebnote{“they”} rebelled, and they grieved his Holy Spirit,\lnBZH{} so\lnBZF{} he became an enemy to them;\lebnote{“was changed to them to one who is hostile”} he himself fought against them.}%
\verse{Then\lnBZG{} his people remembered the days of old, of Moses. Where is the one who led up them from the sea with the shepherds of his flock? Where is the one who puts his Holy Spirit\lnBZH{} inside him,}%
\verse{who made\lebnote{“bringing”} his magnificent arm move at\lebnote{“to”} the right hand of Moses, who divided\lebnote{“dividing”} the waters before them,\lebnote{“from their face”} to make an everlasting name for himself,}%
\verse{who led\lebnote{“leading”} them through the depths? They did not stumble like a horse in the desert;}%
\verse{like cattle in the valley that goes down, the Spirit of Adonai gave him rest,\lebnote{“caused to rest”} so you lead your people to make a magnificent name for yourself.}%
\verseWithHeading{Prayer for Mercy}{Look from heaven, and see from the lofty residence of your holiness and glory. Where are your zeal and strength? Your compassion\lebnote{“The agitation of your intestines”} and mercy to me hold themselves back.}%
\verse{For you are our father, although Abraham does not know us, and Israel does not acknowledge us. You, Adonai are our father, Our Redeemer from of old is your name.}%
\verse{Why do you make us wander, Adonai? You harden our heart from your ways so that we do not fear\lebnote{“from the fear of”} you. Turn back for the sake of your servants, the tribes of your inheritance.}%
\verse{Your holy people\lebnote{“The people of your holiness”} took possession for a little while; our adversaries have trampled down your sanctuary.}%
\verse{We have been since antiquity; you did not rule them; they were not called by your name.\lebnote{“your name was not called over them”}}%
\end{biblechapter}%
\begin{biblechapter}% Isaiah 64
\verse{\lebnote{Isaiah 64:1 in the English Bible begins here. In the Hebrew Bible, 63:19 continues} Would that you would tear the heavens and come down; the mountains would quake before you,}%
\verse{\lebnote{Isaiah 64:2–12 in the English Bible is 64:1–11 in the Hebrew Bible} as fire kindles brushwood, the fire causes water to boil, to make your name known to your adversaries, that the nations might tremble from your presence.\lnBZI{}}%
\verse{When you did terrible deeds which we did not expect,\lebnote{“await”} you came down; the mountains quaked because of your presence.\lnBZI{}}%
\verse{And since ancient times they have not heard, have not listened, no eye has seen a God except you; he acts\lebnote{“makes”} for the one who waits for him.}%
\verse{You meet with the one who rejoices, one who does righteousness. In your ways they remember you. Look! You were angry and we sinned against them in ancient times and we were saved.}%
\verse{And we all have become like the unclean, and all our deeds of justice like a menstrual cloth, And we all wither like a leaf, and our iniquities take us away like the wind.}%
\verse{And there is no one who calls on your name, who pulls himself up to keep hold of you, for you have hidden your face from us,\lebnote{Or “him”} and melted us into the hand of our iniquity.}%
\verse{Yet now Adonai, you are our father; we are the clay and you are our potter,\lebnote{“one who fashions”} and we all are the work of your hand.}%
\verse{You must not be exceedingly angry, Adonai, and you must not remember iniquity forever! Look! Behold, now! We all are your people!}%
\verse{Your holy cities\lebnote{“The cities of your holiness”} have become a wilderness; Zion has become a wilderness, Jerusalem a desolation.}%
\verse{Our holy and beautiful temple,\lebnote{“the house of our holiness and our beauty”} where our ancestors\lebnote{Or “fathers”} praised you has been burned\lebnote{“been for burning”} by fire, and all our precious objects have become ruins.\lebnote{Hebrew “ruin”}}%
\verse{Will you control yourself because of these, Adonai? Will you be silent? And will you oppress us exceedingly?}%
\end{biblechapter}%
\begin{biblechapter}% Isaiah 65
\verseWithHeading{Judgment and Salvation}{“I let myself be sought by those who did not ask; I let myself be found by those who did not seek me. I said, ‘Here I am; here I am!’ to a nation that did not call on my name;}%
\verse{I spread out my hands all day to a stubborn people, those who walk after their thoughts in the way that is not good,}%
\verse{the people who provoke\lebnote{“provoking”} me to anger continually to my face, slaughtering for sacrifices in the garden, and making smoke offerings on bricks,}%
\verse{who sit\lebnote{“sitting”} in graves and spend the night in secret places,\lebnote{“watches”} who eat\lebnote{“eating”} the flesh of swine with\lnBZJ{} a fragment\lebnote{The reading tradition (\textit{Qere}) has “broth”} of impurity in their vessels,}%
\verse{who say,\lebnote{“saying”} “Keep to yourself!\lebnote{“Come near to you”} You must not come near me, for I am too holy for you!” These are a smoke in my nostrils,\lebnote{“nose”} a fire burning all day.}%
\verse{Look! It is written before me: I will not be silent, but I will repay; and I will repay in\lebnote{Or “on”} the fold of their garment}%
\verse{your iniquities and the iniquities of your ancestors\lebnote{Or “fathers”} together, says Adonai, because they made smoke offerings on the mountains and they taunted me on the hills, I will measure their punishment from the beginning into the fold of their garment.”}%
\verse{Thus says Adonai: “Just as the new wine is found in the cluster, and they say ‘You must not destroy it, for there is a blessing in it,’ so I will do for the sake of my servants by\lebnote{“to”} not destroying everyone.\lebnote{“all”}}%
\verse{And I will bring descendant s\lebnote{Hebrew “descendant”} out from Jacob, and a people\lebnote{“man”} from Judah to take possession of my mountain, and my chosen ones shall inherit it, and my servants shall settle there.}%
\verse{And Sharon shall become a pasture for\lnBZK{} flocks,\lebnote{Hebrew “flock”} and the valley of Achor a resting place for\lebnote{Or “of”} herds\lebnote{Hebrew “herd”} for my people who have sought me.}%
\verse{But\lnBZJ{} you who forsake\lebnote{“forsaking”} Adonai, forgetting my holy mountain,\lnBZL{} who set a table for Fortune, and who pour out\lebnote{“fill”} a jug of mixed wine for Destiny,}%
\verse{and I will remit you to the sword, and all of you shall bow down to the slaughter, because I called, but\lnBZJ{} you did not answer; I spoke, but\lnBZJ{} you did not listen, but\lnBZJ{} you did the evil in my eyes, and you chose that in which I do not delight.”}%
\verse{Therefore thus says the Lord Adonai: “Look! My servants shall eat but\lnBZJ{} you, you shall be hungry. Look! My servants shall drink but\lnBZJ{} you, you shall be thirsty. Look! My servants shall rejoice but\lnBZJ{} you, you shall be ashamed.}%
\verse{Look! My servants shall shout for joy,\lebnote{“from goodness of heart”} but\lnBZJ{} you, you shall cry out for pain\lebnote{“from pain of heart”} and howl for sadness.\lebnote{“from a breaking of \textit{the} spirit”}}%
\verse{And you shall leave your name to my chosen ones as a curse, and the Lord Adonai will kill you, and he will give\lebnote{“call”} his servants another name.}%
\verse{Whoever blesses himself in the land shall bless himself by the God of trustworthiness, and the one who swears an oath in the land shall swear by the God of trustworthiness, because the former troubles are forgotten, and they are hidden from my eyes.}%
\verseWithHeading{New Creation}{For look! I am about to create new heavens and a new earth, and the former things shall not be remembered, and they shall not come to mind.\lebnote{“go up to heart”}}%
\verse{But rejoice and shout in exultation forever and ever over what I am about to create! For look! I am about to create Jerusalem as a source of rejoicing, and her people as a source of joy.}%
\verse{And I will shout in exultation over Jerusalem, and I will rejoice over my people, and the sound of weeping shall no longer be heard in it, or\lnBZJ{} the sound of a cry for help.}%
\verse{There will no longer be a nursing infant who lives only a few\lnBZK{} days, or\lnBZJ{} an old man who does not fill his days, for the boy will die a hundred years old,\lnBZM{} and the one who fails to reach\lebnote{“misses”} a hundred years\lnBZM{} will be considered\lebnote{“treated”} accursed.}%
\verse{And they shall build houses and inhabit them, and they shall plant vineyards and eat their fruit.}%
\verse{They shall not build and another inhabit; they shall not plant and another eat. For the days of my people shall be like the days of a tree, and my chosen ones shall enjoy the work of their hands.}%
\verse{They shall not labor for nothing, and they shall not give birth to horror, for they shall be offspring blessed by\lebnote{“the offspring of the blessing of”} Adonai, and their descendants with them.}%
\verse{And this will happen: before they call, I myself will answer; while still they are speaking, I myself will hear.}%
\verse{The wolf and the lamb shall feed like one, and the lion shall eat straw like the ox, but\lnBZJ{} dust shall be the serpent’s food. They shall do no evil, and they shall not destroy on all my holy mountain,”\lnBZL{} says Adonai.}%
\end{biblechapter}%
\begin{biblechapter}% Isaiah 66
\verseWithHeading{Proper Worship}{Thus says Adonai: “Heaven is my throne, and the earth is the footstool for my feet. Where is this house that you would build for me? And where is this resting place for me?}%
\verse{And my hand has made all these things, and all these came to be,”\lebnote{“were”} declares\lnBZN{} Adonai, “but\lnBZO{} I look to this one: to the humble and the contrite of spirit and the one frightened at my word.}%
\verse{The one who slaughters a bull strikes a man; the one who slaughters a lamb for sacrifice breaks the neck of a dog. The one who offers an offering, the blood of swine; the one who offers frankincense blesses an idol. Indeed, they themselves have chosen their ways, and their soul delights in their abhorrence.}%
\verse{Indeed, I myself I will choose their ill treatment, and I will bring them objects of their dread, because I called, and no one answered;\lebnote{“there was no one who answers”} I spoke and they did not listen, but\lnBZO{} they did the evil in my eyes, and they chose that in which I do not delight.”}%
\verse{Hear the word of Adonai, you who are frightened at his word: Your brothers who hate you, who exclude you for my name’s sake have said, “Let Adonai be honored so that\lnBZO{} we may see your joy!” But\lebnote{Or “And”} they themselves shall be ashamed.}%
\verse{A voice, an uproar from the city! A voice from the temple! The voice of Adonai paying back\lebnote{“rewarding retribution to”} his enemies!}%
\verseWithHeading{Zion’s Delivery}{Before she was in labor she gave birth; before labor pains\lebnote{Hebrew “pain”} came to her, she gave birth to a son.}%
\verse{Who has heard anything like this? Who has seen anything like these things? Can a land be born in one day? Or can a nation be born in a moment? Yet when she was in labor, Zion indeed gave birth to her children.}%
\verse{Shall I myself I break open and not deliver?”\lebnote{“cause to bring forth”} says Adonai, “or I who delivers\lebnote{“causes to bring forth”} lock up the womb?” says your God.}%
\verse{“Rejoice with Jerusalem and shout in exultation with her, all those who love her! Rejoice with her in joy, all those who mourn over her,}%
\verse{so that you may suck and be satisfied from her consoling breast,\lebnote{“breast of consolation”} so that you may drink deeply and refresh yourselves from her heavy breast.”\lebnote{“breast of heaviness “}}%
\verse{For thus says Adonai: “Look! I am about to spread prosperity out to her like a river, and the wealth of the nations like an overflowing stream, and you shall suck and be carried on the hip; and you shall you shall be played with on the knees.}%
\verse{As a man whose mother comforts him, so I myself will comfort you, and you shall be comforted in Jerusalem.}%
\verseWithHeading{Adonai’s Final Judgment}{And you shall see and your heart shall rejoice, and your bones shall flourish\lebnote{“sprout”} like the grass, and the hand of Adonai shall make itself known to\lebnote{“with”} his servants, and he shall curse his enemies.}%
\verse{For look! Adonai will come in fire, and his chariots like the storm wind, to give back\lebnote{“cause to turn around”} his anger in wrath, and his rebuke in flames of fire.}%
\verse{For Adonai enters into judgment on all flesh with fire and his sword, and those slain by\lebnote{“the slain of”} Adonai shall be many.}%
\verse{Those who sanctify themselves and those who cleanse themselves to go into\lebnote{“toward”} the gardens after the one in the middle, eating the flesh of swine and detestable things\lebnote{Hebrew “thing”} and rodents\lebnote{Hebrew “rodent”} together shall come to an end!” declares\lnBZN{} Adonai.}%
\verse{“And I — their works and thoughts! — am about to come to gather all nations and tongues, and they shall come and see my glory.}%
\verse{And I will set a sign among them, and I will send survivors from them to the nations: Tarshish, Pul, and Lud, who draw the bow; Tubal and Javan, the faraway coastlands that have not heard of my fame, and have not seen my glory. And they shall declare my glory among the nations,}%
\verse{and bring all your countrymen\lebnote{Or “brothers”} from all the nations as an offering to Adonai on horses and chariots\lebnote{Hebrew “chariot”} and in litters and on mules and camels, to\lebnote{Or “on”} my holy mountain,\lebnote{“the mountain of holiness me”} Jerusalem,” says Adonai, “just as the sons of Israel bring an offering in a clean vessel to the house of Adonai.}%
\verse{And indeed, I will take some of them as priests and the Levites,” says Adonai.}%
\verse{“For just as the new heavens and earth that I am about to make shall stand before me,” declares\lnBZN{} Adonai, “so shall your descendants\lebnote{Hebrew “descendant”} and your name stand.}%
\verse{And this shall happen: From new moon to new moon\lebnote{“enough of new moon in his new moon”} and from Sabbath to Sabbath\lebnote{“enough of Sabbath in his Sabbath”} all flesh shall come to bow in worship before me,” says Adonai.}%
\verse{“And they shall go out and look at the corpses of the people\lebnote{Or “men”} who have rebelled\lebnote{“rebelling”} against me, for their worm shall not die, and their fire shall not be quenched, and they shall be an abhorrence to all flesh.”}%
\end{biblechapter}%
\flushcolsend
\input{leb/content/old-testament/Jer.tex}\flushcolsend
\biblebook{Lamentations}
\begin{biblechapter}% Lamentations 1
\verseWithHeading{The Desolate City}{איכה How ישׁבה sit בדד solitary, העיר doth the city רבתי full עם of people! היתה is she become כאלמנה as a widow! רבתי she great בגוים among the nations, שׂרתי princess במדינות among the provinces, היתה is she become למס׃ tributary!}%
\verse{בכו תבכהלילה in the night, ודמעתה and her tears על on לחיה her cheeks: אין she hath none לה מנחם to comfort מכל among all אהביה her lovers כל all רעיה her friends בגדו have dealt treacherously בה היו with her, they are become לה לאיבים׃ her enemies.}%
\verse{גלתה is gone into captivity יהודה Judah מעני because of affliction, ומרב and because of great עבדה servitude: היא she ישׁבה dwelleth בגוים among the heathen, לא no מצאה she findeth מנוח rest: כל all רדפיה her persecutors השׂיגוה overtook בין her between המצרים׃ the straits.}%
\verse{דרכי The ways ציון of Zion אבלות do mourn, מבלי because none באי come מועד to the solemn feasts: כל all שׁעריה her gates שׁוממין are desolate: כהניה her priests נאנחים sigh, בתולתיה her virgins נוגות are afflicted, והיא and she מר׃}%
\verse{היו are צריה Her adversaries לראשׁ the chief, איביה her enemies שׁלו prosper; כי for יהוה the LORD הוגה hath afflicted על her for רב the multitude פשׁעיה of her transgressions: עולליה her children הלכו are gone שׁבי into captivity לפני before צר׃ the enemy.}%
\verse{ויצא is departed: מן And from בת the daughter ציון כל all הדרה her beauty היו are become שׂריה her princes כאילים like harts לא no מצאו find מרעה pasture, וילכו and they are gone בלא without כח strength לפני before רודף׃ the pursuer.}%
\verse{זכרה remembered ירושׁלם Jerusalem ימי in the days עניה of her affliction ומרודיה and of her miseries כל all מחמדיה her pleasant things אשׁר that היו she had מימי in the days קדם of old, בנפל fell עמה when her people ביד into the hand צר of the enemy, ואין and none עוזר did help לה ראוה saw צרים her: the adversaries שׂחקו her, did mock על at משׁבתה׃ her sabbaths.}%
\verse{חטא hath grievously sinned; חטאה hath grievously sinned; ירושׁלם Jerusalem על therefore כן therefore לנידה removed: היתה she is כל all מכבדיה that honored הזילוה כי her, because ראו they have seen ערותה her nakedness: גם yea, היא she נאנחה sigheth, ותשׁב and turneth אחור׃ backward.}%
\verse{טמאתה Her filthiness בשׁוליה in her skirts; לא not זכרה she remembereth אחריתה her last end; ותרד therefore she came down פלאים wonderfully: אין she had no מנחם comforter. לה ראה behold יהוה O LORD, את עניי my affliction: כי for הגדיל hath magnified אויב׃ the enemy}%
\verse{ידו his hand פרשׂ hath spread out צר The adversary על upon כל all מחמדיה her pleasant things: כי for ראתה she hath seen גוים the heathen באו entered מקדשׁה into her sanctuary, אשׁר whom צויתה thou didst command לא they should not יבאו enter בקהל׃ into thy congregation.}%
\verse{כל All עמה her people נאנחים sigh, מבקשׁים they seek לחם bread; נתנו they have given מחמודיהם their pleasant things באכל for meat להשׁיב to relieve נפשׁ the soul: ראה see, יהוה O LORD, והביטה and consider; כי for הייתי I am become זוללה׃ vile.}%
\verse{לוא nothing אליכם to כל you, all עברי ye that pass by? דרך ye that pass by? הביטו behold, וראו and see אם if ישׁ there be מכאוב any sorrow כמכאבי like unto my sorrow, אשׁר which עולל is done לי אשׁר unto me, wherewith הוגה hath afflicted יהוה the LORD ביום in the day חרון of his fierce אפו׃ anger.}%
\verse{ממרום שׁלח hath he sent אשׁ fire בעצמתי into my bones, וירדנה and it prevaileth against פרשׂ them: he hath spread רשׁת a net לרגלי for my feet, השׁיבני he hath turned אחור me back: נתנני he hath made שׁממה כל all היום the day. דוה׃ faint}%
\verse{נשׂקד is bound על The yoke פשׁעי of my transgressions בידו by his hand: ישׂתרגו they are wreathed, עלו come up על upon צוארי my neck: הכשׁיל to fall, כחי he hath made my strength נתנני hath delivered אדני the Lord בידי me into hands, לא אוכלום׃ to rise up.}%
\verse{סלה hath trodden under foot כל all אבירי my mighty אדני The Lord בקרבי in the midst קרא of me: he hath called עלי against מועד an assembly לשׁבר me to crush בחורי my young men: גת in a winepress. דרך hath trodden אדני the Lord לבתולת the virgin, בת the daughter יהודה׃ of Judah,}%
\verse{על For אלה these אני I בוכיה weep; עיני mine eye, עיני mine eye ירדה runneth down מים with water, כי because רחק is far ממני from מנחם the comforter משׁיב that should relieve נפשׁי my soul היו are בני me: my children שׁוממים desolate, כי because גבר prevailed. אויב׃ the enemy}%
\verse{פרשׂה spreadeth forth ציון Zion בידיה her hands, אין none מנחם to comfort לה צוה hath commanded יהוה her: the LORD ליעקב concerning Jacob, סביביו round about צריו his adversaries היתה is ירושׁלם him: Jerusalem לנדה as a menstruous woman ביניהם׃ among}%
\verse{צדיק is righteous; הוא יהוהי for פיהו against his commandment: מריתי I have rebelled שׁמעו hear, נא I pray you, כל all עמים people, וראו and behold מכאבי my sorrow: בתולתי my virgins ובחורי and my young men הלכו are gone בשׁבי׃ into captivity.}%
\verse{קראתי I called למאהבי for my lovers, המה they רמוני deceived כהני me: my priests וזקני and mine elders בעיר in the city, גועו gave up the ghost כי while בקשׁו they sought אכל their meat למו וישׁיבו to relieve את נפשׁם׃ their souls.}%
\verse{ראה Behold, יהוה O LORD; כי for צר I in distress: לי מעי my bowels חמרמרו are troubled; נהפך is turned לבי mine heart בקרבי within כי me; for מרו מריתיחוץ abroad שׁכלה bereaveth, חרב the sword בבית at home כמות׃ as death.}%
\verse{שׁמעו They have heard כי that נאנחה sigh: אני I אין none מנחם to comfort לי כל me: all איבי mine enemies שׁמעו have heard רעתי of my trouble; שׂשׂו they are glad כי that אתה thou עשׂית hast done הבאת thou wilt bring יום the day קראת thou hast called, ויהיו and they shall be כמוני׃ like unto me.}%
\verse{תבא come כל Let all רעתם their wickedness לפניך before ועולל thee; and do למו כאשׁר unto them, as עוללת thou hast done לי על unto me for כל all פשׁעי my transgressions: כי for רבות many, אנחתי my sighs ולבי and my heart דוי׃ faint.}%
\end{biblechapter}%
\begin{biblechapter}% Lamentations 2
\verseWithHeading{The Lord is Angry}{איכה How יעיב covered באפו with a cloud in his anger, אדני hath the Lord את בת the daughter ציון of Zion השׁליך cast down משׁמים from heaven ארץ unto the earth תפארת the beauty ישׂראל of Israel, ולא not זכר and remembered הדם his footstool רגליו his footstool ביום in the day אפו׃ of his anger!}%
\verse{בלע hath swallowed up אדני The Lord לא and hath not חמל pitied: את כל all נאות the habitations יעקב of Jacob, הרס he hath thrown down בעברתו in his wrath מבצרי the strongholds בת of the daughter יהודה of Judah; הגיע he hath brought down לארץ to the ground: חלל he hath polluted ממלכה the kingdom ושׂריה׃ and the princes}%
\verse{גדע He hath cut off בחרי in fierce אף anger כל all קרן the horn ישׂראל of Israel: השׁיב he hath drawn אחור back ימינו his right hand מפני from before אויב the enemy, ויבער and he burned ביעקב against Jacob כאשׁ fire, להבה like a flaming אכלה devoureth סביב׃ round about.}%
\verse{דרך He hath bent קשׁתו his bow כאויב like an enemy: נצב he stood ימינו with his right hand כצר as an adversary, ויהרג and slew כל all מחמדי pleasant עין to the eye באהל in the tabernacle בת of the daughter ציון of Zion: שׁפך he poured out כאשׁ like fire. חמתו׃ his fury}%
\verse{היה was אדני The Lord כאויב as an enemy: בלע he hath swallowed up ישׂראל Israel, בלע he hath swallowed up כל all ארמנותיה her palaces: שׁחת he hath destroyed מבצריו his strongholds, וירב and hath increased בבת in the daughter יהודה of Judah תאניה mourning ואניה׃ and lamentation.}%
\verse{ויחמס And he hath violently taken away כגן as a garden: שׂכו his tabernacle, שׁחת he hath destroyed מועדו his places of the assembly: שׁכח to be forgotten יהוה the LORD בציון in Zion, מועד hath caused the solemn feasts ושׁבת and sabbaths וינאץ and hath despised בזעם in the indignation אפו of his anger מלך the king וכהן׃ and the priest.}%
\verse{זנח hath cast off אדני The Lord מזבחו his altar, נאר he hath abhorred מקדשׁו his sanctuary, הסגיר he hath given up ביד into the hand אויב of the enemy חומת the walls ארמנותיה of her palaces; קול a noise נתנו they have made בבית in the house יהוה of the LORD, כיום as in the day מועד׃ of a solemn feast.}%
\verse{חשׁב hath purposed יהוה The LORD להשׁחית to destroy חומת the wall בת of the daughter ציון of Zion: נטה he hath stretched out קו a line, לא he hath not השׁיב withdrawn ידו his hand מבלע from destroying: ויאבל to lament; חל therefore he made the rampart וחומה and the wall יחדו together. אמללו׃ they languished}%
\verse{טבעו are sunk בארץ into the ground; שׁעריה Her gates אבד he hath destroyed ושׁבר and broken בריחיה her bars: מלכה her king ושׂריה and her princes בגוים among the Gentiles: אין no תורה the law גם also נביאיה her prophets לא no מצאו find חזון vision מיהוה׃}%
\verse{ישׁבו sit לארץ upon the ground, ידמו keep silence: זקני The elders בת of the daughter ציון of Zion העלו they have cast up עפר dust על upon ראשׁם their heads; חגרו they have girded שׂקים themselves with sackcloth: הורידו hang down לארץ to the ground. ראשׁן their heads בתולת the virgins ירושׁלם׃ of Jerusalem}%
\verse{כלו do fail בדמעות with tears, עיני Mine eyes חמרמרו are troubled, מעי my bowels נשׁפך is poured לארץ upon the earth, כבדי my liver על for שׁבר the destruction בת of the daughter עמי of my people; בעטף swoon עולל because the children ויונק and the sucklings ברחבות in the streets קריה׃ of the city.}%
\verse{לאמתם to their mothers, יאמרו They say איה Where דגן corn ויין and wine? בהתעטפם when they swooned כחלל as the wounded ברחבות in the streets עיר of the city, בהשׁתפך was poured out נפשׁם when their soul אל into חיק bosom. אמתם׃ their mothers'}%
\verse{מה What thing אעידך shall I take to witness מה for thee? what thing אדמה shall I liken לך הבת to thee, O daughter ירושׁלם of Jerusalem? מה what אשׁוה shall I equal לך ואנחמך to thee, that I may comfort בתולת thee, O virgin בת daughter ציון of Zion? כי for גדול great כים like the sea: שׁברך thy breach מי who ירפא׃ can heal}%
\verse{נביאיך Thy prophets חזו have seen לך שׁוא vain ותפל and foolish things ולא for thee: and they have not גלו discovered על discovered עונך thine iniquity, להשׁיב to turn away שׁביתך thy captivity; ויחזו but have seen לך משׂאות burdens שׁוא for thee false ומדוחים׃ and causes of banishment.}%
\verse{ספקו clap עליך at כפים hands כל All עברי that pass by דרך that pass by שׁרקו thee; they hiss וינעו and wag ראשׁם their head על at בת the daughter ירושׁלם of Jerusalem, הזאת this העיר the city שׁיאמרו that call כלילת The perfection יפי of beauty, משׂושׂ The joy לכל of the whole הארץ׃ earth?}%
\verse{פצו have opened עליך against פיהם their mouth כל All אויביך thine enemies שׁרקו thee: they hiss ויחרקו and gnash שׁן the teeth: אמרו they say, בלענו We have swallowed up: אך certainly זה this היום the day שׁקוינהו that we looked for; מצאנו we have found, ראינו׃ we have seen}%
\verse{עשׂה hath done יהוה The LORD אשׁר which זמם he had devised; בצע he hath fulfilled אמרתו his word אשׁר that צוה he had commanded מימי in the days קדם of old: הרס he hath thrown down, ולא and hath not חמל pitied: וישׂמח to rejoice עליך over אויב and he hath caused enemy הרים thee, he hath set up קרן the horn צריך׃ of thine adversaries.}%
\verse{צעק cried לבם Their heart אל unto אדני the Lord, חומת O wall בת of the daughter ציון of Zion, הורידי run down כנחל like a river דמעה let tears יומם day ולילה and night: אל thyself no תתני give פוגת rest; לך אל let not תדם cease. בת the apple עינך׃ of thine eye}%
\verse{קומי Arise, רני cry out בליל in the night: לראשׁ in the beginning אשׁמרות of the watches שׁפכי pour out כמים like water לבך thine heart נכח before פני the face אדני of the Lord: שׂאי lift up אליו toward כפיך thy hands על him for נפשׁ the life עולליך of thy young children, העטופים that faint ברעב for hunger בראשׁ in the top כל of every חוצות׃ street.}%
\verse{ראה Behold, יהוה O LORD, והביטה and consider למי to whom עוללת thou hast done כה this. אם תאכלנה eat נשׁים Shall the women פרים their fruit, עללי children טפחים of a span long? אם יהרג be slain במקדשׁ in the sanctuary אדני of the Lord? כהן shall the priest ונביא׃ and the prophet}%
\verse{שׁכבו lie לארץ on the ground חוצות in the streets: נער The young וזקן and the old בתולתי my virgins ובחורי and my young men נפלו are fallen בחרב by the sword; הרגת thou hast slain ביום in the day אפך of thine anger; טבחת thou hast killed, לא not חמלת׃ pitied.}%
\verse{תקרא Thou hast called כיום day מועד as in a solemn מגורי my terrors מסביב round about, ולא none היה escaped ביום so that in the day אף anger יהוה of the LORD's פליט escaped ושׂריד nor remained: אשׁר those that טפחתי I have swaddled ורביתי and brought up איבי hath mine enemy כלם׃ consumed.}%
\end{biblechapter}%
\begin{biblechapter}% Lamentations 3
\verseWithHeading{Israel’s Affliction}{אני I הגבר the man ראה hath seen עני affliction בשׁבט by the rod עברתו׃ of his wrath.}%
\verse{אותי נהג He hath led וילך me, and brought חשׁך darkness, ולא but not אור׃ light.}%
\verse{אך Surely בי ישׁב against me is he turned; יהפך he turneth ידו his hand כל all היום׃ the day.}%
\verse{בלה hath he made old; בשׂרי My flesh ועורי and my skin שׁבר he hath broken עצמותי׃ my bones.}%
\verse{בנה He hath built עלי against ויקף me, and compassed ראשׁ with gall ותלאה׃ and travail.}%
\verse{במחשׁכים me in dark places, הושׁיבני He hath set כמתי as dead עולם׃ of old.}%
\verse{גדר He hath hedged בעדי me about, ולא that I cannot אצא get out: הכביד he hath made my chain heavy. נחשׁתי׃ he hath made my chain heavy.}%
\verse{גם Also כי when אזעק I cry ואשׁוע and shout, שׂתם he shutteth out תפלתי׃ my prayer.}%
\verse{גדר He hath enclosed דרכי my ways בגזית with hewn stone, נתיבתי he hath made my paths crooked. עוה׃ he hath made my paths crooked.}%
\verse{דב unto me a bear ארב lying in wait, הוא He לי אריה a lion במסתרים׃ in secret places.}%
\verse{דרכי my ways, סורר He hath turned aside ויפשׁחני and pulled me in pieces: שׂמני he hath made שׁמם׃}%
\verse{דרך He hath bent קשׁתו his bow, ויציבני and set כמטרא me as a mark לחץ׃ for the arrow.}%
\verse{הביא to enter בכליותי into my reins. בני He hath caused the arrows אשׁפתו׃ of his quiver}%
\verse{הייתי I was שׂחק a derision לכל to all עמי my people; נגינתם their song כל all היום׃ the day.}%
\verse{השׂביעני He hath filled במרורים me with bitterness, הרוני he hath made me drunken לענה׃ with wormwood.}%
\verse{ויגרס He hath also broken בחצץ with gravel stones, שׁני my teeth הכפישׁני he hath covered באפר׃ me with ashes.}%
\verse{ותזנח משׁלום from peace: נפשׁי נשׁיתי I forgot טובה׃ prosperity.}%
\verse{ואמר And I said, אבד is perished נצחי My strength ותוחלתי and my hope מיהוה׃}%
\verse{זכר Remembering עניי mine affliction ומרודי and my misery, לענה the wormwood וראשׁ׃ and the gall.}%
\verse{זכור hath still in remembrance, תזכור hath still in remembrance, ותשׁיח and is humbled עלי in נפשׁי׃ My soul}%
\verse{זאת This אשׁיב I recall אל to לבי my mind, על therefore כן therefore אוחיל׃ have I hope.}%
\verse{חסדי mercies יהוה the LORD's כי that לא we are not תמנו consumed, כי because לא not. כלו fail רחמיו׃ his compassions}%
\verse{חדשׁים new לבקרים every morning: רבה great אמונתך׃ thy faithfulness.}%
\verse{חלקי my portion, יהוה The LORD אמרה saith נפשׁי my soul; על therefore כן therefore אוחיל׃ will I hope}%
\verse{טוב good יהוה The LORD לקוו unto them that wait for לנפשׁ him, to the soul תדרשׁנו׃ seeketh}%
\verse{טוב good ויחיל ודומם and quietly wait לתשׁועת for the salvation יהוה׃ of the LORD.}%
\verse{טוב good לגבר for a man כי that ישׂא he bear על the yoke בנעוריו׃ in his youth.}%
\verse{ישׁב He sitteth בדד alone וידם and keepeth silence, כי because נטל he hath borne עליו׃ upon}%
\verse{יתן He putteth בעפר in the dust; פיהו his mouth אולי if so be ישׁ there may be תקוה׃ hope.}%
\verse{יתן He giveth למכהו to him that smiteth לחי cheek ישׂבע him: he is filled full בחרפה׃ with reproach.}%
\verse{כי For לא will not יזנח cast off לעולם forever: אדני׃ the Lord}%
\verse{כי But אם though הוגה he cause grief, ורחם yet will he have compassion כרב according to the multitude חסדו׃ of his mercies.}%
\verse{כי For לא he doth not ענה afflict מלבו willingly ויגה nor grieve בני the children אישׁ׃ of men.}%
\verse{לדכא To crush תחת under רגליו his feet כל all אסירי the prisoners ארץ׃ of the earth,}%
\verse{להטות To turn aside משׁפט the right גבר of a man נגד before פני the face עליון׃ of the most High,}%
\verse{לעות To subvert אדם a man בריבו in his cause, אדני the Lord לא not. ראה׃ approveth}%
\verse{מי Who זה he אמר saith, ותהי and it cometh to pass, אדני the Lord לא not? צוה׃ commandeth}%
\verse{מפי עליון of the most High לא not תצא proceedeth הרעות evil והטוב׃ and good?}%
\verse{מה Wherefore יתאונן complain, אדם man חי doth a living גבר a man על for חטאו׃ the punishment of his sins?}%
\verse{נחפשׂה Let us search דרכינו our ways, ונחקרה and try ונשׁובה and turn again עד to יהוה׃ the LORD.}%
\verse{נשׂא Let us lift up לבבנו our heart אל with כפים hands אל unto אל God בשׁמים׃ in the heavens.}%
\verse{נחנו We פשׁענו have transgressed ומרינו and have rebelled: אתה thou לא hast not סלחת׃ pardoned.}%
\verse{סכתה Thou hast covered באף with anger, ותרדפנו and persecuted הרגת us: thou hast slain, לא thou hast not חמלת׃ pitied.}%
\verse{סכותה Thou hast covered בענן thyself with a cloud, לך מעבור should not pass through. תפלה׃ that prayer}%
\verse{סחי us the offscouring ומאוס and refuse תשׂימנו Thou hast made בקרב in the midst העמים׃ of the people.}%
\verse{פצו have opened עלינו against פיהם their mouths כל All איבינו׃ our enemies}%
\verse{פחד Fear ופחת and a snare היה is come לנו השׁאת upon us, desolation והשׁבר׃ and destruction.}%
\verse{פלגי with rivers מים of water תרד runneth down עיני Mine eye על for שׁבר the destruction בת of the daughter עמי׃ of my people.}%
\verse{עיני Mine eye נגרה trickleth down, ולא not, תדמה and ceaseth מאין without any הפגות׃ intermission,}%
\verse{עד Till ישׁקיף look down, וירא and behold יהוה the LORD משׁמים׃ from heaven.}%
\verse{עיני Mine eye עוללה affecteth לנפשׁי mine heart מכל because of all בנות the daughters עירי׃ of my city.}%
\verse{צוד chased me sore, צדוני chased me sore, כצפור like a bird, איבי Mine enemies חנם׃ without cause.}%
\verse{צמתו They have cut off בבור in the dungeon, חיי my life וידו and cast אבן׃ a stone}%
\verse{צפו flowed מים Waters על over ראשׁי mine head; אמרתי I said, נגזרתי׃ I am cut off.}%
\verse{קראתי I called upon שׁמך thy name, יהוה O LORD, מבור תחתיות׃}%
\verse{קולי my voice: שׁמעת Thou hast heard אל not תעלם hide אזנך thine ear לרוחתי at my breathing, לשׁועתי׃ at my cry.}%
\verse{קרבת Thou drewest near ביום in the day אקראך I called upon אמרת thee: thou saidst, אל not. תירא׃ Fear}%
\verse{רבת thou hast pleaded אדני O Lord, ריבי the causes נפשׁי of my soul; גאלת thou hast redeemed חיי׃ my life.}%
\verse{ראיתה thou hast seen יהוה O LORD, עותתי my wrong: שׁפטה judge משׁפטי׃ thou my cause.}%
\verse{ראיתה Thou hast seen כל all נקמתם their vengeance כל all מחשׁבתם׃ their imaginations}%
\verse{שׁמעת Thou hast heard חרפתם their reproach, יהוה O LORD, כל all מחשׁבתם their imaginations עלי׃ against}%
\verse{שׂפתי The lips קמי of those that rose up against והגיונם me, and their device עלי against כל me all היום׃ the day.}%
\verse{שׁבתם their sitting down, וקימתם and their rising up; הביטה Behold אני I מנגינתם׃ their music.}%
\verse{תשׁיב Render להם גמול unto them a recompense, יהוה O LORD, כמעשׂה according to the work ידיהם׃ of their hands.}%
\verse{תתן Give להם מגנת them sorrow לב of heart, תאלתך thy curse להם׃}%
\verse{תרדף Persecute באף them in anger ותשׁמידם and destroy מתחת from under שׁמי the heavens יהוה׃ of the LORD.}%
\end{biblechapter}%
\begin{biblechapter}% Lamentations 4
\verseWithHeading{Zion Is Punished}{איכה How יועם become dim! זהב is the gold ישׁנא changed! הכתם fine gold הטוב is the most תשׁתפכנה are poured out אבני the stones קדשׁ of the sanctuary בראשׁ in the top כל of every חוצות׃ street.}%
\verse{בני sons ציון of Zion, היקרים The precious המסלאים comparable בפז to fine gold, איכה how נחשׁבו are they esteemed לנבלי pitchers, חרשׂ as earthen מעשׂה the work ידי of the hands יוצר׃ of the potter!}%
\verse{גם Even תנין חלצו draw out שׁד the breast, היניקו they give suck גוריהן to their young ones: בת the daughter עמי of my people לאכזר cruel, כי like the ostriches ענים במדבר׃ in the wilderness.}%
\verse{דבק cleaveth לשׁון The tongue יונק of the sucking child אל to חכו the roof of his mouth בצמא for thirst: עוללים the young children שׁאלו ask לחם bread, פרשׂ breaketh אין no man להם׃}%
\verse{האכלים They that did feed למעדנים delicately נשׁמו are desolate בחוצות in the streets: האמנים they that were brought up עלי in תולע scarlet חבקו embrace אשׁפתות׃ dunghills.}%
\verse{ויגדל is greater עון For the punishment of the iniquity בת of the daughter עמי of my people מחטאת than the punishment of the sin סדם of Sodom, ההפוכה that was overthrown כמו as in רגע a moment, ולא and no חלו stayed בה ידים׃ hands}%
\verse{זכו were purer נזיריה Her Nazarites משׁלג than snow, צחו they were whiter מחלב than milk, אדמו they were more ruddy עצם in body מפנינים than rubies, ספיר of sapphire: גזרתם׃ their polishing}%
\verse{חשׁך is blacker משׁחור than a coal; תארם Their visage לא they are not נכרו known בחוצות in the streets: צפד cleaveth עורם their skin על to עצמם their bones; יבשׁ it is withered, היה it is become כעץ׃ like a stick.}%
\verse{טובים better היו are חללי slain חרב with the sword מחללי than slain רעב with hunger: שׁהם for these יזובו pine away, מדקרים stricken through מתנובת for the fruits שׂדי׃ of the field.}%
\verse{ידי The hands נשׁים women רחמניות of the pitiful בשׁלו have sodden ילדיהן their own children: היו they were לברות their meat למו בשׁבר in the destruction בת of the daughter עמי׃ of my people.}%
\verse{כלה hath accomplished יהוה The LORD את חמתו his fury; שׁפך he hath poured out חרון his fierce אפו anger, ויצת and hath kindled אשׁ a fire בציון in Zion, ותאכל and it hath devoured יסודתיה׃ the foundations}%
\verse{לא would not האמינו have believed מלכי The kings ארץ of the earth, וכל and all ישׁבי the inhabitants תבל of the world, כי that יבא should have entered צר the adversary ואויב and the enemy בשׁערי into the gates ירושׁלם׃ of Jerusalem.}%
\verse{מחטאת נביאיה of her prophets, עונות the iniquities כהניה of her priests, השׁפכים that have shed בקרבה in the midst דם the blood צדיקים׃ of the just}%
\verse{נעו They have wandered עורים blind בחוצות in the streets, נגאלו they have polluted בדם themselves with blood, בלא not יוכלו so that men could יגעו touch בלבשׁיהם׃ their garments.}%
\verse{סורו unto them, Depart טמא ye; unclean; קראו They cried למו סורו depart, סורו depart, אל not: תגעו touch כי when נצו they fled away גם and נעו wandered, אמרו they said בגוים among the heathen, לא They shall no יוסיפו more לגור׃ sojourn}%
\verse{פני The anger יהוה of the LORD חלקם hath divided לא them; he will no יוסיף more להביטם regard פני the persons כהנים of the priests, לא not נשׂאו them: they respected זקנים the elders. לא not חננו׃ they favored}%
\verse{עודינה תכלינה failed עינינו As for us, our eyes אל for עזרתנו help: הבל our vain בצפיתנו we have watched צפינו אל for גוי a nation לא could not יושׁע׃ save}%
\verse{צדו They hunt צעדינו our steps, מלכת that we cannot go ברחבתינו in our streets: קרב is near, קצינו our end מלאו are fulfilled; ימינו our days כי for בא is come. קצינו׃ our end}%
\verse{קלים swifter היו are רדפינו Our persecutors מנשׁרי than the eagles שׁמים of the heaven: על us upon ההרים the mountains, דלקנו they pursued במדבר for us in the wilderness. ארבו׃ they laid wait}%
\verse{רוח The breath אפינו of our nostrils, משׁיח the anointed יהוה of the LORD, נלכד was taken בשׁחיתותם in their pits, אשׁר of whom אמרנו we said, בצלו Under his shadow נחיה we shall live בגוים׃ among the heathen.}%
\verse{שׂישׂי Rejoice ושׂמחי בת O daughter אדום of Edom, יושׁבתי that dwellest בארץ in the land עוץ of Uz; גם also עליך unto תעבר shall pass through כוס the cup תשׁכרי thee: thou shalt be drunken, ותתערי׃ and shalt make thyself naked.}%
\verse{תם is accomplished, עונך The punishment of thine iniquity בת O daughter ציון of Zion; לא he will no יוסיף more להגלותך carry thee away into captivity: פקד he will visit עונך thine iniquity, בת O daughter אדום of Edom; גלה he will discover על he will discover חטאתיך׃ thy sins.}%
\end{biblechapter}%
\begin{biblechapter}% Lamentations 5
\verseWithHeading{A Request for Mercy}{זכר Remember, יהוה O LORD, מה what היה is come לנו הביט upon us: consider, וראה and behold את חרפתנו׃ our reproach.}%
\verse{נחלתנו Our inheritance נהפכה is turned לזרים to strangers, בתינו our houses לנכרים׃ to aliens.}%
\verse{יתומים orphans היינו We are אין and fatherless, אב and fatherless, אמתינו our mothers כאלמנות׃ as widows.}%
\verse{מימינו our water בכסף for money; שׁתינו We have drunken עצינו our wood במחיר is sold יבאו׃ is sold}%
\verse{על under צוארנו Our necks נרדפנו persecution: יגענו we labor, לא have no rest. הונח׃ have no rest.}%
\verse{מצרים נתנו We have given יד the hand אשׁור the Assyrians, לשׂבע to be satisfied לחם׃ with bread.}%
\verse{אבתינו Our fathers חטאו have sinned, אינם not; אנחנו and we עונתיהם their iniquities. סבלנו׃ have borne}%
\verse{עבדים Servants משׁלו have ruled over בנו פרק that doth deliver אין us: none מידם׃ out of their hand.}%
\verse{בנפשׁנו with our lives נביא We got לחמנו our bread מפני because of חרב the sword המדבר׃ of the wilderness.}%
\verse{עורנו Our skin כתנור like an oven נכמרו was black מפני because of זלעפות the terrible רעב׃ famine.}%
\verse{נשׁים the women בציון in Zion, ענו They ravished בתלת the maids בערי in the cities יהודה׃ of Judah.}%
\verse{שׂרים Princes בידם by their hand: נתלו are hanged up פני the faces זקנים of elders לא were not נהדרו׃ honored.}%
\verse{בחורים the young men טחון to grind, נשׂאו They took ונערים and the children בעץ under the wood. כשׁלו׃ fell}%
\verse{זקנים The elders משׁער from the gate, שׁבתו have ceased בחורים the young men מנגינתם׃ from their music.}%
\verse{שׁבת is ceased; משׂושׂ The joy לבנו of our heart נהפך is turned לאבל into mourning. מחלנו׃ our dance}%
\verse{נפלה is fallen עטרת The crown ראשׁנו our head: אוי woe נא לנו כי unto us, that חטאנו׃ we have sinned!}%
\verse{על For זה this היה is דוה faint; לבנו our heart על for אלה these חשׁכו are dim. עינינו׃ our eyes}%
\verse{על Because of הר the mountain ציון of Zion, שׁשׁמם is desolate, שׁועלים the foxes הלכו׃ walk}%
\verse{אתה Thou, יהוה לעולם forever; תשׁב remainest כסאך thy throne לדר from generation ודור׃ to generation.}%
\verse{למה Wherefore לנצח us forever, תשׁכחנו dost thou forget תעזבנו forsake לארך us so long ימים׃ time?}%
\verse{השׁיבנו Turn יהוה thee, O LORD, אליך thou us unto ונשׁוב and we shall be turned; חדשׁ renew ימינו our days כקדם׃ as of old.}%
\verse{כי אםאס thou hast utterly rejected מאסתנו קצפת us; thou art very wroth עלינו against עד מאד׃}%
\end{biblechapter}%
\flushcolsend
\biblebook{Ezekiel}
\begin{biblechapter}% Ezekiel 1
\verseWithHeading{Date of Vision of Living Creatures and Glory of the Throne of Adonai}{And it was in the thirtieth year, in the fourth month, on the fifth day of the month, and I was in the midst of\lebnote{Or “among”} the exiles\lebnote{Hebrew “exile”} by the Kebar River.\lnCGR{} The heavens were opened, and I saw visions of God.}%
\verse{On the fifth day of the month — it was the fifth year of the exile of the king Jehoiachin —}%
\verse{the word of Adonai came\lebnote{“was”} clearly\lebnote{Or “certainly”} to Ezekiel the son of Buzi, the priest, in the land of the Chaldeans at the Kebar River,\lnCGR{} and the hand of Adonai was on him there.}%
\verse{And I looked,\lebnote{Or “saw”} and look! A storm wind was coming from the north, a great cloud, and fire flashing back and forth, and brightness around and within it,\lebnote{“for it”} and from its midst\lnCGS{} it was like the outward appearance of amber stone from the midst\lnCGS{} of the fire.}%
\verse{And from its midst\lnCGS{} was the likeness of four living creatures, and this was their appearance: a human form,\lebnote{“this their appearance the likeness of a human to them”}}%
\verse{and each had four faces,\lebnote{“and four faces to each”} and each of them had four wings.\lebnote{“and four wings to each for them”}}%
\verse{And their legs were straight legs, and the sole of their feet was like the sole of the foot of a calf, and they were sparkling like the outward appearance of polished bronze.}%
\verse{And under their wings were human hands\lebnote{“hand his a man”} on their four sides. And their faces and their wings for the four of them were as follows:}%
\verse{their wings were touching one another;\lebnote{“their wings \textit{were} touching each her sister”} each of them went straight forward,\lnCGT{} without turning right or left.\lebnote{“each toward the side of his face they went”}}%
\verse{The likeness of their faces was the face of a human in front, and the face of a lion on the right of each of them,\lnCGU{} and the face of an ox on the left of each of them,\lnCGU{} and the face of an eagle for each of them.\lnCGU{}}%
\verse{So were their faces; their wings were spread out upward;\lnCGV{} each had two touching one another and two covering their bodies.}%
\verse{And each went straight forward;\lebnote{“toward the side of his face they went”} wherever the spirit went\lebnote{“towards which it was there the spirit”} they went, and they did not turn as they went.\lnCGT{}}%
\verse{As for\lnCGW{} the likeness of the living creatures, their appearance was like burning coals of fire, like the appearance of torches. It\lebnote{That is, the fire} was moving to and fro between the living creatures, and the fire was very bright\lebnote{“brightness \textit{was} for the fire”} and lightning was going out from the fire.}%
\verse{And the living creatures were speeding to and fro\lebnote{“sped forth and returned”} like the appearance of lightning.}%
\verse{And I saw the living creatures, and look! A wheel was on the earth\lnCGX{} beside each of the living creatures that had four faces.\lebnote{“the living creatures for the four of his/its faces.” The LXX has “for the four of them.”}}%
\verse{The appearance of the wheels and their construction was like the appearance of beryl,\lebnote{Or “chrysolite,” or “topaz”; this stone is difficult to identify exactly} and they all looked alike,\lebnote{“likeness \textit{was} one for the four of them”} and their appearance and their construction was like a wheel within a wheel.\lebnote{“as that it was a wheel in the midst of a wheel”}}%
\verse{When they moved, they went toward their four sides; they did not veer at all as they went.\lebnote{“turn away when they went”}}%
\verse{And their rims were high and awesome,\lebnote{“height \textit{was} for them and awesomeness \textit{was} for them”} and all four of\lnCGU{} their rims were full of eyes all around.}%
\verse{And at the going of the living creatures, the wheels next to them also went, and when the creatures were lifted from the ground\lebnote{“and at the lifting up of the creatures from the earth”} the wheels also rose.}%
\verse{Wherever\lebnote{“Toward which”} the spirit went they would go there, and the wheels rose, for the spirit of the living creatures\lnCGY{} was in the wheels.}%
\verse{At their\lebnote{That is, the living creatures} going, they\lebnote{That is, the wheels} go, and at their standing, they stood, and at their being lifted up from on the earth,\lnCGX{} the wheels rose,\lebnote{Or “were lifted up”} for the spirit of the living creatures\lnCGY{} was in the wheels.}%
\verse{Now\lnCGW{} the likeness above the heads of the living creatures was an expanse like the outward appearance of awesome\lebnote{“terrible”} ice\lebnote{“frozen water”} spread out above their heads upward.\lebnote{“from to upward”}}%
\verse{And under the expanse their wings were stretched out straight one toward the other;\lebnote{“each towards its sister”} each had two wings covering them, and each had two wings covering their bodies.}%
\verse{And I heard the sound of their wings like the sound of many waters, like the voice of Shaddai,\lebnote{Often translated “Almighty”} and when they moved\lebnote{“at their going”} there was a sound of tumult like the sound of an army; when they stood still\lnCGZ{} they lowered their wings.}%
\verse{And there was\lebnote{Or “it was”} a sound from above the expanse that was above their heads, and when they stood\lnCGZ{} they lowered their wings.}%
\verse{And from above the expanse that was above their heads there was the likeness of a throne, looking like a sapphire,\lebnote{“\textit{the} appearance of a stone of sapphire”} and above the likeness of the throne was a likeness similar to\lebnote{Or “like”} the appearance of a human on it, but above it.\lnCGV{}}%
\verse{And I saw something like the outward appearance of amber, something like the appearance of fire, with a covering around it, from the likeness of his loins and upward.\lebnote{“to above”} And from the likeness of his loins and downward\lebnote{“to down”} I saw something like the appearance of fire, and it was radiant all around.\lebnote{“brightness \textit{was} for it around”}}%
\verse{Like the appearance\lebnote{Or “likeness”} of a bow that is in the cloud on a rainy day,\lebnote{“the day of the rain”} such was the radiance\lebnote{Or “brightness”} around it; thus was the appearance of the likeness of the glory of Adonai. And I saw, and I fell on my face, and I heard a voice speaking.}%
\end{biblechapter}%
\begin{biblechapter}% Ezekiel 2
\verseWithHeading{The Call of Ezekiel to Speak God’s Words}{And he said to me, “Son of man,\lnCHA{} stand on your feet, so that I can speak with you.”}%
\verse{And the Spirit\lebnote{Or “\textit{a} spirit”} came into me as he was speaking to me, and it set me on my feet, and I heard the one speaking to me.}%
\verse{And he said to me, “Son of man,\lnCHA{} I am sending you to the Israelites,\lebnote{“sons/children of Israel”} to nations\lebnote{Or “a people”} who are rebelling, who rebelled against me, they and their ancestors,\lebnote{Hebrew “fathers”} they transgressed against me until this very day.\lebnote{“until the day the this”}}%
\verse{And the children\lebnote{Or “sons”} are impudent\lebnote{“hard of face”} and stubborn,\lebnote{“strong of heart”} and so I am sending you to them, and you must say to them, ‘Thus says the Lord Adonai!’}%
\verse{And they, whether they listen or whether they fail to listen, for they are a rebellious house, they will know that a prophet was in their midst.}%
\verse{And you, son of man,\lnCHA{} you must not be afraid of them, and you must not be afraid of their words, because\lebnote{Or “although”} briers and thorns are with you, and you are sitting among scorpions. You must not be afraid of their words, and you must not be dismayed because of their looks,\lebnote{“faces”} for they are a rebellious house.}%
\verse{And you must speak my words to them whether they listen or whether they fail to listen, for they are rebellious.}%
\verse{And you, son of man,\lnCHA{} hear what I am speaking to you: you must not be rebellious like the house of rebellion. Open your mouth and eat what I am giving to you.”}%
\verse{And I looked,\lebnote{Or “saw”} and look! There was a hand stretched out to me, and look! In it was a scroll with writing.\lebnote{“a scroll of a book”}}%
\verse{And he rolled it out\lebnote{Or “he spread it out”} before me,\lebnote{“to the face of me”} and it was written on the front and back, and there were written on it laments and mourning and wailing.\lebnote{Or “woe”}}%
\end{biblechapter}%
\begin{biblechapter}% Ezekiel 3
\verse{And he said to me, “Son of man,\lnCHB{} what you find, eat! Eat this scroll, and go, speak to the house of Israel.”}%
\verse{And I opened my mouth, and he gave me this scroll to eat,}%
\verse{and he said to me, “Son of man,\lnCHB{} you must give your stomach this to eat, and you must fill your belly with this scroll that I am giving to you.” And I ate, and it became like sweet honey\lebnote{“like honey for sweetness”} in my mouth.}%
\verse{And he said to me, “Son of man,\lnCHB{} come! Go to the house of Israel, and you must speak to them with my words.}%
\verse{For you are sent to the house of Israel, not to a people of obscure speech\lebnote{“obscure of lip”} and of a difficult language,\lnCHC{}}%
\verse{and not to many nations of obscure speech\lebnote{“obscure of tongue”} and a difficult language\lnCHC{} whose words\lebnote{“their words”} you do not understand, for if I had sent you to them they would have listened to you.}%
\verse{And the house of Israel, they are not willing to listen to you, for they are not willing to listen to me, for all of the house of Israel is hard of forehead, and they are hard of heart.}%
\verse{But look, I have made\lebnote{Or “set”} your face hard against\lnCHD{} their faces and your forehead hard against\lnCHD{} their forehead.}%
\verse{Like a diamond harder than flint\lebnote{“hard more than flint”} I have made your forehead; you must not fear them, and you must not be dismayed on account of them,\lebnote{“from the face of them”} for they are a rebellious house.\lebnote{“\textit{are} a house of rebellion”}}%
\verse{And he said to me, “Son of man,\lnCHB{} all of my words that I shall speak to you, receive into\lebnote{Hebrew “in”} your heart and hear with your ears.}%
\verse{And come, go to the exiles,\lnCHE{} to the children\lebnote{Or “sons”} of your people, and you must speak to them, and say to them, ‘Thus says the Lord Adonai!’ whether they listen or whether they fail to listen.”}%
\verse{And the Spirit lifted me up, and I heard behind me the sound of a great earthquake when the glory of Adonai rose from its place.}%
\verse{And it was the sound of the wings of the living creatures touching lightly\lebnote{Or “brushing against”} one against the other,\lebnote{“each against each \textit{its sister}} and the sound of the wheels beside them and the sound of a great earthquake.}%
\verse{And the Spirit lifted me and took me, and I went in bitterness in the heat of my spirit, and the hand of Adonai was strong on me.}%
\verse{And I went to the exiles\lnCHE{} at Tel Abib, who were dwelling near the Kebar River,\lnCHF{} and I sat where they were dwelling. I sat there seven days in the midst of them, stunned.}%
\verse{And then it happened\lebnote{Or “was”} at the end of seven days, the word of Adonai came\lebnote{“and it was the word of Adonai to me to say \textit{saying}”} to me, saying,}%
\verse{“Son of man,\lnCHB{} I have appointed you as a watchman for the house of Israel. When you hear a word from my mouth, then you must warn them from me.}%
\verse{When I say\lebnote{“at/in my saying”} to the wicked, ‘Surely you will die,’ and you do not warn him and you do not speak to warn the wicked from his wicked way so that he may live,\lebnote{“so that \textit{you} let him live”} that wicked person will die because of his guilt, and from your hand I will seek his blood.}%
\verse{And you, if you do warn the wicked and he does not turn from his wickedness and from his wicked way, on account of his guilt he will die, and you yourself will have saved your life.\lnCHG{}}%
\verse{And when the righteous turns from his righteousness\lebnote{“and at \textit{the} to turn of \textit{the} righteous from his righteousness”} and does injustice, and I place\lnCHH{} a stumbling block before him,\lebnote{“to the face of him”} he will die, for you did not warn him. Because of his sin he will die, and his righteousness\lebnote{Or “righteous deeds”} that he did will not be remembered, and his blood I shall seek from your hand.}%
\verse{And if you warn him, the righteous, not to sin, and the righteous does not sin, surely he will live, for he heeded a warning. And you will have saved your life.”\lnCHG{}}%
\verse{And the hand of Adonai was on me there, and he said to me, “Rise up, go out to the valley, and there I will speak with you.”}%
\verse{And I rose up, and I went to the valley, and, look, there the glory of Adonai was standing, like the glory that I saw near the Kebar River,\lnCHF{} and I fell on my face.}%
\verse{And the Spirit\lebnote{Or “\textit{a} spirit”} came into me, and it made me stand on my feet, and he spoke with me and said to me, “Come, shut yourself inside your house,\lebnote{“in the midst of your house”}}%
\verse{and you, son of man,\lnCHB{} look, they will place\lnCHH{} cords on you, and they will tie you up with them. Then you will not go out into the midst of them.}%
\verse{And I will make your tongue cling to the roof of your mouth, and you will be silenced, and you will not be a reproving man for them, for they are a rebellious house.\lnCHI{}}%
\verse{And when I speak with you, I will open your mouth, and you must say to them, ‘Thus says the Lord Adonai: “The one hearing him, let him hear, and the one failing to hear, let him fail,”’ for they are a rebellious house.\lnCHI{}}%
\end{biblechapter}%
\begin{biblechapter}% Ezekiel 4
\verseWithHeading{Siege of Jerusalem Illustrated on a Brick}{“Now,\lebnote{Or “And”} son of man,\lnCHJ{} take for yourself a brick, and you must put it before you,\lebnote{“to the face of you”} and you must portray on it a city, Jerusalem.}%
\verse{And you must build against it siege works, and you must build against it a bulwark, and you must heap against it a siege ramp, and you must set up against it camps and put against it a battering ram all around.}%
\verse{And take for yourself a plate of iron, and you must place it as a wall of iron between you and the city, and you must set your face against it, and it must be under siege,\lebnote{“in the siege”} and you must lay the siege against it; it is a sign for the house of Israel.}%
\verse{And you, lie down on your left side, and you must put\lebnote{Or “you will put”} the guilt of the house of Israel on it. You will carry their guilt the number of days that you will lie on it.}%
\verse{And I will give to you the years of their guilt according to the number of the days, three hundred and ninety days, and you must bear the guilt\lnCHK{} of the house of Israel.}%
\verse{When you have completed these, then you must lie a second time on your right side; and you must bear the guilt\lnCHK{} of the house of Judah forty days, a day for each year, a day for each year I give it to you.}%
\verse{And toward the siege of Jerusalem you must set your face and your bared arm; then you must prophesy against it.}%
\verse{Now\lebnote{Or “and”} look! I will put on you cords, and you may not turn yourself from one side to your other side until you complete the days of your siege.}%
\verseWithHeading{Ezekiel’s Strange Defiled Meal}{“And you, take for yourself\lnCHL{} wheat and barley and beans and lentils and millet\lebnote{Or “sorghum”} and spelt, and you must put them in one vessel, and you must make them for yourself\lnCHL{} into a food during the number of days that you are lying on your side; three hundred and ninety days you shall eat it.}%
\verse{And your food that you will eat\lebnote{Hebrew “will eat it”} will be according to weight; twenty shekels for each day at fixed times\lnCHM{} you shall eat it.}%
\verse{And an amount of water\lebnote{“water by amount”} you shall drink, a sixth of a hin; at fixed times\lnCHM{} you shall drink it.}%
\verse{And as a bread-cake of barley you shall eat it, and with human excrement\lebnote{“with human dung of the excrement of the human”} you shall bake it before their eyes.”}%
\verse{And Adonai said, “Thus shall the Israelites\lebnote{“sons/children of Israel”} eat their unclean food among the nations where I will scatter them.”\lebnote{“which I will scatter them there”}}%
\verse{And I said, “Ah, Lord Adonai! Look! I have not been defiling myself, and a dead body and mangled carcass I have not eaten from my childhood until now, and unclean meat\lebnote{“flesh of unclean meat”} has not come into my mouth!”}%
\verse{And he said to me, “See I will give you cattle manure\lebnote{“the manure of cattle”} in the place of the feces of a human, and you may prepare your food on it.”}%
\verse{And he said to me, “Son of man,\lnCHJ{} look, I am going to break the supply\lebnote{“staff”} of bread in Jerusalem, and they will eat bread by weight, anxiously,\lebnote{“and with worry”} and rationed water,\lebnote{“and water by amount”} and they will drink with horror,}%
\verse{so that they will lack food and water, and they will be appalled with one another,\lebnote{“a man and his brother”} and they will waste away because of their guilt.\lnCHK{}}%
\end{biblechapter}%
\begin{biblechapter}% Ezekiel 5
\verse{“And you, son of man,\lebnote{Or “mortal,” or “son of humankind”} take for yourself\lebnote{Hebrew “you”} a sword, sharp as a barber’s razor.\lebnote{“the razor of the barbers”} Take it for yourself, and you must cause it to pass over your head and over your beard, and you must take for yourself a set of scales for weighing, and you must divide them.\lebnote{That is, the hairs}}%
\verse{A third you must burn with fire in the midst\lebnote{Or “middle”} of the city at the completion\lebnote{“the completing”} of the days of the siege, and you must take a third, and you must strike it with the sword around it, and a third you must scatter to the wind, and I will draw a sword behind them.}%
\verse{And you must take from these a few in number, and you must tuck them in your hem.}%
\verse{And from them again you shall take some, and you must throw them in the middle of the fire, and you must burn them with fire; from it a fire will go out to all of the house of Israel.}%
\verse{Thus says the Lord Adonai: This is Jerusalem in the midst of the nations where I have put her, and countries are around her.}%
\verse{But she has rebelled against my regulations to the point of wickedness more than the nations, and my statutes more than the countries that are around her; for they rejected my regulations, and as for my statutes, they did not walk in them.}%
\verse{Therefore, thus says the Lord Adonai: Because of your commotion more than the nations that are around you — you did not walk in my statutes, and you did not do my regulations, and according to the regulations of the nations that are around you, you did not do.}%
\verse{Therefore thus says the Lord Adonai: Look! I, even I,\lebnote{“surely I”} am against you, and I will execute judgment in the midst of you before the eyes of the nations,}%
\verse{and I will do with you that which I have not done, and which I will not do again, because of all of your detestable things.}%
\verse{Therefore\lebnote{“to thus”} parents\lnCHN{} will eat children\lnCHO{} in the midst of you, and children\lnCHO{} will eat their parents,\lnCHN{} and I will execute judgment in you, and I will scatter your entire remnant to every wind.}%
\verse{Therefore as surely as I live,\lebnote{“alive I \textit{am}”} declares\lebnote{“declaration of”} the Lord Adonai, Surely\lebnote{“If not”} because you have defiled my sanctuary with all of your vile idols and with all of your detestable things, now indeed I\lebnote{“and also I”} will reduce, and my eye will not take pity, and surely I will show no compassion.}%
\verse{A third of you will die because of the plague, and because of the famine they will perish in the midst of you, and a third will fall through the sword around you, and a third I will scatter to every direction of the wind, and I will draw the sword behind them.}%
\verse{And my anger will come to an end, and I will place my rage on them, and I will relent, and they will know that I, Adonai, have spoken in my passion when I fully vent my rage against them.\lebnote{“at my finishing my anger against them”}}%
\verse{And I will make you into a desolate place and into a disgrace among the nations that surround you before the eyes of every one who passes by.\lebnote{“every one of passing by”}}%
\verse{And it will be\lebnote{Or “and you will be”} an object of taunting\lebnote{Hebrew “taunt”} and an object of mockery, a warning and a horror to the nations that are around you whenever I execute judgments against you in anger and in wrath and in furious punishments! I, Adonai, have spoken!}%
\verse{When I send my arrows of deadly famine against them, which will be as destruction that I will send in order to destroy you, I will increase famine against you, and I will break the supply of food\lebnote{“staff of bread”} for you.}%
\verse{And I will send against you famine and fierce animals,\lebnote{Hebrew “animal”} and they will make you childless; and plague and blood will pass through you, and I will bring the sword upon you. I, Adonai, I have spoken!”}%
\end{biblechapter}%
\begin{biblechapter}% Ezekiel 6
\verseWithHeading{Idolatrous Worship and Idolatrous Objects Denounced}{And the word of Adonai came\lebnote{“was”} to me, saying,}%
\verse{“Son of man,\lebnote{Or “mortal,” or “son of humankind”} set your face to the mountains of Israel and prophesy against them,}%
\verse{and you must say, ‘Mountains of Israel, hear the word of the Lord Adonai, thus says the Lord Adonai to the mountains and to the hills, to the ravines and to the valleys: “Look, I am bringing upon you the sword, and I will destroy your high places,}%
\verse{and your altars will be desolate, and your incense altars will be broken, and I will throw down your slain ones before\lnCHP{} your idols,}%
\verse{and I will place\lebnote{Or “give”} the corpses of the children of Israel before\lnCHP{} their idols, and I will scatter your bones around your altars.}%
\verse{In all of your dwellings, the cities will be desolate and the high places will be ruined, so that your altars will be desolate and will suffer punishment. Your idols will be broken and will come to an end, and your incense altars will be cut down, and your works will be destroyed,}%
\verse{and the slain one will fall in the midst of you, and then you will know that I am Adonai.}%
\verse{But\lebnote{Or “And”} I will spare some, so there will be\lebnote{“at their being”} for you fugitives from the sword among the nations when you are scattered\lebnote{“at you to be scattered”} in the countries.}%
\verse{And your fugitives will remember me among the nations to which they were taken captive, that I was shattered by their adulterous heart\lebnote{“heart their which \textit{was} prostituting \textit{itself}”} which departed from me, and by their adulterous eyes\lebnote{“eyes their which \textit{were} prostituting”} which went after their idols, and they will feel loathing for themselves,\lebnote{“for their face”} for the evil that they did, for all of their detestable things.}%
\verse{And they will know that I am Adonai; not in vain\lebnote{“to nothing”} I spoke to bring to them this evil.”’}%
\verse{Thus says the Lord Adonai: ‘Clap your hand and stamp with your foot, and say, “Alas, for all of the detestable things of the evil of the house of Israel, because of which\lebnote{“which”} they will fall with the sword, with the famine, and with the plague.}%
\verse{The one who is far away will die by the plague, and the one who is near will fall by the sword, and the one who is being left behind and being spared will die by the famine, and I will complete my rage on\lebnote{Or “toward”} them.}%
\verse{And you will know that I am Adonai when their slain ones are in the midst of their idols\lebnote{“in the being of their slain ones in the midst of their idols”} around their altars at every high hill, on the tops of all the mountains and under every green tree and under every leafy oak — the place at which they gave a pleasing scent for all of their idols.}%
\verse{And I will stretch out my hand against them, and I will make the land a desolation and a wasteland from the desert to Riblah in all of their dwellings, and they will know that I am Adonai.”’”}%
\end{biblechapter}%
\begin{biblechapter}% Ezekiel 7
\verseWithHeading{Punishment for Abominations Throughout the Land}{And the word of Adonai came\lebnote{“was”} to me, saying,}%
\verse{“And you, son of man,\lebnote{Or “mortal,” or “son of humankind”} thus says the Lord Adonai to the land of Israel: ‘The end comes, the end on the four corners of the land.}%
\verse{Now the end is on you, and I will send my anger on you, and I will judge you according to your ways, and I will bring on you all your detestable things.}%
\verse{And my eye will not take pity on you, and I will not show compassion for your ways; on you I will bring your detestable things; they will be in the midst of you, and you will know that I am Adonai.’}%
\verse{Thus says the Lord Adonai: ‘Look! Disaster after disaster is coming!}%
\verse{The end comes, comes the end! It has awakened against you! Look! It comes!}%
\verse{Doom is coming against you, the dweller\lebnote{Or “inhabitant”} of the land; the time comes, the day is near, panic and not joy is on the mountains.}%
\verse{Soon\lebnote{“now from near”} I will pour out my rage on you, and I will fully vent my anger on you, and I will judge you according to your ways, and I will bring on you all of your detestable things.}%
\verse{And my eye will not take pity, and I will not show compassion. According to your ways I will deal with you,\lebnote{“on you I will bring”} and your detestable things will be in the midst of you, and you will know that I am Adonai who strikes.\lebnote{Or “smites”}}%
\verse{Look! The day is coming; doom goes out, the staff blossoms, pride sprouts.}%
\verse{Violence has grown to become a staff of wickedness; none\lnCHQ{} from them will remain, and none\lnCHQ{} from their abundance nor from their wealth;\lebnote{Or “noise,” or “multitude”; the same word appears in each of the next three verses} and prominence will not be among them.}%
\verse{The time has come, the day has arrived; let not the buyer rejoice, and let the seller not mourn, for anger is on all their multitude.}%
\verse{For the seller will not return to the merchandise while they are still alive,\lebnote{“while still \textit{is} in the life, their life”} for the vision is about all of its multitude; it will not change, and a man because of his guilt will not be able to hold onto his life.}%
\verse{They shall blow on the trumpet and prepare everything,\lebnote{“the all”} but there is no one going to the battle, for my anger is on all of their multitude.}%
\verse{The sword is outside,\lebnote{“\textit{is} in the outside place”} and the plague and the famine are inside;\lebnote{“\textit{are} from \textit{the} house} the one who is in the field will die by the sword, and the one who is in the city, famine and plague will devour him.}%
\verse{And if their survivors will escape, they will be on the mountains, like the doves of the valley, all of them groaning, each because of his guilt.}%
\verse{All of the hands will hang limp, and all of the knees will be wet with urine.\lebnote{“\textit{the} knees will flow \textit{with} water”}}%
\verse{And they will wear sackcloth, and horror will cover them, and on all of the faces will be shame, and baldness on all of their heads.}%
\verse{Their silver they will discard on the streets, and filth will be their gold; their silver and their gold will not be able to rescue them on the day of the wrath of Adonai. They will not satisfy their hunger\lebnote{“selves,” or “desire”} and their stomachs they will not fill, for their guilt will be their\lebnote{Hebrew “the”} stumbling block.}%
\verseWithHeading{Judgment for the Profanation of the Temple and Sanctuary}{“‘And the beauty of its\lebnote{Or “their”} ornament he made into a prideful thing\lebnote{“to \textit{a} prideful \textit{thing} he made it”} and made with it their detestable things and their vile idols; therefore I will make it into an impure thing for them.}%
\verse{And I will give it into the hand of strangers as plunder and to the wicked people of the earth as spoil, and they will defile it.}%
\verse{And I will turn my face from them, and they will defile my treasured place, and violent ones will enter and defile it.}%
\verse{Make a chain for the land; it is full of bloody crimes\lebnote{“\textit{the} judgment of blood”} and the city is full of violence.}%
\verse{And I will bring the worst of the nations, and they will take possession of their houses, and I will put to an end the pride of the mighty ones, and their sanctuaries will be defiled.}%
\verse{Anguish comes, and they will seek peace, and there will be none.}%
\verse{Calamity upon calamity will come, and rumor will be upon rumor. And they will seek a vision from a prophet, but instruction will perish from the priest and counsel from the elders.}%
\verse{The king will mourn, and the prince will be dressed with despair, and the hands of the people of the land will tremble. According to their way I will do to\lebnote{Or “deal with”} them, and according to their judgments I will judge them, and they will know that I am Adonai.’”}%
\end{biblechapter}%
\begin{biblechapter}% Ezekiel 8
\verseWithHeading{Abominations in the Temple and in Jerusalem}{And then\lebnote{“and it happened”} in the sixth year, in the sixth month, on the fifth day of the month, I was sitting in my house, and the elders of Judah were sitting before me.\lebnote{“to the face of me”} And the hand of the Lord Adonai fell on me there,}%
\verse{and I saw,\lebnote{Or “looked”} and look! A figure like the appearance of a man;\lebnote{Hebrew “fire”; but reading the BHS note 2b, “a man.”} from the appearance of his waist and below was fire, and from his waist and upward\lebnote{“to above”} was like the appearance of brightness,\lebnote{Or “shining metal”} like the outward appearance of\lebnote{“eye”} amber.}%
\verse{And he sent out the form of a hand, and he took me by a lock of hair of my head, and the Spirit lifted me between earth and heaven, and it brought me to Jerusalem in visions of God to the doorway of the inner gate that faced north,\lebnote{“the \textit{one} facing north”} at which there was the seat of the image of jealousy, which was making jealous.}%
\verse{And look! There was the glory of the God of Israel like the vision that I saw in the valley.}%
\verse{And he said to me, “Son of man,\lnCHR{} lift up now\lebnote{Particle of entreaty, “please/ please now”} your eyes toward the north.”\lnCHS{} And I lifted up my eyes toward the north,\lnCHS{} and, look, there was to the north of the gate of the altar this image of jealousy at the entrance.}%
\verse{And he said to me, “Son of man,\lnCHR{} “Do you see\lebnote{“\textit{Are} you seeing?”} what they are doing — great detestable things that the house of Israel is committing\lebnote{Or “doing”} here so as to drive me from\lebnote{Hebrew “from upon”} my sanctuary, and yet you will see again\lebnote{“you will return you will see”} greater\lebnote{Or “worse”} detestable things.”}%
\verse{And he brought me to the doorway of the courtyard, and I saw, and look! A hole in the wall.}%
\verse{And he said to me, “Son of man,\lnCHR{} dig now\lebnote{Or “please”} through the wall.” And I dug through the wall, and look! There was a doorway.}%
\verse{And he said to me, “Come and see the detestable things, the evil that they are doing\lebnote{Or “making”} here.}%
\verse{And I came, and I saw, and look, all kinds of creatures\lebnote{Hebrew “creature”} and detestable beasts;\lebnote{Hebrew “beast”} and all of the idols of the house of Israel were carved on the wall all around.\lebnote{“around around”}}%
\verse{And seventy men from the elders of the house of Israel, and Jaazaniah, the son of Shaphan, was standing in the midst of them, and they were standing before them.\lebnote{“to the face of them”} Each one had his censer in his hand, and the fragrance of the cloud of the incense was going up.}%
\verse{And he said to me, “Have you seen, son of man,\lnCHR{} what the elders of the house of Israel are doing in the dark, each in the inner rooms of his idol, for they are saying, ‘Adonai is not seeing us; Adonai has abandoned the land.’”}%
\verse{And he said to me, “Still you will see again\lnCHT{} greater detestable things that they are doing.”}%
\verse{And he brought me to the doorway of the gate of the house of Adonai that is toward the north, and look! There were the women sitting weeping for Tammuz.}%
\verse{And he said to me, “Have you seen, son of man?\lnCHR{} Still you will see again\lnCHT{} greater detestable things than these.”}%
\verse{And he brought me to the inner courtyard of the house of Adonai, and look, at the doorway of the temple of Adonai, between the portico and the altar, there were about twenty-five men with their backs to the temple of Adonai and their faces toward the east, and they were bowing down toward the east before the sun.}%
\verse{And he said to me, “Have you seen, son of man?\lnCHR{} Was it too small a thing\lebnote{“was it \textit{too} little”} for the house of Judah to do\lebnote{“from doing”} the detestable things that they did here? For they filled up the land with violence, and they provoked me to anger again,\lebnote{“they returned to provoke me to anger”} and look! They are putting the branch to their nose.}%
\verse{And so I will act in rage, and my eye will not take pity, and I will not have compassion, and they will cry in my ear with a loud voice, and I will not hear them.”}%
\end{biblechapter}%
\begin{biblechapter}% Ezekiel 9
\verseWithHeading{Adonai’s Avenging Messenger Destroys the Wicked in the City}{And he called in my ears with a loud voice, saying,\lnCHU{} “They have come near,\lebnote{Or “Come near” (imperative)} the punishers of the city, and each with his weapon of destruction in his hand.}%
\verse{And look! Six men coming from the way of the upper gate\lebnote{“the gate of the upper”} that faced northward,\lebnote{“faced north + directive”} and each with his weapon for\lebnote{Or “of”} shattering in his hand; and one man was in the midst of them, dressed in linen, and the\lebnote{Or “his”} writing case of the scribe was at his side. And they came and stood beside the bronze altar.}%
\verse{And the glory of the God of Israel lifted itself up from upon the cherub that he was on\lebnote{Hebrew “which he was on it”} and went to the threshold of the house,\lnCHV{} and he called to the man who was clothed in linen with a scribal writing case at his side.\lebnote{“whom a writing case of a scribe \textit{was} at his side”}}%
\verse{And Adonai said to him, “Go through in the midst of the city, in the midst of Jerusalem, and you must place a mark on the foreheads of the men who are groaning and lamenting\lebnote{“the men the groaning and the lamenting”} about all of the detestable things that are being done\lebnote{“the being done”} in the midst of her.”}%
\verse{And to the others he said in my hearing, “Go through the city behind him and kill! Your eyes shall not take pity, and you shall not have compassion.}%
\verse{You must kill totally\lebnote{“to destruction”} old man, young man and young woman, and little children and women, but concerning\lebnote{“upon”} every man with the mark on him\lebnote{“which is on him the mark”} you must not approach; and from my sanctuary you must begin.” And they began with the old men who were before\lebnote{“\textit{were} to the face of”} the house.\lnCHV{}}%
\verse{And he said to them, “Defile the house\lebnote{Or “temple”} and fill the courtyards with the dead; go out! And they went out, and they killed in the city.}%
\verse{And then\lebnote{“and it was”} as they were striking,\lebnote{“in their striking”} and I was left behind, I fell on my face, and I cried out, and I said, “Ah, Lord Adonai! “Will you be destroying all of the remnant of Israel while you pour out your rage\lebnote{“when pouring out your rage”} on Jerusalem?”}%
\verse{And he said to me, “The guilt of the house of Israel and Judah is exceedingly\lebnote{“with very very”} great, and the land is filled with bloodguilt, and the city is full of injustice. For they say Adonai abandoned the land, and Adonai does not see.\lebnote{“not Adonai \textit{is} seeing}}%
\verse{And I, my eye will not take pity, and I will not have compassion; their way I will bring on their head.”}%
\verse{And look! The man clothed in linen and with a writing case at his side\lebnote{“which/whom the writing-case \textit{was} at his side”} was bringing back a word, saying,\lnCHU{} “I have done all that you commanded me.”}%
\end{biblechapter}%
\begin{biblechapter}% Ezekiel 10
\verseWithHeading{God’s Presence Leaves the Temple}{And I looked, and look! On the expanse that was above the head of the cherubim something like a stone of sapphire,\lebnote{Or “lapis lazuli”} and like the appearance of the shape of a throne it appeared above them.}%
\verse{And he spoke\lebnote{Or “said”} to the man clothed in linen and said, “Go in among the wheel area under the cherubim\lebnote{“to between to the wheel to under the cherub”} and fill the hollow of your hands with coals of fire from among the cherubim, and toss them on the city.” And he went right before my eyes.}%
\verse{Now\lebnote{Or “And”} the cherubim were standing on the south of the temple when the man went,\lebnote{“at his coming the man”} and the cloud filled the inner courtyard.}%
\verse{And the glory of Adonai rose up from on the cherub toward\lebnote{Or “onto”} the threshold of the house,\lnCHW{} and the house\lnCHW{} was filled with the cloud, and the courtyard was filled with the brightness of the glory of Adonai.}%
\verse{And the sound of the wings of the cherubim was heard as far as\lebnote{“up to”} the outer courtyard; it was like the voice of God Shaddai\lebnote{Often translated “Almighty”} when he speaks.\lebnote{“in his speaking”}}%
\verse{And then at his command\lebnote{Or “when he commanded”} to the man clothed in linen, saying,\lebnote{“to say”} “Take fire from among the wheel area from among the cherubim,” he went and stood beside the wheel.}%
\verse{Then the cherub stretched out his hand from among\lnCHX{} the cherubim toward the fire that was among\lnCHX{} the cherubim, and he lifted up and gave\lebnote{Or “placed”} it into the hollow of the hand of the man clothed with linen, and he took it, and he went out.}%
\verse{And there appeared for\lebnote{“to”} the cherubim the form of a human hand\lebnote{“a hand of a man”} under their wings.}%
\verse{And I saw, and look, four wheels beside the cherubim, one wheel beside each cherub,\lebnote{“wheel one beside the cherub one, and wheel one beside the cherub one”} and the appearance of the wheels was like the outward appearance of turquoise stone.\lebnote{“stone of turquoise”}}%
\verse{Their appearance\lebnote{That is, of the wheels} was the same for each of the four of them,\lebnote{“likeness one \textit{was} for the four of them”} as if the wheel was in the midst of a wheel.}%
\verse{When they went\lebnote{“In their going”} to the four of their directions that they went, they did not change direction when they went,\lebnote{“At their going”} for the place to which the head\lebnote{Or “first” or “lead wheel”} turned, they went behind him; they did not change direction at their going.}%
\verse{And their whole body,\lebnote{“the whole of their body”} and their rims, and their spokes, and their wings, and the wheels were full of eyes all around — the wheels for the four of them.\lebnote{“for the four of them their wheels”}}%
\verse{Concerning the wheels, he was calling them “the wheelwork” in my hearing.\lebnote{“in my ears”}}%
\verse{And each one had four faces;\lebnote{“four faces \textit{were} to each”} the face of the one was the face of a cherub, and the face of the second was the face of a human, and the face of the third was the face of a lion, and the fourth was the face of an eagle.}%
\verse{And the cherubim rose; that is, the living creatures\lebnote{Hebrew “creature”} that I saw at the Kebar River.\lnCHY{}}%
\verse{And when the cherubim went,\lebnote{“at the going of the cherubim”} the wheels beside them went; and when the cherubim lifted their wings to rise up from the earth,\lebnote{“and at the lifting of the cherubim wings their to rise from on the earth”} the wheels also did not turn aside from beside them.}%
\verse{When they stood,\lebnote{“at their standing”} they stood, and at their rising, they rose with them, for the spirit of the living creatures was in them.}%
\verse{And the glory of Adonai went out from\lebnote{Hebrew “from on”} the threshold of the temple, and it stood above the cherubim.}%
\verse{And the cherubim lifted up their wings, and they rose from the earth before my eyes. At their going out, the wheels were\lebnote{Hebrew “to”} beside them. And he stood at the doorway of the eastern gate of the temple\lebnote{Or “house”} of Adonai, and the glory of the God of Israel was over them.\lebnote{“on them from to above”}}%
\verse{This was\lebnote{Or “these were”} the living creature that I saw under the God of Israel at the Kebar River,\lnCHY{} and I knew that they were cherubim.}%
\verse{Each one had four faces,\lebnote{“four faces \textit{were} for one/each”} and there were four wings for each, and the likeness of human hands\lebnote{“hands of a man”} was under their wings.}%
\verse{And the likeness of their faces, they were the faces that I saw at the Kebar River;\lebnote{Litearlly “river of Kebar”} thus was their appearance, and they each went straight ahead.\lebnote{“to the opposite of face his”}}%
\end{biblechapter}%
\begin{biblechapter}% Ezekiel 11
\verseWithHeading{Judgment On Evil Rulers}{And the Spirit lifted me up, and it brought me to the eastern gate, the one facing east, of the temple\lebnote{Or “house”} of Adonai. And look, there were twenty-five men in the doorway of the gate, and I saw Jaazaniah the son of Azzur in the midst of them, and Pelatiah the son of Benaiah, the commanders of the people.}%
\verse{And he said to me, “Son of man,\lnCHZ{} these are the men who devise mischief,\lebnote{“the ones devising mischief”} and who are offering bad counsel in this city,}%
\verse{who are saying, “The building of houses is not near; the city is the pot, and we are the flesh.}%
\verse{Therefore prophesy against them! Prophesy, son of man!”\lnCHZ{}}%
\verse{And the Spirit of Adonai fell on me, and he said to me, “Say, ‘thus says Adonai: “This is what you think, house of Israel, and I myself know them,\lebnote{Hebrew “it”} the thoughts of your spirit.}%
\verse{You made your slain ones numerous in this city, and you have filled its streets with slain ones.”’\lebnote{Hebrew “slain one”}}%
\verse{Therefore thus says the Lord Adonai: “Your slain ones whom you put in its midst, they are the flesh, and it is the pot, and I\lebnote{Hebrew “he”} will bring you out from its midst.}%
\verse{The sword you have feared, and the sword I will bring against you!” declares\lnCIA{} the Lord Adonai.}%
\verse{“And I will bring you out from its midst, and I will give you into the hand of strangers, and I will execute judgments against you.}%
\verse{By the sword you will fall at the border of Israel; I will judge you, and you will know that I am Adonai.}%
\verse{It will not be as a pot to you, and so you would be\lebnote{Or “will be”} in the midst of it as flesh, for at the border of Israel I will judge you.}%
\verse{And you will know that I am Adonai, whose\lebnote{Hebrew “which”} rules you did not follow, and whose\lebnote{Hebrew “my”} regulations you did not do, but according to the regulations of the nations that are around you, you acted.”’\lebnote{Or “did”}}%
\verse{And it happened that as I was prophesying,\lebnote{“at my prophesying”} Pelatiahu the son of Benaiahu died! And I fell on my face, and I cried with a loud voice, and I said, “Ah, Lord Adonai, you are making a complete destruction of the remnant of Israel!”}%
\verse{And the word of Adonai came\lebnote{“was”} to me, saying,\lebnote{Or “to say”}}%
\verse{“Son of man,\lnCHZ{} your brothers, your brothers, the men of your redemption, and all of the house of Israel, all of it, who said concerning the inhabitants of Jerusalem,\lebnote{“which they said concerning them the inhabitants of Jerusalem”} ‘They are far from Adonai, therefore to us this land was given as a possession.’}%
\verse{Therefore say, ‘Thus says the Lord Adonai: “Though I have removed them far away among the nations and though I have scattered them among the countries, yet I was a sanctuary to them for a little while in the countries to which they have gone.’”\lebnote{“which they came there”}}%
\verse{Therefore say, ‘Thus says the Lord Adonai: “And I will assemble you from the peoples, and I will gather you from the countries to which you were scattered among them, and I will give the land of Israel to you.}%
\verse{And when they come there, then they will remove all of its vile idols and all of its detestable things from it.}%
\verse{And I will give to them one heart, and a new spirit I will give\lebnote{Or “put”} in their inner parts. And I will remove their\lebnote{Hebrew “the”} heart of stone from their body, and I will give to them a heart of flesh,}%
\verse{so that they may walk in my statutes, and they will keep my regulations, and they will do them, and they will be to me a people, and I myself will be to them as God.}%
\verse{But to\lebnote{Or “and to”} the heart of their abominations and the detestable things their heart is going. I will bring their way on their head,” declares\lnCIA{} the Lord Adonai.’”}%
\verse{Then\lebnote{Or “And”} the cherubim lifted up their wings, and their\lebnote{Or “the”} wheels were beside them, and the glory of the God of Israel was over them.\lebnote{“from to above”}}%
\verse{And the glory of Adonai went up from the midst of the city, and it stood still on the mountain that is to the east of the city.}%
\verse{And the Spirit lifted me up, and it brought me to Chaldea, to the exiles,\lnCIB{} in the vision by the spirit of God; and the vision that I had seen left me.\lebnote{“went up from \textit{being} upon me”}}%
\verse{And I spoke to the exiles\lnCIB{} all of the words of Adonai that he had shown me.}%
\end{biblechapter}%
\begin{biblechapter}% Ezekiel 12
\verseWithHeading{The Announcement of the Imminent Coming of the Exile}{And the word of Adonai came\lnCIC{} to me, saying,\lnCID{}}%
\verse{“Son of man,\lnCIE{} you are dwelling in the midst of the house of rebellion who has eyes to see and they do not see;\lebnote{“which eyes \textit{are} for them to see and not they see”} they have ears to hear,\lebnote{“ears \textit{are} for them to hear”} and they do not hear, for they are a house of rebellion.}%
\verse{And you, son of man,\lnCIE{} prepare for yourself\lnCIF{} the baggage of an exile, and go into exile by day before their eyes. And you must go into\lnCIG{} exile from your place to another place before their eyes; perhaps they will see that they are a house of rebellion.}%
\verse{And you must bring out your baggage like the baggage of an exile by day before their eyes, and you must go out in the evening before their eyes like those who go into exile.\lebnote{“like \textit{the} goers out of an exile”}}%
\verse{Before their eyes dig through for yourself,\lnCIF{} through the wall, and you must bring the baggage out through it.}%
\verse{Before their eyes, on your\lebnote{“the”} shoulder, you must lift up the baggage in the dusk, and your face you must cover, so that you may not see the land, for I make you as a sign to the house of Israel.”}%
\verse{And I did just as\lebnote{“so as that”} I was commanded; my baggage, like the baggage of an exile, I brought out by day, and in the evening I dug through for myself\lebnote{Hebrew “me”} into\lnCIG{} the wall with my hand in the dusk; I brought the baggage on my shoulder; I carried it before their eyes.}%
\verse{And the word of Adonai came\lnCIC{} to me in the morning, saying,\lnCID{}}%
\verse{“Son of man,\lnCIE{} did not they, the house of Israel, the house of rebellion, say to you, ‘What are you doing?’}%
\verse{Say to them, ‘Thus says the Lord Adonai: “This oracle is about the prince in Jerusalem and the entire\lebnote{“all of”} house of Israel who are among them.”’\lebnote{“which they \textit{are} in the midst of them”}}%
\verse{Say, ‘I am your sign, and just as\lebnote{“as that”} I did, so will it be done to them in the exile; into\lnCIG{} captivity they will go.’}%
\verse{And the prince who is in the midst of them, on his shoulder he will carry the baggage in the dusk, and\lebnote{Or “in the dusk he will go out”} he will go out; the wall will be dug through\lebnote{“through the wall they will dig”} to bring him through it; and he will cover his face so that he will\lebnote{Or “may”} not see the land with his\lebnote{Hebrew “the”} eye.}%
\verse{And I will spread out my net on him, and he will be captured in my hunting snare, and I will bring him to Babylon, the land of the Chaldeans, but he will not see it, and there he will die.}%
\verse{And all who are around him, his help\lebnote{Or “support”} and all of his troops I will scatter in every direction,\lebnote{“to every direction of the wind”} and I will draw\lebnote{Or “unsheathe”} the sword behind them.}%
\verse{And they will know that I am Adonai when I scatter them\lebnote{“at my scattering them”} among the nations and I scatter them in the countries.}%
\verse{But\lebnote{Or “And”} I will spare from them a few men\lebnote{“men of number”} from the sword, from the famine and from the plague, so that they may tell of all their abominations among the nations to which they will go, and they will know that I am Adonai.”}%
\verse{And the word of Adonai came\lnCIC{} to me, saying,\lnCID{}}%
\verse{“Son of man,\lnCIE{} you must eat your food with trembling, and your water with shuddering, and with anxiety you must drink.}%
\verse{And you must say to the people of the land, ‘Thus says the Lord Adonai to the inhabitants of Jerusalem about the land of Israel: “They will eat their food with anxiety and their water they will drink with horror, because their land will be desolate from\lebnote{Or “of”} its fullness because of all the violence of those who are dwelling in it.\lebnote{“the ones dwelling in it”}}%
\verse{And the inhabited cites will be desolate,\lebnote{“the being inhabited \textit{cities} they will be desolate”} and the land will be a desolation, and you will know that I am Adonai.”’”}%
\verse{And the word of Adonai came\lnCIC{} to me, saying,\lnCID{}}%
\verse{“Son of man,\lnCIE{} what is this proverb you people have\lnCIH{} about the land of Israel, saying,\lnCIH{} ‘The days are prolonged,\lebnote{“they are long the days”} and every vision has come to nothing.’}%
\verse{Therefore say to them, ‘Thus says the Lord Adonai: “I will put an end to this proverb, and they will not quote it as a proverb again in Israel.”’ But\lebnote{Hebrew “But if”} say to them, ‘The days are near, and also the word of every vision.’}%
\verse{For there will not be any longer any false vision\lebnote{“every vision of falseness”} or flattering divination in the midst of the house of Israel.}%
\verse{For I, Adonai, I will speak what I will speak as a word, and it will be done. It will not prolong itself any longer, for in your days, house of rebellion, I will speak a word and I will fulfill it!” declares\lnCII{} the Lord Adonai.}%
\verse{And the word of Adonai came\lnCIC{} to me, saying,\lnCID{}}%
\verse{“Son of man,\lnCIE{} look! The house of Israel is saying, ‘The vision that he is seeing is for many days from now, and for distant times he is prophesying.’}%
\verse{Therefore say to them, ‘Thus says the Lord Adonai: “None of my words will be prolonged any longer that I speak as a word, and it will be fulfilled!”’” declares\lnCII{} the Lord Adonai.}%
\end{biblechapter}%
\begin{biblechapter}% Ezekiel 13
\verseWithHeading{Condemnation of False Prophets}{And the word of Adonai came\lebnote{“was”} to me, saying,\lebnote{Or “to say”}}%
\verse{“Son of man,\lnCIJ{} prophesy to\lebnote{Or “against”} the prophets of Israel who are prophesying,\lebnote{“the \textit{ones} prophesying”} and you must say to those who are prophets out of their own imagination,\lnCIK{} ‘Hear the word of Adonai!’}%
\verse{Thus says the Lord Adonai: ‘Alas, for the foolish prophets who are going after their own spirit, and they did not see anything!\lebnote{“and not they saw”}}%
\verse{Your prophets have been like foxes\lebnote{Or “jackals”} among ruins.}%
\verse{You did not go up into\lnCIL{} the breaches and repair a wall for the house of Israel to stand firm in the battle on the day of Adonai.}%
\verse{They saw falseness and a lying divination,\lebnote{“a divination of a lie”} the ones saying, ‘Declaration of Adonai!’ and Adonai did not send them, and they wait for the confirmation of their word.\lebnote{“to confirm a word”}}%
\verse{Have you not seen a false vision\lebnote{“vision of falseness”} and spoken a lying divination,\lebnote{“divination of a lie”} and you said, ‘Declaration of Adonai!’ but I myself did not speak.\lebnote{“I not I spoke”}}%
\verse{Therefore thus says the Lord Adonai: ‘Because of your speaking falseness and because you envisioned a lie, therefore, look! I am against you!” declares\lnCIM{} the Lord Adonai.}%
\verse{“And it will be my hand against the prophets who are seeing falseness and who are practicing lying divination. They will not be in the council of my people, and they will not be written down in the record book of the house of Israel, and into\lnCIL{} the land of Israel they will not come, and then you will know that I am the Lord Adonai.}%
\verse{Because, yes, because they led my people astray, saying\lebnote{“to say”} ‘Peace!’ And there is not peace. And when anyone\lebnote{Hebrew “he”} builds a flimsy wall, look, they coat it with whitewash.}%
\verse{Say to those covering it with whitewash that it will fall; there will be a torrent of rain,\lebnote{“rain flooding”} and I will give stones of hail; they will fall!\lebnote{Or “and I will give great hailstones. They will fall”} And a windstorm\lebnote{“wind of a storm”; or “with respect/concerning you stones of hail, they will fall”} will burst forth!}%
\verse{And look! When the wall falls, will it not be said to you, ‘Where is the whitewash with which you covered it?’}%
\verse{Therefore thus says the Lord Adonai: ‘And so I will let burst forth a windstorm\lebnote{“a wind of storm”} in my rage, and there will be a torrent of rain\lebnote{“rain flooding there will be”} in my anger, and hailstones in my rage for complete destruction.\lebnote{“to complete destruction”}}%
\verse{And I will break down the wall that you covered with whitewash, and I will knock it to the ground, and its foundation will be revealed, and it will fall, and you will come to an end in the midst of it, and you will know that I am Adonai!}%
\verse{And I will fully vent my rage against the wall and against those covering it with whitewash, and I will say to you, “The wall is no more,\lebnote{“not \textit{is} the wall”} and the people covering it are no more,\lebnote{“not \textit{are} the \textit{people} whitewashing him”}}%
\verse{that is, the prophets of Israel, the ones prophesying concerning Jerusalem and the ones seeing visions of peace, and there is not peace!’” ” declares\lnCIM{} the Lord Adonai.}%
\verse{“And you, son of man,\lnCIJ{} place your face toward the daughters of your people who prophesy from their imagination,\lnCIK{} and prophesy against them!}%
\verse{And you must say, ‘Thus says the Lord Adonai: “Woe to those who sew\lebnote{“\textit{the} sewers of”} magic charm bands on all the wrists\lebnote{Or “joints”} of the hands of my people and those who make the veils\lebnote{“the makers of the veils”} that are on the head of people of every height, to ensnare people’s lives!\lnCIN{} Will you ensnare the lives\lnCIN{} of my people and keep yourselves alive?\lebnote{“and lives/souls for you, you keep alive”}}%
\verse{And you defiled me among my people for a handful of barley and for morsels of bread to kill persons who should not die and to keep alive persons who should not live by means of your lies to my people who are listening to your lies.”’\lebnote{Hebrew “lie”}}%
\verse{Therefore thus says the Lord Adonai: ‘Look! I am against your magic charm bands with which you are ensnaring persons\lebnote{“which you \textit{are} ensnaring with them the souls/individuals”} as birds; I will tear them from your arms, and I will release the persons that you are ensnaring, treating persons as birds.}%
\verse{And I will tear off your veils, and I will deliver my people from your hand, and they will not any longer be in your hand as prey, and you will know that I am Adonai.}%
\verse{Because you disheartened the heart of the righteous by deception, and I have not\lebnote{Or “did not”} caused him pain, and strengthened the hands of the wicked so that he did not turn\lebnote{Or “not to turn”} from his wicked way to save his life.}%
\verse{Therefore falseness\lebnote{Or “vanity” or “vain/false visions”} you will not see,\lebnote{Or “observe”} and divination you will not practice\lebnote{Or “you will indeed not practice divination”} any longer, and I will rescue my people from your hand, and you will know that I am Adonai!’”}%
\end{biblechapter}%
\begin{biblechapter}% Ezekiel 14
\verseWithHeading{Condemnation of Israel’s Idolatrous Elders}{And men from the elders of Israel came to me, and they sat before me.\lebnote{“to the face of me”}}%
\verse{And the word of Adonai came\lnCIO{} to me, saying,\lnCIP{}}%
\verse{“Son of man,\lnCIQ{} these men took up their idols into their heart and they placed before themselves\lnCIR{} a stumbling block of their guilt.\lnCIS{} Should I really let myself be consulted by them?\lebnote{“Surely I will let me be consulted by them?”}}%
\verse{Therefore speak with them, and you must say to them, ‘Thus says the Lord Adonai: “Every person from the house of Israel who brings up his idols into his heart and places before himself\lnCIR{} a stumbling block of his guilt\lnCIS{} and yet he comes to the prophet, I Adonai, I will respond to him through this act\lebnote{Hebrew “it”} with respect to the multitude\lebnote{According to the \textit{Kethib}; the \textit{Qere} reads “I will respond to him, he who comes with the multitude of his idols”} of his idols,}%
\verse{so that I may take hold of the house of Israel by their heart, those who are estranged from me, all of them, through their idols.’}%
\verse{Therefore say to the house of Israel, ‘Thus says the Lord Adonai: “Return,\lebnote{Or “Repent”} and turn away from your idols and from all of your detestable things! Turn away your faces!”’}%
\verse{For each person\lebnote{“a man, a man” or “each one, each one”} from the house of Israel, and from the alien who dwells as an alien in Israel, who separates himself from following me\lnCIT{} and brings up his idols into his heart and places before himself\lebnote{“he puts beside his face”} a stumbling block of his guilt\lnCIS{}, and yet he comes to the prophet to consult him concerning me, I Adonai will answer\lebnote{Or “am answering”} him myself.}%
\verse{And I will set my face against that person, and I will make him to\lnCIU{} be a sign and make him into\lebnote{Hebrew “to”} the subject of proverbs; and I will cut him off from the midst of my people, and you will know I am Adonai.}%
\verse{And the prophet, if he is deceived and he speaks a word, I Adonai, I have deceived that prophet, and I will stretch out my hand against him, and I will destroy him from the midst of my people Israel.}%
\verse{And they will bear their guilt, like\lnCIU{} the guilt of the inquirer so\lebnote{Or “like”} the guilt of the prophet will be,}%
\verse{so that the house of Israel will not go astray again from me,\lnCIT{} and they will not make themselves unclean again with all of their transgressions, and they will be for me a people, and I will be for them as God,” declares\lnCIV{} the Lord Adonai.}%
\verseWithHeading{Jerusalem will not be Spared}{And the word of Adonai came\lnCIO{} to me, saying,\lnCIP{}}%
\verse{“Son of man,\lnCIQ{} when a land sins against me by acting very unfaithfully,\lebnote{“acting unfaithfully doing infidelity”} then I will stretch out my hand against it, and I will break for it the supply of food,\lebnote{“staff of bread”} and I will send against it famine, and I will cut it off, both human and animal.}%
\verse{And if even the three of these men were in the midst of it — Noah, Daniel, and Job — they, through their righteousness, would save only themselves!”\lnCIW{} declares\lnCIV{} the Lord Adonai.}%
\verse{“If a fierce animal I should let cross through the land, and it should make it childless and it will be a desolation, so that there will not be one crossing over the land due to\lebnote{“from the face of”} the presence of the animal,}%
\verse{even if these three men were in the midst of it, as surely as I live,”\lnCIX{} declares\lnCIV{} the Lord Adonai, “surely they will not save sons and daughters; they themselves alone,\lnCIY{} they will be saved, but the land will be a desolation.}%
\verse{Or, if I bring a sword over that land, and I say, ‘Sword, let it cross over into\lebnote{Hebrew “in”} the land!’ And I will cut off from it both human and animal.}%
\verse{And if these three men were in the midst of it, as surely as I live,”\lnCIX{} declares\lnCIV{} the Lord Adonai, “they will not save sons and daughters, but they alone\lnCIY{} will be saved.}%
\verse{And if I were to send a plague to that land, and I pour out my rage on it with blood to cut it off, both human and animal,}%
\verse{and if Noah, Daniel, and Job were in the midst of her, as surely as I live,”\lnCIX{} declares\lnCIV{} the Lord Adonai, “surely not a son, surely not a daughter will they save by their righteousness; they would save themselves.”\lnCIW{}}%
\verse{For thus says the Lord Adonai: “How much more when I send my four punishments — the evil sword, and famine, and a fierce animal, and a plague — to Jerusalem to cut it off, both human and animal!}%
\verse{But\lebnote{Or “And”} look! A remnant will be left over in it, sons and daughters who will be brought out.\lebnote{“the ones being brought out”} Look! They are coming out to you, and you will see their way, and with their deeds you will be consoled with respect to\lebnote{Hebrew “for”} the evil that I brought over Jerusalem, all of it that I brought over\lebnote{Or “upon”} it.}%
\verse{And they will console you when you see their way and their deeds, and you will know that not for nothing I did all that I did in it,” declares\lnCIV{} the Lord Adonai.}%
\end{biblechapter}%
\begin{biblechapter}% Ezekiel 15
\verseWithHeading{Jerusalem a Useless Vine}{And the word of Adonai came\lebnote{“was”} to me, saying,\lebnote{“to say”}}%
\verse{“Son of man, how will the wood of the vine be better than\lebnote{“more than”} any of the wood of\lebnote{“all of the wood of”} the branch which is among the trees of the forest?\lnCIZ{}}%
\verse{Can wood be taken from it to make anything, or can they take a tent peg from it to hang on it any object?\lebnote{Or “vessel”}}%
\verse{Look! It is given to the fire as fuel. The fire will consume two of its ends, and its middle will be charred. Is it useful for anything?}%
\verse{Look! When it is in perfect condition,\lebnote{“In being it perfect”} it will not be used for anything, how much less\lebnote{“indeed”} when the fire has consumed it\lebnote{“for \textit{if} fire has consumed it”} and it is charred; and then can it be used again for anything?}%
\verse{Therefore thus says the Lord Adonai: Just like\lebnote{“As that”} the wood of the vine among the trees of the forest\lnCIZ{} which I gave to the fire for fuel, so I have given the inhabitants of Jerusalem.}%
\verse{And I will set my face against them. From the fire they mayescape,\lebnote{“go out”} but the fire will yet consume them, and they will know that I am Adonai when I set my face against them.\lebnote{“in to set me my face against them”}}%
\verse{And I will make the land a desolation because they acted very unfaithfully!”\lebnote{“they displayed infidelity”} declares\lebnote{“declaration of”} the Lord Adonai.}%
\end{biblechapter}%
\begin{biblechapter}% Ezekiel 16
\verseWithHeading{God’s Covenant Grace to Unfaithful and Rebellious Israel}{And the word of Adonai came\lebnote{“was”} to me, saying,\lnCJA{}}%
\verse{“Son of man,\lebnote{Or “mortal,” or “son of humankind”} make known to Jerusalem its detestable things,}%
\verse{and you must say, ‘Thus says the Lord Adonai to Jerusalem: Your origin and your birth were from the land of the Canaanites,\lebnote{Hebrew “Canaanite”} your father was an Amorite, and your mother was a Hittite.}%
\verse{And as for your birth, on the day you were born\lnCJB{} your umbilical cord was not cut, and you were not thoroughly washed clean with water,\lebnote{“with water not you were cleansed for cleansing”} and you were not thoroughly rubbed with salt,\lebnote{“being rubbed with salt you were not rubbed with salt”} and you were not carefully wrapped in strips of cloth.\lebnote{“being wrapped in strips of cloth not you were wrapped in strips of cloth”}}%
\verse{No eye took pity on you\lebnote{“not took pity on you an eye”} to do to you one of these things to show compassion for you, and you were thrown into the open field\lebnote{“to the face of the field”} in their despising of you on the day you were born.\lnCJB{}}%
\verse{“‘And I passed by you, and I saw you kicking about in your blood, and I said to you in your blood, “Live!}%
\verse{Grow up;\lebnote{Or “A myriad \textit{I have made you}”} I will make you like a plant of the field.” And you grew up, and you became tall and reached full womanhood.\lebnote{“ornament of ornaments”} Your breasts were formed, and your hair had grown, but you were naked and bare.}%
\verse{“‘And I passed by you, and I saw you, and look, your time of lovemaking had come,\lebnote{“your time \textit{was the} time of lovemaking”} and so I spread out my hem over you, and I covered your nakedness, and I swore to you, and I entered into\lnCJC{} a covenant with you,’ declares\lnCJD{} the Lord Adonai, ‘and you became mine.\lebnote{“and you were to me”}}%
\verse{And I washed you with water, and I rinsed off your blood from on you, and I anointed you with oil.}%
\verse{And I clothed you with beautiful finished cloth, and I put sandals on you of fine leather, and I bound you in fine linen, and I covered you with costly fabric.}%
\verse{And I adorned you with ornaments,\lebnote{Hebrew “ornament”} and I put a bracelet on your arms and a necklace on your neck.}%
\verse{And I put an ornamental ring on your nose and earrings on your ears and a beautiful crown\lebnote{“a crown of splendor”} on your head.}%
\verse{And you adorned yourself with gold and silver, and your clothing was fine linen and costly fabric and beautiful finished cloth; you ate finely milled flour and honey and olive oil, and you became exceedingly beautiful; you were fit to be a queen.\lebnote{“you could succeed to queenship”}}%
\verse{And a name went out for you among the nations because of your beauty, for it was perfect because of my majesty that I bestowed on you,’ declares\lnCJD{} the Lord Adonai.}%
\verse{‘And you trusted in your beauty, and you prostituted on account of your name, and you poured out your fornication on every one passing by, saying, “Let it be his.”}%
\verse{And you took from your clothes and you made for yourself colorful\lebnote{Or “gaudy”} shrines, and you prostituted on them; this had not happened before,\lebnote{“Not \textit{were} happening”} and it will not continue to happen.\lebnote{“and not it will be”; Hebrew text unclear}}%
\verse{And you took your beautiful jewelry\lebnote{“the jewelry of splendor”} made from my gold and from my silver that I had given to you, and you made for yourself male images,\lebnote{“images of a male”} and you prostituted with them.}%
\verse{And you took the clothes of your beautiful finished cloth, and you covered them and my oil and my incense you set before them.\lnCJE{}}%
\verse{And my bread that I gave to you, finely milled flour and oil and honey with which I fed you, then you set it before them\lnCJE{} as a pleasing odor,\lebnote{“a scent of an incense”} and so it was,’ declares\lnCJD{} the Lord Adonai.}%
\verse{‘And you took your sons and your daughters whom you had borne for me, and you sacrificed them to them to be eaten,\lebnote{Or “consumed”} as if your whorings were not enough.\lebnote{“\textit{it was too} little from your whorings”}}%
\verse{And you slaughtered my children, and you gave them to be sacrificed to them.\lebnote{“to cause them to cross over to them”}}%
\verse{And in all of your detestable things and your fornication you did not remember the days of your childhood when you were naked and bare,\lebnote{“at your being naked and bare”} when you were kicking about in your blood.}%
\verse{“‘And then after all of your evil! Woe, woe, to you!’ declares\lnCJD{} the Lord Adonai.}%
\verse{‘And then you built for yourself a mound, and you made for yourself a high place in every public square.}%
\verse{At the head of every street you built your high place and you disgraced your beauty and you spread your feet for every passerby, and you increased your whoring.}%
\verse{And you prostituted with the Egyptians,\lebnote{“sons/children of Egypt”} your neighbors who were sexually aroused,\lebnote{“\textit{those} being great of flesh”} and you increased your fornication to provoke me.}%
\verse{And look! I stretched out my hand against you, and I reduced your portion, and I gave you into\lnCJC{} the desire of your haters, the daughters of the Philistines, who were ashamed because of your lewd conduct.\lebnote{“your way \textit{of life}”}}%
\verse{And you prostituted with the Assyrians\lebnote{“sons/children of Assyria”} on account of your insatiable lust,\lebnote{“of no your satisfaction”} and you prostituted with them, and still you were not satisfied.}%
\verse{And you increased your fornication to the land of traders,\lebnote{Hebrew “trader”} to Chaldea, and even with this you were not satisfied.}%
\verse{How hot with fever was your heart!’ declares\lnCJD{} the Lord Adonai. ‘When you did\lebnote{“at/in you to do”} all of these actions of a headstrong female prostitute,\lebnote{“deeds of a woman prostituting domineering”}}%
\verse{when you built\lebnote{“at/in building your”} your mound at the head of every street and your high place you made on every street, yet\lebnote{Hebrew “and”} you were\lnCJF{} not like a prostitute, as you were scorning your harlot’s wages.}%
\verse{O, adulterous woman! In the place of her husband she received strangers.}%
\verse{To all prostitutes they give a fee, but you, you gave your gifts to all of your lovers, and you bribed them to come to you from all around in your whorings!}%
\verse{And it was the opposite with you from the other women in your whorings; no one solicited you as a prostitute.\lebnote{“and after you not he solicited a prostitute”} You gave\lebnote{“And at your giving”} a harlot’s wages, and a harlot’s wages were not given to you, and so it was just the opposite.}%
\verse{“‘Therefore, prostitute, hear the word of Adonai.}%
\verse{Thus says the Lord Adonai: Because of the pouring out of your shame and because your nakedness was revealed in your whoring with your lovers, and on account of all of the idols of your detestable things, and according to the blood of your children whom you gave to them,}%
\verse{therefore, look! I am gathering all of your lovers whom you were pleased about,\lebnote{“whom you were pleased about them”} and all of whom you loved in addition to all of whom you hated, and I will gather them against you from everywhere,\lebnote{“all around”} and I will uncover your nakedness to them, and they will see all of your nakedness.}%
\verse{And I will judge you with the judgment of women committing adultery and shedding blood, and I will give you the blood of rage and jealousy.}%
\verse{And I will give you into their hand, and they will demolish your mound, and they will break down your high places, and they will strip you of your clothes, and they will take your beautiful jewelry,\lebnote{“the objects of your splendor”} and they will leave you naked and bare.}%
\verse{And they will bring up against you a crowd, and they will stone you with stones,\lebnote{Hebrew “stone”} and they will cut you to pieces with their swords.}%
\verse{And they will burn your houses with fire, and they will execute against you judgments before\lebnote{“to \textit{the} eyes of”} many women, and I will stop you from prostitution, and also a harlot’s wages you will not give\lebnote{Or “pay”} again.}%
\verse{And I will satisfy my rage on you, and my jealousy will turn away from you, and I will be calm, and I will not be angry any longer.}%
\verse{Because\lebnote{“because that”} you did not remember\lebnote{According to the reading tradition (\textit{Qere}) with BHS} the days of your childhood and you made me angry with all of these. And in turn I, look, I will return your way upon your head,’ declares\lnCJD{} the Lord Adonai, ‘and did\lnCJG{} you not do wickedness\lebnote{“the wickedness”} in addition to all of your detestable things?}%
\verse{Look! Everyone quoting a proverb\lebnote{“proverbing a proverb”} against you, he will quote, saying,\lnCJA{} ‘Like her mother is her daughter!’}%
\verse{You are a daughter of your mother who abhorred her husband and her children, and you are a sister of your sisters who abhorred their husbands and their children. Your mother was a Hittite, and your father was an Amorite.}%
\verse{And your elder sister, she is Samaria, and her daughters, who are dwelling on your north, and your younger sister is the one dwelling on your south; she is Sodom and her daughters.}%
\verse{And you have not only gone in their ways, but\lebnote{Or “and”} you also did according to their detestable things. In such a short time\lebnote{“as few little and”} you behaved more corruptly than they in all of your ways.}%
\verse{As surely as I live,’\lebnote{“live I”} declares\lnCJD{} the Lord Adonai, ‘surely your sister Sodom and her daughters did not do as\lebnote{Hebrew “as that”} you and your daughters did.}%
\verse{Look! This was the iniquity of Sodom, your sister: Pride, abundance of food, and prosperous ease\lebnote{“and ease of granting relief”} was to her and to her daughters, and she did not sustain the needy and the poor.\lebnote{“And \textit{the} hand of \textit{the} needy and \textit{the} poor not she did take hold of”}}%
\verse{And they were proud, and they did a detestable thing before me,\lebnote{“to the face of me”} and I removed them because\lebnote{“as that”} I saw it.}%
\verse{And Samaria did not sin according to even half of your sins,\lebnote{“about/as/like the half of sins your not did she sin”} and you caused your detestable things to increase more than they, and you made your sister righteous in comparison with all of your detestable things that you did.\lnCJG{}}%
\verse{Also,\lnCJH{} you bear your disgrace, by which you furnished justification to your sisters\lebnote{Hebrew “sister”} through your sins by which you acted more abominably than they; they were more righteous than you, and also,\lnCJH{} you be ashamed and bear your disgrace through your making your sister righteous.}%
\verse{And I will restore their fortune, the fortunes of Sodom and her daughters, and the fortunes of Samaria and her daughters, and even the fortunes of your captivity in the midst of them,}%
\verse{in order that you may bear your disgrace and you may be put to shame because of all that you did at your consoling them.}%
\verse{And as for your sisters, Sodom and her daughters, they will return to their former state, and Samaria and her daughters, they will return to their former state, and you and your daughters will return to your former state.}%
\verse{And was not Sodom, your sister, a byword in your mouth in the day of your pride}%
\verse{before\lebnote{“at not yet”} your evil was uncovered? It was like the time of the scorn of the daughters of Aram and all those around her, and of the daughters of the Philistines, those all around who are despising you.}%
\verse{Your wickedness and your detestable things, you, you must bear them,’ declares\lnCJD{} Adonai.}%
\verse{‘For thus says the Lord Adonai: And I will do it\lnCJF{} to you according to what you have done,\lebnote{“as that you have done”} who despised the oath to break covenant.}%
\verse{And I, I will remember my covenant with you in the days of your youth, and I will establish with you an everlasting covenant.\lebnote{“a covenant of eternity”}}%
\verse{And you will remember your ways, and you will be ashamed when you take your sisters, both the older and the younger,\lebnote{“the older from you to the younger from you”} and I give them to you as daughters, but not on account of your covenant.}%
\verse{And I, I will establish my covenant with you, and you will know that I am Adonai,}%
\verse{in order that you will remember, and you will be ashamed, and you will not open your mouth again\lebnote{“and not it will be again an opening of mouth”} because of\lebnote{“from the face of”} your disgrace when I forgive you for all that you have done!’” declares\lnCJD{} the Lord Adonai.}%
\end{biblechapter}%
\begin{biblechapter}% Ezekiel 17
\verseWithHeading{Parable of Two Eagles and a Vine}{And the word of Adonai came\lnCJI{} to me, saying,\lebnote{“to say”}}%
\verse{“Son of man,\lebnote{Or “mortal,” or “son of humankind”} tell a riddle and quote a proverb to the house of Israel,}%
\verse{and you must say, ‘Thus says the Lord Adonai: “The great eagle with great wings and long feathers\lebnote{“the eagle the great great of \textit{the} wings long of \textit{the} feather”} and full of variegated plumage\lebnote{“full of \textit{the} plumage which \textit{is} for him \textit{the} variegated”} came to Lebanon and he took the treetop of the cedar.}%
\verse{He plucked the top of its new plant shoot, and he brought it to the land of Canaan and put it in a city of merchants,}%
\verse{and he took from the seed of the land and placed it in fertile soil\lebnote{“a field of seed”} on\lebnote{Or “by”} many waters; like a willow he planted it.}%
\verse{And it sprouted, and it became a vine spreading out,\lebnote{“he was to vine spreading out”} low of height, turning its branches to him, and its roots were under it, and it became a vine,\lebnote{“and it was to a vine”} and it made\lebnote{Or “sprouted”} branches, and it sent out foliage.}%
\verse{“‘“And there was another great eagle, great of wings and with abundant plumage, and look! This vine stretched out its roots toward him and extended its branches to him to water it from the garden bed where it was planted.\lebnote{“from the garden bed of its planting”}}%
\verse{It was planted in\lnCJJ{} a good field by\lnCJJ{} many waters\lebnote{Hebrew “water”} to produce branches\lnCJK{} and to bear fruit to become\lebnote{Hebrew “be”} a beautiful vine.”’\lebnote{“to a vine of splendor”}}%
\verse{Say, ‘Thus says the Lord Adonai: “Will it prosper? Will he not tear out its roots, and will he not make its fruit scaly, and it will wither, and all of the freshness of its vegetation will dry up? And to lift it from its roots will not require great strength or many people.\lebnote{“not with strong arm and with many people”}}%
\verse{And look! Though it is planted, will it prosper? When the east wind strikes it,\lebnote{“at striking her/it the wind of the east”} will it not dry up completely? On the garden bed of its vegetation it will dry up!”’”}%
\verse{And the word of Adonai came\lnCJI{} to me, saying,}%
\verse{“Say now to the rebellious house of Israel, ‘Do you not know what these are?’\lebnote{Or “mean”} Say, ‘Look! The king of Babylon will come to Jerusalem, and he will take its king and its officials, and he will bring\lebnote{Or “take”} them to himself,\lebnote{Hebrew “him”} to Babylon.}%
\verse{And he took from the seed of the kingship, and he made with him a covenant, and he brought him under oath,\lebnote{“and he brought him in\textit{to} oath”} and he took the rulers of the land,}%
\verse{so that they would be a humble kingdom and not lift itself up to keep his covenant in order for it to stand.\lebnote{“to standing her”; or “to let her stand”}}%
\verse{But he rebelled against him by sending his messengers to Egypt to give to him horses and a large army. Will he succeed? Will he escape doing these things, and can he break the covenant and escape?}%
\verse{As I live,’\lnCJL{} declares\lebnote{“declaration of”} the Lord Adonai, ‘surely\lnCJM{} in the place of the king who made\lebnote{“who \textit{he was} making”} him king, who despised his oath and who broke his covenant with him — in the midst of Babylon he will die.}%
\verse{And not with a great army and with a great crowd will Pharaoh work\lebnote{Or “help/aid”} with him in the war, at the pouring out of a siege ramp and the building of siege works to destroy many lives.}%
\verse{And he despised the oath to break covenant. And, look, he gave his hand in pledge, and yet he did all of these things. He will not escape.’}%
\verse{Therefore thus says the Lord Adonai: ‘As I live,\lnCJL{} surely\lnCJM{} my oath that he despised and my covenant that he broke I will return upon his head.}%
\verse{And I will spread my net over him, and he will be caught in my hunting net, and I will bring him to Babylon, and I will enter into judgment with him there about his infidelity that he displayed against me.}%
\verse{And all of his choice troops, among all of his troops, they will fall by the sword, and those who are remaining,\lebnote{Or “left over”} they will be scattered to all the directions of the wind, and you will know that I, Adonai, I have spoken.’}%
\verse{Thus says the Lord Adonai: ‘And I will take, even I, from the treetop of the high cedar, and I will plant it, from the head of its new plant shoot I will pluck a tender one, and I will plant it, even I, on a high and lofty mountain.}%
\verse{On the height of the mountain of Israel I will plant it, and it will carry\lebnote{Or “bear”} branches,\lnCJK{} and it will bear fruit, and it will become a noble cedar,\lebnote{“he will be to a cedar noble”} and all of the birds of all wings\lebnote{Or “kinds”} will dwell under it in the shade of its branches.}%
\verse{And all of the trees of the field will know that I, Adonai, I will bring low a high tree, and I will exalt a low, fresh tree, and I will make a dry tree flourish. I, Adonai, I have spoken, and I will do it.’”}%
\end{biblechapter}%
\begin{biblechapter}% Ezekiel 18
\verseWithHeading{God’s Just Treatment of Individuals, Both Righteous and Wicked}{And the word of Adonai came\lebnote{“was”} to me, saying,\lnCJN{}}%
\verse{“What do you mean by\lebnote{“What \textit{is} to you”} quoting this proverb about the land of Israel, saying,\lnCJN{} ‘The fathers, they ate unripe fruit, and the teeth of the child became blunt.’\lebnote{Or “dull,” set on edge”}}%
\verse{As I live,\lebnote{“life I”} declares\lnCJO{} the Lord Adonai, it will surely not any longer be appropriate for you\lnCJP{} to quote this proverb in Israel!}%
\verse{Look! All lives are mine.\lebnote{“all of the lives \textit{are} for me they”} The lives of father and son alike are mine.\lebnote{“Like the life of the father and like the life of the son \textit{is} to me look”} The person\lebnote{Or “life,” or “soul”} sinning will die.}%
\verse{And if a man is righteous and does justice and righteousness,}%
\verse{and on the mountains he does not eat and he does not lift up his eyes to the idols of the house of Israel, and the wife of his neighbor he does not defile and he does not approach a woman of\lnCJQ{} menstruation,}%
\verse{and he oppresses no one\lebnote{“a person not”} and he returns a pledge for his loan and he commits no robbery\lebnote{“robbed things not he seized”} and he gives\lebnote{Or “shares”} his bread to the hungry and he covers a naked person with a garment,}%
\verse{and he does not charge interest\lebnote{“with the interest not he gives”} and he takes no usury, and he holds back\lebnote{Or “brings back”} his hand from injustice and he executes a judgment of fairness between persons,\lebnote{“man and man”}}%
\verse{and in my statutes he goes about and my regulations he keeps,\lebnote{Or “watches”} performing faithfully\lebnote{“to do with \textit{with} faithfulness”} — then he is righteous, and certainly he will live,” declares\lnCJO{} the Lord Adonai.}%
\verse{“And now he has a son, a violent one, who sheds blood\lebnote{“shedding of blood”} and does any of these things\lebnote{“and he does like from one from these”}}%
\verse{(though\lebnote{Hebrew “And”} he\lebnote{That is, the father} did not do all of these things), for the son also\lebnote{Or “even”} eats on the mountains and he defiles the wife of his neighbor.}%
\verse{He oppresses the needy and the poor, and he commits robbery,\lebnote{“robbed things he seized”} and he does not return a pledge for a loan, and he lifts his eyes to the idols so he does a detestable thing.}%
\verse{He charges interest\lebnote{“with the interest he gives”} and takes usury. Then, shall he live? He shall not live, for he did all of these detestable things. Surely he will die! His blood will be on him.}%
\verse{“And look! He has a son, and he sees all of the sin of his father that he did, and he sees it, but he does not do it.}%
\verse{On the mountains he does not eat, and he does not lift up his eyes to the idols of the house of Israel, and the wife of his neighbor he does not defile.}%
\verse{And he oppresses no one; he requires no pledge for a loan, and he does not commit robbery;\lebnote{“robbed things not does he seize”} he gives his bread to the hungry, and he covers the naked person with a garment.}%
\verse{He brings back\lebnote{Or “holds back”} his hand from iniquity;\lebnote{Cf. verse 8} he does not take interest and usury; he does my regulations; he goes\lebnote{Or “walks”} in my statutes. He will not die because of the guilt of his father; he will surely live!}%
\verse{Because his father oppressed severely;\lebnote{“he oppressed oppression”} he stole from his brother;\lebnote{“he seized robbed things of a brother”} that which is not good he did in the midst of his people, and look! He will die through\lnCJQ{} his guilt.}%
\verse{“Yet\lnCJR{} you say, ‘Why does the son not bear the guilt of the father?’ And since the son does justice and righteousness and he keeps all of my statutes and does them, he shall surely live!}%
\verse{The person,\lebnote{Or “soul,” or “life”} the one sinning, will die. A son shall not bear the guilt of the father, and a father shall not bear the guilt of the son. The righteousness of the righteous shall be on him; the wickedness of the wicked\lnCJS{} shall be on him.}%
\verse{But if the wicked returns from all of his sins\lnCJS{} that he has done and he keeps all of my statutes and he does justice and righteousness, he shall surely live; he shall not die!}%
\verse{All of his transgressions that he committed will not be remembered against\lebnote{Or “for”} him. Through\lebnote{Or “By means of”} his righteousness that he has done he shall live.}%
\verse{Have I delight by any means\lebnote{Or “in any way”} in the death of the wicked, declares\lnCJO{} the Lord Adonai, and not at his turning from his way, so that\lnCJT{} he lives?}%
\verse{And when the righteous turns\lebnote{“at turning of \textit{the} righteous”} from his righteousness, so that\lnCJT{} he does injustice, and does all of the detestable things that the wicked do, then\lnCJU{} will he live because of all of his righteousness\lnCJS{} that he did? Those things will not be remembered because of his infidelity that he displayed, and because of his sin that he committed.\lebnote{“sin that he sinned”} Through\lebnote{Or “in/by”} them he shall die.}%
\verse{“Yet\lnCJR{} you say, ‘The way of the Lord is not fair!’ Listen, now, house of Israel, is my way not fair? Is it not your ways that are not fair?}%
\verse{When the righteous turns from his righteousness, and he does injustice, then\lnCJU{} he will die because of them; because of his injustice that he did he will die!}%
\verse{And when the wicked turns from his wickedness that he did and he does justice and righteousness, he will preserve his life.\lebnote{“his soul/life he will keep alive”}}%
\verse{And if he sees and he returns from all of his transgressions that he did, surely he will live; he will not die!}%
\verse{And yet they, the house of Israel, say, ‘The way of the Lord is not fair!’ Are not my ways fair, house of Israel? Are not your ways unfair?\lebnote{“not ways your not are right”}}%
\verse{“Therefore I will judge you, house of Israel, each one\lebnote{Hebrew “man”} according to his ways,” declares\lnCJO{} the Lord Adonai. “Repent\lebnote{Or “Turn/Return”} and turn around from all of your transgression, and it will not be as a stumbling block of iniquity to you.}%
\verse{Throw away from yourselves\lebnote{“from on you”} all of your transgressions that you committed,\lebnote{Hebrew “committed with them”} and make for yourselves\lnCJP{} a new heart and new spirit, and so why will you die, house of Israel?}%
\verse{For I have no pleasure in the death of the dying,” declares\lnCJO{} the Lord Adonai. “And so repent\lebnote{Or “turn/return”} and live!”}%
\end{biblechapter}%
\begin{biblechapter}% Ezekiel 19
\verseWithHeading{Lament for the Leaders of Israel}{“And you, raise a lament about\lebnote{Or “over”} the leaders of Israel,}%
\verse{and you must say, ‘What a lioness was your mother among the lions. She lay down in the midst of young lions, and she reared her lion cubs.}%
\verse{And she raised up one from her cubs; he became a fierce lion, and he learned to tear prey; he ate\lnCJV{} humans.\lnCJW{}}%
\verse{And nations heard about him; in their pit he was caught, and they brought him with hooks to the land of Egypt.}%
\verse{And she saw\lebnote{Or “realized”} that she was waiting in vain; her hope was destroyed, and she took one from her cubs, and she made him a fierce lion.}%
\verse{And he walked about in the midst of lions; he became a fierce lion, and he learned to tear prey; he ate\lnCJV{} humans.\lnCJW{}}%
\verse{And he knew their widows,\lebnote{Or “ravished/raped”; or “he ravaged their strongholds”} and he devastated their cities, and the land was appalled, and everyone in it\lebnote{“its fullness”} at the sound of his roar.}%
\verse{And nations set out against him from the surrounding provinces,\lebnote{“all around from \textit{the} provinces”} and they spread their net over him, and he was caught in their pit.}%
\verse{And they put him in a collar with hooks, and they brought him to the king of Babylon; they brought him into\lebnote{Hebrew “in”} a prison, so that his voice would not be heard any longer\lebnote{“longer”} on the mountains of Israel.}%
\verse{Your mother was like the vine in your vineyard;\lebnote{Or “in the vineyard”} planted fruitfully\lebnote{“fruitful”} beside water, and it was full of branches from many waters.\lebnote{Or “from waters abundant/many”}}%
\verse{And she produced branches of strength\lebnote{“they became to her branches/rods of strength”} to\lebnote{Or “for”} scepters of rulers; its height became tall between\lebnote{Or “among”} thick foliage, and it was seen\lebnote{Or “visible”} because of its tallness among the abundance of its branches.}%
\verse{But it was uprooted in rage; it was thrown to the earth, and the east wind dried up its fruit; they were stripped off, and its strong branch dried up; fire consumed it.}%
\verse{And now it is planted in the desert, in a dry and thirsty land.\lebnote{“a land of dryness and thirst”}}%
\verse{And so fire has gone out from the stem of its branches; its fruit it has consumed, and there was not in it a strong branch,\lebnote{“a branch of strength”} a scepter for ruling.’” This is a lament, and it will be used as a lament.\lebnote{“she \textit{is} for a lament”}}%
\end{biblechapter}%
\begin{biblechapter}% Ezekiel 20
\verseWithHeading{God’s Dealings with Israel for His Name’s Sake}{And then in the seventh year, in the fifth month, on the tenth day of the month, men from the elders of Israel came to consult Adonai, and they sat before me.\lebnote{“to the face of”}}%
\verse{And the word of Adonai came\lnCJX{} to me, saying,\lnCJY{}}%
\verse{“Son of man,\lnCJZ{} speak with the elders of Israel, and you must say to them, ‘Thus says the Lord Adonai: “Are you coming to consult me? As I live,\lnCKA{} I will surely not allow myself\lebnote{Or “me”} to be consulted by you!”’ declares\lnCKB{} the Lord Adonai.}%
\verse{Will you judge them? Will you judge them, son of man?\lnCJZ{} Make known to them the detestable things of their ancestors.\lnCKC{}}%
\verse{And you must say to them, ‘Thus says the Lord Adonai: “On the day of my choosing Israel I swore\lnCKD{} to the offspring of the house of Jacob, and I made myself known to them in the land of Egypt, and I swore\lnCKD{} to them, saying,\lnCJY{} ‘I am Adonai your God.’}%
\verse{On that day I swore\lnCKE{} to them to bring them out from the land of Egypt to the land that I had searched out for them, flowing with milk and honey — it is the most beautiful of all of the lands.}%
\verse{Then I said to them, ‘Let each one throw away the detestable things of his eyes, and you must not make yourselves unclean with the idols of Egypt! I am Adonai your God.’}%
\verse{But they rebelled against me, and they were not willing to listen to me; each one did not throw away\lebnote{“not they threw away”} the detestable things of their eyes; and they did not abandon\lebnote{“not they abandoned”} the idols of Egypt, and I decided to pour out my rage on them, to fully vent my anger against them in the midst of the land of Egypt.}%
\verse{“But\lnCKF{} I acted for the sake of my name to keep it from being profaned\lebnote{“not to be profaned before the eyes of”} before the eyes of the nations among whom they lived,\lebnote{“which they \textit{were} in their midst”} where I made known to them before their eyes, to bring them out from the land of Egypt.}%
\verse{And I brought them out from the land of Egypt, and I brought them to the desert,}%
\verse{and I gave my statutes to them, and my regulations I made known to them, which, if a person does them, then he will live by them.}%
\verse{And also my Sabbaths I gave to them to be a sign between me and between them so they would know\lnCKG{} that I, Adonai, am the one sanctifying them.}%
\verse{“But\lnCKF{} in the desert the house of Israel rebelled against me; they did not walk in my statutes,\lebnote{“in my statutes not they went”} and they rejected my regulations, which, if a person does them, he will live by them, and they greatly profaned my Sabbaths, and I decided to pour out my rage on them in the desert to destroy them,}%
\verse{and I acted for the sake of my name, that it not be profaned before the eyes of the nations before whom I brought them out.\lebnote{“which I brought out them before their eyes”}}%
\verse{And also I myself swore\lebnote{“I, I raised my hand”} to them in the desert not to bring them into the land that I had given to them, flowing with milk and honey — it is the most beautiful of all of the lands —}%
\verse{because they despised my judgments, and they did not walk in my statutes,\lebnote{“and my statutes not they went in them”} and my Sabbaths they profaned, for their heart was going after their idols.}%
\verse{But my eye took pity on them by not destroying them, and I did not completely destroy them\lebnote{“and not I make them complete destruction”} in the desert.}%
\verse{“And I said to their children in the desert, ‘You must not go in the statutes of your parents;\lnCKC{} you must not keep their regulations, and you must not make yourself unclean with their idols.}%
\verse{I, Adonai, am your God, so go in my statutes and keep my regulations and do them.}%
\verse{And treat my Sabbaths as holy, and they will be a sign between me and between you that you may know\lnCKG{} that I, Adonai, am your God.’}%
\verse{But the children rebelled against me; they did not walk in my statutes,\lebnote{“in statutes my not they went”} and they did not observe my regulations,\lebnote{“my regulations not they kept to do them”} which if a person does them, then he will live by them. My Sabbaths they desecrated, and I decided to pour out my rage on them, to finish my anger against them in the desert.}%
\verse{But I withheld my hand, and I acted for the sake of my name not to be profaned before the eyes of the nations before whom I had brought them out before their eyes.}%
\verse{What is more, I swore\lebnote{“I will raise my hand”} to them in the desert to scatter them among the nations and to disperse them in the lands,}%
\verse{because they did not do my regulations, and my statutes they despised, and my Sabbaths they profaned, and their eyes were after the idols of their ancestors.\lnCKC{}}%
\verse{And in turn I gave to them rules that were not good and regulations by which they will not live.\lebnote{“regulations not they will live by them”}}%
\verse{And I defiled them through their gifts in sacrificing all of the first offspring of the womb, in order that\lebnote{Or “so that”} I will cause them to be stunned, so that they will know that I am Adonai.}%
\verse{“Therefore speak to the house of Israel, son of man,\lnCJZ{} and you must say to them, ‘Thus says the Lord Adonai: “Again in this your ancestors blasphemed me at\lebnote{Or “by”} their display of infidelity toward me.”’\lebnote{“at/in acting unfaithfully they against me in/by unfaithfulness”}}%
\verse{And I brought them to the land that I swore\lnCKE{} to give\lebnote{Hebrew “give it”} to them, and they saw every high hill and every leafy tree, and they offered their sacrifices, and they presented there the provocation of their offering, and they gave there their fragrant incense offering, and they poured out their libations there.}%
\verse{And I said to them, ‘What is the high place\lebnote{Hebrew Bamah, which became its name} to which you are going?’ And it is called\lebnote{“it is called name its”} Bamah until this day.}%
\verse{Therefore thus say to the house of Israel, ‘Thus says the Lord Adonai: “In the way of your ancestors\lnCKC{} will you defile yourself, and after their vile idols will you prostitute yourselves?}%
\verse{And when you lift up your gifts, sacrificing your children\lebnote{“at to lift up your gifts at causing to pass over your children”} through the fire, you are defiling yourself through all of your idols until today,\lebnote{“the day”} and will I let myself\lnCKH{} be consulted by you, house of Israel?”’ As I live,”\lebnote{“Live I”} declares\lnCKB{} the Lord Adonai, “I will not let myself\lnCKH{} be consulted by you!}%
\verse{And what you are planning,\lebnote{“what is going up on your spirit”} surely it will not be — that you are saying, ‘Let us be like the nations, like the clans\lebnote{Or “tribes”} of the lands, serving wood and stone!’\lebnote{“to serve wood and stone”}}%
\verse{“As I live,”\lnCKA{} declares\lnCKB{} the Lord Adonai, “surely\lebnote{“if not”} with a strong hand and with an outstretched arm and with rage pouring forth\lebnote{“rage poured out”} I will reign as king over you!}%
\verse{And I will bring you from the peoples, and I will gather you from the countries to which you were scattered with a strong hand and with an outstretched arm and with rage poured out.}%
\verse{Then I will bring you to the desert of the peoples, and I will execute justice on you there face to face.}%
\verse{Just as\lebnote{“Like what”} I executed justice on your ancestors\lnCKC{} in the desert of the land of Egypt, likewise I will execute justice on you!” declares\lnCKB{} the Lord Adonai.}%
\verse{“And I will make you pass under the rod, and I will bring you into the bond of the covenant.}%
\verse{And I will purge the rebels from among you and the ones transgressing against me; I will bring them out from the land where they are living as aliens,\lebnote{“from the land of living as an alien there”} but into the land of Israel they will not come, and then you will know that I am Adonai.}%
\verse{And you, house of Israel, thus says the Lord Adonai: “Let each one go serve his idols now and after, if you are not listening to me, but my holy name\lebnote{“the name of my holiness”} you will not profane any longer with your gifts and with your idols.}%
\verse{“For on my holy mountain,\lebnote{“the mountain of my holiness”} on the mountain of the height of Israel,” declares\lnCKB{} the Lord Adonai, “there all of the house of Israel will serve me, all of them,\lebnote{Hebrew “it/him”} in the land. I will take pleasure in them, and there I will accept your contributions and the best of your portions with all of your holy objects.}%
\verse{I will accept you as a fragrant incense offering\lebnote{“a fragrance of an incense offering”} when I bring you out from the peoples and I gather you from the lands to which you were scattered, and I will show myself holy among you before the eyes of the nations.}%
\verse{And you will know that I am Adonai when I bring you to the land of Israel, to the land that I swore\lnCKE{} to give to your ancestors.}%
\verse{And you will remember there your ways, and all of your deeds by which you were made unclean,\lebnote{“which you were made unclean by them”} and you will feel a loathing for yourself\lebnote{“for your face”} for all of your evils that you have done.}%
\verse{And you will know that I am Adonai when I deal with you\lebnote{“at/in doing my with you”} for the sake of my name and not according to your evil ways\lebnote{“not like your ways the evil”} or according to\lebnote{“like”} your corrupted deeds, house of Israel,” declares\lnCKB{} the Lord Adonai.}%
\verse{\lebnote{Ezekiel 20:45–21:32 in the English Bible is 21:1–37 in the Hebrew Bible} And the word of Adonai came\lnCJX{} to me, saying,\lnCJY{}}%
\verse{“Son of man,\lnCJZ{} set your face toward the way of the south,\lebnote{Or “Teman”} and preach to the south, and prophesy against the forest of the territory of the Negev.}%
\verse{And you must say to the forest of the Negev, ‘Hear the word of Adonai, thus says the Lord Adonai: “Look! I am kindling against you a fire, and it will devour in you every fresh tree and every dry tree; the blaze of the flame will not be quenched, and all the surfaces from the south to the north will be scorched by it.}%
\verse{And all creatures will see that I, Adonai, I kindled it — it will not be quenched!”’”}%
\verse{Then I said, “Ah, Lord Adonai, they are saying about me, ‘Is he not posing a parable?’”}%
\end{biblechapter}%
\begin{biblechapter}% Ezekiel 21
\verseWithHeading{A Vision of the Avenging Sword of Adonai}{And the word of Adonai came\lnCKI{} to me, saying,\lnCKJ{}}%
\verse{“Son of man,\lnCKK{} set your face toward Jerusalem, and preach to the sanctuaries, and prophesy to the land of Israel.}%
\verse{And you must say to the land of Israel, ‘Thus says Adonai: “Look! I am against you, and I will draw out my sword from its sheath, and I will cut off from you the righteous and the wicked.}%
\verse{Because\lebnote{Hebrew “Because that”} I will cut off from you both righteous and wicked, therefore my sword will go out from its sheath to\lebnote{Or “against”} all creatures\lebnote{Hebrew “creature”} from south to north.}%
\verse{And they will know, all creatures, that I, Adonai, I will bring out my sword from its sheath; it will not return again!”’}%
\verse{And you, son of man,\lnCKK{} groan with shaking hips,\lebnote{“with destruction of loins”} and you must groan with bitterness before their eyes.}%
\verse{And then\lebnote{“and it will be”} when they say to you, ‘On account of what are you groaning?’ then you must say, ‘On account of the report, for it is coming, and every heart will be weak and all hands will hang limp and every spirit will be disheartened, and all knees will go like water.’ Look! It is coming, and it will happen!” declares\lnCKL{} the Lord Adonai.}%
\verse{And the word of Adonai came\lnCKI{} to me, saying,\lnCKJ{}}%
\verse{“Son of man,\lnCKK{} prophesy, and you must say, ‘Thus says the Lord,’ say, ‘A sword, a sword is sharpened and is also polished.}%
\verse{It is sharpened to slaughter a slaughter, polished to flash like lightning!\lebnote{“so that there be for her lightning”} Or will we rejoice?\lebnote{Or “be pleased”} A rod, my son, is despising every tree.\lebnote{Or “stick” of correction; Hebrew is uncertain in meaning here}}%
\verse{And he gives it to be polished, to be seized by the hand.\lebnote{“to take hold of in the palm of the hand”} It is sharpened — a sword — and it is polished to give it into the hand of the killer.}%
\verse{Cry and wail, son of man,\lnCKK{} for it is against my people; it is against all of the princes of Israel. They are thrown to the sword with my people; therefore strike your thigh.\lebnote{“strike on \textit{the} thigh”}}%
\verse{For examine!\lebnote{“it was tested”} And what if also the rod will not be despising?’ declares\lnCKL{} the Lord Adonai.}%
\verse{And you, son of man,\lnCKK{} prophesy and clap your hands.\lebnote{“strike hand to hand”} And the sword will strike twice;\lebnote{“be doubled”} let it happen a third time. It is a sword of the dead, the sword of the great dead that is surrounding them,}%
\verse{so that a heart melts, and the fallen multiply\lebnote{“be made many”} at all of their gates. I gave a sword for slaughter, and alas! It is made for flashing, it is grasped\lebnote{Or “polished”} for slaughtering.}%
\verse{Gather together, strike to the right; cause to go to the left, where your edge\lebnote{“face”} is directed.}%
\verse{And also I myself\lebnote{“I, I”} will clap my hands,\lebnote{“strike my hand to my hand”} and I will satisfy my rage! I, Adonai, I have spoken.”}%
\verse{And the word of Adonai came\lnCKI{} to me, saying,\lnCKJ{}}%
\verse{“And you, son of man,\lnCKK{} mark out for yourself\lebnote{Hebrew “you”} two roads for the coming of the sword of the king of Babylon; they must both\lebnote{“the two of them”} go out from the same land. And hew out a signpost;\lebnote{“hand”} hew it at the head of the road of the city.}%
\verse{You must mark a road for the coming of the sword to Rabbah of the Ammonites\lnCKM{} and to Judah, in Jerusalem the fortified.}%
\verse{For the king of Babylon stands at the fork of the road at the head of the two roads to practice divination. He shakes the arrows,\lebnote{“he casts the lot with the arrows” (cf. NASB, NRSV)} he inquires with the teraphim, he examines\lebnote{“sees with”} the liver.}%
\verse{In his right hand is the divination for Jerusalem, to put up battering rams, to open a mouth for slaughter,\lebnote{Or “open the mouth of Sheol”; or “give the command to slaughter” (cf. NRSV, NASB)} to raise the battle cry,\lebnote{“to raise a voice in shouting”} to put up battering rams against gates, to build a siege ramp, to build siege works.\lebnote{Hebrew “siege work”}}%
\verse{And it will be to them like practicing divination falsely\lebnote{Or “in vain”} in their eyes; they have sworn oaths for themselves.\lebnote{“swearers of oaths \textit{will be} for them”} But\lebnote{Or “And”} he will bring their guilt to remembrance so as to seize them.}%
\verse{“Therefore thus says the Lord Adonai: ‘Because you have brought to remembrance your guilt by the uncovering of your transgressions, so that your sins in all of your deeds appear — because of your being remembered, you will be captured in the hand.’}%
\verse{And as for you, profane one, wicked prince of Israel, whose day has come with the time of the punishment of the end,}%
\verse{thus says the Lord Adonai: ‘Remove the turban and lift off the crown; things are no longer the same. Exalt the low and bring low the high.}%
\verse{A ruin, a ruin, a ruin I will make it! Also\lebnote{Or “indeed”} this has not ever happened;\lebnote{Or “will not happen until” (cf. NASB and NRSV)} it will remain until the coming of the one to whom the judgment belongs and I have given it to him.’}%
\verse{And you, son of man,\lnCKK{} prophesy, and you must say, ‘Thus says the Lord Adonai to the Ammonites\lnCKM{} and concerning their disgrace,’ and you must say, ‘A sword, a sword is drawn for slaughtering; it is polished for holding, for flashing like lightning,\lebnote{“for the sake of lightening”}}%
\verse{when seeing a false vision for you in vain, when practicing divination for you falsely, to give\lebnote{Or “place”} you on the neck of the profane ones of the wicked, whose day has come,\lebnote{“who has come their day”} the time of final punishment.\lebnote{“in \textit{the} time of \textit{the} punishment of \textit{the} end”}}%
\verse{Return it to its sheath in the place where you were created. In the land of your origin I will judge you!}%
\verse{And I will pour out my anger on you; I will blow on you with the fire of my wrath, and I will give you into the hand of brutal men, skilled craftsmen of destruction.}%
\verse{You will be as fuel for the fire; your blood will be in the midst of the earth. You will not be remembered, for I, Adonai, I have spoken.’”}%
\end{biblechapter}%
\begin{biblechapter}% Ezekiel 22
\verseWithHeading{Highlighting the Sins and the Judgments of Israel}{And the word of Adonai came\lnCKN{} to me, saying,\lnCKO{}}%
\verse{“And you, son of man,\lnCKP{} will you judge? Will you judge the bloody city?\lebnote{“city of blood”} Then you must make known to her all of her detestable things!}%
\verse{And you must say, ‘Thus says the Lord Adonai: “A city pouring out blood in the midst of her; its time has come,\lebnote{“to come time her its”} and it made idols for itself, becoming unclean.}%
\verse{By your blood that you poured out you have become guilty, and by your idols that you made you have become unclean, and you have brought your days near, and your appointed years have come.\lebnote{“and it came to years your”} Therefore I have made you a disgrace for the nations and a laughingstock to all of the countries.}%
\verse{The people near and the people far from you will make fun of you, the unclean and the terrified.\lebnote{“the unclean of the name the great of the panic”}}%
\verse{Look! The princes of Israel, each one according to his strength,\lebnote{Or “influence”} they are in you for the shedding of blood.\lebnote{“to shedding of blood”}}%
\verse{They have treated father and mother with contempt in you; they violated\lebnote{“they did with extortion”} the alien with\lebnote{Or “by”} extortion; in the midst of you they mistreated the orphan and widow.}%
\verse{You despised my holy objects; my Sabbaths you profaned.}%
\verse{Slanderous men are among you\lnCKQ{} to shed blood, and they, among you,\lnCKQ{} eat upon the mountains; they do wickedness in the midst of you.}%
\verse{They uncover the nakedness of a father among\lnCKR{} you; they violate a woman unclean of menstruation among\lnCKR{} you.}%
\verse{And a man does a detestable thing with the wife of his neighbor, and a man defiles his daughter-in-law in wickedness, and a man sexually violates among\lnCKR{} you his sister, the daughter of his father.}%
\verse{They take a bribe among\lnCKR{} you in order to shed blood; and you take usury, and you make gain from your neighbors by extortion, and so you have forgotten me, declares\lnCKS{} the Lord Adonai.}%
\verse{And look! I strike my hand for your ill-gotten gain that you have made and at your blood that was in the midst of you.}%
\verse{Can your heart endure,\lebnote{Or “hold up”} or can your hands be strong at the days in which I am dealing with you? I, Adonai, I have spoken, and I will act!}%
\verse{I will scatter you among the nations, and I will disperse you through the countries, and I will purge your uncleanness from you.}%
\verse{And I will be profaned by you before the eyes of the nations, and you will know that I am Adonai.”’”}%
\verse{And the word of Adonai came\lnCKN{} to me, saying,\lnCKO{}}%
\verse{“Son of man,\lnCKP{} the house of Israel has become as silver dross to me; all of them are as bronze and tin and iron and lead in the midst of a furnace, even as silver dross, silver dross they became!}%
\verse{Therefore thus says the Lord Adonai: ‘Because all of you have become as silver dross, therefore look! I am gathering you to the midst of Jerusalem,}%
\verse{like the gathering of silver and bronze and iron and lead and tin to the middle of a furnace to blow fire on it for melting. Thus I will gather in my anger and in my rage, and I will deposit\lebnote{“I will put/place”} you, and I will melt you.}%
\verse{And I will gather you, and I will blow on you with the fire of my wrath, and so I will melt you in the midst of her.}%
\verse{Thus you will be melted like the melting of silver in the midst of a furnace, and you will know that I, Adonai, I have poured out my rage on you.’”}%
\verse{And the word of Adonai came\lnCKN{} to me, saying,\lnCKO{}}%
\verse{“Son of man,\lnCKP{} say to her, ‘You are a land not cleansed; it is not rained upon in the day of indignation.’}%
\verse{The conspiracy of its prophets in the midst of her is like a roaring lion that is tearing prey. They devour people, and they take wealth and treasure; they make its widows numerous in the midst of her.}%
\verse{Its priests treat my law violently, and they profane my holy objects; they do not distinguish between a holy object and what is unholy, or between the clean and the unclean. They do not teach the difference, and they hide their eyes from my Sabbaths, and so I am profaned in the midst of them.}%
\verse{Its officials are like wolves tearing prey in its midst, to pour out blood, to destroy people, to make dishonest gain.}%
\verse{And for them its prophets plaster whitewash; they are seeing falseness and are practicing divination for them by lying, saying, ‘Thus says the Lord Adonai,’ and Adonai has not spoken.}%
\verse{They severely oppress the people of the land, and they committed robbery,\lebnote{“and they seized \textit{a} robbed thing\textit{s}”} and they mistreated the needy and the poor, and they oppressed the alien without\lebnote{“with not”} justice.}%
\verse{And so I sought for them somebody, one repairing the wall and standing in the breach before me\lebnote{“to the face of me”} on behalf of the land not to destroy it, but I did not find anyone,}%
\verse{and so I poured out my indignation on them. With the fire of my wrath I destroyed them; I returned their way upon their head,” declares\lnCKS{} the Lord Adonai.}%
\end{biblechapter}%
\begin{biblechapter}% Ezekiel 23
\verseWithHeading{Oholah and Oholibah as Symbols of God’s Corrupt People}{And the word of Adonai came\lebnote{“was”} to me, saying,\lebnote{“to say”}}%
\verse{“Son of man,\lnCKT{} there were two women, the daughters of one mother,}%
\verse{and they prostituted themselves in Egypt in their childhood; they were prostituting themselves there, and their breasts were fondled, and there they caressed the bosoms of their virginity.}%
\verse{Now as for their names, the older was Oholah, and Oholibah was her sister. And they became mine,\lebnote{“they were for me”} and they bore sons and daughters, and their names\lebnote{Hebrew “name”} are Samaria for Oholah, and Jerusalem for Oholibah.}%
\verse{And Oholah prostituted herself while she was still mine,\lebnote{“under me”} and she lusted for her lovers, for Assyria who was nearby,}%
\verse{clothed in blue cloth, governors and prefects, handsome young men\lebnote{“young men of handsomeness/beauty”} and all of them horsemen, experts on horseback.\lebnote{“riders \textit{of} horsemen”}}%
\verse{And she bestowed her fornication on them, on the choice ones of the Assyrians,\lnCKU{} all of them, and with every one after which she lusted; with all of their idols she defiled herself.}%
\verse{And her whorings from the time of Egypt she did not abandon, for they slept with her in her childhood, and they caressed the bosoms of her virginity, and they poured out their fornication on her.}%
\verse{Therefore I gave her into the hand of her lovers, into the hand of the Assyrians\lnCKU{} after whom she lusted.\lebnote{“which she lusted for them”}}%
\verse{They uncovered her nakedness; they took her sons and her daughters, and they killed her with the sword. And she became a name for the women, and they executed judgments against her.}%
\verse{“And though Oholibah her sister saw, yet in her lust she behaved more corruptly than her,\lebnote{“she behaved corruptly \textit{in} lust from her”} and her whoring was more than the prostitution of her sister.}%
\verse{She lusted after the Assyrians,\lnCKU{} governors and prefects, warriors\lebnote{Hebrew uncertain} clothed in\lebnote{Or “to”} perfection,\lebnote{Or “in full armor” or “wonderfully dressed”} expert horsemen,\lebnote{“horsemen, riders of horses”} all of them handsome young men.\lebnote{“young men of handsomeness”}}%
\verse{And I saw that she was defiled; they had both taken the same path!\lebnote{“way one \textit{was} for the two of them”}}%
\verse{And she increased her whorings, and she saw men carved\lebnote{Or “portrayed”} on the wall, images of Chaldeans\lebnote{According to the reading tradition (\textit{Qere})} carved in red,}%
\verse{belted with a belt at\lebnote{Or “around”} their waist with turbans on their heads, all of them giving the appearance of adjutants,\lnCKV{} the image\lebnote{Or “likeness”} of the Babylonians;\lnCKW{} Chaldea was the land of their birth.}%
\verse{And she lusted for them when her eyes saw them,\lebnote{“at the sight of her eyes”} and she sent messengers to them at Chaldea.}%
\verse{And so the Babylonians\lnCKW{} came to her for the bed of lovemaking; and they defiled her with their fornication, and she was defiled by them, and she turned from them.}%
\verse{And she revealed her whorings, and she revealed her nakedness, and so I turned from her just as\lebnote{“as that”} I turned from her sister.}%
\verse{Yet\lebnote{Or “And”} she increased her whorings, recalling\lebnote{Or “remembering”} the days of her childhood when she was prostituted\lebnote{Or “prostituted \textit{herself}”} in the land of Egypt.}%
\verse{And she lusted after her male lovers whose genitalia were the genitalia of male donkeys and their seminal emission was the seminal emission of horses.}%
\verse{And you longed after the obscene conduct of your youth when your bosom was caressed by Egypt, fondling your young breasts.}%
\verse{“Therefore, Oholibah, thus says the Lord Adonai: ‘Look! I am stirring up your lovers against you concerning whom you turned away, and I will bring them against you from all around:}%
\verse{the Babylonians\lnCKW{} and all of the Chaldeans, Pekod and Shoa and Koa, all of the Assyrians\lnCKU{} along with them, handsome young men,\lebnote{“young men of beauty/handsomeness”} governors and prefects, all of them adjutants\lnCKV{} and excellent horsemen.\lebnote{“ones called riders of horses”}}%
\verse{And they will come against you with an army chariot and wagon and with a crowd\lebnote{Or “horde” or “host of”} of peoples; they will set themselves against you from all around with large shield and small shield and helmet. And I will give before them\lebnote{“to the face of them”} judgment, and they will judge you with their judgments.}%
\verse{And I will direct my zeal against you, and they will deal with you in anger; your nose and your ears they will remove, and those who are left,\lnCKX{} they will fall by the sword, and they will take your sons and your daughters, and your remnant\lnCKX{} will be consumed by fire.}%
\verse{And they will strip you of your clothes, and they will take your splendid jewelry.\lebnote{“the jewelry of splendor your”; or “your beautiful jewelry”}}%
\verse{And I will put an end to your obscene conduct coming from you and your fornication from the land of Egypt, and you will not lift your eyes to them; and you will not remember Egypt again.’}%
\verse{For thus says the Lord Adonai: ‘Look! I am giving you into the hand of those you hated, into the hand of those from whom\lebnote{“them”} you turned away.}%
\verse{And they will deal with you in hatred, and they will take all of your acquisitions, and they will leave you naked and in bareness; and the nakedness of your fornication and your obscene conduct and your whorings will be exposed.}%
\verse{These things are accomplished\lebnote{Or “done”} against\lebnote{Hebrew “in”} you since you prostituted yourself after the nations, and on account of that, you defiled yourself by their idols.}%
\verse{You went in the same way of your sister, and I will give her cup into your hand.”’}%
\verse{Thus says the Lord Adonai: “You will drink the deep and wide\lebnote{Or place “drink deep and wide the cup”} cup of your sister; you will be as laughter and as scorn; the cup holds so much!\lebnote{“a large among to hold”}}%
\verse{You will be filled with drunkenness and sorrow, for a cup of horror and desolation is the cup of your sister, Samaria.}%
\verse{And you will drink it, and you will drain it, and its potsherds you will gnaw, and you will tear out your breasts, for I myself\lebnote{“I, I”} spoke,” declares\lebnote{“declaration of”} the Lord Adonai.}%
\verse{Therefore thus says the Lord Adonai: “Because you have forgotten me, and you threw me behind your back, now in turn\lebnote{Or “also”} you bear your obscene conduct and your whorings.”}%
\verse{And Adonai said to me, “Son of man,\lnCKT{} will you judge Oholah and Oholibah and declare their abominable deeds to\lebnote{Or “for”} them?}%
\verse{For they committed adultery, and blood is on their hands, and they committed adultery with their idols, and even their children that they had borne for me — they sacrificed them as food!}%
\verse{Also they did this to me: they defiled my sanctuary on that day, and they profaned my Sabbaths,}%
\verse{and when they slaughtered their children for their idols, they came to my sanctuary on that day to profane it. And look, this is what they did in my house!\lebnote{“thus they did in the midst of house my”}}%
\verse{What is worse,\lebnote{“and also for/indeed”} they sent for men who come\lebnote{“coming”} from a distant place, to whom a messenger was sent to them, and look! They came! Men for whom you bathed and painted your eyes, and you adorned yourself with an ornament.}%
\verse{And you sat on a magnificent couch and a table prepared before her,\lebnote{“to the face of her”} and my incense and my olive oil you put on her.\lebnote{Or “it”}}%
\verse{And a sound of a carefree crowd was with it,\lebnote{Or “her”} and in addition to these men, a crowd of drunken men\lebnote{A problematic verse; reading according to the reading tradition (\textit{Qere})} was brought in\lebnote{“from an abundance of men being brought in drunken”} from the desert,\lebnote{Or “wilderness”} and they put bracelets on their arms and a crown of splendor on their heads.}%
\verse{And I said to\lebnote{Or “about”} the one worn out with adulteries, ‘Now they will prostitute her concerning her fornication, even her.’\lebnote{Or “that \textit{she} did”}}%
\verse{And so they went to her like going to a female\lebnote{Or “a woman”} prostitute; and thus they went to Oholah and to Oholibah, the women of obscene conduct.}%
\verse{But righteous men, they will judge them with the judgment of committing adultery, and with the judgment of shedding blood; for they were committing adultery, and blood was on their hands.”}%
\verse{For thus says the Lord Adonai: “Bring up against them an assembly, and make them as a thing of horror and as plunder.}%
\verse{And an assembly must stone them with stones,\lebnote{Hebrew “stone”} and they must cut them down. With their swords they shall kill their sons and their daughters, and with fire they shall burn their houses.}%
\verse{And I will cause obscene conduct to cease from the land,\lebnote{Or “I will put an end to obscene conduct from the land”} and all of the women will be warned, and they will not do according to your wickedness.}%
\verse{And they will repay your obscene conduct upon you, and the guilt of your idols you will bear, and you will know that I am the Lord Adonai.}%
\end{biblechapter}%
\begin{biblechapter}% Ezekiel 24
\verseWithHeading{The Boiling Pot and the Death of Ezekiel’s Wife}{And the word of Adonai came\lnCKY{} to me in the ninth year, in the tenth month, on the tenth day of the month, saying,\lnCKZ{}}%
\verse{“Son of man,\lnCLA{} write for yourself the name of the day, this very day.\lebnote{“exactly the day the this”} The king of Babylon laid siege to Jerusalem on exactly this day!}%
\verse{And deliver a proverb\lebnote{“quote a proverb”; “proverb a proverb”} to the rebellious house,\lebnote{“the house of rebellion”} and you must say to them, ‘Thus says the Lord Adonai: “Place the pot! Place it and also pour water into it.}%
\verse{Gather its pieces to it, every good piece, thigh and shoulder, fill it with choice bones;}%
\verse{take the choicest of the flock, and also pile the bones\lebnote{Or possibly “logs”; the difference is one letter in Hebrew} under it; boil it vigorously;\lebnote{“boil its boiling”} indeed,\lebnote{Or “also”} its bones boiled in the midst of it.”’”}%
\verse{Therefore thus says the Lord Adonai: “Woe to the city of bloodguilt! A pot that has its rust in it; its rust did not go out from it.\lebnote{“rust her not went forth from her”} Bring it out piece by piece;\lebnote{“according to its pieces”} one is as good as another.\lebnote{“no lot has fallen on it”}}%
\verse{For her blood was in the midst of her; she put it on a bare rock;\lebnote{“on barrenness of \textit{a} rock she put it”} she did not pour it on the ground to cover it with dust.\lebnote{Or “earth” or “dirt”}}%
\verse{To stir up rage, to avenge myself through vengeance, I placed its blood on the barrenness of a rock, so that it may not be covered.”\lebnote{“so \textit{as} not to be covered”}}%
\verse{Therefore thus says the Lord Adonai: “Woe to the city of bloodguilt! I, even I, will make the pile of wood great!}%
\verse{Pile up the logs; kindle the fire; finish cooking the meat, and mix in the spices, and let the bones be burned.\lebnote{Or “charred”}}%
\verse{And make it stand\lebnote{“cause it to stand”} empty upon its burning coals so that it may become hot, and its copper may become molten and be melted in the midst of it,\lebnote{“her”} so that its uncleanness and its rust be consumed.}%
\verse{Through effort she has caused weariness; its thick rust\lebnote{“the abundance of its rust”} went not out of it. Its rust! Into the fire with its rust!}%
\verse{In your uncleanness is obscene conduct, because I cleansed you, but you were not clean from your uncleanness; you will not be clean again until I fully vent my rage\lebnote{“until causing rest I to my rage”} on you.}%
\verse{I, Adonai, I have spoken. It is coming, and I did it; I will not disregard, and I will not take pity, and I will not reconsider. According to your ways and according to your deeds they will judge you!”\lebnote{Or “I will judge”} declares\lebnote{“declaration of”} the Lord Adonai.}%
\verse{And the word of Adonai came\lnCKY{} to me, saying,\lnCKZ{}}%
\verse{“Son of man,\lnCKZ{} look! I am taking from you what is pleasing to your eyes with a plague,\lebnote{Or “blow”} but you shall not mourn, and you shall not weep, and your tears shall not run down.}%
\verse{Sigh in silence for the dead; you shall not make a mourning ceremony. Bind your turban on you, and you must put your sandals on your feet. You must not cover your upper lip,\lebnote{Or “your mustache” (NASB)} and the bread of mourners\lnCLB{} you shall not eat.”}%
\verse{And I spoke to the people in the morning, and my wife died in the evening, and I did in the morning just as\lnCLC{} I was commanded.}%
\verse{And the people said to me, “Will you not make known to us what these things that you are doing mean for us?”}%
\verse{And I said to them that the word of Adonai came\lnCKY{} to me, saying,\lnCKZ{}}%
\verse{“Say to the house of Israel, ‘Thus says the Lord Adonai: “Look! I will be profaning my sanctuary, the pride of your strength and the delight of your eyes and the object of your own\lebnote{Or “souls’,” or “selves’”} affection; and your sons and your daughters whom you left behind will fall by the sword,}%
\verse{and you shall do just as\lnCLC{} I did: You shall not cover your upper lip,\lebnote{Or “mustache”} and you shall not eat the bread of mourners.\lnCLB{}}%
\verse{And your turban must be on your heads, and your sandals must be on your feet. You shall not mourn, and you shall not weep, but you shall waste away because of your iniquities, and you shall groan to one another.\lebnote{“a man to his brother”}}%
\verse{And Ezekiel will be to you as a sign; everything that he did\lebnote{“like all that he did”} you shall do; and when it comes you will know that I am the Lord Adonai.}%
\verse{“And you, son of man,\lnCLA{} will it not be that on the day when I take from them\lebnote{“on the day of to take me from them”} their stronghold, the joy of their glory, the delight of their eyes, and the longing of their inner self,\lebnote{Or “soul”} their sons and daughters —}%
\verse{on that day a survivor will come to you with the news.\lebnote{“with a message for ears”}}%
\verse{On that day your mouth shall be opened at the arrival of the fugitive,\lebnote{“with the fugitive”; or “to the one who has escaped” (NRSV; cf. NASB)} and you shall speak, and you shall not be silent any longer, and you shall be to them as a sign, and they will know that I am Adonai.”}%
\end{biblechapter}%
\begin{biblechapter}% Ezekiel 25
\verseWithHeading{Judgment upon Various Gentile Nations}{And the word of Adonai came\lebnote{“was”} to me, saying,\lebnote{“to say”}}%
\verse{“Son of man,\lebnote{Or “mortal,” or “son of humankind”} set your face toward the Ammonites\lnCLD{} and prophesy against them,}%
\verse{and you must say to the Ammonites,\lnCLD{} ‘Hear the word of the Lord Adonai! Thus says the Lord Adonai: “Because of your saying, ‘Ah!’ to my sanctuary when it was profaned, and to the land of Israel when it was\lebnote{Or “became”} desolate, and to the house of Judah when they went into the exile,}%
\verse{therefore look! I am giving you to the people\lnCLE{} of the East as a possession, and they will set up their encampments in you, and they will make in you their dwelling places; they themselves\lnCLF{} will eat your fruit and they themselves\lnCLF{} will drink your milk.}%
\verse{And I will make Rabbah as a pasture of camels and the Ammonites\lnCLD{} as a haunt of flocks, and they will know that I am Adonai.”’”}%
\verse{For thus says the Lord Adonai: “Because of clapping your hand and stamping with your foot, and because you rejoiced in\lebnote{Or “to”} yourself with all of your malice over the land of Israel,}%
\verse{therefore look! I stretched out my hand against you, and I will give you as plunder\lebnote{According to the reading tradition (\textit{Qere})} to the nations, and I will cut you off from the peoples, and I will destroy you from the countries, and I will wipe you out, and you will know that I am Adonai.”}%
\verse{Thus says the Lord Adonai: “Because of Moab and Seir saying, ‘The house of Judah is like all of the nations,’}%
\verse{therefore, look! I am opening the side of\lebnote{Or “the flank”} Moab from the cities on its frontier, the glory of the land: Beth Jeshimoth, Baal Meon, and Kiriathaim.}%
\verse{I will give it to the people\lnCLE{} of the East in addition to the Ammonites\lnCLD{} as a possession, so that the Ammonites\lnCLD{} will not be remembered among the nations.}%
\verse{And on Moab I will execute punishments, and they will know that I am Adonai.”}%
\verse{Thus says the Lord Adonai: “Because of the doings of Edom in avenging himself with vengeance for\lebnote{Or “toward”} the house of Judah, and so they became very guilty and\lebnote{Or “since”} they avenged themselves on them,”}%
\verse{therefore thus says the Lord Adonai: “so\lebnote{“and”} I will stretch out my hand against Edom and I will cut off from it both human and animal, and I will make it a ruin from Teman and Dedan; they will fall by the sword.}%
\verse{And I will exact my vengeance on Edom by the hand of my people Israel, and they will do in Edom according to my anger and according to my rage, and they will know my vengeance,” declares\lebnote{“declaration of”} the Lord Adonai.}%
\verse{Thus says the Lord Adonai: “Because of the acting\lebnote{“doing”} of the Philistines in vengeance, so that they avenged themselves relentlessly with malice in themselves for destruction with everlasting hostility,\lebnote{“\textit{with} hostility of ongoing ages/eternity”}}%
\verse{therefore thus says the Lord Adonai: ‘Look! I am stretching out my hand against the Philistines, and I will cut off the Kerethites, and I will destroy the remainder of the seacoast.\lebnote{“the coast of the sea”}}%
\verse{And I will execute on them great vengeance with punishments of rage, and they will know that I am Adonai when I exact my vengeance on them!”}%
\end{biblechapter}%
\begin{biblechapter}% Ezekiel 26
\verseWithHeading{Ezekiel Prophesies Against Tyre}{And it was in the eleventh\lebnote{“one ten”} year, on the first day of the month, the word of Adonai came\lebnote{“was”} to me, saying,\lebnote{“to say”}}%
\verse{“Son of man,\lebnote{Or “mortal,” or “son of humankind”} because\lebnote{Hebrew “because that”} Tyre said concerning Jerusalem, ‘Ah! The gates of the peoples are broken; it has swung open to me; I shall be filled, for it lies in ruins!’}%
\verse{Therefore thus says the Lord Adonai: ‘Look! I am against you, Tyre, and I will bring up against you many nations like the stirring up of the sea stirring up its waves.\lebnote{“\textit{with respect to} its waves”}}%
\verse{And I will destroy the walls of Tyre, and they will demolish its towers, and I will scrape away its earthen dirt from it, and I will make it into a bare rock.\lnCLG{}}%
\verse{It will become a place for spreading out dragnets in the midst of the sea; for I have spoken,’ declares\lnCLH{} the Lord Adonai. ‘And it will become as plunder for the nations,}%
\verse{and its daughters who are in the field with the sword, they will be killed; and they will know that I am Adonai.’”}%
\verse{For thus says the Lord Adonai: “Look! I am bringing to Tyre Nebuchadnezzar, the king of Babylon, from the north, the king of kings, with horse and with chariot and with horsemen and his assembly and many people.}%
\verse{Your daughters he will kill in the field with the sword, and he will place\lebnote{Hebrew “give”} against you siege works, and he will build against you a siege ramp, and he will raise against you a shield,}%
\verse{and the thrust of his battering ram he will direct against your walls, and your towers he will break down with his weapons.}%
\verse{From the abundance of his horses he will cover you with their fine dust; at the sound of horseman and wheel and chariot your walls will shake, at his coming into your gates like the entrance of a city that is being broken through.}%
\verse{With the hooves of his horses he will trample all of your streets. He will kill your people with the sword, and your strong stone pillars\lebnote{“the stone pillars of strength your”} will tumble down to the earth.}%
\verse{And they will plunder your wealth, and they will loot your merchandise, and they will break down your walls, and the houses of your delight they will break down, and your stones and your timbers and your earthen dirt they will cast into the midst of the water.}%
\verse{And, I will put an end to the noise of your songs, and the sound of your lyres will not be heard any longer.}%
\verse{And I will make you into a bare rock,\lnCLG{} a place for the spreading out of dragnets. You will not be built again, for I, Adonai, I have spoken,” declares\lnCLH{} the Lord Adonai.}%
\verse{Thus says the Lord Adonai to Tyre, “Will not the coastlands shake from the sound of your downfall, at the groaning of the wounded, at people being killed\lebnote{“at being killed a killing”} in the midst of you?}%
\verse{And all the princes of the sea will go down from their thrones, and they will remove their robes, and their beautiful garments\lebnote{“the garments of beautiful finished cloth”} of finished cloth they will take off. With terror they will be clothed, and on the ground they will sit, and they will tremble continually,\lebnote{“for moments”} and they will be appalled over you.}%
\verse{And they will raise a lament over you, and they will say to you, ‘How you have been lost who was inhabited from\lebnote{Or “on”} the seas; the city that was praised, that was strong. It is located on the sea, and its inhabitants imposed their terror on all of its inhabitants.\lebnote{“to all of its inhabitants”}}%
\verse{Now the coastlands will tremble at the day of your downfall, and the islands that are in the sea will be horrified because of your departure.’”}%
\verse{For thus says the Lord Adonai: “When I make you a desolate city, like the cities that are not inhabited, when bringing up over you the deep, the great surging waters\lebnote{“the waters the many”} will cover you.}%
\verse{And I will bring you down with those who are going down to the grave,\lnCLI{} an ancient people,\lebnote{“a people of eternity/endless ages”} and I will cause you to dwell in the world\lebnote{Or “land”} of the depths, in the ruins from of old with those who are going down to the grave,\lnCLI{} so that you will not be inhabited and have a place\lebnote{Hebrew “and I will give beauty in \textit{the} land of \textit{the} living”} in the land of the living.}%
\verse{Sudden terrors I will bring on you, and you shall no longer exist; and you will be sought, and you will not be found again forever,”\lebnote{“to eternity/endless ages”} declares\lnCLH{} the Lord Adonai.}%
\end{biblechapter}%
\begin{biblechapter}% Ezekiel 27
\verseWithHeading{A Lament for Doomed Tyre}{And the word of Adonai came\lebnote{“was”} to me, saying,\lebnote{“to say”}}%
\verse{“And you, son of man,\lebnote{Or “mortal,” or “son of humankind”} raise a lament against Tyre.}%
\verse{And you must say to Tyre, the one who sits at the entrance of the sea as the merchant of the peoples to the many coastlands, ‘Thus says the Lord Adonai: Tyre, you yourself said\lebnote{“you you said”} “I am perfect in beauty!”\lebnote{“of beauty”}}%
\verse{In the heart of the seas are your boundaries; your builders perfected your beauty.}%
\verse{They built with pine trees from Senir all of your boards for you; they took cedars from Lebanon to make a sailing mast for you.\lebnote{“on you”}}%
\verse{They made your oars with oaks from Bashan; your deck they made with inlaid ivory, with cypress trees from the coastlands of Cyprus.\lebnote{According to the reading tradition (\textit{Qere})}}%
\verse{Your sail was fine linen with colorful weaving from Egypt to serve as a banner for you;\lebnote{“to be for you as a banner”} blue and purple cloth from the coastlands of Cyprus\lebnote{Or “Elishah”} was your awning.}%
\verse{The inhabitants of Sidon and Arvan were your rowers; your skilled men,\lnCLJ{} O Tyre, were from your own people,\lnCLK{} and they were your seamen.}%
\verse{The elders of Gebal and its skilled men\lnCLJ{} were among you\lnCLK{} as the repairers of the seam of your boat; all of the ships of the sea and their mariners were among you\lnCLK{} to barter your wares.}%
\verse{Persia and Lud and Put were among your soldiers;\lebnote{“in your army \textit{as} the men of war your”} small shield and helmet hung among you\lnCLK{} and gave to you your adornment.}%
\verse{The people\lebnote{Or “children,” or “sons”} of Arvan and Helech were on your walls all around, and Gammadites were in your towers. They hung their quivers\lebnote{Or “shields” (cf. NJPS; NIV)} on your walls all around; they perfected your beauty.}%
\verse{“‘Tarshish was your trader; from the abundance of all of their wealth, with silver, iron, tin, and lead they exchanged for your merchandise.}%
\verse{Javan, Tubal, and Meshech were your traders; in exchange for people\lebnote{Or “for a human person/soul”} and an object of bronze they gave you your wares.}%
\verse{From Beth Togarmah they exchanged horses and war horses and mules for your wares.}%
\verse{The people\lebnote{Or “children”} of Dedan were trading with you, many coastlands composed the region of your influence;\lebnote{“hand”} they brought back horns of ivory and ebony as your payment.}%
\verse{Edom\lebnote{Reading Edom for Aram} was trading with you because of the abundance of your products, trading with malachite, purple wool yarn and colorful weaving, and fine white fabric and black corals and rubies; all these they exchange for your merchandise.}%
\verse{Judah and the land of Israel were trading with you with wheat from Minnith and millet and honey and olive oil and balm; all these they gave for your wares.}%
\verse{Damascus was trading with you because of\lebnote{“for”} the abundance of your products, because of the abundance of all of your wealth, trading with the wine of Helbon and white wool.}%
\verse{Vedan and Javan from Uzal, they exchanged wrought iron, cinnamon,\lebnote{Or “cassia”} and reed spice for your merchandise; all this was for your wares.}%
\verse{Dedan was trading with you, with garments of woven material for riding.}%
\verse{Arabia and all of the leaders of Kedar were your customers;\lebnote{“traders of your hand”} with young rams and adult rams and goats they were trading with them with you.}%
\verse{The traders of Sheba and Raamah were trading with you, with the finest of every spice and with every precious stone and gold; they exchanged all these for your merchandise.}%
\verse{Haran and Canneh and Eden, the traders of Sheba, Assyria, and Kilmad were trading with you.}%
\verse{They were trading with you in finery, in mantles of blue cloth and colorful embroidered work and with rugs of variegated cloth in twisted cords and knotted tightly; these were among your merchandise.\lebnote{“in your trading/marketplace” (cf. NET)}}%
\verse{The ships of Tarshish were carrying for you your wares, and you were filled, and you became very heavy in the heart of the seas.}%
\verse{“‘Into many waters the rowers brought you; but the east wind\lebnote{“the wind of the east”} wrecked you in the heart of the seas.}%
\verse{Your wealth and your merchandise, your wares, your mariners, and your seamen, your shipwrights\lebnote{“the repairers/strengtheners of your seams”} and the barterers of your wares and all of your soldiers\lebnote{“the men of your war”} who are in you, along\lebnote{Hebrew “and”} with all of your crew who are in the midst of you, they will fall in the heart of the seas on the day of your downfall.}%
\verse{At the sound of the shout\lebnote{Or “cry”} of your seamen, the pasturelands will shake.}%
\verse{And they will go down from their ships, all of those holding an oar, mariners, all of the seamen of the sea will stand on the land,}%
\verse{and they will lament\lebnote{Hebrew “lamented”} over you with their voice, and they will cry out\lebnote{Hebrew “cried out”} bitterly, and they will throw\lebnote{Hebrew “threw”} dust on their heads, and they will roll\lebnote{Hebrew “rolled”} in the dust.}%
\verse{They will shave themselves bald for your sake,\lebnote{“And they made bald for you \textit{with} baldness”} and they will dress\lebnote{Hebrew “dressed”} themselves in sackcloth, and they will weep\lebnote{Hebrew “wept”} over you with a bitterness of soul\lebnote{Or “\textit{of their} inner self”} and with bitter wailing.}%
\verse{And they will raise\lebnote{Hebrew “raised”} over you with their wailing a lament, and they will chant\lebnote{Hebrew “chanted”} a lament over you: ‘Who is like Tyre, like this destruction in the midst of the sea?’}%
\verse{When your merchandise went out from the seas, you satisfied many peoples with the abundance of your wealth, and by your wares you made rich the kings of the world.}%
\verse{Now you are broken by seas in the depths of waters; your wares and all of your crew in the midst of you have sunk.}%
\verse{All of the inhabitants of the coastlands are appalled over you, and their kings shudder\lebnote{“they have bristling hair”} in horror; they distort their faces!\lebnote{“they are distorted \textit{as to} faces”}}%
\verse{The traders among the peoples hiss over you; you have become a horror, and you shall be no more forever!’”\lebnote{“there is not you until eternity/endless ages” (cf. NRSV)}}%
\end{biblechapter}%
\begin{biblechapter}% Ezekiel 28
\verseWithHeading{A Prophecy Aimed at the King of Tyre}{And the word of Adonai came\lnCLL{} to me, saying,\lnCLM{}}%
\verse{“Son of man,\lnCLN{} say to the leader of Tyre, ‘Thus says the Lord Adonai: “Because your heart was haughty, and you said, ‘I am a god; I sit in the seat of the gods, I sit in the heart of the seas!’ But\lebnote{Or “And”} you are a human, not a god, and you gave your heart to be like the heart of a god.}%
\verse{Look, are you wiser than Daniel, so that no secret\lebnote{“all of hidden/secret”} is hidden from you?}%
\verse{By your wisdom and by your understanding you have gained for yourself wealth, and you have amassed gold and silver in your treasuries.}%
\verse{By the abundance of your wisdom in your trading you have increased your wealth and your heart was proud in\lebnote{Or “by/through”} your wealth.”’”}%
\verse{Therefore thus says the Lord Adonai: “Because of your regarding your mind\lnCLO{} like the mind\lnCLO{} of a god,}%
\verse{therefore look! I am bringing strangers over you, the most ruthless of the peoples, and they will draw their swords against the beauty\lebnote{Or “splendor”} of your wisdom, and they will defile your splendor.}%
\verse{They will thrust you down to the pit, and you will die a violent death\lebnote{“a death of a slain”} in the heart of the seas.}%
\verse{Will you indeed still say\lebnote{“Will you say, will you say”} “I am a god!” before the face of your killers?\lebnote{Or “slayers”} And in fact you are a human\lebnote{Or “man”} and not a god in the hand of those who pierce you.\lebnote{“piercers/slayers of you”}}%
\verse{You will die the death of the uncircumcised by the hand of strangers, for I myself\lebnote{Hebrew “I, I”} have spoken!” declares\lebnote{“declaration of”} the Lord Adonai.}%
\verse{And the word of Adonai came\lnCLL{} to me, saying,\lnCLM{}}%
\verse{“Son of man,\lnCLN{} raise a lament over the king of Tyre, and you must say to him, ‘thus says the Lord Adonai: “You were a perfect model of\lebnote{Or “for”} an example, full of wisdom and perfect of beauty.}%
\verse{You were in Eden, the garden of God, and every precious stone was your adornment: carnelian, topaz and moonstone,\lebnote{Meaning of Hebrew uncertain} turquoise, onyx and jasper, sapphire,\lebnote{Or “lapis lazuli”} malachite and emerald. And gold was the craftsmanship of your settings and your mountings in you; on the day when you were created they were prepared.}%
\verse{You were an anointed guardian cherub, and I placed you on God’s holy mountain;\lebnote{“on the mountain of the holiness of God”} you walked in the midst of stones of fire.}%
\verse{You were blameless in your ways from the day when you were created,\lebnote{“being created you”} until wickedness was found in you.}%
\verse{In the abundance of your trading, they filled the midst of you with violence, and you sinned; and I cast you as a profane thing from the mountain of God, and I expelled you, the guardian\lebnote{Or “guarding”} cherub, from the midst of the stones of fire.}%
\verse{Your heart was proud because of your beauty; you ruined your wisdom because of your splendor. I threw you on the ground before\lebnote{“in the face of”} kings; I have exposed you for viewing.\lebnote{“to look at you”}}%
\verse{From the abundance of your iniquities in the dishonesty of your trading, I profaned your sanctuaries and I brought fire from your midst; it consumed you, and I have turned you to ashes on the earth before the eyes of everyone who sees you.\lebnote{“before the eyes of all seeing you”}}%
\verse{All who know you\lebnote{“all of the knowers of you”} among the peoples are appalled over you; you have become as horrors, and you shall cease to exist forever.”’”\lebnote{“there is not you until eternity”}}%
\verse{And the word of Adonai came\lnCLL{} to me, saying,\lnCLM{}}%
\verse{“Son of man,\lnCLN{} set your face toward Sidon and prophesy against it,}%
\verse{and you must say, ‘Thus says the Lord Adonai: “Look! I am against you, Sidon, and I display my glory in the midst of you, and they will know that I am Adonai when I execute my judgments,\lebnote{“in/at doing in her judgments”} and I will show myself holy within it.}%
\verse{And I will send into it a plague, and blood will be in its streets. And the dead will fall in the midst of her by the sword that is against it from all around; and they will know that I am Adonai.}%
\verse{And there will not be any longer\lebnote{“and not it will be again”} a painful thorn and a sharp thornbush for\lebnote{Or “against”} the house of Israel from anywhere around them\lebnote{“from all \textit{of} all around them”} from those who are despising them,\lnCLP{} and they will know that I am the Lord Adonai.”’”}%
\verse{Thus says the Lord Adonai: “When I gather the house of Israel from the peoples to which they were scattered about in them, and I show myself holy in\lebnote{Or “among”} them before the eyes of the nations, then they will live on their soil, which I gave to my servant Jacob.}%
\verse{And they will live on it in safety, and they will build houses, and they will plant vineyards, and they will live in safety when I execute my judgments on all those who despise them from all around them,\lnCLP{} and then they will know that I am Adonai their God.”}%
\end{biblechapter}%
\begin{biblechapter}% Ezekiel 29
\verseWithHeading{Adonai’s Judgment upon Egypt}{In the tenth year, in the tenth month, on the twelfth\lebnote{“two ten”} day of the month, the word of Adonai came\lnCLQ{} to me, saying,\lnCLR{}}%
\verse{“Son of man,\lnCLS{} set your face against Pharaoh, the king of Egypt, and prophesy against him and against Egypt, all of it.}%
\verse{Speak, and you must say, ‘thus says the Lord Adonai: “Look! I am against you, Pharaoh, king of Egypt, the great sea monster, the one lying down in the midst of his Nile streams, who says to me, “It is my Nile, and I made it for myself.”\lebnote{Hebrew “me”}}%
\verse{And so I will put hooks in your jawbones, and I will make the fish of your Nile streams stick to your scales, and I will bring you up from the midst of your Nile streams, and all of the fish of your Nile streams which cling to your scales.}%
\verse{And I will fling you to the desert, you and all the fish of your Nile streams. On the surface of the field you will fall; you will not be gathered, and you will not be assembled. To the animals\lebnote{Hebrew “animal”} of the field and to the birds\lebnote{Hebrew “bird”} of the heavens I will give you as food.}%
\verse{And all of the inhabitants of Egypt will know that I am Adonai, because of their\lebnote{LXX, Syriac, and Vulgate: “you”} being a staff of reed for the house of Israel.}%
\verse{When they took hold of you with the hand, you snapped, and you split their every shoulder. And when they leaned on you, you broke, and you caused all of their loins to wobble.”}%
\verse{Therefore thus says the Lord Adonai: “Look! I am bringing on you a sword, and I will cut off from you human and animal.}%
\verse{And the land of Egypt will become\lebnote{Hebrew “be”} a desolation and ruins, and they will know that I am Adonai because he\lebnote{That is, Pharaoh} said, ‘The Nile is mine!\lebnote{“to me”} And I, I made it!’}%
\verse{Therefore look! I am against you and against your Nile streams, and I will make the land of Egypt into ruins, a pile of rubble, a desolation from Migdol to Syene and up to the boundary of Cush.}%
\verse{A foot of a human will not pass over it, and a foot of an animal will not pass over it, and so it will not be inhabited for forty years.\lebnote{Hebrew “year”}}%
\verse{And I will make the land of Egypt a desolation in the midst of desolated countries, and its cities to be in the midst of ruined cities. They will be a desolation for forty years, and I will scatter Egypt among the nations, and I will disperse them among the countries.”}%
\verse{For thus says the Lord Adonai: “At the end of forty years I will gather Egypt from the peoples among whom they were scattered.}%
\verse{And I will restore the fortunes of Egypt, and I will bring them back to the land of Pathros, to the land of their origin, and they will be a lowly kingdom there.}%
\verse{Of all the kingdoms it will be the most lowly, and it will not exalt itself again over the nations, and I will make them small so as not to rule over the nations.}%
\verse{And it will not be again for the house of Israel an object of trust, bringing to remembrance their guilt when they turned back to them,\lebnote{“at their turning”} and they will know that I am the Lord Adonai.”}%
\verse{And then\lebnote{“and it happened”} in the twenty-seventh year,\lebnote{“twenty and seven year”} in the first month, on the first day of the month, the word of Adonai came\lnCLQ{} to me, saying,\lnCLR{}}%
\verse{“Son of man,\lnCLS{} Nebuchadnezzar the king of Babylon, he made his army labor hard against Tyre. Every head was rubbed bare, and every shoulder was rubbed raw, but\lebnote{Or “and”} a wage was not paid for him and for his army from Tyre for the labor that he did against it.”}%
\verse{Therefore thus says the Lord Adonai: “Look! I am giving to Nebuchadnezzar the king of Babylon the land of Egypt, and he will carry its wealth away, and he will plunder its plunder, and he will loot its loot, and it will be a wage for his army.}%
\verse{As his wages that he worked for, I will give to him the land of Egypt, because they worked for me,” declares\lebnote{“declaration of”} the Lord Adonai.}%
\verse{“On that day I will cause power\lebnote{“a horn”} to grow up for the house of Israel, and for you I will give an opening of your mouth in the midst of them, and they will know that I am Adonai.”}%
\end{biblechapter}%
\begin{biblechapter}% Ezekiel 30
\verseWithHeading{A Lament over the Nation and Land of Egypt}{And the word of Adonai came\lnCLT{} to me, saying,\lnCLU{}}%
\verse{“Son of man,\lnCLV{} prophesy, and you must say, ‘thus says the Lord Adonai: “Wail, alas! For the day!}%
\verse{For a day is near; indeed, a day is near for Adonai. A day of cloud, a time of the nations it will be.}%
\verse{And a sword will come in Egypt, and anguish will be in Cush, at the falling of the slain in Egypt. And they will take its wealth, and its foundation will be demolished.}%
\verse{Cush and Put and Lud and all of Arabia and Kub and the people\lebnote{Or “children”} of the land of the treaty\lebnote{Or “covenant”} with them — by the sword they will fall.”}%
\verse{Thus says Adonai: “And the supporters of Egypt will fall, and the majesty of its strength will go down from Migdol to Syene; by the sword they will fall in it,” declares\lebnote{“declaration of”} the Lord Adonai.}%
\verse{“And they will be desolate in the midst of desolate countries, and its cities will be in the midst of ruined cities.}%
\verse{And they will know that I am Adonai when I put my fire in Egypt, and all of its helpers are broken.}%
\verse{On that day messengers will go down from before me\lebnote{“to the face of me”} in the ships to terrify unsuspecting Cush, and anguish will be in them on the day of Egypt; for look, it is coming!”}%
\verse{Thus says the Lord Adonai: “And I will put an end to the crowds\lebnote{Or “hordes”} of Egypt by the hand of Nebuchadnezzar the king of Babylon.}%
\verse{He and his people with him, the most ruthless of nations, will be brought to destroy the land. And they will draw their swords against Egypt, and they will fill the land with the slain,}%
\verse{and I will make the Nile streams dry land, and I will sell the land into the hand of bad people, and I will lay waste the land and its fullness by the hand of strangers. I, Adonai, I have spoken.”}%
\verse{Thus says the Lord Adonai: “And I will destroy the idols, and I will put an end to the worthless idols from Memphis. And there will no longer be\lebnote{“not be any longer”} a prince from the land of Egypt, and I will put fear in the land of Egypt.}%
\verse{And I will lay waste Pathros, and I will set fire in Zoan, and I will execute judgments in No.\lnCLW{}}%
\verse{I will pour out my rage over Sin, the stronghold of Egypt, and I will cut off the crowd\lebnote{Or “horde”} of No.\lnCLW{}}%
\verse{And I will set fire in Egypt, certainly Sin will writhe, and No\lnCLW{} will be breached, and Memphis will face enemies daily.\lebnote{“of by day”}}%
\verse{The young men of On and Pi Beseth will fall by the sword, and they will go into captivity.}%
\verse{And at Tahpanhes the day will become dark when I break the yoke of Egypt, and in it the majesty of its strength will come to an end. A cloud will cover it, and its daughters will go into captivity.}%
\verse{And I will execute judgments in Egypt, and they will know that I am Adonai.”}%
\verse{And then\lebnote{“and it was”} in the eleventh\lebnote{“on one ten”} year, in the first month, on the seventh day of the month, the word of Adonai came\lnCLT{} to me, saying,\lnCLU{}}%
\verse{“Son of man,\lnCLV{} the arm of Pharaoh, the king of Egypt, I have broken. And look, it has not been bound up for giving of a remedy, or for the placing of a splint to bind it up to make it strong to take hold of the sword.”}%
\verse{Therefore thus says the Lord Adonai: “Look! I am against Pharaoh, the king of Egypt, and I will break his strong arm and the broken arm, and I will let the sword fall from his hand.}%
\verse{And I will scatter Egypt among the nations, and I will disperse them among the countries.}%
\verse{And I will strengthen the arm of the king of Babylon, and I will give my sword into his hand, and I will break the arms of Pharaoh, and he will groan with the groaning of the mortally wounded before him.\lebnote{“the face of him”}}%
\verse{And I will strengthen the arm of the king of Babylon, and the arms of Pharaoh will fall. And they will know that I am Adonai when I give my sword into the hand of the king of Babylon, and he will stretch it out to the land of Egypt.}%
\verse{And I will scatter Egypt among the nations, and I will disperse them into the countries, and they will know that I am Adonai.”}%
\end{biblechapter}%
\begin{biblechapter}% Ezekiel 31
\verseWithHeading{Pharaoh Is Warned Through Assyria’s Destruction}{And then\lebnote{“and it happened/it was”} in the eleventh\lebnote{“on one ten”} year, in the third month, on the first day of the month, the word of Adonai came\lebnote{“was”} to me, saying,\lebnote{“to say”}}%
\verse{“Son of man,\lebnote{Or “mortal,” or “son of humankind”} say to Pharaoh, the king of Egypt, and to his crowd,\lnCLX{} ‘To whom are you like in your greatness?}%
\verse{Look! Assyria was a cedar in Lebanon, with beautiful branches\lebnote{Hebrew “branch”} and a forest giving shade, and very high,\lebnote{“high of height”} and its treetop was between the clouds.}%
\verse{Waters made it great, the deep made it grow high; its rivers were going all around its planting area,\lebnote{“base of its tree”; NRSV, “in the place where it was planted”} and its channels it sent out to all of the trees of the field.}%
\verse{Therefore it became tall, with its height more than all of the trees of the field, and its branches became numerous, and its branches became long from its sending its shoots from abundant water.\lebnote{“from water much at sending \textit{shoots} its”}}%
\verse{In its branches all the birds\lnCLY{} of the heaven\lnCLZ{} made their nest, and under its branches all the animals\lebnote{Hebrew “animal”} of the field gave birth, and in its shadow all the many nations lived.}%
\verse{And it was beautiful in its greatness, in the length of its branches, for its root was toward much water.}%
\verse{Cedars in the garden of God could not be equal to it; fir trees\lebnote{Or possibly, juniper} could not resemble its branches, and plane trees were not even like its branches; any tree even in the garden of God could not resemble it in its beauty.}%
\verse{I made it beautiful with the abundance of its branches, and all of the trees of Eden that were in the garden of God envied it.’”}%
\verse{Therefore thus says the Lord Adonai: “Because\lebnote{“because that”} it was tall in height and it set its treetop between\lebnote{Hebrew “to between”} thick clouds, and he took pride in his tallness,\lebnote{“and his heart he raised in his tallness”}}%
\verse{then\lebnote{Or “and”} I gave it into the hand of the leader of nations; he dealt thoroughly\lebnote{“doing he did”} with it according to its wickedness. I drove it out.}%
\verse{And strangers cut it off, the most ruthless of nations, and they abandoned it. On the mountains and in all of the valleys its branches fell, and its branches were broken in all the river channels of the land, and all the peoples of the world went out from its shadow, and they abandoned it.}%
\verse{On its fallen trunk all the birds\lnCLY{} of the heaven\lnCLZ{} now dwell, and all the animals of the field were on its branches.}%
\verse{This occurred so that all of the trees with abundant water will not become tall, and they will not set their treetop between\lebnote{Or “among”} their thick foliage,\lebnote{Or “clouds”} and so thatall of the trees that are abundantly watered\lebnote{“all of \textit{the} drinkers of water”} will not stand up to them\lebnote{“will not stand to them”} in their tallness, for all of them, they have been given over to death, to the world below in the midst of mortals,\lebnote{“sons of men,” or “children of humankind”} to the people going down to the grave.”\lebnote{Or “pit”}}%
\verse{Thus says the Lord Adonai: “On the day of its going down to Sheol, I caused mourning; I covered over it with the deep, and I withheld its rivers, and many waters were restrained, and I brought gloom over it; Lebanon and all of the trees of the field, they had fainted because of it.}%
\verse{From the sound of its downfall I caused nations to shake when I made it go down to Sheol, with the people going down to the grave,\lebnote{Or “the pit”} and so in the world below all of the trees of Eden, the choice and the best of Lebanon, all the well-watered trees\lebnote{“all \textit{of}\textit{the} drinkers of water”} were comforted!}%
\verse{They also went down with it to Sheol to those who died by the sword,\lebnote{“to the slain of \textit{the} sword”} and its army who had lived in its shadow in the midst of nations.}%
\verse{To whom could you be compared, whether in glory or in majesty\lebnote{“To who you are like as in glory and in majesty”} among the trees of Eden? And yet you will be brought down with the trees of Eden to the world below; in the midst of the uncircumcised you will lie with those who died by the sword.\lebnote{“the slain of \textit{the} sword”} That is Pharaoh and his entire\lebnote{“all his hordes/crowd”} crowd!”\lnCLX{} declares\lebnote{“declaration of”} the Lord Adonai.}%
\end{biblechapter}%
\begin{biblechapter}% Ezekiel 32
\verseWithHeading{A Lament for Pharaoh, King of Egypt}{And then\lebnote{“and it was/happened”} in the twelfth\lnCMA{} year, in the twelfth\lebnote{“in two ten month”} month, on the first day of the month, the word of Adonai came\lnCMB{} to me, saying,\lnCMC{}}%
\verse{“Son of man,\lnCMD{} raise a lament over Pharaoh, the king of Egypt, and you must say to him, ‘With a fierce, strong lion among nations you compared yourself, and you are like the sea monster in the seas, and you thrash about in your rivers, and you make water turbid with your feet, and you make your rivers muddy.}%
\verse{Thus says the Lord Adonai: Now I will spread my net over you in the assembly of many peoples, and I will bring you up in my dragnet.}%
\verse{And I will throw you on the ground; on the surface of the open field I will hurl you, and I will cause every bird of the heaven to dwell on you, and I will satisfy the animals\lebnote{Hebrew “animal”} of all of the world from you.}%
\verse{And I will put your flesh on the mountains, and I will fill the valleys with your carcass.}%
\verse{And I will water the land with your discharge from your blood on the mountains, and valleys will be filled from\lebnote{Or “with”} you.}%
\verse{And I will cover you at\lebnote{Or “by”} extinguishing your heavens, and I will make dark their stars, and I will cover the sun with the cloud, and the moon will not give its light.}%
\verse{All\lebnote{Or “every”} sources of light in the heavens, I will make them dark over you, and I will put\lebnote{Or “give”} darkness on your land,’” declares\lnCME{} the Lord Adonai.}%
\verse{“And I will disturb the hearts\lebnote{Hebrew “heart”} of many peoples at my bringing about your captivity among the nations, to countries that you do not know.}%
\verse{And I will cause many peoples to be awestruck over you, and their kings will shudder over you in horror, at\lebnote{Or “when”} my brandishing my sword before their faces, and they will tremble continually,\lebnote{“for moments/seconds”} each person for his life, on the day of your downfall.”}%
\verse{For thus says the Lord Adonai: “The sword of the king of Babylon, it will come to you.}%
\verse{By the swords of warriors I will cause to fall your hordes by the most ruthless of the nations, all of them, and they will devastate the majesty of Egypt, and all of its hordes will be destroyed.}%
\verse{And I will destroy all her domestic livestock beside\lebnote{“from on”} many waters. The feet of humans\lebnote{Hebrew “human”} will not make them turbid again, and the hooves of domestic livestock will not make them turbid.}%
\verse{Then I will make their waters settle, and their rivers as the olive oil I will let flow,” declares\lnCME{} the Lord Adonai.}%
\verse{“Then I make the land of Egypt a desolation, so that the land will be stripped from its fullness when I strike all of those dwelling in it, and they will know that I am Adonai.}%
\verse{This is\lebnote{“she \textit{is}”} a lament, and they will chant it as a lament; the daughters of the nations will chant it as a lament over Egypt, and over all of its hordes they will chant it as a lament,” declares\lnCME{} the Lord Adonai.}%
\verse{And then\lebnote{“and it happened/was”} in the twelfth\lnCMA{} year, on the fifteenth\lebnote{“in the five ten of”} day of the month, the word of Adonai came\lnCMB{} to me, saying,\lnCMC{}}%
\verse{“Son of man,\lnCMD{} mourn over the hordes of Egypt, and make her go down, and with her the daughters of mighty nations, down to the deep underworld\lebnote{“to \textit{the} world/earth deeps/depth”} with the people going down to the grave.\lebnote{Or “pit” or “netherworld”}}%
\verse{You are more lovely than whom?\lebnote{“from whom are you lovely”} Go down and be laid in rest with the uncircumcised.}%
\verse{In the midst of the people slain by the sword it\lebnote{Or “she”; that is, Egypt} is given they will fall to a sword; they carried her\lebnote{Or “it”} off and all of her\lebnote{Or “its”} hordes.}%
\verse{The chiefs of the warriors will speak to him from the middle of Sheol; with his helpers they have gone down; the uncircumcised lie still, those slain by the sword.\lebnote{“\textit{the} slain of \textit{the} sword”}}%
\verse{Assyria is there and all of its assembly; all around it are its graves, all of them killed, those fallen by the sword,}%
\verse{who\lebnote{That is, Assyria} will be given its graves in the remote areas of the pit, and its assembly will be all around its grave, all of them killed, fallen by the sword, those who spread terror in the land of the living.}%
\verse{Elam is there and all of its hordes, all around its grave, all of them dead who fell by the sword, those who went down uncircumcised to the depths of the underworld,\lebnote{“land/earth/world”} those who spread their terror to the land of the living, and now they bear their disgrace with the people going down to the grave.\lebnote{Or “pit” or “sheol”}}%
\verse{In the midst of the slain they made a bed for her with all of her hordes all around its graves, all of them uncircumcised, slain by the sword, for their terror was spread in the land of the living, and they now bear their disgrace with the people going down to the grave;\lnCMF{} in the midst of the slain it was placed.}%
\verse{Meshech and Tubal are there and all of its hordes, all around him its graves, all of them uncircumcised and killed by the sword, for they gave their terror in the land of the living.}%
\verse{And they do not lie with warriors fallen from long ago, who went down to Sheol with their weapons of war,\lebnote{“with the weapons of war their”} and they placed their sword under their heads, and their shields\lebnote{Reading “their shields” for “their guilt”} were on their bones, for the terror of the warriors was in the land of the living.}%
\verse{And you\lebnote{Evidently refers to Egypt} too in the midst of uncircumcised people will be broken, and you will lie with those slain by the sword.}%
\verse{There is Edom, its kings and all of its leaders who are laid along with their might with those killed by the sword; they will lie with the uncircumcised and with the people going down to the grave.\lnCMF{}}%
\verse{There are also the princes of the north, all of them, all of the Sidonians who have gone down, being ashamed to lie with the slain because\lebnote{Or “in spite of” (cf. NJPS, NIV)} of their terror that they caused from their might, and they lie uncircumcised with those killed by the sword, and they bear their disgrace along with the people going down to the grave.\lnCMF{}}%
\verse{Pharaoh will see them, and he will be comforted over all of his hordes killed by the sword, Pharaoh and all of his army,” declares\lnCME{} the Lord Adonai.}%
\verse{“For\lebnote{Or “although”} he spread\lebnote{Hebrew “I gave”} my terror in the land of the living, and so he will be laid down in the midst of the uncircumcised with those killed by the sword: Pharaoh and all of his hordes,”declares\lnCME{} the Lord Adonai.}%
\end{biblechapter}%
\begin{biblechapter}% Ezekiel 33
\verseWithHeading{Ezekiel Appointed as a Watchman and His Responsibility}{And the word of Adonai came\lnCMG{} to me, saying,\lnCMH{}}%
\verse{“Son of man,\lnCMI{} speak to your people,\lnCMJ{} and you must say to them, ‘A land, if I bring over it a sword and the people of the land take a man, one from their number, and they appoint him for them as a watchman,}%
\verse{and he sees the sword coming against the land, and he blows on the horn and he warns the people,}%
\verse{and anyone who listens\lebnote{“the listening”} hears the sound of the horn and he does not take warning\lebnote{“not he takes warning”} and the sword comes and it takes him, his blood will be on his own head.}%
\verse{For he heard the sound of the horn and he did not take warning; his blood will be on him. But if he took warning, he saved his life.}%
\verse{And as for the watchman, if he sees the sword coming and he does not blow\lebnote{“and not he blows on”} the horn, and the people are not warned, and the sword comes and it takes their lives,\lebnote{“it takes from them a life”} he will be taken through his guilt, but his blood from the hand of the watchman I will seek.’}%
\verse{“And you, son of man,\lnCMI{} I have made you a watchman for the house of Israel; if you hear a word from my mouth, then you must warn them from me.}%
\verse{When I say\lebnote{“At my saying”} to the wicked, ‘Wicked one, you will certainly die,’ and you did not speak\lebnote{“and not you spoke”} to warn the wicked from his way, he, the wicked, will die by his guilt, but his blood I will seek from your hand.}%
\verse{But, you, if you warn the wicked from his way, to turn from it, and he does not turn\lebnote{“and not he turns”} from his way, he will die by his guilt, and you will have saved your life.}%
\verse{“And you, son of man,\lnCMI{} say to the house of Israel, saying,\lnCMH{} ‘So you said, saying,\lnCMH{} “Indeed, our transgressions and our sins are on us, and because of them we are rotting, and how can we live?”’}%
\verse{Say to them, ‘As I live,’\lebnote{“live I”} declares\lebnote{“declaration of”} the Lord Adonai, ‘Surely I have no delight\lebnote{“if”} in the death of the wicked, except\lebnote{“for it”} in the wicked returning from his way, and he lives. Turn back! Turn back from your ways, O evil ones, for why should you die, house of Israel?’}%
\verse{“So\lnCMK{} you, son of man,\lnCMI{} say to your people\lnCMJ{} that the righteousness of the righteous shall not save him in the day of his transgression, and the wickedness of the wicked will not cause him to stumble\lebnote{“not he will stumbled by it”} on the day of his returning from his wickedness; and the righteous will not be able to live by it on the day when he returns to his sin.}%
\verse{When I say to the righteous, ‘Certainly he will live,’ and he trusted in\lebnote{Or “on”} his righteousness, and he turns and he does injustice, all of his righteousness\lebnote{According to the reading tradition (\textit{Qere})} will not be remembered, and because of his injustice that he did, because of it he will die.}%
\verse{And when I say to the wicked, ‘Certainly you will die,’ but\lebnote{Or “and”} he returns from his sin and he does justice and righteousness —}%
\verse{for example, the wicked returns a pledge for a loan, he restores stolen property,\lebnote{“robbed/stolen \textit{items}”} he goes in the statutes of life so as not to do injustice — certainly he will live; he will not die.}%
\verse{All of his sins that he committed,\lebnote{“sins that he sinned”} they will not be remembered against\lebnote{Or “for”} him, and he did justice and righteousness; certainly he will live.}%
\verse{“Yet\lnCMK{} your people\lnCMJ{} say, ‘The way of the Lord is not fair!’ yet their way is not fair.}%
\verse{When the righteous turns from his righteousness and does injustice, then he will die because of it.\lebnote{“by/in them”}}%
\verse{And when the wicked turns from his wickedness, and he does justice and righteousness, because of it\lebnote{Hebrew “them”} he will certainly live!\lebnote{“he, he will live”}}%
\verse{Yet you said, ‘The way of the Lord is not fair!’ I will judge you, house of Israel, each person according to his ways.”}%
\verse{And then\lebnote{“and it happened”} it was in the twelfth\lebnote{“two of ten”} year, in the tenth month, on the fifth day of the month of our exile, a survivor from Jerusalem came to me, saying,\lnCMH{} “The city was destroyed!”}%
\verse{And the hand of Adonai was on me on the evening before\lebnote{“to the face of”} the coming of the survivor, and he opened my mouth before the survivor came to me in the morning,\lebnote{“until coming to me in the morning”} and my mouth was opened, and I was no longer\lebnote{“and not I was dumb still”} dumb.\lebnote{Or “silent” or “speechless”}}%
\verse{And the word of Adonai came\lnCMG{} to me, saying,\lnCMH{}}%
\verse{“Son of man,\lnCMI{} the inhabitants of these ruins on the soil of Israel are saying, ‘Abraham was one man, and he took possession of the land, and we are many; the land has been given for\lebnote{Or “to”} us as a possession.’}%
\verse{Therefore say to them, ‘Thus says the Lord Adonai: “You eat blood in your meat, and you raise your eyes to your idols and you pour out blood, and yet you want to take possession of the land?\lebnote{Or “you would take possession of”}}%
\verse{You rely on your sword, you do a detestable thing, and each man defiles\lebnote{“a man the wife of neighbor you defile”} the wife of his neighbor, and yet you want to take possession of the land?”’\lebnote{Or “you would take possession of the land”}}%
\verse{“Thus you must say to them, ‘Thus says the Lord Adonai: “As I live,\lebnote{“I live”} surely\lebnote{“if not”} whoever is in the ruins, by the sword they will fall, and whoever is on the surface of the open field, I will give him to the animals\lebnote{Hebrew “animal”} to eat him, and whoever is in the stronghold and in the cave, they will die by the plague.}%
\verse{And I will make the land a desolation and a wasteland, and the pride of its strength will come to an end, and the mountains of Israel will be desolate with no one traveling through them.”’\lebnote{“from there is not \textit{someone} going over”}}%
\verse{And they will know that I am Adonai when I make my land a desolation and a wasteland, because of all of their detestable things that they have done.}%
\verse{“And you, son of man,\lnCMI{} your people,\lnCMJ{} the ones talking together concerning you beside the walls and in the doorways of the houses, each one with his brother saying,\lnCMH{} ‘Please come and hear what is the word that is going out from Adonai.’\lebnote{“the going out from with Adonai”}}%
\verse{And they come to you as people do,\lebnote{“like \textit{the} coming of a people”} and they sit before you,\lebnote{“to the face of you”} and my people hear your words, and they do not do them, for they are showing\lebnote{Or “expressing”} passion in their mouth, but their heart is going after ill-gotten gain.}%
\verse{Now, look! You are to them like a sensual song,\lebnote{“like a song of lust”} beautiful of voice and played well on an instrument,\lebnote{“doing good to play a music instrument”} and they hear your words, but they are not doing them.\lebnote{“and/but doing there is not they them”}}%
\verse{So\lnCMK{} when it comes,\lebnote{“in/at her coming”} look! It is coming! And then they will know that a prophet was in the midst of them.”}%
\end{biblechapter}%
\begin{biblechapter}% Ezekiel 34
\verseWithHeading{Prophecy Concerning Israel’s Shepherds and Sheep}{And the word of Adonai came\lebnote{“was”} to me, saying,}%
\verse{“Son of man,\lebnote{Or “mortal,” or “son of humankind”} prophesy against the shepherds of Israel, prophesy, and you must say to them, to the shepherds, ‘Thus says the Lord Adonai: “Woe to the shepherds of Israel who were feeding themselves! Must not the shepherds feed the flock?}%
\verse{The fat you eat, and you clothe yourself with the wool; the well-nourished animals you slaughter, but you do not feed the flock.}%
\verse{The weak you have not strengthened, and the sick you have not healed, and with respect to\lebnote{“for”} the hurt\lebnote{Or “wounded”} you have not bound them up, and you have not brought back the scattered, and you have not sought the lost, but rather you ruled over them with force\lebnote{Or “violence”} and with ruthlessness.}%
\verse{And they were scattered without\lebnote{“from not”} a shepherd, and they were\lebnote{Or “became”} as food for all the animals\lnCML{} of the field when they were scattered.}%
\verse{My flock went astray upon all of the mountains and on every high hill, and so upon all the surface of the world\lebnote{Or “earth”} my flock were scattered, and there was no one seeking them,\lebnote{“there was not \textit{someone} seeking \textit{them}”} and there was no one searching for them.”\lebnote{“there was not \textit{someone} searching for \textit{them}”}}%
\verse{Therefore, hear, O shepherds, the word of Adonai:}%
\verse{“As I live,”\lebnote{“live I”} declares\lnCMM{} the Lord Adonai, “Surely\lebnote{“if not”} because my flock have become as plunder, and my flock became as food to\lebnote{Or “for”} all the animals\lnCML{} of the field, since there was not a shepherd, since my shepherds have not sought my flock, but the shepherds fed themselves, and they fed not my flock,’”}%
\verse{therefore, O shepherds, hear the word of Adonai,}%
\verse{‘Thus says the Lord Adonai: “Look! I am against the shepherds, and I will seek my flock from their hand, and I will put an end to them from shepherding flocks, and the shepherds will no longer\lnCMN{} feed themselves, and I will deliver my flocks from their mouth, so that they will not be as food for them.”}%
\verse{“‘For thus says the Lord Adonai: “Look! I, even I, will seek my flock, and I will look after them,}%
\verse{just like the caring of a shepherd for his herd on\lebnote{Or “in”} the day when he is in the midst of his scattered flock. Thus I will look after my flock, and I will deliver them from all the places to which they were scattered\lebnote{“which they were scattered there”} on the day of storm and stress.\lebnote{“on \textit{the} day of cloud and deep gloom”}}%
\verse{And I will bring them out from the peoples, and I will gather them from the countries, and I will bring them to their soil, and I will feed them on the mountains of Israel, in the valleys, and in all of the settlements of the land.}%
\verse{I will feed them in good pasture, and their pasture will be on the mountains of the heights of Israel; there they will lie down in good pasture, and on lush pasture they will feed on the mountains of Israel.}%
\verse{I myself\lnCMO{} will feed my flock and I myself\lnCMO{} will allow them to lie down,” declares\lnCMM{} the Lord Adonai.}%
\verse{“I will seek the lost, and I will bring back the scattered, and I will bind up the one hurt, and I will strengthen the sick; and the fat and the strong I will destroy. I will feed her with justice.”}%
\verse{“‘And you, my flock, thus says the Lord Adonai: “Look! I am judging between one sheep and another,\lebnote{“small sheep and small sheep”} between the rams and between the he-goats.}%
\verse{Is it not enough for you\lebnote{“is it \textit{too} little from you”} that you feed on the good pasture? And still you must trample the remainder of your pasture with your feet, and clear water\lebnote{“and settling of water”} you drink, and the leftover water\lebnote{“\textit{water} being left”} you must make muddy with your feet.}%
\verse{And my flock must graze the pasture treaded upon by\lnCMP{} your feet, and must drink the mud puddle stirred up by\lnCMP{} your feet.”}%
\verse{“‘Therefore thus says the Lord Adonai to them: “Look! I, even I will judge between fat sheep and between lean sheep,}%
\verse{because with your flank and with your shoulder you shoved, and with your horns you pushed all of the sick animals until\lebnote{“until that”} you scattered it\lebnote{Or “them”} to the outside.}%
\verse{And so I will save my flock, and they will no longer be\lebnote{“not they will be longer for plunder”} for plunder, and I will judge between one sheep and another.\lebnote{“sheep and sheep”}}%
\verse{And I will set up over them one shepherd, and he will feed them; that is, my servant David. He will feed them, and he will be for them as\lebnote{Or “like”} a shepherd.}%
\verse{And I, Adonai, I will be for them as God,\lebnote{“to a God”} and my servant David will be a leader in the midst of them. I, Adonai, I have spoken.}%
\verse{And I will make them a covenant of peace, and I will put an end to wild animals\lnCML{} from the land, and they will dwell in the desert safely,\lebnote{“with confidence/trust”} and they will sleep in the forest.}%
\verse{And I will make them and the area all around my hill a blessing, and I will let the rain go down at its appointed time;\lebnote{“in time his”} they will be rains of blessing.}%
\verse{And the tree\lebnote{Or “trees”} of the field will give its fruit, and the land will give its produce, and they will be on their land safely,\lebnote{“with/in conficence”} and they will know that I am Adonai when I break their yoke, and I will deliver them from the hand of the ones enslaving them.}%
\verse{And they will not be any longer plunder to the nations, and the animals of the land will not eat them, and they will dwell in safety,\lebnote{“with confidence”} and there will not be anyone frightening them.}%
\verse{And I will raise for them a garden plot\lebnote{Or “planting”} of renown, and they will no longer\lnCMN{} be victims of famine in the land, and they will not bear any more the insult of the nations.}%
\verse{And they will know that I, Adonai their God, am with them and they are my people, the house of Israel,” declares\lnCMM{} the Lord Adonai,}%
\verse{“and you are my flock, the flock of my pasture. You are my people;\lebnote{“humanity, Mankind, human, mankind”} I am your God,” declares\lnCMM{} the Lord Adonai.’”}%
\end{biblechapter}%
\begin{biblechapter}% Ezekiel 35
\verseWithHeading{Prophecy Directed at Edom}{And the word of Adonai came\lebnote{“was”} to me, saying,\lnCMQ{}}%
\verse{“Son of man,\lebnote{Or “mortal,” or “son of humankind”} set your face against the mountain of Seir and prophesy against it,}%
\verse{and you must say to it, ‘Thus says the Lord Adonai: “Look! I am against you, mountain of Seir, and I will stretch out my hand against you, and I will make you a desolation and a wasteland.}%
\verse{Your cities I will make ruins, and you will be a desolation, and then you will know that I am Adonai,}%
\verse{because there has been to you an ancient hostility,\lebnote{“enmity of eternity”} and you handed over the Israelites\lebnote{“sons/children of Israel”} to the power of the sword at the time of their disaster, at the time of their final punishment.\lebnote{“\textit{the} time of punishment of end”}}%
\verse{Therefore as I live,”\lnCMR{} declares\lnCMS{} the Lord Adonai, “Certainly to blood guilt I will prepare you and blood\lebnote{Or “blood vengeance”} will pursue; since\lebnote{“if not”} you did not hate blood,\lebnote{“not blood you hated”} it will pursue you.}%
\verse{And I will make the mountain of Seir into a desolation and ruin, and I will cut off from it both he who is crossing over and he who is returning.}%
\verse{And I will fill its mountains with its slain; your hills and your valleys and all of your watercourses, the slain by the sword will fall in them.}%
\verse{An everlasting desolation\lebnote{“of eternity/endless ages”} I will make you, and your cities will not return, and you will know that I am Adonai.}%
\verse{I do this because of your saying,\lebnote{Or “thinking”} ‘The two nations and the two lands, they will be mine,\lebnote{“to me”} and we will take possession of it,’ and yet Adonai was there.}%
\verse{Therefore, as I live,”\lnCMR{} declares\lnCMS{} the Lord Adonai, “so I will deal with you according to your anger and according to your jealousy that you did,\lebnote{Or “showed” or “acted on”} because of your hatred against them, and I will make myself known among them when I judge you.}%
\verse{And you will know that I, Adonai, I have heard all of your contemptible words that you said\lebnote{Or “spoke”} against the mountains of Israel, saying,\lnCMQ{} ‘They will be desolate; they are given to us as food.’}%
\verse{And you magnified yourself against me with your insolent speech,\lebnote{“with your mouth”} and you spoke voluminously against me with your words — I heard!}%
\verse{Thus the Lord Adonai says, ‘As the whole world rejoices,\lebnote{“as/like \textit{the} rejoicing of all of the world”} I will make you a desolation.}%
\verse{Like your rejoicing over the inheritance of the house of Israel\lebnote{“as/like rejoicing your for the inheritance of the house of Israel”} because\lebnote{“because that”} it was desolate, I will do to you; you will be a desolation, mountain of Seir and all of Edom, indeed all of it; and they will know that I am Adonai.’”’}%
\end{biblechapter}%
\begin{biblechapter}% Ezekiel 36
\verseWithHeading{Prophecy to the Mountains of Israel and of Israel’s Restoration}{“And you, son of man,\lnCMT{} prophesy to the mountains of Israel; and so you must say, ‘Mountains of Israel, hear the word of Adonai;}%
\verse{thus says the Lord Adonai: “Because the enemy said concerning you, ‘Ah, and\lebnote{Or “now”} the ancient high places will be to us as a possession!’”’}%
\verse{Therefore prophesy, and you must say, ‘Thus says the Lord Adonai: “Because indeed\lebnote{“with because”} when you were desolate,\lebnote{“\textit{you} being desolate”} then they crushed you from all around, so that you became a possession to the remainder\lnCMU{} of the nations, and you became a byword and object of slander for the people.”\lebnote{“on lips of \textit{the} tongue and slander of \textit{the} people”}}%
\verse{Therefore, mountains of Israel, hear the word of the Lord Adonai; thus says the Lord Adonai to the mountains and to the hills, to the river channels and to the valleys and to the ruins that are desolate and to the cities that are abandoned and that have become as plunder and as scorn for the remainder\lnCMU{} of the nations that are from all around.}%
\verse{Therefore thus says the Lord Adonai: “Certainly\lnCMV{} in the fire of my passion I spoke against the rest of the nations and against all\lebnote{“against Edom all of it”} Edom, who made my land as a possession, because of\lebnote{Or “in order to plunder its pasture\textit{land}”} its pastureland for plunder, for themselves in all of the joy of their whole heart and in their inner disdain.”\lebnote{Or “malice/disrespect”}}%
\verse{Therefore prophesy against the land of Israel, and you must say to the mountains and to the hills, to the river channels and to the valleys, thus says the Lord Adonai: “Look! I spoke in my passion and in my anger the insult of the peoples you have endured.”}%
\verse{Therefore thus says the Lord Adonai: “I swear\lebnote{“I, raise my hand”} certainly\lnCMV{} that the nations that are all around you,\lebnote{“that \textit{are} to you from all around”} they indeed will endure their disgrace.}%
\verse{“‘“But you, mountains of Israel, your branch you will shoot, and your fruit you will carry\lebnote{Or “bear”} for my people Israel, for they are soon to come.}%
\verse{For look! I am for you, and I will turn you, and you will be tilled, and you will be planted.}%
\verse{And I will cause your population\lnCMW{} to increase for you, all of the house of Israel, all of it, and the towns will be inhabited, and the ruins will be built.}%
\verse{And I will cause your population\lnCMW{} to increase for you, and domestic animals, even they will be numerous, and they will be fruitful, and I will cause you to be inhabited like your ancient times,\lebnote{“past days/times/ancient times”} and I will do good things more than in your former times, and you will know that I am Adonai.}%
\verse{And I will send over you my people Israel, and they will take possession of you, and you will be for them as an inheritance, and you will no longer bereave them of their children.”\lebnote{“not you will again longer to bring bereavement for them”}}%
\verse{“‘Thus says the Lord Adonai, because they are saying to you, “You are an eater of humans,\lnCMX{} and you are one who makes childless\lebnote{“maker of childless”} your people.”\lebnote{That is, Israel and Judah; Cf. Ezek 37:15–22}}%
\verse{Therefore humans\lnCMX{} you will not eat any longer, and your nation you will not make childless\lebnote{According to the reading tradition (\textit{Qere})} any longer,” declares\lnCMY{} the Lord Adonai.}%
\verse{“And I will not let you hear any longer the insult of the nations, and the disgrace of the peoples you will not bear any longer; you yourself will not cause your nation to stumble,” declares\lnCMY{} the Lord Adonai.’”}%
\verse{And the word of Adonai came\lebnote{“was”} to me, saying,\lebnote{“to say”}}%
\verse{“Son of man,\lnCMT{} the house of Israel, they were dwelling on their soil, and they defiled it with their way and with their deeds; like the uncleanness of menstruation was their way before me.\lebnote{“to the face of me”}}%
\verse{And I poured out my rage on them for the blood that they poured on the land and for their idols with which they defiled it.}%
\verse{And I scattered them among the nations, and I dispersed them in the countries according to their way, and according to their deeds I judged them.}%
\verse{And they came\lebnote{Or “went”} into the nations to which they went,\lebnote{“who/which they went there”} and they profaned my holy name\lnCMZ{} when they said to them, ‘These are the people of Adonai, and from his land they went out.’}%
\verse{And I was concerned for my holy name,\lnCMZ{} which, the house of Israel defiled among the nations to which they went.\lebnote{“which they went there”}}%
\verse{“Therefore thus say to the house of Israel, ‘Thus says the Lord Adonai: “Not for your sake\lebnote{“not for the sake of/on behalf of you”} am I about to act, house of Israel, but\lebnote{“for it”} for my holy name,\lebnote{“for the name of holiness my”} which you defiled among the nations to which you went.\lebnote{“which you went there”}}%
\verse{And I will consecrate my great name, which was profaned among the nations and which you have profaned in the midst of them, and the nations will know that I am Adonai!”’ a declaration of the Lord Adonai, when I show myself holy\lebnote{“at/in showing/to show self holy my”} before\lebnote{“to their eyes”} their eyes.}%
\verse{“‘And I will take you from the nations, and I will gather you from all of the lands, and I will bring you to your land.}%
\verse{And I will sprinkle on you pure water, and you will be clean from all of your uncleanness, and I will cleanse you from all of your idols.}%
\verse{And I will give a new heart to you, and a new spirit I will give\lebnote{Or “put/place”} into your inner parts, and I will remove the\lebnote{Or “your”} heart of stone from your flesh, and I will give to you a heart of flesh.}%
\verse{And I will give\lebnote{Or “place/put”} my spirit into your inner parts, and I will make it so that you will go\lebnote{Or “walk”} in my rules, and my regulations you will remember, and you will do them.}%
\verse{And you will dwell in the land that I gave to your ancestors,\lebnote{Or “fathers”} and you will be to me as a people, and I will be to you as\lebnote{Hebrew “to”} God.}%
\verse{And I will save you from all of your uncleanness, and I will call to the grain, and I will cause it to increase, and I will not bring famine upon you.}%
\verse{And I will cause the fruit of the tree and the crop of the field to increase, so that you will not suffer again the disgrace of famine among the nations.}%
\verse{And you will remember your evil ways and your deeds that were not good, and you will loathe yourself\lebnote{“for face your”} over\lebnote{Or “because of”} your iniquities and over your detestable things.}%
\verse{But not for your sake am I acting,” declares\lnCMY{} the Lord Adonai. “Let it be known to you, be ashamed, and be put to shame because of your ways, house of Israel.}%
\verse{“Thus says the Lord Adonai: ‘On the day when I cleanse you from all of your iniquities, I will cause the cities to be inhabited, and the ruins will be rebuilt.}%
\verse{And the land that was desolate will be cultivated in the very place that\lebnote{“in the place that/which/where”} it was desolate before the eyes of all of the persons crossing over.}%
\verse{And they will say, “This desolate land\lebnote{“the land this the being desolate”} has become like the garden of Eden, and the wasted and desolate and destroyed cities, now being refortified, are inhabited.”}%
\verse{And the nations who are left all around, you will know that I, Adonai, I built that which was destroyed;\lebnote{“the destroyed”} I planted the desolate land; I, Adonai, I have spoken, and I will act.’}%
\verse{“Thus says the Lord Adonai: ‘Again this time I will let myself be inquired of\lebnote{“I will be/let me be searched”} by the house of Israel, to do something for them; I will cause them to increase their population like a flock.\lebnote{“them as the flock humankind”}}%
\verse{Like the flock of the sanctuary, like the flock of Jerusalem at its festival, so the desolate cities will be filled with flocks\lebnote{Hebrew “flock”} of people; and they will know that I am Adonai.’”}%
\end{biblechapter}%
\begin{biblechapter}% Ezekiel 37
\verseWithHeading{A Valley of Dried Bones; A Renewed Nation of Israel}{The hand of Adonai was upon me, and he brought me by the Spirit of Adonai, and he let me rest\lebnote{Or “placed me”; literally “set me”} in the midst of the valley, and it was full of bones.}%
\verse{And he led me all around\lebnote{“all around, all around”} over them, and look, very many on the surface of the valley, and look, very dry.}%
\verse{And he said to me, “Son of man,\lnCNA{} can these bones live?” And I said, “Lord Adonai, you know.”}%
\verse{And he said to me, “Prophesy to these bones, and you must say to them, to the dry bones, ‘Hear the word of Adonai!}%
\verse{Thus says the Lord Adonai to these bones: “Look! I am bringing into you breath, and you will live!}%
\verse{And I will lay on you sinews, and I will let flesh come upon you, and I will cover you over with skin, and I will put breath into you, and you will live, and you will know that I am Adonai.”’”}%
\verse{And I prophesied just as\lnCNB{} I had been commanded, and there was a sound at my prophesying, and look! A rattling, and they came together — the bones! Bone to its bone!}%
\verse{And I looked, and indeed, sinews were on them, and flesh went up,\lebnote{Or “appeared”} and skin covered over\lebnote{Or “on”} them upward,\lebnote{“from to above”} but breath was not in them.}%
\verse{And he said to me, “Prophesy to the breath; prophesy, son of man,\lnCNA{} and you must say to the breath, ‘Thus says the Lord Adonai from the four winds,\lebnote{“from four winds”} “Come, O spirit and breath, on these dead ones, so that they may live!”’”}%
\verse{And I prophesied as\lebnote{“like what”} he commanded me, and the breath came into them, and they became alive, and they stood on their feet, a very, very large group.}%
\verse{And he said to me, “Son of man,\lnCNA{} these bones are all of the house of Israel; look! they are saying, ‘Our bones are dried up, and our hope is destroyed; we are cut off as far as we are concerned.’\lebnote{“we are cut in two for us”}}%
\verse{Therefore prophesy, and you must say to them, ‘Thus says the Lord Adonai: “Look! I am opening your graves, and I will bring you up from your graves, my people, and I will bring you to the land of Israel!}%
\verse{And you will know that I am Adonai when I open your graves when I bring you up from your graves, my people!}%
\verse{And I will put my breath into you so that you may live, and I will cause you to rest on your soil, and you will know that I, Adonai, I have spoken, and I will act!”’ declares\lebnote{“declaration of”} Adonai.”}%
\verse{And the word of Adonai came\lebnote{“was”} to me, saying,\lnCNC{}}%
\verse{“And you, son of man,\lnCNA{} take for yourself a piece of wood,\lebnote{“\textit{a piece of} wood one”} and write on it, ‘For Judah and for the Israelites\lnCND{} his associates,’ and take another piece of wood, and write on it, ‘For Joseph, the piece of wood of\lebnote{Or “for”} Ephraim and all of the house of Israel his associates.’}%
\verse{And join them one to the other with respect to you as one piece of wood, so that they may become one in your hand.}%
\verse{When\lnCNB{} your people\lebnote{“the children of your people”} say to you, saying,\lnCNC{} ‘Will you not inform us as to what these actions mean for you?’}%
\verse{Then speak to them, ‘Thus says the Lord Adonai: “Look! I am taking the piece of wood for Joseph that is in the hand of Ephraim and the tribes of Israel its associates, and I will put them on it, the piece of wood of Judah, and I will make them into one piece of wood, so that they be one in my hand.’”}%
\verse{And the pieces of wood on which you wrote will be in your hand before their eyes.}%
\verse{And speak to them, ‘Thus says the Lord Adonai: “Look! I am taking the Israelites\lnCND{} from among the nations to which they went,\lebnote{“which they went there”} and I will gather them from everywhere,\lebnote{“all around”} and I will bring them to their own soil.}%
\verse{And I will make them into one nation in the land on the mountains of Israel, and they will all have one king as their king,\lebnote{“and king one he will be for all of them to king”} and they\lebnote{According to the reading tradition (\textit{Qere})} will not again be two nations and will not again divide into two kingdoms again.}%
\verse{And they will not defile themselves again with their idols and with vile idols and with all of their transgressions, and I will save them from all of their apostasy by which they sinned,\lebnote{Or “which they sinned by them”} and I will cleanse them and they will be for me as a people and I, I will be for them as God.}%
\verse{“‘“And my servant David will be king over them, and one shepherd will be for all of them, and in my regulations they will go, and my statutes they will observe, and they will do them.}%
\verse{And they will dwell on the land that I gave to my servant, to Jacob, in which your ancestors dwelled,\lebnote{“which they dwelt in it your fathers”} and they will dwell on it, they and their children and the children of their children forever,\lebnote{“until eternity” or “until unlimited time”} and my servant David will be a leader for them forever.\lebnote{“to eternity” or “to endless ages”}}%
\verse{And I will make with them a covenant of peace; an everlasting covenant\lebnote{“a covenant of eternity”} it will be with them. And I will establish them, and I will cause them to increase, and I will put my sanctuary in the midst of them forever.\lnCNE{}}%
\verse{And my tabernacle will be with them, and I will be for them as God, and they will be to me as a nation.}%
\verse{And the nations will know that I, Adonai, am consecrating Israel when my sanctuary is in the midst of them forever.”’”\lnCNE{}}%
\end{biblechapter}%
\begin{biblechapter}% Ezekiel 38
\verseWithHeading{God’s Message to Gog}{And the word of Adonai came\lebnote{“was”} to me, saying,\lebnote{“to say”}}%
\verse{“Son of man,\lnCNF{} set your face toward Gog of the land of Magog, the head\lnCNG{} leader of Meshech and Tubal, and prophesy against him.}%
\verse{And you must say, ‘Thus says the Lord Adonai: “Look! I am against you, Gog, the head\lnCNG{} leader of Meshach and Tubal,}%
\verse{and I will turn you around, and I will place\lebnote{Or “give”} hooks in your cheeks, and I will bring you out and all of your horses and horsemen fully armed,\lebnote{“clothed of fullness”} all of them, a great crowd, holding a shield,\lebnote{Or “buckler”} and small shield, and holding\lebnote{Or “wielding”} swords, all of them.}%
\verse{Persia, Cush, and Put are with them, all of them, with a small shield and helmet.}%
\verse{Gomer and all of its troops, Beth Togarmah, the remote areas of the north, and with all of its troops and many peoples with them.}%
\verse{Be ready, and prepare yourselves,\lebnote{“for you”} you and all of your assembly,\lebnote{Or “battalions” or “hoards”} the assembling\lebnote{Or “assemblies”} around you, and you must be\lebnote{Or “serve”} for them as a guard.}%
\verse{After many days you will be mustered; in the last years\lebnote{Or “final years”; literally “in the last of the years”} you will come to a restored land from the sword, gathered from many peoples on the mountains of Israel which were as permanent ruins\lebnote{“ruins continually/permanently”} but from peoples it was brought out, and they will dwell in safety,\lebnote{“with confidence/assurance/trust”} all of them.}%
\verse{And you will advance like a storm; you will come, and you will be like a cloud covering\lnCNH{} the land, and all of your troops and many nations along with you.”}%
\verse{“‘Thus says the Lord Adonai: “And then\lebnote{“and it will be”} on that day, things\lebnote{Or “thoughts”} will come up on\lebnote{Or “in”} your mind, and you will devise evil plans.}%
\verse{And you will say, ‘I will go up against a land of open country; I will come\lebnote{Or “go”} to the people being at rest in safety,\lebnote{“with confidence”} all of them dwelling without a wall\lebnote{Or “walls”} and crossbars\lebnote{Hebrew “crossbar”} and without doors,\lebnote{“and doors there are not for them”}}%
\verse{to\lnCNI{} loot loot,\lebnote{Or “war-booty”; this word is chosen because of the alliteration in Hebrew} and to plunder\lnCNJ{} plunder, to assail\lebnote{“to bring back your hand against”; NJPS, “to turn your hand against”} inhabited ruins and a people gathered together from various peoples and who are acquiring livestock and goods and dwelling at the center of the world.}%
\verse{Sheba and Dedan and traders of Tarshish and all its strong lions,\lebnote{Or “all of its violent leaders”} they will ask you, ‘To\lnCNI{} loot loot\lebnote{Or “war-booty” this word is chosen because of the alliteration in Hebrew} are you coming? To plunder\lnCNJ{} plunder have you summoned your assembly, to take away silver and gold, to take livestock and goods, to loot great amounts\lebnote{Hebrew “amount”} of loot?”’}%
\verse{“Therefore prophesy, son of man,\lnCNF{} and you must say to Gog, ‘Thus says the Lord Adonai: “Will you not realize\lebnote{LXX reads, “stir up/rouse yourself”} on that day when my people Israel are dwelling in safety,\lebnote{“with confidence/trust”}}%
\verse{and so you will come from your place, from the remote areas of the north, you and many people with you, horsemen\lebnote{“riders of horsemen”} all of them, a great crowd and a vast army,}%
\verse{and you will advance against my people Israel like a cloud covering\lnCNH{} the land; it will be\lebnote{Or “occur”} in the last days, and I will bring you against my land, so that the nations know me, when I show myself holy through you before their eyes, O Gog!”’}%
\verse{“Thus says the Lord Adonai, ‘Are you he of whom I spoke in former days by the hand of my servants the prophets of Israel who were prophesying in those days for years that I would bring you against them?}%
\verse{And so then in that day, on the day of the coming of God against the land of Israel,’ declares\lnCNK{} the Lord Adonai, ‘my rage will come up in my anger.}%
\verse{And in my passion, in the fire of my wrath, I spoke\lebnote{Or “I asserted”} that certainly\lebnote{“if not”} on that day a great earthquake will be on the land of Israel.}%
\verse{And the fish of the sea and the birds\lebnote{Hebrew “bird”} of the heaven\lebnote{Or “sky”} and the animals\lebnote{Hebrew “animal”} of the field and all of the creeping things that creep on the earth and all of the humans\lebnote{Or “every human being”} who are on the surface of the earth will shake at my presence;\lebnote{“from the face of me”} and the mountains will be demolished, and the steep mountain sides will fall, and every wall on\lebnote{Or “to”} the earth will fall.}%
\verse{And I will call against him\lebnote{Or “summon against him” (cf. NJPS)} in all of my mountains a sword,’ declares\lnCNK{} the Lord Adonai, ‘And the sword of each person will be against his brother.}%
\verse{And I will execute justice with\lebnote{Or “with respect to him”} him with a plague and with blood and torrents of rain,\lebnote{“rain flooding/pouring down”} and hailstones; fire and sulfur I will cause to fall on him and on his troops and on many peoples who are with him.}%
\verse{And so I will exalt myself, and I will show myself holy, and I will make myself known before the eyes of many nations, and they will know that I am Adonai.’”}%
\end{biblechapter}%
\begin{biblechapter}% Ezekiel 39
\verseWithHeading{God’s Destruction of Gog and His Hordes}{“And you, son of man,\lnCNL{} prophesy against Gog, and you must say, ‘Thus says the Lord Adonai: “Look! I am against you Gog, the head\lebnote{Or “chief”} leader of Meshech and Tubal,}%
\verse{and I will turn you around, and I will drag you along, and I will bring you up from the remote areas of the north, and I will bring you against the mountains of Israel,}%
\verse{and I will strike your bow from your left hand,\lebnote{“the hand of left your”} and your arrows from your right hand\lebnote{“the hand of right your”} I will cause to fall.}%
\verse{On the mountains of Israel you will fall, you and all of your troops, and the peoples who are with you, to birds\lebnote{Hebrew “bird”} of prey, birds of every wing,\lebnote{Or “kind”} and animals of the field I will give you as food.}%
\verse{On the surface of the field you will fall, for I myself\lebnote{“I, I”} have spoken,” declares\lnCNM{} the Lord Adonai.}%
\verse{“And I will send fire against Magog and among the people inhabiting the coastlands in safety,\lebnote{“with confidence/trust”} and they will know that I am Adonai.}%
\verse{And my holy name\lebnote{“the name of my holiness”} I will make known in the midst of my people Israel, and I will not let the name of my holiness be profaned anymore, and the nations will know that I am Adonai, the holy one in Israel.}%
\verse{Look! it is coming, and it will happen,” declares\lnCNM{} the Lord Adonai. “It is the day about which I have spoken.\lebnote{“which I have spoken \textit{of}”}}%
\verse{“ʻ“And the inhabitants of the towns\lebnote{Or “cities”} of Israel will go out, and they will set ablaze, and they will kindle a fire with a weapon\lebnote{Or “weapons”} and a small shield and a shield,\lebnote{Or “shields and bucklers”; cf. NJPS} with a bow and with arrows, and with hand clubs\lebnote{“with club of hand”} and with spears,\lebnote{Hebrew “spear”} and so they will light a fire with them, a fire lasting seven years.\lebnote{“a fire of seven years”}}%
\verse{And they will not carry trees from the field, and they will not chop wood from the forests, for with the weapons\lebnote{Hebrew “weapon”} they will light a fire; and they will plunder those plundering them, and they will plunder those plundering them,”’ declares\lnCNM{} the Lord Adonai.}%
\verse{“‘“And then\lebnote{“and it will happen”} on that day I will give to Gog a grave there\lebnote{“a place there of a grave”} in Israel, The Valley of the Travelers, east of the sea,\lebnote{That is, the Dead Sea} and it will block\lebnote{“blocking”} the travelers and Gog and all of his hordes\lebnote{Hebrew “horde”} they will bury there, and they will call it the Valley of Hamon-Gog.\lebnote{“Gog’s Multitude”}}%
\verse{And the house of Israel will bury them for seven months to cleanse the land.}%
\verse{All of the people of the land will bury them, and it will be an honor for them on the day when I appear,”’ declares\lnCNM{} the Lord Adonai.}%
\verse{ʻ“And they will set apart men to continually\lebnote{“men of continuity/men who regularly”} go through the land burying the travelers,\lebnote{In a negative sense = “invaders”} the ones left over on the surface of the land to cleanse it; for the whole of\lebnote{“from \textit{the} end of”} seven months they will explore it.}%
\verse{And the ones going through, they will go through within the land, and if anyone sees the bones of a person, then he will build beside it a sign to remain until the buriers have buried him in the Valley of Hamon-Gog.\lebnote{Or “Gog’s multitude”}}%
\verse{And furthermore the name of the city there is Hamonah,\lebnote{That is, “Multitude”} and thus they will cleanse the land.”’}%
\verse{“And you, son of man,\lnCNL{} thus says the Lord Adonai: ‘Say to all kinds of birds and animals of the field,\lebnote{“to bird of every wind and to every animal of the field”} “Be gathered, come be gathered from everywhere\lebnote{“all around”} to my sacrifice that I am sacrificing for you, a great sacrifice on the mountains of Israel, and you will eat flesh, and you will drink blood.}%
\verse{Flesh of warriors you will eat, and the blood of leaders of the land you will drink; rams, young rams and goats, bulls, and the fattened animals of Bashan — all of them!}%
\verse{And you will eat fat until you are satiated,\lebnote{“satiation/gluttony”} and you will drink blood until you are drunk\lebnote{“to drunkenness”} from my sacrifice that I sacrifice for you,}%
\verse{and you will be satisfied at my table with horse and horsemen and warriors and every man of war,”’ declares\lnCNM{} the Lord Adonai.}%
\verse{“And I will display my glory among the nations, and all of the nations will see my judgment that I have executed and my hand that I have laid upon them.}%
\verse{“And the house of Israel will know that I am Adonai their God from that day and beyond,}%
\verse{and the nations will know that because of their guilt the house of Israel went into exile, because\lebnote{“because that”} they acted unfaithfully against me, and I hid my face from them, and I gave them into the hand of their foes, and they fell by the sword, all of them.}%
\verse{According to their uncleanness and according to their transgression I dealt with them, and I hid my face from them.}%
\verse{Therefore thus says the Lord Adonai: “Now I will restore the fortunes of Jacob, and I will have compassion on all of the house of Israel, and I will be jealous\lebnote{Or “zealous”} for my holy name.\lebnote{“for the name of holiness my”}}%
\verse{And they will forget their disgrace and all of their infidelity that they displayed against me when they dwelt on their soil in safety\lebnote{“with confidence”} with nobody making them afraid.\lebnote{“and not \textit{was somebody} making afraid”}}%
\verse{When I restore them from the nations and I gather them from the countries of their enemies, then I will show myself holy through them before the eyes of many nations.}%
\verse{And they will know that I am Adonai their God, because of when I deported them into the nations, and I reassembled them to their soil, and I will not let any of them remain there any longer.\lebnote{“not I will leave over/let remain again/still from them there”}}%
\verse{And I will not hide my face again from them when I pour out my Spirit over\lebnote{Or “upon”} the house of Israel,” declares\lnCNM{} the Lord Adonai.}%
\end{biblechapter}%
\begin{biblechapter}% Ezekiel 40
\verseWithHeading{A Vision of the New Temple}{In the twenty-fifth year of our exile, at the beginning of the year, on the tenth day of the month, in the fourteenth year after the city was destroyed, in this day exactly, the hand of Adonai was on me, and he brought me there}%
\verse{in visions from God. He brought me to the land of Israel and put me on a very high mountain, and on it was something like a structure of a city to the south.}%
\verse{And he brought me there, and look, there was a man whose\lebnote{Hebrew “his”} appearance was like the appearance of bronze, and a cord of linen was in his hand and a reed for measurement; he was standing in the gate.}%
\verse{And the man spoke to me, “Son of man,\lebnote{Or “mortal,” or “son of humankind”} look with your eyes and listen with your ears and apply\lebnote{Or “set/put”} your heart to all that I am showing you, for you were brought here in order to show you this; tell all that you are seeing to the house of Israel.”}%
\verseWithHeading{The Eastern Gate}{And there was a wall on the outside of the temple all the way around it,\lnCNN{} and in the hand of the man the reed\lebnote{Or “rod”} for measurement was six long cubits, according to\lebnote{“with/in”} the cubit\lnCNO{} and a handbreadth,\lebnote{A handbreadth wasabout 3 inches; the reed/rod was 10.5 feet long} and he measured the width of the outer wall as one reed,\lnCNP{} and the height as one reed.\lnCNQ{}}%
\verse{And then he went toward\lebnote{Or “to”} the gate whose face was to the east.\lebnote{“to the way toward the east”} And he went up by\lebnote{Or “with”} its steps, and he measured the threshold of the gate, one reed\lnCNQ{} wide.}%
\verse{And the alcove\lebnote{Or “niches” in the temple tower} was one reed\lnCNQ{} long and one reed\lnCNQ{} wide,\lebnote{That is, 10.5 x 10.5 feet} and between the alcoves was five cubits,\lebnote{That is, about 8.75 feet} and the threshold of the gate along the side of the portico of the gate on the inside\lnCNR{} was one reed.\lnCNQ{}}%
\verse{And then he measured the portico of the gate on the inside\lnCNR{} as one reed.\lebnote{That is, about 10.5 feet.}}%
\verse{And he measured the portico of the gate as eight cubits,\lebnote{That is, about 14 feet} and its pilaster was two cubits\lebnote{That is, 3.5 feet} thick, and he measured the portico of the gate on the inside.\lnCNR{}}%
\verse{And the alcoves\lebnote{Hebrew “alcove”} of the gate toward the way\lebnote{Or “direction”} eastward\lebnote{“of the east”} were three on each side;\lebnote{“\textit{was} three from here and three from here”} the same measurements applied to all three of them;\lebnote{“measurement one \textit{was} to the three of them”} and the same measurement applied to the pilaster\lebnote{“measurement one \textit{was} to the pilaster”} on each side.\lebnote{“from here and from there”}}%
\verse{And he measured the width of the doorway of the gate as ten cubits\lebnote{That is, 17.5 feet} and the length of the gateway\lebnote{Hebrew “gate”} was thirteen cubits.\lebnote{That is, 22.75 feet; or the inside width of the gateway was 22.75 feet}}%
\verse{And a wall\lebnote{Or “guard wall”; others prefer “curb” (NLT)} was before\lebnote{“\textit{was} to the face of”} the alcoves; one cubit\lnCNO{} on either side\lebnote{“cubit one and cubit one”; or 21 inches} was the wall from here.\lebnote{Or “from the front”} And the alcove was six cubits\lnCNP{} on each side.\lebnote{“\textit{was} six cubits from here and six cubits from here”}}%
\verse{And he measured the gate from the top slab of the alcove to its opposite top slab as twenty-five cubits in width, from one entrance to the other one opposite it.\lebnote{“doorway opposite doorway”}}%
\verse{And he made the pilasters sixty cubits,\lebnote{That is, 105 feet (problematic in this context)} and to the pilaster of the courtyard all the way around the gate.\lebnote{“the gate all around, all around”}}%
\verse{And from the front of\lebnote{“and at the face of”} the gate at the entrance to the front of\lebnote{“the face of”} the portico of the inner gate was fifty cubits.\lnCNS{}}%
\verse{And there were narrow windows for the alcoves and for their pilasters to the inside of the gate all the way around\lnCNN{} it. And likewise with respect to the porticos and windows were all the way around\lnCNN{} to the inside, and on a pilaster there were images of palm trees.\lebnote{Hebrew “tree”}}%
\verseWithHeading{The Outer Courtyard}{And he brought me to the outer courtyard, and there were chambers and a pavement made for the courtyard all around it, and thirty chambers were facing to the pavement.}%
\verse{And the pavement flanked\lebnote{“\textit{was} to the side of”} the side of the gates and all along the gates;\lebnote{“to along side the length of the gates”} this describes the lower pavement.}%
\verse{And he measured the width from the front of\lnCNT{} the lower gate to the front of\lnCNT{} the outside of the inner courtyard as a hundred cubits\lnCNU{} to the east and to the north.}%
\verseWithHeading{Measurement of the North Gate}{And as for the gate that had its face toward the north\lebnote{“to the way of the north”} of the outer courtyard, he measured its length and its width.}%
\verse{And its alcoves were three on each side,\lebnote{“three from here and three from here”} and its pilasters and its porticos had the same measurement\lebnote{“it was like the measurement of”} as the first gate: its length was fifty cubits,\lnCNS{} and its width was twenty-five cubits.\lebnote{“in/by the cubit”; 43.5 feet}}%
\verse{And its windows and its porticos and its palm tree images were like the measurement of the gate that was facing toward the east;\lebnote{“its face \textit{to} the way of the east”} and it had\lebnote{“and with steps”} seven steps that go up it, and there were porticos before them.\lnCNV{}}%
\verse{And a gate led to the inner courtyard, opposite the gate to the north and to the east, and he measured from gate to gate as a hundred cubits.\lnCNU{}}%
\verseWithHeading{The Gate to the South}{And then he took me toward the south,\lnCNW{} and look, a gate toward the south,\lnCNW{} and he measured its pilasters and its porticos; they had measurements just like the others.\lnCNX{}}%
\verse{And there were for it windows and for its porticos all the way around it\lnCNN{} like these windows; fifty cubits\lnCNS{} long and twenty-five cubits\lnCNY{} wide.}%
\verse{And seven steps were going up to it and its porticos before them.\lnCNV{} And it had palm tree images\lebnote{“and palm tree images \textit{were} for it”} all along\lebnote{“from here and one from here to”} its pilasters.}%
\verse{And there was to the way of the south\lebnote{That is, “to the south”} a gate for the inner courtyard, and he measured from gate to the gate on the way of the south, a hundred cubits.\lnCNU{}}%
\verseWithHeading{Gates to the Inner Courtyard}{And he brought me to the inner courtyard through\lebnote{Or “at”} the gate of the south, and he measured the south gate, and it had measurements like the others.\lnCNZ{}}%
\verse{And its alcoves and its pilasters and its porticos were just like these measurements. And it had windows for it and for its porticos\lebnote{Hebrew “portico”} all the way around,\lnCNN{} fifty cubits\lebnote{Hebrew “cubit”} along\lebnote{“long”} its length, and twenty-five cubits\lnCNY{} wide.}%
\verse{And there were porticos all the way around\lnCNN{} its length, twenty-five cubits;\lnCNY{} and its width was five cubits.\lebnote{That is, 8.75 feet, which differs from 40:9; the LXX and some Greek versions lack this verse altogether}}%
\verse{And its porticos were to the outer courtyard, and palm tree images were on its pilasters, and eight steps were for its stairs.}%
\verse{And he brought me to the inner courtyard by way of the east gate, and he measured the gate as just like the other measurements.\lnCNZ{}}%
\verse{And its alcoves and its pilasters and its porticos were also like these measurements, and windows were for it and for its porticos all around;\lnCNN{} its length was fifty cubits,\lnCNS{} and it was twenty-five cubits\lnCNY{} wide.}%
\verse{And its porticos were toward the outer courtyard, and palm tree images were on its pilasters one each side,\lnCOA{} and eight steps served as it stairs.\lebnote{“and eight steps its stairs”}}%
\verse{And he brought me to the gate of the north, and he measured it; and he measured the same measurements as these others,\lnCNX{}}%
\verse{its alcoves, its pilasters and porticos, and its windows\lebnote{“its windows \textit{were} for it”} all the way around,\lnCNN{} and a length of fifty cubits\lnCNS{} and twenty-five cubits wide.\lnCNY{}}%
\verse{And its pilasters faced the outer courtyard,\lebnote{“\textit{were} to the courtyard the outer”} and it had palm tree images on its pilasters\lebnote{“and palm tree images \textit{were} for its pilaster”} on each side;\lnCOA{} and eight steps served as its stairs.}%
\verseWithHeading{Rooms for Sacrificial Preparations}{And a chamber with its doorway was in\lebnote{Or “\textit{set} among”} the pilasters at the gates,\lebnote{Or “gate”} and there they rinsed off\lebnote{Or “washed”} the burnt offering.}%
\verse{And in the portico of the gate were two tables on each side\lebnote{“\textit{were} two tables from here and two tables from here”} to slaughter the burnt offering on them and the sin offering and the guilt offering.}%
\verse{And on the outer side as one goes up to the doorway of the gate to the north were two tables, and on the other side, which is toward the portico of the gate, were two tables.}%
\verse{On each side of the gate were four tables,\lebnote{“four tables from here and four tables from here”} eight tables in all, and on them they slaughtered sacrifices.}%
\verse{And four tables were for the burnt offering, and made of dressed stones, one cubit and a half\lebnote{That is, 31.5 inches} long and one cubit and a half wide and one cubit\lebnote{That is, 21 inches} high was their measurements,\lebnote{“\textit{was}to them} and they placed the objects\lebnote{Or “instruments/utensils”} with which they slaughtered the burnt offering on them and the sacrifice.}%
\verse{And there were double-pronged hooks, one handbreadth in width, put in place in the house\lebnote{Or “inside”} all around, and on the tables was the flesh of the offering.}%
\verseWithHeading{Rooms for Priests and Other Worship Leaders}{And on the outside of the inner gate, there were chambers for\lebnote{Hebrew “of”} singers in the inner courtyard, which was to the side of the north gate, and their faces\lebnote{That is, of the chambers} were to the south\lebnote{“\textit{were to} the way of the south”} with respect to one, and on the side of the east gate one facing to the north.}%
\verse{And he said to me, “This chamber with its face toward the south\lebnote{“which its face \textit{is to} the way of the south”} is for the priests who are taking care of the responsibility of the temple.}%
\verse{And the chamber with its face to the north\lebnote{“with its face \textit{is to} the way of the north”} is for the priests who are taking care of the responsibility of the altar. They are the descendants\lnCOB{} of Zadok, the ones who approach from among the descendants\lnCOB{} of Levi to Adonai to serve him.”}%
\verseWithHeading{Inner Courtyard and Temple Measurements}{And he measured the courtyard as to its length, a hundred cubits,\lnCNU{} and a hundred cubits\lnCNU{} wide, squared, and the altar is in front of\lebnote{“before face of”} the temple.}%
\verse{And he brought me to the portico of the temple, and he measured the pilaster of the portico, five cubits on each side,\lebnote{That is, 8.75 feet; literally “five cubits from here and five cubits from here”} and the width of the gate was three cubits on each side.\lebnote{“\textit{was} three cubit\textit{s} from here and three cubits from here”}}%
\verse{The length of the portico was twenty cubits\lebnote{That is, 35 feet} and its width eleven cubits,\lebnote{That is, 19.25 feet; LXX reads 12 cubits or 21 feet} and with ten steps they went up to it, and the pilasters had pillars,\lebnote{“and pillars \textit{were} to the pilasters”} one on each side.\lebnote{“one from here and one from here”}}%
\end{biblechapter}%
\begin{biblechapter}% Ezekiel 41
\verseWithHeading{Sanctuary and Temple}{And he brought me to the temple sanctuary,\lebnote{Or “main room” (of the temple of Jerusalem); “great hall” (JPS); “outer sanctuary” (NIV); word can refer to surrounding courts and halls; HALOT 245.} and he measured the pilasters, six cubits\lnCOC{} wide on each side;\lebnote{“six cubits wide from here and six cubits wide from here”} this was the width of the tent.\lebnote{Or “of the jambs”}}%
\verse{And the width of the doorway was ten cubits,\lebnote{That is, 17.5 feet} and the sidewall of the doorway was five cubits on each side,\lebnote{“\textit{was} five cubits from here and five cubits from here”; five cubits = 8.75 feet} and he measured its\lebnote{That is, the outer sanctuary} length as forty cubits,\lebnote{That is, 70 feet} and its width was twenty cubits.\lnCOD{}}%
\verse{And he went into the inner room, and he measured the pilaster of the doorway as two cubits\lnCOE{} and the doorway as six cubits\lnCOC{} and the width of the doorway seven cubits.\lebnote{That is, 12.25 feet}}%
\verse{And he measured its length as twenty cubits\lnCOD{} and its width as twenty cubits to the front of the temple, and he said to me, “This is the most holy place.\lebnote{“the holy of the holies”}}%
\verse{And he measured the wall of the temple as six cubits,\lnCOC{} and the width of the side room as four cubits\lebnote{That is, 7 feet} all along the outside wall for the temple all around the wall.\lebnote{“all around, all around for the temple all around”}}%
\verse{And the side rooms were side by side\lebnote{“\textit{were} side room to side room”} in three stories and a total of thirty rooms,\lebnote{“times”} and there were offsets\lebnote{Meaning of word is uncertain} in the wall, which was for the temple for the side rooms, all the way around\lnCOF{} to be supports, and so they were not supports extending into the wall of the temple.}%
\verse{And each level widened, and it went around\lebnote{Or “was surrounded”} upward\lnCOG{} to the side rooms for the structure that surrounds\lebnote{Or “goes around”} the temple upward\lnCOG{} all the way around,\lnCOF{} therefore the width increased to the temple upward,\lebnote{“to above”} and thus the lower level goes up to the upper level by means of the middle story.}%
\verse{And I saw for the temple a platform all the way around\lnCOF{} the foundations of the side rooms; it was the length of a full reed, six cubits long.\lnCOC{}}%
\verse{The width of the outside wall, which was for the side room to the outside, was five cubits,\lnCOH{} and a space that was set between the side rooms, which was for the temple}%
\verse{and between the chambers was the width of twenty cubits\lnCOD{} all around the temple, all the way around\lnCOF{} it.}%
\verse{And the doorway of the side room faced\lnCOI{} the open area; one doorway faced\lnCOI{} the north,\lebnote{“the way of the north”} and one doorway was to the south, and the width of the place of the open area was five cubits\lnCOH{} all around.\lnCOF{}}%
\verse{And the building that faced\lebnote{“\textit{was} to the front of”} the courtyard was toward the west,\lebnote{“\textit{was}\textit{to} the side of the way of the sea/west”} and its width was seventy cubits,\lebnote{That is, 122.5 feet} and the wall of the building was five cubits\lebnote{That is, 8.25 feet} wide all the way around,\lnCOF{} and its length was ninety cubits.\lebnote{That is, 157.5 feet}}%
\verse{And he measured the temple, and its length was a hundred cubits,\lnCOJ{} and the courtyard and the building and its walls, their length was a hundred cubits.\lnCOJ{}}%
\verse{And the width of the front of the temple and the courtyard to the east was a hundred cubits.\lnCOJ{}}%
\verse{And he measured the length of the building facing the courtyard\lebnote{“to the front of the courtyard”} at the rear\lebnote{“which \textit{was} on behind her/it”} and its galleries, a hundred cubits\lnCOJ{} on each side,\lebnote{“from here and from here”} and also the temple, the court, and the porticos of the courtyard,}%
\verse{the thresholds and the framed windows and the galleries around the three of them. Before the threshold was a covering of wood all around,\lnCOF{} and from the ground up to the windows, and all around the windows, they were covered.}%
\verse{Above\lebnote{“On from on”} the doorway and up to the inner temple\lebnote{Or “sanctuary”} and on the outside, and on all of the wall all the way around\lnCOF{} in the inner and in the outer areas were patterns,\lebnote{“measurements”}}%
\verse{and it was made of cherubim and palm tree images; a palm tree image between cherub and cherub, and the cherub had two faces.\lebnote{“two faces \textit{were} for the cherub”}}%
\verse{And the face of a human was toward\lnCOI{} the palm tree image on the one side,\lnCOK{} and the face of a fierce strong lion faced\lebnote{“\textit{was} to”} the palm tree image on the other side;\lnCOK{} this work was executed\lebnote{“being made”} for the entire temple all the way around.\lnCOF{}}%
\verse{From the ground up to above\lebnote{“up to form on”} the doorway,\lebnote{Or “entrance”} the cherubim and the palm tree images were made,\lebnote{Or “being made”} and also the outer wall of the temple.}%
\verse{As far as the temple is concerned its doorframe was squared, and before\lebnote{“in the front of”} the sanctuary was the appearance as it were the appearance of}%
\verse{a wooden altar\lebnote{“the altar \textit{made of} wood”} that was three cubits\lebnote{That is, 5.25 feet} high, and its length was two cubits,\lnCOE{} and its corners\lebnote{Hebrew “corner”} for it and its length and its walls were of wood. And he spoke to me, “This is the table that is before\lebnote{“to the face of”} Adonai.”}%
\verse{And the two doors were for the temple and for the sanctuary.}%
\verse{And two leaves\lebnote{That is, two swinging leaves for each door} of a door were for each of the doors, two hinged leaves of a door: two were for the first door, and two leaves of a door were for the other door.}%
\verse{And cherubim were made\lebnote{Or “carved”} on them, that is, on the doors of the temple and palm tree images like the ones prepared for the walls;\lebnote{“like which \textit{they} were made for the walls”} and an overhang of wood was on the surface of the porticos on the outside.}%
\verse{And narrow windows and palm tree images were on either side,\lebnote{“\textit{were} from here and from here”} and on the side walls of the portico, and the side rooms of the temple and their\lebnote{“the”} overhang.}%
\end{biblechapter}%
\begin{biblechapter}% Ezekiel 42
\verseWithHeading{Description of Rooms for the Priests}{And he brought me out to the outer courtyard to the north,\lebnote{“\textit{to} the way of the north”} and he brought me to the chamber which was opposite the courtyard and which is opposite the building to the north.}%
\verse{As to\lebnote{Or “along the face of”} the face of the length of the building with the doorway to the north, it was a hundred cubits,\lnCOL{} and its\lebnote{Or “the”} width was fifty cubits.\lnCOM{}}%
\verse{Opposite the twenty cubits of the inner courtyard,\lebnote{“which \textit{were} to the courtyard the inner”} and opposite the pavement that was to the outer courtyard was a gallery facing\lebnote{“to\textit{ward} in the face/front of”} a gallery in the three stories.}%
\verse{And in front of\lebnote{“to the face of”} the chambers was a passageway ten cubits\lebnote{That is, 17.5 feet} in width toward the inside, a walkway of one cubit,\lebnote{One cubit of 18–21 inches is problematic; LXX and Syriac read 100 cubits (175 feet) running the length of the building} and their doorways were to the north.}%
\verse{And the upper chambers narrowed, for the galleries took away space from them more than they took from the lower levels and more than they took from the middle level in the building.}%
\verse{For they\lebnote{Or “there”} were three stories and they had no pillars\lebnote{“there were not for them pillars”} like\lebnote{Or “as”} the pillars of\lebnote{Or “for”} the courtyards; therefore they were smaller than the lower stories and than the middle stories from the ground up.}%
\verse{And there was a wall that was to the outside alongside\lebnote{“to the outside to corresponding”} the chambers on the walkway to\lebnote{Or “of”} the outer courtyard\lebnote{That is, “to the north”} in front of\lebnote{“to in face of”} the chambers; its length was fifty\lnCOM{} cubits.\lnCON{}}%
\verse{For\lebnote{Or “while”} the length of the chambers which were to the outer courtyard was fifty\lnCON{} cubits,\lnCOM{} and look!\lebnote{Or “yet!”} The chambers on the front of the temple were a hundred\lnCON{} cubits.\lnCOL{}}%
\verse{And from under these chambers was the entrance\lebnote{Reading the Ketiv} from the east for them when one enters\lebnote{“at/in his/its coming”} from the outer courtyard.}%
\verse{All along the width of the wall of the courtyard eastward\lebnote{“on the way of the east”} in front of the courtyard to the front of the building were chambers.}%
\verse{And a walkway was before them\lebnote{“was to the face of them”} like the appearance of the chambers which were on the north,\lebnote{“\textit{were on} the way of the north”} just like them in their length, and so was their width and all their exits, and they were built like their arrangements and like their doorways,}%
\verse{and like the doorways of the chambers which were on the way of the south was a doorway at the head of the way\lebnote{Dropping one “way”} before the stone wall, projecting\lebnote{Or “protecting wall”} on\lebnote{Or “toward”} the way of the east at their coming.\lebnote{That is, when people entered the complex}}%
\verse{And he said to me, “The chambers of the north and the chambers of the south which are before\lebnote{“to the front of”} the courtyard, they are the holy chambers\lebnote{“\textit{are} the chambers of the holiness”} in which the priests, who are near to Adonai, will eat the most holy objects.\lnCOO{} There they shall put the most holy objects,\lnCOO{} and the grain offering and the sin offering and the guilt offering, for the place is holy.}%
\verse{When the priests enter,\lebnote{“at coming their”} then they shall not go out from the sanctuary to the outer courtyard; and there they must put\lebnote{Or “deposit/leave/remove”} their garments in which they serve because they are holy. They must put on other garments and then they may approach the area that is for the people.}%
\verse{And he completed the measurements of the inner temple, and he brought me to the walkway of the gate that faces toward the east\lebnote{“which its face \textit{is to} the way of the east”} and he measured it all the way around.\lnCOP{}}%
\verse{He measured the east side with the reed\lebnote{Or “rod”} for\lebnote{“of measuring”} measuring, five hundred cubits,\lebnote{Hebrew has “reeds” here (= 5,250 feet), which is not correct; 500 cubits = 875 feet.} with respect to reeds with the reed for measurement, he measured it all around.}%
\verse{He measured the north side as five hundred cubits, with respect to reeds with the reed for measurement all around.}%
\verse{Then he measured the south side as five hundred cubits, with respect to reeds with the reed for measurement.}%
\verse{He went around the west side\lebnote{Or “side toward/to the sea”} and he measured five hundred cubits, with respect to reeds with the reed for measurement.}%
\verse{Toward the four sides\lebnote{“toward the four winds”} he measured it; there was a wall for it all the way around.\lnCOP{} Its length was five hundred cubits\lebnote{That is, 875 feet} and its width was five hundred cubits, in order to make a separation between what is holy and what is common.}%
\end{biblechapter}%
\begin{biblechapter}% Ezekiel 43
\verseWithHeading{Return of the Lord’s Glory and Description of Sacrifices and Worship}{And he brought me to the gate which was facing east.\lebnote{“which \textit{was} facing \textit{to} the way of the east”}}%
\verse{And, look! The glory of the God of Israel, it came from the way of the east, and its sound was like the sound of many waters\lebnote{Hebrew “water”} and the land radiated due to his glory!}%
\verse{And\lebnote{Or “and as”; as is not needed in translation} the appearance of the vision which I saw was as\lnCOQ{} the vision which I saw at his\lebnote{Or “my coming”} coming to destroy the city, and these visions were also as\lnCOQ{} the vision which I saw by the Kebar River,\lebnote{“river of Kebar”} and I fell on my face.}%
\verse{And then the glory of Adonai came to the temple by the way of the gate facing east.\lebnote{“\textit{by} the way of \textit{the} gate which its face \textit{was to} the way of the east”}}%
\verse{And the Spirit lifted me and it brought me to the inner courtyard and, look! The glory of Adonai filled the temple!}%
\verse{And I heard someone speaking to me from the house,\lebnote{Or “from inside”} and a man was standing beside me.}%
\verse{And he said to me, “Son of man,\lnCOR{} this is the place of my throne and the place for the soles of my feet where I will dwell\lebnote{“which I will dwell there”} in the midst of the Israelites\lebnote{“sons/children of Israel”} to eternity, and they, the house of Israel, they and their kings, will not again defile my holy name\lnCOS{} with their fornication and with their offerings for the dead\lnCOT{} of their kings on their high places.}%
\verse{When they placed their threshold\lebnote{“in/at their placing their threshold”} with\lebnote{Or “beside/next to”} my threshold, their doorframe beside my doorframe and the\lebnote{Or “their”} wall was between me and between them, then they defiled my holy name,\lnCOS{} with their detestable things that they did, and so I consumed them in my anger.}%
\verse{Now let them send their fornication far away and the offerings for the dead\lnCOT{} of their kings away from me, and I will dwell in the midst of them forever.\lebnote{“to eternity”}}%
\verse{You, son of man,\lnCOR{} describe to the house of Israel the temple and let them be ashamed of their iniquities and let them measure the pattern.}%
\verse{And if they are ashamed of all that they did, then the plan of the temple and its arrangement and its exits and its entrances and all of its plans\lebnote{Or “its entire pattern”} and all of its statutes. And all of its plans\lebnote{Or “purposes”} and all of its laws\lebnote{According to the reading tradition (\textit{Qere}); cf. BHS} make known to them and write them before their eyes, so that they may remember all of its plans and all of its statutes, so that they do them!}%
\verse{This is the law\lebnote{“torah” or “instruction”} of the temple: On the top of the mountain, all of its territory, all the way around it,\lebnote{“all around, all around”} will be most holy.\lebnote{“\textit{will be} a holiness of holinesses”} Look, this is the law of the temple.}%
\verse{And these are the measurements of the altar in the cubits\lebnote{Or “in long cubits”} (a cubit is a cubit and a handbreadth):\lebnote{That is, about 18 inches (regular cubit) + 3 inches = 21 inches (long cubit)} now its gutter\lebnote{Or “trench” (NJPS)} is a cubit\lnCOU{} in depth by a cubit\lnCOU{} in width\lebnote{“the cubit and cubit \textit{its} width”} and its rim along its edge all around is one span,\lebnote{That is, 9 inches} and this is also the height of the altar.}%
\verse{And from the gutter\lnCOV{} at the ground up to the lower ledge is two cubits,\lebnote{That is, 3 feet 6 inches} and its width is one cubit,\lnCOU{} and from the small ledge up to the large ledge is four cubits,\lnCOW{} and its width is one cubit.}%
\verse{And the altar hearth was four cubits,\lnCOW{} and from the altar hearth and upwards\lebnote{“and to above”} were the four horns of the altar.}%
\verse{And the altar hearth was twelve\lnCOX{} cubits in length and twelve\lnCOX{} cubits in width\lebnote{That is, 21 feet square}; it was squared on its four sides.\lebnote{“squared to the four of its sides”}}%
\verse{And the ledge was fourteen\lnCOY{} cubits in length with fourteen\lnCOY{} cubits its width\lebnote{That is, 24.5 feet on all four sides, a square} to all four of its sides, and a rim was all around it of one-half cubit.\lebnote{That is, 10.5 inches} And the gutter\lnCOV{} for it was a cubit\lnCOU{} all around, and its steps were facing east.”}%
\verse{And he said to me, “Son of man,\lnCOR{} thus says the Lord Adonai: ‘These are the statutes of the altar on the day when it is made,\lebnote{“to be made it”} to sacrifice a burnt offering on it and to sprinkle blood on it.}%
\verse{And you must give to the Levitical priests who are from the offspring of Zadok, the ones coming near me,” declares\lnCOZ{} the Lord Adonai, “to serve me, a bull, a calf\lnCPA{} as a sin offering.}%
\verse{You must take some of its blood and put it on the four horns of the altar,\lebnote{“on its four horns”} and the four corners of the ledge, and on the rim all around; thus you must purify it and make atonement for it.}%
\verse{And you must take the bull, the sin offering, and you must burn it in the designated place of the temple outside of the holy place.}%
\verse{And on the second day you must offer a he-goat without defect as a sin offering, and they must purify the altar like\lebnote{“as that”} they purified with the bull.}%
\verse{When you are finished\lebnote{“at your finishing”} from purifying, you must offer a bull, a calf,\lnCPA{} without defect, and a ram from the flock without defect.}%
\verse{And you must bring them near before\lebnote{“to the face of”} Adonai, and the priests must throw salt on them, and they must offer them as a burnt offering to Adonai.}%
\verse{For seven days you must provide a goat of sin offering for the day and a bull, a calf,\lnCPA{} and a ram from the flock without defect; you must provide them.}%
\verse{Seven days they must purify the altar and they must cleanse it and so they will consecrate it.\lebnote{“they must fill its hand”}}%
\verse{And they will finish the days, and then\lebnote{“and it will happen”} on the eighth day and beyond that the priests will offer on the altar your burnt offerings and your fellowship offerings, and then I will be pleased with you,” declares\lnCOZ{} the Lord Adonai.}%
\end{biblechapter}%
\begin{biblechapter}% Ezekiel 44
\verseWithHeading{The Prince, the Levites, and the Priests Described}{And he brought me back by the way of the outer gate of the sanctuary that is facing east, and it was shut.}%
\verse{And Adonai said to me, “This gate will be shut. It shall not be opened, and no one\lebnote{“and a man not”} will go through it, for Adonai, the God of Israel, has entered it, and it will be shut.}%
\verse{The prince, he may sit in it to eat food before\lnCPB{} Adonai; he will come from the way\lebnote{Or “direction”} of the portico of the gate and by means of its way he will also go out.”}%
\verse{Then he brought me by the way of the gate of the north to the front of the temple,\lnCPC{} and I looked,\lebnote{Or “saw”} and look! The glory of Adonai filled the temple\lnCPC{} of Adonai, and I fell on my face.}%
\verse{And Adonai said, “Son of man,\lebnote{Or “mortal,” or “son of humankind”} set your heart\lebnote{Or “pay close attention” or “listen closely”} and look with your eyes and with your ears hear everything\lebnote{“all of which/what”} I am saying to you concerning all the statutes of the temple\lnCPC{} of Adonai and concerning all its laws,\lebnote{According to the reading tradition (\textit{Qere}); cf. BHS} and you must listen with your heart concerning the entrance of the temple\lnCPC{} with all of the exits of the sanctuary.}%
\verse{And you must say to the rebellious, to the house of Israel, ‘Thus says the Lord Adonai: “Enough for you, house of Israel, of\lebnote{Or “from”} all of your detestable things!}%
\verse{At\lebnote{Or “in”} your bringing foreigners\lebnote{“children/people of a foreign land”} who are uncircumcised of heart and uncircumcised of flesh to be in my sanctuary to profane it, my temple,\lnCPC{} as you offered my food, fat, and blood,\lebnote{“in/at your offering my food and fat and blood”} so\lnCPD{} you broke my covenant by all of your detestable things.}%
\verse{And you did not observe the responsibility of my sanctuary, but\lnCPD{} you appointed them\lebnote{That is, foreigners} as the keepers of my responsibility in my sanctuary for you.”}%
\verse{Thus says the Lord Adonai: “Every foreigner\lnCPE{} uncircumcised of heart and uncircumcised of flesh shall not come into my sanctuary — not any of the foreigners\lnCPE{} who are in the midst of the Israelites.\lebnote{“sons/children of Israel”}}%
\verse{But\lebnote{“For if”} the Levites who removed themselves from me at the going astray of Israel,\lebnote{Or “when Israel went astray”} who went astray from me and went after their idols, as a result they will bear their guilt.}%
\verse{Then they will be\lebnote{Or “may be”} in my sanctuary, those serving in my sanctuary, as sentries\lebnote{Or “guards”} at the gates of the temple\lnCPC{} and as those serving the temple;\lnCPC{} they will slaughter the burnt offering and the sacrifice for the people, and they will stand before them\lebnote{“to the face of them”} to serve them,}%
\verse{because\lebnote{“because that”} they used to serve them before\lnCPB{} their idols and they were for the house of Israel like a stumbling block of iniquity. Therefore I swore\lebnote{“I raised my hand”} concerning them,” declares\lnCPF{} the Lord Adonai, “that they will bear their iniquity.}%
\verse{And they shall not approach me to serve as a priest for me and to come near to all of\lebnote{Or “any of”} my holy objects, to the most holy objects,\lebnote{“to the holy objects of the holy objects”} and they will bear their disgrace and the results of their detestable things they have done.}%
\verse{And I will appoint them as the keepers of the responsibility of\lebnote{Or “toward”} the temple, for all of its work\lebnote{Or “tasks”} and everything\lebnote{“all”}which will be done in it.}%
\verse{But the Levitical priests, the descendants\lebnote{Or “sons”} of Zadok, who cared for the responsibility of my sanctuary when the Israelites went astray\lebnote{“at going astray the children of Israel”} from me, they will approach me to serve me, and they will stand before me\lebnote{“to the face of me”} to offer to me fat and blood,” declares\lnCPF{} the Lord Adonai.}%
\verse{“They shall come to my sanctuary, and they shall approach my table to serve me, and they will observe my requirement.}%
\verse{And then\lebnote{“and it will happen”} when they come\lebnote{“at their coming”} to the gates of the inner courtyards, they shall put on their\lebnote{Hebrew “the”} inner linen garments and not wear wool garments when they serve\lebnote{“at their serving”} in the gates of the inner courtyard and inside\lebnote{“and toward”} the temple.}%
\verse{Linen turbans shall be on their head and linen undergarments shall be on their waists; they shall not gird themselves with material causing perspiration.}%
\verse{When they go out to the outer courtyard to the people, they must take off their garment in which they were serving, and they must place them in the holy chambers,\lebnote{“in the chambers of the holiness”} and they must put on other garments, so that they will not make the people holy with their garments.}%
\verse{And their head they shall not shave, or long hair they shall not let grow; short they shall clip their heads.}%
\verse{And no priest shall drink wine when they come into the inner courtyard.}%
\verse{And a widow or divorced woman they shall not take for themselves as wives, but only\lnCPG{} a virgin from the offspring of the house of Israel, or the widow who is a widow for a priest they may take.}%
\verse{And they will teach my people the difference between what is holy and what is unholy, and the difference between unclean and clean they must show them.}%
\verse{And at a legal dispute they themselves shall stand to judge; employing\lebnote{Or “with”} my judgments they shall judge it, and with respect to my laws and my statutes. All my festivals they shall keep and my Sabbaths they shall consecrate.}%
\verse{And near a dead person\lebnote{“and to a dead person of a man”} he\lnCPH{} shall not come to be defiled, but only\lnCPG{} for a father and for a mother and for a son and for a daughter, for a brother and for a sister who was not married\lebnote{“who not she was for a man”} may they defile themselves.}%
\verse{And, after his\lnCPH{} cleansing, they shall count for him seven days,}%
\verse{and then on the day of his coming into the sanctuary to the inner courtyard to serve in the sanctuary, he shall offer his sin offering,” declares\lnCPF{} the Lord Adonai.}%
\verse{And thus it will be to them as regards inheritance, that I am their inheritance, and so you shall not give to them property in Israel; I am their property.}%
\verse{The grain offering and the sin offering and the guilt offering, they themselves may eat them, and also all the consecrated possessions\lebnote{Hebrew “possession”} in Israel will be theirs.}%
\verse{And also what is first of all of the firstfruits of everything, and of all of the contribution of everything from all of your contributions, to the priest it belongs,\lebnote{“it shall be”} and what is first of your dough you shall give to the priest, so that a blessing may rest on your house.}%
\verse{Any dead body or mangled carcass from the birds\lebnote{Hebrew “bird”} or from the animals,\lebnote{Hebrew “animal”} the priests shall not eat.”}%
\end{biblechapter}%
\begin{biblechapter}% Ezekiel 45
\verseWithHeading{Instructions about Divisions of the Renewed Land}{And when you allocate the land as an inheritance, you shall provide a contribution for Adonai as a holy portion from the land, its length being twenty-five thousand cubits\lebnote{Or 8.33 miles} and its width ten thousand cubits;\lebnote{Hebrew 3.5 miles; LXX reads 6.67 miles in width, or 20,000 cubits (cf. 45:3, 5; 48:9)} it is holy in all its territory, all around.}%
\verse{And there shall be from this area five hundred cubits\lebnote{That is, “875 feet by 875 feet”} by five hundred cubits, squared all around, for the sanctuary; and fifty cubits\lebnote{Hebrew “cubit”}\lebnote{That is, “87.5 feet”} of open space shall be for it all around it.}%
\verse{And from this measured area you shall measure a length of twenty-five\lebnote{According to the reading tradition (\textit{Qere})} thousand cubits\lebnote{That is, 8.33 miles; cf. 45:5} and a width of ten thousand cubits,\lebnote{That is, 3.5 miles; cf. 45:5} and in it will be the sanctuary, the most holy place.\lebnote{“the holiness of holinesses”}}%
\verse{It\lebnote{Or “this area”} is a holy portion from the land; it will be for the priests, the servants of the sanctuary who approach\lebnote{“the ones who approach”} to serve Adonai, and it will be for them a place for houses and a holy place for the sanctuary.}%
\verse{And an area twenty-five thousand cubits in length and ten thousand cubits in width will be for the Levites, the servants of the temple. It will serve as property for them as cities to dwell in.\lebnote{“twenty chambers”}}%
\verse{And as property of the city, you must set apart alongside the contribution of the sanctuary a portion five thousand cubits\lebnote{That is, 1.67 miles} in width and twenty-five thousand cubits\lebnote{That is, 8.33 miles} in length, and it shall be for the entire house of Israel.}%
\verse{And a portion will be for\lnCPI{} the prince on both sides\lebnote{“from this and from this”} of the holy district,\lebnote{“the contribution of the holiness”} and both sides of the property of the city, and alongside\lebnote{“to the front of”} the property of the city on the west,\lebnote{“from/on the side of \textit{the} sea”} and alongside the property on the east.\lebnote{“from/on the side of east”} And also its length\lebnote{Or “portion”} corresponds to one of the tribal portions to the west\lebnote{“corresponding \textit{to} one of the portions from \textit{the} boundary of \textit{the} sea”} running to the eastern border.\lebnote{“to \textit{the} boundary of \textit{the} east”}}%
\verse{This shall be to him with respect to the land\lebnote{“to the land”} as property in Israel; and so my princes shall not again oppress my people, but they shall give the land to the house of Israel according to their tribes.”}%
\verse{Thus says the Lord Adonai, “Enough of this for you, the princes of Israel; put away violence and destruction and do justice and righteousness; revoke your acts of dispossession from upon\lebnote{Or “against”} my people!” declares\lnCPJ{} the Lord Adonai.}%
\verse{There shall be for you an honest set of scales\lebnote{“a set of scales of justice/honesty”} and an honest ephah\lebnote{“an ephah of honesty”} and an honest bath.\lebnote{“a bath of honesty”; a bath is a liquid measure}}%
\verse{The ephah and the bath shall be one unit of measurement;\lebnote{Or “shall be the same size”} the tenth part of the homer is the bath, and the tenth of the homer is the ephah; so the homer shall be its\lebnote{Or “the”} unit of measurement.}%
\verse{And the shekel shall weigh twenty gerahs, twenty shekels\lnCPK{} and five and twenty shekels\lnCPK{} and ten and five shekels,\lnCPK{} that shall make the mina for you.\lebnote{Meaning of Hebrew uncertain}}%
\verse{This is the contribution offering which you shall present: a sixth of the ephah from a\lebnote{Or “each”; literally “the”} homer of wheat, and a\lnCPL{} sixth of the ephah from a\lnCPL{} homer of barley.}%
\verse{And the quota of the olive oil, the bath of the olive oil, is the tenth part of a bath from a kor, which is ten baths, or a homer — for ten baths are equal to a homer.}%
\verse{And one\lebnote{Or “a”} sheep from the flock from among two hundred from the pastures of Israel will be taken as a grain offering and as a burnt offering and as a fellowship offering to make atonement for them,” declares\lnCPJ{} the Lord Adonai.}%
\verse{“All of the people of the land shall join in\lebnote{Or “participate/ take part in”} to this contribution with\lebnote{Or possibly “for/to”} the prince in Israel.}%
\verse{“But on the prince shall be the responsibility for the burnt offerings, and the grain offering, and the libation at the feasts, and at the New Moon festivals, and at\lebnote{Or “on”} the Sabbaths at all of the assemblies of the house of Israel; he shall provide the sin offering, and the grain offering, and the burnt offering, and the fellowship offering to make atonement for the house of Israel.}%
\verse{Thus says the Lord Adonai: ‘On the first month on the first day of the month, you shall take a bull, a calf\lebnote{“son of cattle,” or “son of \textit{the} herd”} without defect, and you shall purify the sanctuary.}%
\verse{And the priest shall take from the blood of the sin offering, and he shall put it on the doorframe of the temple\lnCPM{} and on the four corners of the ledge of the altar and on the doorframe of the gate of the inner courtyard.}%
\verse{And so you shall do on the seventh day in the month for anyone doing wrong inadvertently or due to ignorance, and so you must make atonement for the temple.\lnCPM{}}%
\verse{In the first month, on the fourteenth\lebnote{“four ten”} day of the month, you shall have the Passover, a feast lasting for seven days,\lebnote{“a week of days”} when you shall eat unleavened breads.}%
\verse{And the prince shall provide on that day for himself\lebnote{Hebrew “him”} and for all of the people of the land a bull as a sin offering.}%
\verse{And during the seven days of the religious feast he shall provide as a burnt offering to Adonai seven bulls and seven rams without defect for each of the seven days, and as a sin offering a he-goat for each day.}%
\verse{And as a grain offering, an ephah for\lnCPI{} the bull and an ephah for\lnCPI{} the ram he must provide, and a hin of olive oil for each\lnCPL{} ephah.}%
\verse{In the seventh month, on the fifteenth\lebnote{“five ten”} day of the month, during the feast, he shall provide just as he has for these seven days, as he did for the sin offering, as he did for the burnt offering, and as he did for the grain offering, and as he did for the olive oil.’”}%
\end{biblechapter}%
\begin{biblechapter}% Ezekiel 46
\verse{Thus says the Lord Adonai: “The gate of the inner courtyard facing east must be shut on\lebnote{Or “during”} the six of the days for work, but on the day of the Sabbath it must be opened and on the day of the new moon it must be opened.}%
\verse{And the prince shall come by way of the portico of the gate from the outside, and he shall stand at the doorframe of the gate and the priests shall prepare his burnt offering and his fellowship offerings, and he shall bow down at the threshold of the gate. Then he shall go out, but the gate shall not be closed until the evening.}%
\verse{And the people of the land will bow down at the doorway of that gate on the Sabbaths and on the new moons before\lnCPN{} Adonai.}%
\verse{And the burnt offering which the prince will offer to Adonai on the Sabbath day must be six male lambs without defect and a ram without defect.}%
\verse{And the grain offering he will give shall be an ephah for each ram,\lebnote{“to the ram”} and for the male lambs the grain offering shall be as much as he wants to give\lnCPO{} and a hin of olive oil for each ephah.\lnCPP{}}%
\verse{And on the day of the new moon he will offer a bull, a calf\lebnote{“son of cattle,” or “son of \textit{the} herd”} without defect and six male lambs and a ram; they must be without defect.}%
\verse{And an ephah for each bull and an ephah for the ram he must provide as a grain offering, and also for the male lambs as much as he can afford\lebnote{“as that it stretches out his hand”} and a hin of olive oil for each ephah.}%
\verse{And when the prince comes, on\lebnote{Or “by”} the way of the portico of the gate he shall come and by this\lebnote{Or “his”} way he shall go out.}%
\verse{And when the people of the land come before\lnCPN{} Adonai at the festivals, the one coming by way of the gate of the north to bow down\lebnote{Or “to worship”} must go out by the way of the gate of the south, and the one coming by way of the gate of the south must go out by the way of the gate of the north, so he must not return by the way of the gate by which he came,\lebnote{Hebrew “came by it”} but before it he\lebnote{According to the reading tradition (\textit{Qere})} shall go out opposite it.}%
\verse{And the prince will be in the midst of them; he will come in when they come in, and when they go out he will go out.}%
\verse{And at the festivals and at the appointed times, the grain offering will be an ephah with a bull and an ephah with the ram and with the male lambs, as much as he wants to give,\lnCPO{} and a hin of olive oil for each ephah.\lnCPP{}}%
\verse{And when the prince makes a freewill offering, a burnt offering, or a fellowship offering as a freewill offering to Adonai, the gate facing east will be open for him. And he must offer his burnt offering and his fellowship offering just as\lebnote{“as that”} he does on the day of the Sabbath, and then he must go out and the gate will shut after he goes out.\lebnote{“after his going out”}}%
\verse{And a male lamb, a yearling\lebnote{“a son of his year”} without defect, he must provide as a burnt offering daily\lebnote{“for the day”} for Adonai; every morning\lnCPQ{} he must provide it.}%
\verse{And a grain offering he must provide in addition to it every morning,\lnCPQ{} a\lnCPR{} sixth of an\lnCPR{} ephah and a third of a hin of olive oil to moisten the finely milled flour as a grain offering to Adonai as a perpetual statute.\lebnote{“regulations of everlasting continually”}}%
\verse{So they must provide the male lamb and the grain offering and the olive oil every morning\lnCPQ{} as a regular burnt offering.”\lebnote{“as burnt offering of regularity/continuity”}}%
\verse{Thus says the Lord Adonai: “If the prince gives a gift to any one from among his sons, it is his inheritance and it will belong\lnCPS{} to his sons; it is their possession as an inheritance.}%
\verse{And if he give a gift from his inheritance to one from among his servants, then it will belong\lnCPS{} to him until the year of release,\lebnote{That is, the Year of Jubilee} and then it will return to the prince. His inheritance with respect to his sons will be only for them.}%
\verse{But the prince shall not take from the inheritance of the people to rob\lebnote{Or “oppress”} them, separating them from their property; from his own property he shall bestow an inheritance on his sons, so that my people will not be driven away, any one, from his property.”}%
\verse{Then he brought me through the entrance which was on the side of the gate to the holy chambers\lebnote{“to the chamber of the holiness”} to the priests; the chambers were facing northwards and, look, there was a place at the extreme end westwards.}%
\verse{And he said to me, “This is the place where the priests will boil the guilt offering and the sin offering, and where they shall bake the grain offering in order not to bring them out to the outer courtyard to make the people holy.}%
\verse{And he brought me to the outer courtyard and led me past the four corners of the courtyard; and look, there was a courtyard in each corner of the courtyard.\lebnote{“a courtyard in the corner of the courtyard, a courtyard in the corner of the courtyard”}}%
\verse{In the four corners of the courtyard were small\lebnote{Hebrew uncertain} courtyards forty cubits in length and thirty cubits in width, each with the same measurements\lebnote{“measurement one”} for the four of them with corners.\lebnote{Or “\textit{being} cornered”}}%
\verse{And a row was all around inside\lebnote{Hebrew “in”} them, and cooking-places were made under the rows of stones\lebnote{Possibly some kind of stone ledge} all around.}%
\verse{And he said to me, “These are the houses for\lebnote{Hebrew “of”} cooking, where the ones serving the temple shall cook the sacrifices of the people.”}%
\end{biblechapter}%
\begin{biblechapter}% Ezekiel 47
\verseWithHeading{River of the Temple, Division of the Land and Its Boundaries}{And he brought me back to the doorway of the temple and, look! There was water coming out from under the threshold of the temple eastward, because the face of the temple was eastwards; and the water was coming down from under the south side of the temple, from south of the altar.}%
\verse{And he brought me out by the way of the north gate and he led me around by the way leading to the outside of the outer gate\lebnote{“way of \textit{the} outside to the gate of the outside”} by way of the gate facing east, and look, there was water trickling from the south side.}%
\verse{As the man went eastward a measuring line was in his hand, and he measured a thousand cubits.\lebnote{“in the cubit”; about 1,750 feet} And he led me through on into the water; it was ankle deep.\lebnote{“water of ankles”}}%
\verse{And he measured a thousand cubits,\lnCPT{} and he brought me over into the waters and it was knee deep.\lebnote{“water of knees”} And he measured a thousand cubits,\lnCPT{} and he brought me over into the waters and it was waist deep.\lebnote{“water of waist”}}%
\verse{And he measured a thousand cubits,\lnCPT{} and it became a stream that I was not able to cross, because the water rose, waters a person could swim in,\lebnote{“water of swimming”} and became a stream that could not be crossed.}%
\verse{And he asked me, “Son of man,\lebnote{Or “mortal,” or “son of humankind”} did you see this?” And he made me go, and he brought me back along the bank of the stream.}%
\verse{When I returned,\lebnote{“At my returning”} then look! On the bank of the river were many trees on both sides.\lnCPU{}}%
\verse{And he said to me, “These waters are going out to the eastern region, and they go down to the Jordan Valley,\lebnote{“Arabah”} and they come to the sea\lebnote{Clearly refers to the Dead Sea} and flow into the sea where they issue out,\lebnote{“being brought out/are issued out”} and the waters in the sea will be healed.\lebnote{Or “made fresh”}}%
\verse{And then every living creature with which it teems,\lebnote{Or “swarms”} to every place the waters come\lebnote{“to all of which they come there the water”} it will live; and very many fish will live because these waters came there, and so the water will be healed\lebnote{Or “fresh”} and everything will be alive everywhere that\lebnote{“all that”} the stream will come.\lebnote{“that \textit{it} will come there the stream”}}%
\verse{And then fishers\lebnote{Or “fishermen”} from En Gedi and on up to En Eglaim will stand beside it, a sea-girt drying yard for nets; their fish for the dragnets will be of every kind,\lebnote{“to the”} like the fish of the great sea,\lnCPV{} and very many.}%
\verse{But its marshes and its swamps will not be cured, for they were given for salt.}%
\verse{And along the stream will go up on its banks from both sides\lnCPU{} every tree producing food;\lebnote{“of food”} its leaf will not wither and it will not cease producing its fruit. Every month\lebnote{“for its months”} it will bear early fruit, for its waters are going out from the sanctuary, and its fruit will be as food, and its leaf for healing.”}%
\verse{Thus says the Lord Adonai: “This is the boundary by which you shall distribute the land for the twelve\lebnote{“two ten”} tribes of Israel: Two shares shall be for Joseph.}%
\verse{And you must take possession of it, each one a share like his brother, of this land which I swore\lebnote{“I raised my hand”} to give it to your fathers, and so this land shall fall to you as an inheritance.}%
\verse{And\lebnote{Or “Bow”} this is the boundary of the land: On the north side,\lebnote{“to the side of \textit{the} north”} from the great sea by the way of Hethlon until you come to Zedad,\lebnote{“to the arriving at Zedad”}}%
\verse{Hamath\lebnote{Or “Lebo-Hamath to Zeded” (cf. NRSV, NJPS)} to Berothath to Sibraim, which is\lebnote{Or “are”} between the boundary of Damascus and the boundary of Hamath, on to Hazer Hatticon, which is on the boundary of Hauran.}%
\verse{And so the boundary will be from the sea\lnCPV{} to Hazar Enan at the boundary of Damascus northwards and the boundary of Hamath to the north, and this is the boundary on the side of the north.}%
\verse{And the eastern boundary\lebnote{“and \textit{the} side of \textit{the} east”} will be between Hauran and Damascus, and from between Gilead and the land of Israel along the Jordan River from\lebnote{Or “to”} the boundary on the eastern sea to Tamar. And\lebnote{Or “Now”} this is the border on the east.\lebnote{“and this \textit{is the} side of \textit{the} east”}}%
\verse{And on the south side the border\lebnote{“and the side of the south \textit{to} south”} will run from Tamar up to the waters of Meribot Kadesh and along the wadi\lebnote{That is, the Wadi of Egypt} to the Great Sea.\lnCPV{} And this is the boundary to the south.\lebnote{“and this \textit{is} the side of south \textit{to} the south”}}%
\verse{And on the west side\lebnote{“and the side of the west”} the Great Sea\lnCPV{} shall be the boundary up to opposite Lebo-Hamath. This is the western boundary.\lebnote{“this \textit{is the} side of \textit{the} west”}}%
\verse{You shall divide this land for yourselves according to the tribes of Israel.}%
\verse{And then you shall allocate it as an inheritance for yourselves and for the aliens dwelling as aliens in the midst of you who have children in the midst of you, and so they shall be to you just like full citizens among the Israelites.\lebnote{“sons/children of Israel”} With you they shall be allotted with an inheritance in the midst of the tribes of Israel.}%
\verse{And then in the tribe in which an alien who is with you dwells, there you shall give his inheritance,” declares\lebnote{“declaration of”} the Lord Adonai.}%
\end{biblechapter}%
\begin{biblechapter}% Ezekiel 48
\verseWithHeading{The Allotment of the Land and the Gates of the City}{And\lebnote{Or “Now”} these are the names of the tribes: At the end of the north, along the way to Hethlon\lebnote{“to the side of the way of Hethlon”} from Lebo-hamath to Hazar Enan on the boundary of Damascus beside Hamath northward,\lebnote{“northward to the side of Hamat,”} there will be one portion for Dan from the eastern border to the west.\lebnote{“sea”}}%
\verse{And next to the territory of Dan, from the eastern border\lnCPW{} up to the western border,\lnCPX{} one portion for Asher.}%
\verse{And next to the territory of Asher, from the eastern border\lnCPY{} and up to the western border,\lebnote{“\textit{the} side of \textit{the} west/sea”} one portion for Naphtali.}%
\verse{And next to the territory of Naphtali, from the eastern border\lnCPY{} up to the western border,\lnCPX{} one portion for Manasseh.}%
\verse{And next to the territory of Manasseh, from the eastern border\lnCPY{} up to the western border,\lnCPX{} one portion for Ephraim.}%
\verse{And next to the territory of Ephraim, from the eastern border\lnCPY{} and up to the western border,\lnCPZ{} one portion for Reuben.}%
\verse{And next to the territory of Reuben, from the eastern border\lnCPW{} up to the western border,\lnCPZ{} one portion for Judah.}%
\verse{And next to the territory of Judah, from the eastern border\lnCPW{} up to the western border,\lnCPZ{} shall be located the contribution that you set apart, twenty-five thousand cubits\lnCQA{} in width, and its length just like one of the portions from the eastern border\lnCPW{} up to the western border,\lnCPZ{} and the sanctuary shall be in the midst of it.}%
\verse{The contribution that you shall set apart for Adonai, its length shall be twenty-five thousand cubits,\lnCQA{} and its width ten thousand cubits.\lebnote{That is, 3.5 miles; cf. 48:13 and 45:1}}%
\verse{And to these shall be the holy district:\lnCQB{} to the priests northwards twenty-five thousand cubits,\lnCQA{} and westwards its width ten thousand cubits,\lnCQC{} and eastwards its width ten thousand cubits,\lnCQC{} and southwards its length twenty-five thousand cubits,\lnCQA{} and the sanctuary of Adonai shall be in the midst of it.}%
\verse{To the priests, the ones being consecrated from the descendants\lebnote{Or “sons”} of Zadok, who took care of my service and who did not go astray\lebnote{“who not they went astray”} when the Israelites went astray,\lebnote{“at going astray the children of Israel”} just as\lebnote{“as that”} the Levites went astray.}%
\verse{And it will be for them a special gift from the contribution of the land as a most holy object\lebnote{“\textit{as} a holiness of holinesses”} next to the territory of the Levites.}%
\verse{And the Levites alongside\lnCQD{} the territory of the priests shall have twenty-five thousand cubits\lnCQA{} in length and in width ten thousand cubits,\lnCQC{} its full length,\lebnote{“all/every of \textit{its} length ….”} twenty-five thousand cubits\lnCQA{} and its full width, ten thousand cubits.\lnCQC{}}%
\verse{And they shall not sell any part from it, and shall not exchange, and shall not transfer the best of the land, for it is holy to Adonai.}%
\verse{And the remaining part\lebnote{“the being a part left over”} of five thousand cubits\lebnote{That is, 1.75 miles} in the width by\lebnote{“in front of”} twenty-five thousand cubits,\lnCQA{} is unholy;\lebnote{Or “common”} it is for the city as dwelling and as pastureland; and the city will be located in the middle of it.\lebnote{“will be the city in its middle”}}%
\verse{And these shall be its measurements: on its side to the north, four thousand five hundred cubits;\lnCQE{} and on its side to the south, four thousand five hundred cubits;\lnCQE{} and on the eastern side,\lebnote{“from \textit{the} side of \textit{the} east”} four thousand five hundred cubits;\lnCQE{} and on the western side,\lebnote{“and \textit{the} side of \textit{the} sea/west + directive”} four thousand five hundred cubits.\lnCQE{}}%
\verse{And a pastureland shall be for the city northwards, two hundred and fifty cubits,\lnCQF{} southwards two hundred and fifty cubits,\lnCQF{} and eastwards two hundred and fifty cubits,\lnCQF{} and westwards two hundred and fifty cubits.\lnCQF{}}%
\verse{And the rest in its length alongside\lnCQD{} the holy district,\lnCQB{} ten thousand cubits\lnCQC{} eastwards and ten thousand\lnCQC{} westwards; and it shall be alongside\lnCQD{} the holy district\lnCQB{} and its yield\lebnote{According to the reading tradition (\textit{Qere}); cf. BHS} shall be as food for the workers of the city.}%
\verse{And the workers of the city from all the tribes of Israel shall cultivate it.}%
\verse{All of the contribution, twenty-five thousand by twenty five thousand cubits\lebnote{That is, 8.33 square miles} square, you shall set apart as the holy district\lnCQB{} along with the property of the city.}%
\verse{And the rest shall be to\lebnote{Or “for”} the prince, on both sides\lebnote{“from this and from this”} of the holy district\lebnote{“of the contribution of the holiness”} and of the property of the city extending from\lnCQG{} twenty-five thousand cubits of the contribution up to the boundary of the east, and westwards extending\lnCQG{} twenty-five thousand cubits\lnCQA{} to the boundary of the sea alongside\lnCQD{} the tribal portions. They shall be for the prince, and it shall be the holy district,\lnCQB{} and the sanctuary of the temple\lebnote{Or “house”} will be in its middle.}%
\verse{And also the property of the Levites and the property of the city will be in the midst of the property which is for the prince; between the territory of Judah and between the territory of Benjamin the land will be for the prince.}%
\verse{And for the remainder of the tribes: From the side on the east up to the side on the west, Benjamin one portion;}%
\verse{and next to the territory of Benjamin, from the side on the east up to the side of the sea, Simeon one portion;}%
\verse{and next to the territory of Simeon, from the side of the east up to the side of the sea, Issachar one portion;}%
\verse{and next to the territory of Issachar, from the side of the east up to the side of the sea, Zebulun one portion;}%
\verse{and next to the territory of Zebulun, from the side of the east up to the side of the sea, Gad one portion;}%
\verse{and next to the territory of Gad, to the side of the south, southward;\lebnote{“south”} and the territory will be from Tamar to the waters of Meribath Kadesh, toward the wadi,\lebnote{That is, the Wadi of Egypt} on to the Great Sea.\lebnote{That is, the Mediterranean}}%
\verse{This is the land that you shall allot as an inheritance to the tribes of Israel, and these are their portions,” declares\lebnote{“declaration of”} the Lord Adonai.}%
\verse{And these shall be the exits of the city: From\lebnote{Or “on” or “along”} the north side,\lebnote{“\textit{the} side of \textit{the} north”} four thousand and five hundred cubits\lnCQE{} by measurement.}%
\verse{And the gates of the city shall be according to the names of the tribes of Israel: three gates to the north, the gate of Reuben, one; the gate of Judah, one; the gate of Levi, one.}%
\verse{And on the east side, four thousand five hundred cubits,\lnCQE{} and three gates, and so the gate of Joseph, one; the gate of Benjamin, one; the gate of Dan, one.}%
\verse{And on the south side, four thousand five hundred cubits\lnCQE{} by measurement, and so three gates, the gate of Simeon, one; the gate of Issachar, one; the gate of Zebulun, one.}%
\verse{And on the west\lebnote{Or “sea”} side, four thousand five hundred cubits,\lnCQE{} and their gates three, the gate of Gad, one; the gate of Asher, one; the gate of Naphtali, one.}%
\verse{All around the city is eighteen thousand cubits,\lebnote{That is, 6 miles} and the name of the city from that day is “Adonai Is There”!}%
\end{biblechapter}%
\flushcolsend
\biblebook{Daniel}
\begin{biblechapter}% Daniel 1
\verseWithHeading{Daniel’s Development in the King’s Court}{In the third year of the reign of Jehoiakim king of Judah, Nebuchadnezzar the king of Babylon came to Jerusalem and besieged it.}%
\verse{And the Lord gave Jehoiakim the king of Judah into his hand and some of\lebnote{“from the end/extremity/limits”} of the utensils of the temple of God,\lnCQH{} and he brought them to the land of Shinar to the temple of his gods, and he brought the utensils to the treasury\lebnote{“the house of the treasury”} of his gods.}%
\verse{And the king ordered\lebnote{“said to”} Ashpenaz, the commander of his court officials, to bring some of the Israelites\lebnote{“from the sons/children of Israel”} from the royal family\lebnote{“from the seed of the kingship”} and from the lords,\lebnote{Or “nobles”}}%
\verse{youths who have no physical defect,\lebnote{“there is not in them any physical defect”} and who are handsome,\lebnote{“good/pleasing of appearance”} and who are prudent\lebnote{Or “insightful”} in all wisdom and endowed with knowledge,\lebnote{“\textit{who have} knowledge of knowledge”} and who understand insight, and who have the ability\lebnote{“strength”} in them to serve in the palace of the king. And the king ordered him to teach them the literature and the language of the Chaldeans.}%
\verse{And the king assigned to them his daily portion\lebnote{“\textit{the} portion of the day on its day”} from the fine food of the king, and from the wine that he drank,\lnCQI{} and instructed that they were to be educated for three years.\lebnote{“and to be educated them years three”} And at the end of their training, they were to be stationed\lebnote{“they would stand”} before\lnCQJ{} the king.}%
\verse{Now there was among them from the Judeans,\lebnote{“from the children of Judah”} Daniel, Hananiah, Mishael and Azariah.}%
\verse{And the commander of the court officials gave them names, and he called Daniel, Belteshazzar; and Hananiah, Shadrach; and Mishael, Meshach; and Azariah, Abednego.}%
\verseWithHeading{Daniel’s Resolve}{Now Daniel resolved\lebnote{“Daniel placed upon his heart”} that he would not defile himself with the fine food of the king, and with the wine that he drank,\lnCQI{} and so he requested from the commander of the court officials permission so that he would not defile himself.}%
\verse{And God\lnCQH{} gave Daniel favor and compassion before\lnCQJ{} the commander of the court officials,}%
\verse{and the commander of the court officials said to Daniel, “I am afraid of my lord, the king, who has determined your food and your drink, for why should\lebnote{“to what would”} he see your face having a worse appearance than the young men who are your age?\lebnote{“like your age”; “your” is plural} Then you will endanger my head with the king.”}%
\verse{Then\lnCQK{} Daniel asked the guard\lebnote{Or “overseer”} whom the commander of the court officials had appointed over Daniel, Hananiah, Mishael, and Azariah,}%
\verse{“Please test your servants for ten days, and let them give us some of the vegetables,\lebnote{“from the vegetables”} and let us eat and let us drink water.}%
\verse{Then\lnCQK{} let our appearances and the appearance of the young men who are eating the fine food of the king be compared before you,\lebnote{“in your presence”} and then deal with your servants according to what you see.”\lebnote{“and according to/that you see”}}%
\verse{So he agreed to this proposal with them, and he tested them for ten days.}%
\verse{And at the end of ten days their appearances appeared better and they were healthier of body than all the young men who were eating the fine food of the king.}%
\verse{So\lebnote{“And it happened”} the guard continued to withhold\lebnote{“he was … withdrawing”} their fine food and the wine of their drink, and he gave them\lebnote{Hebrew “to them”} vegetables.}%
\verse{And as for these four young men,\lebnote{“And these young men, four of them”} God\lnCQH{} gave to them knowledge and insight into all literature and wisdom, and Daniel had insight into all visions\lebnote{Hebrew “vision”} and dreams.}%
\verse{And at the end of the time the king had set to bring them, the commander of the court officials brought them in before\lnCQJ{} Nebuchadnezzar.}%
\verse{And the king spoke with them, and among all of them no one was found like\lebnote{“and he was not found from all of them like”} Daniel, Hananiah, Mishael, and Azariah; then they stood before\lnCQJ{} the king.}%
\verse{And in every matter of wisdom and understanding about which the king inquired from them, he found them ten times better than all of the magicians\lebnote{Or “soothsayer-priests”} and conjurers\lebnote{Or “enchanters”} that were in his entire kingdom.\lebnote{“in the whole of his kingdom”}}%
\verse{And Daniel was there until the first year of Cyrus the king.}%
\end{biblechapter}%
\begin{biblechapter}% Daniel 2
\verseWithHeading{The King’s Troubling Dream}{Now\lebnote{Hebrew “And”} in the second year of the reign of Nebuchadnezzar, Nebuchadnezzar dreamed dreams; and his spirit was troubled and his sleep left him.}%
\verse{So the king said to call the magicians\lnCQL{} and the conjurers\lnCQM{} and the sorcerers and the astrologers\lnCQN{} to tell to the king his dreams. And they came in and they stood before\lebnote{“to the face of”} the king.}%
\verse{And the king said to them, “I have had a dream\lebnote{“A dream I have dreamed”} and my spirit is anxious to know the dream.”}%
\verse{And the astrologers\lnCQN{} said to the king in Aramaic,\lebnote{The text of Daniel is in Aramaic from this point on through Daniel chapter 7. Then the text resumes in Hebrew to the end of the book} “O king, live forever! Tell the dream to your servants and we will reveal the explanation.”}%
\verse{The king answered and said to the astrologers,\lnCQN{} “The command from me is firm: if you do not make known to me the dream and its explanation,\lnCQO{} then you will be broken into pieces and your houses will be laid in ruins.}%
\verse{But if you tell me the dream and its explanation, you will receive gifts and rewards\lebnote{Aramaic “reward”} and great honor from me. Therefore, tell me the dream and its explanation.”\lnCQO{}}%
\verse{They answered once more and said, “Let the king tell the dream to his servants and we will make the explanation known.”}%
\verse{The king answered and said, “Certainly\lebnote{“From \textit{a} certainty”} I know that you are trying to gain time\lebnote{“the time you are gaining”} because\lebnote{“all because of that”} you have seen that this matter is firmly decreed by me,\lebnote{“that firm from me \textit{is} the decree”}}%
\verse{for if you do not make the dream known to me, your verdict is fixed,\lebnote{“one is your verdict”} and you have conspired to say a lying and deceitful word to me until the circumstances\lebnote{“time”} will change. Therefore, tell me the dream, and I will know that you can tell me its explanation.”}%
\verse{The astrologers\lnCQN{} answered the king and said, “There is not a man on earth that is able to reveal the word of the king; in fact,\lebnote{“all because that”} no great and powerful king has ever asked a thing like this of any magician\lebnote{Or “soothsayer-priest”} or conjurer\lebnote{Or “enchanter”} or astrologer.\lebnote{“Chaldean”}}%
\verse{And the thing that the king is asking is too difficult and there is no one who\lebnote{“there is not another”} can reveal it to the king except the gods whose dwelling is not with mortals.”\lebnote{“concerning their dwelling is not with flesh”}}%
\verse{Because of this the king became angry, and he became very much enraged, and he said that all the wise men of Babylon are to be destroyed.}%
\verse{And the decree was issued, and the wise men were on the verge of being executed,\lebnote{“were in the process of being killed”} and they searched for Daniel and his companions to be executed.}%
\verse{Then Daniel responded prudently and discretely to Arioch, the commander of the imperial guard of the king, who had gone out to execute the wise men of Babylon.}%
\verse{He asked\lebnote{“he answered”} and said to Arioch, the royal official of the king, “Why\lebnote{“Over what”} is the decree from the king so severe?” Then Arioch explained the matter\lebnote{Or “situation”} to Daniel.}%
\verse{And Daniel went in and requested from the king that he would give him time, and he would tell the king\lebnote{“to tell/reveal to the king”} the explanation.\lnCQO{}}%
\verse{Then Daniel went to his home, and he made the matter known to Hananiah, Mishael and Azariah, his companions,}%
\verse{and told them to seek mercy from the God of heaven\lebnote{“from before the God of the heaven”} concerning this mystery, so that Daniel and his companions, along with the remainder of the wise men of Babylon, would not be killed.}%
\verse{Then in a vision of the night the mystery was revealed to Daniel; then Daniel blessed the God of heaven.}%
\verse{Daniel said:\lebnote{“answered and he said”} “Let his name, the name of God,\lebnote{Aramaic “the God”} be blessed throughout the ages,\lebnote{“from the age unto the age”} for the wisdom and the power are his.\lebnote{“of to him it \textit{is}”}}%
\verse{And he changes the times and the seasons, and he deposes kings and he sets up kings; he gives wisdom\lebnote{Aramaic “the wisdom”} to wise men and knowledge to men who know understanding.}%
\verse{He reveals the deep and the hidden things; he knows what is in the darkness, and the light dwells with him.}%
\verse{To you, O God of my ancestors,\lebnote{Or “fathers”} I give thanks and I give praises, for the wisdom and the power you gave to me, and now you have made known to me what we have asked from you, for you have made known to us the matter of the king.”}%
\verseWithHeading{Daniel Praises God and Interprets the King’s Dream}{Therefore\lebnote{“Now because this”} Daniel went to Arioch, whom the king had appointed\lebnote{Aramaic “has appointed”} to destroy the wise men of Babylon; he went and thus he said to him: “You must not destroy the wise men of Babylon; take me in before the king and I will give the explanation\lnCQO{} to the king.”}%
\verse{Then Arioch quickly\lebnote{“in haste”} brought Daniel in before the king and thus he said to him: “I have found a man among the exiles\lebnote{“the children of exiles”} of Judah\lebnote{Aramaic “Jehud”} who can relate\lebnote{“he will make known”} the explanation\lnCQO{} to the king.}%
\verse{The king then asked\lebnote{“The king answered”} and said to Daniel, whose name was Belteshazzar, “Are you able to make known to me the dream that I have seen and its explanation?”\lnCQO{}}%
\verse{Daniel answered the king and said, “The mystery that the king asks, no wise men, conjurers,\lnCQM{} magicians,\lnCQL{} or diviners are able to make known to the king.}%
\verse{But there is a God in heaven who reveals mysteries, and he has made known to King Nebuchadnezzar what it is that will be at the end of days. This is your dream and the visions of your head on your bed.}%
\verse{“As for you, king,\lebnote{Aramaic “the king”} your thoughts on your bed turned to\lebnote{“they went up”} what it was that would be in the future,\lnCQP{} and the revealer of mysteries has made known to you what that would be.}%
\verse{And as for me, it is not because of wisdom that is in me more than any other living person\lebnote{“all of living beings”} that this mystery is revealed to me, but in order that\lebnote{“concerning the matter of”} the explanation\lnCQO{} may be made known to the king and you will understand\lebnote{“know”} the thoughts of your mind.\lebnote{“heart”}}%
\verse{“You, O king, were looking and, look, there was one great statue. This statue was huge and its brilliance extraordinary, standing there before you, and its appearance was frightening.\lebnote{Or “dreadful”}}%
\verse{The head of this statue was of fine gold, its chest and its arms of silver, its belly and its thighs of bronze,}%
\verse{its legs of iron, its feet, part of them of iron and part of them of clay.}%
\verse{You were looking on until\lebnote{“until that”} a stone was chiseled out\lnCQQ{} — that not by hands — and it struck the statue on its feet of iron and clay, and it broke them in pieces.}%
\verse{Then the iron, the clay, the bronze, the silver and the gold all at once\lebnote{“like one”} broke into pieces\lebnote{Or “were crushed”} and they became like chaff from the summer threshing floor, and the wind carried them away and any trace of them could not be found. But the stone that struck the statue became a great mountain and it filled the whole earth.}%
\verse{“This was the dream, and now we will tell its interpretation to the king.}%
\verse{You, O king, the king of kings, to whom the God of heaven has given the kingdom, the power and the might and the glory,}%
\verse{and also human beings wherever they dwell,\lebnote{“and into all that dwell sons of humankind”} the animals of the field and the birds\lebnote{Aramaic “bird”} of heaven\lebnote{Or “the sky”} — he has given into your hand and made you ruler over all of them — you are the head of gold.}%
\verse{And after you another kingdom inferior to yours will arise, and another third kingdom of bronze that will rule over the whole earth.}%
\verse{And a fourth kingdom will be strong as iron, and just as\lebnote{“all of because”} iron crushes and smashes everything,\lebnote{“the whole thing”} and as iron shatters all of these other metals, so it will crush and it will shatter these nations.}%
\verse{And just as\lebnote{“And that”} you saw the feet and the toes, partly potter’s clay\lebnote{“from them clay of the potter”} and partly iron,\lebnote{“from them iron”} it will be a divided kingdom; and some hardness\lebnote{“from the hardness”} of the iron will be in it, just as\lebnote{“all of because that”} you have seen the iron mixed with the wet clay.}%
\verse{And as the toes of the feet were partly iron and partly clay, so part of the kingdom will be strong and partly brittle.\lebnote{“and from it, it will be brittle”}}%
\verse{And in that you saw the iron was mixed with wet clay, so they will mix in marriage\lebnote{“mixing they will in the seed of man”} and they will not cling to one another,\lebnote{“this one with that one”} just as iron does not mix with clay.}%
\verse{And in the day of those kings,\lebnote{“in their days of those kings”} the God of heaven will set up a kingdom that will never\lebnote{“to forever not”} be destroyed, and the kingdom will not be left for another nation,\lebnote{Or “people”} and it will bring an end to all these kingdoms, but it will stand forever.\lebnote{“to eternity”}}%
\verse{Inasmuch as\lebnote{“All of because that”} you saw that a stone from the mountain was chiseled out\lnCQQ{} but not by hands, and that it crushed the iron, the bronze, the clay, the silver and the gold, thereby the great God made known to the king what will be in the future,\lnCQP{} and the dream is certain and its explanation trustworthy.”}%
\verseWithHeading{The King’s Response to Daniel and His God}{Then the king, Nebuchadnezzar, fell on his face and he paid homage to Daniel; and he commanded them to offer a grain offering and incense offering to him.}%
\verse{The king answered Daniel and said, “Truly\lebnote{“From truth that”} your God is the God of gods and the Lord of kings, and he reveals mysteries, for you are able to reveal this mystery.”}%
\verse{Then the king promoted Daniel and gave him many great gifts, and made him ruler over the whole province of Babylon and the chief prefect over all the wise men of Babylon.}%
\verse{And Daniel made a request from the king, and he appointed Shadrach, Meshach and Abednego over the affairs of the province of Babylon, while Daniel remained in the court of the king.}%
\end{biblechapter}%
\begin{biblechapter}% Daniel 3
\verseWithHeading{The Faithfulness of Three Young Israelites and God’s Deliverance}{Nebuchadnezzar the king made a statue of gold. Its height was sixty cubits and its width was six cubits; he set it up in the valley of Dura in the province of Babylon.}%
\verse{Then Nebuchadnezzar the king sent directions to assemble the satraps, the prefects, and the governors, the counselors, the treasurers, the judges, the magistrates and all of the officials of the provinces to come to the dedication of the statue that Nebuchadnezzar the king had set up.}%
\verse{Then the satraps, the prefects, the governors, the counselors, the treasurers, the judges, the magistrates, and all the officials of the province were assembled for the dedication of the statue that Nebuchadnezzar the king had set up, and were standing before\lebnote{“to \textit{the} front”} the statue that Nebuchadnezzar had set up.}%
\verse{Then the herald proclaimed aloud,\lebnote{“with power”} “To you it is commanded,\lebnote{“they \textit{are} saying”} O peoples, nations and people of all languages,\lnCQR{}}%
\verse{that at the time that you hear the sound of the horn, the flute, the lyre, the trigon, the harp, the drum and all kinds of music, you must fall down and you must worship the statue of gold\lnCQS{} that Nebuchadnezzar the king has set up.}%
\verse{And whoever\lebnote{“who that”} does not fall down and worship will be thrown immediately\lebnote{“the time”} into the midst of the furnace of blazing fire.”}%
\verse{Therefore,\lebnote{“All of because this”} at that time,\lebnote{“at the time”} as soon as\lebnote{“as that”} all the peoples heard the sound of the horn, the flute, the lyre, the trigon, the harp and all kinds of music, all the peoples, the nations and people of all languages\lnCQR{} were falling down and were worshiping the statue of gold\lnCQS{} that Nebuchadnezzar the king had set up.}%
\verse{Therefore\lebnote{“All because of this”} at this time\lebnote{“at it the time”} some astrologers\lebnote{“Chaldean men”} came forward and they denounced the Jews.\lebnote{“they ate their pieces of the Jews”}}%
\verse{They responded and said\lebnote{“they answered and \textit{were} saying”} to Nebuchadnezzar the king, “O king, may you live forever!\lebnote{“to eternity”}}%
\verse{You, O king, have made a decree that everyone\lebnote{“all of humanity”} who hears the sound of the horn, the flute, the lyre, the trigon, the harp and the drum and all kinds of music, he shall fall down and shall worship the statue of gold.\lnCQS{}}%
\verse{And whoever\lebnote{“And who that”} does not fall down\lebnote{“not he falls down”} and worship shall be thrown into the midst of the furnace of blazing fire.}%
\verse{However there are Judean men whom you have appointed\lebnote{Aramaic “whom you have appointed them”} over the affairs of the province of Babylon — Shadrach, Meshach, and Abednego — these men, O king, pay you no heed\lebnote{“not they pay heed to you”} and were not serving your god, and the statue of gold\lnCQS{} that you set up they are not worshiping.”}%
\verse{Then Nebuchadnezzar said in rage and anger to bring in Shadrach, Meshach and Abednego; then they brought in these men before the king.}%
\verse{Nebuchadnezzar answered and said to them, “Is it true, Shadrach, Meshach and Abednego, that you are not serving my god, and you are not worshiping the statue of gold\lnCQS{} that I have set up?}%
\verse{Now if you are ready so that when\lebnote{“at that time”} you hear the sound of the horn, the flute, the lyre, the trigon, the harp and the drum and all kinds of music, you fall down and you worship the statue that I have made, that will be good. But if you do not worship it, immediately\lebnote{“at that \textit{very} time”} you will be thrown into the midst of the furnace of blazing fire, and who is the god\lebnote{“who he god”} who\lebnote{Or “that”} will rescue you from my hands?”}%
\verse{Shadrach, Meshach and Abednego answered and said to the king, Nebuchadnezzar, “We have no need on this matter to present a defense to you.\lebnote{“to return to you”}}%
\verse{If it is so,\lebnote{“If there is”} our God, whom we serve, is able to rescue us from the furnace of blazing fire. And from your hand, O king, let him rescue us.}%
\verse{And if not, let it be known to you, O king, that we will not serve your gods,\lebnote{“your gods not there is us serving”} and the statue of gold\lnCQS{} that you have set up we will not worship.”}%
\verseWithHeading{The Wrath of the King and His Fiery Furnace of Death}{Then Nebuchadnezzar was filled with anger and the image of his face was changed toward\lebnote{“against”} Shadrach, Meshach and Abednego, so he ordered and said\lebnote{“answering and saying”} to heat up the one furnace seven times what was usual to heat it up.}%
\verse{And he commanded the strongest men of the guards\lebnote{“to the strong men, men of strength”} who were in his army to bind Shadrach, Meshach and Abednego and to throw them into the furnace of blazing fire.}%
\verse{Then these men were bound with their garments, their trousers and their turbans and their other clothing, and they were thrown into the midst of the furnace of blazing fire.}%
\verse{Therefore\lebnote{“because of this”} because\lebnote{“from that”} the word of the king was severe and the furnace was exceedingly hot, the flame of the fire killed these men who lifted up Shadrach, Meshach and Abednego.}%
\verse{But these men, the three of them, Shadrach, Meshach and Abednego, fell down into the midst of the furnace of blazing fire, and they were bound.}%
\verse{Then Nebuchadnezzar the king was astonished, and he rose up in haste and he asked, saying to his advisers, “Did we not throw three men, bound, into the midst of the fire? They answered, saying to the king, “Certainly, O king!”}%
\verse{He answered,\lebnote{“\textit{was} answering”} saying, “Look, I see four men unbound walking in the midst of the fire and there is no damage to them, and the\lebnote{Aramaic “its”} appearance of the fourth man resembles the son of a god.”}%
\verse{Then Nebuchadnezzar approached the door of the furnace of blazing fire, and he called out, saying, “Shadrach, Meshach and Abednego, servants\lebnote{“his servant”} of the Most High God, come out\lebnote{Or “Go out”} and come here!”}%
\verse{And the satraps, the prefects, the governors and the advisors of the king were assembling, and they saw these men, that the fire had no power over their bodies, and the hair of their heads was\lebnote{Aramaic plural} not singed, and their garments were not harmed, and the smell of fire did not come from them.}%
\verseWithHeading{The King’s Praise for the Great Deliverance of the Three Young Israelites}{Nebuchadnezzar responded,\lebnote{“answering”} saying, “Blessed be their God, the God of Shadrach, Meshach and Abednego, who sent his angel\lebnote{Or “messenger”} and rescued his servants who trusted in him, and the command of the king they disobeyed, and they gave their bodies so that they did not serve and did not worship any god except their God.}%
\verse{And from me is set forth\lebnote{“setting out”} a decree that any people, nation, or language that may utter criticism against their God — the God of Shadrach, Meshach and Abednego — will be broken into pieces and their\lebnote{Aramaic “his”} house will be made like ruins. For\lebnote{“All of because”} there is not another God who is able to rescue like this God.”}%
\verse{Then the king promoted Shadrach, Meshach and Abednego in the province of Babylon.}%
\end{biblechapter}%
\begin{biblechapter}% Daniel 4
\verseWithHeading{Nebuchadnezzar Declares What the Most High God Has Done for Him}{\lebnote{Daniel 4:1–4:37 in the English Bible is 3:31–4:34 in the Hebrew Bible} Nebuchadnezzar the king to all of the nations, the peoples and languages that live in the whole earth: “May your prosperity increase!}%
\verse{It is pleasing to me to recount the signs and wonders that the Most High God worked for me.}%
\verse{How great are his signs and wonders, how strong is his kingdom, an everlasting kingdom;\lebnote{“a kingdom of eternity”} and his sovereignty is from generation to generation.}%
\verseWithHeading{The Dream and the King’s Search for an Interpreter}{“I, Nebuchadnezzar, was content in my house and prospering in my palace.}%
\verse{And I saw a dream and a revelation on my bed and it frightened me, and the visions of my head terrified me.}%
\verse{And a decree was sent out,\lebnote{“put/placed”} ordering that all the wise men of Babylon were to be brought in before me so that they may make known to me the explanation of the dream.}%
\verse{Then the magicians,\lnCQT{} the conjurers,\lebnote{Or “enchanters”} the astrologers\lebnote{“Chaldeans”} and the diviners came in and I told them the dream, but they could not make known to me its explanation.\lnCQU{}}%
\verse{Then at last\lebnote{“until another”} Daniel came before me whose name was\lebnote{“who name his”} Belteshazzar, according to\lebnote{“like”} the name of my god, and in whom was the spirit of the holy gods,\lebnote{“and who \textit{the} spirit of gods holy in him”} and I related the dream to him.\lebnote{“and the dream before him I told”}}%
\verse{“‘O Belteshazzar, chief of the magicians,\lnCQT{} I myself know that the spirit\lebnote{Or “a spirit”} of the holy gods is in you, and no mystery is too difficult\lebnote{“every mystery is not difficult”} for you. Now tell me the visions of my dream that I saw, and its explanation.\lnCQU{}}%
\verse{Now these were the visions of my head as I was lying on my bed: I was gazing\lebnote{“seeing”} and, look, a tree was in the midst of the earth, and its height was exalted.}%
\verse{The tree grew and it became strong, and its height reached to heaven,\lnCQV{} and it was visible to the end of the whole earth.\lebnote{“and its appearance \textit{was} to \textit{the} end of the whole earth”}}%
\verse{Its foliage was beautiful, and its fruit abundant, and in it was provision for all. Under it the animals\lnCQW{} of the field sought shade, and in its branches the birds of heaven\lnCQV{} nested, and from it all the living beings were fed.}%
\verse{“‘I was looking in the vision of my head as I lay on my bed, and look, a watcher,\lebnote{Or “angelic being”} and a holy one, came down from heaven.}%
\verse{He cried aloud\lebnote{“in strength”} and so he said: “Cut down the tree and chop off its branches; shake off its foliage and scatter its fruit. Let the animals flee from under it, and the birds from its branches.}%
\verse{But the stump of its roots leave in the earth, along with\lebnote{“and with”} a band of iron and bronze; leave it in the grass of the field. And in\lebnote{Or “with”} the dew of heaven let it\lebnote{Or “him,” that is, the king} be watered, and with the animals\lnCQW{} let his lot be in the grass of the earth.}%
\verse{Let his mind be changed from that of a human,\lebnote{Or “the human”} and let the mind of an animal be given to him, and let seven times pass over him.}%
\verse{The sentence is by the decree of the watchers, and the decision by the command of the holy ones, in order that\lebnote{“until matter that”} the living will know that the Most High is sovereign over the kingdom of humankind, and to whomever\lebnote{“to whom that”} he wills he gives it, and he even sets the humblest of men over it.”}%
\verse{“‘This is the dream that I, Nebuchadnezzar the king, saw. Now\lebnote{Or “And”} you, Belteshazzar, declare its explanation, for\lebnote{“all of because that”} all of the wise men of my kingdom were not able to make the explanation known to me, but you are able because the spirit of holy gods is in you.’}%
\verseWithHeading{Daniel Relates and Interprets the Dream}{“Then Daniel, whose name was Belteshazzar, was distressed for some time,\lebnote{“for one hour”} and his thoughts disturbed him. The king answered and he said, ‘Belteshazzar, let the dream and its explanation not disturb you.’ Belteshazzar answered and said, ‘My lord, may the dream and its explanation be for those who hate you and for your enemies.}%
\verse{The tree that you saw, which grew and became strong and its height reached to heaven\lnCQV{} and it was visible to the end of the whole earth,\lebnote{“its appearance to the all the earth”}}%
\verse{and its foliage was beautiful and its fruit abundant, and so there was provision for all in it, and the animals\lnCQW{} of the field lived under it and in its branches nest the birds of heaven,\lnCQV{}}%
\verse{— it is you, O king, who have grown great and you have grown strong, and so your greatness has increased and it has reached to heaven\lnCQV{} and your sovereignty to the end of the earth.}%
\verse{And inasmuch that the king saw the watcher, a holy one coming down from heaven\lnCQV{} and he said, “Cut down the tree and destroy it, but the stump of its root in the earth leave with a band of iron and bronze in the grass of the field, and let it be watered with the dew of heaven and let his lot be with the animals of the field until seven times have passed over him.”\lebnote{“until that seven times \textit{they} pass over him”}}%
\verse{This is the explanation, O king, and it is a decree of the Most High that has come upon my lord the king:}%
\verse{you will be driven away from human society\lebnote{“from the humankind”} and you will dwell\lebnote{“your dwelling will be”} with the animals\lnCQW{} of the field, and you will be caused to graze grass like the oxen yourself,\lebnote{“with respect to you”} and you will be watered with the dew of heaven, and seven periods of time will pass over you until that you have acknowledged that the Most High is sovereign over the kingdom of humankind, and to whom he wills\lebnote{“that he wills”} he gives it.}%
\verse{And in that\lebnote{“And for”} they said to leave alone the stump of the tree’s root, so your kingdom will be restored for you when\lebnote{“from that”} you acknowledge that heaven is sovereign.}%
\verse{Therefore, O king, let my advice be acceptable to you and your sin remove\lebnote{Or “replace”} with righteousness and your iniquity with having mercy on the oppressed, in case there might be a prolongation of your prosperity.’”\lebnote{“if will be prolongation for your prosperity”}}%
\verseWithHeading{God Judges the King for His Hubris}{All this happened\lebnote{“It came upon”} to Nebuchadnezzar the king.}%
\verse{At the end of twelve months he was walking on the roof of the royal palace\lebnote{“the place of the kingdom”} of Babylon.}%
\verse{And the king answered and said, “Is this not the great Babylon which I have built as a royal palace by the strength of my own power, and for the glory of my own majesty?”}%
\verse{While the words were still in the mouth of the king, a voice from heaven came, saying, ‘To you, King Nebuchadnezzar, it is declared that the kingdom has departed from you,}%
\verse{and you will be driven away from human society\lnCQX{} and your dwelling will be with the animals\lnCQW{} of the field and they will cause you to graze the grass like oxen, and seven times will pass over you, until you acknowledge that the Most High is sovereign over the kingdom of humankind and that he gives it to whom he wills.’}%
\verse{Immediately\lebnote{“In that hour”} the word was fulfilled concerning\lebnote{“over”} Nebuchadnezzar, and he was expelled from human society\lnCQX{} and he ate grass like oxen, and his body was bathed with the dew of heaven until his hair was like the hair of an eagle and his nails grew like a bird’s claws.}%
\verseWithHeading{The King Praises and Extols the Most High God}{“But at the end of that period,\lebnote{“the day”} I, Nebuchadnezzar, lifted up my eyes to heaven, and then my reason returned to me; and I blessed the Most High and the one who lives forever\lebnote{“\textit{the one} living forever”} I praised and I honored. “For his sovereignty is an everlasting sovereignty, and his kingdom continues from generation to generation.}%
\verse{And all the dwellers of the earth are regarded as nothing, and he does according to\lebnote{“as”} his desire in the host of heaven and among the dwellers of earth, and there is not one who can hold back his hand, or ask him,\lebnote{“or asks to him”} ‘What are you doing?’}%
\verse{“At that time\lebnote{“In it the time”} my reason returned to me, and also the glory of my kingdom and my majesty and splendor returned to me, and my advisers and my lords\lebnote{Or “nobles”} searched me out, and I was established over my kingdom and abundant greatness was added to me.}%
\verse{Now I, Nebuchadnezzar, praise and exalt and honor the king of heaven, for all his works are truth,\lebnote{Or “true”} and his ways are justice and that he is able to humble those who walk in pride.”}%
\end{biblechapter}%
\begin{biblechapter}% Daniel 5
\verseWithHeading{The Mysterious Writing on the Wall}{Belshazzar the king made a great festival for a thousand of his lords,\lnCQY{} and in the presence of\lnCQZ{} the thousand lords he was drinking wine.\lebnote{Or “the wine”}}%
\verse{When he tasted the wine,\lebnote{“At the taste of the wine”} Belshazzar commanded that they bring the vessels of gold and silver that Nebuchadnezzar his predecessor\lnCRA{} had taken from the temple that was in Jerusalem, so that the king and his lords,\lnCQY{} his wives and his concubines may drink from them.}%
\verse{Then they brought in the vessels of gold that they took from the temple, the house of God that was in Jerusalem, and the king and his lords,\lnCQY{} his wives and his concubines drank from them.}%
\verse{They drank the wine and praised the gods of gold and silver, bronze, iron, wood and stone.\lebnote{All these metals have the definite article that does not have to be translated into English to retain the meaning}}%
\verse{Immediately\lebnote{“In its the hour”} human fingers\lebnote{“fingers of \textit{the} hand of a man”} appeared and they wrote opposite\lnCQZ{} the lampstand on the plaster of the wall of the palace of the king, and the king was watching\lebnote{Or “gazing at”} the palm of the hand that was writing.}%
\verse{Then his face changed and his thoughts terrified him, and his hip joints gave way\lebnote{“the limbs of his hip became loose”} and his knees knocked together.\lebnote{“his knees this to this they knocked together”}}%
\verse{The king cried aloud\lebnote{“with strength”} to bring in the conjurers,\lnCRB{} the astrologers\lnCRC{} and the diviners; the king spoke\lnCRD{} and said to the wise men of Babylon, “Any man that can read this writing and can tell me its explanation will be clothed in purple and will have a necklace of gold hung around his neck and he will rule as third in authority in the kingdom.”}%
\verse{Then all the wise men of the king came in, but they were not able to read the writing or to make known its explanation.\lebnote{Aramaic “the explanation/ interpretation”}}%
\verse{Then the king, Belshazzar, became greatly terrified, and his facial features\lebnote{“shining of his face”} changed upon him, and his lords\lnCQY{} were perplexed.}%
\verse{Because of the words of the king and his lords,\lnCQY{} the queen came into the banqueting hall\lebnote{“the house of the drinking”} and the queen spoke up\lnCRD{} and said, “O king, live forever,\lebnote{“to eternity”} and let not your thoughts terrify you and do not let your facial expressions grow pale.\lebnote{“change”}}%
\verse{There is a man in your kingdom who has the spirit of the holy gods in him.\lebnote{“who \textit{the} spirit of holy gods \textit{is} in him”} And in the days of your predecessor,\lnCRA{} enlightenment and insight and wisdom like the wisdom of the gods was found in him; and, O king, Nebuchadnezzar your predecessor\lnCRA{} appointed him as chief of the magicians,\lebnote{Or “soothsayer-priests”} the conjurers,\lnCRB{} the astrologers,\lnCRC{} and the diviners. Your predecessor\lnCRA{} the king did this}%
\verse{because\lnCRE{} there was found in him an excellent spirit and understanding and insight for interpreting dreams and explaining riddles and solving riddles; that is, in Daniel whom the king named\lebnote{“whom the king gave his name”} Belteshazzar. Now, let Daniel be called and he will tell the explanation.”\lnCRF{}}%
\verseWithHeading{Daniel Deciphers and Interprets the Writing}{Then Daniel was brought in before the king, and the king spoke\lnCRD{} and said to Daniel, “You are Daniel who are one of the exiles\lebnote{“who from sons of the exile”} of Judah\lnCRG{} whom my predecessor,\lnCRA{} the king, brought from Judah.\lnCRG{}}%
\verse{And I have heard that a spirit of the gods is in you and enlightenment and insight and excellent wisdom was found in you.}%
\verse{And now the wise men and the conjurers\lnCRB{} were brought in before me so that they could read this writing in order to make its explanation\lnCRF{} known to me, but they were not able to disclose the explanation\lnCRF{} of the matter.}%
\verse{But I have heard concerning you that you are able to produce interpretations\lebnote{“to interpret interpretations”} and to solve riddles; now if you are able to read the writing and to make known its explanation\lnCRF{} to me, you will be clothed in purple and a necklace of gold will be placed around your neck and you will rule as third in command in the kingdom.”}%
\verse{Then Daniel answered and said before the king, “Let your gifts be for yourself or your rewards give to another; nevertheless, I will read the writing to the king and I will make known to him the explanation.\lnCRF{}}%
\verse{O king,\lebnote{“You, O king”} the Most High God gave the kingdom and the greatness and the glory and the majesty to Nebuchadnezzar your predecessor.\lnCRA{}}%
\verse{And because of the greatness that he gave to him, all the peoples, the nations and languages trembled and feared before him; whomever he wanted he killed, and whomever he wanted he let live, and whomever he wanted he honored, and whomever he wanted he humbled.}%
\verse{But when\lebnote{“as that”} his heart became arrogant and his spirit became hard so as to act proudly, he was deposed from the throne of his kingdom and the\lebnote{Or “his”} glory was taken away from him.\lebnote{“they took away”}}%
\verse{And he was driven away from human society\lebnote{“from the sons of the humankind”} and his mind was made like the animals\lebnote{Aramaic “with the animals”} and his dwelling was with the wild asses; and he was given\lebnote{“they gave him”} grass like oxen to eat, and with\lebnote{Aramaic “from”} the dew of heaven his body was bathed, until he acknowledged that the Most High God is sovereign over the kingdom of humankind, and whoever\lebnote{“to from that”} he wants he sets over it.}%
\verse{“But you his successor,\lebnote{“son”} Belshazzar, have not humbled your heart even though\lnCRE{} you knew all this.}%
\verse{And now you have exalted yourself against the Lord of heaven, and the vessels of his temple you have brought in before you, and you and your lords,\lnCQY{} your wives and your concubines, have been drinking wine from\lebnote{Aramaic “with”} them, and you have praised the gods of silver, gold, bronze, iron, wood, and stone\lebnote{All the materials in this list have the definite article that does not have to be translated into English} that do not see and do not hear and do not know, but the God who holds your life in his hand\lebnote{“who your breath \textit{is} in his hand”} and all of your ways come from him,\lebnote{“\textit{belong} to him”} you have not honored.}%
\verse{So then the palm of the hand was sent out from his presence and this writing was inscribed.}%
\verse{“Now this was the writing that was inscribed: ‘Mene, Mene, Tekel and Parsin.’}%
\verse{“This is the explanation of the matter:\lebnote{Or “word, message”} ‘Mene’ — God has numbered your kingdom and brought an end to it.}%
\verse{“‘Tekel’ — you have been weighed on scales and you have been found wanting.\lebnote{Or “deficient”}}%
\verse{“‘Peres’ — your kingdom has been divided and given to the Medes and Persians.’”}%
\verse{Then Belshazzar commanded, and they clothed Daniel with purple and placed a necklace of gold around his neck, and they made a proclamation concerning him that he would be the third ruler in authority in the kingdom.}%
\verse{That same night\lebnote{“In it, in the night”} Belshazzar, the Chaldean king, was killed.}%
\verse{\lebnote{Daniel 5:31–6:28 in the English Bible is 6:1–29 in the Hebrew Bible} And Darius the Mede received the kingdom when he was about sixty-two years old.\lebnote{“when a son of sixty and two years”}}%
\end{biblechapter}%
\begin{biblechapter}% Daniel 6
\verseWithHeading{Daniel’s Integrity and His Entrapment by His Enemies}{It pleased Darius,\lebnote{“It was pleasant before Darius”} and he set up one hundred and twenty satraps over the kingdom, that they were throughout the whole kingdom,\lebnote{“in all the kingdom”}}%
\verse{and over them were three administrators, of whom Daniel was one, so that these satraps were giving account to them, and the king would not be suffering loss.}%
\verse{Then Daniel began distinguishing himself\lebnote{“was distinguishing himself”} above the administrators and the satraps because\lnCRH{} an exceptional spirit was in him, and so the king planned to appoint him over the whole kingdom.}%
\verse{Then the other administrators and satraps began to seek\lebnote{“were seeking”} to find a pretext against Daniel in connection with the kingdom,\lebnote{“from the side of the kingdom”} but they were not able to find any pretext and corruption\lnCRI{} because\lnCRJ{} he was trustworthy, and no\lebnote{Aramaic “any”} negligence or corruption\lnCRI{} could be found in him.\lebnote{“not could be found”}}%
\verse{Then these men said,\lebnote{“\textit{were} saying”} “We will not find any pretext\lebnote{Or “basis for accusation”} against this Daniel unless we find it in connection with the law of his God.”}%
\verse{So the administrators and the satraps conspired with respect to\lebnote{“on/upon”} the king and so they said to him, “Darius, O king, live forever!\lebnote{“to eternity”}}%
\verse{All of the administrators of the kingdom, and the prefects, the satraps, the counselors and the governors took counsel and have agreed to establish an edict of the king and to enforce a decree that whoever\lebnote{“all that”} will seek a prayer from any god or human except from you for up to thirty days will be thrown into the lion pit.\lnCRK{}}%
\verse{Now, O king, establish the edict and you must sign the document so that it cannot be changed, according to the law of the Medes and Persians which cannot be revoked.”}%
\verse{So\lebnote{“Like to before this”} the king, Darius, signed the writing and the interdict.\lebnote{Or “the writing that is the interdict”}}%
\verse{Now when\lebnote{“as that”} Daniel realized that the document was signed, he went to his house (now he had windows in his upper room that were open\lebnote{“and windows were open to him in his upper room”} toward Jerusalem), and three times daily\lebnote{“in the day”} he knelt on his knees and prayed and gave praise before his God, just as\lnCRJ{} he had been doing previously.\lebnote{“from before this”}}%
\verse{Then these men came as a group and they found Daniel praying and pleading for mercy before his God.}%
\verse{Then they approached and spoke with the king\lebnote{“and saying before the king”} concerning the edict of the king, “Did you not sign an edict\lebnote{“Not an edict you signed?”} that any person who would seek anything from any God or human within thirty days except from you, O king, would be thrown into the lion pit?”\lnCRK{} The king answered and said, “The matter as you have just stated is certain according to the law of the Medes and Persians which cannot be revoked.”}%
\verseWithHeading{God’s Miraculous Deliverance of Daniel Again}{Then they responded\lebnote{“they answered”} and said before the king, “Daniel, who is from the exiles\lebnote{“the sons of the exile”} of Judah,\lebnote{Aramaic “Jehud”} is not paying any attention\lebnote{“not he sets \textit{his mind}”} to you, O king, or to the decree that you have signed, and three times daily\lebnote{“and three times on the day”} he says his prayer.”}%
\verse{Then the king, when\lebnote{Aramaic “as ”} he heard that report,\lebnote{“word”} he was extremely distressed over it; and concerning Daniel\lebnote{“upon Daniel”} he was determined\lebnote{“he set \textit{his} heart”} to rescue him. And until the setting of the sun he was making every effort to deliver him.}%
\verse{Then these men came as a group to the king and said,\lebnote{“saying”} “Recall, O king, that with respect to the law of\lebnote{Aramaic “to”} the Medes and Persians that any\lebnote{“all/every”} decree or edict that the king establishes cannot be changed.”}%
\verse{Then the king gave the command, and Daniel was brought in and they threw him into the lion pit.\lnCRK{} The king said\lebnote{“The king answered and saying”} to Daniel, “Your God, whom you serve faithfully,\lebnote{“in the loyalty”} may he rescue you!”}%
\verse{And a\lebnote{Aramaic “one”} stone was brought and it was put on the entrance of the pit, and the king sealed it with his signet ring and with the signet rings of his lords,\lebnote{Or “nobles”} so that nothing would be changed concerning Daniel.}%
\verse{Then the king went to his palace and spent the night in fasting, and no food was brought in before him and his sleep fled from him.}%
\verse{Then the king got up at daybreak, at first light, and he went in haste\lebnote{“to hasten”} to the lion pit.\lnCRK{}}%
\verse{And when he came near\lebnote{“as coming near him”} to the pit, he cried out to Daniel with a distressed voice, and the king spoke\lebnote{“answered”} and said to Daniel, “O Daniel, servant of the living God, your God whom you serve faithfully, was he able to rescue you from the lions?”}%
\verse{Then Daniel spoke to\lebnote{Aramaic “with”} the king, “O king, live forever!\lebnote{“unto endless ages”}}%
\verse{My God sent his angel and he shut the mouth of the lions and they did not\lebnote{Aramaic “would not”} hurt me, because\lnCRH{} before him I was found\lebnote{“it was found for me”} blameless, and also before you, O king, I have not done any wrong.”}%
\verse{Then the king was exceedingly glad over it\lebnote{That is, the situation} and commanded that Daniel be lifted up from the pit; and there was not any wound found on him, because he had trusted in his God.}%
\verse{The king then commanded, and these men were brought who had accused Daniel,\lebnote{“they ate his pieces of Daniel”} and they threw them and their children and their wives into the lion pit,\lnCRK{} and they had not reached the floor of the pit before\lebnote{“until that”} the lions had overpowered them and they had crushed all of their bones.}%
\verseWithHeading{The King Praises the God of Daniel and Daniel Prospers}{Then Darius the king wrote to all the people, the nations, and the languages living in the whole earth, “May your prosperity become great!}%
\verse{I make a decree\lebnote{“From before me is put forth a decree”} that in all the dominion of my kingdom people will be trembling and fearing before the God of Daniel, for he is the living God and endures forever\lebnote{“to eternity/endless ages”} and his\lebnote{“and whose”} kingdom is one that will not be destroyed and his dominion has no end.\lebnote{“dominion his to the end”}}%
\verse{He is rescuing, delivering, and working signs and wonders in the heavens and on earth, for he has rescued Daniel from the power\lebnote{“hand”} of the lions.”}%
\verse{So this Daniel prospered during the kingdom of Darius and during the kingdom of Cyrus the Persian.}%
\end{biblechapter}%
\begin{biblechapter}% Daniel 7
\verseWithHeading{Daniel’s Vision of Four Beasts and the Son of Man}{In the first year of Belshazzar the king of Babylon, Daniel saw a dream and visions\lebnote{Or “the visions”} of his head as he lay on his bed; then he wrote down the dream and the summary\lebnote{“head”} of the words as follows:\lebnote{“saying”}}%
\verse{Daniel explained\lebnote{“answered”} and said, “I was looking in my vision in the night, and look, the four winds of heaven were stirring up the great sea.}%
\verse{And four great beasts were coming up from the sea, differing from one another.\lebnote{“this from this”}}%
\verse{The first was like a lion and had\lebnote{“for him/it”} the wings of an eagle. I was watching until its wings were plucked off, and it was lifted up from the earth and it was raised on its feet like a human, and a human heart was given to it.}%
\verse{And look, another beast, a second one, looking like a bear. And it was raised up on one side and three tusks\lebnote{Or “ribs”} were in its mouth between its teeth, and so it was told,\lebnote{“they were saying”} “Arise, eat much flesh!”}%
\verse{After this I was watching and look, another beast like a leopard; it had four wings\lebnote{“and \textit{there were} to it four wings”} of a bird on its back, and the beast had\lebnote{“to the beast”} four heads, and dominion was given to it.}%
\verse{After this in the visions of the night I was looking and there was a fourth beast, terrifying and frightful and exceedingly strong, and it had\lebnote{“\textit{were} to it”} great iron teeth,\lebnote{“teeth of iron for it great”} and it was devouring and crushing, and it stamped the remainder\lebnote{Or “residue”} with its feet; and it was different from all the other beasts that preceded it\lebnote{“that preceding her/it”} and it had ten horns.\lebnote{“and ten horns for her/it”}}%
\verse{I was considering the horns, and look, another little horn came up among\lebnote{“between”} them, and three of the earlier horns were rooted out\lebnote{“dehorned”} from before it, and there were eyes like the eyes of a human\lebnote{Or “humankind”} in this horn and also a mouth that was speaking boastfully.}%
\verse{“I continued watching\lnCRL{} until thrones were placed and an Ancient of Days sat; his clothing was like white snow and the hair of his head was like pure wool and his throne was a flame\lebnote{Or “ablaze with”} of fire and its wheels\lebnote{Aramaic “wheel”} were burning fire.}%
\verse{A stream of fire issued forth and flowed from his presence;\lebnote{“from before him”} thousands upon thousands served him and ten thousand upon ten thousand stood before him. The judge sat, and the books were opened.}%
\verse{“I continued watching\lnCRL{} then\lebnote{Aramaic “in then”} because of the noise of the boastful words of the horn who was speaking; I continued watching\lnCRL{} until the beast was slain and its body was destroyed, and it was given over to burning with fire.}%
\verse{And as for the remainder of the beasts, their dominion was taken away,\lebnote{“they took away their dominion”} but a prolongation of their life\lebnote{“in life”} was given to them for a season and a time.}%
\verse{“I continued watching\lnCRL{} in the visions of the night, and look, with the clouds of heaven one like a son of man\lebnote{Or “mortal,” or “son of humankind”} was coming, and he came to the Ancient of Days, and was presented\lebnote{“they presented him”} before him.}%
\verse{And to him was given dominion and glory and kingship that\lebnote{Aramaic “and”} all the peoples, the nations, and languages would serve him; his dominion is a dominion without end\lebnote{“of eternity”} that will not cease, and his kingdom is one that will not be destroyed.}%
\verseWithHeading{Daniel Explains and Interprets the Vision}{“As for me, Daniel,\lnCRM{} my spirit was troubled within me,\lebnote{“in the midst of sheath”} and the visions of my head terrified me.}%
\verse{So I approached one of the attendants and I asked him\lebnote{Aramaic “from him”} about the truth concerning all this; and he told me\lebnote{Aramaic “to me”} that he would make known to me the explanation of the matter.}%
\verse{‘These great beasts which are four in number are four kings who will arise from the earth.}%
\verse{But the holy ones\lnCRN{} of the Most High will receive the kingdom, and they will take possession of the kingdom forever, forever and ever.’\lebnote{“forever and to forever”}}%
\verse{“Then I desired to make certain concerning the fourth beast that was different from all the others\lebnote{“all \textit{of} them”} — exceedingly terrifying, with its iron teeth and its claws of bronze; it devoured and crushed and stamped the remainder with its feet —}%
\verse{and concerning the horns that were on its head, and concerning the other horn that came up and before which three horns fell,\lebnote{“and fell from before it three”} and this horn had eyes\lebnote{“and eyes to it”} and a mouth speaking boastfully,\lebnote{“abundantly”} and its appearance was larger than its companions.}%
\verse{I continued watching,\lnCRL{} and this horn made war with the holy ones\lnCRN{} and it prevailed over them,}%
\verse{until the Ancient of Days came and gave judgment to the holy ones\lnCRN{} of the Most High; and the time arrived and the holy ones\lnCRN{} took possession of the kingdom.}%
\verse{“And he said, ‘The fourth beast is the fourth kingdom that will be on the earth that will be different from all the other kingdoms, and it will devour the whole earth and it will trample it and it will crush it.}%
\verse{And as for the ten horns coming from it, from this kingdom\lebnote{Or “the kingdom”} ten kings will arise, and another will arise after them. And he will be different from the earlier ones, and he will subdue three kings.}%
\verse{And he will speak words against the Most High, and he will wear out the holy ones\lnCRN{} of the Most High, and he will attempt to change times and law, and they will be given into his hand for a time and two times and half a time.}%
\verse{Then the court will sit, and his dominion will be removed,\lebnote{Or “taken away”} to be eradicated and to be destroyed totally.\lebnote{“until the end”}}%
\verse{And the kingdom and the dominion and the greatness of the kingdoms under the whole heaven will be given to the nation of the holy ones\lnCRN{} of the Most High; his kingdom is an everlasting kingdom,\lebnote{“\textit{is} a kingdom of eternity”} and all the dominions will serve and obey him.’}%
\verse{This is the end of the account.\lebnote{“Up to here \textit{is} the end of the account”} As for me, Daniel\lnCRM{} — my thoughts terrified me greatly and my face changed over me, but I kept the matter in my heart.”\lebnote{Or “to myself”}}%
\end{biblechapter}%
\begin{biblechapter}% Daniel 8
\verseWithHeading{The Vision of a Small Horn that Oppresses Israel and Threatens the Temple}{In the third year of the kingdom of Belshazzar the king, a vision appeared to me, Daniel,\lebnote{“I Daniel”} after the one that appeared to me previously.\lebnote{“in the beginning”}}%
\verse{And I saw in the vision, and when I saw,\lebnote{“and it was at my seeing”} I was in Susa, the citadel that was in the province of Elam. And I saw in the vision, and I myself\lebnote{“I, I was”} was at the stream of Ulai.\lebnote{Or “Ulai Canal” (NASB); “stream of the Ulai” (NEB)}}%
\verse{And I lifted up my eyes and I saw, and look! A ram\lebnote{Hebrew “One ram”} standing before\lnCRO{} the stream, and it had\lebnote{“for it”} two horns, and the horns were long, but the one was longer than the second, and the longer one came up after the other one.\lebnote{“in the behind of the other”}}%
\verse{I saw the ram charging westward and northward and southward, and none of the beasts stood before it,\lebnote{“all \textit{the} beasts not they withstood to the face of him”} and there was no rescuing from its power,\lebnote{“its hand”} and it did what it wanted\lebnote{“according to its will”} and it became strong.\lebnote{Or “it/he magnified itself/himself”}}%
\verse{And I was considering this, and look, a he-goat coming from the west across the face of the whole earth, and it was not touching the ground;\lebnote{Or “earth,” or “land”} and the he-goat had a conspicuous horn\lebnote{“a horn of vision”} between its eyes.}%
\verse{Then it came toward the ram that had the two horns\lnCRP{} that I saw standing before\lnCRO{} the stream, and it ran at it with\lebnote{Or “in”} the rage of its power.}%
\verse{And I saw it approaching the ram and it was furious at it, and it struck the ram, and it broke its two horns, and there was not strength in the ram to stand before him,\lebnote{“and not was strength in the ram to withstand to the face of him”} and he threw it down to the ground and trampled it, and there was no one who could rescue the ram\lebnote{“and not he/it was one rescuing”} from its power.}%
\verse{And the he-goat grew exceedingly great,\lebnote{“until much”} and at the height of its power\lebnote{“and at his being powerful”} the great horn was broken, and four conspicuous horns\lebnote{LXX reads “others” here for “conspicuous.” The words in Hebrew can be confused for each other} came up in place of it toward the four winds of heaven.}%
\verse{And from one of them\lebnote{“And from the one from them”} came forth a horn, a little one,\lebnote{“one from little”} and it grew exceedingly\lebnote{Or “pre-eminently”} toward the south, and toward the east, and toward the beautiful land.}%
\verse{And it grew up to the host of heaven, and it threw down to the ground some of the host and some of the stars and trampled them.}%
\verse{Even against the prince of the hosts it acted arrogantly and took away from him the regular burnt offering, and the place of his sanctuary was overthrown.}%
\verse{And on account of transgression, the host was given over to the horn in addition to the regular burnt offering. And it cast down truth to the ground, and it acted,\lnCRQ{} and it had success.}%
\verse{And I heard a certain holy one speaking, and a certain other holy one said to the specific one who was speaking,\lebnote{“the \textit{one} speaking”} “For how long\lebnote{“Until when”} is the vision concerning the regular burnt offering, and the transgression that makes desolate, and the giving over of the sanctuary and the host to trampling?”}%
\verse{And he said to me, “For two thousand three hundred evenings and mornings, then the sanctuary will be restored.”}%
\verseWithHeading{Gabriel Gives Angelic Help and Interprets the Vision}{And then\lebnote{“And it happened”} when I, Daniel, saw the vision, and I was seeking understanding, there was one standing before me with the appearance of a man.}%
\verse{And I heard the voice of a human at the Ulai, and he called and said, “Gabriel, make this man understand the vision.”}%
\verse{And he came beside where I was standing,\lebnote{“my standing”} and when he came\lebnote{“at his coming”} I became terrified and I fell prostrate on my face. And he said to me, “Understand, son of man,\lebnote{Or “mortal,” or “son of humankind”} that the vision is for the time of the end.”}%
\verse{And when he spoke\lebnote{“at his speaking”} with me I fell into a trance with my face to the ground, and he touched me and made me stand on by feet.}%
\verse{And he said, “Look, I am making known to you what will happen in the period of wrath,\lebnote{Or “the wrath”} for it refers to the appointed time of the end.}%
\verse{“The ram that you saw who had two horns\lnCRP{} represents the kings of Media and Persia.}%
\verse{“And the hairy he-goat is the kingdom\lebnote{“king”} of Greece, and the great horn that is between his eyes — he is the first king.}%
\verse{And the horn that was broken, and then there arose four horns in place of it — these are four kingdoms that will arise from his nation, but not with his power.}%
\verse{And at the end of their kingdom, when the transgressions are completed, a king will arise, fierce in countenance and skilled in riddles.}%
\verse{And his power will grow, but not by his own power, and he will cause fearful destruction, and he will succeed and he will act,\lebnote{“he will do”} and he will destroy the mighty and the people of the holy ones.\lebnote{Or “saints”}}%
\verse{And by his planning\lebnote{Or “cunning”} he will make a success of deceit by\lebnote{Hebrew “in”} his hand, and in his mind\lebnote{“heart”} he will boast,\lebnote{“magnify himself”} and in their ease he will destroy many, and even against the prince of princes he will rise up, and he will be broken, but not by human hands.}%
\verse{And the vision of the evening and the morning that has been described,\lebnote{“told”} it is true; and you, seal up the vision, for it refers to many days to come.”\lebnote{“for to days many”}}%
\verse{And I, Daniel, was overcome, and I became ill for some days, and I performed\lnCRQ{} the business of the king, and I was dismayed over the vision and I did not understand it.\lebnote{“there was not understanding”}}%
\end{biblechapter}%
\begin{biblechapter}% Daniel 9
\verseWithHeading{Daniel’s Prayer for His People}{In the first year of Darius, the son of Ahasuerus,\lebnote{That is, Xerxes} from the offspring of the Medes, who became king over the kingdom of the Chaldeans —}%
\verse{in the first year of his kingship I, Daniel, observed in the scrolls the number of the years that it was that were to be fulfilled according to the word of Adonai to Jeremiah the prophet for the devastation of Jerusalem — seventy years.}%
\verse{Then I turned my face to the Lord God to seek him by prayer and pleas for mercy,\lebnote{Or “supplications”} in fasting and in sackcloth and ashes.}%
\verse{And I prayed to Adonai my God, and I made confession and I said, “O Lord, the great and awesome God, keeping the covenant and loyal love with those who love him and with those who keep his commandments,}%
\verse{we have sinned and we have done wrong and we acted wickedly and we rebelled and have been turning aside from your commandments and from your ordinances.}%
\verse{We have not listened to your servants the prophets, who spoke in your name to our kings, our princes and our ancestors\lnCRR{} and to all the people of the land.}%
\verse{“Righteousness belongs to you, O Lord,\lebnote{“for you Lord the righteousness”} and on us is open shame, just as it is this day to the people of Judah and to the inhabitants of Jerusalem and to all Israel, those who are near and those who are far off\lebnote{“those near and those far”} in all the lands to which you have driven them,\lebnote{Hebrew “\textit{to} which you have driven them there”} because of their infidelity which they displayed against you.}%
\verse{Adonai, on us is open shame, on our kings, on our princes, and on our ancestors,\lnCRR{} because we have sinned against you.}%
\verse{Compassion and forgiveness belong to the Lord, our God,\lebnote{“to \textit{the} Lord our God the compassion and the forgiveness”} for we have rebelled against him,}%
\verse{and we have not listened to the voice of Adonai our God, by following his law which he placed before us\lebnote{“which he gave to the face of us”} by the hand of his servants the prophets.}%
\verse{“And all Israel transgressed your law and turned aside so as not to listen to your voice, and so the curse and the oath which was written in the law of Moses, the servant of God, has been poured out upon us, because we have sinned against him.}%
\verse{And so he has carried out his words which he spoke against us and against our rulers who ruled us, to bring upon us a great calamity which was not done under all of heaven as it was done in Jerusalem.}%
\verse{Just as it is written in the law of Moses, all of this calamity has come upon us, and we have not implored\lebnote{“we not implored”} the face of Adonai our God so as to turn from our iniquities and to attend closely to your faithfulness.\lebnote{Or “truth”}}%
\verse{So Adonai has kept watch over the\lebnote{Or “his”} calamity, and now he has brought it upon us. Indeed, Adonai our God is righteous concerning all his works\lebnote{Or “acts”} that he has done, but we have not listened to his voice.}%
\verse{“And now, Lord our God, who have brought your people out from the land of Egypt with a strong hand, and you have made for yourself a name until this day — we have sinned, we have acted wickedly.}%
\verse{Lord, according to all your righteousness, please let your anger and your rage turn away from your city Jerusalem, your holy mountain,\lebnote{“the mountain of your holiness”} because through our sins and through the iniquities of our ancestors\lnCRR{} Jerusalem and your people have become an object of mockery among all of our neighbors.}%
\verse{“And now, listen to the prayer of your servant and to his pleas for mercy. Shine\lebnote{Or “let shine” or “cause to shine”} your face upon your desolate sanctuary for your sake, O Lord.\lebnote{“for the sake of the Lord”}}%
\verse{Incline your ear, my God, and listen; open your eyes and look at our desolation and the city that is called by your name, for we are not presenting our pleas for mercy before you\lebnote{“to the face of you”} because of our righteousness, but rather because of your great compassion.}%
\verse{Lord, listen! Lord, forgive! Lord, pay heed and act! You must not delay for your sake,\lebnote{“for the sake of you”} my God; because your city and your people are called by your name.”\lebnote{“your name is called over your city and over your people”}}%
\verseWithHeading{Gabriel’s Announcement and Presentation of the Seventy Sets of Seven}{Now I was still speaking and praying, and confessing my sin and the sin of my people Israel, and presenting my plea before\lebnote{“to the face of”} Adonai my God, on behalf of the holy mountain of my God.\lebnote{“the mountain of the holiness of my God”}}%
\verse{And I was still speaking in prayer, and the man Gabriel, whom I had seen in the vision previously,\lebnote{“in the beginning”} in my weariness touched me\lebnote{“\textit{my} being wearied, touching me”} at the time of the evening offering.}%
\verse{And he instructed me and he spoke with me and he said, “Daniel I have now come out\lebnote{Or “forth”} to teach you understanding.}%
\verse{At the beginning of your pleas for mercy, a word went out, and now I have come to declare\lebnote{Or “deliver”} it, for you are highly esteemed, and so consider the word and understand the vision.}%
\verseWithHeading{The Seventy Sets of Seven Detailed}{“Seventy weeks is decreed for your people and for your holy city,\lebnote{“for \textit{the} city of your holiness”} to put an end to the transgression and to seal up sin\lebnote{Or “to make an end to sin”} and to make atonement for guilt and to bring in everlasting righteousness and to seal vision and prophet and to anoint the most holy place.\lebnote{“the holy place of holy places”}}%
\verse{And you must know and you must understand\lebnote{Or “so you are to know and discern” (NASB), or “know then and understand” (e.g. NEB)} that from the time of the going out of the word to restore and build Jerusalem until an anointed\lnCRS{} one — a leader — will be seven weeks and sixty-two weeks;\lebnote{“weeks seven and weeks sixty and two”} it will be restored and will be built with streets and a moat, but in a time of oppression.\lebnote{“in distress/oppression of the times/time”}}%
\verse{“And after the sixty and two weeks an anointed one\lnCRS{} shall be cut off, and he shall have nothing,\lebnote{“there is not for him”} and the people of the coming leader will destroy the city and the sanctuary,\lebnote{Or “holy place”} and its end will be with the flood and on to the end there shall be war; these desolations are determined.}%
\verse{And he will make a strong covenant with the many for one week, but in half of the week he will let cease sacrifice and offering and in its place\lebnote{“and on a wing of”} a desolating abomination comes even until the determined complete destruction\lebnote{“a complete destruction and being determined”} is poured out on the desolator.”}%
\end{biblechapter}%
\begin{biblechapter}% Daniel 10
\verseWithHeading{Daniel’s Vision of a Dazzling Heavenly Messenger}{In the third year of Cyrus the king of the Persians, a word\lnCRT{} was revealed to Daniel, who was called by his name Belteshazzar, and the word\lnCRT{} was reliable and it concerned a great tribulation,\lebnote{Or “conflict”} and he understood the word\lnCRT{} and he received understanding.\lebnote{“and understanding to him”}}%
\verse{In those days, I, Daniel, I myself\lnCRU{} was in mourning for three whole weeks.\lebnote{“three weeks \textit{of} days”}}%
\verse{I had not eaten any choice food, and meat and wine did not enter my mouth, and I did not use any ointment\lebnote{“I have not anointed at all”} until the end of three whole weeks.\lebnote{“until being full/complete three weeks \textit{of} days”}}%
\verse{And then on the twenty-fourth day of the first month, I myself\lnCRU{} was on the bank of the great river; that is, the Tigris.}%
\verse{And I lifted up my eyes and I saw,\lebnote{Or “looked”} and there was a man, and he was dressed in linen, and his waist was girded with the gold of Uphaz.}%
\verse{Now his body was like turquoise,\lebnote{Or “yellow jasper” or other gold-colored stones; the exact identity of ancient precious stones is difficult} and his face was like the appearance of lightning, and his eyes were like torches of fire, and his arms and his legs were like the gleam of polished bronze, and the sound of his words was like the sound of a multitude.}%
\verse{And I saw, I, Daniel alone,\lebnote{“to alone me”} the vision; and the people who were with me did not see\lebnote{“not they saw”} the vision; nevertheless, a great trembling fell upon them and they fled in order to hide themselves.}%
\verse{And I myself, Daniel, alone saw this great vision, and as a result no strength was left in me\lebnote{“not was left in me strength”} and my complexion grew deathly pale,\lebnote{“and complexion my changed over me to destruction”} and I did not retain any strength.}%
\verse{And I heard the sound of his words, and when I heard\lebnote{“at my hearing”} the sound of his words I myself began falling into a trance on my face, with my face to the ground.\lebnote{“and my face toward the ground”}}%
\verse{And look, a hand touched me and it roused me to my knees\lebnote{Hebrew “on my knees”} and the palms\lebnote{Hebrew “palm”} of my hands.}%
\verse{And he said to me, “Daniel, a man beloved, pay attention to the words that I am speaking to you and stand upright where you are,\lebnote{“stand on your position”} for I have now been sent to you.” And while he was speaking\lnCRV{} with me this word,\lnCRT{} I stood up trembling.}%
\verse{And he said to me, “You must not fear, Daniel, for from the first day that you set your heart to understand and to humble yourself before\lebnote{“to the face of”} your God, your words were heard, and I myself\lnCRU{} have come because of your words.}%
\verse{But the prince of the kingdom of Persia stood before me\lebnote{“to front of me”} for twenty-one days. And look,\lebnote{Or “finally”} Michael, one of the chief princes, came to assist me, and I left him there beside the king of the Persians.}%
\verse{And I have come\lebnote{Hebrew “I came”} to instruct you about what\lebnote{Hebrew “that”} will happen to your people in the future,\lebnote{“at the end of the days”} for there is a further vision here for the future.\lebnote{“for the days”}}%
\verse{And while he was speaking with me according to these words, I turned my face toward the ground and I was speechless.}%
\verse{Then look, there was one in the form of a human;\lebnote{“as \textit{the} likeness of \textit{a} son of man”} he touched my lips and I opened my mouth and I spoke and I said to the one standing before me,\lebnote{“to opposite me”} “My lord, because of the vision my anxieties fell upon me and I have not retained my strength.}%
\verse{So how am I, a servant of my lord\lebnote{“how is \textit{he} able a servant \textit{of} my lord this”} to speak with you, my lord,\lebnote{“with my lord this”} and I just now\lebnote{“from now”} have no strength in me?”\lebnote{“not it remains in me strength”}}%
\verse{And he again touched\lebnote{“he added to and touched”} me, the one in the form of a human,\lebnote{“like vision of a man”} and he strengthened me.}%
\verse{And he said, “You must not fear, O beloved man. Peace be to you;\lebnote{“Peace to/for you”} be strong and be courageous!” And when he spoke\lnCRV{} with me, I was strengthened and I said, “Let my lord speak, for you have strengthened me.”}%
\verse{Then he asked, “Do you know why I have come to you? And now I return to fight against the prince of Persia and I myself\lnCRU{} am going, and look,\lebnote{Or “listen”} the prince of Javan\lebnote{That is, Greece} will come.}%
\verse{However, I will tell you what is inscribed\lebnote{“the being inscribed”} in the book of truth, and there is not one who contends with me against these beings except\lebnote{“but if”} Michael, your prince.”}%
\end{biblechapter}%
\begin{biblechapter}% Daniel 11
\verseWithHeading{A Survey of the Kings of the North and South}{“And I, in the first year of Darius the Mede, I stood\lebnote{“my standing”} as a support and as a protection for him.}%
\verse{And now I will reveal the truth to you. Look! Still three kings will arise in Persia, and the fourth will get abundance and great wealth, even more than all of them, and when he becomes strong\lebnote{“at his strength”} through his wealth, then he will stir up everyone\lebnote{“the all”} against the kingdom of Javan.\lebnote{That is, Greece}}%
\verse{And a mighty king will arise, and he will rule with great dominion, and he will do as he pleases.\lnCRW{}}%
\verse{But as he rises in power,\lebnote{“as his rising/to rise”} his kingdom will be divided toward the four winds of heaven, but not to his posterity, and not according to his dominion by which he ruled, for his kingdom will be uprooted and be given to others besides them.\lebnote{“to others from to alone these”}}%
\verse{“Then the king of the south will grow strong and also one of his officials,\lebnote{“from among his officials”} and he will grow stronger than him\lebnote{“over him”} and he will rule a dominion greater than his dominion.\lebnote{“a dominion great dominion his”}}%
\verse{And at the end of some years they will make an alliance, and the daughter of the king of the south will come to the king of the north to make a peace treaty, but she will not retain her position of power,\lebnote{“she will not retain the power of her arm”} and his offspring\lebnote{Or “power”} will not endure, and she will be given up, she and her attendants and her child supporting her, in those times.\lebnote{Hebrew “in the times”}}%
\verse{And a branch from her roots\lebnote{That is, a member of her family} will rise up in his place, and he will come against the army and he will enter the stronghold of the king of the north and he will take action against them and he will prevail.}%
\verse{And also their gods with their idols and with the precious vessels,\lebnote{“vessels of their desire”} silver and gold he will take to Egypt into captivity, and for years he will leave the king of the north alone.\lebnote{“he will stand back from the king of the north”}}%
\verse{And then he\lebnote{That is, the king of the north} will come into the kingdom of the king of the south, but he will return to his land.}%
\verse{“But his sons will wage war and they will gather a multitude of great forces and he will advance with great force,\lebnote{“he will come, to come”} and he will overflow like a flood and he will pass through and he will return, and they will wage war up to his fortress.}%
\verse{And the king of the south will become furious, and he will go and he will battle against him, against the king of the north; and he will muster a great multitude, and the multitude will be given into his hand.}%
\verse{When the multitude is carried off, his heart will be exalted and he will overthrow tens of thousands,\lebnote{“myriads”} but he will not prevail.}%
\verse{And the king of the north will again raise a multitude, greater than the former, and at the end of some years\lebnote{“to the end of the times years”} he will surely come with a great army and with great supplies.\lebnote{Or “resources”}}%
\verse{“And in these times many will rise up against the king of the south, and the violent ones of your people\lebnote{“the sons of the lawless ones of your people”} will lift themselves to fulfill\lebnote{“to cause to stand”} the vision, but they will fall.}%
\verse{And the king of the north will come, and he will throw up siege ramps\lebnote{Hebrew “ramp”} and capture a city of fortifications and the military forces of the south and his choice troops\lebnote{“the army of his chosen troops”} will not stand, for\lnCRX{} there is no strength left to resist.}%
\verse{And the one coming to him will act according to\lebnote{“as”} his pleasure, and there is no one who will stand\lebnote{“standing”} before him,\lnCRY{} and he will stand in the beautiful land\lebnote{“in the land of the beauty”} and complete destruction will be in his power.}%
\verse{And he will set his face to come with the authority of his whole kingdom and will form an agreement;\lebnote{Or “treaty,” or “peace proposal”; the Hebrew is difficult} and he will act,\lebnote{“do, perform”} and the daughter of women he will give to him to destroy it,\lebnote{Or “her”} but the ploy will not succeed and she will not support him.\lebnote{“and not for him will she be”}}%
\verse{And he will turn his face to the coastlands, and he will capture many, but\lnCRX{} a commander will end his insults to him so that instead his insults will turn back upon him.\lebnote{“so that not his insults he will return against him”}}%
\verse{And he will turn back his face toward\lebnote{Hebrew “to”} the strongholds of his land, but he will stumble and he will fall and will not be found.}%
\verse{“Then in his place will arise one sending an official throughout the glory\lebnote{Or “splendor”} of his kingdom, and in a few days\lebnote{“in some days”} he will be broken, but not in anger and not in battle.}%
\verse{And in his place a despicable person will arise on whom\lebnote{Hebrew “him”} they have not conferred the majesty of the kingdom, and he will come in without warning\lebnote{“in ease”} and he will seize the kingdom by deceit.}%
\verse{And before him\lnCRY{} mighty military forces\lebnote{“military forces of the flood”} will be utterly swept away, and they will be broken, and also the leader of the covenant.}%
\verse{And after an alliance is made with him, he will act deceitfully, and he will rise and he will become powerful with few people backing him.}%
\verse{In a time of ease and in the rich parts of the province, he will come and he will do what his predecessors\lebnote{“his fathers and the father of his fathers”} did not do; he will distribute plunder and spoil and possessions to them, and he will devise his plans against fortifications, but only for a time.\lebnote{“until time”}}%
\verse{And he will stir up his power and his heart against the king of the south and with a much greater and stronger army;\lebnote{“with an army great and numerous exceedingly”} but he will not succeed, for they will devise plans against him.}%
\verse{And those who eat of his royal rations will break him and his army will be overwhelmed, and many will fall, slain.}%
\verse{And two of the kings\lebnote{“And two of them, the kings”} will bend their hearts\lebnote{“their heart”} to evil. And at the same table\lebnote{“one table”} they will speak lies, but what is discussed will not succeed, for still an end is coming at the appointed time.}%
\verse{Then he will return to his land with many possessions, but his heart will be set against the holy covenant,\lebnote{“covenant of holiness”} and he will take action and he will return to his land.}%
\verse{“At the appointed time he will return and he will come into the south, but it will not be as it was before.\lebnote{“as the former and as the last”}}%
\verse{And the ships of Kittim will come against him, and he will lose heart, and he will turn back, and he will be enraged against the holy covenant,\lebnote{“against the covenant of holiness”} and he will take action, and he will turn back, and he will pay attention to those who forsake the holy covenant.\lebnote{“\textit{the} covenant of holiness”}}%
\verse{And military forces from him will occupy\lebnote{“will stand forth”} and will profane\lebnote{Or “desecrate”} the sanctuary stronghold,\lebnote{“the sanctuary, the stronghold”} and they will abolish the regular burnt offering, and they will set up the abomination that causes desolation.}%
\verse{“And those who violate the covenant he will seduce with flattery, but the persons who know their God\lebnote{Hebrew “the ones who know his God”} will stand firm and will take action.}%
\verse{And those who have insight will instruct\lebnote{“make understand”} the many, but they will fall by sword and by flame, by captivity and by plunder for some time.\lebnote{“\textit{during} days”}}%
\verse{And when they fall\lebnote{“in/at their falling”} they will receive little help, and many will join with them in hypocrisy.}%
\verse{And even some of those who have insight\lebnote{“the \textit{ones} who produce insight”} will fall\lebnote{Or “stumble”} in order for them to be refined by it, and to be purified and cleansed until the time of the end, for the appointed time is still to come.}%
\verse{“Then the king will do as he pleases,\lnCRW{} and he will exalt himself and will consider himself above any god, and he will speak horrendous things against\lebnote{“over”} the God of gods, yet he will succeed until the period of anger is finished, for what is determined will be done.}%
\verse{He will not pay respect to the gods of his ancestors,\lnCRZ{} or to the darling of women, and not to any god will he pay respect, for he will consider himself great over all gods.}%
\verse{But instead\lebnote{“in his place”} he will honor the god of fortresses, a god whom his ancestors\lnCRZ{} did not know. He will honor him with gold, and with silver, and with precious stones\lebnote{Hebrew “precious stone”} and with costly gifts.}%
\verse{And he will deal with the fortified strongholds\lebnote{“the fortifications of strongholds”} with the help of a foreign god;\lebnote{“\textit{the} god of a foreign land”} and he will increase wealth for whoever will acknowledge him, and he will cause them to rule\lebnote{Or “appoint”} over the many, and he will distribute land for a price.}%
\verse{“And at the time of the end the king of the south will attack him, and the king of the north will storm against him with chariots\lebnote{Hebrew “chariot”} and with horsemen and with many ships, and he will advance against the countries and he will sweep through like a flood.\lebnote{“and he will overflow \textit{them} and he will pass through”}}%
\verse{And he will come into the beautiful land\lebnote{“into the land of the beauty”} and many will fall victim, but these will escape from his power: Edom and Moab and the best part\lebnote{A difficult word to translate in this context: possibly it means “the foremost” (NASB), or it may mean “the remnant” (NEB)} of the Ammonites.\lebnote{“sons/children of Ammon”}}%
\verse{And he will stretch out his hand against countries and the land of Egypt will not escape.\lebnote{“not it will be to escape”}}%
\verse{And he will rule over the treasures of gold and the silver and over all the precious things of Egypt; and the Libyans and the Cushites will follow in his footsteps.}%
\verse{But reports will terrify him from the east and from the north, and he will go out with great fury to destroy and to exterminate many.}%
\verse{And then he will pitch the tents of his palace between the sea and the beautiful holy mountain,\lebnote{“\textit{the} mountain of the beauty holiness”} and he will come to his end, and there is no one\lebnote{Or “not one”} helping him.”}%
\end{biblechapter}%
\begin{biblechapter}% Daniel 12
\verseWithHeading{The Time of the Judgment and the Resurrection of Two Groups}{“Now at that time, Michael, the great prince, will arise, the protector\lebnote{“the one who stands with”} over the sons of your people, and it will be a time of distress that has not been since your people have been\lebnote{“to be” or “being”} a nation until that time. And at that time your people will escape, everyone who is found\lebnote{“all \textit{of} the being found”} written in the scroll.\lnCSA{}}%
\verse{And many from those sleeping in the dusty ground\lebnote{“\textit{of the} ground of dust”} will awake, some to everlasting life\lebnote{“life of eternity”} and some to disgrace and everlasting contempt.\lebnote{“contempt \textit{of} eternity”}}%
\verse{But the ones having insight will shine like the brightness of the expanse, and the ones providing justice for the many\lebnote{Or “leading the many in righteousness”} will be like the stars forever and ever.\lebnote{“to eternity and ever”}}%
\verse{But you, Daniel, keep the words secret and seal the scroll\lnCSA{} until the time of the end; many will run back and forth and knowledge will increase.”}%
\verse{Then I looked, I myself, Daniel, and look, there were two others standing: one on this bank of the stream and one on the other.\lebnote{“one here on the bank of the stream and one there on the bank of the stream”}}%
\verse{Then he said to the man who was clothed in linen who was above\lnCSB{} the water of the stream,\lnCSC{} “How long until\lebnote{“Until when”} the end of the wonders?”}%
\verse{And I heard the man who was clothed in linen who was above\lnCSB{} the water of the stream,\lnCSC{} and he raised his right hand and his left hand to heaven and he swore by the one who lives forever\lebnote{“by the life of the eternity”} that an appointed time, appointed times, and half an appointed time would pass when the shattering of the power of the holy people\lebnote{“\textit{the} power of \textit{the} people of holiness”} would be completed;\lebnote{“to come to an end”} then all these things will be accomplished.}%
\verseWithHeading{Daniel Seeks Additional Insights}{Now I myself heard, but I did not understand, and I said, “My lord, what will be the outcome of these things?”}%
\verse{And he said, “Go, Daniel, for the words are secret and are sealed up until the time of the end.}%
\verse{Many will be purified and will be cleansed and will be refined, but the wicked will act wickedly and none of the wicked will understand,\lebnote{“and not will understand all the wicked”} but those who have insight\lebnote{“the \textit{ones} having insight”} will understand.}%
\verse{And from the time the regular burnt offering is removed and the abomination that causes desolation is set up\lebnote{“to be set up” or “to put in place”} there will be one thousand two hundred and ninety days.}%
\verse{Happy is the one who is persevering, and attains to the one thousand three hundred and thirty-five days.}%
\verse{But you, go on to the end and rest, and you will arise for your allotted inheritance at the end of the days.”}%
\end{biblechapter}%
\flushcolsend
\input{leb/content/old-testament/Hos.tex}\flushcolsend
\input{leb/content/old-testament/Joel.tex}\flushcolsend
\input{leb/content/old-testament/Am.tex}\flushcolsend
\biblebook{Obadiah}
\begin{biblechapter}% Obadiah 1
\verseWithHeading{God’s Judgment on Edom}{The vision of Obadiah. Thus says my Lord Adonai concerning Edom:\innerVerseHeading{Edom’s Approaching Destruction}We have heard a report from Adonai, and a messenger has been sent among the nations: “Rise up and let us rise against it\lebnote{That is, Edom} for battle.”}%
\verse{“Look, I will\lebnote{The NET Bible note for this verse points out: “The Hebrew perfect verb form used here usually describes past events. However, here and several times in the following verses it is best understood as portraying certain fulfillment of events that at the time of writing were still future. It is the perfect of certitude”} make you insignificant among the nations. You will be utterly despised!}%
\verse{The pride of your heart has deceived you, you who live in the clefts of a rock, the heights of its dwelling, you who say in your heart: ‘Who can bring me down to the ground?’}%
\verse{Even if you soar like the eagle, even if your nest is set among the stars, from there I will bring you down!” declares\lnCSX{} Adonai:}%
\verse{“If thieves came to you, if plunderers of the night — How you have been destroyed! — would they not steal what they wanted?\lebnote{“their sufficiency”} If grape gatherers came, would they not leave gleanings?}%
\verse{How Esau has been pillaged; his treasures have been ransacked!}%
\verse{All of your allies\lebnote{“men of your covenant”} have driven you up to the boundary; your confederates\lebnote{“men of your peace”} have deceived you and have prevailed against you. Those who eat your bread have set an ambush for you, there is no\lebnote{The preposition בּ plus אֵין indicates the absence of a thing within a location} understanding of it.\lebnote{Or “in him”}}%
\verse{On that day,” declares\lnCSX{} Adonai, “will I not destroy the wise men from Edom, and understanding from the mountain of Esau?}%
\verse{And your warriors will be shattered, O Teman, so that everyone\lebnote{“each”} from the mountain of Esau will be cut off because of the slaughter!}%
\verseWithHeading{Edom’s Treachery against Judah}{“Because of the violence done to your brother Jacob, shame will cover you and you will be cut off forever.}%
\verse{On the day you stood nearby,\lebnote{“the day of your standing nearby”} on the day strangers took\lebnote{“the day of the taking captive by strangers”} his wealth, and foreigners entered his gates and cast lots over Jerusalem, you were also like one of them.}%
\verse{But you should not have gloated\lebnote{The Hebrew expression “to look upon” often has the sense of “to feast the eyes upon” or “to gloat over”} over your brother’s day, on the day of his misfortune, and you should not have rejoiced over the people\lebnote{“son”} of Judah on the day of their perishing, and you should not have opened your mouth wide on the day of distress.}%
\verse{You should not have entered the gate of my people on the day of their disaster. You also should not have gloated over his misery on the day of his disaster, and you should not have stretched out your hands on the day of his disaster.}%
\verse{And you should not have stood at the crossroads to cut off his fugitives and you should not have handed over his survivors on the day of distress.}%
\verseWithHeading{Future Judgment of the Nations}{“For the day of Adonai is near against all the nations! Just as you have done, it will be done to you. Your deeds will return on your own head.}%
\verse{For just as you have drunk on my holy mountain,\lebnote{“the mountain of my holiness”} all the nations will drink continually. They will drink and they will slurp, and they will be as if they had never been.}%
\verse{But on Mount Zion there will be an escape, and it will be holy, and the house of Jacob will take possession of their dispossessors.}%
\verse{And the house of Jacob will be a fire and the house of Joseph a flame, and the house of Esau stubble; and they will set them on fire and will consume them. And there will not be a survivor for the house of Esau,” for Adonai has spoken.}%
\verseWithHeading{Future Blessing for Israel}{Those of the Negev will take possession of the mountain of Esau, and those of the Shephelah will possess the land of the Philistines, and they shall take possession of the territory of Ephraim and the territory of Samaria, and Benjamin will take possession of Gilead.}%
\verse{And the exiles\lnCSY{} of this army of the people of Israel will possess Canaan up to Zarephath, and the exiles\lnCSY{} of Jerusalem who are in Sepharad will take possession of the cities of the Negev.}%
\verse{And those who have been saved will go up on Mount Zion to rule the mountain of Esau. And the kingdom will belong to\lebnote{“will be for”} Adonai.}%
\end{biblechapter}%
\flushcolsend
\biblebook{Jonah}
\begin{biblechapter}% Jonah 1
\verseWithHeading{Jonah Disobeys Adonai}{And the word of Adonai came to Jonah the son of Amittai, saying,}%
\verse{“Get up! Go to the great city Nineveh and cry out against her, because their evil has come up before me.”\lebnote{“to my face”}}%
\verse{But Jonah set out to flee toward Tarshish from the presence of\lnCSZ{} Adonai. And he went down to Joppa and found a merchant ship going to Tarshish, and paid her fare, and went on board her to go with them toward Tarshish from the presence of\lnCSZ{} Adonai.}%
\verse{And Adonai hurled a great wind upon the sea, and it was a great storm on the sea, and the merchant ship was in danger of breaking up.\lebnote{“threatened to be broken up”}}%
\verse{And the mariners were afraid, and each cried out to his god. And they threw the contents\lebnote{“objects”} that were in the merchant ship into the sea to lighten it for them. And meanwhile Jonah went down into the hold of the vessel and lay down and fell asleep.}%
\verse{And the captain\lebnote{Or “commander”} of the ship approached him and said to him, “Why are you sound asleep?\lebnote{“in a deep sleep”} Get up! Call on your god! Perhaps your god\lebnote{“the god,” with the article indicating previous reference} will take notice of us and we won’t perish!”}%
\verse{And they said to one another,\lebnote{“each to his friend/companion”} “Come, let us cast lots so that we may know on whose account this disaster has come on us!” And they cast lots, and the lot fell on Jonah.}%
\verse{So they said to him, “Please tell us whoever is responsible that this disaster has come upon us! What is your occupation? And from where do you come? What is your country? And from which people are you?”}%
\verse{And he said to them, “I am a Hebrew, and I fear Adonai, the God of heaven, who made the sea and the dry land.”}%
\verse{Then the men were greatly afraid,\lebnote{“were afraid with a great fear”} and they said to him, “What is this you have done?” because they\lebnote{“the men” but this is redundant in English} knew that he was fleeing from the presence of\lebnote{“from before the face of”} Adonai (because he had told them).}%
\verse{So they said to him, “What shall we do to you so that the sea may quiet down for us?” because the sea was growing more and more tempestuous.\lnCTA{}}%
\verse{And he said to them, “Pick me up and hurl me into the sea so that the sea may quiet down for you, because I know that on account of me this great storm has come upon you all.”}%
\verse{But the men rowed hard to bring the ship\lebnote{the direct object is supplied from context} back to the dry land, and they could not do so because the sea was growing more and more tempestuous\lnCTA{} against them.}%
\verse{So they cried out to Adonai, and they said, “O Adonai! Please do not let us perish because of this man’s life, and do not make us guilty of innocent blood,\lebnote{“do not give innocent blood on us”} because you, O Adonai, did what you wanted.”}%
\verse{And they picked Jonah up and hurled him into the sea, and the sea ceased from its raging.}%
\verse{So the men feared Adonai greatly,\lebnote{“with a great fear”} and they offered a sacrifice to Adonai and made\lebnote{“vowed”} vows.}%
\verseWithHeading{Jonah Is Swallowed by a Fish and Prays to Adonai}{\lebnote{Jonah 1:17–2:10 in the English Bible is 2:1–11 in the Hebrew Bible} And Adonai provided a large fish to swallow up Jonah, and Jonah was in the belly of the fish three days and three nights.}%
\end{biblechapter}%
\begin{biblechapter}% Jonah 2
\verse{And Jonah prayed to Adonai his God from the belly of the fish}%
\verse{and said, “I called from my distress\lebnote{“from distress for me”} to Adonai, and he answered me; from the belly of Sheol\lebnote{“Sheol” is a Hebrew term for the place where the dead reside (i.e. the Underworld)} I cried for help — you heard my voice.}%
\verse{And you threw me into the deep, into the heart of the seas, and the sea currents surrounded me; all your breakers and your surging waves passed over me.}%
\verse{And I said, ‘I am banished\lebnote{Or “expelled,” or “driven away”} from your sight; how\lebnote{Or “however, I will continue to look on your holy temple”} will I continue to look\lebnote{“will I do again to look,” meaning “will I continue to look”} on your holy temple?’\lebnote{“on the temple of your holiness”}}%
\verse{The waters encompassed me up to my neck; the deep surrounded me; seaweed was wrapped around my head.}%
\verse{I went down to the foundations of the mountains; the Underworld — its bars were around me forever. But you brought up my life from the pit, Adonai my God.}%
\verse{When my life was ebbing away from me, I remembered Adonai, and my prayer came to you, to your holy temple.\lebnote{“to the temple of your holiness”}}%
\verse{Those who worship vain idols forsake their loyal love.\lebnote{This could mean (1) they forsake the loyal love they should show to God or (2) they forfeit \textit{forsake} the loyal love that God would have shown to them}}%
\verse{But I, with a voice of thanksgiving, will sacrifice to you; I will fulfill what I have vowed. Deliverance\lebnote{Or “salvation”} belongs to Adonai!”}%
\verse{And Adonai spoke to the fish, and it vomited Jonah out on the dry land.}%
\end{biblechapter}%
\begin{biblechapter}% Jonah 3
\verseWithHeading{The People of Nineveh Repent at Jonah’s Proclamation}{And the word of Adonai came to Jonah a second time, saying,}%
\verse{“Get up! Go to Nineveh, the great city, and proclaim to it the message that I am telling you.”}%
\verse{So Jonah got up\lebnote{Or “set out”} and went to Nineveh according to the word of Adonai. Now Nineveh was an extraordinarily great city\lebnote{“a great city to God” or “a great city to \textit{the} gods,” a disputed phrase which may refer to God’s estimate or ownership of Nineveh, to the presence of many idols, or an idiom referring to the size of Nineveh (this translation takes the last view)} — a journey of three days across.\lebnote{This phrase may also refer to a journey on which business was done, so that “three days” is the total elapsed time}}%
\verse{And Jonah began to go into the city a journey of one day, and he cried out and said, “Forty more days and Nineveh will be demolished!”\lebnote{Or “overthrown”}}%
\verse{And the people of Nineveh believed in God, and they proclaimed a fast and put on sackcloth — from the greatest of them to the least important.\lebnote{“to the smallest of them”}}%
\verseWithHeading{The King’s Proclamation}{And the news reached the king of Nineveh, and he rose from his throne and removed his royal robe, put on sackcloth, and sat in the ashes.}%
\verse{And he had a proclamation made, and said, “In Nineveh, by a decree of the king and his nobles: “No human being or animal, no herd or flock, shall taste anything! They must not eat, and they must not drink water!}%
\verse{And the human beings and the animals must be covered with sackcloth! And they must call forcefully to God, and each must turn from his evil way and from the violence that is in his\lebnote{Hebrew “their”} hands.}%
\verse{Who knows? God may relent and change his mind and turn from his blazing anger\lebnote{“from the heat of his anger”} so that\lebnote{Hebrew “and”} we will not perish.”}%
\verse{And God saw their deeds — that they turned from their evil ways — and God changed his mind about the evil that he had said he would bring upon them, and he did not do it.\lebnote{supplied from English context}}%
\end{biblechapter}%
\begin{biblechapter}% Jonah 4
\verseWithHeading{Jonah Is Angry at Adonai’s Compassion}{And this\lebnote{Hebrew “it”} was greatly displeasing\lebnote{“was displeasing \textit{with} great displeasure”} to Jonah, and he became furious.\lebnote{“it was hot for him”}}%
\verse{And he prayed to Adonai and said, “O Adonai, was this not what I said\lebnote{“my word”} while I was in my homeland? Therefore I originally fled\lebnote{“I did the first time to flee”} to Tarshish, because I knew that you are a gracious and compassionate God, slow to anger and having great steadfast love,\lebnote{“and great of steadfast love”} and one who relents concerning calamity.\lebnote{That is, calamity sent by God as judgment}}%
\verse{And so then, Adonai, please take my life from me, because for me death is better than life!”}%
\verse{And Adonai said, “Is it right for you to be angry?”\lebnote{“Rightfully is it hot for you”; some take this to mean “Are you so very angry?” (\textit{Targum} Jonah 4:4; see also JPS, NEB, NET)}}%
\verse{And Jonah went out from the city and sat down east of the city, and he made for himself a shelter there. And he sat under it in the shade, waiting to see\lebnote{“until he would see”} what would happen with the city.}%
\verse{And Adonai God appointed a plant,\lebnote{Probably a castor oil plant, though some have suggested some type of gourd plant} and he made it grow up over Jonah to be a shade over his head, to save him from his discomfort. And Jonah was very glad\lebnote{“was glad with great joy”} about the plant.}%
\verse{So God appointed a worm at daybreak\lebnote{“at the coming up of the dawn”} the next day, and it attacked the plant, and it withered.}%
\verse{And when the sun rose,\lebnote{“and it happened at the rising of the sun”} God appointed a scorching east wind, and the sun beat down on Jonah’s head and he grew faint. And he asked that he could die\lebnote{“and he asked his soul to die”} and said, “My death is better than my life!”}%
\verse{So God said to Jonah, “Is it right for you to be angry\lebnote{“rightfully is it hot for you”; some take this to mean “Are you so very angry?” (\textit{Targum} Jonah 4:4; see also JPS, NEB, NET)} about the plant?” And he said, “It is right for me to be angry enough to die!”\lebnote{“Rightfully it is hot for me until death”}}%
\verse{But Adonai said, “You are troubled about the plant, for which you did not labor nor cause it to grow. It grew up in a night and it perished in a night!\lebnote{“Which was a son of a night and \textit{as} a son of a night it perished”}}%
\verse{And should I not be concerned about Nineveh, the great city, in which there are\lebnote{“which there are in it”} more than one hundred and twenty thousand\lebnote{“than two ten myriad” (12 x 10,000)} people who do not know right from left,\lebnote{“\textit{hand}” is often supplied, but it is not clear just what deficiency is meant by this expression, which occurs only here in biblical Hebrew} plus many animals?”}%
\end{biblechapter}%
\flushcolsend
\input{leb/content/old-testament/Mic.tex}\flushcolsend
\input{leb/content/old-testament/Nah.tex}\flushcolsend
\biblebook{Habakkuk}
\begin{biblechapter}% Habakkuk 1
\verseWithHeading{Habakkuk’s Complaint}{המשׂא The burden אשׁר which חזה did see. חבקוק Habakkuk הנביא׃ the prophet}%
\verse{עד how long אנה how long יהוה O LORD, שׁועתי shall I cry, ולא and thou wilt not תשׁמע hear! אזעק cry out אליך unto חמס thee violence, ולא and thou wilt not תושׁיע׃ save!}%
\verse{למה Why תראני dost thou show און me iniquity, ועמל grievance? תביט and cause to behold ושׁד for spoiling וחמס and violence לנגדי before ויהי me: and there are ריב strife ומדון and contention. ישׂא׃ raise up}%
\verse{על therefore כן therefore תפוג is slacked, תורה the law ולא doth never יצא go forth: לנצח doth never משׁפט and judgment כי for רשׁע the wicked מכתיר doth compass about את הצדיק the righteous; על כןצא proceedeth. משׁפט judgment מעקל׃ wrong}%
\verseWithHeading{God’s Answer to Habakkuk}{ראו Behold בגוים ye among the heathen, והביטו and regard, והתמהו and wonder marvelously: תמהו and wonder marvelously: כי for פעל a work פעל will work בימיכם in your days, לא ye will not תאמינו believe, כי though יספר׃ it be told}%
\verse{כי For, הנני מקים I raise up את הכשׂדים the Chaldeans, הגוי nation, המר bitter והנמהר and hasty ההולך which shall march למרחבי through the breadth ארץ of the land, לרשׁת to possess משׁכנות the dwelling places לא׃ not}%
\verse{אים terrible ונורא and dreadful: הוא They ממנו of משׁפטו their judgment ושׂאתו and their dignity יצא׃ shall proceed}%
\verse{וקלו also are swifter מנמרים than the leopards, סוסיו Their horses וחדו and are more fierce מזאבי wolves: ערב than the evening ופשׁו shall spread פרשׁיו and their horsemen ופרשׁיו themselves, and their horsemen מרחוק from far; יבאו shall come יעפו they shall fly כנשׁר as the eagle חשׁ hasteth לאכול׃ to eat.}%
\verse{כלה all לחמס for violence: יבוא They shall come מגמת shall sup up פניהם their faces קדימה the east wind, ויאסף and they shall gather כחול as the sand. שׁבי׃ the captivity}%
\verse{והוא And they במלכים at the kings, יתקלס shall scoff ורזנים and the princes משׂחק shall be a scorn לו הוא unto them: they לכל every מבצר stronghold; ישׂחק shall deride ויצבר for they shall heap עפר dust, וילכדה׃ and take}%
\verse{אז Then חלף change, רוח shall mind ויעבר and he shall pass over, ואשׁם and offend, זו this כחו his power לאלהו׃}%
\verseWithHeading{Habakkuk’s Second Complaint}{הלוא not אתה thou מקדם from everlasting, יהוה אלהי my God, קדשׁי mine Holy One? לא we shall not נמות die. יהוה למשׁפט them for judgment; שׂמתו thou hast ordained וצור and, O mighty God, להוכיח them for correction. יסדתו׃ thou hast established}%
\verse{טהור עינים eyes מראות than to behold רע evil, והביט look אל on עמל iniquity: לא not תוכל and canst למה wherefore תביט lookest בוגדים thou upon them that deal treacherously, תחרישׁ holdest thy tongue בבלע devoureth רשׁע when the wicked צדיק more righteous ממנו׃ than to behold}%
\verse{ותעשׂה And makest אדם men כדגי as the fishes הים of the sea, כרמשׂ as the creeping things, לא no משׁל׃ ruler}%
\verse{כלה all בחכה of them with the angle, העלה They take up יגרהו they catch בחרמו them in their net, ויאספהו and gather במכמרתו them in their drag: על therefore כן therefore ישׂמח they rejoice ויגיל׃ and are glad.}%
\verse{על כןזבח they sacrifice לחרמו unto their net, ויקטר and burn incense למכמרתו unto their drag; כי because בהמה by them שׁמן fat, חלקו their portion ומאכלו and their meat בראה׃ plenteous.}%
\verse{העל כןריק empty חרמו their net, ותמיד continually להרג to slay גוים the nations? לא and not יחמול׃ spare}%
\end{biblechapter}%
\begin{biblechapter}% Habakkuk 2
\verseWithHeading{The Righteous Will Live by Faith}{על upon משׁמרתי my watch, אעמדה I will stand ואתיצבה and set על me upon מצור the tower, ואצפה and will watch לראות to see מה what ידבר he will say בי ומה unto me, and what אשׁיב I shall answer על when תוכחתי׃ I am reproved.}%
\verse{ויענני answered יהוה And the LORD ויאמר me, and said, כתוב Write חזון the vision, ובאר and make plain על upon הלחות tables, למען that ירוץ he may run קורא׃ that readeth}%
\verse{כי For עוד yet חזון the vision למועד for an appointed time, ויפח it shall speak, לקץ but at the end ולא and not יכזב lie: אם though יתמהמה it tarry, חכה wait לו כי for it; because בא it will surely come, יבא it will surely come, לא it will not יאחר׃ tarry.}%
\verse{הנה Behold, עפלה is lifted up לא is not upright ישׁרה is not upright נפשׁו his soul בו וצדיק in him: but the just באמונתו by his faith. יחיה׃ shall live}%
\verse{ואף Yea כי also, because היין by wine, בוגד he transgresseth גבר man, יהיר a proud ולא neither ינוה keepeth at home, אשׁר who הרחיב enlargeth כשׁאול as hell, נפשׁו his desire והוא כמות and as death, ולא and cannot ישׂבע be satisfied, ויאסף but gathereth אליו unto כל him all הגוים nations, ויקבץ and heapeth אליו unto כל him all העמים׃ people:}%
\verse{הלוא Shall not אלה these כלם all עליו against משׁל a parable ישׂאו take up ומליצה him, and a taunting חידות proverb לו ויאמר against him, and say, הוי Woe המרבה to him that increaseth לא not לו עד מתימכביד and to him that ladeth עליו and to him that ladeth עבטיט׃ himself with thick clay!}%
\verse{הלוא Shall they not פתע suddenly יקומו rise up נשׁכיך that shall bite ויקצו מזעזעיך that shall vex והיית thee, and thou shalt be למשׁסות׃ for booties}%
\verse{כי Because אתה thou שׁלות hast spoiled גוים nations, רבים many ישׁלוך shall spoil כל all יתר the remnant עמים of the people מדמי blood, אדם thee; because of men's וחמס and the violence ארץ of the land, קריה of the city, וכל and of all ישׁבי׃ that dwell}%
\verse{הוי Woe בצע to him that coveteth בצע covetousness רע an evil לביתו to his house, לשׂום that he may set במרום on high, קנו his nest להנצל that he may be delivered מכף from the power רע׃ of evil!}%
\verse{יעצת Thou hast consulted בשׁת shame לביתך to thy house קצות by cutting off עמים people, רבים many וחוטא and hast sinned נפשׁך׃ thy soul.}%
\verse{כי For אבן the stone מקיר of the wall, תזעק shall cry out וכפיס and the beam מעץ out of the timber יעננה׃ shall answer}%
\verse{הוי Woe בנה to him that buildeth עיר a town בדמים with blood, וכונן and establisheth קריה a city בעולה׃ by iniquity!}%
\verse{הלוא not הנה Behold, מאת יהוה the LORD צבאות of hosts וייגעו shall labor עמים that the people בדי in the very אשׁ fire, ולאמים and the people בדי themselves for very ריק vanity? יעפו׃ shall weary}%
\verse{כי For תמלא shall be filled הארץ the earth לדעת אתבוד of the glory יהוה of the LORD, כמים as the waters יכסו cover על cover ים׃ the sea.}%
\verse{הוי Woe משׁקה unto him that giveth his neighbor drink, רעהו unto him that giveth his neighbor drink, מספח that puttest חמתך thy bottle ואף also, שׁכר to and makest drunken למען that הביט thou mayest look על on מעוריהם׃ their nakedness!}%
\verse{שׂבעת Thou art filled קלון with shame מכבוד for glory: שׁתה drink גם also, אתה thou והערל and let thy foreskin be uncovered: תסוב shall be turned עליך unto כוס the cup ימין right hand יהוה of the LORD's וקיקלון thee, and shameful spewing על on כבודך׃ thy glory.}%
\verse{כי For חמס the violence לבנון of Lebanon יכסך shall cover ושׁד thee, and the spoil בהמות of beasts, יחיתן made them afraid, מדמי blood, אדם because of men's וחמס and for the violence ארץ of the land, קריה of the city, וכל and of all ישׁבי׃ that dwell}%
\verse{מה What הועיל profiteth פסל the graven image כי that פסלו thereof hath graven יצרו of his work מסכה it; the molten image, ומורה and a teacher שׁקר of lies, כי that בטח trusteth יצר the maker יצרו the maker עליו therein, לעשׂות to make אלילים idols? אלמים׃ dumb}%
\verse{הוי Woe אמר unto him that saith לעץ to the wood, הקיצה Awake; עורי Arise, לאבן stone, דומם to the dumb הוא it יורה shall teach! הנה Behold, הוא it תפושׂ laid over זהב with gold וכסף and silver, וכל at all רוח breath אין and no בקרבו׃ in the midst}%
\verse{ויהוה But the LORD בהיכל temple: קדשׁו in his holy הס keep silence מפניו before כל let all הארץ׃ the earth}%
\end{biblechapter}%
\begin{biblechapter}% Habakkuk 3
\verseWithHeading{The Prayer of Habakkuk}{תפלה A prayer לחבקוק of Habakkuk הנביא the prophet על upon שׁגינות׃ Shigionoth.}%
\verse{יהוה O LORD, שׁמעתי I have heard שׁמעך thy speech, יראתי was afraid: יהוה O LORD, פעלך thy work בקרב in the midst שׁנים of the years, חייהו revive בקרב in the midst שׁנים of the years תודיע make known; ברגז in wrath רחם mercy. תזכור׃ remember}%
\verse{אלוה God מתימן יבוא came וקדושׁ and the Holy One מהר from mount פארן Paran. סלה Selah. כסה covered שׁמים the heavens, הודו His glory ותהלתו of his praise. מלאה was full הארץ׃ and the earth}%
\verse{ונגה And brightness כאור as the light; תהיה was קרנים he had horns מידו out of his hand: לו ושׁם and there חביון the hiding עזה׃ of his power.}%
\verse{לפניו Before ילך him went דבר the pestilence, ויצא went forth רשׁף and burning coals לרגליו׃ at his feet.}%
\verse{עמד He stood, וימדד ארץ the earth: ראה he beheld, ויתר and drove asunder גוים the nations; ויתפצצו were scattered, הררי mountains עד and the everlasting שׁחו did bow: גבעות hills עולם the perpetual הליכות his ways עולם׃ everlasting.}%
\verse{תחת in און affliction: ראיתי I saw אהלי the tents כושׁן of Cushan ירגזון did tremble. יריעות the curtains ארץ of the land מדין׃ of Midian}%
\verse{הבנהרים against the rivers? חרה displeased יהוה Was the LORD אם בנהרים against the rivers? אפך thine anger אם בים against the sea, עברתך thy wrath כי that תרכב thou didst ride על upon סוסיך thine horses מרכבתיך thy chariots ישׁועה׃ of salvation?}%
\verse{עריה was made quite naked, תעור was made quite naked, קשׁתך Thy bow שׁבעות to the oaths מטות of the tribes, אמר word. סלה Selah. נהרות with rivers. תבקע Thou didst cleave ארץ׃ the earth}%
\verse{ראוך saw יחילו thee, they trembled: הרים The mountains זרם the overflowing מים of the water עבר passed by: נתן uttered תהום the deep קולו his voice, רום on high. ידיהו his hands נשׂא׃ lifted up}%
\verse{שׁמשׁ The sun ירח moon עמד stood still זבלה in their habitation: לאור at the light חציך of thine arrows יהלכו they went, לנגה at the shining ברק of thy glittering חניתך׃ spear.}%
\verse{בזעם in indignation, תצעד Thou didst march through ארץ the land באף in anger. תדושׁ thou didst thresh גוים׃ the heathen}%
\verse{יצאת Thou wentest forth לישׁע for the salvation עמך of thy people, לישׁע for salvation את משׁיחך thine anointed; מחצת thou woundedst ראשׁ the head מבית out of the house רשׁע of the wicked, ערות by discovering יסוד the foundation עד unto צואר the neck. סלה׃ Selah.}%
\verse{נקבת Thou didst strike through במטיו with his staves ראשׁ the head פרזו of his villages: יסערו they came out as a whirlwind להפיצני to scatter עליצתם me: their rejoicing כמו as לאכל to devour עני the poor במסתר׃ secretly.}%
\verse{דרכת Thou didst walk בים through the sea סוסיך with thine horses, חמר the heap מים waters. רבים׃ of great}%
\verse{שׁמעתי When I heard, ותרגז trembled; בטני my belly לקול at the voice: צללו quivered שׂפתי my lips יבוא entered רקב rottenness בעצמי into my bones, ותחתי in ארגז and I trembled אשׁר myself, that אנוח I might rest ליום in the day צרה of trouble: לעלות when he cometh up לעם unto the people, יגודנו׃ he will invade}%
\verse{כי Although תאנה the fig tree לא shall not תפרח blossom, ואין neither יבול fruit בגפנים in the vines; כחשׁ shall fail, מעשׂה the labor זית of the olive ושׁדמות and the fields לא no עשׂה shall yield אכל meat; גזר shall be cut off ממכלה from the fold, צאן the flock ואין and no בקר herd ברפתים׃ in the stalls:}%
\verse{ואני Yet I ביהוה in the LORD, אעלוזה will rejoice אגילה I will joy באלהי in the God ישׁעי׃ of my salvation.}%
\verse{יהוה אדני God חילי my strength, וישׂם and he will make רגלי my feet כאילות like hinds' ועל upon במותי mine high places. ידרכני and he will make me to walk למנצח To the chief singer בנגינותי׃ on my stringed instruments.}%
\end{biblechapter}%
\flushcolsend
\biblebook{Zephaniah}
\begin{biblechapter}% Zephaniah 1
\verseWithHeading{Judah’s Impending Judgment}{דבר The word יהוה of the LORD אשׁר which היה came אל unto צפניה Zephaniah בן the son כושׁי בן the son גדליה of Gedaliah, בן the son אמריה of Amariah, בן the son חזקיה of Hizkiah, בימי in the days יאשׁיהו of Josiah בן the son אמון of Amon, מלך king יהודה׃ of Judah.}%
\verse{אסף אסףל all מעל פניאדמה the land, נאם saith יהוה׃ the LORD.}%
\verse{אסף אדם man ובהמה and beast; אסף עוף the fowls השׁמים of the heaven, ודגי and the fishes הים of the sea, והמכשׁלות and the stumblingblocks את with הרשׁעים the wicked; והכרתי and I will cut off את האדם man מעל פניאדמה the land, נאם saith יהוה׃ the LORD.}%
\verse{ונטיתי I will also stretch out ידי mine hand על upon יהודה Judah, ועל and upon כל all יושׁבי the inhabitants ירושׁלם of Jerusalem; והכרתי and I will cut off מן from המקום place, הזה this את שׁאר the remnant הבעל of Baal את שׁם the name הכמרים of the Chemarims עם with הכהנים׃ the priests;}%
\verse{ואת המשׁתחוים And them that worship על upon הגגות the housetops; לצבא the host השׁמים of heaven ואת המשׁתחוים and them that worship הנשׁבעים that swear ליהוה by the LORD, והנשׁבעים and that swear במלכם׃ by Malcham;}%
\verse{ואת הנסוגים And them that are turned back מאחרי from יהוה ואשׁר and that לא have not בקשׁו sought את יהוהלא nor דרשׁהו׃ inquired}%
\verse{הס Hold thy peace מפני at the presence אדני of the Lord יהוה GOD: כי for קרוב at hand: יום the day יהוה of the LORD כי for הכין hath prepared יהוה the LORD זבח a sacrifice, הקדישׁ he hath bid קראיו׃ his guests.}%
\verse{והיה And it shall come to pass ביום in the day זבח sacrifice, יהוה of the LORD's ופקדתי עלשׂרים the princes, ועל בני children, המלך and the king's ועל כל and all הלבשׁים such as are clothed מלבושׁ apparel. נכרי׃ with strange}%
\verse{ופקדתי על on כל all הדולג those that leap על המפתן the threshold, ביום day ההוא In the same הממלאים which fill בית houses אדניהם their masters' חמס with violence ומרמה׃ and deceit.}%
\verse{והיה And it shall come to pass ביום day, ההוא in that נאם saith יהוה the LORD, קול the noise צעקה of a cry משׁער gate, הדגים from the fish ויללה and a howling מן gate, המשׁנה the second, ושׁבר crashing גדול and a great מהגבעות׃ from the hills.}%
\verse{הילילו Howl, ישׁבי ye inhabitants המכתשׁ of Maktesh, כי for נדמה are cut down; כל all עם people כנען the merchant נכרתו are cut off. כל all נטילי they that bear כסף׃ silver}%
\verse{והיה And it shall come to pass בעת time, ההיא at that אחפשׂ I will search את ירושׁלם Jerusalem בנרות with candles, ופקדתי and punish על and punish האנשׁים the men הקפאים that are settled על on שׁמריהם their lees: האמרים that say בלבבם in their heart, לא will not ייטיב do good, יהוה The LORD ולא neither ירע׃ will he do evil.}%
\verse{והיה shall become חילם Therefore their goods למשׁסה a booty, ובתיהם and their houses לשׁממה a desolation: ובנו they shall also build בתים houses, ולא but not ישׁבו inhabit ונטעו and they shall plant כרמים vineyards, ולא but not ישׁתו drink את יינם׃ the wine}%
\verse{קרוב near, יום day יהוה of the LORD הגדול The great קרוב near, ומהר and hasteth מאד greatly, קול the voice יום of the day יהוה of the LORD: מר bitterly. צרח shall cry שׁם there גבור׃ the mighty man}%
\verse{יום day עברה of wrath, היום a day ההוא That יום a day צרה of trouble ומצוקה and distress, יום a day שׁאה of wasteness ומשׁואה and desolation, יום a day חשׁך of darkness ואפלה and gloominess, יום a day ענן of clouds וערפל׃ and thick darkness,}%
\verse{יום A day שׁופר of the trumpet ותרועה and alarm על against הערים cities, הבצרות the fenced ועל and against הפנות towers. הגבהות׃ the high}%
\verse{והצרתי And I will bring distress לאדם upon men, והלכו that they shall walk כעורים like blind men, כי because ליהוה against the LORD: חטאו they have sinned ושׁפך shall be poured out דמם and their blood כעפר as dust, ולחמם and their flesh כגללים׃ as the dung.}%
\verse{גם nor כספם their silver גם זהבם their gold לא יוכל shall be able להצילם to deliver ביום them in the day עברת wrath; יהוה of the LORD's ובאשׁ by the fire קנאתו of his jealousy: תאכל shall be devoured כל but the whole הארץ land כי for כלה riddance אך even נבהלה a speedy יעשׂה he shall make את כל all ישׁבי them that dwell הארץ׃ in the land.}%
\end{biblechapter}%
\begin{biblechapter}% Zephaniah 2
\verseWithHeading{Judgment of Judah’s Enemies}{התקושׁשׁו Gather yourselves together, וקושׁו yea, gather together, הגוי O nation לא not נכסף׃ desired;}%
\verse{בטרם Before לדת bring forth, חק the decree כמץ as the chaff, עבר pass יום the day בטרם before לא יבוא come עליכם upon חרון the fierce אף anger יהוה of the LORD בטרם you, before לא יבוא come עליכם upon יום the day אף anger יהוה׃ of the LORD's}%
\verse{בקשׁו Seek את יהוה ye the LORD, כל all ענוי ye meek הארץ of the earth, אשׁר which משׁפטו his judgment; פעלו have wrought בקשׁו seek צדק righteousness, בקשׁו seek ענוה meekness: אולי it may be תסתרו ye shall be hid ביום in the day אף anger. יהוה׃ of the LORD's}%
\verse{כי For עזה Gaza עזובה forsaken, תהיה shall be ואשׁקלון and Ashkelon לשׁממה a desolation: אשׁדוד Ashdod בצהרים at the noon day, יגרשׁוה they shall drive out ועקרון and Ekron תעקר׃ shall be rooted up.}%
\verse{הוי Woe ישׁבי unto the inhabitants חבל coast, הים of the sea גוי the nation כרתים of the Cherethites! דבר the word יהוה of the LORD עליכם against כנען you; O Canaan, ארץ the land פלשׁתים of the Philistines, והאבדתיך I will even destroy מאין thee, that there shall be no יושׁב׃ inhabitant.}%
\verse{והיתה shall be חבל coast הים And the sea נות dwellings כרת cottages רעים for shepherds, וגדרות and folds צאן׃ for flocks.}%
\verse{והיה shall be חבל And the coast לשׁארית for the remnant בית of the house יהודה of Judah; עליהם thereupon: ירעון they shall feed בבתי in the houses אשׁקלון of Ashkelon בערב in the evening: ירבצון shall they lie down כי for יפקדם shall visit יהוה the LORD אלהיהם their God ושׁב them, and turn away שׁבותם׃ their captivity.}%
\verse{שׁמעתי I have heard חרפת the reproach מואב of Moab, וגדופי and the revilings בני of the children עמון of Ammon, אשׁר whereby חרפו they have reproached את עמי my people, ויגדילו and magnified על against גבולם׃ their border.}%
\verse{לכן Therefore חי live, אני I נאם saith יהוה the LORD צבאות of hosts, אלהי the God ישׂראל of Israel, כי Surely מואב Moab כסדם as Sodom, תהיה shall be ובני and the children עמון of Ammon כעמרה as Gomorrah, ממשׁק the breeding חרול of nettles, ומכרה מלח and saltpits, ושׁממה desolation: עד and a perpetual עולם and a perpetual שׁארית the residue עמי of my people יבזום shall spoil ויתר them, and the remnant גוי of my people ינחלום׃ shall possess}%
\verse{זאת This להם תחת shall they have for גאונם their pride, כי because חרפו they have reproached ויגדלו and magnified על against עם the people יהוה צבאות׃ of hosts.}%
\verse{נורא terrible יהוה The LORD עליהם unto כי them: for רזה he will famish את כל all אלהי the gods הארץ of the earth; וישׁתחוו and shall worship לו אישׁ him, every one ממקומו from his place, כל all איי the isles הגוים׃ of the heathen.}%
\verse{גם also, אתם Ye כושׁים Ethiopians חללי slain חרבי by my sword. המה׃ ye}%
\verse{ויט And he will stretch out ידו his hand על against צפון the north, ויאבד and destroy את אשׁור Assyria; וישׂם and will make את נינוה Nineveh לשׁממה a desolation, ציה dry כמדבר׃ like a wilderness.}%
\verse{ורבצו shall lie down בתוכה in the midst עדרים And flocks כל of her, all חיתו the beasts גוי of the nations: גם both קאת the cormorant גם and קפד the bittern בכפתריה in the upper lintels ילינו shall lodge קול of it; voice ישׁורר shall sing בחלון in the windows; חרב desolation בסף in the thresholds: כי for ארזה the cedar work. ערה׃ he shall uncover}%
\verse{זאת This העיר city העליזה the rejoicing היושׁבת that dwelt לבטח carelessly, האמרה that said בלבבה in her heart, אני I ואפסי and none עוד beside איך me: how היתה is she become לשׁמה a desolation, מרבץ to lie down לחיה a place for beasts כל in! every one עובר that passeth עליה by ישׁרק her shall hiss, יניע wag ידו׃ his hand.}%
\end{biblechapter}%
\begin{biblechapter}% Zephaniah 3
\verseWithHeading{Judgment of Jerusalem and the Nations}{הוי Woe מראה ונגאלה and polluted, העיר city! היונה׃ to the oppressing}%
\verse{לא not שׁמעה She obeyed בקול the voice; לא not לקחה she received מוסר correction; ביהוה in the LORD; לא not בטחה she trusted אל to אלהיה her God. לא she drew not near קרבה׃ she drew not near}%
\verse{שׂריה Her princes בקרבה within אריות lions; שׁאגים her roaring שׁפטיה her judges זאבי wolves; ערב evening לא they gnaw not the bones גרמו they gnaw not the bones לבקר׃ till the morrow.}%
\verse{נביאיה Her prophets פחזים light אנשׁי בגדות treacherous כהניה her priests חללו have polluted קדשׁ the sanctuary, חמסו they have done violence תורה׃ to the law.}%
\verse{יהוה LORD צדיק The just בקרבה in the midst לא thereof; he will not יעשׂה do עולה iniquity: בבקר every morning בבקר every morning משׁפטו his judgment יתן doth he bring לאור to light, לא not; נעדר he faileth ולא no יודע knoweth עול but the unjust בשׁת׃ shame.}%
\verse{הכרתי I have cut off גוים the nations: נשׁמו are desolate; פנותם their towers החרבתי חוצותםבלי that none עובר passeth by: נצדו are destroyed, עריהם their cities מבלי so that there is no אישׁ man, מאין that there is none יושׁב׃ inhabitant.}%
\verse{אמרתי I said, אך Surely תיראי thou wilt fear אותי תקחי me, thou wilt receive מוסר instruction; ולא should not יכרת be cut off, מעונה כל howsoever אשׁר howsoever פקדתי עליהכן them: but השׁכימו they rose early, השׁחיתו corrupted כל all עלילותם׃ their doings.}%
\verse{לכן Therefore חכו wait לי נאם ye upon me, saith יהוה the LORD, ליום until the day קומי that I rise up לעד to the prey: כי for משׁפטי my determination לאסף to gather גוים the nations, לקבצי that I may assemble ממלכות the kingdoms, לשׁפך to pour עליהם upon זעמי them mine indignation, כל all חרון my fierce אפי anger: כי for באשׁ with the fire קנאתי of my jealousy. תאכל shall be devoured כל all הארץ׃ the earth}%
\verse{כי For אז then אהפך will I turn אל to עמים the people שׂפה language, ברורה a pure לקרא call כלם that they may all בשׁם upon the name יהוה of the LORD, לעבדו to serve שׁכם consent. אחד׃ him with one}%
\verse{מעבר לנהרי the rivers כושׁ of Ethiopia עתרי my suppliants, בת the daughter פוצי of my dispersed, יובלון shall bring מנחתי׃ mine offering.}%
\verse{ביום day ההוא In that לא shalt thou not תבושׁי be ashamed מכל for all עלילתיך thy doings, אשׁר wherein פשׁעת thou hast transgressed בי כי against me: for אז then אסיר I will take away מקרבך out of the midst עליזי of thee them that rejoice גאותך in thy pride, ולא and thou shalt no תוספי more לגבהה be haughty עוד more בהר mountain. קדשׁי׃ because of my holy}%
\verse{והשׁארתי I will also leave בקרבך in the midst עם people, עני of thee an afflicted ודל and poor וחסו and they shall trust בשׁם in the name יהוה׃ of the LORD.}%
\verse{שׁארית The remnant ישׂראל of Israel לא shall not יעשׂו do עולה iniquity, ולא nor ידברו speak כזב lies; ולא neither ימצא be found בפיהם in their mouth: לשׁון tongue תרמית shall a deceitful כי for המה they ירעו shall feed ורבצו and lie down, ואין and none מחריד׃ shall make afraid.}%
\verse{רני Sing, בת O daughter ציון of Zion; הריעו shout, ישׂראל O Israel; שׂמחי be glad ועלזי and rejoice בכל with all לב the heart, בת O daughter ירושׁלם׃ of Jerusalem.}%
\verse{הסיר hath taken away יהוה The LORD משׁפטיך thy judgments, פנה he hath cast out איבך thine enemy: מלך the king ישׂראל of Israel, יהוה the LORD, בקרבך in the midst לא of thee: thou shalt not תיראי רע evil עוד׃ any more.}%
\verse{ביום day ההוא In that יאמר it shall be said לירושׁלם to Jerusalem, אל thou not: תיראי Fear ציון Zion, אל Let not ירפו be slack. ידיך׃ thine hands}%
\verse{יהוה The LORD אלהיך thy God בקרבך in the midst גבור of thee mighty; יושׁיע he will save, ישׂישׂ he will rejoice עליך over בשׂמחה thee with joy; יחרישׁ he will rest באהבתו in his love, יגיל he will joy עליך over ברנה׃ thee with singing.}%
\verse{נוגי sorrowful ממועד for the solemn assembly, אספתי I will gather ממך for the solemn assembly, היו are משׂאת it a burden. עליה of חרפה׃ thee, the reproach}%
\verse{הנני עשׂה I will undo את כל all מעניך that afflict בעת time ההיא at that והושׁעתי thee: and I will save את הצלעה her that halteth, והנדחה her that was driven out; אקבץ and gather ושׂמתים and I will get לתהלה them praise ולשׁם and fame בכל in every הארץ land בשׁתם׃ where they have been put to shame.}%
\verse{בעת time ההיא At that אביא will I bring אתכם ובעת you even in the time קבצי that I gather אתכם כי you: for אתן I will make אתכם לשׁם you a name ולתהלה and a praise בכל among all עמי people הארץ of the earth, בשׁובי when I turn back את שׁבותיכם your captivity לעיניכם before your eyes, אמר saith יהוה׃ the LORD.}%
\end{biblechapter}%
\flushcolsend
\biblebook{Haggai}
\begin{biblechapter}% Haggai 1
\verseWithHeading{The Temple Lies in Ruins}{In the second year of King Darius, in the sixth month, on the first day, the word of Adonai came through\lnCTM{} Haggai the prophet to Zerubbabel the son of Shealtiel, governor of Judah, and to Joshua the son of Jehozadak, the high priest, saying,}%
\verse{“Thus says Adonai of hosts: ‘This people says, “The time has not come to rebuild the temple of Adonai.’”\lebnote{“The time is not coming, the time for the house of Adonai to be built”}}%
\verse{And the word of Adonai came through\lnCTM{} Haggai the prophet, saying,}%
\verse{“Is it a time for you yourselves to dwell in your houses that have been paneled while this house\lnCTN{} is desolate?”}%
\verse{And so then, thus says Adonai of hosts: ‘Consider your ways!\lnCTO{}}%
\verse{You have sown much but have harvested little. You have eaten without being satisfied; you have drunk without being satiated;\lebnote{“You have drunk, without becoming drunk”} you have worn clothes without being warm; the one who earns wages puts it in a pouch with holes.’\lebnote{“a pierced pouch”}}%
\verse{Thus says Adonai of hosts: ‘Consider your ways!\lnCTO{}}%
\verse{Go up the mountain and bring wood and build the house\lnCTN{} so that I may be pleased with it and honored,’ says Adonai.}%
\verse{‘You have looked for much, and look! It came to little; and when you brought it home, I blew it away. Why?’ declares\lebnote{“a declaration of ”} Adonai of hosts. ‘Because my house is desolate and you are running each to your own house!}%
\verse{Therefore, because of you the heavens have withheld the dew and the earth has withheld its produce.}%
\verse{I have called for a drought on the land and the hills,\lebnote{Or mountains} on the grain, the new wine, the olive oil, on what the soil produces, on human beings and wild animals,\lebnote{Hebrew “animal”} and on all their labor.’”\lebnote{“all the labor of \textit{the} hands”}}%
\verse{Zerubbabel son of Shealtiel, Joshua son of Jehozadak, the high priest, and all the remnant of the people gave heed to the voice of Adonai their God and to the words of Haggai the prophet, as Adonai their God had sent him, and the people feared Adonai.\lebnote{“and the people feared from before Adonai”}}%
\verse{And Haggai the messenger of Adonai spoke to the people with the message of Adonai, saying, “‘I am with you’ declares\lebnote{“a declaration of”} Adonai.”}%
\verse{And Adonai stirred up the spirit of Zerubbabel son of Shealtiel, governor of Judah, and the spirit of Joshua son of Jehozadak, the high priest, and the spirit of all the remnant of the people. And they came and did the work on the house of Adonai of hosts, their God,}%
\verse{on the twenty-fourth day of the sixth month in the second year of King Darius.}%
\end{biblechapter}%
\begin{biblechapter}% Haggai 2
\verseWithHeading{A Promise to Rebuild}{In the seventh month on the twenty-first day, the word of Adonai came through\lebnote{“came through the hand of”} Haggai the prophet, saying,}%
\verse{“Speak now to Zerubbabel son of Shealtiel, governor of Judah, and to Joshua son of Jehozadak, the high priest, and to the remnant of the people, saying,}%
\verse{‘Who among you is left that saw this house in its former glory? And how do you see it now? Does it seem like nothing to you?\lebnote{“in your eyes”}}%
\verse{‘But now take courage,\lnCTP{} Zerubbabel,’ declares\lnCTQ{} Adonai. ‘Take courage,\lebnote{Or “Be strong”} Joshua son of Jehozadak, the high priest, and take courage,\lnCTP{} all the people of the land,’ declares\lnCTQ{} Adonai. ‘Do the work, because I am with you,’ declares\lnCTQ{} Adonai of hosts,}%
\verse{‘according to the promise\lebnote{“the word”} that I covenanted\lebnote{“I cut”} with you when you came out of Egypt. My spirit is with you;\lebnote{“\textit{is} standing in your midst”} do not be afraid.’}%
\verse{For thus says Adonai of hosts: ‘Once again, in a little while, I will shake the heavens and the earth and the sea and dry land.}%
\verse{I will shake all the nations so that the treasure of all the nations will come, and I will fill this house\lebnote{That is, the temple} with glory,’ says Adonai of hosts.}%
\verse{‘The silver is mine and the gold is mine,’ declares\lnCTQ{} Adonai of hosts.}%
\verse{‘The latter glory of this house will be greater than the former,’ says Adonai of hosts, ‘and in this place I will give peace’ declares\lnCTQ{} Adonai of hosts.’”}%
\verse{On the twenty-fourth day of the ninth month, in the second year of Darius, the word of Adonai came to the prophet Haggai, saying,}%
\verse{“Thus says Adonai of hosts: ‘Ask now the priests for a ruling:\lebnote{Or “a torah,” or “a law”}}%
\verse{If a man carries\lebnote{“lifts”} consecrated meat in the hem of his garment, and his hem touches bread, or stew, or wine, or olive oil, or any kind of food, will it become holy?’” The priests answered, “No.”}%
\verse{Then Haggai said, “If one who is unclean from contact with a corpse touches any of these, will it become unclean?” The priests answered, “Yes, it will be become unclean.”}%
\verse{And Haggai answered and said, “‘So it is with this people, and with this nation before me,’ declares\lnCTQ{} Adonai, ‘and so it is with every kind of work of their hands; and what they offer there is unclean.}%
\verse{But now, please consider\lnCTR{} from this day forward, before one stone was placed on another in the temple of Adonai,}%
\verse{from that time when\lebnote{Or “from their being,” or “from the time they were then”} one came to a heap of twenty measures, there were only ten, and when one came to the wine vat to draw out fifty measures,\lebnote{Hebrew “measure”} there were only twenty.}%
\verse{I struck you with blight, and with plant mildew, and hail, all the work of your hands. But you did not come back to me,’\lebnote{“there were none of you to me”} declares\lnCTQ{} Adonai.}%
\verse{‘Please consider\lebnote{“Please set your heart”} from this day forward, from the twenty-fourth day of the ninth month, from the day that the foundation of Adonai’s temple was laid,\lebnote{Or “placed,” or “set”} consider:\lnCTR{}}%
\verse{Is there still seed in the store chamber? Do the vine, the fig tree, the pomegranate, and the olive tree still produce nothing? From this day forward I will bless you.’”}%
\verseWithHeading{The Coming Kingdom}{And the word of Adonai came to Haggai a second time on the twenty-fourth day of the month saying,}%
\verse{“Say to Zerubbabel, governor of Judah: I am going to shake the heavens and the earth,}%
\verse{and I will overthrow the thrones of kingdoms and destroy the military strength of the kingdoms of the nations. I will overthrow chariots and their drivers; horses and their riders will fall, every one by the sword of another!}%
\verse{‘On that day,’ declares\lnCTQ{} Adonai of hosts, ‘I will take you, Zerubbabel son of Shealtiel, my servant,’ declares\lnCTQ{} Adonai, ‘and I will make you a signet ring, for I have chosen you,’ declares\lnCTQ{} Adonai of hosts.”}%
\end{biblechapter}%
\flushcolsend
\biblebook{Zechariah}
\begin{biblechapter}% Zechariah 1
\verse{בחדשׁ month, השׁמיני In the eighth בשׁנת year שׁתים in the second לדריושׁ of Darius, היה came דבר the word יהוה of the LORD אל unto זכריה Zechariah, בן the son ברכיה of Berechiah, בן the son עדו of Iddo הנביא the prophet, לאמר׃ saying,}%
\verse{קצף hath been sore displeased יהוה The LORD על with אבותיכם your fathers. קצף׃ hath been sore displeased}%
\verse{ואמרת Therefore say אלהם thou unto כה them, Thus אמר saith יהוה the LORD צבאות of hosts; שׁובו Turn אלי ye unto נאם me, saith יהוה the LORD צבאות of hosts, ואשׁוב and I will turn אליכם unto אמר you, saith יהוה the LORD צבאות׃ of hosts.}%
\verse{אל ye not תהיו Be כאבתיכם as your fathers, אשׁר whom קראו have cried, אליהם unto הנביאים prophets הראשׁנים the former לאמר saying, כה Thus אמר saith יהוה the LORD צבאות of hosts; שׁובו Turn נא ye now מדרכיכם ways, הרעים from your evil ומעליליכם doings: הרעים and your evil ולא but they did not שׁמעו hear, ולא nor הקשׁיבו hearken אלי unto נאם me, saith יהוה׃ the LORD.}%
\verse{אבותיכם Your fathers, איה where הם they? והנבאים and the prophets, הלעולם forever? יחיו׃ do they live}%
\verse{אך But דברי my words וחקי and my statutes, אשׁר which צויתי I commanded את עבדי my servants הנביאים the prophets, הלוא did they not השׂיגו take hold אבתיכם of your fathers? וישׁובו and they returned ויאמרו and said, כאשׁר Like as זמם thought יהוה the LORD צבאות of hosts לעשׂות to do לנו כדרכינו unto us, according to our ways, וכמעללינו and according to our doings, כן so עשׂה hath he dealt אתנו׃ with}%
\verseWithHeading{Zechariah’s First Vision}{ביום day עשׂרים and twentieth וארבעה Upon the four לעשׁתי of the eleventh עשׂר of the eleventh חדשׁ month, הוא which חדשׁ the month שׁבט Sebat, בשׁנת year שׁתים in the second לדריושׁ of Darius, היה came דבר the word יהוה of the LORD אל unto זכריה Zechariah, בן the son ברכיהו of Berechiah, בן the son עדוא of Iddo הנביא the prophet, לאמר׃ saying,}%
\verse{ראיתי I saw הלילה by night, והנה and behold אישׁ a man רכב riding על upon סוס horse, אדם a red והוא and he עמד stood בין among ההדסים the myrtle trees אשׁר that במצלה in the bottom; ואחריו and behind סוסים horses, אדמים him red שׂרקים speckled, ולבנים׃ and white.}%
\verse{ואמר Then said מה what אלה these? אדני I, O my lord, ויאמר with me said אלי unto המלאך And the angel הדבר that talked בי אני me, I אראך will show מה thee what המה אלה׃ these}%
\verse{ויען answered האישׁ And the man העמד that stood בין among ההדסים the myrtle trees ויאמר and said, אלה These אשׁר whom שׁלח hath sent יהוה the LORD להתהלך to walk to and fro בארץ׃ through the earth.}%
\verse{ויענו And they answered את מלאך the angel יהוה העמד that stood בין among ההדסים the myrtle trees, ויאמרו and said, התהלכנו We have walked to and fro בארץ through the earth, והנה and, behold, כל all הארץ the earth ישׁבת sitteth still, ושׁקטת׃ and is at rest.}%
\verse{ויען answered מלאך Then the angel יהוה of the LORD ויאמר and said, יהוה O LORD צבאות of hosts, עד how long מתי how long אתה wilt thou לא not תרחם have mercy on את ירושׁלם Jerusalem ואת ערי and on the cities יהודה of Judah, אשׁר against which זעמתה thou hast had indignation זה these שׁבעים threescore and ten שׁנה׃ years?}%
\verse{ויען answered יהוה And the LORD את המלאך the angel הדבר that talked בי דברים words טובים with me good דברים words. נחמים׃ comfortable}%
\verse{ויאמר with me said אלי unto המלאך So the angel הדבר that communed בי קרא me, Cry לאמר thou, saying, כה Thus אמר saith יהוה the LORD צבאות of hosts; קנאתי I am jealous לירושׁלם for Jerusalem ולציון and for Zion קנאה jealousy. גדולה׃ with a great}%
\verse{וקצף sore displeased גדול am very אני And I קצף sore displeased על with הגוים the heathen השׁאננים at ease: אשׁר for אני I קצפתי displeased, מעט was but a little והמה and they עזרו helped לרעה׃ forward the affliction.}%
\verse{לכן Therefore כה thus אמר saith יהוה the LORD; שׁבתי I am returned לירושׁלם to Jerusalem ברחמים with mercies: ביתי my house יבנה shall be built בה נאם in it, saith יהוה the LORD צבאות of hosts, וקוה ינטה shall be stretched forth על upon ירושׁלם׃ Jerusalem.}%
\verse{עוד yet, קרא Cry לאמר saying, כה Thus אמר saith יהוה the LORD צבאות of hosts; עוד shall yet תפוצינה be spread abroad; ערי My cities מטוב through prosperity ונחם comfort יהוה and the LORD עוד shall yet את ציון Zion, ובחר choose עוד and shall yet בירושׁלם׃ Jerusalem.}%
\verseWithHeading{Zechariah’s Second Vision}{ואשׂא Then lifted I up את עיני mine eyes, וארא and saw, והנה and behold ארבע four קרנות׃ horns.}%
\verse{ואמר And I said אל unto המלאך the angel הדבר that talked בי מה with me, What אלה these? ויאמר אלילה me, These הקרנות the horns אשׁר which זרו have scattered את יהודה Judah, את ישׂראל Israel, וירושׁלם׃ and Jerusalem.}%
\verse{ויראני showed יהוה And the LORD ארבעה me four חרשׁים׃ carpenters.}%
\verse{ואמר Then said מה I, What אלה these באים come לעשׂות to do? ויאמר And he spoke, לאמר saying, אלה These הקרנות the horns אשׁר which זרו have scattered את יהודה Judah, כפי אישׁ man לא so that no נשׂא did lift up ראשׁו his head: ויבאו are come אלה but these להחריד to frighten אתם לידות them, to cast out את קרנות the horns הגוים of the Gentiles, הנשׂאים which lifted up קרן horn אל over ארץ the land יהודה of Judah לזרותה׃ to scatter}%
\end{biblechapter}%
\begin{biblechapter}% Zechariah 2
\verseWithHeading{Zechariah’s Third Vision}{ואשׂא I lifted up עיני mine eyes וארא again, and looked, והנה and behold אישׁ a man ובידו in his hand. חבל line מדה׃ with a measuring}%
\verse{ואמר Then said אנה I, Whither אתה thou? הלך goest ויאמר And he said אלי unto למד me, To measure את ירושׁלם Jerusalem, לראות to see כמה what רחבה the breadth וכמה thereof, and what ארכה׃ the length}%
\verse{והנה And, behold, המלאך the angel הדבר that talked בי יצא with me went forth, ומלאך angel אחר and another יצא went out לקראתו׃ to meet}%
\verse{ויאמר And said אלו unto רץ him, Run, דבר speak אל to הנער young man, הלז this לאמר saying, פרזות towns without walls תשׁב shall be inhabited ירושׁלם Jerusalem מרב for the multitude אדם of men ובהמה and cattle בתוכה׃ therein:}%
\verse{ואני For I, אהיה will be לה נאם saith יהוה חומת unto her a wall אשׁ of fire סביב round about, ולכבוד the glory אהיה and will be בתוכה׃ in the midst}%
\verse{הוי Ho, הוי ho, ונסו and flee מארץ from the land צפון of the north, נאם saith יהוה the LORD: כי for כארבע as the four רוחות winds השׁמים of the heaven, פרשׂתי אתכםאם saith יהוה׃ the LORD.}%
\verse{הוי O ציון Zion, המלטי Deliver thyself, יושׁבת that dwellest בת the daughter בבל׃ of Babylon.}%
\verse{כי For כה thus אמר saith יהוה the LORD צבאות of hosts; אחר After כבוד the glory שׁלחני hath he sent אל me unto הגוים the nations השׁללים which spoiled אתכם כי you: for הנגע he that toucheth בכם נגע you toucheth בבבת the apple עינו׃ of his eye.}%
\verse{כי For, הנני מניף I will shake את ידי mine hand עליהם upon והיו them, and they shall be שׁלל a spoil לעבדיהם to their servants: וידעתם and ye shall know כי that יהוה the LORD צבאות of hosts שׁלחני׃ hath sent}%
\verse{רני Sing ושׂמחי and rejoice, בת O daughter ציון of Zion: כי for, הנני בא I come, ושׁכנתי and I will dwell בתוכך in the midst נאם of thee, saith יהוה׃ the LORD.}%
\verse{ונלוו shall be joined גוים nations רבים And many אל to יהוה the LORD ביום day, ההוא in that והיו and shall be לי לעם my people: ושׁכנתי and I will dwell בתוכך in the midst וידעת of thee, and thou shalt know כי that יהוה the LORD צבאות of hosts שׁלחני hath sent אליך׃ me unto}%
\verse{ונחל shall inherit יהוה And the LORD את יהודה Judah חלקו his portion על in אדמת land, הקדשׁ the holy ובחר and shall choose עוד again. בירושׁלם׃ Jerusalem}%
\verse{הס Be silent, כל O all בשׂר flesh, מפני before יהוה the LORD: כי for נעור he is raised up ממעון habitation. קדשׁו׃ out of his holy}%
\end{biblechapter}%
\begin{biblechapter}% Zechariah 3
\verseWithHeading{Zechariah’s Fourth Vision}{ויראני And he showed את יהושׁע me Joshua הכהן priest הגדול the high עמד standing לפני before מלאך the angel יהוה of the LORD, והשׂטן and Satan עמד standing על at ימינו his right hand לשׂטנו׃ to resist}%
\verse{ויאמר said יהוה And the LORD אל unto השׂטן Satan, יגער rebuke יהוה The LORD בך השׂטן thee, O Satan; ויגער rebuke יהוה even the LORD בך הבחר that hath chosen בירושׁלם Jerusalem הלוא thee: not זה this אוד a brand מצל plucked מאשׁ׃ out of the fire?}%
\verse{ויהושׁע Now Joshua היה was לבשׁ clothed בגדים garments, צואים with filthy ועמד and stood לפני before המלאך׃ the angel.}%
\verse{ויען And he answered ויאמר and spoke אל unto העמדים those that stood לפניו before לאמר him, saying, הסירו Take away הבגדים garments הצאים the filthy מעליו from ויאמר him he said, אליו him. And unto ראה Behold, העברתי to pass מעליך from עונך I have caused thine iniquity והלבשׁ thee, and I will clothe אתך מחלצות׃ thee with change of raiment.}%
\verse{ואמר And I said, ישׂימו Let them set צניף miter טהור a fair על upon ראשׁו his head. וישׂימו So they set הצניף miter הטהור a fair על upon ראשׁו his head, וילבשׁהו and clothed בגדים him with garments. ומלאך And the angel יהוה of the LORD עמד׃ stood by.}%
\verse{ויעד protested מלאך And the angel יהוה ביהושׁע unto Joshua, לאמר׃ saying,}%
\verse{כה Thus אמר saith יהוה the LORD צבאות of hosts; אם If בדרכי in my ways, תלך thou wilt walk ואם and if את משׁמרתי my charge, תשׁמר thou wilt keep וגם shalt also אתה then thou תדין judge את ביתי my house, וגם and shalt also תשׁמר keep את חצרי my courts, ונתתי and I will give לך מהלכים thee places to walk בין among העמדים that stand האלה׃ these}%
\verse{שׁמע Hear נא now, יהושׁע O Joshua הכהן priest, הגדול the high אתה thou, ורעיך and thy fellows הישׁבים that sit לפניך before כי thee: for אנשׁי men מופת wondered המה they כי at: for, הנני מביא I will bring forth את עבדי my servant צמח׃ the BRANCH.}%
\verse{כי For הנה behold האבן the stone אשׁר that נתתי I have laid לפני before יהושׁע Joshua; על upon אבן stone אחת one שׁבעה seven עינים eyes: הנני מפתח I will engrave פתחה the graving נאם thereof, saith יהוה the LORD צבאות of hosts, ומשׁתי and I will remove את עון the iniquity הארץ land ההיא of that ביום day. אחד׃ in one}%
\verse{ביום day, ההוא In that נאם saith יהוה the LORD צבאות of hosts, תקראו shall ye call אישׁ every man לרעהו his neighbor אל under תחת under גפן the vine ואל and under תחת and under תאנה׃ the fig tree.}%
\end{biblechapter}%
\begin{biblechapter}% Zechariah 4
\verseWithHeading{Zechariah’s Fifth Vision}{וישׁב with me came again, המלאך And the angel הדבר that talked בי ויעירני and waked כאישׁ me, as a man אשׁר that יעור is wakened משׁנתו׃ out of his sleep,}%
\verse{ויאמר And said אלי unto מה me, What אתה thou? ראה seest ויאמר And I said, ראיתי I have looked, והנה and behold מנורת a candlestick זהב gold, כלה all וגלה with a bowl על upon ראשׁה the top ושׁבעה of it, and his seven נרתיה lamps עליה thereon, שׁבעה and seven ושׁבעה to the seven מוצקות pipes לנרות lamps, אשׁר which על upon ראשׁה׃ the top}%
\verse{ושׁנים And two זיתים olive trees עליה by אחד it, one מימין upon the right הגלה of the bowl, ואחד and the other על upon שׂמאלה׃ the left}%
\verse{ואען So I answered ואמר and spoke אל to המלאך the angel הדבר that talked בי לאמר with me, saying, מה What אלה these, אדני׃ my lord?}%
\verse{ויען with me answered המלאך Then the angel הדבר that talked בי ויאמר and said אלי unto הלוא thou not ידעת me, Knowest מה what המה אלה these ואמר be? And I said, לא No, אדני׃ my lord.}%
\verse{ויען Then he answered ויאמר and spoke אלי unto לאמר me, saying, זה This דבר the word יהוה of the LORD אל unto זרבבל Zerubbabel, לאמר saying, לא Not בחיל by might, ולא nor בכח by power, כי but אם but ברוחי by my spirit, אמר saith יהוה the LORD צבאות׃ of hosts.}%
\verse{מי Who אתה thou, הר mountain? הגדול O great לפני before זרבבל Zerubbabel למישׁר a plain: והוציא and he shall bring forth את האבן the headstone הראשׁה the headstone תשׁאות shoutings, חן Grace, חן׃ grace}%
\verse{ויהי came דבר Moreover the word יהוה of the LORD אלי unto לאמר׃ me, saying,}%
\verse{ידי The hands זרבבל of Zerubbabel יסדו הבית house; הזה of this וידיו his hands תבצענה shall also finish וידעת it; and thou shalt know כי that יהוה the LORD צבאות of hosts שׁלחני hath sent אליכם׃ me unto}%
\verse{כי For מי who בז ליום the day קטנות of small things? ושׂמחו for they shall rejoice, וראו and shall see את האבן the plummet הבדיל the plummet ביד in the hand זרבבל of Zerubbabel שׁבעה seven; אלה those עיני the eyes יהוה of the LORD, המה they משׁוטטים which run to and fro בכל through the whole הארץ׃ earth.}%
\verse{ואען Then answered ואמר I, and said אליו unto מה him, What שׁני two הזיתים olive trees האלה these על upon ימין the right המנורה of the candlestick ועל and upon שׂמאולה׃ the left}%
\verse{ואען And I answered שׁנית again, ואמר and said אליו unto מה him, What שׁתי two שׁבלי branches הזיתים olive אשׁר which ביד through שׁני the two צנתרות pipes הזהב golden המריקים empty מעליהם out of הזהב׃ the golden}%
\verse{ויאמר me and said, אלי לאמר And I said, הלוא thou not ידעת Knowest מה what אלה these ואמר לא No, אדני׃ my lord.}%
\verse{ויאמר Then said אלה he, These שׁני the two בני anointed ones, היצהר anointed ones, העמדים that stand על by אדון the Lord כל of the whole הארץ׃ earth.}%
\end{biblechapter}%
\begin{biblechapter}% Zechariah 5
\verseWithHeading{Zechariah’s Sixth Vision}{ואשׁוב Then I turned, ואשׂא and lifted up עיני mine eyes, ואראה and looked, והנה and behold מגלה roll. עפה׃ a flying}%
\verse{ויאמר And he said אלי unto מה me, What אתה thou? ראה seest ואמר And I answered, אני I ראה see מגלה roll; עפה a flying ארכה the length עשׂרים thereof twenty באמה cubits, ורחבה and the breadth עשׂר thereof ten באמה׃ cubits.}%
\verse{ויאמר Then said אלי he unto זאת me, This האלה the curse היוצאת that goeth forth על over פני the face כל of the whole הארץ earth: כי for כל every one הגנב that stealeth מזה on this side כמוה according to it; נקה shall be cut off וכל and every הנשׁבע one that sweareth מזה on that side כמוה according to it. נקה׃ shall be cut off}%
\verse{הוצאתיה I will bring it forth, נאם saith יהוה צבאות of hosts, ובאה and it shall enter אל into בית the house הגנב of the thief, ואל and into בית the house הנשׁבע of him that sweareth בשׁמי by my name: לשׁקר falsely ולנה and it shall remain בתוך in the midst ביתו of his house, וכלתו and shall consume ואת עציו it with the timber ואת אבניו׃ thereof and the stones}%
\verseWithHeading{Zechariah’s Seventh Vision}{ויצא with me went forth, המלאך Then the angel הדבר that talked בי ויאמר and said אלי unto שׂא me, Lift up נא now עיניך thine eyes, וראה and see מה what היוצאת that goeth forth. הזאת׃ this}%
\verse{ואמר And I said, מה What היא it? ויאמר And he said, זאת This האיפה an ephah היוצאת that goeth forth. ויאמר He said זאת moreover, This עינם their resemblance בכל through all הארץ׃ the earth.}%
\verse{והנה And, behold, ככר a talent עפרת of lead: נשׂאת there was lifted up וזאת and this אשׁה woman אחת a יושׁבת that sitteth בתוך in the midst האיפה׃ of the ephah.}%
\verse{ויאמר And he said, זאת This הרשׁעה wickedness. וישׁלך And he cast אתה אל it into תוך the midst האיפה of the ephah; וישׁלך and he cast את אבן the weight העפרת of lead אל upon פיה׃ the mouth}%
\verse{ואשׂא Then lifted I up עיני mine eyes, וארא and looked, והנה and, behold, שׁתים two נשׁים women, יוצאות there came out ורוח and the wind בכנפיהם in their wings; ולהנה for they כנפים had wings ככנפי like the wings החסידה of a stork: ותשׂאנה and they lifted up את האיפה the ephah בין between הארץ the earth ובין השׁמים׃ and the heaven.}%
\verse{ואמר Then said אל I to המלאך the angel הדבר that talked בי אנה with me, Whither המה do these מולכות bear את האיפה׃ the ephah?}%
\verse{ויאמר And he said אלי unto לבנות me, To build לה בית it a house בארץ in the land שׁנער of Shinar: והוכן and it shall be established, והניחה שׁם there על upon מכנתה׃ her own base.}%
\end{biblechapter}%
\begin{biblechapter}% Zechariah 6
\verseWithHeading{Zechariah’s Eighth Vision}{ואשׁב And I turned, ואשׂא and lifted up עיני mine eyes, ואראה and looked, והנה and, behold, ארבע מרכבותצאות there came four chariots out מבין from between שׁני two ההרים mountains; וההרים and the mountains הרי mountains נחשׁת׃ of brass.}%
\verse{במרכבה chariot הראשׁנה In the first סוסים horses; אדמים red ובמרכבה chariot השׁנית and in the second סוסים horses; שׁחרים׃ black}%
\verse{ובמרכבה chariot השׁלשׁית And in the third סוסים horses; לבנים white ובמרכבה chariot הרבעית and in the fourth סוסים horses. ברדים grizzled אמצים׃ and bay}%
\verse{ואען Then I answered ואמר and said אל unto המלאך the angel הדבר that talked בי מה with me, What אלה these, אדני׃ my lord?}%
\verse{ויען answered המלאך And the angel ויאמר and said אלי unto אלה me, These ארבע the four רחות spirits השׁמים of the heavens, יוצאות which go forth מהתיצב from standing על before אדון the Lord כל of all הארץ׃ the earth.}%
\verse{אשׁר which בה הסוסים horses השׁחרים The black יצאים therein go forth אל into ארץ country; צפון the north והלבנים and the white יצאו go forth אל after אחריהם after והברדים them; and the grizzled יצאו go forth אל toward ארץ country. התימן׃ the south}%
\verse{והאמצים And the bay יצאו went forth, ויבקשׁו and sought ללכת to go להתהלך that they might walk to and fro בארץ through the earth: ויאמר and he said, לכו Get you hence, התהלכו walk to and fro בארץ through the earth. ותתהלכנה So they walked to and fro בארץ׃ through the earth.}%
\verse{ויזעק Then cried אתי וידבר he upon me, and spoke אלי unto לאמר me, saying, ראה Behold, היוצאים these that go אל toward ארץ country צפון the north הניחו have quieted את רוחי my spirit בארץ country. צפון׃ in the north}%
\verseWithHeading{The Branch is Crowned}{ויהי came דבר And the word יהוה of the LORD אלי unto לאמר׃ me, saying,}%
\verse{לקוח Take מאת הגולה the captivity, מחלדי ומאתוביה Tobijah, ומאת ידעיה Jedaiah, ובאת are come אתה thou ביום day, ההוא the same ובאת and come בית into the house יאשׁיה of Josiah בן the son צפניה of Zephaniah; אשׁר which באו and go מבבל׃}%
\verse{ולקחת Then take כסף silver וזהב and gold, ועשׂית and make עטרות crowns, ושׂמת and set בראשׁ upon the head יהושׁע of Joshua בן the son יהוצדק of Josedech, הכהן priest; הגדול׃ the high}%
\verse{ואמרת And speak אליו unto לאמר him, saying, כה Thus אמר speaketh יהוה the LORD צבאות of hosts, לאמר saying, הנה Behold אישׁ the man צמח The BRANCH; שׁמו whose name ומתחתיו out of his place, יצמח and he shall grow up ובנה and he shall build את היכל the temple יהוה׃ of the LORD:}%
\verse{והוא Even he יבנה shall build את היכל the temple יהוה of the LORD; והוא and he ישׂא shall bear הוד the glory, וישׁב and shall sit ומשׁל and rule על upon כסאו his throne; והיה and he shall be כהן a priest על upon כסאו his throne: ועצת and the counsel שׁלום of peace תהיה shall be בין between שׁניהם׃ them both.}%
\verse{והעטרת And the crowns תהיה shall be לחלם to Helem, ולטוביה and to Tobijah, ולידעיה and to Jedaiah, ולחן and to Hen בן the son צפניה of Zephaniah, לזכרון for a memorial בהיכל in the temple יהוה׃ of the LORD.}%
\verse{ורחוקים And they far off יבאו shall come ובנו and build בהיכל in the temple יהוה of the LORD, וידעתם and ye shall know כי that יהוה the LORD צבאות of hosts שׁלחני hath sent אליכם me unto והיה you. And shall come to pass, אם if שׁמוע ye will diligently obey תשׁמעון ye will diligently obey בקול the voice יהוה of the LORD אלהיכם׃ your God.}%
\end{biblechapter}%
\begin{biblechapter}% Zechariah 7
\verse{ויהי And it came to pass בשׁנת year ארבע in the fourth לדריושׁ Darius, המלך of king היה came דבר the word יהוה of the LORD אל unto זכריה Zechariah בארבעה in the fourth לחדשׁ month, התשׁעי of the ninth בכסלו׃ in Chisleu;}%
\verse{וישׁלח When they had sent בית אל unto the house of God שׂר אצר Sherezer ורגם מלך and Regem-melech, ואנשׁיו לחלות to pray את פני before יהוה׃}%
\verse{לאמר to speak אל unto הכהנים the priests אשׁר which לבית in the house יהוה of the LORD צבאות of hosts, ואל and to הנביאים the prophets, לאמר saying, האבכה Should I weep בחדשׁ month, החמשׁי in the fifth הנזר separating myself, כאשׁר as עשׂיתי I have done זה these כמה so many שׁנים׃ years?}%
\verse{ויהי Then came דבר the word יהוה of the LORD צבאות of hosts אלי unto לאמר׃ me, saying,}%
\verse{אמר Speak אל unto כל all עם the people הארץ of the land, ואל and to הכהנים the priests, לאמר saying, כי When צמתם ye fasted וספוד and mourned בחמישׁי in the fifth ובשׁביעי and seventh וזה even those שׁבעים seventy שׁנה years, הצום did ye at all fast צמתני did ye at all fast אני׃ unto me, to me?}%
\verse{וכי And when תאכלו ye did eat, וכי and when תשׁתו ye did drink, הלוא did not אתם ye האכלים eat ואתם השׁתים׃ and drink}%
\verse{הלוא not את הדברים the words אשׁר which קרא hath cried יהוה the LORD ביד by הנביאים prophets, הראשׁנים the former בהיות was ירושׁלם when Jerusalem ישׁבת inhabited ושׁלוה and in prosperity, ועריה and the cities סביבתיה thereof round about והנגב the south והשׁפלה and the plain? ישׁב׃ her, when inhabited}%
\verse{ויהי came דבר And the word יהוה of the LORD אל unto זכריה Zechariah, לאמר׃ saying,}%
\verse{כה Thus אמר speaketh יהוה the LORD צבאות of hosts, לאמר saying, משׁפט judgment, אמת true שׁפטו Execute וחסד mercy ורחמים and compassions עשׂו and show אישׁ every man את to אחיו׃ his brother:}%
\verse{ואלמנה the widow, ויתום nor the fatherless, גר the stranger, ועני nor the poor; אל not תעשׁקו And oppress ורעת evil אישׁ evil אחיו against his brother אל and let none תחשׁבו of you imagine בלבבכם׃ in your heart.}%
\verse{וימאנו But they refused להקשׁיב to hearken, ויתנו and pulled כתף the shoulder, סררת away ואזניהם their ears, הכבידו and stopped משׁמוע׃ that they should not hear.}%
\verse{ולבם their hearts שׂמו Yea, they made שׁמיר an adamant stone, משׁמוע lest they should hear את התורה the law, ואת הדברים and the words אשׁר which שׁלח hath sent יהוה the LORD צבאות of hosts ברוחו in his spirit ביד by הנביאים prophets: הראשׁנים the former ויהי therefore came קצף wrath גדול a great מאת יהוה the LORD צבאות׃ of hosts.}%
\verse{ויהי Therefore it is come to pass, כאשׁר as קרא he cried, ולא and they would not שׁמעו hear; כן so יקראו they cried, ולא and I would not אשׁמע hear, אמר saith יהוה the LORD צבאות׃ of hosts:}%
\verse{ואסערם But I scattered them with a whirlwind על among כל all הגוים the nations אשׁר whom לא not. ידעום they knew והארץ Thus the land נשׁמה desolate. אחריהם after מעבר them, that no man passed through ומשׁב nor returned: וישׂימו for they laid ארץ land חמדה the pleasant לשׁמה׃ was desolate}%
\end{biblechapter}%
\begin{biblechapter}% Zechariah 8
\verseWithHeading{Adonai’s Promise to Jerusalem}{ויהי came דבר Again the word יהוה of the LORD צבאות of hosts לאמר׃ saying,}%
\verse{כה Thus אמר saith יהוה the LORD צבאות of hosts; קנאתי I was jealous לציון for Zion קנאה jealousy, גדולה with great וחמה fury. גדולה for her with great קנאתי׃ and I was jealous}%
\verse{כה Thus אמר saith יהוה the LORD; שׁבתי I am returned אל unto ציון Zion, ושׁכנתי and will dwell בתוך in the midst ירושׁלם of Jerusalem: ונקראה shall be called ירושׁלם and Jerusalem עיר a city האמת of truth; והר and the mountain יהוה of the LORD צבאות of hosts הר mountain. הקדשׁ׃ the holy}%
\verse{כה Thus אמר saith יהוה the LORD צבאות of hosts; עד There shall yet ישׁבו dwell זקנים old men וזקנות and old women ברחבות in the streets ירושׁלם of Jerusalem, ואישׁ and every man משׁענתו with his staff בידו in his hand מרב for very ימים׃ age.}%
\verse{ורחבות And the streets העיר of the city ימלאו shall be full ילדים of boys וילדות and girls משׂחקים playing ברחבתיה׃ in the streets}%
\verse{כה Thus אמר saith יהוה צבאות of hosts; כי If יפלא it be marvelous בעיני in the eyes שׁארית of the remnant העם people הזה of this בימים days, ההם in these גם should it also בעיני in mine eyes? יפלא be marvelous נאם saith יהוה צבאות׃ of hosts.}%
\verse{כה Thus אמר saith יהוה the LORD צבאות of hosts; הנני מושׁיע I will save את עמי my people מארץ country, מזרח from the east ומארץ country; מבוא and from the west השׁמשׁ׃ and from the west}%
\verse{והבאתי And I will bring אתם ושׁכנו them, and they shall dwell בתוך in the midst ירושׁלם of Jerusalem: והיו and they shall be לי לעם my people, ואני and I אהיה will be להם לאלהים their God, באמת in truth ובצדקה׃ and in righteousness.}%
\verse{כה Thus אמר saith יהוה the LORD צבאות of hosts; תחזקנה be strong, ידיכם Let your hands השׁמעים ye that hear בימים days האלה in these את הדברים words האלה these מפי by the mouth הנביאים of the prophets, אשׁר which ביום in the day יסד was laid, בית the foundation of the house יהוה of the LORD צבאות of hosts ההיכל that the temple להבנות׃ might be built.}%
\verse{כי For לפני before הימים days ההם these שׂכר hire האדם for man, לא no נהיה there was ושׂכר any hire הבהמה for beast; איננה nor וליוצא to him that went out ולבא or came in אין neither שׁלום peace מן because of הצר the affliction: ואשׁלח for I set את כל all האדם men אישׁ every one ברעהו׃ against his neighbor.}%
\verse{ועתה But now לא not כימים days, הראשׁנים as in the former אני I לשׁארית unto the residue העם people הזה of this נאם saith יהוה the LORD צבאות׃ of hosts.}%
\verse{כי For זרע the seed השׁלום prosperous; הגפן the vine תתן shall give פריה her fruit, והארץ and the ground תתן shall give את יבולה her increase, והשׁמים and the heavens יתנו shall give טלם their dew; והנחלתי to possess את שׁארית and I will cause the remnant העם people הזה of this את כל all אלה׃ these}%
\verse{והיה And it shall come to pass, כאשׁר as הייתם ye were קללה a curse בגוים among the heathen, בית O house יהודה of Judah, ובית and house ישׂראל of Israel; כן so אושׁיע will I save אתכם והייתם you, and ye shall be ברכה a blessing: אל not, תיראו fear תחזקנה be strong. ידיכם׃ let your hands}%
\verse{כי For כה thus אמר saith יהוה the LORD צבאות of hosts; כאשׁר As זממתי I thought להרע to punish לכם בהקציף אבתיכם you, when your fathers אתי אמר saith יהוה the LORD צבאות of hosts, ולא not: נחמתי׃ and I repented}%
\verse{כן So שׁבתי again זממתי have I thought בימים days האלה in these להיטיב to do well את ירושׁלם unto Jerusalem ואת בית and to the house יהודה of Judah: אל ye not. תיראו׃ fear}%
\verse{אלה These הדברים the things אשׁר that תעשׂו ye shall do; דברו Speak אמת the truth אישׁ ye every man את to רעהו his neighbor; אמת of truth ומשׁפט the judgment שׁלום and peace שׁפטו execute בשׁעריכם׃ in your gates:}%
\verse{ואישׁ אתעת evil רעהו against his neighbor; אל no תחשׁבו of you imagine בלבבכם in your hearts ושׁבעת oath: שׁקר false אל תאהבו and love כי for את כל all אלה these אשׁר that שׂנאתי I hate, נאם saith יהוה׃ the LORD.}%
\verse{ויהי came דבר And the word יהוה of the LORD צבאות of hosts אלי unto לאמר׃ me, saying,}%
\verse{כה Thus אמר saith יהוה the LORD צבאות of hosts; צום The fast הרביעי of the fourth וצום and the fast החמישׁי of the fifth, וצום and the fast השׁביעי of the seventh, וצום and the fast העשׂירי of the tenth, יהיה shall be לבית to the house יהודה of Judah לשׂשׂון joy ולשׂמחה and gladness, ולמעדים feasts; טובים and cheerful והאמת the truth והשׁלום and peace. אהבו׃ therefore love}%
\verse{כה Thus אמר saith יהוה the LORD צבאות of hosts; עד yet אשׁר that יבאו there shall come עמים people, וישׁבי and the inhabitants ערים cities: רבות׃ of many}%
\verse{והלכו shall go ישׁבי And the inhabitants אחת of one אל to אחת another, לאמר saying, נלכה will go הלוך לחלות to pray את פני before יהוה the LORD, ולבקשׁ and to seek את יהוה the LORD צבאות of hosts: אלכה גם also. אני׃ I}%
\verse{ובאו shall come עמים people רבים Yea, many וגוים nations עצומים and strong לבקשׁ to seek את יהוה the LORD צבאות of hosts בירושׁלם in Jerusalem, ולחלות and to pray את פני before יהוה׃ the LORD.}%
\verse{כה Thus אמר saith יהוה the LORD צבאות of hosts; בימים days ההמה In those אשׁר that יחזיקו shall take hold עשׂרה ten אנשׁים men מכל out of all לשׁנות languages הגוים of the nations, והחזיקו even shall take hold בכנף of the skirt אישׁ of him יהודי that is a Jew, לאמר saying, נלכה We will go עמכם with כי you: for שׁמענו we have heard אלהים God עמכם׃ with}%
\end{biblechapter}%
\begin{biblechapter}% Zechariah 9
\verse{משׂא The burden דבר of the word יהוה of the LORD בארץ in the land חדרך of Hadrach, ודמשׂק and Damascus מנחתו the rest כי thereof: when ליהוה toward the LORD. עין the eyes אדם of man, וכל as of all שׁבטי the tribes ישׂראל׃ of Israel,}%
\verse{וגם also חמת And Hamath תגבל shall border בה צר thereby; Tyrus, וצידון and Zidon, כי though חכמה wise. מאד׃ it be very}%
\verse{ותבן did build צר And Tyrus מצור herself a stronghold, לה ותצבר and heaped up כסף silver כעפר as the dust, וחרוץ and fine gold כטיט as the mire חוצות׃ of the streets.}%
\verse{הנה Behold, אדני the Lord יורשׁנה will cast her out, והכה and he will smite בים in the sea; חילה her power והיא and she באשׁ with fire. תאכל׃ shall be devoured}%
\verse{תרא shall see אשׁקלון Ashkelon ותירא and fear; ועזה Gaza ותחיל , and be very sorrowful, מאד also and be very sorrowful, ועקרון and Ekron; כי for הבישׁ shall be ashamed; מבטה her expectation ואבד shall perish מלך and the king מעזה ואשׁקלון and Ashkelon לא shall not תשׁב׃ be inhabited.}%
\verse{וישׁב shall dwell ממזר And a bastard באשׁדוד in Ashdod, והכרתי and I will cut off גאון the pride פלשׁתים׃ of the Philistines.}%
\verse{והסרתי And I will take away דמיו his blood מפיו out of his mouth, ושׁקציו and his abominations מבין from between שׁניו his teeth: ונשׁאר but he that remaineth, גם even הוא he, לאלהינו for our God, והיה and he shall be כאלף as a governor ביהודה in Judah, ועקרון and Ekron כיבוסי׃ as a Jebusite.}%
\verse{וחניתי And I will encamp לביתי about mine house מצבה because of the army, מעבר because of him that passeth by, ומשׁב and because of him that returneth: ולא and no יעבר shall pass עליהם through עוד them any more: נגשׂ oppressor כי for עתה now ראיתי have I seen בעיני׃ with mine eyes.}%
\verse{גילי Rejoice מאד greatly, בת O daughter ציון of Zion; הריעי shout, בת O daughter ירושׁלם of Jerusalem: הנה behold, מלכך thy King יבוא cometh לך צדיק just, ונושׁע and having salvation; הוא unto thee: he עני lowly, ורכב and riding על upon חמור an ass, ועל and upon עיר a colt בן the foal אתנות׃ of an ass.}%
\verse{והכרתי And I will cut off רכב the chariot מאפרים וסוס and the horse מירושׁלם ונכרתה shall be cut off: קשׁת bow מלחמה and the battle ודבר and he shall speak שׁלום peace לגוים unto the heathen: ומשׁלו and his dominion מים from sea עד to ים sea, ומנהר and from the river עד to אפסי the ends ארץ׃ of the earth.}%
\verse{גם also, את As for thee בדם by the blood בריתך of thy covenant שׁלחתי I have sent forth אסיריך thy prisoners מבור out of the pit אין wherein no מים׃ water.}%
\verse{שׁובו Turn לבצרון you to the stronghold, אסירי ye prisoners התקוה of hope: גם even היום today מגיד do I declare משׁנה double אשׁיב׃ I will render}%
\verse{כי When דרכתי I have bent לי יהודה Judah קשׁת the bow מלאתי for me, filled אפרים with Ephraim, ועוררתי and raised בניך up thy sons, ציון O Zion, על against בניך thy sons, יון O Greece, ושׂמתיך and made כחרב thee as the sword גבור׃ of a mighty man.}%
\verse{ויהוה And the LORD עליהם over יראה shall be seen ויצא shall go forth כברק as the lightning: חצו them, and his arrow ואדני and the Lord יהוה GOD בשׁופר the trumpet, יתקע shall blow והלך and shall go בסערות with whirlwinds תימן׃ of the south.}%
\verse{יהוה צבאות of hosts יגן shall defend עליהם shall defend ואכלו them; and they shall devour, וכבשׁו and subdue אבני with slingstones; קלע with slingstones; ושׁתו and they shall drink, המו make a noise כמו as through יין wine; ומלאו and they shall be filled כמזרק like bowls, כזויות as the corners מזבח׃ of the altar.}%
\verse{והושׁיעם shall save יהוה And the LORD אלהיהם their God ביום day ההוא them in that כצאן as the flock עמו of his people: כי for אבני they the stones נזר of a crown, מתנוססות lifted up as an ensign על upon אדמתו׃ his land.}%
\verse{כי For מה how טובו great his goodness, ומה and how יפיו great his beauty! דגן corn בחורים shall make the young men ותירושׁ and new wine ינובב cheerful, בתלות׃ the maids.}%
\end{biblechapter}%
\begin{biblechapter}% Zechariah 10
\verseWithHeading{Adonai Gives Rain}{שׁאלו Ask מיהוה מטר rain בעת in the time מלקושׁ of the latter rain; יהוה עשׂה shall make חזיזים bright clouds, ומטר of rain, גשׁם them showers יתן and give להם לאישׁ to every one עשׂב grass בשׂדה׃ in the field.}%
\verse{כי For התרפים the idols דברו have spoken און vanity, והקוסמים and the diviners חזו have seen שׁקר a lie, וחלמות dreams; השׁוא false ידברו and have told הבל in vain: ינחמון they comfort על therefore כן therefore נסעו they went their way כמו as צאן a flock, יענו they were troubled, כי because אין no רעה׃ shepherd.}%
\verse{על against הרעים the shepherds, חרה was kindled אפי Mine anger ועל העתודים the goats: אפקוד hath visited כי for פקד יהוה the LORD צבאות of hosts את עדרו his flock את בית the house יהודה of Judah, ושׂם and hath made אותם כסוס horse הודו them as his goodly במלחמה׃ in the battle.}%
\verse{ממנו Out of פנה the corner, ממנו out of יתד him the nail, ממנו out of קשׁת bow, מלחמה him the battle ממנו out of יצא him came forth כל him every נוגשׂ oppressor יחדו׃ together.}%
\verse{והיו And they shall be כגברים as mighty בוסים which tread down בטיט in the mire חוצות of the streets במלחמה in the battle: ונלחמו and they shall fight, כי because יהוה the LORD עמם with והבישׁו רכבי them, and the riders סוסים׃ on horses}%
\verse{וגברתי And I will strengthen את בית the house יהודה of Judah, ואת בית the house יוסף of Joseph, אושׁיע and I will save והושׁבותים and I will bring them again כי for רחמתים I have mercy upon והיו them: and they shall be כאשׁר as though לא I had not זנחתים cast them off: כי for אני I יהוה the LORD אלהיהם their God, ואענם׃ and will hear}%
\verse{והיו shall be כגבור like a mighty אפרים And Ephraim ושׂמח shall rejoice לבם and their heart כמו as through יין wine: ובניהם yea, their children יראו shall see ושׂמחו יגל shall rejoice לבם their heart ביהוה׃ in the LORD.}%
\verse{אשׁרקה I will hiss להם ואקבצם for them, and gather כי them; for פדיתים I have redeemed ורבו them: and they shall increase כמו as רבו׃ they have increased.}%
\verse{ואזרעם And I will sow בעמים them among the people: ובמרחקים me in far countries; יזכרוני and they shall remember וחיו and they shall live את with בניהם their children, ושׁבו׃ and turn again.}%
\verse{והשׁיבותים I will bring them again מארץ also out of the land מצרים of Egypt, ומאשׁור אקבצם and gather ואל them into ארץ the land גלעד of Gilead ולבנון and Lebanon; אביאם and I will bring ולא and shall not ימצא׃ be found}%
\verse{ועבר And he shall pass בים through the sea צרה with affliction, והכה and shall smite בים in the sea, גלים the waves והבישׁו shall dry up: כל and all מצולות the deeps יאר of the river והורד shall be brought down, גאון and the pride אשׁור of Assyria ושׁבט and the scepter מצרים of Egypt יסור׃ shall depart away.}%
\verse{וגברתים And I will strengthen ביהוה them in the LORD; ובשׁמו in his name, יתהלכו and they shall walk up and down נאם saith יהוה׃ the LORD.}%
\end{biblechapter}%
\begin{biblechapter}% Zechariah 11
\verse{פתח Open לבנון O Lebanon, דלתיך thy doors, ותאכל may devour אשׁ that the fire בארזיך׃ thy cedars.}%
\verse{הילל Howl, ברושׁ fir tree; כי for נפל is fallen; ארז the cedar אשׁר because אדרים the mighty שׁדדו are spoiled: הילילו howl, אלוני O ye oaks בשׁן of Bashan; כי for ירד is come down. יער the forest הבצור׃}%
\verse{קול a voice יללת of the howling הרעים of the shepherds; כי for שׁדדה is spoiled: אדרתם their glory קול a voice שׁאגת of the roaring כפירים of young lions; כי for שׁדד is spoiled. גאון the pride הירדן׃ of Jordan}%
\verseWithHeading{The Two Shepherds}{כה Thus אמר saith יהוה the LORD אלהי my God; רעה Feed את צאן the flock ההרגה׃ of the slaughter;}%
\verse{אשׁר Whose קניהן possessors יהרגן slay ולא them, and hold themselves not guilty: יאשׁמו them, and hold themselves not guilty: ומכריהן and they that sell יאמר them say, ברוך Blessed יהוה the LORD; ואעשׁר for I am rich: ורעיהם and their own shepherds לא them not. יחמול pity עליהן׃ pity}%
\verse{כי For לא I will no אחמול pity עוד more על pity ישׁבי the inhabitants הארץ of the land, נאם saith יהוה the LORD: והנה but, lo, אנכי I ממציא will deliver את האדם the men אישׁ every one ביד hand, רעהו into his neighbor's וביד and into the hand מלכו of his king: וכתתו and they shall smite את הארץ the land, ולא I will not אציל deliver מידם׃ and out of their hand}%
\verse{וארעה And I will feed את צאן the flock ההרגה of slaughter, לכן עניי you, O poor הצאן of the flock. ואקח And I took לי שׁני unto me two מקלות staves; לאחד the one קראתי I called נעם Beauty, ולאחד and the other קראתי I called חבלים Bands; וארעה and I fed את הצאן׃ the flock.}%
\verse{ואכחד also I cut off את שׁלשׁת Three הרעים shepherds בירח month; אחד in one ותקצר loathed נפשׁי and my soul בהם וגם also נפשׁם them, and their soul בחלה׃ abhorred}%
\verse{ואמר Then said לא I, I will not ארעה feed אתכם המתה you: that that dieth, תמות let it die; והנכחדת and that that is to be cut off, תכחד let it be cut off; והנשׁארות and let the rest תאכלנה eat אשׁה every one את בשׂר the flesh רעותה׃ of another.}%
\verse{ואקח And I took את מקלי my staff, את נעם Beauty, ואגדע אתוהפיר that I might break את בריתי my covenant אשׁר which כרתי I had made את with כל all העמים׃ the people.}%
\verse{ותפר And it was broken ביום day: ההוא in that וידעו me knew כן and so עניי the poor הצאן of the flock השׁמרים that waited upon אתי כי that דבר the word יהוה of the LORD. הוא׃ it}%
\verse{ואמר And I said אליהם unto אם them, If טוב בעיניכם ye think good, הבו give שׂכרי my price; ואם and if לא not, חדלו forbear. וישׁקלו So they weighed for את שׂכרי my price שׁלשׁים thirty כסף׃ of silver.}%
\verse{ויאמר said יהוה אלי unto השׁליכהו me, Cast אל it unto היוצר the potter: אדר a goodly היקר price אשׁר that יקרתי I was prised at מעליהם of ואקחה them. And I took שׁלשׁים the thirty הכסף of silver, ואשׁליך and cast אתו בית in the house יהוה אל them to היוצר׃ the potter}%
\verse{ואגדע Then I cut asunder את מקלי staff, השׁני mine other את החבלים Bands, להפר that I might break את האחוה the brotherhood בין between יהודה Judah ובין ישׂראל׃ and Israel.}%
\verse{ויאמר said יהוה And the LORD אלי unto עוד unto thee yet קח me, Take לך כלי the instruments רעה shepherd. אולי׃ of a foolish}%
\verse{כי For, הנה lo, אנכי I מקים will raise רעה up a shepherd בארץ in the land, הנכחדות those that be cut off, לא shall not יפקד visit הנער לא neither יבקשׁ shall seek והנשׁברת that that is broken, לא nor ירפא heal הנצבה that that standeth still: לא nor יכלכל feed ובשׂר the flesh הבריאה of the fat, יאכל but he shall eat ופרסיהן and tear their claws in pieces. יפרק׃ and tear their claws in pieces.}%
\verse{הוי Woe רעי shepherd האליל to the idol עזבי that leaveth הצאן the flock! חרב the sword על upon זרועו his arm, ועל and upon עין eye: ימינו his right זרעו his arm יבושׁ shall be clean dried up, תיבשׁ shall be clean dried up, ועין eye ימינו and his right כהה shall be utterly darkened. תכהה׃ shall be utterly darkened.}%
\end{biblechapter}%
\begin{biblechapter}% Zechariah 12
\verseWithHeading{The Day of Adonai}{משׂא The burden דבר of the word יהוה of the LORD על for ישׂראל Israel, נאם saith יהוה the LORD, נטה which stretcheth forth שׁמים the heavens, ויסד and layeth the foundation ארץ of the earth, ויצר and formeth רוח the spirit אדם of man בקרבו׃ within}%
\verse{הנה Behold, אנכי I שׂם will make את ירושׁלם Jerusalem סף a cup רעל of trembling לכל unto all העמים the people סביב round about, וגם both על against יהודה Judah יהיה when they shall be במצור in the siege על against ירושׁלם׃ Jerusalem.}%
\verse{והיה ביום day ההוא And in that אשׂים will I make את ירושׁלם Jerusalem אבן stone מעמסה a burdensome לכל for all העמים people: כל all עמסיה that burden שׂרוט themselves with it shall be cut in pieces, ישׂרטו themselves with it shall be cut in pieces, ונאספו be gathered together עליה against כל though all גויי the people הארץ׃ of the earth}%
\verse{ביום day, ההוא In that נאם saith יהוה the LORD, אכה I will smite כל every סוס horse בתמהון with astonishment, ורכבו and his rider בשׁגעון with madness: ועל upon בית the house יהודה of Judah, אפקח and I will open את עיני mine eyes וכל every סוס horse העמים of the people אכה and will smite בעורון׃ with blindness.}%
\verse{ואמרו shall say אלפי And the governors יהודה of Judah בלבם in their heart, אמצה my strength לי ישׁבי The inhabitants ירושׁלם of Jerusalem ביהוה in the LORD צבאות of hosts אלהיהם׃ their God.}%
\verse{ביום day ההוא In that אשׂים will I make את אלפי the governors יהודה of Judah ככיור like a hearth אשׁ of fire בעצים among the wood, וכלפיד and like a torch אשׁ of fire בעמיר in a sheaf; ואכלו and they shall devour על on ימין the right hand ועל and on שׂמאול the left: את כל all העמים the people סביב round about, וישׁבה shall be inhabited ירושׁלם and Jerusalem עוד again תחתיה in her own place, בירושׁלם׃ in Jerusalem.}%
\verse{והושׁיע also shall save יהוה The LORD את אהלי the tents יהודה of Judah בראשׁנה first, למען that לא do not תגדל magnify תפארת the glory בית of the house דויד of David ותפארת and the glory ישׁב of the inhabitants ירושׁלם of Jerusalem על against יהודה׃ Judah.}%
\verse{ביום day ההוא In that יגן defend יהוה shall the LORD בעד defend יושׁב the inhabitants ירושׁלם of Jerusalem; והיה shall be הנכשׁל and he that is feeble בהם ביום day ההוא among them at that כדויד as David; ובית and the house דויד of David כאלהים as God, כמלאך as the angel יהוה of the LORD לפניהם׃ before}%
\verse{והיה And it shall come to pass ביום day, ההוא in that אבקשׁ I will seek להשׁמיד to destroy את כל all הגוים the nations הבאים that come על against ירושׁלם׃ Jerusalem.}%
\verse{ושׁפכתי And I will pour על upon בית the house דויד of David, ועל and upon יושׁב the inhabitants ירושׁלם of Jerusalem, רוח the spirit חן of grace ותחנונים and of supplications: והביטו and they shall look אלי upon את אשׁר me whom דקרו they have pierced, וספדו and they shall mourn עליו for כמספד על for היחיד only והמר and shall be in bitterness עליו for כהמר him, as one that is in bitterness על for הבכור׃ firstborn.}%
\verse{ביום day ההוא In that יגדל shall there be a great המספד mourning בירושׁלם in Jerusalem, כמספד as the mourning הדד רמון of Hadadrimmon בבקעת in the valley מגדון׃ of Megiddon.}%
\verse{וספדה shall mourn, הארץ And the land משׁפחות every family משׁפחות every family לבד apart; משׁפחת the family בית of the house דויד of David לבד apart, ונשׁיהם and their wives לבד apart; משׁפחת the family בית of the house נתן of Nathan לבד apart, ונשׁיהם and their wives לבד׃ apart;}%
\verse{משׁפחת The family בית of the house לוי of Levi לבד apart, ונשׁיהם and their wives לבד apart; משׁפחת the family השׁמעי of Shimei לבד apart, ונשׁיהם and their wives לבד׃ apart;}%
\verse{כל All המשׁפחות the families הנשׁארות that remain, משׁפחת every family משׁפחת every family לבד apart, ונשׁיהם and their wives לבד׃ apart.}%
\end{biblechapter}%
\begin{biblechapter}% Zechariah 13
\verse{ביום day ההוא In that יהיה there shall be מקור a fountain נפתח opened לבית to the house דויד of David ולישׁבי and to the inhabitants ירושׁלם of Jerusalem לחטאת for sin ולנדה׃ and for uncleanness.}%
\verse{והיה And it shall come to pass ביום day, ההוא in that נאם saith יהוה the LORD צבאות of hosts, אכרית I will cut off את שׁמות the names העצבים of the idols מן out of הארץ the land, ולא and they shall no יזכרו be remembered: עוד more וגם and also את הנביאים I will cause the prophets ואת רוח spirit הטמאה and the unclean אעביר to pass מן out of הארץ׃ the land.}%
\verse{והיה And it shall come to pass, כי when ינבא prophesy, אישׁ any עוד shall yet ואמרו him shall say אליו unto אביו then his father ואמו and his mother ילדיו that begot לא him, Thou shalt not תחיה live; כי for שׁקר lies דברת thou speakest בשׁם in the name יהוה of the LORD: ודקרהו him shall thrust him through אביהו and his father ואמו and his mother ילדיו that begot בהנבאו׃ when he prophesieth.}%
\verse{והיה And it shall come to pass ביום day, ההוא in that יבשׁו shall be ashamed הנביאים the prophets אישׁ every one מחזינו of his vision, בהנבאתו when he hath prophesied; ולא neither ילבשׁו shall they wear אדרת garment שׂער a rough למען to כחשׁ׃ deceive:}%
\verse{ואמר But he shall say, לא no נביא prophet, אנכי I אישׁ a husbandman; עבד אדמהנכי I כי for אדם man הקנני taught me to keep cattle מנעורי׃ from my youth.}%
\verse{ואמר And shall say אליו unto מה him, What המכות wounds האלה these בין in ידיך thine hands? ואמר Then he shall answer, אשׁר with which הכיתי I was wounded בית the house מאהבי׃ of my friends.}%
\verseWithHeading{The Sword Shall Strike the Shepherd}{חרב O sword, עורי Awake, על against רעי ועל and against גבר the man עמיתי my fellow, נאם saith יהוה צבאות of hosts: הך smite את הרעה my shepherd, ותפוצין shall be scattered: הצאן and the sheep והשׁבתי and I will turn ידי mine hand על upon הצערים׃ the little ones.}%
\verse{והיה And it shall come to pass, בכל in all הארץ the land, נאם saith יהוה the LORD, פי parts שׁנים two בה יכרתו therein shall be cut off יגועו die; והשׁלשׁית but the third יותר׃ shall be left}%
\verse{והבאתי And I will bring את השׁלשׁית the third part באשׁ through the fire, וצרפתים and will refine כצרף is refined, את הכסף them as silver ובחנתים and will try כבחן is tried: את הזהב them as gold הוא they יקרא shall call בשׁמי on my name, ואני and I אענה will hear אתו אמרתי them: I will say, עמי my people: הוא It והוא and they יאמר shall say, יהוה The LORD אלהי׃ my God.}%
\end{biblechapter}%
\begin{biblechapter}% Zechariah 14
\verseWithHeading{Zechariah’s Apocalyptic Vision}{הנה Behold, יום the day בא cometh, ליהוה of the LORD וחלק shall be divided שׁללך and thy spoil בקרבך׃ in the midst}%
\verse{ואספתי For I will gather את כל all הגוים nations אל against ירושׁלם Jerusalem למלחמה to battle; ונלכדה shall be taken, העיר and the city ונשׁסו rifled, הבתים and the houses והנשׁים and the women תשׁגלנה ravished; ויצא shall go forth חצי and half העיר of the city בגולה into captivity, ויתר and the residue העם of the people לא shall not יכרת be cut off מן from העיר׃ the city.}%
\verse{ויצא go forth, יהוה ונלחם and fight against בגוים nations, ההם those כיום as when הלחמו he fought ביום in the day קרב׃ of battle.}%
\verse{ועמדו shall stand רגליו And his feet ביום day ההוא in that על upon הר the mount הזתים of Olives, אשׁר which על before פני before ירושׁלם Jerusalem מקדם on the east, ונבקע shall cleave הר and the mount הזיתים of Olives מחציו in the midst מזרחה thereof toward the east וימה and toward the west, גיא valley; גדולה great מאד a very ומשׁ shall remove חצי and half ההר of the mountain צפונה toward the north, וחציו and half נגבה׃ of it toward the south.}%
\verse{ונסתם And ye shall flee גיא the valley הרי of the mountains; כי for יגיע shall reach גי the valley הרים of the mountains אל unto אצל Azal: ונסתם yea, ye shall flee, כאשׁר like as נסתם ye fled מפני from before הרעשׁ the earthquake בימי in the days עזיה of Uzziah מלך king יהודה of Judah: ובא shall come, יהוה and the LORD אלהי my God כל all קדשׁים the saints עמך׃ with}%
\verse{והיה And it shall come to pass ביום day, ההוא in that לא shall not יהיה be אור the light יקרות clear, יקפאון׃ dark:}%
\verse{והיה But it shall be יום day אחד one הוא which יודע shall be known ליהוה to the LORD, לא not יום day, ולא nor לילה night: והיה but it shall come to pass, לעת time ערב at evening יהיה it shall be אור׃ light.}%
\verse{והיה And it shall be ביום day, ההוא in that יצאו shall go out מים waters חיים living מירושׁלם חצים half אל of them toward הים sea, הקדמוני the former וחצים and half אל of them toward הים sea: האחרון the hinder בקיץ in summer ובחרף and in winter יהיה׃ shall it be.}%
\verse{והיה shall be יהוה And the LORD למלך king על over כל all הארץ the earth: ביום day ההוא in that יהיה shall there be יהוה LORD, אחד one ושׁמו and his name אחד׃ one.}%
\verse{יסוב shall be turned כל All הארץ the land כערבה as a plain מגבע לרמון to Rimmon נגב south ירושׁלם of Jerusalem: וראמה and it shall be lifted up, וישׁבה and inhabited תחתיה in her place, למשׁער gate בנימן from Benjamin's עד unto מקום the place שׁער gate, הראשׁון of the first עד unto שׁער gate, הפנים the corner ומגדל and the tower חננאל of Hananeel עד unto יקבי winepresses. המלך׃ the king's}%
\verse{וישׁבו And shall dwell בה וחרם utter destruction; לא no יהיה in it, and there shall be עוד more וישׁבה inhabited. ירושׁלם but Jerusalem לבטח׃ shall be safely}%
\verse{וזאת And this תהיה shall be המגפה the plague אשׁר wherewith יגף will smite יהוה the LORD את כל all העמים the people אשׁר that צבאו have fought על against ירושׁלם Jerusalem; המק shall consume away בשׂרו Their flesh והוא while they עמד stand על upon their feet, רגליו upon their feet, ועיניו and their eyes תמקנה shall consume away בחריהן in their holes, ולשׁונו and their tongue תמק shall consume away בפיהם׃ in their mouth.}%
\verse{והיה And it shall come to pass ביום day, ההוא in that תהיה shall be מהומת tumult יהוה from the LORD רבה a great בהם והחזיקו among them; and they shall lay hold אישׁ every one יד on the hand רעהו of his neighbor, ועלתה shall rise up ידו and his hand על against יד the hand רעהו׃ of his neighbor.}%
\verse{וגם also יהודה And Judah תלחם shall fight בירושׁלם at Jerusalem; ואסף shall be gathered together, חיל and the wealth כל of all הגוים the heathen סביב round about זהב gold, וכסף and silver, ובגדים and apparel, לרב abundance. מאד׃ in great}%
\verse{וכן And so תהיה shall be מגפת the plague הסוס of the horse, הפרד of the mule, הגמל of the camel, והחמור and of the ass, וכל and of all הבהמה the beasts אשׁר that יהיה shall be במחנות tents, ההמה in these כמגפה plague. הזאת׃ as this}%
\verse{והיה And it shall come to pass, כל every one הנותר that is left מכל of all הגוים the nations הבאים which came על against ירושׁלם Jerusalem ועלו shall even go up מדי from שׁנה year בשׁנה to year להשׁתחות to worship למלך the King, יהוה the LORD צבאות of hosts, ולחג and to keep את חג the feast הסכות׃ of tabernacles.}%
\verse{והיה And it shall be, אשׁר whoso לא will not יעלה come up מאת משׁפחות the families הארץ of the earth אל unto ירושׁלם Jerusalem להשׁתחות to worship למלך the King, יהוה the LORD צבאות of hosts, ולא no עליהם even upon יהיה them shall be הגשׁם׃ rain.}%
\verse{ואם And if משׁפחת the family מצרים of Egypt לא go not up, תעלה go not up, ולא not, באה and come ולא that no עליהם תהיה there shall be המגפה the plague, אשׁר wherewith יגף will smite יהוה the LORD את הגוים the heathen אשׁר that לא come not up יעלו come not up לחג to keep את חג the feast הסכות׃ of tabernacles.}%
\verse{זאת This תהיה shall be חטאת the punishment מצרים of Egypt, וחטאת and the punishment כל of all הגוים nations אשׁר that לא come not up יעלו come not up לחג to keep את חג the feast הסכות׃ of tabernacles.}%
\verse{ביום day ההוא In that יהיה shall there be על upon מצלות the bells הסוס of the horses, קדשׁ HOLINESS ליהוה UNTO THE LORD; והיה shall be הסירות and the pots בבית house יהוה in the LORD's כמזרקים like the bowls לפני before המזבח׃ the altar.}%
\verse{והיה shall be כל Yea, every סיר pot בירושׁלם in Jerusalem וביהודה and in Judah קדשׁ holiness ליהוה unto the LORD צבאות of hosts: ובאו shall come כל and all הזבחים they that sacrifice ולקחו and take מהם ובשׁלו them, and seethe בהם ולא no יהיה there shall be כנעני the Canaanite עוד more בבית in the house יהוה of the LORD צבאות of hosts. ביום day ההוא׃ therein: and in that}%
\end{biblechapter}%
\flushcolsend
\input{leb/content/old-testament/Mal.tex}\flushcolsend
% a blank page on the back of the page - next page on a new right-sided page
\cleardoublepage


\addcontentsline{toc}{part}{The New Testament}%
\onecolumn
\begin{centering}
\includepdf[noautoscale=true, scale=1]{leb/content/pictures/title-page-new-testament.png}
\end{centering}
\twocolumn

% a blank page on the back of the page - next page on a new right-sided page
\clearpage
\newpage
\thispagestyle{empty}
\mbox{}


\biblebook{Matthew}
\begin{biblechapter}% Matthew 1
\verseWithHeading{The Genealogy of Jesus Christ}{The book of the genealogy of Jesus Christ, the son of David, the son of Abraham.}%
\verse{Abraham became the father of Isaac, and Isaac became the father of Jacob, and Jacob became the father of Judah and his brothers,}%
\verse{and Judah became the father of Perez and Zerah by Tamar, and Perez became the father of Hezron, and Hezron became the father of Aram,\lebnote{Although the Greek text reads “Aram,” many English versions substitute the Old Testament form of the name, “Ram” (cf. 1 Chr 2:9; Ruth 4:19), here and in the following verse}}%
\verse{and Aram became the father of Amminadab, and Amminadab became the father of Nahshon, and Nahshon became the father of Salmon,}%
\verse{and Salmon became the father of Boaz by Rahab, and Boaz became the father of Obed by Ruth, and Obed became the father of Jesse,}%
\verse{and Jesse became the father of David the king. And David became the father of Solomon by the wife\lebnote{The word “wife” is not in the Greek text, but is implied idiomatically} of Uriah,}%
\verse{and Solomon became the father of Rehoboam, and Rehoboam became the father of Abijah, and Abijah became the father of Asa,\lebnote{Greek “Asaph,” alternately spelled “Asa” in many English versions here and in the following verse (cf. 1 Chr 3:10)}}%
\verse{and Asa became the father of Jehoshaphat,\lebnote{Greek “Josaphat”; alternately spelled “Jehoshaphat” in many English versions} and Jehoshaphat became the father of Joram, and Joram became the father of Uzziah,}%
\verse{and Uzziah became the father of Jotham, and Jotham became the father of Ahaz, and Ahaz became the father of Hezekiah,}%
\verse{and Hezekiah became the father of Manasseh, and Manasseh became the father of Amon,\lebnote{The earliest and best Greek manuscripts read “Amos,” but many English versions use the Old Testament form of the name here, “Amon” (cf. 2 Kgs 21:18)} and Amon became the father of Josiah,}%
\verse{and Josiah became the father of Jechoniah and his brothers, at the time of the deportation to Babylon.}%
\verse{And after the deportation to Babylon, Jechoniah became the father of Shealtiel,\lebnote{Greek “Salathiel,” but many English versions use the Old Testament form of the name here, “Shealtiel” (cf. Ezra 3:2)} and Shealtiel became the father of Zerubbabel,}%
\verse{and Zerubbabel became the father of Abiud, and Abiud became the father of Eliakim, and Eliakim became the father of Azor,}%
\verse{and Azor became the father of Zadok, and Zadok became the father of Achim, and Achim became the father of Eliud,}%
\verse{and Eliud became the father of Eleazar, and Eleazar became the father of Matthan, and Matthan became the father of Jacob,}%
\verse{and Jacob became the father of Joseph, the husband of Mary by whom\lebnote{The Greek relative pronoun is feminine gender and thus refers only to Mary, not Joseph} was born Jesus, who is called Christ.}%
\verse{Therefore all the generations from Abraham to David are fourteen generations, and from David to the deportation to Babylon are fourteen generations, and from the deportation to Babylon to the Christ are fourteen generations.}%
\verseWithHeading{The Birth of Jesus Christ}{Now the birth of Jesus Christ occurred in this way. His mother Mary had been betrothed to Joseph, but before they came together, she was found to be pregnant\lebnote{“to have in the womb”} by the Holy Spirit.}%
\verse{So Joseph her husband, being righteous and not wanting to disgrace her, intended to divorce her secretly.}%
\verse{But as\lebnote{“\textit{as}” is supplied as a component of the temporal genitive absolute participle (“considering”)} he was considering these things, behold, an angel of the Lord appeared to him in a dream, saying, “Joseph, son of David, do not be afraid to take Mary as your wife, for what has been conceived in her is from the Holy Spirit.}%
\verse{And she will give birth to a son, and you will call his name ‘Jesus,’ because he will save his people from their sins.”}%
\verse{Now all this happened in order that what was spoken by the Lord through the prophet would be fulfilled, saying,}%
\verse{“Behold, the virgin will become pregnant\lebnote{“will have in the womb”} and will give birth to a son, and they will call his name Emmanuel,”\lebnote{from Isa 7:14} which is translated, “God with us.”\lebnote{An allusion to Isa 8:8, 10 in the Greek OT (LXX)}}%
\verse{And Joseph, when he\lebnote{“when” is supplied as a component of the participle (“woke up”) which is understood as temporal} woke up from sleep, did as the angel of the Lord commanded him, and he took his wife}%
\verse{and did not have sexual relations with\lebnote{“did not know”} her until she gave birth to a son. And he called his name Jesus.}%
\end{biblechapter}%
\begin{biblechapter}% Matthew 2
\verseWithHeading{Wise Men Visit Jesus}{Now after\lebnote{“\textit{after}” is supplied as a component of the temporal genitive absolute participle (“was born”)} Jesus was born in Bethlehem of Judea in the days of Herod the king, behold, wise men from the east came to Jerusalem,}%
\verse{saying, “Where is the one who has been born king of the Jews? For we have seen his star at its rising\lnCVA{} and have come to worship him.”}%
\verse{And when\lnCVB{} King Herod heard it,\lebnote{*supplied from English context} he was troubled, and all Jerusalem with him,}%
\verse{and after\lebnote{“\textit{after}” is supplied as a component of the participle (“calling together”) which is understood as temporal} calling together all the chief priests and scribes of the people, he inquired from them where the Christ was to be born.}%
\verse{So they said to him, “In Bethlehem of Judea, for thus it is written by the prophet,}%
\verse{‘And you, Bethlehem, land of Judah, are by no means least among the rulers of Judah, for from you will go out a ruler who will shepherd my people Israel.’”\lebnote{from Mic 5:2}}%
\verse{Then Herod secretly summoned the wise men and\lebnote{“\textit{and}” supplied because previous participle “summoned” translated as a finite verb} determined precisely from them the time when\lebnote{“\textit{when}” is supplied as a component of the temporal genitive absolute participle (“appeared”)} the star appeared.}%
\verse{And he sent them to Bethlehem and\lnCVC{} said, “Go, inquire carefully concerning the child, and when you have found him, report to me so that I also may come and\lebnote{“\textit{and}” supplied because previous participle “may come” translated as a finite verb} worship him.”}%
\verse{After\lebnote{“\textit{after}” is supplied as a component of the participle (“listened to”) which is understood as temporal} they listened to the king, they went out, and behold, the star which they had seen at its rising\lnCVA{} led them until it came and\lnCVD{} stood above the place where the child was.}%
\verse{Now when they\lnCVE{} saw the star, they rejoiced with very great joy.}%
\verse{And when they\lebnote{“\textit{when}” is supplied as a component of the participle (“came”) which is understood as temporal} came into the house, they saw the child with Mary his mother, and they fell down and\lebnote{“\textit{and}” supplied because previous participle “fell down” translated as a finite verb} worshiped him. And opening their treasure boxes, they offered him gifts of gold and frankincense and myrrh.}%
\verse{And being warned in a dream not to return to Herod, they went back to their own country by another route.}%
\verseWithHeading{Joseph, Mary, and Jesus Escape to Egypt}{Now after they had gone away, behold, an angel of the Lord appeared in a dream to Joseph, saying, “Get up, take the child and his mother and flee to Egypt, and stay there until I tell you. For Herod is about to seek the child to destroy him.”}%
\verse{So he got up and\lnCVF{} took the child and his mother during the night and went away to Egypt.}%
\verse{And he was there until the death of Herod, in order that what was said by the Lord through the prophet would be fulfilled, saying, “Out of Egypt I called my son.”}%
\verseWithHeading{Herod Has Innocent Children Murdered}{Then Herod, when he\lnCVE{} saw that he had been deceived by the wise men, became very angry, and he sent soldiers\lebnote{supplied from English context} and\lnCVC{} executed all the children in Bethlehem and in all the region around it from the age of two years old and under, according to the time which he had determined precisely from the wise men.}%
\verse{Then what was spoken by the prophet Jeremiah was fulfilled, saying,}%
\verse{“A voice was heard in Ramah, weeping and great mourning, Rachel weeping for her children, and she did not want to be comforted, because they exist no longer\lebnote{“they are not”}.”\lebnote{from Jer 31:15}}%
\verseWithHeading{Joseph, Mary, and Jesus Return to Nazareth}{Now after\lebnote{“\textit{after}” is supplied as a component of the temporal genitive absolute participle (“had died”)} Herod had died, behold, an angel of the Lord appeared in a dream to Joseph in Egypt,}%
\verse{saying, “Get up, take the child and his mother and go to the land of Israel, for those who were seeking the life of the child are dead.”}%
\verse{So he got up and\lnCVF{} took the child and his mother and entered\lebnote{“entered into”} the land of Israel.}%
\verse{But when he\lnCVB{} heard that Archelaus was reigning over Judea in place of his father Herod, he was afraid to go there, and being warned in a dream, he took refuge in the regions of Galilee.}%
\verse{And he came and\lnCVD{} lived in a town called Nazareth, in order that what was said by the prophets would be fulfilled:\lebnote{“that”; the conjunction could be understood (1) to introduce a direct quotation, serving a function similar to modern English quotation marks, and thus not translated; or (2) to introduce an indirect quotation, in which case it could be translated “that he would be called a Nazarene”} “He will be called a Nazarene.”}%
\end{biblechapter}%
\begin{biblechapter}% Matthew 3
\verseWithHeading{John the Baptist Begins His Ministry}{Now in those days John the Baptist came preaching in the Judean wilderness}%
\verse{and saying, “Repent, for the kingdom of heaven has come near!”}%
\verse{For this is the one who was spoken about by the prophet Isaiah, saying, “The voice of one crying out in the wilderness, ‘Prepare the way of the Lord, make his paths straight.’”\lebnote{from Isa 40:3}}%
\verse{Now John himself had his clothing made from camel’s hair and a belt made of leather around his waist, and his food was locusts and wild honey.}%
\verse{Then Jerusalem and all Judea and all the district around the Jordan were going out to him,}%
\verse{and they were being baptized by him in the Jordan River as they\lebnote{“\textit{as}” is supplied as a component of the participle (“confessed”) which is understood as temporal} confessed their sins.}%
\verse{But when he\lebnote{“\textit{when}” is supplied as a component of the participle (“saw”) which is understood as temporal} saw many of the Pharisees and Sadducees coming to his baptism, he said to them, “Offspring of vipers! Who warned you to flee from the coming wrath?}%
\verse{Therefore produce fruit worthy of repentance!}%
\verse{And do not think to say to yourselves, ‘We have Abraham as father.’ For I say to you that God is able to raise up children for Abraham from these stones!}%
\verse{Already now the ax is positioned at the root of the trees; therefore every tree not producing good fruit is cut down and thrown into the fire.}%
\verse{I baptize you with water for repentance, but the one who comes after me is more powerful than I am, whose sandals I am not worthy to carry. He will baptize you with the Holy Spirit and fire.}%
\verse{His winnowing shovel is in his hand, and he will clean out his threshing floor and will gather his wheat into the storehouse, but he will burn up the chaff with unquenchable fire.”}%
\verseWithHeading{The Baptism of Jesus}{Then Jesus came from Galilee to the Jordan to John in order to be baptized by him.}%
\verse{But John tried to prevent\lebnote{The imperfect verb is understood as conative (“tried to”)} him, saying, “I need\lebnote{“I have need”} to be baptized by you, and do you come to me?”}%
\verse{But Jesus answered and\lebnote{“\textit{and}” supplied because previous participle “answered” translated as a finite verb} said to him, \JesusWords{“Permit it now, for in this way it is right for us to fulfill all righteousness.”} Then he permitted him.}%
\verse{Now after he\lebnote{“\textit{after}” is supplied as a component of the participle (“was baptized”) which is understood as temporal} was baptized, Jesus immediately went up from the water, and behold, the heavens opened\lebnote{Some manuscripts have “opened to him”} and he saw the Spirit of God descending like a dove coming\lebnote{Some manuscripts have “and coming”} upon him.}%
\verse{And behold, there was\lebnote{The words “\textit{there was}” are not in the Greek text, but are supplied to make a complete sentence in English} a voice from heaven saying, “This is my beloved Son, with whom I am well pleased.”}%
\end{biblechapter}%
\begin{biblechapter}% Matthew 4
\verseWithHeading{The Temptation of Jesus}{Then Jesus was led up into the wilderness by the Spirit to be tempted by the devil,}%
\verse{and after he\lebnote{“\textit{after}” is supplied as a component of the participle (“had fasted”) which is understood as temporal} had fasted forty days and forty nights, then he was hungry.}%
\verse{And the tempter approached and\lebnote{“\textit{and}” supplied because previous participle “approached” translated as a finite verb} said to him, “If you are the Son of God, order that these stones become bread.”}%
\verse{But he answered and\lebnote{“\textit{and}” supplied because previous participle “answered” translated as a finite verb} said, \JesusWords{“It is written, ‘Man will not live on bread alone, but on every word that comes out of the mouth of God.”}\lebnote{from Deut 8:3}}%
\verse{Then the devil took him to the holy city\lebnote{That is, Jerusalem} and placed him on the highest point of the temple}%
\verse{and said to him, “If you are the Son of God, throw yourself down! For it is written, ‘He will command his angels concerning you,’\lebnote{from Ps 91:11} and ‘On their hands they will lift you up, lest you strike your foot against a stone.’”\lebnote{from Ps 91:12}}%
\verse{Jesus said to him, \JesusWords{“On the other hand it is written, ‘You are not to put the Lord your God to the test.’”\lebnote{from Deut 6:16}}}%
\verse{Again the devil took him to a very high mountain and showed him all the kingdoms of the world and their glory,}%
\verse{and he said to him, “I will give to you all these things, if you will fall down and\lebnote{“\textit{and}” supplied because previous participle “will fall down” translated as a finite verb} worship me.”}%
\verse{Then Jesus said to him, \JesusWords{“Go away, Satan, for it is written, ‘You shall worship the Lord your God and serve only him.’”\lebnote{from Deut 6:13}}}%
\verse{Then the devil left him, and behold, angels came and began ministering to him.}%
\verseWithHeading{Public Ministry in Galilee}{Now when he\lebnote{“\textit{when}” is supplied as a component of the participle (“heard”) which is understood as temporal} heard that John had been arrested,\lebnote{“had been handed over”} he withdrew into Galilee.}%
\verse{And leaving Nazareth, he went and\lebnote{“\textit{and}” supplied because previous participle “went” translated as a finite verb} lived in Capernaum by the sea, in the region of Zebulun and Naphtali,}%
\verse{in order that what was spoken by the prophet Isaiah would be fulfilled, who said,}%
\verse{“Land of Zebulun and land of Naphtali, toward the sea,\lebnote{“the way of the sea”} on the other side of the Jordan, Galilee of the Gentiles\lebnote{Or “nations”; the same Greek word can be translated “nations” or “Gentiles” depending on the context} —}%
\verse{the people who sit in darkness have seen a great light, and the ones who sit in the land and shadow of death, a light has dawned on them.”\lebnote{from Isa 9:1}}%
\verse{From that time on, Jesus began to preach and to say, \JesusWords{“Repent, because the kingdom of heaven is near.”}}%
\verseWithHeading{Jesus Calls His First Disciples}{Now as he\lebnote{“\textit{as}” is supplied as a component of the participle (“was walking”) which is understood as temporal} was walking beside the Sea of Galilee, he saw two brothers, Simon called Peter and his brother Andrew, throwing a casting net into the sea (for they were fishermen).}%
\verse{And he said to them, \JesusWords{“Follow me\lebnote{“come behind me”} and I will make you fishers of people.”}}%
\verse{And immediately they left their nets and\lnCVG{} followed him.}%
\verse{And going on from there, he saw two other brothers, James the son of Zebedee and his brother John, in the boat with their father Zebedee, mending their nets, and he called them.}%
\verse{And immediately they left the boat and their father and\lnCVG{} followed him.}%
\verseWithHeading{Teaching, Preaching, and Healing throughout Galilee}{And he went around through all of Galilee, teaching in their synagogues and proclaiming the good news of the kingdom and healing every disease and every sickness among the people.}%
\verse{And a report about him went out throughout\lebnote{“in the whole of”} Syria, and they brought to him all those who were sick\lebnote{“having badly”} with various diseases and afflicted by torments, demon-possessed\lebnote{Some manuscripts have “and demon-possessed”} and epileptics and paralytics, and he healed them.}%
\verse{And large crowds followed him from Galilee, Decapolis, Jerusalem, Judea, and from the other side of the Jordan.}%
\end{biblechapter}%
\begin{biblechapter}% Matthew 5
\verseWithHeading{The Sermon on the Mount: The Beatitudes}{Now when he\lebnote{“\textit{when}” is supplied as a component of the participle (“saw”) which is understood as temporal} saw the crowds, he went up the mountain and after he\lebnote{“\textit{after}” is supplied as a component of the participle (“sat down”) which is understood as temporal} sat down, his disciples approached him.}%
\verse{And opening his mouth he began to teach them, saying,}%
\verse{\JesusWords{“Blessed are the poor in spirit, because theirs is the kingdom of heaven.}}%
\verse{\JesusWords{Blessed are the ones who mourn, because they will be comforted.}}%
\verse{\JesusWords{Blessed are the meek, because they will inherit the earth.}}%
\verse{\JesusWords{Blessed are the ones who hunger and thirst for righteousness, because they will be satisfied.}}%
\verse{\JesusWords{Blessed are the merciful, because they will be shown mercy.}}%
\verse{\JesusWords{Blessed are the pure in heart, because they will see God.}}%
\verse{\JesusWords{Blessed are the peacemakers, because they will be called sons of God.}}%
\verse{\JesusWords{Blessed are those who are persecuted because of righteousness, because theirs is the kingdom of heaven.}}%
\verse{\JesusWords{Blessed are you when they insult you and persecute you and say all kinds of evil things against you, lying on account of me.}}%
\verse{\JesusWords{Rejoice and be glad, because your reward is great in heaven, for in the same way they persecuted the prophets before you.}}%
\verseWithHeading{The Sermon on the Mount: Salt and Light}{\JesusWords{“You are the salt of the earth. But if salt becomes tasteless, by what will it be made salty? It is good for nothing any longer except to be thrown outside and\lebnote{“\textit{and}” supplied because previous participle “thrown” translated as a finite verb} trampled under foot by people.}}%
\verse{\JesusWords{You are the light of the world. A city located on top of a hill cannot be hidden,}}%
\verse{\JesusWords{nor do they light a lamp and place it under a basket, but on a lampstand, and it shines on all those in the house.}}%
\verse{\JesusWords{In the same way let your light shine before people, so that they can see your good works and glorify your Father who is in heaven.}}%
\verseWithHeading{The Sermon on the Mount: The Law and the Prophets Fulfilled}{\JesusWords{“Do not think that I have come to destroy the law or the prophets. I have not come to destroy them but to fulfill them.}}%
\verse{\JesusWords{For truly I say to you, until heaven and earth pass away, not one tiny letter or one stroke of a letter will pass away from the law until all takes place.}}%
\verse{\JesusWords{Therefore whoever abolishes one of the least of these commandments and teaches people to do so will be called least in the kingdom of heaven, but whoever keeps them and teaches them, this person will be called great in the kingdom of heaven.}}%
\verse{\JesusWords{For I say to you that unless your righteousness greatly surpasses that of the scribes and Pharisees, you will never enter into the kingdom of heaven.}}%
\verseWithHeading{The Sermon on the Mount: Anger Toward Others}{\JesusWords{“You have heard that it was said to the people of old,\lnCVH{} ‘Do not commit murder,’\lebnote{from Exod 20:13; Deut 5:17} and ‘whoever commits murder will be subject to judgment.’}}%
\verse{\JesusWords{But I say to you that everyone who is angry at his brother will be subject to judgment, and whoever says to his brother, ‘Stupid fool!’\lebnote{Greek “Raca,” a term of verbal abuse involving lack of intelligence} will be subject to the council, and whoever says, ‘Obstinate fool!’\lebnote{Perhaps with the idea of obstinate, godless foolishness; some take the word to be a Greek transliteration of the Hebrew word for “rebel” (Deut 21:18, 20)} will be subject to fiery hell.}}%
\verse{\JesusWords{Therefore if you present your gift at the altar and there remember that your brother has something against you,}}%
\verse{\JesusWords{leave your gift there before the altar and first go be reconciled to your brother, and then come and\lebnote{“\textit{and}” supplied because previous participle “come” translated as a finite verb} present your gift.}}%
\verse{\JesusWords{Settle the case quickly with your accuser\lebnote{“be making friends quickly with your accuser”} while you are with him on the way, lest your accuser hand you over to the judge, and the judge to the officer, and you be thrown into prison.}}%
\verse{\JesusWords{Truly I say to you, you will never come out of there until you have paid back the last penny!}}%
\verseWithHeading{The Sermon on the Mount: Adultery and Lust}{\JesusWords{“You have heard that it was said, ‘Do not commit adultery.’}\lebnote{from Exod 20:14; Deut 5:17}}%
\verse{\JesusWords{But I say to you that everyone who looks at a woman to lust for her has already committed adultery with her in his heart.}}%
\verse{\JesusWords{And if your right eye causes you to sin, tear it out and throw it from you! For it is better for you that one of your members be destroyed than your whole body be thrown into hell.}}%
\verse{\JesusWords{And if your right hand causes you to sin, cut it off and throw it from you! For it is better for you that one of your limbs be destroyed than your whole body go into hell.}}%
\verseWithHeading{The Sermon on the Mount: Divorce}{\JesusWords{“And it was said, ‘Whoever divorces his wife must give her a certificate of divorce.’}\lebnote{from Deut 24:1}}%
\verse{\JesusWords{But I say to you that everyone who divorces his wife, except for a matter of sexual immorality, causes her to commit adultery, and whoever marries a divorced woman commits adultery.}}%
\verseWithHeading{The Sermon on the Mount: Taking Oaths}{\JesusWords{“Again you have heard that it was said to the people of old,\lnCVH{} ‘Do not swear falsely,\lebnote{Or “do not break your oath”} but fulfill your oaths to the Lord.’}\lebnote{from Lev 19:12}}%
\verse{\JesusWords{But I say to you, do not swear at all, either by heaven, because it is the throne of God,}}%
\verse{\JesusWords{or by the earth, because it is the footstool of his feet, or by Jerusalem, because it is the city of the great king.}}%
\verse{\JesusWords{And do not swear by your head, because you are not able to make one hair white or black.}}%
\verse{\JesusWords{But let your statement be ‘Yes, yes; no, no,’ and anything beyond these is from the evil one.}\lebnote{Or “is of evil”}}%
\verseWithHeading{The Sermon on the Mount: Retaliation}{\JesusWords{“You have heard that it was said, ‘An eye for an eye and a tooth for a tooth.’}\lebnote{from Exod 21:24; Lev 24:20}}%
\verse{\JesusWords{But I say to you, do not resist the evildoer, but whoever strikes you on the right cheek,\lebnote{Some manuscripts have “your right cheek”} turn the other to him also.}}%
\verse{\JesusWords{And the one who wants to go to court with you and take your tunic, let him have\lebnote{“leave to him”} your outer garment also.}}%
\verse{\JesusWords{And whoever forces you to go one mile,\lebnote{A Roman mile was originally a thousand paces, but was later fixed at eight stades (1,478.5 meters)} go with him two.}}%
\verse{\JesusWords{Give to the one who asks you, and do not turn away from the one who wants to borrow from you.}}%
\verseWithHeading{The Sermon on the Mount: Love for Enemies}{\JesusWords{“You have heard that it was said, ‘Love your neighbor’\lebnote{from Lev 19:18} and ‘Hate your enemy.’}\lebnote{An allusion to Deut 23:3–6}}%
\verse{\JesusWords{But I say to you, love your enemies and pray for those who persecute you,}}%
\verse{\JesusWords{in order that you may be sons of your Father who is in heaven, because he causes his sun to rise on the evil and the good, and he sends rain on the just and the unjust.}}%
\verse{\JesusWords{For if you love those who love you, what reward do you have? Do not the tax collectors also do the same?}}%
\verse{\JesusWords{And if you greet only your brothers, what are you doing that is remarkable? Do not the Gentiles also do the same?}}%
\verse{\JesusWords{Therefore you be perfect as your heavenly Father is perfect.}}%
\end{biblechapter}%
\begin{biblechapter}% Matthew 6
\verseWithHeading{The Sermon on the Mount: Charitable Giving}{\JesusWords{“And take care not to practice your righteousness before people to be seen by them; otherwise\lebnote{“but if not”} you have no reward from your Father who is in heaven.}}%
\verse{\JesusWords{Therefore whenever you practice charitable giving, do not sound a trumpet in front of you, as the hypocrites do in the synagogues and in the streets, in order that they may be praised by people. Truly I say to you, they have received their reward in full!}}%
\verse{\JesusWords{But you, when you\lebnote{“\textit{when}” is supplied as a component of the participle (“practice”) which is understood as temporal} practice charitable giving, do not let your left hand know what your right hand is doing,}}%
\verse{\JesusWords{in order that your charitable giving may be in secret, and your Father who sees in secret will reward you.}}%
\verseWithHeading{The Sermon on the Mount: How to Pray}{\JesusWords{And whenever you pray, do not be like the hypocrites, because they love to stand and\lebnote{“\textit{and}” supplied because previous participle “stand” translated as a finite verb} pray in the synagogues and on the corners of the streets, in order that they may be seen by people. Truly I say to you, they have received their reward in full!}}%
\verse{\JesusWords{But whenever you pray, enter into your inner room and shut your door and\lebnote{“\textit{and}” supplied because previous participle “shut” translated as a finite verb} pray to your Father who is in secret, and your Father who sees in secret will reward you.}}%
\verse{\JesusWords{“But when you\lebnote{“\textit{when}” is supplied as a component of the participle (“pray”) which is understood as temporal} pray, do not babble repetitiously like the pagans, for they think that because of their many words they will be heard.}}%
\verse{\JesusWords{Therefore do not be like them, for your Father knows what you need\lebnote{“of what you have need”} before you ask him.}}%
\verse{\JesusWords{Therefore you pray in this way: “Our Father who is in heaven, may your name be treated as holy.}}%
\verse{\JesusWords{May your kingdom come, may your will be done on earth as it is in heaven.}}%
\verse{\JesusWords{Give us today our daily bread,}}%
\verse{\JesusWords{and forgive us our debts, as we also have forgiven our debtors.}}%
\verse{\JesusWords{And do not bring us into temptation, but deliver us from the evil one.}\lebnote{Or “evil”; most later Greek manuscripts add the phrase “for yours is the kingdom and the power and the glory forever, amen”}}%
\verse{\JesusWords{For if you forgive people their sins, your heavenly Father will also forgive you.}}%
\verse{\JesusWords{But if you do not forgive people, neither will your Father forgive your sins.}}%
\verseWithHeading{The Sermon on the Mount: How to Fast}{\JesusWords{“Whenever you fast, do not be sullen like the hypocrites, for they make their faces unrecognizable in order that they may be seen fasting by people. Truly I say to you, they have received their reward in full!}}%
\verse{\JesusWords{But when\lebnote{“\textit{when}” is supplied as a component of the participle (“fasting”) which is understood as temporal} you are fasting, put olive oil on your head\lebnote{“anoint your head”} and wash your face}}%
\verse{\JesusWords{so that you will not be seen by people as fasting, but to your Father who is in secret, and your Father who sees in secret will reward you.}}%
\verseWithHeading{The Sermon on the Mount: Treasure in Heaven}{\JesusWords{“Do not store up for yourselves treasures on earth, where moth and consuming insect\lnCVI{} destroy and where thieves break in and steal,}}%
\verse{\JesusWords{but store up for yourselves treasures in heaven, where neither moth nor consuming insect\lnCVI{} destroy and where thieves do not break in or steal.}}%
\verse{\JesusWords{For where your treasure is, there your heart will be also.}}%
\verse{\JesusWords{“The eye is the lamp of the body. Therefore if your eye is sincere, your whole body will be full of light.}}%
\verse{\JesusWords{But if your eye is evil, your whole body will be dark. Therefore if the light in you is darkness, how great is the darkness!}}%
\verse{\JesusWords{“No one is able to serve two masters. For either he will hate the one and love the other, or he will be devoted to one and despise the other. You are not able to serve God and money.}\lebnote{Traditionally transliterated from the Greek as “mammon”}}%
\verseWithHeading{The Sermon on the Mount: Anxiety}{\JesusWords{“For this reason I say to you, do not be anxious for your life, what you will eat,\lebnote{Some manuscripts add “or what you will drink”; other later manuscripts add “and what you will drink”} and not for your body, what you will wear. Is your life not more than food and your body more than clothing?}}%
\verse{\JesusWords{Consider the birds of the sky, that they do not sow or reap or gather produce into barns, and your heavenly Father feeds them. Are you not worth more than they are?}}%
\verse{\JesusWords{And who among you, by\lebnote{“\textit{by}” is supplied as a component of the participle (“being anxious”) which is understood as means} being anxious, is able to add one hour\lebnote{Or “cubit”} to his life span?}}%
\verse{\JesusWords{And why are you anxious about clothing? Observe the lilies of the field, how they grow: they do not toil or spin,}}%
\verse{\JesusWords{but I say to you that not even Solomon in all his glory was dressed like one of these.}}%
\verse{\JesusWords{But if God dresses the grass of the field in this way, although it\lebnote{“\textit{although}” is supplied as a component of the participle (“is”) which is understood as concessive} is here today and tomorrow is thrown into the oven, will he not do so much more for you, you of little faith?}}%
\verse{\JesusWords{Therefore do not be anxious, saying, ‘What will we eat?’ or ‘What will we drink?’ or ‘What will we wear?,’}}%
\verse{\JesusWords{for the pagans seek after all these things. For your heavenly Father knows that you need all these things.}}%
\verse{\JesusWords{But seek first his kingdom and righteousness,\lebnote{Some manuscripts have “the kingdom of God and his righteousness”} and all these things will be added to you.}}%
\verse{\JesusWords{Therefore do not be anxious for tomorrow, because tomorrow will be anxious for itself. Each day has enough trouble of its own.}\lebnote{“sufficient for the day its trouble”}}%
\end{biblechapter}%
\begin{biblechapter}% Matthew 7
\verseWithHeading{The Sermon on the Mount: On Judging Others}{\JesusWords{“Do not judge, so that you will not be judged.}}%
\verse{\JesusWords{For by what judgment you judge, you will be judged, and by what measure you measure out, it will be measured out to you.}}%
\verse{\JesusWords{And why do you see the speck that is in your brother’s eye, but do not notice the beam of wood in your own eye?}}%
\verse{\JesusWords{Or how will you say to your brother, ‘Allow me to remove the speck from your eye,’ and behold, the beam of wood is in your own eye?}}%
\verse{\JesusWords{Hypocrite! First remove the beam of wood from your own eye and then you will see clearly to remove the speck from your brother’s eye!}}%
\verse{\JesusWords{“Do not give what is holy to dogs, or throw your pearls in front of pigs, lest they trample them with their feet, and turn around and\lebnote{“\textit{and}” supplied because previous participle “turn around” translated as a finite verb} tear you to pieces.}}%
\verseWithHeading{The Sermon on the Mount: Ask, Seek, Knock}{\JesusWords{“Ask and it will be given to you; seek and you will find; knock and it will be opened for you.}}%
\verse{\JesusWords{For everyone who asks receives, and the one who seeks finds, and to the one who knocks it will be opened.}}%
\verse{\JesusWords{Or what man is there among you, if his son will ask him for bread, will give him a stone?}}%
\verse{\JesusWords{Or also if he will ask for a fish, will give him a snake?}}%
\verse{\JesusWords{Therefore if you, although you\lebnote{“\textit{although}” is supplied as a component of the participle (“are”) which is understood as concessive} are evil, know how to give good gifts to your children, how much more will your Father who is in heaven give good things to those who ask him?}}%
\verse{\JesusWords{Therefore in all things, whatever you want that people should do to you, thus also you do to them. For this is the law and the prophets.}}%
\verseWithHeading{The Sermon on the Mount: The Narrow Gate}{\JesusWords{“Enter through the narrow gate, because broad is the gate and spacious is the road that leads to destruction, and there are many who enter through it,}}%
\verse{\JesusWords{because narrow\lebnote{Some manuscripts have “how narrow”} is the gate and constricted is the road that leads to life, and there are few who find it!}}%
\verseWithHeading{The Sermon on the Mount: Recognizing False Prophets}{\JesusWords{“Beware of false prophets who come to you in sheep’s clothing, but inside are ravenous wolves.}}%
\verse{\JesusWords{You will recognize them by their fruits: they do not gather grapes from thorn bushes or figs from thistles, do they?}\lebnote{The negative construction in Greek anticipates a negative answer here, indicated by “\textit{do they}”}}%
\verse{\JesusWords{In the same way, every good tree produces good fruit, but a bad tree produces bad fruit.}}%
\verse{\JesusWords{A good tree is not able to produce bad fruit, nor a bad tree to produce good fruit.}}%
\verse{\JesusWords{Every tree that does not produce good fruit is cut down and thrown into the fire.}}%
\verse{\JesusWords{As a result, you will recognize them by their fruits.}}%
\verseWithHeading{The Sermon on the Mount: False Followers}{\JesusWords{“Not everyone who says to me, ‘Lord, Lord,’ will enter into the kingdom of heaven, but the one who does the will of my Father who is in heaven.}}%
\verse{\JesusWords{On that day many will say to me, ‘Lord, Lord, did we not prophesy in your name, and expel demons in your name, and perform many miracles in your name?’}}%
\verse{\JesusWords{And then I will say to them plainly,\lebnote{“I will declare to them”} ‘I never knew you. Depart from me, you who practice lawlessness!’}}%
\verseWithHeading{The Sermon on the Mount: Two Houses and Two Foundations}{\JesusWords{“Therefore everyone who hears these words of mine and does them will be like a wise man who built his house on the rock.}}%
\verse{\JesusWords{And the rain came down and the rivers came and the winds blew and beat against that house, and it did not collapse, because its foundation was laid on the rock.}}%
\verse{\JesusWords{And everyone who hears these words of mine and does not do them will be like a foolish man who built his house on the sand.}}%
\verse{\JesusWords{And the rain came down and the rivers came and the winds blew and beat against that house, and it collapsed, and its fall was great.”}}%
\verseWithHeading{The Sermon on the Mount: Response}{And it happened when Jesus finished these words the crowds were amazed at his teaching,}%
\verse{because he was teaching them like one who had authority, and not like their scribes.}%
\end{biblechapter}%
\begin{biblechapter}% Matthew 8
\verseWithHeading{A Leper Cleansed}{And when\lebnote{“\textit{when}” is supplied as a component of the temporal genitive absolute participle (“came down”)} he came down from the mountain, large crowds followed him.}%
\verse{And behold, a leper approached and\lnCVJ{} worshiped him, saying, “Lord, if you are willing, you are able to make me clean.”}%
\verse{And extending his hand he touched him, saying, \JesusWords{“I am willing, be clean.”} And immediately his leprosy was cleansed.}%
\verse{And Jesus said to him, \JesusWords{“See that you tell no one, but go, show yourself to the priest and offer the gift that Moses commanded, for a testimony to them.”}}%
\verseWithHeading{A Centurion’s Slave Healed}{Now when\lebnote{“\textit{when}” is supplied as a component of the temporal genitive absolute participle (“entered”)} he entered Capernaum, a centurion approached him, appealing to him}%
\verse{and saying, “Lord, my slave\lebnote{Traditionally “servant”; the Greek term here is one often used of a slave who was regarded with some degree of affection, possibly a personal servant (the parallel passage in Luke 7:7 uses the more common term for slave)} is lying paralyzed in my\lebnote{“the”: the Greek article is used here as a possessive pronoun} house, terribly tormented!”}%
\verse{And he said to him, \JesusWords{“I will come and\lebnote{“\textit{and}” supplied because previous participle “come” translated as a finite verb} heal him.”}}%
\verse{And the centurion answered and\lebnote{“\textit{and}” supplied because previous participle “answered” translated as a finite verb} said, “Lord, I am not worthy that you should come in under my roof. But only say the word and my slave will be healed.}%
\verse{For I also am a man under authority who has soldiers under me, and I say to this one, ‘Go!’ and he goes, and to another one, ‘Come!’ and he comes, and to my slave, ‘Do this!’ and he does it\lebnote{supplied from English context (see the parallel in Luke 7:8).}.”}%
\verse{Now when\lebnote{“\textit{when}” is supplied as a component of the participle (“heard”) which is understood as temporal} Jesus heard this,\lnCVK{} he was astonished, and said to those who were following him,\lnCVK{} \JesusWords{“Truly I say to you, I have found such great faith with no one in Israel.}}%
\verse{\JesusWords{But I say to you that many will come from east and west and be seated at the banquet\lebnote{“recline at table”} with Abraham and Isaac and Jacob in the kingdom of heaven.}}%
\verse{\JesusWords{But the sons of the kingdom will be thrown out into the outer darkness. In that place there will be weeping and gnashing of teeth!”}}%
\verse{And Jesus said to the centurion, \JesusWords{“Go, as you have believed it will be done for you.”} And the slave\lebnote{Many later manuscripts have “his slave”} was healed at that hour.}%
\verseWithHeading{Many at Capernaum Are Healed}{And when\lebnote{“\textit{when}” is supplied as a component of the participle (“came”) which is understood as temporal} Jesus came into Peter’s house, he saw his mother-in-law lying down and suffering with a fever.}%
\verse{And he touched her hand and the fever left her, and she got up and began to serve him.}%
\verse{Now when it\lebnote{“\textit{when}” is supplied as a component of the temporal genitive absolute participle (“was”)} was evening, they brought to him many who were demon-possessed, and he expelled the spirits with a word. And he healed all those who were sick,\lebnote{“those who were having badly”}}%
\verse{in order that what was spoken through the prophet Isaiah would be fulfilled, who said, “He himself took away our sicknesses, and carried away our diseases.”\lebnote{from Isa 53:4}}%
\verseWithHeading{Would-be Followers}{Now when\lnCVL{} Jesus saw many crowds\lebnote{Some manuscripts have “a crowd”} around him, he gave orders to depart to the other side.\lnCVM{}}%
\verse{And a scribe approached and\lnCVJ{} said to him, “Teacher, I will follow you wherever you go!”}%
\verse{And Jesus said to him, \JesusWords{“Foxes have dens and birds of the sky have nests, but the Son of Man has no place to lay his head.”}}%
\verse{And another of the disciples\lebnote{Some manuscripts have “of his disciples”} said to him, “Lord, allow me first to go and bury my father.”}%
\verse{But Jesus said to him, \JesusWords{“Follow me, and leave the dead to bury their own dead!”}}%
\verseWithHeading{Calming of a Storm}{And as\lebnote{“\textit{as}” is supplied as a component of the participle (“got”) which is understood as temporal} he got into the boat, his disciples followed him.}%
\verse{And behold, a great storm arose on the sea, so that the boat was being inundated by the waves, but he himself was asleep.}%
\verse{And they came and\lebnote{“\textit{and}” supplied because previous participle “came” translated as a finite verb} woke him, saying, “Lord, save us!\lnCVK{} We are perishing!”}%
\verse{And he said to them, \JesusWords{“Why are you fearful, you of little faith?”} Then he got up and\lebnote{“\textit{and}” supplied because previous participle “got up” translated as a finite verb} rebuked the winds and the sea and there was a great calm.}%
\verse{And the men were astonished, saying, “What sort of man is this, that even the winds and the sea obey him?”}%
\verseWithHeading{Demon-possessed Gadarenes Healed}{And when\lebnote{“\textit{when}” is supplied as a component of the temporal genitive absolute participle (“came”)} he came to the other side,\lnCVM{} to the region of the Gadarenes,\lebnote{Many manuscripts read “Gergesenes”; others read “Gerasenes” (see Luke 8:26)} two demon-possessed men coming from among the tombs met him, very violent, so that no one was able to pass by along that road.}%
\verse{And behold, they cried out, saying, “What do you have to do with us,\lebnote{“what to us and to you”} Son of God? Have you come here to torment us before the time?”\lebnote{That is, before the appointed time of judgment}}%
\verse{Now a long way from them a large herd of pigs was feeding.}%
\verse{So the demons implored him, saying, “If you are going to expel us, send us into the herd of pigs.”}%
\verse{And he said to them, \JesusWords{“Go!”} So they departed and\lebnote{“\textit{and}” supplied because previous participle “departed” translated as a finite verb} went into the pigs, and behold, the whole herd rushed headlong down the steep slope into the sea and drowned in the water.}%
\verse{Now the herdsmen fled and went into the town and\lebnote{“\textit{and}” supplied because previous participle “went” translated as a finite verb} reported everything, including the things concerning the demon-possessed men.}%
\verse{And behold, the whole town came out to meet Jesus, and when they\lnCVL{} saw him, they implored him\lebnote{supplied from English context} that he would depart from their region.}%
\end{biblechapter}%
\begin{biblechapter}% Matthew 9
\verseWithHeading{A Paralytic Healed}{And getting into a boat, he crossed over and came to his own town.\lebnote{A reference to Capernaum}}%
\verse{And behold, they brought to him a paralytic lying on a stretcher, and when\lnCVN{} Jesus saw their faith, he said to the paralytic, \JesusWords{“Have courage, child, your sins are forgiven.”}}%
\verse{And behold, some of the scribes said to themselves, “This man is blaspheming!”}%
\verse{And knowing\lebnote{Some manuscripts have “perceiving”} their thoughts, Jesus said, \JesusWords{“Why do you think evil in your hearts?}}%
\verse{\JesusWords{For which is easier to say, ‘Your sins are forgiven,’ or to say, ‘Get up and walk’?}}%
\verse{\JesusWords{But in order that you may know that the Son of Man has authority on earth to forgive sins,”} then he said to the paralytic, \JesusWords{“Get up, pick up your stretcher and go to your home.”}}%
\verse{And he got up and\lnCVO{} went to his home.}%
\verse{But when\lnCVN{} the crowds saw this,\lnCVP{} they were afraid and glorified God who had given such authority to men.}%
\verseWithHeading{Matthew Called to Follow Jesus}{And as\lebnote{“\textit{as}” is supplied as a component of the participle (“saw”) which is understood as temporal} Jesus was going away from there, he saw a man called Matthew sitting at the tax booth and said to him, \JesusWords{“Follow me!”} And he stood up and\lebnote{“\textit{and}” supplied because previous participle “stood up” translated as a finite verb} followed him.}%
\verse{And it happened as\lebnote{“\textit{as}” is supplied as a component of the temporal genitive absolute participle (“was dining”)} he was dining\lebnote{“was reclining at table”} in the house, behold, many tax collectors and sinners were coming and\lebnote{“\textit{and}” supplied because previous participle “were coming” translated as a finite verb} dining\lebnote{“reclining at table”} with Jesus and his disciples.}%
\verse{And when they\lnCVN{} saw it,\lnCVP{} the Pharisees began to say to his disciples, “Why does your teacher eat with tax collectors and sinners?”}%
\verse{And when he\lebnote{“\textit{when}” is supplied as a component of the participle (“heard”) which is understood as temporal} heard it,\lnCVP{} he said, \JesusWords{“Those who are healthy do not have need of a physician, but those who are sick.}\lebnote{“having badly”}}%
\verse{\JesusWords{But go and\lebnote{“\textit{and}” supplied because previous participle “go” translated as a finite verb} learn what it means, “I want mercy and not sacrifice.”\lebnote{from Hos 6:6 (see also Matt 12:7)} For I did not come to call the righteous, but sinners.”}}%
\verseWithHeading{On Fasting}{Then the disciples of John\lebnote{That is, John the Baptist} approached him, saying, “Why do we and the Pharisees fast often, but your disciples do not fast?”}%
\verse{And Jesus said to them, \JesusWords{“The bridegroom’s attendants\lebnote{“the sons of the bridal chamber”} are not able to mourn as long as the bridegroom is with them. But days are coming when the bridegroom is taken away from them, and then they will fast.}}%
\verse{\JesusWords{But no one puts a patch of unshrunken cloth on an old garment, for its patch pulls away from the garment, and the tear becomes worse.}}%
\verse{\JesusWords{Nor do they put new wine into old wineskins. Otherwise\lebnote{“but if not”} the wineskins burst and the wine is spilled, and the wineskins are destroyed. But they put new wine into new wineskins and both are preserved.”}}%
\verseWithHeading{A Woman Healed and a Daughter Raised}{As\lebnote{“\textit{as}” is supplied as a component of the temporal genitive absolute participle (“was saying”)} he was saying these things to them, behold, one of the rulers came and\lebnote{“\textit{and}” supplied because previous participle “came” translated as a finite verb} knelt down before him, saying, “My daughter has just now died, but come, place your hand on her and she will live!”}%
\verse{And Jesus and his disciples got up and\lnCVO{} followed him.}%
\verse{And behold, a woman who had been suffering with a hemorrhage twelve years approached from behind and\lebnote{“\textit{and}” supplied because previous participle “approached” translated as a finite verb} touched the edge\lebnote{Or “tassel”} of his cloak,}%
\verse{for she said to herself, “If only I touch his cloak I will be healed.”}%
\verse{But Jesus, turning around and seeing her, said, \JesusWords{“Have courage, daughter! Your faith has healed you.”} And the woman was healed from that hour.}%
\verse{And when\lnCVQ{} Jesus came into the ruler’s house and saw the flute players and the disorderly crowd,}%
\verse{he said, \JesusWords{“Go away, because the girl is not dead, but is sleeping.”} And they ridiculed him.}%
\verse{But when the crowd had been sent out, he entered and\lebnote{“\textit{and}” supplied because previous participle “entered” translated as a finite verb} took her hand, and the girl got up.}%
\verse{And this report went out into that whole region.}%
\verseWithHeading{Two Blind Men Healed}{And as\lebnote{“\textit{as}” is supplied as a component of the participle (“going away”) which is understood as temporal} Jesus was going away from there, two blind men followed him, crying out and saying, “Have mercy on us, Son of David!”}%
\verse{And when he\lnCVQ{} came into the house, the blind men approached him, and Jesus said to them, \JesusWords{“Do you believe that I am able to do this?”} They said to him, “Yes, Lord.”}%
\verse{Then he touched their eyes, saying, \JesusWords{“According to your faith let it be done for you.”}}%
\verse{And their eyes were opened, and Jesus sternly warned them, saying, \JesusWords{“See that no one finds out.”}}%
\verse{But they went out and\lebnote{“\textit{and}” supplied because previous participle “went out” translated as a finite verb} spread the report about him in that whole region.}%
\verseWithHeading{A Demon Expelled}{Now as\lebnote{“\textit{as}” is supplied as a component of the temporal genitive absolute participle (“were going away”)} they were going away, behold, they brought to him a demon-possessed man who was unable to speak.}%
\verse{And after\lebnote{“\textit{after}” is supplied as a component of the temporal genitive absolute participle (“had been expelled”)} the demon had been expelled, the one who had been mute spoke, and the crowds were astonished, saying, “This has never been seen before\lebnote{“never has it been seen thus”} in Israel!”}%
\verse{But the Pharisees were saying,\lebnote{Or “began to say”} “By the ruler of demons he expels the demons!”}%
\verseWithHeading{A Plentiful Harvest But Few Workers}{And Jesus was going around all the towns and the villages, teaching in their synagogues and proclaiming the good news of the kingdom and healing every\lnCVR{} disease and every\lnCVR{} sickness.}%
\verse{And when he\lnCVN{} saw the crowds, he had compassion for them, because they were weary and dejected, like sheep that did not have a shepherd.}%
\verse{Then he said to his disciples, \JesusWords{“The harvest is plentiful, but the workers are few.}}%
\verse{\JesusWords{Therefore ask the Lord of the harvest that he send out workers into his harvest.”}}%
\end{biblechapter}%
\begin{biblechapter}% Matthew 10
\verseWithHeading{The Twelve Commissioned and Sent Out}{And summoning his twelve disciples, he gave them authority over unclean spirits, so that they could expel them\lebnote{supplied from English context} and could heal every\lnCVS{} disease and every\lnCVS{} sickness.}%
\verse{Now these are the names of the twelve apostles: first Simon who is called Peter, and Andrew his brother, James\lebnote{Some manuscripts have “and James”} the son of Zebedee, and John his brother,}%
\verse{Philip, and Bartholomew, Thomas, and Matthew the tax collector, James the son of Alphaeus, and Thaddaeus,}%
\verse{Simon the Zealot,\lebnote{“the Cananean,” but according to BDAG 507 s.v., this term has no relation at all to the geographical terms for Cana or Canaan, but is derived from the Aramaic term for “enthusiast, zealot” (see Luke 6:15; Acts 1:13)} and Judas Iscariot — the one who also betrayed him.}%
\verse{Jesus sent out these twelve, instructing them saying, \JesusWords{“Do not go on the road to the Gentiles, and do not enter into a city of the Samaritans,}}%
\verse{\JesusWords{but go instead to the lost sheep of the house of Israel.}}%
\verse{\JesusWords{And as you\lnCVT{} are going, preach, saying, ‘The kingdom of heaven has come near!’}}%
\verse{\JesusWords{Heal those who are sick, raise the dead, cleanse lepers, expel demons. Freely you have received; freely give.}}%
\verse{\JesusWords{Do not procure gold or silver or copper for your belts.}}%
\verse{\JesusWords{Do not take a traveler’s bag for the road, or two tunics, or sandals, or a staff, for the worker is deserving of his provisions.}}%
\verse{\JesusWords{And into whatever town or village you enter, inquire who in it is worthy, and stay there until you depart.}}%
\verse{\JesusWords{And when you\lebnote{“\textit{when}” is supplied as a component of the participle (“enter”) which is understood as temporal} enter into the house, greet it.}}%
\verse{\JesusWords{And if the house is worthy, let your peace come upon it, but if it is not worthy, let your peace return to you.}}%
\verse{\JesusWords{And whoever does not welcome you or listen to your words, shake off the dust from your feet as you\lnCVT{} are going out of that house or that\lebnote{A repetition of “that” is supplied in English; the single Greek term is understood to modify both “house” and “town”} town.}}%
\verse{\JesusWords{Truly I say to you, it will be more bearable for the region of Sodom and Gomorrah on the day of judgment than for that town!}}%
\verseWithHeading{Persecution of Disciples Predicted}{\JesusWords{“Behold, I am sending you out like sheep in the midst of wolves. Therefore be wise as serpents and innocent as doves.}}%
\verse{\JesusWords{But beware of people, because they will hand you over to councils, and they will flog you in their synagogues.}}%
\verse{\JesusWords{And you will be brought before both governors and kings because of me, for a witness to them and to the Gentiles.}}%
\verse{\JesusWords{But whenever they hand you over, do not be anxious how to speak\lebnote{“\textit{to speak}” has been supplied for stylistic reasons, since “how to speak” is more natural in English than “how to say”; in Greek the same verb works with both expressions (“how or what you should say”) and also occurs again at the end of the verse} or what you should say, for what you should say will be given to you at that hour.}}%
\verse{\JesusWords{For you are not the ones who are speaking, but the Spirit of your Father who is speaking through you.}}%
\verse{\JesusWords{“And brother will hand over brother to death, and a father his children, and children will rise up against parents and have them put to death,}}%
\verse{\JesusWords{and you will be hated by everyone because of my name. But the one who endures to the end — this one will be saved.}}%
\verse{\JesusWords{And whenever they persecute you in this town, flee to another, for truly I say to you, you will never finish going through the towns of Israel until the Son of Man comes.}}%
\verse{\JesusWords{“A disciple is not superior to his teacher, nor a slave superior to his master.}}%
\verse{\JesusWords{It is enough for the disciple that he become like his teacher, and the slave like his master. If they have called the master of the house Beelzebul, how much more the members of his household?}}%
\verseWithHeading{Fear God Rather Than People}{\JesusWords{“Therefore do not be afraid of them, because nothing is hidden that will not be revealed, and nothing secret that will not become known.}}%
\verse{\JesusWords{What I say to you in the dark, tell in the light, and what you hear in your ear, proclaim on the housetops.}}%
\verse{\JesusWords{And do not be afraid of those who kill the body but are not able to kill the soul, but instead be afraid of the one who is able to destroy both soul and body in hell.}}%
\verse{\JesusWords{Are not two sparrows sold for a penny?\lebnote{Literally, “an assarion,” a Roman coin worth about 1/16 of a denarius} And one of them will not fall to the ground without the knowledge and consent\lebnote{“without”; the phrase “the knowledge and consent” is implied when this term is used of God} of your Father.}}%
\verse{\JesusWords{And even the hairs of your head are all numbered!}}%
\verse{\JesusWords{Therefore do not be afraid; you are worth more than many sparrows.}}%
\verse{\JesusWords{“Therefore everyone who acknowledges me before people, I also will acknowledge him before my Father who is in heaven.}}%
\verse{\JesusWords{But whoever denies me before people, I also will deny him before my Father who is in heaven.}}%
\verseWithHeading{Not Peace, But a Sword of Divisiveness}{\JesusWords{“Do not think that I have come to bring peace on the earth! I have not come to bring peace, but a sword.}}%
\verse{\JesusWords{For I have come to turn a man against his father, and a daughter against her mother, and a daughter-in-law against her mother-in-law.}}%
\verse{\JesusWords{And the enemies of a man will be the members of his household.}\lebnote{An allusion to Mic 7:6}}%
\verse{\JesusWords{The one who loves father or mother more than me is not worthy of me, and the one who loves son or daughter more than me is not worthy of me.}}%
\verse{\JesusWords{And whoever does not take up his cross and follow me is not worthy of me.}}%
\verse{\JesusWords{The one who finds his life will lose it, and the one who loses his life because of me will find it.}}%
\verseWithHeading{On Rewards}{\JesusWords{“The one who receives you receives me, and the one who receives me receives the one who sent me.}}%
\verse{\JesusWords{The one who receives a prophet in the name of a prophet will receive a prophet’s reward, and the one who receives a righteous person in the name of a righteous person will receive a righteous person’s reward.}}%
\verse{\JesusWords{And whoever gives one of these little ones only a cup of cold water to drink in the name of a disciple, truly I say to you, he will never lose his reward.”}}%
\end{biblechapter}%
\begin{biblechapter}% Matthew 11
\verseWithHeading{A Question from John the Baptist}{And it happened that when Jesus had finished giving orders to his twelve disciples, he went on from there to teach and to preach in their towns.}%
\verse{Now when\lebnote{“\textit{when}” is supplied as a component of the participle (“heard”) which is understood as temporal} John\lebnote{That is, John the Baptist} heard in prison the deeds of Christ, he sent word\lebnote{supplied from English context} by his disciples}%
\verse{and\lebnote{“\textit{and}” is supplied because the previous participle (“sent” in the previous verse) has been translated as a finite verb} said to him, “Are you the one who is to come, or should we look for another?”}%
\verse{And Jesus answered and\lnCVU{} said to them, \JesusWords{“Go and\lebnote{“\textit{and}” supplied because previous participle “go” translated as a finite verb} tell John what you hear and see:}}%
\verse{\JesusWords{the blind receive sight and the lame walk, lepers are cleansed and the deaf hear, and the dead are raised, and the poor have good news announced to them.}\lnCVV{}}%
\verse{\JesusWords{And whoever is not offended by me is blessed.”}}%
\verse{Now as\lebnote{“\textit{as}” is supplied as a component of the temporal genitive absolute participle (“were going away”)} these were going away, Jesus began to speak to the crowds concerning John: \JesusWords{“What did you go out into the wilderness to see? A reed shaken by the wind?}}%
\verse{\JesusWords{But what did you go out to see? A man dressed in soft clothing? Behold, those who wear soft clothing are in the houses of kings.}}%
\verse{\JesusWords{But why did you go out? To see a prophet?\lebnote{Some manuscripts have “But what did you go out to see? A prophet?”} Yes, I tell you, and even more than a prophet!}}%
\verse{\JesusWords{It is this man about whom it is written: ‘Behold, I am sending my messenger before your face, who will prepare your way before you.’}\lebnote{from Mal 3:1; cf. Mark 1:2}}%
\verse{\JesusWords{Truly I say to you, among those born of women there has not arisen one greater than John the Baptist. But the one who is least in the kingdom of heaven is greater than he.}}%
\verse{\JesusWords{But from the days of John the Baptist until now, the kingdom of heaven is treated violently, and the violent claim\lebnote{Or “seize”} it.}}%
\verse{\JesusWords{For all the prophets and the law prophesied until John,}}%
\verse{\JesusWords{and if you are willing to accept it,\lnCVV{} he is Elijah, the one who is going to come.}}%
\verse{\JesusWords{The one who has ears, let him hear!}}%
\verse{\JesusWords{“But to what shall I compare this generation? It is like children sitting in the marketplaces who call out to one another,}}%
\verse{\JesusWords{saying, ‘We played the flute for you and you did not dance; we sang a lament and you did not mourn.’}}%
\verse{\JesusWords{For John came neither eating nor drinking, and they say, ‘He has a demon!’}}%
\verse{\JesusWords{The Son of Man came eating and drinking, and they say, ‘Behold, a man who is a glutton and a drunkard, a friend of tax collectors and sinners!’ But wisdom is vindicated by her deeds.”}\lebnote{Many Greek manuscripts, including most later ones, read “children”}}%
\verseWithHeading{Woes Pronounced on Unrepentant Towns}{Then he began to reproach the towns in which the majority of his miracles had been done, because they did not repent:}%
\verse{\JesusWords{“Woe to you, Chorazin! Woe to you, Bethsaida! For if the miracles done in you had been done in Tyre and Sidon, they would have repented long ago in sackcloth and ashes.}}%
\verse{\JesusWords{Nevertheless I tell you, it will be more bearable for Tyre and for Sidon on the day of judgment than for you!}}%
\verse{\JesusWords{And you, Capernaum, will you be exalted to heaven? No, you will be brought down to Hades! For if the miracles done in you had been done in Sodom, it would have remained until today.}}%
\verse{\JesusWords{Nevertheless I tell you that it will be more bearable for the region of Sodom on the day of judgment than for you!”}}%
\verseWithHeading{A Yoke That Is Easy}{At that time Jesus answered and\lnCVU{} said, \JesusWords{“I praise you, Father, Lord of heaven and earth, because you have hidden these things from the wise and intelligent, and have revealed them to young children.}\lebnote{Or perhaps “to the childlike,” or “to the innocent”}}%
\verse{\JesusWords{Yes, Father, for to do so was your gracious will.}\lebnote{“was pleasing before you”}}%
\verse{\JesusWords{All things have been handed over to me by my Father, and no one knows the Son except the Father, and no one knows the Father except the Son and anyone to whom\lebnote{“to whomever”} the Son wants to reveal him.}\lnCVV{}}%
\verse{\JesusWords{Come to me, all of you who labor and are burdened, and I will give you rest.}}%
\verse{\JesusWords{Take my yoke on you and learn from me, for I am gentle and humble in heart, and you will find rest for your souls.}}%
\verse{\JesusWords{For my yoke is easy to carry and my burden is light.”}}%
\end{biblechapter}%
\begin{biblechapter}% Matthew 12
\verseWithHeading{Plucking Grain on the Sabbath}{At that time Jesus went through the grain fields on the Sabbath. And his disciples were hungry, and they began to pluck off heads of grain and eat them.\lnCVW{}}%
\verse{But when\lebnote{“\textit{when}” is supplied as a component of the participle (“saw”) which is understood as temporal} the Pharisees saw it,\lnCVW{} they said to him, “Behold, your disciples are doing what it is not permitted to do on the Sabbath!”}%
\verse{So he said to them, \JesusWords{“Have you not read what David did when he was hungry, and those with him,}}%
\verse{\JesusWords{how he entered into the house of God and ate the bread of the presentation, which it was not permitted for him or for those with him to eat, but only for the priests?}}%
\verse{\JesusWords{Or have you not read in the law that on the Sabbath the priests in the temple violate the sanctity of the Sabbath and are guiltless?}}%
\verse{\JesusWords{But I tell you that something greater than the temple is here!}}%
\verse{\JesusWords{And if you had known what it means,\lebnote{“it is”} ‘I want mercy and not sacrifice,’ you would not have condemned the guiltless.}}%
\verse{\JesusWords{For the Son of Man is lord of the Sabbath.”}}%
\verseWithHeading{A Man with a Withered Hand Healed}{And going on from there he came into their synagogue.}%
\verse{And behold, there was a man who had a withered hand, and they asked him, saying, “Is it permitted to heal on the Sabbath?” in order that they could accuse him.}%
\verse{But he said to them, \JesusWords{“What man will there be among you who will have one sheep and if this one fell into a pit on the Sabbath, will not take hold of it and lift it\lnCVX{} out?}}%
\verse{\JesusWords{Then to what degree is a man worth more than a sheep? So then, it is permitted to do good on the Sabbath.”}}%
\verse{Then he said to the man, \JesusWords{“Stretch out your hand,”} and he stretched it\lnCVX{} out, and it was restored as healthy as the other one.}%
\verse{But the Pharisees went out and plotted\lebnote{“taking counsel”} against him in order that they could destroy him.}%
\verseWithHeading{God’s Chosen Servant}{Now Jesus, when he\lebnote{“\textit{when}” is supplied as a component of the participle (“learned”) which is understood as temporal} learned of it,\lnCVW{} withdrew from there, and many\lebnote{Some manuscripts have “many crowds”} followed him, and he healed them all.}%
\verse{And he warned them that they should not reveal his identity,\lebnote{“make him known”}}%
\verse{in order that what was spoken through the prophet Isaiah would be fulfilled, who said,}%
\verse{“Behold my servant whom I have chosen, my beloved in whom my soul is well pleased. I will put my Spirit on him, and he will proclaim justice to the Gentiles.}%
\verse{He will not quarrel or cry out, nor will anyone hear his voice in the streets.}%
\verse{A crushed reed he will not break, and a smoldering wick he will not extinguish, until he brings justice to victory.}%
\verse{And in his name the Gentiles\lebnote{Or “the nations”; the same Greek word (in the plural) can be translated as “nations,” “Gentiles,” or “pagans”} will hope.\lebnote{Verses 18–21 are a quotation from Isa 42:1–4}}%
\verseWithHeading{A House Divided Cannot Stand}{Then a demon-possessed man who was blind and mute was brought to him. And he healed him so that the man who was mute could speak and see.}%
\verse{And all the crowds were amazed and began saying, “Perhaps this one is the Son of David!”}%
\verse{But the Pharisees, when they\lebnote{“\textit{when}” is supplied as a component of the participle (“heard”) which is understood as temporal} heard it,\lnCVW{} said, “This man does not expel demons except by Beelzebul the ruler of demons!”}%
\verse{But knowing their thoughts, he said to them, \JesusWords{“Every kingdom divided against itself is laid waste, and every city or household divided against itself will not stand.}}%
\verse{\JesusWords{And if Satan expels Satan, he is divided against himself. How then will his kingdom stand?}}%
\verse{\JesusWords{And if I expel demons by Beelzebul, by whom do your sons expel them?\lnCVW{} For this reason they will be your judges!}}%
\verse{\JesusWords{But if I expel demons by the Spirit of God, then the kingdom of God has come upon you!}}%
\verse{\JesusWords{Or how can someone enter into the house of a strong man and steal his property, unless he first ties up the strong man? And then he can thoroughly plunder his house.}}%
\verse{\JesusWords{The one who is not with me is against me, and the one who does not gather with me scatters.}}%
\verse{\JesusWords{For this reason I tell you, every sin and blasphemy will be forgiven people, but the blasphemy against the Spirit will not be forgiven!}}%
\verse{\JesusWords{And whoever speaks a word against the Son of Man, it will be forgiven him. But whoever speaks against the Holy Spirit, it will not be forgiven him either in this age or in the coming one!}}%
\verseWithHeading{Good Trees and Good Fruit or Bad Trees and Bad Fruit}{\JesusWords{“Either make the tree good and its fruit is good, or make the tree bad and its fruit is bad, for the tree is known by its fruit.}}%
\verse{\JesusWords{Offspring of vipers! How are you able to say good things when you\lebnote{“\textit{when}” is supplied as a component of the participle (“are”) which is understood as temporal} are evil? For from the abundance of the heart the mouth speaks.}}%
\verse{\JesusWords{The good person from his\lnCVY{} good treasury brings out good things, and the evil person from his\lnCVY{} evil treasury brings out evil things.}}%
\verse{\JesusWords{But I tell you that every worthless word that they speak, people will give an account for it on the day of judgment!}}%
\verse{\JesusWords{For by your words you will be vindicated, and by your words you will be condemned.”}}%
\verseWithHeading{The Sign of Jonah}{Then some of the scribes and Pharisees answered him saying, “Teacher, we want to see a sign from you!”}%
\verse{But he answered and\lnCVZ{} said to them, \JesusWords{“An evil and adulterous generation desires a sign, and no sign will be given to it except the sign of the prophet Jonah!}}%
\verse{\JesusWords{For just as Jonah was in the belly of the huge fish three days and three nights, so the Son of Man will be in the heart of the earth three days and three nights.}}%
\verse{\JesusWords{The people of Nineveh will stand up at the judgment with this generation and condemn it, because they repented at the proclamation of Jonah, and behold, something\lnCWA{} greater than Jonah is here!}}%
\verse{\JesusWords{The queen of the south will rise up at the judgment with this generation and condemn it, because she came from the ends of the earth to hear the wisdom of Solomon, and behold, something\lnCWA{} greater than Solomon is here!}}%
\verseWithHeading{An Unclean Spirit Returns}{\JesusWords{“Now whenever an unclean spirit has gone out of a person, it travels through waterless places searching for rest, and does not find it.}\lnCVW{}}%
\verse{\JesusWords{Then it says, ‘I will return to my house from which I came out.’ And when it\lebnote{“\textit{when}” is supplied as a component of the participle (“arrives”) which is understood as temporal} arrives it finds the house\lnCVX{} unoccupied and swept and put in order.}}%
\verse{\JesusWords{Then it goes and brings along with itself seven other spirits more evil than itself, and they go in and\lebnote{“\textit{and}” supplied because previous participle “go in” translated as a finite verb} live there. And the last state of that person becomes worse than the first. So it will be for this evil generation also!”}}%
\verseWithHeading{Jesus’ Mother and Brothers}{And\lebnote{Some manuscripts omit “And”} while\lebnote{“\textit{while}” is supplied as a component of the temporal genitive absolute participle (“speaking”)} he was still speaking to the crowds, behold, his mother and brothers were standing there outside, desiring to speak to him.}%
\verse{And someone told him, “Behold, your mother and your brothers are standing there outside desiring to speak to you.”}%
\verse{But he answered and\lnCVZ{} said to the one who told him, \JesusWords{“Who is my mother, and who are my brothers?”}}%
\verse{And stretching out his hand toward his disciples, he said, \JesusWords{“Behold my mother and my brothers!}}%
\verse{\JesusWords{For whoever does the will of my Father who is in heaven, he is my brother and sister and mother.”}}%
\end{biblechapter}%
\begin{biblechapter}% Matthew 13
\verseWithHeading{The Parable of the Sower}{On that day Jesus went out of the house and\lebnote{“\textit{and}” supplied because previous participle “went out” translated as a finite verb} was sitting by the sea.\lebnote{That is, the Sea of Galilee}}%
\verse{And large crowds gathered close around him, so that he got into a boat to sit down, and all the crowd was standing on the shore.}%
\verse{And he spoke many things to them in parables, saying, \JesusWords{“Behold, the sower went out to sow,}}%
\verse{\JesusWords{and while he was sowing, some seed\lebnote{“some of which”} fell on the side of the path, and the birds came and\lnCWB{} devoured it.}}%
\verse{\JesusWords{And other seed fell on the rocky ground, where it did not have much soil, and it sprang up at once because it did not have any depth of soil.}}%
\verse{\JesusWords{But when\lebnote{“\textit{when}” is supplied as a component of the temporal genitive absolute participle (“rose”)} the sun rose it was scorched, and because it did not have enough root, it withered.}}%
\verse{\JesusWords{And other seed fell among the thorn plants, and the thorn plants came up and choked it.}}%
\verse{\JesusWords{But other seed fell on the good soil and produced grain,\lebnote{“fruit,” describing here the grain harvested from the healthy plants; in contemporary English this would more naturally be expressed by terms like “grain” or “crop”} this one a hundred times as much and this one sixty and this one thirty.}}%
\verse{\JesusWords{The one who has ears, let him hear!”}}%
\verseWithHeading{The Reason for the Parables}{And the disciples came up and\lebnote{“\textit{and}” supplied because previous participle “came up” translated as a finite verb} said to him, “Why\lebnote{“because of what”} do you speak to them in parables?”}%
\verse{And he answered and\lnCWC{} said to them, \JesusWords{“To you it has been granted to know the mysteries of the kingdom of heaven, but to those people it has not been granted.}}%
\verse{\JesusWords{For whoever has, to him more will be given, and he will have an abundance. But whoever does not have, even what he has will be taken away from him.}}%
\verse{\JesusWords{For this reason I speak to them in parables, because seeing they do not see, and hearing they do not hear, nor do they understand,}}%
\verse{\JesusWords{and with reference to them the prophecy of Isaiah is fulfilled that says, “You will listen carefully\lebnote{“with hearing you will hear”} and will never understand, and you will look closely\lebnote{“seeing you will see”} and will never perceive.}}%
\verse{\JesusWords{For the heart of this people has become dull, and with their ears they hear with difficulty, and they have shut their eyes, so that they would not see with their eyes and hear with their ears and understand with their heart and turn, and I would heal them.”}\lebnote{from Isa 6:9–10}}%
\verse{\JesusWords{But your eyes are blessed because they see, and your ears because they hear.}}%
\verse{\JesusWords{For truly I say to you that many prophets and righteous people longed to see what you see, and did not see it,\lnCWD{} and to hear what you hear, and did not hear it!\lnCWD{}}}%
\verseWithHeading{The Parable of the Sower Interpreted}{\JesusWords{“You, therefore, listen to the parable of the sower:}}%
\verse{\JesusWords{When\lebnote{“\textit{when}” is supplied as a component of the temporal genitive absolute participle (“hears”)} anyone hears the word about the kingdom and does not understand it,\lnCWD{} the evil one comes and snatches away what was sown in his heart. This is what was sown on the side of the path.}}%
\verse{\JesusWords{And what was sown on the rocky ground — this is the one who hears the word and immediately receives it with joy.}}%
\verse{\JesusWords{But he does not have a root in himself, but lasts only a little while,\lebnote{“is temporary”} and when\lebnote{“\textit{when}” is supplied as a component of the temporal genitive absolute participle (“happens”)} affliction or persecution happens because of the word, immediately he falls away.}}%
\verse{\JesusWords{And what was sown into the thorn plants — this is the one who hears the word, and the anxiety of this world\lebnote{Some manuscripts have “of the world”} and the deceitfulness of wealth choke the word and it becomes unproductive.}}%
\verse{\JesusWords{But what was sown on the good soil — this is the one who hears the word and understands it,\lnCWD{} who indeed bears fruit and produces, this one a hundred times as much, and this one sixty, and this one thirty.”}}%
\verseWithHeading{The Parable of the Weeds Among the Wheat}{He put before them another parable, saying, \JesusWords{“The kingdom of heaven may be compared to a man who sowed good seed in his field.}}%
\verse{\JesusWords{But while his\lebnote{“the”: the Greek article is used here as a possessive pronoun} people were sleeping, his enemy came and sowed darnel\lebnote{A weed that looks similar to wheat but has poisonous seeds} in the midst of the wheat and went away.}}%
\verse{\JesusWords{So when the wheat\lebnote{“grass,” “hay,” but in this context referring to the good plants as opposed to the weeds} sprouted and yielded grain, then the darnel appeared also.}}%
\verse{\JesusWords{So the slaves of the master of the house came and\lnCWB{} said to him, ‘Master, did you not sow good seed in your field? How then does it have darnel?’}}%
\verse{\JesusWords{And he said to them, ‘An enemy has done this!’ So the slaves said to him, ‘Then do you want us to go and\lebnote{“\textit{and}” supplied because previous participle “go” translated as an English infinitive} gather them?’}}%
\verse{\JesusWords{But he said, “No, lest when you\lebnote{“\textit{when}” is supplied as a component of the participle (“gather”) which is understood as temporal} gather the darnel you uproot the wheat together with it.}}%
\verse{\JesusWords{Let both grow together until the harvest, and at the season of the harvest I will tell the reapers, “First gather the darnel and tie it into bundles to burn them, but gather the wheat into my storehouse.”’”}}%
\verseWithHeading{The Parable of the Mustard Seed}{He put before them another parable, saying, \JesusWords{“The kingdom of heaven is like a mustard seed that a man took and\lnCWE{} sowed in his field.}}%
\verse{\JesusWords{It\lebnote{“which,” but a new sentence is started here in the English translation} is the smallest of all the seeds, but when it is grown it is larger than the garden herbs and becomes a tree, so that the birds of the sky come and nest in its branches.”}}%
\verseWithHeading{The Parable of the Yeast}{He told them another parable: \JesusWords{“The kingdom of heaven is like yeast that a woman took and\lnCWE{} put into three measures of wheat flour until the whole batch was leavened.”}}%
\verseWithHeading{Parables Fulfill Prophecy}{Jesus spoke all these things to the crowds in parables, and he was saying nothing to them without a parable,}%
\verse{in order that what was spoken through the prophet would be fulfilled, who said, “I will open my mouth in parables; I will proclaim what has been hidden since the creation.”\lebnote{from Ps 78:2}\lebnote{Some manuscripts have “since the creation of the world”}}%
\verseWithHeading{The Parable of the Weeds Interpreted}{Then he left the crowds and\lebnote{“\textit{and}” supplied because previous participle “left” translated as a finite verb} came into the house, and his disciples came to him saying, “Explain the parable of the darnel in the field to us.”}%
\verse{So he answered and\lnCWC{} said, \JesusWords{“The one who sows the good seed is the Son of Man,}}%
\verse{\JesusWords{and the field is the world. And the good seed — these are the sons of the kingdom, but the darnel are the sons of the evil one.}}%
\verse{\JesusWords{And the enemy who sowed them is the devil, and the harvest is the end of the age, and the reapers are angels.}}%
\verse{\JesusWords{Thus just as the darnel is gathered and burned\lebnote{Some manuscripts have “burned up”} with fire, so it will be at the end of the age.}}%
\verse{\JesusWords{The Son of Man will send out his angels and they will gather out of his kingdom all the causes of sin and those who do lawless deeds,}}%
\verse{\JesusWords{and throw them into the fiery furnace.\lnCWF{} In that place there will be weeping and gnashing of teeth!}}%
\verse{\JesusWords{Then the righteous will shine like the sun in the kingdom of their Father.\lebnote{An allusion to Dan 12:3} The one who has ears, let him hear!}}%
\verseWithHeading{The Parable of the Treasure Hidden in a Field}{\JesusWords{“The kingdom of heaven is like treasure hidden in a field, that a man found and\lebnote{“\textit{and}” supplied because previous participle “found” translated as a finite verb} concealed, and in his joy he goes and sells everything that he has and buys that field.}}%
\verseWithHeading{The Parable of the Valuable Pearl}{\JesusWords{“Again, the kingdom of heaven is like a merchant searching for fine pearls.}}%
\verse{\JesusWords{And when he\lebnote{“\textit{when}” is supplied as a component of the participle (“found”) which is understood as temporal} found one very valuable pearl, he went and\lebnote{“\textit{and}” supplied because previous participle “went” translated as a finite verb} sold everything that he possessed and purchased it.}}%
\verseWithHeading{The Parable of the Dragnet}{\JesusWords{“Again, the kingdom of heaven is like a dragnet that was thrown into the sea and gathered fish\lebnote{supplied from English context} of every kind,}}%
\verse{\JesusWords{which when it was filled they pulled to shore and sat down and\lebnote{“\textit{and}” supplied because previous participle “sat down” translated as a finite verb} collected the good fish\lebnote{the word “\textit{fish}” is not in the Greek text but is implied} into containers, but the bad they threw out.}}%
\verse{\JesusWords{Thus it will be at the end of the age. The angels will go out and separate the evil from among the righteous}}%
\verse{\JesusWords{and throw them into the fiery furnace.\lnCWF{} In that place there will be weeping and gnashing of teeth!}}%
\verse{\JesusWords{“Have you understood all these things?”} They said to him, “Yes.”}%
\verse{And he said to them, \JesusWords{“For this reason every scribe who has been trained for the kingdom of heaven is like the master of the house who brings out of his storeroom new things and old things.”}}%
\verseWithHeading{Rejected at Nazareth}{And it happened that when Jesus had finished these parables he went away from there.}%
\verse{And he came to his hometown and\lnCWB{} began to teach\lebnote{Imperfect tense as ingressive (“began to teach”)} them in their synagogue, so that they were amazed and said, “From where did this man get this wisdom and these miracles?}%
\verse{Is not this one the son of the carpenter? Is not his mother called Mary and his brothers James and Joseph and Simon and Judas?}%
\verse{And are not all his sisters with us? From where then did this man get all these things?”}%
\verse{And they were offended by him. But Jesus said to them, \JesusWords{“A prophet is not without honor except in his own hometown and in his own household.”}}%
\verse{And he did not perform many miracles in that place because of their unbelief.}%
\end{biblechapter}%
\begin{biblechapter}% Matthew 14
\verseWithHeading{Herod Kills John the Baptist}{At that time Herod the tetrarch heard the report about Jesus}%
\verse{and he said to his servants, “This is John the Baptist! He has been raised from the dead, and for this reason miraculous powers are at work in him.”}%
\verse{For Herod, after\lebnote{“\textit{after}” is supplied as a component of the participle (“arresting”) which is understood as temporal} arresting John, bound him and put him\lnCWG{} in prison on account of Herodias, the wife of his brother Philip,}%
\verse{because John had been saying to him, “It is not permitted for you to have her.”}%
\verse{And although he\lebnote{“\textit{although}” is supplied as a component of the participle (“wanted”) which is understood as concessive} wanted to kill him, he feared the crowd, because they looked upon him as a prophet.}%
\verse{But when\lebnote{“\textit{when}” is supplied as a component of the participle (“took place”) which is understood as temporal} Herod’s birthday celebration took place, the daughter of Herodias danced in the midst of them and pleased Herod.}%
\verse{Therefore he promised with an oath to give her whatever she asked.}%
\verse{And coached by her mother, she said, “Give me the head of John the Baptist here on a platter!”}%
\verse{And although\lebnote{“\textit{although}” is supplied as a component of the participle (“was distressed”) which is understood as concessive} the king was distressed, because of his oaths and his dinner guests\lebnote{“those reclining at table with”} he commanded the request\lnCWG{} to be granted.}%
\verse{And he sent orders\lnCWG{} and\lebnote{“\textit{and}” supplied because previous participle “sent” translated as a finite verb} had John beheaded in the prison,}%
\verse{and his head was brought on a platter and given to the girl, and she brought it\lnCWG{} to her mother.}%
\verse{And his disciples came and\lebnote{“\textit{and}” supplied because previous participle “came” translated as a finite verb} took away the corpse and buried it, and went and\lebnote{“\textit{and}” supplied because previous participle “went” translated as a finite verb} told Jesus.}%
\verseWithHeading{The Feeding of Five Thousand}{Now when\lnCWH{} Jesus heard it,\lnCWI{} he withdrew from there in a boat to an isolated place by himself. And when\lnCWH{} the crowds heard it,\lnCWI{} they followed him by land from the towns.}%
\verse{And as he\lebnote{“\textit{as}” is supplied as a component of the participle (“got out”) which is understood as temporal} got out, he saw the large crowd and had compassion on them and healed their sick.}%
\verse{Now when it\lebnote{“\textit{when}” is supplied as a component of the temporal genitive absolute participle (“was”)} was evening, the disciples came to him saying, “The place is desolate and the hour is late.\lebnote{“has passed away”} Release the crowds so that they can go away into the villages and\lebnote{“\textit{and}” supplied because previous participle “go away” translated as a finite verb} purchase food for themselves.”}%
\verse{But Jesus said to them, \JesusWords{“They do not need\lebnote{“have need”} to go away. You give them something\lnCWG{} to eat.”}}%
\verse{And they said to him, “We do not have anything\lnCWG{} here except five loaves and two fish.”}%
\verse{So he said, \JesusWords{“Bring them here to me.”}}%
\verse{And he commanded the crowds to recline for a meal on the grass. Taking the five loaves and the two fish and\lebnote{“\textit{and}” has been supplied in the English translation for stylistic reasons} looking up to heaven, he gave thanks. And after\lebnote{“\textit{when}” is supplied as a component of the participle (“breaking”) which is understood as temporal} breaking them,\lnCWI{} he gave the loaves to the disciples, and the disciples gave them\lebnote{“\textit{gave them}” is an implied repetition of the earlier verb} to the crowds.}%
\verse{And they all ate and were satisfied, and they picked up what was left over of the broken pieces, twelve baskets full.}%
\verse{Now those who ate were about five thousand men, in addition to women and children.}%
\verseWithHeading{Jesus Walks on the Water}{And immediately he made the disciples get into the boat and go ahead of him to the other side, while he sent away the crowds.}%
\verse{And after he\lebnote{“\textit{after}” is supplied as a component of the participle (“sent away”) which is understood as temporal} sent away the crowds, he went up on the mountain by himself to pray. So when\lebnote{“\textit{when}” is supplied as a component of the temporal genitive absolute participle (“came”)} evening came, he was there alone.}%
\verse{But the boat was already many stadia distant from the land, being beaten by the waves, because the wind was against it.}%
\verse{And in the fourth watch of the night he came to them, walking on the sea.}%
\verse{But the disciples, when they\lnCWJ{} saw him walking on the sea, were terrified, saying, “It is a ghost!” and they cried out in fear.}%
\verse{But immediately Jesus spoke to them, saying, \JesusWords{“Have courage, I am he! Do not be afraid!”}}%
\verse{And Peter answered him and\lebnote{“\textit{and}” supplied because previous participle “answered” translated as a finite verb} said, “Lord, if it is you,\lebnote{“you are” \textit{he}} command me to come to you on the water!”}%
\verse{So he said, \JesusWords{“Come!”} And getting out of the boat, Peter walked on the water and came toward Jesus.}%
\verse{But when he\lnCWJ{} saw the strong wind, he was afraid. And beginning to sink, he cried out, saying, “Lord, save me!”}%
\verse{And immediately Jesus extended his\lebnote{“the”: the Greek article is used here as a possessive pronoun} hand and\lebnote{“\textit{and}” supplied because previous participle “extended” translated as a finite verb} caught him and said to him, \JesusWords{“You of little faith! Why did you doubt?”}}%
\verse{And when\lebnote{“\textit{when}” is supplied as a component of the temporal genitive absolute participle (“got”)} they got into the boat, the wind abated.}%
\verse{So those in the boat worshiped him, saying, “Truly you are the Son of God!”}%
\verseWithHeading{Many Healed at Gennesaret}{And after they\lebnote{“\textit{after}” is supplied as a component of the participle (“had crossed over”) which is understood as temporal} had crossed over, they came to land at Gennesaret.}%
\verse{And when\lebnote{“\textit{when}” is supplied as a component of the participle (“recognized”) which is understood as temporal} the men of that place recognized him, they sent word into that whole surrounding region, and they brought to him all those who were sick.\lebnote{“having badly”}}%
\verse{And they were imploring him that they might only touch the edge\lebnote{Or “tassel”} of his cloak, and all those who touched it\lnCWG{} were cured.}%
\end{biblechapter}%
\begin{biblechapter}% Matthew 15
\verseWithHeading{Human Traditions and God’s Commandments}{Then Pharisees and scribes came to Jesus from Jerusalem, saying,}%
\verse{“Why do your disciples break the tradition of the elders? For they do not wash their hands when they eat a meal.”\lebnote{“bread”}}%
\verse{So he answered and\lnCWK{} said to them, \JesusWords{“Why do you also break the commandment of God because of your tradition?}}%
\verse{\JesusWords{For God said, ‘Honor your\lnCWL{} father and your\lnCWL{} mother,’\lebnote{from Exod 20:12; Deut 5:16} and ‘The one who speaks evil of father or mother must certainly die\lebnote{“let him die the death”}.’\lebnote{from Exod 21:17; Lev 20:9}}}%
\verse{\JesusWords{But you say, ‘Whoever says to his\lnCWL{} father or his\lnCWL{} mother, “Whatever benefit you would have received\lebnote{“you would have been benefited”} from me is a gift to God,”}}%
\verse{\JesusWords{need not honor his father,’\lebnote{Most later manuscripts add “or his mother”} and you make void the word of God for the sake of your tradition.}}%
\verse{\JesusWords{Hypocrites! Isaiah correctly prophesied about you saying,}}%
\verse{\JesusWords{‘This people honors me with their\lnCWL{} lips, but their heart is far, far away from me,}}%
\verse{\JesusWords{and they worship me in vain, teaching as doctrines the commandments of men.’”\lebnote{from Isa 29:13} }}%
\verseWithHeading{Defilement from Within}{And summoning the crowd, he said to them, \JesusWords{“Hear and understand:}}%
\verse{\JesusWords{It is not what goes into the mouth that defiles a person, but what comes out of the mouth — this defiles a person.”}}%
\verse{Then the disciples came and\lnCWM{} said to him, “Do you know that the Pharisees were offended when they\lebnote{“\textit{when}” is supplied as a component of the participle (“heard”) which is understood as temporal} heard this saying?”}%
\verse{And he answered and\lnCWK{} said, \JesusWords{“Every plant that my heavenly Father did not plant will be uprooted.}}%
\verse{\JesusWords{Let\lebnote{Or “Depart from”} them! They are blind guides of the blind. And if the blind guide the blind, both will fall into a pit.”}}%
\verse{But Peter answered and\lnCWK{} said to him, “Explain this parable to us.”}%
\verse{But he said, \JesusWords{“Are you also still without understanding?}}%
\verse{\JesusWords{Do you not understand that everything that enters into the mouth goes into the stomach and is evacuated into the latrine?}}%
\verse{\JesusWords{But the things that come out of the mouth come from the heart, and these defile the person.}}%
\verse{\JesusWords{For from the heart come evil plans, murder, adultery, sexual immorality, theft, false witness, abusive speech.}}%
\verse{\JesusWords{These are the things that defile a person. But eating with unwashed hands does not defile a person!”}}%
\verseWithHeading{A Canaanite Woman’s Great Faith}{And departing from there, Jesus went away to the region of Tyre and Sidon.}%
\verse{And behold, a Canaanite woman from that district came and cried out, saying, “Have mercy on me, Lord, Son of David! My daughter is severely possessed by a demon!”}%
\verse{But he did not answer her a word. And his disciples came up and\lebnote{“\textit{and}” supplied because previous participle “came up” translated as a finite verb} asked him, saying, “Send her away, because she is crying out after us!”}%
\verse{But he answered and\lnCWK{} said, \JesusWords{“I was not sent except to the lost sheep of the house of Israel.”}}%
\verse{But she came and\lnCWM{} knelt down before him, saying, “Lord, help me!”}%
\verse{And he answered and\lnCWK{} said, \JesusWords{“It is not right to take the children’s bread and throw it\lnCWN{} to the dogs!”}}%
\verse{So she said, “Yes, Lord, for even the dogs eat the crumbs that fall from their master’s table.”}%
\verse{Then Jesus answered and said to her, \JesusWords{“O woman, your faith is great! Let it be done for you as you want.”} And her daughter was healed from that hour.}%
\verseWithHeading{Many Others Healed in Galilee}{And departing from there, Jesus went along the Sea of Galilee, and he went up on the mountain and\lebnote{“\textit{and}” supplied because previous participle “went up” translated as a finite verb} was sitting there.}%
\verse{And large crowds came to him, having with them the mute, blind, lame, crippled,\lebnote{Some manuscripts have “\textit{the} lame, blind, crippled, mute”} and many others, and they put them down at his feet, and he healed them.}%
\verse{So then the crowd was astonished when they\lebnote{“\textit{when}” is supplied as a component of the participle (“saw”) which is understood as temporal} saw the mute speaking, the crippled healthy, and the lame walking, and the blind seeing, and they praised the God of Israel.}%
\verseWithHeading{The Feeding of Four Thousand}{And Jesus summoned his disciples and\lebnote{“\textit{and}” supplied because previous participle “summoned” translated as a finite verb} said, \JesusWords{“I have compassion on the crowd, because they have remained with me three days already and do not have anything to eat, and I do not want to send them away hungry lest they give out on the way.”}}%
\verse{And the disciples said to him, “Where in this desolate place can we get\lebnote{“for us”} so much bread that such a great crowd could be satisfied?”}%
\verse{And Jesus said to them, \JesusWords{“How many loaves do you have?”} So they said, “Seven, and a few little fish.”}%
\verse{And commanding the crowd to recline for a meal on the ground,}%
\verse{he took the seven loaves and the fish and after he\lnCWO{} had given thanks, he broke them\lnCWN{} and began giving\lebnote{Imperfect tense as ingressive (“began giving”)} them\lnCWN{} to the disciples, and the disciples gave them to the crowds.}%
\verse{And they all ate and were satisfied, and they picked up what was left over of the broken pieces, seven baskets full.}%
\verse{Now those who ate were four thousand men, in addition to women and children.}%
\verse{And after he\lnCWO{} sent away the crowds, he got into the boat and went to the region of Magadan.}%
\end{biblechapter}%
\begin{biblechapter}% Matthew 16
\verseWithHeading{The Signs of the Times}{And when\lnCWP{} the Pharisees and Sadducees came to test him,\lnCWQ{} they asked him to show them a sign from heaven.}%
\verse{So he answered and\lnCWR{} said to them, \JesusWords{“When\lebnote{“\textit{when}” is supplied as a component of the temporal genitive absolute participle (“comes”)} evening comes you say, ‘It will be fair weather because the sky is red,’}}%
\verse{\JesusWords{and early in the morning, ‘Today it will be stormy weather, because the sky is red and\lebnote{“\textit{and}” is supplied in the translation because of English style} darkening.’ You know how to evaluate correctly the appearance of the sky, but you are not able to evaluate\lebnote{“\textit{to evaluate}” is an implied repetition of the verb earlier in the verse} the signs of the times.}}%
\verse{\JesusWords{An evil and adulterous generation seeks for a sign, and a sign will not be given to it except the sign of Jonah!”} And he left them and\lebnote{“\textit{and}” supplied because previous participle “left” translated as a finite verb} went away.}%
\verseWithHeading{Beware the Leaven of the Pharisees and Sadducees}{And when\lebnote{“\textit{when}” is supplied as a component of the participle (“arrived”) which is understood as temporal} the disciples arrived at the other side,\lebnote{That is, the other side of the Sea of Galilee} they had forgotten to take bread.}%
\verse{And Jesus said to them, \JesusWords{“Watch out for and beware of the leaven of the Pharisees and Sadducees!”}}%
\verse{So they were discussing this\lnCWS{} among themselves, saying, “It is because we did not take bread.”}%
\verse{But knowing this,\lnCWQ{} Jesus said, \JesusWords{“Why are you discussing among yourselves that you did not take bread,\lebnote{Some manuscripts have “you do not have bread”} you of little faith?}}%
\verse{\JesusWords{Do you not yet understand or do you not remember the five loaves for the five thousand, and how many baskets you took up?}}%
\verse{\JesusWords{Or the seven loaves for the four thousand and how many baskets you took up?}}%
\verse{\JesusWords{How do you not understand that I did not speak to you about bread? But beware of the leaven of the Pharisees and Sadducees!”}}%
\verse{Then they understood that he did not say to beware of the leaven of bread, but of the teaching of the Pharisees and Sadducees.}%
\verseWithHeading{Peter’s Confession at Caesarea Philippi}{Now when\lnCWP{} Jesus came to the region of Caesarea Philippi,\lebnote{“of Philip”} he began asking\lebnote{Imperfect tense as ingressive (“began asking”)} his disciples, saying, \JesusWords{“Who do people say that the Son of Man is?”}}%
\verse{And they said, Some\lebnote{“those on the one hand”} say John the Baptist, but others Elijah, and others Jeremiah or one of the prophets.”}%
\verse{He said to them, \JesusWords{“But who do you say that I am?”}}%
\verse{And Simon Peter answered and\lnCWR{} said, “You are the Christ, the Son of the living God!”}%
\verse{And Jesus answered and\lnCWR{} said to him, \JesusWords{“Blessed are you, Simon son of Jonah, because flesh and blood did not reveal this\lnCWS{} to you, but my Father who is in heaven.}}%
\verse{\JesusWords{And I also say to you that you are Peter, and on this rock I will build my church, and the gates of Hades will not overpower it!}}%
\verse{\JesusWords{I will give you the keys of the kingdom of heaven, and whatever you bind on earth will be bound\lebnote{Or “will have been bound”} in heaven, and whatever you release on earth will be released\lebnote{Or “will have been released”} in heaven.”}}%
\verse{Then he commanded the disciples that they should tell no one that he was the Christ.}%
\verseWithHeading{Jesus Predicts His Death and Resurrection}{From that time on Jesus began to show his disciples that he must go to Jerusalem and suffer many things from the elders and chief priests and scribes, and be killed, and be raised on the third day.}%
\verse{And Peter took him aside and\lebnote{“\textit{and}” supplied because previous participle “took … aside” translated as a finite verb} began to rebuke him, saying, God forbid,\lebnote{“merciful to you”} Lord! This will never happen to you!”}%
\verse{But he turned around and\lebnote{“\textit{and}” supplied because previous participle “turned around” translated as a finite verb} said to Peter, \JesusWords{“Get behind me, Satan! You are a cause for stumbling to me, because you are not intent on the things of God, but the things of people!”}}%
\verseWithHeading{Taking Up One’s Cross to Follow Jesus}{Then Jesus said to his disciples, \JesusWords{“If anyone wants to come after me, let him deny himself and take up his cross and follow me.}}%
\verse{\JesusWords{For whoever wants to save his life will lose it, but whoever loses his life on account of me will find it.}}%
\verse{\JesusWords{For what will a person be benefited if he gains the whole world but forfeits his life? Or what will a person give in exchange for his life?}}%
\verse{\JesusWords{For the Son of Man is going to come in the glory of his Father with his angels, and at that time he will reward each one according to what he has done.}\lebnote{“his activity”}}%
\verse{\JesusWords{Truly I say to you, that there are some of those standing here who will never experience death until they see the Son of Man coming in his kingdom.”}}%
\end{biblechapter}%
\begin{biblechapter}% Matthew 17
\verseWithHeading{The Transfiguration}{And after six days Jesus took along Peter and James and John his brother, and led them up on a high mountain by themselves.}%
\verse{And he was transfigured before them, and his face shone like the sun, and his clothing became bright as the light.}%
\verse{And behold, Moses and Elijah appeared to them, talking with him.}%
\verse{So Peter answered and\lnCWT{} said to Jesus, “Lord, it is good that we are here! If you want, I will make here three shelters, one for you and one for Moses and one for Elijah.”}%
\verse{While\lebnote{“\textit{while}” is supplied as a component of the temporal genitive absolute participle (“speaking”)} he was still speaking, behold, a bright cloud overshadowed them, and behold, a voice from the cloud said, “This is my beloved Son, with whom I am well pleased. Listen to him!”}%
\verse{And when\lebnote{“\textit{when}” is supplied as a component of the participle (“heard”) which is understood as temporal} the disciples heard this,\lebnote{*supplied from English context} they fell down on their faces and were extremely frightened.}%
\verse{And Jesus came and touched them and\lebnote{“\textit{and}” supplied because previous participle “touched” translated as a finite verb} said, \JesusWords{“Get up and do not be afraid.”}}%
\verse{And when they\lebnote{“\textit{when}” is supplied as a component of the participle (“lifted up”) which is understood as temporal} lifted up their eyes they saw no one except him — Jesus alone.}%
\verse{And as\lebnote{“\textit{as}” is supplied as a component of the temporal genitive absolute participle (“were coming down”)} they were coming down from the mountain, Jesus commanded them saying, \JesusWords{“Tell no one the vision until the Son of Man is raised from the dead.”}}%
\verse{And the disciples asked him, saying, “Then why do the scribes say that Elijah must come first?”}%
\verse{And he answered and\lnCWT{} said, \JesusWords{“Elijah indeed is coming, and will restore all things.}}%
\verse{\JesusWords{But I say to you that Elijah has already come, and they did not recognize him, but did with him whatever they wanted. In the same way also the Son of Man is going to suffer at their hands.”}}%
\verse{Then the disciples understood that he had spoken to them about John the Baptist.}%
\verseWithHeading{A Demon-possessed Boy Healed}{And when they\lebnote{“\textit{when}” is supplied as a component of the temporal genitive absolute participle (“came”)} came to the crowd, a man approached him, kneeling down before him}%
\verse{and saying, “Lord, have mercy on my son, because he has seizures and suffers severely, for often he falls into the fire and often into the water.}%
\verse{And I brought him to your disciples, and they were not able to heal him.”}%
\verse{And Jesus answered and\lnCWT{} said, \JesusWords{“O unbelieving and perverse generation! How long\lnCWU{} will I be with you? How long\lnCWU{} must I put up with you? Bring him here to me!”}}%
\verse{And Jesus rebuked him, and the demon came out of him and the boy was healed from that hour.}%
\verse{Then the disciples approached Jesus privately and\lebnote{“\textit{and}” supplied because previous participle “approached” translated as a finite verb} said, Why\lebnote{“for what”} were we not able to expel it?”}%
\verse{And he said to them, \JesusWords{“Because of your little faith. For truly I say to you, if you have faith like a mustard seed, you will say to this mountain, ‘Move from here to there,’ and it will move, and nothing will be impossible for you.”}\lebnote{Most later Greek manuscripts add v. 21, “But this kind does not go out except by prayer and fasting.”}}%
\verseWithHeading{Jesus Predicts His Death and Resurrection a Second Time}{}%
\verse{Now as\lebnote{“\textit{as}” is supplied as a component of the temporal genitive absolute participle (“were gathering”)} they were gathering in Galilee, Jesus said to them, \JesusWords{“The Son of Man is going to be betrayed into the hands of men,}\lebnote{Or possibly “of people,” in a generic sense, although the reference here may be to the specific group responsible for Jesus’ arrest, where it is unlikely women were present}}%
\verse{\JesusWords{and they will kill him, and on the third day he will be raised.”} And they were extremely distressed.}%
\verseWithHeading{Paying the Double Drachma Temple Tax}{Now when\lebnote{“\textit{when}” is supplied as a component of the temporal genitive absolute participle (“arrived”)} they arrived in Capernaum, the ones who collected the double drachma tax\lebnote{This refers to the half-shekel annual tax paid by Jewish males to support the temple; over time the name of the coin commonly used to pay the tax came to be used for the tax itself} came up to Peter and said, “Does your teacher not pay the double drachma tax?”}%
\verse{He said, \JesusWords{“Yes.”} And when he\lebnote{“\textit{when}” is supplied as a component of the participle (“came”) which is understood as temporal} came into the house, Jesus spoke to him first, saying, \JesusWords{“What do you think, Simon? From whom do the kings of the earth collect tolls or taxes — from their own sons, or from foreigners?”}}%
\verse{And when he\lebnote{“\textit{when}” is supplied as a component of the participle (“said”) which is understood as temporal} said, “From foreigners,” Jesus said to him, \JesusWords{“Then the sons are free.}}%
\verse{\JesusWords{But so that we do not give offense to them, go out to the sea, cast a line with\lebnote{The words “\textit{a line with}” are not in the Greek text but are implied} a hook, and take the first fish that comes up. And when you\lebnote{“\textit{when}” is supplied as a component of the participle (“open”) which is understood as temporal} open its mouth, you will find a four-drachma coin. Take that and\lebnote{“\textit{and}” supplied because previous participle “take” translated as a finite verb} give it\lebnote{supplied from English context} to them for me and you.”}}%
\end{biblechapter}%
\begin{biblechapter}% Matthew 18
\verseWithHeading{The Question About Who Is Greatest}{At that time the disciples came up to Jesus, saying, “Who then is the greatest in the kingdom of heaven?”}%
\verse{And calling a child to himself, he had him stand in their midst}%
\verse{and said, \JesusWords{“Truly I say to you, unless you turn around and become like young children, you will never enter into the kingdom of heaven!}}%
\verse{\JesusWords{Therefore whoever humbles himself like this child, this person is the greatest in the kingdom of heaven,}}%
\verse{\JesusWords{and whoever welcomes one child such as this in my name welcomes me.}}%
\verse{\JesusWords{But whoever causes one of these little ones who believe in me to sin, it would be better for him that a large millstone\lebnote{“a millstone of a donkey”} be hung on\lebnote{Some manuscripts have “around”} his neck and he be drowned in the depths of the sea.}}%
\verse{\JesusWords{Woe to the world because of causes for stumbling, for it is a necessity that causes for stumbling come; nevertheless, woe to the person through whom the cause for stumbling comes.}}%
\verse{\JesusWords{And if your hand or your foot causes you to sin, cut it off and throw it\lnCWV{} from you! It is better for you to enter into life crippled or lame than, having two hands or two feet, to be thrown into the eternal fire!}}%
\verse{\JesusWords{And if your eye causes you to sin, tear it out and throw it\lnCWV{} from you! It is better for you to enter into life one-eyed than, having two eyes, to be thrown into fiery hell!}}%
\verseWithHeading{The Parable of the Lost Sheep}{\JesusWords{“See to it that you do not despise one of these little ones, for I tell you that their angels in heaven constantly see the face of my Father who is in heaven.}\lebnote{Many later Greek manuscripts include vs. 11, “For the Son of Man came to save those who are lost.”}}%
\verse{\JesusWords{}}%
\verse{\JesusWords{What do you think? If a certain man has\lebnote{“there are to a certain man”} a hundred sheep, and one of them wanders away, will he not leave the ninety-nine on the hills and go and\lebnote{“\textit{and}” supplied because previous participle “go” translated as a finite verb} look for the one that wandered away?}}%
\verse{\JesusWords{And if he happens to find it, truly I say to you that he rejoices over it more than over the ninety-nine that did not wander away.}}%
\verse{\JesusWords{In the same way it is not the will of\lebnote{“in the presence of”} your Father who is in heaven that one of these little ones perish.}}%
\verseWithHeading{Confronting a Brother Who Sins Against You}{\JesusWords{“Now if your brother sins against you, go correct him between you and him alone. If he listens to you, you have gained your brother.}}%
\verse{\JesusWords{But if he does not listen, take with you in addition one or two others, so that by the testimony\lebnote{“the mouth”} of two or three witnesses every matter may be established.}}%
\verse{\JesusWords{And if he refuses to listen to them, tell it\lnCWV{} to the church. But if he refuses to listen to the church also, let him be to you as a Gentile and a tax collector.}}%
\verse{\JesusWords{“Truly I say to you, whatever you bind on earth will be bound in heaven, and whatever you release on earth will be released in heaven.}}%
\verse{\JesusWords{Again, truly I say to you that if two of you agree on earth about any matter that they ask, it will be done for them from my Father who is in heaven.}}%
\verse{\JesusWords{For where two or three are gathered in my name, I am there in the midst of them.”}}%
\verseWithHeading{The Parable of the Unforgiving Slave}{Then Peter came up to him and\lebnote{“\textit{and}” supplied because previous participle “came up” translated as a finite verb} said,\lebnote{Some manuscripts have “Then Peter came up \textit{and} said to him”} “Lord, how many times will my brother sin against me and I will forgive him? Up to seven times?”}%
\verse{Jesus said to him, \JesusWords{“I do not say to you up to seven times, but up to seventy times seven!\lebnote{Or “seventy-seven times”}}}%
\verse{\JesusWords{“For this reason the kingdom of heaven may be compared to a man — a king — who wanted to settle accounts with his slaves.}}%
\verse{\JesusWords{And when\lebnote{“\textit{when}” is supplied as a component of the temporal genitive absolute participle (“began”)} he began to settle them,\lnCWW{} someone was brought to him who owed ten thousand talents.}}%
\verse{\JesusWords{And because\lebnote{“\textit{because}” is supplied as a component of the participle (“have”) which is understood as causal} he did not have enough\lnCWV{} to repay it,\lnCWW{} the master ordered him to be sold, and his\lnCWX{} wife and his\lnCWX{} children and everything that he had, and to be repaid.}}%
\verse{\JesusWords{Then the slave threw himself to the ground and\lnCWY{} began to do obeisance to him, saying, ‘Be patient with me, and I will pay back everything to you!’}}%
\verse{\JesusWords{So the master of that slave, because he\lebnote{“\textit{because}” is supplied as a component of the participle (“had compassion”) which is understood as causal} had compassion, released him and forgave him the loan.}}%
\verse{\JesusWords{But that slave went out and\lebnote{“\textit{and}” supplied because previous participle “went out” translated as a finite verb} found one of his fellow slaves who owed him a hundred denarii, and taking hold of him, he began to choke him,\lnCWW{} saying, ‘Pay back everything that you owe!’}}%
\verse{\JesusWords{Then his fellow slave threw himself to the ground and\lnCWY{} began to implore\lebnote{Imperfect tense as ingressive (“began to implore”)} him, saying, ‘Be patient with me and I will repay you!’}}%
\verse{\JesusWords{But he did not want to, but rather he went and\lnCWZ{} threw him into prison until he would repay what was owed.}}%
\verse{\JesusWords{So when\lebnote{“\textit{when}” is supplied as a component of the participle (“saw”) which is understood as temporal} his fellow slaves saw what had happened, they were extremely distressed, and went and\lnCWZ{} reported to their master everything that had happened.}}%
\verse{\JesusWords{Then his master summoned him and\lebnote{“\textit{and}” supplied because previous participle “summoned” translated as a finite verb} said to him, ‘Wicked slave! I forgave you all that debt because you implored me!}}%
\verse{\JesusWords{Should you not also have shown mercy to your fellow slave as I also showed mercy to you?’}}%
\verse{\JesusWords{And because he\lebnote{“\textit{because}” is supplied as a component of the participle (“was angry”) which is understood as causal} was angry, his master handed him over to the merciless jailers\lebnote{Or “torturers”} until he would repay everything that was owed.}}%
\verse{\JesusWords{So also my heavenly Father will do to you, unless each of you forgives his brother from your hearts!”}}%
\end{biblechapter}%
\begin{biblechapter}% Matthew 19
\verseWithHeading{On Divorce}{And it happened that when Jesus had finished these statements, he went away from Galilee and came into the region of Judea on the other side of the Jordan.}%
\verse{And large crowds followed him, and he healed them there.}%
\verse{And Pharisees came up to him in order to\lebnote{“\textit{in order to}” is supplied as a component of the participle (“test”) which is understood as purpose} test him, and asked\lebnote{the participle (“asked”) is translated as a finite verb because of English style} if it was permitted for a man to divorce his wife for any cause.}%
\verse{And he answered and\lnCXA{} said, \JesusWords{“Have you not read that the one who created them\lnCXB{} from the beginning made them male and female}}%
\verse{\JesusWords{and said, ‘On account of this a man will leave his\lnCXC{} father and his\lnCXC{} mother and will be joined to his wife, and the two will become one flesh’?}\lebnote{from Gen 2:24}}%
\verse{\JesusWords{So then, they are no longer two but one flesh. Therefore what God has joined together, man must not separate.”}}%
\verse{They said to him, “Why then did Moses command us\lnCXB{} to give a document — a certificate of divorce — and to divorce her?”}%
\verse{He said to them, \JesusWords{“Moses, with reference to your hardness of heart, permitted you to divorce your wives, but from the beginning it was not like this.}}%
\verse{\JesusWords{Now I say to you that whoever divorces his wife, except on the basis of sexual immorality, and marries another commits adultery, and whoever marries her who is divorced commits adultery.”}\lebnote{Some manuscripts omit “and whoever marries her who is divorced commits adultery”}}%
\verse{The disciples\lebnote{Some manuscripts have “His disciples”} said to him, “If this is the case of a man with his\lnCXC{} wife, it would be better not to marry!”}%
\verse{But he said to them, \JesusWords{“Not everyone can accept this saying but those to whom it has been given.}}%
\verse{\JesusWords{For there are eunuchs who were born as such from their mother’s womb, and there are eunuchs who were made eunuchs by people, and there are eunuchs who have made themselves eunuchs\lebnote{An understood repetition of the term from earlier in the verse} for the sake of the kingdom of heaven. The one who is able to accept this,\lnCXD{} let him accept it.”}\lnCXD{}}%
\verseWithHeading{Little Children Brought to Jesus}{Then children were brought to him so that he could lay his\lnCXC{} hands on them and pray, but the disciples rebuked them.}%
\verse{But Jesus said, \JesusWords{“Allow the children, and do not forbid them to come to me, for to such belongs\lebnote{“for of such is”} the kingdom of heaven.”}}%
\verse{And he laid his\lnCXC{} hands on them and\lebnote{“\textit{and}” supplied because previous participle “laid” translated as a finite verb} traveled on from there.}%
\verseWithHeading{A Rich Young Man}{And behold, someone came up to him and\lebnote{“\textit{and}” supplied because previous participle “came up” translated as a finite verb} said, “Teacher, what good thing must I do so that I will have eternal life?”}%
\verse{And he said to him, \JesusWords{“Why are you asking me about what is good? There is one who is good. But if you want to enter into life, keep the commandments!”}}%
\verse{He said to him, “Which ones?” And Jesus said, \JesusWords{“Do not commit murder, do not commit adultery, do not steal, do not give false testimony,}}%
\verse{honor your\lnCXC{} father and your\lnCXC{} mother, and love your neighbor as yourself.”}%
\verse{The young man said to him, “All these I have observed. What do I still lack?”}%
\verse{Jesus said to him, \JesusWords{“If you want to be perfect, go, sell your possessions and give the proceeds\lnCXB{} to the poor — and you will have treasure in heaven — and come, follow me.”}}%
\verse{But when\lnCXE{} the young man heard the statement, he went away sorrowful, because he was one who had many possessions.}%
\verse{And Jesus said to his disciples, \JesusWords{“Truly I say to you that with difficulty a rich person will enter into the kingdom of heaven!}}%
\verse{\JesusWords{And again I say to you, it is easier for a camel to go through the eye of a needle than a rich person into\lebnote{Some manuscripts have “to enter into”} the kingdom of God.”}}%
\verse{So when\lnCXE{} the disciples heard this,\lnCXD{} they were extremely amazed, saying, “Then who can be saved?”}%
\verse{But Jesus looked at them\lnCXB{} and\lebnote{“\textit{and}” supplied because previous participle “looked at” translated as a finite verb} said to them, \JesusWords{“With human beings this is impossible, but with God all things are possible.”}}%
\verse{Then Peter answered and\lnCXA{} said to him, “Behold, we have left everything and followed you. What then will there be for us?”}%
\verse{And Jesus said to them, \JesusWords{“Truly I say to you that in the renewal of the world,\lebnote{The words “of the world” are supplied as a clarification of “renewal”} when the Son of Man sits on his glorious throne, you who have followed me — you also will sit on twelve thrones judging the twelve tribes of Israel.}}%
\verse{\JesusWords{And everyone who has left houses or brothers or sisters or father or mother or wife\lebnote{Some manuscripts omit “or wife”} or children or fields on account of my name will receive a hundred times as much, and will inherit eternal life.}}%
\verse{\JesusWords{But many who are first will be last, and the last first.}}%
\end{biblechapter}%
\begin{biblechapter}% Matthew 20
\verseWithHeading{The Parable of the Workers in the Vineyard}{\JesusWords{“For the kingdom of heaven is like a man — the master of the house — who went out early in the morning to hire workers for his vineyard.}}%
\verse{\JesusWords{And after\lebnote{“\textit{after}” is supplied as a component of the participle (“coming to an agreement”) which is understood as temporal} coming to an agreement with the workers for a denarius per day, he sent them into his vineyard.}}%
\verse{\JesusWords{And going out about the third hour, he saw others standing idle in the marketplace.}}%
\verse{\JesusWords{And to those people he said, ‘You also go into the vineyard, and I will give you whatever is right.’}}%
\verse{\JesusWords{So they went. Going out\lebnote{Some manuscripts have “And going out”} again about the sixth and ninth hour he did the same thing.}}%
\verse{\JesusWords{And about the eleventh hour he went out and\lebnote{“\textit{and}” supplied because previous participle “went out” translated as a finite verb} found others standing there and said to them, ‘Why are you standing here the whole day unemployed?’}}%
\verse{\JesusWords{They said to him, ‘Because no one hired us.’ He said to them, ‘You go also into the vineyard.’}}%
\verse{\JesusWords{And when\lebnote{“\textit{when}” is supplied as a component of the temporal genitive absolute participle (“came”)} evening came, the owner of the vineyard said to his manager, ‘Call the workers and pay them their\lebnote{“the”: the Greek article is used here as a possessive pronoun} wages, beginning from the last up to the first.’}}%
\verse{\JesusWords{And when the ones hired about the eleventh hour came, they received a denarius apiece.}}%
\verse{\JesusWords{And when\lebnote{“\textit{when}” is supplied as a component of the participle (“came”) which is understood as temporal} the first came, they thought that they would receive more, and they also received a denarius apiece.}}%
\verse{\JesusWords{And when they\lebnote{“\textit{when}” is supplied as a component of the participle (“received”) which is understood as temporal} received it,\lnCXF{} they began to complain\lebnote{Imperfect tense as ingressive (“began to complain”)} against the master of the house,}}%
\verse{\JesusWords{saying, ‘These last people worked one hour and you made them equal to us who have endured the burden of the day and the burning heat!’}}%
\verse{\JesusWords{But he answered one of them and\lnCXG{} said, ‘Friend, I am not doing you wrong. Did you not come to an agreement with me for a denarius?}}%
\verse{\JesusWords{Take what is yours and go! But I want to give to this last person the same as I gave\lebnote{The words “\textit{I gave}” are an implied repetition from the verb earlier in the verse} to you also.}}%
\verse{\JesusWords{Is it not\lebnote{Some manuscripts have “Or \textit{is it} not”} permitted for me to do whatever I want with what is mine? Or is your eye evil because I am generous?’}}%
\verse{\JesusWords{Thus the last will be first and the first last.”}}%
\verseWithHeading{Jesus Predicts His Death and Resurrection a Third Time}{And as\lebnote{“\textit{as}” is supplied as a component of the participle (“was going up”) which is understood as temporal} Jesus was going up to Jerusalem, he took the twelve disciples by themselves and said to them on the way,}%
\verse{\JesusWords{“Behold, we are going up to Jerusalem, and the Son of Man will be handed over to the chief priests and scribes, and they will condemn him to death,}}%
\verse{\JesusWords{and will hand him over to the Gentiles to mock him\lnCXH{} and flog him\lnCXH{} and crucify him,\lnCXF{} and on the third day he will be raised.”}}%
\verseWithHeading{A Request by the Mother of James and John}{Then the mother of the sons of Zebedee came up to him with her sons, and\lebnote{“\textit{and}” supplied because participle “kneeling down” translated as a finite verb in keeping with English style} kneeling down she asked\lebnote{the participle (“asked”) is translated as a finite verb because of English style} something from him.}%
\verse{And he said to her, \JesusWords{“What do you want?”} She said to him, “Say that these two sons of mine may sit one at your right hand and one at your left in your kingdom.”}%
\verse{But Jesus answered and\lnCXG{} said, \JesusWords{“You do not know what you are asking! Are you able to drink the cup that I am about to drink?”} They said to him, “We are able.”}%
\verse{He said to them, \JesusWords{“You will indeed drink my cup, but to sit at my right hand and at my left is not mine\lebnote{Some manuscripts have “this is not mine”} to grant, but is for those for whom it has been prepared by my Father.”}}%
\verse{And when\lnCXI{} the ten heard this,\lnCXF{} they were indignant concerning the two brothers.}%
\verse{But Jesus called them to himself and\lebnote{“\textit{and}” supplied because previous participle “called … to himself” translated as a finite verb} said, \JesusWords{“You know that the rulers of the Gentiles lord it over them, and those in high positions exercise authority over them.}}%
\verse{\JesusWords{It will not be like this among you! But whoever wants to become great among you must be your servant,}}%
\verse{\JesusWords{and whoever wants to be most prominent among you must be your slave —}}%
\verse{\JesusWords{just as the Son of Man did not come to be served, but to serve, and to give his life as a ransom for many.”}}%
\verseWithHeading{Two Blind Men Healed at Jericho}{And as\lebnote{“\textit{as}” is supplied as a component of the temporal genitive absolute participle (“were going out”)} they were going out of Jericho, a large crowd followed him.}%
\verse{And behold, there were two blind men sitting beside the road. When they\lnCXI{} heard that Jesus was passing by, they called out, saying, “Lord, have mercy on us,\lnCXJ{} Son of David!”}%
\verse{And the crowd rebuked them so that they would be quiet. But they called out all the more, saying, “Lord, have mercy on us,\lnCXJ{} Son of David!”}%
\verse{And Jesus stopped,\lebnote{*Here the participle (“stopped”) is translated as a finite verb because of English style} called them, and said, \JesusWords{“What do you want me to do for you?”}}%
\verse{They said to him, “Lord, that our eyes be opened!”}%
\verse{And having compassion, Jesus touched their eyes, and immediately they received their sight and followed him.}%
\end{biblechapter}%
\begin{biblechapter}% Matthew 21
\verseWithHeading{The Triumphal Entry}{And when they drew near to Jerusalem and came to Bethphage at the Mount of Olives, then Jesus sent two disciples,}%
\verse{saying to them, \JesusWords{“Go into the village before you, and right away you will find a donkey tied and a colt with her. Untie them\lnCXK{} and\lebnote{“\textit{and}” supplied because previous participle “untie” translated as a finite verb} bring them\lnCXK{} to me.}}%
\verse{\JesusWords{And if anyone says anything to you, you will say, ‘The Lord needs them,’\lebnote{“has need of them”} and he will send them at once.”}}%
\verse{Now this took place so that what was spoken through the prophet would be fulfilled, saying,}%
\verse{“Say to the daughter of Zion, ‘Behold, your king is coming to you, humble and mounted on a donkey, and\lebnote{Or “even”} on a colt, the foal of a pack animal.’”\lebnote{from Zech 9:9}}%
\verse{So the disciples went\lebnote{the participle (“went”) is translated as a finite verb because of English style} and did\lebnote{the participle (“did”) is translated as a finite verb because of English style} just as Jesus directed them,}%
\verse{and\lebnote{“\textit{and}” is supplied because the previous participles (“went” and “did” in the previous verse) have been translated as finite verbs} brought the donkey and the colt and put their\lnCXL{} cloaks on them, and he sat on them.}%
\verse{And a very large crowd spread their cloaks on the road, and others were cutting branches from the trees and spreading them\lnCXK{} on the road.}%
\verse{And the crowds who went ahead of him and the ones who followed were shouting, saying, “Hosanna to the Son of David! Blessed is the one who comes in the name of the Lord!\lebnote{from Ps 118:25–26} Hosanna in the highest heaven!”\lebnote{*Here “heaven” is understood}}%
\verse{And when\lebnote{“\textit{when}” is supplied as a component of the temporal genitive absolute participle (“entered”)} he entered into Jerusalem, the whole city was stirred up, saying, “Who is this?”}%
\verse{And the crowds were saying, “This is the prophet Jesus from Nazareth of Galilee!”}%
\verseWithHeading{The Cleansing of the Temple}{And Jesus entered the temple courts\lnCXM{} and drove out all those who were selling and buying in the temple, and overturned the tables of the money changers and the chairs of those who were selling doves.}%
\verse{And he said to them, \JesusWords{“It is written, ‘My house will be called a house of prayer,’\lebnote{from Isa 56:7} but you have made it a cave of robbers!”}}%
\verse{And the blind and the lame came up to him in the temple courts\lnCXM{} and he healed them.}%
\verse{But when\lnCXN{} the chief priests and the scribes saw the wonderful things that he did, and the children shouting in the temple courts\lnCXM{} and saying, “Hosanna to the Son of David!” they were indignant.}%
\verse{And they said to him, \JesusWords{“Do you hear what these children\lebnote{The word “\textit{children}” is not in the Greek text but is implied} are saying?”} So Jesus said to them, \JesusWords{“Yes, have you never read, ‘Out of the mouths of children and nursing babies you have prepared for yourself praise’?”}\lebnote{from Ps 8:2}}%
\verse{And leaving them, he went outside of the city to Bethany and spent the night there.}%
\verseWithHeading{A Barren Fig Tree Cursed}{Now early in the morning, as he\lebnote{“\textit{as}” is supplied as a component of the participle (“was returning”) which is understood as temporal} was returning to the city, he was hungry.}%
\verse{And seeing a single fig tree by the road, he went to it and found nothing on it except leaves only. And he said to it, \JesusWords{“May there be no more fruit from you forever,\lebnote{“to the age”} and the fig tree withered at once.}}%
\verse{And when they\lnCXN{} saw it,\lnCXO{} the disciples were astonished, saying, “How did the fig tree wither at once?”}%
\verse{And Jesus answered and\lnCXP{} said to them, \JesusWords{“Truly I say to you, if you have faith and do not doubt, you will do not only what was done to the fig tree, but even if you say to this mountain, ‘Be lifted up and thrown into the sea,”} it will happen!}%
\verse{\JesusWords{And whatever you ask in prayer, if you\lebnote{“\textit{if}” is supplied as a component of the participle (“believe”) which is understood as conditional} believe, you will receive.”}}%
\verseWithHeading{Jesus’ Authority Challenged}{And after\lebnote{“\textit{after}” is supplied as a component of the temporal genitive absolute participle (“arrived”)} he arrived at the temple, the chief priests and the elders of the people came up to him while he\lebnote{“\textit{while}” is supplied as a component of the participle (“was teaching”) which is understood as temporal} was teaching, saying, “By what authority are you doing these things? And who gave you this authority?”}%
\verse{And Jesus answered and\lnCXP{} said to them, \JesusWords{“I also will ask you one question. If you tell the answer\lnCXK{} to me, I also will tell you by what authority I am doing these things.}}%
\verse{\JesusWords{From where was the baptism of John — from heaven or from men?”} And they began to discuss\lebnote{Imperfect tense as ingressive (“began to discuss”)} this\lnCXK{} among themselves, saying, “If we say ‘From heaven,’ he will say to us, ‘Why then did you not believe him?’}%
\verse{But if we say, ‘From men,’ we are afraid of the crowd, because they all look upon John as a prophet.”}%
\verse{And they answered and\lnCXP{} said to Jesus, “We do not know.” And he said to them, \JesusWords{“Neither will I tell you by what authority I am doing these things.}}%
\verseWithHeading{The Parable of the Two Sons}{\JesusWords{“Now what do you think? A man had two sons. He approached\lebnote{Some manuscripts have “And he approached”} the first and\lnCXQ{} said, ‘Son, go work in the vineyard today.’}}%
\verse{\JesusWords{And he answered and\lnCXP{} said, ‘I do not want to!’ But later he changed his mind and\lebnote{“\textit{and}” supplied because previous participle “changed his mind” translated as a finite verb} went.}}%
\verse{\JesusWords{And he approached the second\lebnote{Some manuscripts have “the other”} and\lnCXQ{} said the same thing. So he answered and\lnCXP{} said, ‘I will, sir,’ and he did not go.}}%
\verse{\JesusWords{Which of the two did the will of his\lnCXL{} father?”} They said, “The first.” Jesus said to them, \JesusWords{“Truly I say to you that the tax collectors and the prostitutes are going ahead of you into the kingdom of God!}}%
\verse{\JesusWords{For John came to you in the way of righteousness and you did not believe him, but the tax collectors and the prostitutes did believe him. And when\lnCXN{} you saw it,\lnCXO{} you did not even change your minds later so as to believe in him.}}%
\verseWithHeading{The Parable of the Tenant Farmers in the Vineyard}{\JesusWords{“Listen to another parable: There was a man — a master of a house — who planted a vineyard, and put a fence around it, and dug a winepress in it, and built a watchtower, and leased it to tenant farmers, and went on a journey.}}%
\verse{\JesusWords{And when the season of fruit drew near, he sent his slaves to the tenant farmers to collect his fruit.}}%
\verse{\JesusWords{And the tenant farmers seized his slaves, one of whom they beat, and one of whom they killed, and one of whom they stoned.}}%
\verse{\JesusWords{Again, he sent other slaves, more than the first ones, and they did the same thing to them.}}%
\verse{\JesusWords{So finally he sent his son to them, saying, ‘They will respect my son.’}}%
\verse{\JesusWords{But when\lnCXN{} the tenant farmers saw the son, they said among themselves, ‘This is the heir. Come, let us kill him and have his inheritance!’}}%
\verse{\JesusWords{And they seized him and\lebnote{“\textit{and}” supplied because previous participle “seized” translated as a finite verb} threw him\lnCXK{} out of the vineyard and killed him.}\lnCXO{}}%
\verse{\JesusWords{Now when the master of the vineyard arrives, what will he do to those tenant farmers?”}}%
\verse{They said to him, “He will destroy those evil men completely and lease the vineyard to other tenant farmers who will give him the fruits in their season.”}%
\verse{Jesus said to them, \JesusWords{“Have you never read in the scriptures, ‘The stone which the builders rejected, this has become the cornerstone.\lebnote{“the head of the corner”} This came about from the Lord, and it is marvelous in our eyes’?}\lebnote{from Ps 118:22–23}}%
\verse{\JesusWords{For this reason, I tell you that the kingdom of God will be taken away from you and will be given to a people\lebnote{Or “nation”} who produce its fruits.}}%
\verse{\JesusWords{And the one who falls on this stone will be broken to pieces, and the one on whom it falls — it will crush him!”}}%
\verse{And when\lebnote{“\textit{when}” is supplied as a component of the participle (“heard”) which is understood as temporal} the chief priests and the Pharisees heard his parables, they knew that he was speaking about them,}%
\verse{and although they\lebnote{“\textit{although}” is supplied as a component of the participle (“wanted”) which is understood as concessive} wanted to arrest him, they were afraid of the crowds, because they looked upon him as a prophet.}%
\end{biblechapter}%
\begin{biblechapter}% Matthew 22
\verseWithHeading{The Parable of the Wedding Celebration}{And continuing, Jesus spoke to them again in parables, saying,}%
\verse{\JesusWords{“The kingdom of heaven may be compared to a man — a king — who gave a wedding celebration for his son.}}%
\verse{\JesusWords{And he sent his slaves to summon those who had been invited to the wedding celebration, and they did not want to come.}}%
\verse{\JesusWords{Again he sent other slaves, saying, ‘Tell those who have been invited, “Behold, I have prepared my dinner; my oxen and fattened cattle have been slaughtered, and everything is ready. Come to the wedding celebration!”’}}%
\verse{\JesusWords{But they paid no attention and\lebnote{“\textit{and}” supplied because previous participle “paid no attention” translated as a finite verb} went away — this one to his own field, that one to his business.}}%
\verse{\JesusWords{And the others, seizing his slaves, mistreated them\lnCXR{} and killed them.}\lnCXS{}}%
\verse{\JesusWords{And the king was angry and sent his troops and\lebnote{“\textit{and}” supplied because previous participle “sent” translated as a finite verb} destroyed those murderers and burned their city.}}%
\verse{\JesusWords{Then he said to his slaves, ‘The wedding celebration is ready, but those who had been invited were not worthy.}}%
\verse{\JesusWords{Therefore, go out to the places where the roads exit the city and invite to the wedding celebration as many people as you find.’}}%
\verse{\JesusWords{And those slaves went out into the roads and\lebnote{“\textit{and}” supplied because previous participle “went out” translated as a finite verb} gathered everyone whom they found, both evil and good, and the wedding celebration was filled with dinner guests.}\lebnote{“with those reclining at table”}}%
\verse{\JesusWords{But when\lebnote{“\textit{when}” is supplied as a component of the participle (“came in”) which is understood as temporal} the king came in to see the dinner guests,\lebnote{“ones reclining at table”} he saw a man there not dressed in wedding clothes.}}%
\verse{\JesusWords{And he said to him, ‘Friend, how did you come in here, not having wedding clothes?’ But he could say nothing.}\lebnote{“he was silent”}}%
\verse{\JesusWords{Then the king said to the servants, ‘Tie him up hand and foot\lebnote{“feet and hands”} and\lebnote{“\textit{and}” supplied because previous participle “tie” translated as a finite verb} throw him into the outer darkness. In that place there will be weeping and gnashing of teeth!’}}%
\verse{\JesusWords{For many are called but few are chosen.”}}%
\verseWithHeading{Paying Taxes to Caesar}{Then the Pharisees went and consulted\lebnote{“took counsel”} so that they could entrap him with a statement.}%
\verse{And they sent their disciples to him with the Herodians, saying, “Teacher, we know that you are truthful and teach the way of God in truth, and you do not care what anyone thinks,\lebnote{“it is not a care to you concerning anyone”} because you do not regard the opinion of people.\lebnote{“because you do not look at the face of men”}}%
\verse{Therefore tell us what you think. Is it permitted to pay taxes to Caesar or not?”}%
\verse{But because he\lebnote{“\textit{because}” is supplied as a component of the participle (“knew”) which is understood as causal} knew their maliciousness, Jesus said, \JesusWords{“Hypocrites! Why are you testing me?}}%
\verse{Show me the coin for the tax!” So they brought him a denarius.}%
\verse{And he said to them, \JesusWords{“Whose image and inscription is this?”}}%
\verse{They said to him, “Caesar’s.” Then he said to them, \JesusWords{“Therefore give to Caesar the things of Caesar, and to God the things of God!”}}%
\verse{And when they\lnCXT{} heard this,\lnCXS{} they were astonished, and they left him and\lebnote{“\textit{and}” supplied because previous participle “left” translated as a finite verb} went away.}%
\verseWithHeading{A Question About Marriage and the Resurrection}{On that day Sadducees — who say there is no resurrection — came up to him and asked him,}%
\verse{saying, “Teacher, Moses said if someone dies without having children, his brother is to marry his wife and father\lebnote{“raise up”} descendants for his brother.}%
\verse{Now there were seven brothers with us. And the first died after\lebnote{“\textit{after}” is supplied as a component of the participle (“getting married”) which is understood as temporal} getting married, and because he\lebnote{“\textit{because}” is supplied as a component of the participle (“have”) which is understood as causal} did not have descendants, he left his wife to his brother.}%
\verse{So also the second and the third, up to the seventh.}%
\verse{And last of all the woman died.}%
\verse{In the resurrection, therefore, whose wife of the seven will she be? For they all had her as wife.”\lebnote{*The words “\textit{as wife}” are not in the Greek text but are implied}}%
\verse{But Jesus answered and\lebnote{“\textit{and}” supplied because previous participle “answered” translated as a finite verb} said to them, \JesusWords{“You are mistaken, because\lebnote{“\textit{because}” is supplied as a component of the participle (“know”) which is understood as causal} you do not know the scriptures or the power of God!}}%
\verse{\JesusWords{For in the resurrection they neither marry nor are given in marriage, but are like angels of God\lebnote{Some manuscripts omit “of God”} in heaven.}}%
\verse{\JesusWords{Now concerning the resurrection of the dead, have you not read what was spoken to you by God, who said,}}%
\verse{\JesusWords{“I am the God of Abraham and the God of Isaac and the God of Jacob”}?\lebnote{from Exod 3:6} He is not the God of the dead, but of the living!”}%
\verse{And when\lnCXT{} the crowds heard this,\lnCXS{} they were amazed at his teaching.}%
\verseWithHeading{The Greatest Commandment}{Now when\lnCXT{} the Pharisees heard that he had silenced the Sadducees, they assembled at the same place.\lebnote{Or “they assembled together”}}%
\verse{And one of them, a legal expert, put a question to him\lnCXR{} to test him:}%
\verse{“Teacher, which commandment is greatest in the law?”}%
\verse{And he said to him, \JesusWords{“‘You shall love the Lord your God with all your heart and with all your soul and with all your mind.’}\lebnote{from Deut 6:5}}%
\verse{\JesusWords{This is the greatest and first commandment.}}%
\verse{\JesusWords{And the second is like it: ‘You shall love your neighbor as yourself.’}\lebnote{from Lev 19:18}}%
\verse{\JesusWords{On these two commandments depend all the law and the prophets.”}}%
\verseWithHeading{David’s Son and Lord}{Now while\lebnote{“\textit{while}” is supplied as a component of the temporal genitive absolute participle (“assembled”)} the Pharisees were assembled, Jesus asked them,}%
\verse{saying, \JesusWords{“What do you think about the Christ? Whose son is he?”} They said to him, “David’s.”}%
\verse{\JesusWords{He said to them, “How then does David, by the Spirit, call him ‘Lord,’ saying,}}%
\verse{\JesusWords{‘The Lord said to my Lord, “Sit at my right hand until I put your enemies under your feet”’?\lebnote{from Ps 110:1} }}%
\verse{\JesusWords{If then David calls him ‘Lord,’ how is he his son?”}}%
\verse{And no one was able to answer him a word, nor did anyone dare from that day on to ask him any more questions.\lebnote{*The word “\textit{questions}” is not in the Greek text but is implied}}%
\end{biblechapter}%
\begin{biblechapter}% Matthew 23
\verseWithHeading{Seven Woes Pronounced on the Scribes and Pharisees}{Then Jesus spoke to the crowds and to his disciples,}%
\verse{\JesusWords{saying, “The scribes and the Pharisees sit on the seat of Moses.}}%
\verse{\JesusWords{Therefore do and observe everything that they tell you, but do not do as they do,\lebnote{“their deeds”} for they tell others to do something\lebnote{The words “\textit{others to do something}” are not in the Greek text but are implied} and do not do it themselves.}\lebnote{*The words “\textit{it themselves}” are not in the Greek text but are implied}}%
\verse{\JesusWords{And they tie up heavy burdens\lebnote{Some manuscripts have “burdens that are heavy and hard to bear”} and put them\lnCXU{} on people’s shoulders, but they themselves are not willing with their finger to move them.}}%
\verse{\JesusWords{And they do all their deeds in order to be seen by people, for they make their phylacteries broad and make their\lnCXV{} tassels long.}}%
\verse{\JesusWords{And they love the place of honor at banquets and the best seats in the synagogues}}%
\verse{\JesusWords{and the greetings in the marketplaces and to be called ‘Rabbi’ by people.}}%
\verse{\JesusWords{But you are not to be called ‘Rabbi,’ because one is your teacher, and you are all brothers,}}%
\verse{\JesusWords{And do not call anyone\lnCXU{} your father on earth, for one is your heavenly Father.}}%
\verse{\JesusWords{And do not be called teachers, because one is your teacher, the Christ.}}%
\verse{\JesusWords{And the greatest among you will be your servant.}}%
\verse{\JesusWords{And whoever exalts himself will be humbled, and whoever humbles himself will be exalted.}}%
\verse{\JesusWords{“But woe to you, scribes and Pharisees — hypocrites! — because you shut the kingdom of heaven before people! For you do not enter, nor permit those wanting to go in\lebnote{the present tense has been translated as voluntative (“wanting to go in”)} to enter.\lebnote{The most important Greek manuscripts omit v. 14, “Woe to you, scribes and Pharisees—hypocrites!—because you devour widows’ houses and for show you pray long prayers! Therefore you will receive the greater condemnation.”}}}%
\verse{\JesusWords{}}%
\verse{\JesusWords{“Woe to you, scribes and Pharisees — hypocrites! — because you travel around the sea and the dry land to make one convert, and when he becomes one,\lebnote{*supplied from English context} you make him twice as much a son of hell as you are!}}%
\verse{\JesusWords{“Woe to you, blind guides, who say, ‘Whoever swears by the temple, it is nothing! But whoever swears by the gold of the temple is bound by his oath.’}\lnCXW{}}%
\verse{\JesusWords{Fools and blind people! For which is greater, the gold or the temple that makes the gold holy?}}%
\verse{\JesusWords{And, ‘Whoever swears by the altar, it is nothing! But whoever swears by the gift that is on it is bound by his oath.’}\lnCXW{}}%
\verse{\JesusWords{Blind people! For which is greater, the gift or the altar that makes the gift holy?}}%
\verse{\JesusWords{Therefore the one who swears by the altar swears by it and by everything that is on it.}}%
\verse{\JesusWords{And the one who swears by the temple swears by it and by the one who dwells in it.}}%
\verse{\JesusWords{And the one who swears by heaven swears by the throne of God and by the one who sits on it.}}%
\verse{\JesusWords{“Woe to you, scribes and Pharisees — hypocrites! — because you pay a tenth of mint and dill and cumin, and neglect the more important matters of the law — justice and mercy and faithfulness! It was necessary\lebnote{Some manuscripts have “But it was necessary”} to do these things while not neglecting those.\lebnote{“and those not to neglect”}}}%
\verse{\JesusWords{Blind guides who filter out a gnat and swallow a camel!}}%
\verse{\JesusWords{“Woe to you, scribes and Pharisees — hypocrites! — because you cleanse the outside of the cup and the dish, but inside they are full of greed and self-indulgence!}}%
\verse{\JesusWords{Blind Pharisee! First clean the inside of the cup and the dish,\lebnote{Some manuscripts omit “and the dish”} so that the outside of it may become clean also.}}%
\verse{\JesusWords{“Woe to you, scribes and Pharisees — hypocrites! — because you are like whitewashed tombs which on the outside appear beautiful, but on the inside are full of the bones of the dead and of everything unclean!}}%
\verse{\JesusWords{In the same way, on the outside you also appear righteous to people, but inside you are full of hypocrisy and lawlessness.}}%
\verse{\JesusWords{“Woe to you, scribes and Pharisees — hypocrites! — because you build the tombs of the prophets and decorate the graves of the righteous,}}%
\verse{\JesusWords{and you say, ‘If we had lived in the days of our fathers, we would not have been partners with them in the blood of the prophets!’}}%
\verse{\JesusWords{Thus you testify against yourselves that you are descendants of those who murdered the prophets!}}%
\verse{\JesusWords{And you — fill up the measure of your fathers!}}%
\verse{\JesusWords{Serpents! Offspring of vipers! How will you escape from the condemnation to hell?}}%
\verse{\JesusWords{For this reason, behold, I am sending to you prophets and wise men and scribes. Some of them you will kill and crucify, and some of them you will flog in your synagogues and will pursue from town to town,}}%
\verse{\JesusWords{so that upon you will come all the righteous blood shed on the earth from the blood of righteous Abel up to the blood of Zechariah son of Barachiah, whom you murdered between the temple and the altar.}}%
\verse{\JesusWords{Truly I say to you, all these things will come upon this generation!}}%
\verseWithHeading{The Lament over Jerusalem}{\JesusWords{“Jerusalem, Jerusalem, who kills the prophets and stones those who are sent to her! How many times I wanted to gather your children together the way\lebnote{“in the manner in which”} a hen gathers her young together under her\lnCXV{} wings, and you were not willing!}}%
\verse{\JesusWords{Behold, your house has been left to you desolate!}}%
\verse{\JesusWords{For I tell you, you will never see me from now on until you say, ‘Blessed is the one who comes in the name of the Lord!’”\lebnote{from Ps 118:26} }}%
\end{biblechapter}%
\begin{biblechapter}% Matthew 24
\verseWithHeading{The Destruction of the Temple Predicted}{And as Jesus went out of the temple courts\lebnote{“\textit{courts}” is supplied to distinguish this area from the interior of the temple building itself} he was going along, and his disciples came up to point out to him the buildings of the temple.}%
\verse{But he answered and\lnCXX{} said to them, \JesusWords{“Do you not see all these things? Truly I say to you, not one stone will be left here on another stone that will not be thrown down!”}}%
\verseWithHeading{Signs of the End of the Age}{And as\lebnote{“\textit{as}” is supplied as a component of the temporal genitive absolute participle (“was sitting”)} he was sitting on the Mount of Olives, the disciples came up to him privately, saying, “Tell us, when will these things happen, and what will be the sign of your coming and of the end of the age?”}%
\verse{And Jesus answered and\lnCXX{} said to them, \JesusWords{“Watch out that no one deceives you!}}%
\verse{\JesusWords{For many will come in my name, saying, ‘I am the Christ,’ and they will deceive many.}}%
\verse{\JesusWords{And you are going to hear about wars and rumors of wars. See to it that you are not alarmed, for this must happen, but the end is not yet.}}%
\verse{\JesusWords{For nation will rise up against nation and kingdom against kingdom, and there will be famines and earthquakes in various places.}\lebnote{Or “in place after place”}}%
\verse{\JesusWords{But all these things are the beginning of birth pains.}}%
\verseWithHeading{Persecution of Disciples Predicted}{\JesusWords{“Then they will hand you over to persecution and will kill you, and you will be hated by all the nations\lnCXY{} because of my name.}}%
\verse{\JesusWords{And then many will be led into sin and will betray one another and will hate one another,}}%
\verse{\JesusWords{and many false prophets will appear and will deceive many,}}%
\verse{\JesusWords{and because lawlessness will increase, the love of many will grow cold.}}%
\verse{\JesusWords{But the one who endures to the end — this person will be saved.}}%
\verse{\JesusWords{And this gospel of the kingdom will be proclaimed in the whole inhabited earth for a testimony to all the nations,\lnCXY{} and then the end will come.}}%
\verseWithHeading{The Abomination of Desolation}{\JesusWords{“So when you see the abomination of desolation\lebnote{An allusion to Dan 9:27} spoken about by the prophet Daniel standing in the holy place” (let the one who reads understand),}}%
\verse{\JesusWords{“then those in Judea must flee to the mountains!}}%
\verse{\JesusWords{The one who is on his\lnCXZ{} housetop must not come down to take things out of his house,}}%
\verse{\JesusWords{and the one who is in the field must not turn back to pick up his cloak.}}%
\verse{\JesusWords{And woe to those who are pregnant\lebnote{“who have in the womb”} and to those who are nursing their babies\lebnote{The words “\textit{their babies}” are not in the Greek text but are supplied as a necessary clarification} in those days!}}%
\verse{\JesusWords{But pray that your flight may not happen in winter or on a Sabbath.}}%
\verse{\JesusWords{For at that time there will be great tribulation, such as has not happened from the beginning of the world until now, nor ever will happen.}}%
\verse{\JesusWords{And unless those days had been shortened, no human being would be saved.\lebnote{“every flesh would not be saved”} But for the sake of the elect, those days will be shortened.}}%
\verse{\JesusWords{“At that time if anyone should say to you, ‘Behold, here is the Christ,’ or ‘Here he is,’ do not believe him!\lnCYA{}}}%
\verse{\JesusWords{For false messiahs and false prophets will appear, and will produce great signs and wonders in order to deceive, if possible, even the elect.}}%
\verse{\JesusWords{Behold, I have told you ahead of time!}}%
\verse{\JesusWords{Therefore if they say to you, ‘Behold, he is in the wilderness,’ do not go out, or\lebnote{“\textit{or}” is supplied because of English style} ‘Behold, he is in the inner rooms,’ do not believe it!\lnCYA{}}}%
\verse{\JesusWords{For just as the lightning comes from the east and flashes to the west, so the coming of the Son of Man will be.}}%
\verse{\JesusWords{Wherever the corpse is, there the vultures will gather.}}%
\verseWithHeading{The Arrival of the Son of Man}{\JesusWords{“And immediately after the tribulation of those days, ‘the sun will be darkened and the moon will not give its light, and the stars will fall from heaven, and the powers of heaven will be shaken.’}\lebnote{from Isa 13:10; 34:4}}%
\verse{\JesusWords{And then the sign of the Son of Man will appear in heaven,\lnCYB{} and then all the tribes of the earth will mourn, and they will see the Son of Man arriving on the clouds of heaven\lnCYB{} with power and great glory.}}%
\verse{\JesusWords{And he will send out his angels with a loud trumpet call, and they will gather his elect together from the four winds, from one end of heaven\lebnote{Or “of the sky”} to the other end of it.}}%
\verseWithHeading{The Parable of the Fig Tree}{\JesusWords{“Now learn the parable from the fig tree: Whenever its branch has already become tender and puts forth its\lnCXZ{} leaves, you know that summer is near.}}%
\verse{\JesusWords{So also you, when you see all these things, know\lebnote{Or “you know”} that he is near, at the door.}}%
\verse{\JesusWords{Truly I say to you that this generation will never pass away until all these things take place!}}%
\verse{\JesusWords{Heaven and earth will pass away, but my words will never pass away.}}%
\verseWithHeading{The Unknown Day and Hour}{\JesusWords{“But concerning that day and hour no one knows — not even the angels of heaven nor the Son — except the Father alone.}}%
\verse{\JesusWords{For just as the days of Noah were, so the coming of the Son of Man will be.}}%
\verse{\JesusWords{For as in the days\lebnote{Some manuscripts have “those days”} before the flood they were eating and drinking, marrying and giving in marriage, until the day Noah entered into the ark.}}%
\verse{\JesusWords{And they did not know anything\lnCYC{} until the deluge came and swept them\lnCYC{} all away. So also the coming of the Son of Man will be.}}%
\verse{\JesusWords{Then there will be two men in the field; one will be taken and one left.}}%
\verse{\JesusWords{Two women will be grinding at the mill; one will be taken and one left.}}%
\verse{\JesusWords{Therefore be on the alert, because you do not know what day your Lord is coming!}}%
\verse{\JesusWords{But understand this: that if the master of the house had known what watch of the night the thief was coming, he would have stayed awake and would not have let his house be broken into.}}%
\verse{\JesusWords{For this reason you also must be ready, because the Son of Man is coming at an hour that you do not think he will come.}\lebnote{*The words “\textit{he will come}” are not in the Greek text but are implied}}%
\verseWithHeading{A Faithful Slave and an Unfaithful Slave}{\JesusWords{“Who then is the faithful and wise slave whom the master has put in charge of his household slaves to give them their\lnCXZ{} food at the right time?}}%
\verse{\JesusWords{Blessed is that slave whom his master will find so doing when he\lebnote{“\textit{when}” is supplied as a component of the participle (“comes back”) which is understood as temporal} comes back.}}%
\verse{\JesusWords{Truly I say to you that he will put him in charge of all his possessions.}}%
\verse{\JesusWords{But if that evil slave should say to himself,\lebnote{“in his heart”} ‘My master is staying away for a long time,’}}%
\verse{\JesusWords{and he begins to beat his fellow slaves and eats and drinks with drunkards,}}%
\verse{\JesusWords{the master of that slave will come on a day that he does not expect and at an hour that he does not know,}}%
\verse{\JesusWords{and will cut him in two and assign his place with the hypocrites. In that place there will be weeping and gnashing of teeth!}}%
\end{biblechapter}%
\begin{biblechapter}% Matthew 25
\verseWithHeading{The Parable of the Ten Virgins}{\JesusWords{“Then the kingdom of heaven may be compared to ten virgins who took their lamps and\lebnote{“\textit{and}” supplied because previous participle “took” translated as a finite verb} went out to meet the bridegroom.}}%
\verse{\JesusWords{Now five of them were foolish and five were wise.}}%
\verse{\JesusWords{For when\lebnote{“\textit{when}” is supplied as a component of the participle (“took”) which is understood as temporal} the foolish ones took their lamps, they did not take olive oil with them.}}%
\verse{\JesusWords{But the wise ones took olive oil in flasks with their lamps.}}%
\verse{\JesusWords{And when\lebnote{“\textit{when}” is supplied as a component of the temporal genitive absolute participle (“was delayed”)} the bridegroom was delayed, they all became drowsy and fell asleep.}\lebnote{Imperfect tense as ingressive (“began to sleep”, “fell asleep”)}}%
\verse{\JesusWords{But in the middle of the night there was a shout, ‘Behold, the bridegroom! Come out to meet him!’}}%
\verse{\JesusWords{Then all those virgins woke up and trimmed their lamps.}}%
\verse{\JesusWords{And the foolish ones said to the wise ones, ‘Give us some of your olive oil, because our lamps are going out!’}}%
\verse{\JesusWords{But the wise ones answered saying, “Certainly there will never be enough for us and for you! Go instead to those who sell olive oil\lnCYD{} and buy some\lnCYD{} for yourselves.’}}%
\verse{\JesusWords{But while\lebnote{“\textit{while}” is supplied as a component of the temporal genitive absolute participle (“had gone away”)} they had gone away to buy it\lnCYD{} the bridegroom arrived, and those who were ready went inside with him to the wedding celebration, and the door was shut.}}%
\verse{\JesusWords{And later the other virgins came also, saying, ‘Lord, lord, open the door\lnCYD{} for us!’}}%
\verse{\JesusWords{But he answered and\lnCYE{} said, ‘Truly I say to you, I do not know you!’}}%
\verse{\JesusWords{Therefore be on the alert, because you do not know the day or the hour!}}%
\verseWithHeading{The Parable of the Talents}{\JesusWords{For it is like a man going on a journey. He summoned his own slaves and handed over his property to them.}}%
\verse{\JesusWords{And to one he gave five talents, and to another two, and to another one, to each one according to his own ability, and he went on a journey immediately.}}%
\verse{\JesusWords{The one who had received the five talents went out and\lebnote{“\textit{and}” supplied because previous participle “went out” translated as a finite verb} traded with them and gained five more.}}%
\verse{\JesusWords{In the same way the one who had the two gained two more.}}%
\verse{\JesusWords{But the one who had received the one went away and\lnCYF{} dug up the ground and hid his master’s money.}}%
\verse{\JesusWords{Now after a long time, the master of those slaves came and settled accounts with them.}}%
\verse{\JesusWords{And the one who had received the five talents came up and\lnCYG{} brought five more talents, saying, ‘Master, you handed over to me five talents. See, I have gained five more talents!’}}%
\verse{\JesusWords{His master said to him, ‘Well done, good and faithful slave! You were faithful over a few things; I will put you in charge over many things. Enter into the joy of your master!’}}%
\verse{\JesusWords{And the one who had the two talents also came up and\lnCYG{} said, ‘Master, you handed over to me two talents. See, I have gained two talents more!’}}%
\verse{\JesusWords{His master said to him, ‘Well done, good and faithful slave! You were faithful over a few things; I will put you in charge over many things. Enter into the joy of your master!’}}%
\verse{\JesusWords{And the one who had received the one talent came up also and\lnCYG{} said, ‘Master, because I\lebnote{“\textit{because}” is supplied as a component of the participle (“knew”) which is understood as causal} knew you, that you are a hard man, reaping where you did not sow and gathering from where you did not scatter seed.}\lnCYH{}}%
\verse{\JesusWords{And because I\lebnote{“\textit{because}” is supplied as a component of the participle (“was afraid”) which is understood as causal} was afraid, I went away and\lnCYF{} hid your talent in the ground. See, you have what is yours!’}}%
\verse{\JesusWords{But his master answered and\lnCYE{} said to him, ‘Evil and lazy slave! You knew that I reap where I did not sow and gather from where I did not scatter seed.}\lnCYH{}}%
\verse{\JesusWords{Then you ought to have deposited my money with the bankers, and when I\lebnote{“\textit{when}” is supplied as a component of the participle (“returned”) which is understood as temporal} returned I would have gotten back what was mine with interest!}}%
\verse{\JesusWords{Therefore take the talent from him and give it\lnCYD{} to the one who has the ten talents.}}%
\verse{\JesusWords{For to everyone who has, more will be given, and he will have an abundance. But from the one who does not have, even what he has will be taken away from him.}}%
\verse{\JesusWords{And throw the worthless slave into the outer darkness — in that place there will be weeping and gnashing of teeth!’}}%
\verseWithHeading{The Judgment of the Sheep and the Goats}{\JesusWords{Now when the Son of Man comes in his glory and all the angels with him, then he will sit on his glorious throne.}}%
\verse{\JesusWords{And all the nations will be gathered before him, and he will separate them from one another like a shepherd separates the sheep from the goats.}}%
\verse{\JesusWords{And he will place the sheep on his right and the goats on the left.}}%
\verse{\JesusWords{Then the king will say to those on his right, ‘Come, you who are blessed by my Father. Inherit the kingdom prepared for you from the foundation of the world!}}%
\verse{\JesusWords{For I was hungry and you gave me something\lnCYD{} to eat, I was thirsty and you gave me something\lnCYD{} to drink, I was a stranger and you welcomed me as a guest,}}%
\verse{\JesusWords{I was naked and you clothed me, I was sick and you cared for me, I was in prison and you came to me.’}}%
\verse{\JesusWords{Then the righteous will answer him, saying, ‘Lord, when did we see you hungry and feed you,\lnCYH{} or thirsty and give you something\lnCYD{} to drink?}}%
\verse{\JesusWords{And when did we see you a stranger and welcome you\lnCYD{} as a guest, or naked and clothe you?}\lnCYH{}}%
\verse{\JesusWords{And when did we see you sick or in prison and come to you?’}}%
\verse{\JesusWords{And the king will answer and\lebnote{“\textit{and}” supplied because previous participle “will answer” translated as a finite verb} say to them, ‘Truly I say to you, in as much as you did it\lnCYD{} to one of the least of these brothers of mine, you did it\lnCYD{} to me.’}}%
\verse{\JesusWords{Then he will also say to those on his left, ‘Depart from me, you accursed ones, into the eternal fire that has been prepared for the devil and his angels!}}%
\verse{\JesusWords{For I was hungry and you did not give me anything\lnCYD{} to eat, I was thirsty and you did not give me anything\lnCYD{} to drink,}}%
\verse{\JesusWords{I was a stranger and you did not welcome me as a guest, naked and you did not clothe me, sick and in prison and you did not care for me.’}}%
\verse{\JesusWords{Then they will also answer, saying, ‘Lord, when did we see you hungry or thirsty or a stranger or naked or sick or in prison and not serve you?’}}%
\verse{\JesusWords{Then he will answer them, saying, ‘Truly I say to you, in as much as you did not do it\lnCYD{} to one of the least of these, you did not do it\lnCYD{} to me.’}}%
\verse{\JesusWords{And these will depart into eternal punishment, but the righteous into eternal life.”}}%
\end{biblechapter}%
\begin{biblechapter}% Matthew 26
\verseWithHeading{The Chief Priests and Elders Plot to Kill Jesus}{And it happened that when Jesus had finished all these sayings, he said to his disciples,}%
\verse{\JesusWords{“You know that after two days the Passover takes place, and the Son of Man will be handed over\lebnote{Or “will be delivered up”} in order to be crucified.”}}%
\verse{Then the chief priests and the elders of the people assembled in the palace of the high priest, who was named Caiaphas,}%
\verse{and plotted in order that they could arrest Jesus by stealth and kill him.\lnCYI{}}%
\verse{But they were saying, “Not during the feast, so that there will not be an uproar among the people.”}%
\verseWithHeading{Jesus’ Anointing at Bethany}{Now while\lebnote{“\textit{while}” is supplied as a component of the temporal genitive absolute participle (“was”)} Jesus was at Bethany in the house of Simon the leper,}%
\verse{a woman came up to him holding an alabaster flask of very expensive perfumed oil, and poured it\lnCYJ{} out on his head while he\lebnote{“\textit{while}” is supplied as a component of the temporal genitive absolute participle (“was reclining at table”)} was reclining at table.}%
\verse{And when\lebnote{“\textit{when}” is supplied as a component of the participle (“saw”) which is understood as temporal} the disciples saw it\lnCYJ{} they were indignant, saying, “Why\lebnote{“for what” \textit{reason}} this waste?}%
\verse{For this could have been sold for a large sum and given to the poor!”}%
\verse{But Jesus, knowing this,\lnCYI{} said to them, \JesusWords{“Why do you cause trouble for the woman? For she has done a good deed for me.}}%
\verse{\JesusWords{For the poor you always have with you, but you do not always have me.}}%
\verse{\JesusWords{For when\lebnote{“\textit{when}” is supplied as a component of the participle (“poured”) which is understood as temporal} this woman poured this ointment on my body, she did it\lnCYJ{} in order to prepare me for burial.}}%
\verse{\JesusWords{Truly I say to you, wherever this gospel is proclaimed in the whole world, what this woman has done will also be told in memory of her.”}}%
\verseWithHeading{Judas Arranges to Betray Jesus}{Then one of the twelve, the one named Judas Iscariot, went to the chief priests}%
\verse{and\lebnote{“\textit{and}” supplied because participle in the previous verse “went” translated as a finite verb} said, “What are you willing to give me if I in turn deliver him to you?” So they set out for him thirty silver coins.}%
\verse{And from that time on, he began seeking a favorable opportunity in order that he could betray him.}%
\verseWithHeading{Jesus’ Final Passover with the Disciples}{Now on the first day\lebnote{the word “\textit{day}” is not in the Greek text but is implied} of the feast of Unleavened Bread the disciples came up to Jesus, saying, “Where do you want us to prepare for you to eat the Passover?”}%
\verse{And he said, \JesusWords{“Go into the city to a certain man and tell him, ‘The Teacher says, “My time is near. I am celebrating the Passover with you with my disciples.”}’}%
\verse{And the disciples did as Jesus directed them, and they prepared the Passover.}%
\verse{And when it\lebnote{“\textit{when}” is supplied as a component of the temporal genitive absolute participle (“was”)} was evening, he was reclining at table with the twelve disciples.\lebnote{Some manuscripts omit “disciples”}}%
\verse{And while\lnCYK{} they were eating he said, \JesusWords{“Truly I say to you, that one of you will betray me.”}}%
\verse{And greatly distressed, each one began to say to him, “Surely I am not he, am I,\lnCYL{} Lord?”}%
\verse{And he answered and\lnCYM{} said, \JesusWords{“The one who dips his\lnCYN{} hand in the bowl with me — this one will betray me.}}%
\verse{\JesusWords{The Son of Man is going just as it is written about him, but woe to that man by whom the Son of Man is betrayed! It would be better for him if that man had not been born.”}}%
\verse{And Judas, the one who was betraying him, answered and\lnCYM{} said, “Surely I am not he, am I,\lnCYL{} Rabbi?” He said to him, \JesusWords{“You have said it.”}\lnCYI{}}%
\verseWithHeading{The Lord’s Supper}{Now while\lnCYK{} they were eating Jesus took bread and, after\lebnote{“\textit{after}” is supplied as a component of the participle (“giving thanks”) which is understood as temporal} giving thanks, he broke it,\lnCYI{} and giving it\lnCYJ{} to the disciples, he said, \JesusWords{“Take, eat, this is my body.”}}%
\verse{And after\lebnote{“\textit{after}” is supplied as a component of the participle (“taking”) which is understood as temporal} taking the cup and giving thanks he gave it\lnCYJ{} to them, saying, \JesusWords{“Drink from it, all of you,}}%
\verse{\JesusWords{for this is my blood of the covenant which is poured out for many for the forgiveness of sins.}}%
\verse{\JesusWords{But I tell you, from now on I will never drink of this fruit of the vine until that day when I drink it new with you in the kingdom of my Father.”}}%
\verse{And after they\lebnote{“\textit{after}” is supplied as a component of the participle (“had sung the hymn”) which is understood as temporal} had sung the hymn, they went out to the Mount of Olives.}%
\verseWithHeading{Jesus Predicts Peter’s Denial}{Then Jesus said to them, \JesusWords{“You will all fall away because of me during this night, for it is written, ‘I will strike the shepherd and the sheep of the flock will be scattered.’}\lebnote{from Zech 13:7}}%
\verse{\JesusWords{But after I am raised, I will go ahead of you into Galilee.”}}%
\verse{But Peter answered and\lnCYM{} said to him, “If they all fall away because of you, I will never fall away!”}%
\verse{Jesus said to him, \JesusWords{“Truly I say to you that during this night, before the rooster crows, you will deny me three times!”}}%
\verse{Peter said to him, “Even if it is necessary for me to die with you, I will never deny you!” And all the disciples said the same thing.}%
\verseWithHeading{The Prayer in Gethsemane}{Then Jesus went with them to a place called Gethsemane, and he said to the disciples, \JesusWords{“Sit here while I go over there and\lebnote{“\textit{and}” supplied because previous participle “go” translated as a finite verb} pray.”}}%
\verse{And taking along Peter and the two sons of Zebedee, he began to be distressed and troubled.}%
\verse{Then he said to them, \JesusWords{“My soul is deeply grieved, to the point of death. Remain here and stay awake with me.”}}%
\verse{And going forward a little he fell down on his face, praying and saying, \JesusWords{“My Father, if it is possible, let this cup pass from me. Nevertheless, not as I will, but as you will.”}\lebnote{*Here the verb “\textit{will}” is an understood repetition of the verb earlier in this verse}}%
\verse{And he came to the disciples and found them sleeping, and he said to Peter, \JesusWords{“So, were you not able to stay awake with me one hour?}}%
\verse{\JesusWords{Stay awake and pray that you will not enter into temptation. The spirit is willing, but the flesh is weak!”}}%
\verse{Again for the second time he went away and\lnCYO{} prayed, saying, \JesusWords{“My Father, if this cannot pass unless I drink it, your will must be done.”}}%
\verse{And he came again and\lebnote{“\textit{and}” supplied because previous participle “came again” translated as a finite verb} found them sleeping, for they could not keep their eyes open.\lebnote{“for their eyes were weighed down”}}%
\verse{And leaving them again, he went away and\lnCYO{} prayed for the third time, saying the same thing again.}%
\verse{Then he came to the disciples and said to them, \JesusWords{“Are you still sleeping and resting? Behold, the hour is near, and the Son of Man is being betrayed into the hands of sinners.}}%
\verse{\JesusWords{Get up, let us go! Behold, the one who is betraying me is approaching!”}}%
\verseWithHeading{The Betrayal and Arrest of Jesus}{And while\lebnote{“\textit{while}” is supplied as a component of the temporal genitive absolute participle (“was … speaking”)} he was still speaking, behold, Judas — one of the twelve — arrived, and with him a large crowd with swords and clubs, from the chief priests and elders of the people.}%
\verse{Now the one who was betraying him had given them a sign, saying, “The one whom I kiss — he is the one.\lebnote{*Here the predicate nominative (“\textit{the one}”) is implied} Arrest him!”}%
\verse{And he came up to Jesus immediately and\lnCYP{} said, “Greetings, Rabbi,” and kissed him.}%
\verse{And Jesus said to him, \JesusWords{“Friend, do that\lebnote{The words “\textit{do that}” are not in the Greek text but are implied} for which you have come.”}\lebnote{The meaning of this phrase is disputed: (1) some take it as a declarative (as in the translation); (2) others understand it as some form of a question, often with supplied words: (a) “Friend, \textit{are you misusing the kiss} for that \textit{purpose} for which you are here?” (b) “Friend, in connection with that for which you have appeared \textit{do you kiss me}?” (c) “Friend, are you here for this purpose?” (d) “Friend, what are you here for?”; this last option, though often suggested, is doubtful because of lack of evidence for the relative pronoun used as an interrogative in direct questions} Then they came up and\lnCYP{} laid hands on Jesus and arrested him.}%
\verse{And behold, one of those with Jesus extended his\lnCYN{} hand and\lebnote{“\textit{and}” supplied because previous participle “extended” translated as a finite verb} drew his sword, and striking the slave of the high priest, cut off his ear.}%
\verse{Then Jesus said to him, \JesusWords{“Put your sword back into its place! For all who take up the sword will die by the sword.}}%
\verse{\JesusWords{Or do you think that I cannot call upon my Father, and he would put at my disposal at once more than twelve legions of angels?}}%
\verse{\JesusWords{How then would the scriptures be fulfilled that it must happen in this way?”}}%
\verse{At that time Jesus said to the crowds, \JesusWords{“Have you come out with swords and clubs, as against a robber, to arrest me? Every day in the temple courts\lebnote{“\textit{courts}” is supplied to distinguish this area from the interior of the temple building itself} I sat teaching, and you did not arrest me!}}%
\verse{But all this has happened in order that the scriptures of the prophets would be fulfilled.” Then the disciples all abandoned him and\lebnote{“\textit{and}” supplied because previous participle “abandoned” translated as a finite verb} fled.}%
\verseWithHeading{Jesus Before the Sanhedrin}{Now those who had arrested Jesus led him\lnCYJ{} away to Caiaphas the high priest, where the scribes and the elders had gathered.}%
\verse{But Peter was following him from a distance, as far as the courtyard of the high priest. And he went inside and\lnCYQ{} was sitting with the officers to see the outcome.}%
\verse{Now the chief priests and the whole Sanhedrin were looking for false testimony against Jesus in order that they could put him to death.}%
\verse{And they did not find it,\lnCYI{} although\lebnote{“\textit{although}” is supplied as a component of the participle (“came forward”) which is understood as concessive} many false witnesses came forward. And finally two came forward}%
\verse{and\lebnote{“\textit{and}” supplied because participle in the previous verse “came forward” translated as a finite verb} said, “This man said, ‘I am able to destroy the temple of God and rebuild it\lnCYJ{} within three days.’”}%
\verse{And the high priest stood up and\lebnote{“\textit{and}” supplied because previous participle “stood up” translated as a finite verb} said to him, “Do you reply nothing? What are these people testifying against you?”}%
\verse{But Jesus was silent. And the high priest said to him, “I put you under oath by the living God, that you tell us if you are the Christ, the Son of God!”}%
\verse{Jesus said to him, \JesusWords{“You have said it.\lnCYI{} But I tell you, from now on you will see the Son of Man sitting at the right hand of the Power\lebnote{An indirect way of referring to God} and coming on the clouds of heaven.”}}%
\verse{Then the high priest tore his robes, saying, “He has blasphemed! What further need do we have of witnesses? Behold, you have just now heard the blasphemy!}%
\verse{What do you think?” And they answered and\lnCYM{} said, “He deserves death!”\lebnote{“he is deserving of death”}}%
\verse{Then they spat in his face and struck him with their fists, and they slapped him,\lnCYI{}}%
\verse{saying, “Prophesy for us, you Christ! Who is it who hit you?”}%
\verseWithHeading{Peter Denies Jesus Three Times}{Now Peter was sitting outside in the courtyard, and a female slave came up to him and\lnCYP{} said, “You also were with Jesus the Galilean.”}%
\verse{But he denied it\lnCYJ{} in the presence of them all, saying, “I do not know what you mean!”}%
\verse{And when he\lebnote{“\textit{when}” is supplied as a component of the participle (“went out”) which is understood as temporal} went out to the gateway, another female slave\lebnote{The words “\textit{female slave}” are not in the Greek text but are implied by the feminine singular form} saw him and said to those who were there, “This man was with Jesus the Nazarene.”}%
\verse{And again he denied it\lnCYJ{} with an oath, “I do not know the man!”}%
\verse{And after a little while those who were standing there came up and\lnCYP{} said to Peter, “You really are one of them also, because even your accent reveals who you are.”\lebnote{“makes you evident”}}%
\verse{Then he began to curse and to swear with an oath, “I do not know the man!” And immediately a rooster crowed.}%
\verse{And Peter remembered the statement Jesus had said, \JesusWords{“Before the rooster crows, you will deny me three times,”} and he went outside and\lnCYQ{} wept bitterly.}%
\end{biblechapter}%
\begin{biblechapter}% Matthew 27
\verseWithHeading{Jesus Taken to Pilate}{Now when it\lebnote{“\textit{when}” is supplied as a component of the temporal genitive absolute participle (“was”)} was early morning, all the chief priests and the elders of the people took counsel against Jesus in order to put him to death.}%
\verse{And after\lebnote{“\textit{after}” is supplied as a component of the participle (“tying”) which is understood as temporal} tying him up, they led him\lnCYR{} away and handed him\lnCYR{} over to Pilate the governor.}%
\verseWithHeading{The Suicide of Judas Iscariot}{Then when\lnCYS{} Judas, the one who had betrayed him, saw that he had been condemned, he regretted what he had done\lnCYR{} and\lebnote{“\textit{and}” supplied because previous participle “regretted” translated as a finite verb} returned the thirty silver coins to the chief priests and elders,}%
\verse{saying, “I have sinned by\lebnote{“\textit{by}” is supplied as a component of the participle (“betraying”) which is understood as means} betraying innocent blood!” But they said, “What is that to us? You see to it!”\lnCYT{}}%
\verse{And throwing the silver coins into the temple he departed. And he went away and\lebnote{“\textit{and}” supplied because previous participle “went away” translated as a finite verb} hanged himself.}%
\verse{But the chief priests took the silver coins and\lnCYU{} said, “It is not permitted to put them into the temple treasury, because it is blood money.”\lebnote{“the price of blood”}}%
\verse{And after\lebnote{“\textit{after}” is supplied as a component of the participle (“taking”) which is understood as temporal} taking counsel, they purchased with\lebnote{“for”} them the Potter’s Field, for a burial place for strangers.}%
\verse{(For this reason that field has been called the Field of Blood until today.)}%
\verse{Then what was spoken by the prophet Jeremiah was fulfilled, who said, “And they took the thirty silver coins, the price of the one who had been priced, on whom a price had been set by the sons of Israel,}%
\verse{and they gave them for the potter’s field, just as the Lord directed me.”\lebnote{from various passages in Jeremiah including 18:2–6; 19:1–13; 32:6–15; see also Zech 11:12–13}}%
\verseWithHeading{Jesus Before Pilate}{So Jesus stood before the governor, and the governor asked him, saying, “Are you the king of the Jews?” And Jesus said, \JesusWords{“You say so.”}}%
\verse{And when he was being accused\lebnote{“in the him being accused”} by the chief priests and elders he answered nothing.}%
\verse{Then Pilate said to him, “Do you not hear how many things they are testifying against you?”}%
\verse{And he did not reply to him, not even with reference to one statement, so that the governor was very astonished.}%
\verseWithHeading{Pilate Releases Barabbas}{Now at each feast, the governor was accustomed to release one prisoner to the crowd — the one whom they wanted.}%
\verse{And at that time they had a notorious prisoner named Jesus\lnCYV{} Barabbas.\lebnote{“Barabbas” means “son of the father” in Aramaic}}%
\verse{So after\lebnote{“\textit{after}” is supplied as a component of the temporal genitive absolute participle (“had assembled”)} they had assembled, Pilate said to them, “Whom do you want me to release for you — Jesus\lnCYV{} Barabbas or Jesus who is called Christ?”}%
\verse{(For he knew that they had handed him over because of envy.}%
\verse{And while\lebnote{“\textit{while}” is supplied as a component of the temporal genitive absolute participle (“was sitting”)} he was sitting on the judgment seat, his wife sent a message\lnCYR{} to him, saying, “Have nothing to do with that righteous man,\lebnote{“nothing to you and to that righteous man”} for I have suffered much as a result of a dream today because of him.”)}%
\verse{But the chief priests and the elders persuaded the crowds that they should ask for Barabbas and put Jesus to death.}%
\verse{So the governor answered and\lnCYW{} said to them, “Which of the two do you want me to release for you?” And they said, “Barabbas!”}%
\verse{Pilate said to them, “What then should I do with Jesus, the one who is called Christ?” They all said, “Let him be crucified!”}%
\verse{And he said, “Why? What wrong has he done?” But they began to shout\lebnote{Imperfect tense as ingressive (“began to shout”)} even louder, saying, “Let him be crucified!”}%
\verse{So Pilate, when he\lnCYS{} saw that he was accomplishing nothing, but instead an uproar was developing, took water and\lnCYU{} washed his\lnCYX{} hands before the crowd, saying, “I am innocent of the blood of this man. You see to it!”\lnCYT{}}%
\verse{And all the people answered and\lnCYW{} said, “His blood be on us and on our children!”}%
\verse{Then he released Barabbas for them, but after\lebnote{“\textit{after}” is supplied as a component of the participle (“flogged”) which is understood as temporal} he had Jesus flogged, he handed him\lnCYR{} over so that he could be crucified.}%
\verseWithHeading{Jesus Is Mocked}{Then the soldiers of the governor took Jesus into the governor’s residence and\lnCYU{} gathered the whole cohort to him.}%
\verse{And they stripped him and\lebnote{“\textit{and}” supplied because previous participle “stripped” translated as a finite verb} put a scarlet military cloak around him,}%
\verse{and weaving a crown of thorns, they put it\lnCYR{} on his head, and put\lebnote{This is an understood repetition of the verb from earlier in the verse} a reed in his right hand. And kneeling down before him, they mocked him, saying, “Hail, king of the Jews!”}%
\verse{And they spat on him and\lebnote{“\textit{and}” supplied because previous participle “spat” translated as a finite verb} took the reed and repeatedly struck\lebnote{The imperfect tense has been translated as iterative here (“repeatedly struck”)} him\lnCYR{} on his head.}%
\verse{And when they had mocked him, they stripped him of the military cloak and put his own clothes on him, and led him away in order to crucify him.\lnCYT{}}%
\verseWithHeading{Jesus Is Crucified}{And as they\lebnote{“\textit{as}” is supplied as a component of the participle (“were going out”) which is understood as temporal} were going out, they found a man of Cyrene named\lebnote{“by name”} Simon. They forced this man to carry his cross.}%
\verse{And when they\lebnote{“\textit{when}” is supplied as a component of the participle (“came”) which is understood as temporal} came to a place called Golgotha (which means Place of a Skull),\lebnote{“is called Place of a Skull”}}%
\verse{they gave him wine mixed with gall to drink, and when he\lebnote{“\textit{when}” is supplied as a component of the participle (“tasted”) which is understood as temporal} tasted it\lnCYR{} he did not want to drink it.\lnCYT{}}%
\verse{And when they\lebnote{“\textit{when}” is supplied as a component of the participle (“had crucified”) which is understood as temporal} had crucified him, they divided his clothes among themselves\lebnote{“among themselves” reflects the middle voice of the verb “divided”} by\lebnote{“\textit{by}” is supplied as a component of the participle (“casting”) which is understood as means} casting lots.}%
\verse{And they sat down and\lebnote{“\textit{and}” supplied because previous participle “sat down” translated as a finite verb} were watching over him there.}%
\verse{And they put above his head the charge against him in writing:\lebnote{“written”} “This is Jesus, the king of the Jews.”}%
\verse{Then two robbers were crucified with him, one on his right and one on his left.}%
\verse{And those who passed by reviled him, shaking their heads}%
\verse{and saying, “The one who would destroy the temple and rebuild it\lnCYR{} in three days, save yourself! If you are the Son of God, come down from the cross!”}%
\verse{In the same way also the chief priests, along with the scribes and elders, were mocking him,\lnCYT{} saying,}%
\verse{“He saved others; he is not able to save himself! He is the king of Israel! Let him come down now from the cross, and we will believe in him!}%
\verse{He trusts in God; let him deliver him now if he wants to,\lebnote{Or “let him deliver \textit{him} now if he wants him”} because he said, ‘I am the Son of God’!”}%
\verse{And in the same way even the robbers who were crucified with him were reviling him.}%
\verseWithHeading{Jesus Dies on the Cross}{Now from the sixth hour, darkness came over all the land until the ninth hour.}%
\verse{And about the ninth hour Jesus cried out with a loud voice, saying, \JesusWords{“Eli, Eli, lema sabachthani?”} (that is, \JesusWords{“My God, my God, why have you forsaken me?”})\lebnote{from Ps 22:1}}%
\verse{And some of those who were standing there, when they\lebnote{“\textit{when}” is supplied as a component of the participle (“heard”) which is understood as temporal} heard it,\lnCYT{} said, “This man is summoning Elijah!”}%
\verse{And immediately one of them ran and took a sponge and filled it\lnCYR{} with sour wine and put it\lnCYR{} on a reed and\lebnote{“\textit{and}” supplied because previous participles “ran … took … filled … put” translated as finite verbs} gave it\lnCYR{} to him to drink.}%
\verse{But the others said, “Leave him\lnCYR{} alone! let us see if Elijah is coming to save him.”}%
\verse{And Jesus cried out again with a loud voice and\lebnote{“\textit{and}” supplied because previous participle “cried out” translated as a finite verb} gave up his\lnCYX{} spirit.}%
\verse{And behold, the curtain of the temple was torn in two from top to bottom, and the earth shook and the rocks were split.}%
\verse{And the tombs were opened, and many bodies of the saints who had fallen asleep were raised,}%
\verse{and coming out of the tombs after his resurrection, they went into the holy city and appeared to many.}%
\verse{Now the centurion and those with him who were guarding Jesus, when they\lnCYS{} saw the earthquake and the things that took place, were extremely frightened, saying, “Truly this man was God’s Son!”}%
\verse{And there were many women there, observing from a distance, who had followed Jesus from Galilee, serving him,}%
\verse{among whom were Mary Magdalene, and Mary the mother of James and Joseph, and the mother of the sons of Zebedee.}%
\verseWithHeading{Jesus Is Buried}{Now when it was evening, a rich man from Arimathea named Joseph came, who also was a disciple of Jesus himself.}%
\verse{This man approached Pilate and\lebnote{“\textit{and}” supplied because previous participle “approached” translated as a finite verb} asked for the body of Jesus. Then Pilate ordered it\lnCYR{} to be given to him.\lebnote{*The words “\textit{to him}” are not in the Greek text but are implied}}%
\verse{And Joseph took the body and\lnCYU{} wrapped it in a clean linen cloth,}%
\verse{and placed it in his own new tomb that he had cut in the rock. And he rolled a large stone to the entrance of the tomb and\lebnote{“\textit{and}” supplied because previous participle “rolled” translated as a finite verb} went away.}%
\verse{Now Mary Magdalene and the other Mary were there, sitting opposite the tomb.}%
\verseWithHeading{The Tomb Is Sealed and Guarded}{Now on the next day, which is after the day of preparation, the chief priests and the Pharisees assembled before Pilate,}%
\verse{saying, \JesusWords{“Sir, we remember that while\lebnote{“\textit{while}” is supplied as a component of the participle (“alive”) which is understood as temporal} that deceiver was still alive he said, ‘After three days I will rise.’}}%
\verse{Therefore give orders that the tomb be made secure until the third day, lest his disciples come and\lebnote{“\textit{and}” supplied because previous participle “come” translated as a finite verb} steal him and tell the people, ‘He has been raised from the dead,’ and the last deception will be worse than the first.”}%
\verse{Pilate said to them, “You have a guard of soldiers. Go, make it\lnCYR{} as secure as you know how.”}%
\verse{So they went with the guard of soldiers and\lebnote{“\textit{and}” supplied because previous participle “went” translated as a finite verb} made the tomb secure by\lebnote{“\textit{by}” is supplied as a component of the participle (“sealing”) which is understood as means} sealing the stone.}%
\end{biblechapter}%
\begin{biblechapter}% Matthew 28
\verseWithHeading{Jesus Is Raised}{Now after the Sabbath, at the dawning on the first day of the week, Mary Magdalene and the other Mary came to view the tomb.}%
\verse{And behold, there was a great earthquake, for an angel of the Lord descended from heaven and came up and\lebnote{“\textit{and}” supplied because previous participles “descended” and “came up” translated as finite verbs} rolled away the stone and sat down\lebnote{Or “was sitting”; here “sat down” reflects an ingressive nuance (beginning of a process or entry into a state) in the translation of the imperfect verb} on it.}%
\verse{Now his appearance was like lightning and his clothing white as snow.}%
\verse{And the guards trembled from the fear of him and became like dead men.}%
\verse{But the angel answered and\lebnote{“\textit{and}” supplied because previous participle “answered” translated as a finite verb} said to the women, “Do not be afraid, for I know that you are looking for Jesus, who was crucified.}%
\verse{He is not here, for he has been raised, just as he said. Come, see the place where he was lying.}%
\verse{And go quickly, tell his disciples, ‘He has been raised from the dead, and behold, he is going ahead of you into Galilee. You will see him there.’ Behold, I have told you.”}%
\verse{And they departed quickly from the tomb with fear and great joy, and\lebnote{“\textit{and}” supplied because previous participle “departed” translated as a finite verb} ran to tell his disciples.}%
\verse{And behold, Jesus met them, saying, \JesusWords{“Greetings!”} And they came up and\lebnote{“\textit{and}” supplied because previous participle “came up” translated as a finite verb} took hold of his feet and worshiped him.}%
\verse{Then Jesus said to them, \JesusWords{“Do not be afraid! Go tell my brothers that they should go to Galilee, and there they will see me.”}}%
\verseWithHeading{The Guards Report the Body Stolen}{And while\lebnote{“\textit{while}” is supplied as a component of the temporal genitive absolute participle (“were going”)} they were going, behold, some of the guard of soldiers went into the city and\lebnote{“\textit{and}” supplied because previous participle “went” translated as a finite verb} reported to the chief priests everything that had happened.}%
\verse{And after they\lebnote{“\textit{after}” is supplied as a component of the participle (“had assembled”) which is understood as temporal} had assembled with the elders and had taken counsel, they gave a rather large sum of money to the soldiers,}%
\verse{telling them,\lnCYY{} “Say ‘His disciples came during the night and\lebnote{“\textit{and}” supplied because previous participle “came” translated as a finite verb} stole him while\lebnote{“\textit{while}” is supplied as a component of the temporal genitive absolute participle (“were sleeping”)} we were sleeping.’}%
\verse{And if this matter is heard before the governor, we will satisfy him and keep you out of trouble.”\lebnote{“make you free from care”}}%
\verse{So they took the money and\lebnote{“\textit{and}” supplied because previous participle “took” translated as a finite verb} did as they were told, and spread abroad this report among the Jews until this very day.}%
\verseWithHeading{The Great Commission}{So the eleven disciples proceeded to Galilee, to the mountain which Jesus had designated for them.}%
\verse{And when they\lebnote{“\textit{when}” is supplied as a component of the participle (“saw”) which is understood as temporal} saw him, they worshiped him,\lnCYY{} but some doubted.}%
\verse{And Jesus approached and\lebnote{“\textit{and}” supplied because previous participle “approached” translated as a finite verb} spoke to them, saying, \JesusWords{“All authority in heaven and on earth has been given to me.}}%
\verse{\JesusWords{Therefore, go\lebnote{As a participle of attendant circumstance this participle carries imperatival force picked up from the main verb (“make disciples”)} and\lebnote{“\textit{and}” supplied because previous participle “go” translated as a finite verb} make disciples of all the nations, baptizing them in the name of the Father and of the Son and of the Holy Spirit,}}%
\verse{\JesusWords{teaching them to observe everything I have commanded you, and behold, I am with you all the days until the end of the age.”}}%
\end{biblechapter}%
\flushcolsend
\biblebook{Mark}
\begin{biblechapter}% Mark 1
\verseWithHeading{John the Baptist Begins His Ministry}{The beginning of the gospel of Jesus Christ.\lebnote{Some manuscripts add “the Son of God”}}%
\verse{Just as it is written in the prophet Isaiah, “Behold, I am sending my messenger before your face, who will prepare your way,}%
\verse{the voice of one shouting in the wilderness, ‘Prepare the way of the Lord, make straight his paths!’”\lebnote{from Isa 40:3}}%
\verse{John was there baptizing in the wilderness, proclaiming\lebnote{Some manuscripts have “and proclaiming”} a baptism of repentance for the forgiveness of sins.}%
\verse{And all the Judean region and all the inhabitants of Jerusalem went out to him and were being baptized by him in the Jordan River, confessing their sins.}%
\verse{And John was dressed in camel’s hair and a belt made of leather around his waist, and he ate locusts and wild honey.}%
\verse{And he was preaching, saying, “One who is more powerful than I is coming after me, of whom I am not worthy to bend down and\lebnote{“\textit{and}” supplied because previous participle “bend down” translated as an infinitive} untie the strap of his sandals.}%
\verse{I baptized you with water, but he will baptize you with the Holy Spirit.”}%
\verseWithHeading{The Baptism of Jesus}{And it happened that in those days Jesus came from Nazareth in Galilee and was baptized in the Jordan by John.}%
\verse{And immediately as he\lebnote{“\textit{as}” is supplied as a component of the participle (“was coming up”) which is understood as temporal} was coming up out of the water, he saw the heavens being split apart and the Spirit descending like a dove on him.}%
\verse{And a voice came from heaven, “You are my beloved Son; with you I am well pleased.”}%
\verseWithHeading{The Temptation of Jesus}{And immediately the Spirit drove him out into the wilderness.}%
\verse{And he was in the wilderness forty days being tempted by Satan. And he was with the wild animals, and the angels were ministering to him.}%
\verseWithHeading{Public Ministry in Galilee}{And after\lebnote{Some manuscripts have “Now after”} John had been taken into custody,\lebnote{“had been handed over”} Jesus went into Galilee proclaiming the gospel of God}%
\verse{and saying, \JesusWords{“The time is fulfilled and the kingdom of God has come near. Repent and believe in the gospel!”}}%
\verseWithHeading{Jesus Calls His First Disciples}{And as he\lebnote{“\textit{as}” is supplied as a component of the participle (“was passing by”) which is understood as temporal} was passing by along the Sea of Galilee, he saw Simon and Andrew, Simon’s brother, casting a net\lnCYZ{} into the sea (for they were fishermen).}%
\verse{And Jesus said to them, \JesusWords{“Follow\lebnote{“come after”} me and I will make you become fishers of people.”}}%
\verse{And immediately they left their nets and\lnCZA{} followed him.}%
\verse{And going on a little farther, he saw James the son of Zebedee and his brother John, and they were in the boat mending the nets.}%
\verse{And immediately he called them, and they left their father Zebedee in the boat with the hired men and\lnCZA{} went away after him.}%
\verse{And they went into Capernaum and immediately on the Sabbath he began to teach in the synagogue.\lebnote{Some manuscripts have “he went into the synagogue and began to teach”}}%
\verseWithHeading{A Man with an Unclean Spirit Healed}{And they were amazed at his teaching, because he was teaching them like one who had authority, and not like the scribes.}%
\verse{And so then there was a man in their synagogue with an unclean spirit, and he cried out,}%
\verse{saying, “Leave us alone,\lebnote{“what to us and to you”} Jesus the Nazarene! Have you come to destroy us? I know who you are — the Holy One of God!”}%
\verse{And Jesus rebuked him, saying, \JesusWords{“Be silent, and come out of him!”}}%
\verse{And after\lebnote{“\textit{after}” is supplied as a component of the participle (“convulsing”) which is understood as temporal} convulsing him and crying out with a loud voice, the unclean spirit came out of him.}%
\verse{And they were all amazed, so that they began to discuss with one another, saying, “What is this? A new teaching with authority! He even commands the unclean spirits and they obey him.”}%
\verse{And the report about him then went out everywhere in the whole surrounding region of Galilee.}%
\verseWithHeading{Many at Capernaum Are Healed}{And so then he departed from the synagogue and\lebnote{“\textit{and}” supplied because previous participle “departed” translated as a finite verb} came into the house of Simon and Andrew with James and John.}%
\verse{Now Simon’s mother-in-law was lying down, suffering with a fever, and at once they told him about her.}%
\verse{And he came and\lebnote{“\textit{and}” supplied because previous participle “came” translated as a finite verb} raised her up by\lebnote{“\textit{by}” is supplied as a component of the participle (“taking hold of”) which is understood as means} taking hold of her\lebnote{“the”: the Greek article is used here as a possessive pronoun} hand, and the fever left her, and she began to serve them.}%
\verse{Now when it\lebnote{“\textit{when}” is supplied as a component of the temporal genitive absolute participle (“was”)} was evening, when the sun had set, they began bringing\lebnote{Imperfect tense as ingressive (“began bringing”)} to him all those who were sick\lnCZB{} and those who were demon-possessed.}%
\verse{And the whole town was gathered together at the door.}%
\verse{And he healed many who were sick\lnCZB{} with various diseases and expelled many demons. And he did not permit the demons to speak, because they knew him.}%
\verseWithHeading{Preaching Throughout Galilee}{And getting up early in the morning while it was very dark, he departed and went to a deserted place, and there he was praying.}%
\verse{And Simon and those who were with him searched diligently for him.}%
\verse{And they found him and said to him, “Everyone is looking for you!”}%
\verse{And he said to them, \JesusWords{“Let us go elsewhere, into the neighboring rural towns, so that I can preach there also, because I have come out for this very reason.”}}%
\verse{And he went into all Galilee preaching in their synagogues and expelling demons.}%
\verseWithHeading{A Leper Cleansed}{And a leper came to him, entreating him and kneeling down, saying\lebnote{Some manuscripts have “and saying”} to him, “If you are willing, you are able to make me clean.”}%
\verse{And becoming angry,\lebnote{Some manuscripts have “having compassion”} he stretched out his hand and\lebnote{“\textit{and}” supplied because previous participle “stretched out” translated as a finite verb} touched him\lnCYZ{}, and said to him, \JesusWords{“I am willing; be made clean.”}}%
\verse{And immediately the leprosy went away from him and he was made clean.}%
\verse{And warning him sternly, he sent him away at once.}%
\verse{And he said to him, \JesusWords{“See to it that you say nothing to anyone, but go, show yourself to the priest and bring for your cleansing the things which Moses commanded, for a testimony to them.}}%
\verse{But he went out and\lebnote{“\textit{and}” supplied because previous participle “went out” translated as a finite verb} began to proclaim it\lnCYZ{} freely and to spread abroad the account, so that he was no longer able to enter publicly into a town. But he was staying outside in deserted places, and they were coming to him from all directions.}%
\end{biblechapter}%
\begin{biblechapter}% Mark 2
\verseWithHeading{A Paralytic Healed}{And when he\lebnote{“\textit{when}” is supplied as a component of the participle (“entered”) which is understood as temporal} entered again into Capernaum after some days, it became known that he was at home.}%
\verse{And many had gathered, so that there was no longer room, not even at the door, and he was speaking the word to them.}%
\verse{And they came bringing to him a paralytic, carried by four of them.}%
\verse{And when\lebnote{“\textit{when}” is supplied as a component of the participle (“able”) which is understood as temporal} they were not able to bring him\lnCZC{} to him because of the crowd, they removed the roof where he was. And after\lebnote{“\textit{after}” is supplied as a component of the participle (“digging through”) which is understood as temporal} digging through, they lowered the stretcher on which the paralytic was lying.}%
\verse{And when\lnCZD{} Jesus saw their faith, he said to the paralytic, \JesusWords{“Child, your sins are forgiven.”}}%
\verse{Now some of the scribes were sitting there and reasoning in their hearts,}%
\verse{“Why does this man speak like this? He is blaspheming! Who is able to forgive sins except God alone?”}%
\verse{And immediately Jesus, perceiving in his spirit that they were reasoning like this within themselves, said to them, \JesusWords{“Why are you considering these things in your hearts?}}%
\verse{\JesusWords{Which is easier to say to the paralytic, ‘Your sins are forgiven,’ or to say ‘Get up and pick up your stretcher and walk’?}}%
\verse{\JesusWords{But so that you may know that the Son of Man has authority on earth to forgive sins,”} — he said to the paralytic —}%
\verse{\JesusWords{“I say to you, get up, pick up your stretcher, and go to your home.”}}%
\verse{And he got up and immediately picked up his\lebnote{“the”: the Greek article is used here as a possessive pronoun} stretcher and\lebnote{“\textit{and}” supplied because previous participle “picked up” translated as a finite verb} went out in front of them all, so that they were all amazed and glorified God, saying, “We have never seen anything\lnCZC{} like this!”}%
\verseWithHeading{Levi Called to Follow Jesus}{And he went out again beside the sea,\lebnote{That is, the Sea of Galilee} and all the crowd was coming to him, and he began to teach\lebnote{Imperfect tense as ingressive (“began to teach”)} them.}%
\verse{And as he\lebnote{“\textit{as}” is supplied as a component of the participle (“was passing by”) which is understood as temporal} was passing by, he saw Levi the son of Alphaeus sitting at the tax booth, and he said to him, \JesusWords{“Follow me!”} And he stood up and\lebnote{“\textit{and}” supplied because previous participle “stood up” translated as a finite verb} followed him.}%
\verse{And it happened that he was dining\lebnote{“was reclining for a meal”} in his house, and many tax collectors and sinners were dining with\lebnote{“were reclining at table with”} Jesus and his disciples, for there were many and they were following him.}%
\verse{And the scribes of the Pharisees, when they\lnCZD{} saw that he was eating with sinners and tax collectors, began to say\lnCZE{} to his disciples, “Why does he eat with tax collectors and sinners?”}%
\verse{And when\lebnote{“\textit{when}” is supplied as a component of the participle (“heard”) which is understood as temporal} Jesus heard it\lnCZC{}, he said to them, \JesusWords{“Those who are healthy do not have need of a physician, but those who are sick.\lebnote{“having badly”} I have not come to call the righteous, but sinners.”}}%
\verseWithHeading{On Fasting}{And John’s disciples and the Pharisees were fasting, and they came and said to him, “Why\lebnote{“for what” \textit{reason}} do the disciples of John and the disciples of the Pharisees fast, but your disciples do not fast?”}%
\verse{And Jesus said to them, \JesusWords{“The bridegroom’s attendants\lebnote{“the sons of the bridal chamber”} are not able to fast while the bridegroom is with them, are they?\lebnote{The negative construction in Greek anticipates a negative answer here, indicated in the translation by the phrase “\textit{are they}”} As long a time as they have the bridegroom with them, they are not able to fast.}}%
\verse{\JesusWords{But days will come when the bridegroom is taken away from them, and then they will fast in that day.}}%
\verse{\JesusWords{No one sews a patch of unshrunken cloth on an old garment. Otherwise\lnCZF{} the patch pulls away from it — the new from the old — and the tear becomes worse.}}%
\verse{\JesusWords{And no one puts new wine into old wineskins. Otherwise\lnCZF{} the wine will burst the wineskins and the wine is destroyed and the wineskins too. But new wine is put into new wineskins.”}}%
\verseWithHeading{Plucking Grain on the Sabbath}{And it happened that he was going through the grain fields on the Sabbath, and his disciples began to make their way while\lebnote{“\textit{while}” is supplied as a component of the participle (“picking”) which is understood as temporal} plucking off the heads of grain.}%
\verse{And the Pharisees began to say\lnCZE{} to him, “Behold, why are they doing what is not permitted on the Sabbath?”}%
\verse{And he said to them, \JesusWords{“Have you never read what David did when he had need and he and those who were with him were hungry —}}%
\verse{\JesusWords{how he entered into the house of God in the time of Abiathar the high priest and ate the bread of the presentation, which it is not permitted to eat (except the priests) and also gave it\lnCZC{} to those who were with him?”}}%
\verse{And he said to them, \JesusWords{“The Sabbath was established for people, and not people for the Sabbath.}}%
\verse{\JesusWords{So then, the Son of Man is lord even of the Sabbath.”}}%
\end{biblechapter}%
\begin{biblechapter}% Mark 3
\verseWithHeading{A Man with a Withered Hand Healed}{And he entered into the synagogue again, and a man who had a withered hand was there.}%
\verse{And they were watching him closely to see if he would heal him on the Sabbath, in order that they could accuse him.}%
\verse{And he said to the man who had the withered hand, \JesusWords{“Come into the middle.”}}%
\verse{And he said to them, \JesusWords{“Is it permitted on the Sabbath to do good or to do evil, to save life or to kill?”} But they were silent.}%
\verse{And looking around at them with anger, grieved at the hardness of their hearts, he said to the man, \JesusWords{“Stretch out your\lebnote{“the”: the Greek article is used here as a possessive pronoun} hand.”} And he stretched it\lebnote{supplied from English context} out, and his hand was restored.}%
\verse{And the Pharisees went out immediately with the Herodians and began to conspire\lebnote{“began to give counsel”; the imperfect tense has been translated as ingressive (“began to”)} against him with regard to how they could destroy him.}%
\verseWithHeading{Jesus Heals Crowds by the Sea}{And Jesus went away with his disciples to the sea,\lebnote{That is, the Sea of Galilee} and a great crowd from Galilee followed him.\lnCZG{} And from Judea}%
\verse{and from Jerusalem and from Idumea and the other side of the Jordan and around Tyre and Sidon a great crowd came to him because they\lebnote{“\textit{because}” is supplied as a component of the participle (“heard”) which is understood as causal} heard all that he was doing.}%
\verse{And he told his disciples that a small boat should stand ready for him because of the crowd, so that they would not press upon him.}%
\verse{For he had healed many, so that all those who were suffering from diseases\lebnote{“had suffering”} pressed about him in order that they could touch him.}%
\verse{And the unclean spirits, whenever they saw him, were falling down before him and crying out, saying, “You are the Son of God!”}%
\verse{And he warned them strictly that they should not make him known.}%
\verseWithHeading{The Selection of the Twelve Apostles}{And he went up on the mountain and summoned those whom he wanted, and they came to him.}%
\verse{And he appointed twelve,\lebnote{Some manuscripts add “whom he also named apostles”} so that they would be with him and so that he could send them out to preach}%
\verse{and to have authority to expel demons.}%
\verse{And he appointed the twelve.\lebnote{Most Greek manuscripts omit the phrase “and he appointed the twelve”} And to Simon he gave the name Peter,}%
\verse{and James the son of Zebedee and John the brother of James (and he gave to them the name Boanerges, that is, “Sons of Thunder”),}%
\verse{and Andrew, and Philip, and Bartholomew, and Matthew, and Thomas, and James the son of Alphaeus, and Thaddaeus, and Simon the Zealot,\lebnote{“the Cananean,” but according to BDAG 507 s.v., this term has no relation at all to the geographical terms for Cana or Canaan, but is derived from the Aramaic term for “enthusiast, zealot” (see Luke 6:15; Acts 1:13)}}%
\verse{and Judas Iscariot, who also betrayed him.}%
\verseWithHeading{A House Divided Cannot Stand}{And he went home, and the crowd gathered again, so that they were not even able to eat a meal.}%
\verse{And when\lebnote{“\textit{when}” is supplied as a component of the participle (“heard”) which is understood as temporal} his family\lebnote{\textit{those} “close to him”} heard this,\lnCZG{} they went out to restrain him, for they were saying, “He has lost his mind!”}%
\verse{And the scribes who had come down from Jerusalem were saying, “He is possessed by Beelzebul!” and “By the ruler of the demons he expels the demons!”}%
\verse{And he called them to himself and\lebnote{“\textit{and}” supplied because previous participle “called” translated as a finite verb} was speaking to them in parables, \JesusWords{“How can Satan expel Satan?}}%
\verse{\JesusWords{And if a kingdom is divided against itself, that kingdom is not able to stand.}}%
\verse{\JesusWords{And if a house is divided against itself, that house will not be able to stand.}}%
\verse{\JesusWords{And if Satan has risen up against himself and is divided, he is not able to stand, but is at an end!}}%
\verse{\JesusWords{But no one is able to enter into the house of a strong man and\lebnote{“\textit{and}” supplied because previous participle “enter” translated as a finite verb} plunder his property unless he first ties up the strong man, and then he can thoroughly plunder his house.}}%
\verse{\JesusWords{“Truly I say to you that all the sins and the blasphemies will be forgiven the sons of men, however much they blaspheme.}}%
\verse{\JesusWords{But whoever blasphemes against the Holy Spirit does not have forgiveness forever,\lebnote{“for the age”} but is guilty of an eternal sin” —}}%
\verse{because they were saying, “He has an unclean spirit.”}%
\verseWithHeading{Jesus’ Mother and Brothers}{And his mother and his brothers arrived, and standing outside, they sent word to him to summon him.}%
\verse{And a crowd was sitting around him, and they told him, “Behold, your mother and your brothers\lebnote{Some manuscripts add “and your sisters”} are outside looking for you.”}%
\verse{And he answered them and\lebnote{“\textit{and}” supplied because previous participle “answered” translated as a finite verb} said, \JesusWords{“Who is my mother or\lebnote{Some manuscripts have “and”} my brothers?”}}%
\verse{And looking around at those who were sitting around him in a circle, he said, \JesusWords{“Behold, my mother and my brothers!}}%
\verse{\JesusWords{For whoever does the will of God, this person is my brother and sister and mother.”}}%
\end{biblechapter}%
\begin{biblechapter}% Mark 4
\verseWithHeading{The Parable of the Sower}{And again he began to teach beside the sea,\lebnote{That is, the Sea of Galilee} and a very large crowd was gathered to him, so that he got into a boat and\lebnote{“and” supplied because previous participle “got” translated as a finite verb} sat on the sea, and the whole crowd was at the sea on the land.}%
\verse{And he began to teach\lebnote{Imperfect tense as ingressive (“began to teach”)} them many things in parables, and was saying to them in his teaching,}%
\verse{\JesusWords{“Listen! Behold, the sower went out to sow.}}%
\verse{\JesusWords{And it happened that while he was sowing, some seed\lebnote{“\textit{some of} which”} fell on the side of the path, and the birds came and devoured it.}}%
\verse{\JesusWords{And other seed fell on the rocky ground where it did not have much soil, and it sprang up at once, because it did not have any depth of soil.}}%
\verse{\JesusWords{And when the sun rose it was scorched, and because it did not have enough root, it withered.}}%
\verse{\JesusWords{And other seed fell among the thorn plants, and the thorn plants came up and choked it, and it did not produce grain.}\lnCZH{}}%
\verse{\JesusWords{And other seed fell on the good soil, and produced grain,\lnCZH{} coming up and increasing, and it bore a crop\lnCZI{} — one thirty and one sixty and one a hundred times as much.\lnCZJ{}}}%
\verse{And he said, \JesusWords{“Whoever has ears to hear, let him hear!”}}%
\verseWithHeading{The Reason for the Parables}{And when he was alone, those around him together with the twelve began asking\lebnote{Imperfect tense as ingressive (“began asking”)} him about the parables.}%
\verse{And he said to them, \JesusWords{“To you has been granted the secret of the kingdom of God, but to those who are outside everything is in parables,}}%
\verse{\JesusWords{so that ‘they may look closely\lebnote{“seeing they may see”} and not perceive, and they may listen carefully\lebnote{“hearing they may hear”} and not understand, lest they turn and it be forgiven them.’”\lebnote{from Isa 6:9–10} }}%
\verseWithHeading{The Parable of the Sower Interpreted}{And he said to them, \JesusWords{“Do you not understand this parable? And how will you understand all the parables?}}%
\verse{\JesusWords{The sower sows the word.}}%
\verse{\JesusWords{And these are the ones beside the path where the word is sown, and whenever they hear it,\lnCZK{} immediately Satan comes and takes away the word that was sown in them.}}%
\verse{\JesusWords{And these are like\lebnote{Some manuscripts omit “like”} the ones sown on the rocky ground, who whenever they hear the word immediately receive it with joy.}}%
\verse{\JesusWords{And they have no root in themselves, but are temporary. Then when\lebnote{“\textit{when}” is supplied as a component of the temporal genitive absolute participle (“comes”)} affliction or persecution comes because of the word, immediately they fall away.}}%
\verse{\JesusWords{And others are the ones sown among the thorn plants — these are the ones who hear the word,}}%
\verse{\JesusWords{and the cares of the world and the deceitfulness of wealth and the desires for other things come in and\lebnote{“\textit{and}” supplied because previous participle “come in” translated as a finite verb} choke the word and it becomes unproductive.}}%
\verse{\JesusWords{And those are the ones sown on the good soil, who hear the word and receive it\lnCZI{} and bear fruit — one thirty and one sixty and one a hundred times as much.”\lnCZJ{}}}%
\verseWithHeading{The Parable of the Lamp}{And he said to them, \JesusWords{“Surely a lamp is not brought so that it may be put under a bushel basket or under a bed, is it?\lebnote{The negative construction in Greek anticipates a negative answer here, indicated in the translation by the phrase “\textit{is it}”} Is it not\lebnote{The negative construction in Greek anticipates a positive answer here} so that it may be put on a lampstand?}}%
\verse{\JesusWords{For nothing is secret except so that it may be revealed, nor has become hidden except so that it will come to light.}}%
\verse{\JesusWords{If anyone has ears to hear, let him hear!”}}%
\verse{And he said to them, \JesusWords{“Take care what you hear! With the measure by which you measure out, it will be measured out to you, and will be added to you.}}%
\verse{\JesusWords{For whoever has, more will be given to him, and whoever does not have, even what he has will be taken away from him.”}}%
\verseWithHeading{The Parable of the Seed that Grows by Itself}{And he said, \JesusWords{“The kingdom of God is like this: like a man scatters seed on the ground.}}%
\verse{\JesusWords{And he sleeps and gets up, night and day, and the seed sprouts and grows — he does not know how.\lebnote{“\textit{in such a way} as he does not know”}}}%
\verse{\JesusWords{By itself the soil produces a crop: first the grass, then the head of grain, then the full grain in the head.}}%
\verse{\JesusWords{But when the crop permits, he sends in the sickle right away, because the harvest has come.”}}%
\verseWithHeading{The Parable of the Mustard Seed}{And he said, \JesusWords{“With what can we compare the kingdom of God, or by what parable can we present it?}}%
\verse{\JesusWords{It is like a mustard seed that when sown on the ground, although it\lebnote{“\textit{although}” is supplied as a component of the participle (“is”) which is understood as concessive} is the smallest of all the seeds that are on the ground,}}%
\verse{\JesusWords{but when it is sown it grows up and becomes the largest of all the garden herbs, and sends out large branches so that the birds of the sky are able to nest in its shade.”}}%
\verse{And with many parables such as these he was speaking the word to them, as they were able to hear it.\lnCZK{}}%
\verse{And he did not speak to them without a parable, but in private he explained everything to his own disciples.}%
\verseWithHeading{Calming of a Storm}{And on that day, when it\lebnote{“\textit{when}” is supplied as a component of the temporal genitive absolute participle (“was”)} was evening, he said to them, \JesusWords{“Let us cross over to the other side.”}}%
\verse{And leaving the crowd, they took him along, as he was, in the boat. And other boats were with him.}%
\verse{And a great storm of wind developed, and the waves were breaking into the boat, so that the boat was already being filled with water.\lebnote{*The words “\textit{with water}” are not in the Greek text but are implied}}%
\verse{And he was in the stern sleeping on the cushion, and they woke him up and said to him, “Teacher, is it not a concern to you that we are perishing?”}%
\verse{And he woke up and\lebnote{“\textit{and}” supplied because previous participle “woke up” translated as a finite verb} rebuked the wind, and said to the sea, \JesusWords{“Be quiet! Be silent!”} And the wind abated and there was a great calm.}%
\verse{And he said to them, \JesusWords{“Why are you fearful? Do you not yet have faith?”}}%
\verse{And they were terribly frightened\lebnote{“they feared a great fear”} and began to say\lebnote{Imperfect tense as ingressive (“began to say”)} to one another, “Who then is this, that even the wind and the sea obey him?”}%
\end{biblechapter}%
\begin{biblechapter}% Mark 5
\verseWithHeading{A Demon-possessed Gerasene Healed}{And they came to the other side of the sea,\lebnote{That is, the Sea of Galilee} to the region of the Gerasenes.\lebnote{Most later manuscripts read “Gadarenes,” while other manuscripts read “Gergesenes” here}}%
\verse{And as\lebnote{“\textit{as}” is supplied as a component of the temporal genitive absolute participle (“was getting out”)} he was getting out of the boat, immediately from the tombs a man with an unclean spirit went to meet him,}%
\verse{who lived\lebnote{“had his dwelling”} among the tombs. And no one was able to bind him any longer, not even with a chain,}%
\verse{because he had often been bound with shackles and chains, and the chains had been torn apart by him, and the shackles had been shattered. And no one was strong enough to subdue him.}%
\verse{And during every night and day among the tombs and on the mountains he was crying out and cutting himself with stones.}%
\verse{And when he\lnCZL{} saw Jesus from a distance, he ran and knelt down before him.}%
\verse{And crying out with a loud voice he said, “What have I to do with you\lebnote{“what to me and to you”}, Jesus, Son of the Most High God? I implore you by God, do not torment me!”}%
\verse{(For he was saying to him, \JesusWords{“Come out of the man, unclean spirit!”})}%
\verse{And he was asking him \JesusWords{“What is your name?”}\lebnote{“what name to you”} And he said to him, “My name is Legion, because we are many.”}%
\verse{And he was imploring him many times that he would not send them out of the region.}%
\verse{Now a large herd of pigs was there at the hill feeding,}%
\verse{and they implored him, saying, “Send us to the pigs so that we may enter into them.”}%
\verse{And he permitted them. And the unclean spirits came out and\lebnote{“\textit{and}” supplied because previous participle “came out” translated as a finite verb} entered into the pigs, and the herd — about two thousand — rushed headlong down the steep slope into the sea and were drowned in the sea.}%
\verse{And their herdsmen fled and reported it\lnCZM{} in the town and in the countryside, and they came to see what it was that had happened.}%
\verse{And they came to Jesus and saw the demon-possessed man sitting there clothed and in his right mind — the one who had had the legion — and they were afraid.}%
\verse{And those who had seen it\lnCZM{} described to them what had happened to the demon-possessed man, and about the pigs.}%
\verse{And they began to urge him to depart from their region.}%
\verse{And as\lebnote{“\textit{as}” is supplied as a component of the temporal genitive absolute participle (“was getting”)} he was getting into the boat, the man who had been demon-possessed began to implore\lebnote{Imperfect tense as ingressive (“began to implore”)} him that he could go with him.}%
\verse{And he did not permit him, but said to him, \JesusWords{“Go to your home to your people and tell them all that the Lord has done for you, and that he has had mercy on you.”}}%
\verse{And he went away and began to proclaim in the Decapolis all that Jesus had done for him, and they were all astonished.}%
\verseWithHeading{A Woman Healed and a Daughter Raised}{And after\lebnote{“\textit{after}” is supplied as a component of the temporal genitive absolute participle (“had crossed over”)} Jesus had crossed over again in the boat to the other side, a large crowd gathered to him, and he was beside the sea.}%
\verse{And one of the rulers of the synagogue came — Jairus by name — and when he\lnCZL{} saw him, he fell down at his feet.}%
\verse{And he was imploring him many times, saying, “My little daughter is at the point of death!\lebnote{“has finally”} Come, lay your\lnCZN{} hands on her, so that she will get well and will live.”}%
\verse{And he went with him, and a large crowd was following him and pressing around him.}%
\verse{And there was a woman who was suffering from hemorrhages\lebnote{“with a flow of blood”} twelve years.}%
\verse{And she had endured many things under many physicians, and had spent all that she had\lebnote{“all with her”} and had received no help at all, but instead became worse.\lebnote{“came back for the worse”}}%
\verse{When she\lebnote{“\textit{when}” is supplied as a component of the participle (“heard”) which is understood as temporal} heard about Jesus, she came up in the crowd behind him and\lebnote{“\textit{and}” supplied because previous participle “came up” translated as a finite verb} touched his cloak,}%
\verse{for she was saying, “If I touch just his clothing, I will be healed!”}%
\verse{And immediately her hemorrhage stopped\lebnote{“flow of blood was dried up”} and she realized in her\lnCZN{} body that she was healed of her\lnCZN{} suffering.}%
\verse{And immediately Jesus, perceiving in himself that power had gone out from himself, turned around in the crowd and\lebnote{“\textit{and}” supplied because previous participle “turned around” translated as a finite verb} said, \JesusWords{“Who touched my clothing?”}}%
\verse{And his disciples said to him, “You see the crowd pressing upon you, and you say ‘Who touched me?’”}%
\verse{And he was looking around to see the one who had done this.}%
\verse{So the woman, frightened and trembling, knowing what had happened to her, came and fell down before him and told him the whole truth.}%
\verse{But he said to her, \JesusWords{“Daughter, your faith has healed you. Go in peace and be well from your suffering.”}}%
\verse{While\lebnote{“\textit{while}” is supplied as a component of the temporal genitive absolute participle (“speaking”)} he was still speaking, they came from the synagogue ruler’s house\lebnote{“\textit{house}” is implied since the synagogue ruler himself is directly addressed (and therefore present) in the remainder of this verse} saying, “Your daughter has died. Why trouble the Teacher further?”}%
\verse{But Jesus, ignoring\lebnote{Or “overhearing”} what was said,\lebnote{“the report that was spoken”} told the ruler of the synagogue, \JesusWords{“Do not be afraid — only believe!”}}%
\verse{And he did not allow anyone to follow along with him except Peter and James and John, the brother of James.}%
\verse{And they came to the house of the ruler of the synagogue and saw a commotion, and people weeping and wailing loudly.}%
\verse{And when he\lebnote{“\textit{when}” is supplied as a component of the participle (“entered”) which is understood as temporal} entered, he said to them, \JesusWords{“Why are you agitated and weeping? The child is not dead, but is sleeping.”}}%
\verse{And they began laughing\lebnote{Imperfect tense as ingressive (“began laughing”)} at him. But he sent them all out and\lebnote{“\textit{and}” supplied because previous participle “sent … out” translated as a finite verb} took along the father and mother of the child, and those who were with him, and went in to where the child was.}%
\verse{And taking hold of the child’s hand, he said to her, \JesusWords{“Talitha koum!”} (which is translated, \JesusWords{“Little girl, I say to you, get up!”}),}%
\verse{and immediately the girl stood up and began walking around\lebnote{Imperfect tense as ingressive (“began walking around”)} (for she was twelve years old). And immediately they were utterly and completely astonished.\lebnote{“immediately they were astonished with great astonishment”}}%
\verse{And he commanded them strictly that no one should learn of this, and said to give her something\lnCZM{} to eat.}%
\end{biblechapter}%
\begin{biblechapter}% Mark 6
\verseWithHeading{Rejected at Nazareth}{And he went out from there and came to his hometown, and his disciples followed him.}%
\verse{And when\lnCZO{} the Sabbath came, he began to teach in the synagogue, and many who heard him\lnCZP{} were amazed, saying, “Where did this man get these things? And what is this wisdom that has been granted to this man, and the miracles such as these performed through his hands?}%
\verse{Is not this the carpenter, the son of Mary and brother of James and Joses and Judas and Simon? And are not his sisters here with us?” And they were offended by him.}%
\verse{And Jesus said to them, \JesusWords{“A prophet is not without honor except in his hometown, and among his relatives, and in his own household.”}}%
\verse{And he was not able to do any miracle in that place except to lay his\lnCZQ{} hands on a few sick people and\lebnote{“\textit{and}” supplied because previous participle “lay” translated as a finite verb} heal them.\lnCZR{}}%
\verse{And he was astonished because of their unbelief.\innerVerseHeading{The Twelve Commissioned and Sent Out}And he was going around among the villages teaching.}%
\verse{And he summoned the twelve and began to send them out two by two, and gave them authority over the unclean spirits.}%
\verse{And he commanded them that they \JesusWords{take along nothing for the journey except only a staff — no bread, no traveler’s bag, no money in their\lnCZQ{} belts —}}%
\verse{\JesusWords{but to put on sandals and not to wear two tunics.}}%
\verse{And he said to them, \JesusWords{“Whenever you enter into a house, stay there until you depart from there.}}%
\verse{\JesusWords{And whatever place does not welcome you or listen to you, as you\lebnote{“\textit{as}” is supplied as a component of the participle (“go out”) which is understood as temporal} go out from there, shake off the dust that is on your feet for a testimony against them.”}}%
\verse{And they went out and\lnCZS{} proclaimed that people\lebnote{“\textit{people}” is supplied as the subject because a third-person pronoun (“they”) would be ambiguous} should repent.}%
\verse{And they were expelling many demons and anointing many sick people with olive oil and healing them.\lnCZR{}}%
\verseWithHeading{Herod Kills John the Baptist}{And King Herod heard it,\lnCZR{} because his name had become known. And they were saying, “John, the one who baptizes, has been raised from the dead, and because of this these miraculous powers are at work in him.”}%
\verse{But others were saying, “He is Elijah,” and others were saying, “He is a prophet like one of the prophets.”}%
\verse{But when\lnCZT{} Herod heard it,\lnCZR{} he said, “John whom I beheaded — this one has been raised!”}%
\verse{For Herod himself had sent and\lebnote{“\textit{and}” supplied because previous participle “had sent” translated as a finite verb} arrested John and bound him in prison because of Herodias, the wife of Philip his brother, because he had married her.}%
\verse{For John had been saying to Herod, “It is not permitted for you to have your brother’s wife.”}%
\verse{So Herodias held a grudge against him and was wanting to kill him, and was not able to do so.}%
\verse{For Herod was afraid of John, because he\lebnote{“\textit{because}” is supplied as a component of the participle (“knew”) which is understood as causal} knew him to be a righteous and holy man and protected him. And when he\lebnote{“\textit{when}” is supplied as a component of the participles (“came in” and “danced”) which are understood as temporal} listened to him, he was greatly perplexed, and yet he listened to him gladly.}%
\verse{And a suitable day came when Herod, on his birthday, gave a banquet for his courtiers and military tribunes and the most prominent men of Galilee.}%
\verse{And when\lebnote{“\textit{when}” is supplied as a component of the temporal genitive absolute participle (“came in”)} the daughter of Herodias herself\lebnote{In place of “the daughter of Herodias herself” some manuscripts have “his daughter Herodias”} came in and danced and pleased\lebnote{Some manuscripts have “danced, she pleased”} Herod and his dinner guests,\lebnote{“those reclining at table with \textit{him}”} the king said to the girl, “Ask me for whatever you want, and I will give it\lnCZP{} to you.”}%
\verse{And he swore to her,\lebnote{Some manuscripts add “at length”} “Anything whatever you ask me for I will give you, up to half my kingdom!”}%
\verse{And she went out and\lnCZS{} said to her mother, “What should I ask for?” And she said, “The head of John the baptizer.”}%
\verse{And she came in immediately with haste to the king and\lebnote{“\textit{and}” supplied because previous participle “came in” translated as a finite verb} asked, saying, “I want you to give me the head of John the Baptist on a platter immediately.”}%
\verse{And although he\lebnote{“\textit{although}” is supplied as a component of the participle (“was”) which is understood as concessive} was deeply grieved, the king, because of his\lnCZQ{} oaths and dinner guests,\lebnote{“those who were reclining at table”} did not want to refuse her.}%
\verse{And immediately the king sent an executioner and\lebnote{“\textit{and}” supplied because previous participle “sent” translated as a finite verb} ordered him\lnCZP{} to bring his head. And he went and\lebnote{“\textit{and}” supplied because previous participle “went” translated as a finite verb} beheaded him in the prison.}%
\verse{And he brought his head on a platter and gave it to the girl, and the girl gave it to her mother.}%
\verse{And when\lnCZT{} his disciples heard this,\lnCZR{} they came and took away his corpse and placed it in a tomb.}%
\verseWithHeading{The Feeding of Five Thousand}{And the apostles regathered to Jesus and reported to him everything that they had done and that they had taught.}%
\verse{And he said to them, \JesusWords{“You yourselves come privately to an isolated place and rest for a short time.”} For those who were coming and going were many, and they did not even have time to eat.}%
\verse{And they went away in the boat to an isolated place by themselves.}%
\verse{And many people saw them leaving and recognized them,\lnCZR{} and ran there together by land from all the towns, and arrived ahead of them.}%
\verse{And getting out of the boat\lebnote{The words “\textit{of the boat}” are not in the Greek text but are implied by this verb, which refers to disembarking from a boat or ship} he saw the large crowd and had compassion on them, because they were like sheep without\lebnote{“not having”} a shepherd, and he began to teach them many things.}%
\verse{And the hour had already become late when\lebnote{“\textit{when}” is supplied as a component of the participle (“came up”) which is understood as temporal} his disciples came up to him, saying, “The place is desolate and the hour is already late.}%
\verse{Send them away so that they can go into the surrounding farms and villages and\lnCZU{} purchase something to eat for themselves.”}%
\verse{But he answered and\lebnote{“\textit{and}” supplied because previous participle “answered” translated as a finite verb} said to them, \JesusWords{“You give them something\lnCZP{} to eat.”} And they said to him, \JesusWords{“Should we go and\lnCZU{} purchase bread for two hundred denarii and give it\lnCZP{} to them to eat?”}}%
\verse{And he said to them, \JesusWords{“How many loaves do you have? Go look!”} And when they\lebnote{“\textit{when}” is supplied as a component of the participle (“found out”) which is understood as temporal} found out, they said, “Five, and two fish.”}%
\verse{And he ordered them all to recline in groups on the green grass.}%
\verse{And they reclined in groups, by hundreds and by fifties.}%
\verse{And taking the five loaves and the two fish and\lebnote{“\textit{and}” is supplied in the translation because of English style} looking up to heaven, he gave thanks and broke the loaves and gave them\lnCZP{} to his disciples so that they could set them\lnCZP{} before them. And he distributed the two fish to them all.}%
\verse{And they all ate and were satisfied.}%
\verse{And they picked up the broken pieces, twelve baskets full, and of the fish.}%
\verse{And those who ate the loaves were five thousand men.}%
\verseWithHeading{Jesus Walks on the Water}{And immediately he made his disciples get into the boat and go on ahead to the other side, to Bethsaida, while he himself dismissed the crowd.}%
\verse{And after he\lebnote{“\textit{after}” is supplied as a component of the participle (“went away”) which is understood as temporal} had said farewell to them, he went away to the mountain to pray.}%
\verse{And when\lnCZO{} evening came, the boat was in the middle of the sea and he was alone on the land.}%
\verse{And he saw them being beaten in their rowing\lebnote{Or “being held up in their progress”} because the wind was against them. Around the fourth watch of the night he came to them, walking on the sea, and he was wanting to pass by them.}%
\verse{But when\lebnote{“\textit{when}” is supplied as a component of the participle (“saw”) which is understood as temporal} they saw him walking on the sea, they thought that it was a ghost, and they cried out.}%
\verse{For they all saw him and were terrified. But immediately he spoke with them and said to them, \JesusWords{“Have courage, I am he! Do not be afraid!”}}%
\verse{And he went up with them into the boat, and the wind abated. And they were extraordinarily\lebnote{“exceedingly extremely”} astounded within themselves,}%
\verse{because they did not understand concerning the loaves, but their hearts were hardened.}%
\verseWithHeading{Many Healed at Gennesaret}{And after they\lebnote{“\textit{after}” is supplied as a component of the participle (“had crossed over”) which is understood as temporal} had crossed over, they came to land at Gennesaret and anchored there.}%
\verse{And as they were getting out of the boat, people\lebnote{“\textit{people}” is supplied as the subject of the verb because the third person pronoun “they” could be confused with the disciples getting out of the boat earlier in this verse} recognized him immediately.}%
\verse{They ran about through that whole region and began to carry around those who were sick\lebnote{“who were having badly”} on stretchers, wherever they heard that he was.}%
\verse{And wherever he would go, into villages or into towns or to farms, they would put those who were sick in the marketplaces and would implore him that if they could touch even the edge of his cloak. And all those who touched it were healed.}%
\end{biblechapter}%
\begin{biblechapter}% Mark 7
\verseWithHeading{Human Traditions and God’s Commandments}{And the Pharisees and some of the scribes who had come from Jerusalem gathered to him.}%
\verse{And they saw that some of his disciples were eating their\lnCZV{} bread with unclean — that is, unwashed — hands.}%
\verse{(For the Pharisees and all the Jews do not eat unless they wash their hands ritually,\lebnote{“with the fist”; although the exact meaning of the phrase is uncertain, there is general agreement it has to do with the ritual or ceremonial nature of the handwashing} thus\lebnote{“\textit{thus}” is supplied as a component of the participle (“holding fast to”) which is understood as result} holding fast to the traditions of the elders.}%
\verse{And when they come\lebnote{The phrase “\textit{when they come}” is not in the Greek text but is implied} from the marketplace, they do not eat unless they wash. And there are many other traditions\lebnote{The word “\textit{traditions}” is not in the Greek text but is implied} which they have received and\lebnote{“\textit{and}” supplied because previous participle “have received” translated as a finite verb} hold fast to — for example,\lebnote{The phrase “\textit{for example}” is not in the Greek text but is supplied as a clarification in the English translation} the washing of cups and pitchers and bronze kettles and dining couches.)\lebnote{Several important manuscripts omit “and dining couches”}}%
\verse{And the Pharisees and the scribes asked him, “Why do your disciples not live according to the tradition of the elders, but eat their\lnCZV{} bread with unclean hands?”}%
\verse{So he said to them, \JesusWords{“Isaiah prophesied correctly about you hypocrites, as it is written, ‘This people honors me with their\lnCZV{} lips, but their heart is far, far away from me.}}%
\verse{\JesusWords{And they worship me in vain, teaching as doctrines the commandments of men.’}\lebnote{from Isa 29:13}}%
\verse{\JesusWords{Abandoning the commandment of God, you hold fast to the tradition of men.”}}%
\verse{And he said to them, \JesusWords{“You splendidly ignore the commandment of God so that you can keep\lebnote{Some manuscripts have “you can maintain”} your tradition.}}%
\verse{\JesusWords{For Moses said, ‘Honor your father and your mother,’\lebnote{from Exod 20:12; Deut 5:16} and, ‘The one who speaks evil of father or mother must certainly die\lebnote{“let him die the death”}.’\lebnote{from Exod 21:17; Lev 20:9}}}%
\verse{\JesusWords{But you say, ‘If a man says to his\lnCZV{} father or to his\lnCZV{} mother, “Whatever benefit you would have received\lebnote{“you would have been benefited”} from me is corban”\lebnote{A Hebrew term referring to something consecrated as a gift to God and thus not available for ordinary use} (that is, a gift to God),}}%
\verse{\JesusWords{you no longer permit him to do anything for his\lnCZV{} father or his\lnCZV{} mother,}}%
\verse{\JesusWords{thus\lebnote{“\textit{thus}” is supplied as a component of the participle (“making void”) which is understood as result} making void the word of God by your tradition that you have handed down, and you do many similar things such as this.”}}%
\verseWithHeading{Defilement from Within}{And summoning the crowd again, he said to them, \JesusWords{“Listen to me, all of you, and understand:}}%
\verse{\JesusWords{There is nothing outside of a person that is able to defile him by\lebnote{“\textit{by}” is supplied as a component of the participle (“going”) which is understood as means} going into him. But the things that go out of a person are the things that defile a person.”}\lebnote{Most later manuscripts add v. 16, “If anyone has ears to hear, let him hear.”}}%
\verse{\JesusWords{}}%
\verse{And when he entered into the house away from the crowd, his disciples asked him about the parable.}%
\verse{And he said to them, \JesusWords{“So are you also without understanding? Do you not understand that everything that is outside that goes into a person is not able to defile him?}}%
\verse{\JesusWords{For it does not enter into his heart but into his\lnCZV{} stomach, and goes out into the latrine” — thus\lebnote{“\textit{thus}” is supplied as a component of the participle (“declaring”) which is understood as result} purifying all the food.}}%
\verse{And he said, \JesusWords{“What comes out of a person, that defiles a person.}}%
\verse{\JesusWords{For from within, from the heart of people, come evil plans, sexual immoralities, thefts, murders,}}%
\verse{\JesusWords{adulteries, acts of greed, malicious deeds, deceit, licentiousness, envy,\lebnote{“the evil eye”} abusive speech, pride, foolishness.}}%
\verse{\JesusWords{All these evil things come from within and defile a person.”}}%
\verseWithHeading{A Syrophoenician Woman’s Great Faith}{And from there he set out and\lebnote{“\textit{and}” supplied because previous participle “set out” translated as a finite verb} went to the region of Tyre. And when he\lebnote{“\textit{when}” is supplied as a component of the participle (“entered”) which is understood as temporal} entered into a house, he wanted no one to know, and yet he was not able to escape notice.}%
\verse{But immediately a woman whose young daughter was possessed by an unclean spirit, when she\lebnote{“\textit{when}” is supplied as a component of the participle (“heard”) which is understood as temporal} heard about him, came and\lebnote{“\textit{and}” supplied because previous participle “came” translated as a finite verb} fell down at his feet,}%
\verse{Now the woman was a Greek — a Syrophoenician by nationality — and she was asking him that he would expel the demon from her daughter.}%
\verse{And he said to her, \JesusWords{“Let the children be satisfied first, for it is not right to take the children’s bread and throw it\lnCZW{} to the dogs!”}}%
\verse{But she answered and said to him, “Lord, even the dogs under the table eat the children’s crumbs.”}%
\verse{And he said to her, \JesusWords{“Because of this statement, go! The demon has gone out of your daughter.”}}%
\verse{And when she\lebnote{“\textit{when}” is supplied as a component of the participle (“went”) which is understood as temporal} went to her home, she found the child lying on the bed and the demon gone.}%
\verseWithHeading{A Man Deaf and Unable to Speak Healed}{And again he went away from the region of Tyre and\lebnote{“\textit{and}” supplied because previous participle “went away” translated as a finite verb} came through Sidon to the Sea of Galilee, within the region of the Decapolis.}%
\verse{And they brought to him a man who was deaf and had difficulty speaking, and they were imploring him that he would place his\lnCZV{} hand on him.}%
\verse{And he took him away from the crowd by himself and\lebnote{“\textit{and}” supplied because previous participle “took … away” translated as a finite verb} put his fingers into his ears, and after\lebnote{“\textit{after}” is supplied as a component of the participle (“spitting”) which is understood as temporal} spitting, he touched his tongue.}%
\verse{And looking up to heaven, he sighed and said to him, \JesusWords{“Ephphatha!”} (that is, \JesusWords{“Be opened!”}).}%
\verse{And\lebnote{Some manuscripts have “And immediately”} his ears were opened and his difficulty in speaking was removed\lebnote{“the bond of his tongue was loosened”} and he began to speak normally.}%
\verse{And he ordered them that they should say nothing, but as much as he ordered them not to, they proclaimed it\lnCZW{} even more instead.}%
\verse{And they were amazed beyond all measure, saying, “He has done all things well! He even makes the deaf hear and the mute speak!”}%
\end{biblechapter}%
\begin{biblechapter}% Mark 8
\verseWithHeading{The Feeding of Four Thousand}{In those days there was\lebnote{the participle (“was”) is translated as a finite verb because of English style} again a large crowd, and they did not have\lebnote{the participle (“have”) is translated as a finite verb because of English style} anything they could eat. Summoning the disciples, he said to them,}%
\verse{\JesusWords{“I have compassion on the crowd, because they have remained with me three days already and do not have anything to eat.}}%
\verse{\JesusWords{And if I send them away hungry to their homes they will give out on the way, and some of them have come from far away.”}}%
\verse{And his disciples answered him, “Where is anyone able to feed these people with bread here in the desert?”}%
\verse{And he asked them, \JesusWords{“How many loaves do you have?”} So they said, “Seven.”}%
\verse{And he commanded the crowd to recline for a meal on the ground, and taking the seven loaves, after he\lebnote{“\textit{after}” is supplied as a component of the participle (“had given thanks”) which is understood as temporal} had given thanks he broke them\lnCZX{} and began giving\lebnote{Imperfect tense as ingressive (“began giving”)} them\lnCZX{} to his disciples so that they could set them\lnCZX{} before them.\lnCZY{} And they set them\lnCZX{} before the crowd.}%
\verse{And they had a few small fish, and after\lebnote{“\textit{after}” is supplied as a component of the participle (“giving thanks”) which is understood as temporal} giving thanks for them, he said to set these before them also.}%
\verse{And they ate and were satisfied, and they picked up the broken pieces that were left, seven baskets full.}%
\verse{Now there were about four thousand. And he sent them away.}%
\verse{And immediately he got into the boat with his disciples and\lebnote{“\textit{and}” supplied because previous participle “got” translated as a finite verb} went to the district of Dalmanutha.}%
\verseWithHeading{Pharisees Demand a Sign}{And the Pharisees came and began to argue with him, demanding from him a sign from heaven in order to\lebnote{“\textit{in order to}” is supplied as a component of the participle (“test”) which is understood as purpose} test him.}%
\verse{And sighing deeply in his spirit, he said, \JesusWords{“Why does this generation demand a sign? Truly I say to you, no sign will be given to this generation!”}}%
\verse{And he left them, got into the boat\lebnote{the words “\textit{the boat}” are not in the Greek text but must be supplied} again, and\lebnote{“\textit{and}” supplied because previous participles “left” and “got into” translated as finite verbs} went to the other side.}%
\verseWithHeading{Beware the Leaven of the Pharisees and Herod}{And they had forgotten to take bread, and except for one loaf, they did not have any\lnCZX{} with them in the boat.}%
\verse{And he ordered them, saying, \JesusWords{“Watch out! Beware of the leaven of the Pharisees and the leaven of Herod!”}}%
\verse{And they began to discuss with one another that they had no bread.}%
\verse{And knowing this,\lnCZY{} he said to them, \JesusWords{“Why are you discussing that you have no bread? Do you not yet perceive or understand? Have your hearts been hardened?}}%
\verse{\JesusWords{Although you\lnCZZ{} have eyes, do you not see? And although you\lnCZZ{} have ears, do you not hear? And do you not remember?}}%
\verse{\JesusWords{When I broke the five loaves for the five thousand how many baskets full of broken pieces did you pick up?”} They said to him, “Twelve.”}%
\verse{\JesusWords{“When I also\lebnote{Some manuscripts omit “also”} broke\lebnote{The words “\textit{I broke}” are not in the Greek text, but are understood based on the previous verse} the seven loaves\lebnote{The word “\textit{loaves}” is not in the Greek text, but is understood based on the previous verse} for the four thousand, how many baskets full of broken pieces did you pick up?”} And they said to him, “Seven.”}%
\verse{And he said to them, \JesusWords{“Do you not yet understand?”}}%
\verseWithHeading{A Blind Man Healed at Bethsaida}{And they came to Bethsaida. And they brought to him a blind man and implored him that he would touch him.}%
\verse{And he took hold of the blind man’s hand and\lebnote{“\textit{and}” supplied because previous participle “took hold of” translated as a finite verb} led him outside the village, and after\lebnote{“\textit{after}” is supplied as a component of the participle (“spitting”) which is understood as temporal} spitting in his eyes, he placed his hands on him and\lebnote{“\textit{and}” supplied because previous participle “placed” translated as a finite verb} asked him, \JesusWords{“Do you see anything?”}}%
\verse{And looking up he said, “I see people, for I see them\lnCZX{} like trees walking around.”}%
\verse{Then he placed his\lebnote{“the”: the Greek article is used here as a possessive pronoun} hands on his eyes again, and he opened his eyes and was cured, and could see everything clearly.}%
\verse{And he sent him to his home, saying, \JesusWords{“Do not even go into the village.”}}%
\verseWithHeading{Peter’s Confession at Caesarea Philippi}{And Jesus and his disciples went out to the villages of Caesarea Philippi, and on the way he asked his disciples, saying to them, \JesusWords{“Who do people say that I am?”}}%
\verse{And they told him, saying, “John the Baptist, and others Elijah, and others that you are one of the prophets.”}%
\verse{And he asked them, \JesusWords{“But who do you say that I am?”} Peter answered and\lebnote{“\textit{and}” supplied because previous participle “answered” translated as a finite verb} said to him, “You are the Christ!”}%
\verse{And he warned them that they should tell no one about him.}%
\verseWithHeading{Jesus Predicts His Death and Resurrection}{And he began to teach them that it was necessary for the Son of Man to suffer many things and to be rejected by the elders and the chief priests and the scribes, and to be killed, and after three days to rise.}%
\verse{And he was speaking openly about the subject, and Peter took him aside and\lebnote{“\textit{and}” supplied because previous participle “took … aside” translated as a finite verb} began to rebuke him.}%
\verse{But turning around and seeing his disciples, he rebuked Peter and said, \JesusWords{“Get behind me, Satan, because you are not setting your mind on the things of God, but the things of people!”}}%
\verseWithHeading{Taking Up One’s Cross to Follow Jesus}{And summoning the crowd together with his disciples, he said to them, \JesusWords{“If anyone wants to come\lebnote{Some manuscripts have “to follow”} after me, let him deny himself and take up his cross and follow me.}}%
\verse{\JesusWords{For whoever wants to save his life will lose it, but whoever loses his life on account of me and of the gospel will save it.}}%
\verse{\JesusWords{For what does it benefit a person to gain the whole world and forfeit his life?}}%
\verse{\JesusWords{For what can a person give in exchange for his life?}}%
\verse{\JesusWords{For whoever is ashamed of me and my words in this adulterous and sinful generation, the Son of Man will also be ashamed of him when he comes in the glory of his Father with the holy angels.”}}%
\end{biblechapter}%
\begin{biblechapter}% Mark 9
\verse{And he said to them, \JesusWords{“Truly I say to you, that there are some of those standing here who will never experience death until they see the kingdom of God having come with power.”}}%
\verseWithHeading{The Transfiguration}{And after six days, Jesus took along Peter and James and John, and led them to a high mountain by themselves alone. And he was transfigured before them,}%
\verse{and his clothing became radiant — extremely white, like no cloth refiner on earth can make so white.\lebnote{“make white like this”}}%
\verse{And Elijah appeared to them together with Moses, and they were talking with Jesus.}%
\verse{And Peter answered and\lnDAA{} said to Jesus, “Rabbi, it is good that we are here! And let us make three shelters, one for you and one for Moses and one for Elijah.”}%
\verse{(For he did not know what he should answer, because they were terrified.)}%
\verse{And a cloud came, overshadowing them, and a voice came from the cloud, “This is my beloved Son. Listen to him!”}%
\verse{And suddenly, looking around, they no longer saw anyone with them but Jesus alone.}%
\verse{And as\lebnote{“\textit{as}” is supplied as a component of the temporal genitive absolute participle (“were coming down”)} they were coming down from the mountain, he ordered them that they should tell no one the things that they had seen, except when the Son of Man had risen from the dead.}%
\verse{And they kept the matter to themselves, discussing what this rising from the dead meant.\lebnote{“is”}}%
\verse{And they asked him, saying, “Why do\lebnote{“\textit{what is it} that”} the scribes say that Elijah must come first?”}%
\verse{And he said to them, \JesusWords{“Elijah indeed does come first and\lebnote{“\textit{and}” supplied because previous participle “does come” translated as a finite verb} restores all things. And how is it written concerning the Son of Man that he should suffer many things and be treated with contempt?}}%
\verse{\JesusWords{But I tell you that indeed Elijah has come, and they did to him whatever they wanted, just as it is written about him.”}}%
\verseWithHeading{A Demon-possessed Boy Healed}{And when they\lebnote{“\textit{when}” is supplied as a component of the participle (“came”) which is understood as temporal} came to the disciples, they saw a large crowd around them and scribes arguing with them.}%
\verse{And immediately the whole crowd, when they\lnDAB{} saw him, were amazed, and ran up to him\lnDAC{} and\lebnote{“\textit{and}” supplied because previous participle “ran up to” translated as a finite verb} greeted him.}%
\verse{And he asked them, \JesusWords{“What are you arguing about with them?”}}%
\verse{And one individual from the crowd answered him, “Teacher, I brought to you my son who has a spirit that makes him mute.\lebnote{“a mute spirit”}}%
\verse{And whenever it seizes him, it throws him down and he foams at the mouth and grinds his\lebnote{“the”: the Greek article is used here as a possessive pronoun} teeth and becomes paralyzed. And I told your disciples that they should expel it, and they were not able to do so.\lebnote{*The words “\textit{to do so}” are not in the Greek text but are implied}}%
\verse{And he answered them and\lnDAA{} said, \JesusWords{“O unbelieving generation! How long\lnDAD{} will I be with you? How long\lnDAD{} must I put up with you? Bring him to me!”}}%
\verse{And they brought him to him. And when he\lnDAB{} saw him, the spirit immediately convulsed him, and falling on the ground, he began to roll around, foaming at the mouth.}%
\verse{And he asked his father \JesusWords{how long it was since this had been happening to him.} And he said, “From childhood.}%
\verse{And often it has thrown him both into fire and into water, in order that it could destroy him. But if you are able to do anything, have compassion on us and\lebnote{“\textit{and}” is supplied and the following participle (“have compassion”) has been translated as a finite verb and placed before the verb “help” in keeping with English style} help us!”}%
\verse{But Jesus said to him, \JesusWords{“If you are able! All things are possible for the one who believes!”}}%
\verse{Immediately the father of the child cried out and\lebnote{“\textit{and}” supplied because previous participle “cried out” translated as a finite verb} said, “I believe! Help my unbelief!”}%
\verse{Now when\lnDAB{} Jesus saw that a crowd was running together, he rebuked the unclean spirit, saying to it, \JesusWords{“Mute and deaf spirit, I command you, come out of him, and enter into him no more!”}}%
\verse{And it came out, screaming and convulsing him\lnDAC{} greatly, and he became as if he were dead, so that most of them said, “He has died!”}%
\verse{But Jesus took hold of his hand and\lebnote{“\textit{and}” supplied because previous participle “took hold of” translated as a finite verb} raised him up, and he stood up.}%
\verse{And after\lebnote{“\textit{after}” is supplied as a component of the temporal genitive absolute participle (“had entered”)} he had entered into the house, his disciples asked him privately, “Why were we not able to expel it?”}%
\verse{And he said to them, \JesusWords{“This kind can come out by nothing except by prayer.”}\lebnote{Some manuscripts add “and fasting”}}%
\verseWithHeading{Jesus Predicts His Death and Resurrection a Second Time}{And from there they went out and\lebnote{“\textit{and}” supplied because previous participle “went out” translated as a finite verb} passed through Galilee. And he did not want anyone to know,}%
\verse{for he was teaching his disciples and was telling them, \JesusWords{“The Son of Man is being betrayed into the hands of men, and they will kill him. And when he\lebnote{“\textit{when}” is supplied as a component of the participle (“is killed”) which is understood as temporal} is killed, after three days he will rise.”}}%
\verse{But they did not understand the statement, and they were afraid to ask him.}%
\verseWithHeading{The Question About Who Is Greatest}{And they came to Capernaum. And after he\lebnote{“\textit{after}” is supplied as a component of the participle (“was”) which is understood as temporal} was in the house, he asked them, \JesusWords{“What were you discussing on the way?”}}%
\verse{But they were silent, because they had argued with one another on the way about who was greatest.}%
\verse{And he sat down and\lebnote{“\textit{and}” supplied because previous participle “sat down” translated as a finite verb} called the twelve and said to them, \JesusWords{“If anyone wants to be first, he will be last of all and servant of all.”}}%
\verse{And he took a young child and\lebnote{“\textit{and}” supplied because previous participle “took” translated as a finite verb} had him stand among them.\lebnote{“in the midst of them”} And taking him in his arms, he said to them,}%
\verse{\JesusWords{“Whoever welcomes one of the young children such as these in my name welcomes me, and whoever welcomes me does not welcome me, but the one who sent me.”}}%
\verseWithHeading{Whoever Is Not Against Us Is for Us}{John said to him, “Teacher, we saw someone expelling demons in your name, and we tried to prevent him because he was not following us.”}%
\verse{But Jesus said, \JesusWords{“Do not prevent him, because there is no one who does a miracle in my name and will be able soon afterward to speak evil of me.}}%
\verse{\JesusWords{For whoever is not against us is for us.}}%
\verse{\JesusWords{For whoever gives you a cup of water to drink in my name because you are Christ’s, truly I say to you that he will never lose his reward.}}%
\verseWithHeading{Temptations to Sin}{\JesusWords{“And whoever causes one of these little ones who believe in me to sin, it is better for him if instead a large millstone\lebnote{“a millstone of a donkey”} is placed around his neck and he is thrown into the sea.}}%
\verse{\JesusWords{And if your hand causes you to sin, cut it off! It is better for you to enter into life crippled than, having two hands, to go into hell — into the unquenchable fire!\lebnote{Most later Greek manuscripts add v. 44 after v. 43, “where their worm does not die and the fire is not extinguished” (identical to v. 48)}}}%
\verse{\JesusWords{}}%
\verse{\JesusWords{And if your foot causes you to sin, cut it off! It is better for you to enter into life lame than, having two feet, to be thrown into hell!}\lebnote{Most later Greek manuscripts add v. 46 after v. 45, “where their worm does not die and the fire is not extinguished” (identical to v. 48)}}%
\verse{\JesusWords{}}%
\verse{\JesusWords{And if your eye causes you to sin, tear it out! It is better for you to enter into the kingdom of God with one eye than, having two eyes, to be thrown into hell,}}%
\verse{\JesusWords{‘where their worm does not die and the fire is not extinguished.’}\lebnote{from Isa 66:24}}%
\verse{\JesusWords{For everyone will be salted with fire.}}%
\verse{\JesusWords{Salt is good, but if the salt becomes deprived of its salt content, by what can you make it salty? Have salt among yourselves, and be at peace with one another.”}}%
\end{biblechapter}%
\begin{biblechapter}% Mark 10
\verseWithHeading{On Divorce}{And from there he set out and\lebnote{“\textit{and}” supplied because previous participle “set out” translated as a finite verb} came to the region of Judea and the other side of the Jordan, and again crowds came together to him. And again, as he was accustomed to do,\lebnote{*The words “\textit{to do}” are not in the Greek text but are implied} he began to teach\lebnote{Imperfect tense as ingressive (“began to teach”)} them.}%
\verse{And they asked\lebnote{Some manuscripts have “And Pharisees came up and asked”} him if it was permitted for a man to divorce his\lebnote{The pronoun “\textit{his}” is not in the Greek text but is implied} wife, in order to\lebnote{“\textit{in order to}” is supplied as a component of the participle (“test”) which is understood as purpose} test him.}%
\verse{And he answered and\lnDAE{} said to them, \JesusWords{“What did Moses command you?”}}%
\verse{So they said, “Moses permitted a man\lnDAF{} to write a certificate of divorce and to send her\lnDAF{} away.”}%
\verse{But Jesus said to them, \JesusWords{“He wrote this commandment for you because of your hardness of heart.}}%
\verse{\JesusWords{But from the beginning of creation ‘he made them male and female.}\lebnote{from Gen 1:27; 5:2}}%
\verse{\JesusWords{Because of this a man will leave his father and mother and will be joined to his wife,}\lebnote{The earliest and most important manuscripts do not contain the phrase “and be joined to his wife”}}%
\verse{\JesusWords{and the two will become one flesh,’\lebnote{from Gen 2:24} so that they are no longer two but one flesh.}}%
\verse{\JesusWords{Therefore what God has joined together, man must not separate.”}}%
\verse{And in the house again the disciples began to ask\lebnote{Imperfect tense as ingressive (“began to ask”)} him about this.}%
\verse{And he said to them, \JesusWords{“Whoever divorces his wife and marries another commits adultery against her.}}%
\verse{\JesusWords{And if she divorces her husband and\lebnote{“\textit{and}” supplied because previous participle “divorces” translated as a finite verb} marries another, she commits adultery.”}}%
\verseWithHeading{Little Children Brought to Jesus}{And they were bringing young children to him so that he could touch them, but the disciples rebuked them.}%
\verse{But when\lebnote{“\textit{when}” is supplied as a component of the participle (“saw”) which is understood as temporal} Jesus saw it,\lnDAG{} he was indignant, and said to them, \JesusWords{“Let the young children come to me. Do not forbid them, for to such belongs\lebnote{“for of such is”} the kingdom of God.}}%
\verse{\JesusWords{Truly I say to you, whoever does not welcome the kingdom of God like a young child will never enter into it.”}}%
\verse{And after\lebnote{“\textit{after}” is supplied as a component of the participle (“taking … into his arms”) which is understood as temporal} taking them\lnDAF{} into his arms, he blessed them, placing his\lebnote{“the”: the Greek article is used here as a possessive pronoun} hands on them.}%
\verseWithHeading{A Rich Young Man}{And as\lnDAH{} he was setting out on his way, one individual ran up and knelt down before him and\lebnote{“\textit{and}” supplied because two previous participles “ran up” and “knelt down before” translated as finite verbs} asked him, “Good Teacher, what must I do so that I will inherit eternal life?”}%
\verse{So Jesus said to him, \JesusWords{“Why do you call me good? No one is good except God alone.}}%
\verse{\JesusWords{You know the commandments: ‘Do not murder, do not commit adultery, do not steal, do not give false testimony, do not defraud, honor your father and mother.’”\lebnote{from Exod 20:12–16; Deut 5:16–20, except for “do not defraud” which is an allusion to Deut 24:14}}}%
\verse{And he said to him, “Teacher, all these I have observed from my youth.”}%
\verse{And Jesus, looking at him, loved him, and said to him, \JesusWords{“You lack one thing: Go, sell all that you have, and give the proceeds\lnDAF{} to the poor — and you will have treasure in heaven — and come, follow me.”}}%
\verse{But he looked gloomy at the statement and\lebnote{“\textit{and}” supplied because previous participle “looked gloomy” translated as a finite verb} went away sorrowful, because he had\lebnote{“because he was having”} many possessions.}%
\verse{And Jesus looked around and\lebnote{“\textit{and}” supplied because previous participle “looked around” translated as a finite verb} said to his disciples, \JesusWords{“How difficult it is for\lebnote{“with difficulty”} those who possess wealth to enter into the kingdom of God!”}}%
\verse{And the disciples were astounded at his words. But Jesus answered and\lnDAE{} said to them again, \JesusWords{“Children, how difficult it is to enter into the kingdom of God!}}%
\verse{\JesusWords{It is easier for a camel to go through the eye of a needle than for a rich person to enter into the kingdom of God.”}}%
\verse{And they were very astounded, saying to one another, “And who can be saved?”}%
\verse{Jesus looked at them and\lebnote{“\textit{and}” supplied because previous participle “looked at” translated as a finite verb} said, \JesusWords{“With human beings it is impossible, but not with God. For all things are possible with God.”}}%
\verse{Peter began to say to him, “Behold, we have left everything and followed you.”}%
\verse{Jesus said, \JesusWords{“Truly I say to you, there is no one who has left house or brothers or sisters or mother or father or children or fields on account of me and on account of the gospel}}%
\verse{\JesusWords{who will not\lebnote{“unless \textit{he will} not”} receive a hundred times as much now in this time — houses and brothers and sisters and mothers and children and fields, together with persecutions — and in the age to come, eternal life.}}%
\verse{\JesusWords{But many who are first will be last, and the last first.”}}%
\verseWithHeading{Jesus Predicts His Death and Resurrection a Third Time}{Now they were on the road going up to Jerusalem, and Jesus was going on ahead of them. And they were astounded, but those who were following him\lnDAF{} were afraid. And taking aside the twelve again, he began to tell them the things that were about to happen to him:}%
\verse{\JesusWords{“Behold, we are going up to Jerusalem, and the Son of Man will be handed over to the chief priests and the scribes, and they will condemn him to death and will hand him over to the Gentiles.}}%
\verse{\JesusWords{And they will mock him and spit on him and flog him and kill him,\lnDAG{} and after three days he will rise.”}}%
\verseWithHeading{A Request by James and John}{And James and John, the sons of Zebedee, came up to him and\lebnote{“\textit{and}” supplied because participle “said” translated as a finite verb in keeping with English style} said to him, “Teacher, we want you to do for us whatever we ask you.”}%
\verse{And he said to them, \JesusWords{“What do you want that I do\lebnote{Some manuscripts have “do you want me to do”} for you?”}}%
\verse{So they said to him, “Grant to us that we may sit one at your right hand and one at your left in your glory.”}%
\verse{But Jesus said to them, \JesusWords{“You do not know what you are asking! Are you able to drink the cup that I drink, or to be baptized with the baptism that I am baptized with?”}}%
\verse{And they said to him, “We are able.” So Jesus said to them, \JesusWords{“You will drink the cup that I drink, and you will be baptized with the baptism that I am baptized with,}}%
\verse{\JesusWords{but to sit at my right hand or at my left is not mine to grant, but is for those for whom it has been prepared.”}}%
\verse{And when they\lnDAI{} heard this,\lnDAG{} the ten began to be indignant about James and John.}%
\verse{And Jesus called them to himself and\lebnote{“\textit{and}” supplied because previous participle “called … to himself” translated as a finite verb} said to them, \JesusWords{“You know that those who are considered to rule over the Gentiles lord it over them, and their people in high positions exercise authority over them.}}%
\verse{\JesusWords{But it is not like this among you! But whoever wants to become great among you must be your servant,}}%
\verse{\JesusWords{and whoever wants to be most prominent among you must be the slave of all.}}%
\verse{\JesusWords{For even the Son of Man did not come to be served, but to serve, and to give his life as a ransom for many.”}}%
\verseWithHeading{A Blind Man Healed at Jericho}{And they came to Jericho. And as\lnDAH{} he was setting out from Jericho along with his disciples and a large crowd, a blind beggar, Bartimaeus the son of Timaeus, was sitting beside the road.}%
\verse{And when he\lnDAI{} heard that it was Jesus the Nazarene, he began to cry out and say, “Jesus, Son of David, have mercy on me!”}%
\verse{And many people warned him that he should be quiet. But he was crying out even more loudly,\lebnote{“by much more”} “Son of David, have mercy on me!”}%
\verse{And Jesus stopped and\lebnote{“\textit{and}” supplied because previous participle “stopped” translated as a finite verb} said, \JesusWords{“Call him.”} And they called the blind man and\lebnote{“\textit{and}” supplied because previous participle “called” translated as a finite verb} said to him, “Have courage! Get up! He is calling you.”}%
\verse{And he threw off his cloak, jumped up, and\lebnote{“\textit{and}” supplied because two previous participles “threw off” and “jumped up” translated as finite verbs} came to Jesus.}%
\verse{And Jesus answered him and\lnDAE{} said, \JesusWords{“What do you want me to do\lebnote{“that I do”} for you?”} And the blind man said to him, “Rabboni,\lebnote{The Aramaic term is an elevated form of Rabbi} that I may regain my sight.”}%
\verse{And Jesus said to him, \JesusWords{“Go, your faith has healed you.”} And immediately he regained his sight and began to follow\lebnote{Imperfect tense as ingressive (“began to follow”)} him on the road.}%
\end{biblechapter}%
\begin{biblechapter}% Mark 11
\verseWithHeading{The Triumphal Entry}{And when they came near to Jerusalem, to Bethphage and Bethany at the Mount of Olives, he sent two of his disciples}%
\verse{and said to them, \JesusWords{“Go into the village before you, and right away as you\lebnote{“\textit{as}” is supplied as a component of the participle (“enter”) which is understood as temporal} enter into it you will find a colt tied, on which no one has ever sat. Untie it and bring it.}\lnDAJ{}}%
\verse{\JesusWords{And if anyone says to you, ‘Why are you doing this?’ say\lebnote{Some manuscripts omit “that” after “say” here; though understood to be present in the underlying Greek text, it introduces direct discourse here and is left untranslated, functioning much like English quotation marks} ‘The Lord has need of it, and will send it here again at once.’”}}%
\verse{And they went away and found a colt tied at a door outside in the street, and they untied it.}%
\verse{And some of those who were standing there said to them, “What are you doing, untying the colt?”}%
\verse{So they told them, just as Jesus had said, and they allowed them to take it.\lebnote{The phrase “\textit{to take it}” is not in the Greek text, but is implied}}%
\verse{And they brought the colt to Jesus and threw their cloaks over it, and he sat on it.}%
\verse{And many people spread their cloaks on the road, and others spread\lebnote{the verb “\textit{spread}” is an understood repetition of the verb earlier in this verse} leafy branches they\lebnote{the participle “had cut” has been translated as a finite verb; it agrees in number, gender, and case with “others,” so “\textit{they}’ has been supplied to indicate this} had cut from the fields.}%
\verse{And those who went ahead and those who were following were shouting, “Hosanna! Blessed is the one who comes in the name of the Lord!\lebnote{from Ps 118:25–26}}%
\verse{Blessed is the coming kingdom of our father David! Hosanna in the highest heaven!”\lebnote{*Here “heaven” is understood}}%
\verse{And he went into Jerusalem to the temple, and after\lebnote{“\textit{after}” is supplied as a component of the participle (“looking around”) which is understood as temporal} looking around at everything, because\lebnote{“\textit{because}” is supplied as a component of the participle (“was”) which is understood as causal} the hour was already late, he went out to Bethany with the twelve.}%
\verseWithHeading{A Barren Fig Tree Cursed}{And on the next day as\lebnote{“\textit{as}” is supplied as a component of the temporal genitive absolute participle (“were departing”)} they were departing from Bethany, he was hungry.}%
\verse{And when he\lebnote{“\textit{when}” is supplied as a component of the participle (“saw”) which is understood as temporal} saw from a distance a fig tree that had leaves, he went to see if perhaps he would find anything on it. And when he\lebnote{“\textit{when}” is supplied as a component of the participle (“came up”) which is understood as temporal} came up to it he found nothing except leaves, because it was not the season for figs.}%
\verse{And he responded and\lebnote{“\textit{and}” supplied because previous participle “responded” translated as a finite verb} said to it, \JesusWords{“Let no one eat fruit from you any more forever!”}\lebnote{“for the age”} And his disciples heard it.\lnDAJ{}}%
\verseWithHeading{The Cleansing of the Temple}{And they came to Jerusalem. And he entered into the temple courts\lebnote{“\textit{courts}” is supplied to distinguish this area from the interior of the temple building itself} and\lebnote{“\textit{and}” supplied because previous participle “entered” translated as a finite verb} began to drive out those who were selling and those who were buying in the temple courts,\lnDAK{} and overturned the tables of the money changers and the chairs of those who were selling doves.}%
\verse{And he did not permit anyone to carry objects\lebnote{Or “merchandise”} through the temple courts.\lnDAK{}}%
\verse{And he began to teach\lebnote{Imperfect tense as ingressive (“began to teach”)} and was saying to them, \JesusWords{“Is it not written, ‘My house will be called a house of prayer for all the nations,’\lebnote{from Isa 56:7} but you have made it a cave of robbers!”}}%
\verse{And the chief priests and the scribes heard it,\lnDAJ{} and began considering\lebnote{Imperfect tense as ingressive (“began considering”)} how they could destroy him. For they were afraid of him because the whole crowd was astounded by his teaching.}%
\verse{And when evening came they went out of the city.}%
\verseWithHeading{The Barren Fig Tree Withered}{And as they\lebnote{“\textit{as}” is supplied as a component of the participle (“passed by”) which is understood as temporal} passed by early in the morning, they saw the fig tree withered from the roots.}%
\verse{And Peter remembered and\lebnote{“\textit{and}” supplied because previous participle “remembered” translated as a finite verb} said to him, “Rabbi, look! The fig tree that you cursed has withered!”}%
\verse{And Jesus answered and\lebnote{“\textit{and}” supplied because previous participle “answered” translated as a finite verb} said to them, \JesusWords{“Have faith in God!}}%
\verse{\JesusWords{Truly I say to you that whoever says to this mountain, ‘Be lifted up and thrown into the sea!’ and does not doubt in his heart, but believes that what he says will happen, it will be done for him.}}%
\verse{\JesusWords{For this reason I say to you, whatever you pray and ask for, believe that you have received it,\lnDAJ{} and it will be done for you.}}%
\verse{\JesusWords{And whenever you stand praying, if you have anything against anyone, forgive him,\lnDAJ{} so that your Father who is in heaven will also forgive you your sins.”}\lebnote{Most later Greek manuscripts add v. 26 after v. 25, “But if you do not forgive, neither will your Father in heaven forgive your sins”}}%
\verseWithHeading{Jesus’ Authority Challenged}{}%
\verse{And they came again to Jerusalem. And as\lebnote{“\textit{as}” is supplied as a component of the temporal genitive absolute participle (“was walking”)} he was walking in the temple courts,\lnDAK{} the chief priests and the scribes and the elders came up to him}%
\verse{and said to him, “By what authority are you doing these things, or who gave you this authority that you do these things?”}%
\verse{So Jesus said to them, \JesusWords{“I will ask you one question. Answer me and I will tell you by what authority I am doing these things.}}%
\verse{\JesusWords{The baptism of John — was it from heaven or from men? Answer me!”}}%
\verse{And they began to discuss\lebnote{Imperfect tense as ingressive (“began to discuss”)} this\lebnote{supplied from English context} with one another, saying, “What should we say?\lebnote{Some manuscripts omit “What should we say?”} If we say ‘From heaven,’ he will say, ‘Why then did you not believe him?’}%
\verse{But if we say, ‘From men’” — they were afraid of the crowd, because they all looked upon John as truly a prophet.\lebnote{“that he was truly a prophet”}}%
\verse{And they replied to Jesus saying, “We do not know.” And Jesus said to them, \JesusWords{“Neither will I tell you by what authority I am doing these things.”}}%
\end{biblechapter}%
\begin{biblechapter}% Mark 12
\verseWithHeading{The Parable of the Tenant Farmers in the Vineyard}{And he began to speak to them in parables: \JesusWords{“A man planted a vineyard, and put a fence around it, and dug a trough for the winepress, and built a watchtower, and leased it to tenant farmers, and went on a journey.}}%
\verse{\JesusWords{And he sent a slave to the tenant farmers at the proper time, so that he could collect some of the fruit of the vineyard from the tenant farmers.}}%
\verse{\JesusWords{And they seized him and\lnDAL{} beat him\lnDAM{} and sent him\lnDAM{} away empty-handed.}}%
\verse{\JesusWords{And again he sent to them another slave, and that one they struck on the head and dishonored.}}%
\verse{\JesusWords{And he sent another, and that one they killed. And he sent\lebnote{The words “\textit{he sent}” are not in the Greek text, but are an implied repetition from earlier in the verse} many others, some of whom they beat and some of whom they killed.}}%
\verse{\JesusWords{He had one more, a beloved son. Last of all he sent him to them, saying, ‘They will respect my son.’}}%
\verse{\JesusWords{But those tenant farmers said to one another, ‘This is the heir. Come, let us kill him and the inheritance will be ours!’}}%
\verse{\JesusWords{And they seized and\lnDAL{} killed him and threw him out of the vineyard.}}%
\verse{\JesusWords{What\lebnote{Some manuscripts have “What then”} will the owner of the vineyard do? He will come and destroy the tenant farmers and give the vineyard to others.}}%
\verse{\JesusWords{Have you not read this scripture: ‘The stone which the builders rejected, this has become the cornerstone.}\lebnote{“the head of the corner”}}%
\verse{\JesusWords{This came about from the Lord, and it is marvelous in our eyes’?”}\lebnote{from Ps 118:22–23}}%
\verse{And they were seeking to arrest him, and they were afraid of the crowd, because they knew that he had told the parable with reference to them. And they left him and\lebnote{“\textit{and}” supplied because previous participle “left” translated as a finite verb} went away.}%
\verseWithHeading{Paying Taxes to Caesar}{And they sent some of the Pharisees and the Herodians to him so that they could catch him unawares in a statement.}%
\verse{And when they\lebnote{“\textit{when}” is supplied as a component of the participle (“came”) which is understood as temporal} came, they said to him, “Teacher, we know that you are truthful and you do not care what anyone thinks,\lebnote{“it is not a care to you concerning anyone”} because you do not regard the opinion of people\lebnote{“because you do not look at the face of men”} but teach the way of God in truth. Is it permitted to pay taxes to Caesar or not? Should we pay or should we not pay?”}%
\verse{But because he\lebnote{“\textit{because}” is supplied as a component of the participle (“knew”) which is understood as causal} knew their hypocrisy, he said to them, \JesusWords{“Why are you testing me? Bring me a denarius so that I can look at it!”\lnDAN{}}}%
\verse{So they brought one.\lnDAN{} And he said to them, \JesusWords{“Whose image and inscription is this?”} And they said to him, “Caesar’s.”}%
\verse{And Jesus said to them, \JesusWords{“Give to Caesar the things of Caesar, and to God the things of God!”} And they were utterly amazed at him.}%
\verseWithHeading{A Question About Marriage and the Resurrection}{And Sadducees — who say there is no resurrection — came up to him and began to ask\lebnote{Imperfect tense as ingressive (“began to ask”)} him, saying,}%
\verse{“Teacher, Moses wrote for us that if someone’s brother dies and he leaves behind a wife and does not leave a child, that his brother should take the wife and father\lebnote{“raise up”} descendants for his brother.}%
\verse{There were seven brothers, and the first took a wife. And when he\lebnote{“\textit{when}” is supplied as a component of the participle (“died”) which is understood as temporal} died, he did not leave descendants.}%
\verse{And the second took her, and he died without leaving descendants. And the third likewise.}%
\verse{And the seven did not leave descendants. Last of all the woman also died.}%
\verse{In the resurrection, when they rise, whose\lebnote{“who of them”} wife will she be? For the seven had her as wife.}%
\verse{Jesus said to them, \JesusWords{“Are you not deceived because of this, because you\lebnote{“\textit{because}” is supplied as a component of the participle (“know”) which is understood as causal} do not know the scriptures or the power of God?}}%
\verse{\JesusWords{For when they rise from the dead, they neither marry nor are given in marriage, but are like angels in heaven.}}%
\verse{\JesusWords{Now concerning the dead, that they are raised, have you not read in the book of Moses in the passage about the bush\lebnote{“at the bush”} how God spoke to him, saying, ‘I am the God of Abraham and the God of Isaac and the God of Jacob’?}\lebnote{from Exod 3:6}}%
\verse{\JesusWords{He is not God of the dead, but of the living. You are very much mistaken!”}}%
\verseWithHeading{The Greatest Commandment}{And one of the scribes came up and\lebnote{“\textit{and}” supplied because previous participle “came up” translated as a finite verb} heard them debating. When he\lnDAO{} saw that he answered them well, he asked him, “Which commandment is the most important of all?”}%
\verse{Jesus answered, \JesusWords{“The most important is, ‘Listen, Israel! The Lord our God, the Lord is one.}}%
\verse{\JesusWords{And you shall love the Lord your God from your whole heart and from your whole soul and from your whole mind and from your whole strength.’}\lebnote{from Deut 6:4–5; Josh 22:5}}%
\verse{\JesusWords{The second is this: ‘You shall love your neighbor as yourself.’\lebnote{from Lev 19:18} There is no other commandment greater than these.”}}%
\verse{And the scribe said to him, “That is true, Teacher. You have said correctly\lebnote{“in accordance with truth”} that he is one and there is no other except him.}%
\verse{And to love him from your\lnDAP{} whole heart and from your\lnDAP{} whole understanding and from your\lnDAP{} whole strength, and to love your\lnDAP{} neighbor as yourself, is much more than all whole burnt offerings and sacrifices.”}%
\verse{And Jesus, when he\lnDAO{} saw that he had answered thoughtfully, said to him, \JesusWords{“You are not far from the kingdom of God.”} And no one dared to put a question to him any longer.}%
\verseWithHeading{David’s Son and Lord}{And continuing, Jesus said while\lebnote{“\textit{while}” is supplied as a component of the participle (“teaching”) which is understood as temporal} teaching in the temple courts,\lebnote{*Here “\textit{courts}” is supplied to distinguish this area from the interior of the temple building itself} \JesusWords{“How can the scribes say that the Christ is David’s son?}}%
\verse{\JesusWords{David himself said by the Holy Spirit, ‘The Lord said to my Lord, “Sit at my right hand, until I put your enemies under your feet.”’\lebnote{from Ps 110:1} }}%
\verse{David himself calls him ‘Lord,’ and how is he his son?” And the large crowd was listening to him gladly.}%
\verseWithHeading{Warning to Beware of the Scribes}{And in his teaching he said, \JesusWords{“Beware of the scribes, who like walking around in long robes and greetings in the marketplaces}}%
\verse{\JesusWords{and the best seats in the synagogues and the places of honor at banquets,}}%
\verse{\JesusWords{who devour the houses of widows and pray lengthy prayers for the sake of appearance. These will receive more severe condemnation!”}}%
\verseWithHeading{A Poor Widow’s Offering}{And he sat down opposite the contribution box and\lebnote{“\textit{and}” supplied because previous participle “sat down” translated as a finite verb} was observing how the crowd was putting coins into the contribution box. And many rich people were putting in many coins.\lebnote{Although often translated “large sums,” the plural here suggests large numbers of individual coins, which would make an impressive noise}}%
\verse{And one poor widow came and\lebnote{“\textit{and}” supplied because previous participle “came” translated as a finite verb} put in two small copper coins\lebnote{This coin was the \textit{lepton}, worth 1/128 of a denarius} (that is, a penny).\lebnote{This coin was the \textit{quadrans}, the smallest Roman coin, worth 2 \textit{lepta}}}%
\verse{And summoning his disciples, he said to them, \JesusWords{“Truly I say to you that this poor widow put in more than all those who put offerings\lnDAM{} into the contribution box.}}%
\verse{\JesusWords{For they all contributed\lebnote{“put in”} out of their abundance, but she out of her poverty put in everything she had, her whole means of subsistence.”}}%
\end{biblechapter}%
\begin{biblechapter}% Mark 13
\verseWithHeading{The Destruction of the Temple Predicted}{And as\lebnote{“\textit{as}” is supplied as a component of the temporal genitive absolute participle (“was going out”)} he was going out of the temple courts,\lebnote{*Here “\textit{courts}” is supplied to distinguish this area from the interior of the temple building itself} one of his disciples said to him, “Teacher, look! What great stones and what wonderful buildings!”}%
\verse{And Jesus said to him, \JesusWords{“Do you see these great buildings? Not one stone will be left here on another stone that will not be thrown down!”}}%
\verseWithHeading{Signs of the End of the Age}{And as\lebnote{“\textit{as}” is supplied as a component of the temporal genitive absolute participle (“was sitting”)} he was sitting on the Mount of Olives opposite the temple, Peter and James and John and Andrew asked him privately,}%
\verse{“Tell us, when will these things happen, and what will be the sign when all these things are about to be accomplished?”}%
\verse{So Jesus began to say to them, \JesusWords{“Watch out that no one deceives you!}}%
\verse{\JesusWords{Many will come in my name, saying, ‘I am he,’ and they will deceive many.}}%
\verse{\JesusWords{And when you hear about wars and rumors of wars, do not be alarmed. This must happen, but the end is not yet.}}%
\verse{\JesusWords{For nation will rise up against nation and kingdom against kingdom. There will be earthquakes in various places. There will be famines. These things are the beginning of birth pains.}}%
\verseWithHeading{Persecution of Disciples Predicted}{\JesusWords{“But you, watch out for yourselves! They will hand you over to councils and you will be beaten in the synagogues and will have to stand before governors and kings because of me, for a witness to them.}}%
\verse{\JesusWords{And the gospel must first be proclaimed to all the nations.}\lebnote{Or “Gentiles”; the same Greek word can be translated “nations” or “Gentiles” depending on the context}}%
\verse{\JesusWords{And when they arrest you and\lebnote{“\textit{and}” supplied because participle “hand you over” translated as a finite verb in keeping with English style} hand you over, do not be anxious beforehand what you should say, but whatever is given to you at that hour, say this. For you are not the ones who are speaking, but the Holy Spirit.}}%
\verse{\JesusWords{And brother will hand over brother to death, and a father his\lebnote{The word “\textit{his}” is not in the Greek text but is implied} child, and children will rise up against parents and have them put to death.}}%
\verse{\JesusWords{And you will be hated by all because of my name. But the one who endures to the end — this one will be saved.}}%
\verseWithHeading{The Abomination of Desolation}{\JesusWords{“But when you see the abomination of desolation standing where it should not be” (let the one who reads understand), “then those in Judea must flee to the mountains!}}%
\verse{\JesusWords{The one\lebnote{Some manuscripts have “And the one”} who is on his\lnDAQ{} housetop must not come down or go inside to take anything out of his house,}}%
\verse{\JesusWords{and the one who is in the field must not turn back to pick up his cloak.}}%
\verse{\JesusWords{And woe to those who are pregnant\lebnote{“who have in the womb”} and to those who are nursing their babies\lebnote{The words “\textit{their babies}” are not in the Greek text but are supplied as a necessary clarification} in those days!}}%
\verse{\JesusWords{But pray that it will not happen in winter.}}%
\verse{\JesusWords{For in those days there will be tribulation of such a kind as has not happened from the beginning of the creation that God created until now, and never will happen.}}%
\verse{\JesusWords{And if the Lord had not shortened the days, no human being would be saved.\lebnote{“every flesh would not be saved”} But for the sake of the elect, whom he chose, he has shortened the days.}}%
\verse{\JesusWords{“And at that time if anyone should say to you, “Behold, here is the Christ,’ ‘Behold, there he is,’ do not believe him!\lebnote{*supplied from English context}}}%
\verse{\JesusWords{For false messiahs and false prophets will appear, and will produce signs and wonders in order to mislead, if possible, the elect.}}%
\verse{\JesusWords{But you, watch out! I have told you everything ahead of time!}}%
\verseWithHeading{The Arrival of the Son of Man}{\JesusWords{“But in those days, after that tribulation, ‘the sun will be darkened and the moon will not give its light,}}%
\verse{\JesusWords{and the stars will be falling from heaven, and the powers in the heavens will be shaken.’}\lebnote{from Isa 13:10; 34:4}}%
\verse{\JesusWords{And then they will see the Son of Man arriving in the clouds with great power and glory.}}%
\verse{\JesusWords{And then he will send out the angels, and will gather the\lebnote{Some manuscripts have “his”} elect together from the four winds, from the end of the earth to the end of heaven.}}%
\verseWithHeading{The Parable of the Fig Tree}{\JesusWords{“Now learn the parable from the fig tree: Whenever its branch has already become tender and puts forth its\lnDAQ{} leaves, you know that summer is near.}}%
\verse{\JesusWords{So also you, when you see these things happening, know\lebnote{Or “you know”} that he is near, at the door.}}%
\verse{\JesusWords{Truly I say to you that this generation will never pass away until all these things take place!}}%
\verse{\JesusWords{Heaven and earth will pass away, but my words will never pass away.}}%
\verseWithHeading{The Unknown Day and Hour}{\JesusWords{“But concerning that day or hour no one knows — not even the angels in heaven nor the Son — except the Father.}}%
\verse{\JesusWords{Watch out! Be alert, because you do not know when the time is!}}%
\verse{\JesusWords{It is like a man away on a journey, who left his house and gave his slaves authority — to each one his work — and to the doorkeeper he gave orders that he should be on the alert.}}%
\verse{\JesusWords{Therefore be on the alert, for you do not know when the master of the house is coming — whether in the evening, or at midnight, or when the rooster crows, or early in the morning —}}%
\verse{\JesusWords{lest he arrive suddenly and\lebnote{“\textit{and}” supplied because previous participle “arrive” translated as a finite verb} find you sleeping.}}%
\verse{\JesusWords{And what I say to you, I say to everyone: Be on the alert!”}}%
\end{biblechapter}%
\begin{biblechapter}% Mark 14
\verseWithHeading{The Chief Priests and Scribes Plot to Kill Jesus}{Now after two days it was the Passover and the feast of Unleavened Bread, and the chief priests and the scribes were seeking how, after\lebnote{“\textit{after}” is supplied as a component of the participle (“arresting”) which is understood as temporal} arresting him by stealth, they could kill him.\lnDAR{}}%
\verse{For they said, “Not at the feast, lest there be an uproar by the people.”}%
\verseWithHeading{Jesus’ Anointing at Bethany}{And while\lnDAS{} he was at Bethany in the house of Simon the leper, as\lebnote{“\textit{as}” is supplied as a component of the temporal genitive absolute participle (“was reclining for a meal”)} he was reclining for a meal, a woman came holding an alabaster flask of very costly perfumed oil of genuine nard. After\lebnote{“\textit{after}” is supplied as a component of the participle (“breaking”) which is understood as temporal} breaking the alabaster flask, she poured it\lnDAT{} out on his head.}%
\verse{But some were expressing indignation to one another:\lebnote{Or perhaps “within themselves”} “Why has there been this waste of perfumed oil?}%
\verse{For this perfumed oil could have been sold for more than three hundred denarii and given to the poor!” And they began to scold\lebnote{Imperfect tense as ingressive (“began to scold”)} her.}%
\verse{But Jesus said, \JesusWords{“Leave her alone. Why do you cause trouble for her? She has done a good deed to me.}}%
\verse{\JesusWords{For the poor you always have with you, and you can do good for them whenever you want, but you do not always have me.}}%
\verse{\JesusWords{She has done what she could; she has anointed my body beforehand\lebnote{“she has anticipated to anoint my body”} for burial.}}%
\verse{\JesusWords{And truly I say to you, wherever the gospel is proclaimed in the whole world, what she has done will also be told in memory of her.}}%
\verseWithHeading{Judas Arranges to Betray Jesus}{And Judas Iscariot, who was one of the twelve, went to the chief priests in order to betray him to them.}%
\verse{And when\lebnote{“\textit{when}” is supplied as a component of the participle (“heard”) which is understood as temporal} they heard this,\lnDAR{} they were delighted, and promised to give him money. And he began seeking\lebnote{Imperfect tense as ingressive (“began seeking”)} how he could betray him conveniently.}%
\verseWithHeading{Jesus’ Final Passover with the Disciples}{And on the first day of the feast of Unleavened Bread, when they sacrificed the Passover lamb, his disciples said to him, “Where do you want us to go and\lebnote{“and” supplied because previous participle “go” translated as an English infinitive} prepare, so that you can eat the Passover?”}%
\verse{And he sent two of his disciples and said to them, \JesusWords{“Go into the city and a man carrying a jar of water will meet you. Follow him,}}%
\verse{\JesusWords{and wherever he enters, say to the master of the house, ‘The Teacher says, “Where is my guest room where I may eat the Passover with my disciples?”’}}%
\verse{\JesusWords{And he will show you a large upstairs room furnished\lebnote{Or perhaps “paved” or “panelled”} and\lnDAU{} ready, and prepare for us there.”}}%
\verse{And the disciples went out and came into the city and found everything\lnDAT{} just as he had told them, and they prepared the Passover.}%
\verse{And when it\lebnote{“\textit{when}” is supplied as a component of the temporal genitive absolute participle (“was”)} was evening, he arrived with the twelve.}%
\verse{And while\lebnote{“\textit{while}” is supplied as a component of the temporal genitive absolute participle (“were reclining at table”)} they were reclining at table and eating, Jesus said, \JesusWords{“Truly I say to you, that one of you who is eating with me will betray me.”}}%
\verse{They began to be distressed and to say to him one by one, “Surely not I?”\lebnote{The negative construction in Greek anticipates a negative answer here}}%
\verse{But he said to them, \JesusWords{“It is one of the twelve — the one who is dipping bread\lnDAT{} into the bowl with me.}}%
\verse{\JesusWords{For the Son of Man is going just as it is written about him, but woe to that man by whom the Son of Man is betrayed! It would be better for him if that man had not been born.”}}%
\verseWithHeading{The Lord’s Supper}{And while\lebnote{“\textit{while}” is supplied as a component of the temporal genitive absolute participle (“were eating”)} they were eating, he took bread and,\lebnote{*Here “\textit{and}” is supplied in the translation because of English style} after\lebnote{“\textit{after}” is supplied as a component of the participle (“giving thanks”) which is understood as temporal} giving thanks, he broke it\lnDAT{} and gave it\lnDAT{} to them and said, \JesusWords{“Take it,\lnDAR{} this is my body.”}}%
\verse{And after\lebnote{“\textit{after}” is supplied as a component of the participle (“taking”) which is understood as temporal} taking the cup and\lnDAU{} giving thanks, he gave it\lnDAT{} to them, and they all drank from it.}%
\verse{And he said to them, \JesusWords{“This is my blood of the covenant which is poured out for many.}}%
\verse{\JesusWords{Truly I say to you that I will never drink of the fruit of the vine any longer until that day when I drink it new in the kingdom of God.”}}%
\verse{And after they\lebnote{“\textit{after}” is supplied as a component of the participle (“had sung the hymn”) which is understood as temporal} had sung the hymn, they went out to the Mount of Olives.}%
\verseWithHeading{Jesus Predicts Peter’s Denial}{And Jesus said to them, \JesusWords{“You will all fall away, because it is written, ‘I will strike the shepherd and the sheep will be scattered.’}\lebnote{from Zech 13:7}}%
\verse{\JesusWords{But after I am raised, I will go ahead of you into Galilee.”}}%
\verse{But Peter said to him, “Even if they all fall away, certainly I will not!”}%
\verse{And Jesus said to him, \JesusWords{“Truly I say to you that today — this night — before the rooster crows twice, you will deny me three times!”}}%
\verse{But he kept saying emphatically, “If it is necessary for me to die with you, I will never deny you!” And they all were saying the same thing also.}%
\verseWithHeading{The Prayer in Gethsemane}{And they came to a place named\lebnote{“the name of which”} Gethsemane, and he said to his disciples, \JesusWords{“Sit here while I pray.”}}%
\verse{And he took along Peter and James and John with him, and he began to be distressed and troubled.}%
\verse{And he said to them, \JesusWords{“My soul is deeply grieved, to the point of death. Remain here and stay awake.”}}%
\verse{And going forward a little he fell to the ground and began to pray\lebnote{Imperfect tense as ingressive (“began to pray”)} that, if it were possible, the hour would pass from him.}%
\verse{And he said, \JesusWords{“Abba,\lebnote{The word “Abba” means “father” in Aramaic} Father, all things are possible for you! Take away this cup from me! Yet not what I will, but what you will.”}\lebnote{*Here the verb “\textit{will}” is an understood repetition of the verb earlier in this verse}}%
\verse{And he came and found them sleeping, and he said to Peter, \JesusWords{“Simon, are you sleeping? Were you not able to stay awake one hour?}}%
\verse{\JesusWords{Stay awake and pray that you will not enter into temptation. The spirit is willing, but the flesh is weak!”}}%
\verse{And again he went away and\lebnote{“\textit{and}” supplied because previous participle “went away” translated as a finite verb} prayed, saying the same thing.}%
\verse{And again he came and\lebnote{“\textit{and}” supplied because previous participle “came” translated as a finite verb} found them sleeping, for they could not keep their eyes open,\lebnote{“for their eyes were weighed down”} and they did not know what to reply to him.}%
\verse{And he came the third time and said to them, \JesusWords{“Are you still sleeping and resting? It is enough! The hour has come. Behold, the Son of Man is being betrayed into the hands of sinners.}}%
\verse{\JesusWords{Get up, let us go! Behold, the one who is betraying me is approaching!”}}%
\verseWithHeading{The Betrayal and Arrest of Jesus}{And immediately, while\lebnote{“\textit{while}” is supplied as a component of the temporal genitive absolute participle (“was … speaking”)} he was still speaking, Judas — one of the twelve — arrived, and with him a crowd with swords and clubs, from the chief priests and the scribes and the elders.}%
\verse{Now the one who was betraying him had given them a sign, saying, “The one whom I kiss — he is the one.\lebnote{*Here the predicate nominative (“\textit{the one}”) is implied} Arrest him and lead him\lnDAT{} away under guard!”}%
\verse{And when he\lebnote{“\textit{when}” is supplied as a component of the participle (“arrived”) which is understood as temporal} arrived, he came up to him immediately and\lebnote{“\textit{and}” supplied because previous participle “came up” translated as a finite verb} said, “Rabbi,” and kissed him.}%
\verse{So they laid hands on him and arrested him.}%
\verse{But a certain one of the bystanders, drawing his\lnDAV{} sword, struck the slave of the high priest and cut off his ear.}%
\verse{And Jesus answered and\lebnote{“\textit{and}” supplied because previous participle “answered” translated as a finite verb} said to them, \JesusWords{“Have you come out with swords and clubs, as against a robber, to arrest me?}}%
\verse{\JesusWords{Every day I was with you in the temple courts\lebnote{“\textit{courts}” is supplied to distinguish this area from the interior of the temple building itself} teaching, and you did not arrest me! But this has happened\lebnote{The phrase “\textit{this has happened}” is not in the Greek text, but is understood and must be supplied in the translation because of English style; cf. the parallel in Matt 26:56} in order that the scriptures would be fulfilled.}}%
\verse{And they all abandoned him and\lebnote{“\textit{and}” supplied because previous participle “abandoned” translated as a finite verb} fled.}%
\verse{And a certain young man was following him, clothed only in a linen cloth on his naked body. And they attempted to seize\lebnote{the present tense is translated as a conative present (“attempted to”)} him,}%
\verse{but he left behind the linen cloth and\lebnote{“\textit{and}” supplied because previous participle “left behind” translated as a finite verb} fled naked.}%
\verseWithHeading{Jesus Before the Sanhedrin}{And they led Jesus away to the high priest, and all the chief priests and the elders and the scribes came together.}%
\verse{And Peter followed him from a distance, right inside, into the courtyard of the high priest. And he was sitting with the officers and warming himself by the fire.}%
\verse{Now the chief priests and the whole Sanhedrin were looking for testimony against Jesus in order to put him to death, and they did not find it.\lnDAR{}}%
\verse{For many gave false testimony against him, and their\lnDAV{} testimony was not consistent.}%
\verse{And some stood up and\lnDAW{} began to give false testimony\lebnote{Imperfect tense as ingressive (“began to give false testimony”)} against him, saying,}%
\verse{“We heard him saying, \JesusWords{‘I will destroy this temple made by hands, and within three days I will build another not made by hands.”}}%
\verse{And their testimony was not even consistent about this.}%
\verse{And the high priest stood up in the midst of them and\lnDAW{} asked Jesus, saying, “Do you not reply anything? What are these people testifying against you?”}%
\verse{But he was silent and did not reply anything. Again the high priest asked him and said to him, “Are you the Christ, the Son of the Blessed One?”}%
\verse{And Jesus said, \JesusWords{“I am, and you will see the Son of Man sitting at the right hand of the Power\lebnote{An indirect way of referring to God} and coming with the clouds of heaven.”}}%
\verse{And the high priest tore his clothes and\lebnote{“\textit{and}” supplied because previous participle “tore” translated as a finite verb} said, “What further need do we have of witnesses?}%
\verse{You have heard the blasphemy! What do you think?”\lebnote{“does it seem to you”} And they all condemned him as deserving death.\lebnote{“to be deserving of death”}}%
\verse{And some began to spit on him and to cover his face and to strike him with their fists, and to say to him “Prophesy!” And the officers received him with slaps in the face.\lebnote{Or “with blows” (either meaning is possible here)}}%
\verseWithHeading{Peter Denies Jesus Three Times}{And while\lnDAS{} Peter was below in the courtyard, one of the female slaves of the high priest came up}%
\verse{And when\lnDAX{} she saw Peter warming himself, she looked intently at him and\lebnote{“\textit{and}” supplied because previous participle “looked intently at” translated as a finite verb} said, “You also were with the Nazarene, Jesus.”}%
\verse{But he denied it,\lnDAR{} saying, “I neither know nor understand what you mean!” And he went out into the gateway, and a rooster crowed.\lebnote{Several important and early manuscripts lack the words “and a rooster crowed”}}%
\verse{And the female slave, when she\lnDAX{} saw him, began to say again to the bystanders, “This man is one of them!”}%
\verse{But he denied it\lnDAT{} again. And after a little while, again the bystanders began to say\lebnote{Imperfect tense as ingressive (“began to say”)} to Peter, “You really are one of them, because you also are a Galilean, and your accent shows it!”\lebnote{“is like”}\lebnote{Some manuscripts omit “and your accent shows it”}}%
\verse{And he began to curse and to swear with an oath, “I do not know this man whom you are talking about!”}%
\verse{And immediately a rooster crowed for the second time. And Peter remembered the statement, how Jesus had said to him, \JesusWords{“Before the rooster crows twice, you will deny me three times,”} and throwing himself down, he began to weep.\lebnote{Imperfect tense as ingressive (“began to weep”)}}%
\end{biblechapter}%
\begin{biblechapter}% Mark 15
\verseWithHeading{Jesus Taken to Pilate}{And as soon as morning came, after\lebnote{“\textit{after}” is supplied as a component of the participle (“formulating”) which is understood as temporal} formulating a plan, the chief priests, with the elders and scribes and the whole Sanhedrin, tied up Jesus, led him\lnDAY{} away, and handed him\lnDAY{} over to Pilate.}%
\verse{And Pilate asked him, “Are you the king of the Jews?” And he answered him and\lnDAZ{} said, \JesusWords{“You say so.”}}%
\verse{And the chief priests began to accuse\lebnote{Imperfect tense as ingressive (“began to accuse”)} him of many things.}%
\verse{So Pilate asked him again, saying, “Do you not answer anything? See how many charges\lebnote{The word “charges” is not in the Greek text but is implied} they are bringing against you!”}%
\verse{But Jesus did not answer anything further, so that Pilate was astonished.}%
\verseWithHeading{Pilate Releases Barabbas}{Now at each feast he customarily released\lebnote{The imperfect tense has been translated as customary here (“customarily released”)} for them one prisoner whom they requested.}%
\verse{And the one named Barabbas\lebnote{“Barabbas” means “son of the father” in Aramaic} was imprisoned with the rebels who had committed murder in the rebellion.}%
\verse{And the crowd came up and\lebnote{“\textit{and}” supplied because previous participle “came up” translated as a finite verb} began to ask him to do as he customarily did\lebnote{The imperfect tense has been translated as customary here (“customarily did”)} for them.}%
\verse{So Pilate answered them, saying, “Do you want me to release for you the king of the Jews?”}%
\verse{(For he realized that the chief priests had handed him over because of envy.)}%
\verse{But the chief priests incited the crowd so that he would release for them Barabbas\lnDAZ{} instead.}%
\verse{So Pilate answered and said to them again, “Then what do you want me to do with the one whom you call the king of the Jews?”}%
\verse{And they shouted again, “Crucify him!”}%
\verse{And Pilate said to them, “Why? What evil has he done?” But they shouted even louder, “Crucify him!”}%
\verse{So Pilate, because he\lebnote{“\textit{because}” is supplied as a component of the participle (“wanted”) which is understood as causal} wanted to satisfy\lebnote{“to make sufficient”} the crowd, released for them Barabbas. And after\lebnote{“\textit{after}” is supplied as a component of the participle (“flogged”) which is understood as temporal} he had Jesus flogged, he handed him\lnDAY{} over so that he could be crucified.}%
\verseWithHeading{Jesus Is Mocked}{So the soldiers led him away into the palace (that is, the governor’s residence) and called together the whole cohort.}%
\verse{And they put a purple cloak on him, and after\lebnote{“\textit{after}” is supplied as a component of the participle (“weaving”) which is understood as temporal} weaving a crown of thorns they placed it\lnDAY{} on him.}%
\verse{And they began to greet him, “Hail, king of the Jews!”}%
\verse{And they repeatedly struck\lebnote{The imperfect tense has been translated as iterative here (“repeatedly struck”)} him on the head with a reed, and were spitting on him, and they knelt down\lebnote{“bending the knees”} and\lebnote{“\textit{and}” supplied because previous participle “knelt down” translated as a finite verb} did obeisance to him.}%
\verse{And when they had mocked him, they stripped him of the purple cloak and put his own clothes\lebnote{Some manuscripts have “his clothes” in place of “his own clothes”} on him, and they led him out so that they could crucify him.}%
\verseWithHeading{Jesus Is Crucified}{And they forced a certain man who was passing by, Simon of Cyrene (the father of Alexander and Rufus), who was coming from the country, to carry his cross.}%
\verse{And they brought him to the place Golgotha (which is translated “Place of a Skull”).}%
\verse{And they attempted to give\lebnote{the imperfect tense is translated as a conative imperfect (“attempted to give”)} him wine mixed with myrrh, but he did not take it.}%
\verse{And they crucified him and divided his clothes among themselves\lebnote{“among themselves” reflects the middle voice of the verb “divided”} by\lebnote{“\textit{by}” is supplied as a component of the participle (“casting”) which is understood as means} casting lots for them to see who should take what.}%
\verse{Now it was the third hour when they crucified him.}%
\verse{And the inscription of the charge against him was written, “The king of the Jews.”}%
\verse{And with him they crucified two robbers, one on his right and one on his left.\lebnote{Most later Greek manuscripts add v. 28 (a quotation from Isa 53:12) after v. 27, “And the scripture was fulfilled that says, ‘And he was counted with the lawless ones’”}}%
\verse{}%
\verse{And those who passed by reviled him, shaking their heads and saying, “Aha! The one who would destroy the temple and rebuild it\lnDAY{} in three days,}%
\verse{save yourself by\lebnote{“\textit{by}” is supplied as a component of the participle (“coming down”) which is understood as means} coming down from the cross!”}%
\verse{In the same way also the chief priests, along with the scribes, were mocking him\lnDAY{} to one another, saying, “He saved others; he is not able to save himself!}%
\verse{Let the Christ, the king of Israel, come down now from the cross, so that we may see and believe! Even those who were crucified with him were reviling him.}%
\verseWithHeading{Jesus Dies on the Cross}{And when\lebnote{“\textit{when}” is supplied as a component of the temporal genitive absolute participle (“came”)} the sixth hour came, darkness came over the whole land until the ninth hour.}%
\verse{And at the ninth hour Jesus cried out with a loud voice, \JesusWords{“Eloi, Eloi, lema sabachthani?”} (which is translated, \JesusWords{“My God, my God, why\lebnote{“for what \textit{reason}”} have you forsaken me?”})\lebnote{from Ps 22:1}}%
\verse{And some of the bystanders, when they\lebnote{“\textit{when}” is supplied as a component of the participle (“heard”) which is understood as temporal} heard it,\lebnote{*supplied from English context} said, “Behold, he is summoning Elijah!”}%
\verse{And someone ran and filled a sponge with sour wine, put it\lnDAY{} on a reed, and\lebnote{“\textit{and}” supplied because two previous participles “ran” and “filled” translated as finite verbs} gave it\lnDAY{} to him to drink, saying, “Leave him\lnDAY{} alone! Let us see if Elijah is coming to take him down.”}%
\verse{But Jesus uttered a loud cry and\lebnote{“\textit{and}” supplied because previous participle “uttered” translated as a finite verb} expired.}%
\verse{And the curtain of the temple was torn in two from top to bottom.}%
\verse{And when\lebnote{“\textit{when}” is supplied as a component of the participle (“saw”) which is understood as temporal} the centurion who was standing opposite him saw that he expired like this, he said, “Truly this man was God’s Son!”}%
\verse{And there were also women observing from a distance, among whom were Mary Magdalene, and Mary the mother of James the younger\lebnote{Or perhaps “the short,” referring to stature} and Joses,\lebnote{This name appears in Matt 27:56 as “Joseph”} and Salome,}%
\verse{who used to follow\lebnote{The imperfect tense has been translated as customary here (“used to follow”)} him and serve him when he was in Galilee, and many other women who went up with him to Jerusalem.}%
\verseWithHeading{Jesus Is Buried}{And when it\lebnote{“\textit{when}” is supplied as a component of the temporal genitive absolute participle (“was”)} was already evening, since it was the day of preparation (that is, the day before the Sabbath),}%
\verse{Joseph of Arimathea, a prominent member of the council who was also himself looking forward to\lebnote{Or “waiting for”} the kingdom of God, came acting courageously and\lebnote{“\textit{and}” supplied because previous participle “came” translated as a finite verb} went in to Pilate and asked for the body of Jesus.}%
\verse{And Pilate was surprised that he was already dead, and summoning the centurion, asked him whether he had died already.}%
\verse{And when he\lebnote{“\textit{when}” is supplied as a component of the participle (“learned of”) which is understood as temporal} learned of it\lnDAY{} from the centurion, he granted the corpse to Joseph.}%
\verse{And after\lebnote{“\textit{after}” is supplied as a component of the participle (“purchasing”) which is understood as temporal} purchasing a linen cloth and\lebnote{“\textit{and}” supplied because participle “taking...down” translated as a finite verb in keeping with English style} taking him down, he wrapped him\lnDAY{} in the linen cloth and placed him in a tomb that had been cut from the rock. And he rolled a stone over the entrance of the tomb.}%
\verse{Now Mary Magdalene and Mary the mother of Joses saw where he was placed.}%
\end{biblechapter}%
\begin{biblechapter}% Mark 16
\verseWithHeading{Jesus Is Raised}{And when\lebnote{“\textit{when}” is supplied as a component of the temporal genitive absolute participle (“was over”)} the Sabbath was over, Mary Magdalene, and Mary the mother of James, and Salome purchased fragrant spices so that they could go and\lnDBA{} anoint him.}%
\verse{And very early in the morning on the first day of the week they came to the tomb after\lebnote{“\textit{after}” is supplied as a component of the temporal genitive absolute participle (“had risen”)} the sun had risen.}%
\verse{And they were saying to one another, “Who will roll away the stone for us from the entrance of the tomb?”}%
\verse{And when they\lebnote{“\textit{when}” is supplied as a component of the participle (“looked up”) which is understood as temporal} looked up, they saw that the stone had been rolled away (for it was very large).}%
\verse{And as they\lebnote{“\textit{as}” is supplied as a component of the participle (“going”) which is understood as temporal} were going into the tomb, they saw a young man dressed in a white robe sitting on the right side, and they were alarmed.}%
\verse{But he said to them, “Do not be alarmed. You are looking for Jesus the Nazarene who was crucified. He has been raised, he is not here! See the place where they laid him!}%
\verse{But go, tell his disciples and Peter that he is going ahead of you to Galilee. You will see him there, just as he told you.”}%
\verse{And they went out and\lnDBB{} fled from the tomb, because trembling and amazement had seized them. And they said nothing to anyone, because they were afraid.\lebnote{The Gospel of Mark ends at this point in some manuscripts, including two of the most important ones, while other manuscripts supply a shorter ending (sometimes included as part of v. 8), others supply the traditional longer ending (vv. 9–20), and still other manuscripts supply both the shorter ending and vv. 9–20; due to significant questions about the authenticity of these alternative endings, many scholars regard 16:8 as the last verse of the Gospel of Mark}\innerVerseHeading{The Shorter Ending of Mark}So they promptly reported all the things they had been commanded to those around Peter. And after these things, Jesus himself also sent out through them from the east even as far as the west the holy and imperishable proclamation of eternal salvation. Amen.}%
\verseWithHeading{The Longer Ending of Mark}{Now early on the first day of the week, after he\lebnote{“\textit{after}” is supplied as a component of the participle (“rose”) which is understood as temporal} rose, he appeared first to Mary Magdalene, from whom he had expelled seven demons.}%
\verse{She went out and\lnDBB{} announced it\lnDBC{} to those who were with him while they\lebnote{“\textit{while}” is supplied as a component of the participle (“were”) which is understood as temporal} were mourning and weeping.}%
\verse{And those, when they\lebnote{“\textit{when}” is supplied as a component of the participle (“heard”) which is understood as temporal} heard that he was alive and had been seen by her, refused to believe it.\lebnote{*supplied from English context}}%
\verse{And after these things, he appeared in a different form to two of them as they\lebnote{“\textit{as}” is supplied as a component of the participle (“were walking”) which is understood as temporal} were walking, while they\lebnote{“\textit{while}” is supplied as a component of the participle (“were going out”) which is understood as temporal} were going out into the countryside.}%
\verse{And these went and\lebnote{“\textit{and}” supplied because previous participle “went” translated as a finite verb} reported it\lnDBC{} to the others, and they did not believe them.}%
\verse{And later, while\lebnote{“\textit{while}” is supplied as a component of the participle (“were reclining at table”) which is understood as temporal} they were reclining at table, he appeared to the eleven. And he reprimanded their unbelief and hardness of heart, because they did not believe those who had seen him after he\lebnote{“\textit{after}” is supplied as a component of the participle (“had been raised”) which is understood as temporal} had been raised.}%
\verse{And he said to them, \JesusWords{“Go\lebnote{As a participle of attendant circumstance this participle carries imperatival force picked up from the main verb (“preach”)} into all the world and\lnDBA{} preach the gospel to all creation.}}%
\verse{\JesusWords{The one who believes and is baptized will be saved, but the one who refuses to believe will be condemned.}}%
\verse{\JesusWords{And these signs will accompany those who believe: in my name they will expel demons, they will speak in new tongues,}}%
\verse{\JesusWords{they will pick up\lebnote{Some manuscripts have “and they will pick up”} snakes.\lebnote{Some manuscripts add “with their hands”} And if they drink any deadly poison it will never hurt them; they will lay hands on the sick and they will get\lebnote{“they will have”} well.”}}%
\verse{Then the Lord Jesus, after he had spoken to them, was taken up into heaven and sat down at the right hand of God.}%
\verse{And they went out and\lnDBB{} proclaimed everywhere, while\lebnote{“\textit{while}” is supplied as a component of the temporal genitive absolute participle (“was working together with”)} the Lord was working together with them\lnDBC{} and confirming the message through the accompanying signs.}%
\end{biblechapter}%
\flushcolsend
\biblebook{Luke}
\begin{biblechapter}% Luke 1
\verseWithHeading{The Preface to Luke’s Gospel}{Since many have attempted to compile an account concerning the events that have been fulfilled among us,}%
\verse{just as those who were eyewitnesses and servants of the word from the beginning passed on to us,}%
\verse{it seemed best to me also — because I\lebnote{“\textit{because}” is supplied as a component of the participle (“have followed”) which is understood as causal} have followed all things carefully from the beginning — to write them\lnDBD{} down in orderly sequence for you, most excellent Theophilus,}%
\verse{so that you may know the certainty concerning the things about which you were taught.}%
\verseWithHeading{The Prediction of John the Baptist’s Birth}{It happened that in the days of Herod, king of Judea, there was a certain priest, Zechariah by name, of the division of Abijah. And he had a wife\lebnote{“a wife to him”} from the daughters of Aaron, and her name was Elizabeth.}%
\verse{And they were both righteous in the sight of God, living blamelessly in all the commandments and regulations of the Lord.}%
\verse{And they did not have\lebnote{“there was not to them”} a child, because Elizabeth was barren. And they were both advanced in years.\lebnote{“in their days”}}%
\verse{And it happened that while\lebnote{“\textit{while}” is supplied as a component of the temporal infinitive (“was serving as priest”)} he was serving as priest before God in the order of his division,}%
\verse{according to the custom of the priesthood he was chosen by lot to enter into the temple of the Lord to burn incense.}%
\verse{And the whole crowd of the people were praying outside at the hour of the incense offering.}%
\verse{And an angel of the Lord appeared to him, standing at the right side of the altar of incense.}%
\verse{And Zechariah was terrified when he\lebnote{“\textit{when}” is supplied as a component of the participle (“saw”) which is understood as temporal} saw the angel,\lnDBE{} and fear fell upon him.}%
\verse{But the angel said to him, “Do not be afraid, Zechariah, because your prayer has been heard, and your wife Elizabeth will bear you a son, and you will call his name John.}%
\verse{And you will experience joy and exultation,\lebnote{“joy and exultation will be to you”} and many will rejoice at his birth.}%
\verse{For he will be great in the sight of the Lord, and he must never drink wine or beer, and he will be filled with the Holy Spirit while he is\lebnote{the phrase “while he is,” including the verb, is understood in Greek and is supplied in the translation} still in his mother’s womb.}%
\verse{And he will turn many of the sons of Israel to the Lord their God.}%
\verse{And he will go on before him in the spirit and power of Elijah, to turn the hearts of the fathers to the children, and the disobedient to the wisdom of the righteous, to prepare for the Lord a people made ready.”}%
\verse{And Zechariah said to the angel, “By what will I know this? For I am an old man, and my wife is advanced in years!”\lebnote{“in her days”}}%
\verse{And the angel answered and\lnDBF{} said to him, “I am Gabriel, who stands in the presence of God, and I was sent to speak to you and to announce to you this good news.}%
\verse{And behold, you will be silent and not able to speak until the day these things take place, because\lebnote{“in return for which”} you did not believe my words, which will be fulfilled in their time.”}%
\verse{And the people were waiting for Zechariah, and began to wonder\lebnote{Imperfect tense as ingressive (“began to wonder”)} when\lebnote{“\textit{when}” is supplied as a component of the temporal infinitive (“was delayed”)} he was delayed in the temple.}%
\verse{And when he\lebnote{“\textit{when}” is supplied as a component of the participle (“came out”) which is understood as temporal} came out he was not able to speak to them, and they realized that he had seen a vision in the temple. And he kept making signs to them, and remained unable to speak.}%
\verse{And it happened that when the days of his service came to an end, he went away to his home.}%
\verse{Now after these days, his wife Elizabeth conceived, and she kept herself in seclusion for five months, saying,}%
\verse{“Thus the Lord has done for me in the days in which he has concerned himself with me,\lnDBE{} to take away my disgrace among people.”}%
\verseWithHeading{The Prediction of Jesus’ Birth}{Now in the sixth month, the angel Gabriel was sent from God to a town of Galilee named\lebnote{“to which the name”} Nazareth,}%
\verse{to a virgin legally promised in marriage to a man named\lebnote{“to whom the name”} Joseph of the house of David. And the name of the virgin was Mary.}%
\verse{And he came to her and\lebnote{“\textit{and}” supplied because previous participle “came” translated as a finite verb} said, “Greetings, favored one! The Lord is with you.”}%
\verse{But she was greatly perplexed at the statement, and was pondering what sort of greeting this might be.}%
\verse{And the angel said to her, “Do not be afraid, Mary, for you have found favor with God.}%
\verse{And behold, you will conceive in the womb and will give birth to a son, and you will call his name Jesus.}%
\verse{This one will be great, and he will be called the Son of the Most High, and the Lord God will give him the throne of his father David.}%
\verse{And he will reign over the house of Jacob forever,\lebnote{“for the ages”} and of his kingdom there will be no end.}%
\verse{And Mary said to the angel, “How will this be, since I have not had sexual relations with a man?”}%
\verse{And the angel answered and\lnDBF{} said to her, “The Holy Spirit will come upon you, and the power of the Most High will overshadow you. Therefore also the one to be born will be called holy, the Son of God.}%
\verse{And behold, your relative Elizabeth — she also has conceived a son in her old age, and this is the sixth month for her who was called barren.}%
\verse{For nothing will be impossible with God.”\lebnote{“every thing will not be impossible with God”}}%
\verse{So Mary said, “Behold, the Lord’s female slave! May it happen to me according to your word.” And the angel departed from her.}%
\verseWithHeading{Mary Visits Elizabeth}{Now in those days Mary set out and\lebnote{“\textit{and}” supplied because previous participle “set out” translated as a finite verb} traveled with haste into the hill country, to a town of Judah,}%
\verse{and entered into the house of Zechariah, and greeted Elizabeth.}%
\verse{And it happened that when Elizabeth heard the greeting of Mary, the baby in her womb leaped and Elizabeth was filled with the Holy Spirit.}%
\verse{And she cried out with a loud shout and said, “Blessed are you among women, and blessed is the fruit of your womb!}%
\verse{And why is this granted to me, that the mother of my Lord should come to me?}%
\verse{For behold, when the sound of your greeting came to my ears, the baby in my womb leaped for joy!}%
\verse{And blessed is she who believed that there will be a fulfillment to what was spoken to her from the Lord!”}%
\verseWithHeading{Mary’s Hymn of Praise to God}{And Mary said, “My soul exalts the Lord,}%
\verse{and my spirit has rejoiced greatly in God my Savior,}%
\verse{because he has looked upon the humble state of his female slave, for behold, from now on all generations will consider me blessed,}%
\verse{because the Mighty One has done great things for me, and holy is his name.}%
\verse{And his mercy is for generation after generation to those who fear him.}%
\verse{He has done a mighty deed with his arm; he has dispersed the proud in the thoughts of their hearts.}%
\verse{He has brought down rulers from their thrones, and has exalted the lowly.}%
\verse{He has filled those who are hungry with good things, and those who are rich he has sent away empty-handed.}%
\verse{He has helped Israel his servant, remembering his mercy,}%
\verse{just as he spoke to our fathers, to Abraham and to his descendants forever.”\lebnote{“for the age”}}%
\verse{And Mary stayed with her about three months, and returned to her home.}%
\verseWithHeading{The Birth of John the Baptist}{Now the time came for Elizabeth that she should give birth, and she gave birth to a son.}%
\verse{And her neighbors and relatives heard that the Lord had shown his great mercy to her,\lebnote{“the Lord had made great his mercy with her”} and they rejoiced with her.}%
\verse{And it happened that on the eighth day they came to circumcise the child, and they were wanting to name him after\lebnote{“in the name of”} his father Zechariah.}%
\verse{And his mother answered and\lnDBF{} said, “No, but he will be named John.”}%
\verse{And they said to her, “There is no one of your relatives who is called by this name.”}%
\verse{So they made signs to his father asking what he wanted him to be named,}%
\verse{and he asked for a writing tablet and\lebnote{“\textit{and}” supplied because previous participle “asked for” translated as a finite verb} wrote, saying, “John is his name.” And they were all astonished.}%
\verse{And his mouth and his tongue were opened immediately, and he began to speak,\lebnote{Imperfect tense (“began to speak”)} praising God.}%
\verse{And fear came on all those who lived near them, and in all the hill country of Judea all these events were discussed.}%
\verse{And all those who heard kept these things\lnDBD{} in their hearts, saying, “What then will this child be? For indeed the hand of the Lord was with him!”}%
\verseWithHeading{The Praise and Prophecy of Zechariah}{And his father Zechariah was filled with the Holy Spirit and prophesied, saying,}%
\verse{“Blessed be the Lord, the God of Israel, because he has visited to help and has redeemed\lebnote{“has done redemption for”} his people,}%
\verse{and has raised up a horn of salvation for us in the house of his servant David,}%
\verse{just as he spoke through the mouth of his holy prophets from earliest times —}%
\verse{salvation from our enemies and from the hand of all those who hate us,}%
\verse{to show mercy to our fathers and to remember his holy covenant,}%
\verse{the oath that he swore to Abraham our father, to grant us}%
\verse{that we, being rescued from the hand of our enemies, could serve him without fear}%
\verse{in holiness and righteousness before him all our days.}%
\verse{And so you, child, will be called the prophet of the Most High, for you will go on before the Lord to prepare his ways,}%
\verse{to give knowledge of salvation to his people by the forgiveness of their sins,}%
\verse{because of the merciful compassion\lebnote{Or “heart”} of our God by which the dawn will visit to help us from on high,}%
\verse{to give light to those who sit in darkness and in the shadow of death, to direct our feet into the way of peace.”}%
\verse{And the child kept growing and becoming strong in spirit, and was in the wilderness until the day of his public appearance to Israel.}%
\end{biblechapter}%
\begin{biblechapter}% Luke 2
\verseWithHeading{The Birth of Jesus Christ}{Now it happened that in those days a decree went out from Caesar\lebnote{Or “the emperor”} Augustus to register all the empire.}%
\verse{(This first registration took place when\lebnote{Or perhaps “\textit{before}”; here “\textit{when}” is supplied as a component of the temporal genitive absolute participle (“was governor”)} Quirinius was governor of Syria.)}%
\verse{And everyone went to be registered, each one to his own town.}%
\verse{So Joseph also went up from Galilee, from the town of Nazareth, to Judea, to the city of David which is called Bethlehem, because he was of the house and family line of David,}%
\verse{to be registered together with Mary, who was legally promised in marriage to him and\lebnote{“\textit{and}” is supplied in keeping with English style} was pregnant.}%
\verse{And it happened that while they were there, the time came\lebnote{“the days were completed”} for her to give birth.}%
\verse{And she gave birth to her firstborn son, and wrapped him in strips of cloth and laid him in a manger, because there was no place for them in the inn.}%
\verseWithHeading{The Shepherds and the Angels}{And there were shepherds in the same region, living out of doors and keeping watch, guarding over their flock by night.}%
\verse{And an angel of the Lord stood near them, and the glory of the Lord shone around them, and they were terribly frightened.\lebnote{“they were afraid with great fear”}}%
\verse{And the angel said to them, “Do not be afraid, for behold, I bring good news to you of great joy which will be for all the people:}%
\verse{that today a Savior, who is Christ the Lord, was born for you in the city of David.}%
\verse{And this will be the sign for you: you will find the baby wrapped in strips of cloth and lying in a manger.”}%
\verse{And suddenly there was with the angel a multitude of the heavenly army, praising God and saying,}%
\verse{“Glory to God in the highest, and on earth peace among people with whom he is pleased!”\lebnote{“of good pleasure”}}%
\verse{And it happened that when the angels had departed from them into heaven, the shepherds began to say\lebnote{Imperfect tense as ingressive (“began to say”)} to one another, “Let us go now to Bethlehem and see this thing that has happened, which the Lord has revealed to us!”}%
\verse{And they went hurrying and found both Mary and Joseph, and the baby who was lying in the manger.}%
\verse{And when they\lnDBG{} saw it,\lnDBH{} they made known the statement that had been told to them about this child.}%
\verse{And all who heard it\lnDBI{} were astonished concerning what had been said to them by the shepherds.}%
\verse{But Mary treasured up all these words, pondering them\lnDBI{} in her heart.}%
\verse{And the shepherds returned, glorifying and praising God for all that they had heard and seen, just as it had been told to them.}%
\verse{And when eight days were completed so that he could be circumcised,\lebnote{“to circumcise him”} he was named Jesus, his name that he was called by the angel before he was conceived in the womb.}%
\verseWithHeading{The Presentation of Jesus at the Temple}{And when the days of their purification were completed according to the law of Moses, they brought him up to Jerusalem to present him\lnDBI{} to the Lord}%
\verse{(just as it is written in the law of the Lord, “Every male that opens the womb will be called holy to the Lord”)\lebnote{An allusion to Exod 13:2, 12, 15}}%
\verse{and to offer a sacrifice according to what was stated in the law of the Lord, “a pair of turtledoves or two young pigeons.”\lebnote{from Lev 5:11; 12:8}}%
\verseWithHeading{The Prophecy of Simeon}{And behold, there was a man in Jerusalem whose name was\lebnote{“to whom the name”} Simeon, and this man was righteous and devout, looking forward to the consolation of Israel, and the Holy Spirit was upon him.}%
\verse{And it had been revealed to him by the Holy Spirit that he would not see death before he would see the Lord’s Christ.\lebnote{Or “Messiah”}}%
\verse{And he came in the Spirit into the temple, and when the parents brought in the child Jesus so that they could do for him according to what was customary under the law,}%
\verse{he took him in his\lnDBJ{} arms and praised God and said,}%
\verse{“Now dismiss your slave in peace, Lord, according to your word.}%
\verse{For my eyes have seen your salvation}%
\verse{that you have prepared in the presence of all the peoples,}%
\verse{a light for revelation to the Gentiles, and glory to your people Israel.”}%
\verse{And his father and mother were astonished at what was said about him.}%
\verse{And Simeon blessed them and said to his mother Mary, “Behold, this child is appointed for the fall and rise of many in Israel, and for a sign that is opposed\lebnote{Or “rejected”} —}%
\verse{and a sword will pierce your own soul also, so that the thoughts of many hearts will be revealed!”}%
\verseWithHeading{The Testimony of Anna}{And there was a prophetess, Anna the daughter of Phanuel of the tribe of Asher (she was advanced in years,\lebnote{“with many days”} having lived with her husband seven years after her marriage,\lebnote{“from her virginity”}}%
\verse{and herself as a widow up to eighty-four years)\lebnote{Or “eighty-four years as a widow”} who did not depart from the temple with fastings and prayers, serving night and day.}%
\verse{And at that same hour she approached and\lebnote{“\textit{and}” supplied because previous participle “approached” translated as a finite verb} began to give thanks\lebnote{Imperfect tense as ingressive (“began to give thanks”)} to God, and to speak about him to all those who were waiting for the redemption of Jerusalem.}%
\verse{And when they had completed everything according to the law of the Lord, they returned to Galilee, to their own town of Nazareth.}%
\verse{And the child was growing and becoming strong, filled with wisdom, and the favor of God was upon him.}%
\verseWithHeading{Jesus in the Temple at Twelve Years Old}{And his parents went every year to Jerusalem for the feast of the Passover.}%
\verse{And when he was twelve years old, they went up according to the custom of the feast.}%
\verse{And after\lebnote{“\textit{after}” is supplied as a component of the participle (“were completed”) which is understood as temporal} the days were completed, while they were returning, the boy Jesus stayed behind in Jerusalem. And his parents did not know it,\lnDBH{}}%
\verse{but believing him to be in the group of travelers, they went a day’s journey. And they began searching for\lebnote{Imperfect tense as ingressive (“began searching for”)} him among their\lnDBJ{} relatives and their\lnDBJ{} acquaintances,}%
\verse{and when they\lebnote{“\textit{when}” is supplied as a component of the participle (“find”) which is understood as temporal} did not find him,\lnDBH{} they returned to Jerusalem to search for him.}%
\verse{And it happened that after three days they found him in the temple courts,\lebnote{*Here “\textit{courts}” is supplied to distinguish this area from the interior of the temple building itself} sitting in the midst of the teachers and listening to them and asking them questions.}%
\verse{And all who heard him were amazed at his insight and his\lebnote{This is an understood repetition of “\textit{his}” due to English style} answers.}%
\verse{And when they\lnDBG{} saw him, they were astounded and his mother said to him, “Child, why have you done this to us? Look, your father and I have been searching for you anxiously!”}%
\verse{And he said to them, \JesusWords{“Why\lebnote{“what \textit{is it} that”} were you searching for me? Did you not know that it was necessary for me to be in the house\lebnote{Or “things” (= business)} of my Father?”}}%
\verse{And they did not understand the statement that he spoke to them.}%
\verse{And he went down with them and came to Nazareth, and was submitting to them. And his mother treasured all these things in her heart.}%
\verse{And Jesus was advancing in wisdom and stature and in favor with God and with people.}%
\end{biblechapter}%
\begin{biblechapter}% Luke 3
\verseWithHeading{John the Baptist Begins His Ministry}{Now in the fifteenth year of the reign of Tiberius Caesar,\lebnote{Or “the emperor Tiberius”} when Pontius Pilate was governor of Judea, and Herod was tetrarch of Galilee, and his brother Philip was tetrarch of the region of Iturea and Trachonitis, and Lysanias was tetrarch of Abilene,}%
\verse{in the time of the high priest Annas and Caiaphas, the word of God came to John the son of Zechariah in the wilderness.}%
\verse{And he went into all the surrounding region of the Jordan, preaching a baptism of repentance for the forgiveness of sins,}%
\verse{as it is written in the book of the words of the prophet Isaiah, “The voice of one crying out in the wilderness, ‘Prepare the way of the Lord, make his paths straight!}%
\verse{Every valley will be filled, and every mountain and hill will be leveled, and the crooked will become straight, and the rough road will become\lebnote{“\textit{will become}” is an implied repetition of the verb earlier in the verse} smooth,}%
\verse{and all flesh will see the salvation of God.’”\lebnote{from Isa 40:3–5}}%
\verse{Therefore he was saying to the crowds that came out to be baptized by him, “Offspring of vipers! Who warned you to flee from the coming wrath?}%
\verse{Therefore produce fruit worthy of repentance! And do not begin to say to yourselves, ‘We have Abraham as father.’ For I say to you that God is able to raise up children for Abraham from these stones!}%
\verse{And even now the ax is positioned at the root of the trees; therefore every tree not producing good fruit is cut down and thrown into the fire.”}%
\verse{And the crowds were asking him, saying, “What then should we do?”}%
\verse{And he answered and\lebnote{“\textit{and}” supplied because previous participle “answered” translated as a finite verb} said to them, “The one who has two tunics must share with the one who does not have one,\lnDBK{} and the one who has food must do likewise.”}%
\verse{And tax collectors also came to be baptized, and they said to him, “Teacher, what should we do?”}%
\verse{And he said to them, “Collect no more than what you are ordered to.”\lebnote{“what is ordered to you”}}%
\verse{And those who served in the army were also asking him, saying, “What should we also do?” And he said to them, “Extort from no one, and do not blackmail anyone,\lnDBK{} and be content with your pay.”}%
\verse{And while\lebnote{“\textit{while}” is supplied as a component of the temporal genitive absolute participle (“were waiting expectantly”)} the people were waiting expectantly and all were pondering in their hearts concerning John, whether perhaps he might be the Christ,\lebnote{Or “Messiah”}}%
\verse{John answered them all, saying, “I baptize you with water, but the one who is more powerful than I am is coming, of whom I am not worthy to untie the strap of his sandals. He will baptize you with the Holy Spirit and fire.}%
\verse{His winnowing shovel is in his hand, to clean out his threshing floor and to gather the wheat into his storehouse, but he will burn up the chaff with unquenchable fire.”}%
\verse{So with many other exhortations also he proclaimed good news to the people.}%
\verse{But Herod the tetrarch, who had been reproved by him concerning Herodias, his brother’s wife, and concerning all the evil deeds that Herod had done,}%
\verse{added this also to them all: he also locked up John in prison.}%
\verseWithHeading{The Baptism of Jesus}{Now it happened that when all the people were baptized, Jesus also was baptized, and while he\lebnote{“\textit{while}” is supplied as a component of the temporal genitive absolute participle (“was praying”)} was praying, heaven was opened,}%
\verse{and the Holy Spirit descended on him in bodily form like a dove, and a voice came from heaven, “You are my beloved Son; with you I am well pleased.”}%
\verseWithHeading{The Genealogy of Jesus Christ}{And Jesus, when he\lebnote{“\textit{when}” is supplied as a component of the participle (“began”) which is understood as temporal} began his ministry,\lnDBK{} was himself about thirty years old, being the son (as it was believed) of Joseph the son of Eli,}%
\verse{the son of Matthat, the son of Levi, the son of Melchi, the son of Jannai, the son of Joseph,}%
\verse{the son of Mattathias, the son of Amos, the son of Nahum, the son of Esli, the son of Naggai,}%
\verse{the son of Maath, the son of Mattathias, the son of Semein, the son of Josech, the son of Joda,}%
\verse{the son of Joanan, the son of Rhesa, the son of Zerubbabel, the son of Shealtiel, the son of Neri,}%
\verse{the son of Melchi, the son of Addi, the son of Cosam, the son of Elmadam, the son of Er,}%
\verse{the son of Joshua, the son of Eliezer, the son of Jorim, the son of Matthat, the son of Levi,}%
\verse{the son of Simeon, the son of Judah, the son of Joseph, the son of Jonam, the son of Eliakim,}%
\verse{the son of Melea, the son of Menna, the son of Mattatha, the son of Nathan, the son of David,}%
\verse{the son of Jesse, the son of Obed, the son of Boaz, the son of Sala, the son of Nahshon,}%
\verse{the son of Amminadab, the son of Admin, the son of Arni, the son of Hezron, the son of Perez, the son of Judah,}%
\verse{the son of Jacob, the son of Isaac, the son of Abraham, the son of Terah, the son of Nahor,}%
\verse{the son of Serug, the son of Reu, the son of Peleg, the son of Eber, the son of Shelah,\lebnote{Greek “Sala”}}%
\verse{the son of Cainan, the son of Arphaxad, the son of Shem, the son of Noah, the son of Lamech,}%
\verse{the son of Methuselah, the son of Enoch, the son of Jared, the son of Mahalaleel, the son of Cainan,}%
\verse{the son of Enosh, the son of Seth, the son of Adam, the son of God.}%
\end{biblechapter}%
\begin{biblechapter}% Luke 4
\verseWithHeading{The Temptation of Jesus}{And Jesus, full of the Holy Spirit, returned from the Jordan and was led by the Spirit in the wilderness}%
\verse{forty days, being tempted by the devil. And he ate nothing during those days, and when\lebnote{“\textit{when}” is supplied as a component of the temporal genitive absolute participle (“were completed”)} they were completed, he was hungry.}%
\verse{So the devil said to him, “If you are the Son of God, order this stone that it become bread!”}%
\verse{And Jesus replied to him, \JesusWords{“It is written, ‘Man will not live on bread alone.’”\lebnote{from Deut 8:3; most manuscripts add “but by every word of God” here}}}%
\verse{And he led him up and\lebnote{“\textit{and}” supplied because previous participle “led … up” translated as a finite verb} showed him all the kingdoms of the world in a moment of time.}%
\verse{And the devil said to him, “I will give you all this domain and their glory, because it has been handed over to me, and I can give it to whomever I want.}%
\verse{So if you will worship before me, all this will be yours.”}%
\verse{And Jesus answered and\lnDBL{} said to him, \JesusWords{“It is written, ‘You shall worship the Lord your God, and serve only him.’”\lebnote{from Deut 6:13}}}%
\verse{And he brought him to Jerusalem, and had him stand on the highest point of the temple and said to him, “If you are the Son of God, throw yourself down from here,}%
\verse{for it is written, ‘He will command his angels concerning you, to protect you,’\lebnote{from Ps 91:11}}%
\verse{and ‘on their hands they will lift you up, lest you strike your foot against a stone.’”\lebnote{from Ps 91:12}}%
\verse{And Jesus answered and\lnDBL{} said to him, \JesusWords{“It is said, ‘You are not to put to the test the Lord your God.’”\lebnote{from Deut 6:16}}}%
\verse{And when\lebnote{“\textit{when}” is supplied as a component of the participle (“had completed”) which is understood as temporal} the devil had completed every temptation, he departed from him until a favorable time.\lebnote{Or “for a while”}}%
\verseWithHeading{Public Ministry in Galilee}{And Jesus returned in the power of the Spirit to Galilee, and news about him went out throughout all the surrounding region.}%
\verse{And he began to teach\lebnote{Imperfect tense as ingressive (“began to teach”)} in their synagogues, and\lebnote{the participle (“was praised”) is translated as a finite verb because of English style} was praised by all.}%
\verseWithHeading{Rejected at Nazareth}{And he came to Nazareth, where he had been brought up,\lebnote{“he was having been brought up”} and according to his custom\lebnote{“what he was accustomed to for him”} he entered into the synagogue on the day of the Sabbath and stood up to read.}%
\verse{And the scroll of the prophet Isaiah was given to him, and unrolling the scroll he found the place where it was written,}%
\verse{\JesusWords{“The Spirit of the Lord is upon me, because of which he has anointed me to proclaim good news to the poor. He has sent me to proclaim release to the captives, and recovery of sight to the blind, to send out in freedom those who are oppressed,}}%
\verse{\JesusWords{to proclaim the favorable year of the Lord.”}\lebnote{from Isa 61:1–2, with one line from Isa 58:6}}%
\verse{And he rolled up the scroll and\lebnote{“\textit{and}” supplied because previous participle “rolled up” translated as a finite verb} gave it\lnDBM{} back to the attendant and\lebnote{“\textit{and}” supplied because previous participle “gave … back” translated as a finite verb} sat down. And the eyes of everyone in the synagogue were looking intently at him.}%
\verse{And he began to say to them, \JesusWords{“Today this scripture has been fulfilled in your hearing.”}}%
\verse{And they were all speaking well of him, and were astonished at the gracious words that were coming out of his mouth. And they were saying, “Is this man not the son of Joseph?”}%
\verse{And he said to them, \JesusWords{“Doubtless you will tell me this parable: ‘Physician, heal yourself!’ Whatever we have heard that took place in Capernaum, do here in your hometown also!”}}%
\verse{And he said, \JesusWords{“Truly I say to you that no prophet is acceptable in his own hometown.}}%
\verse{\JesusWords{But in truth I say to you, there were many widows in Israel in the days of Elijah, when the sky was shut for three years and six months while a great famine took place over all the land.}}%
\verse{\JesusWords{And Elijah was sent to none of them, but only to Zarephath in the region of Sidon, to a woman who was a widow.}}%
\verse{\JesusWords{And there were many lepers in Israel in the time of the prophet Elisha, and none of them was made clean except Naaman the Syrian.”}}%
\verse{And all those in the synagogue were filled with anger when they\lebnote{“\textit{when}” is supplied as a component of the participle (“heard”) which is understood as temporal} heard these things.}%
\verse{And they stood up and\lebnote{“\textit{and}” supplied because previous participle “stood up” translated as a finite verb} forced him out of the town and brought him up to the edge of the hill on which their town was built, so that they could throw him down the cliff.}%
\verse{But he passed through their midst and\lebnote{“\textit{and}” supplied because previous participle “passed” translated as a finite verb} went on his way.}%
\verseWithHeading{Jesus Teaches and Heals Many in Capernaum}{And he came down to Capernaum, a town of Galilee, and was teaching them on the Sabbath.}%
\verse{And they were astounded at his teaching, because he spoke\lebnote{“his word was”} with authority.}%
\verse{And in the synagogue there was a man who had the spirit of an unclean demon,\lebnote{Or “an unclean demonic spirit”} and he cried out with a loud voice,}%
\verse{“Ha! Leave us alone,\lebnote{“what to us and to you”} Jesus the Nazarene! Have you come to destroy us? I know who you are — the Holy One of God!”}%
\verse{And Jesus rebuked him, saying, \JesusWords{“Be silent and come out of him!”} And after\lebnote{“\textit{after}” is supplied as a component of the participle (“throwing … down”) which is understood as temporal} throwing him down in their midst, the demon came out of him without hurting him at all.}%
\verse{And amazement came upon them all, and they began to talk\lebnote{Imperfect tense as ingressive (“began to talk”)} with one another, saying, “What word\lebnote{Or “command”} is this? For he commands the unclean spirits with authority and power, and they come out!”}%
\verse{And news about him went out into every place of the surrounding region.}%
\verse{And after he\lebnote{“\textit{after}” is supplied as a component of the participle (“set out”) which is understood as temporal} set out from the synagogue, he went into Simon’s house. And Simon’s mother-in-law was afflicted with a high fever, and they asked him on behalf of her.}%
\verse{And he stood over her and\lebnote{“\textit{and}” supplied because previous participle “stood” translated as a finite verb} rebuked the fever, and it left her. And immediately she got up and\lebnote{“\textit{and}” supplied because previous participle “got up” translated as a finite verb} began to serve\lebnote{Imperfect tense as ingressive (“began to serve”)} them.}%
\verse{Now as\lebnote{“\textit{as}” is supplied as a component of the temporal genitive absolute participle (“was setting”)} the sun was setting, all who had those who were sick with various diseases brought them to him, and placing his\lebnote{“the”: the Greek article is used here as a possessive pronoun} hands on every one of them, he healed them.}%
\verse{And demons also were coming out of many, crying out and saying, “You are the Son of God!” And he rebuked them\lnDBM{} and did not permit them to speak, because they knew that he was the Christ.\lebnote{Or “Messiah”}}%
\verse{And when it\lebnote{“\textit{when}” is supplied as a component of the temporal genitive absolute participle (“was”)} was day, he departed and\lebnote{“\textit{and}” supplied because previous participle “departed” translated as a finite verb} went to an isolated place. And the crowds were seeking him, and came to him and were trying to prevent him from departing from them.}%
\verse{But he said to them, \JesusWords{“It is necessary for me to proclaim the good news of the kingdom of God to the other towns also, because I was sent for this purpose.}}%
\verse{And he was preaching in the synagogues of Judea.}%
\end{biblechapter}%
\begin{biblechapter}% Luke 5
\verseWithHeading{Jesus Calls His First Disciples}{Now it happened that while the crowd was pressing around him and hearing the word of God, he was standing beside the lake of Gennesaret,\lebnote{Another name for the Sea of Galilee}}%
\verse{and he saw two boats there beside the lake, but the fishermen had gotten out of them and\lebnote{“\textit{and}” supplied because previous participle “had gotten out” translated as a finite verb} were washing their nets.}%
\verse{And he got into one of the boats, which was Simon’s, and\lebnote{“\textit{and}” supplied because previous participle “got” translated as a finite verb} asked him to put out from the land a little. And he sat down and\lebnote{“\textit{and}” supplied because previous participle “sat down” translated as a finite verb} began to teach\lebnote{Imperfect tense as ingressive (“began to teach”)} the crowds from the boat.}%
\verse{And when he stopped speaking, he said to Simon, \JesusWords{“Put out into the deep water and let down your nets for a catch.”}}%
\verse{And Simon answered and\lnDBN{} said, “Master, although we\lebnote{“\textit{although}” is supplied as a component of the participle (“worked hard”) which is understood as concessive} worked hard through the whole night, we caught nothing. But at your word I will let down the nets.”}%
\verse{And when they\lebnote{“\textit{when}” is supplied as a component of the participle (“did”) which is understood as temporal} did this, they caught a very large number of fish, and their nets began to tear.\lebnote{Imperfect tense (“began to tear”)}}%
\verse{And they signaled to their partners in the other boat to come and\lebnote{“\textit{and}” supplied because previous participle “come” translated as a finite verb} help them, and they came and filled both the boats so that they began to sink.}%
\verse{And when he\lnDBO{} saw it,\lebnote{*supplied from English context} Simon Peter fell down at Jesus’ knees, saying, “Depart from me, Lord, because I am a sinful man!”\lebnote{“a man, a sinner”}}%
\verse{For amazement had seized him and all those who were with him at the catch of fish that they had caught,}%
\verse{and so also were James and John, the sons of Zebedee, who were business partners with Simon. And Jesus said to Simon, \JesusWords{“Do not be afraid! From now on you will be catching people!”}}%
\verse{And after they\lebnote{“\textit{after}” is supplied as a component of the participle (“brought”) which is understood as temporal} brought their\lnDBP{} boats to the land, they left everything and\lebnote{“\textit{and}” supplied because previous participle “left” translated as a finite verb} followed him.}%
\verseWithHeading{A Leper Cleansed}{And it happened that while he was in one of the towns, there was\lebnote{“behold”} a man covered with leprosy.\lebnote{“full of leprosy”} And when he\lnDBO{} saw Jesus, he fell down on his face and\lebnote{“\textit{and}” supplied because previous participle “fell down” translated as a finite verb} begged him, saying, “Lord, if you are willing, you are able to make me clean.”}%
\verse{And extending his\lnDBP{} hand he touched him, saying, \JesusWords{“I am willing; be clean.”} And immediately the leprosy went away from him.}%
\verse{And he ordered him, \JesusWords{“Tell no one, but go and\lebnote{“\textit{and}” supplied because previous participle “go” translated as a finite verb} show yourself to the priest and bring the offering\lnDBQ{} for your cleansing just as Moses commanded, for a testimony to them.}}%
\verse{But the report about him spread even more, and large crowds were gathering to hear him\lnDBQ{} and to be healed of their illnesses.}%
\verse{But he himself was withdrawing in the wilderness and praying.}%
\verseWithHeading{A Paralytic Healed}{And it happened that on one of the days as he was teaching, Pharisees and teachers of the law were sitting there who had come\lebnote{“were having come”} from every village of Galilee and Judea and from Jerusalem, and the power of the Lord was there in order for him to heal.}%
\verse{And behold, men came carrying on a stretcher a man who was paralyzed, and they were seeking to bring him in and place him\lebnote{Some manuscripts Some manuscripts include the pronoun “him” after “place”}\lebnote{Since Greek routinely omits direct objects when they are clear from context, the pronoun is not necessary here in the Greek text, but it must be supplied in the English translation} before him.}%
\verse{And when they\lebnote{“\textit{when}” is supplied as a component of the participle (“find”) which is understood as temporal} did not find a way to bring him in because of the crowd, they went up on the roof and\lebnote{“\textit{and}” supplied because previous participle “went up” translated as a finite verb} let him down through the roof tiles with the stretcher into the midst of them, in front of Jesus.}%
\verse{And when he\lnDBO{} saw their faith, he said, \JesusWords{“Friend, your sins are forgiven you.”}}%
\verse{And the scribes and the Pharisees began to reason, saying, “Who is this man who speaks blasphemies? Who is able to forgive sins except God alone?”}%
\verse{But Jesus, perceiving their thoughts, answered and\lnDBN{} said to them, \JesusWords{“Why are you reasoning in your hearts?}}%
\verse{\JesusWords{Which is easier to say, ‘Your sins are forgiven you,’ or to say, ‘Get up and walk?’}}%
\verse{But in order that you may know that the Son of Man has authority on earth to forgive sins,” he said to the one who was paralyzed, \JesusWords{“I say to you, ‘Get up and pick up your stretcher and\lebnote{“\textit{and}” supplied because previous participle “pick up” translated as a finite verb} go to your home.’”}}%
\verse{And immediately he stood up before them, picked up what he had been lying on, and\lebnote{“\textit{and}” supplied because previous participles “stood up” and “picked up” translated as finite verbs} went away to his home, glorifying God.}%
\verse{And amazement seized them all, and they began to glorify\lebnote{Imperfect tense as ingressive (“began to glorify”)} God. And they were filled with fear, saying, “We have seen wonderful things today!”}%
\verseWithHeading{Levi Called to Follow Jesus}{And after these things, he went out and saw a tax collector named\lebnote{“by name”} Levi sitting at the tax booth, and he said to him, \JesusWords{“Follow me!”}}%
\verse{And leaving everything behind, he got up and\lebnote{“\textit{and}” supplied because previous participle “got up” translated as a finite verb} began to follow\lebnote{Imperfect tense as ingressive (“began to follow”)} him.}%
\verse{And Levi gave a great banquet for him in his house, and there was a large crowd of tax collectors and others who were reclining for the meal with them.}%
\verse{And the Pharisees and their scribes began to complain\lebnote{Imperfect tense as ingressive (“began to complain”)} to his disciples, saying, “Why do you eat and drink with the tax collectors and sinners?”}%
\verse{And Jesus answered and\lnDBN{} said to them, \JesusWords{“Those who are healthy do not have need of a physician, but those who are sick.\lebnote{“having badly”}}}%
\verse{\JesusWords{I have not come to call the righteous but sinners to repentance.”}}%
\verseWithHeading{On Fasting}{And they said to him, “The disciples of John fast often and make prayers — likewise also the disciples\lebnote{The word “\textit{disciples}” is not in the Greek text but is implied} of the Pharisees — but yours are eating and drinking!”}%
\verse{So he\lnDBP{} said\lebnote{Some manuscripts have “So Jesus said”} to them, \JesusWords{“You are not able to make the bridegroom’s attendants\lebnote{“sons of the bridal chamber”} fast as long as the bridegroom is with them, are you?}\lebnote{*The negative construction in Greek anticipates a negative answer here, indicated in the translation by the phrase “\textit{are you}”}}%
\verse{\JesusWords{But days will come, and when the bridegroom is taken away from them, then they will fast in those days.”}}%
\verse{And he also told a parable to them: \JesusWords{“No one tears a patch from a new garment and\lebnote{“\textit{and}” supplied because previous participle “tears” translated as a finite verb} puts it\lnDBQ{} on an old garment. Otherwise, he will have torn the new also, and the old will not match the patch that is from the new.}}%
\verse{\JesusWords{And no one pours new wine into old wineskins. Otherwise, the new wine will burst the wineskins, and it will be spilled and the wineskins will be destroyed.}}%
\verse{\JesusWords{But new wine must be put into new wineskins.}}%
\verse{\JesusWords{And no one after\lebnote{“\textit{after}” is supplied as a component of the participle (“drinking”) which is understood as temporal} drinking old wine\lebnote{The word “\textit{wine}” is not in the Greek text but is implied} wants new, because he says, ‘The old is just fine!’”}}%
\end{biblechapter}%
\begin{biblechapter}% Luke 6
\verseWithHeading{Plucking Grain on the Sabbath}{Now it happened that on a Sabbath he went through the grain fields, and his disciples were picking and eating the heads of grain, rubbing them\lnDBR{} in their\lnDBS{} hands.}%
\verse{But some of the Pharisees said, “Why are you doing what is not permitted on the Sabbath?}%
\verse{And Jesus answered and\lebnote{“\textit{and}” supplied because previous participle “answered” translated as a finite verb} said to them, \JesusWords{“Have you not read this, what David did when he and those who were with him were hungry —}}%
\verse{\JesusWords{how he entered into the house of God and took the bread of the presentation, which it is not permitted to eat (except the priests alone), and\lebnote{“\textit{and}” supplied because previous participle “took” translated as a finite verb} ate it\lnDBR{} and gave it\lnDBR{} to those with him?”}}%
\verse{And he said to them, \JesusWords{“The Son of Man is Lord of the Sabbath.”}}%
\verseWithHeading{A Man with a Withered Hand Healed}{Now it happened that on another Sabbath he entered into the synagogue and was teaching, and a man was there, and his right hand was withered.}%
\verse{So the scribes and the Pharisees were watching closely\lebnote{Some manuscripts have “were watching him closely”} to see if he would heal on the Sabbath, in order that they could find a reason\lnDBR{} to accuse him.}%
\verse{But he knew their thoughts and said to the man who had the withered hand, \JesusWords{“Get up and stand in the middle,”} and he got up and\lebnote{“\textit{and}” supplied because previous participle “get up” translated as a finite verb} stood there.}%
\verse{And Jesus said to them, \JesusWords{“I ask you whether it is permitted on the Sabbath to do good or to do evil, to save a life or to destroy it?”}\lnDBT{}}%
\verse{And after\lebnote{“\textit{after}” is supplied as a component of the participle (“looking around”) which is understood as temporal} looking around at them all, he said to him, \JesusWords{“Stretch out your hand,”} and he did, and his hand was restored.}%
\verse{But they were filled with fury, and began discussing\lebnote{Imperfect tense as ingressive (“began discussing”)} with one another what they might do to Jesus.}%
\verseWithHeading{The Selection of the Twelve Apostles}{Now it happened that in these days he went away to the mountain to pray, and was spending the whole night in prayer to God.}%
\verse{And when day came, he summoned his disciples and chose from them twelve, whom he also named apostles:}%
\verse{Simon (whom he also named Peter) and his brother Andrew, and James, and John, and Philip, and Bartholomew,}%
\verse{and Matthew, and Thomas, and James the son of Alphaeus, and Simon who was called the Zealot,}%
\verse{and Judas the son of James, and Judas Iscariot, who became a traitor.}%
\verseWithHeading{The Sermon on the Plain: The Beatitudes}{And he came down with them and\lebnote{“\textit{and}” supplied because previous participle “came down” translated as a finite verb} stood on a level place, and a large crowd of his disciples and a great multitude of people from all of Judea and Jerusalem and the seacoast district of Tyre and Sidon,}%
\verse{who came to hear him and to be healed of their diseases, and those who were troubled by unclean spirits were cured.}%
\verse{And the whole crowd was seeking to touch him, because power was going out from him and healing them all.}%
\verse{And he lifted up his eyes to his disciples and\lebnote{“\textit{and}” supplied because previous participle “lifted up” translated as a finite verb} said, \JesusWords{“Blessed are the poor, because yours is the kingdom of God.}}%
\verse{\JesusWords{Blessed are those who are hungry now, because you will be satisfied. Blessed are those who weep now, Because you will laugh.}}%
\verse{\JesusWords{Blessed are you when people hate you, and when they exclude you and revile you and spurn your name as evil on account of the Son of Man.}}%
\verse{\JesusWords{Rejoice in that day, and leap for joy, for behold, your reward is great in heaven. For their fathers used to do the same things to the prophets.}}%
\verseWithHeading{The Sermon on the Plain: Woes}{\JesusWords{“But woe to you who are rich, because you have received your comfort.}}%
\verse{\JesusWords{Woe to you who are satisfied now, because you will be hungry. Woe, you who laugh now, because you will mourn and weep.}}%
\verse{\JesusWords{Woe whenever all people speak well of you, for their fathers used to do the same things to the false prophets.}}%
\verseWithHeading{The Sermon on the Plain: Love for Enemies}{\JesusWords{“But to you who are listening I say: Love your enemies, do good to those who hate you,}}%
\verse{\JesusWords{bless those who curse you, pray for those who mistreat you.}}%
\verse{\JesusWords{To the one who strikes you on the cheek, offer the other also, and from the one who takes away your cloak, do not withhold your tunic also.}}%
\verse{\JesusWords{Give to everyone who asks you, and from the one who takes away your things, do not ask for them back.}\lnDBT{}}%
\verse{\JesusWords{And just as you want people to do\lebnote{“would do”} to you, do the same\lebnote{“likewise”} to them.}}%
\verse{\JesusWords{“And if you love those who love you, what kind of credit is that to you? For even sinners love those who love them!}}%
\verse{\JesusWords{And if\lebnote{Some manuscripts have “For even if”} you do good to those who do good to you, what kind of credit is that to you? Even the sinners do the same!}}%
\verse{\JesusWords{And if you lend to those from whom you expect to receive back, what kind of credit is that to you? Even sinners lend to sinners, so that they may get back an equal amount!}}%
\verse{\JesusWords{But love your enemies, and do good, and lend expecting back nothing, and your reward will be great, and you will be sons of the Most High, because he is kind to the ungrateful and wicked.}}%
\verse{\JesusWords{Be merciful, just as your Father is merciful!}\lebnote{Some manuscripts have “also is merciful”}}%
\verseWithHeading{The Sermon on the Plain: On Judging Others}{\JesusWords{“And do not judge, and you will never be judged. And do not condemn, and you will never be condemned. Pardon, and you will be pardoned.}}%
\verse{\JesusWords{Give, and it will be given to you, a good measure — pressed down, shaken, overflowing — they will pour out into your lap. For with the measure by which you measure out, it will be measured out to you in return.”}}%
\verse{And he also told them a parable: \JesusWords{“Surely a blind person cannot lead the blind, can he?\lebnote{*The negative construction in Greek anticipates a negative answer here, indicated in the translation by the phrase “\textit{can he}”} Will they not both fall into a pit?}}%
\verse{\JesusWords{A disciple is not superior to his\lnDBS{} teacher, but everyone, when he\lebnote{“\textit{when}” is supplied as a component of the participle (“is fully trained”) which is understood as temporal} is fully trained, will be like his teacher.}}%
\verse{\JesusWords{And why do you see the speck that is in your brother’s eye, but do not notice the beam of wood that is in your own eye?}}%
\verse{\JesusWords{How are you able to say to your brother, “Brother, allow me to remove the speck that is in your eye,” while\lebnote{“\textit{while}” is supplied as a component of the participle (“see”) which is understood as temporal} you yourself do not see the beam of wood in your own eye? Hypocrite! First remove the beam of wood from your own eye, and then you will see clearly to remove the speck that is in your brother’s eye!}}%
\verseWithHeading{The Sermon on the Plain: Trees and Their Fruit}{\JesusWords{“For there is no good tree that produces bad fruit, nor on the other hand a bad tree that produces good fruit,}}%
\verse{\JesusWords{for each tree is known by its own fruit. For figs are not gathered from thorn plants, nor are grapes harvested from thorn bushes.}}%
\verse{\JesusWords{The good person out of the good treasury of his heart brings forth good, and the evil person out of his\lnDBS{} evil treasury\lebnote{The word “\textit{treasury”} here is an understood repetition from earlier in the verse} brings forth evil. For out of the abundance of the heart his mouth speaks.}}%
\verse{\JesusWords{“And why do you call me ‘Lord, Lord,’ and do not do what I tell you?}\lnDBT{}}%
\verseWithHeading{The Sermon on the Plain: Two Houses and Two Foundations}{\JesusWords{“Everyone who comes to me and listens to my words and does them — I will show you what he is like:}}%
\verse{\JesusWords{he is like a man building a house, who dug and went down deep and laid the foundation on the rock. And when\lebnote{“\textit{when}” is supplied as a component of the temporal genitive absolute participle (“came”)} a flood came, the river burst against that house and was not able to shake it, because it had been built well.}}%
\verse{\JesusWords{But the one who hears my words\lnDBR{} and does not do them\lnDBR{} is like a man who built a house on the ground without a foundation, which the river burst against, and immediately it collapsed — and the collapse of that house was great!”}}%
\end{biblechapter}%
\begin{biblechapter}% Luke 7
\verseWithHeading{A Centurion’s Slave Healed}{After he had finished all his statements in the hearing of the people, he entered into Capernaum.}%
\verse{Now a certain centurion’s slave, who was esteemed by him, was sick\lebnote{“was having badly”} and\lebnote{“\textit{and}” supplied because previous participle “was having” translated as a finite verb} was about to die.}%
\verse{So when he\lnDBU{} heard about Jesus, he sent Jewish elders to him, asking him that he would come and\lebnote{“\textit{and}” supplied because previous participle “come” translated as a finite verb} cure his slave.}%
\verse{And when they\lnDBV{} came to Jesus, they began imploring\lebnote{Imperfect tense as ingressive (“began imploring”)} him earnestly, saying, “He is worthy that you grant this for him,}%
\verse{because he loves our nation and he himself built the synagogue for us.”}%
\verse{So Jesus went with them. Now by this time he was not far away from the house, and\lebnote{“\textit{and}” supplied because previous participle “away” translated as a finite verb} the centurion sent friends, saying to him, “Lord, do not trouble yourself, for I am not worthy that you should come in under my roof.}%
\verse{For this reason neither did I consider myself worthy to come to you. But say the word and my slave must be healed.}%
\verse{For I also am a man placed under authority, who has soldiers under me, and I say to this one, ‘Go!’ and he goes, and to another one, ‘Come!’ and he comes, and to my slave, ‘Do this!’ and he does it.”\lnDBW{}}%
\verse{And when\lnDBU{} Jesus heard these things, he marveled at him, and turning around to the crowd that was following him, he said, \JesusWords{“I tell you, not even in Israel have I found such great faith!”}}%
\verse{And when they\lebnote{“\textit{when}” is supplied as a component of the participle (“returned”) which is understood as temporal} returned to the house, those who had been sent found the slave healthy.}%
\verseWithHeading{A Widow’s Son Raised}{And it happened that on the next day he went to a town called Nain, and his disciples and a large crowd went with him.}%
\verse{And as he approached the gate of the town, behold, a man who had died was being carried out, his mother’s only son, and she was a widow. And a large crowd from the town was with her.}%
\verse{And when\lnDBX{} the Lord saw her, he had compassion for her and said to her, \JesusWords{“Do not weep!”}}%
\verse{And he came up and\lebnote{“\textit{and}” supplied because previous participle “came up” translated as a finite verb} touched the bier, and those who were carrying it\lnDBY{} stopped. And he said, \JesusWords{“Young man, I say to you, get up!”}}%
\verse{And the dead man sat up and began to speak, and he gave him to his mother.}%
\verse{And fear seized them all, and they began to glorify\lebnote{Imperfect tense as ingressive (“began to glorify”)} God, saying, “A great prophet has appeared among us!” and “God has visited to help his people!”}%
\verse{And this report about him went out in the whole of Judea and in all the surrounding region.}%
\verseWithHeading{A Question from John the Baptist}{And his disciples reported to John about all these things. And summoning a certain two of his disciples, John}%
\verse{sent them\lnDBY{} to the Lord,\lebnote{A number of significant manuscripts read “Jesus”} saying, “Are you the one who is to come, or should we look for another?”}%
\verse{And when\lnDBV{} the men came to him, they said, “John the Baptist sent us to you, saying, ‘Are you the one who is to come, or should we look for another?’”}%
\verse{In that hour he healed many people of diseases and suffering and evil spirits, and he granted sight to many blind people.}%
\verse{And he answered and\lnDBZ{} said to them, \JesusWords{“Go and\lebnote{“\textit{and}” supplied because previous participle “go” translated as a finite verb} tell John what you have seen and heard: the blind receive sight, the lame walk, lepers are cleansed, the deaf hear;\lebnote{Some manuscripts have “and the deaf hear”} the dead are raised, the poor have good news announced to them.}\lnDBW{}}%
\verse{\JesusWords{And whoever is not offended by me is blessed.”}}%
\verse{And when\lebnote{“\textit{when}” is supplied as a component of the temporal genitive absolute participle (“had departed”)} the messengers of John had departed, he began to speak to the crowds concerning John: \JesusWords{“What did you go out into the wilderness to see? A reed shaken by the wind?}}%
\verse{\JesusWords{But what did you go out to see? A man dressed in soft clothing? Behold, those who are in splendid clothing and luxury are in the royal palaces.}}%
\verse{\JesusWords{But what did you go out to see? A prophet? Yes, I tell you, and even more than a prophet!}}%
\verse{\JesusWords{It is this man about whom it is written: ‘Behold, I am sending my messenger before your face, who will prepare your way before you.’}\lebnote{from Mal 3:1; cf. Mark 1:2; Matt 11:10}}%
\verse{\JesusWords{I tell you, there is no one greater among those born of women than John, but the one who is least in the kingdom of God is greater than he.}}%
\verse{(And all the people, when they\lnDBU{} heard this\lnDBY{} — even the tax collectors — affirmed the righteousness of God, because they\lebnote{“\textit{because}” is supplied as a component of the participle (“had been baptized”) which is understood as causal} had been baptized with the baptism of John,}%
\verse{but the Pharisees and the legal experts rejected the purpose of God for themselves, because they\lebnote{“\textit{because}” is supplied as a component of the participle (“been baptized”) which is understood as causal} had not been baptized by him.)}%
\verse{\JesusWords{“To what then shall I compare the people of this generation, and what are they like?}}%
\verse{\JesusWords{They are like children sitting in the marketplace and calling out to one another, who say, ‘We played the flute for you and you did not dance; we sang a lament and you did not weep.’}}%
\verse{\JesusWords{For John the Baptist has come not eating bread or drinking wine, and you say, ‘He has a demon!’}}%
\verse{\JesusWords{The Son of Man has come eating and drinking, and you say, ‘Behold, a man who is a glutton and a drunkard, a friend of tax collectors and sinners!’}}%
\verse{\JesusWords{And wisdom is vindicated by all her children.”}}%
\verseWithHeading{A Sinful Woman Anoints Jesus’ Feet}{Now one of the Pharisees asked him to eat with him, and he entered into the house of the Pharisee and\lebnote{“\textit{and}” supplied because previous participle “entered” translated as a finite verb} reclined at the table.}%
\verse{And behold, a woman in the town who was a sinner, when she\lebnote{“\textit{when}” is supplied as a component of the participle (“learned”) which is understood as temporal} learned that he was dining in the Pharisee’s house, brought an alabaster flask of perfumed oil,}%
\verse{and standing behind him at his feet weeping, she began to wet his feet with her tears and was wiping them\lnDBY{} with the hair of her head and was kissing his feet and anointing them\lnDBY{} with the perfumed oil.}%
\verse{Now when\lnDBX{} the Pharisee who invited him saw this,\lnDBW{} he spoke to himself, saying, “If this man were a prophet, he would have known who and what kind of woman this is who is touching him, that she is a sinner.”}%
\verse{And Jesus answered and\lnDBZ{} said to him, \JesusWords{“Simon, I have something to say to you.”} And he said, “Teacher, say it.”\lnDBW{}}%
\verse{\JesusWords{“There were two debtors who owed a certain creditor. One owed five hundred denarii and the other fifty.}}%
\verse{\JesusWords{When\lebnote{“\textit{when}” is supplied as a component of the temporal genitive absolute participle (“able”)} they were not able to repay him,\lnDBW{} he forgave the debts\lnDBY{} of both. Now which of them will love him more?”}}%
\verse{Simon answered and\lnDBZ{} said, “I suppose that it is the one to whom he forgave more.” And he said to him, \JesusWords{“You have judged correctly.”}}%
\verse{And turning toward the woman, he said to Simon, \JesusWords{“Do you see this woman? I entered into your house. You did not give me water for my feet, but she wet my feet with her tears and wiped them\lnDBY{} with her hair.}}%
\verse{\JesusWords{You did not give me a kiss, but from the time I entered, she has not stopped kissing my feet.}}%
\verse{\JesusWords{You did not anoint my head with olive oil, but she anointed my feet with perfumed oil.}}%
\verse{\JesusWords{For this reason\lebnote{“on account of which”} I tell you, her sins — which were many — have been forgiven, for she loved much. But the one to whom little is forgiven loves little.”}}%
\verse{And he said to her, \JesusWords{“Your sins are forgiven.”}}%
\verse{And those who were reclining at the table with him began to say among themselves, “Who is this who even forgives sins?”}%
\verse{And he said to the woman, \JesusWords{“Your faith has saved you. Go in peace.”}}%
\end{biblechapter}%
\begin{biblechapter}% Luke 8
\verseWithHeading{Some Women Accompany Jesus}{And it happened that afterward\lebnote{“in what follows”} also he was going about from one town and village to another preaching and proclaiming the good news concerning the kingdom of God, and the twelve were with him,}%
\verse{and some women who had been healed of evil spirits and diseases: Mary (who was called Magdalene), from whom seven demons had gone out,}%
\verse{and Joanna the wife of Chuza (Herod’s household manager), and Susanna, and many others who were helping to support them from their possessions.}%
\verseWithHeading{The Parable of the Sower}{And while\lebnote{“\textit{while}” is supplied as a component of the temporal genitive absolute participle (“was gathering”)} a large crowd was gathering and they were going to him from town after town, he spoke by means of a parable:}%
\verse{\JesusWords{“The sower went out to sow his seed, and while he was sowing, some seed\lebnote{“some of which”} fell on the side of the path and was trampled under foot, and the birds of the sky devoured it.}}%
\verse{\JesusWords{And other seed fell on the rock, and when it\lnDCA{} came up, it withered, because it did not have moisture.}}%
\verse{\JesusWords{And other seed fell in the midst of the thorn plants, and the thorn plants grew up with it\lnDCB{} and\lebnote{“\textit{and}” supplied because previous participle “grew up with” translated as a finite verb} choked it.}}%
\verse{\JesusWords{And other seed fell on the good soil, and when it\lnDCA{} came up, it produced a hundred times as much grain.”} As he\lebnote{“\textit{as}” is supplied as a component of the participle (“said”) which is understood as temporal} said these things, he called out, \JesusWords{“The one who has ears to hear, let him hear!”}}%
\verseWithHeading{The Reason for the Parables}{And his disciples asked him what this parable meant.}%
\verse{And he said, \JesusWords{“To you it has been given to know the mysteries of the kingdom of God, but to the rest they are in parables, so that ‘Seeing they may not see, and hearing they may not understand.’}\lebnote{from Isa 6:9}}%
\verseWithHeading{The Parable of the Sower Interpreted}{\JesusWords{Now the parable means this: the seed is the word of God,}}%
\verse{\JesusWords{and those beside the path are the ones who have heard. Then the devil comes and takes away the word from their heart, so that they may not believe and\lebnote{“\textit{and}” supplied because previous participle “believe” translated as a finite verb} be saved.}}%
\verse{\JesusWords{And those on the rock are those who receive the word with joy when they hear it,\lnDCC{} and these do not have enough root, who believe for a time and in a time of testing fall away.}}%
\verse{\JesusWords{And the seed that fell into the thorn plants — these are the ones who hear and as they\lebnote{“\textit{as}” is supplied as a component of the participle (“go along”) which is understood as temporal} go along are choked by the worries and riches and pleasures of life, and they do not bear fruit to maturity.}}%
\verse{\JesusWords{But the seed on the good soil — these are the ones who, after\lebnote{“\textit{after}” is supplied as a component of the participle (“hearing”) which is understood as temporal} hearing the word, hold fast to it\lnDCB{} with a noble and good heart, and bear fruit with patient endurance.}}%
\verseWithHeading{The Parable of the Lamp}{\JesusWords{“And no one, after\lebnote{“\textit{after}” is supplied as a component of the participle (“lighting”) which is understood as temporal} lighting a lamp, covers it with a jar or puts it\lnDCB{} under a bed, but puts it\lnDCB{} on a lampstand, so that those who come in can see the light.}}%
\verse{\JesusWords{For nothing is secret that will not become evident, and nothing hidden that will never be known and come to light.}}%
\verse{\JesusWords{Therefore consider how you listen, for whoever has, to him more will be given, and whoever does not have, even what he thinks that he has will be taken away from him.”}}%
\verseWithHeading{Jesus’ Mother and Brothers}{Now his mother and brothers came to him, and they were not able to meet with him because of the crowd.}%
\verse{And it was reported to him, “Your mother and your brothers are standing outside wanting to see you.”}%
\verse{But he answered and\lebnote{“\textit{and}” supplied because previous participle “answered” translated as a finite verb} said to them, \JesusWords{“These are my mother and my brothers — the ones who hear the word of God and do it.\lnDCC{}}}%
\verseWithHeading{Calming of a Storm}{Now it happened that on one of the days both he and his disciples got into a boat, and he said to them, \JesusWords{“Let us cross over to the other side of the lake.”} And they set sail,}%
\verse{and as\lebnote{“\textit{as}” is supplied as a component of the temporal genitive absolute participle (“were sailing”)} they were sailing, he fell asleep. And a storm of wind came down on the lake, and they were being swamped and were in danger.}%
\verse{And they came and\lebnote{“\textit{and}” supplied because previous participle “came” translated as a finite verb} woke him up, saying, “Master, master! We are perishing!” So he got up and\lebnote{“\textit{and}” supplied because previous participle “got up” translated as a finite verb} rebuked the wind and the billowing waves of water and they ceased, and it became calm.}%
\verse{And he said to them, \JesusWords{“Where is your faith?”} But they were afraid and\lebnote{“\textit{and}” supplied because previous participle “were afraid” translated as a finite verb} were astonished, saying to one another, “Who then is this, that he commands even the winds and the water and they obey him?”}%
\verseWithHeading{A Demon-possessed Gerasene Healed}{And they sailed to the region of the Gerasenes, which is opposite Galilee.}%
\verse{And as\lebnote{“\textit{as}” is supplied as a component of the participle (“got out”) which is understood as temporal} he got out on the land, a certain man from the town met him\lnDCB{} who had demons and for a considerable time had not worn clothes and did not live in a house, but among the tombs.}%
\verse{And when he\lnDCD{} saw Jesus, he cried out, fell down before him, and said with a loud voice, “What do I have to do with you,\lebnote{“what to me and to you”} Jesus, Son of the Most High God? I beg you, do not torment me!”}%
\verse{For he had commanded the unclean spirit to come out of the man. (For it had seized him many times, and he was bound with chains and shackles and\lebnote{“\textit{and}” supplied because previous participle “was bound” translated as a finite verb} was guarded, and breaking the bonds he would be driven by the demon into the deserted places.)}%
\verse{So Jesus asked him, \JesusWords{“What is your name?”} And he said, “Legion,” because many demons had entered into him.}%
\verse{And they began imploring\lnDCE{} him that he would not order them to depart into the abyss.}%
\verse{Now there was a large herd of pigs feeding there on the hill, and they implored him that he would permit them to enter into those pigs. And he permitted them.}%
\verse{So the demons came out of the man and\lebnote{“\textit{and}” supplied because previous participle “came out” translated as a finite verb} entered into the pigs, and the herd rushed headlong down the steep slope into the lake and were drowned.}%
\verse{And when\lnDCD{} the herdsmen saw what had happened, they fled and reported it\lnDCB{} in the town and in the countryside.}%
\verse{So they went out to see what had happened, and they came to Jesus and found the man from whom the demons had gone out sitting there clothed and in his right mind, at the feet of Jesus, and they were afraid.}%
\verse{And those who had seen it\lnDCB{} reported to them how the man who had been demon-possessed had been healed.}%
\verse{And all the people of the surrounding region of the Gerasenes asked him to depart from them, because they had been seized with great fear. So he got into the boat and\lebnote{“\textit{and}” supplied because previous participle “got” translated as a finite verb} returned.}%
\verse{And the man from who the demons had gone out was begging him to stay with him, but he sent him away, saying,}%
\verse{\JesusWords{“Return to your home and tell all that God has done for you.”} And he went away, proclaiming throughout the whole town all that Jesus had done for him.}%
\verseWithHeading{A Woman Healed and a Daughter Raised}{Now when Jesus returned, the crowd welcomed him, because they were all waiting for him.}%
\verse{And behold, a man who was named\lebnote{“to whom the name”} Jairus came, and this man was a ruler of the synagogue. And he fell down at the feet of Jesus and\lebnote{“\textit{and}” supplied because previous participle “fell down” translated as a finite verb} began imploring\lnDCE{} him to come to his house,}%
\verse{because he had\lebnote{“there was to him”} an only daughter, about twelve years old, and she was dying. Now as he was going, the crowds were pressing against him.}%
\verse{And a woman who was suffering from hemorrhages\lebnote{Literally, “with a flow of blood”} for twelve years (who, although she\lebnote{“\textit{although}” is supplied as a component of the participle (“had spent”) which is understood as concessive} had spent all her\lebnote{“the”: the Greek article is used here as a possessive pronoun} assets on physicians, was not able to be healed by anyone)}%
\verse{came up behind him and\lebnote{“\textit{and}” supplied because previous participle “came up” translated as a finite verb} touched the edge of his cloak, and immediately her hemorrhaging\lebnote{“the flow of her blood”} stopped.}%
\verse{And Jesus said, \JesusWords{“Who is the one who touched me?”} And when they\lebnote{“\textit{when}” is supplied as a component of the temporal genitive absolute participle (“denied”)} all denied it,\lnDCC{} Peter said, “Master, the crowds are pressing you hard and crowding you!”\lnDCC{}}%
\verse{But Jesus said, \JesusWords{“Someone touched me, because I know power has gone out from me.”}}%
\verse{And when\lnDCD{} the woman saw that she did not escape notice, she came trembling and falling down before him. In the presence of all the people, she told for what reason she had touched him, and that she was healed immediately.}%
\verse{And he said to her, \JesusWords{“Daughter, your faith has saved you. Go in peace.”}}%
\verse{While\lebnote{“\textit{while}” is supplied as a component of the temporal genitive absolute participle (“speaking”)} he was still speaking, someone came from the synagogue ruler’s house, saying, “Your daughter is dead! Trouble the Teacher no longer!”}%
\verse{But Jesus, when he\lebnote{“\textit{when}” is supplied as a component of the participle (“heard”) which is understood as temporal} heard this,\lnDCC{} replied to him, \JesusWords{“Do not be afraid! Only believe, and she will be healed.”}}%
\verse{Now when he\lebnote{“\textit{when}” is supplied as a component of the participle (“came”) which is understood as temporal} came to the house, he did not allow anyone to enter with him except Peter and John and James and the father and mother of the child.}%
\verse{And they were all weeping and mourning for her, but he said, \JesusWords{“Do not weep! For she is not dead, but is sleeping.”}}%
\verse{And they began laughing\lebnote{Imperfect tense as ingressive (“began laughing”)} at him, because they\lebnote{“\textit{because}” is supplied as a component of the participle (“knew”) which is understood as causal} knew that she was dead.}%
\verse{But he took her hand and\lebnote{“\textit{and}” supplied because previous participle “took” translated as a finite verb} called, saying, \JesusWords{“Child, get up.”}}%
\verse{And her spirit returned, and she got up immediately, and he ordered something\lnDCB{} to be given to her to eat.}%
\verse{And her parents were astonished, but he ordered them to tell no one what had happened.}%
\end{biblechapter}%
\begin{biblechapter}% Luke 9
\verseWithHeading{The Twelve Commissioned and Sent Out}{And summoning the twelve, he gave them power and authority over all the demons and to cure diseases,}%
\verse{and he sent them out to proclaim the kingdom of God and to heal the sick.}%
\verse{And he said to them, \JesusWords{“Take along nothing for the journey — neither a staff, nor a traveler’s bag, nor bread, nor money, nor to have two tunics apiece.}}%
\verse{\JesusWords{And into whatever house you enter, stay there and depart from there.}}%
\verse{\JesusWords{And as for all those who do not welcome you — when you\lebnote{“\textit{when}” is supplied as a component of the participle (“depart”) which is understood as temporal} depart from that town, shake off the dust from your feet for a testimony against them.”}}%
\verse{So they departed and\lebnote{“\textit{and}” supplied because previous participle “departed” translated as a finite verb} went throughout the villages, proclaiming the good news and healing everywhere.}%
\verseWithHeading{Herod Perplexed About Jesus}{Now Herod the tetrarch heard about all that was happening, and he was greatly perplexed, because it was said by some that John has been raised from the dead,}%
\verse{and by some that Elijah had appeared, and others that some prophet of ancient times had risen.}%
\verse{And Herod said, “John I beheaded, but who is this about whom I hear such things?” And he was wanting to see him.}%
\verseWithHeading{The Feeding of Five Thousand}{And when they\lebnote{“\textit{when}” is supplied as a component of the participle (“returned”) which is understood as temporal} returned, the apostles described to him all that they had done. And he took them along and\lebnote{“\textit{and}” supplied because previous participle “took … along” translated as a finite verb} withdrew privately to a town called Bethsaida.}%
\verse{But when\lebnote{“\textit{when}” is supplied as a component of the participle (“found out”) which is understood as temporal} the crowds found out, they followed him, and welcoming them, he began to speak\lnDCF{} to them about the kingdom of God, and he cured those who had need of healing.}%
\verse{Now the day began to be far spent, and the twelve came up and\lebnote{“\textit{and}” supplied because previous participle “came up” translated as a finite verb} said to him, “Send away the crowd so that they can go into the surrounding villages and farms to obtain lodging and find provisions, because we are here in a desolate place.}%
\verse{But he said to them, \JesusWords{“You give them something to eat!”} And they said, “We have no\lebnote{“there is not to us”} more than five loaves and two fish, unless perhaps we go and\lnDCG{} purchase food for all these people.”}%
\verse{(For there were about five thousand men.) So he said to his disciples, \JesusWords{“Have them sit down in groups of about fifty each.”}}%
\verse{And they did so, and had them all sit down.}%
\verse{And taking the five loaves and the two fish, and\lebnote{“\textit{and}” is supplied before the participle (“looking up”) in keeping with English style} looking up to heaven, he gave thanks and broke them and began giving\lebnote{Imperfect tense as ingressive (“began giving”)} them\lnDCF{} to the disciples to set before the crowd.}%
\verse{And they all ate and were satisfied, and what was left over was picked up by them — twelve baskets of broken pieces.}%
\verseWithHeading{Peter’s Confession}{And it happened that while he was praying alone, the disciples were with him. And he asked them, saying, \JesusWords{“Who do the crowds say that I am?”}}%
\verse{And they answered and\lnDCH{} said, “John the Baptist, but others, Elijah, and others, that one of the ancient prophets has risen.”}%
\verse{And he said to them, \JesusWords{“But who do you say that I am?”} And Peter answered and\lnDCH{} said, “The Christ of God.”}%
\verseWithHeading{Jesus Predicts His Death and Resurrection}{But he warned and\lebnote{“\textit{and}” supplied because previous participle “warned” translated as a finite verb} commanded them to tell this to no one,}%
\verse{saying, \JesusWords{“It is necessary for the Son of Man to suffer many things and to be rejected by the elders and chief priests and scribes, and to be killed, and to be raised on the third day.}}%
\verseWithHeading{Taking Up One’s Cross to Follow Jesus}{And he said to them all, \JesusWords{“If anyone wants to come after me, he must deny himself and take up his cross every day and follow me.}}%
\verse{\JesusWords{For whoever wants to save his life will lose it, but whoever loses his life on account of me, this person will save it.}}%
\verse{\JesusWords{For what is a person benefited if he\lebnote{“\textit{if}” is supplied as a component of the participle (“gains”) which is understood as conditional} gains the whole world but loses or forfeits himself?}}%
\verse{\JesusWords{For whoever is ashamed of me and my words, the Son of Man will be ashamed of this person when he comes in his glory and the glory\lebnote{“\textit{glory}” is an understood repetition of the same word earlier in this verse} of the Father and of the holy angels.}}%
\verse{\JesusWords{But I tell you truly, there are some of those standing here who will never experience death until they see the kingdom of God.”}}%
\verseWithHeading{The Transfiguration}{Now it happened that about eight days after these words, he took along Peter and John and James and\lebnote{“\textit{and}” supplied because previous participle “took along” translated as a finite verb} went up on the mountain to pray.}%
\verse{And as he was praying, the appearance of his face became different, and his clothing became white, gleaming like lightning.\lebnote{Or “became brilliant as light”; or “became dazzling white”}}%
\verse{And behold, two men were talking with him, who were Moses and Elijah,}%
\verse{who appeared in glory and\lebnote{“\textit{and}” supplied because previous participle “appeared” translated as a finite verb} were speaking about his departure which he was about to fulfill\lebnote{Or “to accomplish”} in Jerusalem.}%
\verse{Now Peter and those with him were very sleepy,\lebnote{“burdened with sleep”} but when they\lebnote{“\textit{when}” is supplied as a component of the participle (“became fully awake”) which is understood as temporal} became fully awake, they saw his glory and the two men who were standing with him.}%
\verse{And it happened that as they were going away from him, Peter said to Jesus, “Master, it is good for us to be here. And let us make three shelters, one for you and one for Moses and one for Elijah,” not knowing what he was saying.}%
\verse{And while\lebnote{“\textit{while}” is supplied as a component of the temporal genitive absolute participle (“saying”)} he was saying these things, a cloud came and overshadowed them, and they were afraid as they entered into the cloud.}%
\verse{And a voice came from the cloud, saying, “This is my Son, my Chosen One. Listen to him!”}%
\verse{And after the voice had occurred, Jesus was found alone. And they kept silent and told no one in those days anything of what they had seen.}%
\verseWithHeading{A Demon-possessed Boy Healed}{Now it happened that on the next day, when\lebnote{“\textit{when}” is supplied as a component of the temporal genitive absolute participle (“had come down”)} they had come down from the mountain, a large crowd met him.}%
\verse{And behold, a man from the crowd cried out, saying, “Teacher, I beg you to look with concern on my son, because he is my only son!}%
\verse{And behold, a spirit seizes him and suddenly he screams, and it convulses him with foam and rarely withdraws from him, battering him severely.}%
\verse{And I begged your disciples that they would expel it, and they were not able to do so.”}%
\verse{So Jesus answered and\lnDCH{} said, \JesusWords{“O unbelieving and perverted generation! How long\lebnote{“until when”} will I be with you and put up with you? Bring your son here!”}}%
\verse{And while\lebnote{“\textit{while}” is supplied as a component of the temporal genitive absolute participle (“approaching”)} he was still approaching, the demon threw him down and convulsed him.\lnDCI{} But Jesus rebuked the unclean spirit and healed the boy, and gave him back to his father.}%
\verse{And they were all astounded at the impressiveness of God.\innerVerseHeading{Jesus Predicts His Suffering}But while they\lebnote{“\textit{while}” is supplied as a component of the temporal genitive absolute participle (“marveling”)} were all marveling at all the things that he was doing, he said to his disciples,}%
\verse{\JesusWords{“You take these words to heart,\lebnote{“you put these words into your ears”} for the Son of Man is about to be betrayed into the hands of men.”}}%
\verse{But they did not understand this statement, and it was concealed from them so that they could not understand it. And they were afraid to ask him about this statement.}%
\verseWithHeading{The Question About Who Is Greatest}{And an argument developed among them as to who of them might be greatest.}%
\verse{But Jesus, because he\lebnote{“\textit{because}” is supplied as a component of the participle (“knew”) which is understood as causal} knew the thoughts of their hearts, took hold of a child and had him stand beside him}%
\verse{and said to them, \JesusWords{“Whoever welcomes this child in my name welcomes me, and whoever welcomes me welcomes the one who sent me. For the one who is least among you all — this one is great.”}}%
\verseWithHeading{Whoever Is Not Against Us Is for Us}{And John answered and\lnDCH{} said, “Master, we saw someone expelling demons in your name, and we tried to prevent him, because he does not follow in company with us.”}%
\verse{But Jesus said to him, \JesusWords{“Do not prevent him,\lnDCI{} because whoever is not against you is for you.”}}%
\verseWithHeading{Jesus Rejected in a Samaritan Village}{Now it happened that when the days were approaching for him to be taken up,\lebnote{“of his taking up”} he set his\lnDCJ{} face to go to Jerusalem.}%
\verse{And he sent messengers before him,\lebnote{“his face”} and they went and\lebnote{“\textit{and}” supplied because previous participle “went” translated as a finite verb} entered into a village of the Samaritans in order to prepare for him.}%
\verse{And they did not welcome him because he was determined to go\lebnote{“his face was going”} to Jerusalem.}%
\verse{Now when\lebnote{“\textit{when}” is supplied as a component of the participle (“saw”) which is understood as temporal} the disciples James and John saw it,\lnDCI{} they said, “Lord, do you want us to call fire to come down from heaven and consume them?”}%
\verse{But he turned around and\lebnote{“\textit{and}” supplied because previous participle “turned around” translated as a finite verb} rebuked them,}%
\verse{and they proceeded to another village.}%
\verseWithHeading{Would-be Followers}{And as\lebnote{“\textit{as}” is supplied as a component of the temporal genitive absolute participle (“were going”)} they were traveling on the road, someone said to him, “I will follow you wherever you go!”}%
\verse{And Jesus said to him, \JesusWords{“Foxes have dens and birds of the sky have nests, but the Son of Man has no place to lay his\lnDCJ{} head.”}}%
\verse{And he said to another, \JesusWords{“Follow me!”} But he said, “Lord, first allow me to go and\lebnote{“\textit{and}” supplied because previous participle “go” translated as an infinitive} bury my father.”}%
\verse{But he said to him, \JesusWords{“Leave the dead to bury their own dead! But you go and\lnDCG{} proclaim the kingdom of God.”}}%
\verse{And another person also said, “I will follow you, Lord, but first allow me to say farewell to those in my house.”}%
\verse{But Jesus said,\lebnote{Some manuscripts have “said to him”} \JesusWords{“No one who puts his\lnDCJ{} hand on the plow and looks back is fit for the kingdom of God!”}}%
\end{biblechapter}%
\begin{biblechapter}% Luke 10
\verseWithHeading{The Seventy-Two Appointed and Sent Out}{And after these things, the Lord also\lebnote{Some manuscripts omit “also”} appointed seventy-two others and sent them out two by two before him\lebnote{“his presence”} into every town and place where he was about to go.}%
\verse{And he said to them, \JesusWords{“The harvest is plentiful, but the workers are few. Therefore ask the Lord of the harvest that he send out workers into his harvest.}}%
\verse{\JesusWords{Go! Behold, I am sending you out like lambs in the midst of wolves!}}%
\verse{\JesusWords{Do not carry a money bag or a traveler’s bag or sandals, and greet no one along the road.}}%
\verse{\JesusWords{And into whatever house you enter, first say, “Peace be to this household!”}}%
\verse{\JesusWords{And if a son of peace is there, your peace will rest on him. But if not, it will return to you.}}%
\verse{\JesusWords{And remain in the same house, eating and drinking whatever they provide,\lebnote{“the things from them”} for the worker is worthy of his pay. Do not move from house to house.}}%
\verse{\JesusWords{And into whatever town you enter and they welcome you, eat whatever is\lebnote{“the things”} set before you,}}%
\verse{\JesusWords{and heal the sick in it, and say to them, “The kingdom of God has come near to you.”}}%
\verse{\JesusWords{But into whatever town you enter and they do not welcome you, go out into its streets and\lebnote{“\textit{and}” supplied because previous participle “go out” translated as a finite verb} say,}}%
\verse{\JesusWords{“Even the dust of your town that clings to our feet we wipe off against you! Nevertheless know this: that the kingdom of God has come near!”}\lebnote{Or “has come”}}%
\verse{\JesusWords{I tell you that it will be more bearable on that day for Sodom than for that town!}}%
\verse{\JesusWords{Woe to you, Chorazin! Woe to you, Bethsaida! For if the miracles that were done in you had been done in Tyre and Sidon, they would have repented long ago, sitting in sackcloth and ashes!}}%
\verse{\JesusWords{But it will be more bearable for Tyre and for Sidon in the judgment than for you!}}%
\verse{\JesusWords{And you, Capernaum, will you be exalted to heaven? No! You will be brought down to Hades!}}%
\verse{\JesusWords{The one who listens to you listens to me, and the one who rejects you rejects me. But the one who rejects me rejects the one who sent me.”}}%
\verse{And the seventy-two returned with joy, saying, “Lord, even the demons are subject to us in your name!”}%
\verse{So he said to them, \JesusWords{“I saw Satan falling like lightning from heaven.}}%
\verse{\JesusWords{Behold, I have given you the authority to tread on snakes and scorpions, and over all the power of the enemy, and nothing will ever harm you.}}%
\verse{\JesusWords{Nevertheless, do not rejoice in this, that the spirits are subject to you, but rejoice that your names are inscribed in heaven.”}}%
\verseWithHeading{Jesus Rejoices and Prays}{At that same time he rejoiced in the Holy Spirit and said, \JesusWords{“I praise you, Father, Lord of heaven and earth, that you have hidden these things from the wise and intelligent and have revealed them to young children. Yes, Father, for this was pleasing before you.}}%
\verse{\JesusWords{All things have been handed over to me by my Father, and no one knows who the Son is except the Father and who the Father is except the Son, and anyone to whom the Son wants to reveal him.”}\lnDCK{}}%
\verse{And turning to the disciples, he said privately, \JesusWords{“Blessed are the eyes that see the things which you see!}}%
\verse{\JesusWords{For I tell you that many prophets and kings desired to see the things which you see, and did not see them,\lnDCK{} and to hear the things which you hear, and did not hear them.”}\lnDCK{}}%
\verseWithHeading{The Parable of the Good Samaritan}{And behold, a certain legal expert stood up to test him, saying, “Teacher, what must I do so that I will inherit eternal life?”}%
\verse{And he said to him, \JesusWords{“What is written in the law? How do you read it?”}\lnDCK{}}%
\verse{And he answered and\lnDCL{} said, “You shall love the Lord your God from all your heart, and with all your soul, and with all your strength, and with all your mind,\lebnote{from Deut 6:5} and your neighbor as yourself.”\lebnote{from Lev 19:18}}%
\verse{And he said to him, \JesusWords{“You have answered correctly. Do this and you will live.”}}%
\verse{But he, wanting to justify himself, said to Jesus, “And who is my neighbor?”}%
\verse{And\lebnote{Some manuscripts omit “and”} Jesus replied and\lebnote{“\textit{and}” supplied because previous participle “replied” translated as a finite verb} said, \JesusWords{“A certain man was going down from Jerusalem to Jericho, and fell into the hands of robbers, who both stripped him and beat him.\lnDCK{} After\lebnote{“\textit{after}” is supplied as a component of the participle (“inflicting blows on”) which is understood as temporal} inflicting blows on him,\lnDCK{} they went away, leaving him\lnDCM{} half dead.}}%
\verse{\JesusWords{Now by coincidence a certain priest was going down on that road, and when he\lnDCN{} saw him, he passed by on the opposite side.}}%
\verse{\JesusWords{And in the same way also a Levite, when he\lebnote{“\textit{when}” is supplied as a component of the participle (“came”) which is understood as temporal} came down to the place\lebnote{Some manuscripts have “who happened by the place, when he came up to him”} and saw him,\lnDCK{} passed by on the opposite side.}}%
\verse{\JesusWords{But a certain Samaritan who was traveling came up to him and, when he\lnDCN{} saw him,\lnDCK{} had compassion.}}%
\verse{\JesusWords{And he came up and\lebnote{“\textit{and}” supplied because previous participle “came up” translated as a finite verb} bandaged his wounds, pouring on olive oil and wine, and he put him on his own animal and\lebnote{“\textit{and}” supplied because previous participle “put” translated as a finite verb} brought him to an inn and took care of him.}}%
\verse{\JesusWords{And on the next day, he took out two denarii and\lebnote{“\textit{and}” supplied because previous participle “took out” translated as a finite verb} gave them\lnDCM{}\lebnote{Some manuscripts have “he took out \textit{and} gave two denarii”} to the innkeeper, and said, “Take care of him, and whatever you spend in addition, I will repay to you when I return.}}%
\verse{\JesusWords{Which of these three do you suppose became a neighbor of the man who fell among the robbers?”}}%
\verse{So he said, “The one who showed mercy to him.” And Jesus said to him, \JesusWords{“You go and do likewise.”}}%
\verseWithHeading{Martha and Mary}{Now as they traveled along, he entered into a certain village. And a certain woman named\lebnote{“by name”} Martha welcomed him.\lebnote{Most manuscripts add some form of the location such as “into her house” but there is considerable variation in the exact wording, so the shorter reading is preferred}}%
\verse{And she had\lebnote{“this was”} a sister named Mary, who also sat at the feet of Jesus\lebnote{Some manuscripts have “of the Lord”} and\lebnote{“\textit{and}” supplied because previous participle “sat” translated as a finite verb} was listening to his teaching.}%
\verse{But Martha was distracted with much preparation, so she approached and\lebnote{“\textit{and}” supplied because previous participle “approached” translated as a finite verb} said, “Lord, is it not a concern to you that my sister has left me alone to make preparations? Then tell her that she should help me!”}%
\verse{But the Lord answered and\lnDCL{} said to her, \JesusWords{“Martha, Martha, you are anxious and troubled about many things!}}%
\verse{\JesusWords{But few things are necessary, or only one thing,\lebnote{Some manuscripts have “But one thing is necessary”} for Mary has chosen the better part, which will not be taken away from her.”}}%
\end{biblechapter}%
\begin{biblechapter}% Luke 11
\verseWithHeading{How to Pray}{And it happened that while he was in a certain place praying, when he stopped a certain one of his disciples said to him, “Lord, teach us to pray, just as John also taught his disciples.”}%
\verse{And he said to them, \JesusWords{“When you pray, say, “Father, may your name be treated as holy. May your kingdom come.}}%
\verse{\JesusWords{Give us each day our daily bread.}}%
\verse{\JesusWords{And forgive us our sins, for we ourselves also forgive everyone who is indebted to us. And do not lead us into temptation.”}}%
\verseWithHeading{Ask, Seek, Knock}{And he said to them, \JesusWords{“Who of you will have a friend, and will go to him at midnight and say to him, ‘Friend, lend me three loaves,}}%
\verse{\JesusWords{because a friend of mine has come to me on a journey, and I do not have anything to set before him.’}}%
\verse{\JesusWords{And that one will answer from inside and\lebnote{“\textit{and}” supplied because previous participle “will answer” translated as a finite verb} say, ‘Do not cause me trouble! The door has already been shut and my children are with me in bed! I am not able to get up to give you anything.’}\lnDCO{}}%
\verse{\JesusWords{I tell you, even if he does not give him anything\lnDCP{} after he\lebnote{“\textit{after}” is supplied as a component of the participle (“gets up”) which is understood as temporal} gets up because he is his friend, at any rate because of his impudence\lebnote{Or “shamelessness”; some translate as “persistence” based on the context, though this is not the normal meaning of the word} he will get up and\lebnote{“\textit{and}” supplied because previous participle “will get up” translated as a finite verb} give him whatever he needs.}}%
\verse{\JesusWords{And I tell you, ask and it will be given to you; seek and you will find; knock and it will be opened for you.}}%
\verse{\JesusWords{For everyone who asks receives, and the one who seeks finds, and to the one who knocks it will be opened.}}%
\verse{\JesusWords{But what father from among you, if his\lebnote{“the”: the Greek article is used here as a possessive pronoun} son will ask for a fish, instead of a fish will give him a snake?}}%
\verse{\JesusWords{Or also, if he will ask for an egg, will give him a scorpion?}}%
\verse{\JesusWords{Therefore if you, although you\lebnote{“\textit{although}” is supplied as a component of the participle (“are”) which is understood as concessive} are evil, know how to give good gifts to your children, how much more will the Father from heaven give the Holy Spirit to those who ask him?”}}%
\verseWithHeading{A House Divided Cannot Stand}{And he was expelling a mute demon.\lebnote{Some manuscripts have “a demon, and it was mute”} Now it happened that when\lebnote{“\textit{when}” is supplied as a component of the temporal genitive absolute participle (“came out”)} the demon came out, the man who had been mute spoke, and the crowds were astonished.}%
\verse{But some of them said, “By Beelzebul the ruler of demons he expels demons!”}%
\verse{And others, in order to\lebnote{“\textit{in order to}” is supplied as a component of the participle (“test”) which is understood as purpose} test him,\lnDCO{} were demanding from him a sign from heaven.}%
\verse{But he, knowing their thoughts, said to them, \JesusWords{“Every kingdom divided against itself is laid waste, and a divided household\lebnote{“house against house”} falls.}}%
\verse{\JesusWords{So if Satan also is divided against himself, how will his kingdom stand? For you say that I expel demons by Beelzebul.}}%
\verse{\JesusWords{But if I expel demons by Beelzebul, by whom do your sons expel them?\lnDCO{} For this reason they will be your judges!}}%
\verse{\JesusWords{But if I expel demons by the finger of God, then the kingdom of God has come upon you!}}%
\verse{\JesusWords{When a strong man, fully armed, guards his own palace, his possessions are safe.}\lebnote{“in peace”}}%
\verse{\JesusWords{But when a stronger man attacks him and\lebnote{“\textit{and}” supplied because previous participle “attacks” translated as a finite verb} conquers him, he takes away his full armor in which he trusted and distributes his plunder.}}%
\verse{\JesusWords{The one who is not with me is against me, and the one who does not gather with me scatters.}}%
\verseWithHeading{An Unclean Spirit Returns}{\JesusWords{“Whenever an unclean spirit has gone out of a person, it travels through waterless places searching for rest, and does not find it.\lnDCO{} It says,\lebnote{Some manuscripts have “Then it says”} ‘I will return to my house from which I came out.’}}%
\verse{\JesusWords{And when it\lebnote{“\textit{when}” is supplied as a component of the participle (“arrives”) which is understood as temporal} arrives it finds the house\lnDCP{} swept and put in order.}}%
\verse{\JesusWords{Then it goes and brings along seven other spirits more evil than itself, and they go in and\lebnote{“\textit{and}” supplied because previous participle “go in” translated as a finite verb} live there. And the last state of that person becomes worse than the first!”}}%
\verse{Now it happened that as he said these things, a certain woman from the crowd raised her voice and\lebnote{“\textit{and}” supplied because previous participle “raised” translated as a finite verb} said to him, “Blessed is the womb that bore you, and the breasts at which you nursed!”}%
\verse{But he said, \JesusWords{“On the contrary, blessed are those who hear the word of God and follow it!”\lnDCO{}}}%
\verseWithHeading{The Sign of Jonah}{And as\lebnote{“\textit{as}” is supplied as a component of the temporal genitive absolute participle (“were increasing”)} the crowds were increasing, he began to say, \JesusWords{“This generation is an evil generation! It demands a sign, and no sign will be given to it except the sign of Jonah!}}%
\verse{\JesusWords{For just as Jonah became a sign to the Ninevites, so also the Son of Man will be to this generation.}}%
\verse{\JesusWords{The queen of the south will rise up at the judgment with the people of this generation and condemn them, because she came from the ends of the earth to hear the wisdom of Solomon, and behold, something\lnDCQ{} greater than Solomon is here!}}%
\verse{\JesusWords{The people of Nineveh will stand up at the judgment with this generation and condemn it, because they repented at the proclamation of Jonah, and behold, something\lnDCQ{} greater than Jonah is here!}}%
\verseWithHeading{Light and Darkness}{\JesusWords{“No one after\lebnote{“\textit{after}” is supplied as a component of the participle (“lighting”) which is understood as temporal} lighting a lamp puts it\lnDCP{} in a cellar or under a bushel basket, but on a lampstand, so that those who come in can see the light.}}%
\verse{\JesusWords{Your eye is the lamp of the body. When your eye is sincere, your whole body is full of light also. But when it is evil, your body is dark also.}}%
\verse{\JesusWords{Therefore pay careful attention that the light in you is not darkness!}}%
\verse{\JesusWords{If therefore your whole body is full of light, not having any part dark, it will be completely full of light, as when the lamp with its light gives light to you.”}}%
\verseWithHeading{Pharisees and Legal Experts Denounced}{And as he was speaking, a Pharisee asked him to have a meal\lebnote{“that he would have a meal”} with him, and he went in and\lebnote{“\textit{and}” supplied because previous participle “went in” translated as a finite verb} reclined at table.}%
\verse{And the Pharisee, when he\lebnote{“\textit{when}” is supplied as a component of the participle (“saw”) which is understood as temporal} saw it,\lnDCO{} was astonished that he did not first wash before the meal.}%
\verse{But the Lord said to him, \JesusWords{“Now you Pharisees cleanse the outside of the cup and of the dish, but your inside is full of greediness and wickedness.}}%
\verse{\JesusWords{Fools! Did not the one who made the outside make the inside also?}}%
\verse{\JesusWords{But give as charitable giving the things that are within, and behold, everything is clean for you.}}%
\verse{\JesusWords{“But woe to you, Pharisees, because you pay a tenth of mint and rue and every garden herb, and neglect justice and love for God! But it was necessary to do these things without neglecting those things also.}}%
\verse{\JesusWords{Woe to you, Pharisees, because you love the best seat in the synagogues and the greetings in the marketplaces!}}%
\verse{\JesusWords{Woe to you, because you are like unmarked graves, and the people who walk over them\lnDCP{} do not know it!\lnDCO{}}}%
\verse{And one of the legal experts answered and\lebnote{“\textit{and}” supplied because previous participle “answered” translated as a finite verb} said to him, “Teacher, when you\lebnote{“\textit{when}” is supplied as a component of the participle (“say”) which is understood as temporal} say these things, you insult us also!”}%
\verse{So he said, \JesusWords{“Woe to you also, legal experts, because you load people with burdens hard to bear, and you yourselves do not touch the burdens with one of your fingers!}}%
\verse{\JesusWords{Woe to you, because you build the tombs of the prophets, and your fathers killed them!}}%
\verse{\JesusWords{As a result you are witnesses, and you approve of the deeds of your fathers, because they killed them and you build their tombs!\lebnote{A large number of later manuscripts add the words “their tombs” here, with variations of wording; although the words are not likely to be original, it is necessary to supply them in keeping with English style}}}%
\verse{\JesusWords{For this reason also the wisdom of God said, ‘I will send to them prophets and apostles, and some of them they will kill and persecute,’}}%
\verse{\JesusWords{so that the blood of all the prophets that has been shed from the foundation of the world may be required of this generation,}}%
\verse{\JesusWords{from the blood of Abel to the blood of Zechariah, who perished between the altar and the temple building.\lebnote{“the house,” here a reference to the temple} Yes, I tell you, it will be required of this generation!}}%
\verse{\JesusWords{Woe to you, legal experts, because you have taken away the key to knowledge! You did not enter yourselves, and you hindered those who were entering!”}}%
\verse{And when\lebnote{“\textit{when}” is supplied as a component of the temporal genitive absolute participle (“departed”)} he departed from there, the scribes and the Pharisees began to be terribly hostile, and to question him closely about many things,}%
\verse{plotting to catch him with reference to something he might say.\lebnote{“from his mouth”}}%
\end{biblechapter}%
\begin{biblechapter}% Luke 12
\verseWithHeading{Warning Against Hypocrisy}{During this time\lebnote{“which time”} when\lebnote{“\textit{when}” is supplied as a component of the temporal genitive absolute participle (“had gathered together”)} a crowd of many thousands had gathered together, so that they were trampling one another, he began to say to his disciples first, \JesusWords{“Beware for yourselves of the leaven of the Pharisees, which is hypocrisy.}}%
\verse{\JesusWords{But nothing is concealed that will not be revealed, and secret that will not be made known.}}%
\verse{\JesusWords{Therefore everything that you have said in the dark will be heard in the light, and what you have whispered\lebnote{“you have spoken to the ear”} in the inner rooms will be proclaimed on the housetops.}}%
\verseWithHeading{Fear God Rather Than People}{\JesusWords{“And I tell you, my friends, do not be afraid of those who kill the body, and after these things do not have anything more to do.}}%
\verse{\JesusWords{But I will show you whom you should fear: fear the one who has authority, after the killing, to throw you\lnDCR{} into hell! Yes, I tell you, fear this one!}}%
\verse{\JesusWords{Are not five sparrows sold for two pennies? And not one of them is forgotten in the sight of God.}}%
\verse{\JesusWords{But even the hairs of your head are all numbered! Do not be afraid; you are worth more than many sparrows.}}%
\verseWithHeading{Acknowledgement of Christ and Persecution of Disciples}{\JesusWords{“And I tell you, everyone who acknowledges me before people, the Son of Man also will acknowledge him before the angels of God,}}%
\verse{\JesusWords{but the one who denies me before people will be denied before the angels of God.}}%
\verse{\JesusWords{And everyone who speaks a word against the Son of Man, it will be forgiven him, but to the one who blasphemes against the Holy Spirit, it will not be forgiven.}}%
\verse{\JesusWords{But when they bring you before the synagogues and the rulers and the authorities, do not be anxious how or what you should speak in your own defense or what you should say,}}%
\verse{\JesusWords{for the Holy Spirit will teach you in that same hour what it is necessary to say.”}}%
\verseWithHeading{The Parable of the Rich Landowner Who Was a Fool}{Now someone from the crowd said to him, “Teacher, tell my brother to divide the inheritance with me!”}%
\verse{But he said to him, \JesusWords{“Man, who made me a judge or an arbitrator over you?”}}%
\verse{And he said to them, \JesusWords{“Watch out and guard yourselves from all greediness, because not even when someone has an abundance does\lebnote{“is”} his life consist of his possessions.”}}%
\verse{And he told a parable to them, saying, \JesusWords{“The land of a certain rich man yielded an abundant harvest.}}%
\verse{\JesusWords{And he reasoned to himself, saying, ‘What should I do? For I do not have anywhere I can gather in my crops.’}}%
\verse{\JesusWords{And he said, ‘I will do this: I will tear down my barns and build larger ones, and I will gather in there all my grain and possessions.}}%
\verse{\JesusWords{And I will say to my soul, “Soul, you have many possessions stored up for many years. Relax, eat, drink, celebrate!”’}}%
\verse{\JesusWords{But God said to him, ‘Fool! This night your life\lebnote{The same Greek word can be translated “soul” or “life” depending on the context} is demanded from you, and the things which you have prepared — whose will they be?’}}%
\verse{\JesusWords{So is the one who stores up treasure for himself, and who is not rich toward God!”}}%
\verseWithHeading{Anxiety}{And he said to his disciples, \JesusWords{“For this reason I tell you, do not be anxious for your\lnDCS{} life, what you will eat, or for your\lnDCS{} body, what you will wear.}}%
\verse{\JesusWords{For life is more than food, and the body more than clothing.}}%
\verse{\JesusWords{Consider the ravens, that they neither sow nor reap; to them there is neither storeroom nor barn, and God feeds them. How much more are you worth than the birds?}}%
\verse{\JesusWords{And which of you by\lebnote{“\textit{by}” is supplied as a component of the participle (“being anxious”) which is understood as means} being anxious is able to add an hour\lebnote{Or “a cubit” (the literal meaning); most scholars understand this to refer figuratively to an “hour” of life here, though some take it as a literal measurement of height} to his life span?}}%
\verse{\JesusWords{If then you are not even able to do a very little thing, why are you anxious about the rest?}}%
\verse{\JesusWords{Consider the lilies, how they grow: they do not toil or spin, but I say to you, not even Solomon in all his glory was dressed like one of these.}}%
\verse{\JesusWords{But if God clothes the grass in the field in this way, although it\lebnote{“\textit{although}” is supplied as a component of the participle (“is”) which is understood as concessive} is here today and tomorrow is thrown into the oven, how much more will he do so for\lebnote{The phrase “\textit{will he do so for}” is not in the Greek text but is implied} you, you of little faith?}}%
\verse{\JesusWords{And you, do not consider what you will eat and what you will drink, and do not be anxious.}}%
\verse{\JesusWords{For all the nations of the world seek after these things, and your Father knows that you need these things.}}%
\verse{\JesusWords{But seek his kingdom and these things will be added to you.}}%
\verse{\JesusWords{“Do not be afraid, little flock, because your Father is well pleased to give you the kingdom.}}%
\verse{\JesusWords{Sell your possessions and give charitable gifts. Make for yourselves money bags that do not wear out, an inexhaustible treasure in heaven\lebnote{Or “in the heavens”} where thief does not approach or moth destroy.}}%
\verse{\JesusWords{For where your treasure is, there your heart will be also.}}%
\verseWithHeading{On the Alert for the Master’s Return}{\JesusWords{“You must be prepared for action\lebnote{“your loins must be girded”} and your\lnDCS{} lamps burning.}}%
\verse{\JesusWords{And you, be like people who are waiting for their master when he returns from the wedding feast,\lebnote{Or perhaps simply “feast”} so that when he\lebnote{“\textit{when}” is supplied as a component of the temporal genitive absolute participle (“comes back”)} comes back and knocks, they can open the door\lnDCR{} for him immediately.}}%
\verse{\JesusWords{Blessed are those slaves whom the master will find on the alert when he returns! Truly I say to you that he will dress himself for service and have them recline at the table and will come by and\lebnote{“\textit{and}” supplied because previous participle “will come by” translated as a finite verb} serve them.}}%
\verse{\JesusWords{Even if he should come back in the second or in the third watch of the night and find them\lnDCR{} like this, blessed are they!}}%
\verse{\JesusWords{But understand this, that if the master of the house had known what hour the thief was coming, he would not have left his house to be broken into.}}%
\verse{\JesusWords{You also must be ready, because the Son of Man is coming at an hour that you do not think he will come.”}\lebnote{*The words “\textit{he will come}” are not in the Greek text but are implied}}%
\verseWithHeading{A Faithful Slave and an Unfaithful Slave}{And Peter said, “Lord, are you telling this parable for us, or also for everyone?”}%
\verse{And the Lord said, \JesusWords{“Who then is the faithful wise manager whom the master will put in charge over his servants to give them\lnDCR{} their\lnDCS{} food allowance at the right time?}}%
\verse{\JesusWords{Blessed is that slave whom his master will find so doing when he\lebnote{“\textit{when}” is supplied as a component of the participle (“comes back”) which is understood as temporal} comes back.}}%
\verse{\JesusWords{Truly I say to you that he will put him in charge of all his possessions.}}%
\verse{\JesusWords{But if that slave should say to himself,\lebnote{“in his heart”} ‘My master is taking a long time to return,’ and he begins to beat the male slaves and the female slaves and to eat and drink and get drunk,}}%
\verse{\JesusWords{the master of that slave will come on a day that he does not expect and at an hour that he does not know, and will cut him in two and assign his place with the unbelievers.}}%
\verse{\JesusWords{And that slave who knew the will of his master and did not prepare or do according to his will will be given a severe beating.}\lebnote{“will be beaten much”}}%
\verse{\JesusWords{But the one who did not know and did things deserving blows will be given a light beating.\lebnote{“will be beaten a few times”} And from everyone to whom much has been given, much will be demanded, and from him to whom they entrusted much, they will ask him for even more.}}%
\verseWithHeading{Not Peace, But a Sword of Divisiveness}{\JesusWords{“I have come to bring fire on the earth, and how I wish that it had been kindled already!}}%
\verse{\JesusWords{But I have a baptism to be baptized with, and how I am distressed until it is accomplished!}}%
\verse{\JesusWords{Do you think that I have come to grant peace on the earth? No, I tell you, but rather division!}}%
\verse{\JesusWords{For from now on there will be five in one household, divided three against two and two against three.}}%
\verse{\JesusWords{They will be divided, father against son and son against father, mother against daughter and daughter against mother, mother-in-law against her daughter-in-law and daughter-in-law against mother-in-law.”}}%
\verseWithHeading{The Signs of the Times}{And he also said to the crowds, \JesusWords{“When you see a cloud coming up in the west, you say at once, ‘A rainstorm is coming,’ and so it happens.}}%
\verse{\JesusWords{And when you see the south wind blowing, you say, ‘There will be burning heat,’ and it happens.}}%
\verse{\JesusWords{Hypocrites! You know how to evaluate the appearance of the earth and the sky, but how is it you do not know how to evaluate this present time?}}%
\verseWithHeading{Settle Accounts Quickly}{\JesusWords{And why do you not also judge for yourselves what is right?}}%
\verse{\JesusWords{For as you are going with your accuser before the magistrate, make an effort to come to a settlement with him on the way, so that he will not drag you to the judge, and the judge will hand you over to the bailiff, and the bailiff will throw you into prison.}}%
\verse{\JesusWords{I tell you, you will never get out of there until you have paid back even the last cent!”}\lebnote{“lepton,” a small copper coin worth 1/128 of a denarius}}%
\end{biblechapter}%
\begin{biblechapter}% Luke 13
\verseWithHeading{Repent or Perish}{Now at the same time some had come to tell him about the Galileans whose blood Pilate had mixed with their sacrifices.}%
\verse{And he answered and\lnDCT{} said to them, \JesusWords{“Do you think that these Galileans were sinners worse than all the Galileans, because they suffered these things?}}%
\verse{\JesusWords{No, I tell you, but unless you repent you will all perish as well!}}%
\verse{\JesusWords{Or those eighteen on whom the tower in Siloam fell and killed them — do you think that they were sinners worse than all the people who live in Jerusalem?}}%
\verse{\JesusWords{No, I tell you, but unless you repent, you will all perish as well!”}}%
\verseWithHeading{The Parable of the Barren Fig Tree}{And he told this parable: \JesusWords{“A certain man had a fig tree planted in his vineyard, and he came looking for fruit on it and did not find any.}\lnDCU{}}%
\verse{\JesusWords{So he said to the gardener, ‘Behold, for three years\lebnote{“three years from which”} I have come looking for fruit on this fig tree and did not find any.\lnDCU{} Cut it down!\lebnote{Some manuscripts have “Therefore cut it down!”} Why should it even exhaust the soil?’}}%
\verse{\JesusWords{But he answered and\lnDCT{} said to him, ‘Sir, leave it alone this year also, until I dig around it and put manure on it.}\lnDCU{}}%
\verse{\JesusWords{And if indeed it produces fruit in the coming year, so much the better,\lebnote{*The phrase “\textit{so much the better}” is not in the Greek text but is implied} but if not, you can cut it down.’”}}%
\verseWithHeading{A Woman with a Disabling Spirit Healed}{Now he was teaching in one of the synagogues on the Sabbath.}%
\verse{And behold, a woman was there\lebnote{The phrase “\textit{was there}” is not in the Greek text but is supplied in keeping with English style} who had a spirit that had disabled her\lebnote{“of weakness”} for eighteen years, and she was bent over and not able to straighten herself up completely.\lebnote{Or “at all”}}%
\verse{And when he\lebnote{“\textit{when}” is supplied as a component of the participle (“saw”) which is understood as temporal} saw her, Jesus summoned her\lnDCV{} and said to her, \JesusWords{“Woman, you are freed from your disability!”}}%
\verse{And he placed his\lnDCW{} hands on her, and immediately she straightened up and glorified God.}%
\verse{But the ruler of the synagogue, indignant because Jesus had healed on the Sabbath, answered and\lnDCT{} said to the crowd, “There are six days on which it is necessary to work. Therefore come and\lebnote{“\textit{and}” supplied because previous participle “come” translated as a finite verb} be healed on them, and not on the day of the Sabbath!”}%
\verse{But the Lord answered and said to him, \JesusWords{“Hypocrites! Does not each one of you untie his ox or his\lnDCW{} donkey from the feeding trough on the Sabbath and lead it\lnDCV{} away to water it?}\lnDCU{}}%
\verse{\JesusWords{And this woman, who is a daughter of Abraham, whom Satan bound eighteen\lebnote{“ten and eight”} long years — is it not necessary that she be released from this bond on the day of the Sabbath?”}}%
\verse{And when\lebnote{“\textit{when}” is supplied as a component of the temporal genitive absolute participle (“said”)} he said these things, all those who opposed him were humiliated, and the whole crowd was rejoicing at all the splendid things that were being done by him.}%
\verseWithHeading{The Parable of the Mustard Seed}{Therefore he said, \JesusWords{“What is the kingdom of God like, and to what shall I compare it?}}%
\verse{\JesusWords{It is like a mustard seed that a man took and\lnDCX{} sowed in his own garden, and it grew and became a tree, and the birds of the sky nested in its branches.”}}%
\verseWithHeading{The Parable of the Yeast}{And again he said, \JesusWords{“To what shall I compare the kingdom of God?}}%
\verse{\JesusWords{It is like yeast that a woman took and\lnDCX{} hid in\lebnote{Some manuscripts have “put into”} three measures of wheat flour until the whole batch was leavened.”}}%
\verseWithHeading{The Narrow Door}{And he was going throughout towns and villages, teaching and making his journey toward Jerusalem.}%
\verse{And someone said to him, “Lord, are there only\lebnote{“if”} a few who are saved?” And he said to them,}%
\verse{\JesusWords{“Make every effort to enter through the narrow door, because many, I tell you, will seek to enter and will not be able to,}}%
\verse{\JesusWords{when once the master of the house has gotten up and shut the door, and you begin to stand outside and knock on the door, saying, ‘Lord, open the door\lnDCV{} for us!’ And he will answer and\lebnote{“\textit{and}” supplied because previous participle “answer” translated as a finite verb} say to you, ‘I do not know where you are from!’}}%
\verse{\JesusWords{Then you will begin to say, ‘We ate and drank in your presence, and you taught in our streets!’}}%
\verse{\JesusWords{And he will reply, saying to you, ‘I do not know where you are from! Go away from me, all you evildoers!’\lebnote{“workers of unrighteousness”}}}%
\verse{\JesusWords{In that place there will be weeping and gnashing of teeth, when you see Abraham and Isaac and Jacob and all the prophets in the kingdom of God, but yourselves thrown outside!}}%
\verse{\JesusWords{And they will come from east and west, and from north and south, and will recline at the table in the kingdom of God.}}%
\verse{\JesusWords{And behold, some are last who will be first, and some are first who will be last.”}}%
\verseWithHeading{The Lament over Jerusalem}{At that same hour some Pharisees came up and\lebnote{“\textit{and}” supplied because previous participle “came up” translated as a finite verb} said to him, “Go out and depart from here, because Herod wants to kill you!”}%
\verse{And he said to them, \JesusWords{“Go and\lebnote{“\textit{and}” supplied because previous participle “go” translated as a finite verb} tell that fox, ‘Behold, I am expelling demons and performing healings today and tomorrow, and on the third day I will complete my work.’}\lnDCU{}}%
\verse{\JesusWords{Nevertheless, it is necessary for me to be on the way today and tomorrow and on the next day, because it is not possible for a prophet to perish outside Jerusalem.}}%
\verse{\JesusWords{“Jerusalem, Jerusalem, the one who kills the prophets and stones those who are sent to her! How many times I wanted to gather your children together the way\lebnote{“in the manner in which”} a hen gathers\lebnote{the term “\textit{gathers}” is not in the Greek text but is implied} her own brood under her\lnDCW{} wings, and you were not willing!}}%
\verse{\JesusWords{Behold, your house has been left to you! And I tell you, you will never see me until the time\lebnote{The words “\textit{the time}” are not in the Greek text but are implied} will come when you say, ‘Blessed is the one who comes in the name of the Lord!’”\lebnote{from Ps 118:26}}}%
\end{biblechapter}%
\begin{biblechapter}% Luke 14
\verseWithHeading{A Man Suffering from Edema Healed}{And it happened that when he came to the house of a certain one of the leaders of the Pharisees on a Sabbath to eat a meal,\lebnote{“bread”} they were watching him closely.}%
\verse{And behold, a certain man was in front of him, suffering from edema.}%
\verse{And Jesus answered and\lebnote{“\textit{and}” supplied because previous participle “answered” translated as a finite verb} said to the legal experts and Pharisees, saying, \JesusWords{“Is it permitted to heal on the Sabbath, or not?”}}%
\verse{But they remained silent. And he took hold of him\lnDCY{} and\lebnote{“\textit{and}” supplied because previous participle “took hold of” translated as a finite verb} healed him, and sent him\lnDCY{} away.}%
\verse{And he said to them, \JesusWords{“Who among you, if your\lebnote{The words “\textit{if your}” are not in the Greek text but are implied} son or your ox falls into a well\lebnote{Or “cistern”} on the day of the Sabbath, will not immediately pull him out?”}}%
\verse{And they were not able to make a reply to these things.}%
\verseWithHeading{The Parable of the Guests at the Wedding Feast}{Now he told a parable to those who had been invited when he\lebnote{“\textit{when}” is supplied as a component of the participle (“noticed”) which is understood as temporal} noticed how they were choosing for themselves the places of honor, saying to them,}%
\verse{\JesusWords{“When you are invited by someone to a wedding feast,\lebnote{Or perhaps simply “a feast”} do not recline at the table in the place of honor, lest someone more distinguished than you has been invited by him,}}%
\verse{\JesusWords{and the one who invited you both\lebnote{“and him”} will come and\lebnote{“\textit{and}” supplied because previous participle “will come” translated as a finite verb} say to you, ‘Give the place to this person,’ and then with shame you will begin to take the last place.}}%
\verse{\JesusWords{But when you are invited, go and\lebnote{“\textit{and}” supplied because previous participle “go” translated as a finite verb} recline at the table in the last place, so that when the one who invited you comes, he will say to you, ‘Friend, move up higher.’ Then it will be an honor to you in the presence of all those who are reclining at the table with you.}}%
\verse{\JesusWords{For everyone who exalts himself will be humbled, and the one who humbles himself will be exalted.”}}%
\verseWithHeading{The Parable of the Great Banquet}{And he also said to the one who had invited him, \JesusWords{“When you give a dinner or a banquet, do not invite your friends or your brothers or your relatives or wealthy neighbors, lest they also invite you in return, and repayment come to you.}}%
\verse{\JesusWords{But whenever you give a banquet, invite the poor, the crippled, the lame, the blind,}}%
\verse{\JesusWords{and you will be blessed, because they are not able to repay you. For it will be paid back to you at the resurrection of the righteous.”}}%
\verse{Now when\lebnote{“\textit{when}” is supplied as a component of the participle (“heard”) which is understood as temporal} one of those reclining at the table with him heard these things, he said to him, “Blessed is everyone who\lebnote{“whoever”} will eat bread in the kingdom of God!”}%
\verse{But he said to him, \JesusWords{“A certain man was giving a large banquet and invited many.}}%
\verse{\JesusWords{And he sent his slave at the hour of the banquet to say to those who have been invited, ‘Come, because now it is ready!’}}%
\verse{\JesusWords{And they all alike\lebnote{“by one”} began to excuse themselves. The first said to him, ‘I have purchased a field, and I must\lebnote{“I have necessity”} go out to look at it. I ask you, consider me excused.’}}%
\verse{\JesusWords{And another said, ‘I have purchased five yoke of oxen, and I am going to examine them. I ask you, consider me excused.’}}%
\verse{\JesusWords{And another said, ‘I have married a wife, and for this reason I am not able to come.’}}%
\verse{\JesusWords{And the slave came and\lebnote{“\textit{and}” supplied because previous participle “came” translated as a finite verb} reported these things to his master. Then the master of the house became angry and\lebnote{“\textit{and}” supplied because previous participle “became angry” translated as a finite verb} said to his slave, ‘Go out quickly into the streets and alleys of the city and bring in here the poor and crippled and blind and lame!’}}%
\verse{\JesusWords{And the slave said, ‘Sir, what you ordered has been done, and there is still room.’}}%
\verse{\JesusWords{And the master said to the slave, ‘Go out into the highways and hedges and press them\lnDCY{} to come in, so that my house will be filled!}}%
\verse{\JesusWords{For I say to you that none of those persons who were invited will taste my banquet!’”}}%
\verseWithHeading{The Cost of Discipleship}{Now large crowds were going along with him, and he turned around and\lebnote{“\textit{and}” supplied because previous participle “turned around” translated as a finite verb} said to them,}%
\verse{\JesusWords{“If anyone comes to me and does not hate his own father and mother and wife and children and brothers and sisters, and furthermore, even his own life, he cannot be\lebnote{“he is not able to be”} my disciple.}}%
\verse{\JesusWords{Whoever does not carry his own cross and follow\lebnote{“come after”} me cannot be\lnDCZ{} my disciple.}}%
\verse{\JesusWords{For which of you, wanting to build a tower, does not first sit down and\lnDDA{} calculate the cost to see if he has enough\lnDCY{} to complete it?}\lebnote{“for completion”}}%
\verse{\JesusWords{Otherwise\lebnote{“so that lest”} after\lebnote{“\textit{after}” is supplied as a component of the temporal genitive absolute participle (“has laid”)} he has laid the foundation and is not able to finish it,\lebnote{*supplied from English context} all who see it\lnDCY{} will begin to ridicule him,}}%
\verse{\JesusWords{saying, ‘This man began to build and was not able to finish!’}}%
\verse{\JesusWords{Or what king, going out to engage another king in battle, does not sit down first and\lnDDA{} deliberate whether he is able with ten thousand to oppose the one coming against him with twenty thousand.}}%
\verse{\JesusWords{But if not, while\lebnote{“\textit{while}” is supplied as a component of the temporal genitive absolute participle (“is”)} the other is still far away, he sends an ambassador and\lebnote{“\textit{and}” supplied because previous participle “asks” translated as a finite verb} asks for terms of\lebnote{“the things with reference to”} peace.}}%
\verse{\JesusWords{In the same way, therefore, every one of you who does not renounce all his own possessions cannot be\lnDCZ{} my disciple.}}%
\verse{\JesusWords{“Now salt is good, but if salt becomes tasteless, with what will it be made salty?}}%
\verse{\JesusWords{It is usable neither for the soil nor for the manure pile; they throw it out. The one who has ears to hear, let him hear!”}}%
\end{biblechapter}%
\begin{biblechapter}% Luke 15
\verseWithHeading{The Parable of the Lost Sheep}{Now all the tax collectors and the sinners were drawing near to hear him.}%
\verse{And both the Pharisees and the scribes were complaining, saying, “This man welcomes sinners and eats with them!”}%
\verse{So he told them this parable, saying,}%
\verse{\JesusWords{“What man of you, having a hundred sheep and losing one of them, does not leave the ninety-nine in the grassland and go after the one that was lost until he finds it?}}%
\verse{\JesusWords{And when he\lnDDB{} has found it,\lnDDC{} he places it\lnDDD{} on his shoulders, rejoicing.}}%
\verse{\JesusWords{And when he\lebnote{“\textit{when}” is supplied as a component of the participle (“returns”) which is understood as temporal} returns to his\lnDDE{} home, he calls together his\lnDDE{} friends and neighbors, saying to them, ‘Rejoice with me, because I have found my sheep that was lost!’}}%
\verse{\JesusWords{I tell you that in the same way, there will be more joy in heaven over one sinner who repents than over ninety-nine righteous people who have no need of repentance.}}%
\verseWithHeading{The Parable of the Lost Coin}{\JesusWords{Or what woman who has ten drachmas, if she loses one drachma, does not light a lamp and sweep the house and search carefully until she finds it?}\lnDDC{}}%
\verse{\JesusWords{And when she\lnDDB{} has found it,\lnDDC{} she calls together her\lnDDE{} friends and neighbors, saying, ‘Rejoice with me, because I have found the drachma that I had lost!’}}%
\verse{\JesusWords{In the same way, I tell you, there is joy in the presence of the angels of God over one sinner who repents.”}}%
\verseWithHeading{The Parable of the Lost Son}{And he said, \JesusWords{“A certain man had two sons.}}%
\verse{\JesusWords{And the younger of them said to his\lnDDE{} father, ‘Father, give me the share of the property that is coming to me.’ So he divided his\lnDDE{} assets between them.}}%
\verse{\JesusWords{And after not many days, the younger son gathered everything and\lebnote{“\textit{and}” supplied because previous participle “gathered” translated as a finite verb} went on a journey to a distant country, and there he squandered his wealth by\lebnote{“\textit{by}” is supplied as a component of the adverbial participle of manner (“living”)} living wastefully.}}%
\verse{\JesusWords{And after\lebnote{“\textit{after}” is supplied as a component of the temporal genitive absolute participle (“had spent”)} he had spent everything, there was a severe famine throughout that country, and he began to be in need.}}%
\verse{\JesusWords{And he went and\lebnote{“\textit{and}” supplied because previous participle “went” translated as a finite verb} hired himself out to one of the citizens of that country, and he sent him into his fields to tend pigs.}}%
\verse{\JesusWords{And he was longing to fill his stomach with\lebnote{Some manuscripts have “to stuff himself with”} the carob pods that the pigs were eating, and no one was giving anything\lnDDD{} to him.}}%
\verse{\JesusWords{“But when he\lebnote{“\textit{when}” is supplied as a component of the participle (“came”) which is understood as temporal} came to himself, he said, ‘How many of my father’s hired workers have an abundance of food,\lebnote{“of bread”} and I am dying here from hunger!}}%
\verse{\JesusWords{I will set out and\lnDDF{} go to my father and will say to him, ‘Father, I have sinned against heaven and in your sight!\lnDDG{}}}%
\verse{\JesusWords{I am no longer worthy to be called your son! Make me like one of your hired workers.’}}%
\verse{\JesusWords{And he set out and\lnDDF{} came to his own father. But while\lebnote{“\textit{while}” is supplied as a component of the temporal genitive absolute participle (“away”)} he was still a long way away, his father saw him and had compassion, and ran and embraced him\lebnote{“fell on his neck”} and kissed him.}}%
\verse{\JesusWords{And his\lnDDE{} son said to him, ‘Father, I have sinned against heaven and in your sight!\lnDDG{} I am no longer worthy to be called your son!’}}%
\verse{\JesusWords{But his\lnDDE{} father said to his slaves, ‘Quickly bring out the best robe and put it\lnDDD{} on him, and put a ring on his finger\lebnote{“hand,” but this is a metonymy of whole (“hand”) for part (“finger”)} and sandals on his\lnDDE{} feet!}}%
\verse{\JesusWords{And bring the fattened calf — kill it\lnDDD{} and let us eat and\lebnote{“\textit{and}” supplied because previous participle “eat” translated as a finite verb} celebrate,}}%
\verse{\JesusWords{because this son of mine was dead, and is alive again! He was lost and is found!’ And they began to celebrate.}}%
\verse{\JesusWords{“Now his older son was in the field, and when he came and\lebnote{“\textit{and}” supplied because previous participle “came” translated as a finite verb} approached the house, he heard music and dancing.}}%
\verse{\JesusWords{And he summoned one of the slaves and\lebnote{“\textit{and}” supplied because previous participle “summoned” translated as a finite verb} asked what these things meant.}}%
\verse{\JesusWords{And he said to him, ‘Your brother has come, and your father has killed the fattened calf because he has gotten him back healthy.’}}%
\verse{\JesusWords{But he became angry and did not want to go in. So his father came out and\lebnote{“\textit{and}” supplied because previous participle “came out” translated as a finite verb} began to implore\lebnote{Imperfect tense as ingressive (“began to implore”)} him.}}%
\verse{\JesusWords{But he answered and\lebnote{“\textit{and}” supplied because previous participle “answered” translated as a finite verb} said to his father, ‘Behold, so many years I have served you, and have never disobeyed your command! And you never gave me a young goat so that I could celebrate with my friends!}}%
\verse{\JesusWords{But when this son of yours returned — who has consumed your assets with prostitutes — you killed the fattened calf for him!’}}%
\verse{\JesusWords{But he said to him, ‘Child, you are always with me, and everything I have belongs to you.}\lebnote{“all my \textit{things} are yours”}}%
\verse{\JesusWords{But it was necessary to celebrate and to rejoice, because this brother of yours was dead, and is alive, and was lost, and is found!’”}}%
\end{biblechapter}%
\begin{biblechapter}% Luke 16
\verseWithHeading{The Parable of the Dishonest Manager}{And he also said to the disciples, \JesusWords{“A certain man was rich, who had a manager. And charges were brought to him that this person was squandering his possessions.}}%
\verse{\JesusWords{And he summoned him and\lnDDH{} said to him, ‘What is this I hear about you? Give the account of your management, because you can no longer manage.’}}%
\verse{\JesusWords{And the manager said to himself, ‘What should I do, because my master is taking away the management from me? I am not strong enough to dig; I am ashamed to beg.}}%
\verse{\JesusWords{I know what I should do, so that when I am removed from the management they will welcome me into their homes!’}}%
\verse{\JesusWords{And he summoned each one of his own master’s debtors and\lnDDH{} said to the first, ‘How much do you owe my master?’}}%
\verse{\JesusWords{And he said, ‘A hundred measures of olive oil.’ So he said to him, ‘Take your promissory note and sit down quickly and\lebnote{“\textit{and}” supplied because previous participle “sit down” translated as a finite verb} write fifty.’}}%
\verse{\JesusWords{Then he said to another, ‘And how much do you owe?’ And he said, ‘A hundred measures of wheat.’ He said to him, ‘Take your promissory note and write eighty.’}}%
\verse{\JesusWords{And the master praised the dishonest manager, because he had acted shrewdly. For the sons of this age are shrewder than the sons of light with regard to their own generation.}\lebnote{Or “kind”}}%
\verse{\JesusWords{And I tell you, make friends for yourselves by means of unrighteous wealth, so that when it runs out they will welcome you into the eternal dwellings.}}%
\verse{\JesusWords{“The one who is faithful in very little is also faithful in much, and the one who is dishonest in very little is also dishonest in much.}}%
\verse{\JesusWords{If then you have not been faithful with unrighteous wealth, who will entrust to you the true riches?}\lebnote{*The word “\textit{riches}” is not in the Greek text but is implied}}%
\verse{\JesusWords{And if you have not been faithful with what belongs to another, who will give you your own?}}%
\verse{\JesusWords{No domestic slave is able to serve two masters, for either he will hate the one and love the other, or he will be devoted to one and will despise the other. You are not able to serve God and money.”}}%
\verseWithHeading{Hypocrisy, Law, and the Kingdom of God}{Now the Pharisees, who were lovers of money, heard all these things, and they ridiculed him.}%
\verse{And he said to them, \JesusWords{“You are the ones who justify themselves in the sight of men, but God knows your hearts! For what is considered exalted among men is an abomination in the sight of God.}}%
\verse{\JesusWords{“The law and the prophets were until John; from that time on the kingdom of God has been proclaimed, and everyone is urgently pressed\lebnote{The verb is translated here as a passive; some English versions translate the verb as active (“forces \textit{their way} into it”)} into it.}}%
\verse{\JesusWords{But it is easier for heaven and earth to pass away than for one stroke of a letter of the law to become invalid.}}%
\verseWithHeading{On Divorce}{\JesusWords{“Everyone who divorces his wife and marries another commits adultery, and the one who marries a woman divorced from her husband commits adultery.}}%
\verseWithHeading{The Rich Man and Lazarus}{\JesusWords{“Now a certain man was rich, and dressed in purple cloth and fine linen, feasting sumptuously every day.}}%
\verse{\JesusWords{And a certain poor man named\lebnote{“by name”} Lazarus, covered with sores, lay at his gate,}}%
\verse{\JesusWords{and was longing to be filled with what fell from the table of the rich man. But even the dogs came and\lebnote{“\textit{and}” supplied because previous participle “came” translated as a finite verb} licked his sores.}}%
\verse{\JesusWords{Now it happened that the poor man died, and he was carried away by the angels to Abraham’s side.\lebnote{“the bosom of Abraham”} And the rich man also died and was buried.}}%
\verse{\JesusWords{And in Hades he lifted up his eyes as he\lebnote{“\textit{as}” is supplied as a component of the participle (“was”) which is understood as temporal} was in torment and\lebnote{“\textit{and}” supplied because previous participle “lifted up” translated as a finite verb} saw Abraham from a distance, and Lazarus at his side.}\lebnote{“in his bosom”}}%
\verse{\JesusWords{And he called out and\lebnote{“\textit{and}” supplied because previous participle “called out” translated as a finite verb} said, ‘Father Abraham, have mercy on me, and send Lazarus so that he could dip the tip of his finger in water and cool my tongue, because I am suffering pain in this flame!’}}%
\verse{\JesusWords{But Abraham said, ‘Child, remember that you received your good things during your life, and Lazarus likewise bad things. But now he is comforted here, but you are suffering pain.}}%
\verse{\JesusWords{And in addition to all these things, a great chasm has been established between us and you, so that those who want to cross over from here to you are not able to do so,\lebnote{*The words “to do so” are not in the Greek text but are implied} nor can they cross over from there to us.’}}%
\verse{\JesusWords{So he said, ‘Then I ask you, father, that you send him to my father’s house,}}%
\verse{\JesusWords{for I have five brothers, so that he could warn them, in order that they also should not come to this place of torment!’}}%
\verse{\JesusWords{But Abraham said, ‘They have Moses and the prophets; they must listen to them.’}}%
\verse{\JesusWords{And he said, ‘No, father Abraham, but if someone from the dead goes to them, they will repent!’}}%
\verse{\JesusWords{But he said to him, ‘If they do not listen to Moses and the prophets, neither will they be convinced if someone rises from the dead.’”}}%
\end{biblechapter}%
\begin{biblechapter}% Luke 17
\verseWithHeading{Sin, Forgiveness, Faith, and Service}{\JesusWords{And he said to his disciples, “It is impossible for causes for stumbling not to come, but woe to him through whom they come!}}%
\verse{\JesusWords{It would be better for him if a millstone\lebnote{“a stone belonging to a mill”} is placed around his neck and he is thrown into the sea than that he causes one of these little ones to sin.}}%
\verse{\JesusWords{“Be concerned about yourselves! If your brother sins, rebuke him, and if he repents, forgive him.}}%
\verse{\JesusWords{And if he sins against you seven times in a day, and seven times he returns to you saying, ‘I repent,’ you must forgive him.”}}%
\verse{And the apostles said to the Lord, “Increase our faith!”}%
\verse{So the Lord said, \JesusWords{“If you have faith like a mustard seed, you could say to this mulberry tree, ‘Be uprooted and planted in the sea,’ and it would obey you.}}%
\verse{\JesusWords{“And which of you who has a slave plowing or shepherding sheep\lnDDI{} who comes in from the field will say to him, ‘Come here at once and\lebnote{“\textit{and}” supplied because previous participle “come here” translated as a finite verb} recline at the table’?}}%
\verse{\JesusWords{Will he not rather say to him, ‘Prepare something that I may eat, and dress yourself to serve me while I eat and drink, and after these things you will eat and drink.’}}%
\verse{\JesusWords{He will not be grateful\lebnote{“have gratitude”} to the slave because he did what was ordered, will he?}\lebnote{*The negative construction in Greek anticipates a negative answer here, indicated by “\textit{will he}”}}%
\verse{\JesusWords{Thus you also, when you have done all the things you were ordered to do,\lebnote{“things that were ordered to you”} say, ‘We are unworthy slaves; we have done what we were obligated to do.’”}}%
\verseWithHeading{Ten Lepers Cleansed}{And it happened that while traveling toward Jerusalem, he was passing through the region between\lebnote{“through the midst”} Samaria and Galilee.}%
\verse{And as\lebnote{“\textit{as}” is supplied as a component of the temporal genitive absolute participle (“was entering”)} he was entering into a certain village, ten men met him\lnDDI{}\lebnote{Some manuscripts explicitly state “him”} — lepers, who stood at a distance.}%
\verse{And they raised their voices, saying, “Jesus, Master, have mercy on us!”}%
\verse{And when he\lnDDJ{} saw them\lnDDI{} he said to them, \JesusWords{“Go and\lebnote{“\textit{and}” supplied because previous participle “go” translated as a finite verb} show yourselves to the priests.”} And it happened that as they were going, they were cleansed.}%
\verse{But one of them, when he\lnDDJ{} saw that he was healed, turned back, praising God with a loud voice.}%
\verse{And he fell on his face at his feet, giving thanks to him. And he was a Samaritan.}%
\verse{So Jesus answered and\lnDDK{} said, \JesusWords{“Were not ten cleansed? And where are the nine?}}%
\verse{\JesusWords{Was no one found to turn back and\lebnote{“\textit{and}” supplied because previous participle “turn back” translated as an infinitive} give praise to God except this foreigner?”}}%
\verse{And he said to him, \JesusWords{“Get up and\lebnote{“\textit{and}” supplied because previous participle “get up” translated as a finite verb} go your way. Your faith has saved you.”}}%
\verseWithHeading{The Coming of the Kingdom of God}{Now when he\lebnote{“\textit{when}” is supplied as a component of the participle (“was asked”) which is understood as temporal} was asked by the Pharisees when the kingdom of God would come, he answered them and said, \JesusWords{“The kingdom of God does not come with things that can be observed,\lebnote{“observation”}}}%
\verse{\JesusWords{nor will they say, ‘Behold, here it is!’ or ‘There!’ For behold, the kingdom of God is in your midst.”}}%
\verseWithHeading{The Coming of the Son of Man}{And he said to the disciples, \JesusWords{“Days will come when you will desire to see one of the days of the Son of Man, and you will not see it.}\lnDDL{}}%
\verse{\JesusWords{And they will say to you, ‘Behold, there!’ ‘Behold, here!’\lebnote{Some manuscripts have “‘Behold, there!’ or ‘Behold, here!’”} Do not go out or run after them!\lnDDL{}}}%
\verse{\JesusWords{For just as the lightning shines forth, flashing from one place under heaven to another place under heaven, so the Son of Man will be in his day.}}%
\verse{\JesusWords{But first it is necessary for him to suffer many things, and to be rejected by this generation.}}%
\verse{\JesusWords{And just as it was in the days of Noah, so also it will be in the days of the Son of Man —}}%
\verse{\JesusWords{they were eating, they were drinking, they were marrying, they were being given in marriage, until the day Noah entered into the ark, and the flood came and destroyed them all.}}%
\verse{\JesusWords{Likewise, just as it was in the days of Lot — they were eating, they were drinking, they were buying, they were selling, they were planting, they were building.}}%
\verse{\JesusWords{But on the day that Lot went out from Sodom, it rained fire and sulphur from heaven and destroyed them all.}}%
\verse{\JesusWords{It will be just the same\lebnote{“according to the same”} on the day that the Son of Man is revealed.}}%
\verse{\JesusWords{On that day, whoever is on the housetop and his goods are in the house must not come down to take them away. And likewise the one who is in the field must not turn back}}%
\verse{\JesusWords{Remember Lot’s wife!}}%
\verse{\JesusWords{Whoever seeks to preserve his life will lose it, but whoever loses it\lnDDI{} will keep it.}}%
\verse{\JesusWords{I tell you that in that night there will be two in one bed; one will be taken and the other will be left.}}%
\verse{\JesusWords{There will be two women\lebnote{“\textit{women}” is supplied because the form (“two”) is feminine gender in Greek} grinding at the same place; one will be taken and the other will be left.”}\lebnote{A few manuscripts add v. 36 (with some variations): “There will be two in the field; one will be taken and the other will be left.”}}%
\verse{}%
\verse{And they answered and\lnDDK{} said to him, “Where, Lord?” So he said to them, \JesusWords{“Where the dead body is, there also the vultures will be gathered.”}}%
\end{biblechapter}%
\begin{biblechapter}% Luke 18
\verseWithHeading{The Parable of the Unjust Judge}{And he told them a parable to show that they must always pray and not be discouraged,}%
\verse{saying, \JesusWords{“There was a certain judge in a certain town who did not fear God and did not respect people.}}%
\verse{\JesusWords{And there was a widow in that town, and she kept coming to him, saying, ‘Grant me justice against my adversary!’}}%
\verse{\JesusWords{And he was not willing for a time, but after these things he said to himself, ‘Even if I do not fear God or respect people,}}%
\verse{\JesusWords{yet because this widow is causing trouble for me, I will grant her justice, so that she does not wear me down in the end by her\lebnote{“\textit{by}” is supplied as a component of the participle (“coming back”) which is understood as means} coming back!’”}}%
\verse{And the Lord said, \JesusWords{“Listen to what the unrighteous judge is saying!}}%
\verse{\JesusWords{And will not God surely see to it that justice is done\lebnote{“carry out the giving of justice”} to his chosen ones who cry out to him day and night, and will he delay toward them?}}%
\verse{\JesusWords{I tell you that he will see to it that justice is done\lebnote{“he will carry out the giving of justice”} for them soon! Nevertheless, when\lebnote{“\textit{when}” is supplied as a component of the participle (“comes”) which is understood as temporal} the Son of Man comes, then will he find faith on earth?”}}%
\verseWithHeading{The Parable of the Pharisee and the Tax Collector}{And he also told this parable to some who trusted in themselves that they were righteous, and looked down on everyone else:\lebnote{“the rest”}}%
\verse{\JesusWords{“Two men went up to the temple to pray, one a Pharisee and the other a tax collector.}}%
\verse{\JesusWords{The Pharisee stood and\lebnote{“\textit{and}” supplied because previous participle “stood” translated as a finite verb} prayed these things with reference to himself: ‘God, I give thanks to you that I am not like other people — swindlers, unrighteous people, adulterers, or even like this tax collector!}}%
\verse{\JesusWords{I fast twice a week; I give a tenth of all that I get.’}}%
\verse{\JesusWords{But the tax collector, standing far away, did not want even to raise his eyes to heaven, but was beating his breast, saying, ‘God, be merciful to me, a sinner!’}}%
\verse{\JesusWords{I tell you, this man went down to his house justified rather than that one! For everyone who exalts himself will be humbled, but the one who humbles himself will be exalted.”}}%
\verseWithHeading{Little Children Brought to Jesus}{Now they were bringing even their\lebnote{“the”: the Greek article is used here as a possessive pronoun} babies to him so that he could touch them. But when\lnDDM{} the disciples saw it,\lnDDN{} they rebuked them.}%
\verse{But Jesus called them to himself, saying, \JesusWords{“Allow the children to come to me, and do not forbid them, for to such belongs\lebnote{“for of such is”} the kingdom of God.}}%
\verse{\JesusWords{Truly I say to you, whoever does not welcome the kingdom of God like a young child will never enter into it.”}}%
\verseWithHeading{A Rich Young Man}{And a certain ruler asked him, saying, “Good Teacher, by\lebnote{“\textit{by}” is supplied as a component of the participle (“doing”) which is understood as means} doing what will I inherit eternal life?”}%
\verse{And Jesus said to him, \JesusWords{“Why do you call me good? No one is good except God alone.}}%
\verse{\JesusWords{You know the commandments: ‘Do not commit adultery, do not murder, do not steal, do not give false testimony, honor your father and mother.’”\lebnote{from Exod 20:12–16; Deut 5:16–20}}}%
\verse{And he said, “All these I have observed from my\lebnote{Some manuscripts omit “my”} youth.”}%
\verse{And when he\lnDDO{} heard this,\lnDDN{} Jesus said to him, \JesusWords{“You still lack one thing:\lebnote{“one still lacking to you”} Sell all that you have, and distribute the proceeds\lnDDP{} to the poor — and you will have treasure in heaven — and come, follow me.”}}%
\verse{But when he\lnDDO{} heard these things he became very sad, because he was extremely wealthy.}%
\verse{And Jesus took notice of him\lebnote{Some manuscripts add “becoming very sad”} and\lebnote{“\textit{and}” supplied because previous participle “took notice of” translated as a finite verb} said, \JesusWords{“How difficult it is for\lebnote{“with difficulty”} those who possess wealth to enter into the kingdom of God!}}%
\verse{\JesusWords{For it is easier for a camel to go through the eye of a needle than for a rich person to enter into the kingdom of God.}}%
\verse{So those who heard this\lnDDP{} said, “And who can be saved?”}%
\verse{But he said, \JesusWords{“What is impossible with men is possible with God.”}}%
\verse{And Peter said, “Behold, we have left all that is ours\lebnote{“our own”} and\lebnote{“\textit{and}” supplied because previous participle “have left” translated as a finite verb} followed you.”}%
\verse{And he said to them, \JesusWords{“Truly I say to you that there is no one who has left house or wife or brothers or parents or children on account of the kingdom of God,}}%
\verse{\JesusWords{who will not receive many times more in this time and in the age to come, eternal life.”}}%
\verseWithHeading{Jesus Predicts His Death and Resurrection a Third Time}{And taking aside the twelve, he said to them, \JesusWords{“Behold, we are going up to Jerusalem, and all the things that are written by the prophets with reference to the Son of Man will be accomplished.}}%
\verse{\JesusWords{For he will be handed over to the Gentiles and will be mocked and mistreated and spit on,}}%
\verse{\JesusWords{and after\lebnote{“\textit{after}” is supplied as a component of the participle (“flogging”) which is understood as temporal} flogging him\lnDDP{} they will kill him, and on the third day he will rise.”}}%
\verse{And they understood none of these things, and this saying was concealed from them, and they did not comprehend the things that were said.}%
\verseWithHeading{A Blind Man Healed at Jericho}{Now it happened that as he drew near to Jericho, a certain blind man was sitting on the side of the road begging.}%
\verse{And when he\lnDDO{} heard a crowd going by, he inquired what this meant.}%
\verse{And they told him, “Jesus the Nazarene is passing by.”}%
\verse{And he called out, saying, “Jesus, Son of David, have mercy on me!”}%
\verse{And those who were in front rebuked him, that he should be silent, but he cried out even more loudly,\lebnote{“by much more”} “Son of David, have mercy on me!”}%
\verse{So Jesus stopped and\lebnote{“\textit{and}” supplied because previous participle “stopped” translated as a finite verb} ordered him to be brought to him. And when\lebnote{“\textit{when}” is supplied as a component of the temporal genitive absolute participle (“approached”)} he approached, he asked him,}%
\verse{\JesusWords{“What do you want me to do for you?”} And he said, “Lord, that I may regain my sight.}%
\verse{And Jesus said to him, \JesusWords{“Regain your sight! Your faith has saved you.”}}%
\verse{And immediately he regained his sight and began to follow\lebnote{Imperfect tense as ingressive (“began to follow”)} him, glorifying God. And all the people, when they\lnDDM{} saw it,\lnDDN{} gave praise to God.}%
\end{biblechapter}%
\begin{biblechapter}% Luke 19
\verseWithHeading{Jesus and Zacchaeus}{And he entered and\lnDDQ{} traveled through Jericho.}%
\verse{And there was\lebnote{“behold”} a man named\lebnote{“called by name”} Zacchaeus, and he was a chief tax collector, and he was rich.}%
\verse{And he was seeking to see Jesus — who he was — and he was not able to as a result of the crowd, because he was short in stature.}%
\verse{And he ran on ahead and\lebnote{“\textit{and}” supplied because previous participle “ran on” translated as a finite verb} climbed up into a sycamore tree so that he could see him, because he was going to go through that way.}%
\verse{And when he came to the place, Jesus looked up and\lebnote{“\textit{and}” supplied because previous participle “looked up” translated as a finite verb} said to him, \JesusWords{“Zacchaeus, come down quickly, because it is necessary for me to stay at your house today!”}}%
\verse{And he came down quickly and welcomed him joyfully.}%
\verse{And when they\lebnote{“\textit{when}” is supplied as a component of the participle (“saw”) which is understood as temporal} saw it,\lnDDR{} they all began to complain,\lebnote{Imperfect tense (“began to complain”)} saying, “He has gone in to find lodging with a man who is a sinner!”}%
\verse{And Zacchaeus stopped and\lebnote{“\textit{and}” supplied because previous participle “stopped” translated as a finite verb} said to the Lord, “Behold, half of my possessions, Lord, I am giving to the poor, and if I have extorted anything from anyone, I am paying it\lnDDS{} back four times as much!”}%
\verse{And Jesus said to him, \JesusWords{“Today salvation has come to this house, because he too is a son of Abraham.}}%
\verse{\JesusWords{For the Son of Man came to seek and to save those who are lost.”}}%
\verseWithHeading{The Parable of the Ten Minas}{Now while\lebnote{“\textit{while}” is supplied as a component of the temporal genitive absolute participle (“were listening to”)} they were listening to these things, he went on and\lebnote{“\textit{and}” supplied because previous participle “went on” translated as a finite verb} told a parable, because he was near Jerusalem and they thought that the kingdom of God was going to appear immediately.}%
\verse{Therefore he said, \JesusWords{“A certain nobleman traveled to a distant country to receive for himself a kingdom and to return.}}%
\verse{\JesusWords{And summoning ten of his own slaves, he gave them ten minas\lebnote{A Greek monetary unit equal to 100 drachmas} and said to them, ‘Do business until I come back.’}\lebnote{“in which \textit{time} I am coming back”}}%
\verse{\JesusWords{But his citizens hated him, and sent a delegation after him, saying, ‘We do not want this man to be king over us!’}}%
\verse{\JesusWords{And it happened that when he returned after\lebnote{“\textit{after}” is supplied as a component of the participle (“receiving”) which is understood as temporal} receiving the kingdom, he ordered these slaves to whom he had given the money to be summoned to him, so that he could know what they had gained by trading.}}%
\verse{\JesusWords{So the first arrived, saying, ‘Sir, your mina has made ten minas more!’}}%
\verse{\JesusWords{And he said to him, ‘Well done, good slave! Because you have been faithful in a very small thing, have authority\lebnote{“be having authority”} over ten cities.’}}%
\verse{\JesusWords{And the second came, saying, ‘Sir, your mina has made five minas.’}}%
\verse{\JesusWords{So he said to this one also, ‘And you be over five cities.’}}%
\verse{\JesusWords{And another came, saying, ‘Sir, behold your mina, which I had put away for safekeeping in a piece of cloth.}}%
\verse{\JesusWords{For I was afraid of you, because you are a severe man — you withdraw what you did not deposit, and you reap what you did not sow!’}}%
\verse{\JesusWords{He said to him, ‘By your own words\lebnote{“from your \textit{own} mouth”} I will judge you, wicked slave! You knew that I am a severe man, withdrawing what I did not deposit and reaping what I did not sow.}}%
\verse{\JesusWords{And why did you not give my money to the bank, and I, when I\lebnote{“\textit{when}” is supplied as a component of the participle (“returned”) which is understood as temporal} returned, would have collected it with interest?’}}%
\verse{\JesusWords{And to the bystanders he said, ‘Take away from him the mina and give it\lnDDS{} to the one who has the ten minas!’}}%
\verse{\JesusWords{And they said to him, ‘Sir, he has ten minas.’}}%
\verse{\JesusWords{‘I tell you that to everyone who has, more will be given. But from the one who does not have, even what he has will be taken away.}}%
\verse{\JesusWords{But these enemies of mine who did not want me to be king over them — bring them\lnDDS{} here and slaughter them in my presence!’”}}%
\verseWithHeading{The Triumphal Entry}{And after he\lebnote{“\textit{after}” is supplied as a component of the participle (“had said”) which is understood as temporal} had said these things, he traveled on ahead, going up to Jerusalem.}%
\verse{And it happened that when he drew near to Bethphage and Bethany, to the hill\lebnote{Or “mountain”} called the Mount of Olives, he sent two of the disciples,}%
\verse{saying, ‘Go into the village in front of you, in which as you\lebnote{“\textit{as}” is supplied as a component of the participle (“enter”) which is understood as temporal} enter you will find a colt tied, on which no person has ever\lebnote{“no one of men ever”} sat, and untie it and\lebnote{“\textit{and}” supplied because previous participle “untie” translated as a finite verb} bring it.\lnDDR{}}%
\verse{\JesusWords{And if anyone asks you, ‘Why are you untying it?’\lnDDR{} you will say this: ‘The Lord has need of it.’”}}%
\verse{So those who were sent went and\lebnote{“\textit{and}” supplied because previous participle “went” translated as a finite verb} found it\lnDDS{} just as he had told them.}%
\verse{And as\lebnote{“\textit{as}” is supplied as a component of the temporal genitive absolute participle (“were untying”)} they were untying the colt, its owners said to them, ‘Why are you untying the colt?’}%
\verse{So they said, ‘The Lord has need of it.’}%
\verse{And they brought it to Jesus, and throwing their cloaks on the colt, they put Jesus on it.}%
\verse{And as\lebnote{“\textit{as}” is supplied as a component of the temporal genitive absolute participle (“was going along”)} he was going along, they were spreading out their cloaks on the road.}%
\verse{Now as\lebnote{“\textit{as}” is supplied as a component of the temporal genitive absolute participle (“was drawing near”)} he was drawing near by this time to the descent from the Mount of Olives, the whole crowd of the disciples began rejoicing to praise God with a loud voice for all the miracles that they had seen,}%
\verse{saying, “Blessed is the king, the one who comes in the name of the Lord!\lebnote{from Ps 118:26, with “the king” added as a clarification} Peace in heaven and glory in the highest!”}%
\verse{And some of the Pharisees from the crowd said to him, “Teacher, rebuke your disciples!”}%
\verse{And he answered and\lebnote{“\textit{and}” supplied because previous participle “answered” translated as a finite verb} said, \JesusWords{“I tell you that\lebnote{Some manuscripts omit “that”} if these keep silent, the stones will cry out!”}}%
\verseWithHeading{Jesus Weeps over Jerusalem}{And when he approached and\lebnote{“\textit{and}” supplied because previous participle “approached” translated as a finite verb} saw the city, he wept over it,}%
\verse{saying, \JesusWords{“If you had known on this day — even you — the things that make for peace! But now they are hidden from your eyes.}}%
\verse{\JesusWords{For days will come upon you and your enemies will put up an embankment\lebnote{Or “a palisade” (the term can refer to either a wooden or an earthen barricade)} against you, and will surround you and press you hard from all directions.}}%
\verse{\JesusWords{And they will raze you to the ground, you and your children within you, and will not leave a stone upon a stone within you, because\lebnote{“in return for which”} you did not recognize the time of your visitation.”}}%
\verseWithHeading{The Cleansing of the Temple}{And he entered into the temple courts\lebnote{“\textit{courts}” is supplied to distinguish this area from the interior of the temple building itself} and\lnDDQ{} began to drive out those who were selling,}%
\verse{saying to them, \JesusWords{“It is written, ‘And my house will be a house of prayer,’\lebnote{from Isa 56:7} but you have made it a cave of robbers!”}}%
\verse{And he was teaching every day in the temple courts,\lebnote{*Here “\textit{courts}” is supplied to distinguish this area from the interior of the temple building itself} and the chief priests and the scribes and the most prominent men of the people were seeking to destroy him.}%
\verse{And they did not find anything they could do, because all the people were paying close attention to what they were hearing from him.}%
\end{biblechapter}%
\begin{biblechapter}% Luke 20
\verseWithHeading{Jesus’ Authority Challenged}{And it happened that on one of the days while\lebnote{“\textit{while}” is supplied as a component of the temporal genitive absolute participle (“was teaching”)} he was teaching the people in the temple courts\lebnote{“\textit{courts}” is supplied to distinguish this area from the interior of the temple building itself} and proclaiming the gospel, the chief priests and the scribes approached together with the elders}%
\verse{and said, saying to him, “Tell us, by what authority you are doing these things, or who is the one who gave you this authority?}%
\verse{And he answered and\lnDDT{} said to them, \JesusWords{“I also will ask you a question, and you tell me:}}%
\verse{\JesusWords{The baptism of John — was it from heaven or from men?}}%
\verse{And they discussed this\lnDDU{} with one another, saying, “If we say ‘From heaven,’ he will say, ‘Why did you not believe him?’}%
\verse{But if we say, ‘From men,’ all the people will stone us to death, because they are convinced that John was a prophet.”}%
\verse{And they replied that they did not know where it was from.}%
\verse{And Jesus said to them, \JesusWords{“Neither will I tell you by what authority I am doing these things.”}}%
\verseWithHeading{The Parable of the Tenant Farmers in the Vineyard}{And he began to tell the people this parable: \JesusWords{“A man\lebnote{Some manuscripts have “A certain man”} planted a vineyard, and leased it to tenant farmers, and went on a journey for a long time.}}%
\verse{\JesusWords{And at the proper time he sent a slave to the tenant farmers, so that they would give him some of the fruit of the vineyard. But the tenant farmers sent him away empty-handed after\lebnote{“\textit{after}” is supplied as a component of the participle (“beating”) which is understood as temporal} beating him.}\lnDDV{}}%
\verse{\JesusWords{And he proceeded to send another slave, but they beat and dishonored that one also, and\lebnote{“\textit{and}” supplied because two previous participles “beat” and “dishonored” translated as finite verbs} sent him\lnDDU{} away empty-handed.}}%
\verse{\JesusWords{And he proceeded to send a third, but they wounded and\lebnote{“\textit{and}” supplied because previous participle “wounded” translated as a finite verb} threw out this one also.}}%
\verse{\JesusWords{So the owner of the vineyard said, ‘What should I do? I will send my beloved son; perhaps they will respect him.’}}%
\verse{\JesusWords{But when\lebnote{“\textit{when}” is supplied as a component of the participle (“saw”) which is understood as temporal} the tenant farmers saw him, they began to reason\lebnote{Imperfect tense as ingressive (“began to reason”)} with one another, saying, ‘This is the heir. Let us kill him so that the inheritance will become ours!’}}%
\verse{\JesusWords{And they threw him out of the vineyard and\lebnote{“\textit{and}” supplied because previous participle “threw” translated as a finite verb} killed him.\lnDDV{} What then will the owner of the vineyard do to them?}}%
\verse{\JesusWords{He will come and destroy those tenant farmers and give the vineyard to others.”} And when they\lebnote{“\textit{when}” is supplied as a component of the participle (“heard”) which is understood as temporal} heard this,\lnDDV{} they said, “May this never happen!”}%
\verse{But he looked intently at them and\lebnote{“\textit{and}” supplied because previous participle “looked intently at” translated as a finite verb} said, \JesusWords{“What then is this that is written: ‘The stone which the builders rejected, this has become the cornerstone.’}\lebnote{“the head of the corner”}}%
\verse{\JesusWords{Everyone who falls on that stone will be broken to pieces, and the one on whom it falls — it will crush him!”}}%
\verse{And the scribes and the chief priests sought to lay their\lebnote{“the”: the Greek article is used here as a possessive pronoun} hands on him at that same hour, and they were afraid of the people, for they knew that he had told this parable with reference to them.}%
\verseWithHeading{Paying Taxes to Caesar}{And they watched him\lnDDU{} closely and\lebnote{“\textit{and}” supplied because previous participle “watched closely” translated as a finite verb} sent spies who pretended they were upright, in order that they could catch him in a statement, so that they could hand him over to the authority and the jurisdiction of the governor.}%
\verse{And they asked him, saying, “Teacher, we know that you speak and teach rightly, and do not show partiality,\lebnote{“receive face”} but teach the way of God in truth.}%
\verse{Is it permitted for us to pay taxes\lebnote{Or “the tribute tax”} to Caesar or not?”}%
\verse{But seeing through their craftiness, he said to them,}%
\verse{\JesusWords{“Show me a denarius! Whose image and inscription does it have?”} And they answered and\lnDDT{} said,\lebnote{Some manuscripts have “And they said”} “Caesar’s.”}%
\verse{So he said to them, \JesusWords{“Well then, give to Caesar the things of Caesar, and to God the things of God!”}}%
\verse{And they were not able to catch him\lnDDU{}\lebnote{Some manuscripts explicitly state “him”} in a statement in the sight of the people, and astonished at his answer, they became silent.}%
\verseWithHeading{A Question About Marriage and the Resurrection}{Now some of the Sadducees — who deny that there is a resurrection\lebnote{“resurrection not to exist”} — came up and\lebnote{“\textit{and}” supplied because previous participle “came up” translated as a finite verb} asked him,}%
\verse{saying, “Teacher, Moses wrote for us if someone’s brother dies having a wife, and this man is childless, that his brother should take the wife and father\lebnote{“raise up”} descendants for his brother.}%
\verse{Now there were seven brothers, and the first took a wife and\lebnote{“\textit{and}” supplied because previous participle “took” translated as a finite verb} died childless,}%
\verse{and the second,}%
\verse{and the third took her, and likewise also the seven did not leave children and died.}%
\verse{Finally the woman also died.}%
\verse{Therefore in the resurrection, the woman — whose wife will she be? For the seven had her as wife.”}%
\verse{And Jesus said to them, \JesusWords{“The sons of this age marry and are given in marriage,}}%
\verse{\JesusWords{but those who are considered worthy to attain to that age and to the resurrection from the dead neither marry nor are given in marriage,}}%
\verse{\JesusWords{for they are not even able to die any longer, because they are like the angels and are sons of God, because they\lebnote{“\textit{because}” is supplied as a component of the participle (“are”) which is understood as causal} are sons of the resurrection.}}%
\verse{\JesusWords{But that the dead are raised, even Moses revealed in the passage about\lebnote{The words “\textit{the passage about}” are not in the Greek text but are implied; here a common form of rabbinic citation is being used to refer to an Old Testament passage} the bush, when he calls the Lord the God of Abraham and the God of Isaac and the God of Jacob.}}%
\verse{\JesusWords{Now he is not God of the dead, but of the living, for all live to him!”}}%
\verse{And some of the scribes answered and\lnDDT{} said, “Teacher, you have spoken well.”}%
\verse{For they no longer dared to ask him anything.}%
\verseWithHeading{David’s Son and Lord}{\JesusWords{But he said to them, “In what sense do they say that the Christ is David’s son?}}%
\verse{\JesusWords{For David himself says in the book of Psalms, ‘The Lord said to my Lord, “Sit at my right hand,}}%
\verse{\JesusWords{until I make your enemies a footstool for your feet.”’\lebnote{from Ps 110:1} }}%
\verse{\JesusWords{David therefore calls him ‘Lord,’ and how is he his son?”}}%
\verseWithHeading{Warning to Beware of the Scribes}{And while\lebnote{“\textit{while}” is supplied as a component of the temporal genitive absolute participle (“were listening”)} all the people were listening, he said to the disciples,\lebnote{Some manuscripts have “to his disciples”}}%
\verse{\JesusWords{“Beware of the scribes, who like walking around in long robes and who love greetings in the marketplaces and the best seats in the synagogues and the places of honor at banquets,}}%
\verse{\JesusWords{who devour the houses of widows and pray lengthy prayers for the sake of appearance. These will receive more severe condemnation!”}}%
\end{biblechapter}%
\begin{biblechapter}% Luke 21
\verseWithHeading{A Poor Widow’s Offering}{And he looked up and\lebnote{“\textit{and}” supplied because previous participle “looked up” translated as a finite verb} saw the rich putting their gifts into the contribution box,}%
\verse{and he saw a certain poor widow putting in there two small copper coins.\lebnote{This coin was the \textit{lepton}, worth 1/128 of a denarius}}%
\verse{And he said, \JesusWords{“Truly I say to you that this poor widow put in more than all of them.}}%
\verse{\JesusWords{For these all put gifts\lnDDW{} into the offering out of their abundance, but this woman out of her poverty put in all the means of subsistence that she had.”}}%
\verseWithHeading{The Destruction of the Temple Predicted}{And while\lebnote{“\textit{while}” is supplied as a component of the temporal genitive absolute participle (“were speaking”)} some were speaking about the temple, that it was adorned with precious stones and votive offerings, he said,}%
\verse{\JesusWords{“As for these things that you see — days will come in which not one stone will be left on another stone that will not be thrown down!”}}%
\verseWithHeading{Signs of the End of the Age}{And they asked him, saying, “Teacher, when therefore will these things happen, and what will be the sign when these things are about to take place?”}%
\verse{And he said, Watch out that you are not deceived! For many will come in my name, saying, ‘I am he,’ and ‘The time is near!’ Do not go after them!}%
\verse{\JesusWords{And when you hear about wars and insurrections, do not be terrified, for these things must happen first, but the end will not be at once.”}}%
\verse{Then he said to them, \JesusWords{“nation will rise up against nation and kingdom against kingdom.}}%
\verse{\JesusWords{There will be great earthquakes and famines and plagues in various places. There will be terrible sights and great signs from heaven.}}%
\verseWithHeading{Persecution of Disciples Predicted}{\JesusWords{“But before all these things, they will lay their hands on you and will persecute you,\lnDDX{} handing you\lnDDW{} over to the synagogues and prisons. You will be brought before\lebnote{the participle (“be brought before”) is translated as a finite verb because of English style} kings and governors because of my name.}}%
\verse{\JesusWords{This will turn out to you for a time of witness.}}%
\verse{\JesusWords{Therefore make up your minds\lebnote{“therefore place in your hearts”} not to prepare in advance to speak in your own defense,}}%
\verse{\JesusWords{for I will give you a mouth and wisdom that all your opponents will not be able to resist or contradict.}}%
\verse{\JesusWords{And you will be handed over even by parents and brothers and relatives and friends, and they will put to death some of you.}}%
\verse{\JesusWords{And you will be hated by all because of my name.}}%
\verse{\JesusWords{Even a hair of your head will never perish!}}%
\verse{\JesusWords{By your patient endurance you will gain your lives.}}%
\verseWithHeading{The Desolation of Jerusalem}{\JesusWords{“But when you see Jerusalem surrounded by armies, then know that its desolation has come near.}}%
\verse{\JesusWords{Then those in Judea must flee to the mountains, and those inside it\lebnote{“in the midst of it”} must depart, and those in the fields must not enter into it,}}%
\verse{\JesusWords{because these are days of vengeance, so that all the things that are written can be fulfilled.}}%
\verse{\JesusWords{Woe to those who are pregnant\lebnote{“who have in the womb”} and to those who are nursing their babies\lebnote{The words “\textit{their babies}” are not in the Greek text but are supplied as a necessary clarification} in those days! For there will be great distress on the earth and wrath against this people,}}%
\verse{\JesusWords{and they will fall by the edge of the sword, and will be led captive into all the nations,\lebnote{The same Greek word, occurring three times in this verse, can be translated “nations” or “Gentiles” depending on the context} and Jerusalem will be trampled down by the Gentiles until the times of the Gentiles are fulfilled.}}%
\verseWithHeading{The Arrival of the Son of Man}{\JesusWords{“And there will be signs in the sun and moon and stars, and on the earth distress of nations in perplexity from the noise of the sea and its surging,}}%
\verse{\JesusWords{people fainting from fear and expectation of what is coming on the inhabited earth, for the powers of the heavens will be shaken.}\lebnote{An allusion to Isa 34:4}}%
\verse{\JesusWords{And then they will see the Son of Man arriving in a cloud\lebnote{An allusion to Dan 7:13} with power and great glory.}}%
\verse{\JesusWords{But when\lebnote{“\textit{when}” is supplied as a component of the temporal genitive absolute participle (“begin”)} these things begin to happen, stand up straight and raise your heads, because your redemption is drawing near!”}}%
\verseWithHeading{The Parable of the Fig Tree}{And he told them a parable: \JesusWords{“Look at the fig tree and all the trees.}}%
\verse{\JesusWords{When they put out foliage,\lnDDX{} now you see for yourselves and\lebnote{“\textit{and}” supplied because previous participle “see” translated as a finite verb} know that by this time the summer is near.}}%
\verse{\JesusWords{So also you, when you see these things happening, know\lebnote{Or “you know”} that the kingdom of God is near.}}%
\verse{\JesusWords{Truly I say to you that this generation will never pass away until all things take place!}}%
\verse{\JesusWords{Heaven and earth will pass away, but my words will never pass away.}}%
\verseWithHeading{Be Alert}{\JesusWords{“But take care for yourselves, lest your hearts are weighed down with dissipation and drunkenness and the worries of daily life, and that day come upon you suddenly}}%
\verse{\JesusWords{like a trap. For it will come upon all who reside on the face of the whole earth.}}%
\verse{\JesusWords{But be alert at all times, praying that you may have strength to escape all these things that are going to happen, and to stand before the Son of Man.”}}%
\verse{So throughout the days he was teaching in the temple courts,\lebnote{*Here “\textit{courts}” is supplied to distinguish this area from the interior of the temple building itself} and throughout the nights he was going out and\lebnote{“\textit{and}” supplied because previous participle “was going out” translated as a finite verb} spending the night on the hill that is called the Mount of Olives.}%
\verse{And all the people were getting up very early in the morning to come\lebnote{The words “\textit{to come}” are not in the Greek text but are implied} to him in the temple courts\lebnote{“\textit{courts}” is supplied to distinguish this area from the interior of the temple building itself} to listen to him.}%
\end{biblechapter}%
\begin{biblechapter}% Luke 22
\verseWithHeading{The Chief Priests and Scribes Plot to Kill Jesus}{Now the feast of Unleavened Bread (which is called Passover) was drawing near.}%
\verse{And the chief priests and the scribes were seeking how they could destroy him, because they were afraid of the people.}%
\verseWithHeading{Judas Arranges to Betray Jesus}{And Satan entered into Judas, the one called Iscariot, who was of the number of the twelve.}%
\verse{And he went away and\lnDDY{} discussed with the chief priests and officers of the temple guard how he could betray him to them.}%
\verse{And they were delighted, and came to an agreement with him to give him\lnDDZ{} money.}%
\verse{And he agreed, and began looking\lebnote{Imperfect tense as ingressive (“began looking”)} for a favorable opportunity to betray him to them apart from the crowd.}%
\verseWithHeading{Jesus’ Final Passover with the Disciples}{And the day of the feast of Unleavened Bread came, on which it was necessary for the Passover lamb to be sacrificed.}%
\verse{And he sent Peter and John, saying, \JesusWords{“Go and\lebnote{“\textit{and}” supplied because previous participle “go” translated as a finite verb} prepare the Passover for us, so that we may eat it.}\lnDEA{}}%
\verse{So they said to him, “Where do you want us to prepare it?”\lnDEA{}}%
\verse{And he said to them, \JesusWords{“Behold, when\lebnote{“\textit{when}” is supplied as a component of the temporal genitive absolute participle (“have entered”)} you have entered into the city, a man carrying a jar of water will meet you. Follow him into the house which he enters.}}%
\verse{\JesusWords{And you will say to the master of the house, ‘The Teacher says to you, “Where is the guest room where I may eat the Passover with my disciples?”’}}%
\verse{\JesusWords{And he will show you a large furnished\lebnote{Or perhaps “paved” or “panelled”} upstairs room. Make preparations there.”}}%
\verse{So they went and\lnDEB{} found everything\lnDDZ{} just as he had told them, and they prepared the Passover.}%
\verseWithHeading{The Lord’s Supper}{And when the hour came, he reclined at the table, and the apostles with him.}%
\verse{And he said to them, \JesusWords{“I have earnestly desired\lebnote{“I have desired with desire”} to eat this Passover with you before I suffer.}}%
\verse{\JesusWords{For I tell you that I will not eat it until it is fulfilled in the kingdom of God.”}}%
\verse{And he took in hand a cup, and\lebnote{“\textit{and}” supplied because previous participle “took in hand” translated as a finite verb} after\lnDEC{} giving thanks he said, \JesusWords{“Take this and share it\lnDDZ{} among yourselves.}}%
\verse{\JesusWords{For I tell you,\lebnote{Some manuscripts have “I tell you that”} from now on I will not drink of the product of the vine until the kingdom of God comes.”}}%
\verse{And he took bread, and\lebnote{“\textit{and}” supplied because previous participle “took” translated as a finite verb} after\lnDEC{} giving thanks, he broke it\lnDDZ{} and gave it\lnDDZ{} to them, saying, \JesusWords{“This is my body which is given for you. Do this in remembrance of me.”}}%
\verse{And in the same way the cup after they had eaten, saying, \JesusWords{“This cup is the new covenant in my blood which is poured out for you.}}%
\verse{\JesusWords{“But behold, the hand of the one who is betraying me is with me on the table!}}%
\verse{\JesusWords{For the Son of Man is going according to what has been determined, but woe to that man by whom he is betrayed!”}}%
\verse{And they began to debate with one another who then of them it could be who was going to do this.}%
\verseWithHeading{A Dispute About Who Is Greatest}{And a dispute also occurred among them as to which of them was recognized as being greatest.}%
\verse{So he said to them, \JesusWords{“The kings of the Gentiles\lebnote{The same Greek word can be translated “nations” or “Gentiles” depending on the context} lord it over them, and those who have authority over them are called benefactors.}}%
\verse{\JesusWords{But you are not to be like this! But the one who is greatest among you must become like the youngest, and the one who leads like the one who serves.}}%
\verse{\JesusWords{For who is greater, the one who reclines at the table or the one who serves? Is it not the one who reclines at the table? But I am in your midst as the one who serves.}}%
\verse{\JesusWords{“And you are the ones who have remained\lebnote{Or “ones who have continued”} with me in my trials,}}%
\verse{\JesusWords{and I confer on you a kingdom, just as my Father conferred on me,}}%
\verse{\JesusWords{that you may eat and drink at my table in my kingdom, and you will sit on thrones judging the twelve tribes of Israel.}}%
\verseWithHeading{Jesus Predicts Peter’s Denial}{\JesusWords{“Simon, Simon, behold, Satan has demanded to sift you like wheat,}}%
\verse{\JesusWords{but I have prayed for you, that your faith may not fail. And you, when\lebnote{“\textit{when}” is supplied as a component of the participle (“have turned back”) which is understood as temporal} once you have turned back,\lebnote{Or “have turned around”} strengthen your brothers.”}}%
\verse{But he said to him, “Lord, I am ready to go with you both to prison and to death!”}%
\verse{And he said, \JesusWords{“I tell you, Peter, the rooster will not crow today until you have denied three times that you know me!”}}%
\verseWithHeading{The Two Swords}{And he said to them, \JesusWords{“When I sent you out without a money bag and a traveler’s bag and sandals, you did not lack anything, did you?”}\lebnote{*The negative construction in Greek anticipates a negative answer here, indicated in the translation by “\textit{did you}”} And they said, “Nothing.”}%
\verse{And he said to them, \JesusWords{“But now the one who has a money bag must take it,\lnDEA{} and likewise a traveler’s bag. And the one who does not have a sword must sell his cloak and buy one.}}%
\verse{\JesusWords{For I tell you that this that is written must be fulfilled in me: ‘And he was counted with the criminals.’\lebnote{from Isa 53:12} For indeed, what is written\lebnote{The phrase “\textit{what is written}” is not in the Greek text but is an understood repetition of the similar phrase at the beginning of the verse} about me is being fulfilled.”}\lebnote{“is having an end”}}%
\verse{So they said, “Lord, behold, here are two swords!” And he said to them, \JesusWords{“It is adequate.”}}%
\verseWithHeading{The Prayer in Gethsemane}{And he went away and\lnDDY{} proceeded, according to his\lnDED{} custom, to the Mount of Olives, and the disciples also followed him.}%
\verse{And when\lebnote{“\textit{when}” is supplied as a component of the participle (“came”) which is understood as temporal} he came to the place, he said to them, \JesusWords{“Pray that you will not enter into temptation.”}}%
\verse{And he withdrew from them about a stone’s throw and knelt down\lebnote{“bent his knees”} and\lebnote{“\textit{and}” supplied because previous participle “knelt down”; literally “bent his knees” translated as a finite verb} began to pray,\lebnote{Imperfect tense (“began to pray”)}}%
\verse{saying, \JesusWords{“Father, if you are willing, take away this cup from me. Nevertheless, not my will but yours be done.”}}%
\verse{And an angel from heaven appeared to him, strengthening him.}%
\verse{And being in anguish, he began praying\lebnote{Imperfect tense as ingressive (“began praying”)} more fervently and his sweat became like drops of blood falling down to the ground.\lebnote{A number of early and important Greek manuscripts lack verses 43 and 44}}%
\verse{And when he\lebnote{“\textit{when}” is supplied as a component of the participle (“got up”) which is understood as temporal} got up from the prayer and\lebnote{“\textit{and}” supplied because participle “came” translated as a finite verb in keeping with English style} came to the disciples, he found them sleeping from sorrow,}%
\verse{and he said to them, \JesusWords{“Why are you sleeping? Get up and\lebnote{“\textit{and}” supplied because previous participle “get up” translated as a finite verb} pray that you will not enter into temptation!”}}%
\verseWithHeading{The Betrayal and Arrest of Jesus}{While\lebnote{“\textit{while}” is supplied as a component of the temporal genitive absolute participle (“speaking”)} he was still speaking, behold, there came a crowd, and the one named Judas, one of the twelve, leading them. And he approached Jesus to kiss him.}%
\verse{But Jesus said to him, \JesusWords{“Judas, are you betraying the Son of Man with a kiss?”}}%
\verse{And when\lebnote{“\textit{when}” is supplied as a component of the participle (“saw”) which is understood as temporal} those around him saw what was about to happen, they said, “Lord, should we strike with the sword?”}%
\verse{And a certain one of them struck the slave of the high priest and cut off his right ear.}%
\verse{But Jesus answered and\lebnote{“\textit{and}” supplied because previous participle “answered” translated as a finite verb} said, \JesusWords{“Stop! No more of this!”}\lebnote{“leave off to this”} And he touched his\lnDED{} ear and\lebnote{“\textit{and}” supplied because previous participle “touched” translated as a finite verb} healed him.}%
\verse{And Jesus said to the chief priests and officers of the temple and elders who had come out against him, \JesusWords{“Have you come out with swords and clubs, as against a robber?}}%
\verse{\JesusWords{Every day when\lebnote{“\textit{when}” is supplied as a component of the temporal genitive absolute participle (“was”)} I was with you in the temple courts,\lebnote{*Here “\textit{courts}” is supplied to distinguish this area from the interior of the temple building itself} you did not stretch out your\lnDED{} hands against me! But this is your hour and the domain of darkness!”}}%
\verseWithHeading{Jesus Before the Sanhedrin}{And they arrested him and\lebnote{“\textit{and}” supplied because previous participle “arrested” translated as a finite verb} led him\lnDDZ{} away and brought him\lnDDZ{} into the house of the high priest. But Peter was following at a distance.}%
\verse{And when they\lebnote{“\textit{when}” is supplied as a component of the participle (“had kindled”) which is understood as temporal} had kindled a fire in the middle of the courtyard and sat down together, Peter sat down among them.}%
\verse{And a certain female slave, seeing him sitting in the light and looking intently at him, said, “This man also was with him!”}%
\verse{But he denied it,\lnDEA{} saying, “Woman, I do not know him!”}%
\verse{And after a short time another person saw him and\lebnote{“\textit{and}” supplied because previous participle “saw” translated as a finite verb} said, “You also are one of them!” But Peter said, “Man, I am not!”}%
\verse{And after\lebnote{“\textit{after}” is supplied as a component of the temporal genitive absolute participle (“had passed”)} about one hour had passed, someone else was insisting, saying, “In truth this man also was with him, because he is also a Galilean!”}%
\verse{But Peter said, “Man, I do not know what you are talking about!” And immediately, while\lebnote{“\textit{while}” is supplied as a component of the temporal genitive absolute participle (“was speaking”)} he was still speaking, a rooster crowed.}%
\verse{And the Lord turned around and\lebnote{“\textit{and}” supplied because previous participle “turned around” translated as a finite verb} looked intently at Peter. And Peter remembered the word of the Lord,\lebnote{Some manuscripts have “the Lord’s statement”} how he said to him, \JesusWords{“Before the rooster crows today, you will deny me three times.”}}%
\verse{And he went outside and\lnDEB{} wept bitterly.}%
\verse{And the men who were guarding him began to mock\lebnote{Imperfect tense as ingressive (“began to mock”)} him while\lebnote{“\textit{when}” is supplied as a component of the participle (“beat”) which is understood as temporal} they beat him,\lnDEA{}}%
\verse{and after\lebnote{“\textit{after}” is supplied as a component of the participle (“blindfolding”) which is understood as temporal} blindfolding him they repeatedly asked\lebnote{This imperfect verb is translated as iterative (“repeatedly asked”)} him,\lnDEA{} saying, “Prophesy! Who is the one who struck you?”}%
\verse{And they were saying many other things against him, reviling him.\lnDEA{}}%
\verseWithHeading{Jesus Before the Sanhedrin}{And when day came, the council of elders of the people gathered, both chief priests and scribes, and they led him away to their Sanhedrin,\lebnote{Or “council”}}%
\verse{saying, “If you are the Christ, tell us!” But he said to them, \JesusWords{“If I tell you, you will never believe,}}%
\verse{\JesusWords{and if I ask you,\lnDEA{} you will never answer!}}%
\verse{\JesusWords{But from now on the Son of Man will be seated at the right hand of the power of God.”}}%
\verse{So they all said, “Are you then the Son of God?” And he said to them, \JesusWords{“You say that I am.”}}%
\verse{And they said, “Why do we have need of further testimony? For we ourselves have heard it\lnDDZ{} from his mouth!”}%
\end{biblechapter}%
\begin{biblechapter}% Luke 23
\verseWithHeading{Jesus Brought Before Pilate}{And the whole assembly of them rose up and\lebnote{“\textit{and}” supplied because previous participle “rose up” translated as a finite verb} brought him before Pilate.}%
\verse{And they began to accuse him, saying, “We have found this man misleading our nation and forbidding us\lnDEE{} to pay taxes to Caesar, and saying he himself is Christ, a king!”}%
\verse{And Pilate asked him, saying, “Are you the king of the Jews?” And he answered him and\lebnote{“\textit{and}” supplied because previous participle “answered” translated as a finite verb} said, \JesusWords{“You say so.”}}%
\verse{So Pilate said to the chief priests and the crowds, “I find no basis for an accusation against this man.”}%
\verse{But they insisted, saying, “He incites the people, teaching throughout the whole of Judea and beginning from Galilee as far as here.”}%
\verseWithHeading{Jesus Brought Before Herod}{Now when\lebnote{“\textit{when}” is supplied as a component of the participle (“heard”) which is understood as temporal} Pilate heard this,\lnDEF{} he asked if the man was a Galilean.}%
\verse{And when he\lebnote{“\textit{when}” is supplied as a component of the participle (“found out”) which is understood as temporal} found out that he was from the jurisdiction of Herod, he sent him over to Herod, who was also in Jerusalem in those days.}%
\verse{And when\lnDEG{} Herod saw Jesus, he was very glad, for he had been wanting to see him for a long time, because he had heard about him and was hoping to see some miracle performed by him.}%
\verse{So he questioned him at considerable length,\lebnote{“with many words”} but he answered nothing to him.}%
\verse{And the chief priests and the scribes were standing there vehemently accusing him.}%
\verse{And Herod with his soldiers also treated him with contempt, and after\lebnote{“\textit{after}” is supplied as a component of the participle (“mocking”) which is understood as temporal} mocking him\lnDEE{} and\lebnote{“\textit{and}” is supplied to connect the two participles (“mocking” and “dressing”) in keeping with English style} dressing him\lnDEE{} in glistening clothing, he sent him back to Pilate.}%
\verse{And both Herod and Pilate became friends with one another on that same day, for they had previously been enemies of one another.\lebnote{“for they had previously existed being at enmity with each other”}}%
\verseWithHeading{Pilate Releases Barabbas}{So Pilate called together the chief priests and the rulers and the people}%
\verse{and\lebnote{“\textit{and}” supplied because participle in the previous verse “called together” translated as a finite verb} said to them, “You brought me this man as one who was misleading the people, and behold, when I\lebnote{“\textit{when}” is supplied as a component of the participle (“examined”) which is understood as temporal} examined him\lnDEE{} before you, I found nothing in this man as basis for the accusation which you are making\lebnote{“you are accusing”} against him.}%
\verse{But neither did Herod, because he sent him back to us. And behold, nothing deserving death has been done\lebnote{“is having been done”} by him.}%
\verse{Therefore I will punish him and\lnDEH{} release him.”\lnDEF{}\lebnote{Many of the most important Greek manuscripts lack v. 17, “Now he was obligated to release for them at the feast one person.”}}%
\verse{}%
\verse{But they all cried out in unison, saying, “Take this man away, and release for us Barabbas!”}%
\verse{(who had been thrown in prison because of a certain insurrection that had taken place in the city, and for murder).}%
\verse{And Pilate, wanting to release Jesus, addressed them\lnDEE{}\lebnote{Some manuscripts explicitly state “them”} again,}%
\verse{but they kept crying out, saying, “Crucify! Crucify him!”}%
\verse{So he said to them a third time, “Why? What wrong has this man done? I found no basis for an accusation deserving death\lebnote{“of death”} in him. Therefore I will punish him and\lnDEH{} release him.”\lnDEF{}}%
\verse{But they were urgent, demanding with loud cries that he be crucified. And their cries prevailed.}%
\verse{And Pilate decided that their demand should be granted.}%
\verse{And he released the one who had been thrown into prison because of insurrection and murder, whom they were asking for, but Jesus he handed over to their will.}%
\verseWithHeading{Jesus Is Crucified}{And as they led him away, they seized Simon, a certain man of Cyrene, who was coming from the country, and\lebnote{“\textit{and}” supplied because previous participle “seized” translated as a finite verb} placed the cross on him, to carry it\lnDEE{} behind Jesus.}%
\verse{And a great crowd of the people were following him, and of women who were mourning and lamenting him.}%
\verse{But turning to them, Jesus said, \JesusWords{“Daughters of Jerusalem, do not weep for me, but weep for yourselves and for your children!}}%
\verse{\JesusWords{For behold, days are coming in which they will say, ‘Blessed are the barren, and the wombs that did not give birth, and the breasts that did not nurse!’}}%
\verse{\JesusWords{Then they will begin to say to the mountains, ‘Fall on us!’ and to the hills, ‘Cover us!’}}%
\verse{\JesusWords{For if they do these things when the wood is green,\lebnote{“in the green wood”} what will happen when it is dry?”}\lebnote{“in the dry”}}%
\verse{And two other criminals were also led away to be executed with him.}%
\verse{And when they came to the place that is called “The Skull,” there they crucified him, and the criminals, the one on his right and the other on his left.}%
\verse{But Jesus said, \JesusWords{“Father, forgive them, for they do not know what they are doing.”}\lebnote{Many important manuscripts lack v. 34a, “But Jesus said, ‘Father, forgive them, for they do not know what they are doing.’”} And they cast lots to divide his clothes.}%
\verse{And the people stood there watching, but the rulers also ridiculed him,\lnDEF{} saying, “He saved others; let him save himself, if this man is the Christ\lebnote{Or “Messiah”} of God, the Chosen One!”}%
\verse{And the soldiers also mocked him, coming up and\lebnote{“\textit{and}” is supplied to connect the two participles (“coming up” and “offering”) in keeping with English style} offering him sour wine}%
\verse{and saying, “If you are the king of the Jews, save yourself!”}%
\verse{And there was also an inscription over him, “This is the king of the Jews.”}%
\verse{And one of the criminals who were hanged there reviled him, saying, “Are you not the Christ? Save yourself — and us!”}%
\verse{But the other answered and\lebnote{“\textit{and}” is supplied to connect the two participles (“answered” and “rebuked”) in keeping with English style} rebuked him, saying, “Do you not even fear God, because you are undergoing the same condemnation?}%
\verse{And we indeed justly, for we are receiving what we deserve\lebnote{“\textit{things} worthy”} for what we have done. But this man has done nothing wrong!”}%
\verse{And he said, “Jesus, remember me when you come into your kingdom!”}%
\verse{And he said to him, \JesusWords{“Truly I say to you, today you will be with me in paradise.”}}%
\verseWithHeading{Jesus Dies on the Cross}{And by this time it was about the sixth hour, and darkness came over the whole land until the ninth hour}%
\verse{because\lebnote{“\textit{because}” is supplied as a component of the participle (“failed”) which is understood as causal} the light of the sun failed. And the curtain of the temple was torn apart down the middle.}%
\verse{And Jesus, calling out with a loud voice, said, \JesusWords{“Father, into your hands I entrust my spirit!”}\lebnote{from Ps 31:5} And after he\lebnote{“\textit{after}” is supplied as a component of the participle (“said”) which is understood as temporal} said this, he expired.}%
\verse{Now when\lnDEG{} the centurion saw what had happened, he began to praise\lebnote{Imperfect tense as ingressive (“began to praise”)} God, saying, “Certainly this man was righteous!”}%
\verse{And all the crowds that had come together for this spectacle, when they\lnDEG{} saw the things that had happened, returned home\lebnote{The word “\textit{home}” is not in the Greek text, but is implied} beating their\lebnote{“the”: the Greek article is used here as a possessive pronoun} breasts.}%
\verse{And all his acquaintances, and the women who had followed him from Galilee who saw these things, stood at a distance.}%
\verseWithHeading{Jesus Is Buried}{And behold, a man named\lebnote{“by name”} Joseph, who was a member of the council,\lebnote{Or “a member of the Sanhedrin”} a good\lebnote{Some manuscripts have “and a good”} and righteous man}%
\verse{(this man was not consenting to their plan and deed), from Arimathea, a Judean town,\lebnote{“a town of the Jews”} who was looking forward to the kingdom of God.}%
\verse{This man approached Pilate and\lebnote{“\textit{and}” supplied because previous participle “approached” translated as a finite verb} asked for the body of Jesus.}%
\verse{And he took it\lnDEE{} down and\lebnote{“\textit{and}” supplied because previous participle “wrapped” translated as a finite verb} wrapped it in a linen cloth and placed him in a tomb cut into the rock where no one had ever been placed.}%
\verse{And it was the day of preparation, and the Sabbath was drawing near.}%
\verse{And the women who had been accompanying him from Galilee followed and\lebnote{“\textit{and}” supplied because previous participle “followed” translated as a finite verb} saw the tomb and how his body was placed.}%
\verse{And they returned and\lebnote{“\textit{and}” supplied because previous participle “returned” translated as a finite verb} prepared fragrant spices and perfumes, and on the Sabbath they rested according to the commandment.}%
\end{biblechapter}%
\begin{biblechapter}% Luke 24
\verseWithHeading{Jesus Is Raised}{Now on the first day of the week, at very early dawn, they came back to the tomb bringing the fragrant spices which they had prepared.}%
\verse{And they found the stone had been rolled away from the tomb,}%
\verse{but when they\lebnote{“\textit{when}” is supplied as a component of the participle (“went in”) which is understood as temporal} went in, they did not find the body.\lebnote{Some manuscripts add “of the Lord Jesus”}}%
\verse{And it happened that while they were perplexed about this, behold, two men in gleaming clothing stood near them.}%
\verse{And as\lebnote{“\textit{as}” is supplied as a component of the temporal genitive absolute participles (“were” and “bowed”)} they were terrified and bowed their faces to the ground, they said to them, “Why are you looking for the living among the dead?}%
\verse{He is not here, but has been raised! Remember how he spoke to you while he\lnDEI{} was still in Galilee,}%
\verse{saying that \JesusWords{the Son of Man must be delivered into the hands of men who are sinners, and be crucified, and on the third day rise?}”}%
\verse{And they remembered his words,}%
\verse{and when they\lebnote{“\textit{when}” is supplied as a component of the participle (“returned”) which is understood as temporal} returned from the tomb, they reported all these things to the eleven and to all the rest.}%
\verse{Now Mary Magdalene and Joanna and Mary the mother of James and the others with them were telling these things to the apostles.}%
\verse{And these words appeared to them as nonsense, and they refused to believe them.}%
\verse{But Peter got up and\lnDEJ{} ran to the tomb, and bending over to look, he saw only the strips of linen cloth, and he went away to his home\lebnote{Or “wondering to himself,” if the prepositional phrase modifies the following participle} wondering what had happened.}%
\verseWithHeading{Jesus Encountered on the Road to Emmaus}{And behold, on that same day, two of them were traveling to a village named\lebnote{“to which the name”} Emmaus that was sixty stadia\lebnote{A “stade” or “stadium” (plur. “stadia”) is about 607 ft (187 m)} distant from Jerusalem,}%
\verse{and they were conversing with one another about all these things that had happened.}%
\verse{And it happened that while they were conversing, and discussing, Jesus himself also approached and\lebnote{“\textit{and}” supplied because previous participle “approached” translated as a finite verb} began to go along with\lebnote{Imperfect tense as ingressive (“began to go along with”)} them,}%
\verse{but their eyes were prevented from recognizing him.}%
\verse{And he said to them, \JesusWords{“What are these matters that you are discussing with one another as you\lebnote{“\textit{as}” is supplied as a component of the participle (“walking along”) which is understood as temporal} are walking along?”} And they stood still, looking sad.}%
\verse{And one of them, named\lebnote{“by name”} Cleopas, answered and\lebnote{“\textit{and}” supplied because previous participle “answered” translated as a finite verb} said to him, “Are you the only one living near Jerusalem and not knowing the things that have happened in it in these days?”}%
\verse{And he said to them, \JesusWords{“What things?”} So they said to him, “The things concerning Jesus the Nazarene, a man who was a prophet, powerful in deed and word before God and all the people,}%
\verse{and how our chief priests and rulers handed him over to a sentence of death, and crucified him.}%
\verse{But we were hoping that he was the one who was going to redeem Israel. But in addition to all these things, this is the third day since\lebnote{“he spends from which”} these things took place.}%
\verse{But also some women from among us astonished us, who were at the tomb early in the morning,}%
\verse{and when they\lebnote{“\textit{when}” is supplied as a component of the participle (“find”) which is understood as temporal} did not find his body, they came back saying they had seen even a vision of angels, who said that he was alive!}%
\verse{And some of those with us went out to the tomb and found it\lnDEK{} like this, just as the women had also said, but him they did not see.”}%
\verse{And he said to them, \JesusWords{“O foolish and slow in heart to believe in all that the prophets have spoken!}}%
\verse{\JesusWords{Was it not necessary that the Christ suffer these things and enter into his glory?”}}%
\verse{And beginning from Moses and from all the prophets, he interpreted to them the things concerning himself in all the scriptures.}%
\verse{And they drew near to the village where they were going, and he acted as though he was going farther.}%
\verse{And they urged him strongly, saying, “Stay with us, because it is getting toward evening, and by this time the day is far spent.” And he went in to stay with them.}%
\verse{And it happened that when he reclined at the table with them, he took the bread and\lnDEL{} gave thanks, and after\lebnote{“\textit{after}” is supplied as a component of the participle (“breaking”) which is understood as temporal} breaking it,\lebnote{*supplied from English context} he gave it\lnDEK{} to them.}%
\verse{And their eyes were opened, and they recognized him, and he became invisible to them.}%
\verse{And they said to one another, “Were not our hearts burning within us while he was speaking with us on the road, while he was explaining the scriptures to us?”}%
\verse{And they got up that same hour and\lnDEJ{} returned to Jerusalem and found the eleven and those with them assembled,}%
\verse{saying, “The Lord has really been raised, and has appeared to Simon!”}%
\verse{And they began describing\lebnote{Imperfect tense as ingressive (“began describing”)} what happened\lebnote{“the things”} on the road, and how he was recognized by them in the breaking of the bread.}%
\verseWithHeading{Jesus Appears to His Disciples}{And while\lebnote{“\textit{while}” is supplied as a component of the temporal genitive absolute participle (“were saying”)} they were saying these things, he himself stood there among them.\lebnote{Some manuscripts add “and said to them, ‘Peace to you!’”}}%
\verse{But they were startled and became terrified, and\lebnote{“\textit{and}” supplied because two previous participles “startled” and “terrified” translated as finite verbs} thought they had seen a ghost.}%
\verse{And he said to them, \JesusWords{“Why are you frightened? And for what reason do doubts arise in your hearts?}}%
\verse{\JesusWords{Look at my hands and my feet, that I am I myself! Touch me and see, because a ghost does not have flesh and bones, as you see that I have.”}\lebnote{“you see me having”}}%
\verse{And when he\lebnote{“\textit{when}” is supplied as a component of the participle (“had said”) which is understood as temporal} had said this, he showed them his\lnDEM{} hands and his\lnDEM{} feet.}%
\verse{And while\lebnote{“\textit{while}” is supplied as a component of the temporal genitive absolute participle (“disbelieving”)} they were still disbelieving because of joy and were marveling, he said to them, \JesusWords{“Do you have anything to eat\lebnote{“eatable”} here?”}}%
\verse{So they gave him a piece of broiled fish,}%
\verse{and he took it\lnDEK{} and\lnDEL{} ate it\lnDEK{} in front of them.}%
\verseWithHeading{Jesus Commissions His Disciples}{And he said to them, \JesusWords{“These are my words that I spoke to you while I\lnDEI{} was still with you, that everything that is written about me in the law of Moses and the prophets and psalms must be fulfilled.”}}%
\verse{Then he opened their minds to understand the scriptures,}%
\verse{and said to them, \JesusWords{“Thus it is written that the Christ would suffer and would rise from the dead on the third day,}}%
\verse{\JesusWords{and repentance and the forgiveness\lebnote{Some manuscripts have “repentance for the forgiveness”} of sins would be proclaimed in his name to all the nations,\lebnote{The same Greek word can be translated “nations” or “Gentiles” depending on the context} beginning from Jerusalem.}}%
\verse{\JesusWords{You are witnesses of these things.}}%
\verse{\JesusWords{And behold, I am sending out\lebnote{Some manuscripts have “am sending”} what was promised by my Father upon you, but you stay in the city until you are clothed with power from on high.”}}%
\verseWithHeading{The Ascension}{And he led them out as far as Bethany, and lifting up his hands, he blessed them.}%
\verse{And it happened that while he was blessing them, he parted from them and was taken up into heaven.}%
\verse{And they worshiped him and\lebnote{“\textit{and}” supplied because previous participle “worshiped” translated as a finite verb} returned to Jerusalem with great joy.}%
\verse{And they were continually\lebnote{“through everything”} in the temple courts\lebnote{“\textit{courts}” is supplied to distinguish this area from the interior of the temple building itself} praising God.}%
\end{biblechapter}%
\flushcolsend
\biblebook{John}
\begin{biblechapter}% John 1
\verseWithHeading{The Prologue to John’s Gospel}{In the beginning was the Word, and the Word was with God, and the Word was God.}%
\verse{This one was in the beginning with God.}%
\verse{All things came into being through him, and apart from him not one thing came into being that\lebnote{Or “came into being. What …,” beginning a new sentence connected with the following verse. A major punctuation problem is involved, since the earliest manuscripts have no punctuation, but some important later ones place the punctuation before this phrase, effectively connecting it to v. 4: “What has come into being was life in him”} has come into being.}%
\verse{In him was life, and the life was the light of humanity.\lebnote{Or “humankind”}}%
\verse{And the light shines in the darkness, and the darkness did not overcome\lebnote{Or “comprehend” (if primarily referring to people in the world)} it.}%
\verse{A man came, sent from God, whose name was\lebnote{“the name to him”} John.}%
\verse{This one came for a witness, in order that he could testify about the light, so that all would believe through him.}%
\verse{That one was not the light, but came\lebnote{The verb is implied from the previous verse, and must be supplied in the English translation} in order that he could testify about the light.}%
\verse{The true light, who gives light to every person, was coming into the world.}%
\verse{He was in the world, and the world came into being through him, and the world did not recognize\lebnote{Or “acknowledge”} him.}%
\verse{He came to his own things, and his own people did not receive him.}%
\verse{But as many as received him — to those who believe in his name — he gave to them authority to become children of God,}%
\verse{who were born not of blood, nor of the will of the flesh, nor of the will of a man, but of God.}%
\verse{And the Word became flesh and took up residence among us, and we saw his glory, glory as of the one and only from the Father, full of grace and truth.}%
\verse{John testified about him and cried out, saying, “This one was he about whom I said, ‘The one who comes after me is ahead of me, because he existed before me.’”}%
\verse{For from his fullness we have all received therefore grace because of grace.}%
\verse{For the law was given through Moses; grace and truth came about through Jesus Christ.}%
\verse{No one has seen God at any time; the one and only, God, the one who is in the bosom of the Father — that one has made him\lebnote{supplied from English context} known.}%
\verseWithHeading{John the Baptist Testifies to Jesus}{And this is the testimony of John, when the Jews sent\lebnote{Some manuscripts have “sent to him”} priests and Levites from Jerusalem so that they could ask him, “Who are you?”}%
\verse{And he confessed — and he did not deny, and confessed — “I am not the Christ!”}%
\verse{And they asked him, “Then who are you? Are you Elijah?” And he said, “I am not!” “Are you the Prophet?”\lebnote{A reference to the “Prophet like Moses” of Deut 18:15 (see Acts 3:22)} And he answered, “No!”}%
\verse{Then they said to him, “Who are you, so that we can give an answer to those who sent us? What do you say about yourself?”}%
\verse{He said, “I am ‘the voice of one crying out in the wilderness, “Make straight the way of the Lord,”’\lebnote{from Isa 40:3} just as Isaiah the prophet said.”}%
\verse{(And they had been sent from the Pharisees.)}%
\verse{And they asked him and said to him, “Why then are you baptizing, if you are not the Christ, nor Elijah, nor the Prophet?”}%
\verse{John answered them, saying, “I baptize with water. In your midst stands one whom you do not know —}%
\verse{the one who comes after me, of whom I am not worthy to untie\lebnote{“that I might untie”} the strap of his sandal!”}%
\verse{These things took place in Bethany on the other side of the Jordan, where John was baptizing.}%
\verse{On the next day he saw Jesus coming to him and said, “Look! The Lamb of God who takes away the sin of the world!}%
\verse{This one is the one about whom I said, ‘After me is coming a man who is ahead of me, because he existed before me.’}%
\verse{And I did not know him, but in order that he could be revealed to Israel, because of this I came baptizing with water.”}%
\verse{And John testified, saying, “I have seen the Spirit descending like a dove from heaven and remaining upon him.}%
\verse{And I did not know him, but the one who sent me to baptize with water, that one said to me, ‘The one upon whom you see the Spirit descending and remaining upon him — this one is the one who baptizes with the Holy Spirit.’}%
\verse{And I have seen and testify that this one is the Chosen One\lebnote{Some manuscripts have “the Son of God”} of God.}%
\verseWithHeading{Two of John’s Disciples Follow Jesus}{On the next day again John was standing there,\lebnote{*The word “\textit{there}” is not in the Greek text but is implied} and two of his disciples,}%
\verse{and looking at Jesus as he\lebnote{“\textit{as}” is supplied as a component of the participle (“walking by”) which is understood as temporal} was walking by, he said, “Look! The Lamb of God!”}%
\verse{And the two disciples heard him speaking, and they followed Jesus.}%
\verse{And Jesus, turning around and seeing them following him,\lebnote{*supplied from English context} said to them, \JesusWords{“What do you seek?”} And they said to him, “Rabbi” (which means when\lebnote{“\textit{when}” is supplied as a component of the participle (“translated”) which is understood as temporal} translated “Teacher”), “where are you staying?”}%
\verse{He said to them, \JesusWords{“Come and you will see!”} So they came and saw where he was staying, and they stayed with him that day (it was about the tenth hour).}%
\verseWithHeading{Andrew Declares Jesus to be the Messiah}{Andrew, the brother of Simon Peter, was one of the two who heard John and followed him.}%
\verse{This one first found his own brother Simon and said to him, “We have found the Messiah!” (which is translated “Christ”).}%
\verse{He brought him to Jesus. Looking at him, Jesus said, \JesusWords{“You are Simon the son of John. You will be called Cephas”} (which is interpreted “Peter”).}%
\verseWithHeading{Jesus Calls Philip and Nathanael}{On the next day he wanted to depart for Galilee, and he found Philip. And Jesus said to him, \JesusWords{“Follow me!”}}%
\verse{(Now Philip was from Bethsaida, the town of Andrew and Peter.)}%
\verse{Philip found Nathanael and said to him, “We have found the one whom Moses wrote about in the law, and the prophets wrote about — Jesus son of Joseph from Nazareth!”}%
\verse{And Nathanael said to him, “Can anything good come out of Nazareth?” Philip said to him, “Come and see!”}%
\verse{Jesus saw Nathanael coming toward him and said about him, \JesusWords{“Look! A true Israelite\lebnote{“truly an Israelite”} in whom is no deceit!”}}%
\verse{Nathanael said to him, “From where do you know me?” Jesus answered and said to him, \JesusWords{“Before Philip called you, when you\lebnote{“\textit{when}” is supplied as a component of the participle (“were”) which is understood as temporal} were under the fig tree, I saw you.”}}%
\verse{Nathanael answered him, “Rabbi, you are the Son of God! You are the king of Israel!”}%
\verse{Jesus answered and said to him, \JesusWords{“Because I said to you that I saw you under the fig tree, do you believe? You will see greater things than these!”}}%
\verse{And he said to him, \JesusWords{“Truly, truly I say to all of you, you will see heaven opened and the angels of God ascending and descending upon the Son of Man.”}}%
\end{biblechapter}%
\begin{biblechapter}% John 2
\verseWithHeading{The Wedding at Cana: Water Turned into Wine}{And on the third day, there was a wedding at Cana in Galilee, and the mother of Jesus was there.}%
\verse{And both Jesus and his disciples were invited to the wedding.}%
\verse{And when the\lebnote{“\textit{when}” is supplied as a component of the temporal genitive absolute participle (“ran out”)} wine ran out, the mother of Jesus said to him, “They have no wine!”}%
\verse{And Jesus said to her, \JesusWords{“What does your concern have to do with me,\lebnote{“to me and to you”} woman? My hour has not yet come.”}}%
\verse{His mother said to the servants, “Whatever he says to you, do it!”\lnDEN{}}%
\verse{Now six stone water jars were set there, in accordance with the ceremonial cleansing of the Jews, each holding two or three measures.\lebnote{A “measure” was about 9 gallons (40 liters)}}%
\verse{Jesus said to them, \JesusWords{“Fill the water jars with water.”} And they filled them to the brim.}%
\verse{And he said to them, \JesusWords{“Now draw some\lnDEO{} out and take it\lnDEO{} to the head steward. So they took it.}\lnDEN{}}%
\verse{Now when the head steward tasted the water which had become wine and did not know where it was from — but the servants who had drawn the water knew — the head steward summoned the bridegroom}%
\verse{and said to him, “Everyone\lebnote{“every man”} serves the good wine first, and whenever they are drunk, the inferior. You have kept the good wine until now!”}%
\verse{This beginning of signs Jesus performed at Cana in Galilee, and revealed his glory, and his disciples believed in him.}%
\verseWithHeading{Jesus’ First Journey to Jerusalem}{After this he went down to Capernaum, and his mother and brothers\lebnote{Some manuscripts have “his brothers”} and his disciples, and they stayed there a few\lebnote{“not many”} days.}%
\verse{And the Passover of the Jews was near, and Jesus went up to Jerusalem.}%
\verseWithHeading{The Cleansing of the Temple}{And he found in the temple courts\lebnote{“\textit{courts}” is supplied to distinguish this area from the interior of the temple building itself} those who were selling oxen and sheep and doves, and the money changers seated.}%
\verse{And he made a whip of cords and\lebnote{“\textit{and}” supplied because previous participle “made” translated as a finite verb} drove them\lnDEO{} all out of the temple courts,\lebnote{*Here “\textit{courts}” is supplied to distinguish this area from the interior of the temple building itself} both the sheep and the oxen, and he poured out the coins of the money changers and overturned their\lebnote{“the”: the Greek article is used here as a possessive pronoun} tables.}%
\verse{And to the ones selling the doves he said, \JesusWords{“Take these things away from here! Do not make my Father’s house a marketplace!”\lebnote{“a market house”; or “a house of merchants” (an allusion to Zech 14:21)}}}%
\verse{His disciples remembered that it is written, “Zeal for your house will consume me.”\lebnote{from Ps 69:9}}%
\verse{So the Jews answered and said to him, “What sign do you show to us, because you are doing these things?”}%
\verse{Jesus answered and said to them, \JesusWords{“Destroy this temple, and in three days I will raise it up!”}}%
\verse{Then the Jews said, “This temple has been under construction\lebnote{This translation of the aorist verb is based on a very close parallel in Ezra 5:16 (LXX), where it is clear from the following verb that the construction had not yet been completed} forty-six years, and will you raise it up in three days?”}%
\verse{But he was speaking about the temple of his body.}%
\verse{So when he was raised from the dead, his disciples remembered that he had said this, and they believed the scripture and the saying that Jesus had spoken.}%
\verseWithHeading{Jesus at the Passover}{Now while he was in Jerusalem at the Passover, during the feast, many believed in his name because they\lebnote{“\textit{because}” is supplied as a component of the participle (“saw”) which is understood as causal} saw his signs which he was doing.}%
\verse{But Jesus himself did not entrust himself to them, because he knew all people,\lebnote{The Greek term is masculine and thus refers to “all \textit{people}” rather than “all \textit{things}” (which would be neuter)}}%
\verse{and because he did not need\lebnote{“have need that”} anyone to testify\lebnote{“should testify”} about man, for he himself knew what was in man.\lebnote{*Here “man” has been retained rather than the generic “people” to maintain the connection with the following verse}}%
\end{biblechapter}%
\begin{biblechapter}% John 3
\verseWithHeading{A Meeting with Nicodemus}{Now there was a man of the Pharisees whose name was\lebnote{“the name to him”} Nicodemus, a ruler of the Jews.}%
\verse{This man came to him at night and said to him, “Rabbi, we know that you are\lebnote{both the pronoun and verb are understood in Greek and are supplied in the translation} a teacher who has come from God, for no one is able to perform these signs that you are performing unless God were with him.”}%
\verse{Jesus answered and said to him, \JesusWords{“Truly, truly I say to you, unless someone is born from above,\lebnote{The same Greek word can mean either “from above” or “again,” which allows for the misunderstanding by Nicodemus here; Jesus was speaking of new birth “from above,” while Nicodemus misunderstood him to mean a second physical birth} he is not able to see the kingdom of God.”}}%
\verse{Nicodemus said to him, “How can a man be born when he is an old man? He is not able to enter into his mother’s womb for the second time and be born, can he?”\lebnote{*The negative construction in Greek anticipates a negative answer here, indicated in the translation by the phrase “\textit{can he}”}}%
\verse{Jesus answered, \JesusWords{“Truly, truly I say to you, unless someone is born of water and spirit, he is not able to enter into the kingdom of God.}}%
\verse{\JesusWords{What is born of the flesh is flesh, and what is born of the Spirit is spirit.}}%
\verse{\JesusWords{Do not be astonished that I said to you, ‘It is necessary for you to be born from above.’}\lebnote{The same Greek word can mean either “from above” or “again” (see also v. 3)}}%
\verse{\JesusWords{The wind blows wherever it wishes, and you hear the sound of it, but you do not know where it comes from and where it is going. So is everyone who is born of the Spirit.”}}%
\verse{Nicodemus answered and said to him, “How can these things be?”}%
\verse{Jesus answered and said to him, \JesusWords{“Are you the teacher of Israel, and you do not understand these things?}}%
\verse{\JesusWords{Truly, truly I say to you, we speak what we know, and we testify about what we have seen, and you do not accept our testimony!}}%
\verse{\JesusWords{If I tell you earthly things and you do not believe, how will you believe if I tell you heavenly things?}}%
\verse{\JesusWords{And no one has ascended into heaven except the one who descended from heaven — the Son of Man.}}%
\verse{\JesusWords{And just as Moses lifted up the snake in the wilderness,\lebnote{An allusion to Num 21:5–9} thus it is necessary that the Son of Man be lifted up,}}%
\verse{\JesusWords{so that everyone who believes in him will have eternal life.”}\lebnote{Some interpreters and Bible translations extend the quotation of Jesus’ words through v. 21}}%
\verseWithHeading{God’s Love for the World}{\JesusWords{For in this way God loved the world, that he gave his one and only Son, in order that all who are believing in him should not perish, but should have life without end.}}%
\verse{\JesusWords{For God did not send his Son into the world in order that he should judge\lebnote{Or “he should condemn”} the world, but in order that the world should be saved through him.}}%
\verse{\JesusWords{The one who believes in him is not judged,\lebnote{Or “condemned”} but the one who does not believe has already been judged,\lebnote{Or “been condemned”} because he has not believed in the name of the one and only Son of God.}}%
\verse{\JesusWords{And this is the judgment: that the light has come into the world, and people loved the darkness rather than the light, because their deeds were evil.}}%
\verse{\JesusWords{For everyone who practices evil hates the light and does not come to the light, lest his deeds be exposed.}}%
\verse{\JesusWords{But the one who practices the truth comes to the light, in order that his deeds may be revealed, that they are done in God.}}%
\verseWithHeading{Additional Testimony by John the Baptist About Jesus}{After these things Jesus and his disciples came into Judean territory, and there he spent time with them and was baptizing.}%
\verse{Now John was also baptizing at Aenon near Salim, because water was plentiful there, and they were coming and were being baptized.}%
\verse{(For John had not yet been thrown into prison.)}%
\verse{So a dispute occurred on the part of John’s disciples with a Jew\lebnote{Some significant early manuscripts read “the Jews”} concerning purification.}%
\verse{And they came to John and said to him, “Rabbi, he who was with you on the other side of the Jordan, about whom you testified — look, this one is baptizing, and all are coming to him!”}%
\verse{John answered and said, “A man can receive not one thing unless it is granted to him from heaven!}%
\verse{You yourselves testify about me that I said, ‘I am not the Christ, but I am sent before that one.’}%
\verse{The one who has the bride is the bridegroom. But the friend of the bridegroom, who stands and hears him, rejoices greatly\lebnote{“with joy”} because of the bridegroom’s voice. So this joy of mine is complete.}%
\verse{It is necessary for that one to increase, but for me to decrease.}%
\verse{The one who comes from above is over all. The one who is from the earth is from the earth and speaks from the earth; the one who comes from heaven is over all.}%
\verse{What he has seen and heard, this he testifies, and no one accepts his testimony.}%
\verse{The one who accepts his testimony has attested that God is true.}%
\verse{For the one whom God sent speaks the words of God, for he does not give the Spirit by measure.}%
\verse{The Father loves the Son and has given all things into his hand.}%
\verse{The one who believes in the Son has eternal life, but the one who disobeys the Son will not see life — but the wrath of God remains on him.”\lebnote{Some interpreters and Bible translations extend the quotation of John the Baptist’s words through v. 36}}%
\end{biblechapter}%
\begin{biblechapter}% John 4
\verseWithHeading{The Samaritan Woman at Jacob’s Well}{Now when Jesus knew that the Pharisees had heard that Jesus was making and baptizing more disciples than John}%
\verse{(although Jesus himself was not baptizing, but his disciples),}%
\verse{he left Judea and departed again for Galilee.}%
\verse{And it was necessary for him to go through Samaria.}%
\verse{Now he came to a town of Samaria called Sychar, near the piece of land that Jacob had given to his son Joseph.}%
\verse{And Jacob’s well was there, so Jesus, because he had become tired from the journey, simply sat down at the well. It was about the sixth hour.}%
\verse{A woman of Samaria came to draw water. Jesus said to her, \JesusWords{“Give me water\lnDEP{} to drink.”}}%
\verse{(For his disciples had gone away into the town so that they could buy food.)}%
\verse{So the Samaritan woman said to him, “How do you, being a Jew, ask from me water\lebnote{“water” is supplied in the translation as the understood direct object of the verb “ask”} to drink, since I\lebnote{“\textit{since}” is supplied as a component of the participle (“am”) which is understood as causal} am a Samaritan woman?” (For Jews have no dealings with Samaritans.)}%
\verse{Jesus answered and said to her, \JesusWords{“If you had known the gift of God and who it is who says to you, ‘Give me water\lnDEP{} to drink,’ you would have asked him, and he would have given you living water.”}}%
\verse{The woman said to him, “Sir, you have no bucket and the well is deep! From where then do you get this living water?}%
\verse{You are not greater than our father Jacob, are you,\lebnote{*The negative construction in Greek anticipates a negative answer here, indicated by the supplied phrase “\textit{are you}” in the translation} who gave us the well and drank from it himself, and his sons and his livestock?”}%
\verse{Jesus answered and said to her, \JesusWords{“Everyone who drinks of this water will be thirsty again.}}%
\verse{\JesusWords{But whoever drinks of this water which I will give to him will never be thirsty for eternity, but the water which I will give to him will become in him a well of water springing up to eternal life.”}}%
\verse{The woman said to him, “Sir, give me this water, so that I will not be thirsty or come here to draw water!”\lnDEQ{}}%
\verse{He said to her, \JesusWords{“Go, call your husband and come here.”}}%
\verse{The woman answered and said to him, “I do not have a husband.” Jesus said to her, \JesusWords{“You have said rightly, ‘I do not have a husband,’}}%
\verse{\JesusWords{for you have had five husbands, and the one whom you have now is not your husband; this you have said truthfully!”}}%
\verse{The woman said to him, “Sir, I see that you are a prophet.}%
\verse{Our fathers worshiped on this mountain, and you people\lebnote{“\textit{people}” is supplied in the translation because the Greek pronoun is plural} say that in Jerusalem is the place where it is necessary to worship.”}%
\verse{Jesus said to her, \JesusWords{“Believe me, woman, that an hour is coming when neither on this mountain nor in Jerusalem will you worship the Father.}}%
\verse{\JesusWords{You worship what you do not know. We worship what we know, because salvation is from the Jews.}}%
\verse{\JesusWords{But an hour is coming — and now is here\lebnote{The word “\textit{here}” is not in the Greek text but is implied} — when the true worshipers will worship the Father in spirit and truth, for indeed the Father seeks such people to be his worshipers.}}%
\verse{\JesusWords{God is spirit, and the ones who worship him must worship in spirit and truth.”}}%
\verse{The woman said to him, “I know that Messiah is coming” (the one called Christ); “whenever that one comes, he will proclaim all things to us.”}%
\verse{Jesus said to her, \JesusWords{“I, the one speaking to you, am he.\lebnote{*Here the predicate nominative is supplied from context in the English translation}}}%
\verseWithHeading{The Disciples and the Harvest}{And at this point\lnDER{} his disciples came, and they were astonished that he was speaking with a woman. However, no one said, “What do you seek?” or “Why are you speaking with her?”}%
\verse{So the woman left her water jar and went away into the town and said to the people,\lebnote{Assuming the term is used here in a generic sense to refer to persons of either gender, it should be translated “people”; if instead the term here refers only to the town leaders or elders who met at the town gate, then “men” would be appropriate}}%
\verse{“Come, see a man who told me everything I have ever done! Perhaps this one is the Christ?”}%
\verse{They went out from the town and were coming to him.}%
\verse{In the meanwhile the disciples were asking him, saying, “Rabbi, eat something!”\lnDEQ{}}%
\verse{But he said to them, \JesusWords{“I have food to eat that you do not know about.”}}%
\verse{So the disciples began to say\lebnote{Imperfect tense as ingressive (“began to say”)} to one another, “No one brought him anything\lebnote{supplied from English context} to eat, did they?”\lebnote{*The negative construction in Greek anticipates a negative answer here, indicated by the supplied phrase “\textit{did they}” in the translation}}%
\verse{Jesus said to them, \JesusWords{“My food is that I do the will of the one who sent me and complete his work.}}%
\verse{\JesusWords{Do you not say, ‘There are yet four months and the harvest comes’? Behold, I say to you, lift up your eyes and look at the fields, that they are white for harvest already.}\lebnote{Some interpreters and Bible translations place the word “already” at the beginning of the next verse: “Already the one who reaps receives wages …”}}%
\verse{\JesusWords{The one who reaps receives wages and gathers fruit for eternal life, in order that the one who sows and the one who reaps can rejoice together.}}%
\verse{\JesusWords{For in this instance\lnDER{} the saying is true, ‘It is one who sows and another who reaps.’}}%
\verse{\JesusWords{I sent you to reap what you did not work for; others have worked, and you have entered into their work.”}}%
\verseWithHeading{The Samaritans and the Savior of the World}{Now from that town many of the Samaritans believed in him because of the word of the woman who testified, “He told me everything that I have done.”}%
\verse{So when the Samaritans came to him, they began asking\lebnote{Imperfect tense as ingressive (“began asking”)} him to stay with them. And he stayed there two days.}%
\verse{And many more believed because of his word,}%
\verse{And they were saying to the woman, “No longer because of what you said\lebnote{“your speaking”} do we believe, for we ourselves have heard, and we know that this one is truly the Savior of the world!”}%
\verseWithHeading{Return to Galilee}{And after the two days he departed from there into Galilee.}%
\verse{For Jesus himself testified that a prophet has no honor in his own homeland.}%
\verse{So when he came to Galilee, the Galileans welcomed him, because they\lebnote{“\textit{because}” is supplied as a component of the participle (“had seen”) which is understood as causal} had seen all the things he had done in Jerusalem at the feast (for they themselves had also come to the feast).}%
\verseWithHeading{A Royal Official’s Son Is Healed}{Now he came again to Cana in Galilee, where he had made the water wine. And there was at Capernaum a certain royal official whose son was sick.}%
\verse{This man, when he\lebnote{“\textit{when}” is supplied as a component of the participle (“heard”) which is understood as temporal} heard that Jesus had come from Judea into Galilee, went to him and asked that he come down and heal his son, for he was about to die.}%
\verse{So Jesus said to him, \JesusWords{“Unless you people\lebnote{“\textit{people}” is supplied in the translation because the Greek verb (“see”) is plural} see signs and wonders, you will never believe!”}}%
\verse{The royal official said to him, “Sir, come down before my child dies!”}%
\verse{Jesus said to him, \JesusWords{“Go, your son will live.”} The man believed the word that Jesus spoke to him, and he departed.}%
\verse{Now as\lebnote{“\textit{as}” is supplied as a component of the temporal genitive absolute participle (“was going down”)} he was going down, his slaves met him, saying that his child was alive.}%
\verse{So he inquired from them the hour at which he had gotten better. Then they said to him, “Yesterday at the seventh hour the fever left him.”}%
\verse{So the father knew that it was that\lebnote{Some manuscripts have “that \textit{it was} at that same hour”} same hour at which Jesus said to him, \JesusWords{“Your son will live,”} and he himself believed, and his whole household.}%
\verse{Now this is again a second sign Jesus performed when he\lebnote{“\textit{when}” is supplied as a component of the participle (“came”) which is understood as temporal} came from Judea into Galilee.}%
\end{biblechapter}%
\begin{biblechapter}% John 5
\verseWithHeading{A Paralytic Is Healed}{After these things there was a feast of the Jews, and Jesus went up to Jerusalem.}%
\verse{Now there is in Jerusalem near the Sheep Gate a pool called in Aramaic Bethzatha,\lebnote{The majority of later manuscripts read “Bethesda,” while other early manuscripts read “Bethsaida”} which has five porticoes.}%
\verse{In these were lying a large number of those who were sick, blind, lame, paralyzed.\lebnote{The majority of later manuscripts add the following words: “waiting for the moving of the water. (4 ) For an angel of the Lord from time to time went down in the pool and stirred up the water. So the one who went in first after the stirring of the water was healed of whatever disease he suffered.”}}%
\verse{}%
\verse{And a certain man was there who had been thirty-eight years in his sickness.}%
\verse{Jesus, when he\lebnote{“\textit{when}” is supplied as a component of the participle (“saw”) which is understood as temporal} saw this one lying there and knew that he had been sick\lebnote{The phrase “\textit{been sick}” is not in the Greek text, but is supplied from the context} a long time already, said to him, \JesusWords{“Do you want to become well?”}}%
\verse{The one who was sick answered him, “Sir, I do not have anyone that, whenever the water is stirred up, could put me into the pool. But while\lebnote{“during which \textit{time}”} I am coming, another goes down before me.”}%
\verse{Jesus said to him, \JesusWords{“Get up! Pick up your mat and walk!”}}%
\verse{And immediately the man became well and picked up his mat and began to walk.\lebnote{Imperfect tense (“began to walk”)} (Now it was the Sabbath on that day.)}%
\verse{So the Jews were saying to the one who had been healed, “It is the Sabbath, and it is not permitted for you to pick up the mat!”\lebnote{Some manuscripts have “your mat”}}%
\verse{But he answered them, “The one who made me well — that one said to me, \JesusWords{‘Pick up your mat and walk!’}”}%
\verse{So they asked him,\lebnote{Some manuscripts have “They asked him”} “Who is the man who said to you, ‘\JesusWords{Pick up your mat\lebnote{In Greek the direct object (“\textit{your mat}”) is not in the Greek text but the repetition is implied from the previous verse} and walk?’}”}%
\verse{But the one who was healed did not know who it was, for Jesus had withdrawn while\lebnote{“\textit{while}” is supplied as a component of the temporal genitive absolute participle (“was”)} a crowd was in the place.}%
\verseWithHeading{Equal with God}{After these things Jesus found him at the temple and said to him, \JesusWords{“Look, you have become well! Sin no longer, lest something worse happen to you.”}}%
\verse{The man went and reported to the Jews that Jesus was the one who made him well.}%
\verse{And on account of this the Jews began to persecute\lebnote{Imperfect tense as ingressive (“began to persecute”)} Jesus, because he was doing these things on the Sabbath.}%
\verse{But he answered\lebnote{Some manuscripts have “Jesus answered”} them, \JesusWords{“My Father is working until now, and I am working.”}}%
\verse{So on account of this the Jews were seeking even more to kill him, because he not only was breaking the Sabbath, but also was calling God his own Father, thus\lebnote{“\textit{thus}” is supplied as a component of the participle (“making”) which is understood as result} making himself equal with God.\lebnote{Consider John 8:41 They said to him, “We were not born from sexual immorality! We have one father, God!”}}%
\verseWithHeading{The Authority of the Son}{So Jesus answered and said to them, \JesusWords{“Truly, truly I say to you, the Son can do nothing from himself except what he sees the Father doing. For whatever that one does, these things also the Son does likewise.}}%
\verse{\JesusWords{For the Father loves the Son and shows him everything that he himself is doing. And greater works than these he will show him, so that you will be astonished.}}%
\verse{\JesusWords{For just as the Father raises the dead and makes them\lebnote{supplied from English context} alive, thus also the Son makes alive whomever he wishes.}}%
\verse{\JesusWords{For the Father does not judge anyone, but he has given all judgment to the Son,}}%
\verse{\JesusWords{in order that all people\lebnote{The word “\textit{people}” is not in the Greek text but is implied} will honor the Son, just as they honor the Father. The one who does not honor the Son does not honor the Father who sent him.}}%
\verse{\JesusWords{Truly, truly I say to you that the one who hears my word and who believes the one who sent me has eternal life, and does not come into judgment, but has passed from death into life.}}%
\verse{\JesusWords{“Truly, truly I say to you, that an hour is coming — and now is here — when the dead will hear the voice of the Son of God, and the ones who hear will live.}}%
\verse{\JesusWords{For just as the Father has life in himself, thus also he has granted to the Son to have life in himself.}}%
\verse{\JesusWords{And he has granted him authority to carry out judgment, because he is the Son of Man.}}%
\verse{\JesusWords{“Do not be astonished at this, because an hour is coming in which all those in the tombs will hear his voice}}%
\verse{\JesusWords{and they will come out — those who have done good things to a resurrection of life, but those who have practiced evil things to a resurrection of judgment.}}%
\verse{\JesusWords{I am able to do nothing from myself. Just as I hear, I judge, and my judgment is just, because I do not seek my own will, but the will of the one who sent me.}}%
\verseWithHeading{Further Testimony About the Son}{\JesusWords{“If I testify about myself, my testimony is not true.}}%
\verse{\JesusWords{There is another who testifies about me, and I know that the testimony which he testifies about me is true.}}%
\verse{\JesusWords{You have sent to John and he has testified to the truth.}}%
\verse{\JesusWords{(And I do not receive testimony from people, but I say these things in order that you may be saved.)}}%
\verse{\JesusWords{That one was the lamp which was burning and shining, and you wanted to rejoice for an hour in his light.}}%
\verse{\JesusWords{“But I have a testimony greater than John’s, for the works which the Father has given to me that I should complete them — the very works which I am doing — these testify about me, that the Father has sent me.}}%
\verse{\JesusWords{And the Father who sent me, that one has testified about me. You have neither heard his voice at any time nor seen his form.}}%
\verse{\JesusWords{And you do not have his word residing in yourselves, because the one whom that one sent, in this one you do not believe.}}%
\verse{\JesusWords{You search\lebnote{Or “Search” (an imperative)} the scriptures because you think that you have eternal life in them, and it is these that testify about me.}}%
\verse{\JesusWords{And you are not willing to come to me so that you may have life.}}%
\verse{\JesusWords{“I do not accept glory\lebnote{Or “honor”} from people,}}%
\verse{\JesusWords{but I know you, that you do not have the love of God in yourselves.}}%
\verse{\JesusWords{I have come in my Father’s name, and you do not accept me. If another should come in his own name, you would accept that one!}}%
\verse{\JesusWords{How are you able to believe, if you\lebnote{“\textit{if}” is supplied as a component of the participle (“accept”) which is understood as conditional} accept glory from one another, and do not seek the glory which is from the only God?}}%
\verse{\JesusWords{Do not think that I will accuse you before the Father! The one who accuses you is Moses, in whom you have put your hope!}}%
\verse{\JesusWords{For if you had believed Moses, you would believe me, for that one wrote about me.}}%
\verse{\JesusWords{But if you do not believe that one’s writings, how will you believe my words?”}}%
\end{biblechapter}%
\begin{biblechapter}% John 6
\verseWithHeading{The Feeding of Five Thousand}{After these things Jesus went away to the other side of the sea of Galilee (that is, Tiberias).}%
\verse{And a large crowd was following him because they were observing the signs that he was doing on those who were sick.}%
\verse{So Jesus went up on the mountain and sat down there with his disciples.}%
\verse{(Now the Passover, the feast of the Jews, was near.)}%
\verse{Then Jesus, when he looked up\lebnote{“then Jesus lifting up the eyes”}\lebnote{*Here “\textit{when}” in the translation is supplied as a component of the participle “lifting up” which is understood as temporal} and saw that a large crowd was coming to him, said to Philip, \JesusWords{“Where can we buy bread so that these people can eat?”}}%
\verse{(Now he said this to test him, because he knew what he was going to do.)}%
\verse{Philip replied to him, “Two hundred denarii worth of bread would not be enough for them, in order that each one could receive a little.”}%
\verse{One of his disciples, Andrew the brother of Simon Peter, said to him,}%
\verse{“Here is a boy who has five barley loaves and two fish, but what are these for so many people?”}%
\verse{Jesus said, \JesusWords{“Make the people recline.”} (Now there was a lot of grass in the place.) So the men reclined, approximately five thousand in number.}%
\verse{Then Jesus took the bread, and after he\lebnote{“\textit{after}” is supplied as a component of the participle (“had given thanks”) which is understood as temporal} had given thanks, he distributed it\lnDES{} to those who were reclining — likewise also of the fish, as much as they wanted.}%
\verse{And when they were satisfied, he said to his disciples, \JesusWords{“Gather the remaining fragments so that nothing is lost.”}}%
\verse{So they gathered them,\lnDET{} and filled twelve baskets with fragments from the five barley loaves which were left over by those who had eaten.}%
\verse{Now when\lebnote{“\textit{when}” is supplied as a component of the participle (“saw”) which is understood as temporal} the people saw the sign that he performed, they began to say,\lebnote{Imperfect tense (“began to say”)} “This one is truly the Prophet who is to come into the world!”}%
\verse{Then Jesus, because he\lnDEU{} knew that they were about to come and seize him in order to make him\lnDES{} king, withdrew again up the mountain by himself alone.}%
\verseWithHeading{Jesus Walks on the Water}{Now when evening came, his disciples went down to the sea.}%
\verse{And getting into a boat, they began to go\lebnote{Imperfect tense as ingressive (“began to go”)} to the other side of the sea, to Capernaum. And it had already become dark, and Jesus had not yet come to them.}%
\verse{And the sea began to be stirred up\lebnote{Imperfect tense as ingressive (“began to be stirred up”)} because\lebnote{“\textit{because}” is supplied as a component of the participle (“was blowing”) which is understood as causal} a strong wind was blowing.}%
\verse{Then when they\lebnote{“\textit{when}” is supplied as a component of the participle (“had rowed”) which is understood as temporal} had rowed about twenty-five or thirty stadia,\lebnote{A “stade” or “stadium” (plur. “stadia”) is about 607 ft (187 m), so this was around 3 miles (5 km)} they saw Jesus walking on the sea and coming near the boat, and they were afraid.}%
\verse{But he said to them, \JesusWords{“It is I! Do not be afraid!”}}%
\verse{So they were wanting to take him into the boat, and immediately the boat came to the land to which they were going.}%
\verseWithHeading{Discourse About the Bread of Life}{On the next day, the crowd that was on the other side of the sea saw that other boats were not there (except one), and that Jesus had not entered with his disciples into the boat, but his disciples had departed alone.}%
\verse{Other boats from Tiberias came near the place where they had eaten the bread after\lebnote{“\textit{after}” is supplied as a component of the temporal genitive absolute participle (“had given thanks”)} the Lord had given thanks.}%
\verse{So when the crowd saw that Jesus was not there, nor his disciples, they themselves got into the boats and came to Capernaum seeking Jesus.}%
\verse{And when they\lebnote{“\textit{when}” is supplied as a component of the participle (“found”) which is understood as temporal} found him on the other side of the sea, they said to him, “Rabbi, when did you get here?”}%
\verse{Jesus replied to them and said, \JesusWords{“Truly, truly I say to you, you seek me not because you saw signs, but because you ate of the loaves and were satisfied!}}%
\verse{\JesusWords{Do not work for the food that perishes, but the food that remains to eternal life, which the Son of Man will give to you. For God the Father has set his seal on this one.”}}%
\verse{So they said to him, “What shall we do that we can accomplish the works of God?”}%
\verse{Jesus answered and said to them, \JesusWords{“This is the work of God: that you believe in the one whom that one sent.”}}%
\verse{So they said to him, “Then what sign will you perform, so that we can see it\lnDES{} and believe you? What will you do?}%
\verse{Our fathers ate the manna in the wilderness, just as it is written, ‘He gave them bread from heaven to eat.’\lebnote{from Ps 78:24 which refers to the events of Exod 16:4–36}}%
\verse{Then Jesus said to them, \JesusWords{“Truly, truly I say to you, Moses did not give you bread from heaven, but my Father is giving you the true bread from heaven!}}%
\verse{\JesusWords{For the bread of God is the one who comes down from heaven and gives life to the world.”}}%
\verse{So they said to him, “Sir, always give us this bread!”}%
\verse{Jesus said to them, \JesusWords{“I am the bread of life. The one who comes to me will never be hungry, and the one who believes in me will never be thirsty again.}}%
\verse{\JesusWords{But I said to you that you have seen me and do not believe.}}%
\verse{\JesusWords{Everyone whom the Father gives to me will come to me, and the one who comes to me I will never throw out,}}%
\verse{\JesusWords{because I have come down from heaven not that I should do my will, but the will of the one who sent me.}}%
\verse{\JesusWords{Now this is the will of the one who sent me: that everyone whom he has given me, I would not lose any of them,\lnDEV{} but raise them\lnDEV{} up on the last day.}}%
\verse{\JesusWords{For this is the will of my Father, that everyone who looks at the Son and believes in him would have eternal life, and I will raise him up on the last day.”}}%
\verse{Now the Jews began to grumble\lebnote{Imperfect tense as ingressive (“began to grumble”)} about him because he said, \JesusWords{“I am the bread that came down from heaven,”}}%
\verse{and they were saying, “Is this one not Jesus the son of Joseph, whose father and mother we know? How does he now say, \JesusWords{‘I have come down from heaven’}?”}%
\verse{Jesus answered and said to them, \JesusWords{“Do not grumble among yourselves!\lnDEW{}}}%
\verse{\JesusWords{No one is able to come to me unless the Father who sent me draws him, and I will raise him up on the last day.}}%
\verse{\JesusWords{It is written in the prophets, ‘And they will all be taught by God.’\lebnote{from Isa 54:13} Everyone who hears from the Father and learns comes to me.}}%
\verse{\JesusWords{(Not that anyone has seen the Father except the one who is from God — this one has seen the Father.)\lebnote{The switch from first person in vv. 44–45 to third person here and back to first person in vv. 47–51 suggests that this verse is a parenthetical comment by the author rather than the words of Jesus}}}%
\verse{\JesusWords{Truly, truly I say to you, the one who believes has eternal life.}}%
\verse{\JesusWords{I am the bread of life.}}%
\verse{\JesusWords{Your fathers ate the manna in the wilderness and they died.}}%
\verse{\JesusWords{This is the bread that comes down from heaven so that someone may eat from it and not die.}}%
\verse{\JesusWords{I am the living bread that came down from heaven. If anyone eats from this bread, he will live forever.\lnDEX{} And the bread that I will give for the life of the world is my flesh.”}}%
\verse{So the Jews began to quarrel\lebnote{Imperfect tense as ingressive (“began to quarrel”)} among themselves,\lnDEW{} saying, “How can this man give us his flesh to eat?”}%
\verse{Then Jesus said to them, \JesusWords{“Truly, truly I say to you, unless you eat the flesh of the Son of Man and drink his blood, you do not have life in yourselves!}}%
\verse{\JesusWords{The one who eats my flesh and drinks my blood has eternal life, and I will raise him up on the last day.}}%
\verse{\JesusWords{For my flesh is true food, and my blood is true drink.}}%
\verse{\JesusWords{The one who eats\lnDEY{} my flesh and drinks my blood resides in me and I in him.}}%
\verse{\JesusWords{Just as the living Father sent me, and I live because of the Father, so also the one who eats\lnDEY{} me — that one will live because of me.}}%
\verse{\JesusWords{This is the bread that came down from heaven, not as the fathers ate and died. The one who eats\lnDEY{} this bread will live forever.”}\lnDEX{}}%
\verseWithHeading{Many of Jesus’ Disciples Offended by His Teaching}{He said these things while\lebnote{“\textit{when}” is supplied as a component of the participle (“teaching”) which is understood as temporal} teaching in the synagogue in Capernaum.}%
\verse{Thus many of his disciples, when they\lebnote{“\textit{when}” is supplied as a component of the participle (“heard”) which is understood as temporal} heard it,\lnDET{} said, “This saying is hard! Who can understand it?”}%
\verse{But Jesus, because he\lnDEU{} knew within himself that his disciples were grumbling about this, said to them, \JesusWords{“Does this cause you to be offended?}}%
\verse{\JesusWords{Then what if you see the Son of Man ascending where he was before?}}%
\verse{\JesusWords{The Spirit is the one who gives life; the flesh profits nothing. The words that I have spoken to you are spirit and are life.}}%
\verse{\JesusWords{But there are some of you who do not believe.”} (For Jesus knew from the beginning who they were who did not believe, and who it was who would betray him.)}%
\verse{And he said, \JesusWords{“Because of this I said to you that no one can come to me unless it has been granted to him by the Father.”}}%
\verseWithHeading{Peter’s Confession}{For this reason many of his disciples drew back\lebnote{“went away to the things behind”} and were not walking with him any longer.}%
\verse{So Jesus said to the twelve, \JesusWords{“You do not want to go away also, do you?”}\lebnote{*The negative construction in Greek anticipates a negative answer here, indicated in the translation by the phrase “\textit{do you}”}}%
\verse{Simon Peter answered him, “Lord, to whom would we go? You have the words of eternal life.}%
\verse{And we have believed, and have come to know, that you are the Holy One of God.”}%
\verse{Jesus replied to them, \JesusWords{“Did I not choose you, the twelve, and one of you is the devil?”}}%
\verse{(Now he was speaking about Judas son of Simon Iscariot, because this one — one of the twelve — was going to betray him.)}%
\end{biblechapter}%
\begin{biblechapter}% John 7
\verseWithHeading{Jesus’ Brothers Do Not Believe in Him}{And after these things Jesus was going about in Galilee. For he did not want to go about in Judea, because the Jews were seeking to kill him.}%
\verse{Now the feast of the Jews — the feast of Tabernacles — was near.}%
\verse{So his brothers said to him, “Depart from here and go to Judea, so that your disciples also can see your works that you are doing.}%
\verse{For no one does anything in secret and yet he himself desires to be publicly recognized.\lebnote{“with openness”} If you are doing these things, reveal yourself to the world!”}%
\verse{(For not even his brothers believed in him.)}%
\verseWithHeading{Jesus at the Feast of Tabernacles}{So Jesus said to them, \JesusWords{“My time has not yet come, but your time is always ready.}}%
\verse{\JesusWords{The world cannot hate you, but it hates me, because I am testifying about it, that its deeds are evil.}}%
\verse{\JesusWords{You go up to the feast. I am not\lebnote{Most manuscripts read “not yet” here, but this is obviously an easier reading intended to reconcile the statement with Jesus’ later actions} going up to this feast, because my time is not yet completed.}\lebnote{Or “fulfilled”}}%
\verse{And when he\lebnote{“\textit{when}” is supplied as a component of the participle (“had said”) which is understood as temporal} had said these things, he remained in Galilee.}%
\verse{But when his brothers had gone up to the feast, then he also went up, not openly, but (as it were) in secret.}%
\verse{So the Jews were looking for him at the feast, and were saying, “Where is he?”}%
\verse{And there was a lot of grumbling concerning him among the crowds; some were saying, “He is a good man,” but others were saying, “No, but he deceives the crowd.”}%
\verse{However, no one was speaking openly about him for fear of the Jews.}%
\verse{Now when the feast was already half over,\lebnote{“now it being already in the middle of the feast”} Jesus went to the temple courts\lebnote{“\textit{courts}” is supplied to distinguish this area from the interior of the temple building itself} and began to teach.\lebnote{Imperfect tense (“began to teach”)}}%
\verse{Then the Jews were astonished, saying, “How does this man possess knowledge,\lebnote{“know letters”} because he\lebnote{“\textit{because}” is supplied as a component of the participle (“been taught”) which is understood as causal} has not been taught?”}%
\verse{So Jesus answered them and said, \JesusWords{“My teaching is not mine, but is from the one who sent me.}}%
\verse{\JesusWords{If anyone wants to do his will, he will know about my\lebnote{“the”: the Greek article is used here as a possessive pronoun} teaching, whether it is from God or I am speaking from myself.}}%
\verse{\JesusWords{The one who speaks from himself seeks his own glory. But the one who seeks the glory of the one who sent him — this one is true, and there is no unrighteousness in him.}}%
\verse{\JesusWords{Has not Moses given you the law, and none of you carries out the law? Why do you seek to kill me?”}}%
\verse{The crowd replied, “You have a demon! Who is seeking to kill you?”}%
\verse{Jesus answered and said to them, \JesusWords{“I performed one work, and you are all astonished.}}%
\verse{\JesusWords{Because of this Moses has given you circumcision (not that it is from Moses, but from the fathers), and you circumcise a man on the Sabbath.}}%
\verse{\JesusWords{If a man receives circumcision on the Sabbath so that the law of Moses would not be broken, are you angry with me because I made a whole man well\lebnote{Or “a man entirely well”} on the Sabbath?}}%
\verse{\JesusWords{Do not judge according to outward appearance, but judge according to righteous judgment!”}}%
\verseWithHeading{Is Jesus the Christ?}{Then some of the inhabitants of Jerusalem began to say,\lnDEZ{} “Is this not the one whom they are seeking to kill?}%
\verse{And behold, he is speaking openly and they are saying nothing to him! Can it be that the rulers truly know that this man is the Christ?}%
\verse{Yet we know where this man is from, but the Christ, whenever he comes — no one knows where he is from!”}%
\verse{Then Jesus cried out in the temple courts,\lebnote{*Here “\textit{courts}” is supplied to distinguish this area from the interior of the temple building itself} teaching and saying, \JesusWords{“You both know me and you know where I am from! And I have not come from myself, but the one who sent me is true, whom you do not know.}}%
\verse{\JesusWords{I know him, because I am from him and he sent me.”}}%
\verse{So they were seeking to seize him, and no one laid a hand on him, because his hour had not yet come.}%
\verse{But from the crowd many believed in him and were saying, “Whenever the Christ comes, he will not perform more signs than this man has done, will he?”\lebnote{*The negative construction in Greek anticipates a negative answer here, indicated in the translation by the phrase “\textit{will he}”}}%
\verse{The Pharisees heard the crowd murmuring these things about him, and the chief priests and the Pharisees sent officers in order to take him into custody .\lebnote{“that they could seize him”}}%
\verse{Then Jesus said, \JesusWords{“Yet a little time I am with you, and I am going to the one who sent me.}}%
\verse{\JesusWords{You will seek me and will not find me,\lnDFA{}\lnDFB{} and where I am, you cannot come.”}}%
\verse{So the Jews said to one another, “Where is this one going to go, that we will not find him? He is not going to go to the Dispersion among the Greeks and teach the Greeks, is he?\lebnote{*The negative construction in Greek anticipates a negative answer here, indicated in the translation by the phrase “\textit{is he}”}}%
\verse{What is this saying that he said, \JesusWords{‘You will seek me and will not find me,\lnDFA{}\lnDFB{} and where I am, you cannot come’?”}}%
\verseWithHeading{The Promise of the Spirit}{Now on the last day of the feast — the great day — Jesus stood and cried out, saying, \JesusWords{“If anyone is thirsty, let him come to me, and let him drink,}}%
\verse{\JesusWords{the one who believes in me.\lebnote{An alternative punctuation of vv. 37–38 reads: “If anyone is thirsty, let him come to me and let him drink. (38 ) The one who believes in me, just as the scripture said, ‘Out of his belly will flow rivers of living water.’”} Just as the scripture said, ‘Out of his belly will flow rivers of living water.’”\lebnote{from the Old Testament of uncertain origin; texts most often suggested are Isa 44:3; 55:1; 58:11; Zech 14:8}}}%
\verse{Now he said this concerning the Spirit, whom those who believed in him were about to receive. For the Spirit was not yet given,\lebnote{A few manuscripts supply the participle “given” here; while it is unlikely this represents the original reading, many English versions nevertheless supply “given” to avoid the impression that the Spirit did not exist prior to this point} because Jesus had not yet been glorified.)}%
\verseWithHeading{Different Opinions About Jesus}{Then, when they\lebnote{“\textit{when}” is supplied as a component of the participle (“heard”) which is understood as temporal} heard these words, some from the crowd began to say,\lnDEZ{} “This man is truly the Prophet!”}%
\verse{Others were saying, “This man is the Christ!” But others were saying, “No, for the Christ does not come from Galilee, does he?\lebnote{*The negative construction in Greek anticipates a negative answer here, indicated in the translation by the phrase “\textit{does he}”}}%
\verse{Has not the scripture said that the Christ comes from the descendants of David, and from Bethlehem, the village where David was?”}%
\verse{So there was a division in the crowd because of him.}%
\verse{And some of them were wanting to seize him, but no one laid hands on him.}%
\verse{So the officers came to the chief priests and Pharisees. And they said to them, “Why\lebnote{“because of what”} did you not bring him?”}%
\verse{The officers replied, “Never has a man spoken like this!”}%
\verse{Then the Pharisees replied to them, “You have not also been deceived, have you?\lebnote{*The negative construction in Greek anticipates a negative answer here, indicated in the translation by the phrase “\textit{have you}”}}%
\verse{None\lebnote{“not anyone”} of the rulers or of the Pharisees have believed in him, have they?\lebnote{*The negative construction in Greek anticipates a negative answer here, indicated in the translation by the phrase “\textit{have they}”}}%
\verse{But this crowd who does not know the law is accursed!”}%
\verse{Nicodemus, the one who came to him previously — who was one of them — said to them,}%
\verse{“Our law does not condemn a man unless it first hears from him and knows what he is doing, does it?”\lebnote{*The negative construction in Greek anticipates a negative answer here, indicated in the translation by the phrase “\textit{does it}”}}%
\verse{They answered and said to him, “You are not also from Galilee, are you?\lebnote{*The negative construction in Greek anticipates a negative answer here, indicated in the translation by the phrase “\textit{are you}”} Investigate and see that a prophet does not arise from Galilee!”}%
\verse{And each one went to his own house.\lebnote{John 7:53–8:11 is not found in the earliest and best manuscripts and was almost certainly not an original part of the Gospel of John; one significant group of Greek manuscripts places it after Luke 21:38}}%
\end{biblechapter}%
\begin{biblechapter}% John 8
\verseWithHeading{A Woman Caught in Adultery}{But Jesus went to the Mount of Olives.}%
\verse{Now early in the morning he came again to the temple courts.\lnDFC{} And all the people were coming,\lebnote{Some manuscripts have “were coming to him”} and he sat down and\lebnote{“\textit{and}” supplied because previous participle “sat down” translated as a finite verb} began to teach\lebnote{Imperfect tense as ingressive (“began to teach”)} them.}%
\verse{Now the scribes and the Pharisees brought to him a woman\lebnote{Some manuscripts have “brought a woman”} caught in adultery. And standing her in their midst,}%
\verse{they said to him, testing him,\lebnote{Some manuscripts omit “testing \textit{him}”} “Teacher, this woman was caught in the very act of committing adultery!}%
\verse{Now in the law, Moses commanded us to stone such women. So what do you say?”}%
\verse{(Now they were saying this to test him, so that they would have an occasion\lnDFD{} to bring charges against him.) But Jesus, bending down, began to write\lebnote{Imperfect tense as ingressive (“began to write”)} with his\lebnote{“the”: the Greek article is used here as a possessive pronoun} finger on the ground, taking no notice.\lebnote{Some manuscripts omit “taking no notice”}}%
\verse{And when they persisted in asking him, straightening up he said\lebnote{Some manuscripts have “he straightened up and said”} to them, \JesusWords{“The one of you without sin, let him throw the first stone at her!”}}%
\verse{And bending down again, he wrote on the ground.}%
\verse{Now when they\lebnote{“\textit{when}” is supplied as a component of the participle (“heard”) which is understood as temporal} heard it,\lebnote{*supplied from English context} being convicted by their conscience,\lebnote{Some manuscripts omit “being convicted by their conscience”} they began to depart,\lebnote{Imperfect tense (“began to depart”)} one by one, beginning with the older ones, and Jesus\lebnote{Some manuscripts have “he”} was left alone — and the woman who was in their midst.}%
\verse{So Jesus, straightening up and seeing no one except the woman,\lebnote{Some manuscripts omit “and seeing no one except the woman”} said to her, \JesusWords{“Where are those accusers of yours?\lebnote{Some manuscripts have “said to her, ‘Woman, where are they?”} Does no one condemn you?”}}%
\verse{And she said, “No one, Lord.” So Jesus said, \JesusWords{“Neither do I condemn you. Go, and\lebnote{Some manuscripts have “and from now on”} sin no more.”}\lebnote{John 7:53–8:11 is not found in the earliest and best manuscripts and was almost certainly not an original part of the Gospel of John; one significant group of Greek manuscripts places it after Luke 21:38}}%
\verseWithHeading{Jesus, the Light of the World}{Then Jesus spoke to them again, saying, \JesusWords{“I am the light of the world! The one who follows me will never walk in darkness, but will have the light of life.”}}%
\verse{So the Pharisees said to him, “You testify concerning yourself! Your testimony is not true.”}%
\verse{Jesus answered and said to them, \JesusWords{“Even if I testify concerning myself, my testimony is true, because I know where I have come from and where I am going. But you do not know where I have come from or where I am going.}}%
\verse{\JesusWords{You judge according to externals; I do not judge anyone.}}%
\verse{\JesusWords{But even if I judge, my judgment is true, because I am not alone, but I and the Father who sent me.}}%
\verse{\JesusWords{And even in your law it is written that the testimony of two men is true.}\lebnote{An allusion to Deut 17:6}}%
\verse{\JesusWords{I am the one who testifies concerning myself, and the Father who sent me testifies concerning me.”}}%
\verse{So they were saying to him, “Where is your father?” Jesus replied, \JesusWords{“You know neither me nor my Father! If you had known me, you would have known my Father also.”}}%
\verse{He spoke these words by the treasury while\lebnote{“\textit{while}” is supplied as a component of the participle (“teaching”) which is understood as temporal} teaching in the temple courts,\lnDFC{} and no one seized him, because his hour had not yet come.}%
\verseWithHeading{Jesus Predicts His Death}{So he said to them again, \JesusWords{“I am going away, and you will seek me and will die in your sin. Where I am going you cannot come!”}}%
\verse{Then the Jews began to say,\lnDFE{} “Perhaps he will kill himself, because he is saying, \JesusWords{‘Where I am going you cannot come.’}”}%
\verse{And he said to them, \JesusWords{“You are from below; I am from above. You are from this world; I am not from this world.}}%
\verse{\JesusWords{Thus I said to you that you will die in your sins. For if you do not believe that I am he, you will die in your sins.”}}%
\verse{So they began to say to him,\lnDFE{} “Who are you?” Jesus said to them, \JesusWords{“What\lebnote{“that which”} I have been saying to you from the beginning.}}%
\verse{\JesusWords{I have many things to say and to judge concerning you, but the one who sent me is true, and the things which I heard from him, these things I say to the world.”}}%
\verse{(They did not know that he was speaking to them about the Father.)}%
\verse{Then Jesus said,\lebnote{Some manuscripts have “said to them”} \JesusWords{“When you lift up the Son of Man, then you will recognize that I am he, and I do nothing from myself, but just as the Father taught me, I say these things.}}%
\verse{\JesusWords{And the one who sent me is with me. He has not left me alone, because I always do the things that are pleasing to him.”}}%
\verse{While\lebnote{“\textit{while}” is supplied as a component of the temporal genitive absolute participle (“was saying”)} he was saying these things, many believed in him.}%
\verseWithHeading{The Truth Will Set You Free}{Then Jesus said to those Jews who had believed him, \JesusWords{“If you continue in my word you are truly my disciples,}}%
\verse{\JesusWords{and you will know the truth, and the truth will set you free.”}}%
\verse{They replied to him, “We are descendants of Abraham and have not been enslaved to anyone at any time. How do you say, \JesusWords{‘You will become free’}?”}%
\verse{Jesus replied to them, \JesusWords{“Truly, truly I say to you, that everyone who commits sin is a slave of sin.}}%
\verse{\JesusWords{And the slave does not remain in the household forever;\lnDFF{} the son remains forever.}\lnDFF{}}%
\verse{\JesusWords{So if the son sets you free, you will be truly free.}}%
\verse{\JesusWords{I know that you are descendants of Abraham. But you are seeking to kill me, because my word makes no progress among you.}}%
\verse{\JesusWords{I speak the things that I have seen with the Father; so also you do the things that you have heard from the Father.”}}%
\verseWithHeading{The Priority of Jesus Over Abraham}{They answered and said to him, “Abraham is our father!” Jesus said to them, \JesusWords{“If you are children of Abraham, do the deeds of Abraham!}}%
\verse{\JesusWords{But now you are seeking to kill me, a man who spoke to you the truth which I heard from God. This Abraham did not do.}}%
\verse{\JesusWords{You are doing the deeds of your father!”} They said\lebnote{Some manuscripts have “Then they said”} to him, “We were not born from sexual immorality! We have one father, God!”}%
\verse{Jesus said to them, \JesusWords{“If God were your father, you would love me, for I have come forth from God and have come. For I have not come from myself, but that one sent me.}}%
\verse{\JesusWords{Why\lnDFG{} do you not understand my way of speaking? Because you are not able to listen to my message.}}%
\verse{\JesusWords{You are of your father the devil, and you want to do the desires of your father! That one was a murderer from the beginning, and does not stand firm in the truth, because truth is not in him. Whenever he speaks the lie, he speaks from his own nature,\lebnote{*The word “\textit{nature}” is not in the Greek text but is implied} because he is a liar and the father of lies.}\lebnote{“of it”}}%
\verse{\JesusWords{But because I am telling the truth, you do not believe me.}}%
\verse{\JesusWords{Who among you convicts me concerning sin? If I am telling the truth, why\lnDFG{} do you not believe me?}}%
\verse{\JesusWords{The one who is from God listens to the words of God. Because of this you do not listen — because you are not of God.”}}%
\verse{The Jews answered and said to him, “Do we not correctly say that you are a Samaritan and have a demon?”}%
\verse{Jesus replied, \JesusWords{“I do not have a demon, but I honor my Father, and you dishonor me!}}%
\verse{\JesusWords{But I do not seek my own glory. There is one who seeks and judges!}}%
\verse{\JesusWords{Truly, truly I say to you, if anyone keeps my word, he will never experience death forever.”}\lnDFF{}}%
\verse{The Jews\lebnote{Some manuscripts have “Then the Jews”} said to him, “Now we know that you have a demon! Abraham and the prophets died, and you say, \JesusWords{‘If anyone keeps my word, he will never taste death forever.’}\lnDFF{}}%
\verse{You are not greater than our father Abraham who died, are you?\lebnote{*The negative construction in Greek anticipates a negative answer here, indicated in the translation by the phrase “\textit{are you}”} And the prophets died! Who do you make yourself to be?”}%
\verse{Jesus replied, \JesusWords{“If I glorify myself, my glory is nothing. The one who glorifies me is my Father, about whom you say, ‘He is our God.’}}%
\verse{\JesusWords{And you have not known him, but I know him. And if I were to say that I do not know him, I would be a liar like you! But I know him and I keep his word.}}%
\verse{\JesusWords{Abraham your father rejoiced that he would see my day, and he saw it\lnDFD{} and was glad.”}}%
\verse{So the Jews said to him, “You are\lebnote{“you have”} not yet fifty years old, and have you seen Abraham?”}%
\verse{Jesus said to them, \JesusWords{“Truly, truly I say to you, before Abraham was, I am!”}}%
\verse{Then they picked up stones in order to throw them\lnDFD{} at him. But Jesus was hidden and went out of the temple courts.\lnDFC{}}%
\end{biblechapter}%
\begin{biblechapter}% John 9
\verseWithHeading{A Man Born Blind Is Given Sight}{And as he\lebnote{“\textit{as}” is supplied as a component of the participle (“went away”) which is understood as temporal} went away, he saw a man blind from birth.}%
\verse{And his disciples asked him, saying, “Rabbi, who sinned, this man or his parents, that he should be born blind?”}%
\verse{Jesus replied, \JesusWords{“Neither this man sinned nor his parents, but it happened\lebnote{The words “\textit{it happened}” are not in the Greek text but are implied} so that the works of God could be revealed in him.}}%
\verse{\JesusWords{It is necessary for us to do the deeds of the one who sent me while it is day; night is coming, when no one can work!}}%
\verse{\JesusWords{While I am in the world, I am the light of the world.”}}%
\verse{When he\lebnote{“\textit{when}” is supplied as a component of the participle (“had said”) which is understood as temporal} had said these things, he spat on the ground and made clay with the saliva, and smeared the clay on his eyes.}%
\verse{And he said to him, \JesusWords{“Go, wash in the pool of Siloam”} (which is translated “sent”). So he went and washed and came back seeing.}%
\verse{Then the neighbors and those who saw him previously (because he was a beggar) began to say,\lebnote{Imperfect tense (“began to say”)} “Is this man not the one who used to sit and beg?”}%
\verse{Others were saying, “It is this man”; others were saying, “No, but he is like him.” That one was saying, “I am he!”}%
\verse{So they began to say\lebnote{Imperfect tense as ingressive (“began to say”)} to him, “How\lebnote{Some manuscripts have “Then how”} were your eyes opened?”}%
\verse{He replied, “The man who is called Jesus made clay and smeared it\lnDFH{} on my eyes and said to me, \JesusWords{‘Go to Siloam and wash!’} So I went, and I washed, and\lebnote{“\textit{and}” supplied because two previous participles “went” and “washed” translated as finite verbs} I received sight.”}%
\verse{And they said to him, “Where is that man?” He said, “I do not know.”}%
\verseWithHeading{The Reaction of the Pharisees to the Healing}{They brought him — the one formerly blind — to the Pharisees.}%
\verse{(Now the day on which Jesus made the clay and opened his eyes was the Sabbath.)}%
\verse{So the Pharisees also were asking him again how he received sight. And he said to them, “He put clay on my eyes, and I washed, and I see.”}%
\verse{So some of the Pharisees were saying, “This man is not from God, because he does not observe the Sabbath!” Others\lebnote{Some manuscripts have “But others”} were saying, “How can a man who is a sinner perform such signs?” And there was a division among them.}%
\verse{So they said to the blind man again, “What do you say about him, because he opened your eyes?” And he said, “He is a prophet.”}%
\verse{So the Jews did not believe concerning him that he had been blind and received sight, until they summoned the parents of the one\lebnote{“of him”} who received sight.}%
\verse{And they asked them, saying, “Is this man your son, whom you say was born blind? Then how does he now see?”}%
\verse{So his parents answered and said, “We know that this man is our son, and that he was born blind.}%
\verse{But how he now sees we do not know, or who opened his eyes we do not know. Ask him! He is a mature adult;\lnDFI{} he will speak for himself!”}%
\verse{(His parents said these things because they were afraid of the Jews, for the Jews had already decided that if anyone should confess him to be Christ, he would be expelled from the synagogue.}%
\verse{Because of this his parents said, “He is a mature adult;\lnDFI{} ask him.”)}%
\verse{So they summoned the man who had been blind for the second time and said to him, “Give glory to God! We know that this man is a sinner!”}%
\verse{Then that man replied, “Whether he is a sinner I do not know. One thing I know — that although I\lebnote{“\textit{although}” is supplied as a component of the participle (“was”) which is understood as concessive} was blind, now I see!”}%
\verse{So they said to him, “What did he do to you? How did he open your eyes?”}%
\verse{He replied to them, “I told you already and you did not listen! Why do you want to hear it\lnDFH{} again? You do not want to become his disciples also, do you?”\lebnote{*The negative construction in Greek anticipates a negative answer here, indicated in the translation by the phrase “\textit{do you}”}}%
\verse{They reviled\lebnote{Some manuscripts have “And they reviled”} him and said, “You are his disciple! But we are disciples of Moses!}%
\verse{We know that God has spoken to Moses, but we do not know where this man is from.”}%
\verse{The man answered and said to them, “For the remarkable thing is this, that you do not know where he is from, and he opened my eyes!}%
\verse{We know that God does not listen to sinners, but if someone is devout and does his will, he listens to this one.}%
\verse{From time immemorial\lebnote{“the age”} it has not been heard that someone opened the eyes of one born blind.}%
\verse{If this man were not from God, he would not be able to do anything!”}%
\verse{They answered and said to him, “You were born completely in sin, and are you attempting to teach\lebnote{the present tense is translated as a conative present (“attempting to teach”)} us?” And they threw him out.}%
\verseWithHeading{Jesus as the Son of Man}{Jesus heard that they had thrown him out, and finding him, he said, \JesusWords{“Do you believe in the Son of Man?”}}%
\verse{He answered and said, “And who is he, sir, that I may believe in him?”}%
\verse{Jesus said to him, \JesusWords{“You have both seen him, and he is the one who is speaking with you.”}}%
\verse{And he said, “I believe, Lord!” and he worshiped him.}%
\verse{And Jesus said,\lebnote{A number of important manuscripts lack v. 38 and the first part of v. 39 (“and Jesus said”)} \JesusWords{“For judgment I have come into this world, so that those who do not see may see, and those who see may become blind!”}}%
\verse{Some of the Pharisees who were with him heard these things and said to him, “We are not also blind, are we?”\lebnote{*The negative construction in Greek anticipates a negative answer here, indicated in the translation by the phrase “\textit{are we}”}}%
\verse{Jesus said to them, \JesusWords{“If you were blind, you would not have sin. But now you say, ‘We see,’ your sin remains.}}%
\end{biblechapter}%
\begin{biblechapter}% John 10
\verseWithHeading{Jesus as the Good Shepherd}{\JesusWords{“Truly, truly I say to you, the one who does not enter through the door into the fold of the sheep, but climbs up at some other place — that one is a thief and a robber.}}%
\verse{\JesusWords{But the one who enters through the door is the shepherd of the sheep.}}%
\verse{\JesusWords{For this one the doorkeeper opens, and the sheep hear his voice, and he calls his own sheep by name and leads them out.}}%
\verse{\JesusWords{Whenever he sends out all his own, he goes before them, and the sheep follow him because they know his voice.}}%
\verse{\JesusWords{And they will never follow a stranger, but will flee from him, because they do not know the voice of strangers.”}}%
\verse{Jesus told them this parable, but they did not understand what it was that he was saying to them.}%
\verse{Then Jesus said to them\lebnote{Some manuscripts omit “to them”} again, \JesusWords{“Truly, truly, I say to you, I am the door of the sheep.}}%
\verse{\JesusWords{All those who came before me are thieves and robbers, but the sheep do not listen to them.}}%
\verse{\JesusWords{I am the door. If anyone enters through me, he will be saved, and will come in and will go out and will find pasture.}}%
\verse{\JesusWords{The thief comes only\lebnote{“does not come except”} so that he can steal and kill and destroy; I have come so that they may have life, and have it\lnDFJ{} abundantly.}}%
\verse{\JesusWords{“I am the good shepherd. The good shepherd lays down his life for the sheep.}}%
\verse{\JesusWords{The hired hand, who is not the shepherd, whose own the sheep are not, sees the wolf approaching and abandons the sheep and runs away — and the wolf seizes them and scatters them\lnDFJ{} —}}%
\verse{\JesusWords{because he is a hired hand and he is not concerned\lebnote{“it is not a concern to him”} about the sheep.}}%
\verse{\JesusWords{“I am the good shepherd, and I know my own, and my own know me,}}%
\verse{\JesusWords{just as the Father knows me and I know the Father, and I lay down my life for the sheep.}}%
\verse{\JesusWords{And I have other sheep which are not from this fold. I must bring these also, and they will hear my voice, and they will become one flock — one shepherd.}}%
\verse{\JesusWords{Because of this the Father loves me, because I lay down my life so that I may take possession of it again.}}%
\verse{\JesusWords{No one takes it from me, but I lay it down voluntarily.\lebnote{“from myself”} I have authority to lay it down, and I have authority to take possession of it again. This commandment I received from my Father.”}}%
\verse{Again there was a division among the Jews because of these words.}%
\verse{And many of them were saying, “He has a demon and is out of his mind! Why do you listen to him?”}%
\verse{Others were saying, “These are not the words of one who is possessed by a demon! A demon is not able to open the eyes of the blind, is it?”\lebnote{*The negative construction in Greek anticipates a negative answer here, indicated in the translation by the phrase “\textit{is it}”}}%
\verseWithHeading{Jesus at the Feast of the Dedication}{Then the feast of the Dedication took place in Jerusalem. It was winter,}%
\verse{and Jesus was walking in the temple in the Portico of Solomon.}%
\verse{So the Jews surrounded him and began to say\lebnote{Imperfect tense as ingressive (“began to say”)} to him, “How long will you keep us in suspense?\lebnote{“until when will you take away our life”} If you are the Christ, tell us plainly!”}%
\verse{Jesus answered them, \JesusWords{“I told you and you do not believe! The deeds that I do in the name of my Father, these testify about me.}}%
\verse{\JesusWords{But you do not believe, because you are not of my sheep!}}%
\verse{\JesusWords{My sheep listen to my voice, and I know them, and they follow me.}}%
\verse{\JesusWords{And I give them eternal life, and they will never perish forever,\lebnote{“for the age”} and no one will seize them out of my hand.}}%
\verse{\JesusWords{My Father, who has given them\lnDFJ{} to me, is greater than all, and no one can seize them\lnDFJ{} from the Father’s hand.}}%
\verse{\JesusWords{The Father and I are one.”}}%
\verse{Then\lebnote{Some manuscripts omit “Then”} the Jews picked up stones again so that they could stone him.}%
\verse{Jesus answered them, \JesusWords{“I have shown you many good deeds from the Father. For which one of them are you going to stone me?”}}%
\verse{The Jews answered him, “We are not going to stone you concerning a good deed, but concerning blasphemy, and because you, although you\lebnote{“\textit{although}” is supplied as a component of the participle (“are”) which is understood as concessive} are a man, make yourself to be God!”}%
\verse{Jesus answered them, \JesusWords{“Is it not written in your law, ‘I said, “You are gods”’?}\lebnote{from Ps 82:6 (in common usage “law” could refer to the entire Old Testament)}}%
\verse{\JesusWords{If he called them ‘gods’ to whom the word of God came — and the scripture cannot be broken —}}%
\verse{\JesusWords{do you say about he whom the Father set apart and sent into the world, ‘You are blaspheming,’ because I said, ‘I am the Son of God’?}}%
\verse{\JesusWords{If I do not do the deeds of my Father, do not believe me.}}%
\verse{\JesusWords{But if I am doing them,\lebnote{*supplied from English context} even if you do not believe me, believe the deeds, so that you may know and understand that the Father is in me and I am in the Father.”}}%
\verse{So they were seeking again to seize him, and he departed out of their hand.}%
\verse{And he went away again on the other side of the Jordan, to the place where John was baptizing at an earlier time, and he stayed there.}%
\verse{And many came to him and began to say,\lebnote{Imperfect tense (“began to say”)} “John performed no sign, but everything John said about this man was true!”}%
\verse{And many believed in him there.}%
\end{biblechapter}%
\begin{biblechapter}% John 11
\verseWithHeading{Lazarus Dies}{Now a certain man was sick, Lazarus from Bethany, the village of Mary and her sister Martha.}%
\verse{(Now it was Mary who anointed the Lord with perfumed oil and wiped his feet with her hair, whose brother Lazarus was sick.)}%
\verse{So the sisters sent word\lnDFK{} to him, saying, “Lord, behold, the one whom you love is sick.”}%
\verse{And when he\lebnote{“\textit{when}” is supplied as a component of the participle (“heard”) which is understood as temporal} heard it,\lnDFL{} Jesus said, \JesusWords{“This sickness is not to death, but for the glory of God, in order that the Son of God may be glorified through it.”}}%
\verse{(Now Jesus loved Martha and her sister and Lazarus.)}%
\verse{So when he heard that he was sick, then he remained in the place where\lebnote{“in which”} he was two days.}%
\verse{Then after this he said to the disciples, \JesusWords{“Let us go to Judea again.”}}%
\verse{The disciples said to him, “Rabbi, the Jews were seeking just now to stone you, and are you going there again?”}%
\verse{Jesus replied, Are there not twelve hours in the day? If anyone walks around in the daylight, he does not stumble, because he sees the light of this world.}%
\verse{\JesusWords{But if anyone walks around in the night, he stumbles, because the light is not in him.}}%
\verse{He said these things, and after this he said to them, \JesusWords{“Our friend Lazarus has fallen asleep, but I am going so that I can awaken him.”}}%
\verse{So the disciples said to him, “Lord, if he has fallen asleep, he will get well.”}%
\verse{(Now Jesus had been speaking about his death, but they thought that he was speaking about real sleep.\lebnote{“the sleep of slumber”})}%
\verse{So Jesus then said to them plainly, \JesusWords{“Lazarus has died,}}%
\verse{\JesusWords{and I am glad for your sake\lebnote{“for the sake of you”} that I was not there, so that you may believe. But let us go to him.”}}%
\verse{Then Thomas (the one who is called Didymus)\lebnote{“Didymus” means “the twin” in Greek} said to his fellow disciples, “Let us go also, so that we may die with him.”}%
\verseWithHeading{Jesus the Resurrection and the Life}{So when he\lebnote{“\textit{when}” is supplied as a component of the participle (“arrived”) which is understood as temporal} arrived, Jesus found he had already been four days in the tomb.}%
\verse{(Now Bethany was near Jerusalem, about fifteen stadia.\lebnote{A “stade” or “stadium” (plur. “stadia”) is about 607 ft (187 m), so this was just under two miles (3 km)}}%
\verse{So many of the Jews came to Martha and Mary in order to console them concerning their\lnDFM{} brother.)}%
\verse{Now Martha, when she heard that Jesus was coming, went to meet him, but Mary was sitting in the house.}%
\verse{So Martha said to Jesus, “Lord, if you had been here, my brother would not have died.}%
\verse{Even\lebnote{Some manuscripts have “But even”} now I know that whatever you ask God, God will grant you.”}%
\verse{Jesus said to her, \JesusWords{“Your brother will rise again.”}}%
\verse{Martha said to him, “I know that he will rise again in the resurrection at the last day.”}%
\verse{Jesus said to her, \JesusWords{“I am the resurrection and the life. The one who believes in me, even if he dies, will live,}}%
\verse{\JesusWords{and everyone who lives and believes in me will never die forever.\lebnote{“for the age”} Do you believe this?”}}%
\verse{She said to him, “Yes, Lord, I have believed that you are the Christ, the Son of God, who comes into the world.”}%
\verseWithHeading{Jesus Weeps}{And when she\lnDFN{} had said this, she went and called her sister Mary privately, saying, “The Teacher is here and is calling for you.”}%
\verse{So that one, when she heard it,\lnDFL{} got up quickly and went to him.}%
\verse{(Now Jesus has not yet come into the village, but was still in the place where Martha went to meet him.)}%
\verse{So the Jews who were with her in the house and were consoling her, when they\lebnote{“\textit{when}” is supplied as a component of the participle (“saw”) which is understood as temporal} saw Mary — that she stood up quickly and went out — followed her, because they\lebnote{“\textit{because}” is supplied as a component of the participle (“thought”) which is understood as causal} thought that she was going to the tomb in order to weep there.}%
\verse{Then Mary, when she came where Jesus was and\lebnote{“\textit{and}” supplied because participle “saw” translated as a finite verb in keeping with English style} saw him, fell at his feet, saying to him, “Lord, if you had been here, my brother would not have died.”}%
\verse{Then Jesus, when he saw her weeping and the Jews who came with her weeping, was deeply moved in spirit and was troubled within himself.}%
\verse{And he said, \JesusWords{“Where have you laid him?”} They said to him, “Lord, come and see.”}%
\verse{Jesus wept.}%
\verse{So the Jews were saying, “See how he loved him!”}%
\verse{But some of them said, “Was not this man who opened the eyes of the blind able to do something\lnDFK{} so that this man also would not have died?”}%
\verseWithHeading{Lazarus Is Raised}{Then Jesus, deeply moved within himself again, came to the tomb. Now it was a cave, and a stone was lying on it.}%
\verse{Jesus said, \JesusWords{“Take away the stone.”} Martha, the sister of the one who had died, said to him, “Lord, he is stinking already, because it has been four days.”}%
\verse{Jesus said to her, \JesusWords{“Did I not say to you that if you believed, you would see the glory of God?”}}%
\verse{So they took away the stone. And Jesus lifted up his\lnDFM{} eyes above and said, \JesusWords{“Father, I give thanks to you that you hear me.}}%
\verse{\JesusWords{And I know that you always hear me, but for the sake of the crowd standing around I said it,\lnDFL{} so that they may believe that you sent me.”}}%
\verse{And when he\lnDFN{} had said these things, he cried out with a loud voice, \JesusWords{“Lazarus, come out!”}}%
\verse{The one who had died came out, his\lnDFM{} feet and his\lnDFM{} hands bound with strips of cloth, and his face wrapped with a facecloth. Jesus said to them, \JesusWords{“Untie him and let him go.”}}%
\verseWithHeading{The Jewish Leaders Plot to Kill Jesus}{Then many of the Jews who had come with Mary and saw the things which he did believed in him.}%
\verse{But some of them went to the Pharisees and told them the things which Jesus had done.}%
\verse{So the chief priests and the Pharisees called together the Sanhedrin and said, “What are we doing? For this man is performing many signs!}%
\verse{If we allow him to go on in this way, everyone will believe in him, and the Romans will come and take away both our place\lebnote{Generally understood to be a reference to the Jerusalem temple} and our\lebnote{“both the place and the nation of us”; the possessive pronoun is repeated in the translation (rather than the article) in keeping with English style} nation.”}%
\verse{But a certain one of them, Caiaphas (who was high priest in that year), said to them, “You do not know anything at all!}%
\verse{Nor do you consider that it is profitable for you that one man should die for the people, and the whole nation not perish.”}%
\verse{(Now he did not say this from himself, but being high priest in that year, he prophesied that Jesus was going to die for the nation,}%
\verse{and not for the nation only, but also that the children of God who are scattered would be gathered into one.)}%
\verse{So from that day they resolved that they should kill him.}%
\verse{So Jesus was no longer walking openly among the Jews, but went away from there to the region near the wilderness, to a city called Ephraim, and there he stayed with the disciples.}%
\verse{Now the Passover of the Jews was near, and many went up to Jerusalem from the surrounding country before the Passover, so that they could purify themselves.}%
\verse{So they were looking for Jesus, and were speaking with one another while\lebnote{“\textit{while}” is supplied as a component of the participle (“standing”) which is understood as temporal} standing in the temple courts,\lebnote{*Here “\textit{courts}” is supplied to distinguish this area from the interior of the temple building itself} “What do you think? That he will not come to the feast?”}%
\verse{(Now the chief priests and the Pharisees had given orders that if anyone knew where he was, they should report it,\lnDFL{} in order that they could arrest him.)}%
\end{biblechapter}%
\begin{biblechapter}% John 12
\verseWithHeading{Jesus Is Anointed at Bethany}{Then, six days before the Passover, Jesus came to Bethany, where Lazarus was, whom Jesus had raised from the dead.}%
\verse{So they made him a dinner there, and Martha was serving, but Lazarus was one of the ones reclining at table with him.}%
\verse{Then Mary took a pound\lebnote{The Greek term refers to a Roman pound, 327.45 grams (approximately 12 ounces)} of ointment of very valuable genuine nard and\lebnote{“\textit{and}” supplied because previous participle “took” translated as a finite verb} anointed the feet of Jesus, and wiped his feet with her hair. And the house was filled with the fragrance of the ointment.}%
\verse{But Judas Iscariot, one of his disciples (the one who was going to betray him) said,}%
\verse{“Why\lebnote{“because of what”} was this ointment not sold for three hundred denarii and given to the poor?”}%
\verse{(Now he said this not because he was concerned\lebnote{“it was a concern to him”} about the poor, but because he was a thief, and having the money box, he used to steal what was put into it.)\lnDFO{}}%
\verse{So Jesus said, \JesusWords{“Leave her alone, so that she may keep it for the day of my preparation for burial.}}%
\verse{\JesusWords{For you have the poor with you always, but you do not always have me.”}}%
\verseWithHeading{The Decision to Kill Lazarus}{Now a large crowd\lebnote{Some manuscripts have “the large crowd”} of Jews found out that he was there, and they came, not only because of Jesus, but so that they could see Lazarus also, whom he raised from the dead.}%
\verse{So the chief priests decided that they would kill Lazarus also,}%
\verse{because on account of him many of the Jews were going and believing in Jesus.}%
\verseWithHeading{The Triumphal Entry}{On the next day the large crowd who had come to the feast, when they\lebnote{“\textit{when}” is supplied as a component of the participle (“heard”) which is understood as temporal} heard that Jesus was coming to Jerusalem,}%
\verse{took the branches of palm trees and went out to meet him, and began crying out, “Hosanna! Blessed is the one who comes in the name of the Lord,\lebnote{from Ps 118:25–26} even the king of Israel!”}%
\verse{So Jesus found a young donkey and\lebnote{“\textit{and}” supplied because previous participle “found” translated as a finite verb} sat on it, just as it is written,}%
\verse{“Do not be afraid, daughter of Zion! Behold, your king is coming, seated on the foal of a donkey!”\lebnote{from Zech 9:9}}%
\verse{(His disciples did not understand these things at first, but when Jesus was glorified, then they remembered that these things had been written about him and they did these things to him.)}%
\verse{So the crowd who was with him when he called Lazarus out of the tomb and raised him from the dead were continuing to testify.}%
\verse{Because of this also the crowd went to meet him, for they had heard that he had performed this sign.}%
\verse{So the Pharisees said to one another, “You see that you are accomplishing nothing! Behold, the world has gone after him.”}%
\verseWithHeading{Greeks Seeking Jesus}{Now some Greeks were among those who had gone up in order to worship at the feast.}%
\verse{So these approached Philip, who was from Bethsaida in Galilee, and began asking him saying, “Sir, we want to see Jesus.”}%
\verse{Philip went and told Andrew. Andrew and Philip went and told Jesus.}%
\verse{And Jesus answered them, saying, \JesusWords{“The hour has come that the Son of Man will be glorified.}}%
\verse{\JesusWords{Truly, truly I say to you, unless a grain of wheat falls into the earth and\lebnote{“\textit{and}” supplied because previous participle “falls” translated as a finite verb} dies, it remains by itself alone. But if it dies, it bears much fruit.}}%
\verse{\JesusWords{The one who loves his life loses it, and the one who hates his life in this world preserves it for eternal life.}}%
\verse{\JesusWords{If anyone serves me, he must follow me, and where I am, there my servant will be also. If anyone serves me, the Father will honor him.}}%
\verseWithHeading{Jesus Predicts His Death}{\JesusWords{“Now my soul is troubled, and what shall I say? ‘Father, deliver me from this hour’? But for this reason I have come to this hour!}}%
\verse{Father, glorify your name!” Then a voice came from heaven, “I have both glorified it,\lnDFO{} and I will glorify it\lnDFP{} again.”}%
\verse{Now the crowd that stood there and heard it\lnDFP{} said it had thundered.\lebnote{“thunder had happened”} Others were saying, “An angel has spoken to him!”}%
\verse{Jesus answered and said, \JesusWords{“This voice has not happened for my sake, but for your sake.}}%
\verse{\JesusWords{Now is the judgment of this world! Now the ruler of this world will be thrown out!}}%
\verse{\JesusWords{And I, when I am lifted up from the earth, will draw all people to myself.”}}%
\verse{(Now he said this to indicate by what sort of death he was going to die.)}%
\verse{Then the crowd replied to him, “We have heard from the law that the Christ remains forever!\lebnote{“for the age”; probably an allusion to Ps 89:35–37 (in common usage “law” could refer to the entire Old Testament)} And how do you say that the Son of Man must be lifted up? Who is this Son of Man?”}%
\verse{So Jesus said to them, \JesusWords{“Yet a little time the light is with you! Walk while you have the light, so that the darkness does not overtake you! And the one who walks in the darkness does not know where he is going.}}%
\verse{While you have the light, believe in the light, in order that you may become sons of light.” Jesus said these things, and then he went away and\lebnote{“\textit{and}” supplied because previous participle “went away” translated as a finite verb} was hidden from them.}%
\verseWithHeading{The People Still Refuse to Believe}{But as many signs as he had performed before them, they did not believe in him,}%
\verse{in order that the word of the prophet Isaiah would be fulfilled, who said, “Lord, who has believed our message? And to whom has the arm of the Lord been revealed?”\lebnote{from Isa 53:1}}%
\verse{For this reason they were not able to believe, because again Isaiah said,}%
\verse{“He has blinded their eyes and hardened their hearts, lest they see with their\lnDFQ{} eyes and understand with their\lnDFQ{} hearts and turn, and I heal them.”\lebnote{from Isa 6:10}}%
\verse{Isaiah said these things because he saw his glory, and he spoke about him.}%
\verse{Yet despite that, even many of the rulers believed in him, but because of the Pharisees they did not confess it,\lnDFO{} so that they would not be expelled from the synagogue.}%
\verse{For they loved the praise of men more than praise from God.}%
\verseWithHeading{Jesus’ Final Public Appeal}{But Jesus cried out and said, \JesusWords{“The one who believes in me does not believe in me, but in the one who sent me,}}%
\verse{\JesusWords{and the one who sees me sees the one who sent me.}}%
\verse{\JesusWords{I have come as a light into the world, in order that everyone who believes in me will not remain in the darkness.}}%
\verse{\JesusWords{And if anyone hears my words and does not observe them,\lnDFO{} I will not judge him. For I have not come to judge the world, but to save the world.}}%
\verse{\JesusWords{The one who rejects me and does not accept my words has one who judges him; the word that I have spoken will judge him on the last day.}}%
\verse{\JesusWords{For I have not spoken from myself, but the Father himself who sent me has commanded me\lebnote{“has given me commandment”} what I should say and what I should speak.}}%
\verse{\JesusWords{And I know that his commandment is eternal life. So the things that I say, just as the Father said to me, thus I say.”}}%
\end{biblechapter}%
\begin{biblechapter}% John 13
\verseWithHeading{Jesus Washes His Disciples’ Feet}{Now before the feast of Passover, Jesus, knowing that his hour had come that he would depart from this world to the Father, and\lebnote{“\textit{and}” is supplied in keeping with English style} having loved his\lnDFR{} own in the world, loved them to the end.}%
\verse{And as\lebnote{“\textit{as}” is supplied as a component of the temporal genitive absolute participle (“was taking place”)} a dinner was taking place, when\lebnote{“\textit{when}” is supplied as a component of the temporal genitive absolute participle (“put”)} the devil had already put into the heart of Judas son of Simon Iscariot that he should betray him,}%
\verse{because he\lebnote{“\textit{because}” is supplied as a component of the participle (“knew”) which is understood as causal} knew that the Father had given him all things into his\lnDFR{} hands, and that he had come forth from God and was going away to God,}%
\verse{he got up from the dinner and took off his\lnDFR{} outer clothing, and taking a towel, tied it\lnDFS{} around himself.}%
\verse{Then he poured water into the washbasin and began to wash the feet of the disciples, and to wipe them\lnDFS{} dry with the towel which he had tied around himself.\lebnote{“with which he was girded”}}%
\verse{Then he came to Simon Peter. He said to him, “Lord, are you going to wash my feet?”}%
\verse{Jesus answered and said to him, \JesusWords{“What I am doing you do not understand now, but you will understand after these things.”}}%
\verse{Peter said to him, “You will never wash my feet forever!”\lebnote{“for the age”} Jesus replied to him, \JesusWords{“Unless I wash you, you do not have a share with me.”}}%
\verse{Simon Peter said to him, “Lord, not my feet only, but also my\lnDFR{} hands and my\lnDFR{} head!”}%
\verse{Jesus said to him, \JesusWords{“The one who has bathed only needs\lebnote{“does not have need except”} to wash his\lnDFR{} feet, but is completely clean. And you are clean, but not all of you.”}}%
\verse{(For he knew the one who would betray him; because of this he said, \JesusWords{“Not all of you are clean.”})}%
\verse{So when he had washed their feet and taken his outer clothing and reclined at table again, he said to them, \JesusWords{“Do you understand what I have done for you?}}%
\verse{\JesusWords{You call me ‘Teacher’ and ‘Lord,’ and you speak correctly, for I am.}}%
\verse{\JesusWords{If then I — your\lnDFR{} Lord and Teacher — wash your feet, you also ought to wash one another’s feet.}}%
\verse{\JesusWords{For I have given you an example, that just as I have done for you, you also do.}}%
\verse{\JesusWords{Truly, truly I say to you, a slave is not greater than his master, nor a messenger greater than the one who sent him.}}%
\verse{\JesusWords{If you understand these things, you are blessed if you do them.}}%
\verse{\JesusWords{“I am not speaking about all of you. I know whom I have chosen, but in order that the scripture would be fulfilled, ‘The one who eats my bread has lifted up his heel against me.’}\lebnote{from Ps 41:9}}%
\verse{\JesusWords{From now on I am telling you before it happens, in order that when it happens you may believe that I am he.}\lebnote{*Here the predicate nominative (“\textit{he}”) is understood, but must be supplied in the translation}}%
\verse{\JesusWords{Truly, truly I say to you, the one who receives anyone I send receives me, and the one who receives me receives the one who sent me.”}}%
\verseWithHeading{Jesus Predicts Judas’ Betrayal}{When he\lebnote{“\textit{when}” is supplied as a component of the participle (“had said”) which is understood as temporal} had said these things, Jesus was troubled in spirit and testified and said, \JesusWords{“Truly, truly I say to you that one of you will betray me.”}}%
\verse{The disciples began looking\lebnote{Imperfect tense as ingressive (“began looking”)} at one another, uncertain about whom he was speaking.}%
\verse{One of his disciples — the one whom Jesus loved — was reclining close beside\lebnote{“in the bosom of” (a position dictated by ancient banqueting practice)} Jesus.}%
\verse{So Simon Peter gestured for this one to inquire who it was about whom he was speaking.}%
\verse{He leaned back\lebnote{Some manuscripts have “Then he leaned back”} accordingly against Jesus’ chest and\lebnote{“\textit{and}” supplied because previous participle “leaned back” translated as a finite verb} said to him, “Lord, who is it?”}%
\verse{Jesus replied, \JesusWords{“It is he to whom I dip the piece of bread and give it\lnDFS{} to him.”} Then after\lebnote{“\textit{after}” is supplied as a component of the participle (“dipping”) which is understood as temporal} dipping the piece of bread, he gave it\lnDFS{}\lebnote{Some manuscripts have “after dipping the piece of bread, he took \textit{it} and gave \textit{it}”} to Judas son of Simon Iscariot.}%
\verse{And after the piece of bread, then Satan entered into him. Then Jesus said to him, \JesusWords{“What you are doing, do quickly!”}}%
\verse{(Now no one of those reclining at table knew for what reason he said this to him.}%
\verse{For some were thinking because Judas had the money box, Jesus was telling him, “Purchase what we need\lebnote{“of which we have need”} for the feast,” or that he should give something to the poor.)}%
\verse{So after he\lebnote{“\textit{after}” is supplied as a component of the participle (“had taken”) which is understood as temporal} had taken the piece of bread, he went out immediately. And it was night.}%
\verseWithHeading{Jesus Predicts Peter’s Denial}{Then, when he had gone out, Jesus said, \JesusWords{“Now the Son of Man is glorified, and God is glorified in him.}}%
\verse{\JesusWords{If God is glorified in him, God will also glorify him in himself, and will glorify him immediately.}}%
\verse{\JesusWords{Children, yet a little time I am with you. You will seek me and just as I said to the Jews, “Where I am going you cannot come,” now I say also to you.}}%
\verse{\JesusWords{“A new commandment I give to you: that you love one another — just as I have loved you, that you also love one another.}}%
\verse{\JesusWords{By this everyone will know that you are my disciples — if you have love for one another.”}}%
\verse{Simon Peter said to him, “Lord, where are you going?” Jesus replied,\lebnote{Some manuscripts have “replied to him”} \JesusWords{“Where I am going you cannot follow me now, but you will follow later.”}}%
\verse{Peter said to him, “Lord, why am I not able to follow you now? I will lay down my life for you!”}%
\verse{Jesus replied, \JesusWords{“Will you lay down your life for me? Truly, truly I say to you, the rooster will not crow until you have denied me three times!}}%
\end{biblechapter}%
\begin{biblechapter}% John 14
\verseWithHeading{Jesus’ Farewell Discourse}{\JesusWords{“Do not let your hearts be troubled. You believe\lebnote{Or simply “Believe”; the verb form can be either indicative (e.g., KJV, NAB, NLT) or imperative (e.g., NIV, NRSV, ESV)} in God; believe\lebnote{Like the previous verb “believe” this form could also be either indicative or imperative, though most English versions regard it as imperative} also in me.}}%
\verse{\JesusWords{In my Father’s house there are many dwelling places; but if not, I would have told you, because\lebnote{A large number of manuscripts, many of them later, lack “because”} I am going away to prepare a place for you.}}%
\verse{\JesusWords{And if I go and prepare a place for you, I will come again and receive you to myself, so that where I am, you may be also.}}%
\verse{\JesusWords{And you know the way to the place I am going.”}}%
\verse{Thomas said to him, “Lord, we do not know where you are going. How are we able to know the way?”}%
\verse{Jesus said to him, \JesusWords{“I am the way, and the truth, and the life. No one comes to the Father except through me.}}%
\verse{\JesusWords{If you had known me, you would have known\lebnote{Some manuscripts have “If you have known me, you will know”} my Father also. From now on\lebnote{Some manuscripts have “And from now on”} you know him and have seen him.”}}%
\verse{Philip said to him, “Lord, show us the Father, and it is enough for us.”}%
\verse{Jesus said to him, \JesusWords{“Am I with you so long a time and you have not known me, Philip? The one who has seen me has seen the Father! How can you say, ‘Show us the Father?’}}%
\verse{\JesusWords{Do you not believe that I am in the Father and the Father is in me? The words that I say to you I do not speak from myself, but the Father residing in me does his works.}}%
\verse{\JesusWords{Believe me that I am in the Father and the Father is in me; but if not, believe because of the works themselves.}}%
\verse{\JesusWords{Truly, truly I say to you, the one who believes in me, the works that I am doing he will do also, and he will do greater works\lnDFT{} than these because I am going to the Father.}}%
\verse{\JesusWords{And whatever\lebnote{“anything which”} you ask in my name, I will do this, in order that the Father may be glorified in the Son.}}%
\verse{\JesusWords{If you ask me anything in my name, I will do it.}\lnDFU{}}%
\verseWithHeading{Jesus Promises the Holy Spirit}{\JesusWords{“If you love me, you will keep my commandments.}}%
\verse{\JesusWords{And I will ask the Father, and he will give you another Advocate, in order that he may be with you forever\lebnote{“for the age”} —}}%
\verse{\JesusWords{the Spirit of truth, whom the world is not able to receive, because it does not see him or know him.\lnDFU{} You know him, because he resides with you and will be in you.}}%
\verse{\JesusWords{“I will not leave you as orphans; I am coming to you.}}%
\verse{\JesusWords{Yet a little time and the world will see me no longer, but you will see me; because I live, you also will live.}}%
\verse{\JesusWords{On that day you will know that I am in my Father, and you are in me, and I am in you.}}%
\verse{\JesusWords{The one who has my commandments and keeps them — that one is the one who loves me. And the one who loves me will be loved by my Father, and I will love him and will reveal myself to him.”}}%
\verse{Judas (not Iscariot) said to him, “Lord, why\lebnote{Some manuscripts have “and why”} is it that you are going to reveal yourself to us and not to the world?”}%
\verse{Jesus answered and said to him, \JesusWords{“If anyone loves me he will keep my word, and my Father will love him, and we will come to him and will take up residence with him.}\lebnote{“will make abode with him”}}%
\verse{\JesusWords{The one who does not love me does not keep my words, and the word that you hear is not mine, but the Father’s who sent me.}}%
\verse{\JesusWords{These things I have spoken to you while\lebnote{“\textit{while}” is supplied as a component of the participle (“residing”) which is understood as temporal} residing with you.}}%
\verse{\JesusWords{But the Advocate, the Holy Spirit, whom the Father will send in my name — that one will teach you all things, and will remind you of everything that I said to you.}}%
\verse{\JesusWords{“Peace I leave with you; my peace I give to you — not as the world gives, I give to you. Do not let your hearts be troubled, and do not let them\lnDFT{} be afraid.}}%
\verse{\JesusWords{You have heard that I said to you, ‘I am going away, and I am coming to you.’ If you loved me, you would have rejoiced that I am going to the Father, because the Father is greater than I am.}}%
\verse{\JesusWords{And now I have told you before it happens, so that when it happens, you may believe.}}%
\verse{\JesusWords{I will no longer speak much with you, for the ruler of the world is coming, and he has no power\lebnote{“nothing”} over\lebnote{“in”} me.}}%
\verse{\JesusWords{But so that the world may know that I love the Father, and just as the Father has commanded me, thus I am doing. Get up, let us go from here!}}%
\end{biblechapter}%
\begin{biblechapter}% John 15
\verseWithHeading{The Vine and the Branches}{\JesusWords{“I am the true vine, and my Father is the vinedresser.}}%
\verse{\JesusWords{Every branch that does not bear fruit in me, he removes it, and every branch that bears fruit, he prunes it in order that it may bear more fruit.}}%
\verse{\JesusWords{You are already clean because of the word that I have spoken to you.}}%
\verse{\JesusWords{Remain in me, and I in you. Just as the branch is not able to bear fruit from itself unless it remains in the vine, so neither can you, unless you remain in me.}}%
\verse{\JesusWords{“I am the vine; you are the branches. The one who remains in me and I in him — this one bears much fruit, for apart from me you are not able to do anything.}}%
\verse{\JesusWords{If anyone does not remain in me, he is thrown out as a branch, and dries up, and they gather them and throw them\lebnote{supplied from English context} into the fire, and they are burned.}}%
\verse{\JesusWords{If you remain in me and my words remain in you, ask whatever you want and it will be done for you.}}%
\verse{\JesusWords{My Father is glorified by this: that you bear much fruit, and prove to be my disciples.}}%
\verse{\JesusWords{“Just as the Father has loved me, I also have loved you. Remain in my love.}}%
\verse{\JesusWords{If you keep my commandments, you will remain in my love, just as I have kept my Father’s commandments and remain in his love.}}%
\verse{\JesusWords{I have spoken these things to you in order that my joy may be in you, and your joy may be made complete.}}%
\verse{\JesusWords{This is my commandment: that you love one another just as I have loved you.}}%
\verse{\JesusWords{No one has greater love than this: that someone lay down his life for his friends.}}%
\verse{\JesusWords{You are my friends if you do what I command you.}}%
\verse{\JesusWords{No longer do I call you slaves, because the slave does not know what his master is doing. But I have called you friends, because everything that I have heard from my Father I have revealed to you.}}%
\verse{\JesusWords{You did not choose me, but I chose you and appointed you that you should go and bear fruit, and your fruit should remain, in order that whatever you ask the Father in my name he will give you.}}%
\verse{\JesusWords{These things I command you: that you love one another.}}%
\verseWithHeading{The World’s Hatred for the Disciples}{\JesusWords{If the world hates you, you know that it has hated me before it hated\lebnote{the verb “\textit{hated}” is an understood repetition of the verb earlier in this verse} you.}}%
\verse{\JesusWords{If you were of the world, the world would love its own. But because you are not of the world, but I chose you out of the world, for this reason the world hates you.}}%
\verse{\JesusWords{Remember the word that I said to you: ‘A slave is not greater than his master.’ If they persecuted me, they will also persecute you. If they kept my word, they will keep yours also.}}%
\verse{\JesusWords{But they will do all these things to you on account of my name, because they do not know the one who sent me.}}%
\verse{\JesusWords{If I had not come and spoken to them, they would not have sin. But now they do not have a valid excuse for their sin.}}%
\verse{\JesusWords{The one who hates me hates my Father also.}}%
\verse{\JesusWords{If I had not done among them the works that no one else did, they would not have sin. But now they have both seen and hated both me and my Father.}}%
\verse{\JesusWords{But this happened\lebnote{The phrase “\textit{this happened}” is not in the Greek text but is implied} so that the word that is written in their law would be fulfilled, ‘They hated me without cause.’}}%
\verse{\JesusWords{“When the Advocate comes, whom I will send to you from the Father — the Spirit of truth, who proceeds from the Father — that one will testify about me.}}%
\verse{\JesusWords{And you also will testify, because you have been with me from the beginning.}}%
\end{biblechapter}%
\begin{biblechapter}% John 16
\verseWithHeading{Persecution of Disciples Predicted}{\JesusWords{“I have said these things to you so that you will not fall away.}}%
\verse{\JesusWords{They will expel you from the synagogue,\lebnote{“they will make you expelled from the synagogue”} but an hour is coming that everyone who kills you will think they are offering service to God.}}%
\verse{\JesusWords{And they will do these things because they do not know the Father or me.}}%
\verse{\JesusWords{But I have said these things to you so that when their hour comes, you may remember that I told you about them.\innerVerseHeading{Jesus’ Departure and the Coming of the Holy Spirit}“And I have not said these things to you from the beginning, because I was with you.}}%
\verse{\JesusWords{But now I am going away to the one who sent me, and none of you is asking me, ‘Where are you going?’}}%
\verse{\JesusWords{But because I have said these things to you, sorrow has filled your hearts.}}%
\verse{\JesusWords{But I tell you the truth, it is better for you that I go away. For if I do not go away, the Advocate will not come to you; but if I go, I will send him to you.}}%
\verse{\JesusWords{And when he\lebnote{“\textit{when}” is supplied as a component of the participle (“comes”) which is understood as temporal} comes, he will convict the world concerning sin and concerning righteousness and concerning judgment:}}%
\verse{\JesusWords{concerning sin, because they do not believe in me,}}%
\verse{\JesusWords{and concerning righteousness, because I am going away to the Father and you will see me no more,}}%
\verse{\JesusWords{and concerning judgment, because the ruler of this world has been condemned.}}%
\verse{\JesusWords{I still have many things to say to you, but you are not able to bear them\lnDFV{} now.}}%
\verse{\JesusWords{But when he — the Spirit of truth — comes, he will guide you into all the truth. For he will not speak from himself, but whatever he hears he will speak, and he will proclaim to you the things to come.}}%
\verse{\JesusWords{He will glorify me, because he will take from what is mine and will proclaim it\lnDFV{} to you.}}%
\verse{\JesusWords{Everything that the Father has is mine. For this reason I said that he takes from what is mine and will proclaim it\lnDFV{} to you.}}%
\verseWithHeading{Jesus Predicts His Return to the Disciples}{\JesusWords{“A little while and you will see me no more, and again a little while and you will see me.}}%
\verse{So some of his disciples said to one another, “What is this that he is saying to us, \JesusWords{‘A little while and you will not see me, and again a little while and you will see me,’} and \JesusWords{‘Because I am going away to the Father’}?”}%
\verse{So they kept on saying, “What is this that he is saying, \JesusWords{‘A little while’}? We do not understand what he is speaking about!”}%
\verse{Jesus knew that they were wanting to ask him, and he said to them, \JesusWords{“Are you deliberating with one another about this — that I said, ‘A little while, and you will not see me, and again a little while and you will see me’?}}%
\verse{\JesusWords{Truly, truly I say to you, that you will weep and lament, but the world will rejoice; you will become sorrowful, but your sorrow will change to joy.}}%
\verse{\JesusWords{A woman, when she gives birth, experiences pain because her hour has come. But when her\lebnote{“the”: the Greek article is used here as a possessive pronoun} child is born, she no longer remembers the affliction, on account of the joy that a human being has been born into the world.}}%
\verse{\JesusWords{So you also are experiencing sorrow now, but I will see you again, and your hearts will rejoice, and no one will take away your joy from you.}}%
\verse{\JesusWords{And on that day you will ask me nothing. Truly, truly I say to you, whatever you ask the Father in my name, he will give you.}}%
\verse{\JesusWords{Until now you have asked nothing in my name. Ask and you will receive, so that your joy may be complete.}}%
\verseWithHeading{Jesus’ Victory Over the World}{\JesusWords{“I have said these things to you in figurative sayings. An hour is coming when I will speak to you in figurative sayings no longer, but I will tell you plainly about the Father.}}%
\verse{\JesusWords{On that day you will ask in my name, and I do not say to you that I will ask the Father on your behalf.}}%
\verse{\JesusWords{For the Father himself loves you, because you have loved me and have believed that I came from God.}}%
\verse{\JesusWords{I have gone out from the Father and have come into the world; again, I am leaving the world and am going to the Father.”}}%
\verse{His disciples said, “Behold, now you are speaking plainly\lebnote{“with plainness”} and are telling us\lnDFV{} no figurative saying!}%
\verse{Now we know that you know everything and do not need for\lebnote{“have need that”} anyone to ask you questions.\lebnote{*Here the word “\textit{questions}” is not in the Greek text but is implied} By this we believe that you have come from God.”}%
\verse{Jesus replied to them, \JesusWords{“Now do you believe?}}%
\verse{\JesusWords{Behold, an hour is coming — and has come — that you will be scattered each one to his own home,\lebnote{Or “to his own things”; or “to his own people” (i.e., family); the Greek text is somewhat ambiguous here} and you will leave me alone. And I am not alone, because the Father is with me.}}%
\verse{\JesusWords{I have said these things to you so that in me you may have peace. In the world you have affliction, but have courage! I have conquered the world.”}}%
\end{biblechapter}%
\begin{biblechapter}% John 17
\verseWithHeading{Jesus Prays to be Glorified}{Jesus said these things, and lifting up his eyes to heaven he said, \JesusWords{“Father, the hour has come! Glorify your Son, in order that your Son may glorify you —}}%
\verse{\JesusWords{just as you have given him authority over all flesh, in order that he would give eternal life to them — everyone whom you have given him.}}%
\verse{\JesusWords{Now this is eternal life: that they know you, the only true God, and Jesus Christ, whom you have sent.}}%
\verse{\JesusWords{I have glorified you on earth by\lebnote{“\textit{by}” is supplied as a component of the participle (“completing”) which is understood as means} completing the work that you have given me to do.}\lebnote{“that I may do it”}}%
\verse{\JesusWords{And now, Father, you glorify me at your side\lebnote{“by the side of yourself”} with the glory that I had at your side\lebnote{“by the side of you”} before the world existed.}}%
\verseWithHeading{Jesus Prays for His Disciples}{\JesusWords{“I have revealed your name to the men whom you gave me out of the world. They were yours, and you have given them to me, and they have kept your word.}}%
\verse{\JesusWords{Now they understand that all the things that\lebnote{“whatever”} you have given me are from you,}}%
\verse{\JesusWords{because the words that you gave to me I have given to them, and they received them\lnDFW{} and know truly that I have come from you, and they have believed that you have sent me.}}%
\verse{\JesusWords{I am asking on behalf of them. I am not asking on behalf of the world, but on behalf of those whom you have given me, because they are yours,}}%
\verse{\JesusWords{and all my things are yours, and your things are mine, and I have been glorified in\lebnote{Or “by”; or “through”} them.}}%
\verse{\JesusWords{And I am no longer in the world, and they are in the world, and I am coming to you. Holy Father, keep them in your name, which you have given to me, so that they may be one, just as we are.}}%
\verse{\JesusWords{When I was with them, I kept them in your name, which you have given to me, and guarded them,\lebnote{*supplied from English context} and none of them has perished except the son of destruction, in order that the scripture would be fulfilled.}}%
\verse{\JesusWords{And now I am coming to you, and I am saying these things in the world so that they may have my joy completed in themselves.}}%
\verse{\JesusWords{I have given them your word, and the world has hated them, because they are not of the world just as I am not of the world.}}%
\verse{\JesusWords{I do not ask that you take them out of the world, but that you protect them from the evil one.}}%
\verse{\JesusWords{They are not of the world, just as I am not of the world.}}%
\verse{\JesusWords{Sanctify them in the truth — your word is truth.}}%
\verse{\JesusWords{Just as you sent me into the world, I also have sent them into the world.}}%
\verse{\JesusWords{And for them I sanctify myself, so that they themselves also may be sanctified in the truth.}}%
\verseWithHeading{Jesus Prays for the Unity of Believers}{\JesusWords{“And I do not ask on behalf of these only, but also on behalf of those who believe in me through their word,}}%
\verse{\JesusWords{that they all may be one, just as you, Father, are in me and I am in you, that they also may be in us, in order that the world may believe that you sent me.}}%
\verse{\JesusWords{And the glory that you have given to me, I have given to them, in order that they may be one, just as we are one —}}%
\verse{\JesusWords{I in them, and you in me, in order that they may be completed in one, so that the world may know that you sent me, and you have loved them just as you have loved me.}}%
\verse{\JesusWords{“Father, those whom you have given to me — I want that those also may be with me where I am, in order that they may see my glory that you have given me because you loved me before the foundation of the world.}}%
\verse{\JesusWords{Righteous Father, although the world does not know you, yet I have known you, and these men have come to know that you sent me.}}%
\verse{\JesusWords{And I made known to them your name, and will make it\lnDFW{} known, in order that the love with which you loved me may be in them, and I may be in them.”}}%
\end{biblechapter}%
\begin{biblechapter}% John 18
\verseWithHeading{Jesus Is Betrayed and Arrested}{When\lnDFX{} Jesus had said these things, he went out with his disciples to the other side of the ravine of the Kidron, where there was a garden into which he and his disciples entered.}%
\verse{(Now Judas, the one who betrayed him, also knew about the place, because Jesus often gathered there with his disciples.)}%
\verse{So Judas, taking the cohort and officers from the chief priests and from the Pharisees, came there with lanterns and torches and weapons.}%
\verse{Then Jesus, because he\lebnote{“\textit{because}” is supplied as a component of the participle (“knew”) which is understood as causal} knew all the things that were coming upon him, went out and said to them, \JesusWords{“Who are you looking for?”}}%
\verse{They replied to him, “Jesus the Nazarene.” He said to them, \JesusWords{“I am he.”}\lnDFY{} (Now Judas, the one who betrayed him, was also standing with them.)}%
\verse{So when he said to them, \JesusWords{“I am he,”}\lnDFY{} they drew back\lebnote{“they went to the back”} and fell to the ground.}%
\verse{Then he asked them again, \JesusWords{“Who are you looking for?”} And they said, “Jesus the Nazarene.”}%
\verse{Jesus replied, \JesusWords{“I said to you that I am he!\lnDFY{} So if you are looking for me, let these men go,”}}%
\verse{in order that the word that he had spoken would be fulfilled: \JesusWords{“Those whom you have given to me — I have not lost anyone of them.”\lebnote{A paraphrase of the statement in John 17:12}}}%
\verse{Then Simon Peter, who had a sword, drew it and struck the slave of the high priest and cut off his right ear. (Now the name of the slave was Malchus.)}%
\verse{So Jesus said to Peter, \JesusWords{“Put the sword into its\lebnote{“the”: the Greek article is used here as a possessive pronoun} sheath! The cup that the Father has given me — shall I not drink it?”}}%
\verseWithHeading{Jesus Taken to Annas}{Then the cohort and the military tribune and the officers of the Jews seized Jesus and tied him up,}%
\verse{and brought him\lnDFZ{} to Annas first, for he was the father-in-law of Caiaphas, who was high priest that year.}%
\verse{(Now it was Caiaphas who had advised the Jews that it was better that one man die for the people.)}%
\verseWithHeading{Peter Denies Jesus the First Time}{So Simon Peter and another disciple followed Jesus. (Now that disciple was known to the high priest, and entered with Jesus into the courtyard of the high priest.)}%
\verse{But Peter was standing by the door outside. So the other disciple who was known to the high priest went out and spoke to the doorkeeper and brought Peter in.}%
\verse{Then the female slave who was the doorkeeper said to Peter, “You are not also one of the disciples of this man, are you?”\lnDGA{} He said, “I am not!”}%
\verse{(Now the slaves and the officers were standing there, having made a charcoal fire because it was cold, and they were warming themselves. And Peter was also standing there with them and warming himself.)}%
\verseWithHeading{Jesus Before Annas}{So the high priest questioned Jesus about his disciples and about his teaching.}%
\verse{Jesus replied to him, \JesusWords{“I have spoken openly to the world. I always taught in the synagogue and in the temple courts\lebnote{“\textit{courts}” is supplied to distinguish this area from the interior of the temple building itself} where all the Jews assemble, and I have said nothing in secret.}}%
\verse{\JesusWords{Why are you asking me? Ask those who heard what I have said to them! Behold, these people know what I said.”}}%
\verse{Now when\lebnote{“\textit{when}” is supplied as a component of the temporal genitive absolute participle (“had said”)} he had said these things, one of the officers who was standing by gave a slap in the face to Jesus, saying, “Do you reply to the high priest in this way?”}%
\verse{Jesus replied to him, \JesusWords{“If I have spoken wrongly, testify about what is wrong! But if I have spoken\lebnote{The phrase “I have spoken” is an implied repetition of the earlier phrase in this verse} correctly, why do you strike me?”}}%
\verse{Then Annas sent him, tied up, to Caiaphas the high priest.}%
\verseWithHeading{Peter Denies Jesus the Second and Third Times}{Now Simon Peter was standing there and warming himself. So they said to him, “You are not also one of his disciples, are you?”\lnDGA{} He denied it\lnDFZ{} and said, “I am not!”}%
\verse{One of the slaves of the high priest, who was related to the one whose ear Peter had cut off, said, “Did I not see you in the garden with him?”}%
\verse{So Peter denied it\lnDFZ{} again, and immediately a rooster crowed.}%
\verseWithHeading{Jesus Brought Before Pilate}{Then they brought Jesus from Caiaphas to the governor’s residence. Now it was early, and they did not enter into the governor’s residence so that they would not be defiled, but could eat the Passover.}%
\verse{So Pilate came outside to them and said, “What accusation do you bring against this man?”}%
\verse{They answered and said to him, “If this man were not doing evil, we would not have handed him over to you!”}%
\verse{So Pilate said to them, “You take him and judge him according to your law!” The Jews said to him, “It is not permitted for us to kill anyone,”}%
\verse{in order that the word of Jesus would be fulfilled that he had spoken, indicating by what sort of death he was going to die.}%
\verseWithHeading{Pilate Questions Jesus}{Then Pilate entered again into the governor’s residence and summoned Jesus and said to him, “Are you the king of the Jews?”}%
\verse{Jesus replied, \JesusWords{“Do you say this from yourself, or have others said this\lnDFZ{} to you about me?”}}%
\verse{Pilate replied, “I am not a Jew, am I?\lebnote{*The negative construction in Greek anticipates a negative answer here, indicated in the translation by the phrase “\textit{am I}”} Your people and the chief priests handed you over to me! What have you done?”}%
\verse{Jesus replied, \JesusWords{“My kingdom is not of this world. If my kingdom were of this world, my servants would be fighting so that I would not be handed over to the Jews. But now my kingdom is not from here.”}}%
\verse{Then Pilate said to him, “So then you are a king!” Jesus replied, \JesusWords{“You say that I am a king. For this reason I was born, and for this reason I have come into the world: in order that I can testify to the truth. Everyone who is of the truth hears my voice.”}}%
\verse{Pilate said to him, “What is truth?” And when he\lnDFX{} had said this, he went out again to the Jews and said to them, “I find no basis for an accusation against him.}%
\verse{But it is your custom that I release for you one prisoner\lebnote{The word “\textit{prisoner}” is not in the Greek text, but is implied} at the Passover. So do you want me\lnDFZ{} to release for you the king of the Jews?”}%
\verse{Then they shouted again, saying, “Not this man, but Barabbas!” (Now Barabbas was a revolutionary.)\lebnote{Or perhaps “robber”}}%
\end{biblechapter}%
\begin{biblechapter}% John 19
\verseWithHeading{Pilate Attempts to Release Jesus}{So then Pilate took Jesus and had him flogged.\lebnote{*This verb has causative force in context; Pilate did not personally carry out the sentence}}%
\verse{And the soldiers wove a crown of thorns and placed it\lnDGB{} on his head, and put a purple robe on him,}%
\verse{and were coming up to him and saying, “Hail, king of the Jews!” and were giving him slaps in the face.}%
\verse{And Pilate came outside again and said to them, “Behold, I am bringing him outside to you, so that you will know that I find no basis for an accusation against him.”}%
\verse{Then Jesus came outside wearing the crown of thorns and the purple robe, and he said to them, “Behold the man!”}%
\verse{So when they saw him, the chief priests and the officers shouted, saying, “Crucify! Crucify!” Pilate said to them, “You take him and crucify him!\lnDGC{} For I do not find a basis for an accusation against him.”}%
\verse{The Jews replied to him, “We have a law, and according to the law he ought to die, because he made himself out to be the Son of God!”}%
\verse{So when Pilate heard this statement, he was even more afraid,}%
\verse{and he entered into the governor’s residence again and said to Jesus, “Where are you from?” But Jesus did not give him an answer.}%
\verse{So Pilate said to him, “Will you not speak to me? Do you not know that I have authority to release you, and I have authority to crucify you?”}%
\verse{Jesus replied to him, \JesusWords{“You would not have any authority over me unless it was given to you from above. For this reason the one who handed me over to you has greater sin.”}}%
\verse{From this point on Pilate was seeking to release him, but the Jews shouted, saying, “If you release this man, you are not a friend of Caesar! Everyone who makes himself out to be a king opposes Caesar!”}%
\verse{So Pilate, when he\lebnote{“\textit{when}” is supplied as a component of the participle (“heard”) which is understood as temporal} heard these words, brought Jesus outside and sat down on the judgment seat, in the place called The Stone Pavement (but Gabbatha in Aramaic).}%
\verse{(Now it was the day of preparation of the Passover; it was about the sixth hour.) And he said to the Jews, “Behold your king!”}%
\verse{Then those shouted, “Away with him! Away with him! Crucify him!” Pilate said to them, “Shall I crucify your king?” The chief priests replied, “We do not have a king except Caesar!”}%
\verse{So then he handed him over to them in order that he could be crucified.\innerVerseHeading{Jesus Is Crucified}So they took Jesus,}%
\verse{and carrying for himself the cross, he went out to the place called The Place of a Skull (which is called Golgotha in Aramaic),}%
\verse{where they crucified him, and with him two others, one on each side,\lebnote{“from here and from here”} and Jesus in the middle.}%
\verse{And Pilate also wrote a notice and placed it\lnDGB{} on the cross, and it was written: “Jesus the Nazarene, the king of the Jews.”}%
\verse{So many of the Jews read this notice, because the place where Jesus was crucified was near the city. And it was written in Aramaic, in Latin, and in Greek.}%
\verse{Then the chief priests of the Jews said to Pilate, “Do not write ‘The king of the Jews,’ but, ‘He said, I am king of the Jews.’”}%
\verse{Pilate replied, “What I have written, I have written.”}%
\verse{Then the soldiers, when they had crucified Jesus, took his clothing and made four shares — for each soldier a share — and the tunic. (Now the tunic was seamless, woven from the top in a single piece.)\lebnote{“through the whole”}}%
\verse{So they said to one another, “Let us not tear it apart, but cast lots for it, to see whose it will be,” so that the scripture would be fulfilled that says, “They divided my garments among themselves, and for my clothing they cast lots.”\lebnote{from Ps 22:18} Thus the soldiers did these things.}%
\verse{Now his mother and the sister of his mother, Mary the wife of Clopas, and Mary Magdalene were standing near the cross of Jesus.}%
\verse{So Jesus, seeing his\lnDGD{} mother and the disciple whom he loved standing there, said to his\lnDGD{} mother, \JesusWords{“Woman, behold your son!”}}%
\verse{Then he said to the disciple, \JesusWords{“Behold your mother!”} And from that hour the disciple took her into his own home.}%
\verseWithHeading{Jesus Dies on the Cross}{After this, Jesus, knowing that now at last everything was completed, in order that the scripture would be fulfilled, said, \JesusWords{“I am thirsty.”}}%
\verse{A jar full of sour wine was standing there, so they put a sponge full of the sour wine on a branch of hyssop and\lebnote{“\textit{and}” supplied because previous participle “put” translated as a finite verb} brought it\lnDGB{} to his mouth.}%
\verse{Then when he had received the sour wine, Jesus said, \JesusWords{“It is completed,”}\lebnote{Or (traditionally) “it is finished”} and bowing his\lnDGD{} head, he gave up his\lnDGD{} spirit.}%
\verse{Then the Jews, because it was the day of preparation, so that the bodies would not remain on the cross on the Sabbath (for that Sabbath was an important day), asked Pilate that their legs could be broken and they could be taken away.}%
\verse{So the soldiers came and broke the legs of the first and of the other who had been crucified with him.}%
\verse{But when they\lebnote{“\textit{when}” is supplied as a component of the participle (“came”) which is understood as temporal} came to Jesus, after they saw he was already dead, they did not break his legs.}%
\verse{But one of the soldiers pierced his side with a spear, and blood and water came out immediately.}%
\verse{And the one who has seen it\lnDGB{} has testified, and his testimony is true, and that person knows that he is telling the truth, so that you also may believe.}%
\verse{For these things happened in order that the scripture would be fulfilled: “Not a bone of his will be broken.”\lebnote{from Exod 12:46, Num 9:12, and Ps 34:20}}%
\verse{And again another scripture says, “They will look on the one whom they have pierced.”\lebnote{from Zech 12:10}}%
\verseWithHeading{Jesus Is Buried}{And after these things, Joseph who was from Arimathea, who was a disciple of Jesus (but a secret one for fear of the Jews), asked Pilate that he might take away the body of Jesus. And Pilate allowed it,\lnDGC{} so he came and took away his body.}%
\verse{And Nicodemus — the one who had come to him formerly at night — also came, bringing a mixture of myrrh and aloes weighing about a hundred pounds.\lebnote{The Greek term refers to a Roman pound, 327.45 grams (approximately 12 ounces)}}%
\verse{So they took the body of Jesus and wrapped it in strips of linen cloth with the fragrant spices, as is the Jews’ custom to prepare for burial.}%
\verse{Now there was a garden at the place where he was crucified, and in the garden a new tomb in which no one was yet buried.}%
\verse{So there, on account of the day of preparation of the Jews, because the tomb was close by, they buried Jesus.}%
\end{biblechapter}%
\begin{biblechapter}% John 20
\verseWithHeading{Jesus Is Raised}{Now on the first day of the week, Mary Magdalene came to the tomb early, while it\lebnote{“\textit{while}” is supplied as a component of the temporal genitive absolute participle (“was”)} was still dark, and saw the stone had been taken away from the tomb.}%
\verse{So she ran and came to Simon Peter and to the other disciple whom Jesus loved and said to them, “They have taken away the Lord from the tomb and we do not know where they have put him!”}%
\verse{Then Peter and the other disciple went out and were going to the tomb.}%
\verse{And the two were running together, and the other disciple ran ahead, faster than Peter, and came to the tomb first.}%
\verse{And bending over to look, he saw the strips of linen cloth lying there, though he did not go in.}%
\verse{Then Simon Peter also came following him, and he went into the tomb and saw the strips of linen cloth lying there,}%
\verse{and the facecloth that was on his head — not lying with the strips of linen cloth, but folded up separately in one place.}%
\verse{So then the other disciple who had come to the tomb first also went in, and he saw and believed.}%
\verse{(For they did not yet know the scripture that it was necessary for him to rise from the dead.)}%
\verseWithHeading{Jesus Appears to Mary Magdalene}{Then the disciples went away again to their own homes.\lebnote{*The phrase “\textit{own homes}” is not in the Greek text but is implied}}%
\verse{But Mary stood outside at the tomb, weeping. Then, while she was weeping, she bent over to look into the tomb,}%
\verse{and she saw two angels in white, seated one at the head and one at the feet where the body of Jesus had been lying.}%
\verse{And they said to her, “Woman, why are you weeping?” She said to them, “They have taken away my Lord, and I do not know where they have put him!”}%
\verse{When she\lnDGE{} had said these things, she turned around\lebnote{“to the back”} and saw Jesus standing there, and she did not know that it was Jesus.}%
\verse{Jesus said to her, \JesusWords{“Woman, why are you weeping? Who are you looking for?”} She thought that it was the gardener, and\lebnote{“\textit{and}” supplied because previous participle “thought” translated as a finite verb} said to him, “Sir, if you have carried him away, tell me where you have put him, and I will take him.”}%
\verse{Jesus said to her, \JesusWords{“Mary.”} She turned around and\lebnote{“\textit{and}” supplied because previous participle “turned around” translated as a finite verb} said to him in Aramaic, “Rabboni” (which means “Teacher”).}%
\verse{Jesus said to her, \JesusWords{“Do not touch me, for I have not yet ascended to the Father. But go to my brothers and tell them, ‘I am ascending to my Father and your Father, and my God and your God.’”}}%
\verse{Mary Magdalene came and\lebnote{“\textit{and}” supplied because participle “announced” translated as a finite verb in keeping with English style} announced to the disciples, “I have seen the Lord,” and he had said these things to her.}%
\verseWithHeading{Jesus Appears to the Disciples}{Now when it\lebnote{“\textit{when}” is supplied as a component of the temporal genitive absolute participle (“was”)} was evening on that day — the first day of the week — and the doors had been shut where the disciples were because of fear of the Jews, Jesus came and stood in their midst and said to them, \JesusWords{“Peace to you.”}}%
\verse{And when he\lnDGE{} had said this, he showed his\lnDGF{} hands and his\lnDGF{} side to them. Then the disciples rejoiced when they\lebnote{“\textit{when}” is supplied as a component of the participle (“saw”) which is understood as temporal} saw the Lord.}%
\verse{So Jesus said to them again, \JesusWords{“Peace to you. As the Father has sent me, I also send you.”}}%
\verse{And when he\lnDGE{} had said this, he breathed on them\lnDGG{} and said to them, \JesusWords{“Receive the Holy Spirit.}}%
\verse{\JesusWords{If you forgive the sins of any, they are forgiven them. If you retain the sins\lebnote{An understood repetition of the phrase from earlier in the verse} of any, they are retained.”}}%
\verseWithHeading{Thomas Doubts But Later Believes}{Now Thomas, one of the twelve, who was called Didymus,\lebnote{The Greek term means “the Twin”} was not with them when Jesus came.}%
\verse{So the other disciples said to him, “We have seen the Lord!” But he said to them, “Unless I see in his hands the mark of the nails, and put my finger into the mark of the nails, and put my hand into his side, I will never believe!”}%
\verse{And after eight days his disciples were again inside, and Thomas with them. Although\lebnote{“\textit{although}” is supplied as a component of the participle (“had been shut”) which is understood as concessive} the doors had been shut, Jesus came and stood in their midst and said, \JesusWords{“Peace to you.”}}%
\verse{Then he said to Thomas, \JesusWords{“Place your finger here and see my hands, and place your hand and put it\lnDGG{} into my side. And do not be unbelieving, but believing!”}}%
\verse{Thomas answered and said to him, “My Lord and my God!”}%
\verse{Jesus said to him, \JesusWords{“Because you have seen me, have you believed? Blessed are those who have not seen and have believed.”}}%
\verseWithHeading{Why This Book Was Written}{Now Jesus also performed many other signs in the presence of the disciples\lebnote{Some manuscripts have “his disciples”} which are not recorded in this book,}%
\verse{but these things are recorded in order that you may believe that Jesus is the Christ, the Son of God, and that by\lebnote{“\textit{by}” is supplied as a component of the participle (“believing”) which is understood as means} believing you may have life in his name.}%
\end{biblechapter}%
\begin{biblechapter}% John 21
\verseWithHeading{Jesus Appears to the Disciples in Galilee}{After these things Jesus revealed himself again to the disciples by the Sea of Tiberias. Now he revealed himself\lnDGH{} in this way:}%
\verse{Simon Peter and Thomas (who was called Didymus)\lebnote{The Greek term means “the Twin”} and Nathanael from Cana in Galilee and the sons of Zebedee and two others of his disciples were together.}%
\verse{Simon Peter said to them, “I am going fishing!” They said to him, “We also are coming with you.” They went out and got into the boat, and during that night they caught nothing.}%
\verse{Now when it\lebnote{“\textit{when}” is supplied as a component of the temporal genitive absolute participle (“was”)} was already early morning, Jesus stood on the beach. However, the disciples did not know that it was Jesus.}%
\verse{So Jesus said to them, \JesusWords{“Children, you do not have any fish to eat, do you?}\lebnote{*The negative construction in Greek anticipates a negative answer here, indicated in the translation by the phrase “\textit{do you}”} They answered him, “No.”}%
\verse{And he said to them, \JesusWords{“Throw the net on the right side of the boat, and you will find some.”}\lnDGI{} So they threw it,\lnDGI{} and were no longer able to haul it in from the large number of the fish.}%
\verse{Then that disciple whom Jesus loved said to Peter, “It is the Lord!” So Simon Peter, when he\lebnote{“\textit{when}” is supplied as a component of the participle (“heard”) which is understood as temporal} heard that it was the Lord, tied around himself his outer garment (for he was naked)\lebnote{I.e., “he was naked underneath the outer garment,” which he tucked into his belt; alternatively, this could mean “for he was stripped for work”} and threw himself into the sea.}%
\verse{But the other disciples came in the boat, dragging the net of fish, because they were not far from the land, but about two hundred cubits\lebnote{Approximately 100 yards or 92 meters, based on a cubit of 18 inches.} away.}%
\verse{So when they got out on the land, they saw a charcoal fire laid there, and a fish lying on it,\lnDGI{} and bread.}%
\verse{Jesus said to them, \JesusWords{“Bring some of the fish that you have just now caught.”}}%
\verse{So Simon Peter got into the boat\lnDGH{} and hauled the net to the land, full of large fish — one hundred fifty-three — and although there\lebnote{“\textit{although}” is supplied as a component of the participle (“were”) which is understood as concessive} were so many, the net was not torn.}%
\verse{Jesus said to them, \JesusWords{“Come, eat breakfast!”} But none of the disciples dared to ask him, “Who are you?” because they\lebnote{“\textit{because}” is supplied as a component of the participle (“knew”) which is understood as causal} knew that it was the Lord.}%
\verse{Jesus came and took the bread and gave it\lnDGH{} to them, and the fish likewise.}%
\verse{This was now the third time Jesus was revealed to the disciples after he\lebnote{“\textit{after}” is supplied as a component of the participle (“had been raised”) which is understood as temporal} had been raised from the dead.}%
\verseWithHeading{Peter Is Restored Three Times}{Now when they had eaten breakfast, Jesus said to Simon Peter, \JesusWords{“Simon son of John, do you love me more than these?”} He said to him, “Yes, Lord, you know that I love you.” He said to him, \JesusWords{“Feed my lambs!”}}%
\verse{He said to him again a second time, \JesusWords{“Simon son of John, do you love me?”} He said to him, “Yes, Lord, you know that I love you.” He said to him, \JesusWords{“Shepherd my sheep!”}}%
\verse{He said to him a third time, \JesusWords{“Simon son of John, do you love me?”} Peter was distressed because he said to him a third time, \JesusWords{“Do you love me?”} and he said to him, “Lord, you know everything! You know that I love you!” Jesus said to him, \JesusWords{“Feed my sheep!}}%
\verse{\JesusWords{Truly, truly I say to you, when you were young, you tied your clothes\lebnote{The words “\textit{your clothes}” are not in the Greek text but are implied} around yourself and walked wherever you wanted. But when you grow old, you will stretch out your hands, and someone else will tie you up and carry you where you do not want to go.}\lebnote{*The words “\textit{to go}” are not in the Greek text but are implied}}%
\verse{(Now he said this to indicate by what kind of death he would glorify God.) And after he\lebnote{“\textit{after}” is supplied as a component of the participle (“had said”) which is understood as temporal} had said this, he said to him, \JesusWords{“Follow me!”}}%
\verseWithHeading{Peter and the Other Disciple Jesus Loved}{Peter turned and\lebnote{“\textit{and}” supplied because previous participle “turned” translated as a finite verb} saw the disciple whom Jesus loved following them\lnDGH{} (who also leaned back on his chest at the dinner and said, “Lord, who is the one betraying you?”)}%
\verse{So when he\lebnote{“\textit{when}” is supplied as a component of the participle (“saw”) which is understood as temporal} saw this one, Peter said to Jesus, “Lord, but what about this one?”}%
\verse{Jesus said to him, \JesusWords{“If I want him to remain until I come, what is that\lnDGJ{} to you? You follow me!”}}%
\verse{So this saying went out to the brothers that that disciple would not die. But Jesus did not say to him that he would not die, but \JesusWords{“If I want him to remain until I come, what is that\lnDGJ{} to you?”}}%
\verseWithHeading{A Concluding Word of Testimony}{This is the disciple who is testifying about these things, and who has written down these things. And we know that his testimony is true.}%
\verse{Now there are also many other things that Jesus did, which — if they were written down one after the other — I suppose not even the world itself could contain the books that would be written.}%
\end{biblechapter}%
\flushcolsend
\input{leb/content/new-testament/Acts.tex}\flushcolsend
\biblebook{Romans}
\begin{biblechapter}% Romans 1
\verseWithHeading{Greeting}{Paul, a slave of Christ Jesus, called to be an apostle, set apart for the gospel of God,}%
\verse{which he promised previously through his prophets in the holy scriptures,}%
\verse{concerning his Son, who was born a descendant\lebnote{“of the seed”} of David according to the flesh,}%
\verse{who was declared Son of God in power according to the Holy Spirit\lebnote{“the Spirit of holiness”} by the resurrection from the dead of Jesus Christ our Lord,}%
\verse{through whom we have received grace and apostleship for the obedience of faith among all the Gentiles\lnDKI{} on behalf of his name,}%
\verse{among whom you also are the called of Jesus Christ.}%
\verse{To all those in Rome who are loved by God, called to be saints. Grace to you and peace from God our Father and the Lord Jesus Christ.}%
\verseWithHeading{Paul Wants to Visit Rome}{First, I give thanks to my God through Jesus Christ for all of you, because your faith is being proclaimed in the whole world.}%
\verse{For God, whom I serve with my spirit in the gospel of his Son, is my witness, how constantly I make mention of you,}%
\verse{always asking in my prayers if somehow now at last I may succeed to come to you in the will of God.}%
\verse{For I desire to see you, in order that I may impart some spiritual gift to you, in order to strengthen you,}%
\verse{that is, to be encouraged together with you through our mutual faith\lebnote{“the in one another faith”}, both yours and mine.}%
\verse{Now I do not want you to be ignorant, brothers, that often I intended to come to you, and was prevented until now, in order that I might have some fruit among you also, just as also among the rest of the Gentiles.\lnDKI{}}%
\verse{I am under obligation both to Greeks and to barbarians, both to the wise and to the foolish.}%
\verse{Thus I am eager\lebnote{“the according to me eagerness”} to proclaim the gospel also to you who are in Rome.}%
\verseWithHeading{The Gospel’s Power for Salvation}{For I am not ashamed of the gospel, for it is the power of God for salvation to everyone who believes, to the Jew first and also to the Greek.}%
\verse{For the righteousness of God is revealed in it from faith to faith, just as it is written, “But the one who is righteous by faith will live.”\lebnote{Or “But the one who is righteous will live by faith” (differing only in word order)}}%
\verseWithHeading{God’s Wrath Revealed Against Sinful Humanity}{For the wrath of God is revealed from heaven against all impiety and unrighteousness of people, who suppress the truth in unrighteousness,}%
\verse{because what can be known about God is evident among\lebnote{Or “in”; or “within”} them, for God made it clear to them.}%
\verse{For from the creation of the world, his invisible attributes, both his eternal power and deity, are discerned clearly, being understood in the things created, so that they are without excuse.}%
\verse{For although they knew God, they did not honor him as God or give thanks, but they became futile in their reasoning, and their senseless hearts were darkened.}%
\verse{Claiming to be wise, they became fools,}%
\verse{and exchanged the glory of the immortal God with the likeness of an image of mortal human beings and birds and quadrupeds and reptiles.}%
\verse{Therefore God gave them over in the desires of their hearts to immorality, that their bodies would be dishonored among themselves,}%
\verse{who exchanged the truth of God with a lie, and worshiped and served the creation rather than the Creator, who is blessed for eternity. Amen.}%
\verseWithHeading{God Hands Sinful Humanity over to Depravity}{Because of this, God gave them over to degrading passions, for their females exchanged the natural relations for those contrary to nature,}%
\verse{and likewise also the males, abandoning the natural relations with the female, were inflamed in their desire toward one another, males with males committing the shameless deed, and receiving in themselves the penalty that was necessary for their error.}%
\verse{And just as they did not see fit to recognize God\lebnote{“to have God in recognition”}, God gave them over to a debased mind, to do the things that are not proper,}%
\verse{being filled with all unrighteousness, wickedness, greediness, malice, full of envy, murder, strife, deceit, malevolence. They are gossipers,}%
\verse{slanderers, haters of God, insolent, arrogant, boasters, contrivers of evil, disobedient to parents,}%
\verse{senseless, faithless, unfeeling, unmerciful,}%
\verse{who, although they\lebnote{“\textit{although}” is supplied as a component of the participle (“know”) which is understood as concessive} know the requirements of God, that those who do such things are worthy of death, not only do they do the same things, but also they approve of those who do them.}%
\end{biblechapter}%
\begin{biblechapter}% Romans 2
\verseWithHeading{The Righteous and Impartial Judgment of God}{Therefore you are without excuse, O man, every one of you who passes judgment. For in that which you pass judgment on someone else, you condemn yourself, for you who are passing judgment are doing the same things.}%
\verse{Now we know that the judgment of God is according to truth against those who do such things.}%
\verse{But do you think this, O man who passes judgment on those who do such things, and who does the same things, that you will escape the judgment of God?}%
\verse{Or do you despise the wealth of his kindness and forbearance and patience, not knowing that the kindness of God leads you to repentance?}%
\verse{But because of your stubbornness and unrepentant heart, you are storing up for yourself wrath in the day of wrath and of the revelation of the righteous judgment of God,}%
\verse{who will reward each one according to his works:}%
\verse{to those who, by perseverance in good work, seek glory and honor and immortality, eternal life,}%
\verse{but to those who act from selfish ambition and who disobey the truth, but who obey unrighteousness, wrath and anger.}%
\verse{There will be affliction and distress for every human being\lebnote{“soul of man”} who does evil, of the Jew first and of the Greek,}%
\verse{but glory and honor and peace to everyone who does good, to the Jew first and to the Greek.}%
\verse{For there is no partiality with God.}%
\verse{For as many as have sinned without law will also perish without law, and as many as have sinned under the law will be judged by the law.}%
\verse{For it is not the hearers of the law who are righteous in the sight of God, but the doers of the law will be declared righteous.\lebnote{Or “will be justified”}}%
\verse{For whenever the Gentiles, who do not have the law, do by nature the things of the law, these, although they\lebnote{“\textit{although}” is supplied as a component of the participle (“have”) which is understood as concessive} do not have the law, are a law to themselves,}%
\verse{who show the work of the law written on their hearts, their conscience bearing witness and their thoughts one after another accusing or even defending them}%
\verse{on the day when God judges the secret things of people, according to my gospel, through\lebnote{Or “by”} Christ Jesus.}%
\verseWithHeading{Jews also Condemned by the Law}{But if you call yourself a Jew and rely on the law and boast in God}%
\verse{and know his will and approve the things that are superior, because you\lebnote{“\textit{because}” is supplied as a component of the participle (“are instructed”) which is understood as causal} are instructed by the law,}%
\verse{and are confident that you yourself are a guide of the blind, a light of those in darkness,}%
\verse{an instructor of the foolish, a teacher of the immature, having the embodiment of knowledge and of the truth in the law.}%
\verse{Therefore, the one who teaches someone else, do you not teach yourself? The one who preaches not to steal, do you steal?}%
\verse{The one who says not to commit adultery, do you commit adultery? The one who abhors idols, do you rob temples?}%
\verse{Who boast in the law, by the transgression of the law you dishonor God!\lebnote{Or “do you dishonor God?” (a number of translators and interpreters take this phrase as a final rhetorical question; the present translation regards it as a final summary statement to be taken ironically)}}%
\verse{For just as it is written, “The name of God is blasphemed among the Gentiles because of you.”\lebnote{from Isa 52:5}}%
\verse{For circumcision is of value if you do the law, but if you should be a transgressor of the law, your circumcision has become uncircumcision.}%
\verse{Therefore, if the uncircumcised person follows the requirements of the law, will not his uncircumcision be credited for circumcision?}%
\verse{And the uncircumcised person by nature who carries out the law will judge you who, though provided with the precise written code\lebnote{“the letter”} and circumcision are a transgressor of the law.}%
\verse{For the Jew is not one outwardly\lnDKJ{}, nor is circumcision outwardly\lnDKJ{}, in the flesh.}%
\verse{But the Jew is one inwardly\lebnote{“in secret”}, and circumcision is of the heart, by the Spirit, not by the letter, whose praise is not from people but from God.}%
\end{biblechapter}%
\begin{biblechapter}% Romans 3
\verseWithHeading{Jews Still Have an Advantage}{Therefore, what is the advantage of the Jew, or what is the use of circumcision?}%
\verse{Much in every way. For first, that they were entrusted with the oracles of God.}%
\verse{What is the result\lebnote{“for what”} if some refused to believe? Their unbelief will not nullify the faithfulness of God, will it?}%
\verse{May it never be! But let God be true but every human being a liar, just as it is written, “In order that you may be justified in your words, and may prevail when you are\lebnote{Or, if the form is understood as middle voice, “when you yourself judge”} judged.”\lebnote{from Ps 51:4}}%
\verse{But if our unrighteousness demonstrates the righteousness of God, what shall we say? God, who inflicts wrath, is not unjust, is he? (I am speaking according to a human perspective.)}%
\verse{May it never be! For otherwise, how will God judge the world?}%
\verse{But if by my lying, the truth of God abounded to his glory, why am I also still condemned as a sinner?}%
\verse{And why not (as we are slandered, and as some affirm that we say), “Let us do evil, in order that good may come of it? Their\lebnote{“whose”} condemnation is just!}%
\verseWithHeading{The Entire World Guilty of Sin}{What then? Do we have an advantage? Not at all. For we have already charged both Jews and Greeks are all under sin,}%
\verse{just as it is written, “There is no one righteous, not even one;}%
\verse{there is no one who understands; there is no one who seeks God.}%
\verse{All have turned aside together; they have become worthless; There is no one who practices kindness; there is not even one.\lebnote{Verses 10–12 are a quotation from Ps 14:1–3}}%
\verse{Their throat is an opened grave; they deceive with their tongues; the venom of asps is under their lips,\lebnote{from Ps 5:9 and Ps 140:3}}%
\verse{whose mouth is full of cursing and bitterness.\lebnote{from Ps 10:7}}%
\verse{Their feet are swift to shed blood;}%
\verse{destruction and distress are in their paths,}%
\verse{and they have not known the way of peace.\lebnote{Verses 15–17 are a quotation from Isa 59:7–8}}%
\verse{The fear of God is not before their eyes.”\lebnote{from Ps 36:1}}%
\verse{Now we know that whatever the law says, it speaks to those under the law, in order that every mouth may be closed and the whole world may become accountable to God.}%
\verse{For by the works of the law no person will be declared righteous\lebnote{“all flesh will not be declared righteous”} before him, for through the law comes knowledge of sin.}%
\verseWithHeading{Righteousness through Faith Revealed}{But now, apart from the law, the righteousness of God has been revealed, being testified about by the law and the prophets —}%
\verse{that is, the righteousness of God through faith in Jesus Christ\lebnote{Or “through the faithfulness of Jesus Christ”} to all who believe. For there is no distinction,}%
\verse{for all have sinned and fall short of the glory of God,}%
\verse{being justified as a gift by his grace, through the redemption which is in Christ Jesus,}%
\verse{whom God made publicly available as the mercy seat\lebnote{Or “as the place of propitiation”} through faith in his blood, for a demonstration of his righteousness, because of the passing over of previously committed sins,}%
\verse{in the forbearance of God, for the demonstration of his righteousness in the present time, so that he should be just and the one who justifies the person by faith\lebnote{Or “by Jesus’ faithfulness”} in Jesus.}%
\verse{Therefore, where is boasting? It has been excluded. By what kind of law? Of works? No, but by a law\lebnote{Or “a principle”} of faith.}%
\verse{For we consider a person to be justified by faith apart from the works of the law.}%
\verse{Or is God the God of the Jews only? Is he not also the God of the Gentiles? Yes, also of the Gentiles,}%
\verse{since God is one, who will justify those who are circumcised\lebnote{“circumcision”} by faith and those who are uncircumcised\lebnote{“uncircumcision”} through faith.}%
\verse{Therefore, do we nullify the law through faith? May it never be! But we uphold the law.}%
\end{biblechapter}%
\begin{biblechapter}% Romans 4
\verseWithHeading{Abraham’s Faith Counted as Righteousness}{What then shall we say that Abraham, our ancestor according to the flesh, has found?}%
\verse{For if Abraham was justified by works, he has something to boast about, but not before God.}%
\verse{For what does the scripture say? “And Abraham believed God, and it was credited to him for righteousness.”\lnDKK{}}%
\verse{Now to the one who works, his pay is not credited according to grace, but according to his due.}%
\verse{But to the one who does not work, but who believes in the one who justifies the ungodly, his faith is credited for righteousness,}%
\verse{just as David also speaks about the blessing of the person to whom God credits righteousness apart from works:}%
\verse{“Blessed are they whose lawless deeds have been forgiven, and whose sins are covered over.}%
\verse{Blessed is the person against whom the Lord will never count sin.”\lebnote{from Ps 32:1–2}}%
\verse{Therefore, is this blessing for those who are circumcised\lebnote{“the circumcision”}, or also for those who are uncircumcised\lebnote{“the uncircumcision”}? For we say, “Faith was credited to Abraham for righteousness.”\lnDKK{}}%
\verse{How then was it credited? While he\lebnote{“\textit{while}” is supplied as a component of the participle (“was”) which is understood as temporal} was circumcised\lnDKL{} or uncircumcised\lnDKM{}? Not while circumcised\lnDKL{} but while uncircumcised\lnDKM{}!}%
\verse{And he received the sign of circumcision as a seal\lebnote{Or “confirmation”} of the righteousness by faith which he had while uncircumcised\lnDKM{}, so that he could be the father of all who believe although they are uncircumcised\lebnote{“through uncircumcision”}, so that righteousness could be credited to them,\lebnote{Some manuscripts have “could be credited to them also”}}%
\verse{and the father of those who are circumcised\lebnote{“of the circumcision”} to those who are not only from the circumcision, but who also follow in the footsteps of the faith of our father Abraham which he had while uncircumcised\lebnote{“of the in uncircumcision faith of our father Abraham”}.}%
\verseWithHeading{The Promise to Abraham Secured through Faith}{For the promise to Abraham or to his descendants, that he would be heir of the world, was not through the law, but through the righteousness by faith.}%
\verse{For if those of the law are heirs, faith is rendered void and the promise is nullified.}%
\verse{For the law produces wrath, but where there is no law, neither is there transgression.}%
\verse{Because of this, it is by faith, in order that it may be according to grace, so that the promise may be secure to all the descendants, not only to those of the law, but also to those of the faith of Abraham, who is the father of us all}%
\verse{(just as it is written, “I have made you the father of many nations”)\lebnote{from Gen 17:5} before God, in whom he believed, the one who makes the dead alive and who calls the things that are not as though they are,}%
\verse{who against hope believed in hope, so that he became the father of many nations, according to what was said, “so will your descendants be.”\lebnote{from Gen 15:5}}%
\verse{And not being weak in faith, he considered his own body as good as dead,\lebnote{Some manuscripts have “already as good as dead”} because he\lebnote{“\textit{because}” is supplied as a component of the participle (“was”) which is understood as causal} was approximately a hundred years old, and the deadness of Sarah’s womb.}%
\verse{And he did not waver in unbelief at the promise of God, but was strengthened in faith, giving glory to God}%
\verse{and being fully convinced that what he had promised, he was also able to do.}%
\verse{Therefore\lebnote{Some manuscripts have “Therefore, indeed,”} it was credited to him for righteousness.}%
\verse{But it was not written for the sake of him alone that it was credited to him,}%
\verse{but also for the sake of us to whom it is going to be credited, to those who believe in the one who raised Jesus our Lord from the dead,}%
\verse{who was handed over on account of our trespasses, and was raised up in the interest of our justification.\lebnote{Or “vindication”; or “acquittal”}}%
\end{biblechapter}%
\begin{biblechapter}% Romans 5
\verseWithHeading{Reconciliation with God through Faith in Christ}{Therefore, because we\lnDKN{} have been declared righteous by faith, we have\lebnote{Although a number of important manuscripts read the subjunctive mood here (“let us have”), almost all English versions prefer the indicative mood (“we have”) which is supported by many other manuscripts} peace with God through our Lord Jesus Christ,}%
\verse{through whom also we have obtained access by faith into this grace in which we stand, and we boast in the hope of the glory of God.}%
\verse{And not only this, but we also boast in our afflictions, because we\lebnote{“\textit{because}” is supplied as a component of the participle (“know”) which is understood as causal} know that affliction produces patient endurance,}%
\verse{and patient endurance, proven character, and proven character, hope,}%
\verse{and hope does not disappoint, because the love of God has been poured out in our hearts through the Holy Spirit who was given to us.}%
\verse{For while\lnDKO{} we were still helpless, yet at the proper time Christ died for the ungodly.}%
\verse{For only rarely will someone die on behalf of a righteous person (for on behalf of a good person possibly someone might even dare to die),}%
\verse{but God demonstrates his own love for us, in that while\lnDKO{} we were still sinners, Christ died for us.}%
\verse{Therefore, by much more, because we\lnDKN{} have been declared righteous now by his blood, we will be saved through him from the wrath.}%
\verse{For if, while we\lnDKO{} were enemies, we were reconciled to God through the death of his Son, by much more, having been reconciled, we will be saved by his life.}%
\verse{And not only this, but also we are boasting in God through our Lord Jesus Christ, through whom we have now received the reconciliation.}%
\verseWithHeading{Death Came through Adam but Life Comes through Christ}{Because of this, just as sin entered into the world through one man, and death through sin, so also death spread to all people because all sinned.}%
\verse{For until the law, sin was in the world, but sin is not charged to one’s account when there\lebnote{“\textit{when}” is supplied as a component of the participle (“is”) which is understood as temporal} is no law.}%
\verse{But death reigned from Adam until Moses even over those who did not sin in the likeness of the transgression of Adam, who is a type of the one who is to come.}%
\verse{But the gift is not like the trespass\lebnote{“but not like the trespass so also the gift”}, for if by the trespass of the one, the many died, by much more did the grace of God and the gift by the grace of the one man, Jesus Christ, multiply to the many.}%
\verse{And the gift is not as through the one who sinned, for on the one hand, judgment from the one sin led to condemnation, but the gift, from many trespasses, led to justification.}%
\verse{For if by the trespass of the one man, death reigned through the one man, much more will those who receive the abundance of grace and of the gift of righteousness reign in life through the one, Jesus Christ.}%
\verse{Consequently therefore, as through one trespass came condemnation to all people, so also through one righteous deed came justification of life to all people.}%
\verse{For just as through the disobedience of the one man, the many were made sinners, so also through the obedience of the one, the many will be made righteous.}%
\verse{Now the law came in as a side issue, in order that the trespass could increase, but where sin increased, grace was present in greater abundance,}%
\verse{so that just as sin reigned in death, so also grace would reign through righteousness to eternal life through Jesus Christ our Lord.}%
\end{biblechapter}%
\begin{biblechapter}% Romans 6
\verseWithHeading{Formerly Dead to Sin, Now Alive in Christ}{What therefore shall we say? Shall we continue in sin, in order that grace may increase?}%
\verse{May it never be! How can we who died to sin still live in it?}%
\verse{Or do you not know that as many as were baptized into Christ Jesus were baptized into his death?}%
\verse{Therefore we have been buried with him through baptism into death, in order that just as Christ was raised from the dead through the glory of the Father, so also we may live a new way of life\lebnote{“in newness of life”}.}%
\verse{For if we have become identified with him in the likeness of his death, certainly also we will be identified with him in the likeness\lebnote{The elliptical phrase “identified with him in the likeness” has been supplied in the translation for clarity} of his resurrection,}%
\verse{knowing this, that our old man was crucified together with him, in order that the body of sin may be done away with, that we may no longer be enslaved to sin.}%
\verse{For the one who has died has been freed from sin.}%
\verse{Now if we died with Christ, we believe that we will also live with him,}%
\verse{knowing that Christ, because he\lebnote{“\textit{because}” is supplied as a component of the participle (“has been raised”) which is understood as causal} has been raised from the dead, is going to die no more, death no longer being master over him.}%
\verse{For that death he died, he died to sin once and never again, but that life he lives, he lives to God.}%
\verse{So also you, consider yourselves to be dead to sin, but alive to God in Christ Jesus.}%
\verse{Therefore do not let sin reign in your mortal body, so that you obey its desires,}%
\verse{and do not present your members to sin as instruments of unrighteousness, but present yourselves to God as those who are alive from the dead, and your members to God as instruments of righteousness.}%
\verse{For sin will not be master over you, because you are not under law, but under grace.}%
\verseWithHeading{Set Free from Sin}{What then? Shall we sin because we are not under law but under grace? May it never be!}%
\verse{Do you not know that to whomever you present yourselves as slaves for obedience, you are slaves to whomever you obey, whether sin, leading to death, or obedience, leading to righteousness?}%
\verse{But thanks be to God that you were slaves of sin, but you have obeyed from the heart the pattern of teaching to which you were entrusted,}%
\verse{and having been set free from sin, you became enslaved to righteousness.}%
\verse{(I am speaking in human terms because of the weakness of your flesh.) For just as you presented your members as slaves to immorality and lawlessness, leading to lawlessness, so now present your members as slaves to righteousness, leading to sanctification.}%
\verse{For when you were slaves of sin, you were free with respect to righteousness.}%
\verse{Therefore what sort of fruit did you have then, about which you are now ashamed? For the end of those things is death.}%
\verse{But now, having been set free from sin and having been enslaved to God, you have your fruit leading to sanctification, and its end is eternal life.}%
\verse{For the compensation due sin is death, but the gift of God is eternal life in Christ Jesus our Lord.}%
\end{biblechapter}%
\begin{biblechapter}% Romans 7
\verseWithHeading{Released from the Law through Death}{Or do you not know, brothers (for I am speaking to those who know the law), that the law is master of a person for as long a time as he lives?}%
\verse{For the married woman is bound by law to her husband while he lives, but if her husband dies, she is released from the law of the husband.}%
\verse{Therefore as a result, if she belongs to another man while\lebnote{“\textit{while}” is supplied as a component of the participle (“is living”) which is understood as temporal} her husband is living, she will be called an adulteress. But if her husband dies, she is free from the law, so that she is not an adulteress if she\lebnote{“\textit{if}” is supplied as a component of the participle (“belongs”) which is understood as conditional} belongs to another man.}%
\verse{So then, my brothers, you also were brought to death with respect to the law through the body of Christ, so that you may belong to another, to the one who was raised from the dead, in order that we may bear fruit for God.}%
\verse{For when we were in the flesh, sinful desires were working through the law in our members, to bear fruit for death.}%
\verse{But now we have been released from the law, because we\lebnote{“\textit{because}” is supplied as a component of the participle (“have died”) which is understood as causal} have died to that by which we were bound, so that we may serve in newness of the Spirit and not in oldness of the letter of the law.}%
\verseWithHeading{Knowledge of Sin Comes through the Law}{What then shall we say? Is the law sin? May it never be! But I would not have known sin except through the law, for I would not have known covetousness if the law had not said, “Do not covet.”\lebnote{from Exod 20:17; Deut 5:21}}%
\verse{But sin, seizing an opportunity through the commandment, produced in me all kinds of covetousness. For apart from the law, sin is dead.}%
\verse{And I was alive once, apart from the law, but when\lebnote{“\textit{when}” is supplied as a component of the participle (“came”) which is understood as temporal} the commandment came, sin sprang to life}%
\verse{and I died, and this commandment which was to lead to life was found with respect to me to lead to death.}%
\verse{For sin, seizing the opportunity through the commandment, deceived me and through it killed me.}%
\verse{So then, the law is holy, and the commandment is holy and righteous and good.}%
\verseWithHeading{Internal Conflict with Sin}{Therefore, did that which is good become death to me? May it never be! Rather it was sin, in order that it might be recognized as sin, producing death through what is good for me, in order that sin might become sinful to an extraordinary degree through the commandment.}%
\verse{For we know that the law is spiritual, but I am fleshly, sold into slavery to sin\lebnote{“sold under sin”}.}%
\verse{For what I am doing I do not understand, because what I want to do, this I do not practice, but what I hate, this I do.}%
\verse{But if what I do not want to do, this I do, I agree with the law that it is good.}%
\verse{But now I am no longer the one doing it, but sin that lives in me.}%
\verse{For I know that good does not live in me, that is, in my flesh. For the willing is present in me, but the doing of the good is not.}%
\verse{For the good that I want to do, I do not do, but the evil that I do not want to do, this I do.}%
\verse{But if what I do not want to do, this I am doing, I am no longer the one doing it, but sin that lives in me.}%
\verse{Consequently, I find the principle with me, the one who wants to do good, that evil is present with me.\lebnote{Or “in me”}}%
\verse{For I joyfully agree with the law of God in my inner person,}%
\verse{but I observe another law in my members, at war with the law of my mind and making me captive to the law of sin that exists in my members.}%
\verse{Wretched man that I am! Who will rescue me from this body of death?}%
\verse{Thanks be\lebnote{Some manuscripts have “But thanks \textit{be}”} to God through Jesus Christ our Lord! So then, I myself with my mind am enslaved to the law of God, but with my flesh I am enslaved to the law of sin.}%
\end{biblechapter}%
\begin{biblechapter}% Romans 8
\verseWithHeading{Set Free from the Law of Sin and Death}{Consequently, there is now no condemnation for those who are in Christ Jesus.}%
\verse{For the law of the Spirit of life in Christ Jesus has set you free from the law of sin and death.}%
\verse{For what was impossible for the law, in that it was weak through the flesh, God did. By\lebnote{“\textit{by}” is supplied as a component of the participle (“sending”) which is understood as means} sending his own Son in the likeness of sinful flesh and concerning sin, he condemned sin in the flesh,}%
\verse{in order that the requirement of the law would be fulfilled in us, who do not live according to the flesh but according to the Spirit.}%
\verse{For those who are living according to the flesh are intent on the things of the flesh, but those who are living according to the Spirit are intent on the things of the Spirit.}%
\verse{For the mindset of the flesh is death, but the mindset of the Spirit is life and peace,}%
\verse{because the mindset of the flesh is enmity toward God, for it is not subjected to the law of God, for it is not able to do so,}%
\verse{and those who are in the flesh are not able to please God.}%
\verse{But you are not in the flesh but in the Spirit, if indeed the Spirit of God lives in you. But if anyone does not have the Spirit of Christ, this person does not belong to him\lebnote{“is not of him”}.}%
\verse{But if Christ is in you, the body is dead because of sin, but the Spirit is life because of righteousness.}%
\verse{And if the Spirit of the one who raised Jesus from the dead lives in you, the one who raised Christ Jesus\lebnote{Some manuscripts omit “Jesus”} from the dead will also make alive your mortal bodies through his Spirit who lives in you.}%
\verse{So then, brothers, we are obligated not to the flesh, to live according to the flesh.}%
\verse{For if you live according to the flesh, you are going to die, but if by the Spirit you put to death the deeds of the body, you will live.}%
\verse{For all those who are led by the Spirit of God, these are sons of God.}%
\verse{For you have not received a spirit of slavery leading to fear again, but you have received the Spirit of adoption, by whom we cry out, “Abba!\lebnote{“Abba” means “father” in Aramaic} Father!”}%
\verse{The Spirit himself confirms to our spirit that we are children of God,}%
\verse{and if children, also heirs — heirs of God and fellow heirs with Christ, if indeed we suffer together with him so that we may also be glorified together with him.}%
\verseWithHeading{The Glory that is to be Revealed}{For I consider that the sufferings of the present time are not worthy to be compared with the glory that is about to be revealed to us.}%
\verse{For the eagerly expecting creation awaits eagerly the revelation of the sons of God.}%
\verse{For the creation has been subjected to futility, not willingly, but because of the one who subjected it, in hope}%
\verse{that the creation itself also will be set free from its servility to decay, into the glorious freedom of the children of God.}%
\verse{For we know that the whole creation groans together and suffers agony together until now.}%
\verse{Not only this, but we ourselves also, having the first fruits of the Spirit, even we ourselves groan within ourselves while we\lebnote{“\textit{while}” is supplied as a component of the participle (“await eagerly”) which is understood as temporal} await eagerly our adoption, the redemption of our body.}%
\verse{For in hope we were saved, but hope that is seen is not hope, for who hopes for what he sees?}%
\verse{But if we hope for what we do not see, we await it eagerly with patient endurance.}%
\verse{And likewise also, the Spirit helps us in our weakness, for we do not know how to pray as one ought, but the Spirit himself intercedes for us with unexpressed groanings.}%
\verse{And the one who searches our hearts knows what the mindset of the Spirit is, because he intercedes on behalf of the saints according to the will of God.}%
\verse{And we know that all things work together for good for those who love God, for those who are called according to his purpose,}%
\verse{because those whom he foreknew, he also predestined to be conformed to the image of his Son, so that he should be the firstborn among many brothers.}%
\verse{And those whom he predestined, these he also called, and those whom he called, these he also justified, and those whom he justified, these he also glorified.}%
\verseWithHeading{Victory in Christ}{What then shall we say about these things? If God is for us, who can be against us?}%
\verse{Indeed, he who did not spare his own Son, but gave him up for us all, how will he not also, together with him, freely give us all things?}%
\verse{Who will bring charges against God’s elect? God is the one who justifies.}%
\verse{Who is the one who condemns? Christ\lebnote{Some manuscripts have “Christ Jesus”} is the one who died, and more than that, who was raised, who is also at the right hand of God, who also intercedes for us.}%
\verse{Who will separate us from the love of Christ? Will affliction or distress or persecution or hunger or lack of sufficient clothing or danger or the sword?}%
\verse{Just as it is written, “On account of you we are being put to death the whole day long; we are considered as sheep for slaughter.”\lebnote{from Ps 44:22}}%
\verse{No, but in all these things we prevail completely through the one who loved us.}%
\verse{For I am convinced that neither death, nor life, nor angels, nor rulers, nor things present, nor things to come, nor powers,}%
\verse{nor height, nor depth, nor any other created thing, will be able to separate us from the love of God that is in Christ Jesus our Lord.}%
\end{biblechapter}%
\begin{biblechapter}% Romans 9
\verseWithHeading{Israel’s Rejection}{I am telling the truth in Christ — I am not lying; my conscience bears witness to me in the Holy Spirit —}%
\verse{that my grief is great and there is constant distress in my heart.}%
\verse{For I could wish myself to be accursed from Christ for the sake of my brothers, my fellow countrymen according to the flesh,}%
\verse{who are Israelites, to whom belong the adoption, and the glory, and the covenants, and the giving of the law, and the temple service, and the promises,}%
\verse{to whom belong the patriarchs, and from whom is the Christ according to human descent, who is God over all, blessed forever\lebnote{“for eternity”}! Amen.}%
\verse{But it is not as if the word of God had failed. For not all those who are descended from Israel are truly Israel,}%
\verse{nor are they all children because they are descendants of Abraham, but “In Isaac will your descendants be named.”\lebnote{from Gen 21:12}}%
\verse{That is, it is not the children by human descent\lebnote{“of the flesh”} who are children of God, but the children of the promise are counted as descendants.}%
\verse{For the statement of the promise is this: “At this time I will return and Sarah will have\lebnote{“there will be to Sarah”} a son.”\lebnote{from Gen 18:10, 14}}%
\verse{And not only this, but also when\lebnote{“\textit{when}” is supplied as a component of the participle (“having conception” = “conceived”) which is understood as temporal} Rebecca conceived children by one man,\lebnote{Or perhaps “by one \textit{act of sexual intercourse}”} Isaac our father —}%
\verse{for although they\lebnote{“\textit{although}” is supplied as a component of the participle (“been born”) which is understood as concessive} had not yet been born, or done anything good or evil, in order that the purpose of God according to election might remain,}%
\verse{not by works but by the one who calls — it was said to her, “The older will serve the younger,”\lebnote{from Gen 25:23}}%
\verse{just as it is written, “Jacob I loved, but Esau I hated.”\lebnote{from Mal 1:2–3}}%
\verseWithHeading{God’s Sovereign Choice to Show Mercy}{What then shall we say? There is no injustice with God, is there?\lebnote{*The negative construction in Greek anticipates a negative answer here} May it never be!}%
\verse{For to Moses he says, “I will have mercy on whomever I have mercy, and I will have compassion on whomever I have compassion.”\lebnote{from Exod 33:19}}%
\verse{Consequently therefore, it does not depend on the\lebnote{“not of the”} one who wills or on the one who runs, but on God who shows mercy.}%
\verse{For the scripture says to Pharaoh, “For this very reason I have raised you up, so that I may demonstrate my power in you, and so that my name might be proclaimed in all the earth.”\lebnote{from Exod 9:16}}%
\verse{Consequently therefore, he has mercy on whomever he wishes, and he hardens whomever he wishes.}%
\verse{Therefore you will say to me, “Why then does he still find fault? For who has resisted\lebnote{Or “who resists”} his will?}%
\verse{On the contrary, O man, who are you who answers back to God? Will what is molded say to the one who molded it, “Why did you make me like this”?\lebnote{from Isa 29:16; 45:9}}%
\verse{Or does the potter not have authority over the clay, to make from the same lump a vessel that is for honorable use\lebnote{“honor”} and one that is for ordinary use\lebnote{“dishonor”}?}%
\verse{And what if God, wanting to demonstrate his wrath and to make known his power, endured with much patience vessels of wrath prepared for destruction?}%
\verse{And he did so\lebnote{The words “he did so” are not in the Greek text, but are an understood repetition from the previous clause} in order that he could make known the riches of his glory upon vessels of mercy that he prepared beforehand for glory,}%
\verse{us whom he also called, not only from the Jews but also from the Gentiles?}%
\verse{As he also says in Hosea, “I will call those who were not my people, ‘My people,’ and those who were not loved, ‘Loved.’\lebnote{from Hos 2:23}}%
\verse{And it will be in the place where it was said to them, ‘You are not my people,’ there they will be called ‘sons of the living God.’”\lebnote{from Hos 1:10}}%
\verse{And Isaiah cries out concerning Israel, “Even if the number of the sons of Israel is like the sand of the sea, the remnant will be saved,}%
\verse{for the Lord will execute his sentence thoroughly and decisively\lebnote{“for the Lord will act, closing the account and cutting short”} upon the earth.”\lebnote{A paraphrased quotation from Isa 10:22–23}}%
\verse{And just as Isaiah foretold, “If the Lord of hosts had not left us descendants, we would have become like Sodom and would have resembled Gomorrah.”\lebnote{from Isa 1:9}}%
\verse{What then shall we say? That the Gentiles, who did not pursue righteousness, attained righteousness — even the righteousness that is by faith.}%
\verse{But Israel, pursuing the law of righteousness, did not attain to the law.}%
\verse{Why that? Because they did not pursue it by faith, but as if by works. They stumbled over the stone that causes people to stumble\lebnote{“stone of stumbling”},}%
\verse{just as it is written, “Behold, I am laying in Zion a stone that causes people to stumble\lebnote{“a stone of stumbling”}, and a rock that causes them to fall\lebnote{“a rock of offense”}, and the one who believes in him will not be put to shame.”\lebnote{from Isa 28:16; 8:14}}%
\end{biblechapter}%
\begin{biblechapter}% Romans 10
\verseWithHeading{The Righteousness of God through Faith in Christ}{Brothers, the desire of my heart and my prayer to God on behalf of them is for their salvation.}%
\verse{For I testify about them that they have a zeal for God, but not according to knowledge.}%
\verse{For ignoring the righteousness of God, and seeking to establish their own,\lebnote{Some manuscripts have “their own righteousness”} they did not subject themselves to the righteousness of God.}%
\verse{For Christ is the end of the law for righteousness to everyone who believes.}%
\verse{For Moses writes about the righteousness that is from the law: “The person who does this\lebnote{Some manuscripts explicitly state “these \textit{things}”} will live by it.”\lebnote{from Lev 18:5}\lebnote{Some manuscripts have “them”}}%
\verse{But the righteousness from faith speaks like this: “Do not say in your heart,\lebnote{from Deut 9:4} ‘Who will ascend into heaven?’”\lebnote{from Deut 30:12} (that is, to bring Christ down),}%
\verse{or “Who will descend into the abyss?”\lebnote{from Deut 30:13} (that is, to bring Christ up from the dead).}%
\verse{But what does it say? “The word is near to you, in your mouth and in your heart”\lebnote{from Deut 30:14} (that is, the word of faith that we proclaim),}%
\verse{that\lebnote{Or “because”} if you confess with your mouth “Jesus is Lord” and believe in your heart that God raised him from the dead, you will be saved.}%
\verse{For with the heart one believes, resulting in righteousness, and with the mouth one confesses, resulting in salvation.}%
\verse{For the scripture says, “Everyone who believes in him will not be put to shame.”\lebnote{from Isa 28:16}}%
\verse{For there is no distinction between Jew and Greek, for the same Lord is Lord of all, who is rich to all who call upon him.}%
\verse{For “everyone who calls upon the name of the Lord will be saved.”\lebnote{from Joel 2:32}}%
\verse{How then will they call upon him in whom they have not believed? And how will they believe in him about whom they have not heard? And how will they hear about him without one who preaches to them?}%
\verse{And how will they preach, unless they are sent? Just as it is written, “How timely are the feet of those who bring good news of good things.”\lebnote{from Isa 52:7; Nah 1:15}}%
\verse{But not all have obeyed the good news, for Isaiah says, “Lord, who has believed our report?”\lebnote{from Isa 53:1}}%
\verse{Consequently, faith comes by hearing, and hearing through the word about Christ.}%
\verse{But I say, they have not heard, have they?\lnDKP{} On the contrary, “Their voice has gone out to all the earth, and their words to the ends of the inhabited world.”\lebnote{from Ps 19:4}}%
\verse{But I say, Israel did not know, did they?\lnDKP{} First, Moses says, “I will provoke you to jealousy by those who are not a nation; by a senseless nation I will provoke you to anger.”\lebnote{from Deut 32:21}}%
\verse{And Isaiah is very bold and says, “I was found by those who did not seek me; I became known to those who did not ask for me.”\lebnote{from Isa 65:1}}%
\verse{But about Israel he says, “The whole day long I held out my hands to a disobedient and resistant people.”\lebnote{from Isa 65:2}}%
\end{biblechapter}%
\begin{biblechapter}% Romans 11
\verseWithHeading{A Remnant of Israel Remains}{Therefore I say, God has not rejected his people, has he?\lnDKQ{} May it never be! For I also am an Israelite, from the descendants of Abraham, of the tribe of Benjamin.}%
\verse{God has not rejected his people, whom he foreknew! Or do you not know, in the passage about\lebnote{The words “the passage about” are not in the Greek text, but are supplied for clarity} Elijah, what the scripture says — how he appeals to God against Israel?}%
\verse{“Lord, they have killed your prophets, they have torn down your altars, and I alone am left, and they are seeking my life!”\lebnote{from 1 Kgs 19:10, 14}}%
\verse{But what does the divine response say to him? “I have left for myself seven thousand people\lebnote{Or perhaps “males,” referring to men only} who have not bent the knee to Baal.”\lebnote{from 1 Kgs 19:18}}%
\verse{So in this way also at the present time, there is a remnant selected by grace\lebnote{“according to selection of grace”}.}%
\verse{But if by grace, it is no longer by works, for otherwise grace would no longer be grace.}%
\verse{What then? What Israel was searching for, this it did not obtain. But the elect obtained it, and the rest were hardened,}%
\verse{just as it is written, “God gave them a spirit of stupor, eyes that do not see and ears that do not hear, until this very day.”\lebnote{from Deut 29:4; Isa 29:10}}%
\verse{And David says, “Let their table become a snare and a trap, and a cause for stumbling and a retribution to them;}%
\verse{let their eyes be darkened so that they do not see, and cause their backs to bend continually\lebnote{“throughout everything”}.”\lebnote{from Ps 69:22–23}}%
\verse{I say then, they did not stumble so that they fell, did they?\lnDKQ{} May it never be! But by their trespass, salvation has come to the Gentiles, in order to provoke them to jealousy.\lebnote{*The words “to jealousy” are not in the Greek text, but are supplied for clarity}}%
\verse{And if their trespass means riches for the world and their loss means riches for the Gentiles, how much more will their fullness mean?}%
\verseWithHeading{Gentile Branches Grafted in}{Now I am speaking to you Gentiles. Therefore, inasmuch as I am apostle to the Gentiles, I promote my ministry,}%
\verse{if somehow I may provoke my people to jealousy and save some of them.}%
\verse{For if their rejection means the reconciliation of the world, what will their acceptance mean except life from the dead?}%
\verse{Now if the first fruits are holy, so also is the whole batch of dough, and if the root is holy, so also are the branches.}%
\verse{Now if some of the branches were broken off, and you, although you\lebnote{“\textit{although}” is supplied as a component of the participle (“were”) which is understood as concessive} were a wild olive tree, were grafted in among them and became a sharer of the root of the olive tree’s richness,}%
\verse{do not boast against the branches. But if you boast against them, you do not support the root, but the root supports you.}%
\verse{Then you will say, “Branches were broken off in order that I could be grafted in.”}%
\verse{Well said! They were broken off because of unbelief, but you stand firm because of faith. Do not think arrogant thoughts, but be afraid.}%
\verse{For if God did not spare the natural\lebnote{“according to nature”} branches, neither will he spare you.\lebnote{Some manuscripts have “perhaps he will not spare you either”}}%
\verse{See, then, the kindness and severity of God: severity upon those who have fallen, but upon you the kindness of God — if you continue in his kindness, for otherwise you also will be cut off.}%
\verse{And those also, if they do not persist in unbelief, will be grafted in, because God is able to graft them in again.}%
\verse{For if you were cut off from what is by nature a wild olive tree, and contrary to nature were grafted into a cultivated olive tree, how much more will these who are natural branches\lebnote{“by nature”} be grafted into their own olive tree?}%
\verseWithHeading{All Israel to be Saved}{For I do not want you to be ignorant, brothers, of this mystery, so that you will not be wise in your own sight,\lebnote{“in yourselves”} that a partial hardening has happened to Israel, until the full number of the Gentiles has come in,}%
\verse{and so all Israel will be saved, just as it is written, “The deliverer will come out of Zion; he will turn away ungodliness from Jacob.}%
\verse{And this is the covenant from me with them\lebnote{from Isa 59:20–21} when I take away their sins.”\lebnote{from Isa 27:9; Jer 31:33–34}}%
\verse{With respect to the gospel, they are enemies for your sake, but with respect to election, they are dearly loved for the sake of the fathers.}%
\verse{For the gifts and the calling of God are irrevocable.}%
\verse{For just as you formerly were disobedient to God, but now have been shown mercy because of the disobedience of these,}%
\verse{so also these have now been disobedient for your mercy, in order that they also may now be shown mercy.}%
\verse{For God confined them all in disobedience, in order that he could have mercy on them all.}%
\verse{Oh, the depth of the riches and the wisdom and the knowledge of God! How unsearchable are his judgments and how incomprehensible are his ways!}%
\verse{“For who has known the mind of the Lord, or who has been his counselor?\lebnote{from Isa 40:13}}%
\verse{Or who has given in advance to him, and it will be paid back to him?”\lebnote{from Job 41:11}}%
\verse{For from him and through him and to him are all things. To him be glory for eternity! Amen.}%
\end{biblechapter}%
\begin{biblechapter}% Romans 12
\verseWithHeading{A Life Dedicated to God}{Therefore I exhort you, brothers, through the mercies of God, to present your bodies as a living sacrifice, holy and pleasing to God, which is your reasonable service.}%
\verse{And do not be conformed to this age, but be transformed by the renewal of your mind, so that you may approve what is the good and well-pleasing and perfect will of God.}%
\verseWithHeading{A Variety of Gifts in the Body of Christ}{For by the grace given to me I say to everyone who is among you not to think more highly of yourself than what one ought to think, but to think sensibly\lebnote{“so as to be sensible”}, as God has apportioned a measure of faith to each one.}%
\verse{For just as in one body we have many members, but all the members do not have the same function,}%
\verse{in the same way we who are many are one body in Christ, and individually\lebnote{“with respect to one”} members of one another,}%
\verse{but having different gifts according to the grace given to us: if it is prophecy, according to the proportion of his faith;}%
\verse{if it is service, by service; if it is one who teaches, by teaching;}%
\verse{if it is one who exhorts, by exhortation; one who gives, with sincerity; one who leads, with diligence; one who shows mercy, with cheerfulness.}%
\verseWithHeading{Living in Love}{Love must be without hypocrisy. Abhor what is evil; be attached to what is good,}%
\verse{being devoted to one another in brotherly love, esteeming one another more highly in honor,}%
\verse{not lagging in diligence, being enthusiastic in spirit, serving the Lord,}%
\verse{rejoicing in hope, enduring in affliction, being devoted to prayer,}%
\verse{contributing to the needs of the saints, pursuing hospitality.}%
\verse{Bless those who persecute,\lebnote{Some manuscripts have “who persecute you”} bless and do not curse them.}%
\verse{Rejoice with those who rejoice; weep with those who weep.}%
\verse{Think the same thing toward one another; do not think arrogantly\lebnote{“think not the arrogant”}, but associate with the lowly. Do not be wise in your own sight\lebnote{“in the sight of yourselves”}.}%
\verse{Pay back no one evil for evil. Take thought for what is good in the sight of all people.}%
\verse{If it is possible on your part, be at peace with all people.}%
\verse{Do not take revenge yourselves, dear friends, but give place to God’s wrath, for it is written, “Vengeance is mine, I will repay,”\lebnote{from Deut 32:35} says the Lord.}%
\verse{But “if your enemy is hungry, feed him; if he is thirsty, give him something to drink; for by\lebnote{“\textit{by}” is supplied as a component of the participle (“doing”) which is understood as means} doing this, you will heap up coals of fire upon his head.”\lebnote{from Prov 25:21–22}}%
\verse{Do not be overcome by evil, but overcome evil with good.}%
\end{biblechapter}%
\begin{biblechapter}% Romans 13
\verseWithHeading{Obedience to the Governing Authorities}{Let every person be subject to the governing authorities, for there is no authority except by God, and those that exist are put in place by God.}%
\verse{So then, the one who resists authority resists the ordinance which is from God, and those who resist will receive condemnation on themselves.}%
\verse{For rulers are not a cause of terror for a good deed, but for bad conduct. So do you want not to be afraid of authority? Do what is good, and you will have praise from it,}%
\verse{for it is God’s servant to you for what is good. But if you do what is bad, be afraid, because it does not bear the sword to no purpose. For it is God’s servant, the one who avenges for punishment on the one who does what is bad.}%
\verse{Therefore it is necessary to be in subjection, not only because of wrath but also because of conscience.}%
\verse{For because of this you also pay taxes, for the authorities\lebnote{“they”} are servants of God, busily engaged in this very thing.}%
\verse{Pay to everyone what is owed: pay taxes to whom taxes are due; pay customs duties to whom customs duties are due; pay respect to whom respect is due; pay honor to whom honor is due.\lebnote{Due to the very compressed style in this verse, many words must be supplied to make sense in English}}%
\verseWithHeading{Love Fulfills the Law}{Owe nothing to anyone, except to love one another, for the one who loves someone else has fulfilled the law.}%
\verse{For the commandments, “You shall not commit adultery, you shall not commit murder, you shall not steal, you shall not covet,”\lebnote{from Exod 20:13–15, 17; Deut 5:17–19, 21} and if there is any other commandment, are summed up in this statement: “You shall love your neighbor as yourself.”\lebnote{from Lev 19:18}}%
\verse{Love does not commit evil against a neighbor. Therefore love is the fulfillment of the law.}%
\verse{And do this because you\lebnote{“\textit{because}” is supplied as a component of the participle (“know”) which is understood as causal} know the time, that it is already the hour for you to wake up from sleep. For our salvation is nearer now than when we believed.}%
\verse{The night is far gone, and the day has drawn near. Therefore let us throw off\lebnote{Some manuscripts have “let us lay aside”} the deeds of darkness and put on the weapons of light.}%
\verse{Let us live decently, as in the day, not in carousing and drunkenness, not in sexual immorality and licentiousness, not in strife and jealousy.}%
\verse{But put on the Lord Jesus Christ and do not make provision for the desires of the flesh.}%
\end{biblechapter}%
\begin{biblechapter}% Romans 14
\verseWithHeading{Do Not Pass Judgment on One Another}{Now receive the one who is weak in faith, but not for quarrels about opinions.}%
\verse{One believes he may eat all things, but the one who is weak eats only vegetables.}%
\verse{The one who eats must not despise the one who does not eat, and the one who does not eat must not judge the one who eats, because God has accepted him.}%
\verse{Who are you, who passes judgment on the domestic slave belonging to someone else? To his own master he stands or falls, and he will stand, for the Lord is able to make him stand.}%
\verse{One person\lebnote{Some manuscripts have “For one person”} prefers one day over another day, and another person regards every day alike. Each one must be fully convinced in his own mind.}%
\verse{The one who is intent on the day is intent on it for the Lord, and the one who eats eats for the Lord, because he is thankful to God, and the one who does not eat does not eat for the Lord, and he is thankful to God.}%
\verse{For none of us lives for himself and none dies for himself.}%
\verse{For if we live, we live for the Lord, and if we die, we die for the Lord. Therefore whether we live or whether we die, we are the Lord’s.}%
\verse{For Christ died and became alive again for this reason, in order that he might be Lord of both the dead and the living.}%
\verse{But why do you judge your brother? Or also, why do you despise your brother? For we will all stand before the judgment seat of God.}%
\verse{For it is written, “As I live, says the Lord, every knee will bow to me, and every tongue will praise God.”\lebnote{from Isa 45:23}}%
\verse{So\lebnote{Some manuscripts have “So then,”} each one of us will give an account concerning himself.\lebnote{Some manuscripts have “an account concerning himself to God”}}%
\verse{Therefore, let us no longer pass judgment on one another, but rather decide this: not to place a cause for stumbling or a temptation before a brother.}%
\verse{I know and am convinced in the Lord Jesus that nothing is unclean of itself, except to the one who considers something to be unclean; to that person it is unclean.}%
\verse{For if because of food, your brother is grieved, you are no longer living according to love. Do not destroy by your food that person for whom Christ died.}%
\verse{Therefore do not let your good be slandered.}%
\verse{For the kingdom of God is not eating and drinking, but righteousness and peace and joy in the Holy Spirit.}%
\verse{For the one who serves Christ in this way is well-pleasing to God and approved by people.}%
\verse{So then, let us pursue what promotes peace\lebnote{“the things of peace”} and what edifies one another\lebnote{“the things of edification toward one another”}.}%
\verse{Do not destroy the work of God on account of food. All things are clean, but it is wrong for the person who eats and stumbles in the process\lebnote{“who eats with stumbling”}.}%
\verse{It is good not to eat meat or to drink wine or to do anything by which your brother stumbles or is offended or is weakened.\lebnote{Some manuscripts omit “or is offended or is weakened”}}%
\verse{The faith that you have, have with respect to yourself before God. Blessed is the one who does not pass judgment on himself by what he approves.}%
\verse{But the one who doubts is condemned if he eats, because he does not do so from faith, and everything that is not from faith is sin.}%
\end{biblechapter}%
\begin{biblechapter}% Romans 15
\verseWithHeading{Accept One Another according to Christ’s Example}{But we who are strong ought to bear the weaknesses of the weak, and not to please ourselves.}%
\verse{Let each one of us please his neighbor for his good, for the purpose of edification.}%
\verse{For even Christ did not please himself, but just as it is written, “The insults of those who insult you have fallen on me.”\lebnote{from Ps 69:9}}%
\verse{For whatever was written beforehand was written for our instruction, in order that through patient endurance and through the encouragement of the scriptures we may have hope.}%
\verse{Now may the God of patient endurance and of encouragement grant you to be in agreement\lebnote{“to think the same”} with one another, in accordance with Christ Jesus,}%
\verse{so that with one mind you may glorify with one mouth the God and Father of our Lord Jesus Christ.}%
\verse{Therefore accept one another, just as Christ also has accepted you, to the glory of God.}%
\verse{For I say, Christ has become a servant of the circumcision on behalf of the truth of God, in order to confirm the promises to the fathers,}%
\verse{and that the Gentiles may glorify God for his mercy, just as it is written, “Because of this, I will praise you among the Gentiles, and I will sing praise to your name.”\lebnote{from Ps 18:49}}%
\verse{And again it says, “Rejoice, Gentiles, with his people.”\lebnote{from Deut 32:43}}%
\verse{And again, “Praise the Lord, all the Gentiles, and let all the peoples praise him.”\lebnote{from Ps 117:1}}%
\verse{And again Isaiah says, “The root of Jesse will come, even the one who rises to rule over the Gentiles; in him the Gentiles will put their hope.”\lebnote{from Isa 11:10}}%
\verse{Now may the God of hope fill you with all joy and peace in believing, so that you may abound in hope by the power of the Holy Spirit.}%
\verseWithHeading{Paul’s Ministry to the Gentiles}{Now I myself also am convinced about you, my brothers, that you yourselves also are full of goodness, filled with all knowledge, able also to instruct one another.}%
\verse{But I have written to you more boldly on some points, so as to remind you again because of the grace that has been given to me by God,}%
\verse{with the result that I am a servant of Christ Jesus to the Gentiles, serving the gospel of God as a priest, in order that the offering of the Gentiles may become acceptable, sanctified by the Holy Spirit.}%
\verse{Therefore I have a reason for boasting in Christ Jesus regarding the things concerning God.}%
\verse{For I will not dare to speak about anything except that which Christ has accomplished through me, resulting in the obedience of the Gentiles by word and deed,}%
\verse{by the power of signs and wonders, by the power of the Spirit,\lebnote{Some manuscripts have “of the Spirit of God”} so that from Jerusalem and traveling around as far as Illyricum I have fully proclaimed the gospel of Christ.}%
\verse{And so, having as my ambition to proclaim the gospel where Christ has not been named, in order that I will not build on the foundation belonging to someone else,}%
\verse{but just as it is written, “Those to whom it was not announced concerning him will see, and those who have not heard will understand.”\lebnote{from Isa 52:15}}%
\verseWithHeading{Paul’s Travel Plans}{For this reason also I was hindered many times from coming to you,}%
\verse{and now, no longer having a place in these regions, but having a desire for many years to come to you}%
\verse{whenever I travel to Spain. For I hope while I\lebnote{“\textit{while}” is supplied as a component of the participle (“passing through”) which is understood as temporal} am passing through to see you and to be sent on my way by you, whenever I have first enjoyed your company for a while.}%
\verse{But now I am traveling to Jerusalem, serving the saints.}%
\verse{For Macedonia and Achaia were pleased to make some contribution for the poor among the saints in Jerusalem.}%
\verse{For they were pleased to do so, and they are obligated to them. For if the Gentiles have shared in their spiritual things, they ought also to serve them in material things.}%
\verse{Therefore, after I\lebnote{“\textit{after}” is supplied as a component of the participle (“have accomplished”) which is understood as temporal} have accomplished this and sealed this fruit for delivery to them, I will depart by way of you for Spain,}%
\verse{and I know that when I\lebnote{“\textit{when}” is supplied as a component of the participle (“come”) which is understood as temporal} come to you, I will come in the fullness of the blessing of Christ.}%
\verse{Now I exhort you, brothers, through our Lord Jesus Christ and through the love of the Spirit, to contend along with me in your prayers on my behalf to God,}%
\verse{that I may be rescued from those who are disobedient in Judea, and my ministry in Jerusalem may be acceptable to the saints,}%
\verse{so that, coming to you with joy by the will of God, I may rest with you.}%
\verse{Now may the God of peace be with all of you. Amen.}%
\end{biblechapter}%
\begin{biblechapter}% Romans 16
\verseWithHeading{Many Personal Greetings}{Now I commend to you Phoebe our sister, who is also a servant\lebnote{Or “a deaconess”; some interpreters understand this term to refer to a specific office (deacon/deaconess) which Phoebe held in the local church at Cenchrea} of the church in Cenchrea,}%
\verse{in order that you may welcome her in the Lord in a manner worthy of the saints, and help her in whatever task she may have need from you, for she herself also has been a helper of many, even me myself.}%
\verse{Greet Prisca and Aquila, my fellow workers in Christ Jesus,}%
\verse{who risked their own necks for my life, for which not only I am thankful, but also all the churches of the Gentiles;}%
\verse{also greet\lebnote{The verb is supplied as an understood repetition from v. 3} the church in their house. Greet Epenetus my dear friend, who is the first convert\lebnote{“the first fruits”} of Asia for Christ.}%
\verse{Greet Mary, who has worked hard\lnDKR{} for you.}%
\verse{Greet Andronicus and Junia,\lebnote{Or “Junias,” the masculine form of the same name} my compatriots\lnDKS{} and my fellow prisoners, who are well known to\lebnote{Or “are outstanding among”} the apostles, who were also in Christ before me.}%
\verse{Greet Ampliatus, my dear friend in the Lord.}%
\verse{Greet Urbanus, our fellow worker in Christ, and my dear friend Stachys.}%
\verse{Greet Apelles, who is approved in Christ. Greet those of the household of Aristobulus.}%
\verse{Greet Herodion my compatriot.\lebnote{Or “relative”} Greet those of the household of Narcissus who are in the Lord.}%
\verse{Greet Tryphena and Tryphosa, the laborers in the Lord. Greet Persis, the dear friend who has worked hard\lnDKR{} in the Lord.}%
\verse{Greet Rufus, the chosen one in the Lord, and his mother and mine.}%
\verse{Greet Asyncritus, Phlegon, Hermes, Patrobas, Hermas, and the brothers with them.}%
\verse{Greet Philologus and Julia, Nereus and his sister, and Olympas, and all the saints who are with them.}%
\verse{Greet one another with a holy kiss. All the churches of Christ greet you.}%
\verseWithHeading{Concluding Exhortations}{Now I exhort you, brothers, to look out for those who cause dissensions and temptations contrary to the teaching which you learned, and stay away from them.}%
\verse{For such people do not serve our Lord Christ, but their own stomach, and by smooth speech and flattery they deceive the hearts of the unsuspecting.}%
\verse{For the report of your obedience has reached to all; therefore I am rejoicing over you, and I want you to be wise toward what is good, but innocent toward what is evil.}%
\verse{And in a short time the God of peace will crush Satan under your feet. The grace of our Lord Jesus Christ\lebnote{Some manuscripts omit “Christ”} be with you.}%
\verseWithHeading{Greetings from Paul’s Associates}{Timothy, my fellow worker, greets you, and Lucius and Jason and Sosipater, my compatriots.\lnDKS{}}%
\verse{I, Tertius, the one who wrote this letter, greet you in the Lord.}%
\verse{Gaius, my host and the host of the whole church, greets you. Erastus the city treasurer greets you, and Quartus the brother.}%
\verseWithHeading{Benediction}{The grace of our Lord Jesus Christ be with all of you. Amen.\lebnote{Some manuscripts include vv. 25–27, “25 Now to the one who is able to strengthen you according to my gospel and the preaching of Jesus Christ, according to the revelation of the mystery that had been kept secret for eternal ages, 26 but now has been revealed, and through the prophetic scriptures has been made known according to the command of the eternal God, resulting in obedience of faith to all the Gentiles, 27 to the only wise God, through Jesus Christ, to whom \textit{be} the glory for eternity. Amen.”}}%
\end{biblechapter}%
\flushcolsend
\input{leb/content/new-testament/1Cor.tex}\flushcolsend
\input{leb/content/new-testament/2Cor.tex}\flushcolsend
\input{leb/content/new-testament/Gal.tex}\flushcolsend
\input{leb/content/new-testament/Eph.tex}\flushcolsend
\input{leb/content/new-testament/Phil.tex}\flushcolsend
\biblebook{Colossians}
\begin{biblechapter}% Colossians 1
\verseWithHeading{Greeting}{Paul, an apostle of Christ Jesus by the will of God, and Timothy our brother,}%
\verse{To the saints and faithful brothers in Christ at Colossae: Grace to you and peace from God our Father.}%
\verseWithHeading{Thanksgiving and Prayer}{We always thank God, the Father of our Lord Jesus Christ, when we pray for you,}%
\verse{since we heard of your faith in Christ Jesus and of the love that you have for all the saints,}%
\verse{because of the hope laid up for you in heaven. Of this you have heard before in the word of the truth, the gospel,}%
\verse{which has come to you, as indeed in the whole world it is bearing fruit and increasing— as it also does among you, since the day you heard it and understood the grace of God in truth,}%
\verse{just as you learned it from Epaphras our beloved fellow servant. He is a faithful minister of Christ on your behalf}%
\verse{and has made known to us your love in the Spirit.}%
\verse{And so, from the day we heard, we have not ceased to pray for you, asking that you may be filled with the knowledge of his will in all spiritual wisdom and understanding,}%
\verse{so as to walk in a manner worthy of the Lord, fully pleasing to him: bearing fruit in every good work and increasing in the knowledge of God;}%
\verse{being strengthened with all power, according to his glorious might, for all endurance and patience with joy;}%
\verse{giving thanks to the Father, who has qualified you to share in the inheritance of the saints in light.}%
\verse{He has delivered us from the domain of darkness and transferred us to the kingdom of his beloved Son,}%
\verse{in whom we have redemption, the forgiveness of sins.}%
\verseWithHeading{The Preeminence of Christ}{He is the image of the invisible God, the firstborn of all creation.}%
\verse{For by him all things were created, in heaven and on earth, visible and invisible, whether thrones or dominions or rulers or authorities— all things were created through him and for him.}%
\verse{And he is before all things, and in him all things hold together.}%
\verse{And he is the head of the body, the church. He is the beginning, the firstborn from the dead, that in everything he might be preeminent.}%
\verse{For in him all the fullness of God was pleased to dwell,}%
\verse{and through him to reconcile to himself all things, whether on earth or in heaven, making peace by the blood of his cross.}%
\verse{And you, who once were alienated and hostile in mind, doing evil deeds,}%
\verse{he has now reconciled in his body of flesh by his death, in order to present you holy and blameless and above reproach before him,}%
\verse{if indeed you continue in the faith, stable and steadfast, not shifting from the hope of the gospel that you heard, which has been proclaimed in all creation under heaven, and of which I, Paul, became a minister.}%
\verseWithHeading{Paul’s Ministry to the Church}{Now I rejoice in my sufferings for your sake, and in my flesh I am filling up what is lacking in Christ's afflictions for the sake of his body, that is, the church,}%
\verse{of which I became a minister according to the stewardship from God that was given to me for you, to make the word of God fully known,}%
\verse{the mystery hidden for ages and generations but now revealed to his saints.}%
\verse{To them God chose to make known how great among the Gentiles are the riches of the glory of this mystery, which is Christ in you, the hope of glory.}%
\verse{Him we proclaim, warning everyone and teaching everyone with all wisdom, that we may present everyone mature in Christ.}%
\verse{For this I toil, struggling with all his energy that he powerfully works within me.}%
\end{biblechapter}%
\begin{biblechapter}% Colossians 2
\verse{For I want you to know how great a struggle I have for you and for those at Laodicea and for all who have not seen me face to face,}%
\verse{that their hearts may be encouraged, being knit together in love, to reach all the riches of full assurance of understanding and the knowledge of God's mystery, which is Christ,}%
\verse{in whom are hidden all the treasures of wisdom and knowledge.}%
\verse{I say this in order that no one may delude you with plausible arguments.}%
\verse{For though I am absent in body, yet I am with you in spirit, rejoicing to see your good order and the firmness of your faith in Christ.}%
\verseWithHeading{Alive in Christ}{Therefore, as you received Christ Jesus the Lord, so walk in him,}%
\verse{rooted and built up in him and established in the faith, just as you were taught, abounding in thanksgiving.}%
\verse{See to it that no one takes you captive by philosophy and empty deceit, according to human tradition, according to the elemental spirits of the world, and not according to Christ.}%
\verse{For in him the whole fullness of deity dwells bodily,}%
\verse{and you have been filled in him, who is the head of all rule and authority.}%
\verse{In him also you were circumcised with a circumcision made without hands, by putting off the body of the flesh, by the circumcision of Christ,}%
\verse{having been buried with him in baptism, in which you were also raised with him through faith in the powerful working of God, who raised him from the dead.}%
\verse{And you, who were dead in your trespasses and the uncircumcision of your flesh, God made alive together with him, having forgiven us all our trespasses,}%
\verse{by canceling the record of debt that stood against us with its legal demands. This he set aside, nailing it to the cross.}%
\verse{He disarmed the rulers and authorities and put them to open shame, by triumphing over them in him.}%
\verseWithHeading{Let No One Disqualify You}{Therefore let no one pass judgment on you in questions of food and drink, or with regard to a festival or a new moon or a Sabbath.}%
\verse{These are a shadow of the things to come, but the substance belongs to Christ.}%
\verse{Let no one disqualify you, insisting on asceticism and worship of angels, going on in detail about visions, puffed up without reason by his sensuous mind,}%
\verse{and not holding fast to the Head, from whom the whole body, nourished and knit together through its joints and ligaments, grows with a growth that is from God.}%
\verse{If with Christ you died to the elemental spirits of the world, why, as if you were still alive in the world, do you submit to regulations—}%
\verse{“Do not handle, Do not taste, Do not touch”}%
\verse{(referring to things that all perish as they are used)— according to human precepts and teachings?}%
\verse{These have indeed an appearance of wisdom in promoting self-made religion and asceticism and severity to the body, but they are of no value in stopping the indulgence of the flesh.}%
\end{biblechapter}%
\begin{biblechapter}% Colossians 3
\verseWithHeading{Put On the New Self}{If then you have been raised with Christ, seek the things that are above, where Christ is, seated at the right hand of God.}%
\verse{Set your minds on things that are above, not on things that are on earth.}%
\verse{For you have died, and your life is hidden with Christ in God.}%
\verse{When Christ who is your life appears, then you also will appear with him in glory.}%
\verse{Put to death therefore what is earthly in you: sexual immorality, impurity, passion, evil desire, and covetousness, which is idolatry.}%
\verse{On account of these the wrath of God is coming.}%
\verse{In these you too once walked, when you were living in them.}%
\verse{But now you must put them all away: anger, wrath, malice, slander, and obscene talk from your mouth.}%
\verse{Do not lie to one another, seeing that you have put off the old self with its practices}%
\verse{and have put on the new self, which is being renewed in knowledge after the image of its creator.}%
\verse{Here there is not Greek and Jew, circumcised and uncircumcised, barbarian, Scythian, slave, free; but Christ is all, and in all.}%
\verse{Put on then, as God's chosen ones, holy and beloved, compassionate hearts, kindness, humility, meekness, and patience,}%
\verse{bearing with one another and, if one has a complaint against another, forgiving each other; as the Lord has forgiven you, so you also must forgive.}%
\verse{And above all these put on love, which binds everything together in perfect harmony.}%
\verse{And let the peace of Christ rule in your hearts, to which indeed you were called in one body. And be thankful.}%
\verse{Let the word of Christ dwell in you richly, teaching and admonishing one another in all wisdom, singing psalms and hymns and spiritual songs, with thankfulness in your hearts to God.}%
\verse{And whatever you do, in word or deed, do everything in the name of the Lord Jesus, giving thanks to God the Father through him.}%
\verseWithHeading{Rules for Christian Households}{Wives, submit to your husbands, as is fitting in the Lord.}%
\verse{Husbands, love your wives, and do not be harsh with them.}%
\verse{Children, obey your parents in everything, for this pleases the Lord.}%
\verse{Fathers, do not provoke your children, lest they become discouraged.}%
\verse{Bondservants, obey in everything those who are your earthly masters, not by way of eye-service, as people-pleasers, but with sincerity of heart, fearing the Lord.}%
\verse{Whatever you do, work heartily, as for the Lord and not for men,}%
\verse{knowing that from the Lord you will receive the inheritance as your reward. You are serving the Lord Christ.}%
\verse{For the wrongdoer will be paid back for the wrong he has done, and there is no partiality.}%
\end{biblechapter}%
\begin{biblechapter}% Colossians 4
\verse{Masters, treat your bondservants justly and fairly, knowing that you also have a Master in heaven.}%
\verseWithHeading{Further Instructions}{Continue steadfastly in prayer, being watchful in it with thanksgiving.}%
\verse{At the same time, pray also for us, that God may open to us a door for the word, to declare the mystery of Christ, on account of which I am in prison—}%
\verse{that I may make it clear, which is how I ought to speak.}%
\verse{Walk in wisdom toward outsiders, making the best use of the time.}%
\verse{Let your speech always be gracious, seasoned with salt, so that you may know how you ought to answer each person.}%
\verseWithHeading{Final Greetings}{Tychicus will tell you all about my activities. He is a beloved brother and faithful minister and fellow servant in the Lord.}%
\verse{I have sent him to you for this very purpose, that you may know how we are and that he may encourage your hearts,}%
\verse{and with him Onesimus, our faithful and beloved brother, who is one of you. They will tell you of everything that has taken place here.}%
\verse{Aristarchus my fellow prisoner greets you, and Mark the cousin of Barnabas (concerning whom you have received instructions— if he comes to you, welcome him),}%
\verse{and Jesus who is called Justus. These are the only men of the circumcision among my fellow workers for the kingdom of God, and they have been a comfort to me.}%
\verse{Epaphras, who is one of you, a servant of Christ Jesus, greets you, always struggling on your behalf in his prayers, that you may stand mature and fully assured in all the will of God.}%
\verse{For I bear him witness that he has worked hard for you and for those in Laodicea and in Hierapolis.}%
\verse{Luke the beloved physician greets you, as does Demas.}%
\verse{Give my greetings to the brothers at Laodicea, and to Nympha and the church in her house.}%
\verse{And when this letter has been read among you, have it also read in the church of the Laodiceans; and see that you also read the letter from Laodicea.}%
\verse{And say to Archippus, “See that you fulfill the ministry that you have received in the Lord.”}%
\verse{I, Paul, write this greeting with my own hand. Remember my chains. Grace be with you.}%
\end{biblechapter}%
\flushcolsend
\input{leb/content/new-testament/1Ths.tex}\flushcolsend
\biblebook{2 Thessalonians}
\begin{biblechapter}% 2 Thessalonians 1
\verseWithHeading{Greeting}{Paul, Silvanus, and Timothy, To the church of the Thessalonians in God our Father and the Lord Jesus Christ:}%
\verse{Grace to you and peace from God our Father and the Lord Jesus Christ.}%
\verseWithHeading{Thanksgiving}{We ought always to give thanks to God for you, brothers, as is right, because your faith is growing abundantly, and the love of every one of you for one another is increasing.}%
\verse{Therefore we ourselves boast about you in the churches of God for your steadfastness and faith in all your persecutions and in the afflictions that you are enduring.}%
\verseWithHeading{The Judgment at Christ’s Coming}{This is evidence of the righteous judgment of God, that you may be considered worthy of the kingdom of God, for which you are also suffering—}%
\verse{since indeed God considers it just to repay with affliction those who afflict you,}%
\verse{and to grant relief to you who are afflicted as well as to us, when the Lord Jesus is revealed from heaven with his mighty angels}%
\verse{in flaming fire, inflicting vengeance on those who do not know God and on those who do not obey the gospel of our Lord Jesus.}%
\verse{They will suffer the punishment of eternal destruction, away from the presence of the Lord and from the glory of his might,}%
\verse{when he comes on that day to be glorified in his saints, and to be marveled at among all who have believed, because our testimony to you was believed.}%
\verse{To this end we always pray for you, that our God may make you worthy of his calling and may fulfill every resolve for good and every work of faith by his power,}%
\verse{so that the name of our Lord Jesus may be glorified in you, and you in him, according to the grace of our God and the Lord Jesus Christ.}%
\end{biblechapter}%
\begin{biblechapter}% 2 Thessalonians 2
\verseWithHeading{The Man of Lawlessness}{Now concerning the coming of our Lord Jesus Christ and our being gathered together to him, we ask you, brothers,}%
\verse{not to be quickly shaken in mind or alarmed, either by a spirit or a spoken word, or a letter seeming to be from us, to the effect that the day of the Lord has come.}%
\verse{Let no one deceive you in any way. For that day will not come, unless the rebellion comes first, and the man of lawlessness is revealed, the son of destruction,}%
\verse{who opposes and exalts himself against every so-called god or object of worship, so that he takes his seat in the temple of God, proclaiming himself to be God.}%
\verse{Do you not remember that when I was still with you I told you these things?}%
\verse{And you know what is restraining him now so that he may be revealed in his time.}%
\verse{For the mystery of lawlessness is already at work. Only he who now restrains it will do so until he is out of the way.}%
\verse{And then the lawless one will be revealed, whom the Lord Jesus will kill with the breath of his mouth and bring to nothing by the appearance of his coming.}%
\verse{The coming of the lawless one is by the activity of Satan with all power and false signs and wonders,}%
\verse{and with all wicked deception for those who are perishing, because they refused to love the truth and so be saved.}%
\verse{Therefore God sends them a strong delusion, so that they may believe what is false,}%
\verse{in order that all may be condemned who did not believe the truth but had pleasure in unrighteousness.}%
\verseWithHeading{Stand Firm}{But we ought always to give thanks to God for you, brothers beloved by the Lord, because God chose you as the firstfruits to be saved, through sanctification by the Spirit and belief in the truth.}%
\verse{To this he called you through our gospel, so that you may obtain the glory of our Lord Jesus Christ.}%
\verse{So then, brothers, stand firm and hold to the traditions that you were taught by us, either by our spoken word or by our letter.}%
\verse{Now may our Lord Jesus Christ himself, and God our Father, who loved us and gave us eternal comfort and good hope through grace,}%
\verse{comfort your hearts and establish them in every good work and word.}%
\end{biblechapter}%
\begin{biblechapter}% 2 Thessalonians 3
\verseWithHeading{Pray for Us}{Finally, brothers, pray for us, that the word of the Lord may speed ahead and be honored, as happened among you,}%
\verse{and that we may be delivered from wicked and evil men. For not all have faith.}%
\verse{But the Lord is faithful. He will establish you and guard you against the evil one.}%
\verse{And we have confidence in the Lord about you, that you are doing and will do the things that we command.}%
\verse{May the Lord direct your hearts to the love of God and to the steadfastness of Christ.}%
\verseWithHeading{Warning Against Idleness}{Now we command you, brothers, in the name of our Lord Jesus Christ, that you keep away from any brother who is walking in idleness and not in accord with the tradition that you received from us.}%
\verse{For you yourselves know how you ought to imitate us, because we were not idle when we were with you,}%
\verse{nor did we eat anyone's bread without paying for it, but with toil and labor we worked night and day, that we might not be a burden to any of you.}%
\verse{It was not because we do not have that right, but to give you in ourselves an example to imitate.}%
\verse{For even when we were with you, we would give you this command: If anyone is not willing to work, let him not eat.}%
\verse{For we hear that some among you walk in idleness, not busy at work, but busybodies.}%
\verse{Now such persons we command and encourage in the Lord Jesus Christ to do their work quietly and to earn their own living.}%
\verse{As for you, brothers, do not grow weary in doing good.}%
\verse{If anyone does not obey what we say in this letter, take note of that person, and have nothing to do with him, that he may be ashamed.}%
\verse{Do not regard him as an enemy, but warn him as a brother.}%
\verseWithHeading{Benediction}{Now may the Lord of peace himself give you peace at all times in every way. The Lord be with you all.}%
\verse{I, Paul, write this greeting with my own hand. This is the sign of genuineness in every letter of mine; it is the way I write.}%
\verse{The grace of our Lord Jesus Christ be with you all.}%
\end{biblechapter}%
\flushcolsend
\biblebook{1 Timothy}
\begin{biblechapter}% 1 Timothy 1
\verseWithHeading{Greeting}{Paul, an apostle of Christ Jesus according to the command of God our Savior and of Christ Jesus our hope,}%
\verse{to Timothy, my true child in the faith. Grace, mercy, and peace from God the Father and Christ Jesus our Lord.}%
\verseWithHeading{Instructions for Timothy in Ephesus}{Just as I urged you when I\lebnote{“\textit{when}” is supplied as a component of the participle (“traveled”) which is understood as temporal} traveled to Macedonia, remain in Ephesus, so that you may instruct certain people not to teach other doctrine,}%
\verse{and not to pay attention to myths and endless genealogies, which cause useless speculations rather than God’s plan that is by faith.}%
\verse{But the goal of our instruction is love from a pure heart and a good conscience and a faith without hypocrisy,}%
\verse{from which some have deviated, and have turned away into fruitless discussion,}%
\verse{wanting to be teachers of the law, although they\lebnote{“\textit{although}” is supplied as a component of the participle (“understand”) which is understood as concessive} do not understand either what they are saying or the things about which they confidently assert.}%
\verse{But we know that the law is good, if anyone makes use of it lawfully,}%
\verse{knowing this, that the law is not given for a righteous person but for the lawless and rebellious, for the ungodly and sinners, for the unholy and totally worldly, for the one who kills his father and the one who kills his mother, for murderers,}%
\verse{sexually immoral people, homosexuals, kidnappers, liars, perjurers, and whatever\lebnote{“if anything”} else is opposed to sound teaching,}%
\verse{according to the glorious gospel of the blessed God that I was entrusted with.}%
\verseWithHeading{Paul’s Thankfulness for the Mercy Shown to Him}{I give thanks\lebnote{“I have thankfulness”} to the one who strengthens me, Christ Jesus our Lord, because he considered me faithful, placing me into ministry,\lebnote{Or “service”}}%
\verse{although I\lebnote{“\textit{although}” is supplied as a component of the participle (“was”) which is understood as concessive} was formerly a blasphemer and a persecutor and a violent man, but I was shown mercy because, being ignorant, I did it in unbelief.}%
\verse{Surpassingly increased then the grace of our Lord with the faith and love that are in Christ Jesus!}%
\verse{The saying is trustworthy and worthy of all acceptance: Christ Jesus came into the world to save sinners, of whom I am the foremost.}%
\verse{But because of this I was shown mercy, in order that in me foremost, Christ Jesus might demonstrate his total patience, for an example for those who are going to believe in him for eternal life.}%
\verse{Now to the King of the ages, immortal, invisible, to the only God, be honor and glory forever and ever\lebnote{“to the ages of the ages”}. Amen.}%
\verseWithHeading{Paul’s Charge to Timothy}{I am setting before you this instruction, Timothy my child, in accordance with the prophecies spoken long ago about you, in order that by them you may fight the good fight,}%
\verse{having faith and a good conscience, which some, because they\lebnote{“\textit{because}” is supplied as a component of the participle (“have rejected”) which is understood as causal} have rejected these, have suffered shipwreck concerning their faith,}%
\verse{among whom are Hymenaeus and Alexander, whom I have handed over to Satan, in order that they may be taught not to blaspheme.}%
\end{biblechapter}%
\begin{biblechapter}% 1 Timothy 2
\verseWithHeading{Instructions to Pray for All People}{Therefore, I urge first of all that petitions, prayers, requests, and thanksgiving be made on behalf of all people,}%
\verse{on behalf of kings and all those who are in authority, in order that we may live a tranquil and quiet life in all godliness and dignity.}%
\verse{This is good and acceptable before God our Savior,}%
\verse{who wants all people to be saved and to come to a knowledge of the truth.}%
\verse{For there is one God and one mediator between God and human beings, the man Christ Jesus,}%
\verse{who gave himself a ransom for all, the testimony at the proper time,}%
\verse{for which I was appointed a herald and an apostle — I am speaking the truth, I am not lying — a teacher of the Gentiles in faith and truth.}%
\verse{Therefore I want the men in every place to pray, lifting up holy hands without anger and dispute.}%
\verseWithHeading{Instructions for Women}{Likewise also the women should adorn themselves in appropriate clothing, with modesty and self-control, not with braided hair and gold jewelry or pearls or expensive clothing,}%
\verse{but with good deeds which are fitting for women who profess godliness.}%
\verse{A woman must learn in quietness with all submission.}%
\verse{But I do not permit a woman to teach or to exercise authority over a man, but to remain quiet\lebnote{“to be in quietness”}.}%
\verse{For Adam was formed first, then Eve,}%
\verse{and Adam was not deceived, but the woman, because she\lebnote{“\textit{because}” is supplied as a component of the participle (“was deceived”) which is understood as causal} was deceived, came into transgression.}%
\verse{But she will be saved through the bearing of children, if she continues in faith and love and holiness with self-control.}%
\end{biblechapter}%
\begin{biblechapter}% 1 Timothy 3
\verseWithHeading{Qualifications for Overseers}{The saying is trustworthy: if anyone aspires to supervision, he desires a good work.}%
\verse{Therefore the overseer must be irreproachable, the husband of one wife, temperate, self-controlled, respectable, hospitable, skillful in teaching,}%
\verse{not addicted to wine, not a violent person, but gentle, peaceable, not loving money,}%
\verse{managing his own household well, having children in submission with all dignity}%
\verse{(but if someone does not know how to manage his own household, how will he take care of the church of God?),}%
\verse{not newly converted, lest he become conceited and fall into the condemnation of the devil.}%
\verse{But he must also have a good testimony from those outside, in order that he may not fall into disgrace and the trap of the devil.}%
\verseWithHeading{Qualifications for Deacons}{Deacons likewise must be dignified, not insincere, not devoted to much wine, not fond of dishonest gain,}%
\verse{holding the mystery of the faith with a clear conscience,}%
\verse{and these also must be tested first; then let them serve if they\lebnote{“\textit{if}” is supplied as a component of the participle (“are”) which is understood as conditional} are above reproach.}%
\verse{The wives\lebnote{Or “The women”} likewise must be dignified, not slanderous, temperate, faithful in all things.}%
\verse{Deacons must be husbands of one wife, managing their children and their own households well.}%
\verse{For those who have served well acquire a good standing for themselves, and great boldness in the faith that is in Christ Jesus.}%
\verseWithHeading{The Mystery of Godliness Described}{I am writing these things to you, hoping to come to you in a short time.}%
\verse{But if I am delayed, I am writing\lebnote{The words “I am writing” are not in the Greek text, but are an understood repetition from the previous clause} in order that you may know how one must conduct oneself in the household of God, which is the church of the living God, the pillar and mainstay\lebnote{Or “basis”; or “support”} of the truth.}%
\verse{And most certainly, great is the mystery of godliness: Who was revealed in the flesh, was vindicated by\lebnote{Or perhaps “in”} the Spirit, was seen by angels, was proclaimed among the Gentiles\lebnote{Or “nations”; the same Greek word can be translated “nations” or “Gentiles” depending on the context}, was believed on in the world, was taken up in glory.}%
\end{biblechapter}%
\begin{biblechapter}% 1 Timothy 4
\verseWithHeading{The Coming Apostasy}{Now the Spirit explicitly says that in the last times some will depart from the faith, paying attention to deceitful spirits and teachings of demons,}%
\verse{by the hypocrisy of liars, who are seared in their own conscience,}%
\verse{who forbid marrying and insist on abstaining from foods that God created for sharing in with thankfulness by those who believe and who know the truth,}%
\verse{because everything created by God is good and nothing is to be rejected if it is\lebnote{“\textit{if}” is supplied as a component of the participle (“received”) which is understood as conditional} received with thankfulness,}%
\verse{for it is made holy by the word of God and prayer.}%
\verseWithHeading{The Good Servant of Christ}{By\lebnote{“\textit{by}” is supplied as a component of the participle (“teaching”) which is understood as means} teaching these things to the brothers, you will be a good servant of Christ Jesus, trained in the words of the faith and of the good teaching that you have followed faithfully.}%
\verse{But reject those worthless myths told by elderly women\lebnote{“worthless and characteristic of an elderly woman myths”}, and train yourself for godliness.}%
\verse{For the training of the body is somewhat\lebnote{“for a little”} profitable, but godliness is profitable for everything, because it\lebnote{“\textit{because}” is supplied as a component of the participle (“holds”) which is understood as causal} holds promise for the present life and for the life to come.}%
\verse{The statement is trustworthy and deserving of complete acceptance.}%
\verse{For to this end we labor and suffer reproach,\lebnote{Some manuscripts have “and strive”} because we have put our hope in the living God, who is the Savior of all people, especially of believers.}%
\verse{Command these things and teach them.}%
\verse{Let no one look down on your youth, but be an example for the believers in word, in conduct, in love, in faith, in purity.}%
\verse{Until I come, pay attention to the public reading,\lebnote{Many English translations supply “of scripture” here to clarify what is to be read aloud} to exhortation, to teaching.}%
\verse{Do not neglect the gift that is in you, that was granted to you through prophecy with the laying on of hands by the council of elders.}%
\verse{Practice these things. Be diligent\lebnote{“be in these \textit{things},” though most English versions supply a predicate here} in these things, in order that your progress may be evident to everyone.}%
\verse{Fix your attention on yourself and on your teaching. Continue in them, for by\lebnote{“\textit{by}” is supplied as a component of the participle (“doing”) which is understood as means} doing this you will save both yourself and those who hear you.}%
\end{biblechapter}%
\begin{biblechapter}% 1 Timothy 5
\verseWithHeading{Instructions About Widows}{Do not rebuke an older man, but appeal to him as a father, younger men as brothers,}%
\verse{older women as mothers, younger women as sisters, with all purity.}%
\verse{Honor widows who are truly widows.}%
\verse{But if any widow has children or grandchildren, they must learn to show profound respect for their own household first, and to pay back recompense to their parents, for this is pleasing in the sight of God.}%
\verse{But the widow who is one truly, and is left alone, has put her hope in God and continues in her petitions and prayers night and day.}%
\verse{But the one who lives for sensual pleasure is dead even though she\lebnote{“\textit{even though}” is supplied as a component of the participle (“lives”) which is understood as concessive} lives.}%
\verse{And command these things, in order that they may be irreproachable.}%
\verse{But if someone does not provide for his own relatives, and especially the members of his household, he has denied the faith and is worse than an unbeliever.}%
\verse{Let a widow be put on the list if she\lebnote{“\textit{if}” is supplied as a component of the participle (“is”) which is understood as conditional} is not less than sixty years old, the wife of one husband,}%
\verse{being well-attested by good works, if she has brought up children, if she has shown hospitality, if she has washed the feet of the saints, if she has helped those who are oppressed, if she has devoted herself to every good work.}%
\verse{But refuse younger widows, for whenever their physical desires lead them away from Christ, they want to marry,}%
\verse{thus incurring condemnation because they have broken their former pledge.}%
\verse{And at the same time also, going around from house to house, they learn to be idle, and not only idle, but also gossipy and busybodies, saying the things that are not necessary.}%
\verse{Therefore I want younger widows to marry, to bear children, to manage a household, to give the adversary no opportunity for reproach.}%
\verse{For already some have turned away and followed after Satan.}%
\verse{If any believing woman has widows, she must help them, and the church must not be burdened, in order that it may help those who are truly widows.}%
\verseWithHeading{Honoring Worthy Elders and Dealing With Sinners}{The elders who lead well must be considered worthy of double honor, especially those who labor by speaking and teaching.}%
\verse{For the scripture says, “You must not muzzle an ox while it\lebnote{“\textit{while}” is supplied as a component of the participle (“threshing”) which is understood as temporal} is threshing,”\lebnote{from Deut 25:4} and “The worker is worthy of his wages.”\lebnote{from Luke 10:7}}%
\verse{Do not accept an accusation against an elder except on the evidence of two or three witnesses.}%
\verse{Reprove those who sin in the presence of all, in order that the rest also may experience fear.}%
\verse{I testify solemnly before God and Christ Jesus and the elect angels that you observe these things without prejudice, doing nothing according to partiality.}%
\verse{Lay hands on no one hastily, and do not participate in the sins of others. Keep yourself pure.}%
\verse{(No longer drink only water, but use a little wine for your stomach and your frequent illnesses.)}%
\verse{The sins of some people are evident, preceding them to judgment, but for some also they follow after them.}%
\verse{Likewise also good works are evident, and those considered otherwise are not able to be hidden.}%
\end{biblechapter}%
\begin{biblechapter}% 1 Timothy 6
\verseWithHeading{Slaves and Masters}{All those who are under the yoke as slaves must regard their own masters as worthy of all honor, lest the name of God and the teaching be slandered.}%
\verse{And those who have believing masters must not look down on them because they are brothers, but rather they must serve, because those who benefit by their service are believers and dearly loved.\innerVerseHeading{False Teachers and the Love of Money}Teach and encourage these things.}%
\verse{If anyone teaches other doctrine and does not devote himself to the sound words of our Lord Jesus Christ and the teaching that is in accordance with godliness,}%
\verse{he is conceited, understanding nothing, but having a morbid interest concerning controversies and disputes about words, from which come envy, strife, slanders, evil suspicions,}%
\verse{constant wrangling by people of depraved mind and deprived of the truth, who consider godliness to be a means of gain.}%
\verse{But godliness with contentment is a great means of gain.}%
\verse{For we have brought nothing into the world, so that neither can we bring anything out.}%
\verse{But if we\lebnote{“\textit{if}” is supplied as a component of the participle (“having”) which is understood as conditional} have food and clothing, with these things we will be content.}%
\verse{But those who want to be rich fall into temptation and a trap and many foolish and harmful desires, which plunge those people into ruin and destruction.}%
\verse{For the love of money is a root of all evil, by which some, because they\lebnote{“\textit{because}” is supplied as a component of the participle (“desire”) which is understood as causal} desire it, have gone astray from the faith and have pierced themselves with many pains.}%
\verseWithHeading{Renewed Charge to Timothy}{But you, O man of God, flee from these things, and pursue righteousness, godliness, faith, love, patient endurance, gentleness.}%
\verse{Fight the good fight of the faith; take hold of the eternal life to which you were called, and confessed the good confession in the presence of many witnesses.}%
\verse{I command you, in the sight of God who gives life to all things and Christ Jesus who testified the good confession before Pontius Pilate,}%
\verse{that you observe the commandment without fault, irreproachable until the appearing of our Lord Jesus Christ,}%
\verse{which he will make known in his own time, the blessed and only Sovereign, the King of those who reign as kings and Lord of those who rule as lords,}%
\verse{the one who alone possesses immortality, who lives in unapproachable light, whom no human being has seen nor is able to see, to whom be honor and eternal power. Amen.}%
\verseWithHeading{Instructions to the Rich}{Command those who are rich in this present age not to be proud and not to put their hope in the uncertainty of riches, but in God, who provides us all things richly for enjoyment,}%
\verse{to do good, to be rich in good works, to be generous, sharing freely,}%
\verse{storing up for themselves a good foundation for the future, in order that they may take hold of what is truly life.}%
\verseWithHeading{Final Charge and Benediction}{O Timothy, guard what has been entrusted to you. Turn away from pointless empty talk and contradictions of what is falsely called knowledge,}%
\verse{which some, by\lebnote{“\textit{by}” is supplied as a component of the participle (“professing”) which is understood as means} professing it, have deviated concerning the faith. Grace be with you all.}%
\end{biblechapter}%
\flushcolsend
\input{leb/content/new-testament/2Tim.tex}\flushcolsend
\biblebook{Titus}
\begin{biblechapter}% Titus 1
\verseWithHeading{Greeting}{Paul, a servant of God and an apostle of Jesus Christ, for the sake of the faith of God's elect and their knowledge of the truth, which accords with godliness,}%
\verse{in hope of eternal life, which God, who never lies, promised before the ages began}%
\verse{and at the proper time manifested in his word through the preaching with which I have been entrusted by the command of God our Savior;}%
\verse{To Titus, my true child in a common faith: Grace and peace from God the Father and Christ Jesus our Savior.}%
\verseWithHeading{Qualifications for Elders}{This is why I left you in Crete, so that you might put what remained into order, and appoint elders in every town as I directed you—}%
\verse{if anyone is above reproach, the husband of one wife, and his children are believers and not open to the charge of debauchery or insubordination.}%
\verse{For an overseer, as God's steward, must be above reproach. He must not be arrogant or quick-tempered or a drunkard or violent or greedy for gain,}%
\verse{but hospitable, a lover of good, self-controlled, upright, holy, and disciplined.}%
\verse{He must hold firm to the trustworthy word as taught, so that he may be able to give instruction in sound doctrine and also to rebuke those who contradict it.}%
\verse{For there are many who are insubordinate, empty talkers and deceivers, especially those of the circumcision party.}%
\verse{They must be silenced, since they are upsetting whole families by teaching for shameful gain what they ought not to teach.}%
\verse{One of the Cretans, a prophet of their own, said, “Cretans are always liars, evil beasts, lazy gluttons.”}%
\verse{This testimony is true. Therefore rebuke them sharply, that they may be sound in the faith,}%
\verse{not devoting themselves to Jewish myths and the commands of people who turn away from the truth.}%
\verse{To the pure, all things are pure, but to the defiled and unbelieving, nothing is pure; but both their minds and their consciences are defiled.}%
\verse{They profess to know God, but they deny him by their works. They are detestable, disobedient, unfit for any good work.}%
\end{biblechapter}%
\begin{biblechapter}% Titus 2
\verseWithHeading{Teach Sound Doctrine}{But as for you, teach what accords with sound doctrine.}%
\verse{Older men are to be sober-minded, dignified, self-controlled, sound in faith, in love, and in steadfastness.}%
\verse{Older women likewise are to be reverent in behavior, not slanderers or slaves to much wine. They are to teach what is good,}%
\verse{and so train the young women to love their husbands and children,}%
\verse{to be self-controlled, pure, working at home, kind, and submissive to their own husbands, that the word of God may not be reviled.}%
\verse{Likewise, urge the younger men to be self-controlled.}%
\verse{Show yourself in all respects to be a model of good works, and in your teaching show integrity, dignity,}%
\verse{and sound speech that cannot be condemned, so that an opponent may be put to shame, having nothing evil to say about us.}%
\verse{Bondservants are to be submissive to their own masters in everything; they are to be well-pleasing, not argumentative,}%
\verse{not pilfering, but showing all good faith, so that in everything they may adorn the doctrine of God our Savior.}%
\verse{For the grace of God has appeared, bringing salvation for all people,}%
\verse{training us to renounce ungodliness and worldly passions, and to live self-controlled, upright, and godly lives in the present age,}%
\verse{waiting for our blessed hope, the appearing of the glory of our great God and Savior Jesus Christ,}%
\verse{who gave himself for us to redeem us from all lawlessness and to purify for himself a people for his own possession who are zealous for good works.}%
\verse{Declare these things; exhort and rebuke with all authority. Let no one disregard you.}%
\end{biblechapter}%
\begin{biblechapter}% Titus 3
\verseWithHeading{Be Ready for Every Good Work}{Remind them to be submissive to rulers and authorities, to be obedient, to be ready for every good work,}%
\verse{to speak evil of no one, to avoid quarreling, to be gentle, and to show perfect courtesy toward all people.}%
\verse{For we ourselves were once foolish, disobedient, led astray, slaves to various passions and pleasures, passing our days in malice and envy, hated by others and hating one another.}%
\verse{But when the goodness and loving kindness of God our Savior appeared,}%
\verse{he saved us, not because of works done by us in righteousness, but according to his own mercy, by the washing of regeneration and renewal of the Holy Spirit,}%
\verse{whom he poured out on us richly through Jesus Christ our Savior,}%
\verse{so that being justified by his grace we might become heirs according to the hope of eternal life.}%
\verse{The saying is trustworthy, and I want you to insist on these things, so that those who have believed in God may be careful to devote themselves to good works. These things are excellent and profitable for people.}%
\verse{But avoid foolish controversies, genealogies, dissensions, and quarrels about the law, for they are unprofitable and worthless.}%
\verse{As for a person who stirs up division, after warning him once and then twice, have nothing more to do with him,}%
\verse{knowing that such a person is warped and sinful; he is self-condemned.}%
\verseWithHeading{Final Instructions and Greetings}{When I send Artemas or Tychicus to you, do your best to come to me at Nicopolis, for I have decided to spend the winter there.}%
\verse{Do your best to speed Zenas the lawyer and Apollos on their way; see that they lack nothing.}%
\verse{And let our people learn to devote themselves to good works, so as to help cases of urgent need, and not be unfruitful.}%
\verse{All who are with me send greetings to you. Greet those who love us in the faith. Grace be with you all.}%
\end{biblechapter}%
\flushcolsend
\biblebook{Philemon}
\begin{biblechapter}% Philemon 1
\verseWithHeading{Greeting}{Paul, a prisoner for Christ Jesus, and Timothy our brother, To Philemon our beloved fellow worker}%
\verse{and Apphia our sister and Archippus our fellow soldier, and the church in your house:}%
\verse{Grace to you and peace from God our Father and the Lord Jesus Christ.}%
\verseWithHeading{Philemon’s Love and Faith}{I thank my God always when I remember you in my prayers,}%
\verse{because I hear of your love and of the faith that you have toward the Lord Jesus and for all the saints,}%
\verse{and I pray that the sharing of your faith may become effective for the full knowledge of every good thing that is in us for the sake of Christ.}%
\verse{For I have derived much joy and comfort from your love, my brother, because the hearts of the saints have been refreshed through you.}%
\verseWithHeading{Paul’s Plea for Onesimus}{Accordingly, though I am bold enough in Christ to command you to do what is required,}%
\verse{yet for love's sake I prefer to appeal to you— I, Paul, an old man and now a prisoner also for Christ Jesus—}%
\verse{I appeal to you for my child, Onesimus, whose father I became in my imprisonment.}%
\verse{(Formerly he was useless to you, but now he is indeed useful to you and to me.)}%
\verse{I am sending him back to you, sending my very heart.}%
\verse{I would have been glad to keep him with me, in order that he might serve me on your behalf during my imprisonment for the gospel,}%
\verse{but I preferred to do nothing without your consent in order that your goodness might not be by compulsion but of your own accord.}%
\verse{For this perhaps is why he was parted from you for a while, that you might have him back forever,}%
\verse{no longer as a bondservant but more than a bondservant, as a beloved brother— especially to me, but how much more to you, both in the flesh and in the Lord.}%
\verse{So if you consider me your partner, receive him as you would receive me.}%
\verse{If he has wronged you at all, or owes you anything, charge that to my account.}%
\verse{I, Paul, write this with my own hand: I will repay it— to say nothing of your owing me even your own self.}%
\verse{Yes, brother, I want some benefit from you in the Lord. Refresh my heart in Christ.}%
\verse{Confident of your obedience, I write to you, knowing that you will do even more than I say.}%
\verse{At the same time, prepare a guest room for me, for I am hoping that through your prayers I will be graciously given to you.}%
\verseWithHeading{Final Greetings}{Epaphras, my fellow prisoner in Christ Jesus, sends greetings to you,}%
\verse{and so do Mark, Aristarchus, Demas, and Luke, my fellow workers.}%
\verse{The grace of the Lord Jesus Christ be with your spirit.}%
\end{biblechapter}%
\flushcolsend
\input{leb/content/new-testament/Heb.tex}\flushcolsend
\input{leb/content/new-testament/Jam.tex}\flushcolsend
\input{leb/content/new-testament/1Pet.tex}\flushcolsend
\input{leb/content/new-testament/2Pet.tex}\flushcolsend
\input{leb/content/new-testament/1Jn.tex}\flushcolsend
\input{leb/content/new-testament/2Jn.tex}\flushcolsend
\biblebook{3 John}
\begin{biblechapter}% 3 John 1
\verseWithHeading{Greeting}{The elder, to Gaius the beloved, whom I love in the truth.}%
\verse{Dear friend, I pray you may prosper concerning everything and be healthy, just as your soul prospers.}%
\verse{For I rejoiced exceedingly when the\lebnote{“\textit{when}” is supplied as a component of the participle (“came”) which is understood as temporal} brothers came and testified to your truth, just as you are walking in the truth.}%
\verse{I have no greater joy than this: that I hear my children are walking in the truth.}%
\verseWithHeading{Instructions to Gaius}{Dear friend, you act faithfully\lebnote{Or “you act loyally”} in whatever you do for the brothers, even though they are strangers\lebnote{“and this strangers”}.}%
\verse{They have testified to your love before the church; you will do well to send them\lebnote{“whom”} on their way in a manner worthy of God.}%
\verse{For they have gone out on behalf of the name, accepting nothing from the pagans.\lebnote{That is, Gentile unbelievers (as opposed to Gentile Christians)}}%
\verse{Therefore we ought to support such people, so that we become fellow workers with the truth.}%
\verseWithHeading{Diotrephes Causes Trouble}{I wrote something to the church, but Diotrephes, who wants to be first among them, does not acknowledge us.}%
\verse{Therefore, if I come, I will call attention to the deeds he is doing\lebnote{“his deeds which he is doing”}, disparaging us with evil words. And not being content with these, he does not receive the brothers himself, and he hinders those wanting to do so and throws them out of the church.}%
\verse{Dear friend, do not imitate what is evil, but what is good. The one who does good is of God; the one who does evil has not seen God.}%
\verseWithHeading{Demetrius Commended}{Demetrius has been testified to by all, even by the truth itself. And we also testify to him, and you know that our testimony is true.}%
\verseWithHeading{Conclusion and Final Greeting}{I have many things to write to you, but I do not want to write to you by means of ink and pen.}%
\verse{But I hope to see you right away, and to speak face to face\lebnote{“mouth to mouth”}.}%
\verse{Peace be to you. The friends greet you. Greet the friends by name.}%
\end{biblechapter}%
\flushcolsend
\input{leb/content/new-testament/Jud.tex}\flushcolsend
\input{leb/content/new-testament/Rev.tex}\flushcolsend
% a blank page on the back of the page - next page on a new right-sided page
\cleardoublepage

% a blank page follows the book of Revelation
\newpage
\thispagestyle{empty}
\mbox{}


\addcontentsline{toc}{part}{\headings{Extra Content}}

\onecolumn

\addcontentsline{toc}{chapter}{\headings{Dictionary}}


% one page of dictionary is a nice idea - some words used in this translation are a bit rare in everyday language
\thispagestyle{empty}

\section*{Useful Definitions}
\begin{multicols}{2}
\dictionaryentry{Adjure}{Transitive Verb}{to command solemnly under or as if under oath or penalty of a curse}
\dictionaryentry{Admonish}{Transitive Verb}{to say (something) as advice or a warning}
\dictionaryentry{Affliction}{Noun}{a cause of persistent pain or distress}
\dictionaryentry{Atonement}{Noun}{a thing given to balance wrong or injury; amends}
\dictionaryentry{Carousing}{Verb}{to spend time drinking alcohol, laughing and enjoying yourself in a noisy way with other people}
\dictionaryentry{Defiled}{Transitive Verb}{to corrupt the purity or perfection of}
\dictionaryentry{Dissension}{Noun}{strong disagreement, contention, discord; quarreling}
\dictionaryentry{Dissipation}{Noun}{indulgence in extravagant, intemperate, or dissolute pleasure}
\dictionaryentry{Eternal}{Adjective}{valid or existing at all times; timeless}
\dictionaryentry{Exhortation}{Transitive Verb}{to incite by argument or advice: urge strongly}
\dictionaryentry{Expiation}{the act of extinguishing the guilt incurred by something}
\dictionaryentry{Lest}{Conjunction}{"for fear that"; so that a thing should not happen}
\dictionaryentry{Licentiousness}{Adjective}{lacking legal or moral restraints}
\dictionaryentry{Mammon}{Noun}{material wealth or possessions especially as having a debasing influence}
\dictionaryentry{Nicolaitan}{Noun}{from Revelation 2:6,15. A compound word in Greek: Nico ("victory") - laitan ("the laity"), describes the factionism of the ignorant class (see Philippians 1:15-17, Jude 1:10-16, 2 Peter 3:16)}
\dictionaryentry{Parable}{Noun}{a usually short fictitious story that illustrates a moral attitude or a religious principle}
\dictionaryentry{Perish}{Verb}{to become destroyed or ruined: cease to exist}
\dictionaryentry{Persecution}{Transitive Verb}{to harass or punish in a manner designed to injure, grieve, or afflict - specifically: to cause to suffer because of belief}
\dictionaryentry{Propitiation}{Noun}{the act of gaining or regaining the favor or goodwill of someone}
\dictionaryentry{Repent}{Verb}{to change one's way of thinking, intentions, etc (usually from a sincere sorrow for sin or fault); be penitent}
\dictionaryentry{Righteous}{Adjective}{morally right or justifiable}
\dictionaryentry{Salvation}{Noun}{deliverance from danger or difficulty}
\dictionaryentry{Sanctified}{Verb}{made productive of holiness or piety}
\dictionaryentry{Satiated}{Verb}{to give somebody so much of something that they do not feel they want any more}
\dictionaryentry{Servility}{Noun}{excessively wanting to please somebody and obey them}
\dictionaryentry{Usury}{Noun}{lending one’s wealth to the poor to make a profit in addition to repayment. Often implemented as interest charged for loans.}
\dictionaryentry{Vacillation}{Intransitive Verb}{to waver in mind, will, or feeling: hesitate in choice of opinions or courses}
\end{multicols}


\flushcolsend
\clearpage
\newpage

% now we are going to include a concordance. The NWT has a fairly handy one and it is easily accesible.
% @TOTO: create a customised concordance


\addcontentsline{toc}{chapter}{\headings{Concordance}}



\includepdf[pages=-]{leb/content/concordance-6x9.pdf}


% let's also include a few of the maps from the NWT. It is nice to look at maps sometimes, when considering the geography relating to events.
\addcontentsline{toc}{chapter}{\headings{Diagrams and Maps}}
\includepdf[noautoscale=true, scale=1]{leb/content/pictures/second-temple-mount.png}
\includepdf[noautoscale=true, scale=1]{leb/content/pictures/settlement-of-the-promised-land.png}
\includepdf[noautoscale=true, scale=1]{leb/content/pictures/israel-in-times-of-king-david-and-solomon.png}
\includepdf[noautoscale=true, scale=1]{leb/content/pictures/israel-in-time-of-jesus.png}
\includepdf[noautoscale=true, scale=1]{leb/content/pictures/spread-of-christianity.png}
\includepdf[noautoscale=true, scale=1]{leb/content/pictures/units-of-currency.png}
\includepdf[noautoscale=true, scale=1]{leb/content/pictures/units-of-measurement.png}







% an empty page at the back
\cleardoublepage

\end{document}
